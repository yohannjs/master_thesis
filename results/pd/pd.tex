\section{Case Study: Patient Diagnosis}

\subsection{Time-series Clustering}

\begin{figure}[H]
    \centering
    % \includegraphics[width=\textwidth]{results/tsc_ind_dor_sens_spec_dist.png}
    %% Creator: Matplotlib, PGF backend
%%
%% To include the figure in your LaTeX document, write
%%   \input{<filename>.pgf}
%%
%% Make sure the required packages are loaded in your preamble
%%   \usepackage{pgf}
%%
%% Figures using additional raster images can only be included by \input if
%% they are in the same directory as the main LaTeX file. For loading figures
%% from other directories you can use the `import` package
%%   \usepackage{import}
%% and then include the figures with
%%   \import{<path to file>}{<filename>.pgf}
%%
%% Matplotlib used the following preamble
%%
\begingroup%
\makeatletter%
\begin{pgfpicture}%
\pgfpathrectangle{\pgfpointorigin}{\pgfqpoint{6.364000in}{2.540000in}}%
\pgfusepath{use as bounding box, clip}%
\begin{pgfscope}%
\pgfsetbuttcap%
\pgfsetmiterjoin%
\definecolor{currentfill}{rgb}{1.000000,1.000000,1.000000}%
\pgfsetfillcolor{currentfill}%
\pgfsetlinewidth{0.000000pt}%
\definecolor{currentstroke}{rgb}{1.000000,1.000000,1.000000}%
\pgfsetstrokecolor{currentstroke}%
\pgfsetdash{}{0pt}%
\pgfpathmoveto{\pgfqpoint{0.000000in}{0.000000in}}%
\pgfpathlineto{\pgfqpoint{6.364000in}{0.000000in}}%
\pgfpathlineto{\pgfqpoint{6.364000in}{2.540000in}}%
\pgfpathlineto{\pgfqpoint{0.000000in}{2.540000in}}%
\pgfpathclose%
\pgfusepath{fill}%
\end{pgfscope}%
\begin{pgfscope}%
\pgfsetbuttcap%
\pgfsetmiterjoin%
\definecolor{currentfill}{rgb}{0.917647,0.917647,0.949020}%
\pgfsetfillcolor{currentfill}%
\pgfsetlinewidth{0.000000pt}%
\definecolor{currentstroke}{rgb}{0.000000,0.000000,0.000000}%
\pgfsetstrokecolor{currentstroke}%
\pgfsetstrokeopacity{0.000000}%
\pgfsetdash{}{0pt}%
\pgfpathmoveto{\pgfqpoint{0.650810in}{0.557870in}}%
\pgfpathlineto{\pgfqpoint{3.096898in}{0.557870in}}%
\pgfpathlineto{\pgfqpoint{3.096898in}{2.242604in}}%
\pgfpathlineto{\pgfqpoint{0.650810in}{2.242604in}}%
\pgfpathclose%
\pgfusepath{fill}%
\end{pgfscope}%
\begin{pgfscope}%
\pgfpathrectangle{\pgfqpoint{0.650810in}{0.557870in}}{\pgfqpoint{2.446088in}{1.684734in}}%
\pgfusepath{clip}%
\pgfsetroundcap%
\pgfsetroundjoin%
\pgfsetlinewidth{1.003750pt}%
\definecolor{currentstroke}{rgb}{1.000000,1.000000,1.000000}%
\pgfsetstrokecolor{currentstroke}%
\pgfsetdash{}{0pt}%
\pgfpathmoveto{\pgfqpoint{0.761996in}{0.557870in}}%
\pgfpathlineto{\pgfqpoint{0.761996in}{2.242604in}}%
\pgfusepath{stroke}%
\end{pgfscope}%
\begin{pgfscope}%
\definecolor{textcolor}{rgb}{0.150000,0.150000,0.150000}%
\pgfsetstrokecolor{textcolor}%
\pgfsetfillcolor{textcolor}%
\pgftext[x=0.761996in,y=0.425926in,,top]{\color{textcolor}\sffamily\fontsize{11.000000}{13.200000}\selectfont \(\displaystyle 0\)}%
\end{pgfscope}%
\begin{pgfscope}%
\pgfpathrectangle{\pgfqpoint{0.650810in}{0.557870in}}{\pgfqpoint{2.446088in}{1.684734in}}%
\pgfusepath{clip}%
\pgfsetroundcap%
\pgfsetroundjoin%
\pgfsetlinewidth{1.003750pt}%
\definecolor{currentstroke}{rgb}{1.000000,1.000000,1.000000}%
\pgfsetstrokecolor{currentstroke}%
\pgfsetdash{}{0pt}%
\pgfpathmoveto{\pgfqpoint{1.426412in}{0.557870in}}%
\pgfpathlineto{\pgfqpoint{1.426412in}{2.242604in}}%
\pgfusepath{stroke}%
\end{pgfscope}%
\begin{pgfscope}%
\definecolor{textcolor}{rgb}{0.150000,0.150000,0.150000}%
\pgfsetstrokecolor{textcolor}%
\pgfsetfillcolor{textcolor}%
\pgftext[x=1.426412in,y=0.425926in,,top]{\color{textcolor}\sffamily\fontsize{11.000000}{13.200000}\selectfont \(\displaystyle 10\)}%
\end{pgfscope}%
\begin{pgfscope}%
\pgfpathrectangle{\pgfqpoint{0.650810in}{0.557870in}}{\pgfqpoint{2.446088in}{1.684734in}}%
\pgfusepath{clip}%
\pgfsetroundcap%
\pgfsetroundjoin%
\pgfsetlinewidth{1.003750pt}%
\definecolor{currentstroke}{rgb}{1.000000,1.000000,1.000000}%
\pgfsetstrokecolor{currentstroke}%
\pgfsetdash{}{0pt}%
\pgfpathmoveto{\pgfqpoint{2.090828in}{0.557870in}}%
\pgfpathlineto{\pgfqpoint{2.090828in}{2.242604in}}%
\pgfusepath{stroke}%
\end{pgfscope}%
\begin{pgfscope}%
\definecolor{textcolor}{rgb}{0.150000,0.150000,0.150000}%
\pgfsetstrokecolor{textcolor}%
\pgfsetfillcolor{textcolor}%
\pgftext[x=2.090828in,y=0.425926in,,top]{\color{textcolor}\sffamily\fontsize{11.000000}{13.200000}\selectfont \(\displaystyle 20\)}%
\end{pgfscope}%
\begin{pgfscope}%
\pgfpathrectangle{\pgfqpoint{0.650810in}{0.557870in}}{\pgfqpoint{2.446088in}{1.684734in}}%
\pgfusepath{clip}%
\pgfsetroundcap%
\pgfsetroundjoin%
\pgfsetlinewidth{1.003750pt}%
\definecolor{currentstroke}{rgb}{1.000000,1.000000,1.000000}%
\pgfsetstrokecolor{currentstroke}%
\pgfsetdash{}{0pt}%
\pgfpathmoveto{\pgfqpoint{2.755243in}{0.557870in}}%
\pgfpathlineto{\pgfqpoint{2.755243in}{2.242604in}}%
\pgfusepath{stroke}%
\end{pgfscope}%
\begin{pgfscope}%
\definecolor{textcolor}{rgb}{0.150000,0.150000,0.150000}%
\pgfsetstrokecolor{textcolor}%
\pgfsetfillcolor{textcolor}%
\pgftext[x=2.755243in,y=0.425926in,,top]{\color{textcolor}\sffamily\fontsize{11.000000}{13.200000}\selectfont \(\displaystyle 30\)}%
\end{pgfscope}%
\begin{pgfscope}%
\definecolor{textcolor}{rgb}{0.150000,0.150000,0.150000}%
\pgfsetstrokecolor{textcolor}%
\pgfsetfillcolor{textcolor}%
\pgftext[x=1.873854in,y=0.235185in,,top]{\color{textcolor}\sffamily\fontsize{11.000000}{13.200000}\selectfont DOR}%
\end{pgfscope}%
\begin{pgfscope}%
\pgfpathrectangle{\pgfqpoint{0.650810in}{0.557870in}}{\pgfqpoint{2.446088in}{1.684734in}}%
\pgfusepath{clip}%
\pgfsetroundcap%
\pgfsetroundjoin%
\pgfsetlinewidth{1.003750pt}%
\definecolor{currentstroke}{rgb}{1.000000,1.000000,1.000000}%
\pgfsetstrokecolor{currentstroke}%
\pgfsetdash{}{0pt}%
\pgfpathmoveto{\pgfqpoint{0.650810in}{0.557870in}}%
\pgfpathlineto{\pgfqpoint{3.096898in}{0.557870in}}%
\pgfusepath{stroke}%
\end{pgfscope}%
\begin{pgfscope}%
\definecolor{textcolor}{rgb}{0.150000,0.150000,0.150000}%
\pgfsetstrokecolor{textcolor}%
\pgfsetfillcolor{textcolor}%
\pgftext[x=0.442824in,y=0.505064in,left,base]{\color{textcolor}\sffamily\fontsize{11.000000}{13.200000}\selectfont \(\displaystyle 0\)}%
\end{pgfscope}%
\begin{pgfscope}%
\pgfpathrectangle{\pgfqpoint{0.650810in}{0.557870in}}{\pgfqpoint{2.446088in}{1.684734in}}%
\pgfusepath{clip}%
\pgfsetroundcap%
\pgfsetroundjoin%
\pgfsetlinewidth{1.003750pt}%
\definecolor{currentstroke}{rgb}{1.000000,1.000000,1.000000}%
\pgfsetstrokecolor{currentstroke}%
\pgfsetdash{}{0pt}%
\pgfpathmoveto{\pgfqpoint{0.650810in}{0.934515in}}%
\pgfpathlineto{\pgfqpoint{3.096898in}{0.934515in}}%
\pgfusepath{stroke}%
\end{pgfscope}%
\begin{pgfscope}%
\definecolor{textcolor}{rgb}{0.150000,0.150000,0.150000}%
\pgfsetstrokecolor{textcolor}%
\pgfsetfillcolor{textcolor}%
\pgftext[x=0.366783in,y=0.881709in,left,base]{\color{textcolor}\sffamily\fontsize{11.000000}{13.200000}\selectfont \(\displaystyle 50\)}%
\end{pgfscope}%
\begin{pgfscope}%
\pgfpathrectangle{\pgfqpoint{0.650810in}{0.557870in}}{\pgfqpoint{2.446088in}{1.684734in}}%
\pgfusepath{clip}%
\pgfsetroundcap%
\pgfsetroundjoin%
\pgfsetlinewidth{1.003750pt}%
\definecolor{currentstroke}{rgb}{1.000000,1.000000,1.000000}%
\pgfsetstrokecolor{currentstroke}%
\pgfsetdash{}{0pt}%
\pgfpathmoveto{\pgfqpoint{0.650810in}{1.311161in}}%
\pgfpathlineto{\pgfqpoint{3.096898in}{1.311161in}}%
\pgfusepath{stroke}%
\end{pgfscope}%
\begin{pgfscope}%
\definecolor{textcolor}{rgb}{0.150000,0.150000,0.150000}%
\pgfsetstrokecolor{textcolor}%
\pgfsetfillcolor{textcolor}%
\pgftext[x=0.290741in,y=1.258354in,left,base]{\color{textcolor}\sffamily\fontsize{11.000000}{13.200000}\selectfont \(\displaystyle 100\)}%
\end{pgfscope}%
\begin{pgfscope}%
\pgfpathrectangle{\pgfqpoint{0.650810in}{0.557870in}}{\pgfqpoint{2.446088in}{1.684734in}}%
\pgfusepath{clip}%
\pgfsetroundcap%
\pgfsetroundjoin%
\pgfsetlinewidth{1.003750pt}%
\definecolor{currentstroke}{rgb}{1.000000,1.000000,1.000000}%
\pgfsetstrokecolor{currentstroke}%
\pgfsetdash{}{0pt}%
\pgfpathmoveto{\pgfqpoint{0.650810in}{1.687806in}}%
\pgfpathlineto{\pgfqpoint{3.096898in}{1.687806in}}%
\pgfusepath{stroke}%
\end{pgfscope}%
\begin{pgfscope}%
\definecolor{textcolor}{rgb}{0.150000,0.150000,0.150000}%
\pgfsetstrokecolor{textcolor}%
\pgfsetfillcolor{textcolor}%
\pgftext[x=0.290741in,y=1.634999in,left,base]{\color{textcolor}\sffamily\fontsize{11.000000}{13.200000}\selectfont \(\displaystyle 150\)}%
\end{pgfscope}%
\begin{pgfscope}%
\pgfpathrectangle{\pgfqpoint{0.650810in}{0.557870in}}{\pgfqpoint{2.446088in}{1.684734in}}%
\pgfusepath{clip}%
\pgfsetroundcap%
\pgfsetroundjoin%
\pgfsetlinewidth{1.003750pt}%
\definecolor{currentstroke}{rgb}{1.000000,1.000000,1.000000}%
\pgfsetstrokecolor{currentstroke}%
\pgfsetdash{}{0pt}%
\pgfpathmoveto{\pgfqpoint{0.650810in}{2.064451in}}%
\pgfpathlineto{\pgfqpoint{3.096898in}{2.064451in}}%
\pgfusepath{stroke}%
\end{pgfscope}%
\begin{pgfscope}%
\definecolor{textcolor}{rgb}{0.150000,0.150000,0.150000}%
\pgfsetstrokecolor{textcolor}%
\pgfsetfillcolor{textcolor}%
\pgftext[x=0.290741in,y=2.011644in,left,base]{\color{textcolor}\sffamily\fontsize{11.000000}{13.200000}\selectfont \(\displaystyle 200\)}%
\end{pgfscope}%
\begin{pgfscope}%
\definecolor{textcolor}{rgb}{0.150000,0.150000,0.150000}%
\pgfsetstrokecolor{textcolor}%
\pgfsetfillcolor{textcolor}%
\pgftext[x=0.235185in,y=1.400237in,,bottom,rotate=90.000000]{\color{textcolor}\sffamily\fontsize{11.000000}{13.200000}\selectfont Occurance}%
\end{pgfscope}%
\begin{pgfscope}%
\pgfpathrectangle{\pgfqpoint{0.650810in}{0.557870in}}{\pgfqpoint{2.446088in}{1.684734in}}%
\pgfusepath{clip}%
\pgfsetbuttcap%
\pgfsetmiterjoin%
\definecolor{currentfill}{rgb}{0.298039,0.447059,0.690196}%
\pgfsetfillcolor{currentfill}%
\pgfsetfillopacity{0.400000}%
\pgfsetlinewidth{1.003750pt}%
\definecolor{currentstroke}{rgb}{1.000000,1.000000,1.000000}%
\pgfsetstrokecolor{currentstroke}%
\pgfsetstrokeopacity{0.400000}%
\pgfsetdash{}{0pt}%
\pgfpathmoveto{\pgfqpoint{0.761996in}{0.557870in}}%
\pgfpathlineto{\pgfqpoint{0.984368in}{0.557870in}}%
\pgfpathlineto{\pgfqpoint{0.984368in}{2.162379in}}%
\pgfpathlineto{\pgfqpoint{0.761996in}{2.162379in}}%
\pgfpathclose%
\pgfusepath{stroke,fill}%
\end{pgfscope}%
\begin{pgfscope}%
\pgfpathrectangle{\pgfqpoint{0.650810in}{0.557870in}}{\pgfqpoint{2.446088in}{1.684734in}}%
\pgfusepath{clip}%
\pgfsetbuttcap%
\pgfsetmiterjoin%
\definecolor{currentfill}{rgb}{0.298039,0.447059,0.690196}%
\pgfsetfillcolor{currentfill}%
\pgfsetfillopacity{0.400000}%
\pgfsetlinewidth{1.003750pt}%
\definecolor{currentstroke}{rgb}{1.000000,1.000000,1.000000}%
\pgfsetstrokecolor{currentstroke}%
\pgfsetstrokeopacity{0.400000}%
\pgfsetdash{}{0pt}%
\pgfpathmoveto{\pgfqpoint{0.984368in}{0.557870in}}%
\pgfpathlineto{\pgfqpoint{1.206739in}{0.557870in}}%
\pgfpathlineto{\pgfqpoint{1.206739in}{0.610601in}}%
\pgfpathlineto{\pgfqpoint{0.984368in}{0.610601in}}%
\pgfpathclose%
\pgfusepath{stroke,fill}%
\end{pgfscope}%
\begin{pgfscope}%
\pgfpathrectangle{\pgfqpoint{0.650810in}{0.557870in}}{\pgfqpoint{2.446088in}{1.684734in}}%
\pgfusepath{clip}%
\pgfsetbuttcap%
\pgfsetmiterjoin%
\definecolor{currentfill}{rgb}{0.298039,0.447059,0.690196}%
\pgfsetfillcolor{currentfill}%
\pgfsetfillopacity{0.400000}%
\pgfsetlinewidth{1.003750pt}%
\definecolor{currentstroke}{rgb}{1.000000,1.000000,1.000000}%
\pgfsetstrokecolor{currentstroke}%
\pgfsetstrokeopacity{0.400000}%
\pgfsetdash{}{0pt}%
\pgfpathmoveto{\pgfqpoint{1.206739in}{0.557870in}}%
\pgfpathlineto{\pgfqpoint{1.429111in}{0.557870in}}%
\pgfpathlineto{\pgfqpoint{1.429111in}{0.588002in}}%
\pgfpathlineto{\pgfqpoint{1.206739in}{0.588002in}}%
\pgfpathclose%
\pgfusepath{stroke,fill}%
\end{pgfscope}%
\begin{pgfscope}%
\pgfpathrectangle{\pgfqpoint{0.650810in}{0.557870in}}{\pgfqpoint{2.446088in}{1.684734in}}%
\pgfusepath{clip}%
\pgfsetbuttcap%
\pgfsetmiterjoin%
\definecolor{currentfill}{rgb}{0.298039,0.447059,0.690196}%
\pgfsetfillcolor{currentfill}%
\pgfsetfillopacity{0.400000}%
\pgfsetlinewidth{1.003750pt}%
\definecolor{currentstroke}{rgb}{1.000000,1.000000,1.000000}%
\pgfsetstrokecolor{currentstroke}%
\pgfsetstrokeopacity{0.400000}%
\pgfsetdash{}{0pt}%
\pgfpathmoveto{\pgfqpoint{1.429111in}{0.557870in}}%
\pgfpathlineto{\pgfqpoint{1.651483in}{0.557870in}}%
\pgfpathlineto{\pgfqpoint{1.651483in}{0.866719in}}%
\pgfpathlineto{\pgfqpoint{1.429111in}{0.866719in}}%
\pgfpathclose%
\pgfusepath{stroke,fill}%
\end{pgfscope}%
\begin{pgfscope}%
\pgfpathrectangle{\pgfqpoint{0.650810in}{0.557870in}}{\pgfqpoint{2.446088in}{1.684734in}}%
\pgfusepath{clip}%
\pgfsetbuttcap%
\pgfsetmiterjoin%
\definecolor{currentfill}{rgb}{0.298039,0.447059,0.690196}%
\pgfsetfillcolor{currentfill}%
\pgfsetfillopacity{0.400000}%
\pgfsetlinewidth{1.003750pt}%
\definecolor{currentstroke}{rgb}{1.000000,1.000000,1.000000}%
\pgfsetstrokecolor{currentstroke}%
\pgfsetstrokeopacity{0.400000}%
\pgfsetdash{}{0pt}%
\pgfpathmoveto{\pgfqpoint{1.651483in}{0.557870in}}%
\pgfpathlineto{\pgfqpoint{1.873854in}{0.557870in}}%
\pgfpathlineto{\pgfqpoint{1.873854in}{0.685930in}}%
\pgfpathlineto{\pgfqpoint{1.651483in}{0.685930in}}%
\pgfpathclose%
\pgfusepath{stroke,fill}%
\end{pgfscope}%
\begin{pgfscope}%
\pgfpathrectangle{\pgfqpoint{0.650810in}{0.557870in}}{\pgfqpoint{2.446088in}{1.684734in}}%
\pgfusepath{clip}%
\pgfsetbuttcap%
\pgfsetmiterjoin%
\definecolor{currentfill}{rgb}{0.298039,0.447059,0.690196}%
\pgfsetfillcolor{currentfill}%
\pgfsetfillopacity{0.400000}%
\pgfsetlinewidth{1.003750pt}%
\definecolor{currentstroke}{rgb}{1.000000,1.000000,1.000000}%
\pgfsetstrokecolor{currentstroke}%
\pgfsetstrokeopacity{0.400000}%
\pgfsetdash{}{0pt}%
\pgfpathmoveto{\pgfqpoint{1.873854in}{0.557870in}}%
\pgfpathlineto{\pgfqpoint{2.096226in}{0.557870in}}%
\pgfpathlineto{\pgfqpoint{2.096226in}{0.580469in}}%
\pgfpathlineto{\pgfqpoint{1.873854in}{0.580469in}}%
\pgfpathclose%
\pgfusepath{stroke,fill}%
\end{pgfscope}%
\begin{pgfscope}%
\pgfpathrectangle{\pgfqpoint{0.650810in}{0.557870in}}{\pgfqpoint{2.446088in}{1.684734in}}%
\pgfusepath{clip}%
\pgfsetbuttcap%
\pgfsetmiterjoin%
\definecolor{currentfill}{rgb}{0.298039,0.447059,0.690196}%
\pgfsetfillcolor{currentfill}%
\pgfsetfillopacity{0.400000}%
\pgfsetlinewidth{1.003750pt}%
\definecolor{currentstroke}{rgb}{1.000000,1.000000,1.000000}%
\pgfsetstrokecolor{currentstroke}%
\pgfsetstrokeopacity{0.400000}%
\pgfsetdash{}{0pt}%
\pgfpathmoveto{\pgfqpoint{2.096226in}{0.557870in}}%
\pgfpathlineto{\pgfqpoint{2.318598in}{0.557870in}}%
\pgfpathlineto{\pgfqpoint{2.318598in}{0.618134in}}%
\pgfpathlineto{\pgfqpoint{2.096226in}{0.618134in}}%
\pgfpathclose%
\pgfusepath{stroke,fill}%
\end{pgfscope}%
\begin{pgfscope}%
\pgfpathrectangle{\pgfqpoint{0.650810in}{0.557870in}}{\pgfqpoint{2.446088in}{1.684734in}}%
\pgfusepath{clip}%
\pgfsetbuttcap%
\pgfsetmiterjoin%
\definecolor{currentfill}{rgb}{0.298039,0.447059,0.690196}%
\pgfsetfillcolor{currentfill}%
\pgfsetfillopacity{0.400000}%
\pgfsetlinewidth{1.003750pt}%
\definecolor{currentstroke}{rgb}{1.000000,1.000000,1.000000}%
\pgfsetstrokecolor{currentstroke}%
\pgfsetstrokeopacity{0.400000}%
\pgfsetdash{}{0pt}%
\pgfpathmoveto{\pgfqpoint{2.318598in}{0.557870in}}%
\pgfpathlineto{\pgfqpoint{2.540969in}{0.557870in}}%
\pgfpathlineto{\pgfqpoint{2.540969in}{0.588002in}}%
\pgfpathlineto{\pgfqpoint{2.318598in}{0.588002in}}%
\pgfpathclose%
\pgfusepath{stroke,fill}%
\end{pgfscope}%
\begin{pgfscope}%
\pgfpathrectangle{\pgfqpoint{0.650810in}{0.557870in}}{\pgfqpoint{2.446088in}{1.684734in}}%
\pgfusepath{clip}%
\pgfsetbuttcap%
\pgfsetmiterjoin%
\definecolor{currentfill}{rgb}{0.298039,0.447059,0.690196}%
\pgfsetfillcolor{currentfill}%
\pgfsetfillopacity{0.400000}%
\pgfsetlinewidth{1.003750pt}%
\definecolor{currentstroke}{rgb}{1.000000,1.000000,1.000000}%
\pgfsetstrokecolor{currentstroke}%
\pgfsetstrokeopacity{0.400000}%
\pgfsetdash{}{0pt}%
\pgfpathmoveto{\pgfqpoint{2.540969in}{0.557870in}}%
\pgfpathlineto{\pgfqpoint{2.763341in}{0.557870in}}%
\pgfpathlineto{\pgfqpoint{2.763341in}{0.572936in}}%
\pgfpathlineto{\pgfqpoint{2.540969in}{0.572936in}}%
\pgfpathclose%
\pgfusepath{stroke,fill}%
\end{pgfscope}%
\begin{pgfscope}%
\pgfpathrectangle{\pgfqpoint{0.650810in}{0.557870in}}{\pgfqpoint{2.446088in}{1.684734in}}%
\pgfusepath{clip}%
\pgfsetbuttcap%
\pgfsetmiterjoin%
\definecolor{currentfill}{rgb}{0.298039,0.447059,0.690196}%
\pgfsetfillcolor{currentfill}%
\pgfsetfillopacity{0.400000}%
\pgfsetlinewidth{1.003750pt}%
\definecolor{currentstroke}{rgb}{1.000000,1.000000,1.000000}%
\pgfsetstrokecolor{currentstroke}%
\pgfsetstrokeopacity{0.400000}%
\pgfsetdash{}{0pt}%
\pgfpathmoveto{\pgfqpoint{2.763341in}{0.557870in}}%
\pgfpathlineto{\pgfqpoint{2.985712in}{0.557870in}}%
\pgfpathlineto{\pgfqpoint{2.985712in}{0.588002in}}%
\pgfpathlineto{\pgfqpoint{2.763341in}{0.588002in}}%
\pgfpathclose%
\pgfusepath{stroke,fill}%
\end{pgfscope}%
\begin{pgfscope}%
\pgfsetrectcap%
\pgfsetmiterjoin%
\pgfsetlinewidth{1.254687pt}%
\definecolor{currentstroke}{rgb}{1.000000,1.000000,1.000000}%
\pgfsetstrokecolor{currentstroke}%
\pgfsetdash{}{0pt}%
\pgfpathmoveto{\pgfqpoint{0.650810in}{0.557870in}}%
\pgfpathlineto{\pgfqpoint{0.650810in}{2.242604in}}%
\pgfusepath{stroke}%
\end{pgfscope}%
\begin{pgfscope}%
\pgfsetrectcap%
\pgfsetmiterjoin%
\pgfsetlinewidth{1.254687pt}%
\definecolor{currentstroke}{rgb}{1.000000,1.000000,1.000000}%
\pgfsetstrokecolor{currentstroke}%
\pgfsetdash{}{0pt}%
\pgfpathmoveto{\pgfqpoint{3.096898in}{0.557870in}}%
\pgfpathlineto{\pgfqpoint{3.096898in}{2.242604in}}%
\pgfusepath{stroke}%
\end{pgfscope}%
\begin{pgfscope}%
\pgfsetrectcap%
\pgfsetmiterjoin%
\pgfsetlinewidth{1.254687pt}%
\definecolor{currentstroke}{rgb}{1.000000,1.000000,1.000000}%
\pgfsetstrokecolor{currentstroke}%
\pgfsetdash{}{0pt}%
\pgfpathmoveto{\pgfqpoint{0.650810in}{0.557870in}}%
\pgfpathlineto{\pgfqpoint{3.096898in}{0.557870in}}%
\pgfusepath{stroke}%
\end{pgfscope}%
\begin{pgfscope}%
\pgfsetrectcap%
\pgfsetmiterjoin%
\pgfsetlinewidth{1.254687pt}%
\definecolor{currentstroke}{rgb}{1.000000,1.000000,1.000000}%
\pgfsetstrokecolor{currentstroke}%
\pgfsetdash{}{0pt}%
\pgfpathmoveto{\pgfqpoint{0.650810in}{2.242604in}}%
\pgfpathlineto{\pgfqpoint{3.096898in}{2.242604in}}%
\pgfusepath{stroke}%
\end{pgfscope}%
\begin{pgfscope}%
\definecolor{textcolor}{rgb}{0.150000,0.150000,0.150000}%
\pgfsetstrokecolor{textcolor}%
\pgfsetfillcolor{textcolor}%
\pgftext[x=1.873854in,y=2.325938in,,base]{\color{textcolor}\sffamily\fontsize{11.000000}{13.200000}\selectfont (a)}%
\end{pgfscope}%
\begin{pgfscope}%
\pgfsetbuttcap%
\pgfsetmiterjoin%
\definecolor{currentfill}{rgb}{0.917647,0.917647,0.949020}%
\pgfsetfillcolor{currentfill}%
\pgfsetlinewidth{0.000000pt}%
\definecolor{currentstroke}{rgb}{0.000000,0.000000,0.000000}%
\pgfsetstrokecolor{currentstroke}%
\pgfsetstrokeopacity{0.000000}%
\pgfsetdash{}{0pt}%
\pgfpathmoveto{\pgfqpoint{3.793912in}{0.557870in}}%
\pgfpathlineto{\pgfqpoint{6.240000in}{0.557870in}}%
\pgfpathlineto{\pgfqpoint{6.240000in}{2.242604in}}%
\pgfpathlineto{\pgfqpoint{3.793912in}{2.242604in}}%
\pgfpathclose%
\pgfusepath{fill}%
\end{pgfscope}%
\begin{pgfscope}%
\pgfpathrectangle{\pgfqpoint{3.793912in}{0.557870in}}{\pgfqpoint{2.446088in}{1.684734in}}%
\pgfusepath{clip}%
\pgfsetroundcap%
\pgfsetroundjoin%
\pgfsetlinewidth{1.003750pt}%
\definecolor{currentstroke}{rgb}{1.000000,1.000000,1.000000}%
\pgfsetstrokecolor{currentstroke}%
\pgfsetdash{}{0pt}%
\pgfpathmoveto{\pgfqpoint{3.905098in}{0.557870in}}%
\pgfpathlineto{\pgfqpoint{3.905098in}{2.242604in}}%
\pgfusepath{stroke}%
\end{pgfscope}%
\begin{pgfscope}%
\definecolor{textcolor}{rgb}{0.150000,0.150000,0.150000}%
\pgfsetstrokecolor{textcolor}%
\pgfsetfillcolor{textcolor}%
\pgftext[x=3.905098in,y=0.425926in,,top]{\color{textcolor}\sffamily\fontsize{11.000000}{13.200000}\selectfont \(\displaystyle 0.00\)}%
\end{pgfscope}%
\begin{pgfscope}%
\pgfpathrectangle{\pgfqpoint{3.793912in}{0.557870in}}{\pgfqpoint{2.446088in}{1.684734in}}%
\pgfusepath{clip}%
\pgfsetroundcap%
\pgfsetroundjoin%
\pgfsetlinewidth{1.003750pt}%
\definecolor{currentstroke}{rgb}{1.000000,1.000000,1.000000}%
\pgfsetstrokecolor{currentstroke}%
\pgfsetdash{}{0pt}%
\pgfpathmoveto{\pgfqpoint{4.461027in}{0.557870in}}%
\pgfpathlineto{\pgfqpoint{4.461027in}{2.242604in}}%
\pgfusepath{stroke}%
\end{pgfscope}%
\begin{pgfscope}%
\definecolor{textcolor}{rgb}{0.150000,0.150000,0.150000}%
\pgfsetstrokecolor{textcolor}%
\pgfsetfillcolor{textcolor}%
\pgftext[x=4.461027in,y=0.425926in,,top]{\color{textcolor}\sffamily\fontsize{11.000000}{13.200000}\selectfont \(\displaystyle 0.25\)}%
\end{pgfscope}%
\begin{pgfscope}%
\pgfpathrectangle{\pgfqpoint{3.793912in}{0.557870in}}{\pgfqpoint{2.446088in}{1.684734in}}%
\pgfusepath{clip}%
\pgfsetroundcap%
\pgfsetroundjoin%
\pgfsetlinewidth{1.003750pt}%
\definecolor{currentstroke}{rgb}{1.000000,1.000000,1.000000}%
\pgfsetstrokecolor{currentstroke}%
\pgfsetdash{}{0pt}%
\pgfpathmoveto{\pgfqpoint{5.016956in}{0.557870in}}%
\pgfpathlineto{\pgfqpoint{5.016956in}{2.242604in}}%
\pgfusepath{stroke}%
\end{pgfscope}%
\begin{pgfscope}%
\definecolor{textcolor}{rgb}{0.150000,0.150000,0.150000}%
\pgfsetstrokecolor{textcolor}%
\pgfsetfillcolor{textcolor}%
\pgftext[x=5.016956in,y=0.425926in,,top]{\color{textcolor}\sffamily\fontsize{11.000000}{13.200000}\selectfont \(\displaystyle 0.50\)}%
\end{pgfscope}%
\begin{pgfscope}%
\pgfpathrectangle{\pgfqpoint{3.793912in}{0.557870in}}{\pgfqpoint{2.446088in}{1.684734in}}%
\pgfusepath{clip}%
\pgfsetroundcap%
\pgfsetroundjoin%
\pgfsetlinewidth{1.003750pt}%
\definecolor{currentstroke}{rgb}{1.000000,1.000000,1.000000}%
\pgfsetstrokecolor{currentstroke}%
\pgfsetdash{}{0pt}%
\pgfpathmoveto{\pgfqpoint{5.572885in}{0.557870in}}%
\pgfpathlineto{\pgfqpoint{5.572885in}{2.242604in}}%
\pgfusepath{stroke}%
\end{pgfscope}%
\begin{pgfscope}%
\definecolor{textcolor}{rgb}{0.150000,0.150000,0.150000}%
\pgfsetstrokecolor{textcolor}%
\pgfsetfillcolor{textcolor}%
\pgftext[x=5.572885in,y=0.425926in,,top]{\color{textcolor}\sffamily\fontsize{11.000000}{13.200000}\selectfont \(\displaystyle 0.75\)}%
\end{pgfscope}%
\begin{pgfscope}%
\pgfpathrectangle{\pgfqpoint{3.793912in}{0.557870in}}{\pgfqpoint{2.446088in}{1.684734in}}%
\pgfusepath{clip}%
\pgfsetroundcap%
\pgfsetroundjoin%
\pgfsetlinewidth{1.003750pt}%
\definecolor{currentstroke}{rgb}{1.000000,1.000000,1.000000}%
\pgfsetstrokecolor{currentstroke}%
\pgfsetdash{}{0pt}%
\pgfpathmoveto{\pgfqpoint{6.128814in}{0.557870in}}%
\pgfpathlineto{\pgfqpoint{6.128814in}{2.242604in}}%
\pgfusepath{stroke}%
\end{pgfscope}%
\begin{pgfscope}%
\definecolor{textcolor}{rgb}{0.150000,0.150000,0.150000}%
\pgfsetstrokecolor{textcolor}%
\pgfsetfillcolor{textcolor}%
\pgftext[x=6.128814in,y=0.425926in,,top]{\color{textcolor}\sffamily\fontsize{11.000000}{13.200000}\selectfont \(\displaystyle 1.00\)}%
\end{pgfscope}%
\begin{pgfscope}%
\definecolor{textcolor}{rgb}{0.150000,0.150000,0.150000}%
\pgfsetstrokecolor{textcolor}%
\pgfsetfillcolor{textcolor}%
\pgftext[x=5.016956in,y=0.235185in,,top]{\color{textcolor}\sffamily\fontsize{11.000000}{13.200000}\selectfont Specificity}%
\end{pgfscope}%
\begin{pgfscope}%
\pgfpathrectangle{\pgfqpoint{3.793912in}{0.557870in}}{\pgfqpoint{2.446088in}{1.684734in}}%
\pgfusepath{clip}%
\pgfsetroundcap%
\pgfsetroundjoin%
\pgfsetlinewidth{1.003750pt}%
\definecolor{currentstroke}{rgb}{1.000000,1.000000,1.000000}%
\pgfsetstrokecolor{currentstroke}%
\pgfsetdash{}{0pt}%
\pgfpathmoveto{\pgfqpoint{3.793912in}{0.634449in}}%
\pgfpathlineto{\pgfqpoint{6.240000in}{0.634449in}}%
\pgfusepath{stroke}%
\end{pgfscope}%
\begin{pgfscope}%
\definecolor{textcolor}{rgb}{0.150000,0.150000,0.150000}%
\pgfsetstrokecolor{textcolor}%
\pgfsetfillcolor{textcolor}%
\pgftext[x=3.391597in,y=0.581642in,left,base]{\color{textcolor}\sffamily\fontsize{11.000000}{13.200000}\selectfont \(\displaystyle 0.00\)}%
\end{pgfscope}%
\begin{pgfscope}%
\pgfpathrectangle{\pgfqpoint{3.793912in}{0.557870in}}{\pgfqpoint{2.446088in}{1.684734in}}%
\pgfusepath{clip}%
\pgfsetroundcap%
\pgfsetroundjoin%
\pgfsetlinewidth{1.003750pt}%
\definecolor{currentstroke}{rgb}{1.000000,1.000000,1.000000}%
\pgfsetstrokecolor{currentstroke}%
\pgfsetdash{}{0pt}%
\pgfpathmoveto{\pgfqpoint{3.793912in}{1.017343in}}%
\pgfpathlineto{\pgfqpoint{6.240000in}{1.017343in}}%
\pgfusepath{stroke}%
\end{pgfscope}%
\begin{pgfscope}%
\definecolor{textcolor}{rgb}{0.150000,0.150000,0.150000}%
\pgfsetstrokecolor{textcolor}%
\pgfsetfillcolor{textcolor}%
\pgftext[x=3.391597in,y=0.964536in,left,base]{\color{textcolor}\sffamily\fontsize{11.000000}{13.200000}\selectfont \(\displaystyle 0.25\)}%
\end{pgfscope}%
\begin{pgfscope}%
\pgfpathrectangle{\pgfqpoint{3.793912in}{0.557870in}}{\pgfqpoint{2.446088in}{1.684734in}}%
\pgfusepath{clip}%
\pgfsetroundcap%
\pgfsetroundjoin%
\pgfsetlinewidth{1.003750pt}%
\definecolor{currentstroke}{rgb}{1.000000,1.000000,1.000000}%
\pgfsetstrokecolor{currentstroke}%
\pgfsetdash{}{0pt}%
\pgfpathmoveto{\pgfqpoint{3.793912in}{1.400237in}}%
\pgfpathlineto{\pgfqpoint{6.240000in}{1.400237in}}%
\pgfusepath{stroke}%
\end{pgfscope}%
\begin{pgfscope}%
\definecolor{textcolor}{rgb}{0.150000,0.150000,0.150000}%
\pgfsetstrokecolor{textcolor}%
\pgfsetfillcolor{textcolor}%
\pgftext[x=3.391597in,y=1.347431in,left,base]{\color{textcolor}\sffamily\fontsize{11.000000}{13.200000}\selectfont \(\displaystyle 0.50\)}%
\end{pgfscope}%
\begin{pgfscope}%
\pgfpathrectangle{\pgfqpoint{3.793912in}{0.557870in}}{\pgfqpoint{2.446088in}{1.684734in}}%
\pgfusepath{clip}%
\pgfsetroundcap%
\pgfsetroundjoin%
\pgfsetlinewidth{1.003750pt}%
\definecolor{currentstroke}{rgb}{1.000000,1.000000,1.000000}%
\pgfsetstrokecolor{currentstroke}%
\pgfsetdash{}{0pt}%
\pgfpathmoveto{\pgfqpoint{3.793912in}{1.783131in}}%
\pgfpathlineto{\pgfqpoint{6.240000in}{1.783131in}}%
\pgfusepath{stroke}%
\end{pgfscope}%
\begin{pgfscope}%
\definecolor{textcolor}{rgb}{0.150000,0.150000,0.150000}%
\pgfsetstrokecolor{textcolor}%
\pgfsetfillcolor{textcolor}%
\pgftext[x=3.391597in,y=1.730325in,left,base]{\color{textcolor}\sffamily\fontsize{11.000000}{13.200000}\selectfont \(\displaystyle 0.75\)}%
\end{pgfscope}%
\begin{pgfscope}%
\pgfpathrectangle{\pgfqpoint{3.793912in}{0.557870in}}{\pgfqpoint{2.446088in}{1.684734in}}%
\pgfusepath{clip}%
\pgfsetroundcap%
\pgfsetroundjoin%
\pgfsetlinewidth{1.003750pt}%
\definecolor{currentstroke}{rgb}{1.000000,1.000000,1.000000}%
\pgfsetstrokecolor{currentstroke}%
\pgfsetdash{}{0pt}%
\pgfpathmoveto{\pgfqpoint{3.793912in}{2.166025in}}%
\pgfpathlineto{\pgfqpoint{6.240000in}{2.166025in}}%
\pgfusepath{stroke}%
\end{pgfscope}%
\begin{pgfscope}%
\definecolor{textcolor}{rgb}{0.150000,0.150000,0.150000}%
\pgfsetstrokecolor{textcolor}%
\pgfsetfillcolor{textcolor}%
\pgftext[x=3.391597in,y=2.113219in,left,base]{\color{textcolor}\sffamily\fontsize{11.000000}{13.200000}\selectfont \(\displaystyle 1.00\)}%
\end{pgfscope}%
\begin{pgfscope}%
\definecolor{textcolor}{rgb}{0.150000,0.150000,0.150000}%
\pgfsetstrokecolor{textcolor}%
\pgfsetfillcolor{textcolor}%
\pgftext[x=3.336042in,y=1.400237in,,bottom,rotate=90.000000]{\color{textcolor}\sffamily\fontsize{11.000000}{13.200000}\selectfont Sensitivity}%
\end{pgfscope}%
\begin{pgfscope}%
\pgfpathrectangle{\pgfqpoint{3.793912in}{0.557870in}}{\pgfqpoint{2.446088in}{1.684734in}}%
\pgfusepath{clip}%
\pgfsetbuttcap%
\pgfsetroundjoin%
\definecolor{currentfill}{rgb}{0.298039,0.447059,0.690196}%
\pgfsetfillcolor{currentfill}%
\pgfsetlinewidth{1.003750pt}%
\definecolor{currentstroke}{rgb}{0.298039,0.447059,0.690196}%
\pgfsetstrokecolor{currentstroke}%
\pgfsetdash{}{0pt}%
\pgfpathmoveto{\pgfqpoint{3.905098in}{2.125798in}}%
\pgfpathcurveto{\pgfqpoint{3.913334in}{2.125798in}}{\pgfqpoint{3.921234in}{2.129070in}}{\pgfqpoint{3.927058in}{2.134894in}}%
\pgfpathcurveto{\pgfqpoint{3.932882in}{2.140718in}}{\pgfqpoint{3.936155in}{2.148618in}}{\pgfqpoint{3.936155in}{2.156854in}}%
\pgfpathcurveto{\pgfqpoint{3.936155in}{2.165091in}}{\pgfqpoint{3.932882in}{2.172991in}}{\pgfqpoint{3.927058in}{2.178814in}}%
\pgfpathcurveto{\pgfqpoint{3.921234in}{2.184638in}}{\pgfqpoint{3.913334in}{2.187911in}}{\pgfqpoint{3.905098in}{2.187911in}}%
\pgfpathcurveto{\pgfqpoint{3.896862in}{2.187911in}}{\pgfqpoint{3.888962in}{2.184638in}}{\pgfqpoint{3.883138in}{2.178814in}}%
\pgfpathcurveto{\pgfqpoint{3.877314in}{2.172991in}}{\pgfqpoint{3.874042in}{2.165091in}}{\pgfqpoint{3.874042in}{2.156854in}}%
\pgfpathcurveto{\pgfqpoint{3.874042in}{2.148618in}}{\pgfqpoint{3.877314in}{2.140718in}}{\pgfqpoint{3.883138in}{2.134894in}}%
\pgfpathcurveto{\pgfqpoint{3.888962in}{2.129070in}}{\pgfqpoint{3.896862in}{2.125798in}}{\pgfqpoint{3.905098in}{2.125798in}}%
\pgfpathclose%
\pgfusepath{stroke,fill}%
\end{pgfscope}%
\begin{pgfscope}%
\pgfpathrectangle{\pgfqpoint{3.793912in}{0.557870in}}{\pgfqpoint{2.446088in}{1.684734in}}%
\pgfusepath{clip}%
\pgfsetbuttcap%
\pgfsetroundjoin%
\definecolor{currentfill}{rgb}{0.298039,0.447059,0.690196}%
\pgfsetfillcolor{currentfill}%
\pgfsetlinewidth{1.003750pt}%
\definecolor{currentstroke}{rgb}{0.298039,0.447059,0.690196}%
\pgfsetstrokecolor{currentstroke}%
\pgfsetdash{}{0pt}%
\pgfpathmoveto{\pgfqpoint{5.975454in}{1.364595in}}%
\pgfpathcurveto{\pgfqpoint{5.983691in}{1.364595in}}{\pgfqpoint{5.991591in}{1.367867in}}{\pgfqpoint{5.997415in}{1.373691in}}%
\pgfpathcurveto{\pgfqpoint{6.003239in}{1.379515in}}{\pgfqpoint{6.006511in}{1.387415in}}{\pgfqpoint{6.006511in}{1.395652in}}%
\pgfpathcurveto{\pgfqpoint{6.006511in}{1.403888in}}{\pgfqpoint{6.003239in}{1.411788in}}{\pgfqpoint{5.997415in}{1.417612in}}%
\pgfpathcurveto{\pgfqpoint{5.991591in}{1.423436in}}{\pgfqpoint{5.983691in}{1.426708in}}{\pgfqpoint{5.975454in}{1.426708in}}%
\pgfpathcurveto{\pgfqpoint{5.967218in}{1.426708in}}{\pgfqpoint{5.959318in}{1.423436in}}{\pgfqpoint{5.953494in}{1.417612in}}%
\pgfpathcurveto{\pgfqpoint{5.947670in}{1.411788in}}{\pgfqpoint{5.944398in}{1.403888in}}{\pgfqpoint{5.944398in}{1.395652in}}%
\pgfpathcurveto{\pgfqpoint{5.944398in}{1.387415in}}{\pgfqpoint{5.947670in}{1.379515in}}{\pgfqpoint{5.953494in}{1.373691in}}%
\pgfpathcurveto{\pgfqpoint{5.959318in}{1.367867in}}{\pgfqpoint{5.967218in}{1.364595in}}{\pgfqpoint{5.975454in}{1.364595in}}%
\pgfpathclose%
\pgfusepath{stroke,fill}%
\end{pgfscope}%
\begin{pgfscope}%
\pgfpathrectangle{\pgfqpoint{3.793912in}{0.557870in}}{\pgfqpoint{2.446088in}{1.684734in}}%
\pgfusepath{clip}%
\pgfsetbuttcap%
\pgfsetroundjoin%
\definecolor{currentfill}{rgb}{0.298039,0.447059,0.690196}%
\pgfsetfillcolor{currentfill}%
\pgfsetlinewidth{1.003750pt}%
\definecolor{currentstroke}{rgb}{0.298039,0.447059,0.690196}%
\pgfsetstrokecolor{currentstroke}%
\pgfsetdash{}{0pt}%
\pgfpathmoveto{\pgfqpoint{5.975454in}{1.364595in}}%
\pgfpathcurveto{\pgfqpoint{5.983691in}{1.364595in}}{\pgfqpoint{5.991591in}{1.367867in}}{\pgfqpoint{5.997415in}{1.373691in}}%
\pgfpathcurveto{\pgfqpoint{6.003239in}{1.379515in}}{\pgfqpoint{6.006511in}{1.387415in}}{\pgfqpoint{6.006511in}{1.395652in}}%
\pgfpathcurveto{\pgfqpoint{6.006511in}{1.403888in}}{\pgfqpoint{6.003239in}{1.411788in}}{\pgfqpoint{5.997415in}{1.417612in}}%
\pgfpathcurveto{\pgfqpoint{5.991591in}{1.423436in}}{\pgfqpoint{5.983691in}{1.426708in}}{\pgfqpoint{5.975454in}{1.426708in}}%
\pgfpathcurveto{\pgfqpoint{5.967218in}{1.426708in}}{\pgfqpoint{5.959318in}{1.423436in}}{\pgfqpoint{5.953494in}{1.417612in}}%
\pgfpathcurveto{\pgfqpoint{5.947670in}{1.411788in}}{\pgfqpoint{5.944398in}{1.403888in}}{\pgfqpoint{5.944398in}{1.395652in}}%
\pgfpathcurveto{\pgfqpoint{5.944398in}{1.387415in}}{\pgfqpoint{5.947670in}{1.379515in}}{\pgfqpoint{5.953494in}{1.373691in}}%
\pgfpathcurveto{\pgfqpoint{5.959318in}{1.367867in}}{\pgfqpoint{5.967218in}{1.364595in}}{\pgfqpoint{5.975454in}{1.364595in}}%
\pgfpathclose%
\pgfusepath{stroke,fill}%
\end{pgfscope}%
\begin{pgfscope}%
\pgfpathrectangle{\pgfqpoint{3.793912in}{0.557870in}}{\pgfqpoint{2.446088in}{1.684734in}}%
\pgfusepath{clip}%
\pgfsetbuttcap%
\pgfsetroundjoin%
\definecolor{currentfill}{rgb}{0.298039,0.447059,0.690196}%
\pgfsetfillcolor{currentfill}%
\pgfsetlinewidth{1.003750pt}%
\definecolor{currentstroke}{rgb}{0.298039,0.447059,0.690196}%
\pgfsetstrokecolor{currentstroke}%
\pgfsetdash{}{0pt}%
\pgfpathmoveto{\pgfqpoint{5.975454in}{1.437964in}}%
\pgfpathcurveto{\pgfqpoint{5.983691in}{1.437964in}}{\pgfqpoint{5.991591in}{1.441236in}}{\pgfqpoint{5.997415in}{1.447060in}}%
\pgfpathcurveto{\pgfqpoint{6.003239in}{1.452884in}}{\pgfqpoint{6.006511in}{1.460784in}}{\pgfqpoint{6.006511in}{1.469021in}}%
\pgfpathcurveto{\pgfqpoint{6.006511in}{1.477257in}}{\pgfqpoint{6.003239in}{1.485157in}}{\pgfqpoint{5.997415in}{1.490981in}}%
\pgfpathcurveto{\pgfqpoint{5.991591in}{1.496805in}}{\pgfqpoint{5.983691in}{1.500077in}}{\pgfqpoint{5.975454in}{1.500077in}}%
\pgfpathcurveto{\pgfqpoint{5.967218in}{1.500077in}}{\pgfqpoint{5.959318in}{1.496805in}}{\pgfqpoint{5.953494in}{1.490981in}}%
\pgfpathcurveto{\pgfqpoint{5.947670in}{1.485157in}}{\pgfqpoint{5.944398in}{1.477257in}}{\pgfqpoint{5.944398in}{1.469021in}}%
\pgfpathcurveto{\pgfqpoint{5.944398in}{1.460784in}}{\pgfqpoint{5.947670in}{1.452884in}}{\pgfqpoint{5.953494in}{1.447060in}}%
\pgfpathcurveto{\pgfqpoint{5.959318in}{1.441236in}}{\pgfqpoint{5.967218in}{1.437964in}}{\pgfqpoint{5.975454in}{1.437964in}}%
\pgfpathclose%
\pgfusepath{stroke,fill}%
\end{pgfscope}%
\begin{pgfscope}%
\pgfpathrectangle{\pgfqpoint{3.793912in}{0.557870in}}{\pgfqpoint{2.446088in}{1.684734in}}%
\pgfusepath{clip}%
\pgfsetbuttcap%
\pgfsetroundjoin%
\definecolor{currentfill}{rgb}{0.298039,0.447059,0.690196}%
\pgfsetfillcolor{currentfill}%
\pgfsetlinewidth{1.003750pt}%
\definecolor{currentstroke}{rgb}{0.298039,0.447059,0.690196}%
\pgfsetstrokecolor{currentstroke}%
\pgfsetdash{}{0pt}%
\pgfpathmoveto{\pgfqpoint{5.975454in}{1.538846in}}%
\pgfpathcurveto{\pgfqpoint{5.983691in}{1.538846in}}{\pgfqpoint{5.991591in}{1.542119in}}{\pgfqpoint{5.997415in}{1.547943in}}%
\pgfpathcurveto{\pgfqpoint{6.003239in}{1.553767in}}{\pgfqpoint{6.006511in}{1.561667in}}{\pgfqpoint{6.006511in}{1.569903in}}%
\pgfpathcurveto{\pgfqpoint{6.006511in}{1.578139in}}{\pgfqpoint{6.003239in}{1.586039in}}{\pgfqpoint{5.997415in}{1.591863in}}%
\pgfpathcurveto{\pgfqpoint{5.991591in}{1.597687in}}{\pgfqpoint{5.983691in}{1.600959in}}{\pgfqpoint{5.975454in}{1.600959in}}%
\pgfpathcurveto{\pgfqpoint{5.967218in}{1.600959in}}{\pgfqpoint{5.959318in}{1.597687in}}{\pgfqpoint{5.953494in}{1.591863in}}%
\pgfpathcurveto{\pgfqpoint{5.947670in}{1.586039in}}{\pgfqpoint{5.944398in}{1.578139in}}{\pgfqpoint{5.944398in}{1.569903in}}%
\pgfpathcurveto{\pgfqpoint{5.944398in}{1.561667in}}{\pgfqpoint{5.947670in}{1.553767in}}{\pgfqpoint{5.953494in}{1.547943in}}%
\pgfpathcurveto{\pgfqpoint{5.959318in}{1.542119in}}{\pgfqpoint{5.967218in}{1.538846in}}{\pgfqpoint{5.975454in}{1.538846in}}%
\pgfpathclose%
\pgfusepath{stroke,fill}%
\end{pgfscope}%
\begin{pgfscope}%
\pgfpathrectangle{\pgfqpoint{3.793912in}{0.557870in}}{\pgfqpoint{2.446088in}{1.684734in}}%
\pgfusepath{clip}%
\pgfsetbuttcap%
\pgfsetroundjoin%
\definecolor{currentfill}{rgb}{0.298039,0.447059,0.690196}%
\pgfsetfillcolor{currentfill}%
\pgfsetlinewidth{1.003750pt}%
\definecolor{currentstroke}{rgb}{0.298039,0.447059,0.690196}%
\pgfsetstrokecolor{currentstroke}%
\pgfsetdash{}{0pt}%
\pgfpathmoveto{\pgfqpoint{4.058458in}{1.924033in}}%
\pgfpathcurveto{\pgfqpoint{4.066694in}{1.924033in}}{\pgfqpoint{4.074594in}{1.927306in}}{\pgfqpoint{4.080418in}{1.933129in}}%
\pgfpathcurveto{\pgfqpoint{4.086242in}{1.938953in}}{\pgfqpoint{4.089514in}{1.946853in}}{\pgfqpoint{4.089514in}{1.955090in}}%
\pgfpathcurveto{\pgfqpoint{4.089514in}{1.963326in}}{\pgfqpoint{4.086242in}{1.971226in}}{\pgfqpoint{4.080418in}{1.977050in}}%
\pgfpathcurveto{\pgfqpoint{4.074594in}{1.982874in}}{\pgfqpoint{4.066694in}{1.986146in}}{\pgfqpoint{4.058458in}{1.986146in}}%
\pgfpathcurveto{\pgfqpoint{4.050221in}{1.986146in}}{\pgfqpoint{4.042321in}{1.982874in}}{\pgfqpoint{4.036498in}{1.977050in}}%
\pgfpathcurveto{\pgfqpoint{4.030674in}{1.971226in}}{\pgfqpoint{4.027401in}{1.963326in}}{\pgfqpoint{4.027401in}{1.955090in}}%
\pgfpathcurveto{\pgfqpoint{4.027401in}{1.946853in}}{\pgfqpoint{4.030674in}{1.938953in}}{\pgfqpoint{4.036498in}{1.933129in}}%
\pgfpathcurveto{\pgfqpoint{4.042321in}{1.927306in}}{\pgfqpoint{4.050221in}{1.924033in}}{\pgfqpoint{4.058458in}{1.924033in}}%
\pgfpathclose%
\pgfusepath{stroke,fill}%
\end{pgfscope}%
\begin{pgfscope}%
\pgfpathrectangle{\pgfqpoint{3.793912in}{0.557870in}}{\pgfqpoint{2.446088in}{1.684734in}}%
\pgfusepath{clip}%
\pgfsetbuttcap%
\pgfsetroundjoin%
\definecolor{currentfill}{rgb}{0.298039,0.447059,0.690196}%
\pgfsetfillcolor{currentfill}%
\pgfsetlinewidth{1.003750pt}%
\definecolor{currentstroke}{rgb}{0.298039,0.447059,0.690196}%
\pgfsetstrokecolor{currentstroke}%
\pgfsetdash{}{0pt}%
\pgfpathmoveto{\pgfqpoint{5.745415in}{1.850664in}}%
\pgfpathcurveto{\pgfqpoint{5.753651in}{1.850664in}}{\pgfqpoint{5.761551in}{1.853937in}}{\pgfqpoint{5.767375in}{1.859761in}}%
\pgfpathcurveto{\pgfqpoint{5.773199in}{1.865584in}}{\pgfqpoint{5.776471in}{1.873484in}}{\pgfqpoint{5.776471in}{1.881721in}}%
\pgfpathcurveto{\pgfqpoint{5.776471in}{1.889957in}}{\pgfqpoint{5.773199in}{1.897857in}}{\pgfqpoint{5.767375in}{1.903681in}}%
\pgfpathcurveto{\pgfqpoint{5.761551in}{1.909505in}}{\pgfqpoint{5.753651in}{1.912777in}}{\pgfqpoint{5.745415in}{1.912777in}}%
\pgfpathcurveto{\pgfqpoint{5.737179in}{1.912777in}}{\pgfqpoint{5.729279in}{1.909505in}}{\pgfqpoint{5.723455in}{1.903681in}}%
\pgfpathcurveto{\pgfqpoint{5.717631in}{1.897857in}}{\pgfqpoint{5.714358in}{1.889957in}}{\pgfqpoint{5.714358in}{1.881721in}}%
\pgfpathcurveto{\pgfqpoint{5.714358in}{1.873484in}}{\pgfqpoint{5.717631in}{1.865584in}}{\pgfqpoint{5.723455in}{1.859761in}}%
\pgfpathcurveto{\pgfqpoint{5.729279in}{1.853937in}}{\pgfqpoint{5.737179in}{1.850664in}}{\pgfqpoint{5.745415in}{1.850664in}}%
\pgfpathclose%
\pgfusepath{stroke,fill}%
\end{pgfscope}%
\begin{pgfscope}%
\pgfpathrectangle{\pgfqpoint{3.793912in}{0.557870in}}{\pgfqpoint{2.446088in}{1.684734in}}%
\pgfusepath{clip}%
\pgfsetbuttcap%
\pgfsetroundjoin%
\definecolor{currentfill}{rgb}{0.298039,0.447059,0.690196}%
\pgfsetfillcolor{currentfill}%
\pgfsetlinewidth{1.003750pt}%
\definecolor{currentstroke}{rgb}{0.298039,0.447059,0.690196}%
\pgfsetstrokecolor{currentstroke}%
\pgfsetdash{}{0pt}%
\pgfpathmoveto{\pgfqpoint{3.905098in}{2.125798in}}%
\pgfpathcurveto{\pgfqpoint{3.913334in}{2.125798in}}{\pgfqpoint{3.921234in}{2.129070in}}{\pgfqpoint{3.927058in}{2.134894in}}%
\pgfpathcurveto{\pgfqpoint{3.932882in}{2.140718in}}{\pgfqpoint{3.936155in}{2.148618in}}{\pgfqpoint{3.936155in}{2.156854in}}%
\pgfpathcurveto{\pgfqpoint{3.936155in}{2.165091in}}{\pgfqpoint{3.932882in}{2.172991in}}{\pgfqpoint{3.927058in}{2.178814in}}%
\pgfpathcurveto{\pgfqpoint{3.921234in}{2.184638in}}{\pgfqpoint{3.913334in}{2.187911in}}{\pgfqpoint{3.905098in}{2.187911in}}%
\pgfpathcurveto{\pgfqpoint{3.896862in}{2.187911in}}{\pgfqpoint{3.888962in}{2.184638in}}{\pgfqpoint{3.883138in}{2.178814in}}%
\pgfpathcurveto{\pgfqpoint{3.877314in}{2.172991in}}{\pgfqpoint{3.874042in}{2.165091in}}{\pgfqpoint{3.874042in}{2.156854in}}%
\pgfpathcurveto{\pgfqpoint{3.874042in}{2.148618in}}{\pgfqpoint{3.877314in}{2.140718in}}{\pgfqpoint{3.883138in}{2.134894in}}%
\pgfpathcurveto{\pgfqpoint{3.888962in}{2.129070in}}{\pgfqpoint{3.896862in}{2.125798in}}{\pgfqpoint{3.905098in}{2.125798in}}%
\pgfpathclose%
\pgfusepath{stroke,fill}%
\end{pgfscope}%
\begin{pgfscope}%
\pgfpathrectangle{\pgfqpoint{3.793912in}{0.557870in}}{\pgfqpoint{2.446088in}{1.684734in}}%
\pgfusepath{clip}%
\pgfsetbuttcap%
\pgfsetroundjoin%
\definecolor{currentfill}{rgb}{0.298039,0.447059,0.690196}%
\pgfsetfillcolor{currentfill}%
\pgfsetlinewidth{1.003750pt}%
\definecolor{currentstroke}{rgb}{0.298039,0.447059,0.690196}%
\pgfsetstrokecolor{currentstroke}%
\pgfsetdash{}{0pt}%
\pgfpathmoveto{\pgfqpoint{4.441857in}{1.924033in}}%
\pgfpathcurveto{\pgfqpoint{4.450093in}{1.924033in}}{\pgfqpoint{4.457993in}{1.927306in}}{\pgfqpoint{4.463817in}{1.933129in}}%
\pgfpathcurveto{\pgfqpoint{4.469641in}{1.938953in}}{\pgfqpoint{4.472914in}{1.946853in}}{\pgfqpoint{4.472914in}{1.955090in}}%
\pgfpathcurveto{\pgfqpoint{4.472914in}{1.963326in}}{\pgfqpoint{4.469641in}{1.971226in}}{\pgfqpoint{4.463817in}{1.977050in}}%
\pgfpathcurveto{\pgfqpoint{4.457993in}{1.982874in}}{\pgfqpoint{4.450093in}{1.986146in}}{\pgfqpoint{4.441857in}{1.986146in}}%
\pgfpathcurveto{\pgfqpoint{4.433621in}{1.986146in}}{\pgfqpoint{4.425721in}{1.982874in}}{\pgfqpoint{4.419897in}{1.977050in}}%
\pgfpathcurveto{\pgfqpoint{4.414073in}{1.971226in}}{\pgfqpoint{4.410801in}{1.963326in}}{\pgfqpoint{4.410801in}{1.955090in}}%
\pgfpathcurveto{\pgfqpoint{4.410801in}{1.946853in}}{\pgfqpoint{4.414073in}{1.938953in}}{\pgfqpoint{4.419897in}{1.933129in}}%
\pgfpathcurveto{\pgfqpoint{4.425721in}{1.927306in}}{\pgfqpoint{4.433621in}{1.924033in}}{\pgfqpoint{4.441857in}{1.924033in}}%
\pgfpathclose%
\pgfusepath{stroke,fill}%
\end{pgfscope}%
\begin{pgfscope}%
\pgfpathrectangle{\pgfqpoint{3.793912in}{0.557870in}}{\pgfqpoint{2.446088in}{1.684734in}}%
\pgfusepath{clip}%
\pgfsetbuttcap%
\pgfsetroundjoin%
\definecolor{currentfill}{rgb}{0.298039,0.447059,0.690196}%
\pgfsetfillcolor{currentfill}%
\pgfsetlinewidth{1.003750pt}%
\definecolor{currentstroke}{rgb}{0.298039,0.447059,0.690196}%
\pgfsetstrokecolor{currentstroke}%
\pgfsetdash{}{0pt}%
\pgfpathmoveto{\pgfqpoint{3.905098in}{2.125798in}}%
\pgfpathcurveto{\pgfqpoint{3.913334in}{2.125798in}}{\pgfqpoint{3.921234in}{2.129070in}}{\pgfqpoint{3.927058in}{2.134894in}}%
\pgfpathcurveto{\pgfqpoint{3.932882in}{2.140718in}}{\pgfqpoint{3.936155in}{2.148618in}}{\pgfqpoint{3.936155in}{2.156854in}}%
\pgfpathcurveto{\pgfqpoint{3.936155in}{2.165091in}}{\pgfqpoint{3.932882in}{2.172991in}}{\pgfqpoint{3.927058in}{2.178814in}}%
\pgfpathcurveto{\pgfqpoint{3.921234in}{2.184638in}}{\pgfqpoint{3.913334in}{2.187911in}}{\pgfqpoint{3.905098in}{2.187911in}}%
\pgfpathcurveto{\pgfqpoint{3.896862in}{2.187911in}}{\pgfqpoint{3.888962in}{2.184638in}}{\pgfqpoint{3.883138in}{2.178814in}}%
\pgfpathcurveto{\pgfqpoint{3.877314in}{2.172991in}}{\pgfqpoint{3.874042in}{2.165091in}}{\pgfqpoint{3.874042in}{2.156854in}}%
\pgfpathcurveto{\pgfqpoint{3.874042in}{2.148618in}}{\pgfqpoint{3.877314in}{2.140718in}}{\pgfqpoint{3.883138in}{2.134894in}}%
\pgfpathcurveto{\pgfqpoint{3.888962in}{2.129070in}}{\pgfqpoint{3.896862in}{2.125798in}}{\pgfqpoint{3.905098in}{2.125798in}}%
\pgfpathclose%
\pgfusepath{stroke,fill}%
\end{pgfscope}%
\begin{pgfscope}%
\pgfpathrectangle{\pgfqpoint{3.793912in}{0.557870in}}{\pgfqpoint{2.446088in}{1.684734in}}%
\pgfusepath{clip}%
\pgfsetbuttcap%
\pgfsetroundjoin%
\definecolor{currentfill}{rgb}{0.298039,0.447059,0.690196}%
\pgfsetfillcolor{currentfill}%
\pgfsetlinewidth{1.003750pt}%
\definecolor{currentstroke}{rgb}{0.298039,0.447059,0.690196}%
\pgfsetstrokecolor{currentstroke}%
\pgfsetdash{}{0pt}%
\pgfpathmoveto{\pgfqpoint{4.978616in}{1.648900in}}%
\pgfpathcurveto{\pgfqpoint{4.986852in}{1.648900in}}{\pgfqpoint{4.994753in}{1.652172in}}{\pgfqpoint{5.000576in}{1.657996in}}%
\pgfpathcurveto{\pgfqpoint{5.006400in}{1.663820in}}{\pgfqpoint{5.009673in}{1.671720in}}{\pgfqpoint{5.009673in}{1.679956in}}%
\pgfpathcurveto{\pgfqpoint{5.009673in}{1.688193in}}{\pgfqpoint{5.006400in}{1.696093in}}{\pgfqpoint{5.000576in}{1.701917in}}%
\pgfpathcurveto{\pgfqpoint{4.994753in}{1.707740in}}{\pgfqpoint{4.986852in}{1.711013in}}{\pgfqpoint{4.978616in}{1.711013in}}%
\pgfpathcurveto{\pgfqpoint{4.970380in}{1.711013in}}{\pgfqpoint{4.962480in}{1.707740in}}{\pgfqpoint{4.956656in}{1.701917in}}%
\pgfpathcurveto{\pgfqpoint{4.950832in}{1.696093in}}{\pgfqpoint{4.947560in}{1.688193in}}{\pgfqpoint{4.947560in}{1.679956in}}%
\pgfpathcurveto{\pgfqpoint{4.947560in}{1.671720in}}{\pgfqpoint{4.950832in}{1.663820in}}{\pgfqpoint{4.956656in}{1.657996in}}%
\pgfpathcurveto{\pgfqpoint{4.962480in}{1.652172in}}{\pgfqpoint{4.970380in}{1.648900in}}{\pgfqpoint{4.978616in}{1.648900in}}%
\pgfpathclose%
\pgfusepath{stroke,fill}%
\end{pgfscope}%
\begin{pgfscope}%
\pgfpathrectangle{\pgfqpoint{3.793912in}{0.557870in}}{\pgfqpoint{2.446088in}{1.684734in}}%
\pgfusepath{clip}%
\pgfsetbuttcap%
\pgfsetroundjoin%
\definecolor{currentfill}{rgb}{0.298039,0.447059,0.690196}%
\pgfsetfillcolor{currentfill}%
\pgfsetlinewidth{1.003750pt}%
\definecolor{currentstroke}{rgb}{0.298039,0.447059,0.690196}%
\pgfsetstrokecolor{currentstroke}%
\pgfsetdash{}{0pt}%
\pgfpathmoveto{\pgfqpoint{3.905098in}{2.125798in}}%
\pgfpathcurveto{\pgfqpoint{3.913334in}{2.125798in}}{\pgfqpoint{3.921234in}{2.129070in}}{\pgfqpoint{3.927058in}{2.134894in}}%
\pgfpathcurveto{\pgfqpoint{3.932882in}{2.140718in}}{\pgfqpoint{3.936155in}{2.148618in}}{\pgfqpoint{3.936155in}{2.156854in}}%
\pgfpathcurveto{\pgfqpoint{3.936155in}{2.165091in}}{\pgfqpoint{3.932882in}{2.172991in}}{\pgfqpoint{3.927058in}{2.178814in}}%
\pgfpathcurveto{\pgfqpoint{3.921234in}{2.184638in}}{\pgfqpoint{3.913334in}{2.187911in}}{\pgfqpoint{3.905098in}{2.187911in}}%
\pgfpathcurveto{\pgfqpoint{3.896862in}{2.187911in}}{\pgfqpoint{3.888962in}{2.184638in}}{\pgfqpoint{3.883138in}{2.178814in}}%
\pgfpathcurveto{\pgfqpoint{3.877314in}{2.172991in}}{\pgfqpoint{3.874042in}{2.165091in}}{\pgfqpoint{3.874042in}{2.156854in}}%
\pgfpathcurveto{\pgfqpoint{3.874042in}{2.148618in}}{\pgfqpoint{3.877314in}{2.140718in}}{\pgfqpoint{3.883138in}{2.134894in}}%
\pgfpathcurveto{\pgfqpoint{3.888962in}{2.129070in}}{\pgfqpoint{3.896862in}{2.125798in}}{\pgfqpoint{3.905098in}{2.125798in}}%
\pgfpathclose%
\pgfusepath{stroke,fill}%
\end{pgfscope}%
\begin{pgfscope}%
\pgfpathrectangle{\pgfqpoint{3.793912in}{0.557870in}}{\pgfqpoint{2.446088in}{1.684734in}}%
\pgfusepath{clip}%
\pgfsetbuttcap%
\pgfsetroundjoin%
\definecolor{currentfill}{rgb}{0.298039,0.447059,0.690196}%
\pgfsetfillcolor{currentfill}%
\pgfsetlinewidth{1.003750pt}%
\definecolor{currentstroke}{rgb}{0.298039,0.447059,0.690196}%
\pgfsetstrokecolor{currentstroke}%
\pgfsetdash{}{0pt}%
\pgfpathmoveto{\pgfqpoint{3.905098in}{2.125798in}}%
\pgfpathcurveto{\pgfqpoint{3.913334in}{2.125798in}}{\pgfqpoint{3.921234in}{2.129070in}}{\pgfqpoint{3.927058in}{2.134894in}}%
\pgfpathcurveto{\pgfqpoint{3.932882in}{2.140718in}}{\pgfqpoint{3.936155in}{2.148618in}}{\pgfqpoint{3.936155in}{2.156854in}}%
\pgfpathcurveto{\pgfqpoint{3.936155in}{2.165091in}}{\pgfqpoint{3.932882in}{2.172991in}}{\pgfqpoint{3.927058in}{2.178814in}}%
\pgfpathcurveto{\pgfqpoint{3.921234in}{2.184638in}}{\pgfqpoint{3.913334in}{2.187911in}}{\pgfqpoint{3.905098in}{2.187911in}}%
\pgfpathcurveto{\pgfqpoint{3.896862in}{2.187911in}}{\pgfqpoint{3.888962in}{2.184638in}}{\pgfqpoint{3.883138in}{2.178814in}}%
\pgfpathcurveto{\pgfqpoint{3.877314in}{2.172991in}}{\pgfqpoint{3.874042in}{2.165091in}}{\pgfqpoint{3.874042in}{2.156854in}}%
\pgfpathcurveto{\pgfqpoint{3.874042in}{2.148618in}}{\pgfqpoint{3.877314in}{2.140718in}}{\pgfqpoint{3.883138in}{2.134894in}}%
\pgfpathcurveto{\pgfqpoint{3.888962in}{2.129070in}}{\pgfqpoint{3.896862in}{2.125798in}}{\pgfqpoint{3.905098in}{2.125798in}}%
\pgfpathclose%
\pgfusepath{stroke,fill}%
\end{pgfscope}%
\begin{pgfscope}%
\pgfpathrectangle{\pgfqpoint{3.793912in}{0.557870in}}{\pgfqpoint{2.446088in}{1.684734in}}%
\pgfusepath{clip}%
\pgfsetbuttcap%
\pgfsetroundjoin%
\definecolor{currentfill}{rgb}{0.298039,0.447059,0.690196}%
\pgfsetfillcolor{currentfill}%
\pgfsetlinewidth{1.003750pt}%
\definecolor{currentstroke}{rgb}{0.298039,0.447059,0.690196}%
\pgfsetstrokecolor{currentstroke}%
\pgfsetdash{}{0pt}%
\pgfpathmoveto{\pgfqpoint{3.905098in}{2.125798in}}%
\pgfpathcurveto{\pgfqpoint{3.913334in}{2.125798in}}{\pgfqpoint{3.921234in}{2.129070in}}{\pgfqpoint{3.927058in}{2.134894in}}%
\pgfpathcurveto{\pgfqpoint{3.932882in}{2.140718in}}{\pgfqpoint{3.936155in}{2.148618in}}{\pgfqpoint{3.936155in}{2.156854in}}%
\pgfpathcurveto{\pgfqpoint{3.936155in}{2.165091in}}{\pgfqpoint{3.932882in}{2.172991in}}{\pgfqpoint{3.927058in}{2.178814in}}%
\pgfpathcurveto{\pgfqpoint{3.921234in}{2.184638in}}{\pgfqpoint{3.913334in}{2.187911in}}{\pgfqpoint{3.905098in}{2.187911in}}%
\pgfpathcurveto{\pgfqpoint{3.896862in}{2.187911in}}{\pgfqpoint{3.888962in}{2.184638in}}{\pgfqpoint{3.883138in}{2.178814in}}%
\pgfpathcurveto{\pgfqpoint{3.877314in}{2.172991in}}{\pgfqpoint{3.874042in}{2.165091in}}{\pgfqpoint{3.874042in}{2.156854in}}%
\pgfpathcurveto{\pgfqpoint{3.874042in}{2.148618in}}{\pgfqpoint{3.877314in}{2.140718in}}{\pgfqpoint{3.883138in}{2.134894in}}%
\pgfpathcurveto{\pgfqpoint{3.888962in}{2.129070in}}{\pgfqpoint{3.896862in}{2.125798in}}{\pgfqpoint{3.905098in}{2.125798in}}%
\pgfpathclose%
\pgfusepath{stroke,fill}%
\end{pgfscope}%
\begin{pgfscope}%
\pgfpathrectangle{\pgfqpoint{3.793912in}{0.557870in}}{\pgfqpoint{2.446088in}{1.684734in}}%
\pgfusepath{clip}%
\pgfsetbuttcap%
\pgfsetroundjoin%
\definecolor{currentfill}{rgb}{0.298039,0.447059,0.690196}%
\pgfsetfillcolor{currentfill}%
\pgfsetlinewidth{1.003750pt}%
\definecolor{currentstroke}{rgb}{0.298039,0.447059,0.690196}%
\pgfsetstrokecolor{currentstroke}%
\pgfsetdash{}{0pt}%
\pgfpathmoveto{\pgfqpoint{3.905098in}{2.125798in}}%
\pgfpathcurveto{\pgfqpoint{3.913334in}{2.125798in}}{\pgfqpoint{3.921234in}{2.129070in}}{\pgfqpoint{3.927058in}{2.134894in}}%
\pgfpathcurveto{\pgfqpoint{3.932882in}{2.140718in}}{\pgfqpoint{3.936155in}{2.148618in}}{\pgfqpoint{3.936155in}{2.156854in}}%
\pgfpathcurveto{\pgfqpoint{3.936155in}{2.165091in}}{\pgfqpoint{3.932882in}{2.172991in}}{\pgfqpoint{3.927058in}{2.178814in}}%
\pgfpathcurveto{\pgfqpoint{3.921234in}{2.184638in}}{\pgfqpoint{3.913334in}{2.187911in}}{\pgfqpoint{3.905098in}{2.187911in}}%
\pgfpathcurveto{\pgfqpoint{3.896862in}{2.187911in}}{\pgfqpoint{3.888962in}{2.184638in}}{\pgfqpoint{3.883138in}{2.178814in}}%
\pgfpathcurveto{\pgfqpoint{3.877314in}{2.172991in}}{\pgfqpoint{3.874042in}{2.165091in}}{\pgfqpoint{3.874042in}{2.156854in}}%
\pgfpathcurveto{\pgfqpoint{3.874042in}{2.148618in}}{\pgfqpoint{3.877314in}{2.140718in}}{\pgfqpoint{3.883138in}{2.134894in}}%
\pgfpathcurveto{\pgfqpoint{3.888962in}{2.129070in}}{\pgfqpoint{3.896862in}{2.125798in}}{\pgfqpoint{3.905098in}{2.125798in}}%
\pgfpathclose%
\pgfusepath{stroke,fill}%
\end{pgfscope}%
\begin{pgfscope}%
\pgfpathrectangle{\pgfqpoint{3.793912in}{0.557870in}}{\pgfqpoint{2.446088in}{1.684734in}}%
\pgfusepath{clip}%
\pgfsetbuttcap%
\pgfsetroundjoin%
\definecolor{currentfill}{rgb}{0.298039,0.447059,0.690196}%
\pgfsetfillcolor{currentfill}%
\pgfsetlinewidth{1.003750pt}%
\definecolor{currentstroke}{rgb}{0.298039,0.447059,0.690196}%
\pgfsetstrokecolor{currentstroke}%
\pgfsetdash{}{0pt}%
\pgfpathmoveto{\pgfqpoint{5.131976in}{1.639729in}}%
\pgfpathcurveto{\pgfqpoint{5.140212in}{1.639729in}}{\pgfqpoint{5.148112in}{1.643001in}}{\pgfqpoint{5.153936in}{1.648825in}}%
\pgfpathcurveto{\pgfqpoint{5.159760in}{1.654649in}}{\pgfqpoint{5.163032in}{1.662549in}}{\pgfqpoint{5.163032in}{1.670785in}}%
\pgfpathcurveto{\pgfqpoint{5.163032in}{1.679021in}}{\pgfqpoint{5.159760in}{1.686921in}}{\pgfqpoint{5.153936in}{1.692745in}}%
\pgfpathcurveto{\pgfqpoint{5.148112in}{1.698569in}}{\pgfqpoint{5.140212in}{1.701842in}}{\pgfqpoint{5.131976in}{1.701842in}}%
\pgfpathcurveto{\pgfqpoint{5.123740in}{1.701842in}}{\pgfqpoint{5.115840in}{1.698569in}}{\pgfqpoint{5.110016in}{1.692745in}}%
\pgfpathcurveto{\pgfqpoint{5.104192in}{1.686921in}}{\pgfqpoint{5.100919in}{1.679021in}}{\pgfqpoint{5.100919in}{1.670785in}}%
\pgfpathcurveto{\pgfqpoint{5.100919in}{1.662549in}}{\pgfqpoint{5.104192in}{1.654649in}}{\pgfqpoint{5.110016in}{1.648825in}}%
\pgfpathcurveto{\pgfqpoint{5.115840in}{1.643001in}}{\pgfqpoint{5.123740in}{1.639729in}}{\pgfqpoint{5.131976in}{1.639729in}}%
\pgfpathclose%
\pgfusepath{stroke,fill}%
\end{pgfscope}%
\begin{pgfscope}%
\pgfpathrectangle{\pgfqpoint{3.793912in}{0.557870in}}{\pgfqpoint{2.446088in}{1.684734in}}%
\pgfusepath{clip}%
\pgfsetbuttcap%
\pgfsetroundjoin%
\definecolor{currentfill}{rgb}{0.298039,0.447059,0.690196}%
\pgfsetfillcolor{currentfill}%
\pgfsetlinewidth{1.003750pt}%
\definecolor{currentstroke}{rgb}{0.298039,0.447059,0.690196}%
\pgfsetstrokecolor{currentstroke}%
\pgfsetdash{}{0pt}%
\pgfpathmoveto{\pgfqpoint{3.905098in}{2.125798in}}%
\pgfpathcurveto{\pgfqpoint{3.913334in}{2.125798in}}{\pgfqpoint{3.921234in}{2.129070in}}{\pgfqpoint{3.927058in}{2.134894in}}%
\pgfpathcurveto{\pgfqpoint{3.932882in}{2.140718in}}{\pgfqpoint{3.936155in}{2.148618in}}{\pgfqpoint{3.936155in}{2.156854in}}%
\pgfpathcurveto{\pgfqpoint{3.936155in}{2.165091in}}{\pgfqpoint{3.932882in}{2.172991in}}{\pgfqpoint{3.927058in}{2.178814in}}%
\pgfpathcurveto{\pgfqpoint{3.921234in}{2.184638in}}{\pgfqpoint{3.913334in}{2.187911in}}{\pgfqpoint{3.905098in}{2.187911in}}%
\pgfpathcurveto{\pgfqpoint{3.896862in}{2.187911in}}{\pgfqpoint{3.888962in}{2.184638in}}{\pgfqpoint{3.883138in}{2.178814in}}%
\pgfpathcurveto{\pgfqpoint{3.877314in}{2.172991in}}{\pgfqpoint{3.874042in}{2.165091in}}{\pgfqpoint{3.874042in}{2.156854in}}%
\pgfpathcurveto{\pgfqpoint{3.874042in}{2.148618in}}{\pgfqpoint{3.877314in}{2.140718in}}{\pgfqpoint{3.883138in}{2.134894in}}%
\pgfpathcurveto{\pgfqpoint{3.888962in}{2.129070in}}{\pgfqpoint{3.896862in}{2.125798in}}{\pgfqpoint{3.905098in}{2.125798in}}%
\pgfpathclose%
\pgfusepath{stroke,fill}%
\end{pgfscope}%
\begin{pgfscope}%
\pgfpathrectangle{\pgfqpoint{3.793912in}{0.557870in}}{\pgfqpoint{2.446088in}{1.684734in}}%
\pgfusepath{clip}%
\pgfsetbuttcap%
\pgfsetroundjoin%
\definecolor{currentfill}{rgb}{0.298039,0.447059,0.690196}%
\pgfsetfillcolor{currentfill}%
\pgfsetlinewidth{1.003750pt}%
\definecolor{currentstroke}{rgb}{0.298039,0.447059,0.690196}%
\pgfsetstrokecolor{currentstroke}%
\pgfsetdash{}{0pt}%
\pgfpathmoveto{\pgfqpoint{5.131976in}{1.401280in}}%
\pgfpathcurveto{\pgfqpoint{5.140212in}{1.401280in}}{\pgfqpoint{5.148112in}{1.404552in}}{\pgfqpoint{5.153936in}{1.410376in}}%
\pgfpathcurveto{\pgfqpoint{5.159760in}{1.416200in}}{\pgfqpoint{5.163032in}{1.424100in}}{\pgfqpoint{5.163032in}{1.432336in}}%
\pgfpathcurveto{\pgfqpoint{5.163032in}{1.440572in}}{\pgfqpoint{5.159760in}{1.448472in}}{\pgfqpoint{5.153936in}{1.454296in}}%
\pgfpathcurveto{\pgfqpoint{5.148112in}{1.460120in}}{\pgfqpoint{5.140212in}{1.463393in}}{\pgfqpoint{5.131976in}{1.463393in}}%
\pgfpathcurveto{\pgfqpoint{5.123740in}{1.463393in}}{\pgfqpoint{5.115840in}{1.460120in}}{\pgfqpoint{5.110016in}{1.454296in}}%
\pgfpathcurveto{\pgfqpoint{5.104192in}{1.448472in}}{\pgfqpoint{5.100919in}{1.440572in}}{\pgfqpoint{5.100919in}{1.432336in}}%
\pgfpathcurveto{\pgfqpoint{5.100919in}{1.424100in}}{\pgfqpoint{5.104192in}{1.416200in}}{\pgfqpoint{5.110016in}{1.410376in}}%
\pgfpathcurveto{\pgfqpoint{5.115840in}{1.404552in}}{\pgfqpoint{5.123740in}{1.401280in}}{\pgfqpoint{5.131976in}{1.401280in}}%
\pgfpathclose%
\pgfusepath{stroke,fill}%
\end{pgfscope}%
\begin{pgfscope}%
\pgfpathrectangle{\pgfqpoint{3.793912in}{0.557870in}}{\pgfqpoint{2.446088in}{1.684734in}}%
\pgfusepath{clip}%
\pgfsetbuttcap%
\pgfsetroundjoin%
\definecolor{currentfill}{rgb}{0.298039,0.447059,0.690196}%
\pgfsetfillcolor{currentfill}%
\pgfsetlinewidth{1.003750pt}%
\definecolor{currentstroke}{rgb}{0.298039,0.447059,0.690196}%
\pgfsetstrokecolor{currentstroke}%
\pgfsetdash{}{0pt}%
\pgfpathmoveto{\pgfqpoint{3.905098in}{2.125798in}}%
\pgfpathcurveto{\pgfqpoint{3.913334in}{2.125798in}}{\pgfqpoint{3.921234in}{2.129070in}}{\pgfqpoint{3.927058in}{2.134894in}}%
\pgfpathcurveto{\pgfqpoint{3.932882in}{2.140718in}}{\pgfqpoint{3.936155in}{2.148618in}}{\pgfqpoint{3.936155in}{2.156854in}}%
\pgfpathcurveto{\pgfqpoint{3.936155in}{2.165091in}}{\pgfqpoint{3.932882in}{2.172991in}}{\pgfqpoint{3.927058in}{2.178814in}}%
\pgfpathcurveto{\pgfqpoint{3.921234in}{2.184638in}}{\pgfqpoint{3.913334in}{2.187911in}}{\pgfqpoint{3.905098in}{2.187911in}}%
\pgfpathcurveto{\pgfqpoint{3.896862in}{2.187911in}}{\pgfqpoint{3.888962in}{2.184638in}}{\pgfqpoint{3.883138in}{2.178814in}}%
\pgfpathcurveto{\pgfqpoint{3.877314in}{2.172991in}}{\pgfqpoint{3.874042in}{2.165091in}}{\pgfqpoint{3.874042in}{2.156854in}}%
\pgfpathcurveto{\pgfqpoint{3.874042in}{2.148618in}}{\pgfqpoint{3.877314in}{2.140718in}}{\pgfqpoint{3.883138in}{2.134894in}}%
\pgfpathcurveto{\pgfqpoint{3.888962in}{2.129070in}}{\pgfqpoint{3.896862in}{2.125798in}}{\pgfqpoint{3.905098in}{2.125798in}}%
\pgfpathclose%
\pgfusepath{stroke,fill}%
\end{pgfscope}%
\begin{pgfscope}%
\pgfpathrectangle{\pgfqpoint{3.793912in}{0.557870in}}{\pgfqpoint{2.446088in}{1.684734in}}%
\pgfusepath{clip}%
\pgfsetbuttcap%
\pgfsetroundjoin%
\definecolor{currentfill}{rgb}{0.298039,0.447059,0.690196}%
\pgfsetfillcolor{currentfill}%
\pgfsetlinewidth{1.003750pt}%
\definecolor{currentstroke}{rgb}{0.298039,0.447059,0.690196}%
\pgfsetstrokecolor{currentstroke}%
\pgfsetdash{}{0pt}%
\pgfpathmoveto{\pgfqpoint{3.905098in}{2.125798in}}%
\pgfpathcurveto{\pgfqpoint{3.913334in}{2.125798in}}{\pgfqpoint{3.921234in}{2.129070in}}{\pgfqpoint{3.927058in}{2.134894in}}%
\pgfpathcurveto{\pgfqpoint{3.932882in}{2.140718in}}{\pgfqpoint{3.936155in}{2.148618in}}{\pgfqpoint{3.936155in}{2.156854in}}%
\pgfpathcurveto{\pgfqpoint{3.936155in}{2.165091in}}{\pgfqpoint{3.932882in}{2.172991in}}{\pgfqpoint{3.927058in}{2.178814in}}%
\pgfpathcurveto{\pgfqpoint{3.921234in}{2.184638in}}{\pgfqpoint{3.913334in}{2.187911in}}{\pgfqpoint{3.905098in}{2.187911in}}%
\pgfpathcurveto{\pgfqpoint{3.896862in}{2.187911in}}{\pgfqpoint{3.888962in}{2.184638in}}{\pgfqpoint{3.883138in}{2.178814in}}%
\pgfpathcurveto{\pgfqpoint{3.877314in}{2.172991in}}{\pgfqpoint{3.874042in}{2.165091in}}{\pgfqpoint{3.874042in}{2.156854in}}%
\pgfpathcurveto{\pgfqpoint{3.874042in}{2.148618in}}{\pgfqpoint{3.877314in}{2.140718in}}{\pgfqpoint{3.883138in}{2.134894in}}%
\pgfpathcurveto{\pgfqpoint{3.888962in}{2.129070in}}{\pgfqpoint{3.896862in}{2.125798in}}{\pgfqpoint{3.905098in}{2.125798in}}%
\pgfpathclose%
\pgfusepath{stroke,fill}%
\end{pgfscope}%
\begin{pgfscope}%
\pgfpathrectangle{\pgfqpoint{3.793912in}{0.557870in}}{\pgfqpoint{2.446088in}{1.684734in}}%
\pgfusepath{clip}%
\pgfsetbuttcap%
\pgfsetroundjoin%
\definecolor{currentfill}{rgb}{0.298039,0.447059,0.690196}%
\pgfsetfillcolor{currentfill}%
\pgfsetlinewidth{1.003750pt}%
\definecolor{currentstroke}{rgb}{0.298039,0.447059,0.690196}%
\pgfsetstrokecolor{currentstroke}%
\pgfsetdash{}{0pt}%
\pgfpathmoveto{\pgfqpoint{3.905098in}{2.125798in}}%
\pgfpathcurveto{\pgfqpoint{3.913334in}{2.125798in}}{\pgfqpoint{3.921234in}{2.129070in}}{\pgfqpoint{3.927058in}{2.134894in}}%
\pgfpathcurveto{\pgfqpoint{3.932882in}{2.140718in}}{\pgfqpoint{3.936155in}{2.148618in}}{\pgfqpoint{3.936155in}{2.156854in}}%
\pgfpathcurveto{\pgfqpoint{3.936155in}{2.165091in}}{\pgfqpoint{3.932882in}{2.172991in}}{\pgfqpoint{3.927058in}{2.178814in}}%
\pgfpathcurveto{\pgfqpoint{3.921234in}{2.184638in}}{\pgfqpoint{3.913334in}{2.187911in}}{\pgfqpoint{3.905098in}{2.187911in}}%
\pgfpathcurveto{\pgfqpoint{3.896862in}{2.187911in}}{\pgfqpoint{3.888962in}{2.184638in}}{\pgfqpoint{3.883138in}{2.178814in}}%
\pgfpathcurveto{\pgfqpoint{3.877314in}{2.172991in}}{\pgfqpoint{3.874042in}{2.165091in}}{\pgfqpoint{3.874042in}{2.156854in}}%
\pgfpathcurveto{\pgfqpoint{3.874042in}{2.148618in}}{\pgfqpoint{3.877314in}{2.140718in}}{\pgfqpoint{3.883138in}{2.134894in}}%
\pgfpathcurveto{\pgfqpoint{3.888962in}{2.129070in}}{\pgfqpoint{3.896862in}{2.125798in}}{\pgfqpoint{3.905098in}{2.125798in}}%
\pgfpathclose%
\pgfusepath{stroke,fill}%
\end{pgfscope}%
\begin{pgfscope}%
\pgfpathrectangle{\pgfqpoint{3.793912in}{0.557870in}}{\pgfqpoint{2.446088in}{1.684734in}}%
\pgfusepath{clip}%
\pgfsetbuttcap%
\pgfsetroundjoin%
\definecolor{currentfill}{rgb}{0.298039,0.447059,0.690196}%
\pgfsetfillcolor{currentfill}%
\pgfsetlinewidth{1.003750pt}%
\definecolor{currentstroke}{rgb}{0.298039,0.447059,0.690196}%
\pgfsetstrokecolor{currentstroke}%
\pgfsetdash{}{0pt}%
\pgfpathmoveto{\pgfqpoint{3.905098in}{2.125798in}}%
\pgfpathcurveto{\pgfqpoint{3.913334in}{2.125798in}}{\pgfqpoint{3.921234in}{2.129070in}}{\pgfqpoint{3.927058in}{2.134894in}}%
\pgfpathcurveto{\pgfqpoint{3.932882in}{2.140718in}}{\pgfqpoint{3.936155in}{2.148618in}}{\pgfqpoint{3.936155in}{2.156854in}}%
\pgfpathcurveto{\pgfqpoint{3.936155in}{2.165091in}}{\pgfqpoint{3.932882in}{2.172991in}}{\pgfqpoint{3.927058in}{2.178814in}}%
\pgfpathcurveto{\pgfqpoint{3.921234in}{2.184638in}}{\pgfqpoint{3.913334in}{2.187911in}}{\pgfqpoint{3.905098in}{2.187911in}}%
\pgfpathcurveto{\pgfqpoint{3.896862in}{2.187911in}}{\pgfqpoint{3.888962in}{2.184638in}}{\pgfqpoint{3.883138in}{2.178814in}}%
\pgfpathcurveto{\pgfqpoint{3.877314in}{2.172991in}}{\pgfqpoint{3.874042in}{2.165091in}}{\pgfqpoint{3.874042in}{2.156854in}}%
\pgfpathcurveto{\pgfqpoint{3.874042in}{2.148618in}}{\pgfqpoint{3.877314in}{2.140718in}}{\pgfqpoint{3.883138in}{2.134894in}}%
\pgfpathcurveto{\pgfqpoint{3.888962in}{2.129070in}}{\pgfqpoint{3.896862in}{2.125798in}}{\pgfqpoint{3.905098in}{2.125798in}}%
\pgfpathclose%
\pgfusepath{stroke,fill}%
\end{pgfscope}%
\begin{pgfscope}%
\pgfpathrectangle{\pgfqpoint{3.793912in}{0.557870in}}{\pgfqpoint{2.446088in}{1.684734in}}%
\pgfusepath{clip}%
\pgfsetbuttcap%
\pgfsetroundjoin%
\definecolor{currentfill}{rgb}{0.298039,0.447059,0.690196}%
\pgfsetfillcolor{currentfill}%
\pgfsetlinewidth{1.003750pt}%
\definecolor{currentstroke}{rgb}{0.298039,0.447059,0.690196}%
\pgfsetstrokecolor{currentstroke}%
\pgfsetdash{}{0pt}%
\pgfpathmoveto{\pgfqpoint{5.975454in}{1.428793in}}%
\pgfpathcurveto{\pgfqpoint{5.983691in}{1.428793in}}{\pgfqpoint{5.991591in}{1.432065in}}{\pgfqpoint{5.997415in}{1.437889in}}%
\pgfpathcurveto{\pgfqpoint{6.003239in}{1.443713in}}{\pgfqpoint{6.006511in}{1.451613in}}{\pgfqpoint{6.006511in}{1.459849in}}%
\pgfpathcurveto{\pgfqpoint{6.006511in}{1.468086in}}{\pgfqpoint{6.003239in}{1.475986in}}{\pgfqpoint{5.997415in}{1.481810in}}%
\pgfpathcurveto{\pgfqpoint{5.991591in}{1.487634in}}{\pgfqpoint{5.983691in}{1.490906in}}{\pgfqpoint{5.975454in}{1.490906in}}%
\pgfpathcurveto{\pgfqpoint{5.967218in}{1.490906in}}{\pgfqpoint{5.959318in}{1.487634in}}{\pgfqpoint{5.953494in}{1.481810in}}%
\pgfpathcurveto{\pgfqpoint{5.947670in}{1.475986in}}{\pgfqpoint{5.944398in}{1.468086in}}{\pgfqpoint{5.944398in}{1.459849in}}%
\pgfpathcurveto{\pgfqpoint{5.944398in}{1.451613in}}{\pgfqpoint{5.947670in}{1.443713in}}{\pgfqpoint{5.953494in}{1.437889in}}%
\pgfpathcurveto{\pgfqpoint{5.959318in}{1.432065in}}{\pgfqpoint{5.967218in}{1.428793in}}{\pgfqpoint{5.975454in}{1.428793in}}%
\pgfpathclose%
\pgfusepath{stroke,fill}%
\end{pgfscope}%
\begin{pgfscope}%
\pgfpathrectangle{\pgfqpoint{3.793912in}{0.557870in}}{\pgfqpoint{2.446088in}{1.684734in}}%
\pgfusepath{clip}%
\pgfsetbuttcap%
\pgfsetroundjoin%
\definecolor{currentfill}{rgb}{0.298039,0.447059,0.690196}%
\pgfsetfillcolor{currentfill}%
\pgfsetlinewidth{1.003750pt}%
\definecolor{currentstroke}{rgb}{0.298039,0.447059,0.690196}%
\pgfsetstrokecolor{currentstroke}%
\pgfsetdash{}{0pt}%
\pgfpathmoveto{\pgfqpoint{5.975454in}{1.529675in}}%
\pgfpathcurveto{\pgfqpoint{5.983691in}{1.529675in}}{\pgfqpoint{5.991591in}{1.532948in}}{\pgfqpoint{5.997415in}{1.538771in}}%
\pgfpathcurveto{\pgfqpoint{6.003239in}{1.544595in}}{\pgfqpoint{6.006511in}{1.552495in}}{\pgfqpoint{6.006511in}{1.560732in}}%
\pgfpathcurveto{\pgfqpoint{6.006511in}{1.568968in}}{\pgfqpoint{6.003239in}{1.576868in}}{\pgfqpoint{5.997415in}{1.582692in}}%
\pgfpathcurveto{\pgfqpoint{5.991591in}{1.588516in}}{\pgfqpoint{5.983691in}{1.591788in}}{\pgfqpoint{5.975454in}{1.591788in}}%
\pgfpathcurveto{\pgfqpoint{5.967218in}{1.591788in}}{\pgfqpoint{5.959318in}{1.588516in}}{\pgfqpoint{5.953494in}{1.582692in}}%
\pgfpathcurveto{\pgfqpoint{5.947670in}{1.576868in}}{\pgfqpoint{5.944398in}{1.568968in}}{\pgfqpoint{5.944398in}{1.560732in}}%
\pgfpathcurveto{\pgfqpoint{5.944398in}{1.552495in}}{\pgfqpoint{5.947670in}{1.544595in}}{\pgfqpoint{5.953494in}{1.538771in}}%
\pgfpathcurveto{\pgfqpoint{5.959318in}{1.532948in}}{\pgfqpoint{5.967218in}{1.529675in}}{\pgfqpoint{5.975454in}{1.529675in}}%
\pgfpathclose%
\pgfusepath{stroke,fill}%
\end{pgfscope}%
\begin{pgfscope}%
\pgfpathrectangle{\pgfqpoint{3.793912in}{0.557870in}}{\pgfqpoint{2.446088in}{1.684734in}}%
\pgfusepath{clip}%
\pgfsetbuttcap%
\pgfsetroundjoin%
\definecolor{currentfill}{rgb}{0.298039,0.447059,0.690196}%
\pgfsetfillcolor{currentfill}%
\pgfsetlinewidth{1.003750pt}%
\definecolor{currentstroke}{rgb}{0.298039,0.447059,0.690196}%
\pgfsetstrokecolor{currentstroke}%
\pgfsetdash{}{0pt}%
\pgfpathmoveto{\pgfqpoint{5.975454in}{1.437964in}}%
\pgfpathcurveto{\pgfqpoint{5.983691in}{1.437964in}}{\pgfqpoint{5.991591in}{1.441236in}}{\pgfqpoint{5.997415in}{1.447060in}}%
\pgfpathcurveto{\pgfqpoint{6.003239in}{1.452884in}}{\pgfqpoint{6.006511in}{1.460784in}}{\pgfqpoint{6.006511in}{1.469021in}}%
\pgfpathcurveto{\pgfqpoint{6.006511in}{1.477257in}}{\pgfqpoint{6.003239in}{1.485157in}}{\pgfqpoint{5.997415in}{1.490981in}}%
\pgfpathcurveto{\pgfqpoint{5.991591in}{1.496805in}}{\pgfqpoint{5.983691in}{1.500077in}}{\pgfqpoint{5.975454in}{1.500077in}}%
\pgfpathcurveto{\pgfqpoint{5.967218in}{1.500077in}}{\pgfqpoint{5.959318in}{1.496805in}}{\pgfqpoint{5.953494in}{1.490981in}}%
\pgfpathcurveto{\pgfqpoint{5.947670in}{1.485157in}}{\pgfqpoint{5.944398in}{1.477257in}}{\pgfqpoint{5.944398in}{1.469021in}}%
\pgfpathcurveto{\pgfqpoint{5.944398in}{1.460784in}}{\pgfqpoint{5.947670in}{1.452884in}}{\pgfqpoint{5.953494in}{1.447060in}}%
\pgfpathcurveto{\pgfqpoint{5.959318in}{1.441236in}}{\pgfqpoint{5.967218in}{1.437964in}}{\pgfqpoint{5.975454in}{1.437964in}}%
\pgfpathclose%
\pgfusepath{stroke,fill}%
\end{pgfscope}%
\begin{pgfscope}%
\pgfpathrectangle{\pgfqpoint{3.793912in}{0.557870in}}{\pgfqpoint{2.446088in}{1.684734in}}%
\pgfusepath{clip}%
\pgfsetbuttcap%
\pgfsetroundjoin%
\definecolor{currentfill}{rgb}{0.298039,0.447059,0.690196}%
\pgfsetfillcolor{currentfill}%
\pgfsetlinewidth{1.003750pt}%
\definecolor{currentstroke}{rgb}{0.298039,0.447059,0.690196}%
\pgfsetstrokecolor{currentstroke}%
\pgfsetdash{}{0pt}%
\pgfpathmoveto{\pgfqpoint{5.975454in}{1.364595in}}%
\pgfpathcurveto{\pgfqpoint{5.983691in}{1.364595in}}{\pgfqpoint{5.991591in}{1.367867in}}{\pgfqpoint{5.997415in}{1.373691in}}%
\pgfpathcurveto{\pgfqpoint{6.003239in}{1.379515in}}{\pgfqpoint{6.006511in}{1.387415in}}{\pgfqpoint{6.006511in}{1.395652in}}%
\pgfpathcurveto{\pgfqpoint{6.006511in}{1.403888in}}{\pgfqpoint{6.003239in}{1.411788in}}{\pgfqpoint{5.997415in}{1.417612in}}%
\pgfpathcurveto{\pgfqpoint{5.991591in}{1.423436in}}{\pgfqpoint{5.983691in}{1.426708in}}{\pgfqpoint{5.975454in}{1.426708in}}%
\pgfpathcurveto{\pgfqpoint{5.967218in}{1.426708in}}{\pgfqpoint{5.959318in}{1.423436in}}{\pgfqpoint{5.953494in}{1.417612in}}%
\pgfpathcurveto{\pgfqpoint{5.947670in}{1.411788in}}{\pgfqpoint{5.944398in}{1.403888in}}{\pgfqpoint{5.944398in}{1.395652in}}%
\pgfpathcurveto{\pgfqpoint{5.944398in}{1.387415in}}{\pgfqpoint{5.947670in}{1.379515in}}{\pgfqpoint{5.953494in}{1.373691in}}%
\pgfpathcurveto{\pgfqpoint{5.959318in}{1.367867in}}{\pgfqpoint{5.967218in}{1.364595in}}{\pgfqpoint{5.975454in}{1.364595in}}%
\pgfpathclose%
\pgfusepath{stroke,fill}%
\end{pgfscope}%
\begin{pgfscope}%
\pgfpathrectangle{\pgfqpoint{3.793912in}{0.557870in}}{\pgfqpoint{2.446088in}{1.684734in}}%
\pgfusepath{clip}%
\pgfsetbuttcap%
\pgfsetroundjoin%
\definecolor{currentfill}{rgb}{0.298039,0.447059,0.690196}%
\pgfsetfillcolor{currentfill}%
\pgfsetlinewidth{1.003750pt}%
\definecolor{currentstroke}{rgb}{0.298039,0.447059,0.690196}%
\pgfsetstrokecolor{currentstroke}%
\pgfsetdash{}{0pt}%
\pgfpathmoveto{\pgfqpoint{5.822095in}{1.456306in}}%
\pgfpathcurveto{\pgfqpoint{5.830331in}{1.456306in}}{\pgfqpoint{5.838231in}{1.459579in}}{\pgfqpoint{5.844055in}{1.465403in}}%
\pgfpathcurveto{\pgfqpoint{5.849879in}{1.471226in}}{\pgfqpoint{5.853151in}{1.479127in}}{\pgfqpoint{5.853151in}{1.487363in}}%
\pgfpathcurveto{\pgfqpoint{5.853151in}{1.495599in}}{\pgfqpoint{5.849879in}{1.503499in}}{\pgfqpoint{5.844055in}{1.509323in}}%
\pgfpathcurveto{\pgfqpoint{5.838231in}{1.515147in}}{\pgfqpoint{5.830331in}{1.518419in}}{\pgfqpoint{5.822095in}{1.518419in}}%
\pgfpathcurveto{\pgfqpoint{5.813858in}{1.518419in}}{\pgfqpoint{5.805958in}{1.515147in}}{\pgfqpoint{5.800134in}{1.509323in}}%
\pgfpathcurveto{\pgfqpoint{5.794311in}{1.503499in}}{\pgfqpoint{5.791038in}{1.495599in}}{\pgfqpoint{5.791038in}{1.487363in}}%
\pgfpathcurveto{\pgfqpoint{5.791038in}{1.479127in}}{\pgfqpoint{5.794311in}{1.471226in}}{\pgfqpoint{5.800134in}{1.465403in}}%
\pgfpathcurveto{\pgfqpoint{5.805958in}{1.459579in}}{\pgfqpoint{5.813858in}{1.456306in}}{\pgfqpoint{5.822095in}{1.456306in}}%
\pgfpathclose%
\pgfusepath{stroke,fill}%
\end{pgfscope}%
\begin{pgfscope}%
\pgfpathrectangle{\pgfqpoint{3.793912in}{0.557870in}}{\pgfqpoint{2.446088in}{1.684734in}}%
\pgfusepath{clip}%
\pgfsetbuttcap%
\pgfsetroundjoin%
\definecolor{currentfill}{rgb}{0.298039,0.447059,0.690196}%
\pgfsetfillcolor{currentfill}%
\pgfsetlinewidth{1.003750pt}%
\definecolor{currentstroke}{rgb}{0.298039,0.447059,0.690196}%
\pgfsetstrokecolor{currentstroke}%
\pgfsetdash{}{0pt}%
\pgfpathmoveto{\pgfqpoint{4.058458in}{2.006573in}}%
\pgfpathcurveto{\pgfqpoint{4.066694in}{2.006573in}}{\pgfqpoint{4.074594in}{2.009846in}}{\pgfqpoint{4.080418in}{2.015669in}}%
\pgfpathcurveto{\pgfqpoint{4.086242in}{2.021493in}}{\pgfqpoint{4.089514in}{2.029393in}}{\pgfqpoint{4.089514in}{2.037630in}}%
\pgfpathcurveto{\pgfqpoint{4.089514in}{2.045866in}}{\pgfqpoint{4.086242in}{2.053766in}}{\pgfqpoint{4.080418in}{2.059590in}}%
\pgfpathcurveto{\pgfqpoint{4.074594in}{2.065414in}}{\pgfqpoint{4.066694in}{2.068686in}}{\pgfqpoint{4.058458in}{2.068686in}}%
\pgfpathcurveto{\pgfqpoint{4.050221in}{2.068686in}}{\pgfqpoint{4.042321in}{2.065414in}}{\pgfqpoint{4.036498in}{2.059590in}}%
\pgfpathcurveto{\pgfqpoint{4.030674in}{2.053766in}}{\pgfqpoint{4.027401in}{2.045866in}}{\pgfqpoint{4.027401in}{2.037630in}}%
\pgfpathcurveto{\pgfqpoint{4.027401in}{2.029393in}}{\pgfqpoint{4.030674in}{2.021493in}}{\pgfqpoint{4.036498in}{2.015669in}}%
\pgfpathcurveto{\pgfqpoint{4.042321in}{2.009846in}}{\pgfqpoint{4.050221in}{2.006573in}}{\pgfqpoint{4.058458in}{2.006573in}}%
\pgfpathclose%
\pgfusepath{stroke,fill}%
\end{pgfscope}%
\begin{pgfscope}%
\pgfpathrectangle{\pgfqpoint{3.793912in}{0.557870in}}{\pgfqpoint{2.446088in}{1.684734in}}%
\pgfusepath{clip}%
\pgfsetbuttcap%
\pgfsetroundjoin%
\definecolor{currentfill}{rgb}{0.298039,0.447059,0.690196}%
\pgfsetfillcolor{currentfill}%
\pgfsetlinewidth{1.003750pt}%
\definecolor{currentstroke}{rgb}{0.298039,0.447059,0.690196}%
\pgfsetstrokecolor{currentstroke}%
\pgfsetdash{}{0pt}%
\pgfpathmoveto{\pgfqpoint{5.898775in}{1.557189in}}%
\pgfpathcurveto{\pgfqpoint{5.907011in}{1.557189in}}{\pgfqpoint{5.914911in}{1.560461in}}{\pgfqpoint{5.920735in}{1.566285in}}%
\pgfpathcurveto{\pgfqpoint{5.926559in}{1.572109in}}{\pgfqpoint{5.929831in}{1.580009in}}{\pgfqpoint{5.929831in}{1.588245in}}%
\pgfpathcurveto{\pgfqpoint{5.929831in}{1.596481in}}{\pgfqpoint{5.926559in}{1.604381in}}{\pgfqpoint{5.920735in}{1.610205in}}%
\pgfpathcurveto{\pgfqpoint{5.914911in}{1.616029in}}{\pgfqpoint{5.907011in}{1.619302in}}{\pgfqpoint{5.898775in}{1.619302in}}%
\pgfpathcurveto{\pgfqpoint{5.890538in}{1.619302in}}{\pgfqpoint{5.882638in}{1.616029in}}{\pgfqpoint{5.876814in}{1.610205in}}%
\pgfpathcurveto{\pgfqpoint{5.870990in}{1.604381in}}{\pgfqpoint{5.867718in}{1.596481in}}{\pgfqpoint{5.867718in}{1.588245in}}%
\pgfpathcurveto{\pgfqpoint{5.867718in}{1.580009in}}{\pgfqpoint{5.870990in}{1.572109in}}{\pgfqpoint{5.876814in}{1.566285in}}%
\pgfpathcurveto{\pgfqpoint{5.882638in}{1.560461in}}{\pgfqpoint{5.890538in}{1.557189in}}{\pgfqpoint{5.898775in}{1.557189in}}%
\pgfpathclose%
\pgfusepath{stroke,fill}%
\end{pgfscope}%
\begin{pgfscope}%
\pgfpathrectangle{\pgfqpoint{3.793912in}{0.557870in}}{\pgfqpoint{2.446088in}{1.684734in}}%
\pgfusepath{clip}%
\pgfsetbuttcap%
\pgfsetroundjoin%
\definecolor{currentfill}{rgb}{0.298039,0.447059,0.690196}%
\pgfsetfillcolor{currentfill}%
\pgfsetlinewidth{1.003750pt}%
\definecolor{currentstroke}{rgb}{0.298039,0.447059,0.690196}%
\pgfsetstrokecolor{currentstroke}%
\pgfsetdash{}{0pt}%
\pgfpathmoveto{\pgfqpoint{4.058458in}{1.621386in}}%
\pgfpathcurveto{\pgfqpoint{4.066694in}{1.621386in}}{\pgfqpoint{4.074594in}{1.624659in}}{\pgfqpoint{4.080418in}{1.630483in}}%
\pgfpathcurveto{\pgfqpoint{4.086242in}{1.636307in}}{\pgfqpoint{4.089514in}{1.644207in}}{\pgfqpoint{4.089514in}{1.652443in}}%
\pgfpathcurveto{\pgfqpoint{4.089514in}{1.660679in}}{\pgfqpoint{4.086242in}{1.668579in}}{\pgfqpoint{4.080418in}{1.674403in}}%
\pgfpathcurveto{\pgfqpoint{4.074594in}{1.680227in}}{\pgfqpoint{4.066694in}{1.683499in}}{\pgfqpoint{4.058458in}{1.683499in}}%
\pgfpathcurveto{\pgfqpoint{4.050221in}{1.683499in}}{\pgfqpoint{4.042321in}{1.680227in}}{\pgfqpoint{4.036498in}{1.674403in}}%
\pgfpathcurveto{\pgfqpoint{4.030674in}{1.668579in}}{\pgfqpoint{4.027401in}{1.660679in}}{\pgfqpoint{4.027401in}{1.652443in}}%
\pgfpathcurveto{\pgfqpoint{4.027401in}{1.644207in}}{\pgfqpoint{4.030674in}{1.636307in}}{\pgfqpoint{4.036498in}{1.630483in}}%
\pgfpathcurveto{\pgfqpoint{4.042321in}{1.624659in}}{\pgfqpoint{4.050221in}{1.621386in}}{\pgfqpoint{4.058458in}{1.621386in}}%
\pgfpathclose%
\pgfusepath{stroke,fill}%
\end{pgfscope}%
\begin{pgfscope}%
\pgfpathrectangle{\pgfqpoint{3.793912in}{0.557870in}}{\pgfqpoint{2.446088in}{1.684734in}}%
\pgfusepath{clip}%
\pgfsetbuttcap%
\pgfsetroundjoin%
\definecolor{currentfill}{rgb}{0.298039,0.447059,0.690196}%
\pgfsetfillcolor{currentfill}%
\pgfsetlinewidth{1.003750pt}%
\definecolor{currentstroke}{rgb}{0.298039,0.447059,0.690196}%
\pgfsetstrokecolor{currentstroke}%
\pgfsetdash{}{0pt}%
\pgfpathmoveto{\pgfqpoint{5.898775in}{1.584702in}}%
\pgfpathcurveto{\pgfqpoint{5.907011in}{1.584702in}}{\pgfqpoint{5.914911in}{1.587974in}}{\pgfqpoint{5.920735in}{1.593798in}}%
\pgfpathcurveto{\pgfqpoint{5.926559in}{1.599622in}}{\pgfqpoint{5.929831in}{1.607522in}}{\pgfqpoint{5.929831in}{1.615758in}}%
\pgfpathcurveto{\pgfqpoint{5.929831in}{1.623995in}}{\pgfqpoint{5.926559in}{1.631895in}}{\pgfqpoint{5.920735in}{1.637719in}}%
\pgfpathcurveto{\pgfqpoint{5.914911in}{1.643543in}}{\pgfqpoint{5.907011in}{1.646815in}}{\pgfqpoint{5.898775in}{1.646815in}}%
\pgfpathcurveto{\pgfqpoint{5.890538in}{1.646815in}}{\pgfqpoint{5.882638in}{1.643543in}}{\pgfqpoint{5.876814in}{1.637719in}}%
\pgfpathcurveto{\pgfqpoint{5.870990in}{1.631895in}}{\pgfqpoint{5.867718in}{1.623995in}}{\pgfqpoint{5.867718in}{1.615758in}}%
\pgfpathcurveto{\pgfqpoint{5.867718in}{1.607522in}}{\pgfqpoint{5.870990in}{1.599622in}}{\pgfqpoint{5.876814in}{1.593798in}}%
\pgfpathcurveto{\pgfqpoint{5.882638in}{1.587974in}}{\pgfqpoint{5.890538in}{1.584702in}}{\pgfqpoint{5.898775in}{1.584702in}}%
\pgfpathclose%
\pgfusepath{stroke,fill}%
\end{pgfscope}%
\begin{pgfscope}%
\pgfpathrectangle{\pgfqpoint{3.793912in}{0.557870in}}{\pgfqpoint{2.446088in}{1.684734in}}%
\pgfusepath{clip}%
\pgfsetbuttcap%
\pgfsetroundjoin%
\definecolor{currentfill}{rgb}{0.298039,0.447059,0.690196}%
\pgfsetfillcolor{currentfill}%
\pgfsetlinewidth{1.003750pt}%
\definecolor{currentstroke}{rgb}{0.298039,0.447059,0.690196}%
\pgfsetstrokecolor{currentstroke}%
\pgfsetdash{}{0pt}%
\pgfpathmoveto{\pgfqpoint{4.058458in}{1.878178in}}%
\pgfpathcurveto{\pgfqpoint{4.066694in}{1.878178in}}{\pgfqpoint{4.074594in}{1.881450in}}{\pgfqpoint{4.080418in}{1.887274in}}%
\pgfpathcurveto{\pgfqpoint{4.086242in}{1.893098in}}{\pgfqpoint{4.089514in}{1.900998in}}{\pgfqpoint{4.089514in}{1.909234in}}%
\pgfpathcurveto{\pgfqpoint{4.089514in}{1.917470in}}{\pgfqpoint{4.086242in}{1.925370in}}{\pgfqpoint{4.080418in}{1.931194in}}%
\pgfpathcurveto{\pgfqpoint{4.074594in}{1.937018in}}{\pgfqpoint{4.066694in}{1.940291in}}{\pgfqpoint{4.058458in}{1.940291in}}%
\pgfpathcurveto{\pgfqpoint{4.050221in}{1.940291in}}{\pgfqpoint{4.042321in}{1.937018in}}{\pgfqpoint{4.036498in}{1.931194in}}%
\pgfpathcurveto{\pgfqpoint{4.030674in}{1.925370in}}{\pgfqpoint{4.027401in}{1.917470in}}{\pgfqpoint{4.027401in}{1.909234in}}%
\pgfpathcurveto{\pgfqpoint{4.027401in}{1.900998in}}{\pgfqpoint{4.030674in}{1.893098in}}{\pgfqpoint{4.036498in}{1.887274in}}%
\pgfpathcurveto{\pgfqpoint{4.042321in}{1.881450in}}{\pgfqpoint{4.050221in}{1.878178in}}{\pgfqpoint{4.058458in}{1.878178in}}%
\pgfpathclose%
\pgfusepath{stroke,fill}%
\end{pgfscope}%
\begin{pgfscope}%
\pgfpathrectangle{\pgfqpoint{3.793912in}{0.557870in}}{\pgfqpoint{2.446088in}{1.684734in}}%
\pgfusepath{clip}%
\pgfsetbuttcap%
\pgfsetroundjoin%
\definecolor{currentfill}{rgb}{0.298039,0.447059,0.690196}%
\pgfsetfillcolor{currentfill}%
\pgfsetlinewidth{1.003750pt}%
\definecolor{currentstroke}{rgb}{0.298039,0.447059,0.690196}%
\pgfsetstrokecolor{currentstroke}%
\pgfsetdash{}{0pt}%
\pgfpathmoveto{\pgfqpoint{4.058458in}{1.878178in}}%
\pgfpathcurveto{\pgfqpoint{4.066694in}{1.878178in}}{\pgfqpoint{4.074594in}{1.881450in}}{\pgfqpoint{4.080418in}{1.887274in}}%
\pgfpathcurveto{\pgfqpoint{4.086242in}{1.893098in}}{\pgfqpoint{4.089514in}{1.900998in}}{\pgfqpoint{4.089514in}{1.909234in}}%
\pgfpathcurveto{\pgfqpoint{4.089514in}{1.917470in}}{\pgfqpoint{4.086242in}{1.925370in}}{\pgfqpoint{4.080418in}{1.931194in}}%
\pgfpathcurveto{\pgfqpoint{4.074594in}{1.937018in}}{\pgfqpoint{4.066694in}{1.940291in}}{\pgfqpoint{4.058458in}{1.940291in}}%
\pgfpathcurveto{\pgfqpoint{4.050221in}{1.940291in}}{\pgfqpoint{4.042321in}{1.937018in}}{\pgfqpoint{4.036498in}{1.931194in}}%
\pgfpathcurveto{\pgfqpoint{4.030674in}{1.925370in}}{\pgfqpoint{4.027401in}{1.917470in}}{\pgfqpoint{4.027401in}{1.909234in}}%
\pgfpathcurveto{\pgfqpoint{4.027401in}{1.900998in}}{\pgfqpoint{4.030674in}{1.893098in}}{\pgfqpoint{4.036498in}{1.887274in}}%
\pgfpathcurveto{\pgfqpoint{4.042321in}{1.881450in}}{\pgfqpoint{4.050221in}{1.878178in}}{\pgfqpoint{4.058458in}{1.878178in}}%
\pgfpathclose%
\pgfusepath{stroke,fill}%
\end{pgfscope}%
\begin{pgfscope}%
\pgfpathrectangle{\pgfqpoint{3.793912in}{0.557870in}}{\pgfqpoint{2.446088in}{1.684734in}}%
\pgfusepath{clip}%
\pgfsetbuttcap%
\pgfsetroundjoin%
\definecolor{currentfill}{rgb}{0.298039,0.447059,0.690196}%
\pgfsetfillcolor{currentfill}%
\pgfsetlinewidth{1.003750pt}%
\definecolor{currentstroke}{rgb}{0.298039,0.447059,0.690196}%
\pgfsetstrokecolor{currentstroke}%
\pgfsetdash{}{0pt}%
\pgfpathmoveto{\pgfqpoint{3.905098in}{2.125798in}}%
\pgfpathcurveto{\pgfqpoint{3.913334in}{2.125798in}}{\pgfqpoint{3.921234in}{2.129070in}}{\pgfqpoint{3.927058in}{2.134894in}}%
\pgfpathcurveto{\pgfqpoint{3.932882in}{2.140718in}}{\pgfqpoint{3.936155in}{2.148618in}}{\pgfqpoint{3.936155in}{2.156854in}}%
\pgfpathcurveto{\pgfqpoint{3.936155in}{2.165091in}}{\pgfqpoint{3.932882in}{2.172991in}}{\pgfqpoint{3.927058in}{2.178814in}}%
\pgfpathcurveto{\pgfqpoint{3.921234in}{2.184638in}}{\pgfqpoint{3.913334in}{2.187911in}}{\pgfqpoint{3.905098in}{2.187911in}}%
\pgfpathcurveto{\pgfqpoint{3.896862in}{2.187911in}}{\pgfqpoint{3.888962in}{2.184638in}}{\pgfqpoint{3.883138in}{2.178814in}}%
\pgfpathcurveto{\pgfqpoint{3.877314in}{2.172991in}}{\pgfqpoint{3.874042in}{2.165091in}}{\pgfqpoint{3.874042in}{2.156854in}}%
\pgfpathcurveto{\pgfqpoint{3.874042in}{2.148618in}}{\pgfqpoint{3.877314in}{2.140718in}}{\pgfqpoint{3.883138in}{2.134894in}}%
\pgfpathcurveto{\pgfqpoint{3.888962in}{2.129070in}}{\pgfqpoint{3.896862in}{2.125798in}}{\pgfqpoint{3.905098in}{2.125798in}}%
\pgfpathclose%
\pgfusepath{stroke,fill}%
\end{pgfscope}%
\begin{pgfscope}%
\pgfpathrectangle{\pgfqpoint{3.793912in}{0.557870in}}{\pgfqpoint{2.446088in}{1.684734in}}%
\pgfusepath{clip}%
\pgfsetbuttcap%
\pgfsetroundjoin%
\definecolor{currentfill}{rgb}{0.298039,0.447059,0.690196}%
\pgfsetfillcolor{currentfill}%
\pgfsetlinewidth{1.003750pt}%
\definecolor{currentstroke}{rgb}{0.298039,0.447059,0.690196}%
\pgfsetstrokecolor{currentstroke}%
\pgfsetdash{}{0pt}%
\pgfpathmoveto{\pgfqpoint{3.905098in}{2.125798in}}%
\pgfpathcurveto{\pgfqpoint{3.913334in}{2.125798in}}{\pgfqpoint{3.921234in}{2.129070in}}{\pgfqpoint{3.927058in}{2.134894in}}%
\pgfpathcurveto{\pgfqpoint{3.932882in}{2.140718in}}{\pgfqpoint{3.936155in}{2.148618in}}{\pgfqpoint{3.936155in}{2.156854in}}%
\pgfpathcurveto{\pgfqpoint{3.936155in}{2.165091in}}{\pgfqpoint{3.932882in}{2.172991in}}{\pgfqpoint{3.927058in}{2.178814in}}%
\pgfpathcurveto{\pgfqpoint{3.921234in}{2.184638in}}{\pgfqpoint{3.913334in}{2.187911in}}{\pgfqpoint{3.905098in}{2.187911in}}%
\pgfpathcurveto{\pgfqpoint{3.896862in}{2.187911in}}{\pgfqpoint{3.888962in}{2.184638in}}{\pgfqpoint{3.883138in}{2.178814in}}%
\pgfpathcurveto{\pgfqpoint{3.877314in}{2.172991in}}{\pgfqpoint{3.874042in}{2.165091in}}{\pgfqpoint{3.874042in}{2.156854in}}%
\pgfpathcurveto{\pgfqpoint{3.874042in}{2.148618in}}{\pgfqpoint{3.877314in}{2.140718in}}{\pgfqpoint{3.883138in}{2.134894in}}%
\pgfpathcurveto{\pgfqpoint{3.888962in}{2.129070in}}{\pgfqpoint{3.896862in}{2.125798in}}{\pgfqpoint{3.905098in}{2.125798in}}%
\pgfpathclose%
\pgfusepath{stroke,fill}%
\end{pgfscope}%
\begin{pgfscope}%
\pgfpathrectangle{\pgfqpoint{3.793912in}{0.557870in}}{\pgfqpoint{2.446088in}{1.684734in}}%
\pgfusepath{clip}%
\pgfsetbuttcap%
\pgfsetroundjoin%
\definecolor{currentfill}{rgb}{0.298039,0.447059,0.690196}%
\pgfsetfillcolor{currentfill}%
\pgfsetlinewidth{1.003750pt}%
\definecolor{currentstroke}{rgb}{0.298039,0.447059,0.690196}%
\pgfsetstrokecolor{currentstroke}%
\pgfsetdash{}{0pt}%
\pgfpathmoveto{\pgfqpoint{3.905098in}{2.125798in}}%
\pgfpathcurveto{\pgfqpoint{3.913334in}{2.125798in}}{\pgfqpoint{3.921234in}{2.129070in}}{\pgfqpoint{3.927058in}{2.134894in}}%
\pgfpathcurveto{\pgfqpoint{3.932882in}{2.140718in}}{\pgfqpoint{3.936155in}{2.148618in}}{\pgfqpoint{3.936155in}{2.156854in}}%
\pgfpathcurveto{\pgfqpoint{3.936155in}{2.165091in}}{\pgfqpoint{3.932882in}{2.172991in}}{\pgfqpoint{3.927058in}{2.178814in}}%
\pgfpathcurveto{\pgfqpoint{3.921234in}{2.184638in}}{\pgfqpoint{3.913334in}{2.187911in}}{\pgfqpoint{3.905098in}{2.187911in}}%
\pgfpathcurveto{\pgfqpoint{3.896862in}{2.187911in}}{\pgfqpoint{3.888962in}{2.184638in}}{\pgfqpoint{3.883138in}{2.178814in}}%
\pgfpathcurveto{\pgfqpoint{3.877314in}{2.172991in}}{\pgfqpoint{3.874042in}{2.165091in}}{\pgfqpoint{3.874042in}{2.156854in}}%
\pgfpathcurveto{\pgfqpoint{3.874042in}{2.148618in}}{\pgfqpoint{3.877314in}{2.140718in}}{\pgfqpoint{3.883138in}{2.134894in}}%
\pgfpathcurveto{\pgfqpoint{3.888962in}{2.129070in}}{\pgfqpoint{3.896862in}{2.125798in}}{\pgfqpoint{3.905098in}{2.125798in}}%
\pgfpathclose%
\pgfusepath{stroke,fill}%
\end{pgfscope}%
\begin{pgfscope}%
\pgfpathrectangle{\pgfqpoint{3.793912in}{0.557870in}}{\pgfqpoint{2.446088in}{1.684734in}}%
\pgfusepath{clip}%
\pgfsetbuttcap%
\pgfsetroundjoin%
\definecolor{currentfill}{rgb}{0.298039,0.447059,0.690196}%
\pgfsetfillcolor{currentfill}%
\pgfsetlinewidth{1.003750pt}%
\definecolor{currentstroke}{rgb}{0.298039,0.447059,0.690196}%
\pgfsetstrokecolor{currentstroke}%
\pgfsetdash{}{0pt}%
\pgfpathmoveto{\pgfqpoint{4.901936in}{1.621386in}}%
\pgfpathcurveto{\pgfqpoint{4.910173in}{1.621386in}}{\pgfqpoint{4.918073in}{1.624659in}}{\pgfqpoint{4.923897in}{1.630483in}}%
\pgfpathcurveto{\pgfqpoint{4.929721in}{1.636307in}}{\pgfqpoint{4.932993in}{1.644207in}}{\pgfqpoint{4.932993in}{1.652443in}}%
\pgfpathcurveto{\pgfqpoint{4.932993in}{1.660679in}}{\pgfqpoint{4.929721in}{1.668579in}}{\pgfqpoint{4.923897in}{1.674403in}}%
\pgfpathcurveto{\pgfqpoint{4.918073in}{1.680227in}}{\pgfqpoint{4.910173in}{1.683499in}}{\pgfqpoint{4.901936in}{1.683499in}}%
\pgfpathcurveto{\pgfqpoint{4.893700in}{1.683499in}}{\pgfqpoint{4.885800in}{1.680227in}}{\pgfqpoint{4.879976in}{1.674403in}}%
\pgfpathcurveto{\pgfqpoint{4.874152in}{1.668579in}}{\pgfqpoint{4.870880in}{1.660679in}}{\pgfqpoint{4.870880in}{1.652443in}}%
\pgfpathcurveto{\pgfqpoint{4.870880in}{1.644207in}}{\pgfqpoint{4.874152in}{1.636307in}}{\pgfqpoint{4.879976in}{1.630483in}}%
\pgfpathcurveto{\pgfqpoint{4.885800in}{1.624659in}}{\pgfqpoint{4.893700in}{1.621386in}}{\pgfqpoint{4.901936in}{1.621386in}}%
\pgfpathclose%
\pgfusepath{stroke,fill}%
\end{pgfscope}%
\begin{pgfscope}%
\pgfpathrectangle{\pgfqpoint{3.793912in}{0.557870in}}{\pgfqpoint{2.446088in}{1.684734in}}%
\pgfusepath{clip}%
\pgfsetbuttcap%
\pgfsetroundjoin%
\definecolor{currentfill}{rgb}{0.298039,0.447059,0.690196}%
\pgfsetfillcolor{currentfill}%
\pgfsetlinewidth{1.003750pt}%
\definecolor{currentstroke}{rgb}{0.298039,0.447059,0.690196}%
\pgfsetstrokecolor{currentstroke}%
\pgfsetdash{}{0pt}%
\pgfpathmoveto{\pgfqpoint{3.905098in}{2.125798in}}%
\pgfpathcurveto{\pgfqpoint{3.913334in}{2.125798in}}{\pgfqpoint{3.921234in}{2.129070in}}{\pgfqpoint{3.927058in}{2.134894in}}%
\pgfpathcurveto{\pgfqpoint{3.932882in}{2.140718in}}{\pgfqpoint{3.936155in}{2.148618in}}{\pgfqpoint{3.936155in}{2.156854in}}%
\pgfpathcurveto{\pgfqpoint{3.936155in}{2.165091in}}{\pgfqpoint{3.932882in}{2.172991in}}{\pgfqpoint{3.927058in}{2.178814in}}%
\pgfpathcurveto{\pgfqpoint{3.921234in}{2.184638in}}{\pgfqpoint{3.913334in}{2.187911in}}{\pgfqpoint{3.905098in}{2.187911in}}%
\pgfpathcurveto{\pgfqpoint{3.896862in}{2.187911in}}{\pgfqpoint{3.888962in}{2.184638in}}{\pgfqpoint{3.883138in}{2.178814in}}%
\pgfpathcurveto{\pgfqpoint{3.877314in}{2.172991in}}{\pgfqpoint{3.874042in}{2.165091in}}{\pgfqpoint{3.874042in}{2.156854in}}%
\pgfpathcurveto{\pgfqpoint{3.874042in}{2.148618in}}{\pgfqpoint{3.877314in}{2.140718in}}{\pgfqpoint{3.883138in}{2.134894in}}%
\pgfpathcurveto{\pgfqpoint{3.888962in}{2.129070in}}{\pgfqpoint{3.896862in}{2.125798in}}{\pgfqpoint{3.905098in}{2.125798in}}%
\pgfpathclose%
\pgfusepath{stroke,fill}%
\end{pgfscope}%
\begin{pgfscope}%
\pgfpathrectangle{\pgfqpoint{3.793912in}{0.557870in}}{\pgfqpoint{2.446088in}{1.684734in}}%
\pgfusepath{clip}%
\pgfsetbuttcap%
\pgfsetroundjoin%
\definecolor{currentfill}{rgb}{0.298039,0.447059,0.690196}%
\pgfsetfillcolor{currentfill}%
\pgfsetlinewidth{1.003750pt}%
\definecolor{currentstroke}{rgb}{0.298039,0.447059,0.690196}%
\pgfsetstrokecolor{currentstroke}%
\pgfsetdash{}{0pt}%
\pgfpathmoveto{\pgfqpoint{3.905098in}{2.125798in}}%
\pgfpathcurveto{\pgfqpoint{3.913334in}{2.125798in}}{\pgfqpoint{3.921234in}{2.129070in}}{\pgfqpoint{3.927058in}{2.134894in}}%
\pgfpathcurveto{\pgfqpoint{3.932882in}{2.140718in}}{\pgfqpoint{3.936155in}{2.148618in}}{\pgfqpoint{3.936155in}{2.156854in}}%
\pgfpathcurveto{\pgfqpoint{3.936155in}{2.165091in}}{\pgfqpoint{3.932882in}{2.172991in}}{\pgfqpoint{3.927058in}{2.178814in}}%
\pgfpathcurveto{\pgfqpoint{3.921234in}{2.184638in}}{\pgfqpoint{3.913334in}{2.187911in}}{\pgfqpoint{3.905098in}{2.187911in}}%
\pgfpathcurveto{\pgfqpoint{3.896862in}{2.187911in}}{\pgfqpoint{3.888962in}{2.184638in}}{\pgfqpoint{3.883138in}{2.178814in}}%
\pgfpathcurveto{\pgfqpoint{3.877314in}{2.172991in}}{\pgfqpoint{3.874042in}{2.165091in}}{\pgfqpoint{3.874042in}{2.156854in}}%
\pgfpathcurveto{\pgfqpoint{3.874042in}{2.148618in}}{\pgfqpoint{3.877314in}{2.140718in}}{\pgfqpoint{3.883138in}{2.134894in}}%
\pgfpathcurveto{\pgfqpoint{3.888962in}{2.129070in}}{\pgfqpoint{3.896862in}{2.125798in}}{\pgfqpoint{3.905098in}{2.125798in}}%
\pgfpathclose%
\pgfusepath{stroke,fill}%
\end{pgfscope}%
\begin{pgfscope}%
\pgfpathrectangle{\pgfqpoint{3.793912in}{0.557870in}}{\pgfqpoint{2.446088in}{1.684734in}}%
\pgfusepath{clip}%
\pgfsetbuttcap%
\pgfsetroundjoin%
\definecolor{currentfill}{rgb}{0.298039,0.447059,0.690196}%
\pgfsetfillcolor{currentfill}%
\pgfsetlinewidth{1.003750pt}%
\definecolor{currentstroke}{rgb}{0.298039,0.447059,0.690196}%
\pgfsetstrokecolor{currentstroke}%
\pgfsetdash{}{0pt}%
\pgfpathmoveto{\pgfqpoint{3.905098in}{2.125798in}}%
\pgfpathcurveto{\pgfqpoint{3.913334in}{2.125798in}}{\pgfqpoint{3.921234in}{2.129070in}}{\pgfqpoint{3.927058in}{2.134894in}}%
\pgfpathcurveto{\pgfqpoint{3.932882in}{2.140718in}}{\pgfqpoint{3.936155in}{2.148618in}}{\pgfqpoint{3.936155in}{2.156854in}}%
\pgfpathcurveto{\pgfqpoint{3.936155in}{2.165091in}}{\pgfqpoint{3.932882in}{2.172991in}}{\pgfqpoint{3.927058in}{2.178814in}}%
\pgfpathcurveto{\pgfqpoint{3.921234in}{2.184638in}}{\pgfqpoint{3.913334in}{2.187911in}}{\pgfqpoint{3.905098in}{2.187911in}}%
\pgfpathcurveto{\pgfqpoint{3.896862in}{2.187911in}}{\pgfqpoint{3.888962in}{2.184638in}}{\pgfqpoint{3.883138in}{2.178814in}}%
\pgfpathcurveto{\pgfqpoint{3.877314in}{2.172991in}}{\pgfqpoint{3.874042in}{2.165091in}}{\pgfqpoint{3.874042in}{2.156854in}}%
\pgfpathcurveto{\pgfqpoint{3.874042in}{2.148618in}}{\pgfqpoint{3.877314in}{2.140718in}}{\pgfqpoint{3.883138in}{2.134894in}}%
\pgfpathcurveto{\pgfqpoint{3.888962in}{2.129070in}}{\pgfqpoint{3.896862in}{2.125798in}}{\pgfqpoint{3.905098in}{2.125798in}}%
\pgfpathclose%
\pgfusepath{stroke,fill}%
\end{pgfscope}%
\begin{pgfscope}%
\pgfpathrectangle{\pgfqpoint{3.793912in}{0.557870in}}{\pgfqpoint{2.446088in}{1.684734in}}%
\pgfusepath{clip}%
\pgfsetbuttcap%
\pgfsetroundjoin%
\definecolor{currentfill}{rgb}{0.298039,0.447059,0.690196}%
\pgfsetfillcolor{currentfill}%
\pgfsetlinewidth{1.003750pt}%
\definecolor{currentstroke}{rgb}{0.298039,0.447059,0.690196}%
\pgfsetstrokecolor{currentstroke}%
\pgfsetdash{}{0pt}%
\pgfpathmoveto{\pgfqpoint{3.905098in}{2.125798in}}%
\pgfpathcurveto{\pgfqpoint{3.913334in}{2.125798in}}{\pgfqpoint{3.921234in}{2.129070in}}{\pgfqpoint{3.927058in}{2.134894in}}%
\pgfpathcurveto{\pgfqpoint{3.932882in}{2.140718in}}{\pgfqpoint{3.936155in}{2.148618in}}{\pgfqpoint{3.936155in}{2.156854in}}%
\pgfpathcurveto{\pgfqpoint{3.936155in}{2.165091in}}{\pgfqpoint{3.932882in}{2.172991in}}{\pgfqpoint{3.927058in}{2.178814in}}%
\pgfpathcurveto{\pgfqpoint{3.921234in}{2.184638in}}{\pgfqpoint{3.913334in}{2.187911in}}{\pgfqpoint{3.905098in}{2.187911in}}%
\pgfpathcurveto{\pgfqpoint{3.896862in}{2.187911in}}{\pgfqpoint{3.888962in}{2.184638in}}{\pgfqpoint{3.883138in}{2.178814in}}%
\pgfpathcurveto{\pgfqpoint{3.877314in}{2.172991in}}{\pgfqpoint{3.874042in}{2.165091in}}{\pgfqpoint{3.874042in}{2.156854in}}%
\pgfpathcurveto{\pgfqpoint{3.874042in}{2.148618in}}{\pgfqpoint{3.877314in}{2.140718in}}{\pgfqpoint{3.883138in}{2.134894in}}%
\pgfpathcurveto{\pgfqpoint{3.888962in}{2.129070in}}{\pgfqpoint{3.896862in}{2.125798in}}{\pgfqpoint{3.905098in}{2.125798in}}%
\pgfpathclose%
\pgfusepath{stroke,fill}%
\end{pgfscope}%
\begin{pgfscope}%
\pgfpathrectangle{\pgfqpoint{3.793912in}{0.557870in}}{\pgfqpoint{2.446088in}{1.684734in}}%
\pgfusepath{clip}%
\pgfsetbuttcap%
\pgfsetroundjoin%
\definecolor{currentfill}{rgb}{0.298039,0.447059,0.690196}%
\pgfsetfillcolor{currentfill}%
\pgfsetlinewidth{1.003750pt}%
\definecolor{currentstroke}{rgb}{0.298039,0.447059,0.690196}%
\pgfsetstrokecolor{currentstroke}%
\pgfsetdash{}{0pt}%
\pgfpathmoveto{\pgfqpoint{4.978616in}{1.502162in}}%
\pgfpathcurveto{\pgfqpoint{4.986852in}{1.502162in}}{\pgfqpoint{4.994753in}{1.505434in}}{\pgfqpoint{5.000576in}{1.511258in}}%
\pgfpathcurveto{\pgfqpoint{5.006400in}{1.517082in}}{\pgfqpoint{5.009673in}{1.524982in}}{\pgfqpoint{5.009673in}{1.533218in}}%
\pgfpathcurveto{\pgfqpoint{5.009673in}{1.541455in}}{\pgfqpoint{5.006400in}{1.549355in}}{\pgfqpoint{5.000576in}{1.555179in}}%
\pgfpathcurveto{\pgfqpoint{4.994753in}{1.561003in}}{\pgfqpoint{4.986852in}{1.564275in}}{\pgfqpoint{4.978616in}{1.564275in}}%
\pgfpathcurveto{\pgfqpoint{4.970380in}{1.564275in}}{\pgfqpoint{4.962480in}{1.561003in}}{\pgfqpoint{4.956656in}{1.555179in}}%
\pgfpathcurveto{\pgfqpoint{4.950832in}{1.549355in}}{\pgfqpoint{4.947560in}{1.541455in}}{\pgfqpoint{4.947560in}{1.533218in}}%
\pgfpathcurveto{\pgfqpoint{4.947560in}{1.524982in}}{\pgfqpoint{4.950832in}{1.517082in}}{\pgfqpoint{4.956656in}{1.511258in}}%
\pgfpathcurveto{\pgfqpoint{4.962480in}{1.505434in}}{\pgfqpoint{4.970380in}{1.502162in}}{\pgfqpoint{4.978616in}{1.502162in}}%
\pgfpathclose%
\pgfusepath{stroke,fill}%
\end{pgfscope}%
\begin{pgfscope}%
\pgfpathrectangle{\pgfqpoint{3.793912in}{0.557870in}}{\pgfqpoint{2.446088in}{1.684734in}}%
\pgfusepath{clip}%
\pgfsetbuttcap%
\pgfsetroundjoin%
\definecolor{currentfill}{rgb}{0.298039,0.447059,0.690196}%
\pgfsetfillcolor{currentfill}%
\pgfsetlinewidth{1.003750pt}%
\definecolor{currentstroke}{rgb}{0.298039,0.447059,0.690196}%
\pgfsetstrokecolor{currentstroke}%
\pgfsetdash{}{0pt}%
\pgfpathmoveto{\pgfqpoint{3.905098in}{2.125798in}}%
\pgfpathcurveto{\pgfqpoint{3.913334in}{2.125798in}}{\pgfqpoint{3.921234in}{2.129070in}}{\pgfqpoint{3.927058in}{2.134894in}}%
\pgfpathcurveto{\pgfqpoint{3.932882in}{2.140718in}}{\pgfqpoint{3.936155in}{2.148618in}}{\pgfqpoint{3.936155in}{2.156854in}}%
\pgfpathcurveto{\pgfqpoint{3.936155in}{2.165091in}}{\pgfqpoint{3.932882in}{2.172991in}}{\pgfqpoint{3.927058in}{2.178814in}}%
\pgfpathcurveto{\pgfqpoint{3.921234in}{2.184638in}}{\pgfqpoint{3.913334in}{2.187911in}}{\pgfqpoint{3.905098in}{2.187911in}}%
\pgfpathcurveto{\pgfqpoint{3.896862in}{2.187911in}}{\pgfqpoint{3.888962in}{2.184638in}}{\pgfqpoint{3.883138in}{2.178814in}}%
\pgfpathcurveto{\pgfqpoint{3.877314in}{2.172991in}}{\pgfqpoint{3.874042in}{2.165091in}}{\pgfqpoint{3.874042in}{2.156854in}}%
\pgfpathcurveto{\pgfqpoint{3.874042in}{2.148618in}}{\pgfqpoint{3.877314in}{2.140718in}}{\pgfqpoint{3.883138in}{2.134894in}}%
\pgfpathcurveto{\pgfqpoint{3.888962in}{2.129070in}}{\pgfqpoint{3.896862in}{2.125798in}}{\pgfqpoint{3.905098in}{2.125798in}}%
\pgfpathclose%
\pgfusepath{stroke,fill}%
\end{pgfscope}%
\begin{pgfscope}%
\pgfpathrectangle{\pgfqpoint{3.793912in}{0.557870in}}{\pgfqpoint{2.446088in}{1.684734in}}%
\pgfusepath{clip}%
\pgfsetbuttcap%
\pgfsetroundjoin%
\definecolor{currentfill}{rgb}{0.298039,0.447059,0.690196}%
\pgfsetfillcolor{currentfill}%
\pgfsetlinewidth{1.003750pt}%
\definecolor{currentstroke}{rgb}{0.298039,0.447059,0.690196}%
\pgfsetstrokecolor{currentstroke}%
\pgfsetdash{}{0pt}%
\pgfpathmoveto{\pgfqpoint{4.901936in}{1.676413in}}%
\pgfpathcurveto{\pgfqpoint{4.910173in}{1.676413in}}{\pgfqpoint{4.918073in}{1.679685in}}{\pgfqpoint{4.923897in}{1.685509in}}%
\pgfpathcurveto{\pgfqpoint{4.929721in}{1.691333in}}{\pgfqpoint{4.932993in}{1.699233in}}{\pgfqpoint{4.932993in}{1.707470in}}%
\pgfpathcurveto{\pgfqpoint{4.932993in}{1.715706in}}{\pgfqpoint{4.929721in}{1.723606in}}{\pgfqpoint{4.923897in}{1.729430in}}%
\pgfpathcurveto{\pgfqpoint{4.918073in}{1.735254in}}{\pgfqpoint{4.910173in}{1.738526in}}{\pgfqpoint{4.901936in}{1.738526in}}%
\pgfpathcurveto{\pgfqpoint{4.893700in}{1.738526in}}{\pgfqpoint{4.885800in}{1.735254in}}{\pgfqpoint{4.879976in}{1.729430in}}%
\pgfpathcurveto{\pgfqpoint{4.874152in}{1.723606in}}{\pgfqpoint{4.870880in}{1.715706in}}{\pgfqpoint{4.870880in}{1.707470in}}%
\pgfpathcurveto{\pgfqpoint{4.870880in}{1.699233in}}{\pgfqpoint{4.874152in}{1.691333in}}{\pgfqpoint{4.879976in}{1.685509in}}%
\pgfpathcurveto{\pgfqpoint{4.885800in}{1.679685in}}{\pgfqpoint{4.893700in}{1.676413in}}{\pgfqpoint{4.901936in}{1.676413in}}%
\pgfpathclose%
\pgfusepath{stroke,fill}%
\end{pgfscope}%
\begin{pgfscope}%
\pgfpathrectangle{\pgfqpoint{3.793912in}{0.557870in}}{\pgfqpoint{2.446088in}{1.684734in}}%
\pgfusepath{clip}%
\pgfsetbuttcap%
\pgfsetroundjoin%
\definecolor{currentfill}{rgb}{0.298039,0.447059,0.690196}%
\pgfsetfillcolor{currentfill}%
\pgfsetlinewidth{1.003750pt}%
\definecolor{currentstroke}{rgb}{0.298039,0.447059,0.690196}%
\pgfsetstrokecolor{currentstroke}%
\pgfsetdash{}{0pt}%
\pgfpathmoveto{\pgfqpoint{3.905098in}{2.125798in}}%
\pgfpathcurveto{\pgfqpoint{3.913334in}{2.125798in}}{\pgfqpoint{3.921234in}{2.129070in}}{\pgfqpoint{3.927058in}{2.134894in}}%
\pgfpathcurveto{\pgfqpoint{3.932882in}{2.140718in}}{\pgfqpoint{3.936155in}{2.148618in}}{\pgfqpoint{3.936155in}{2.156854in}}%
\pgfpathcurveto{\pgfqpoint{3.936155in}{2.165091in}}{\pgfqpoint{3.932882in}{2.172991in}}{\pgfqpoint{3.927058in}{2.178814in}}%
\pgfpathcurveto{\pgfqpoint{3.921234in}{2.184638in}}{\pgfqpoint{3.913334in}{2.187911in}}{\pgfqpoint{3.905098in}{2.187911in}}%
\pgfpathcurveto{\pgfqpoint{3.896862in}{2.187911in}}{\pgfqpoint{3.888962in}{2.184638in}}{\pgfqpoint{3.883138in}{2.178814in}}%
\pgfpathcurveto{\pgfqpoint{3.877314in}{2.172991in}}{\pgfqpoint{3.874042in}{2.165091in}}{\pgfqpoint{3.874042in}{2.156854in}}%
\pgfpathcurveto{\pgfqpoint{3.874042in}{2.148618in}}{\pgfqpoint{3.877314in}{2.140718in}}{\pgfqpoint{3.883138in}{2.134894in}}%
\pgfpathcurveto{\pgfqpoint{3.888962in}{2.129070in}}{\pgfqpoint{3.896862in}{2.125798in}}{\pgfqpoint{3.905098in}{2.125798in}}%
\pgfpathclose%
\pgfusepath{stroke,fill}%
\end{pgfscope}%
\begin{pgfscope}%
\pgfpathrectangle{\pgfqpoint{3.793912in}{0.557870in}}{\pgfqpoint{2.446088in}{1.684734in}}%
\pgfusepath{clip}%
\pgfsetbuttcap%
\pgfsetroundjoin%
\definecolor{currentfill}{rgb}{0.298039,0.447059,0.690196}%
\pgfsetfillcolor{currentfill}%
\pgfsetlinewidth{1.003750pt}%
\definecolor{currentstroke}{rgb}{0.298039,0.447059,0.690196}%
\pgfsetstrokecolor{currentstroke}%
\pgfsetdash{}{0pt}%
\pgfpathmoveto{\pgfqpoint{3.905098in}{2.125798in}}%
\pgfpathcurveto{\pgfqpoint{3.913334in}{2.125798in}}{\pgfqpoint{3.921234in}{2.129070in}}{\pgfqpoint{3.927058in}{2.134894in}}%
\pgfpathcurveto{\pgfqpoint{3.932882in}{2.140718in}}{\pgfqpoint{3.936155in}{2.148618in}}{\pgfqpoint{3.936155in}{2.156854in}}%
\pgfpathcurveto{\pgfqpoint{3.936155in}{2.165091in}}{\pgfqpoint{3.932882in}{2.172991in}}{\pgfqpoint{3.927058in}{2.178814in}}%
\pgfpathcurveto{\pgfqpoint{3.921234in}{2.184638in}}{\pgfqpoint{3.913334in}{2.187911in}}{\pgfqpoint{3.905098in}{2.187911in}}%
\pgfpathcurveto{\pgfqpoint{3.896862in}{2.187911in}}{\pgfqpoint{3.888962in}{2.184638in}}{\pgfqpoint{3.883138in}{2.178814in}}%
\pgfpathcurveto{\pgfqpoint{3.877314in}{2.172991in}}{\pgfqpoint{3.874042in}{2.165091in}}{\pgfqpoint{3.874042in}{2.156854in}}%
\pgfpathcurveto{\pgfqpoint{3.874042in}{2.148618in}}{\pgfqpoint{3.877314in}{2.140718in}}{\pgfqpoint{3.883138in}{2.134894in}}%
\pgfpathcurveto{\pgfqpoint{3.888962in}{2.129070in}}{\pgfqpoint{3.896862in}{2.125798in}}{\pgfqpoint{3.905098in}{2.125798in}}%
\pgfpathclose%
\pgfusepath{stroke,fill}%
\end{pgfscope}%
\begin{pgfscope}%
\pgfpathrectangle{\pgfqpoint{3.793912in}{0.557870in}}{\pgfqpoint{2.446088in}{1.684734in}}%
\pgfusepath{clip}%
\pgfsetbuttcap%
\pgfsetroundjoin%
\definecolor{currentfill}{rgb}{0.298039,0.447059,0.690196}%
\pgfsetfillcolor{currentfill}%
\pgfsetlinewidth{1.003750pt}%
\definecolor{currentstroke}{rgb}{0.298039,0.447059,0.690196}%
\pgfsetstrokecolor{currentstroke}%
\pgfsetdash{}{0pt}%
\pgfpathmoveto{\pgfqpoint{3.905098in}{2.125798in}}%
\pgfpathcurveto{\pgfqpoint{3.913334in}{2.125798in}}{\pgfqpoint{3.921234in}{2.129070in}}{\pgfqpoint{3.927058in}{2.134894in}}%
\pgfpathcurveto{\pgfqpoint{3.932882in}{2.140718in}}{\pgfqpoint{3.936155in}{2.148618in}}{\pgfqpoint{3.936155in}{2.156854in}}%
\pgfpathcurveto{\pgfqpoint{3.936155in}{2.165091in}}{\pgfqpoint{3.932882in}{2.172991in}}{\pgfqpoint{3.927058in}{2.178814in}}%
\pgfpathcurveto{\pgfqpoint{3.921234in}{2.184638in}}{\pgfqpoint{3.913334in}{2.187911in}}{\pgfqpoint{3.905098in}{2.187911in}}%
\pgfpathcurveto{\pgfqpoint{3.896862in}{2.187911in}}{\pgfqpoint{3.888962in}{2.184638in}}{\pgfqpoint{3.883138in}{2.178814in}}%
\pgfpathcurveto{\pgfqpoint{3.877314in}{2.172991in}}{\pgfqpoint{3.874042in}{2.165091in}}{\pgfqpoint{3.874042in}{2.156854in}}%
\pgfpathcurveto{\pgfqpoint{3.874042in}{2.148618in}}{\pgfqpoint{3.877314in}{2.140718in}}{\pgfqpoint{3.883138in}{2.134894in}}%
\pgfpathcurveto{\pgfqpoint{3.888962in}{2.129070in}}{\pgfqpoint{3.896862in}{2.125798in}}{\pgfqpoint{3.905098in}{2.125798in}}%
\pgfpathclose%
\pgfusepath{stroke,fill}%
\end{pgfscope}%
\begin{pgfscope}%
\pgfpathrectangle{\pgfqpoint{3.793912in}{0.557870in}}{\pgfqpoint{2.446088in}{1.684734in}}%
\pgfusepath{clip}%
\pgfsetbuttcap%
\pgfsetroundjoin%
\definecolor{currentfill}{rgb}{0.298039,0.447059,0.690196}%
\pgfsetfillcolor{currentfill}%
\pgfsetlinewidth{1.003750pt}%
\definecolor{currentstroke}{rgb}{0.298039,0.447059,0.690196}%
\pgfsetstrokecolor{currentstroke}%
\pgfsetdash{}{0pt}%
\pgfpathmoveto{\pgfqpoint{5.898775in}{1.557189in}}%
\pgfpathcurveto{\pgfqpoint{5.907011in}{1.557189in}}{\pgfqpoint{5.914911in}{1.560461in}}{\pgfqpoint{5.920735in}{1.566285in}}%
\pgfpathcurveto{\pgfqpoint{5.926559in}{1.572109in}}{\pgfqpoint{5.929831in}{1.580009in}}{\pgfqpoint{5.929831in}{1.588245in}}%
\pgfpathcurveto{\pgfqpoint{5.929831in}{1.596481in}}{\pgfqpoint{5.926559in}{1.604381in}}{\pgfqpoint{5.920735in}{1.610205in}}%
\pgfpathcurveto{\pgfqpoint{5.914911in}{1.616029in}}{\pgfqpoint{5.907011in}{1.619302in}}{\pgfqpoint{5.898775in}{1.619302in}}%
\pgfpathcurveto{\pgfqpoint{5.890538in}{1.619302in}}{\pgfqpoint{5.882638in}{1.616029in}}{\pgfqpoint{5.876814in}{1.610205in}}%
\pgfpathcurveto{\pgfqpoint{5.870990in}{1.604381in}}{\pgfqpoint{5.867718in}{1.596481in}}{\pgfqpoint{5.867718in}{1.588245in}}%
\pgfpathcurveto{\pgfqpoint{5.867718in}{1.580009in}}{\pgfqpoint{5.870990in}{1.572109in}}{\pgfqpoint{5.876814in}{1.566285in}}%
\pgfpathcurveto{\pgfqpoint{5.882638in}{1.560461in}}{\pgfqpoint{5.890538in}{1.557189in}}{\pgfqpoint{5.898775in}{1.557189in}}%
\pgfpathclose%
\pgfusepath{stroke,fill}%
\end{pgfscope}%
\begin{pgfscope}%
\pgfpathrectangle{\pgfqpoint{3.793912in}{0.557870in}}{\pgfqpoint{2.446088in}{1.684734in}}%
\pgfusepath{clip}%
\pgfsetbuttcap%
\pgfsetroundjoin%
\definecolor{currentfill}{rgb}{0.298039,0.447059,0.690196}%
\pgfsetfillcolor{currentfill}%
\pgfsetlinewidth{1.003750pt}%
\definecolor{currentstroke}{rgb}{0.298039,0.447059,0.690196}%
\pgfsetstrokecolor{currentstroke}%
\pgfsetdash{}{0pt}%
\pgfpathmoveto{\pgfqpoint{4.058458in}{1.621386in}}%
\pgfpathcurveto{\pgfqpoint{4.066694in}{1.621386in}}{\pgfqpoint{4.074594in}{1.624659in}}{\pgfqpoint{4.080418in}{1.630483in}}%
\pgfpathcurveto{\pgfqpoint{4.086242in}{1.636307in}}{\pgfqpoint{4.089514in}{1.644207in}}{\pgfqpoint{4.089514in}{1.652443in}}%
\pgfpathcurveto{\pgfqpoint{4.089514in}{1.660679in}}{\pgfqpoint{4.086242in}{1.668579in}}{\pgfqpoint{4.080418in}{1.674403in}}%
\pgfpathcurveto{\pgfqpoint{4.074594in}{1.680227in}}{\pgfqpoint{4.066694in}{1.683499in}}{\pgfqpoint{4.058458in}{1.683499in}}%
\pgfpathcurveto{\pgfqpoint{4.050221in}{1.683499in}}{\pgfqpoint{4.042321in}{1.680227in}}{\pgfqpoint{4.036498in}{1.674403in}}%
\pgfpathcurveto{\pgfqpoint{4.030674in}{1.668579in}}{\pgfqpoint{4.027401in}{1.660679in}}{\pgfqpoint{4.027401in}{1.652443in}}%
\pgfpathcurveto{\pgfqpoint{4.027401in}{1.644207in}}{\pgfqpoint{4.030674in}{1.636307in}}{\pgfqpoint{4.036498in}{1.630483in}}%
\pgfpathcurveto{\pgfqpoint{4.042321in}{1.624659in}}{\pgfqpoint{4.050221in}{1.621386in}}{\pgfqpoint{4.058458in}{1.621386in}}%
\pgfpathclose%
\pgfusepath{stroke,fill}%
\end{pgfscope}%
\begin{pgfscope}%
\pgfpathrectangle{\pgfqpoint{3.793912in}{0.557870in}}{\pgfqpoint{2.446088in}{1.684734in}}%
\pgfusepath{clip}%
\pgfsetbuttcap%
\pgfsetroundjoin%
\definecolor{currentfill}{rgb}{0.298039,0.447059,0.690196}%
\pgfsetfillcolor{currentfill}%
\pgfsetlinewidth{1.003750pt}%
\definecolor{currentstroke}{rgb}{0.298039,0.447059,0.690196}%
\pgfsetstrokecolor{currentstroke}%
\pgfsetdash{}{0pt}%
\pgfpathmoveto{\pgfqpoint{5.898775in}{1.584702in}}%
\pgfpathcurveto{\pgfqpoint{5.907011in}{1.584702in}}{\pgfqpoint{5.914911in}{1.587974in}}{\pgfqpoint{5.920735in}{1.593798in}}%
\pgfpathcurveto{\pgfqpoint{5.926559in}{1.599622in}}{\pgfqpoint{5.929831in}{1.607522in}}{\pgfqpoint{5.929831in}{1.615758in}}%
\pgfpathcurveto{\pgfqpoint{5.929831in}{1.623995in}}{\pgfqpoint{5.926559in}{1.631895in}}{\pgfqpoint{5.920735in}{1.637719in}}%
\pgfpathcurveto{\pgfqpoint{5.914911in}{1.643543in}}{\pgfqpoint{5.907011in}{1.646815in}}{\pgfqpoint{5.898775in}{1.646815in}}%
\pgfpathcurveto{\pgfqpoint{5.890538in}{1.646815in}}{\pgfqpoint{5.882638in}{1.643543in}}{\pgfqpoint{5.876814in}{1.637719in}}%
\pgfpathcurveto{\pgfqpoint{5.870990in}{1.631895in}}{\pgfqpoint{5.867718in}{1.623995in}}{\pgfqpoint{5.867718in}{1.615758in}}%
\pgfpathcurveto{\pgfqpoint{5.867718in}{1.607522in}}{\pgfqpoint{5.870990in}{1.599622in}}{\pgfqpoint{5.876814in}{1.593798in}}%
\pgfpathcurveto{\pgfqpoint{5.882638in}{1.587974in}}{\pgfqpoint{5.890538in}{1.584702in}}{\pgfqpoint{5.898775in}{1.584702in}}%
\pgfpathclose%
\pgfusepath{stroke,fill}%
\end{pgfscope}%
\begin{pgfscope}%
\pgfpathrectangle{\pgfqpoint{3.793912in}{0.557870in}}{\pgfqpoint{2.446088in}{1.684734in}}%
\pgfusepath{clip}%
\pgfsetbuttcap%
\pgfsetroundjoin%
\definecolor{currentfill}{rgb}{0.298039,0.447059,0.690196}%
\pgfsetfillcolor{currentfill}%
\pgfsetlinewidth{1.003750pt}%
\definecolor{currentstroke}{rgb}{0.298039,0.447059,0.690196}%
\pgfsetstrokecolor{currentstroke}%
\pgfsetdash{}{0pt}%
\pgfpathmoveto{\pgfqpoint{4.058458in}{1.878178in}}%
\pgfpathcurveto{\pgfqpoint{4.066694in}{1.878178in}}{\pgfqpoint{4.074594in}{1.881450in}}{\pgfqpoint{4.080418in}{1.887274in}}%
\pgfpathcurveto{\pgfqpoint{4.086242in}{1.893098in}}{\pgfqpoint{4.089514in}{1.900998in}}{\pgfqpoint{4.089514in}{1.909234in}}%
\pgfpathcurveto{\pgfqpoint{4.089514in}{1.917470in}}{\pgfqpoint{4.086242in}{1.925370in}}{\pgfqpoint{4.080418in}{1.931194in}}%
\pgfpathcurveto{\pgfqpoint{4.074594in}{1.937018in}}{\pgfqpoint{4.066694in}{1.940291in}}{\pgfqpoint{4.058458in}{1.940291in}}%
\pgfpathcurveto{\pgfqpoint{4.050221in}{1.940291in}}{\pgfqpoint{4.042321in}{1.937018in}}{\pgfqpoint{4.036498in}{1.931194in}}%
\pgfpathcurveto{\pgfqpoint{4.030674in}{1.925370in}}{\pgfqpoint{4.027401in}{1.917470in}}{\pgfqpoint{4.027401in}{1.909234in}}%
\pgfpathcurveto{\pgfqpoint{4.027401in}{1.900998in}}{\pgfqpoint{4.030674in}{1.893098in}}{\pgfqpoint{4.036498in}{1.887274in}}%
\pgfpathcurveto{\pgfqpoint{4.042321in}{1.881450in}}{\pgfqpoint{4.050221in}{1.878178in}}{\pgfqpoint{4.058458in}{1.878178in}}%
\pgfpathclose%
\pgfusepath{stroke,fill}%
\end{pgfscope}%
\begin{pgfscope}%
\pgfpathrectangle{\pgfqpoint{3.793912in}{0.557870in}}{\pgfqpoint{2.446088in}{1.684734in}}%
\pgfusepath{clip}%
\pgfsetbuttcap%
\pgfsetroundjoin%
\definecolor{currentfill}{rgb}{0.298039,0.447059,0.690196}%
\pgfsetfillcolor{currentfill}%
\pgfsetlinewidth{1.003750pt}%
\definecolor{currentstroke}{rgb}{0.298039,0.447059,0.690196}%
\pgfsetstrokecolor{currentstroke}%
\pgfsetdash{}{0pt}%
\pgfpathmoveto{\pgfqpoint{4.058458in}{1.878178in}}%
\pgfpathcurveto{\pgfqpoint{4.066694in}{1.878178in}}{\pgfqpoint{4.074594in}{1.881450in}}{\pgfqpoint{4.080418in}{1.887274in}}%
\pgfpathcurveto{\pgfqpoint{4.086242in}{1.893098in}}{\pgfqpoint{4.089514in}{1.900998in}}{\pgfqpoint{4.089514in}{1.909234in}}%
\pgfpathcurveto{\pgfqpoint{4.089514in}{1.917470in}}{\pgfqpoint{4.086242in}{1.925370in}}{\pgfqpoint{4.080418in}{1.931194in}}%
\pgfpathcurveto{\pgfqpoint{4.074594in}{1.937018in}}{\pgfqpoint{4.066694in}{1.940291in}}{\pgfqpoint{4.058458in}{1.940291in}}%
\pgfpathcurveto{\pgfqpoint{4.050221in}{1.940291in}}{\pgfqpoint{4.042321in}{1.937018in}}{\pgfqpoint{4.036498in}{1.931194in}}%
\pgfpathcurveto{\pgfqpoint{4.030674in}{1.925370in}}{\pgfqpoint{4.027401in}{1.917470in}}{\pgfqpoint{4.027401in}{1.909234in}}%
\pgfpathcurveto{\pgfqpoint{4.027401in}{1.900998in}}{\pgfqpoint{4.030674in}{1.893098in}}{\pgfqpoint{4.036498in}{1.887274in}}%
\pgfpathcurveto{\pgfqpoint{4.042321in}{1.881450in}}{\pgfqpoint{4.050221in}{1.878178in}}{\pgfqpoint{4.058458in}{1.878178in}}%
\pgfpathclose%
\pgfusepath{stroke,fill}%
\end{pgfscope}%
\begin{pgfscope}%
\pgfpathrectangle{\pgfqpoint{3.793912in}{0.557870in}}{\pgfqpoint{2.446088in}{1.684734in}}%
\pgfusepath{clip}%
\pgfsetbuttcap%
\pgfsetroundjoin%
\definecolor{currentfill}{rgb}{0.298039,0.447059,0.690196}%
\pgfsetfillcolor{currentfill}%
\pgfsetlinewidth{1.003750pt}%
\definecolor{currentstroke}{rgb}{0.298039,0.447059,0.690196}%
\pgfsetstrokecolor{currentstroke}%
\pgfsetdash{}{0pt}%
\pgfpathmoveto{\pgfqpoint{5.975454in}{1.538846in}}%
\pgfpathcurveto{\pgfqpoint{5.983691in}{1.538846in}}{\pgfqpoint{5.991591in}{1.542119in}}{\pgfqpoint{5.997415in}{1.547943in}}%
\pgfpathcurveto{\pgfqpoint{6.003239in}{1.553767in}}{\pgfqpoint{6.006511in}{1.561667in}}{\pgfqpoint{6.006511in}{1.569903in}}%
\pgfpathcurveto{\pgfqpoint{6.006511in}{1.578139in}}{\pgfqpoint{6.003239in}{1.586039in}}{\pgfqpoint{5.997415in}{1.591863in}}%
\pgfpathcurveto{\pgfqpoint{5.991591in}{1.597687in}}{\pgfqpoint{5.983691in}{1.600959in}}{\pgfqpoint{5.975454in}{1.600959in}}%
\pgfpathcurveto{\pgfqpoint{5.967218in}{1.600959in}}{\pgfqpoint{5.959318in}{1.597687in}}{\pgfqpoint{5.953494in}{1.591863in}}%
\pgfpathcurveto{\pgfqpoint{5.947670in}{1.586039in}}{\pgfqpoint{5.944398in}{1.578139in}}{\pgfqpoint{5.944398in}{1.569903in}}%
\pgfpathcurveto{\pgfqpoint{5.944398in}{1.561667in}}{\pgfqpoint{5.947670in}{1.553767in}}{\pgfqpoint{5.953494in}{1.547943in}}%
\pgfpathcurveto{\pgfqpoint{5.959318in}{1.542119in}}{\pgfqpoint{5.967218in}{1.538846in}}{\pgfqpoint{5.975454in}{1.538846in}}%
\pgfpathclose%
\pgfusepath{stroke,fill}%
\end{pgfscope}%
\begin{pgfscope}%
\pgfpathrectangle{\pgfqpoint{3.793912in}{0.557870in}}{\pgfqpoint{2.446088in}{1.684734in}}%
\pgfusepath{clip}%
\pgfsetbuttcap%
\pgfsetroundjoin%
\definecolor{currentfill}{rgb}{0.298039,0.447059,0.690196}%
\pgfsetfillcolor{currentfill}%
\pgfsetlinewidth{1.003750pt}%
\definecolor{currentstroke}{rgb}{0.298039,0.447059,0.690196}%
\pgfsetstrokecolor{currentstroke}%
\pgfsetdash{}{0pt}%
\pgfpathmoveto{\pgfqpoint{5.975454in}{1.667242in}}%
\pgfpathcurveto{\pgfqpoint{5.983691in}{1.667242in}}{\pgfqpoint{5.991591in}{1.670514in}}{\pgfqpoint{5.997415in}{1.676338in}}%
\pgfpathcurveto{\pgfqpoint{6.003239in}{1.682162in}}{\pgfqpoint{6.006511in}{1.690062in}}{\pgfqpoint{6.006511in}{1.698298in}}%
\pgfpathcurveto{\pgfqpoint{6.006511in}{1.706535in}}{\pgfqpoint{6.003239in}{1.714435in}}{\pgfqpoint{5.997415in}{1.720259in}}%
\pgfpathcurveto{\pgfqpoint{5.991591in}{1.726083in}}{\pgfqpoint{5.983691in}{1.729355in}}{\pgfqpoint{5.975454in}{1.729355in}}%
\pgfpathcurveto{\pgfqpoint{5.967218in}{1.729355in}}{\pgfqpoint{5.959318in}{1.726083in}}{\pgfqpoint{5.953494in}{1.720259in}}%
\pgfpathcurveto{\pgfqpoint{5.947670in}{1.714435in}}{\pgfqpoint{5.944398in}{1.706535in}}{\pgfqpoint{5.944398in}{1.698298in}}%
\pgfpathcurveto{\pgfqpoint{5.944398in}{1.690062in}}{\pgfqpoint{5.947670in}{1.682162in}}{\pgfqpoint{5.953494in}{1.676338in}}%
\pgfpathcurveto{\pgfqpoint{5.959318in}{1.670514in}}{\pgfqpoint{5.967218in}{1.667242in}}{\pgfqpoint{5.975454in}{1.667242in}}%
\pgfpathclose%
\pgfusepath{stroke,fill}%
\end{pgfscope}%
\begin{pgfscope}%
\pgfpathrectangle{\pgfqpoint{3.793912in}{0.557870in}}{\pgfqpoint{2.446088in}{1.684734in}}%
\pgfusepath{clip}%
\pgfsetbuttcap%
\pgfsetroundjoin%
\definecolor{currentfill}{rgb}{0.298039,0.447059,0.690196}%
\pgfsetfillcolor{currentfill}%
\pgfsetlinewidth{1.003750pt}%
\definecolor{currentstroke}{rgb}{0.298039,0.447059,0.690196}%
\pgfsetstrokecolor{currentstroke}%
\pgfsetdash{}{0pt}%
\pgfpathmoveto{\pgfqpoint{5.975454in}{1.630558in}}%
\pgfpathcurveto{\pgfqpoint{5.983691in}{1.630558in}}{\pgfqpoint{5.991591in}{1.633830in}}{\pgfqpoint{5.997415in}{1.639654in}}%
\pgfpathcurveto{\pgfqpoint{6.003239in}{1.645478in}}{\pgfqpoint{6.006511in}{1.653378in}}{\pgfqpoint{6.006511in}{1.661614in}}%
\pgfpathcurveto{\pgfqpoint{6.006511in}{1.669850in}}{\pgfqpoint{6.003239in}{1.677750in}}{\pgfqpoint{5.997415in}{1.683574in}}%
\pgfpathcurveto{\pgfqpoint{5.991591in}{1.689398in}}{\pgfqpoint{5.983691in}{1.692671in}}{\pgfqpoint{5.975454in}{1.692671in}}%
\pgfpathcurveto{\pgfqpoint{5.967218in}{1.692671in}}{\pgfqpoint{5.959318in}{1.689398in}}{\pgfqpoint{5.953494in}{1.683574in}}%
\pgfpathcurveto{\pgfqpoint{5.947670in}{1.677750in}}{\pgfqpoint{5.944398in}{1.669850in}}{\pgfqpoint{5.944398in}{1.661614in}}%
\pgfpathcurveto{\pgfqpoint{5.944398in}{1.653378in}}{\pgfqpoint{5.947670in}{1.645478in}}{\pgfqpoint{5.953494in}{1.639654in}}%
\pgfpathcurveto{\pgfqpoint{5.959318in}{1.633830in}}{\pgfqpoint{5.967218in}{1.630558in}}{\pgfqpoint{5.975454in}{1.630558in}}%
\pgfpathclose%
\pgfusepath{stroke,fill}%
\end{pgfscope}%
\begin{pgfscope}%
\pgfpathrectangle{\pgfqpoint{3.793912in}{0.557870in}}{\pgfqpoint{2.446088in}{1.684734in}}%
\pgfusepath{clip}%
\pgfsetbuttcap%
\pgfsetroundjoin%
\definecolor{currentfill}{rgb}{0.298039,0.447059,0.690196}%
\pgfsetfillcolor{currentfill}%
\pgfsetlinewidth{1.003750pt}%
\definecolor{currentstroke}{rgb}{0.298039,0.447059,0.690196}%
\pgfsetstrokecolor{currentstroke}%
\pgfsetdash{}{0pt}%
\pgfpathmoveto{\pgfqpoint{5.975454in}{1.694755in}}%
\pgfpathcurveto{\pgfqpoint{5.983691in}{1.694755in}}{\pgfqpoint{5.991591in}{1.698028in}}{\pgfqpoint{5.997415in}{1.703852in}}%
\pgfpathcurveto{\pgfqpoint{6.003239in}{1.709675in}}{\pgfqpoint{6.006511in}{1.717576in}}{\pgfqpoint{6.006511in}{1.725812in}}%
\pgfpathcurveto{\pgfqpoint{6.006511in}{1.734048in}}{\pgfqpoint{6.003239in}{1.741948in}}{\pgfqpoint{5.997415in}{1.747772in}}%
\pgfpathcurveto{\pgfqpoint{5.991591in}{1.753596in}}{\pgfqpoint{5.983691in}{1.756868in}}{\pgfqpoint{5.975454in}{1.756868in}}%
\pgfpathcurveto{\pgfqpoint{5.967218in}{1.756868in}}{\pgfqpoint{5.959318in}{1.753596in}}{\pgfqpoint{5.953494in}{1.747772in}}%
\pgfpathcurveto{\pgfqpoint{5.947670in}{1.741948in}}{\pgfqpoint{5.944398in}{1.734048in}}{\pgfqpoint{5.944398in}{1.725812in}}%
\pgfpathcurveto{\pgfqpoint{5.944398in}{1.717576in}}{\pgfqpoint{5.947670in}{1.709675in}}{\pgfqpoint{5.953494in}{1.703852in}}%
\pgfpathcurveto{\pgfqpoint{5.959318in}{1.698028in}}{\pgfqpoint{5.967218in}{1.694755in}}{\pgfqpoint{5.975454in}{1.694755in}}%
\pgfpathclose%
\pgfusepath{stroke,fill}%
\end{pgfscope}%
\begin{pgfscope}%
\pgfpathrectangle{\pgfqpoint{3.793912in}{0.557870in}}{\pgfqpoint{2.446088in}{1.684734in}}%
\pgfusepath{clip}%
\pgfsetbuttcap%
\pgfsetroundjoin%
\definecolor{currentfill}{rgb}{0.298039,0.447059,0.690196}%
\pgfsetfillcolor{currentfill}%
\pgfsetlinewidth{1.003750pt}%
\definecolor{currentstroke}{rgb}{0.298039,0.447059,0.690196}%
\pgfsetstrokecolor{currentstroke}%
\pgfsetdash{}{0pt}%
\pgfpathmoveto{\pgfqpoint{5.975454in}{1.318740in}}%
\pgfpathcurveto{\pgfqpoint{5.983691in}{1.318740in}}{\pgfqpoint{5.991591in}{1.322012in}}{\pgfqpoint{5.997415in}{1.327836in}}%
\pgfpathcurveto{\pgfqpoint{6.003239in}{1.333660in}}{\pgfqpoint{6.006511in}{1.341560in}}{\pgfqpoint{6.006511in}{1.349796in}}%
\pgfpathcurveto{\pgfqpoint{6.006511in}{1.358032in}}{\pgfqpoint{6.003239in}{1.365932in}}{\pgfqpoint{5.997415in}{1.371756in}}%
\pgfpathcurveto{\pgfqpoint{5.991591in}{1.377580in}}{\pgfqpoint{5.983691in}{1.380853in}}{\pgfqpoint{5.975454in}{1.380853in}}%
\pgfpathcurveto{\pgfqpoint{5.967218in}{1.380853in}}{\pgfqpoint{5.959318in}{1.377580in}}{\pgfqpoint{5.953494in}{1.371756in}}%
\pgfpathcurveto{\pgfqpoint{5.947670in}{1.365932in}}{\pgfqpoint{5.944398in}{1.358032in}}{\pgfqpoint{5.944398in}{1.349796in}}%
\pgfpathcurveto{\pgfqpoint{5.944398in}{1.341560in}}{\pgfqpoint{5.947670in}{1.333660in}}{\pgfqpoint{5.953494in}{1.327836in}}%
\pgfpathcurveto{\pgfqpoint{5.959318in}{1.322012in}}{\pgfqpoint{5.967218in}{1.318740in}}{\pgfqpoint{5.975454in}{1.318740in}}%
\pgfpathclose%
\pgfusepath{stroke,fill}%
\end{pgfscope}%
\begin{pgfscope}%
\pgfpathrectangle{\pgfqpoint{3.793912in}{0.557870in}}{\pgfqpoint{2.446088in}{1.684734in}}%
\pgfusepath{clip}%
\pgfsetbuttcap%
\pgfsetroundjoin%
\definecolor{currentfill}{rgb}{0.298039,0.447059,0.690196}%
\pgfsetfillcolor{currentfill}%
\pgfsetlinewidth{1.003750pt}%
\definecolor{currentstroke}{rgb}{0.298039,0.447059,0.690196}%
\pgfsetstrokecolor{currentstroke}%
\pgfsetdash{}{0pt}%
\pgfpathmoveto{\pgfqpoint{5.975454in}{1.318740in}}%
\pgfpathcurveto{\pgfqpoint{5.983691in}{1.318740in}}{\pgfqpoint{5.991591in}{1.322012in}}{\pgfqpoint{5.997415in}{1.327836in}}%
\pgfpathcurveto{\pgfqpoint{6.003239in}{1.333660in}}{\pgfqpoint{6.006511in}{1.341560in}}{\pgfqpoint{6.006511in}{1.349796in}}%
\pgfpathcurveto{\pgfqpoint{6.006511in}{1.358032in}}{\pgfqpoint{6.003239in}{1.365932in}}{\pgfqpoint{5.997415in}{1.371756in}}%
\pgfpathcurveto{\pgfqpoint{5.991591in}{1.377580in}}{\pgfqpoint{5.983691in}{1.380853in}}{\pgfqpoint{5.975454in}{1.380853in}}%
\pgfpathcurveto{\pgfqpoint{5.967218in}{1.380853in}}{\pgfqpoint{5.959318in}{1.377580in}}{\pgfqpoint{5.953494in}{1.371756in}}%
\pgfpathcurveto{\pgfqpoint{5.947670in}{1.365932in}}{\pgfqpoint{5.944398in}{1.358032in}}{\pgfqpoint{5.944398in}{1.349796in}}%
\pgfpathcurveto{\pgfqpoint{5.944398in}{1.341560in}}{\pgfqpoint{5.947670in}{1.333660in}}{\pgfqpoint{5.953494in}{1.327836in}}%
\pgfpathcurveto{\pgfqpoint{5.959318in}{1.322012in}}{\pgfqpoint{5.967218in}{1.318740in}}{\pgfqpoint{5.975454in}{1.318740in}}%
\pgfpathclose%
\pgfusepath{stroke,fill}%
\end{pgfscope}%
\begin{pgfscope}%
\pgfpathrectangle{\pgfqpoint{3.793912in}{0.557870in}}{\pgfqpoint{2.446088in}{1.684734in}}%
\pgfusepath{clip}%
\pgfsetbuttcap%
\pgfsetroundjoin%
\definecolor{currentfill}{rgb}{0.298039,0.447059,0.690196}%
\pgfsetfillcolor{currentfill}%
\pgfsetlinewidth{1.003750pt}%
\definecolor{currentstroke}{rgb}{0.298039,0.447059,0.690196}%
\pgfsetstrokecolor{currentstroke}%
\pgfsetdash{}{0pt}%
\pgfpathmoveto{\pgfqpoint{3.905098in}{2.125798in}}%
\pgfpathcurveto{\pgfqpoint{3.913334in}{2.125798in}}{\pgfqpoint{3.921234in}{2.129070in}}{\pgfqpoint{3.927058in}{2.134894in}}%
\pgfpathcurveto{\pgfqpoint{3.932882in}{2.140718in}}{\pgfqpoint{3.936155in}{2.148618in}}{\pgfqpoint{3.936155in}{2.156854in}}%
\pgfpathcurveto{\pgfqpoint{3.936155in}{2.165091in}}{\pgfqpoint{3.932882in}{2.172991in}}{\pgfqpoint{3.927058in}{2.178814in}}%
\pgfpathcurveto{\pgfqpoint{3.921234in}{2.184638in}}{\pgfqpoint{3.913334in}{2.187911in}}{\pgfqpoint{3.905098in}{2.187911in}}%
\pgfpathcurveto{\pgfqpoint{3.896862in}{2.187911in}}{\pgfqpoint{3.888962in}{2.184638in}}{\pgfqpoint{3.883138in}{2.178814in}}%
\pgfpathcurveto{\pgfqpoint{3.877314in}{2.172991in}}{\pgfqpoint{3.874042in}{2.165091in}}{\pgfqpoint{3.874042in}{2.156854in}}%
\pgfpathcurveto{\pgfqpoint{3.874042in}{2.148618in}}{\pgfqpoint{3.877314in}{2.140718in}}{\pgfqpoint{3.883138in}{2.134894in}}%
\pgfpathcurveto{\pgfqpoint{3.888962in}{2.129070in}}{\pgfqpoint{3.896862in}{2.125798in}}{\pgfqpoint{3.905098in}{2.125798in}}%
\pgfpathclose%
\pgfusepath{stroke,fill}%
\end{pgfscope}%
\begin{pgfscope}%
\pgfpathrectangle{\pgfqpoint{3.793912in}{0.557870in}}{\pgfqpoint{2.446088in}{1.684734in}}%
\pgfusepath{clip}%
\pgfsetbuttcap%
\pgfsetroundjoin%
\definecolor{currentfill}{rgb}{0.298039,0.447059,0.690196}%
\pgfsetfillcolor{currentfill}%
\pgfsetlinewidth{1.003750pt}%
\definecolor{currentstroke}{rgb}{0.298039,0.447059,0.690196}%
\pgfsetstrokecolor{currentstroke}%
\pgfsetdash{}{0pt}%
\pgfpathmoveto{\pgfqpoint{3.905098in}{2.125798in}}%
\pgfpathcurveto{\pgfqpoint{3.913334in}{2.125798in}}{\pgfqpoint{3.921234in}{2.129070in}}{\pgfqpoint{3.927058in}{2.134894in}}%
\pgfpathcurveto{\pgfqpoint{3.932882in}{2.140718in}}{\pgfqpoint{3.936155in}{2.148618in}}{\pgfqpoint{3.936155in}{2.156854in}}%
\pgfpathcurveto{\pgfqpoint{3.936155in}{2.165091in}}{\pgfqpoint{3.932882in}{2.172991in}}{\pgfqpoint{3.927058in}{2.178814in}}%
\pgfpathcurveto{\pgfqpoint{3.921234in}{2.184638in}}{\pgfqpoint{3.913334in}{2.187911in}}{\pgfqpoint{3.905098in}{2.187911in}}%
\pgfpathcurveto{\pgfqpoint{3.896862in}{2.187911in}}{\pgfqpoint{3.888962in}{2.184638in}}{\pgfqpoint{3.883138in}{2.178814in}}%
\pgfpathcurveto{\pgfqpoint{3.877314in}{2.172991in}}{\pgfqpoint{3.874042in}{2.165091in}}{\pgfqpoint{3.874042in}{2.156854in}}%
\pgfpathcurveto{\pgfqpoint{3.874042in}{2.148618in}}{\pgfqpoint{3.877314in}{2.140718in}}{\pgfqpoint{3.883138in}{2.134894in}}%
\pgfpathcurveto{\pgfqpoint{3.888962in}{2.129070in}}{\pgfqpoint{3.896862in}{2.125798in}}{\pgfqpoint{3.905098in}{2.125798in}}%
\pgfpathclose%
\pgfusepath{stroke,fill}%
\end{pgfscope}%
\begin{pgfscope}%
\pgfpathrectangle{\pgfqpoint{3.793912in}{0.557870in}}{\pgfqpoint{2.446088in}{1.684734in}}%
\pgfusepath{clip}%
\pgfsetbuttcap%
\pgfsetroundjoin%
\definecolor{currentfill}{rgb}{0.298039,0.447059,0.690196}%
\pgfsetfillcolor{currentfill}%
\pgfsetlinewidth{1.003750pt}%
\definecolor{currentstroke}{rgb}{0.298039,0.447059,0.690196}%
\pgfsetstrokecolor{currentstroke}%
\pgfsetdash{}{0pt}%
\pgfpathmoveto{\pgfqpoint{3.905098in}{2.125798in}}%
\pgfpathcurveto{\pgfqpoint{3.913334in}{2.125798in}}{\pgfqpoint{3.921234in}{2.129070in}}{\pgfqpoint{3.927058in}{2.134894in}}%
\pgfpathcurveto{\pgfqpoint{3.932882in}{2.140718in}}{\pgfqpoint{3.936155in}{2.148618in}}{\pgfqpoint{3.936155in}{2.156854in}}%
\pgfpathcurveto{\pgfqpoint{3.936155in}{2.165091in}}{\pgfqpoint{3.932882in}{2.172991in}}{\pgfqpoint{3.927058in}{2.178814in}}%
\pgfpathcurveto{\pgfqpoint{3.921234in}{2.184638in}}{\pgfqpoint{3.913334in}{2.187911in}}{\pgfqpoint{3.905098in}{2.187911in}}%
\pgfpathcurveto{\pgfqpoint{3.896862in}{2.187911in}}{\pgfqpoint{3.888962in}{2.184638in}}{\pgfqpoint{3.883138in}{2.178814in}}%
\pgfpathcurveto{\pgfqpoint{3.877314in}{2.172991in}}{\pgfqpoint{3.874042in}{2.165091in}}{\pgfqpoint{3.874042in}{2.156854in}}%
\pgfpathcurveto{\pgfqpoint{3.874042in}{2.148618in}}{\pgfqpoint{3.877314in}{2.140718in}}{\pgfqpoint{3.883138in}{2.134894in}}%
\pgfpathcurveto{\pgfqpoint{3.888962in}{2.129070in}}{\pgfqpoint{3.896862in}{2.125798in}}{\pgfqpoint{3.905098in}{2.125798in}}%
\pgfpathclose%
\pgfusepath{stroke,fill}%
\end{pgfscope}%
\begin{pgfscope}%
\pgfpathrectangle{\pgfqpoint{3.793912in}{0.557870in}}{\pgfqpoint{2.446088in}{1.684734in}}%
\pgfusepath{clip}%
\pgfsetbuttcap%
\pgfsetroundjoin%
\definecolor{currentfill}{rgb}{0.298039,0.447059,0.690196}%
\pgfsetfillcolor{currentfill}%
\pgfsetlinewidth{1.003750pt}%
\definecolor{currentstroke}{rgb}{0.298039,0.447059,0.690196}%
\pgfsetstrokecolor{currentstroke}%
\pgfsetdash{}{0pt}%
\pgfpathmoveto{\pgfqpoint{4.518537in}{1.492991in}}%
\pgfpathcurveto{\pgfqpoint{4.526773in}{1.492991in}}{\pgfqpoint{4.534673in}{1.496263in}}{\pgfqpoint{4.540497in}{1.502087in}}%
\pgfpathcurveto{\pgfqpoint{4.546321in}{1.507911in}}{\pgfqpoint{4.549593in}{1.515811in}}{\pgfqpoint{4.549593in}{1.524047in}}%
\pgfpathcurveto{\pgfqpoint{4.549593in}{1.532284in}}{\pgfqpoint{4.546321in}{1.540184in}}{\pgfqpoint{4.540497in}{1.546008in}}%
\pgfpathcurveto{\pgfqpoint{4.534673in}{1.551831in}}{\pgfqpoint{4.526773in}{1.555104in}}{\pgfqpoint{4.518537in}{1.555104in}}%
\pgfpathcurveto{\pgfqpoint{4.510301in}{1.555104in}}{\pgfqpoint{4.502401in}{1.551831in}}{\pgfqpoint{4.496577in}{1.546008in}}%
\pgfpathcurveto{\pgfqpoint{4.490753in}{1.540184in}}{\pgfqpoint{4.487480in}{1.532284in}}{\pgfqpoint{4.487480in}{1.524047in}}%
\pgfpathcurveto{\pgfqpoint{4.487480in}{1.515811in}}{\pgfqpoint{4.490753in}{1.507911in}}{\pgfqpoint{4.496577in}{1.502087in}}%
\pgfpathcurveto{\pgfqpoint{4.502401in}{1.496263in}}{\pgfqpoint{4.510301in}{1.492991in}}{\pgfqpoint{4.518537in}{1.492991in}}%
\pgfpathclose%
\pgfusepath{stroke,fill}%
\end{pgfscope}%
\begin{pgfscope}%
\pgfpathrectangle{\pgfqpoint{3.793912in}{0.557870in}}{\pgfqpoint{2.446088in}{1.684734in}}%
\pgfusepath{clip}%
\pgfsetbuttcap%
\pgfsetroundjoin%
\definecolor{currentfill}{rgb}{0.298039,0.447059,0.690196}%
\pgfsetfillcolor{currentfill}%
\pgfsetlinewidth{1.003750pt}%
\definecolor{currentstroke}{rgb}{0.298039,0.447059,0.690196}%
\pgfsetstrokecolor{currentstroke}%
\pgfsetdash{}{0pt}%
\pgfpathmoveto{\pgfqpoint{3.905098in}{2.125798in}}%
\pgfpathcurveto{\pgfqpoint{3.913334in}{2.125798in}}{\pgfqpoint{3.921234in}{2.129070in}}{\pgfqpoint{3.927058in}{2.134894in}}%
\pgfpathcurveto{\pgfqpoint{3.932882in}{2.140718in}}{\pgfqpoint{3.936155in}{2.148618in}}{\pgfqpoint{3.936155in}{2.156854in}}%
\pgfpathcurveto{\pgfqpoint{3.936155in}{2.165091in}}{\pgfqpoint{3.932882in}{2.172991in}}{\pgfqpoint{3.927058in}{2.178814in}}%
\pgfpathcurveto{\pgfqpoint{3.921234in}{2.184638in}}{\pgfqpoint{3.913334in}{2.187911in}}{\pgfqpoint{3.905098in}{2.187911in}}%
\pgfpathcurveto{\pgfqpoint{3.896862in}{2.187911in}}{\pgfqpoint{3.888962in}{2.184638in}}{\pgfqpoint{3.883138in}{2.178814in}}%
\pgfpathcurveto{\pgfqpoint{3.877314in}{2.172991in}}{\pgfqpoint{3.874042in}{2.165091in}}{\pgfqpoint{3.874042in}{2.156854in}}%
\pgfpathcurveto{\pgfqpoint{3.874042in}{2.148618in}}{\pgfqpoint{3.877314in}{2.140718in}}{\pgfqpoint{3.883138in}{2.134894in}}%
\pgfpathcurveto{\pgfqpoint{3.888962in}{2.129070in}}{\pgfqpoint{3.896862in}{2.125798in}}{\pgfqpoint{3.905098in}{2.125798in}}%
\pgfpathclose%
\pgfusepath{stroke,fill}%
\end{pgfscope}%
\begin{pgfscope}%
\pgfpathrectangle{\pgfqpoint{3.793912in}{0.557870in}}{\pgfqpoint{2.446088in}{1.684734in}}%
\pgfusepath{clip}%
\pgfsetbuttcap%
\pgfsetroundjoin%
\definecolor{currentfill}{rgb}{0.298039,0.447059,0.690196}%
\pgfsetfillcolor{currentfill}%
\pgfsetlinewidth{1.003750pt}%
\definecolor{currentstroke}{rgb}{0.298039,0.447059,0.690196}%
\pgfsetstrokecolor{currentstroke}%
\pgfsetdash{}{0pt}%
\pgfpathmoveto{\pgfqpoint{3.905098in}{2.125798in}}%
\pgfpathcurveto{\pgfqpoint{3.913334in}{2.125798in}}{\pgfqpoint{3.921234in}{2.129070in}}{\pgfqpoint{3.927058in}{2.134894in}}%
\pgfpathcurveto{\pgfqpoint{3.932882in}{2.140718in}}{\pgfqpoint{3.936155in}{2.148618in}}{\pgfqpoint{3.936155in}{2.156854in}}%
\pgfpathcurveto{\pgfqpoint{3.936155in}{2.165091in}}{\pgfqpoint{3.932882in}{2.172991in}}{\pgfqpoint{3.927058in}{2.178814in}}%
\pgfpathcurveto{\pgfqpoint{3.921234in}{2.184638in}}{\pgfqpoint{3.913334in}{2.187911in}}{\pgfqpoint{3.905098in}{2.187911in}}%
\pgfpathcurveto{\pgfqpoint{3.896862in}{2.187911in}}{\pgfqpoint{3.888962in}{2.184638in}}{\pgfqpoint{3.883138in}{2.178814in}}%
\pgfpathcurveto{\pgfqpoint{3.877314in}{2.172991in}}{\pgfqpoint{3.874042in}{2.165091in}}{\pgfqpoint{3.874042in}{2.156854in}}%
\pgfpathcurveto{\pgfqpoint{3.874042in}{2.148618in}}{\pgfqpoint{3.877314in}{2.140718in}}{\pgfqpoint{3.883138in}{2.134894in}}%
\pgfpathcurveto{\pgfqpoint{3.888962in}{2.129070in}}{\pgfqpoint{3.896862in}{2.125798in}}{\pgfqpoint{3.905098in}{2.125798in}}%
\pgfpathclose%
\pgfusepath{stroke,fill}%
\end{pgfscope}%
\begin{pgfscope}%
\pgfpathrectangle{\pgfqpoint{3.793912in}{0.557870in}}{\pgfqpoint{2.446088in}{1.684734in}}%
\pgfusepath{clip}%
\pgfsetbuttcap%
\pgfsetroundjoin%
\definecolor{currentfill}{rgb}{0.298039,0.447059,0.690196}%
\pgfsetfillcolor{currentfill}%
\pgfsetlinewidth{1.003750pt}%
\definecolor{currentstroke}{rgb}{0.298039,0.447059,0.690196}%
\pgfsetstrokecolor{currentstroke}%
\pgfsetdash{}{0pt}%
\pgfpathmoveto{\pgfqpoint{3.905098in}{2.125798in}}%
\pgfpathcurveto{\pgfqpoint{3.913334in}{2.125798in}}{\pgfqpoint{3.921234in}{2.129070in}}{\pgfqpoint{3.927058in}{2.134894in}}%
\pgfpathcurveto{\pgfqpoint{3.932882in}{2.140718in}}{\pgfqpoint{3.936155in}{2.148618in}}{\pgfqpoint{3.936155in}{2.156854in}}%
\pgfpathcurveto{\pgfqpoint{3.936155in}{2.165091in}}{\pgfqpoint{3.932882in}{2.172991in}}{\pgfqpoint{3.927058in}{2.178814in}}%
\pgfpathcurveto{\pgfqpoint{3.921234in}{2.184638in}}{\pgfqpoint{3.913334in}{2.187911in}}{\pgfqpoint{3.905098in}{2.187911in}}%
\pgfpathcurveto{\pgfqpoint{3.896862in}{2.187911in}}{\pgfqpoint{3.888962in}{2.184638in}}{\pgfqpoint{3.883138in}{2.178814in}}%
\pgfpathcurveto{\pgfqpoint{3.877314in}{2.172991in}}{\pgfqpoint{3.874042in}{2.165091in}}{\pgfqpoint{3.874042in}{2.156854in}}%
\pgfpathcurveto{\pgfqpoint{3.874042in}{2.148618in}}{\pgfqpoint{3.877314in}{2.140718in}}{\pgfqpoint{3.883138in}{2.134894in}}%
\pgfpathcurveto{\pgfqpoint{3.888962in}{2.129070in}}{\pgfqpoint{3.896862in}{2.125798in}}{\pgfqpoint{3.905098in}{2.125798in}}%
\pgfpathclose%
\pgfusepath{stroke,fill}%
\end{pgfscope}%
\begin{pgfscope}%
\pgfpathrectangle{\pgfqpoint{3.793912in}{0.557870in}}{\pgfqpoint{2.446088in}{1.684734in}}%
\pgfusepath{clip}%
\pgfsetbuttcap%
\pgfsetroundjoin%
\definecolor{currentfill}{rgb}{0.298039,0.447059,0.690196}%
\pgfsetfillcolor{currentfill}%
\pgfsetlinewidth{1.003750pt}%
\definecolor{currentstroke}{rgb}{0.298039,0.447059,0.690196}%
\pgfsetstrokecolor{currentstroke}%
\pgfsetdash{}{0pt}%
\pgfpathmoveto{\pgfqpoint{3.905098in}{2.125798in}}%
\pgfpathcurveto{\pgfqpoint{3.913334in}{2.125798in}}{\pgfqpoint{3.921234in}{2.129070in}}{\pgfqpoint{3.927058in}{2.134894in}}%
\pgfpathcurveto{\pgfqpoint{3.932882in}{2.140718in}}{\pgfqpoint{3.936155in}{2.148618in}}{\pgfqpoint{3.936155in}{2.156854in}}%
\pgfpathcurveto{\pgfqpoint{3.936155in}{2.165091in}}{\pgfqpoint{3.932882in}{2.172991in}}{\pgfqpoint{3.927058in}{2.178814in}}%
\pgfpathcurveto{\pgfqpoint{3.921234in}{2.184638in}}{\pgfqpoint{3.913334in}{2.187911in}}{\pgfqpoint{3.905098in}{2.187911in}}%
\pgfpathcurveto{\pgfqpoint{3.896862in}{2.187911in}}{\pgfqpoint{3.888962in}{2.184638in}}{\pgfqpoint{3.883138in}{2.178814in}}%
\pgfpathcurveto{\pgfqpoint{3.877314in}{2.172991in}}{\pgfqpoint{3.874042in}{2.165091in}}{\pgfqpoint{3.874042in}{2.156854in}}%
\pgfpathcurveto{\pgfqpoint{3.874042in}{2.148618in}}{\pgfqpoint{3.877314in}{2.140718in}}{\pgfqpoint{3.883138in}{2.134894in}}%
\pgfpathcurveto{\pgfqpoint{3.888962in}{2.129070in}}{\pgfqpoint{3.896862in}{2.125798in}}{\pgfqpoint{3.905098in}{2.125798in}}%
\pgfpathclose%
\pgfusepath{stroke,fill}%
\end{pgfscope}%
\begin{pgfscope}%
\pgfpathrectangle{\pgfqpoint{3.793912in}{0.557870in}}{\pgfqpoint{2.446088in}{1.684734in}}%
\pgfusepath{clip}%
\pgfsetbuttcap%
\pgfsetroundjoin%
\definecolor{currentfill}{rgb}{0.298039,0.447059,0.690196}%
\pgfsetfillcolor{currentfill}%
\pgfsetlinewidth{1.003750pt}%
\definecolor{currentstroke}{rgb}{0.298039,0.447059,0.690196}%
\pgfsetstrokecolor{currentstroke}%
\pgfsetdash{}{0pt}%
\pgfpathmoveto{\pgfqpoint{3.905098in}{2.125798in}}%
\pgfpathcurveto{\pgfqpoint{3.913334in}{2.125798in}}{\pgfqpoint{3.921234in}{2.129070in}}{\pgfqpoint{3.927058in}{2.134894in}}%
\pgfpathcurveto{\pgfqpoint{3.932882in}{2.140718in}}{\pgfqpoint{3.936155in}{2.148618in}}{\pgfqpoint{3.936155in}{2.156854in}}%
\pgfpathcurveto{\pgfqpoint{3.936155in}{2.165091in}}{\pgfqpoint{3.932882in}{2.172991in}}{\pgfqpoint{3.927058in}{2.178814in}}%
\pgfpathcurveto{\pgfqpoint{3.921234in}{2.184638in}}{\pgfqpoint{3.913334in}{2.187911in}}{\pgfqpoint{3.905098in}{2.187911in}}%
\pgfpathcurveto{\pgfqpoint{3.896862in}{2.187911in}}{\pgfqpoint{3.888962in}{2.184638in}}{\pgfqpoint{3.883138in}{2.178814in}}%
\pgfpathcurveto{\pgfqpoint{3.877314in}{2.172991in}}{\pgfqpoint{3.874042in}{2.165091in}}{\pgfqpoint{3.874042in}{2.156854in}}%
\pgfpathcurveto{\pgfqpoint{3.874042in}{2.148618in}}{\pgfqpoint{3.877314in}{2.140718in}}{\pgfqpoint{3.883138in}{2.134894in}}%
\pgfpathcurveto{\pgfqpoint{3.888962in}{2.129070in}}{\pgfqpoint{3.896862in}{2.125798in}}{\pgfqpoint{3.905098in}{2.125798in}}%
\pgfpathclose%
\pgfusepath{stroke,fill}%
\end{pgfscope}%
\begin{pgfscope}%
\pgfpathrectangle{\pgfqpoint{3.793912in}{0.557870in}}{\pgfqpoint{2.446088in}{1.684734in}}%
\pgfusepath{clip}%
\pgfsetbuttcap%
\pgfsetroundjoin%
\definecolor{currentfill}{rgb}{0.298039,0.447059,0.690196}%
\pgfsetfillcolor{currentfill}%
\pgfsetlinewidth{1.003750pt}%
\definecolor{currentstroke}{rgb}{0.298039,0.447059,0.690196}%
\pgfsetstrokecolor{currentstroke}%
\pgfsetdash{}{0pt}%
\pgfpathmoveto{\pgfqpoint{3.905098in}{2.125798in}}%
\pgfpathcurveto{\pgfqpoint{3.913334in}{2.125798in}}{\pgfqpoint{3.921234in}{2.129070in}}{\pgfqpoint{3.927058in}{2.134894in}}%
\pgfpathcurveto{\pgfqpoint{3.932882in}{2.140718in}}{\pgfqpoint{3.936155in}{2.148618in}}{\pgfqpoint{3.936155in}{2.156854in}}%
\pgfpathcurveto{\pgfqpoint{3.936155in}{2.165091in}}{\pgfqpoint{3.932882in}{2.172991in}}{\pgfqpoint{3.927058in}{2.178814in}}%
\pgfpathcurveto{\pgfqpoint{3.921234in}{2.184638in}}{\pgfqpoint{3.913334in}{2.187911in}}{\pgfqpoint{3.905098in}{2.187911in}}%
\pgfpathcurveto{\pgfqpoint{3.896862in}{2.187911in}}{\pgfqpoint{3.888962in}{2.184638in}}{\pgfqpoint{3.883138in}{2.178814in}}%
\pgfpathcurveto{\pgfqpoint{3.877314in}{2.172991in}}{\pgfqpoint{3.874042in}{2.165091in}}{\pgfqpoint{3.874042in}{2.156854in}}%
\pgfpathcurveto{\pgfqpoint{3.874042in}{2.148618in}}{\pgfqpoint{3.877314in}{2.140718in}}{\pgfqpoint{3.883138in}{2.134894in}}%
\pgfpathcurveto{\pgfqpoint{3.888962in}{2.129070in}}{\pgfqpoint{3.896862in}{2.125798in}}{\pgfqpoint{3.905098in}{2.125798in}}%
\pgfpathclose%
\pgfusepath{stroke,fill}%
\end{pgfscope}%
\begin{pgfscope}%
\pgfpathrectangle{\pgfqpoint{3.793912in}{0.557870in}}{\pgfqpoint{2.446088in}{1.684734in}}%
\pgfusepath{clip}%
\pgfsetbuttcap%
\pgfsetroundjoin%
\definecolor{currentfill}{rgb}{0.298039,0.447059,0.690196}%
\pgfsetfillcolor{currentfill}%
\pgfsetlinewidth{1.003750pt}%
\definecolor{currentstroke}{rgb}{0.298039,0.447059,0.690196}%
\pgfsetstrokecolor{currentstroke}%
\pgfsetdash{}{0pt}%
\pgfpathmoveto{\pgfqpoint{4.595217in}{1.566360in}}%
\pgfpathcurveto{\pgfqpoint{4.603453in}{1.566360in}}{\pgfqpoint{4.611353in}{1.569632in}}{\pgfqpoint{4.617177in}{1.575456in}}%
\pgfpathcurveto{\pgfqpoint{4.623001in}{1.581280in}}{\pgfqpoint{4.626273in}{1.589180in}}{\pgfqpoint{4.626273in}{1.597416in}}%
\pgfpathcurveto{\pgfqpoint{4.626273in}{1.605652in}}{\pgfqpoint{4.623001in}{1.613553in}}{\pgfqpoint{4.617177in}{1.619376in}}%
\pgfpathcurveto{\pgfqpoint{4.611353in}{1.625200in}}{\pgfqpoint{4.603453in}{1.628473in}}{\pgfqpoint{4.595217in}{1.628473in}}%
\pgfpathcurveto{\pgfqpoint{4.586981in}{1.628473in}}{\pgfqpoint{4.579081in}{1.625200in}}{\pgfqpoint{4.573257in}{1.619376in}}%
\pgfpathcurveto{\pgfqpoint{4.567433in}{1.613553in}}{\pgfqpoint{4.564160in}{1.605652in}}{\pgfqpoint{4.564160in}{1.597416in}}%
\pgfpathcurveto{\pgfqpoint{4.564160in}{1.589180in}}{\pgfqpoint{4.567433in}{1.581280in}}{\pgfqpoint{4.573257in}{1.575456in}}%
\pgfpathcurveto{\pgfqpoint{4.579081in}{1.569632in}}{\pgfqpoint{4.586981in}{1.566360in}}{\pgfqpoint{4.595217in}{1.566360in}}%
\pgfpathclose%
\pgfusepath{stroke,fill}%
\end{pgfscope}%
\begin{pgfscope}%
\pgfpathrectangle{\pgfqpoint{3.793912in}{0.557870in}}{\pgfqpoint{2.446088in}{1.684734in}}%
\pgfusepath{clip}%
\pgfsetbuttcap%
\pgfsetroundjoin%
\definecolor{currentfill}{rgb}{0.298039,0.447059,0.690196}%
\pgfsetfillcolor{currentfill}%
\pgfsetlinewidth{1.003750pt}%
\definecolor{currentstroke}{rgb}{0.298039,0.447059,0.690196}%
\pgfsetstrokecolor{currentstroke}%
\pgfsetdash{}{0pt}%
\pgfpathmoveto{\pgfqpoint{3.905098in}{2.125798in}}%
\pgfpathcurveto{\pgfqpoint{3.913334in}{2.125798in}}{\pgfqpoint{3.921234in}{2.129070in}}{\pgfqpoint{3.927058in}{2.134894in}}%
\pgfpathcurveto{\pgfqpoint{3.932882in}{2.140718in}}{\pgfqpoint{3.936155in}{2.148618in}}{\pgfqpoint{3.936155in}{2.156854in}}%
\pgfpathcurveto{\pgfqpoint{3.936155in}{2.165091in}}{\pgfqpoint{3.932882in}{2.172991in}}{\pgfqpoint{3.927058in}{2.178814in}}%
\pgfpathcurveto{\pgfqpoint{3.921234in}{2.184638in}}{\pgfqpoint{3.913334in}{2.187911in}}{\pgfqpoint{3.905098in}{2.187911in}}%
\pgfpathcurveto{\pgfqpoint{3.896862in}{2.187911in}}{\pgfqpoint{3.888962in}{2.184638in}}{\pgfqpoint{3.883138in}{2.178814in}}%
\pgfpathcurveto{\pgfqpoint{3.877314in}{2.172991in}}{\pgfqpoint{3.874042in}{2.165091in}}{\pgfqpoint{3.874042in}{2.156854in}}%
\pgfpathcurveto{\pgfqpoint{3.874042in}{2.148618in}}{\pgfqpoint{3.877314in}{2.140718in}}{\pgfqpoint{3.883138in}{2.134894in}}%
\pgfpathcurveto{\pgfqpoint{3.888962in}{2.129070in}}{\pgfqpoint{3.896862in}{2.125798in}}{\pgfqpoint{3.905098in}{2.125798in}}%
\pgfpathclose%
\pgfusepath{stroke,fill}%
\end{pgfscope}%
\begin{pgfscope}%
\pgfpathrectangle{\pgfqpoint{3.793912in}{0.557870in}}{\pgfqpoint{2.446088in}{1.684734in}}%
\pgfusepath{clip}%
\pgfsetbuttcap%
\pgfsetroundjoin%
\definecolor{currentfill}{rgb}{0.298039,0.447059,0.690196}%
\pgfsetfillcolor{currentfill}%
\pgfsetlinewidth{1.003750pt}%
\definecolor{currentstroke}{rgb}{0.298039,0.447059,0.690196}%
\pgfsetstrokecolor{currentstroke}%
\pgfsetdash{}{0pt}%
\pgfpathmoveto{\pgfqpoint{3.905098in}{2.125798in}}%
\pgfpathcurveto{\pgfqpoint{3.913334in}{2.125798in}}{\pgfqpoint{3.921234in}{2.129070in}}{\pgfqpoint{3.927058in}{2.134894in}}%
\pgfpathcurveto{\pgfqpoint{3.932882in}{2.140718in}}{\pgfqpoint{3.936155in}{2.148618in}}{\pgfqpoint{3.936155in}{2.156854in}}%
\pgfpathcurveto{\pgfqpoint{3.936155in}{2.165091in}}{\pgfqpoint{3.932882in}{2.172991in}}{\pgfqpoint{3.927058in}{2.178814in}}%
\pgfpathcurveto{\pgfqpoint{3.921234in}{2.184638in}}{\pgfqpoint{3.913334in}{2.187911in}}{\pgfqpoint{3.905098in}{2.187911in}}%
\pgfpathcurveto{\pgfqpoint{3.896862in}{2.187911in}}{\pgfqpoint{3.888962in}{2.184638in}}{\pgfqpoint{3.883138in}{2.178814in}}%
\pgfpathcurveto{\pgfqpoint{3.877314in}{2.172991in}}{\pgfqpoint{3.874042in}{2.165091in}}{\pgfqpoint{3.874042in}{2.156854in}}%
\pgfpathcurveto{\pgfqpoint{3.874042in}{2.148618in}}{\pgfqpoint{3.877314in}{2.140718in}}{\pgfqpoint{3.883138in}{2.134894in}}%
\pgfpathcurveto{\pgfqpoint{3.888962in}{2.129070in}}{\pgfqpoint{3.896862in}{2.125798in}}{\pgfqpoint{3.905098in}{2.125798in}}%
\pgfpathclose%
\pgfusepath{stroke,fill}%
\end{pgfscope}%
\begin{pgfscope}%
\pgfpathrectangle{\pgfqpoint{3.793912in}{0.557870in}}{\pgfqpoint{2.446088in}{1.684734in}}%
\pgfusepath{clip}%
\pgfsetbuttcap%
\pgfsetroundjoin%
\definecolor{currentfill}{rgb}{0.298039,0.447059,0.690196}%
\pgfsetfillcolor{currentfill}%
\pgfsetlinewidth{1.003750pt}%
\definecolor{currentstroke}{rgb}{0.298039,0.447059,0.690196}%
\pgfsetstrokecolor{currentstroke}%
\pgfsetdash{}{0pt}%
\pgfpathmoveto{\pgfqpoint{3.905098in}{2.125798in}}%
\pgfpathcurveto{\pgfqpoint{3.913334in}{2.125798in}}{\pgfqpoint{3.921234in}{2.129070in}}{\pgfqpoint{3.927058in}{2.134894in}}%
\pgfpathcurveto{\pgfqpoint{3.932882in}{2.140718in}}{\pgfqpoint{3.936155in}{2.148618in}}{\pgfqpoint{3.936155in}{2.156854in}}%
\pgfpathcurveto{\pgfqpoint{3.936155in}{2.165091in}}{\pgfqpoint{3.932882in}{2.172991in}}{\pgfqpoint{3.927058in}{2.178814in}}%
\pgfpathcurveto{\pgfqpoint{3.921234in}{2.184638in}}{\pgfqpoint{3.913334in}{2.187911in}}{\pgfqpoint{3.905098in}{2.187911in}}%
\pgfpathcurveto{\pgfqpoint{3.896862in}{2.187911in}}{\pgfqpoint{3.888962in}{2.184638in}}{\pgfqpoint{3.883138in}{2.178814in}}%
\pgfpathcurveto{\pgfqpoint{3.877314in}{2.172991in}}{\pgfqpoint{3.874042in}{2.165091in}}{\pgfqpoint{3.874042in}{2.156854in}}%
\pgfpathcurveto{\pgfqpoint{3.874042in}{2.148618in}}{\pgfqpoint{3.877314in}{2.140718in}}{\pgfqpoint{3.883138in}{2.134894in}}%
\pgfpathcurveto{\pgfqpoint{3.888962in}{2.129070in}}{\pgfqpoint{3.896862in}{2.125798in}}{\pgfqpoint{3.905098in}{2.125798in}}%
\pgfpathclose%
\pgfusepath{stroke,fill}%
\end{pgfscope}%
\begin{pgfscope}%
\pgfpathrectangle{\pgfqpoint{3.793912in}{0.557870in}}{\pgfqpoint{2.446088in}{1.684734in}}%
\pgfusepath{clip}%
\pgfsetbuttcap%
\pgfsetroundjoin%
\definecolor{currentfill}{rgb}{0.298039,0.447059,0.690196}%
\pgfsetfillcolor{currentfill}%
\pgfsetlinewidth{1.003750pt}%
\definecolor{currentstroke}{rgb}{0.298039,0.447059,0.690196}%
\pgfsetstrokecolor{currentstroke}%
\pgfsetdash{}{0pt}%
\pgfpathmoveto{\pgfqpoint{5.975454in}{1.538846in}}%
\pgfpathcurveto{\pgfqpoint{5.983691in}{1.538846in}}{\pgfqpoint{5.991591in}{1.542119in}}{\pgfqpoint{5.997415in}{1.547943in}}%
\pgfpathcurveto{\pgfqpoint{6.003239in}{1.553767in}}{\pgfqpoint{6.006511in}{1.561667in}}{\pgfqpoint{6.006511in}{1.569903in}}%
\pgfpathcurveto{\pgfqpoint{6.006511in}{1.578139in}}{\pgfqpoint{6.003239in}{1.586039in}}{\pgfqpoint{5.997415in}{1.591863in}}%
\pgfpathcurveto{\pgfqpoint{5.991591in}{1.597687in}}{\pgfqpoint{5.983691in}{1.600959in}}{\pgfqpoint{5.975454in}{1.600959in}}%
\pgfpathcurveto{\pgfqpoint{5.967218in}{1.600959in}}{\pgfqpoint{5.959318in}{1.597687in}}{\pgfqpoint{5.953494in}{1.591863in}}%
\pgfpathcurveto{\pgfqpoint{5.947670in}{1.586039in}}{\pgfqpoint{5.944398in}{1.578139in}}{\pgfqpoint{5.944398in}{1.569903in}}%
\pgfpathcurveto{\pgfqpoint{5.944398in}{1.561667in}}{\pgfqpoint{5.947670in}{1.553767in}}{\pgfqpoint{5.953494in}{1.547943in}}%
\pgfpathcurveto{\pgfqpoint{5.959318in}{1.542119in}}{\pgfqpoint{5.967218in}{1.538846in}}{\pgfqpoint{5.975454in}{1.538846in}}%
\pgfpathclose%
\pgfusepath{stroke,fill}%
\end{pgfscope}%
\begin{pgfscope}%
\pgfpathrectangle{\pgfqpoint{3.793912in}{0.557870in}}{\pgfqpoint{2.446088in}{1.684734in}}%
\pgfusepath{clip}%
\pgfsetbuttcap%
\pgfsetroundjoin%
\definecolor{currentfill}{rgb}{0.298039,0.447059,0.690196}%
\pgfsetfillcolor{currentfill}%
\pgfsetlinewidth{1.003750pt}%
\definecolor{currentstroke}{rgb}{0.298039,0.447059,0.690196}%
\pgfsetstrokecolor{currentstroke}%
\pgfsetdash{}{0pt}%
\pgfpathmoveto{\pgfqpoint{5.975454in}{1.667242in}}%
\pgfpathcurveto{\pgfqpoint{5.983691in}{1.667242in}}{\pgfqpoint{5.991591in}{1.670514in}}{\pgfqpoint{5.997415in}{1.676338in}}%
\pgfpathcurveto{\pgfqpoint{6.003239in}{1.682162in}}{\pgfqpoint{6.006511in}{1.690062in}}{\pgfqpoint{6.006511in}{1.698298in}}%
\pgfpathcurveto{\pgfqpoint{6.006511in}{1.706535in}}{\pgfqpoint{6.003239in}{1.714435in}}{\pgfqpoint{5.997415in}{1.720259in}}%
\pgfpathcurveto{\pgfqpoint{5.991591in}{1.726083in}}{\pgfqpoint{5.983691in}{1.729355in}}{\pgfqpoint{5.975454in}{1.729355in}}%
\pgfpathcurveto{\pgfqpoint{5.967218in}{1.729355in}}{\pgfqpoint{5.959318in}{1.726083in}}{\pgfqpoint{5.953494in}{1.720259in}}%
\pgfpathcurveto{\pgfqpoint{5.947670in}{1.714435in}}{\pgfqpoint{5.944398in}{1.706535in}}{\pgfqpoint{5.944398in}{1.698298in}}%
\pgfpathcurveto{\pgfqpoint{5.944398in}{1.690062in}}{\pgfqpoint{5.947670in}{1.682162in}}{\pgfqpoint{5.953494in}{1.676338in}}%
\pgfpathcurveto{\pgfqpoint{5.959318in}{1.670514in}}{\pgfqpoint{5.967218in}{1.667242in}}{\pgfqpoint{5.975454in}{1.667242in}}%
\pgfpathclose%
\pgfusepath{stroke,fill}%
\end{pgfscope}%
\begin{pgfscope}%
\pgfpathrectangle{\pgfqpoint{3.793912in}{0.557870in}}{\pgfqpoint{2.446088in}{1.684734in}}%
\pgfusepath{clip}%
\pgfsetbuttcap%
\pgfsetroundjoin%
\definecolor{currentfill}{rgb}{0.298039,0.447059,0.690196}%
\pgfsetfillcolor{currentfill}%
\pgfsetlinewidth{1.003750pt}%
\definecolor{currentstroke}{rgb}{0.298039,0.447059,0.690196}%
\pgfsetstrokecolor{currentstroke}%
\pgfsetdash{}{0pt}%
\pgfpathmoveto{\pgfqpoint{5.975454in}{1.630558in}}%
\pgfpathcurveto{\pgfqpoint{5.983691in}{1.630558in}}{\pgfqpoint{5.991591in}{1.633830in}}{\pgfqpoint{5.997415in}{1.639654in}}%
\pgfpathcurveto{\pgfqpoint{6.003239in}{1.645478in}}{\pgfqpoint{6.006511in}{1.653378in}}{\pgfqpoint{6.006511in}{1.661614in}}%
\pgfpathcurveto{\pgfqpoint{6.006511in}{1.669850in}}{\pgfqpoint{6.003239in}{1.677750in}}{\pgfqpoint{5.997415in}{1.683574in}}%
\pgfpathcurveto{\pgfqpoint{5.991591in}{1.689398in}}{\pgfqpoint{5.983691in}{1.692671in}}{\pgfqpoint{5.975454in}{1.692671in}}%
\pgfpathcurveto{\pgfqpoint{5.967218in}{1.692671in}}{\pgfqpoint{5.959318in}{1.689398in}}{\pgfqpoint{5.953494in}{1.683574in}}%
\pgfpathcurveto{\pgfqpoint{5.947670in}{1.677750in}}{\pgfqpoint{5.944398in}{1.669850in}}{\pgfqpoint{5.944398in}{1.661614in}}%
\pgfpathcurveto{\pgfqpoint{5.944398in}{1.653378in}}{\pgfqpoint{5.947670in}{1.645478in}}{\pgfqpoint{5.953494in}{1.639654in}}%
\pgfpathcurveto{\pgfqpoint{5.959318in}{1.633830in}}{\pgfqpoint{5.967218in}{1.630558in}}{\pgfqpoint{5.975454in}{1.630558in}}%
\pgfpathclose%
\pgfusepath{stroke,fill}%
\end{pgfscope}%
\begin{pgfscope}%
\pgfpathrectangle{\pgfqpoint{3.793912in}{0.557870in}}{\pgfqpoint{2.446088in}{1.684734in}}%
\pgfusepath{clip}%
\pgfsetbuttcap%
\pgfsetroundjoin%
\definecolor{currentfill}{rgb}{0.298039,0.447059,0.690196}%
\pgfsetfillcolor{currentfill}%
\pgfsetlinewidth{1.003750pt}%
\definecolor{currentstroke}{rgb}{0.298039,0.447059,0.690196}%
\pgfsetstrokecolor{currentstroke}%
\pgfsetdash{}{0pt}%
\pgfpathmoveto{\pgfqpoint{5.975454in}{1.694755in}}%
\pgfpathcurveto{\pgfqpoint{5.983691in}{1.694755in}}{\pgfqpoint{5.991591in}{1.698028in}}{\pgfqpoint{5.997415in}{1.703852in}}%
\pgfpathcurveto{\pgfqpoint{6.003239in}{1.709675in}}{\pgfqpoint{6.006511in}{1.717576in}}{\pgfqpoint{6.006511in}{1.725812in}}%
\pgfpathcurveto{\pgfqpoint{6.006511in}{1.734048in}}{\pgfqpoint{6.003239in}{1.741948in}}{\pgfqpoint{5.997415in}{1.747772in}}%
\pgfpathcurveto{\pgfqpoint{5.991591in}{1.753596in}}{\pgfqpoint{5.983691in}{1.756868in}}{\pgfqpoint{5.975454in}{1.756868in}}%
\pgfpathcurveto{\pgfqpoint{5.967218in}{1.756868in}}{\pgfqpoint{5.959318in}{1.753596in}}{\pgfqpoint{5.953494in}{1.747772in}}%
\pgfpathcurveto{\pgfqpoint{5.947670in}{1.741948in}}{\pgfqpoint{5.944398in}{1.734048in}}{\pgfqpoint{5.944398in}{1.725812in}}%
\pgfpathcurveto{\pgfqpoint{5.944398in}{1.717576in}}{\pgfqpoint{5.947670in}{1.709675in}}{\pgfqpoint{5.953494in}{1.703852in}}%
\pgfpathcurveto{\pgfqpoint{5.959318in}{1.698028in}}{\pgfqpoint{5.967218in}{1.694755in}}{\pgfqpoint{5.975454in}{1.694755in}}%
\pgfpathclose%
\pgfusepath{stroke,fill}%
\end{pgfscope}%
\begin{pgfscope}%
\pgfpathrectangle{\pgfqpoint{3.793912in}{0.557870in}}{\pgfqpoint{2.446088in}{1.684734in}}%
\pgfusepath{clip}%
\pgfsetbuttcap%
\pgfsetroundjoin%
\definecolor{currentfill}{rgb}{0.298039,0.447059,0.690196}%
\pgfsetfillcolor{currentfill}%
\pgfsetlinewidth{1.003750pt}%
\definecolor{currentstroke}{rgb}{0.298039,0.447059,0.690196}%
\pgfsetstrokecolor{currentstroke}%
\pgfsetdash{}{0pt}%
\pgfpathmoveto{\pgfqpoint{5.975454in}{1.318740in}}%
\pgfpathcurveto{\pgfqpoint{5.983691in}{1.318740in}}{\pgfqpoint{5.991591in}{1.322012in}}{\pgfqpoint{5.997415in}{1.327836in}}%
\pgfpathcurveto{\pgfqpoint{6.003239in}{1.333660in}}{\pgfqpoint{6.006511in}{1.341560in}}{\pgfqpoint{6.006511in}{1.349796in}}%
\pgfpathcurveto{\pgfqpoint{6.006511in}{1.358032in}}{\pgfqpoint{6.003239in}{1.365932in}}{\pgfqpoint{5.997415in}{1.371756in}}%
\pgfpathcurveto{\pgfqpoint{5.991591in}{1.377580in}}{\pgfqpoint{5.983691in}{1.380853in}}{\pgfqpoint{5.975454in}{1.380853in}}%
\pgfpathcurveto{\pgfqpoint{5.967218in}{1.380853in}}{\pgfqpoint{5.959318in}{1.377580in}}{\pgfqpoint{5.953494in}{1.371756in}}%
\pgfpathcurveto{\pgfqpoint{5.947670in}{1.365932in}}{\pgfqpoint{5.944398in}{1.358032in}}{\pgfqpoint{5.944398in}{1.349796in}}%
\pgfpathcurveto{\pgfqpoint{5.944398in}{1.341560in}}{\pgfqpoint{5.947670in}{1.333660in}}{\pgfqpoint{5.953494in}{1.327836in}}%
\pgfpathcurveto{\pgfqpoint{5.959318in}{1.322012in}}{\pgfqpoint{5.967218in}{1.318740in}}{\pgfqpoint{5.975454in}{1.318740in}}%
\pgfpathclose%
\pgfusepath{stroke,fill}%
\end{pgfscope}%
\begin{pgfscope}%
\pgfpathrectangle{\pgfqpoint{3.793912in}{0.557870in}}{\pgfqpoint{2.446088in}{1.684734in}}%
\pgfusepath{clip}%
\pgfsetbuttcap%
\pgfsetroundjoin%
\definecolor{currentfill}{rgb}{0.298039,0.447059,0.690196}%
\pgfsetfillcolor{currentfill}%
\pgfsetlinewidth{1.003750pt}%
\definecolor{currentstroke}{rgb}{0.298039,0.447059,0.690196}%
\pgfsetstrokecolor{currentstroke}%
\pgfsetdash{}{0pt}%
\pgfpathmoveto{\pgfqpoint{5.975454in}{1.318740in}}%
\pgfpathcurveto{\pgfqpoint{5.983691in}{1.318740in}}{\pgfqpoint{5.991591in}{1.322012in}}{\pgfqpoint{5.997415in}{1.327836in}}%
\pgfpathcurveto{\pgfqpoint{6.003239in}{1.333660in}}{\pgfqpoint{6.006511in}{1.341560in}}{\pgfqpoint{6.006511in}{1.349796in}}%
\pgfpathcurveto{\pgfqpoint{6.006511in}{1.358032in}}{\pgfqpoint{6.003239in}{1.365932in}}{\pgfqpoint{5.997415in}{1.371756in}}%
\pgfpathcurveto{\pgfqpoint{5.991591in}{1.377580in}}{\pgfqpoint{5.983691in}{1.380853in}}{\pgfqpoint{5.975454in}{1.380853in}}%
\pgfpathcurveto{\pgfqpoint{5.967218in}{1.380853in}}{\pgfqpoint{5.959318in}{1.377580in}}{\pgfqpoint{5.953494in}{1.371756in}}%
\pgfpathcurveto{\pgfqpoint{5.947670in}{1.365932in}}{\pgfqpoint{5.944398in}{1.358032in}}{\pgfqpoint{5.944398in}{1.349796in}}%
\pgfpathcurveto{\pgfqpoint{5.944398in}{1.341560in}}{\pgfqpoint{5.947670in}{1.333660in}}{\pgfqpoint{5.953494in}{1.327836in}}%
\pgfpathcurveto{\pgfqpoint{5.959318in}{1.322012in}}{\pgfqpoint{5.967218in}{1.318740in}}{\pgfqpoint{5.975454in}{1.318740in}}%
\pgfpathclose%
\pgfusepath{stroke,fill}%
\end{pgfscope}%
\begin{pgfscope}%
\pgfpathrectangle{\pgfqpoint{3.793912in}{0.557870in}}{\pgfqpoint{2.446088in}{1.684734in}}%
\pgfusepath{clip}%
\pgfsetbuttcap%
\pgfsetroundjoin%
\definecolor{currentfill}{rgb}{0.298039,0.447059,0.690196}%
\pgfsetfillcolor{currentfill}%
\pgfsetlinewidth{1.003750pt}%
\definecolor{currentstroke}{rgb}{0.298039,0.447059,0.690196}%
\pgfsetstrokecolor{currentstroke}%
\pgfsetdash{}{0pt}%
\pgfpathmoveto{\pgfqpoint{3.905098in}{2.125798in}}%
\pgfpathcurveto{\pgfqpoint{3.913334in}{2.125798in}}{\pgfqpoint{3.921234in}{2.129070in}}{\pgfqpoint{3.927058in}{2.134894in}}%
\pgfpathcurveto{\pgfqpoint{3.932882in}{2.140718in}}{\pgfqpoint{3.936155in}{2.148618in}}{\pgfqpoint{3.936155in}{2.156854in}}%
\pgfpathcurveto{\pgfqpoint{3.936155in}{2.165091in}}{\pgfqpoint{3.932882in}{2.172991in}}{\pgfqpoint{3.927058in}{2.178814in}}%
\pgfpathcurveto{\pgfqpoint{3.921234in}{2.184638in}}{\pgfqpoint{3.913334in}{2.187911in}}{\pgfqpoint{3.905098in}{2.187911in}}%
\pgfpathcurveto{\pgfqpoint{3.896862in}{2.187911in}}{\pgfqpoint{3.888962in}{2.184638in}}{\pgfqpoint{3.883138in}{2.178814in}}%
\pgfpathcurveto{\pgfqpoint{3.877314in}{2.172991in}}{\pgfqpoint{3.874042in}{2.165091in}}{\pgfqpoint{3.874042in}{2.156854in}}%
\pgfpathcurveto{\pgfqpoint{3.874042in}{2.148618in}}{\pgfqpoint{3.877314in}{2.140718in}}{\pgfqpoint{3.883138in}{2.134894in}}%
\pgfpathcurveto{\pgfqpoint{3.888962in}{2.129070in}}{\pgfqpoint{3.896862in}{2.125798in}}{\pgfqpoint{3.905098in}{2.125798in}}%
\pgfpathclose%
\pgfusepath{stroke,fill}%
\end{pgfscope}%
\begin{pgfscope}%
\pgfpathrectangle{\pgfqpoint{3.793912in}{0.557870in}}{\pgfqpoint{2.446088in}{1.684734in}}%
\pgfusepath{clip}%
\pgfsetbuttcap%
\pgfsetroundjoin%
\definecolor{currentfill}{rgb}{0.298039,0.447059,0.690196}%
\pgfsetfillcolor{currentfill}%
\pgfsetlinewidth{1.003750pt}%
\definecolor{currentstroke}{rgb}{0.298039,0.447059,0.690196}%
\pgfsetstrokecolor{currentstroke}%
\pgfsetdash{}{0pt}%
\pgfpathmoveto{\pgfqpoint{4.058458in}{1.584702in}}%
\pgfpathcurveto{\pgfqpoint{4.066694in}{1.584702in}}{\pgfqpoint{4.074594in}{1.587974in}}{\pgfqpoint{4.080418in}{1.593798in}}%
\pgfpathcurveto{\pgfqpoint{4.086242in}{1.599622in}}{\pgfqpoint{4.089514in}{1.607522in}}{\pgfqpoint{4.089514in}{1.615758in}}%
\pgfpathcurveto{\pgfqpoint{4.089514in}{1.623995in}}{\pgfqpoint{4.086242in}{1.631895in}}{\pgfqpoint{4.080418in}{1.637719in}}%
\pgfpathcurveto{\pgfqpoint{4.074594in}{1.643543in}}{\pgfqpoint{4.066694in}{1.646815in}}{\pgfqpoint{4.058458in}{1.646815in}}%
\pgfpathcurveto{\pgfqpoint{4.050221in}{1.646815in}}{\pgfqpoint{4.042321in}{1.643543in}}{\pgfqpoint{4.036498in}{1.637719in}}%
\pgfpathcurveto{\pgfqpoint{4.030674in}{1.631895in}}{\pgfqpoint{4.027401in}{1.623995in}}{\pgfqpoint{4.027401in}{1.615758in}}%
\pgfpathcurveto{\pgfqpoint{4.027401in}{1.607522in}}{\pgfqpoint{4.030674in}{1.599622in}}{\pgfqpoint{4.036498in}{1.593798in}}%
\pgfpathcurveto{\pgfqpoint{4.042321in}{1.587974in}}{\pgfqpoint{4.050221in}{1.584702in}}{\pgfqpoint{4.058458in}{1.584702in}}%
\pgfpathclose%
\pgfusepath{stroke,fill}%
\end{pgfscope}%
\begin{pgfscope}%
\pgfpathrectangle{\pgfqpoint{3.793912in}{0.557870in}}{\pgfqpoint{2.446088in}{1.684734in}}%
\pgfusepath{clip}%
\pgfsetbuttcap%
\pgfsetroundjoin%
\definecolor{currentfill}{rgb}{0.298039,0.447059,0.690196}%
\pgfsetfillcolor{currentfill}%
\pgfsetlinewidth{1.003750pt}%
\definecolor{currentstroke}{rgb}{0.298039,0.447059,0.690196}%
\pgfsetstrokecolor{currentstroke}%
\pgfsetdash{}{0pt}%
\pgfpathmoveto{\pgfqpoint{5.975454in}{1.392109in}}%
\pgfpathcurveto{\pgfqpoint{5.983691in}{1.392109in}}{\pgfqpoint{5.991591in}{1.395381in}}{\pgfqpoint{5.997415in}{1.401205in}}%
\pgfpathcurveto{\pgfqpoint{6.003239in}{1.407029in}}{\pgfqpoint{6.006511in}{1.414929in}}{\pgfqpoint{6.006511in}{1.423165in}}%
\pgfpathcurveto{\pgfqpoint{6.006511in}{1.431401in}}{\pgfqpoint{6.003239in}{1.439301in}}{\pgfqpoint{5.997415in}{1.445125in}}%
\pgfpathcurveto{\pgfqpoint{5.991591in}{1.450949in}}{\pgfqpoint{5.983691in}{1.454222in}}{\pgfqpoint{5.975454in}{1.454222in}}%
\pgfpathcurveto{\pgfqpoint{5.967218in}{1.454222in}}{\pgfqpoint{5.959318in}{1.450949in}}{\pgfqpoint{5.953494in}{1.445125in}}%
\pgfpathcurveto{\pgfqpoint{5.947670in}{1.439301in}}{\pgfqpoint{5.944398in}{1.431401in}}{\pgfqpoint{5.944398in}{1.423165in}}%
\pgfpathcurveto{\pgfqpoint{5.944398in}{1.414929in}}{\pgfqpoint{5.947670in}{1.407029in}}{\pgfqpoint{5.953494in}{1.401205in}}%
\pgfpathcurveto{\pgfqpoint{5.959318in}{1.395381in}}{\pgfqpoint{5.967218in}{1.392109in}}{\pgfqpoint{5.975454in}{1.392109in}}%
\pgfpathclose%
\pgfusepath{stroke,fill}%
\end{pgfscope}%
\begin{pgfscope}%
\pgfpathrectangle{\pgfqpoint{3.793912in}{0.557870in}}{\pgfqpoint{2.446088in}{1.684734in}}%
\pgfusepath{clip}%
\pgfsetbuttcap%
\pgfsetroundjoin%
\definecolor{currentfill}{rgb}{0.298039,0.447059,0.690196}%
\pgfsetfillcolor{currentfill}%
\pgfsetlinewidth{1.003750pt}%
\definecolor{currentstroke}{rgb}{0.298039,0.447059,0.690196}%
\pgfsetstrokecolor{currentstroke}%
\pgfsetdash{}{0pt}%
\pgfpathmoveto{\pgfqpoint{5.975454in}{1.419622in}}%
\pgfpathcurveto{\pgfqpoint{5.983691in}{1.419622in}}{\pgfqpoint{5.991591in}{1.422894in}}{\pgfqpoint{5.997415in}{1.428718in}}%
\pgfpathcurveto{\pgfqpoint{6.003239in}{1.434542in}}{\pgfqpoint{6.006511in}{1.442442in}}{\pgfqpoint{6.006511in}{1.450678in}}%
\pgfpathcurveto{\pgfqpoint{6.006511in}{1.458915in}}{\pgfqpoint{6.003239in}{1.466815in}}{\pgfqpoint{5.997415in}{1.472639in}}%
\pgfpathcurveto{\pgfqpoint{5.991591in}{1.478463in}}{\pgfqpoint{5.983691in}{1.481735in}}{\pgfqpoint{5.975454in}{1.481735in}}%
\pgfpathcurveto{\pgfqpoint{5.967218in}{1.481735in}}{\pgfqpoint{5.959318in}{1.478463in}}{\pgfqpoint{5.953494in}{1.472639in}}%
\pgfpathcurveto{\pgfqpoint{5.947670in}{1.466815in}}{\pgfqpoint{5.944398in}{1.458915in}}{\pgfqpoint{5.944398in}{1.450678in}}%
\pgfpathcurveto{\pgfqpoint{5.944398in}{1.442442in}}{\pgfqpoint{5.947670in}{1.434542in}}{\pgfqpoint{5.953494in}{1.428718in}}%
\pgfpathcurveto{\pgfqpoint{5.959318in}{1.422894in}}{\pgfqpoint{5.967218in}{1.419622in}}{\pgfqpoint{5.975454in}{1.419622in}}%
\pgfpathclose%
\pgfusepath{stroke,fill}%
\end{pgfscope}%
\begin{pgfscope}%
\pgfpathrectangle{\pgfqpoint{3.793912in}{0.557870in}}{\pgfqpoint{2.446088in}{1.684734in}}%
\pgfusepath{clip}%
\pgfsetbuttcap%
\pgfsetroundjoin%
\definecolor{currentfill}{rgb}{0.298039,0.447059,0.690196}%
\pgfsetfillcolor{currentfill}%
\pgfsetlinewidth{1.003750pt}%
\definecolor{currentstroke}{rgb}{0.298039,0.447059,0.690196}%
\pgfsetstrokecolor{currentstroke}%
\pgfsetdash{}{0pt}%
\pgfpathmoveto{\pgfqpoint{5.975454in}{1.428793in}}%
\pgfpathcurveto{\pgfqpoint{5.983691in}{1.428793in}}{\pgfqpoint{5.991591in}{1.432065in}}{\pgfqpoint{5.997415in}{1.437889in}}%
\pgfpathcurveto{\pgfqpoint{6.003239in}{1.443713in}}{\pgfqpoint{6.006511in}{1.451613in}}{\pgfqpoint{6.006511in}{1.459849in}}%
\pgfpathcurveto{\pgfqpoint{6.006511in}{1.468086in}}{\pgfqpoint{6.003239in}{1.475986in}}{\pgfqpoint{5.997415in}{1.481810in}}%
\pgfpathcurveto{\pgfqpoint{5.991591in}{1.487634in}}{\pgfqpoint{5.983691in}{1.490906in}}{\pgfqpoint{5.975454in}{1.490906in}}%
\pgfpathcurveto{\pgfqpoint{5.967218in}{1.490906in}}{\pgfqpoint{5.959318in}{1.487634in}}{\pgfqpoint{5.953494in}{1.481810in}}%
\pgfpathcurveto{\pgfqpoint{5.947670in}{1.475986in}}{\pgfqpoint{5.944398in}{1.468086in}}{\pgfqpoint{5.944398in}{1.459849in}}%
\pgfpathcurveto{\pgfqpoint{5.944398in}{1.451613in}}{\pgfqpoint{5.947670in}{1.443713in}}{\pgfqpoint{5.953494in}{1.437889in}}%
\pgfpathcurveto{\pgfqpoint{5.959318in}{1.432065in}}{\pgfqpoint{5.967218in}{1.428793in}}{\pgfqpoint{5.975454in}{1.428793in}}%
\pgfpathclose%
\pgfusepath{stroke,fill}%
\end{pgfscope}%
\begin{pgfscope}%
\pgfpathrectangle{\pgfqpoint{3.793912in}{0.557870in}}{\pgfqpoint{2.446088in}{1.684734in}}%
\pgfusepath{clip}%
\pgfsetbuttcap%
\pgfsetroundjoin%
\definecolor{currentfill}{rgb}{0.298039,0.447059,0.690196}%
\pgfsetfillcolor{currentfill}%
\pgfsetlinewidth{1.003750pt}%
\definecolor{currentstroke}{rgb}{0.298039,0.447059,0.690196}%
\pgfsetstrokecolor{currentstroke}%
\pgfsetdash{}{0pt}%
\pgfpathmoveto{\pgfqpoint{5.975454in}{1.419622in}}%
\pgfpathcurveto{\pgfqpoint{5.983691in}{1.419622in}}{\pgfqpoint{5.991591in}{1.422894in}}{\pgfqpoint{5.997415in}{1.428718in}}%
\pgfpathcurveto{\pgfqpoint{6.003239in}{1.434542in}}{\pgfqpoint{6.006511in}{1.442442in}}{\pgfqpoint{6.006511in}{1.450678in}}%
\pgfpathcurveto{\pgfqpoint{6.006511in}{1.458915in}}{\pgfqpoint{6.003239in}{1.466815in}}{\pgfqpoint{5.997415in}{1.472639in}}%
\pgfpathcurveto{\pgfqpoint{5.991591in}{1.478463in}}{\pgfqpoint{5.983691in}{1.481735in}}{\pgfqpoint{5.975454in}{1.481735in}}%
\pgfpathcurveto{\pgfqpoint{5.967218in}{1.481735in}}{\pgfqpoint{5.959318in}{1.478463in}}{\pgfqpoint{5.953494in}{1.472639in}}%
\pgfpathcurveto{\pgfqpoint{5.947670in}{1.466815in}}{\pgfqpoint{5.944398in}{1.458915in}}{\pgfqpoint{5.944398in}{1.450678in}}%
\pgfpathcurveto{\pgfqpoint{5.944398in}{1.442442in}}{\pgfqpoint{5.947670in}{1.434542in}}{\pgfqpoint{5.953494in}{1.428718in}}%
\pgfpathcurveto{\pgfqpoint{5.959318in}{1.422894in}}{\pgfqpoint{5.967218in}{1.419622in}}{\pgfqpoint{5.975454in}{1.419622in}}%
\pgfpathclose%
\pgfusepath{stroke,fill}%
\end{pgfscope}%
\begin{pgfscope}%
\pgfpathrectangle{\pgfqpoint{3.793912in}{0.557870in}}{\pgfqpoint{2.446088in}{1.684734in}}%
\pgfusepath{clip}%
\pgfsetbuttcap%
\pgfsetroundjoin%
\definecolor{currentfill}{rgb}{0.298039,0.447059,0.690196}%
\pgfsetfillcolor{currentfill}%
\pgfsetlinewidth{1.003750pt}%
\definecolor{currentstroke}{rgb}{0.298039,0.447059,0.690196}%
\pgfsetstrokecolor{currentstroke}%
\pgfsetdash{}{0pt}%
\pgfpathmoveto{\pgfqpoint{4.058458in}{1.722269in}}%
\pgfpathcurveto{\pgfqpoint{4.066694in}{1.722269in}}{\pgfqpoint{4.074594in}{1.725541in}}{\pgfqpoint{4.080418in}{1.731365in}}%
\pgfpathcurveto{\pgfqpoint{4.086242in}{1.737189in}}{\pgfqpoint{4.089514in}{1.745089in}}{\pgfqpoint{4.089514in}{1.753325in}}%
\pgfpathcurveto{\pgfqpoint{4.089514in}{1.761561in}}{\pgfqpoint{4.086242in}{1.769461in}}{\pgfqpoint{4.080418in}{1.775285in}}%
\pgfpathcurveto{\pgfqpoint{4.074594in}{1.781109in}}{\pgfqpoint{4.066694in}{1.784382in}}{\pgfqpoint{4.058458in}{1.784382in}}%
\pgfpathcurveto{\pgfqpoint{4.050221in}{1.784382in}}{\pgfqpoint{4.042321in}{1.781109in}}{\pgfqpoint{4.036498in}{1.775285in}}%
\pgfpathcurveto{\pgfqpoint{4.030674in}{1.769461in}}{\pgfqpoint{4.027401in}{1.761561in}}{\pgfqpoint{4.027401in}{1.753325in}}%
\pgfpathcurveto{\pgfqpoint{4.027401in}{1.745089in}}{\pgfqpoint{4.030674in}{1.737189in}}{\pgfqpoint{4.036498in}{1.731365in}}%
\pgfpathcurveto{\pgfqpoint{4.042321in}{1.725541in}}{\pgfqpoint{4.050221in}{1.722269in}}{\pgfqpoint{4.058458in}{1.722269in}}%
\pgfpathclose%
\pgfusepath{stroke,fill}%
\end{pgfscope}%
\begin{pgfscope}%
\pgfpathrectangle{\pgfqpoint{3.793912in}{0.557870in}}{\pgfqpoint{2.446088in}{1.684734in}}%
\pgfusepath{clip}%
\pgfsetbuttcap%
\pgfsetroundjoin%
\definecolor{currentfill}{rgb}{0.298039,0.447059,0.690196}%
\pgfsetfillcolor{currentfill}%
\pgfsetlinewidth{1.003750pt}%
\definecolor{currentstroke}{rgb}{0.298039,0.447059,0.690196}%
\pgfsetstrokecolor{currentstroke}%
\pgfsetdash{}{0pt}%
\pgfpathmoveto{\pgfqpoint{3.905098in}{2.125798in}}%
\pgfpathcurveto{\pgfqpoint{3.913334in}{2.125798in}}{\pgfqpoint{3.921234in}{2.129070in}}{\pgfqpoint{3.927058in}{2.134894in}}%
\pgfpathcurveto{\pgfqpoint{3.932882in}{2.140718in}}{\pgfqpoint{3.936155in}{2.148618in}}{\pgfqpoint{3.936155in}{2.156854in}}%
\pgfpathcurveto{\pgfqpoint{3.936155in}{2.165091in}}{\pgfqpoint{3.932882in}{2.172991in}}{\pgfqpoint{3.927058in}{2.178814in}}%
\pgfpathcurveto{\pgfqpoint{3.921234in}{2.184638in}}{\pgfqpoint{3.913334in}{2.187911in}}{\pgfqpoint{3.905098in}{2.187911in}}%
\pgfpathcurveto{\pgfqpoint{3.896862in}{2.187911in}}{\pgfqpoint{3.888962in}{2.184638in}}{\pgfqpoint{3.883138in}{2.178814in}}%
\pgfpathcurveto{\pgfqpoint{3.877314in}{2.172991in}}{\pgfqpoint{3.874042in}{2.165091in}}{\pgfqpoint{3.874042in}{2.156854in}}%
\pgfpathcurveto{\pgfqpoint{3.874042in}{2.148618in}}{\pgfqpoint{3.877314in}{2.140718in}}{\pgfqpoint{3.883138in}{2.134894in}}%
\pgfpathcurveto{\pgfqpoint{3.888962in}{2.129070in}}{\pgfqpoint{3.896862in}{2.125798in}}{\pgfqpoint{3.905098in}{2.125798in}}%
\pgfpathclose%
\pgfusepath{stroke,fill}%
\end{pgfscope}%
\begin{pgfscope}%
\pgfpathrectangle{\pgfqpoint{3.793912in}{0.557870in}}{\pgfqpoint{2.446088in}{1.684734in}}%
\pgfusepath{clip}%
\pgfsetbuttcap%
\pgfsetroundjoin%
\definecolor{currentfill}{rgb}{0.298039,0.447059,0.690196}%
\pgfsetfillcolor{currentfill}%
\pgfsetlinewidth{1.003750pt}%
\definecolor{currentstroke}{rgb}{0.298039,0.447059,0.690196}%
\pgfsetstrokecolor{currentstroke}%
\pgfsetdash{}{0pt}%
\pgfpathmoveto{\pgfqpoint{4.288497in}{1.850664in}}%
\pgfpathcurveto{\pgfqpoint{4.296734in}{1.850664in}}{\pgfqpoint{4.304634in}{1.853937in}}{\pgfqpoint{4.310458in}{1.859761in}}%
\pgfpathcurveto{\pgfqpoint{4.316282in}{1.865584in}}{\pgfqpoint{4.319554in}{1.873484in}}{\pgfqpoint{4.319554in}{1.881721in}}%
\pgfpathcurveto{\pgfqpoint{4.319554in}{1.889957in}}{\pgfqpoint{4.316282in}{1.897857in}}{\pgfqpoint{4.310458in}{1.903681in}}%
\pgfpathcurveto{\pgfqpoint{4.304634in}{1.909505in}}{\pgfqpoint{4.296734in}{1.912777in}}{\pgfqpoint{4.288497in}{1.912777in}}%
\pgfpathcurveto{\pgfqpoint{4.280261in}{1.912777in}}{\pgfqpoint{4.272361in}{1.909505in}}{\pgfqpoint{4.266537in}{1.903681in}}%
\pgfpathcurveto{\pgfqpoint{4.260713in}{1.897857in}}{\pgfqpoint{4.257441in}{1.889957in}}{\pgfqpoint{4.257441in}{1.881721in}}%
\pgfpathcurveto{\pgfqpoint{4.257441in}{1.873484in}}{\pgfqpoint{4.260713in}{1.865584in}}{\pgfqpoint{4.266537in}{1.859761in}}%
\pgfpathcurveto{\pgfqpoint{4.272361in}{1.853937in}}{\pgfqpoint{4.280261in}{1.850664in}}{\pgfqpoint{4.288497in}{1.850664in}}%
\pgfpathclose%
\pgfusepath{stroke,fill}%
\end{pgfscope}%
\begin{pgfscope}%
\pgfpathrectangle{\pgfqpoint{3.793912in}{0.557870in}}{\pgfqpoint{2.446088in}{1.684734in}}%
\pgfusepath{clip}%
\pgfsetbuttcap%
\pgfsetroundjoin%
\definecolor{currentfill}{rgb}{0.298039,0.447059,0.690196}%
\pgfsetfillcolor{currentfill}%
\pgfsetlinewidth{1.003750pt}%
\definecolor{currentstroke}{rgb}{0.298039,0.447059,0.690196}%
\pgfsetstrokecolor{currentstroke}%
\pgfsetdash{}{0pt}%
\pgfpathmoveto{\pgfqpoint{3.905098in}{2.125798in}}%
\pgfpathcurveto{\pgfqpoint{3.913334in}{2.125798in}}{\pgfqpoint{3.921234in}{2.129070in}}{\pgfqpoint{3.927058in}{2.134894in}}%
\pgfpathcurveto{\pgfqpoint{3.932882in}{2.140718in}}{\pgfqpoint{3.936155in}{2.148618in}}{\pgfqpoint{3.936155in}{2.156854in}}%
\pgfpathcurveto{\pgfqpoint{3.936155in}{2.165091in}}{\pgfqpoint{3.932882in}{2.172991in}}{\pgfqpoint{3.927058in}{2.178814in}}%
\pgfpathcurveto{\pgfqpoint{3.921234in}{2.184638in}}{\pgfqpoint{3.913334in}{2.187911in}}{\pgfqpoint{3.905098in}{2.187911in}}%
\pgfpathcurveto{\pgfqpoint{3.896862in}{2.187911in}}{\pgfqpoint{3.888962in}{2.184638in}}{\pgfqpoint{3.883138in}{2.178814in}}%
\pgfpathcurveto{\pgfqpoint{3.877314in}{2.172991in}}{\pgfqpoint{3.874042in}{2.165091in}}{\pgfqpoint{3.874042in}{2.156854in}}%
\pgfpathcurveto{\pgfqpoint{3.874042in}{2.148618in}}{\pgfqpoint{3.877314in}{2.140718in}}{\pgfqpoint{3.883138in}{2.134894in}}%
\pgfpathcurveto{\pgfqpoint{3.888962in}{2.129070in}}{\pgfqpoint{3.896862in}{2.125798in}}{\pgfqpoint{3.905098in}{2.125798in}}%
\pgfpathclose%
\pgfusepath{stroke,fill}%
\end{pgfscope}%
\begin{pgfscope}%
\pgfpathrectangle{\pgfqpoint{3.793912in}{0.557870in}}{\pgfqpoint{2.446088in}{1.684734in}}%
\pgfusepath{clip}%
\pgfsetbuttcap%
\pgfsetroundjoin%
\definecolor{currentfill}{rgb}{0.298039,0.447059,0.690196}%
\pgfsetfillcolor{currentfill}%
\pgfsetlinewidth{1.003750pt}%
\definecolor{currentstroke}{rgb}{0.298039,0.447059,0.690196}%
\pgfsetstrokecolor{currentstroke}%
\pgfsetdash{}{0pt}%
\pgfpathmoveto{\pgfqpoint{4.978616in}{1.832322in}}%
\pgfpathcurveto{\pgfqpoint{4.986852in}{1.832322in}}{\pgfqpoint{4.994753in}{1.835594in}}{\pgfqpoint{5.000576in}{1.841418in}}%
\pgfpathcurveto{\pgfqpoint{5.006400in}{1.847242in}}{\pgfqpoint{5.009673in}{1.855142in}}{\pgfqpoint{5.009673in}{1.863379in}}%
\pgfpathcurveto{\pgfqpoint{5.009673in}{1.871615in}}{\pgfqpoint{5.006400in}{1.879515in}}{\pgfqpoint{5.000576in}{1.885339in}}%
\pgfpathcurveto{\pgfqpoint{4.994753in}{1.891163in}}{\pgfqpoint{4.986852in}{1.894435in}}{\pgfqpoint{4.978616in}{1.894435in}}%
\pgfpathcurveto{\pgfqpoint{4.970380in}{1.894435in}}{\pgfqpoint{4.962480in}{1.891163in}}{\pgfqpoint{4.956656in}{1.885339in}}%
\pgfpathcurveto{\pgfqpoint{4.950832in}{1.879515in}}{\pgfqpoint{4.947560in}{1.871615in}}{\pgfqpoint{4.947560in}{1.863379in}}%
\pgfpathcurveto{\pgfqpoint{4.947560in}{1.855142in}}{\pgfqpoint{4.950832in}{1.847242in}}{\pgfqpoint{4.956656in}{1.841418in}}%
\pgfpathcurveto{\pgfqpoint{4.962480in}{1.835594in}}{\pgfqpoint{4.970380in}{1.832322in}}{\pgfqpoint{4.978616in}{1.832322in}}%
\pgfpathclose%
\pgfusepath{stroke,fill}%
\end{pgfscope}%
\begin{pgfscope}%
\pgfpathrectangle{\pgfqpoint{3.793912in}{0.557870in}}{\pgfqpoint{2.446088in}{1.684734in}}%
\pgfusepath{clip}%
\pgfsetbuttcap%
\pgfsetroundjoin%
\definecolor{currentfill}{rgb}{0.298039,0.447059,0.690196}%
\pgfsetfillcolor{currentfill}%
\pgfsetlinewidth{1.003750pt}%
\definecolor{currentstroke}{rgb}{0.298039,0.447059,0.690196}%
\pgfsetstrokecolor{currentstroke}%
\pgfsetdash{}{0pt}%
\pgfpathmoveto{\pgfqpoint{3.905098in}{2.125798in}}%
\pgfpathcurveto{\pgfqpoint{3.913334in}{2.125798in}}{\pgfqpoint{3.921234in}{2.129070in}}{\pgfqpoint{3.927058in}{2.134894in}}%
\pgfpathcurveto{\pgfqpoint{3.932882in}{2.140718in}}{\pgfqpoint{3.936155in}{2.148618in}}{\pgfqpoint{3.936155in}{2.156854in}}%
\pgfpathcurveto{\pgfqpoint{3.936155in}{2.165091in}}{\pgfqpoint{3.932882in}{2.172991in}}{\pgfqpoint{3.927058in}{2.178814in}}%
\pgfpathcurveto{\pgfqpoint{3.921234in}{2.184638in}}{\pgfqpoint{3.913334in}{2.187911in}}{\pgfqpoint{3.905098in}{2.187911in}}%
\pgfpathcurveto{\pgfqpoint{3.896862in}{2.187911in}}{\pgfqpoint{3.888962in}{2.184638in}}{\pgfqpoint{3.883138in}{2.178814in}}%
\pgfpathcurveto{\pgfqpoint{3.877314in}{2.172991in}}{\pgfqpoint{3.874042in}{2.165091in}}{\pgfqpoint{3.874042in}{2.156854in}}%
\pgfpathcurveto{\pgfqpoint{3.874042in}{2.148618in}}{\pgfqpoint{3.877314in}{2.140718in}}{\pgfqpoint{3.883138in}{2.134894in}}%
\pgfpathcurveto{\pgfqpoint{3.888962in}{2.129070in}}{\pgfqpoint{3.896862in}{2.125798in}}{\pgfqpoint{3.905098in}{2.125798in}}%
\pgfpathclose%
\pgfusepath{stroke,fill}%
\end{pgfscope}%
\begin{pgfscope}%
\pgfpathrectangle{\pgfqpoint{3.793912in}{0.557870in}}{\pgfqpoint{2.446088in}{1.684734in}}%
\pgfusepath{clip}%
\pgfsetbuttcap%
\pgfsetroundjoin%
\definecolor{currentfill}{rgb}{0.298039,0.447059,0.690196}%
\pgfsetfillcolor{currentfill}%
\pgfsetlinewidth{1.003750pt}%
\definecolor{currentstroke}{rgb}{0.298039,0.447059,0.690196}%
\pgfsetstrokecolor{currentstroke}%
\pgfsetdash{}{0pt}%
\pgfpathmoveto{\pgfqpoint{3.905098in}{2.125798in}}%
\pgfpathcurveto{\pgfqpoint{3.913334in}{2.125798in}}{\pgfqpoint{3.921234in}{2.129070in}}{\pgfqpoint{3.927058in}{2.134894in}}%
\pgfpathcurveto{\pgfqpoint{3.932882in}{2.140718in}}{\pgfqpoint{3.936155in}{2.148618in}}{\pgfqpoint{3.936155in}{2.156854in}}%
\pgfpathcurveto{\pgfqpoint{3.936155in}{2.165091in}}{\pgfqpoint{3.932882in}{2.172991in}}{\pgfqpoint{3.927058in}{2.178814in}}%
\pgfpathcurveto{\pgfqpoint{3.921234in}{2.184638in}}{\pgfqpoint{3.913334in}{2.187911in}}{\pgfqpoint{3.905098in}{2.187911in}}%
\pgfpathcurveto{\pgfqpoint{3.896862in}{2.187911in}}{\pgfqpoint{3.888962in}{2.184638in}}{\pgfqpoint{3.883138in}{2.178814in}}%
\pgfpathcurveto{\pgfqpoint{3.877314in}{2.172991in}}{\pgfqpoint{3.874042in}{2.165091in}}{\pgfqpoint{3.874042in}{2.156854in}}%
\pgfpathcurveto{\pgfqpoint{3.874042in}{2.148618in}}{\pgfqpoint{3.877314in}{2.140718in}}{\pgfqpoint{3.883138in}{2.134894in}}%
\pgfpathcurveto{\pgfqpoint{3.888962in}{2.129070in}}{\pgfqpoint{3.896862in}{2.125798in}}{\pgfqpoint{3.905098in}{2.125798in}}%
\pgfpathclose%
\pgfusepath{stroke,fill}%
\end{pgfscope}%
\begin{pgfscope}%
\pgfpathrectangle{\pgfqpoint{3.793912in}{0.557870in}}{\pgfqpoint{2.446088in}{1.684734in}}%
\pgfusepath{clip}%
\pgfsetbuttcap%
\pgfsetroundjoin%
\definecolor{currentfill}{rgb}{0.298039,0.447059,0.690196}%
\pgfsetfillcolor{currentfill}%
\pgfsetlinewidth{1.003750pt}%
\definecolor{currentstroke}{rgb}{0.298039,0.447059,0.690196}%
\pgfsetstrokecolor{currentstroke}%
\pgfsetdash{}{0pt}%
\pgfpathmoveto{\pgfqpoint{3.905098in}{2.125798in}}%
\pgfpathcurveto{\pgfqpoint{3.913334in}{2.125798in}}{\pgfqpoint{3.921234in}{2.129070in}}{\pgfqpoint{3.927058in}{2.134894in}}%
\pgfpathcurveto{\pgfqpoint{3.932882in}{2.140718in}}{\pgfqpoint{3.936155in}{2.148618in}}{\pgfqpoint{3.936155in}{2.156854in}}%
\pgfpathcurveto{\pgfqpoint{3.936155in}{2.165091in}}{\pgfqpoint{3.932882in}{2.172991in}}{\pgfqpoint{3.927058in}{2.178814in}}%
\pgfpathcurveto{\pgfqpoint{3.921234in}{2.184638in}}{\pgfqpoint{3.913334in}{2.187911in}}{\pgfqpoint{3.905098in}{2.187911in}}%
\pgfpathcurveto{\pgfqpoint{3.896862in}{2.187911in}}{\pgfqpoint{3.888962in}{2.184638in}}{\pgfqpoint{3.883138in}{2.178814in}}%
\pgfpathcurveto{\pgfqpoint{3.877314in}{2.172991in}}{\pgfqpoint{3.874042in}{2.165091in}}{\pgfqpoint{3.874042in}{2.156854in}}%
\pgfpathcurveto{\pgfqpoint{3.874042in}{2.148618in}}{\pgfqpoint{3.877314in}{2.140718in}}{\pgfqpoint{3.883138in}{2.134894in}}%
\pgfpathcurveto{\pgfqpoint{3.888962in}{2.129070in}}{\pgfqpoint{3.896862in}{2.125798in}}{\pgfqpoint{3.905098in}{2.125798in}}%
\pgfpathclose%
\pgfusepath{stroke,fill}%
\end{pgfscope}%
\begin{pgfscope}%
\pgfpathrectangle{\pgfqpoint{3.793912in}{0.557870in}}{\pgfqpoint{2.446088in}{1.684734in}}%
\pgfusepath{clip}%
\pgfsetbuttcap%
\pgfsetroundjoin%
\definecolor{currentfill}{rgb}{0.298039,0.447059,0.690196}%
\pgfsetfillcolor{currentfill}%
\pgfsetlinewidth{1.003750pt}%
\definecolor{currentstroke}{rgb}{0.298039,0.447059,0.690196}%
\pgfsetstrokecolor{currentstroke}%
\pgfsetdash{}{0pt}%
\pgfpathmoveto{\pgfqpoint{3.905098in}{2.125798in}}%
\pgfpathcurveto{\pgfqpoint{3.913334in}{2.125798in}}{\pgfqpoint{3.921234in}{2.129070in}}{\pgfqpoint{3.927058in}{2.134894in}}%
\pgfpathcurveto{\pgfqpoint{3.932882in}{2.140718in}}{\pgfqpoint{3.936155in}{2.148618in}}{\pgfqpoint{3.936155in}{2.156854in}}%
\pgfpathcurveto{\pgfqpoint{3.936155in}{2.165091in}}{\pgfqpoint{3.932882in}{2.172991in}}{\pgfqpoint{3.927058in}{2.178814in}}%
\pgfpathcurveto{\pgfqpoint{3.921234in}{2.184638in}}{\pgfqpoint{3.913334in}{2.187911in}}{\pgfqpoint{3.905098in}{2.187911in}}%
\pgfpathcurveto{\pgfqpoint{3.896862in}{2.187911in}}{\pgfqpoint{3.888962in}{2.184638in}}{\pgfqpoint{3.883138in}{2.178814in}}%
\pgfpathcurveto{\pgfqpoint{3.877314in}{2.172991in}}{\pgfqpoint{3.874042in}{2.165091in}}{\pgfqpoint{3.874042in}{2.156854in}}%
\pgfpathcurveto{\pgfqpoint{3.874042in}{2.148618in}}{\pgfqpoint{3.877314in}{2.140718in}}{\pgfqpoint{3.883138in}{2.134894in}}%
\pgfpathcurveto{\pgfqpoint{3.888962in}{2.129070in}}{\pgfqpoint{3.896862in}{2.125798in}}{\pgfqpoint{3.905098in}{2.125798in}}%
\pgfpathclose%
\pgfusepath{stroke,fill}%
\end{pgfscope}%
\begin{pgfscope}%
\pgfpathrectangle{\pgfqpoint{3.793912in}{0.557870in}}{\pgfqpoint{2.446088in}{1.684734in}}%
\pgfusepath{clip}%
\pgfsetbuttcap%
\pgfsetroundjoin%
\definecolor{currentfill}{rgb}{0.298039,0.447059,0.690196}%
\pgfsetfillcolor{currentfill}%
\pgfsetlinewidth{1.003750pt}%
\definecolor{currentstroke}{rgb}{0.298039,0.447059,0.690196}%
\pgfsetstrokecolor{currentstroke}%
\pgfsetdash{}{0pt}%
\pgfpathmoveto{\pgfqpoint{4.978616in}{1.603044in}}%
\pgfpathcurveto{\pgfqpoint{4.986852in}{1.603044in}}{\pgfqpoint{4.994753in}{1.606316in}}{\pgfqpoint{5.000576in}{1.612140in}}%
\pgfpathcurveto{\pgfqpoint{5.006400in}{1.617964in}}{\pgfqpoint{5.009673in}{1.625864in}}{\pgfqpoint{5.009673in}{1.634101in}}%
\pgfpathcurveto{\pgfqpoint{5.009673in}{1.642337in}}{\pgfqpoint{5.006400in}{1.650237in}}{\pgfqpoint{5.000576in}{1.656061in}}%
\pgfpathcurveto{\pgfqpoint{4.994753in}{1.661885in}}{\pgfqpoint{4.986852in}{1.665157in}}{\pgfqpoint{4.978616in}{1.665157in}}%
\pgfpathcurveto{\pgfqpoint{4.970380in}{1.665157in}}{\pgfqpoint{4.962480in}{1.661885in}}{\pgfqpoint{4.956656in}{1.656061in}}%
\pgfpathcurveto{\pgfqpoint{4.950832in}{1.650237in}}{\pgfqpoint{4.947560in}{1.642337in}}{\pgfqpoint{4.947560in}{1.634101in}}%
\pgfpathcurveto{\pgfqpoint{4.947560in}{1.625864in}}{\pgfqpoint{4.950832in}{1.617964in}}{\pgfqpoint{4.956656in}{1.612140in}}%
\pgfpathcurveto{\pgfqpoint{4.962480in}{1.606316in}}{\pgfqpoint{4.970380in}{1.603044in}}{\pgfqpoint{4.978616in}{1.603044in}}%
\pgfpathclose%
\pgfusepath{stroke,fill}%
\end{pgfscope}%
\begin{pgfscope}%
\pgfpathrectangle{\pgfqpoint{3.793912in}{0.557870in}}{\pgfqpoint{2.446088in}{1.684734in}}%
\pgfusepath{clip}%
\pgfsetbuttcap%
\pgfsetroundjoin%
\definecolor{currentfill}{rgb}{0.298039,0.447059,0.690196}%
\pgfsetfillcolor{currentfill}%
\pgfsetlinewidth{1.003750pt}%
\definecolor{currentstroke}{rgb}{0.298039,0.447059,0.690196}%
\pgfsetstrokecolor{currentstroke}%
\pgfsetdash{}{0pt}%
\pgfpathmoveto{\pgfqpoint{3.905098in}{2.125798in}}%
\pgfpathcurveto{\pgfqpoint{3.913334in}{2.125798in}}{\pgfqpoint{3.921234in}{2.129070in}}{\pgfqpoint{3.927058in}{2.134894in}}%
\pgfpathcurveto{\pgfqpoint{3.932882in}{2.140718in}}{\pgfqpoint{3.936155in}{2.148618in}}{\pgfqpoint{3.936155in}{2.156854in}}%
\pgfpathcurveto{\pgfqpoint{3.936155in}{2.165091in}}{\pgfqpoint{3.932882in}{2.172991in}}{\pgfqpoint{3.927058in}{2.178814in}}%
\pgfpathcurveto{\pgfqpoint{3.921234in}{2.184638in}}{\pgfqpoint{3.913334in}{2.187911in}}{\pgfqpoint{3.905098in}{2.187911in}}%
\pgfpathcurveto{\pgfqpoint{3.896862in}{2.187911in}}{\pgfqpoint{3.888962in}{2.184638in}}{\pgfqpoint{3.883138in}{2.178814in}}%
\pgfpathcurveto{\pgfqpoint{3.877314in}{2.172991in}}{\pgfqpoint{3.874042in}{2.165091in}}{\pgfqpoint{3.874042in}{2.156854in}}%
\pgfpathcurveto{\pgfqpoint{3.874042in}{2.148618in}}{\pgfqpoint{3.877314in}{2.140718in}}{\pgfqpoint{3.883138in}{2.134894in}}%
\pgfpathcurveto{\pgfqpoint{3.888962in}{2.129070in}}{\pgfqpoint{3.896862in}{2.125798in}}{\pgfqpoint{3.905098in}{2.125798in}}%
\pgfpathclose%
\pgfusepath{stroke,fill}%
\end{pgfscope}%
\begin{pgfscope}%
\pgfpathrectangle{\pgfqpoint{3.793912in}{0.557870in}}{\pgfqpoint{2.446088in}{1.684734in}}%
\pgfusepath{clip}%
\pgfsetbuttcap%
\pgfsetroundjoin%
\definecolor{currentfill}{rgb}{0.298039,0.447059,0.690196}%
\pgfsetfillcolor{currentfill}%
\pgfsetlinewidth{1.003750pt}%
\definecolor{currentstroke}{rgb}{0.298039,0.447059,0.690196}%
\pgfsetstrokecolor{currentstroke}%
\pgfsetdash{}{0pt}%
\pgfpathmoveto{\pgfqpoint{4.978616in}{1.639729in}}%
\pgfpathcurveto{\pgfqpoint{4.986852in}{1.639729in}}{\pgfqpoint{4.994753in}{1.643001in}}{\pgfqpoint{5.000576in}{1.648825in}}%
\pgfpathcurveto{\pgfqpoint{5.006400in}{1.654649in}}{\pgfqpoint{5.009673in}{1.662549in}}{\pgfqpoint{5.009673in}{1.670785in}}%
\pgfpathcurveto{\pgfqpoint{5.009673in}{1.679021in}}{\pgfqpoint{5.006400in}{1.686921in}}{\pgfqpoint{5.000576in}{1.692745in}}%
\pgfpathcurveto{\pgfqpoint{4.994753in}{1.698569in}}{\pgfqpoint{4.986852in}{1.701842in}}{\pgfqpoint{4.978616in}{1.701842in}}%
\pgfpathcurveto{\pgfqpoint{4.970380in}{1.701842in}}{\pgfqpoint{4.962480in}{1.698569in}}{\pgfqpoint{4.956656in}{1.692745in}}%
\pgfpathcurveto{\pgfqpoint{4.950832in}{1.686921in}}{\pgfqpoint{4.947560in}{1.679021in}}{\pgfqpoint{4.947560in}{1.670785in}}%
\pgfpathcurveto{\pgfqpoint{4.947560in}{1.662549in}}{\pgfqpoint{4.950832in}{1.654649in}}{\pgfqpoint{4.956656in}{1.648825in}}%
\pgfpathcurveto{\pgfqpoint{4.962480in}{1.643001in}}{\pgfqpoint{4.970380in}{1.639729in}}{\pgfqpoint{4.978616in}{1.639729in}}%
\pgfpathclose%
\pgfusepath{stroke,fill}%
\end{pgfscope}%
\begin{pgfscope}%
\pgfpathrectangle{\pgfqpoint{3.793912in}{0.557870in}}{\pgfqpoint{2.446088in}{1.684734in}}%
\pgfusepath{clip}%
\pgfsetbuttcap%
\pgfsetroundjoin%
\definecolor{currentfill}{rgb}{0.298039,0.447059,0.690196}%
\pgfsetfillcolor{currentfill}%
\pgfsetlinewidth{1.003750pt}%
\definecolor{currentstroke}{rgb}{0.298039,0.447059,0.690196}%
\pgfsetstrokecolor{currentstroke}%
\pgfsetdash{}{0pt}%
\pgfpathmoveto{\pgfqpoint{3.905098in}{2.125798in}}%
\pgfpathcurveto{\pgfqpoint{3.913334in}{2.125798in}}{\pgfqpoint{3.921234in}{2.129070in}}{\pgfqpoint{3.927058in}{2.134894in}}%
\pgfpathcurveto{\pgfqpoint{3.932882in}{2.140718in}}{\pgfqpoint{3.936155in}{2.148618in}}{\pgfqpoint{3.936155in}{2.156854in}}%
\pgfpathcurveto{\pgfqpoint{3.936155in}{2.165091in}}{\pgfqpoint{3.932882in}{2.172991in}}{\pgfqpoint{3.927058in}{2.178814in}}%
\pgfpathcurveto{\pgfqpoint{3.921234in}{2.184638in}}{\pgfqpoint{3.913334in}{2.187911in}}{\pgfqpoint{3.905098in}{2.187911in}}%
\pgfpathcurveto{\pgfqpoint{3.896862in}{2.187911in}}{\pgfqpoint{3.888962in}{2.184638in}}{\pgfqpoint{3.883138in}{2.178814in}}%
\pgfpathcurveto{\pgfqpoint{3.877314in}{2.172991in}}{\pgfqpoint{3.874042in}{2.165091in}}{\pgfqpoint{3.874042in}{2.156854in}}%
\pgfpathcurveto{\pgfqpoint{3.874042in}{2.148618in}}{\pgfqpoint{3.877314in}{2.140718in}}{\pgfqpoint{3.883138in}{2.134894in}}%
\pgfpathcurveto{\pgfqpoint{3.888962in}{2.129070in}}{\pgfqpoint{3.896862in}{2.125798in}}{\pgfqpoint{3.905098in}{2.125798in}}%
\pgfpathclose%
\pgfusepath{stroke,fill}%
\end{pgfscope}%
\begin{pgfscope}%
\pgfpathrectangle{\pgfqpoint{3.793912in}{0.557870in}}{\pgfqpoint{2.446088in}{1.684734in}}%
\pgfusepath{clip}%
\pgfsetbuttcap%
\pgfsetroundjoin%
\definecolor{currentfill}{rgb}{0.298039,0.447059,0.690196}%
\pgfsetfillcolor{currentfill}%
\pgfsetlinewidth{1.003750pt}%
\definecolor{currentstroke}{rgb}{0.298039,0.447059,0.690196}%
\pgfsetstrokecolor{currentstroke}%
\pgfsetdash{}{0pt}%
\pgfpathmoveto{\pgfqpoint{3.905098in}{2.125798in}}%
\pgfpathcurveto{\pgfqpoint{3.913334in}{2.125798in}}{\pgfqpoint{3.921234in}{2.129070in}}{\pgfqpoint{3.927058in}{2.134894in}}%
\pgfpathcurveto{\pgfqpoint{3.932882in}{2.140718in}}{\pgfqpoint{3.936155in}{2.148618in}}{\pgfqpoint{3.936155in}{2.156854in}}%
\pgfpathcurveto{\pgfqpoint{3.936155in}{2.165091in}}{\pgfqpoint{3.932882in}{2.172991in}}{\pgfqpoint{3.927058in}{2.178814in}}%
\pgfpathcurveto{\pgfqpoint{3.921234in}{2.184638in}}{\pgfqpoint{3.913334in}{2.187911in}}{\pgfqpoint{3.905098in}{2.187911in}}%
\pgfpathcurveto{\pgfqpoint{3.896862in}{2.187911in}}{\pgfqpoint{3.888962in}{2.184638in}}{\pgfqpoint{3.883138in}{2.178814in}}%
\pgfpathcurveto{\pgfqpoint{3.877314in}{2.172991in}}{\pgfqpoint{3.874042in}{2.165091in}}{\pgfqpoint{3.874042in}{2.156854in}}%
\pgfpathcurveto{\pgfqpoint{3.874042in}{2.148618in}}{\pgfqpoint{3.877314in}{2.140718in}}{\pgfqpoint{3.883138in}{2.134894in}}%
\pgfpathcurveto{\pgfqpoint{3.888962in}{2.129070in}}{\pgfqpoint{3.896862in}{2.125798in}}{\pgfqpoint{3.905098in}{2.125798in}}%
\pgfpathclose%
\pgfusepath{stroke,fill}%
\end{pgfscope}%
\begin{pgfscope}%
\pgfpathrectangle{\pgfqpoint{3.793912in}{0.557870in}}{\pgfqpoint{2.446088in}{1.684734in}}%
\pgfusepath{clip}%
\pgfsetbuttcap%
\pgfsetroundjoin%
\definecolor{currentfill}{rgb}{0.298039,0.447059,0.690196}%
\pgfsetfillcolor{currentfill}%
\pgfsetlinewidth{1.003750pt}%
\definecolor{currentstroke}{rgb}{0.298039,0.447059,0.690196}%
\pgfsetstrokecolor{currentstroke}%
\pgfsetdash{}{0pt}%
\pgfpathmoveto{\pgfqpoint{5.822095in}{1.795638in}}%
\pgfpathcurveto{\pgfqpoint{5.830331in}{1.795638in}}{\pgfqpoint{5.838231in}{1.798910in}}{\pgfqpoint{5.844055in}{1.804734in}}%
\pgfpathcurveto{\pgfqpoint{5.849879in}{1.810558in}}{\pgfqpoint{5.853151in}{1.818458in}}{\pgfqpoint{5.853151in}{1.826694in}}%
\pgfpathcurveto{\pgfqpoint{5.853151in}{1.834930in}}{\pgfqpoint{5.849879in}{1.842830in}}{\pgfqpoint{5.844055in}{1.848654in}}%
\pgfpathcurveto{\pgfqpoint{5.838231in}{1.854478in}}{\pgfqpoint{5.830331in}{1.857751in}}{\pgfqpoint{5.822095in}{1.857751in}}%
\pgfpathcurveto{\pgfqpoint{5.813858in}{1.857751in}}{\pgfqpoint{5.805958in}{1.854478in}}{\pgfqpoint{5.800134in}{1.848654in}}%
\pgfpathcurveto{\pgfqpoint{5.794311in}{1.842830in}}{\pgfqpoint{5.791038in}{1.834930in}}{\pgfqpoint{5.791038in}{1.826694in}}%
\pgfpathcurveto{\pgfqpoint{5.791038in}{1.818458in}}{\pgfqpoint{5.794311in}{1.810558in}}{\pgfqpoint{5.800134in}{1.804734in}}%
\pgfpathcurveto{\pgfqpoint{5.805958in}{1.798910in}}{\pgfqpoint{5.813858in}{1.795638in}}{\pgfqpoint{5.822095in}{1.795638in}}%
\pgfpathclose%
\pgfusepath{stroke,fill}%
\end{pgfscope}%
\begin{pgfscope}%
\pgfpathrectangle{\pgfqpoint{3.793912in}{0.557870in}}{\pgfqpoint{2.446088in}{1.684734in}}%
\pgfusepath{clip}%
\pgfsetbuttcap%
\pgfsetroundjoin%
\definecolor{currentfill}{rgb}{0.298039,0.447059,0.690196}%
\pgfsetfillcolor{currentfill}%
\pgfsetlinewidth{1.003750pt}%
\definecolor{currentstroke}{rgb}{0.298039,0.447059,0.690196}%
\pgfsetstrokecolor{currentstroke}%
\pgfsetdash{}{0pt}%
\pgfpathmoveto{\pgfqpoint{5.975454in}{1.263713in}}%
\pgfpathcurveto{\pgfqpoint{5.983691in}{1.263713in}}{\pgfqpoint{5.991591in}{1.266985in}}{\pgfqpoint{5.997415in}{1.272809in}}%
\pgfpathcurveto{\pgfqpoint{6.003239in}{1.278633in}}{\pgfqpoint{6.006511in}{1.286533in}}{\pgfqpoint{6.006511in}{1.294769in}}%
\pgfpathcurveto{\pgfqpoint{6.006511in}{1.303006in}}{\pgfqpoint{6.003239in}{1.310906in}}{\pgfqpoint{5.997415in}{1.316730in}}%
\pgfpathcurveto{\pgfqpoint{5.991591in}{1.322554in}}{\pgfqpoint{5.983691in}{1.325826in}}{\pgfqpoint{5.975454in}{1.325826in}}%
\pgfpathcurveto{\pgfqpoint{5.967218in}{1.325826in}}{\pgfqpoint{5.959318in}{1.322554in}}{\pgfqpoint{5.953494in}{1.316730in}}%
\pgfpathcurveto{\pgfqpoint{5.947670in}{1.310906in}}{\pgfqpoint{5.944398in}{1.303006in}}{\pgfqpoint{5.944398in}{1.294769in}}%
\pgfpathcurveto{\pgfqpoint{5.944398in}{1.286533in}}{\pgfqpoint{5.947670in}{1.278633in}}{\pgfqpoint{5.953494in}{1.272809in}}%
\pgfpathcurveto{\pgfqpoint{5.959318in}{1.266985in}}{\pgfqpoint{5.967218in}{1.263713in}}{\pgfqpoint{5.975454in}{1.263713in}}%
\pgfpathclose%
\pgfusepath{stroke,fill}%
\end{pgfscope}%
\begin{pgfscope}%
\pgfpathrectangle{\pgfqpoint{3.793912in}{0.557870in}}{\pgfqpoint{2.446088in}{1.684734in}}%
\pgfusepath{clip}%
\pgfsetbuttcap%
\pgfsetroundjoin%
\definecolor{currentfill}{rgb}{0.298039,0.447059,0.690196}%
\pgfsetfillcolor{currentfill}%
\pgfsetlinewidth{1.003750pt}%
\definecolor{currentstroke}{rgb}{0.298039,0.447059,0.690196}%
\pgfsetstrokecolor{currentstroke}%
\pgfsetdash{}{0pt}%
\pgfpathmoveto{\pgfqpoint{5.975454in}{1.282055in}}%
\pgfpathcurveto{\pgfqpoint{5.983691in}{1.282055in}}{\pgfqpoint{5.991591in}{1.285327in}}{\pgfqpoint{5.997415in}{1.291151in}}%
\pgfpathcurveto{\pgfqpoint{6.003239in}{1.296975in}}{\pgfqpoint{6.006511in}{1.304875in}}{\pgfqpoint{6.006511in}{1.313112in}}%
\pgfpathcurveto{\pgfqpoint{6.006511in}{1.321348in}}{\pgfqpoint{6.003239in}{1.329248in}}{\pgfqpoint{5.997415in}{1.335072in}}%
\pgfpathcurveto{\pgfqpoint{5.991591in}{1.340896in}}{\pgfqpoint{5.983691in}{1.344168in}}{\pgfqpoint{5.975454in}{1.344168in}}%
\pgfpathcurveto{\pgfqpoint{5.967218in}{1.344168in}}{\pgfqpoint{5.959318in}{1.340896in}}{\pgfqpoint{5.953494in}{1.335072in}}%
\pgfpathcurveto{\pgfqpoint{5.947670in}{1.329248in}}{\pgfqpoint{5.944398in}{1.321348in}}{\pgfqpoint{5.944398in}{1.313112in}}%
\pgfpathcurveto{\pgfqpoint{5.944398in}{1.304875in}}{\pgfqpoint{5.947670in}{1.296975in}}{\pgfqpoint{5.953494in}{1.291151in}}%
\pgfpathcurveto{\pgfqpoint{5.959318in}{1.285327in}}{\pgfqpoint{5.967218in}{1.282055in}}{\pgfqpoint{5.975454in}{1.282055in}}%
\pgfpathclose%
\pgfusepath{stroke,fill}%
\end{pgfscope}%
\begin{pgfscope}%
\pgfpathrectangle{\pgfqpoint{3.793912in}{0.557870in}}{\pgfqpoint{2.446088in}{1.684734in}}%
\pgfusepath{clip}%
\pgfsetbuttcap%
\pgfsetroundjoin%
\definecolor{currentfill}{rgb}{0.298039,0.447059,0.690196}%
\pgfsetfillcolor{currentfill}%
\pgfsetlinewidth{1.003750pt}%
\definecolor{currentstroke}{rgb}{0.298039,0.447059,0.690196}%
\pgfsetstrokecolor{currentstroke}%
\pgfsetdash{}{0pt}%
\pgfpathmoveto{\pgfqpoint{3.981778in}{2.052429in}}%
\pgfpathcurveto{\pgfqpoint{3.990014in}{2.052429in}}{\pgfqpoint{3.997914in}{2.055701in}}{\pgfqpoint{4.003738in}{2.061525in}}%
\pgfpathcurveto{\pgfqpoint{4.009562in}{2.067349in}}{\pgfqpoint{4.012834in}{2.075249in}}{\pgfqpoint{4.012834in}{2.083485in}}%
\pgfpathcurveto{\pgfqpoint{4.012834in}{2.091722in}}{\pgfqpoint{4.009562in}{2.099622in}}{\pgfqpoint{4.003738in}{2.105446in}}%
\pgfpathcurveto{\pgfqpoint{3.997914in}{2.111270in}}{\pgfqpoint{3.990014in}{2.114542in}}{\pgfqpoint{3.981778in}{2.114542in}}%
\pgfpathcurveto{\pgfqpoint{3.973542in}{2.114542in}}{\pgfqpoint{3.965642in}{2.111270in}}{\pgfqpoint{3.959818in}{2.105446in}}%
\pgfpathcurveto{\pgfqpoint{3.953994in}{2.099622in}}{\pgfqpoint{3.950721in}{2.091722in}}{\pgfqpoint{3.950721in}{2.083485in}}%
\pgfpathcurveto{\pgfqpoint{3.950721in}{2.075249in}}{\pgfqpoint{3.953994in}{2.067349in}}{\pgfqpoint{3.959818in}{2.061525in}}%
\pgfpathcurveto{\pgfqpoint{3.965642in}{2.055701in}}{\pgfqpoint{3.973542in}{2.052429in}}{\pgfqpoint{3.981778in}{2.052429in}}%
\pgfpathclose%
\pgfusepath{stroke,fill}%
\end{pgfscope}%
\begin{pgfscope}%
\pgfpathrectangle{\pgfqpoint{3.793912in}{0.557870in}}{\pgfqpoint{2.446088in}{1.684734in}}%
\pgfusepath{clip}%
\pgfsetbuttcap%
\pgfsetroundjoin%
\definecolor{currentfill}{rgb}{0.298039,0.447059,0.690196}%
\pgfsetfillcolor{currentfill}%
\pgfsetlinewidth{1.003750pt}%
\definecolor{currentstroke}{rgb}{0.298039,0.447059,0.690196}%
\pgfsetstrokecolor{currentstroke}%
\pgfsetdash{}{0pt}%
\pgfpathmoveto{\pgfqpoint{3.905098in}{2.125798in}}%
\pgfpathcurveto{\pgfqpoint{3.913334in}{2.125798in}}{\pgfqpoint{3.921234in}{2.129070in}}{\pgfqpoint{3.927058in}{2.134894in}}%
\pgfpathcurveto{\pgfqpoint{3.932882in}{2.140718in}}{\pgfqpoint{3.936155in}{2.148618in}}{\pgfqpoint{3.936155in}{2.156854in}}%
\pgfpathcurveto{\pgfqpoint{3.936155in}{2.165091in}}{\pgfqpoint{3.932882in}{2.172991in}}{\pgfqpoint{3.927058in}{2.178814in}}%
\pgfpathcurveto{\pgfqpoint{3.921234in}{2.184638in}}{\pgfqpoint{3.913334in}{2.187911in}}{\pgfqpoint{3.905098in}{2.187911in}}%
\pgfpathcurveto{\pgfqpoint{3.896862in}{2.187911in}}{\pgfqpoint{3.888962in}{2.184638in}}{\pgfqpoint{3.883138in}{2.178814in}}%
\pgfpathcurveto{\pgfqpoint{3.877314in}{2.172991in}}{\pgfqpoint{3.874042in}{2.165091in}}{\pgfqpoint{3.874042in}{2.156854in}}%
\pgfpathcurveto{\pgfqpoint{3.874042in}{2.148618in}}{\pgfqpoint{3.877314in}{2.140718in}}{\pgfqpoint{3.883138in}{2.134894in}}%
\pgfpathcurveto{\pgfqpoint{3.888962in}{2.129070in}}{\pgfqpoint{3.896862in}{2.125798in}}{\pgfqpoint{3.905098in}{2.125798in}}%
\pgfpathclose%
\pgfusepath{stroke,fill}%
\end{pgfscope}%
\begin{pgfscope}%
\pgfpathrectangle{\pgfqpoint{3.793912in}{0.557870in}}{\pgfqpoint{2.446088in}{1.684734in}}%
\pgfusepath{clip}%
\pgfsetbuttcap%
\pgfsetroundjoin%
\definecolor{currentfill}{rgb}{0.298039,0.447059,0.690196}%
\pgfsetfillcolor{currentfill}%
\pgfsetlinewidth{1.003750pt}%
\definecolor{currentstroke}{rgb}{0.298039,0.447059,0.690196}%
\pgfsetstrokecolor{currentstroke}%
\pgfsetdash{}{0pt}%
\pgfpathmoveto{\pgfqpoint{3.905098in}{2.125798in}}%
\pgfpathcurveto{\pgfqpoint{3.913334in}{2.125798in}}{\pgfqpoint{3.921234in}{2.129070in}}{\pgfqpoint{3.927058in}{2.134894in}}%
\pgfpathcurveto{\pgfqpoint{3.932882in}{2.140718in}}{\pgfqpoint{3.936155in}{2.148618in}}{\pgfqpoint{3.936155in}{2.156854in}}%
\pgfpathcurveto{\pgfqpoint{3.936155in}{2.165091in}}{\pgfqpoint{3.932882in}{2.172991in}}{\pgfqpoint{3.927058in}{2.178814in}}%
\pgfpathcurveto{\pgfqpoint{3.921234in}{2.184638in}}{\pgfqpoint{3.913334in}{2.187911in}}{\pgfqpoint{3.905098in}{2.187911in}}%
\pgfpathcurveto{\pgfqpoint{3.896862in}{2.187911in}}{\pgfqpoint{3.888962in}{2.184638in}}{\pgfqpoint{3.883138in}{2.178814in}}%
\pgfpathcurveto{\pgfqpoint{3.877314in}{2.172991in}}{\pgfqpoint{3.874042in}{2.165091in}}{\pgfqpoint{3.874042in}{2.156854in}}%
\pgfpathcurveto{\pgfqpoint{3.874042in}{2.148618in}}{\pgfqpoint{3.877314in}{2.140718in}}{\pgfqpoint{3.883138in}{2.134894in}}%
\pgfpathcurveto{\pgfqpoint{3.888962in}{2.129070in}}{\pgfqpoint{3.896862in}{2.125798in}}{\pgfqpoint{3.905098in}{2.125798in}}%
\pgfpathclose%
\pgfusepath{stroke,fill}%
\end{pgfscope}%
\begin{pgfscope}%
\pgfpathrectangle{\pgfqpoint{3.793912in}{0.557870in}}{\pgfqpoint{2.446088in}{1.684734in}}%
\pgfusepath{clip}%
\pgfsetbuttcap%
\pgfsetroundjoin%
\definecolor{currentfill}{rgb}{0.298039,0.447059,0.690196}%
\pgfsetfillcolor{currentfill}%
\pgfsetlinewidth{1.003750pt}%
\definecolor{currentstroke}{rgb}{0.298039,0.447059,0.690196}%
\pgfsetstrokecolor{currentstroke}%
\pgfsetdash{}{0pt}%
\pgfpathmoveto{\pgfqpoint{4.288497in}{1.703926in}}%
\pgfpathcurveto{\pgfqpoint{4.296734in}{1.703926in}}{\pgfqpoint{4.304634in}{1.707199in}}{\pgfqpoint{4.310458in}{1.713023in}}%
\pgfpathcurveto{\pgfqpoint{4.316282in}{1.718847in}}{\pgfqpoint{4.319554in}{1.726747in}}{\pgfqpoint{4.319554in}{1.734983in}}%
\pgfpathcurveto{\pgfqpoint{4.319554in}{1.743219in}}{\pgfqpoint{4.316282in}{1.751119in}}{\pgfqpoint{4.310458in}{1.756943in}}%
\pgfpathcurveto{\pgfqpoint{4.304634in}{1.762767in}}{\pgfqpoint{4.296734in}{1.766039in}}{\pgfqpoint{4.288497in}{1.766039in}}%
\pgfpathcurveto{\pgfqpoint{4.280261in}{1.766039in}}{\pgfqpoint{4.272361in}{1.762767in}}{\pgfqpoint{4.266537in}{1.756943in}}%
\pgfpathcurveto{\pgfqpoint{4.260713in}{1.751119in}}{\pgfqpoint{4.257441in}{1.743219in}}{\pgfqpoint{4.257441in}{1.734983in}}%
\pgfpathcurveto{\pgfqpoint{4.257441in}{1.726747in}}{\pgfqpoint{4.260713in}{1.718847in}}{\pgfqpoint{4.266537in}{1.713023in}}%
\pgfpathcurveto{\pgfqpoint{4.272361in}{1.707199in}}{\pgfqpoint{4.280261in}{1.703926in}}{\pgfqpoint{4.288497in}{1.703926in}}%
\pgfpathclose%
\pgfusepath{stroke,fill}%
\end{pgfscope}%
\begin{pgfscope}%
\pgfpathrectangle{\pgfqpoint{3.793912in}{0.557870in}}{\pgfqpoint{2.446088in}{1.684734in}}%
\pgfusepath{clip}%
\pgfsetbuttcap%
\pgfsetroundjoin%
\definecolor{currentfill}{rgb}{0.298039,0.447059,0.690196}%
\pgfsetfillcolor{currentfill}%
\pgfsetlinewidth{1.003750pt}%
\definecolor{currentstroke}{rgb}{0.298039,0.447059,0.690196}%
\pgfsetstrokecolor{currentstroke}%
\pgfsetdash{}{0pt}%
\pgfpathmoveto{\pgfqpoint{3.905098in}{2.125798in}}%
\pgfpathcurveto{\pgfqpoint{3.913334in}{2.125798in}}{\pgfqpoint{3.921234in}{2.129070in}}{\pgfqpoint{3.927058in}{2.134894in}}%
\pgfpathcurveto{\pgfqpoint{3.932882in}{2.140718in}}{\pgfqpoint{3.936155in}{2.148618in}}{\pgfqpoint{3.936155in}{2.156854in}}%
\pgfpathcurveto{\pgfqpoint{3.936155in}{2.165091in}}{\pgfqpoint{3.932882in}{2.172991in}}{\pgfqpoint{3.927058in}{2.178814in}}%
\pgfpathcurveto{\pgfqpoint{3.921234in}{2.184638in}}{\pgfqpoint{3.913334in}{2.187911in}}{\pgfqpoint{3.905098in}{2.187911in}}%
\pgfpathcurveto{\pgfqpoint{3.896862in}{2.187911in}}{\pgfqpoint{3.888962in}{2.184638in}}{\pgfqpoint{3.883138in}{2.178814in}}%
\pgfpathcurveto{\pgfqpoint{3.877314in}{2.172991in}}{\pgfqpoint{3.874042in}{2.165091in}}{\pgfqpoint{3.874042in}{2.156854in}}%
\pgfpathcurveto{\pgfqpoint{3.874042in}{2.148618in}}{\pgfqpoint{3.877314in}{2.140718in}}{\pgfqpoint{3.883138in}{2.134894in}}%
\pgfpathcurveto{\pgfqpoint{3.888962in}{2.129070in}}{\pgfqpoint{3.896862in}{2.125798in}}{\pgfqpoint{3.905098in}{2.125798in}}%
\pgfpathclose%
\pgfusepath{stroke,fill}%
\end{pgfscope}%
\begin{pgfscope}%
\pgfpathrectangle{\pgfqpoint{3.793912in}{0.557870in}}{\pgfqpoint{2.446088in}{1.684734in}}%
\pgfusepath{clip}%
\pgfsetbuttcap%
\pgfsetroundjoin%
\definecolor{currentfill}{rgb}{0.298039,0.447059,0.690196}%
\pgfsetfillcolor{currentfill}%
\pgfsetlinewidth{1.003750pt}%
\definecolor{currentstroke}{rgb}{0.298039,0.447059,0.690196}%
\pgfsetstrokecolor{currentstroke}%
\pgfsetdash{}{0pt}%
\pgfpathmoveto{\pgfqpoint{5.055296in}{1.364595in}}%
\pgfpathcurveto{\pgfqpoint{5.063532in}{1.364595in}}{\pgfqpoint{5.071432in}{1.367867in}}{\pgfqpoint{5.077256in}{1.373691in}}%
\pgfpathcurveto{\pgfqpoint{5.083080in}{1.379515in}}{\pgfqpoint{5.086353in}{1.387415in}}{\pgfqpoint{5.086353in}{1.395652in}}%
\pgfpathcurveto{\pgfqpoint{5.086353in}{1.403888in}}{\pgfqpoint{5.083080in}{1.411788in}}{\pgfqpoint{5.077256in}{1.417612in}}%
\pgfpathcurveto{\pgfqpoint{5.071432in}{1.423436in}}{\pgfqpoint{5.063532in}{1.426708in}}{\pgfqpoint{5.055296in}{1.426708in}}%
\pgfpathcurveto{\pgfqpoint{5.047060in}{1.426708in}}{\pgfqpoint{5.039160in}{1.423436in}}{\pgfqpoint{5.033336in}{1.417612in}}%
\pgfpathcurveto{\pgfqpoint{5.027512in}{1.411788in}}{\pgfqpoint{5.024240in}{1.403888in}}{\pgfqpoint{5.024240in}{1.395652in}}%
\pgfpathcurveto{\pgfqpoint{5.024240in}{1.387415in}}{\pgfqpoint{5.027512in}{1.379515in}}{\pgfqpoint{5.033336in}{1.373691in}}%
\pgfpathcurveto{\pgfqpoint{5.039160in}{1.367867in}}{\pgfqpoint{5.047060in}{1.364595in}}{\pgfqpoint{5.055296in}{1.364595in}}%
\pgfpathclose%
\pgfusepath{stroke,fill}%
\end{pgfscope}%
\begin{pgfscope}%
\pgfpathrectangle{\pgfqpoint{3.793912in}{0.557870in}}{\pgfqpoint{2.446088in}{1.684734in}}%
\pgfusepath{clip}%
\pgfsetbuttcap%
\pgfsetroundjoin%
\definecolor{currentfill}{rgb}{0.298039,0.447059,0.690196}%
\pgfsetfillcolor{currentfill}%
\pgfsetlinewidth{1.003750pt}%
\definecolor{currentstroke}{rgb}{0.298039,0.447059,0.690196}%
\pgfsetstrokecolor{currentstroke}%
\pgfsetdash{}{0pt}%
\pgfpathmoveto{\pgfqpoint{3.905098in}{2.125798in}}%
\pgfpathcurveto{\pgfqpoint{3.913334in}{2.125798in}}{\pgfqpoint{3.921234in}{2.129070in}}{\pgfqpoint{3.927058in}{2.134894in}}%
\pgfpathcurveto{\pgfqpoint{3.932882in}{2.140718in}}{\pgfqpoint{3.936155in}{2.148618in}}{\pgfqpoint{3.936155in}{2.156854in}}%
\pgfpathcurveto{\pgfqpoint{3.936155in}{2.165091in}}{\pgfqpoint{3.932882in}{2.172991in}}{\pgfqpoint{3.927058in}{2.178814in}}%
\pgfpathcurveto{\pgfqpoint{3.921234in}{2.184638in}}{\pgfqpoint{3.913334in}{2.187911in}}{\pgfqpoint{3.905098in}{2.187911in}}%
\pgfpathcurveto{\pgfqpoint{3.896862in}{2.187911in}}{\pgfqpoint{3.888962in}{2.184638in}}{\pgfqpoint{3.883138in}{2.178814in}}%
\pgfpathcurveto{\pgfqpoint{3.877314in}{2.172991in}}{\pgfqpoint{3.874042in}{2.165091in}}{\pgfqpoint{3.874042in}{2.156854in}}%
\pgfpathcurveto{\pgfqpoint{3.874042in}{2.148618in}}{\pgfqpoint{3.877314in}{2.140718in}}{\pgfqpoint{3.883138in}{2.134894in}}%
\pgfpathcurveto{\pgfqpoint{3.888962in}{2.129070in}}{\pgfqpoint{3.896862in}{2.125798in}}{\pgfqpoint{3.905098in}{2.125798in}}%
\pgfpathclose%
\pgfusepath{stroke,fill}%
\end{pgfscope}%
\begin{pgfscope}%
\pgfpathrectangle{\pgfqpoint{3.793912in}{0.557870in}}{\pgfqpoint{2.446088in}{1.684734in}}%
\pgfusepath{clip}%
\pgfsetbuttcap%
\pgfsetroundjoin%
\definecolor{currentfill}{rgb}{0.298039,0.447059,0.690196}%
\pgfsetfillcolor{currentfill}%
\pgfsetlinewidth{1.003750pt}%
\definecolor{currentstroke}{rgb}{0.298039,0.447059,0.690196}%
\pgfsetstrokecolor{currentstroke}%
\pgfsetdash{}{0pt}%
\pgfpathmoveto{\pgfqpoint{3.905098in}{2.125798in}}%
\pgfpathcurveto{\pgfqpoint{3.913334in}{2.125798in}}{\pgfqpoint{3.921234in}{2.129070in}}{\pgfqpoint{3.927058in}{2.134894in}}%
\pgfpathcurveto{\pgfqpoint{3.932882in}{2.140718in}}{\pgfqpoint{3.936155in}{2.148618in}}{\pgfqpoint{3.936155in}{2.156854in}}%
\pgfpathcurveto{\pgfqpoint{3.936155in}{2.165091in}}{\pgfqpoint{3.932882in}{2.172991in}}{\pgfqpoint{3.927058in}{2.178814in}}%
\pgfpathcurveto{\pgfqpoint{3.921234in}{2.184638in}}{\pgfqpoint{3.913334in}{2.187911in}}{\pgfqpoint{3.905098in}{2.187911in}}%
\pgfpathcurveto{\pgfqpoint{3.896862in}{2.187911in}}{\pgfqpoint{3.888962in}{2.184638in}}{\pgfqpoint{3.883138in}{2.178814in}}%
\pgfpathcurveto{\pgfqpoint{3.877314in}{2.172991in}}{\pgfqpoint{3.874042in}{2.165091in}}{\pgfqpoint{3.874042in}{2.156854in}}%
\pgfpathcurveto{\pgfqpoint{3.874042in}{2.148618in}}{\pgfqpoint{3.877314in}{2.140718in}}{\pgfqpoint{3.883138in}{2.134894in}}%
\pgfpathcurveto{\pgfqpoint{3.888962in}{2.129070in}}{\pgfqpoint{3.896862in}{2.125798in}}{\pgfqpoint{3.905098in}{2.125798in}}%
\pgfpathclose%
\pgfusepath{stroke,fill}%
\end{pgfscope}%
\begin{pgfscope}%
\pgfpathrectangle{\pgfqpoint{3.793912in}{0.557870in}}{\pgfqpoint{2.446088in}{1.684734in}}%
\pgfusepath{clip}%
\pgfsetbuttcap%
\pgfsetroundjoin%
\definecolor{currentfill}{rgb}{0.298039,0.447059,0.690196}%
\pgfsetfillcolor{currentfill}%
\pgfsetlinewidth{1.003750pt}%
\definecolor{currentstroke}{rgb}{0.298039,0.447059,0.690196}%
\pgfsetstrokecolor{currentstroke}%
\pgfsetdash{}{0pt}%
\pgfpathmoveto{\pgfqpoint{4.211818in}{2.125798in}}%
\pgfpathcurveto{\pgfqpoint{4.220054in}{2.125798in}}{\pgfqpoint{4.227954in}{2.129070in}}{\pgfqpoint{4.233778in}{2.134894in}}%
\pgfpathcurveto{\pgfqpoint{4.239602in}{2.140718in}}{\pgfqpoint{4.242874in}{2.148618in}}{\pgfqpoint{4.242874in}{2.156854in}}%
\pgfpathcurveto{\pgfqpoint{4.242874in}{2.165091in}}{\pgfqpoint{4.239602in}{2.172991in}}{\pgfqpoint{4.233778in}{2.178814in}}%
\pgfpathcurveto{\pgfqpoint{4.227954in}{2.184638in}}{\pgfqpoint{4.220054in}{2.187911in}}{\pgfqpoint{4.211818in}{2.187911in}}%
\pgfpathcurveto{\pgfqpoint{4.203581in}{2.187911in}}{\pgfqpoint{4.195681in}{2.184638in}}{\pgfqpoint{4.189857in}{2.178814in}}%
\pgfpathcurveto{\pgfqpoint{4.184033in}{2.172991in}}{\pgfqpoint{4.180761in}{2.165091in}}{\pgfqpoint{4.180761in}{2.156854in}}%
\pgfpathcurveto{\pgfqpoint{4.180761in}{2.148618in}}{\pgfqpoint{4.184033in}{2.140718in}}{\pgfqpoint{4.189857in}{2.134894in}}%
\pgfpathcurveto{\pgfqpoint{4.195681in}{2.129070in}}{\pgfqpoint{4.203581in}{2.125798in}}{\pgfqpoint{4.211818in}{2.125798in}}%
\pgfpathclose%
\pgfusepath{stroke,fill}%
\end{pgfscope}%
\begin{pgfscope}%
\pgfpathrectangle{\pgfqpoint{3.793912in}{0.557870in}}{\pgfqpoint{2.446088in}{1.684734in}}%
\pgfusepath{clip}%
\pgfsetbuttcap%
\pgfsetroundjoin%
\definecolor{currentfill}{rgb}{0.298039,0.447059,0.690196}%
\pgfsetfillcolor{currentfill}%
\pgfsetlinewidth{1.003750pt}%
\definecolor{currentstroke}{rgb}{0.298039,0.447059,0.690196}%
\pgfsetstrokecolor{currentstroke}%
\pgfsetdash{}{0pt}%
\pgfpathmoveto{\pgfqpoint{3.905098in}{2.125798in}}%
\pgfpathcurveto{\pgfqpoint{3.913334in}{2.125798in}}{\pgfqpoint{3.921234in}{2.129070in}}{\pgfqpoint{3.927058in}{2.134894in}}%
\pgfpathcurveto{\pgfqpoint{3.932882in}{2.140718in}}{\pgfqpoint{3.936155in}{2.148618in}}{\pgfqpoint{3.936155in}{2.156854in}}%
\pgfpathcurveto{\pgfqpoint{3.936155in}{2.165091in}}{\pgfqpoint{3.932882in}{2.172991in}}{\pgfqpoint{3.927058in}{2.178814in}}%
\pgfpathcurveto{\pgfqpoint{3.921234in}{2.184638in}}{\pgfqpoint{3.913334in}{2.187911in}}{\pgfqpoint{3.905098in}{2.187911in}}%
\pgfpathcurveto{\pgfqpoint{3.896862in}{2.187911in}}{\pgfqpoint{3.888962in}{2.184638in}}{\pgfqpoint{3.883138in}{2.178814in}}%
\pgfpathcurveto{\pgfqpoint{3.877314in}{2.172991in}}{\pgfqpoint{3.874042in}{2.165091in}}{\pgfqpoint{3.874042in}{2.156854in}}%
\pgfpathcurveto{\pgfqpoint{3.874042in}{2.148618in}}{\pgfqpoint{3.877314in}{2.140718in}}{\pgfqpoint{3.883138in}{2.134894in}}%
\pgfpathcurveto{\pgfqpoint{3.888962in}{2.129070in}}{\pgfqpoint{3.896862in}{2.125798in}}{\pgfqpoint{3.905098in}{2.125798in}}%
\pgfpathclose%
\pgfusepath{stroke,fill}%
\end{pgfscope}%
\begin{pgfscope}%
\pgfpathrectangle{\pgfqpoint{3.793912in}{0.557870in}}{\pgfqpoint{2.446088in}{1.684734in}}%
\pgfusepath{clip}%
\pgfsetbuttcap%
\pgfsetroundjoin%
\definecolor{currentfill}{rgb}{0.298039,0.447059,0.690196}%
\pgfsetfillcolor{currentfill}%
\pgfsetlinewidth{1.003750pt}%
\definecolor{currentstroke}{rgb}{0.298039,0.447059,0.690196}%
\pgfsetstrokecolor{currentstroke}%
\pgfsetdash{}{0pt}%
\pgfpathmoveto{\pgfqpoint{4.901936in}{1.447135in}}%
\pgfpathcurveto{\pgfqpoint{4.910173in}{1.447135in}}{\pgfqpoint{4.918073in}{1.450408in}}{\pgfqpoint{4.923897in}{1.456231in}}%
\pgfpathcurveto{\pgfqpoint{4.929721in}{1.462055in}}{\pgfqpoint{4.932993in}{1.469955in}}{\pgfqpoint{4.932993in}{1.478192in}}%
\pgfpathcurveto{\pgfqpoint{4.932993in}{1.486428in}}{\pgfqpoint{4.929721in}{1.494328in}}{\pgfqpoint{4.923897in}{1.500152in}}%
\pgfpathcurveto{\pgfqpoint{4.918073in}{1.505976in}}{\pgfqpoint{4.910173in}{1.509248in}}{\pgfqpoint{4.901936in}{1.509248in}}%
\pgfpathcurveto{\pgfqpoint{4.893700in}{1.509248in}}{\pgfqpoint{4.885800in}{1.505976in}}{\pgfqpoint{4.879976in}{1.500152in}}%
\pgfpathcurveto{\pgfqpoint{4.874152in}{1.494328in}}{\pgfqpoint{4.870880in}{1.486428in}}{\pgfqpoint{4.870880in}{1.478192in}}%
\pgfpathcurveto{\pgfqpoint{4.870880in}{1.469955in}}{\pgfqpoint{4.874152in}{1.462055in}}{\pgfqpoint{4.879976in}{1.456231in}}%
\pgfpathcurveto{\pgfqpoint{4.885800in}{1.450408in}}{\pgfqpoint{4.893700in}{1.447135in}}{\pgfqpoint{4.901936in}{1.447135in}}%
\pgfpathclose%
\pgfusepath{stroke,fill}%
\end{pgfscope}%
\begin{pgfscope}%
\pgfpathrectangle{\pgfqpoint{3.793912in}{0.557870in}}{\pgfqpoint{2.446088in}{1.684734in}}%
\pgfusepath{clip}%
\pgfsetbuttcap%
\pgfsetroundjoin%
\definecolor{currentfill}{rgb}{0.298039,0.447059,0.690196}%
\pgfsetfillcolor{currentfill}%
\pgfsetlinewidth{1.003750pt}%
\definecolor{currentstroke}{rgb}{0.298039,0.447059,0.690196}%
\pgfsetstrokecolor{currentstroke}%
\pgfsetdash{}{0pt}%
\pgfpathmoveto{\pgfqpoint{3.905098in}{2.116627in}}%
\pgfpathcurveto{\pgfqpoint{3.913334in}{2.116627in}}{\pgfqpoint{3.921234in}{2.119899in}}{\pgfqpoint{3.927058in}{2.125723in}}%
\pgfpathcurveto{\pgfqpoint{3.932882in}{2.131547in}}{\pgfqpoint{3.936155in}{2.139447in}}{\pgfqpoint{3.936155in}{2.147683in}}%
\pgfpathcurveto{\pgfqpoint{3.936155in}{2.155919in}}{\pgfqpoint{3.932882in}{2.163819in}}{\pgfqpoint{3.927058in}{2.169643in}}%
\pgfpathcurveto{\pgfqpoint{3.921234in}{2.175467in}}{\pgfqpoint{3.913334in}{2.178740in}}{\pgfqpoint{3.905098in}{2.178740in}}%
\pgfpathcurveto{\pgfqpoint{3.896862in}{2.178740in}}{\pgfqpoint{3.888962in}{2.175467in}}{\pgfqpoint{3.883138in}{2.169643in}}%
\pgfpathcurveto{\pgfqpoint{3.877314in}{2.163819in}}{\pgfqpoint{3.874042in}{2.155919in}}{\pgfqpoint{3.874042in}{2.147683in}}%
\pgfpathcurveto{\pgfqpoint{3.874042in}{2.139447in}}{\pgfqpoint{3.877314in}{2.131547in}}{\pgfqpoint{3.883138in}{2.125723in}}%
\pgfpathcurveto{\pgfqpoint{3.888962in}{2.119899in}}{\pgfqpoint{3.896862in}{2.116627in}}{\pgfqpoint{3.905098in}{2.116627in}}%
\pgfpathclose%
\pgfusepath{stroke,fill}%
\end{pgfscope}%
\begin{pgfscope}%
\pgfpathrectangle{\pgfqpoint{3.793912in}{0.557870in}}{\pgfqpoint{2.446088in}{1.684734in}}%
\pgfusepath{clip}%
\pgfsetbuttcap%
\pgfsetroundjoin%
\definecolor{currentfill}{rgb}{0.298039,0.447059,0.690196}%
\pgfsetfillcolor{currentfill}%
\pgfsetlinewidth{1.003750pt}%
\definecolor{currentstroke}{rgb}{0.298039,0.447059,0.690196}%
\pgfsetstrokecolor{currentstroke}%
\pgfsetdash{}{0pt}%
\pgfpathmoveto{\pgfqpoint{4.978616in}{1.474649in}}%
\pgfpathcurveto{\pgfqpoint{4.986852in}{1.474649in}}{\pgfqpoint{4.994753in}{1.477921in}}{\pgfqpoint{5.000576in}{1.483745in}}%
\pgfpathcurveto{\pgfqpoint{5.006400in}{1.489569in}}{\pgfqpoint{5.009673in}{1.497469in}}{\pgfqpoint{5.009673in}{1.505705in}}%
\pgfpathcurveto{\pgfqpoint{5.009673in}{1.513941in}}{\pgfqpoint{5.006400in}{1.521841in}}{\pgfqpoint{5.000576in}{1.527665in}}%
\pgfpathcurveto{\pgfqpoint{4.994753in}{1.533489in}}{\pgfqpoint{4.986852in}{1.536762in}}{\pgfqpoint{4.978616in}{1.536762in}}%
\pgfpathcurveto{\pgfqpoint{4.970380in}{1.536762in}}{\pgfqpoint{4.962480in}{1.533489in}}{\pgfqpoint{4.956656in}{1.527665in}}%
\pgfpathcurveto{\pgfqpoint{4.950832in}{1.521841in}}{\pgfqpoint{4.947560in}{1.513941in}}{\pgfqpoint{4.947560in}{1.505705in}}%
\pgfpathcurveto{\pgfqpoint{4.947560in}{1.497469in}}{\pgfqpoint{4.950832in}{1.489569in}}{\pgfqpoint{4.956656in}{1.483745in}}%
\pgfpathcurveto{\pgfqpoint{4.962480in}{1.477921in}}{\pgfqpoint{4.970380in}{1.474649in}}{\pgfqpoint{4.978616in}{1.474649in}}%
\pgfpathclose%
\pgfusepath{stroke,fill}%
\end{pgfscope}%
\begin{pgfscope}%
\pgfpathrectangle{\pgfqpoint{3.793912in}{0.557870in}}{\pgfqpoint{2.446088in}{1.684734in}}%
\pgfusepath{clip}%
\pgfsetbuttcap%
\pgfsetroundjoin%
\definecolor{currentfill}{rgb}{0.298039,0.447059,0.690196}%
\pgfsetfillcolor{currentfill}%
\pgfsetlinewidth{1.003750pt}%
\definecolor{currentstroke}{rgb}{0.298039,0.447059,0.690196}%
\pgfsetstrokecolor{currentstroke}%
\pgfsetdash{}{0pt}%
\pgfpathmoveto{\pgfqpoint{3.905098in}{2.125798in}}%
\pgfpathcurveto{\pgfqpoint{3.913334in}{2.125798in}}{\pgfqpoint{3.921234in}{2.129070in}}{\pgfqpoint{3.927058in}{2.134894in}}%
\pgfpathcurveto{\pgfqpoint{3.932882in}{2.140718in}}{\pgfqpoint{3.936155in}{2.148618in}}{\pgfqpoint{3.936155in}{2.156854in}}%
\pgfpathcurveto{\pgfqpoint{3.936155in}{2.165091in}}{\pgfqpoint{3.932882in}{2.172991in}}{\pgfqpoint{3.927058in}{2.178814in}}%
\pgfpathcurveto{\pgfqpoint{3.921234in}{2.184638in}}{\pgfqpoint{3.913334in}{2.187911in}}{\pgfqpoint{3.905098in}{2.187911in}}%
\pgfpathcurveto{\pgfqpoint{3.896862in}{2.187911in}}{\pgfqpoint{3.888962in}{2.184638in}}{\pgfqpoint{3.883138in}{2.178814in}}%
\pgfpathcurveto{\pgfqpoint{3.877314in}{2.172991in}}{\pgfqpoint{3.874042in}{2.165091in}}{\pgfqpoint{3.874042in}{2.156854in}}%
\pgfpathcurveto{\pgfqpoint{3.874042in}{2.148618in}}{\pgfqpoint{3.877314in}{2.140718in}}{\pgfqpoint{3.883138in}{2.134894in}}%
\pgfpathcurveto{\pgfqpoint{3.888962in}{2.129070in}}{\pgfqpoint{3.896862in}{2.125798in}}{\pgfqpoint{3.905098in}{2.125798in}}%
\pgfpathclose%
\pgfusepath{stroke,fill}%
\end{pgfscope}%
\begin{pgfscope}%
\pgfpathrectangle{\pgfqpoint{3.793912in}{0.557870in}}{\pgfqpoint{2.446088in}{1.684734in}}%
\pgfusepath{clip}%
\pgfsetbuttcap%
\pgfsetroundjoin%
\definecolor{currentfill}{rgb}{0.298039,0.447059,0.690196}%
\pgfsetfillcolor{currentfill}%
\pgfsetlinewidth{1.003750pt}%
\definecolor{currentstroke}{rgb}{0.298039,0.447059,0.690196}%
\pgfsetstrokecolor{currentstroke}%
\pgfsetdash{}{0pt}%
\pgfpathmoveto{\pgfqpoint{3.905098in}{2.125798in}}%
\pgfpathcurveto{\pgfqpoint{3.913334in}{2.125798in}}{\pgfqpoint{3.921234in}{2.129070in}}{\pgfqpoint{3.927058in}{2.134894in}}%
\pgfpathcurveto{\pgfqpoint{3.932882in}{2.140718in}}{\pgfqpoint{3.936155in}{2.148618in}}{\pgfqpoint{3.936155in}{2.156854in}}%
\pgfpathcurveto{\pgfqpoint{3.936155in}{2.165091in}}{\pgfqpoint{3.932882in}{2.172991in}}{\pgfqpoint{3.927058in}{2.178814in}}%
\pgfpathcurveto{\pgfqpoint{3.921234in}{2.184638in}}{\pgfqpoint{3.913334in}{2.187911in}}{\pgfqpoint{3.905098in}{2.187911in}}%
\pgfpathcurveto{\pgfqpoint{3.896862in}{2.187911in}}{\pgfqpoint{3.888962in}{2.184638in}}{\pgfqpoint{3.883138in}{2.178814in}}%
\pgfpathcurveto{\pgfqpoint{3.877314in}{2.172991in}}{\pgfqpoint{3.874042in}{2.165091in}}{\pgfqpoint{3.874042in}{2.156854in}}%
\pgfpathcurveto{\pgfqpoint{3.874042in}{2.148618in}}{\pgfqpoint{3.877314in}{2.140718in}}{\pgfqpoint{3.883138in}{2.134894in}}%
\pgfpathcurveto{\pgfqpoint{3.888962in}{2.129070in}}{\pgfqpoint{3.896862in}{2.125798in}}{\pgfqpoint{3.905098in}{2.125798in}}%
\pgfpathclose%
\pgfusepath{stroke,fill}%
\end{pgfscope}%
\begin{pgfscope}%
\pgfpathrectangle{\pgfqpoint{3.793912in}{0.557870in}}{\pgfqpoint{2.446088in}{1.684734in}}%
\pgfusepath{clip}%
\pgfsetbuttcap%
\pgfsetroundjoin%
\definecolor{currentfill}{rgb}{0.298039,0.447059,0.690196}%
\pgfsetfillcolor{currentfill}%
\pgfsetlinewidth{1.003750pt}%
\definecolor{currentstroke}{rgb}{0.298039,0.447059,0.690196}%
\pgfsetstrokecolor{currentstroke}%
\pgfsetdash{}{0pt}%
\pgfpathmoveto{\pgfqpoint{3.905098in}{2.116627in}}%
\pgfpathcurveto{\pgfqpoint{3.913334in}{2.116627in}}{\pgfqpoint{3.921234in}{2.119899in}}{\pgfqpoint{3.927058in}{2.125723in}}%
\pgfpathcurveto{\pgfqpoint{3.932882in}{2.131547in}}{\pgfqpoint{3.936155in}{2.139447in}}{\pgfqpoint{3.936155in}{2.147683in}}%
\pgfpathcurveto{\pgfqpoint{3.936155in}{2.155919in}}{\pgfqpoint{3.932882in}{2.163819in}}{\pgfqpoint{3.927058in}{2.169643in}}%
\pgfpathcurveto{\pgfqpoint{3.921234in}{2.175467in}}{\pgfqpoint{3.913334in}{2.178740in}}{\pgfqpoint{3.905098in}{2.178740in}}%
\pgfpathcurveto{\pgfqpoint{3.896862in}{2.178740in}}{\pgfqpoint{3.888962in}{2.175467in}}{\pgfqpoint{3.883138in}{2.169643in}}%
\pgfpathcurveto{\pgfqpoint{3.877314in}{2.163819in}}{\pgfqpoint{3.874042in}{2.155919in}}{\pgfqpoint{3.874042in}{2.147683in}}%
\pgfpathcurveto{\pgfqpoint{3.874042in}{2.139447in}}{\pgfqpoint{3.877314in}{2.131547in}}{\pgfqpoint{3.883138in}{2.125723in}}%
\pgfpathcurveto{\pgfqpoint{3.888962in}{2.119899in}}{\pgfqpoint{3.896862in}{2.116627in}}{\pgfqpoint{3.905098in}{2.116627in}}%
\pgfpathclose%
\pgfusepath{stroke,fill}%
\end{pgfscope}%
\begin{pgfscope}%
\pgfpathrectangle{\pgfqpoint{3.793912in}{0.557870in}}{\pgfqpoint{2.446088in}{1.684734in}}%
\pgfusepath{clip}%
\pgfsetbuttcap%
\pgfsetroundjoin%
\definecolor{currentfill}{rgb}{0.298039,0.447059,0.690196}%
\pgfsetfillcolor{currentfill}%
\pgfsetlinewidth{1.003750pt}%
\definecolor{currentstroke}{rgb}{0.298039,0.447059,0.690196}%
\pgfsetstrokecolor{currentstroke}%
\pgfsetdash{}{0pt}%
\pgfpathmoveto{\pgfqpoint{3.905098in}{2.125798in}}%
\pgfpathcurveto{\pgfqpoint{3.913334in}{2.125798in}}{\pgfqpoint{3.921234in}{2.129070in}}{\pgfqpoint{3.927058in}{2.134894in}}%
\pgfpathcurveto{\pgfqpoint{3.932882in}{2.140718in}}{\pgfqpoint{3.936155in}{2.148618in}}{\pgfqpoint{3.936155in}{2.156854in}}%
\pgfpathcurveto{\pgfqpoint{3.936155in}{2.165091in}}{\pgfqpoint{3.932882in}{2.172991in}}{\pgfqpoint{3.927058in}{2.178814in}}%
\pgfpathcurveto{\pgfqpoint{3.921234in}{2.184638in}}{\pgfqpoint{3.913334in}{2.187911in}}{\pgfqpoint{3.905098in}{2.187911in}}%
\pgfpathcurveto{\pgfqpoint{3.896862in}{2.187911in}}{\pgfqpoint{3.888962in}{2.184638in}}{\pgfqpoint{3.883138in}{2.178814in}}%
\pgfpathcurveto{\pgfqpoint{3.877314in}{2.172991in}}{\pgfqpoint{3.874042in}{2.165091in}}{\pgfqpoint{3.874042in}{2.156854in}}%
\pgfpathcurveto{\pgfqpoint{3.874042in}{2.148618in}}{\pgfqpoint{3.877314in}{2.140718in}}{\pgfqpoint{3.883138in}{2.134894in}}%
\pgfpathcurveto{\pgfqpoint{3.888962in}{2.129070in}}{\pgfqpoint{3.896862in}{2.125798in}}{\pgfqpoint{3.905098in}{2.125798in}}%
\pgfpathclose%
\pgfusepath{stroke,fill}%
\end{pgfscope}%
\begin{pgfscope}%
\pgfpathrectangle{\pgfqpoint{3.793912in}{0.557870in}}{\pgfqpoint{2.446088in}{1.684734in}}%
\pgfusepath{clip}%
\pgfsetbuttcap%
\pgfsetroundjoin%
\definecolor{currentfill}{rgb}{0.298039,0.447059,0.690196}%
\pgfsetfillcolor{currentfill}%
\pgfsetlinewidth{1.003750pt}%
\definecolor{currentstroke}{rgb}{0.298039,0.447059,0.690196}%
\pgfsetstrokecolor{currentstroke}%
\pgfsetdash{}{0pt}%
\pgfpathmoveto{\pgfqpoint{5.975454in}{1.327911in}}%
\pgfpathcurveto{\pgfqpoint{5.983691in}{1.327911in}}{\pgfqpoint{5.991591in}{1.331183in}}{\pgfqpoint{5.997415in}{1.337007in}}%
\pgfpathcurveto{\pgfqpoint{6.003239in}{1.342831in}}{\pgfqpoint{6.006511in}{1.350731in}}{\pgfqpoint{6.006511in}{1.358967in}}%
\pgfpathcurveto{\pgfqpoint{6.006511in}{1.367203in}}{\pgfqpoint{6.003239in}{1.375104in}}{\pgfqpoint{5.997415in}{1.380927in}}%
\pgfpathcurveto{\pgfqpoint{5.991591in}{1.386751in}}{\pgfqpoint{5.983691in}{1.390024in}}{\pgfqpoint{5.975454in}{1.390024in}}%
\pgfpathcurveto{\pgfqpoint{5.967218in}{1.390024in}}{\pgfqpoint{5.959318in}{1.386751in}}{\pgfqpoint{5.953494in}{1.380927in}}%
\pgfpathcurveto{\pgfqpoint{5.947670in}{1.375104in}}{\pgfqpoint{5.944398in}{1.367203in}}{\pgfqpoint{5.944398in}{1.358967in}}%
\pgfpathcurveto{\pgfqpoint{5.944398in}{1.350731in}}{\pgfqpoint{5.947670in}{1.342831in}}{\pgfqpoint{5.953494in}{1.337007in}}%
\pgfpathcurveto{\pgfqpoint{5.959318in}{1.331183in}}{\pgfqpoint{5.967218in}{1.327911in}}{\pgfqpoint{5.975454in}{1.327911in}}%
\pgfpathclose%
\pgfusepath{stroke,fill}%
\end{pgfscope}%
\begin{pgfscope}%
\pgfpathrectangle{\pgfqpoint{3.793912in}{0.557870in}}{\pgfqpoint{2.446088in}{1.684734in}}%
\pgfusepath{clip}%
\pgfsetbuttcap%
\pgfsetroundjoin%
\definecolor{currentfill}{rgb}{0.298039,0.447059,0.690196}%
\pgfsetfillcolor{currentfill}%
\pgfsetlinewidth{1.003750pt}%
\definecolor{currentstroke}{rgb}{0.298039,0.447059,0.690196}%
\pgfsetstrokecolor{currentstroke}%
\pgfsetdash{}{0pt}%
\pgfpathmoveto{\pgfqpoint{5.975454in}{1.263713in}}%
\pgfpathcurveto{\pgfqpoint{5.983691in}{1.263713in}}{\pgfqpoint{5.991591in}{1.266985in}}{\pgfqpoint{5.997415in}{1.272809in}}%
\pgfpathcurveto{\pgfqpoint{6.003239in}{1.278633in}}{\pgfqpoint{6.006511in}{1.286533in}}{\pgfqpoint{6.006511in}{1.294769in}}%
\pgfpathcurveto{\pgfqpoint{6.006511in}{1.303006in}}{\pgfqpoint{6.003239in}{1.310906in}}{\pgfqpoint{5.997415in}{1.316730in}}%
\pgfpathcurveto{\pgfqpoint{5.991591in}{1.322554in}}{\pgfqpoint{5.983691in}{1.325826in}}{\pgfqpoint{5.975454in}{1.325826in}}%
\pgfpathcurveto{\pgfqpoint{5.967218in}{1.325826in}}{\pgfqpoint{5.959318in}{1.322554in}}{\pgfqpoint{5.953494in}{1.316730in}}%
\pgfpathcurveto{\pgfqpoint{5.947670in}{1.310906in}}{\pgfqpoint{5.944398in}{1.303006in}}{\pgfqpoint{5.944398in}{1.294769in}}%
\pgfpathcurveto{\pgfqpoint{5.944398in}{1.286533in}}{\pgfqpoint{5.947670in}{1.278633in}}{\pgfqpoint{5.953494in}{1.272809in}}%
\pgfpathcurveto{\pgfqpoint{5.959318in}{1.266985in}}{\pgfqpoint{5.967218in}{1.263713in}}{\pgfqpoint{5.975454in}{1.263713in}}%
\pgfpathclose%
\pgfusepath{stroke,fill}%
\end{pgfscope}%
\begin{pgfscope}%
\pgfpathrectangle{\pgfqpoint{3.793912in}{0.557870in}}{\pgfqpoint{2.446088in}{1.684734in}}%
\pgfusepath{clip}%
\pgfsetbuttcap%
\pgfsetroundjoin%
\definecolor{currentfill}{rgb}{0.298039,0.447059,0.690196}%
\pgfsetfillcolor{currentfill}%
\pgfsetlinewidth{1.003750pt}%
\definecolor{currentstroke}{rgb}{0.298039,0.447059,0.690196}%
\pgfsetstrokecolor{currentstroke}%
\pgfsetdash{}{0pt}%
\pgfpathmoveto{\pgfqpoint{5.975454in}{1.318740in}}%
\pgfpathcurveto{\pgfqpoint{5.983691in}{1.318740in}}{\pgfqpoint{5.991591in}{1.322012in}}{\pgfqpoint{5.997415in}{1.327836in}}%
\pgfpathcurveto{\pgfqpoint{6.003239in}{1.333660in}}{\pgfqpoint{6.006511in}{1.341560in}}{\pgfqpoint{6.006511in}{1.349796in}}%
\pgfpathcurveto{\pgfqpoint{6.006511in}{1.358032in}}{\pgfqpoint{6.003239in}{1.365932in}}{\pgfqpoint{5.997415in}{1.371756in}}%
\pgfpathcurveto{\pgfqpoint{5.991591in}{1.377580in}}{\pgfqpoint{5.983691in}{1.380853in}}{\pgfqpoint{5.975454in}{1.380853in}}%
\pgfpathcurveto{\pgfqpoint{5.967218in}{1.380853in}}{\pgfqpoint{5.959318in}{1.377580in}}{\pgfqpoint{5.953494in}{1.371756in}}%
\pgfpathcurveto{\pgfqpoint{5.947670in}{1.365932in}}{\pgfqpoint{5.944398in}{1.358032in}}{\pgfqpoint{5.944398in}{1.349796in}}%
\pgfpathcurveto{\pgfqpoint{5.944398in}{1.341560in}}{\pgfqpoint{5.947670in}{1.333660in}}{\pgfqpoint{5.953494in}{1.327836in}}%
\pgfpathcurveto{\pgfqpoint{5.959318in}{1.322012in}}{\pgfqpoint{5.967218in}{1.318740in}}{\pgfqpoint{5.975454in}{1.318740in}}%
\pgfpathclose%
\pgfusepath{stroke,fill}%
\end{pgfscope}%
\begin{pgfscope}%
\pgfpathrectangle{\pgfqpoint{3.793912in}{0.557870in}}{\pgfqpoint{2.446088in}{1.684734in}}%
\pgfusepath{clip}%
\pgfsetbuttcap%
\pgfsetroundjoin%
\definecolor{currentfill}{rgb}{0.298039,0.447059,0.690196}%
\pgfsetfillcolor{currentfill}%
\pgfsetlinewidth{1.003750pt}%
\definecolor{currentstroke}{rgb}{0.298039,0.447059,0.690196}%
\pgfsetstrokecolor{currentstroke}%
\pgfsetdash{}{0pt}%
\pgfpathmoveto{\pgfqpoint{3.905098in}{2.107456in}}%
\pgfpathcurveto{\pgfqpoint{3.913334in}{2.107456in}}{\pgfqpoint{3.921234in}{2.110728in}}{\pgfqpoint{3.927058in}{2.116552in}}%
\pgfpathcurveto{\pgfqpoint{3.932882in}{2.122376in}}{\pgfqpoint{3.936155in}{2.130276in}}{\pgfqpoint{3.936155in}{2.138512in}}%
\pgfpathcurveto{\pgfqpoint{3.936155in}{2.146748in}}{\pgfqpoint{3.932882in}{2.154648in}}{\pgfqpoint{3.927058in}{2.160472in}}%
\pgfpathcurveto{\pgfqpoint{3.921234in}{2.166296in}}{\pgfqpoint{3.913334in}{2.169569in}}{\pgfqpoint{3.905098in}{2.169569in}}%
\pgfpathcurveto{\pgfqpoint{3.896862in}{2.169569in}}{\pgfqpoint{3.888962in}{2.166296in}}{\pgfqpoint{3.883138in}{2.160472in}}%
\pgfpathcurveto{\pgfqpoint{3.877314in}{2.154648in}}{\pgfqpoint{3.874042in}{2.146748in}}{\pgfqpoint{3.874042in}{2.138512in}}%
\pgfpathcurveto{\pgfqpoint{3.874042in}{2.130276in}}{\pgfqpoint{3.877314in}{2.122376in}}{\pgfqpoint{3.883138in}{2.116552in}}%
\pgfpathcurveto{\pgfqpoint{3.888962in}{2.110728in}}{\pgfqpoint{3.896862in}{2.107456in}}{\pgfqpoint{3.905098in}{2.107456in}}%
\pgfpathclose%
\pgfusepath{stroke,fill}%
\end{pgfscope}%
\begin{pgfscope}%
\pgfpathrectangle{\pgfqpoint{3.793912in}{0.557870in}}{\pgfqpoint{2.446088in}{1.684734in}}%
\pgfusepath{clip}%
\pgfsetbuttcap%
\pgfsetroundjoin%
\definecolor{currentfill}{rgb}{0.298039,0.447059,0.690196}%
\pgfsetfillcolor{currentfill}%
\pgfsetlinewidth{1.003750pt}%
\definecolor{currentstroke}{rgb}{0.298039,0.447059,0.690196}%
\pgfsetstrokecolor{currentstroke}%
\pgfsetdash{}{0pt}%
\pgfpathmoveto{\pgfqpoint{5.975454in}{1.309568in}}%
\pgfpathcurveto{\pgfqpoint{5.983691in}{1.309568in}}{\pgfqpoint{5.991591in}{1.312841in}}{\pgfqpoint{5.997415in}{1.318665in}}%
\pgfpathcurveto{\pgfqpoint{6.003239in}{1.324489in}}{\pgfqpoint{6.006511in}{1.332389in}}{\pgfqpoint{6.006511in}{1.340625in}}%
\pgfpathcurveto{\pgfqpoint{6.006511in}{1.348861in}}{\pgfqpoint{6.003239in}{1.356761in}}{\pgfqpoint{5.997415in}{1.362585in}}%
\pgfpathcurveto{\pgfqpoint{5.991591in}{1.368409in}}{\pgfqpoint{5.983691in}{1.371681in}}{\pgfqpoint{5.975454in}{1.371681in}}%
\pgfpathcurveto{\pgfqpoint{5.967218in}{1.371681in}}{\pgfqpoint{5.959318in}{1.368409in}}{\pgfqpoint{5.953494in}{1.362585in}}%
\pgfpathcurveto{\pgfqpoint{5.947670in}{1.356761in}}{\pgfqpoint{5.944398in}{1.348861in}}{\pgfqpoint{5.944398in}{1.340625in}}%
\pgfpathcurveto{\pgfqpoint{5.944398in}{1.332389in}}{\pgfqpoint{5.947670in}{1.324489in}}{\pgfqpoint{5.953494in}{1.318665in}}%
\pgfpathcurveto{\pgfqpoint{5.959318in}{1.312841in}}{\pgfqpoint{5.967218in}{1.309568in}}{\pgfqpoint{5.975454in}{1.309568in}}%
\pgfpathclose%
\pgfusepath{stroke,fill}%
\end{pgfscope}%
\begin{pgfscope}%
\pgfpathrectangle{\pgfqpoint{3.793912in}{0.557870in}}{\pgfqpoint{2.446088in}{1.684734in}}%
\pgfusepath{clip}%
\pgfsetbuttcap%
\pgfsetroundjoin%
\definecolor{currentfill}{rgb}{0.298039,0.447059,0.690196}%
\pgfsetfillcolor{currentfill}%
\pgfsetlinewidth{1.003750pt}%
\definecolor{currentstroke}{rgb}{0.298039,0.447059,0.690196}%
\pgfsetstrokecolor{currentstroke}%
\pgfsetdash{}{0pt}%
\pgfpathmoveto{\pgfqpoint{3.905098in}{2.125798in}}%
\pgfpathcurveto{\pgfqpoint{3.913334in}{2.125798in}}{\pgfqpoint{3.921234in}{2.129070in}}{\pgfqpoint{3.927058in}{2.134894in}}%
\pgfpathcurveto{\pgfqpoint{3.932882in}{2.140718in}}{\pgfqpoint{3.936155in}{2.148618in}}{\pgfqpoint{3.936155in}{2.156854in}}%
\pgfpathcurveto{\pgfqpoint{3.936155in}{2.165091in}}{\pgfqpoint{3.932882in}{2.172991in}}{\pgfqpoint{3.927058in}{2.178814in}}%
\pgfpathcurveto{\pgfqpoint{3.921234in}{2.184638in}}{\pgfqpoint{3.913334in}{2.187911in}}{\pgfqpoint{3.905098in}{2.187911in}}%
\pgfpathcurveto{\pgfqpoint{3.896862in}{2.187911in}}{\pgfqpoint{3.888962in}{2.184638in}}{\pgfqpoint{3.883138in}{2.178814in}}%
\pgfpathcurveto{\pgfqpoint{3.877314in}{2.172991in}}{\pgfqpoint{3.874042in}{2.165091in}}{\pgfqpoint{3.874042in}{2.156854in}}%
\pgfpathcurveto{\pgfqpoint{3.874042in}{2.148618in}}{\pgfqpoint{3.877314in}{2.140718in}}{\pgfqpoint{3.883138in}{2.134894in}}%
\pgfpathcurveto{\pgfqpoint{3.888962in}{2.129070in}}{\pgfqpoint{3.896862in}{2.125798in}}{\pgfqpoint{3.905098in}{2.125798in}}%
\pgfpathclose%
\pgfusepath{stroke,fill}%
\end{pgfscope}%
\begin{pgfscope}%
\pgfpathrectangle{\pgfqpoint{3.793912in}{0.557870in}}{\pgfqpoint{2.446088in}{1.684734in}}%
\pgfusepath{clip}%
\pgfsetbuttcap%
\pgfsetroundjoin%
\definecolor{currentfill}{rgb}{0.298039,0.447059,0.690196}%
\pgfsetfillcolor{currentfill}%
\pgfsetlinewidth{1.003750pt}%
\definecolor{currentstroke}{rgb}{0.298039,0.447059,0.690196}%
\pgfsetstrokecolor{currentstroke}%
\pgfsetdash{}{0pt}%
\pgfpathmoveto{\pgfqpoint{5.898775in}{1.419622in}}%
\pgfpathcurveto{\pgfqpoint{5.907011in}{1.419622in}}{\pgfqpoint{5.914911in}{1.422894in}}{\pgfqpoint{5.920735in}{1.428718in}}%
\pgfpathcurveto{\pgfqpoint{5.926559in}{1.434542in}}{\pgfqpoint{5.929831in}{1.442442in}}{\pgfqpoint{5.929831in}{1.450678in}}%
\pgfpathcurveto{\pgfqpoint{5.929831in}{1.458915in}}{\pgfqpoint{5.926559in}{1.466815in}}{\pgfqpoint{5.920735in}{1.472639in}}%
\pgfpathcurveto{\pgfqpoint{5.914911in}{1.478463in}}{\pgfqpoint{5.907011in}{1.481735in}}{\pgfqpoint{5.898775in}{1.481735in}}%
\pgfpathcurveto{\pgfqpoint{5.890538in}{1.481735in}}{\pgfqpoint{5.882638in}{1.478463in}}{\pgfqpoint{5.876814in}{1.472639in}}%
\pgfpathcurveto{\pgfqpoint{5.870990in}{1.466815in}}{\pgfqpoint{5.867718in}{1.458915in}}{\pgfqpoint{5.867718in}{1.450678in}}%
\pgfpathcurveto{\pgfqpoint{5.867718in}{1.442442in}}{\pgfqpoint{5.870990in}{1.434542in}}{\pgfqpoint{5.876814in}{1.428718in}}%
\pgfpathcurveto{\pgfqpoint{5.882638in}{1.422894in}}{\pgfqpoint{5.890538in}{1.419622in}}{\pgfqpoint{5.898775in}{1.419622in}}%
\pgfpathclose%
\pgfusepath{stroke,fill}%
\end{pgfscope}%
\begin{pgfscope}%
\pgfpathrectangle{\pgfqpoint{3.793912in}{0.557870in}}{\pgfqpoint{2.446088in}{1.684734in}}%
\pgfusepath{clip}%
\pgfsetbuttcap%
\pgfsetroundjoin%
\definecolor{currentfill}{rgb}{0.298039,0.447059,0.690196}%
\pgfsetfillcolor{currentfill}%
\pgfsetlinewidth{1.003750pt}%
\definecolor{currentstroke}{rgb}{0.298039,0.447059,0.690196}%
\pgfsetstrokecolor{currentstroke}%
\pgfsetdash{}{0pt}%
\pgfpathmoveto{\pgfqpoint{3.981778in}{2.125798in}}%
\pgfpathcurveto{\pgfqpoint{3.990014in}{2.125798in}}{\pgfqpoint{3.997914in}{2.129070in}}{\pgfqpoint{4.003738in}{2.134894in}}%
\pgfpathcurveto{\pgfqpoint{4.009562in}{2.140718in}}{\pgfqpoint{4.012834in}{2.148618in}}{\pgfqpoint{4.012834in}{2.156854in}}%
\pgfpathcurveto{\pgfqpoint{4.012834in}{2.165091in}}{\pgfqpoint{4.009562in}{2.172991in}}{\pgfqpoint{4.003738in}{2.178814in}}%
\pgfpathcurveto{\pgfqpoint{3.997914in}{2.184638in}}{\pgfqpoint{3.990014in}{2.187911in}}{\pgfqpoint{3.981778in}{2.187911in}}%
\pgfpathcurveto{\pgfqpoint{3.973542in}{2.187911in}}{\pgfqpoint{3.965642in}{2.184638in}}{\pgfqpoint{3.959818in}{2.178814in}}%
\pgfpathcurveto{\pgfqpoint{3.953994in}{2.172991in}}{\pgfqpoint{3.950721in}{2.165091in}}{\pgfqpoint{3.950721in}{2.156854in}}%
\pgfpathcurveto{\pgfqpoint{3.950721in}{2.148618in}}{\pgfqpoint{3.953994in}{2.140718in}}{\pgfqpoint{3.959818in}{2.134894in}}%
\pgfpathcurveto{\pgfqpoint{3.965642in}{2.129070in}}{\pgfqpoint{3.973542in}{2.125798in}}{\pgfqpoint{3.981778in}{2.125798in}}%
\pgfpathclose%
\pgfusepath{stroke,fill}%
\end{pgfscope}%
\begin{pgfscope}%
\pgfpathrectangle{\pgfqpoint{3.793912in}{0.557870in}}{\pgfqpoint{2.446088in}{1.684734in}}%
\pgfusepath{clip}%
\pgfsetbuttcap%
\pgfsetroundjoin%
\definecolor{currentfill}{rgb}{0.298039,0.447059,0.690196}%
\pgfsetfillcolor{currentfill}%
\pgfsetlinewidth{1.003750pt}%
\definecolor{currentstroke}{rgb}{0.298039,0.447059,0.690196}%
\pgfsetstrokecolor{currentstroke}%
\pgfsetdash{}{0pt}%
\pgfpathmoveto{\pgfqpoint{5.975454in}{1.263713in}}%
\pgfpathcurveto{\pgfqpoint{5.983691in}{1.263713in}}{\pgfqpoint{5.991591in}{1.266985in}}{\pgfqpoint{5.997415in}{1.272809in}}%
\pgfpathcurveto{\pgfqpoint{6.003239in}{1.278633in}}{\pgfqpoint{6.006511in}{1.286533in}}{\pgfqpoint{6.006511in}{1.294769in}}%
\pgfpathcurveto{\pgfqpoint{6.006511in}{1.303006in}}{\pgfqpoint{6.003239in}{1.310906in}}{\pgfqpoint{5.997415in}{1.316730in}}%
\pgfpathcurveto{\pgfqpoint{5.991591in}{1.322554in}}{\pgfqpoint{5.983691in}{1.325826in}}{\pgfqpoint{5.975454in}{1.325826in}}%
\pgfpathcurveto{\pgfqpoint{5.967218in}{1.325826in}}{\pgfqpoint{5.959318in}{1.322554in}}{\pgfqpoint{5.953494in}{1.316730in}}%
\pgfpathcurveto{\pgfqpoint{5.947670in}{1.310906in}}{\pgfqpoint{5.944398in}{1.303006in}}{\pgfqpoint{5.944398in}{1.294769in}}%
\pgfpathcurveto{\pgfqpoint{5.944398in}{1.286533in}}{\pgfqpoint{5.947670in}{1.278633in}}{\pgfqpoint{5.953494in}{1.272809in}}%
\pgfpathcurveto{\pgfqpoint{5.959318in}{1.266985in}}{\pgfqpoint{5.967218in}{1.263713in}}{\pgfqpoint{5.975454in}{1.263713in}}%
\pgfpathclose%
\pgfusepath{stroke,fill}%
\end{pgfscope}%
\begin{pgfscope}%
\pgfpathrectangle{\pgfqpoint{3.793912in}{0.557870in}}{\pgfqpoint{2.446088in}{1.684734in}}%
\pgfusepath{clip}%
\pgfsetbuttcap%
\pgfsetroundjoin%
\definecolor{currentfill}{rgb}{0.298039,0.447059,0.690196}%
\pgfsetfillcolor{currentfill}%
\pgfsetlinewidth{1.003750pt}%
\definecolor{currentstroke}{rgb}{0.298039,0.447059,0.690196}%
\pgfsetstrokecolor{currentstroke}%
\pgfsetdash{}{0pt}%
\pgfpathmoveto{\pgfqpoint{3.905098in}{2.125798in}}%
\pgfpathcurveto{\pgfqpoint{3.913334in}{2.125798in}}{\pgfqpoint{3.921234in}{2.129070in}}{\pgfqpoint{3.927058in}{2.134894in}}%
\pgfpathcurveto{\pgfqpoint{3.932882in}{2.140718in}}{\pgfqpoint{3.936155in}{2.148618in}}{\pgfqpoint{3.936155in}{2.156854in}}%
\pgfpathcurveto{\pgfqpoint{3.936155in}{2.165091in}}{\pgfqpoint{3.932882in}{2.172991in}}{\pgfqpoint{3.927058in}{2.178814in}}%
\pgfpathcurveto{\pgfqpoint{3.921234in}{2.184638in}}{\pgfqpoint{3.913334in}{2.187911in}}{\pgfqpoint{3.905098in}{2.187911in}}%
\pgfpathcurveto{\pgfqpoint{3.896862in}{2.187911in}}{\pgfqpoint{3.888962in}{2.184638in}}{\pgfqpoint{3.883138in}{2.178814in}}%
\pgfpathcurveto{\pgfqpoint{3.877314in}{2.172991in}}{\pgfqpoint{3.874042in}{2.165091in}}{\pgfqpoint{3.874042in}{2.156854in}}%
\pgfpathcurveto{\pgfqpoint{3.874042in}{2.148618in}}{\pgfqpoint{3.877314in}{2.140718in}}{\pgfqpoint{3.883138in}{2.134894in}}%
\pgfpathcurveto{\pgfqpoint{3.888962in}{2.129070in}}{\pgfqpoint{3.896862in}{2.125798in}}{\pgfqpoint{3.905098in}{2.125798in}}%
\pgfpathclose%
\pgfusepath{stroke,fill}%
\end{pgfscope}%
\begin{pgfscope}%
\pgfpathrectangle{\pgfqpoint{3.793912in}{0.557870in}}{\pgfqpoint{2.446088in}{1.684734in}}%
\pgfusepath{clip}%
\pgfsetbuttcap%
\pgfsetroundjoin%
\definecolor{currentfill}{rgb}{0.298039,0.447059,0.690196}%
\pgfsetfillcolor{currentfill}%
\pgfsetlinewidth{1.003750pt}%
\definecolor{currentstroke}{rgb}{0.298039,0.447059,0.690196}%
\pgfsetstrokecolor{currentstroke}%
\pgfsetdash{}{0pt}%
\pgfpathmoveto{\pgfqpoint{4.365177in}{2.116627in}}%
\pgfpathcurveto{\pgfqpoint{4.373414in}{2.116627in}}{\pgfqpoint{4.381314in}{2.119899in}}{\pgfqpoint{4.387137in}{2.125723in}}%
\pgfpathcurveto{\pgfqpoint{4.392961in}{2.131547in}}{\pgfqpoint{4.396234in}{2.139447in}}{\pgfqpoint{4.396234in}{2.147683in}}%
\pgfpathcurveto{\pgfqpoint{4.396234in}{2.155919in}}{\pgfqpoint{4.392961in}{2.163819in}}{\pgfqpoint{4.387137in}{2.169643in}}%
\pgfpathcurveto{\pgfqpoint{4.381314in}{2.175467in}}{\pgfqpoint{4.373414in}{2.178740in}}{\pgfqpoint{4.365177in}{2.178740in}}%
\pgfpathcurveto{\pgfqpoint{4.356941in}{2.178740in}}{\pgfqpoint{4.349041in}{2.175467in}}{\pgfqpoint{4.343217in}{2.169643in}}%
\pgfpathcurveto{\pgfqpoint{4.337393in}{2.163819in}}{\pgfqpoint{4.334121in}{2.155919in}}{\pgfqpoint{4.334121in}{2.147683in}}%
\pgfpathcurveto{\pgfqpoint{4.334121in}{2.139447in}}{\pgfqpoint{4.337393in}{2.131547in}}{\pgfqpoint{4.343217in}{2.125723in}}%
\pgfpathcurveto{\pgfqpoint{4.349041in}{2.119899in}}{\pgfqpoint{4.356941in}{2.116627in}}{\pgfqpoint{4.365177in}{2.116627in}}%
\pgfpathclose%
\pgfusepath{stroke,fill}%
\end{pgfscope}%
\begin{pgfscope}%
\pgfpathrectangle{\pgfqpoint{3.793912in}{0.557870in}}{\pgfqpoint{2.446088in}{1.684734in}}%
\pgfusepath{clip}%
\pgfsetbuttcap%
\pgfsetroundjoin%
\definecolor{currentfill}{rgb}{0.298039,0.447059,0.690196}%
\pgfsetfillcolor{currentfill}%
\pgfsetlinewidth{1.003750pt}%
\definecolor{currentstroke}{rgb}{0.298039,0.447059,0.690196}%
\pgfsetstrokecolor{currentstroke}%
\pgfsetdash{}{0pt}%
\pgfpathmoveto{\pgfqpoint{3.905098in}{2.125798in}}%
\pgfpathcurveto{\pgfqpoint{3.913334in}{2.125798in}}{\pgfqpoint{3.921234in}{2.129070in}}{\pgfqpoint{3.927058in}{2.134894in}}%
\pgfpathcurveto{\pgfqpoint{3.932882in}{2.140718in}}{\pgfqpoint{3.936155in}{2.148618in}}{\pgfqpoint{3.936155in}{2.156854in}}%
\pgfpathcurveto{\pgfqpoint{3.936155in}{2.165091in}}{\pgfqpoint{3.932882in}{2.172991in}}{\pgfqpoint{3.927058in}{2.178814in}}%
\pgfpathcurveto{\pgfqpoint{3.921234in}{2.184638in}}{\pgfqpoint{3.913334in}{2.187911in}}{\pgfqpoint{3.905098in}{2.187911in}}%
\pgfpathcurveto{\pgfqpoint{3.896862in}{2.187911in}}{\pgfqpoint{3.888962in}{2.184638in}}{\pgfqpoint{3.883138in}{2.178814in}}%
\pgfpathcurveto{\pgfqpoint{3.877314in}{2.172991in}}{\pgfqpoint{3.874042in}{2.165091in}}{\pgfqpoint{3.874042in}{2.156854in}}%
\pgfpathcurveto{\pgfqpoint{3.874042in}{2.148618in}}{\pgfqpoint{3.877314in}{2.140718in}}{\pgfqpoint{3.883138in}{2.134894in}}%
\pgfpathcurveto{\pgfqpoint{3.888962in}{2.129070in}}{\pgfqpoint{3.896862in}{2.125798in}}{\pgfqpoint{3.905098in}{2.125798in}}%
\pgfpathclose%
\pgfusepath{stroke,fill}%
\end{pgfscope}%
\begin{pgfscope}%
\pgfpathrectangle{\pgfqpoint{3.793912in}{0.557870in}}{\pgfqpoint{2.446088in}{1.684734in}}%
\pgfusepath{clip}%
\pgfsetbuttcap%
\pgfsetroundjoin%
\definecolor{currentfill}{rgb}{0.298039,0.447059,0.690196}%
\pgfsetfillcolor{currentfill}%
\pgfsetlinewidth{1.003750pt}%
\definecolor{currentstroke}{rgb}{0.298039,0.447059,0.690196}%
\pgfsetstrokecolor{currentstroke}%
\pgfsetdash{}{0pt}%
\pgfpathmoveto{\pgfqpoint{4.901936in}{1.795638in}}%
\pgfpathcurveto{\pgfqpoint{4.910173in}{1.795638in}}{\pgfqpoint{4.918073in}{1.798910in}}{\pgfqpoint{4.923897in}{1.804734in}}%
\pgfpathcurveto{\pgfqpoint{4.929721in}{1.810558in}}{\pgfqpoint{4.932993in}{1.818458in}}{\pgfqpoint{4.932993in}{1.826694in}}%
\pgfpathcurveto{\pgfqpoint{4.932993in}{1.834930in}}{\pgfqpoint{4.929721in}{1.842830in}}{\pgfqpoint{4.923897in}{1.848654in}}%
\pgfpathcurveto{\pgfqpoint{4.918073in}{1.854478in}}{\pgfqpoint{4.910173in}{1.857751in}}{\pgfqpoint{4.901936in}{1.857751in}}%
\pgfpathcurveto{\pgfqpoint{4.893700in}{1.857751in}}{\pgfqpoint{4.885800in}{1.854478in}}{\pgfqpoint{4.879976in}{1.848654in}}%
\pgfpathcurveto{\pgfqpoint{4.874152in}{1.842830in}}{\pgfqpoint{4.870880in}{1.834930in}}{\pgfqpoint{4.870880in}{1.826694in}}%
\pgfpathcurveto{\pgfqpoint{4.870880in}{1.818458in}}{\pgfqpoint{4.874152in}{1.810558in}}{\pgfqpoint{4.879976in}{1.804734in}}%
\pgfpathcurveto{\pgfqpoint{4.885800in}{1.798910in}}{\pgfqpoint{4.893700in}{1.795638in}}{\pgfqpoint{4.901936in}{1.795638in}}%
\pgfpathclose%
\pgfusepath{stroke,fill}%
\end{pgfscope}%
\begin{pgfscope}%
\pgfpathrectangle{\pgfqpoint{3.793912in}{0.557870in}}{\pgfqpoint{2.446088in}{1.684734in}}%
\pgfusepath{clip}%
\pgfsetbuttcap%
\pgfsetroundjoin%
\definecolor{currentfill}{rgb}{0.298039,0.447059,0.690196}%
\pgfsetfillcolor{currentfill}%
\pgfsetlinewidth{1.003750pt}%
\definecolor{currentstroke}{rgb}{0.298039,0.447059,0.690196}%
\pgfsetstrokecolor{currentstroke}%
\pgfsetdash{}{0pt}%
\pgfpathmoveto{\pgfqpoint{3.905098in}{2.125798in}}%
\pgfpathcurveto{\pgfqpoint{3.913334in}{2.125798in}}{\pgfqpoint{3.921234in}{2.129070in}}{\pgfqpoint{3.927058in}{2.134894in}}%
\pgfpathcurveto{\pgfqpoint{3.932882in}{2.140718in}}{\pgfqpoint{3.936155in}{2.148618in}}{\pgfqpoint{3.936155in}{2.156854in}}%
\pgfpathcurveto{\pgfqpoint{3.936155in}{2.165091in}}{\pgfqpoint{3.932882in}{2.172991in}}{\pgfqpoint{3.927058in}{2.178814in}}%
\pgfpathcurveto{\pgfqpoint{3.921234in}{2.184638in}}{\pgfqpoint{3.913334in}{2.187911in}}{\pgfqpoint{3.905098in}{2.187911in}}%
\pgfpathcurveto{\pgfqpoint{3.896862in}{2.187911in}}{\pgfqpoint{3.888962in}{2.184638in}}{\pgfqpoint{3.883138in}{2.178814in}}%
\pgfpathcurveto{\pgfqpoint{3.877314in}{2.172991in}}{\pgfqpoint{3.874042in}{2.165091in}}{\pgfqpoint{3.874042in}{2.156854in}}%
\pgfpathcurveto{\pgfqpoint{3.874042in}{2.148618in}}{\pgfqpoint{3.877314in}{2.140718in}}{\pgfqpoint{3.883138in}{2.134894in}}%
\pgfpathcurveto{\pgfqpoint{3.888962in}{2.129070in}}{\pgfqpoint{3.896862in}{2.125798in}}{\pgfqpoint{3.905098in}{2.125798in}}%
\pgfpathclose%
\pgfusepath{stroke,fill}%
\end{pgfscope}%
\begin{pgfscope}%
\pgfpathrectangle{\pgfqpoint{3.793912in}{0.557870in}}{\pgfqpoint{2.446088in}{1.684734in}}%
\pgfusepath{clip}%
\pgfsetbuttcap%
\pgfsetroundjoin%
\definecolor{currentfill}{rgb}{0.298039,0.447059,0.690196}%
\pgfsetfillcolor{currentfill}%
\pgfsetlinewidth{1.003750pt}%
\definecolor{currentstroke}{rgb}{0.298039,0.447059,0.690196}%
\pgfsetstrokecolor{currentstroke}%
\pgfsetdash{}{0pt}%
\pgfpathmoveto{\pgfqpoint{4.901936in}{1.859835in}}%
\pgfpathcurveto{\pgfqpoint{4.910173in}{1.859835in}}{\pgfqpoint{4.918073in}{1.863108in}}{\pgfqpoint{4.923897in}{1.868932in}}%
\pgfpathcurveto{\pgfqpoint{4.929721in}{1.874756in}}{\pgfqpoint{4.932993in}{1.882656in}}{\pgfqpoint{4.932993in}{1.890892in}}%
\pgfpathcurveto{\pgfqpoint{4.932993in}{1.899128in}}{\pgfqpoint{4.929721in}{1.907028in}}{\pgfqpoint{4.923897in}{1.912852in}}%
\pgfpathcurveto{\pgfqpoint{4.918073in}{1.918676in}}{\pgfqpoint{4.910173in}{1.921948in}}{\pgfqpoint{4.901936in}{1.921948in}}%
\pgfpathcurveto{\pgfqpoint{4.893700in}{1.921948in}}{\pgfqpoint{4.885800in}{1.918676in}}{\pgfqpoint{4.879976in}{1.912852in}}%
\pgfpathcurveto{\pgfqpoint{4.874152in}{1.907028in}}{\pgfqpoint{4.870880in}{1.899128in}}{\pgfqpoint{4.870880in}{1.890892in}}%
\pgfpathcurveto{\pgfqpoint{4.870880in}{1.882656in}}{\pgfqpoint{4.874152in}{1.874756in}}{\pgfqpoint{4.879976in}{1.868932in}}%
\pgfpathcurveto{\pgfqpoint{4.885800in}{1.863108in}}{\pgfqpoint{4.893700in}{1.859835in}}{\pgfqpoint{4.901936in}{1.859835in}}%
\pgfpathclose%
\pgfusepath{stroke,fill}%
\end{pgfscope}%
\begin{pgfscope}%
\pgfpathrectangle{\pgfqpoint{3.793912in}{0.557870in}}{\pgfqpoint{2.446088in}{1.684734in}}%
\pgfusepath{clip}%
\pgfsetbuttcap%
\pgfsetroundjoin%
\definecolor{currentfill}{rgb}{0.298039,0.447059,0.690196}%
\pgfsetfillcolor{currentfill}%
\pgfsetlinewidth{1.003750pt}%
\definecolor{currentstroke}{rgb}{0.298039,0.447059,0.690196}%
\pgfsetstrokecolor{currentstroke}%
\pgfsetdash{}{0pt}%
\pgfpathmoveto{\pgfqpoint{3.905098in}{2.125798in}}%
\pgfpathcurveto{\pgfqpoint{3.913334in}{2.125798in}}{\pgfqpoint{3.921234in}{2.129070in}}{\pgfqpoint{3.927058in}{2.134894in}}%
\pgfpathcurveto{\pgfqpoint{3.932882in}{2.140718in}}{\pgfqpoint{3.936155in}{2.148618in}}{\pgfqpoint{3.936155in}{2.156854in}}%
\pgfpathcurveto{\pgfqpoint{3.936155in}{2.165091in}}{\pgfqpoint{3.932882in}{2.172991in}}{\pgfqpoint{3.927058in}{2.178814in}}%
\pgfpathcurveto{\pgfqpoint{3.921234in}{2.184638in}}{\pgfqpoint{3.913334in}{2.187911in}}{\pgfqpoint{3.905098in}{2.187911in}}%
\pgfpathcurveto{\pgfqpoint{3.896862in}{2.187911in}}{\pgfqpoint{3.888962in}{2.184638in}}{\pgfqpoint{3.883138in}{2.178814in}}%
\pgfpathcurveto{\pgfqpoint{3.877314in}{2.172991in}}{\pgfqpoint{3.874042in}{2.165091in}}{\pgfqpoint{3.874042in}{2.156854in}}%
\pgfpathcurveto{\pgfqpoint{3.874042in}{2.148618in}}{\pgfqpoint{3.877314in}{2.140718in}}{\pgfqpoint{3.883138in}{2.134894in}}%
\pgfpathcurveto{\pgfqpoint{3.888962in}{2.129070in}}{\pgfqpoint{3.896862in}{2.125798in}}{\pgfqpoint{3.905098in}{2.125798in}}%
\pgfpathclose%
\pgfusepath{stroke,fill}%
\end{pgfscope}%
\begin{pgfscope}%
\pgfpathrectangle{\pgfqpoint{3.793912in}{0.557870in}}{\pgfqpoint{2.446088in}{1.684734in}}%
\pgfusepath{clip}%
\pgfsetbuttcap%
\pgfsetroundjoin%
\definecolor{currentfill}{rgb}{0.298039,0.447059,0.690196}%
\pgfsetfillcolor{currentfill}%
\pgfsetlinewidth{1.003750pt}%
\definecolor{currentstroke}{rgb}{0.298039,0.447059,0.690196}%
\pgfsetstrokecolor{currentstroke}%
\pgfsetdash{}{0pt}%
\pgfpathmoveto{\pgfqpoint{3.905098in}{2.125798in}}%
\pgfpathcurveto{\pgfqpoint{3.913334in}{2.125798in}}{\pgfqpoint{3.921234in}{2.129070in}}{\pgfqpoint{3.927058in}{2.134894in}}%
\pgfpathcurveto{\pgfqpoint{3.932882in}{2.140718in}}{\pgfqpoint{3.936155in}{2.148618in}}{\pgfqpoint{3.936155in}{2.156854in}}%
\pgfpathcurveto{\pgfqpoint{3.936155in}{2.165091in}}{\pgfqpoint{3.932882in}{2.172991in}}{\pgfqpoint{3.927058in}{2.178814in}}%
\pgfpathcurveto{\pgfqpoint{3.921234in}{2.184638in}}{\pgfqpoint{3.913334in}{2.187911in}}{\pgfqpoint{3.905098in}{2.187911in}}%
\pgfpathcurveto{\pgfqpoint{3.896862in}{2.187911in}}{\pgfqpoint{3.888962in}{2.184638in}}{\pgfqpoint{3.883138in}{2.178814in}}%
\pgfpathcurveto{\pgfqpoint{3.877314in}{2.172991in}}{\pgfqpoint{3.874042in}{2.165091in}}{\pgfqpoint{3.874042in}{2.156854in}}%
\pgfpathcurveto{\pgfqpoint{3.874042in}{2.148618in}}{\pgfqpoint{3.877314in}{2.140718in}}{\pgfqpoint{3.883138in}{2.134894in}}%
\pgfpathcurveto{\pgfqpoint{3.888962in}{2.129070in}}{\pgfqpoint{3.896862in}{2.125798in}}{\pgfqpoint{3.905098in}{2.125798in}}%
\pgfpathclose%
\pgfusepath{stroke,fill}%
\end{pgfscope}%
\begin{pgfscope}%
\pgfpathrectangle{\pgfqpoint{3.793912in}{0.557870in}}{\pgfqpoint{2.446088in}{1.684734in}}%
\pgfusepath{clip}%
\pgfsetbuttcap%
\pgfsetroundjoin%
\definecolor{currentfill}{rgb}{0.298039,0.447059,0.690196}%
\pgfsetfillcolor{currentfill}%
\pgfsetlinewidth{1.003750pt}%
\definecolor{currentstroke}{rgb}{0.298039,0.447059,0.690196}%
\pgfsetstrokecolor{currentstroke}%
\pgfsetdash{}{0pt}%
\pgfpathmoveto{\pgfqpoint{3.905098in}{2.125798in}}%
\pgfpathcurveto{\pgfqpoint{3.913334in}{2.125798in}}{\pgfqpoint{3.921234in}{2.129070in}}{\pgfqpoint{3.927058in}{2.134894in}}%
\pgfpathcurveto{\pgfqpoint{3.932882in}{2.140718in}}{\pgfqpoint{3.936155in}{2.148618in}}{\pgfqpoint{3.936155in}{2.156854in}}%
\pgfpathcurveto{\pgfqpoint{3.936155in}{2.165091in}}{\pgfqpoint{3.932882in}{2.172991in}}{\pgfqpoint{3.927058in}{2.178814in}}%
\pgfpathcurveto{\pgfqpoint{3.921234in}{2.184638in}}{\pgfqpoint{3.913334in}{2.187911in}}{\pgfqpoint{3.905098in}{2.187911in}}%
\pgfpathcurveto{\pgfqpoint{3.896862in}{2.187911in}}{\pgfqpoint{3.888962in}{2.184638in}}{\pgfqpoint{3.883138in}{2.178814in}}%
\pgfpathcurveto{\pgfqpoint{3.877314in}{2.172991in}}{\pgfqpoint{3.874042in}{2.165091in}}{\pgfqpoint{3.874042in}{2.156854in}}%
\pgfpathcurveto{\pgfqpoint{3.874042in}{2.148618in}}{\pgfqpoint{3.877314in}{2.140718in}}{\pgfqpoint{3.883138in}{2.134894in}}%
\pgfpathcurveto{\pgfqpoint{3.888962in}{2.129070in}}{\pgfqpoint{3.896862in}{2.125798in}}{\pgfqpoint{3.905098in}{2.125798in}}%
\pgfpathclose%
\pgfusepath{stroke,fill}%
\end{pgfscope}%
\begin{pgfscope}%
\pgfpathrectangle{\pgfqpoint{3.793912in}{0.557870in}}{\pgfqpoint{2.446088in}{1.684734in}}%
\pgfusepath{clip}%
\pgfsetbuttcap%
\pgfsetroundjoin%
\definecolor{currentfill}{rgb}{0.298039,0.447059,0.690196}%
\pgfsetfillcolor{currentfill}%
\pgfsetlinewidth{1.003750pt}%
\definecolor{currentstroke}{rgb}{0.298039,0.447059,0.690196}%
\pgfsetstrokecolor{currentstroke}%
\pgfsetdash{}{0pt}%
\pgfpathmoveto{\pgfqpoint{3.905098in}{2.125798in}}%
\pgfpathcurveto{\pgfqpoint{3.913334in}{2.125798in}}{\pgfqpoint{3.921234in}{2.129070in}}{\pgfqpoint{3.927058in}{2.134894in}}%
\pgfpathcurveto{\pgfqpoint{3.932882in}{2.140718in}}{\pgfqpoint{3.936155in}{2.148618in}}{\pgfqpoint{3.936155in}{2.156854in}}%
\pgfpathcurveto{\pgfqpoint{3.936155in}{2.165091in}}{\pgfqpoint{3.932882in}{2.172991in}}{\pgfqpoint{3.927058in}{2.178814in}}%
\pgfpathcurveto{\pgfqpoint{3.921234in}{2.184638in}}{\pgfqpoint{3.913334in}{2.187911in}}{\pgfqpoint{3.905098in}{2.187911in}}%
\pgfpathcurveto{\pgfqpoint{3.896862in}{2.187911in}}{\pgfqpoint{3.888962in}{2.184638in}}{\pgfqpoint{3.883138in}{2.178814in}}%
\pgfpathcurveto{\pgfqpoint{3.877314in}{2.172991in}}{\pgfqpoint{3.874042in}{2.165091in}}{\pgfqpoint{3.874042in}{2.156854in}}%
\pgfpathcurveto{\pgfqpoint{3.874042in}{2.148618in}}{\pgfqpoint{3.877314in}{2.140718in}}{\pgfqpoint{3.883138in}{2.134894in}}%
\pgfpathcurveto{\pgfqpoint{3.888962in}{2.129070in}}{\pgfqpoint{3.896862in}{2.125798in}}{\pgfqpoint{3.905098in}{2.125798in}}%
\pgfpathclose%
\pgfusepath{stroke,fill}%
\end{pgfscope}%
\begin{pgfscope}%
\pgfpathrectangle{\pgfqpoint{3.793912in}{0.557870in}}{\pgfqpoint{2.446088in}{1.684734in}}%
\pgfusepath{clip}%
\pgfsetbuttcap%
\pgfsetroundjoin%
\definecolor{currentfill}{rgb}{0.298039,0.447059,0.690196}%
\pgfsetfillcolor{currentfill}%
\pgfsetlinewidth{1.003750pt}%
\definecolor{currentstroke}{rgb}{0.298039,0.447059,0.690196}%
\pgfsetstrokecolor{currentstroke}%
\pgfsetdash{}{0pt}%
\pgfpathmoveto{\pgfqpoint{3.905098in}{2.125798in}}%
\pgfpathcurveto{\pgfqpoint{3.913334in}{2.125798in}}{\pgfqpoint{3.921234in}{2.129070in}}{\pgfqpoint{3.927058in}{2.134894in}}%
\pgfpathcurveto{\pgfqpoint{3.932882in}{2.140718in}}{\pgfqpoint{3.936155in}{2.148618in}}{\pgfqpoint{3.936155in}{2.156854in}}%
\pgfpathcurveto{\pgfqpoint{3.936155in}{2.165091in}}{\pgfqpoint{3.932882in}{2.172991in}}{\pgfqpoint{3.927058in}{2.178814in}}%
\pgfpathcurveto{\pgfqpoint{3.921234in}{2.184638in}}{\pgfqpoint{3.913334in}{2.187911in}}{\pgfqpoint{3.905098in}{2.187911in}}%
\pgfpathcurveto{\pgfqpoint{3.896862in}{2.187911in}}{\pgfqpoint{3.888962in}{2.184638in}}{\pgfqpoint{3.883138in}{2.178814in}}%
\pgfpathcurveto{\pgfqpoint{3.877314in}{2.172991in}}{\pgfqpoint{3.874042in}{2.165091in}}{\pgfqpoint{3.874042in}{2.156854in}}%
\pgfpathcurveto{\pgfqpoint{3.874042in}{2.148618in}}{\pgfqpoint{3.877314in}{2.140718in}}{\pgfqpoint{3.883138in}{2.134894in}}%
\pgfpathcurveto{\pgfqpoint{3.888962in}{2.129070in}}{\pgfqpoint{3.896862in}{2.125798in}}{\pgfqpoint{3.905098in}{2.125798in}}%
\pgfpathclose%
\pgfusepath{stroke,fill}%
\end{pgfscope}%
\begin{pgfscope}%
\pgfpathrectangle{\pgfqpoint{3.793912in}{0.557870in}}{\pgfqpoint{2.446088in}{1.684734in}}%
\pgfusepath{clip}%
\pgfsetbuttcap%
\pgfsetroundjoin%
\definecolor{currentfill}{rgb}{0.298039,0.447059,0.690196}%
\pgfsetfillcolor{currentfill}%
\pgfsetlinewidth{1.003750pt}%
\definecolor{currentstroke}{rgb}{0.298039,0.447059,0.690196}%
\pgfsetstrokecolor{currentstroke}%
\pgfsetdash{}{0pt}%
\pgfpathmoveto{\pgfqpoint{3.905098in}{2.125798in}}%
\pgfpathcurveto{\pgfqpoint{3.913334in}{2.125798in}}{\pgfqpoint{3.921234in}{2.129070in}}{\pgfqpoint{3.927058in}{2.134894in}}%
\pgfpathcurveto{\pgfqpoint{3.932882in}{2.140718in}}{\pgfqpoint{3.936155in}{2.148618in}}{\pgfqpoint{3.936155in}{2.156854in}}%
\pgfpathcurveto{\pgfqpoint{3.936155in}{2.165091in}}{\pgfqpoint{3.932882in}{2.172991in}}{\pgfqpoint{3.927058in}{2.178814in}}%
\pgfpathcurveto{\pgfqpoint{3.921234in}{2.184638in}}{\pgfqpoint{3.913334in}{2.187911in}}{\pgfqpoint{3.905098in}{2.187911in}}%
\pgfpathcurveto{\pgfqpoint{3.896862in}{2.187911in}}{\pgfqpoint{3.888962in}{2.184638in}}{\pgfqpoint{3.883138in}{2.178814in}}%
\pgfpathcurveto{\pgfqpoint{3.877314in}{2.172991in}}{\pgfqpoint{3.874042in}{2.165091in}}{\pgfqpoint{3.874042in}{2.156854in}}%
\pgfpathcurveto{\pgfqpoint{3.874042in}{2.148618in}}{\pgfqpoint{3.877314in}{2.140718in}}{\pgfqpoint{3.883138in}{2.134894in}}%
\pgfpathcurveto{\pgfqpoint{3.888962in}{2.129070in}}{\pgfqpoint{3.896862in}{2.125798in}}{\pgfqpoint{3.905098in}{2.125798in}}%
\pgfpathclose%
\pgfusepath{stroke,fill}%
\end{pgfscope}%
\begin{pgfscope}%
\pgfpathrectangle{\pgfqpoint{3.793912in}{0.557870in}}{\pgfqpoint{2.446088in}{1.684734in}}%
\pgfusepath{clip}%
\pgfsetbuttcap%
\pgfsetroundjoin%
\definecolor{currentfill}{rgb}{0.298039,0.447059,0.690196}%
\pgfsetfillcolor{currentfill}%
\pgfsetlinewidth{1.003750pt}%
\definecolor{currentstroke}{rgb}{0.298039,0.447059,0.690196}%
\pgfsetstrokecolor{currentstroke}%
\pgfsetdash{}{0pt}%
\pgfpathmoveto{\pgfqpoint{4.978616in}{1.392109in}}%
\pgfpathcurveto{\pgfqpoint{4.986852in}{1.392109in}}{\pgfqpoint{4.994753in}{1.395381in}}{\pgfqpoint{5.000576in}{1.401205in}}%
\pgfpathcurveto{\pgfqpoint{5.006400in}{1.407029in}}{\pgfqpoint{5.009673in}{1.414929in}}{\pgfqpoint{5.009673in}{1.423165in}}%
\pgfpathcurveto{\pgfqpoint{5.009673in}{1.431401in}}{\pgfqpoint{5.006400in}{1.439301in}}{\pgfqpoint{5.000576in}{1.445125in}}%
\pgfpathcurveto{\pgfqpoint{4.994753in}{1.450949in}}{\pgfqpoint{4.986852in}{1.454222in}}{\pgfqpoint{4.978616in}{1.454222in}}%
\pgfpathcurveto{\pgfqpoint{4.970380in}{1.454222in}}{\pgfqpoint{4.962480in}{1.450949in}}{\pgfqpoint{4.956656in}{1.445125in}}%
\pgfpathcurveto{\pgfqpoint{4.950832in}{1.439301in}}{\pgfqpoint{4.947560in}{1.431401in}}{\pgfqpoint{4.947560in}{1.423165in}}%
\pgfpathcurveto{\pgfqpoint{4.947560in}{1.414929in}}{\pgfqpoint{4.950832in}{1.407029in}}{\pgfqpoint{4.956656in}{1.401205in}}%
\pgfpathcurveto{\pgfqpoint{4.962480in}{1.395381in}}{\pgfqpoint{4.970380in}{1.392109in}}{\pgfqpoint{4.978616in}{1.392109in}}%
\pgfpathclose%
\pgfusepath{stroke,fill}%
\end{pgfscope}%
\begin{pgfscope}%
\pgfpathrectangle{\pgfqpoint{3.793912in}{0.557870in}}{\pgfqpoint{2.446088in}{1.684734in}}%
\pgfusepath{clip}%
\pgfsetbuttcap%
\pgfsetroundjoin%
\definecolor{currentfill}{rgb}{0.298039,0.447059,0.690196}%
\pgfsetfillcolor{currentfill}%
\pgfsetlinewidth{1.003750pt}%
\definecolor{currentstroke}{rgb}{0.298039,0.447059,0.690196}%
\pgfsetstrokecolor{currentstroke}%
\pgfsetdash{}{0pt}%
\pgfpathmoveto{\pgfqpoint{3.905098in}{2.125798in}}%
\pgfpathcurveto{\pgfqpoint{3.913334in}{2.125798in}}{\pgfqpoint{3.921234in}{2.129070in}}{\pgfqpoint{3.927058in}{2.134894in}}%
\pgfpathcurveto{\pgfqpoint{3.932882in}{2.140718in}}{\pgfqpoint{3.936155in}{2.148618in}}{\pgfqpoint{3.936155in}{2.156854in}}%
\pgfpathcurveto{\pgfqpoint{3.936155in}{2.165091in}}{\pgfqpoint{3.932882in}{2.172991in}}{\pgfqpoint{3.927058in}{2.178814in}}%
\pgfpathcurveto{\pgfqpoint{3.921234in}{2.184638in}}{\pgfqpoint{3.913334in}{2.187911in}}{\pgfqpoint{3.905098in}{2.187911in}}%
\pgfpathcurveto{\pgfqpoint{3.896862in}{2.187911in}}{\pgfqpoint{3.888962in}{2.184638in}}{\pgfqpoint{3.883138in}{2.178814in}}%
\pgfpathcurveto{\pgfqpoint{3.877314in}{2.172991in}}{\pgfqpoint{3.874042in}{2.165091in}}{\pgfqpoint{3.874042in}{2.156854in}}%
\pgfpathcurveto{\pgfqpoint{3.874042in}{2.148618in}}{\pgfqpoint{3.877314in}{2.140718in}}{\pgfqpoint{3.883138in}{2.134894in}}%
\pgfpathcurveto{\pgfqpoint{3.888962in}{2.129070in}}{\pgfqpoint{3.896862in}{2.125798in}}{\pgfqpoint{3.905098in}{2.125798in}}%
\pgfpathclose%
\pgfusepath{stroke,fill}%
\end{pgfscope}%
\begin{pgfscope}%
\pgfpathrectangle{\pgfqpoint{3.793912in}{0.557870in}}{\pgfqpoint{2.446088in}{1.684734in}}%
\pgfusepath{clip}%
\pgfsetbuttcap%
\pgfsetroundjoin%
\definecolor{currentfill}{rgb}{0.298039,0.447059,0.690196}%
\pgfsetfillcolor{currentfill}%
\pgfsetlinewidth{1.003750pt}%
\definecolor{currentstroke}{rgb}{0.298039,0.447059,0.690196}%
\pgfsetstrokecolor{currentstroke}%
\pgfsetdash{}{0pt}%
\pgfpathmoveto{\pgfqpoint{3.905098in}{2.125798in}}%
\pgfpathcurveto{\pgfqpoint{3.913334in}{2.125798in}}{\pgfqpoint{3.921234in}{2.129070in}}{\pgfqpoint{3.927058in}{2.134894in}}%
\pgfpathcurveto{\pgfqpoint{3.932882in}{2.140718in}}{\pgfqpoint{3.936155in}{2.148618in}}{\pgfqpoint{3.936155in}{2.156854in}}%
\pgfpathcurveto{\pgfqpoint{3.936155in}{2.165091in}}{\pgfqpoint{3.932882in}{2.172991in}}{\pgfqpoint{3.927058in}{2.178814in}}%
\pgfpathcurveto{\pgfqpoint{3.921234in}{2.184638in}}{\pgfqpoint{3.913334in}{2.187911in}}{\pgfqpoint{3.905098in}{2.187911in}}%
\pgfpathcurveto{\pgfqpoint{3.896862in}{2.187911in}}{\pgfqpoint{3.888962in}{2.184638in}}{\pgfqpoint{3.883138in}{2.178814in}}%
\pgfpathcurveto{\pgfqpoint{3.877314in}{2.172991in}}{\pgfqpoint{3.874042in}{2.165091in}}{\pgfqpoint{3.874042in}{2.156854in}}%
\pgfpathcurveto{\pgfqpoint{3.874042in}{2.148618in}}{\pgfqpoint{3.877314in}{2.140718in}}{\pgfqpoint{3.883138in}{2.134894in}}%
\pgfpathcurveto{\pgfqpoint{3.888962in}{2.129070in}}{\pgfqpoint{3.896862in}{2.125798in}}{\pgfqpoint{3.905098in}{2.125798in}}%
\pgfpathclose%
\pgfusepath{stroke,fill}%
\end{pgfscope}%
\begin{pgfscope}%
\pgfpathrectangle{\pgfqpoint{3.793912in}{0.557870in}}{\pgfqpoint{2.446088in}{1.684734in}}%
\pgfusepath{clip}%
\pgfsetbuttcap%
\pgfsetroundjoin%
\definecolor{currentfill}{rgb}{0.298039,0.447059,0.690196}%
\pgfsetfillcolor{currentfill}%
\pgfsetlinewidth{1.003750pt}%
\definecolor{currentstroke}{rgb}{0.298039,0.447059,0.690196}%
\pgfsetstrokecolor{currentstroke}%
\pgfsetdash{}{0pt}%
\pgfpathmoveto{\pgfqpoint{3.905098in}{2.125798in}}%
\pgfpathcurveto{\pgfqpoint{3.913334in}{2.125798in}}{\pgfqpoint{3.921234in}{2.129070in}}{\pgfqpoint{3.927058in}{2.134894in}}%
\pgfpathcurveto{\pgfqpoint{3.932882in}{2.140718in}}{\pgfqpoint{3.936155in}{2.148618in}}{\pgfqpoint{3.936155in}{2.156854in}}%
\pgfpathcurveto{\pgfqpoint{3.936155in}{2.165091in}}{\pgfqpoint{3.932882in}{2.172991in}}{\pgfqpoint{3.927058in}{2.178814in}}%
\pgfpathcurveto{\pgfqpoint{3.921234in}{2.184638in}}{\pgfqpoint{3.913334in}{2.187911in}}{\pgfqpoint{3.905098in}{2.187911in}}%
\pgfpathcurveto{\pgfqpoint{3.896862in}{2.187911in}}{\pgfqpoint{3.888962in}{2.184638in}}{\pgfqpoint{3.883138in}{2.178814in}}%
\pgfpathcurveto{\pgfqpoint{3.877314in}{2.172991in}}{\pgfqpoint{3.874042in}{2.165091in}}{\pgfqpoint{3.874042in}{2.156854in}}%
\pgfpathcurveto{\pgfqpoint{3.874042in}{2.148618in}}{\pgfqpoint{3.877314in}{2.140718in}}{\pgfqpoint{3.883138in}{2.134894in}}%
\pgfpathcurveto{\pgfqpoint{3.888962in}{2.129070in}}{\pgfqpoint{3.896862in}{2.125798in}}{\pgfqpoint{3.905098in}{2.125798in}}%
\pgfpathclose%
\pgfusepath{stroke,fill}%
\end{pgfscope}%
\begin{pgfscope}%
\pgfpathrectangle{\pgfqpoint{3.793912in}{0.557870in}}{\pgfqpoint{2.446088in}{1.684734in}}%
\pgfusepath{clip}%
\pgfsetbuttcap%
\pgfsetroundjoin%
\definecolor{currentfill}{rgb}{0.298039,0.447059,0.690196}%
\pgfsetfillcolor{currentfill}%
\pgfsetlinewidth{1.003750pt}%
\definecolor{currentstroke}{rgb}{0.298039,0.447059,0.690196}%
\pgfsetstrokecolor{currentstroke}%
\pgfsetdash{}{0pt}%
\pgfpathmoveto{\pgfqpoint{3.905098in}{2.125798in}}%
\pgfpathcurveto{\pgfqpoint{3.913334in}{2.125798in}}{\pgfqpoint{3.921234in}{2.129070in}}{\pgfqpoint{3.927058in}{2.134894in}}%
\pgfpathcurveto{\pgfqpoint{3.932882in}{2.140718in}}{\pgfqpoint{3.936155in}{2.148618in}}{\pgfqpoint{3.936155in}{2.156854in}}%
\pgfpathcurveto{\pgfqpoint{3.936155in}{2.165091in}}{\pgfqpoint{3.932882in}{2.172991in}}{\pgfqpoint{3.927058in}{2.178814in}}%
\pgfpathcurveto{\pgfqpoint{3.921234in}{2.184638in}}{\pgfqpoint{3.913334in}{2.187911in}}{\pgfqpoint{3.905098in}{2.187911in}}%
\pgfpathcurveto{\pgfqpoint{3.896862in}{2.187911in}}{\pgfqpoint{3.888962in}{2.184638in}}{\pgfqpoint{3.883138in}{2.178814in}}%
\pgfpathcurveto{\pgfqpoint{3.877314in}{2.172991in}}{\pgfqpoint{3.874042in}{2.165091in}}{\pgfqpoint{3.874042in}{2.156854in}}%
\pgfpathcurveto{\pgfqpoint{3.874042in}{2.148618in}}{\pgfqpoint{3.877314in}{2.140718in}}{\pgfqpoint{3.883138in}{2.134894in}}%
\pgfpathcurveto{\pgfqpoint{3.888962in}{2.129070in}}{\pgfqpoint{3.896862in}{2.125798in}}{\pgfqpoint{3.905098in}{2.125798in}}%
\pgfpathclose%
\pgfusepath{stroke,fill}%
\end{pgfscope}%
\begin{pgfscope}%
\pgfpathrectangle{\pgfqpoint{3.793912in}{0.557870in}}{\pgfqpoint{2.446088in}{1.684734in}}%
\pgfusepath{clip}%
\pgfsetbuttcap%
\pgfsetroundjoin%
\definecolor{currentfill}{rgb}{0.298039,0.447059,0.690196}%
\pgfsetfillcolor{currentfill}%
\pgfsetlinewidth{1.003750pt}%
\definecolor{currentstroke}{rgb}{0.298039,0.447059,0.690196}%
\pgfsetstrokecolor{currentstroke}%
\pgfsetdash{}{0pt}%
\pgfpathmoveto{\pgfqpoint{5.898775in}{1.346253in}}%
\pgfpathcurveto{\pgfqpoint{5.907011in}{1.346253in}}{\pgfqpoint{5.914911in}{1.349525in}}{\pgfqpoint{5.920735in}{1.355349in}}%
\pgfpathcurveto{\pgfqpoint{5.926559in}{1.361173in}}{\pgfqpoint{5.929831in}{1.369073in}}{\pgfqpoint{5.929831in}{1.377309in}}%
\pgfpathcurveto{\pgfqpoint{5.929831in}{1.385546in}}{\pgfqpoint{5.926559in}{1.393446in}}{\pgfqpoint{5.920735in}{1.399270in}}%
\pgfpathcurveto{\pgfqpoint{5.914911in}{1.405094in}}{\pgfqpoint{5.907011in}{1.408366in}}{\pgfqpoint{5.898775in}{1.408366in}}%
\pgfpathcurveto{\pgfqpoint{5.890538in}{1.408366in}}{\pgfqpoint{5.882638in}{1.405094in}}{\pgfqpoint{5.876814in}{1.399270in}}%
\pgfpathcurveto{\pgfqpoint{5.870990in}{1.393446in}}{\pgfqpoint{5.867718in}{1.385546in}}{\pgfqpoint{5.867718in}{1.377309in}}%
\pgfpathcurveto{\pgfqpoint{5.867718in}{1.369073in}}{\pgfqpoint{5.870990in}{1.361173in}}{\pgfqpoint{5.876814in}{1.355349in}}%
\pgfpathcurveto{\pgfqpoint{5.882638in}{1.349525in}}{\pgfqpoint{5.890538in}{1.346253in}}{\pgfqpoint{5.898775in}{1.346253in}}%
\pgfpathclose%
\pgfusepath{stroke,fill}%
\end{pgfscope}%
\begin{pgfscope}%
\pgfpathrectangle{\pgfqpoint{3.793912in}{0.557870in}}{\pgfqpoint{2.446088in}{1.684734in}}%
\pgfusepath{clip}%
\pgfsetbuttcap%
\pgfsetroundjoin%
\definecolor{currentfill}{rgb}{0.298039,0.447059,0.690196}%
\pgfsetfillcolor{currentfill}%
\pgfsetlinewidth{1.003750pt}%
\definecolor{currentstroke}{rgb}{0.298039,0.447059,0.690196}%
\pgfsetstrokecolor{currentstroke}%
\pgfsetdash{}{0pt}%
\pgfpathmoveto{\pgfqpoint{5.975454in}{1.291226in}}%
\pgfpathcurveto{\pgfqpoint{5.983691in}{1.291226in}}{\pgfqpoint{5.991591in}{1.294499in}}{\pgfqpoint{5.997415in}{1.300322in}}%
\pgfpathcurveto{\pgfqpoint{6.003239in}{1.306146in}}{\pgfqpoint{6.006511in}{1.314046in}}{\pgfqpoint{6.006511in}{1.322283in}}%
\pgfpathcurveto{\pgfqpoint{6.006511in}{1.330519in}}{\pgfqpoint{6.003239in}{1.338419in}}{\pgfqpoint{5.997415in}{1.344243in}}%
\pgfpathcurveto{\pgfqpoint{5.991591in}{1.350067in}}{\pgfqpoint{5.983691in}{1.353339in}}{\pgfqpoint{5.975454in}{1.353339in}}%
\pgfpathcurveto{\pgfqpoint{5.967218in}{1.353339in}}{\pgfqpoint{5.959318in}{1.350067in}}{\pgfqpoint{5.953494in}{1.344243in}}%
\pgfpathcurveto{\pgfqpoint{5.947670in}{1.338419in}}{\pgfqpoint{5.944398in}{1.330519in}}{\pgfqpoint{5.944398in}{1.322283in}}%
\pgfpathcurveto{\pgfqpoint{5.944398in}{1.314046in}}{\pgfqpoint{5.947670in}{1.306146in}}{\pgfqpoint{5.953494in}{1.300322in}}%
\pgfpathcurveto{\pgfqpoint{5.959318in}{1.294499in}}{\pgfqpoint{5.967218in}{1.291226in}}{\pgfqpoint{5.975454in}{1.291226in}}%
\pgfpathclose%
\pgfusepath{stroke,fill}%
\end{pgfscope}%
\begin{pgfscope}%
\pgfpathrectangle{\pgfqpoint{3.793912in}{0.557870in}}{\pgfqpoint{2.446088in}{1.684734in}}%
\pgfusepath{clip}%
\pgfsetbuttcap%
\pgfsetroundjoin%
\definecolor{currentfill}{rgb}{0.298039,0.447059,0.690196}%
\pgfsetfillcolor{currentfill}%
\pgfsetlinewidth{1.003750pt}%
\definecolor{currentstroke}{rgb}{0.298039,0.447059,0.690196}%
\pgfsetstrokecolor{currentstroke}%
\pgfsetdash{}{0pt}%
\pgfpathmoveto{\pgfqpoint{4.058458in}{1.492991in}}%
\pgfpathcurveto{\pgfqpoint{4.066694in}{1.492991in}}{\pgfqpoint{4.074594in}{1.496263in}}{\pgfqpoint{4.080418in}{1.502087in}}%
\pgfpathcurveto{\pgfqpoint{4.086242in}{1.507911in}}{\pgfqpoint{4.089514in}{1.515811in}}{\pgfqpoint{4.089514in}{1.524047in}}%
\pgfpathcurveto{\pgfqpoint{4.089514in}{1.532284in}}{\pgfqpoint{4.086242in}{1.540184in}}{\pgfqpoint{4.080418in}{1.546008in}}%
\pgfpathcurveto{\pgfqpoint{4.074594in}{1.551831in}}{\pgfqpoint{4.066694in}{1.555104in}}{\pgfqpoint{4.058458in}{1.555104in}}%
\pgfpathcurveto{\pgfqpoint{4.050221in}{1.555104in}}{\pgfqpoint{4.042321in}{1.551831in}}{\pgfqpoint{4.036498in}{1.546008in}}%
\pgfpathcurveto{\pgfqpoint{4.030674in}{1.540184in}}{\pgfqpoint{4.027401in}{1.532284in}}{\pgfqpoint{4.027401in}{1.524047in}}%
\pgfpathcurveto{\pgfqpoint{4.027401in}{1.515811in}}{\pgfqpoint{4.030674in}{1.507911in}}{\pgfqpoint{4.036498in}{1.502087in}}%
\pgfpathcurveto{\pgfqpoint{4.042321in}{1.496263in}}{\pgfqpoint{4.050221in}{1.492991in}}{\pgfqpoint{4.058458in}{1.492991in}}%
\pgfpathclose%
\pgfusepath{stroke,fill}%
\end{pgfscope}%
\begin{pgfscope}%
\pgfpathrectangle{\pgfqpoint{3.793912in}{0.557870in}}{\pgfqpoint{2.446088in}{1.684734in}}%
\pgfusepath{clip}%
\pgfsetbuttcap%
\pgfsetroundjoin%
\definecolor{currentfill}{rgb}{0.298039,0.447059,0.690196}%
\pgfsetfillcolor{currentfill}%
\pgfsetlinewidth{1.003750pt}%
\definecolor{currentstroke}{rgb}{0.298039,0.447059,0.690196}%
\pgfsetstrokecolor{currentstroke}%
\pgfsetdash{}{0pt}%
\pgfpathmoveto{\pgfqpoint{3.981778in}{2.125798in}}%
\pgfpathcurveto{\pgfqpoint{3.990014in}{2.125798in}}{\pgfqpoint{3.997914in}{2.129070in}}{\pgfqpoint{4.003738in}{2.134894in}}%
\pgfpathcurveto{\pgfqpoint{4.009562in}{2.140718in}}{\pgfqpoint{4.012834in}{2.148618in}}{\pgfqpoint{4.012834in}{2.156854in}}%
\pgfpathcurveto{\pgfqpoint{4.012834in}{2.165091in}}{\pgfqpoint{4.009562in}{2.172991in}}{\pgfqpoint{4.003738in}{2.178814in}}%
\pgfpathcurveto{\pgfqpoint{3.997914in}{2.184638in}}{\pgfqpoint{3.990014in}{2.187911in}}{\pgfqpoint{3.981778in}{2.187911in}}%
\pgfpathcurveto{\pgfqpoint{3.973542in}{2.187911in}}{\pgfqpoint{3.965642in}{2.184638in}}{\pgfqpoint{3.959818in}{2.178814in}}%
\pgfpathcurveto{\pgfqpoint{3.953994in}{2.172991in}}{\pgfqpoint{3.950721in}{2.165091in}}{\pgfqpoint{3.950721in}{2.156854in}}%
\pgfpathcurveto{\pgfqpoint{3.950721in}{2.148618in}}{\pgfqpoint{3.953994in}{2.140718in}}{\pgfqpoint{3.959818in}{2.134894in}}%
\pgfpathcurveto{\pgfqpoint{3.965642in}{2.129070in}}{\pgfqpoint{3.973542in}{2.125798in}}{\pgfqpoint{3.981778in}{2.125798in}}%
\pgfpathclose%
\pgfusepath{stroke,fill}%
\end{pgfscope}%
\begin{pgfscope}%
\pgfpathrectangle{\pgfqpoint{3.793912in}{0.557870in}}{\pgfqpoint{2.446088in}{1.684734in}}%
\pgfusepath{clip}%
\pgfsetbuttcap%
\pgfsetroundjoin%
\definecolor{currentfill}{rgb}{0.298039,0.447059,0.690196}%
\pgfsetfillcolor{currentfill}%
\pgfsetlinewidth{1.003750pt}%
\definecolor{currentstroke}{rgb}{0.298039,0.447059,0.690196}%
\pgfsetstrokecolor{currentstroke}%
\pgfsetdash{}{0pt}%
\pgfpathmoveto{\pgfqpoint{3.905098in}{2.125798in}}%
\pgfpathcurveto{\pgfqpoint{3.913334in}{2.125798in}}{\pgfqpoint{3.921234in}{2.129070in}}{\pgfqpoint{3.927058in}{2.134894in}}%
\pgfpathcurveto{\pgfqpoint{3.932882in}{2.140718in}}{\pgfqpoint{3.936155in}{2.148618in}}{\pgfqpoint{3.936155in}{2.156854in}}%
\pgfpathcurveto{\pgfqpoint{3.936155in}{2.165091in}}{\pgfqpoint{3.932882in}{2.172991in}}{\pgfqpoint{3.927058in}{2.178814in}}%
\pgfpathcurveto{\pgfqpoint{3.921234in}{2.184638in}}{\pgfqpoint{3.913334in}{2.187911in}}{\pgfqpoint{3.905098in}{2.187911in}}%
\pgfpathcurveto{\pgfqpoint{3.896862in}{2.187911in}}{\pgfqpoint{3.888962in}{2.184638in}}{\pgfqpoint{3.883138in}{2.178814in}}%
\pgfpathcurveto{\pgfqpoint{3.877314in}{2.172991in}}{\pgfqpoint{3.874042in}{2.165091in}}{\pgfqpoint{3.874042in}{2.156854in}}%
\pgfpathcurveto{\pgfqpoint{3.874042in}{2.148618in}}{\pgfqpoint{3.877314in}{2.140718in}}{\pgfqpoint{3.883138in}{2.134894in}}%
\pgfpathcurveto{\pgfqpoint{3.888962in}{2.129070in}}{\pgfqpoint{3.896862in}{2.125798in}}{\pgfqpoint{3.905098in}{2.125798in}}%
\pgfpathclose%
\pgfusepath{stroke,fill}%
\end{pgfscope}%
\begin{pgfscope}%
\pgfpathrectangle{\pgfqpoint{3.793912in}{0.557870in}}{\pgfqpoint{2.446088in}{1.684734in}}%
\pgfusepath{clip}%
\pgfsetbuttcap%
\pgfsetroundjoin%
\definecolor{currentfill}{rgb}{0.298039,0.447059,0.690196}%
\pgfsetfillcolor{currentfill}%
\pgfsetlinewidth{1.003750pt}%
\definecolor{currentstroke}{rgb}{0.298039,0.447059,0.690196}%
\pgfsetstrokecolor{currentstroke}%
\pgfsetdash{}{0pt}%
\pgfpathmoveto{\pgfqpoint{5.975454in}{1.447135in}}%
\pgfpathcurveto{\pgfqpoint{5.983691in}{1.447135in}}{\pgfqpoint{5.991591in}{1.450408in}}{\pgfqpoint{5.997415in}{1.456231in}}%
\pgfpathcurveto{\pgfqpoint{6.003239in}{1.462055in}}{\pgfqpoint{6.006511in}{1.469955in}}{\pgfqpoint{6.006511in}{1.478192in}}%
\pgfpathcurveto{\pgfqpoint{6.006511in}{1.486428in}}{\pgfqpoint{6.003239in}{1.494328in}}{\pgfqpoint{5.997415in}{1.500152in}}%
\pgfpathcurveto{\pgfqpoint{5.991591in}{1.505976in}}{\pgfqpoint{5.983691in}{1.509248in}}{\pgfqpoint{5.975454in}{1.509248in}}%
\pgfpathcurveto{\pgfqpoint{5.967218in}{1.509248in}}{\pgfqpoint{5.959318in}{1.505976in}}{\pgfqpoint{5.953494in}{1.500152in}}%
\pgfpathcurveto{\pgfqpoint{5.947670in}{1.494328in}}{\pgfqpoint{5.944398in}{1.486428in}}{\pgfqpoint{5.944398in}{1.478192in}}%
\pgfpathcurveto{\pgfqpoint{5.944398in}{1.469955in}}{\pgfqpoint{5.947670in}{1.462055in}}{\pgfqpoint{5.953494in}{1.456231in}}%
\pgfpathcurveto{\pgfqpoint{5.959318in}{1.450408in}}{\pgfqpoint{5.967218in}{1.447135in}}{\pgfqpoint{5.975454in}{1.447135in}}%
\pgfpathclose%
\pgfusepath{stroke,fill}%
\end{pgfscope}%
\begin{pgfscope}%
\pgfpathrectangle{\pgfqpoint{3.793912in}{0.557870in}}{\pgfqpoint{2.446088in}{1.684734in}}%
\pgfusepath{clip}%
\pgfsetbuttcap%
\pgfsetroundjoin%
\definecolor{currentfill}{rgb}{0.298039,0.447059,0.690196}%
\pgfsetfillcolor{currentfill}%
\pgfsetlinewidth{1.003750pt}%
\definecolor{currentstroke}{rgb}{0.298039,0.447059,0.690196}%
\pgfsetstrokecolor{currentstroke}%
\pgfsetdash{}{0pt}%
\pgfpathmoveto{\pgfqpoint{3.905098in}{2.125798in}}%
\pgfpathcurveto{\pgfqpoint{3.913334in}{2.125798in}}{\pgfqpoint{3.921234in}{2.129070in}}{\pgfqpoint{3.927058in}{2.134894in}}%
\pgfpathcurveto{\pgfqpoint{3.932882in}{2.140718in}}{\pgfqpoint{3.936155in}{2.148618in}}{\pgfqpoint{3.936155in}{2.156854in}}%
\pgfpathcurveto{\pgfqpoint{3.936155in}{2.165091in}}{\pgfqpoint{3.932882in}{2.172991in}}{\pgfqpoint{3.927058in}{2.178814in}}%
\pgfpathcurveto{\pgfqpoint{3.921234in}{2.184638in}}{\pgfqpoint{3.913334in}{2.187911in}}{\pgfqpoint{3.905098in}{2.187911in}}%
\pgfpathcurveto{\pgfqpoint{3.896862in}{2.187911in}}{\pgfqpoint{3.888962in}{2.184638in}}{\pgfqpoint{3.883138in}{2.178814in}}%
\pgfpathcurveto{\pgfqpoint{3.877314in}{2.172991in}}{\pgfqpoint{3.874042in}{2.165091in}}{\pgfqpoint{3.874042in}{2.156854in}}%
\pgfpathcurveto{\pgfqpoint{3.874042in}{2.148618in}}{\pgfqpoint{3.877314in}{2.140718in}}{\pgfqpoint{3.883138in}{2.134894in}}%
\pgfpathcurveto{\pgfqpoint{3.888962in}{2.129070in}}{\pgfqpoint{3.896862in}{2.125798in}}{\pgfqpoint{3.905098in}{2.125798in}}%
\pgfpathclose%
\pgfusepath{stroke,fill}%
\end{pgfscope}%
\begin{pgfscope}%
\pgfpathrectangle{\pgfqpoint{3.793912in}{0.557870in}}{\pgfqpoint{2.446088in}{1.684734in}}%
\pgfusepath{clip}%
\pgfsetbuttcap%
\pgfsetroundjoin%
\definecolor{currentfill}{rgb}{0.298039,0.447059,0.690196}%
\pgfsetfillcolor{currentfill}%
\pgfsetlinewidth{1.003750pt}%
\definecolor{currentstroke}{rgb}{0.298039,0.447059,0.690196}%
\pgfsetstrokecolor{currentstroke}%
\pgfsetdash{}{0pt}%
\pgfpathmoveto{\pgfqpoint{4.058458in}{1.575531in}}%
\pgfpathcurveto{\pgfqpoint{4.066694in}{1.575531in}}{\pgfqpoint{4.074594in}{1.578803in}}{\pgfqpoint{4.080418in}{1.584627in}}%
\pgfpathcurveto{\pgfqpoint{4.086242in}{1.590451in}}{\pgfqpoint{4.089514in}{1.598351in}}{\pgfqpoint{4.089514in}{1.606587in}}%
\pgfpathcurveto{\pgfqpoint{4.089514in}{1.614824in}}{\pgfqpoint{4.086242in}{1.622724in}}{\pgfqpoint{4.080418in}{1.628548in}}%
\pgfpathcurveto{\pgfqpoint{4.074594in}{1.634372in}}{\pgfqpoint{4.066694in}{1.637644in}}{\pgfqpoint{4.058458in}{1.637644in}}%
\pgfpathcurveto{\pgfqpoint{4.050221in}{1.637644in}}{\pgfqpoint{4.042321in}{1.634372in}}{\pgfqpoint{4.036498in}{1.628548in}}%
\pgfpathcurveto{\pgfqpoint{4.030674in}{1.622724in}}{\pgfqpoint{4.027401in}{1.614824in}}{\pgfqpoint{4.027401in}{1.606587in}}%
\pgfpathcurveto{\pgfqpoint{4.027401in}{1.598351in}}{\pgfqpoint{4.030674in}{1.590451in}}{\pgfqpoint{4.036498in}{1.584627in}}%
\pgfpathcurveto{\pgfqpoint{4.042321in}{1.578803in}}{\pgfqpoint{4.050221in}{1.575531in}}{\pgfqpoint{4.058458in}{1.575531in}}%
\pgfpathclose%
\pgfusepath{stroke,fill}%
\end{pgfscope}%
\begin{pgfscope}%
\pgfpathrectangle{\pgfqpoint{3.793912in}{0.557870in}}{\pgfqpoint{2.446088in}{1.684734in}}%
\pgfusepath{clip}%
\pgfsetbuttcap%
\pgfsetroundjoin%
\definecolor{currentfill}{rgb}{0.298039,0.447059,0.690196}%
\pgfsetfillcolor{currentfill}%
\pgfsetlinewidth{1.003750pt}%
\definecolor{currentstroke}{rgb}{0.298039,0.447059,0.690196}%
\pgfsetstrokecolor{currentstroke}%
\pgfsetdash{}{0pt}%
\pgfpathmoveto{\pgfqpoint{3.905098in}{2.125798in}}%
\pgfpathcurveto{\pgfqpoint{3.913334in}{2.125798in}}{\pgfqpoint{3.921234in}{2.129070in}}{\pgfqpoint{3.927058in}{2.134894in}}%
\pgfpathcurveto{\pgfqpoint{3.932882in}{2.140718in}}{\pgfqpoint{3.936155in}{2.148618in}}{\pgfqpoint{3.936155in}{2.156854in}}%
\pgfpathcurveto{\pgfqpoint{3.936155in}{2.165091in}}{\pgfqpoint{3.932882in}{2.172991in}}{\pgfqpoint{3.927058in}{2.178814in}}%
\pgfpathcurveto{\pgfqpoint{3.921234in}{2.184638in}}{\pgfqpoint{3.913334in}{2.187911in}}{\pgfqpoint{3.905098in}{2.187911in}}%
\pgfpathcurveto{\pgfqpoint{3.896862in}{2.187911in}}{\pgfqpoint{3.888962in}{2.184638in}}{\pgfqpoint{3.883138in}{2.178814in}}%
\pgfpathcurveto{\pgfqpoint{3.877314in}{2.172991in}}{\pgfqpoint{3.874042in}{2.165091in}}{\pgfqpoint{3.874042in}{2.156854in}}%
\pgfpathcurveto{\pgfqpoint{3.874042in}{2.148618in}}{\pgfqpoint{3.877314in}{2.140718in}}{\pgfqpoint{3.883138in}{2.134894in}}%
\pgfpathcurveto{\pgfqpoint{3.888962in}{2.129070in}}{\pgfqpoint{3.896862in}{2.125798in}}{\pgfqpoint{3.905098in}{2.125798in}}%
\pgfpathclose%
\pgfusepath{stroke,fill}%
\end{pgfscope}%
\begin{pgfscope}%
\pgfpathrectangle{\pgfqpoint{3.793912in}{0.557870in}}{\pgfqpoint{2.446088in}{1.684734in}}%
\pgfusepath{clip}%
\pgfsetbuttcap%
\pgfsetroundjoin%
\definecolor{currentfill}{rgb}{0.298039,0.447059,0.690196}%
\pgfsetfillcolor{currentfill}%
\pgfsetlinewidth{1.003750pt}%
\definecolor{currentstroke}{rgb}{0.298039,0.447059,0.690196}%
\pgfsetstrokecolor{currentstroke}%
\pgfsetdash{}{0pt}%
\pgfpathmoveto{\pgfqpoint{3.905098in}{2.125798in}}%
\pgfpathcurveto{\pgfqpoint{3.913334in}{2.125798in}}{\pgfqpoint{3.921234in}{2.129070in}}{\pgfqpoint{3.927058in}{2.134894in}}%
\pgfpathcurveto{\pgfqpoint{3.932882in}{2.140718in}}{\pgfqpoint{3.936155in}{2.148618in}}{\pgfqpoint{3.936155in}{2.156854in}}%
\pgfpathcurveto{\pgfqpoint{3.936155in}{2.165091in}}{\pgfqpoint{3.932882in}{2.172991in}}{\pgfqpoint{3.927058in}{2.178814in}}%
\pgfpathcurveto{\pgfqpoint{3.921234in}{2.184638in}}{\pgfqpoint{3.913334in}{2.187911in}}{\pgfqpoint{3.905098in}{2.187911in}}%
\pgfpathcurveto{\pgfqpoint{3.896862in}{2.187911in}}{\pgfqpoint{3.888962in}{2.184638in}}{\pgfqpoint{3.883138in}{2.178814in}}%
\pgfpathcurveto{\pgfqpoint{3.877314in}{2.172991in}}{\pgfqpoint{3.874042in}{2.165091in}}{\pgfqpoint{3.874042in}{2.156854in}}%
\pgfpathcurveto{\pgfqpoint{3.874042in}{2.148618in}}{\pgfqpoint{3.877314in}{2.140718in}}{\pgfqpoint{3.883138in}{2.134894in}}%
\pgfpathcurveto{\pgfqpoint{3.888962in}{2.129070in}}{\pgfqpoint{3.896862in}{2.125798in}}{\pgfqpoint{3.905098in}{2.125798in}}%
\pgfpathclose%
\pgfusepath{stroke,fill}%
\end{pgfscope}%
\begin{pgfscope}%
\pgfpathrectangle{\pgfqpoint{3.793912in}{0.557870in}}{\pgfqpoint{2.446088in}{1.684734in}}%
\pgfusepath{clip}%
\pgfsetbuttcap%
\pgfsetroundjoin%
\definecolor{currentfill}{rgb}{0.298039,0.447059,0.690196}%
\pgfsetfillcolor{currentfill}%
\pgfsetlinewidth{1.003750pt}%
\definecolor{currentstroke}{rgb}{0.298039,0.447059,0.690196}%
\pgfsetstrokecolor{currentstroke}%
\pgfsetdash{}{0pt}%
\pgfpathmoveto{\pgfqpoint{3.905098in}{2.125798in}}%
\pgfpathcurveto{\pgfqpoint{3.913334in}{2.125798in}}{\pgfqpoint{3.921234in}{2.129070in}}{\pgfqpoint{3.927058in}{2.134894in}}%
\pgfpathcurveto{\pgfqpoint{3.932882in}{2.140718in}}{\pgfqpoint{3.936155in}{2.148618in}}{\pgfqpoint{3.936155in}{2.156854in}}%
\pgfpathcurveto{\pgfqpoint{3.936155in}{2.165091in}}{\pgfqpoint{3.932882in}{2.172991in}}{\pgfqpoint{3.927058in}{2.178814in}}%
\pgfpathcurveto{\pgfqpoint{3.921234in}{2.184638in}}{\pgfqpoint{3.913334in}{2.187911in}}{\pgfqpoint{3.905098in}{2.187911in}}%
\pgfpathcurveto{\pgfqpoint{3.896862in}{2.187911in}}{\pgfqpoint{3.888962in}{2.184638in}}{\pgfqpoint{3.883138in}{2.178814in}}%
\pgfpathcurveto{\pgfqpoint{3.877314in}{2.172991in}}{\pgfqpoint{3.874042in}{2.165091in}}{\pgfqpoint{3.874042in}{2.156854in}}%
\pgfpathcurveto{\pgfqpoint{3.874042in}{2.148618in}}{\pgfqpoint{3.877314in}{2.140718in}}{\pgfqpoint{3.883138in}{2.134894in}}%
\pgfpathcurveto{\pgfqpoint{3.888962in}{2.129070in}}{\pgfqpoint{3.896862in}{2.125798in}}{\pgfqpoint{3.905098in}{2.125798in}}%
\pgfpathclose%
\pgfusepath{stroke,fill}%
\end{pgfscope}%
\begin{pgfscope}%
\pgfpathrectangle{\pgfqpoint{3.793912in}{0.557870in}}{\pgfqpoint{2.446088in}{1.684734in}}%
\pgfusepath{clip}%
\pgfsetbuttcap%
\pgfsetroundjoin%
\definecolor{currentfill}{rgb}{0.298039,0.447059,0.690196}%
\pgfsetfillcolor{currentfill}%
\pgfsetlinewidth{1.003750pt}%
\definecolor{currentstroke}{rgb}{0.298039,0.447059,0.690196}%
\pgfsetstrokecolor{currentstroke}%
\pgfsetdash{}{0pt}%
\pgfpathmoveto{\pgfqpoint{4.978616in}{1.804809in}}%
\pgfpathcurveto{\pgfqpoint{4.986852in}{1.804809in}}{\pgfqpoint{4.994753in}{1.808081in}}{\pgfqpoint{5.000576in}{1.813905in}}%
\pgfpathcurveto{\pgfqpoint{5.006400in}{1.819729in}}{\pgfqpoint{5.009673in}{1.827629in}}{\pgfqpoint{5.009673in}{1.835865in}}%
\pgfpathcurveto{\pgfqpoint{5.009673in}{1.844101in}}{\pgfqpoint{5.006400in}{1.852002in}}{\pgfqpoint{5.000576in}{1.857825in}}%
\pgfpathcurveto{\pgfqpoint{4.994753in}{1.863649in}}{\pgfqpoint{4.986852in}{1.866922in}}{\pgfqpoint{4.978616in}{1.866922in}}%
\pgfpathcurveto{\pgfqpoint{4.970380in}{1.866922in}}{\pgfqpoint{4.962480in}{1.863649in}}{\pgfqpoint{4.956656in}{1.857825in}}%
\pgfpathcurveto{\pgfqpoint{4.950832in}{1.852002in}}{\pgfqpoint{4.947560in}{1.844101in}}{\pgfqpoint{4.947560in}{1.835865in}}%
\pgfpathcurveto{\pgfqpoint{4.947560in}{1.827629in}}{\pgfqpoint{4.950832in}{1.819729in}}{\pgfqpoint{4.956656in}{1.813905in}}%
\pgfpathcurveto{\pgfqpoint{4.962480in}{1.808081in}}{\pgfqpoint{4.970380in}{1.804809in}}{\pgfqpoint{4.978616in}{1.804809in}}%
\pgfpathclose%
\pgfusepath{stroke,fill}%
\end{pgfscope}%
\begin{pgfscope}%
\pgfpathrectangle{\pgfqpoint{3.793912in}{0.557870in}}{\pgfqpoint{2.446088in}{1.684734in}}%
\pgfusepath{clip}%
\pgfsetbuttcap%
\pgfsetroundjoin%
\definecolor{currentfill}{rgb}{0.298039,0.447059,0.690196}%
\pgfsetfillcolor{currentfill}%
\pgfsetlinewidth{1.003750pt}%
\definecolor{currentstroke}{rgb}{0.298039,0.447059,0.690196}%
\pgfsetstrokecolor{currentstroke}%
\pgfsetdash{}{0pt}%
\pgfpathmoveto{\pgfqpoint{3.905098in}{2.125798in}}%
\pgfpathcurveto{\pgfqpoint{3.913334in}{2.125798in}}{\pgfqpoint{3.921234in}{2.129070in}}{\pgfqpoint{3.927058in}{2.134894in}}%
\pgfpathcurveto{\pgfqpoint{3.932882in}{2.140718in}}{\pgfqpoint{3.936155in}{2.148618in}}{\pgfqpoint{3.936155in}{2.156854in}}%
\pgfpathcurveto{\pgfqpoint{3.936155in}{2.165091in}}{\pgfqpoint{3.932882in}{2.172991in}}{\pgfqpoint{3.927058in}{2.178814in}}%
\pgfpathcurveto{\pgfqpoint{3.921234in}{2.184638in}}{\pgfqpoint{3.913334in}{2.187911in}}{\pgfqpoint{3.905098in}{2.187911in}}%
\pgfpathcurveto{\pgfqpoint{3.896862in}{2.187911in}}{\pgfqpoint{3.888962in}{2.184638in}}{\pgfqpoint{3.883138in}{2.178814in}}%
\pgfpathcurveto{\pgfqpoint{3.877314in}{2.172991in}}{\pgfqpoint{3.874042in}{2.165091in}}{\pgfqpoint{3.874042in}{2.156854in}}%
\pgfpathcurveto{\pgfqpoint{3.874042in}{2.148618in}}{\pgfqpoint{3.877314in}{2.140718in}}{\pgfqpoint{3.883138in}{2.134894in}}%
\pgfpathcurveto{\pgfqpoint{3.888962in}{2.129070in}}{\pgfqpoint{3.896862in}{2.125798in}}{\pgfqpoint{3.905098in}{2.125798in}}%
\pgfpathclose%
\pgfusepath{stroke,fill}%
\end{pgfscope}%
\begin{pgfscope}%
\pgfpathrectangle{\pgfqpoint{3.793912in}{0.557870in}}{\pgfqpoint{2.446088in}{1.684734in}}%
\pgfusepath{clip}%
\pgfsetbuttcap%
\pgfsetroundjoin%
\definecolor{currentfill}{rgb}{0.298039,0.447059,0.690196}%
\pgfsetfillcolor{currentfill}%
\pgfsetlinewidth{1.003750pt}%
\definecolor{currentstroke}{rgb}{0.298039,0.447059,0.690196}%
\pgfsetstrokecolor{currentstroke}%
\pgfsetdash{}{0pt}%
\pgfpathmoveto{\pgfqpoint{5.131976in}{1.758953in}}%
\pgfpathcurveto{\pgfqpoint{5.140212in}{1.758953in}}{\pgfqpoint{5.148112in}{1.762225in}}{\pgfqpoint{5.153936in}{1.768049in}}%
\pgfpathcurveto{\pgfqpoint{5.159760in}{1.773873in}}{\pgfqpoint{5.163032in}{1.781773in}}{\pgfqpoint{5.163032in}{1.790010in}}%
\pgfpathcurveto{\pgfqpoint{5.163032in}{1.798246in}}{\pgfqpoint{5.159760in}{1.806146in}}{\pgfqpoint{5.153936in}{1.811970in}}%
\pgfpathcurveto{\pgfqpoint{5.148112in}{1.817794in}}{\pgfqpoint{5.140212in}{1.821066in}}{\pgfqpoint{5.131976in}{1.821066in}}%
\pgfpathcurveto{\pgfqpoint{5.123740in}{1.821066in}}{\pgfqpoint{5.115840in}{1.817794in}}{\pgfqpoint{5.110016in}{1.811970in}}%
\pgfpathcurveto{\pgfqpoint{5.104192in}{1.806146in}}{\pgfqpoint{5.100919in}{1.798246in}}{\pgfqpoint{5.100919in}{1.790010in}}%
\pgfpathcurveto{\pgfqpoint{5.100919in}{1.781773in}}{\pgfqpoint{5.104192in}{1.773873in}}{\pgfqpoint{5.110016in}{1.768049in}}%
\pgfpathcurveto{\pgfqpoint{5.115840in}{1.762225in}}{\pgfqpoint{5.123740in}{1.758953in}}{\pgfqpoint{5.131976in}{1.758953in}}%
\pgfpathclose%
\pgfusepath{stroke,fill}%
\end{pgfscope}%
\begin{pgfscope}%
\pgfpathrectangle{\pgfqpoint{3.793912in}{0.557870in}}{\pgfqpoint{2.446088in}{1.684734in}}%
\pgfusepath{clip}%
\pgfsetbuttcap%
\pgfsetroundjoin%
\definecolor{currentfill}{rgb}{0.298039,0.447059,0.690196}%
\pgfsetfillcolor{currentfill}%
\pgfsetlinewidth{1.003750pt}%
\definecolor{currentstroke}{rgb}{0.298039,0.447059,0.690196}%
\pgfsetstrokecolor{currentstroke}%
\pgfsetdash{}{0pt}%
\pgfpathmoveto{\pgfqpoint{3.905098in}{2.125798in}}%
\pgfpathcurveto{\pgfqpoint{3.913334in}{2.125798in}}{\pgfqpoint{3.921234in}{2.129070in}}{\pgfqpoint{3.927058in}{2.134894in}}%
\pgfpathcurveto{\pgfqpoint{3.932882in}{2.140718in}}{\pgfqpoint{3.936155in}{2.148618in}}{\pgfqpoint{3.936155in}{2.156854in}}%
\pgfpathcurveto{\pgfqpoint{3.936155in}{2.165091in}}{\pgfqpoint{3.932882in}{2.172991in}}{\pgfqpoint{3.927058in}{2.178814in}}%
\pgfpathcurveto{\pgfqpoint{3.921234in}{2.184638in}}{\pgfqpoint{3.913334in}{2.187911in}}{\pgfqpoint{3.905098in}{2.187911in}}%
\pgfpathcurveto{\pgfqpoint{3.896862in}{2.187911in}}{\pgfqpoint{3.888962in}{2.184638in}}{\pgfqpoint{3.883138in}{2.178814in}}%
\pgfpathcurveto{\pgfqpoint{3.877314in}{2.172991in}}{\pgfqpoint{3.874042in}{2.165091in}}{\pgfqpoint{3.874042in}{2.156854in}}%
\pgfpathcurveto{\pgfqpoint{3.874042in}{2.148618in}}{\pgfqpoint{3.877314in}{2.140718in}}{\pgfqpoint{3.883138in}{2.134894in}}%
\pgfpathcurveto{\pgfqpoint{3.888962in}{2.129070in}}{\pgfqpoint{3.896862in}{2.125798in}}{\pgfqpoint{3.905098in}{2.125798in}}%
\pgfpathclose%
\pgfusepath{stroke,fill}%
\end{pgfscope}%
\begin{pgfscope}%
\pgfpathrectangle{\pgfqpoint{3.793912in}{0.557870in}}{\pgfqpoint{2.446088in}{1.684734in}}%
\pgfusepath{clip}%
\pgfsetbuttcap%
\pgfsetroundjoin%
\definecolor{currentfill}{rgb}{0.298039,0.447059,0.690196}%
\pgfsetfillcolor{currentfill}%
\pgfsetlinewidth{1.003750pt}%
\definecolor{currentstroke}{rgb}{0.298039,0.447059,0.690196}%
\pgfsetstrokecolor{currentstroke}%
\pgfsetdash{}{0pt}%
\pgfpathmoveto{\pgfqpoint{3.905098in}{1.777295in}}%
\pgfpathcurveto{\pgfqpoint{3.913334in}{1.777295in}}{\pgfqpoint{3.921234in}{1.780568in}}{\pgfqpoint{3.927058in}{1.786392in}}%
\pgfpathcurveto{\pgfqpoint{3.932882in}{1.792216in}}{\pgfqpoint{3.936155in}{1.800116in}}{\pgfqpoint{3.936155in}{1.808352in}}%
\pgfpathcurveto{\pgfqpoint{3.936155in}{1.816588in}}{\pgfqpoint{3.932882in}{1.824488in}}{\pgfqpoint{3.927058in}{1.830312in}}%
\pgfpathcurveto{\pgfqpoint{3.921234in}{1.836136in}}{\pgfqpoint{3.913334in}{1.839408in}}{\pgfqpoint{3.905098in}{1.839408in}}%
\pgfpathcurveto{\pgfqpoint{3.896862in}{1.839408in}}{\pgfqpoint{3.888962in}{1.836136in}}{\pgfqpoint{3.883138in}{1.830312in}}%
\pgfpathcurveto{\pgfqpoint{3.877314in}{1.824488in}}{\pgfqpoint{3.874042in}{1.816588in}}{\pgfqpoint{3.874042in}{1.808352in}}%
\pgfpathcurveto{\pgfqpoint{3.874042in}{1.800116in}}{\pgfqpoint{3.877314in}{1.792216in}}{\pgfqpoint{3.883138in}{1.786392in}}%
\pgfpathcurveto{\pgfqpoint{3.888962in}{1.780568in}}{\pgfqpoint{3.896862in}{1.777295in}}{\pgfqpoint{3.905098in}{1.777295in}}%
\pgfpathclose%
\pgfusepath{stroke,fill}%
\end{pgfscope}%
\begin{pgfscope}%
\pgfpathrectangle{\pgfqpoint{3.793912in}{0.557870in}}{\pgfqpoint{2.446088in}{1.684734in}}%
\pgfusepath{clip}%
\pgfsetbuttcap%
\pgfsetroundjoin%
\definecolor{currentfill}{rgb}{0.298039,0.447059,0.690196}%
\pgfsetfillcolor{currentfill}%
\pgfsetlinewidth{1.003750pt}%
\definecolor{currentstroke}{rgb}{0.298039,0.447059,0.690196}%
\pgfsetstrokecolor{currentstroke}%
\pgfsetdash{}{0pt}%
\pgfpathmoveto{\pgfqpoint{3.905098in}{2.125798in}}%
\pgfpathcurveto{\pgfqpoint{3.913334in}{2.125798in}}{\pgfqpoint{3.921234in}{2.129070in}}{\pgfqpoint{3.927058in}{2.134894in}}%
\pgfpathcurveto{\pgfqpoint{3.932882in}{2.140718in}}{\pgfqpoint{3.936155in}{2.148618in}}{\pgfqpoint{3.936155in}{2.156854in}}%
\pgfpathcurveto{\pgfqpoint{3.936155in}{2.165091in}}{\pgfqpoint{3.932882in}{2.172991in}}{\pgfqpoint{3.927058in}{2.178814in}}%
\pgfpathcurveto{\pgfqpoint{3.921234in}{2.184638in}}{\pgfqpoint{3.913334in}{2.187911in}}{\pgfqpoint{3.905098in}{2.187911in}}%
\pgfpathcurveto{\pgfqpoint{3.896862in}{2.187911in}}{\pgfqpoint{3.888962in}{2.184638in}}{\pgfqpoint{3.883138in}{2.178814in}}%
\pgfpathcurveto{\pgfqpoint{3.877314in}{2.172991in}}{\pgfqpoint{3.874042in}{2.165091in}}{\pgfqpoint{3.874042in}{2.156854in}}%
\pgfpathcurveto{\pgfqpoint{3.874042in}{2.148618in}}{\pgfqpoint{3.877314in}{2.140718in}}{\pgfqpoint{3.883138in}{2.134894in}}%
\pgfpathcurveto{\pgfqpoint{3.888962in}{2.129070in}}{\pgfqpoint{3.896862in}{2.125798in}}{\pgfqpoint{3.905098in}{2.125798in}}%
\pgfpathclose%
\pgfusepath{stroke,fill}%
\end{pgfscope}%
\begin{pgfscope}%
\pgfpathrectangle{\pgfqpoint{3.793912in}{0.557870in}}{\pgfqpoint{2.446088in}{1.684734in}}%
\pgfusepath{clip}%
\pgfsetbuttcap%
\pgfsetroundjoin%
\definecolor{currentfill}{rgb}{0.298039,0.447059,0.690196}%
\pgfsetfillcolor{currentfill}%
\pgfsetlinewidth{1.003750pt}%
\definecolor{currentstroke}{rgb}{0.298039,0.447059,0.690196}%
\pgfsetstrokecolor{currentstroke}%
\pgfsetdash{}{0pt}%
\pgfpathmoveto{\pgfqpoint{3.905098in}{2.107456in}}%
\pgfpathcurveto{\pgfqpoint{3.913334in}{2.107456in}}{\pgfqpoint{3.921234in}{2.110728in}}{\pgfqpoint{3.927058in}{2.116552in}}%
\pgfpathcurveto{\pgfqpoint{3.932882in}{2.122376in}}{\pgfqpoint{3.936155in}{2.130276in}}{\pgfqpoint{3.936155in}{2.138512in}}%
\pgfpathcurveto{\pgfqpoint{3.936155in}{2.146748in}}{\pgfqpoint{3.932882in}{2.154648in}}{\pgfqpoint{3.927058in}{2.160472in}}%
\pgfpathcurveto{\pgfqpoint{3.921234in}{2.166296in}}{\pgfqpoint{3.913334in}{2.169569in}}{\pgfqpoint{3.905098in}{2.169569in}}%
\pgfpathcurveto{\pgfqpoint{3.896862in}{2.169569in}}{\pgfqpoint{3.888962in}{2.166296in}}{\pgfqpoint{3.883138in}{2.160472in}}%
\pgfpathcurveto{\pgfqpoint{3.877314in}{2.154648in}}{\pgfqpoint{3.874042in}{2.146748in}}{\pgfqpoint{3.874042in}{2.138512in}}%
\pgfpathcurveto{\pgfqpoint{3.874042in}{2.130276in}}{\pgfqpoint{3.877314in}{2.122376in}}{\pgfqpoint{3.883138in}{2.116552in}}%
\pgfpathcurveto{\pgfqpoint{3.888962in}{2.110728in}}{\pgfqpoint{3.896862in}{2.107456in}}{\pgfqpoint{3.905098in}{2.107456in}}%
\pgfpathclose%
\pgfusepath{stroke,fill}%
\end{pgfscope}%
\begin{pgfscope}%
\pgfpathrectangle{\pgfqpoint{3.793912in}{0.557870in}}{\pgfqpoint{2.446088in}{1.684734in}}%
\pgfusepath{clip}%
\pgfsetbuttcap%
\pgfsetroundjoin%
\definecolor{currentfill}{rgb}{0.298039,0.447059,0.690196}%
\pgfsetfillcolor{currentfill}%
\pgfsetlinewidth{1.003750pt}%
\definecolor{currentstroke}{rgb}{0.298039,0.447059,0.690196}%
\pgfsetstrokecolor{currentstroke}%
\pgfsetdash{}{0pt}%
\pgfpathmoveto{\pgfqpoint{3.905098in}{2.125798in}}%
\pgfpathcurveto{\pgfqpoint{3.913334in}{2.125798in}}{\pgfqpoint{3.921234in}{2.129070in}}{\pgfqpoint{3.927058in}{2.134894in}}%
\pgfpathcurveto{\pgfqpoint{3.932882in}{2.140718in}}{\pgfqpoint{3.936155in}{2.148618in}}{\pgfqpoint{3.936155in}{2.156854in}}%
\pgfpathcurveto{\pgfqpoint{3.936155in}{2.165091in}}{\pgfqpoint{3.932882in}{2.172991in}}{\pgfqpoint{3.927058in}{2.178814in}}%
\pgfpathcurveto{\pgfqpoint{3.921234in}{2.184638in}}{\pgfqpoint{3.913334in}{2.187911in}}{\pgfqpoint{3.905098in}{2.187911in}}%
\pgfpathcurveto{\pgfqpoint{3.896862in}{2.187911in}}{\pgfqpoint{3.888962in}{2.184638in}}{\pgfqpoint{3.883138in}{2.178814in}}%
\pgfpathcurveto{\pgfqpoint{3.877314in}{2.172991in}}{\pgfqpoint{3.874042in}{2.165091in}}{\pgfqpoint{3.874042in}{2.156854in}}%
\pgfpathcurveto{\pgfqpoint{3.874042in}{2.148618in}}{\pgfqpoint{3.877314in}{2.140718in}}{\pgfqpoint{3.883138in}{2.134894in}}%
\pgfpathcurveto{\pgfqpoint{3.888962in}{2.129070in}}{\pgfqpoint{3.896862in}{2.125798in}}{\pgfqpoint{3.905098in}{2.125798in}}%
\pgfpathclose%
\pgfusepath{stroke,fill}%
\end{pgfscope}%
\begin{pgfscope}%
\pgfpathrectangle{\pgfqpoint{3.793912in}{0.557870in}}{\pgfqpoint{2.446088in}{1.684734in}}%
\pgfusepath{clip}%
\pgfsetbuttcap%
\pgfsetroundjoin%
\definecolor{currentfill}{rgb}{0.298039,0.447059,0.690196}%
\pgfsetfillcolor{currentfill}%
\pgfsetlinewidth{1.003750pt}%
\definecolor{currentstroke}{rgb}{0.298039,0.447059,0.690196}%
\pgfsetstrokecolor{currentstroke}%
\pgfsetdash{}{0pt}%
\pgfpathmoveto{\pgfqpoint{5.131976in}{1.447135in}}%
\pgfpathcurveto{\pgfqpoint{5.140212in}{1.447135in}}{\pgfqpoint{5.148112in}{1.450408in}}{\pgfqpoint{5.153936in}{1.456231in}}%
\pgfpathcurveto{\pgfqpoint{5.159760in}{1.462055in}}{\pgfqpoint{5.163032in}{1.469955in}}{\pgfqpoint{5.163032in}{1.478192in}}%
\pgfpathcurveto{\pgfqpoint{5.163032in}{1.486428in}}{\pgfqpoint{5.159760in}{1.494328in}}{\pgfqpoint{5.153936in}{1.500152in}}%
\pgfpathcurveto{\pgfqpoint{5.148112in}{1.505976in}}{\pgfqpoint{5.140212in}{1.509248in}}{\pgfqpoint{5.131976in}{1.509248in}}%
\pgfpathcurveto{\pgfqpoint{5.123740in}{1.509248in}}{\pgfqpoint{5.115840in}{1.505976in}}{\pgfqpoint{5.110016in}{1.500152in}}%
\pgfpathcurveto{\pgfqpoint{5.104192in}{1.494328in}}{\pgfqpoint{5.100919in}{1.486428in}}{\pgfqpoint{5.100919in}{1.478192in}}%
\pgfpathcurveto{\pgfqpoint{5.100919in}{1.469955in}}{\pgfqpoint{5.104192in}{1.462055in}}{\pgfqpoint{5.110016in}{1.456231in}}%
\pgfpathcurveto{\pgfqpoint{5.115840in}{1.450408in}}{\pgfqpoint{5.123740in}{1.447135in}}{\pgfqpoint{5.131976in}{1.447135in}}%
\pgfpathclose%
\pgfusepath{stroke,fill}%
\end{pgfscope}%
\begin{pgfscope}%
\pgfpathrectangle{\pgfqpoint{3.793912in}{0.557870in}}{\pgfqpoint{2.446088in}{1.684734in}}%
\pgfusepath{clip}%
\pgfsetbuttcap%
\pgfsetroundjoin%
\definecolor{currentfill}{rgb}{0.298039,0.447059,0.690196}%
\pgfsetfillcolor{currentfill}%
\pgfsetlinewidth{1.003750pt}%
\definecolor{currentstroke}{rgb}{0.298039,0.447059,0.690196}%
\pgfsetstrokecolor{currentstroke}%
\pgfsetdash{}{0pt}%
\pgfpathmoveto{\pgfqpoint{3.905098in}{2.125798in}}%
\pgfpathcurveto{\pgfqpoint{3.913334in}{2.125798in}}{\pgfqpoint{3.921234in}{2.129070in}}{\pgfqpoint{3.927058in}{2.134894in}}%
\pgfpathcurveto{\pgfqpoint{3.932882in}{2.140718in}}{\pgfqpoint{3.936155in}{2.148618in}}{\pgfqpoint{3.936155in}{2.156854in}}%
\pgfpathcurveto{\pgfqpoint{3.936155in}{2.165091in}}{\pgfqpoint{3.932882in}{2.172991in}}{\pgfqpoint{3.927058in}{2.178814in}}%
\pgfpathcurveto{\pgfqpoint{3.921234in}{2.184638in}}{\pgfqpoint{3.913334in}{2.187911in}}{\pgfqpoint{3.905098in}{2.187911in}}%
\pgfpathcurveto{\pgfqpoint{3.896862in}{2.187911in}}{\pgfqpoint{3.888962in}{2.184638in}}{\pgfqpoint{3.883138in}{2.178814in}}%
\pgfpathcurveto{\pgfqpoint{3.877314in}{2.172991in}}{\pgfqpoint{3.874042in}{2.165091in}}{\pgfqpoint{3.874042in}{2.156854in}}%
\pgfpathcurveto{\pgfqpoint{3.874042in}{2.148618in}}{\pgfqpoint{3.877314in}{2.140718in}}{\pgfqpoint{3.883138in}{2.134894in}}%
\pgfpathcurveto{\pgfqpoint{3.888962in}{2.129070in}}{\pgfqpoint{3.896862in}{2.125798in}}{\pgfqpoint{3.905098in}{2.125798in}}%
\pgfpathclose%
\pgfusepath{stroke,fill}%
\end{pgfscope}%
\begin{pgfscope}%
\pgfpathrectangle{\pgfqpoint{3.793912in}{0.557870in}}{\pgfqpoint{2.446088in}{1.684734in}}%
\pgfusepath{clip}%
\pgfsetbuttcap%
\pgfsetroundjoin%
\definecolor{currentfill}{rgb}{0.298039,0.447059,0.690196}%
\pgfsetfillcolor{currentfill}%
\pgfsetlinewidth{1.003750pt}%
\definecolor{currentstroke}{rgb}{0.298039,0.447059,0.690196}%
\pgfsetstrokecolor{currentstroke}%
\pgfsetdash{}{0pt}%
\pgfpathmoveto{\pgfqpoint{4.058458in}{2.125798in}}%
\pgfpathcurveto{\pgfqpoint{4.066694in}{2.125798in}}{\pgfqpoint{4.074594in}{2.129070in}}{\pgfqpoint{4.080418in}{2.134894in}}%
\pgfpathcurveto{\pgfqpoint{4.086242in}{2.140718in}}{\pgfqpoint{4.089514in}{2.148618in}}{\pgfqpoint{4.089514in}{2.156854in}}%
\pgfpathcurveto{\pgfqpoint{4.089514in}{2.165091in}}{\pgfqpoint{4.086242in}{2.172991in}}{\pgfqpoint{4.080418in}{2.178814in}}%
\pgfpathcurveto{\pgfqpoint{4.074594in}{2.184638in}}{\pgfqpoint{4.066694in}{2.187911in}}{\pgfqpoint{4.058458in}{2.187911in}}%
\pgfpathcurveto{\pgfqpoint{4.050221in}{2.187911in}}{\pgfqpoint{4.042321in}{2.184638in}}{\pgfqpoint{4.036498in}{2.178814in}}%
\pgfpathcurveto{\pgfqpoint{4.030674in}{2.172991in}}{\pgfqpoint{4.027401in}{2.165091in}}{\pgfqpoint{4.027401in}{2.156854in}}%
\pgfpathcurveto{\pgfqpoint{4.027401in}{2.148618in}}{\pgfqpoint{4.030674in}{2.140718in}}{\pgfqpoint{4.036498in}{2.134894in}}%
\pgfpathcurveto{\pgfqpoint{4.042321in}{2.129070in}}{\pgfqpoint{4.050221in}{2.125798in}}{\pgfqpoint{4.058458in}{2.125798in}}%
\pgfpathclose%
\pgfusepath{stroke,fill}%
\end{pgfscope}%
\begin{pgfscope}%
\pgfpathrectangle{\pgfqpoint{3.793912in}{0.557870in}}{\pgfqpoint{2.446088in}{1.684734in}}%
\pgfusepath{clip}%
\pgfsetbuttcap%
\pgfsetroundjoin%
\definecolor{currentfill}{rgb}{0.298039,0.447059,0.690196}%
\pgfsetfillcolor{currentfill}%
\pgfsetlinewidth{1.003750pt}%
\definecolor{currentstroke}{rgb}{0.298039,0.447059,0.690196}%
\pgfsetstrokecolor{currentstroke}%
\pgfsetdash{}{0pt}%
\pgfpathmoveto{\pgfqpoint{3.905098in}{2.125798in}}%
\pgfpathcurveto{\pgfqpoint{3.913334in}{2.125798in}}{\pgfqpoint{3.921234in}{2.129070in}}{\pgfqpoint{3.927058in}{2.134894in}}%
\pgfpathcurveto{\pgfqpoint{3.932882in}{2.140718in}}{\pgfqpoint{3.936155in}{2.148618in}}{\pgfqpoint{3.936155in}{2.156854in}}%
\pgfpathcurveto{\pgfqpoint{3.936155in}{2.165091in}}{\pgfqpoint{3.932882in}{2.172991in}}{\pgfqpoint{3.927058in}{2.178814in}}%
\pgfpathcurveto{\pgfqpoint{3.921234in}{2.184638in}}{\pgfqpoint{3.913334in}{2.187911in}}{\pgfqpoint{3.905098in}{2.187911in}}%
\pgfpathcurveto{\pgfqpoint{3.896862in}{2.187911in}}{\pgfqpoint{3.888962in}{2.184638in}}{\pgfqpoint{3.883138in}{2.178814in}}%
\pgfpathcurveto{\pgfqpoint{3.877314in}{2.172991in}}{\pgfqpoint{3.874042in}{2.165091in}}{\pgfqpoint{3.874042in}{2.156854in}}%
\pgfpathcurveto{\pgfqpoint{3.874042in}{2.148618in}}{\pgfqpoint{3.877314in}{2.140718in}}{\pgfqpoint{3.883138in}{2.134894in}}%
\pgfpathcurveto{\pgfqpoint{3.888962in}{2.129070in}}{\pgfqpoint{3.896862in}{2.125798in}}{\pgfqpoint{3.905098in}{2.125798in}}%
\pgfpathclose%
\pgfusepath{stroke,fill}%
\end{pgfscope}%
\begin{pgfscope}%
\pgfpathrectangle{\pgfqpoint{3.793912in}{0.557870in}}{\pgfqpoint{2.446088in}{1.684734in}}%
\pgfusepath{clip}%
\pgfsetbuttcap%
\pgfsetroundjoin%
\definecolor{currentfill}{rgb}{0.298039,0.447059,0.690196}%
\pgfsetfillcolor{currentfill}%
\pgfsetlinewidth{1.003750pt}%
\definecolor{currentstroke}{rgb}{0.298039,0.447059,0.690196}%
\pgfsetstrokecolor{currentstroke}%
\pgfsetdash{}{0pt}%
\pgfpathmoveto{\pgfqpoint{5.975454in}{1.373766in}}%
\pgfpathcurveto{\pgfqpoint{5.983691in}{1.373766in}}{\pgfqpoint{5.991591in}{1.377039in}}{\pgfqpoint{5.997415in}{1.382863in}}%
\pgfpathcurveto{\pgfqpoint{6.003239in}{1.388686in}}{\pgfqpoint{6.006511in}{1.396586in}}{\pgfqpoint{6.006511in}{1.404823in}}%
\pgfpathcurveto{\pgfqpoint{6.006511in}{1.413059in}}{\pgfqpoint{6.003239in}{1.420959in}}{\pgfqpoint{5.997415in}{1.426783in}}%
\pgfpathcurveto{\pgfqpoint{5.991591in}{1.432607in}}{\pgfqpoint{5.983691in}{1.435879in}}{\pgfqpoint{5.975454in}{1.435879in}}%
\pgfpathcurveto{\pgfqpoint{5.967218in}{1.435879in}}{\pgfqpoint{5.959318in}{1.432607in}}{\pgfqpoint{5.953494in}{1.426783in}}%
\pgfpathcurveto{\pgfqpoint{5.947670in}{1.420959in}}{\pgfqpoint{5.944398in}{1.413059in}}{\pgfqpoint{5.944398in}{1.404823in}}%
\pgfpathcurveto{\pgfqpoint{5.944398in}{1.396586in}}{\pgfqpoint{5.947670in}{1.388686in}}{\pgfqpoint{5.953494in}{1.382863in}}%
\pgfpathcurveto{\pgfqpoint{5.959318in}{1.377039in}}{\pgfqpoint{5.967218in}{1.373766in}}{\pgfqpoint{5.975454in}{1.373766in}}%
\pgfpathclose%
\pgfusepath{stroke,fill}%
\end{pgfscope}%
\begin{pgfscope}%
\pgfpathrectangle{\pgfqpoint{3.793912in}{0.557870in}}{\pgfqpoint{2.446088in}{1.684734in}}%
\pgfusepath{clip}%
\pgfsetbuttcap%
\pgfsetroundjoin%
\definecolor{currentfill}{rgb}{0.298039,0.447059,0.690196}%
\pgfsetfillcolor{currentfill}%
\pgfsetlinewidth{1.003750pt}%
\definecolor{currentstroke}{rgb}{0.298039,0.447059,0.690196}%
\pgfsetstrokecolor{currentstroke}%
\pgfsetdash{}{0pt}%
\pgfpathmoveto{\pgfqpoint{3.905098in}{2.125798in}}%
\pgfpathcurveto{\pgfqpoint{3.913334in}{2.125798in}}{\pgfqpoint{3.921234in}{2.129070in}}{\pgfqpoint{3.927058in}{2.134894in}}%
\pgfpathcurveto{\pgfqpoint{3.932882in}{2.140718in}}{\pgfqpoint{3.936155in}{2.148618in}}{\pgfqpoint{3.936155in}{2.156854in}}%
\pgfpathcurveto{\pgfqpoint{3.936155in}{2.165091in}}{\pgfqpoint{3.932882in}{2.172991in}}{\pgfqpoint{3.927058in}{2.178814in}}%
\pgfpathcurveto{\pgfqpoint{3.921234in}{2.184638in}}{\pgfqpoint{3.913334in}{2.187911in}}{\pgfqpoint{3.905098in}{2.187911in}}%
\pgfpathcurveto{\pgfqpoint{3.896862in}{2.187911in}}{\pgfqpoint{3.888962in}{2.184638in}}{\pgfqpoint{3.883138in}{2.178814in}}%
\pgfpathcurveto{\pgfqpoint{3.877314in}{2.172991in}}{\pgfqpoint{3.874042in}{2.165091in}}{\pgfqpoint{3.874042in}{2.156854in}}%
\pgfpathcurveto{\pgfqpoint{3.874042in}{2.148618in}}{\pgfqpoint{3.877314in}{2.140718in}}{\pgfqpoint{3.883138in}{2.134894in}}%
\pgfpathcurveto{\pgfqpoint{3.888962in}{2.129070in}}{\pgfqpoint{3.896862in}{2.125798in}}{\pgfqpoint{3.905098in}{2.125798in}}%
\pgfpathclose%
\pgfusepath{stroke,fill}%
\end{pgfscope}%
\begin{pgfscope}%
\pgfpathrectangle{\pgfqpoint{3.793912in}{0.557870in}}{\pgfqpoint{2.446088in}{1.684734in}}%
\pgfusepath{clip}%
\pgfsetbuttcap%
\pgfsetroundjoin%
\definecolor{currentfill}{rgb}{0.298039,0.447059,0.690196}%
\pgfsetfillcolor{currentfill}%
\pgfsetlinewidth{1.003750pt}%
\definecolor{currentstroke}{rgb}{0.298039,0.447059,0.690196}%
\pgfsetstrokecolor{currentstroke}%
\pgfsetdash{}{0pt}%
\pgfpathmoveto{\pgfqpoint{5.975454in}{1.575531in}}%
\pgfpathcurveto{\pgfqpoint{5.983691in}{1.575531in}}{\pgfqpoint{5.991591in}{1.578803in}}{\pgfqpoint{5.997415in}{1.584627in}}%
\pgfpathcurveto{\pgfqpoint{6.003239in}{1.590451in}}{\pgfqpoint{6.006511in}{1.598351in}}{\pgfqpoint{6.006511in}{1.606587in}}%
\pgfpathcurveto{\pgfqpoint{6.006511in}{1.614824in}}{\pgfqpoint{6.003239in}{1.622724in}}{\pgfqpoint{5.997415in}{1.628548in}}%
\pgfpathcurveto{\pgfqpoint{5.991591in}{1.634372in}}{\pgfqpoint{5.983691in}{1.637644in}}{\pgfqpoint{5.975454in}{1.637644in}}%
\pgfpathcurveto{\pgfqpoint{5.967218in}{1.637644in}}{\pgfqpoint{5.959318in}{1.634372in}}{\pgfqpoint{5.953494in}{1.628548in}}%
\pgfpathcurveto{\pgfqpoint{5.947670in}{1.622724in}}{\pgfqpoint{5.944398in}{1.614824in}}{\pgfqpoint{5.944398in}{1.606587in}}%
\pgfpathcurveto{\pgfqpoint{5.944398in}{1.598351in}}{\pgfqpoint{5.947670in}{1.590451in}}{\pgfqpoint{5.953494in}{1.584627in}}%
\pgfpathcurveto{\pgfqpoint{5.959318in}{1.578803in}}{\pgfqpoint{5.967218in}{1.575531in}}{\pgfqpoint{5.975454in}{1.575531in}}%
\pgfpathclose%
\pgfusepath{stroke,fill}%
\end{pgfscope}%
\begin{pgfscope}%
\pgfpathrectangle{\pgfqpoint{3.793912in}{0.557870in}}{\pgfqpoint{2.446088in}{1.684734in}}%
\pgfusepath{clip}%
\pgfsetbuttcap%
\pgfsetroundjoin%
\definecolor{currentfill}{rgb}{0.298039,0.447059,0.690196}%
\pgfsetfillcolor{currentfill}%
\pgfsetlinewidth{1.003750pt}%
\definecolor{currentstroke}{rgb}{0.298039,0.447059,0.690196}%
\pgfsetstrokecolor{currentstroke}%
\pgfsetdash{}{0pt}%
\pgfpathmoveto{\pgfqpoint{3.905098in}{2.125798in}}%
\pgfpathcurveto{\pgfqpoint{3.913334in}{2.125798in}}{\pgfqpoint{3.921234in}{2.129070in}}{\pgfqpoint{3.927058in}{2.134894in}}%
\pgfpathcurveto{\pgfqpoint{3.932882in}{2.140718in}}{\pgfqpoint{3.936155in}{2.148618in}}{\pgfqpoint{3.936155in}{2.156854in}}%
\pgfpathcurveto{\pgfqpoint{3.936155in}{2.165091in}}{\pgfqpoint{3.932882in}{2.172991in}}{\pgfqpoint{3.927058in}{2.178814in}}%
\pgfpathcurveto{\pgfqpoint{3.921234in}{2.184638in}}{\pgfqpoint{3.913334in}{2.187911in}}{\pgfqpoint{3.905098in}{2.187911in}}%
\pgfpathcurveto{\pgfqpoint{3.896862in}{2.187911in}}{\pgfqpoint{3.888962in}{2.184638in}}{\pgfqpoint{3.883138in}{2.178814in}}%
\pgfpathcurveto{\pgfqpoint{3.877314in}{2.172991in}}{\pgfqpoint{3.874042in}{2.165091in}}{\pgfqpoint{3.874042in}{2.156854in}}%
\pgfpathcurveto{\pgfqpoint{3.874042in}{2.148618in}}{\pgfqpoint{3.877314in}{2.140718in}}{\pgfqpoint{3.883138in}{2.134894in}}%
\pgfpathcurveto{\pgfqpoint{3.888962in}{2.129070in}}{\pgfqpoint{3.896862in}{2.125798in}}{\pgfqpoint{3.905098in}{2.125798in}}%
\pgfpathclose%
\pgfusepath{stroke,fill}%
\end{pgfscope}%
\begin{pgfscope}%
\pgfpathrectangle{\pgfqpoint{3.793912in}{0.557870in}}{\pgfqpoint{2.446088in}{1.684734in}}%
\pgfusepath{clip}%
\pgfsetbuttcap%
\pgfsetroundjoin%
\definecolor{currentfill}{rgb}{0.298039,0.447059,0.690196}%
\pgfsetfillcolor{currentfill}%
\pgfsetlinewidth{1.003750pt}%
\definecolor{currentstroke}{rgb}{0.298039,0.447059,0.690196}%
\pgfsetstrokecolor{currentstroke}%
\pgfsetdash{}{0pt}%
\pgfpathmoveto{\pgfqpoint{3.905098in}{2.125798in}}%
\pgfpathcurveto{\pgfqpoint{3.913334in}{2.125798in}}{\pgfqpoint{3.921234in}{2.129070in}}{\pgfqpoint{3.927058in}{2.134894in}}%
\pgfpathcurveto{\pgfqpoint{3.932882in}{2.140718in}}{\pgfqpoint{3.936155in}{2.148618in}}{\pgfqpoint{3.936155in}{2.156854in}}%
\pgfpathcurveto{\pgfqpoint{3.936155in}{2.165091in}}{\pgfqpoint{3.932882in}{2.172991in}}{\pgfqpoint{3.927058in}{2.178814in}}%
\pgfpathcurveto{\pgfqpoint{3.921234in}{2.184638in}}{\pgfqpoint{3.913334in}{2.187911in}}{\pgfqpoint{3.905098in}{2.187911in}}%
\pgfpathcurveto{\pgfqpoint{3.896862in}{2.187911in}}{\pgfqpoint{3.888962in}{2.184638in}}{\pgfqpoint{3.883138in}{2.178814in}}%
\pgfpathcurveto{\pgfqpoint{3.877314in}{2.172991in}}{\pgfqpoint{3.874042in}{2.165091in}}{\pgfqpoint{3.874042in}{2.156854in}}%
\pgfpathcurveto{\pgfqpoint{3.874042in}{2.148618in}}{\pgfqpoint{3.877314in}{2.140718in}}{\pgfqpoint{3.883138in}{2.134894in}}%
\pgfpathcurveto{\pgfqpoint{3.888962in}{2.129070in}}{\pgfqpoint{3.896862in}{2.125798in}}{\pgfqpoint{3.905098in}{2.125798in}}%
\pgfpathclose%
\pgfusepath{stroke,fill}%
\end{pgfscope}%
\begin{pgfscope}%
\pgfpathrectangle{\pgfqpoint{3.793912in}{0.557870in}}{\pgfqpoint{2.446088in}{1.684734in}}%
\pgfusepath{clip}%
\pgfsetbuttcap%
\pgfsetroundjoin%
\definecolor{currentfill}{rgb}{0.298039,0.447059,0.690196}%
\pgfsetfillcolor{currentfill}%
\pgfsetlinewidth{1.003750pt}%
\definecolor{currentstroke}{rgb}{0.298039,0.447059,0.690196}%
\pgfsetstrokecolor{currentstroke}%
\pgfsetdash{}{0pt}%
\pgfpathmoveto{\pgfqpoint{3.905098in}{2.125798in}}%
\pgfpathcurveto{\pgfqpoint{3.913334in}{2.125798in}}{\pgfqpoint{3.921234in}{2.129070in}}{\pgfqpoint{3.927058in}{2.134894in}}%
\pgfpathcurveto{\pgfqpoint{3.932882in}{2.140718in}}{\pgfqpoint{3.936155in}{2.148618in}}{\pgfqpoint{3.936155in}{2.156854in}}%
\pgfpathcurveto{\pgfqpoint{3.936155in}{2.165091in}}{\pgfqpoint{3.932882in}{2.172991in}}{\pgfqpoint{3.927058in}{2.178814in}}%
\pgfpathcurveto{\pgfqpoint{3.921234in}{2.184638in}}{\pgfqpoint{3.913334in}{2.187911in}}{\pgfqpoint{3.905098in}{2.187911in}}%
\pgfpathcurveto{\pgfqpoint{3.896862in}{2.187911in}}{\pgfqpoint{3.888962in}{2.184638in}}{\pgfqpoint{3.883138in}{2.178814in}}%
\pgfpathcurveto{\pgfqpoint{3.877314in}{2.172991in}}{\pgfqpoint{3.874042in}{2.165091in}}{\pgfqpoint{3.874042in}{2.156854in}}%
\pgfpathcurveto{\pgfqpoint{3.874042in}{2.148618in}}{\pgfqpoint{3.877314in}{2.140718in}}{\pgfqpoint{3.883138in}{2.134894in}}%
\pgfpathcurveto{\pgfqpoint{3.888962in}{2.129070in}}{\pgfqpoint{3.896862in}{2.125798in}}{\pgfqpoint{3.905098in}{2.125798in}}%
\pgfpathclose%
\pgfusepath{stroke,fill}%
\end{pgfscope}%
\begin{pgfscope}%
\pgfpathrectangle{\pgfqpoint{3.793912in}{0.557870in}}{\pgfqpoint{2.446088in}{1.684734in}}%
\pgfusepath{clip}%
\pgfsetbuttcap%
\pgfsetroundjoin%
\definecolor{currentfill}{rgb}{0.298039,0.447059,0.690196}%
\pgfsetfillcolor{currentfill}%
\pgfsetlinewidth{1.003750pt}%
\definecolor{currentstroke}{rgb}{0.298039,0.447059,0.690196}%
\pgfsetstrokecolor{currentstroke}%
\pgfsetdash{}{0pt}%
\pgfpathmoveto{\pgfqpoint{3.905098in}{2.125798in}}%
\pgfpathcurveto{\pgfqpoint{3.913334in}{2.125798in}}{\pgfqpoint{3.921234in}{2.129070in}}{\pgfqpoint{3.927058in}{2.134894in}}%
\pgfpathcurveto{\pgfqpoint{3.932882in}{2.140718in}}{\pgfqpoint{3.936155in}{2.148618in}}{\pgfqpoint{3.936155in}{2.156854in}}%
\pgfpathcurveto{\pgfqpoint{3.936155in}{2.165091in}}{\pgfqpoint{3.932882in}{2.172991in}}{\pgfqpoint{3.927058in}{2.178814in}}%
\pgfpathcurveto{\pgfqpoint{3.921234in}{2.184638in}}{\pgfqpoint{3.913334in}{2.187911in}}{\pgfqpoint{3.905098in}{2.187911in}}%
\pgfpathcurveto{\pgfqpoint{3.896862in}{2.187911in}}{\pgfqpoint{3.888962in}{2.184638in}}{\pgfqpoint{3.883138in}{2.178814in}}%
\pgfpathcurveto{\pgfqpoint{3.877314in}{2.172991in}}{\pgfqpoint{3.874042in}{2.165091in}}{\pgfqpoint{3.874042in}{2.156854in}}%
\pgfpathcurveto{\pgfqpoint{3.874042in}{2.148618in}}{\pgfqpoint{3.877314in}{2.140718in}}{\pgfqpoint{3.883138in}{2.134894in}}%
\pgfpathcurveto{\pgfqpoint{3.888962in}{2.129070in}}{\pgfqpoint{3.896862in}{2.125798in}}{\pgfqpoint{3.905098in}{2.125798in}}%
\pgfpathclose%
\pgfusepath{stroke,fill}%
\end{pgfscope}%
\begin{pgfscope}%
\pgfpathrectangle{\pgfqpoint{3.793912in}{0.557870in}}{\pgfqpoint{2.446088in}{1.684734in}}%
\pgfusepath{clip}%
\pgfsetbuttcap%
\pgfsetroundjoin%
\definecolor{currentfill}{rgb}{0.298039,0.447059,0.690196}%
\pgfsetfillcolor{currentfill}%
\pgfsetlinewidth{1.003750pt}%
\definecolor{currentstroke}{rgb}{0.298039,0.447059,0.690196}%
\pgfsetstrokecolor{currentstroke}%
\pgfsetdash{}{0pt}%
\pgfpathmoveto{\pgfqpoint{5.975454in}{1.272884in}}%
\pgfpathcurveto{\pgfqpoint{5.983691in}{1.272884in}}{\pgfqpoint{5.991591in}{1.276156in}}{\pgfqpoint{5.997415in}{1.281980in}}%
\pgfpathcurveto{\pgfqpoint{6.003239in}{1.287804in}}{\pgfqpoint{6.006511in}{1.295704in}}{\pgfqpoint{6.006511in}{1.303941in}}%
\pgfpathcurveto{\pgfqpoint{6.006511in}{1.312177in}}{\pgfqpoint{6.003239in}{1.320077in}}{\pgfqpoint{5.997415in}{1.325901in}}%
\pgfpathcurveto{\pgfqpoint{5.991591in}{1.331725in}}{\pgfqpoint{5.983691in}{1.334997in}}{\pgfqpoint{5.975454in}{1.334997in}}%
\pgfpathcurveto{\pgfqpoint{5.967218in}{1.334997in}}{\pgfqpoint{5.959318in}{1.331725in}}{\pgfqpoint{5.953494in}{1.325901in}}%
\pgfpathcurveto{\pgfqpoint{5.947670in}{1.320077in}}{\pgfqpoint{5.944398in}{1.312177in}}{\pgfqpoint{5.944398in}{1.303941in}}%
\pgfpathcurveto{\pgfqpoint{5.944398in}{1.295704in}}{\pgfqpoint{5.947670in}{1.287804in}}{\pgfqpoint{5.953494in}{1.281980in}}%
\pgfpathcurveto{\pgfqpoint{5.959318in}{1.276156in}}{\pgfqpoint{5.967218in}{1.272884in}}{\pgfqpoint{5.975454in}{1.272884in}}%
\pgfpathclose%
\pgfusepath{stroke,fill}%
\end{pgfscope}%
\begin{pgfscope}%
\pgfpathrectangle{\pgfqpoint{3.793912in}{0.557870in}}{\pgfqpoint{2.446088in}{1.684734in}}%
\pgfusepath{clip}%
\pgfsetbuttcap%
\pgfsetroundjoin%
\definecolor{currentfill}{rgb}{0.298039,0.447059,0.690196}%
\pgfsetfillcolor{currentfill}%
\pgfsetlinewidth{1.003750pt}%
\definecolor{currentstroke}{rgb}{0.298039,0.447059,0.690196}%
\pgfsetstrokecolor{currentstroke}%
\pgfsetdash{}{0pt}%
\pgfpathmoveto{\pgfqpoint{3.905098in}{2.125798in}}%
\pgfpathcurveto{\pgfqpoint{3.913334in}{2.125798in}}{\pgfqpoint{3.921234in}{2.129070in}}{\pgfqpoint{3.927058in}{2.134894in}}%
\pgfpathcurveto{\pgfqpoint{3.932882in}{2.140718in}}{\pgfqpoint{3.936155in}{2.148618in}}{\pgfqpoint{3.936155in}{2.156854in}}%
\pgfpathcurveto{\pgfqpoint{3.936155in}{2.165091in}}{\pgfqpoint{3.932882in}{2.172991in}}{\pgfqpoint{3.927058in}{2.178814in}}%
\pgfpathcurveto{\pgfqpoint{3.921234in}{2.184638in}}{\pgfqpoint{3.913334in}{2.187911in}}{\pgfqpoint{3.905098in}{2.187911in}}%
\pgfpathcurveto{\pgfqpoint{3.896862in}{2.187911in}}{\pgfqpoint{3.888962in}{2.184638in}}{\pgfqpoint{3.883138in}{2.178814in}}%
\pgfpathcurveto{\pgfqpoint{3.877314in}{2.172991in}}{\pgfqpoint{3.874042in}{2.165091in}}{\pgfqpoint{3.874042in}{2.156854in}}%
\pgfpathcurveto{\pgfqpoint{3.874042in}{2.148618in}}{\pgfqpoint{3.877314in}{2.140718in}}{\pgfqpoint{3.883138in}{2.134894in}}%
\pgfpathcurveto{\pgfqpoint{3.888962in}{2.129070in}}{\pgfqpoint{3.896862in}{2.125798in}}{\pgfqpoint{3.905098in}{2.125798in}}%
\pgfpathclose%
\pgfusepath{stroke,fill}%
\end{pgfscope}%
\begin{pgfscope}%
\pgfpathrectangle{\pgfqpoint{3.793912in}{0.557870in}}{\pgfqpoint{2.446088in}{1.684734in}}%
\pgfusepath{clip}%
\pgfsetbuttcap%
\pgfsetroundjoin%
\definecolor{currentfill}{rgb}{0.298039,0.447059,0.690196}%
\pgfsetfillcolor{currentfill}%
\pgfsetlinewidth{1.003750pt}%
\definecolor{currentstroke}{rgb}{0.298039,0.447059,0.690196}%
\pgfsetstrokecolor{currentstroke}%
\pgfsetdash{}{0pt}%
\pgfpathmoveto{\pgfqpoint{5.975454in}{1.318740in}}%
\pgfpathcurveto{\pgfqpoint{5.983691in}{1.318740in}}{\pgfqpoint{5.991591in}{1.322012in}}{\pgfqpoint{5.997415in}{1.327836in}}%
\pgfpathcurveto{\pgfqpoint{6.003239in}{1.333660in}}{\pgfqpoint{6.006511in}{1.341560in}}{\pgfqpoint{6.006511in}{1.349796in}}%
\pgfpathcurveto{\pgfqpoint{6.006511in}{1.358032in}}{\pgfqpoint{6.003239in}{1.365932in}}{\pgfqpoint{5.997415in}{1.371756in}}%
\pgfpathcurveto{\pgfqpoint{5.991591in}{1.377580in}}{\pgfqpoint{5.983691in}{1.380853in}}{\pgfqpoint{5.975454in}{1.380853in}}%
\pgfpathcurveto{\pgfqpoint{5.967218in}{1.380853in}}{\pgfqpoint{5.959318in}{1.377580in}}{\pgfqpoint{5.953494in}{1.371756in}}%
\pgfpathcurveto{\pgfqpoint{5.947670in}{1.365932in}}{\pgfqpoint{5.944398in}{1.358032in}}{\pgfqpoint{5.944398in}{1.349796in}}%
\pgfpathcurveto{\pgfqpoint{5.944398in}{1.341560in}}{\pgfqpoint{5.947670in}{1.333660in}}{\pgfqpoint{5.953494in}{1.327836in}}%
\pgfpathcurveto{\pgfqpoint{5.959318in}{1.322012in}}{\pgfqpoint{5.967218in}{1.318740in}}{\pgfqpoint{5.975454in}{1.318740in}}%
\pgfpathclose%
\pgfusepath{stroke,fill}%
\end{pgfscope}%
\begin{pgfscope}%
\pgfpathrectangle{\pgfqpoint{3.793912in}{0.557870in}}{\pgfqpoint{2.446088in}{1.684734in}}%
\pgfusepath{clip}%
\pgfsetbuttcap%
\pgfsetroundjoin%
\definecolor{currentfill}{rgb}{0.298039,0.447059,0.690196}%
\pgfsetfillcolor{currentfill}%
\pgfsetlinewidth{1.003750pt}%
\definecolor{currentstroke}{rgb}{0.298039,0.447059,0.690196}%
\pgfsetstrokecolor{currentstroke}%
\pgfsetdash{}{0pt}%
\pgfpathmoveto{\pgfqpoint{3.905098in}{2.125798in}}%
\pgfpathcurveto{\pgfqpoint{3.913334in}{2.125798in}}{\pgfqpoint{3.921234in}{2.129070in}}{\pgfqpoint{3.927058in}{2.134894in}}%
\pgfpathcurveto{\pgfqpoint{3.932882in}{2.140718in}}{\pgfqpoint{3.936155in}{2.148618in}}{\pgfqpoint{3.936155in}{2.156854in}}%
\pgfpathcurveto{\pgfqpoint{3.936155in}{2.165091in}}{\pgfqpoint{3.932882in}{2.172991in}}{\pgfqpoint{3.927058in}{2.178814in}}%
\pgfpathcurveto{\pgfqpoint{3.921234in}{2.184638in}}{\pgfqpoint{3.913334in}{2.187911in}}{\pgfqpoint{3.905098in}{2.187911in}}%
\pgfpathcurveto{\pgfqpoint{3.896862in}{2.187911in}}{\pgfqpoint{3.888962in}{2.184638in}}{\pgfqpoint{3.883138in}{2.178814in}}%
\pgfpathcurveto{\pgfqpoint{3.877314in}{2.172991in}}{\pgfqpoint{3.874042in}{2.165091in}}{\pgfqpoint{3.874042in}{2.156854in}}%
\pgfpathcurveto{\pgfqpoint{3.874042in}{2.148618in}}{\pgfqpoint{3.877314in}{2.140718in}}{\pgfqpoint{3.883138in}{2.134894in}}%
\pgfpathcurveto{\pgfqpoint{3.888962in}{2.129070in}}{\pgfqpoint{3.896862in}{2.125798in}}{\pgfqpoint{3.905098in}{2.125798in}}%
\pgfpathclose%
\pgfusepath{stroke,fill}%
\end{pgfscope}%
\begin{pgfscope}%
\pgfpathrectangle{\pgfqpoint{3.793912in}{0.557870in}}{\pgfqpoint{2.446088in}{1.684734in}}%
\pgfusepath{clip}%
\pgfsetbuttcap%
\pgfsetroundjoin%
\definecolor{currentfill}{rgb}{0.298039,0.447059,0.690196}%
\pgfsetfillcolor{currentfill}%
\pgfsetlinewidth{1.003750pt}%
\definecolor{currentstroke}{rgb}{0.298039,0.447059,0.690196}%
\pgfsetstrokecolor{currentstroke}%
\pgfsetdash{}{0pt}%
\pgfpathmoveto{\pgfqpoint{4.058458in}{1.593873in}}%
\pgfpathcurveto{\pgfqpoint{4.066694in}{1.593873in}}{\pgfqpoint{4.074594in}{1.597145in}}{\pgfqpoint{4.080418in}{1.602969in}}%
\pgfpathcurveto{\pgfqpoint{4.086242in}{1.608793in}}{\pgfqpoint{4.089514in}{1.616693in}}{\pgfqpoint{4.089514in}{1.624930in}}%
\pgfpathcurveto{\pgfqpoint{4.089514in}{1.633166in}}{\pgfqpoint{4.086242in}{1.641066in}}{\pgfqpoint{4.080418in}{1.646890in}}%
\pgfpathcurveto{\pgfqpoint{4.074594in}{1.652714in}}{\pgfqpoint{4.066694in}{1.655986in}}{\pgfqpoint{4.058458in}{1.655986in}}%
\pgfpathcurveto{\pgfqpoint{4.050221in}{1.655986in}}{\pgfqpoint{4.042321in}{1.652714in}}{\pgfqpoint{4.036498in}{1.646890in}}%
\pgfpathcurveto{\pgfqpoint{4.030674in}{1.641066in}}{\pgfqpoint{4.027401in}{1.633166in}}{\pgfqpoint{4.027401in}{1.624930in}}%
\pgfpathcurveto{\pgfqpoint{4.027401in}{1.616693in}}{\pgfqpoint{4.030674in}{1.608793in}}{\pgfqpoint{4.036498in}{1.602969in}}%
\pgfpathcurveto{\pgfqpoint{4.042321in}{1.597145in}}{\pgfqpoint{4.050221in}{1.593873in}}{\pgfqpoint{4.058458in}{1.593873in}}%
\pgfpathclose%
\pgfusepath{stroke,fill}%
\end{pgfscope}%
\begin{pgfscope}%
\pgfpathrectangle{\pgfqpoint{3.793912in}{0.557870in}}{\pgfqpoint{2.446088in}{1.684734in}}%
\pgfusepath{clip}%
\pgfsetbuttcap%
\pgfsetroundjoin%
\definecolor{currentfill}{rgb}{0.298039,0.447059,0.690196}%
\pgfsetfillcolor{currentfill}%
\pgfsetlinewidth{1.003750pt}%
\definecolor{currentstroke}{rgb}{0.298039,0.447059,0.690196}%
\pgfsetstrokecolor{currentstroke}%
\pgfsetdash{}{0pt}%
\pgfpathmoveto{\pgfqpoint{3.905098in}{2.125798in}}%
\pgfpathcurveto{\pgfqpoint{3.913334in}{2.125798in}}{\pgfqpoint{3.921234in}{2.129070in}}{\pgfqpoint{3.927058in}{2.134894in}}%
\pgfpathcurveto{\pgfqpoint{3.932882in}{2.140718in}}{\pgfqpoint{3.936155in}{2.148618in}}{\pgfqpoint{3.936155in}{2.156854in}}%
\pgfpathcurveto{\pgfqpoint{3.936155in}{2.165091in}}{\pgfqpoint{3.932882in}{2.172991in}}{\pgfqpoint{3.927058in}{2.178814in}}%
\pgfpathcurveto{\pgfqpoint{3.921234in}{2.184638in}}{\pgfqpoint{3.913334in}{2.187911in}}{\pgfqpoint{3.905098in}{2.187911in}}%
\pgfpathcurveto{\pgfqpoint{3.896862in}{2.187911in}}{\pgfqpoint{3.888962in}{2.184638in}}{\pgfqpoint{3.883138in}{2.178814in}}%
\pgfpathcurveto{\pgfqpoint{3.877314in}{2.172991in}}{\pgfqpoint{3.874042in}{2.165091in}}{\pgfqpoint{3.874042in}{2.156854in}}%
\pgfpathcurveto{\pgfqpoint{3.874042in}{2.148618in}}{\pgfqpoint{3.877314in}{2.140718in}}{\pgfqpoint{3.883138in}{2.134894in}}%
\pgfpathcurveto{\pgfqpoint{3.888962in}{2.129070in}}{\pgfqpoint{3.896862in}{2.125798in}}{\pgfqpoint{3.905098in}{2.125798in}}%
\pgfpathclose%
\pgfusepath{stroke,fill}%
\end{pgfscope}%
\begin{pgfscope}%
\pgfpathrectangle{\pgfqpoint{3.793912in}{0.557870in}}{\pgfqpoint{2.446088in}{1.684734in}}%
\pgfusepath{clip}%
\pgfsetbuttcap%
\pgfsetroundjoin%
\definecolor{currentfill}{rgb}{0.298039,0.447059,0.690196}%
\pgfsetfillcolor{currentfill}%
\pgfsetlinewidth{1.003750pt}%
\definecolor{currentstroke}{rgb}{0.298039,0.447059,0.690196}%
\pgfsetstrokecolor{currentstroke}%
\pgfsetdash{}{0pt}%
\pgfpathmoveto{\pgfqpoint{4.671897in}{1.648900in}}%
\pgfpathcurveto{\pgfqpoint{4.680133in}{1.648900in}}{\pgfqpoint{4.688033in}{1.652172in}}{\pgfqpoint{4.693857in}{1.657996in}}%
\pgfpathcurveto{\pgfqpoint{4.699681in}{1.663820in}}{\pgfqpoint{4.702953in}{1.671720in}}{\pgfqpoint{4.702953in}{1.679956in}}%
\pgfpathcurveto{\pgfqpoint{4.702953in}{1.688193in}}{\pgfqpoint{4.699681in}{1.696093in}}{\pgfqpoint{4.693857in}{1.701917in}}%
\pgfpathcurveto{\pgfqpoint{4.688033in}{1.707740in}}{\pgfqpoint{4.680133in}{1.711013in}}{\pgfqpoint{4.671897in}{1.711013in}}%
\pgfpathcurveto{\pgfqpoint{4.663660in}{1.711013in}}{\pgfqpoint{4.655760in}{1.707740in}}{\pgfqpoint{4.649936in}{1.701917in}}%
\pgfpathcurveto{\pgfqpoint{4.644113in}{1.696093in}}{\pgfqpoint{4.640840in}{1.688193in}}{\pgfqpoint{4.640840in}{1.679956in}}%
\pgfpathcurveto{\pgfqpoint{4.640840in}{1.671720in}}{\pgfqpoint{4.644113in}{1.663820in}}{\pgfqpoint{4.649936in}{1.657996in}}%
\pgfpathcurveto{\pgfqpoint{4.655760in}{1.652172in}}{\pgfqpoint{4.663660in}{1.648900in}}{\pgfqpoint{4.671897in}{1.648900in}}%
\pgfpathclose%
\pgfusepath{stroke,fill}%
\end{pgfscope}%
\begin{pgfscope}%
\pgfpathrectangle{\pgfqpoint{3.793912in}{0.557870in}}{\pgfqpoint{2.446088in}{1.684734in}}%
\pgfusepath{clip}%
\pgfsetbuttcap%
\pgfsetroundjoin%
\definecolor{currentfill}{rgb}{0.298039,0.447059,0.690196}%
\pgfsetfillcolor{currentfill}%
\pgfsetlinewidth{1.003750pt}%
\definecolor{currentstroke}{rgb}{0.298039,0.447059,0.690196}%
\pgfsetstrokecolor{currentstroke}%
\pgfsetdash{}{0pt}%
\pgfpathmoveto{\pgfqpoint{3.905098in}{2.116627in}}%
\pgfpathcurveto{\pgfqpoint{3.913334in}{2.116627in}}{\pgfqpoint{3.921234in}{2.119899in}}{\pgfqpoint{3.927058in}{2.125723in}}%
\pgfpathcurveto{\pgfqpoint{3.932882in}{2.131547in}}{\pgfqpoint{3.936155in}{2.139447in}}{\pgfqpoint{3.936155in}{2.147683in}}%
\pgfpathcurveto{\pgfqpoint{3.936155in}{2.155919in}}{\pgfqpoint{3.932882in}{2.163819in}}{\pgfqpoint{3.927058in}{2.169643in}}%
\pgfpathcurveto{\pgfqpoint{3.921234in}{2.175467in}}{\pgfqpoint{3.913334in}{2.178740in}}{\pgfqpoint{3.905098in}{2.178740in}}%
\pgfpathcurveto{\pgfqpoint{3.896862in}{2.178740in}}{\pgfqpoint{3.888962in}{2.175467in}}{\pgfqpoint{3.883138in}{2.169643in}}%
\pgfpathcurveto{\pgfqpoint{3.877314in}{2.163819in}}{\pgfqpoint{3.874042in}{2.155919in}}{\pgfqpoint{3.874042in}{2.147683in}}%
\pgfpathcurveto{\pgfqpoint{3.874042in}{2.139447in}}{\pgfqpoint{3.877314in}{2.131547in}}{\pgfqpoint{3.883138in}{2.125723in}}%
\pgfpathcurveto{\pgfqpoint{3.888962in}{2.119899in}}{\pgfqpoint{3.896862in}{2.116627in}}{\pgfqpoint{3.905098in}{2.116627in}}%
\pgfpathclose%
\pgfusepath{stroke,fill}%
\end{pgfscope}%
\begin{pgfscope}%
\pgfpathrectangle{\pgfqpoint{3.793912in}{0.557870in}}{\pgfqpoint{2.446088in}{1.684734in}}%
\pgfusepath{clip}%
\pgfsetbuttcap%
\pgfsetroundjoin%
\definecolor{currentfill}{rgb}{0.298039,0.447059,0.690196}%
\pgfsetfillcolor{currentfill}%
\pgfsetlinewidth{1.003750pt}%
\definecolor{currentstroke}{rgb}{0.298039,0.447059,0.690196}%
\pgfsetstrokecolor{currentstroke}%
\pgfsetdash{}{0pt}%
\pgfpathmoveto{\pgfqpoint{4.288497in}{1.813980in}}%
\pgfpathcurveto{\pgfqpoint{4.296734in}{1.813980in}}{\pgfqpoint{4.304634in}{1.817252in}}{\pgfqpoint{4.310458in}{1.823076in}}%
\pgfpathcurveto{\pgfqpoint{4.316282in}{1.828900in}}{\pgfqpoint{4.319554in}{1.836800in}}{\pgfqpoint{4.319554in}{1.845036in}}%
\pgfpathcurveto{\pgfqpoint{4.319554in}{1.853273in}}{\pgfqpoint{4.316282in}{1.861173in}}{\pgfqpoint{4.310458in}{1.866997in}}%
\pgfpathcurveto{\pgfqpoint{4.304634in}{1.872821in}}{\pgfqpoint{4.296734in}{1.876093in}}{\pgfqpoint{4.288497in}{1.876093in}}%
\pgfpathcurveto{\pgfqpoint{4.280261in}{1.876093in}}{\pgfqpoint{4.272361in}{1.872821in}}{\pgfqpoint{4.266537in}{1.866997in}}%
\pgfpathcurveto{\pgfqpoint{4.260713in}{1.861173in}}{\pgfqpoint{4.257441in}{1.853273in}}{\pgfqpoint{4.257441in}{1.845036in}}%
\pgfpathcurveto{\pgfqpoint{4.257441in}{1.836800in}}{\pgfqpoint{4.260713in}{1.828900in}}{\pgfqpoint{4.266537in}{1.823076in}}%
\pgfpathcurveto{\pgfqpoint{4.272361in}{1.817252in}}{\pgfqpoint{4.280261in}{1.813980in}}{\pgfqpoint{4.288497in}{1.813980in}}%
\pgfpathclose%
\pgfusepath{stroke,fill}%
\end{pgfscope}%
\begin{pgfscope}%
\pgfpathrectangle{\pgfqpoint{3.793912in}{0.557870in}}{\pgfqpoint{2.446088in}{1.684734in}}%
\pgfusepath{clip}%
\pgfsetbuttcap%
\pgfsetroundjoin%
\definecolor{currentfill}{rgb}{0.298039,0.447059,0.690196}%
\pgfsetfillcolor{currentfill}%
\pgfsetlinewidth{1.003750pt}%
\definecolor{currentstroke}{rgb}{0.298039,0.447059,0.690196}%
\pgfsetstrokecolor{currentstroke}%
\pgfsetdash{}{0pt}%
\pgfpathmoveto{\pgfqpoint{3.905098in}{2.125798in}}%
\pgfpathcurveto{\pgfqpoint{3.913334in}{2.125798in}}{\pgfqpoint{3.921234in}{2.129070in}}{\pgfqpoint{3.927058in}{2.134894in}}%
\pgfpathcurveto{\pgfqpoint{3.932882in}{2.140718in}}{\pgfqpoint{3.936155in}{2.148618in}}{\pgfqpoint{3.936155in}{2.156854in}}%
\pgfpathcurveto{\pgfqpoint{3.936155in}{2.165091in}}{\pgfqpoint{3.932882in}{2.172991in}}{\pgfqpoint{3.927058in}{2.178814in}}%
\pgfpathcurveto{\pgfqpoint{3.921234in}{2.184638in}}{\pgfqpoint{3.913334in}{2.187911in}}{\pgfqpoint{3.905098in}{2.187911in}}%
\pgfpathcurveto{\pgfqpoint{3.896862in}{2.187911in}}{\pgfqpoint{3.888962in}{2.184638in}}{\pgfqpoint{3.883138in}{2.178814in}}%
\pgfpathcurveto{\pgfqpoint{3.877314in}{2.172991in}}{\pgfqpoint{3.874042in}{2.165091in}}{\pgfqpoint{3.874042in}{2.156854in}}%
\pgfpathcurveto{\pgfqpoint{3.874042in}{2.148618in}}{\pgfqpoint{3.877314in}{2.140718in}}{\pgfqpoint{3.883138in}{2.134894in}}%
\pgfpathcurveto{\pgfqpoint{3.888962in}{2.129070in}}{\pgfqpoint{3.896862in}{2.125798in}}{\pgfqpoint{3.905098in}{2.125798in}}%
\pgfpathclose%
\pgfusepath{stroke,fill}%
\end{pgfscope}%
\begin{pgfscope}%
\pgfpathrectangle{\pgfqpoint{3.793912in}{0.557870in}}{\pgfqpoint{2.446088in}{1.684734in}}%
\pgfusepath{clip}%
\pgfsetbuttcap%
\pgfsetroundjoin%
\definecolor{currentfill}{rgb}{0.298039,0.447059,0.690196}%
\pgfsetfillcolor{currentfill}%
\pgfsetlinewidth{1.003750pt}%
\definecolor{currentstroke}{rgb}{0.298039,0.447059,0.690196}%
\pgfsetstrokecolor{currentstroke}%
\pgfsetdash{}{0pt}%
\pgfpathmoveto{\pgfqpoint{3.905098in}{2.089113in}}%
\pgfpathcurveto{\pgfqpoint{3.913334in}{2.089113in}}{\pgfqpoint{3.921234in}{2.092386in}}{\pgfqpoint{3.927058in}{2.098210in}}%
\pgfpathcurveto{\pgfqpoint{3.932882in}{2.104033in}}{\pgfqpoint{3.936155in}{2.111933in}}{\pgfqpoint{3.936155in}{2.120170in}}%
\pgfpathcurveto{\pgfqpoint{3.936155in}{2.128406in}}{\pgfqpoint{3.932882in}{2.136306in}}{\pgfqpoint{3.927058in}{2.142130in}}%
\pgfpathcurveto{\pgfqpoint{3.921234in}{2.147954in}}{\pgfqpoint{3.913334in}{2.151226in}}{\pgfqpoint{3.905098in}{2.151226in}}%
\pgfpathcurveto{\pgfqpoint{3.896862in}{2.151226in}}{\pgfqpoint{3.888962in}{2.147954in}}{\pgfqpoint{3.883138in}{2.142130in}}%
\pgfpathcurveto{\pgfqpoint{3.877314in}{2.136306in}}{\pgfqpoint{3.874042in}{2.128406in}}{\pgfqpoint{3.874042in}{2.120170in}}%
\pgfpathcurveto{\pgfqpoint{3.874042in}{2.111933in}}{\pgfqpoint{3.877314in}{2.104033in}}{\pgfqpoint{3.883138in}{2.098210in}}%
\pgfpathcurveto{\pgfqpoint{3.888962in}{2.092386in}}{\pgfqpoint{3.896862in}{2.089113in}}{\pgfqpoint{3.905098in}{2.089113in}}%
\pgfpathclose%
\pgfusepath{stroke,fill}%
\end{pgfscope}%
\begin{pgfscope}%
\pgfpathrectangle{\pgfqpoint{3.793912in}{0.557870in}}{\pgfqpoint{2.446088in}{1.684734in}}%
\pgfusepath{clip}%
\pgfsetbuttcap%
\pgfsetroundjoin%
\definecolor{currentfill}{rgb}{0.298039,0.447059,0.690196}%
\pgfsetfillcolor{currentfill}%
\pgfsetlinewidth{1.003750pt}%
\definecolor{currentstroke}{rgb}{0.298039,0.447059,0.690196}%
\pgfsetstrokecolor{currentstroke}%
\pgfsetdash{}{0pt}%
\pgfpathmoveto{\pgfqpoint{3.905098in}{2.125798in}}%
\pgfpathcurveto{\pgfqpoint{3.913334in}{2.125798in}}{\pgfqpoint{3.921234in}{2.129070in}}{\pgfqpoint{3.927058in}{2.134894in}}%
\pgfpathcurveto{\pgfqpoint{3.932882in}{2.140718in}}{\pgfqpoint{3.936155in}{2.148618in}}{\pgfqpoint{3.936155in}{2.156854in}}%
\pgfpathcurveto{\pgfqpoint{3.936155in}{2.165091in}}{\pgfqpoint{3.932882in}{2.172991in}}{\pgfqpoint{3.927058in}{2.178814in}}%
\pgfpathcurveto{\pgfqpoint{3.921234in}{2.184638in}}{\pgfqpoint{3.913334in}{2.187911in}}{\pgfqpoint{3.905098in}{2.187911in}}%
\pgfpathcurveto{\pgfqpoint{3.896862in}{2.187911in}}{\pgfqpoint{3.888962in}{2.184638in}}{\pgfqpoint{3.883138in}{2.178814in}}%
\pgfpathcurveto{\pgfqpoint{3.877314in}{2.172991in}}{\pgfqpoint{3.874042in}{2.165091in}}{\pgfqpoint{3.874042in}{2.156854in}}%
\pgfpathcurveto{\pgfqpoint{3.874042in}{2.148618in}}{\pgfqpoint{3.877314in}{2.140718in}}{\pgfqpoint{3.883138in}{2.134894in}}%
\pgfpathcurveto{\pgfqpoint{3.888962in}{2.129070in}}{\pgfqpoint{3.896862in}{2.125798in}}{\pgfqpoint{3.905098in}{2.125798in}}%
\pgfpathclose%
\pgfusepath{stroke,fill}%
\end{pgfscope}%
\begin{pgfscope}%
\pgfpathrectangle{\pgfqpoint{3.793912in}{0.557870in}}{\pgfqpoint{2.446088in}{1.684734in}}%
\pgfusepath{clip}%
\pgfsetbuttcap%
\pgfsetroundjoin%
\definecolor{currentfill}{rgb}{0.298039,0.447059,0.690196}%
\pgfsetfillcolor{currentfill}%
\pgfsetlinewidth{1.003750pt}%
\definecolor{currentstroke}{rgb}{0.298039,0.447059,0.690196}%
\pgfsetstrokecolor{currentstroke}%
\pgfsetdash{}{0pt}%
\pgfpathmoveto{\pgfqpoint{3.905098in}{2.125798in}}%
\pgfpathcurveto{\pgfqpoint{3.913334in}{2.125798in}}{\pgfqpoint{3.921234in}{2.129070in}}{\pgfqpoint{3.927058in}{2.134894in}}%
\pgfpathcurveto{\pgfqpoint{3.932882in}{2.140718in}}{\pgfqpoint{3.936155in}{2.148618in}}{\pgfqpoint{3.936155in}{2.156854in}}%
\pgfpathcurveto{\pgfqpoint{3.936155in}{2.165091in}}{\pgfqpoint{3.932882in}{2.172991in}}{\pgfqpoint{3.927058in}{2.178814in}}%
\pgfpathcurveto{\pgfqpoint{3.921234in}{2.184638in}}{\pgfqpoint{3.913334in}{2.187911in}}{\pgfqpoint{3.905098in}{2.187911in}}%
\pgfpathcurveto{\pgfqpoint{3.896862in}{2.187911in}}{\pgfqpoint{3.888962in}{2.184638in}}{\pgfqpoint{3.883138in}{2.178814in}}%
\pgfpathcurveto{\pgfqpoint{3.877314in}{2.172991in}}{\pgfqpoint{3.874042in}{2.165091in}}{\pgfqpoint{3.874042in}{2.156854in}}%
\pgfpathcurveto{\pgfqpoint{3.874042in}{2.148618in}}{\pgfqpoint{3.877314in}{2.140718in}}{\pgfqpoint{3.883138in}{2.134894in}}%
\pgfpathcurveto{\pgfqpoint{3.888962in}{2.129070in}}{\pgfqpoint{3.896862in}{2.125798in}}{\pgfqpoint{3.905098in}{2.125798in}}%
\pgfpathclose%
\pgfusepath{stroke,fill}%
\end{pgfscope}%
\begin{pgfscope}%
\pgfpathrectangle{\pgfqpoint{3.793912in}{0.557870in}}{\pgfqpoint{2.446088in}{1.684734in}}%
\pgfusepath{clip}%
\pgfsetbuttcap%
\pgfsetroundjoin%
\definecolor{currentfill}{rgb}{0.298039,0.447059,0.690196}%
\pgfsetfillcolor{currentfill}%
\pgfsetlinewidth{1.003750pt}%
\definecolor{currentstroke}{rgb}{0.298039,0.447059,0.690196}%
\pgfsetstrokecolor{currentstroke}%
\pgfsetdash{}{0pt}%
\pgfpathmoveto{\pgfqpoint{4.748577in}{1.492991in}}%
\pgfpathcurveto{\pgfqpoint{4.756813in}{1.492991in}}{\pgfqpoint{4.764713in}{1.496263in}}{\pgfqpoint{4.770537in}{1.502087in}}%
\pgfpathcurveto{\pgfqpoint{4.776361in}{1.507911in}}{\pgfqpoint{4.779633in}{1.515811in}}{\pgfqpoint{4.779633in}{1.524047in}}%
\pgfpathcurveto{\pgfqpoint{4.779633in}{1.532284in}}{\pgfqpoint{4.776361in}{1.540184in}}{\pgfqpoint{4.770537in}{1.546008in}}%
\pgfpathcurveto{\pgfqpoint{4.764713in}{1.551831in}}{\pgfqpoint{4.756813in}{1.555104in}}{\pgfqpoint{4.748577in}{1.555104in}}%
\pgfpathcurveto{\pgfqpoint{4.740340in}{1.555104in}}{\pgfqpoint{4.732440in}{1.551831in}}{\pgfqpoint{4.726616in}{1.546008in}}%
\pgfpathcurveto{\pgfqpoint{4.720792in}{1.540184in}}{\pgfqpoint{4.717520in}{1.532284in}}{\pgfqpoint{4.717520in}{1.524047in}}%
\pgfpathcurveto{\pgfqpoint{4.717520in}{1.515811in}}{\pgfqpoint{4.720792in}{1.507911in}}{\pgfqpoint{4.726616in}{1.502087in}}%
\pgfpathcurveto{\pgfqpoint{4.732440in}{1.496263in}}{\pgfqpoint{4.740340in}{1.492991in}}{\pgfqpoint{4.748577in}{1.492991in}}%
\pgfpathclose%
\pgfusepath{stroke,fill}%
\end{pgfscope}%
\begin{pgfscope}%
\pgfpathrectangle{\pgfqpoint{3.793912in}{0.557870in}}{\pgfqpoint{2.446088in}{1.684734in}}%
\pgfusepath{clip}%
\pgfsetbuttcap%
\pgfsetroundjoin%
\definecolor{currentfill}{rgb}{0.298039,0.447059,0.690196}%
\pgfsetfillcolor{currentfill}%
\pgfsetlinewidth{1.003750pt}%
\definecolor{currentstroke}{rgb}{0.298039,0.447059,0.690196}%
\pgfsetstrokecolor{currentstroke}%
\pgfsetdash{}{0pt}%
\pgfpathmoveto{\pgfqpoint{3.905098in}{2.116627in}}%
\pgfpathcurveto{\pgfqpoint{3.913334in}{2.116627in}}{\pgfqpoint{3.921234in}{2.119899in}}{\pgfqpoint{3.927058in}{2.125723in}}%
\pgfpathcurveto{\pgfqpoint{3.932882in}{2.131547in}}{\pgfqpoint{3.936155in}{2.139447in}}{\pgfqpoint{3.936155in}{2.147683in}}%
\pgfpathcurveto{\pgfqpoint{3.936155in}{2.155919in}}{\pgfqpoint{3.932882in}{2.163819in}}{\pgfqpoint{3.927058in}{2.169643in}}%
\pgfpathcurveto{\pgfqpoint{3.921234in}{2.175467in}}{\pgfqpoint{3.913334in}{2.178740in}}{\pgfqpoint{3.905098in}{2.178740in}}%
\pgfpathcurveto{\pgfqpoint{3.896862in}{2.178740in}}{\pgfqpoint{3.888962in}{2.175467in}}{\pgfqpoint{3.883138in}{2.169643in}}%
\pgfpathcurveto{\pgfqpoint{3.877314in}{2.163819in}}{\pgfqpoint{3.874042in}{2.155919in}}{\pgfqpoint{3.874042in}{2.147683in}}%
\pgfpathcurveto{\pgfqpoint{3.874042in}{2.139447in}}{\pgfqpoint{3.877314in}{2.131547in}}{\pgfqpoint{3.883138in}{2.125723in}}%
\pgfpathcurveto{\pgfqpoint{3.888962in}{2.119899in}}{\pgfqpoint{3.896862in}{2.116627in}}{\pgfqpoint{3.905098in}{2.116627in}}%
\pgfpathclose%
\pgfusepath{stroke,fill}%
\end{pgfscope}%
\begin{pgfscope}%
\pgfpathrectangle{\pgfqpoint{3.793912in}{0.557870in}}{\pgfqpoint{2.446088in}{1.684734in}}%
\pgfusepath{clip}%
\pgfsetbuttcap%
\pgfsetroundjoin%
\definecolor{currentfill}{rgb}{0.298039,0.447059,0.690196}%
\pgfsetfillcolor{currentfill}%
\pgfsetlinewidth{1.003750pt}%
\definecolor{currentstroke}{rgb}{0.298039,0.447059,0.690196}%
\pgfsetstrokecolor{currentstroke}%
\pgfsetdash{}{0pt}%
\pgfpathmoveto{\pgfqpoint{4.978616in}{1.740611in}}%
\pgfpathcurveto{\pgfqpoint{4.986852in}{1.740611in}}{\pgfqpoint{4.994753in}{1.743883in}}{\pgfqpoint{5.000576in}{1.749707in}}%
\pgfpathcurveto{\pgfqpoint{5.006400in}{1.755531in}}{\pgfqpoint{5.009673in}{1.763431in}}{\pgfqpoint{5.009673in}{1.771667in}}%
\pgfpathcurveto{\pgfqpoint{5.009673in}{1.779904in}}{\pgfqpoint{5.006400in}{1.787804in}}{\pgfqpoint{5.000576in}{1.793628in}}%
\pgfpathcurveto{\pgfqpoint{4.994753in}{1.799452in}}{\pgfqpoint{4.986852in}{1.802724in}}{\pgfqpoint{4.978616in}{1.802724in}}%
\pgfpathcurveto{\pgfqpoint{4.970380in}{1.802724in}}{\pgfqpoint{4.962480in}{1.799452in}}{\pgfqpoint{4.956656in}{1.793628in}}%
\pgfpathcurveto{\pgfqpoint{4.950832in}{1.787804in}}{\pgfqpoint{4.947560in}{1.779904in}}{\pgfqpoint{4.947560in}{1.771667in}}%
\pgfpathcurveto{\pgfqpoint{4.947560in}{1.763431in}}{\pgfqpoint{4.950832in}{1.755531in}}{\pgfqpoint{4.956656in}{1.749707in}}%
\pgfpathcurveto{\pgfqpoint{4.962480in}{1.743883in}}{\pgfqpoint{4.970380in}{1.740611in}}{\pgfqpoint{4.978616in}{1.740611in}}%
\pgfpathclose%
\pgfusepath{stroke,fill}%
\end{pgfscope}%
\begin{pgfscope}%
\pgfpathrectangle{\pgfqpoint{3.793912in}{0.557870in}}{\pgfqpoint{2.446088in}{1.684734in}}%
\pgfusepath{clip}%
\pgfsetbuttcap%
\pgfsetroundjoin%
\definecolor{currentfill}{rgb}{0.298039,0.447059,0.690196}%
\pgfsetfillcolor{currentfill}%
\pgfsetlinewidth{1.003750pt}%
\definecolor{currentstroke}{rgb}{0.298039,0.447059,0.690196}%
\pgfsetstrokecolor{currentstroke}%
\pgfsetdash{}{0pt}%
\pgfpathmoveto{\pgfqpoint{3.905098in}{2.125798in}}%
\pgfpathcurveto{\pgfqpoint{3.913334in}{2.125798in}}{\pgfqpoint{3.921234in}{2.129070in}}{\pgfqpoint{3.927058in}{2.134894in}}%
\pgfpathcurveto{\pgfqpoint{3.932882in}{2.140718in}}{\pgfqpoint{3.936155in}{2.148618in}}{\pgfqpoint{3.936155in}{2.156854in}}%
\pgfpathcurveto{\pgfqpoint{3.936155in}{2.165091in}}{\pgfqpoint{3.932882in}{2.172991in}}{\pgfqpoint{3.927058in}{2.178814in}}%
\pgfpathcurveto{\pgfqpoint{3.921234in}{2.184638in}}{\pgfqpoint{3.913334in}{2.187911in}}{\pgfqpoint{3.905098in}{2.187911in}}%
\pgfpathcurveto{\pgfqpoint{3.896862in}{2.187911in}}{\pgfqpoint{3.888962in}{2.184638in}}{\pgfqpoint{3.883138in}{2.178814in}}%
\pgfpathcurveto{\pgfqpoint{3.877314in}{2.172991in}}{\pgfqpoint{3.874042in}{2.165091in}}{\pgfqpoint{3.874042in}{2.156854in}}%
\pgfpathcurveto{\pgfqpoint{3.874042in}{2.148618in}}{\pgfqpoint{3.877314in}{2.140718in}}{\pgfqpoint{3.883138in}{2.134894in}}%
\pgfpathcurveto{\pgfqpoint{3.888962in}{2.129070in}}{\pgfqpoint{3.896862in}{2.125798in}}{\pgfqpoint{3.905098in}{2.125798in}}%
\pgfpathclose%
\pgfusepath{stroke,fill}%
\end{pgfscope}%
\begin{pgfscope}%
\pgfpathrectangle{\pgfqpoint{3.793912in}{0.557870in}}{\pgfqpoint{2.446088in}{1.684734in}}%
\pgfusepath{clip}%
\pgfsetbuttcap%
\pgfsetroundjoin%
\definecolor{currentfill}{rgb}{0.298039,0.447059,0.690196}%
\pgfsetfillcolor{currentfill}%
\pgfsetlinewidth{1.003750pt}%
\definecolor{currentstroke}{rgb}{0.298039,0.447059,0.690196}%
\pgfsetstrokecolor{currentstroke}%
\pgfsetdash{}{0pt}%
\pgfpathmoveto{\pgfqpoint{3.905098in}{2.125798in}}%
\pgfpathcurveto{\pgfqpoint{3.913334in}{2.125798in}}{\pgfqpoint{3.921234in}{2.129070in}}{\pgfqpoint{3.927058in}{2.134894in}}%
\pgfpathcurveto{\pgfqpoint{3.932882in}{2.140718in}}{\pgfqpoint{3.936155in}{2.148618in}}{\pgfqpoint{3.936155in}{2.156854in}}%
\pgfpathcurveto{\pgfqpoint{3.936155in}{2.165091in}}{\pgfqpoint{3.932882in}{2.172991in}}{\pgfqpoint{3.927058in}{2.178814in}}%
\pgfpathcurveto{\pgfqpoint{3.921234in}{2.184638in}}{\pgfqpoint{3.913334in}{2.187911in}}{\pgfqpoint{3.905098in}{2.187911in}}%
\pgfpathcurveto{\pgfqpoint{3.896862in}{2.187911in}}{\pgfqpoint{3.888962in}{2.184638in}}{\pgfqpoint{3.883138in}{2.178814in}}%
\pgfpathcurveto{\pgfqpoint{3.877314in}{2.172991in}}{\pgfqpoint{3.874042in}{2.165091in}}{\pgfqpoint{3.874042in}{2.156854in}}%
\pgfpathcurveto{\pgfqpoint{3.874042in}{2.148618in}}{\pgfqpoint{3.877314in}{2.140718in}}{\pgfqpoint{3.883138in}{2.134894in}}%
\pgfpathcurveto{\pgfqpoint{3.888962in}{2.129070in}}{\pgfqpoint{3.896862in}{2.125798in}}{\pgfqpoint{3.905098in}{2.125798in}}%
\pgfpathclose%
\pgfusepath{stroke,fill}%
\end{pgfscope}%
\begin{pgfscope}%
\pgfpathrectangle{\pgfqpoint{3.793912in}{0.557870in}}{\pgfqpoint{2.446088in}{1.684734in}}%
\pgfusepath{clip}%
\pgfsetbuttcap%
\pgfsetroundjoin%
\definecolor{currentfill}{rgb}{0.298039,0.447059,0.690196}%
\pgfsetfillcolor{currentfill}%
\pgfsetlinewidth{1.003750pt}%
\definecolor{currentstroke}{rgb}{0.298039,0.447059,0.690196}%
\pgfsetstrokecolor{currentstroke}%
\pgfsetdash{}{0pt}%
\pgfpathmoveto{\pgfqpoint{3.905098in}{2.125798in}}%
\pgfpathcurveto{\pgfqpoint{3.913334in}{2.125798in}}{\pgfqpoint{3.921234in}{2.129070in}}{\pgfqpoint{3.927058in}{2.134894in}}%
\pgfpathcurveto{\pgfqpoint{3.932882in}{2.140718in}}{\pgfqpoint{3.936155in}{2.148618in}}{\pgfqpoint{3.936155in}{2.156854in}}%
\pgfpathcurveto{\pgfqpoint{3.936155in}{2.165091in}}{\pgfqpoint{3.932882in}{2.172991in}}{\pgfqpoint{3.927058in}{2.178814in}}%
\pgfpathcurveto{\pgfqpoint{3.921234in}{2.184638in}}{\pgfqpoint{3.913334in}{2.187911in}}{\pgfqpoint{3.905098in}{2.187911in}}%
\pgfpathcurveto{\pgfqpoint{3.896862in}{2.187911in}}{\pgfqpoint{3.888962in}{2.184638in}}{\pgfqpoint{3.883138in}{2.178814in}}%
\pgfpathcurveto{\pgfqpoint{3.877314in}{2.172991in}}{\pgfqpoint{3.874042in}{2.165091in}}{\pgfqpoint{3.874042in}{2.156854in}}%
\pgfpathcurveto{\pgfqpoint{3.874042in}{2.148618in}}{\pgfqpoint{3.877314in}{2.140718in}}{\pgfqpoint{3.883138in}{2.134894in}}%
\pgfpathcurveto{\pgfqpoint{3.888962in}{2.129070in}}{\pgfqpoint{3.896862in}{2.125798in}}{\pgfqpoint{3.905098in}{2.125798in}}%
\pgfpathclose%
\pgfusepath{stroke,fill}%
\end{pgfscope}%
\begin{pgfscope}%
\pgfpathrectangle{\pgfqpoint{3.793912in}{0.557870in}}{\pgfqpoint{2.446088in}{1.684734in}}%
\pgfusepath{clip}%
\pgfsetbuttcap%
\pgfsetroundjoin%
\definecolor{currentfill}{rgb}{0.298039,0.447059,0.690196}%
\pgfsetfillcolor{currentfill}%
\pgfsetlinewidth{1.003750pt}%
\definecolor{currentstroke}{rgb}{0.298039,0.447059,0.690196}%
\pgfsetstrokecolor{currentstroke}%
\pgfsetdash{}{0pt}%
\pgfpathmoveto{\pgfqpoint{3.905098in}{2.125798in}}%
\pgfpathcurveto{\pgfqpoint{3.913334in}{2.125798in}}{\pgfqpoint{3.921234in}{2.129070in}}{\pgfqpoint{3.927058in}{2.134894in}}%
\pgfpathcurveto{\pgfqpoint{3.932882in}{2.140718in}}{\pgfqpoint{3.936155in}{2.148618in}}{\pgfqpoint{3.936155in}{2.156854in}}%
\pgfpathcurveto{\pgfqpoint{3.936155in}{2.165091in}}{\pgfqpoint{3.932882in}{2.172991in}}{\pgfqpoint{3.927058in}{2.178814in}}%
\pgfpathcurveto{\pgfqpoint{3.921234in}{2.184638in}}{\pgfqpoint{3.913334in}{2.187911in}}{\pgfqpoint{3.905098in}{2.187911in}}%
\pgfpathcurveto{\pgfqpoint{3.896862in}{2.187911in}}{\pgfqpoint{3.888962in}{2.184638in}}{\pgfqpoint{3.883138in}{2.178814in}}%
\pgfpathcurveto{\pgfqpoint{3.877314in}{2.172991in}}{\pgfqpoint{3.874042in}{2.165091in}}{\pgfqpoint{3.874042in}{2.156854in}}%
\pgfpathcurveto{\pgfqpoint{3.874042in}{2.148618in}}{\pgfqpoint{3.877314in}{2.140718in}}{\pgfqpoint{3.883138in}{2.134894in}}%
\pgfpathcurveto{\pgfqpoint{3.888962in}{2.129070in}}{\pgfqpoint{3.896862in}{2.125798in}}{\pgfqpoint{3.905098in}{2.125798in}}%
\pgfpathclose%
\pgfusepath{stroke,fill}%
\end{pgfscope}%
\begin{pgfscope}%
\pgfpathrectangle{\pgfqpoint{3.793912in}{0.557870in}}{\pgfqpoint{2.446088in}{1.684734in}}%
\pgfusepath{clip}%
\pgfsetbuttcap%
\pgfsetroundjoin%
\definecolor{currentfill}{rgb}{0.298039,0.447059,0.690196}%
\pgfsetfillcolor{currentfill}%
\pgfsetlinewidth{1.003750pt}%
\definecolor{currentstroke}{rgb}{0.298039,0.447059,0.690196}%
\pgfsetstrokecolor{currentstroke}%
\pgfsetdash{}{0pt}%
\pgfpathmoveto{\pgfqpoint{5.975454in}{1.327911in}}%
\pgfpathcurveto{\pgfqpoint{5.983691in}{1.327911in}}{\pgfqpoint{5.991591in}{1.331183in}}{\pgfqpoint{5.997415in}{1.337007in}}%
\pgfpathcurveto{\pgfqpoint{6.003239in}{1.342831in}}{\pgfqpoint{6.006511in}{1.350731in}}{\pgfqpoint{6.006511in}{1.358967in}}%
\pgfpathcurveto{\pgfqpoint{6.006511in}{1.367203in}}{\pgfqpoint{6.003239in}{1.375104in}}{\pgfqpoint{5.997415in}{1.380927in}}%
\pgfpathcurveto{\pgfqpoint{5.991591in}{1.386751in}}{\pgfqpoint{5.983691in}{1.390024in}}{\pgfqpoint{5.975454in}{1.390024in}}%
\pgfpathcurveto{\pgfqpoint{5.967218in}{1.390024in}}{\pgfqpoint{5.959318in}{1.386751in}}{\pgfqpoint{5.953494in}{1.380927in}}%
\pgfpathcurveto{\pgfqpoint{5.947670in}{1.375104in}}{\pgfqpoint{5.944398in}{1.367203in}}{\pgfqpoint{5.944398in}{1.358967in}}%
\pgfpathcurveto{\pgfqpoint{5.944398in}{1.350731in}}{\pgfqpoint{5.947670in}{1.342831in}}{\pgfqpoint{5.953494in}{1.337007in}}%
\pgfpathcurveto{\pgfqpoint{5.959318in}{1.331183in}}{\pgfqpoint{5.967218in}{1.327911in}}{\pgfqpoint{5.975454in}{1.327911in}}%
\pgfpathclose%
\pgfusepath{stroke,fill}%
\end{pgfscope}%
\begin{pgfscope}%
\pgfpathrectangle{\pgfqpoint{3.793912in}{0.557870in}}{\pgfqpoint{2.446088in}{1.684734in}}%
\pgfusepath{clip}%
\pgfsetbuttcap%
\pgfsetroundjoin%
\definecolor{currentfill}{rgb}{0.298039,0.447059,0.690196}%
\pgfsetfillcolor{currentfill}%
\pgfsetlinewidth{1.003750pt}%
\definecolor{currentstroke}{rgb}{0.298039,0.447059,0.690196}%
\pgfsetstrokecolor{currentstroke}%
\pgfsetdash{}{0pt}%
\pgfpathmoveto{\pgfqpoint{5.898775in}{1.667242in}}%
\pgfpathcurveto{\pgfqpoint{5.907011in}{1.667242in}}{\pgfqpoint{5.914911in}{1.670514in}}{\pgfqpoint{5.920735in}{1.676338in}}%
\pgfpathcurveto{\pgfqpoint{5.926559in}{1.682162in}}{\pgfqpoint{5.929831in}{1.690062in}}{\pgfqpoint{5.929831in}{1.698298in}}%
\pgfpathcurveto{\pgfqpoint{5.929831in}{1.706535in}}{\pgfqpoint{5.926559in}{1.714435in}}{\pgfqpoint{5.920735in}{1.720259in}}%
\pgfpathcurveto{\pgfqpoint{5.914911in}{1.726083in}}{\pgfqpoint{5.907011in}{1.729355in}}{\pgfqpoint{5.898775in}{1.729355in}}%
\pgfpathcurveto{\pgfqpoint{5.890538in}{1.729355in}}{\pgfqpoint{5.882638in}{1.726083in}}{\pgfqpoint{5.876814in}{1.720259in}}%
\pgfpathcurveto{\pgfqpoint{5.870990in}{1.714435in}}{\pgfqpoint{5.867718in}{1.706535in}}{\pgfqpoint{5.867718in}{1.698298in}}%
\pgfpathcurveto{\pgfqpoint{5.867718in}{1.690062in}}{\pgfqpoint{5.870990in}{1.682162in}}{\pgfqpoint{5.876814in}{1.676338in}}%
\pgfpathcurveto{\pgfqpoint{5.882638in}{1.670514in}}{\pgfqpoint{5.890538in}{1.667242in}}{\pgfqpoint{5.898775in}{1.667242in}}%
\pgfpathclose%
\pgfusepath{stroke,fill}%
\end{pgfscope}%
\begin{pgfscope}%
\pgfpathrectangle{\pgfqpoint{3.793912in}{0.557870in}}{\pgfqpoint{2.446088in}{1.684734in}}%
\pgfusepath{clip}%
\pgfsetbuttcap%
\pgfsetroundjoin%
\definecolor{currentfill}{rgb}{0.298039,0.447059,0.690196}%
\pgfsetfillcolor{currentfill}%
\pgfsetlinewidth{1.003750pt}%
\definecolor{currentstroke}{rgb}{0.298039,0.447059,0.690196}%
\pgfsetstrokecolor{currentstroke}%
\pgfsetdash{}{0pt}%
\pgfpathmoveto{\pgfqpoint{5.975454in}{1.327911in}}%
\pgfpathcurveto{\pgfqpoint{5.983691in}{1.327911in}}{\pgfqpoint{5.991591in}{1.331183in}}{\pgfqpoint{5.997415in}{1.337007in}}%
\pgfpathcurveto{\pgfqpoint{6.003239in}{1.342831in}}{\pgfqpoint{6.006511in}{1.350731in}}{\pgfqpoint{6.006511in}{1.358967in}}%
\pgfpathcurveto{\pgfqpoint{6.006511in}{1.367203in}}{\pgfqpoint{6.003239in}{1.375104in}}{\pgfqpoint{5.997415in}{1.380927in}}%
\pgfpathcurveto{\pgfqpoint{5.991591in}{1.386751in}}{\pgfqpoint{5.983691in}{1.390024in}}{\pgfqpoint{5.975454in}{1.390024in}}%
\pgfpathcurveto{\pgfqpoint{5.967218in}{1.390024in}}{\pgfqpoint{5.959318in}{1.386751in}}{\pgfqpoint{5.953494in}{1.380927in}}%
\pgfpathcurveto{\pgfqpoint{5.947670in}{1.375104in}}{\pgfqpoint{5.944398in}{1.367203in}}{\pgfqpoint{5.944398in}{1.358967in}}%
\pgfpathcurveto{\pgfqpoint{5.944398in}{1.350731in}}{\pgfqpoint{5.947670in}{1.342831in}}{\pgfqpoint{5.953494in}{1.337007in}}%
\pgfpathcurveto{\pgfqpoint{5.959318in}{1.331183in}}{\pgfqpoint{5.967218in}{1.327911in}}{\pgfqpoint{5.975454in}{1.327911in}}%
\pgfpathclose%
\pgfusepath{stroke,fill}%
\end{pgfscope}%
\begin{pgfscope}%
\pgfpathrectangle{\pgfqpoint{3.793912in}{0.557870in}}{\pgfqpoint{2.446088in}{1.684734in}}%
\pgfusepath{clip}%
\pgfsetbuttcap%
\pgfsetroundjoin%
\definecolor{currentfill}{rgb}{0.298039,0.447059,0.690196}%
\pgfsetfillcolor{currentfill}%
\pgfsetlinewidth{1.003750pt}%
\definecolor{currentstroke}{rgb}{0.298039,0.447059,0.690196}%
\pgfsetstrokecolor{currentstroke}%
\pgfsetdash{}{0pt}%
\pgfpathmoveto{\pgfqpoint{3.905098in}{2.125798in}}%
\pgfpathcurveto{\pgfqpoint{3.913334in}{2.125798in}}{\pgfqpoint{3.921234in}{2.129070in}}{\pgfqpoint{3.927058in}{2.134894in}}%
\pgfpathcurveto{\pgfqpoint{3.932882in}{2.140718in}}{\pgfqpoint{3.936155in}{2.148618in}}{\pgfqpoint{3.936155in}{2.156854in}}%
\pgfpathcurveto{\pgfqpoint{3.936155in}{2.165091in}}{\pgfqpoint{3.932882in}{2.172991in}}{\pgfqpoint{3.927058in}{2.178814in}}%
\pgfpathcurveto{\pgfqpoint{3.921234in}{2.184638in}}{\pgfqpoint{3.913334in}{2.187911in}}{\pgfqpoint{3.905098in}{2.187911in}}%
\pgfpathcurveto{\pgfqpoint{3.896862in}{2.187911in}}{\pgfqpoint{3.888962in}{2.184638in}}{\pgfqpoint{3.883138in}{2.178814in}}%
\pgfpathcurveto{\pgfqpoint{3.877314in}{2.172991in}}{\pgfqpoint{3.874042in}{2.165091in}}{\pgfqpoint{3.874042in}{2.156854in}}%
\pgfpathcurveto{\pgfqpoint{3.874042in}{2.148618in}}{\pgfqpoint{3.877314in}{2.140718in}}{\pgfqpoint{3.883138in}{2.134894in}}%
\pgfpathcurveto{\pgfqpoint{3.888962in}{2.129070in}}{\pgfqpoint{3.896862in}{2.125798in}}{\pgfqpoint{3.905098in}{2.125798in}}%
\pgfpathclose%
\pgfusepath{stroke,fill}%
\end{pgfscope}%
\begin{pgfscope}%
\pgfpathrectangle{\pgfqpoint{3.793912in}{0.557870in}}{\pgfqpoint{2.446088in}{1.684734in}}%
\pgfusepath{clip}%
\pgfsetbuttcap%
\pgfsetroundjoin%
\definecolor{currentfill}{rgb}{0.298039,0.447059,0.690196}%
\pgfsetfillcolor{currentfill}%
\pgfsetlinewidth{1.003750pt}%
\definecolor{currentstroke}{rgb}{0.298039,0.447059,0.690196}%
\pgfsetstrokecolor{currentstroke}%
\pgfsetdash{}{0pt}%
\pgfpathmoveto{\pgfqpoint{5.592055in}{1.841493in}}%
\pgfpathcurveto{\pgfqpoint{5.600291in}{1.841493in}}{\pgfqpoint{5.608191in}{1.844765in}}{\pgfqpoint{5.614015in}{1.850589in}}%
\pgfpathcurveto{\pgfqpoint{5.619839in}{1.856413in}}{\pgfqpoint{5.623112in}{1.864313in}}{\pgfqpoint{5.623112in}{1.872550in}}%
\pgfpathcurveto{\pgfqpoint{5.623112in}{1.880786in}}{\pgfqpoint{5.619839in}{1.888686in}}{\pgfqpoint{5.614015in}{1.894510in}}%
\pgfpathcurveto{\pgfqpoint{5.608191in}{1.900334in}}{\pgfqpoint{5.600291in}{1.903606in}}{\pgfqpoint{5.592055in}{1.903606in}}%
\pgfpathcurveto{\pgfqpoint{5.583819in}{1.903606in}}{\pgfqpoint{5.575919in}{1.900334in}}{\pgfqpoint{5.570095in}{1.894510in}}%
\pgfpathcurveto{\pgfqpoint{5.564271in}{1.888686in}}{\pgfqpoint{5.560999in}{1.880786in}}{\pgfqpoint{5.560999in}{1.872550in}}%
\pgfpathcurveto{\pgfqpoint{5.560999in}{1.864313in}}{\pgfqpoint{5.564271in}{1.856413in}}{\pgfqpoint{5.570095in}{1.850589in}}%
\pgfpathcurveto{\pgfqpoint{5.575919in}{1.844765in}}{\pgfqpoint{5.583819in}{1.841493in}}{\pgfqpoint{5.592055in}{1.841493in}}%
\pgfpathclose%
\pgfusepath{stroke,fill}%
\end{pgfscope}%
\begin{pgfscope}%
\pgfpathrectangle{\pgfqpoint{3.793912in}{0.557870in}}{\pgfqpoint{2.446088in}{1.684734in}}%
\pgfusepath{clip}%
\pgfsetbuttcap%
\pgfsetroundjoin%
\definecolor{currentfill}{rgb}{0.298039,0.447059,0.690196}%
\pgfsetfillcolor{currentfill}%
\pgfsetlinewidth{1.003750pt}%
\definecolor{currentstroke}{rgb}{0.298039,0.447059,0.690196}%
\pgfsetstrokecolor{currentstroke}%
\pgfsetdash{}{0pt}%
\pgfpathmoveto{\pgfqpoint{3.905098in}{2.125798in}}%
\pgfpathcurveto{\pgfqpoint{3.913334in}{2.125798in}}{\pgfqpoint{3.921234in}{2.129070in}}{\pgfqpoint{3.927058in}{2.134894in}}%
\pgfpathcurveto{\pgfqpoint{3.932882in}{2.140718in}}{\pgfqpoint{3.936155in}{2.148618in}}{\pgfqpoint{3.936155in}{2.156854in}}%
\pgfpathcurveto{\pgfqpoint{3.936155in}{2.165091in}}{\pgfqpoint{3.932882in}{2.172991in}}{\pgfqpoint{3.927058in}{2.178814in}}%
\pgfpathcurveto{\pgfqpoint{3.921234in}{2.184638in}}{\pgfqpoint{3.913334in}{2.187911in}}{\pgfqpoint{3.905098in}{2.187911in}}%
\pgfpathcurveto{\pgfqpoint{3.896862in}{2.187911in}}{\pgfqpoint{3.888962in}{2.184638in}}{\pgfqpoint{3.883138in}{2.178814in}}%
\pgfpathcurveto{\pgfqpoint{3.877314in}{2.172991in}}{\pgfqpoint{3.874042in}{2.165091in}}{\pgfqpoint{3.874042in}{2.156854in}}%
\pgfpathcurveto{\pgfqpoint{3.874042in}{2.148618in}}{\pgfqpoint{3.877314in}{2.140718in}}{\pgfqpoint{3.883138in}{2.134894in}}%
\pgfpathcurveto{\pgfqpoint{3.888962in}{2.129070in}}{\pgfqpoint{3.896862in}{2.125798in}}{\pgfqpoint{3.905098in}{2.125798in}}%
\pgfpathclose%
\pgfusepath{stroke,fill}%
\end{pgfscope}%
\begin{pgfscope}%
\pgfpathrectangle{\pgfqpoint{3.793912in}{0.557870in}}{\pgfqpoint{2.446088in}{1.684734in}}%
\pgfusepath{clip}%
\pgfsetbuttcap%
\pgfsetroundjoin%
\definecolor{currentfill}{rgb}{0.298039,0.447059,0.690196}%
\pgfsetfillcolor{currentfill}%
\pgfsetlinewidth{1.003750pt}%
\definecolor{currentstroke}{rgb}{0.298039,0.447059,0.690196}%
\pgfsetstrokecolor{currentstroke}%
\pgfsetdash{}{0pt}%
\pgfpathmoveto{\pgfqpoint{5.975454in}{1.337082in}}%
\pgfpathcurveto{\pgfqpoint{5.983691in}{1.337082in}}{\pgfqpoint{5.991591in}{1.340354in}}{\pgfqpoint{5.997415in}{1.346178in}}%
\pgfpathcurveto{\pgfqpoint{6.003239in}{1.352002in}}{\pgfqpoint{6.006511in}{1.359902in}}{\pgfqpoint{6.006511in}{1.368138in}}%
\pgfpathcurveto{\pgfqpoint{6.006511in}{1.376375in}}{\pgfqpoint{6.003239in}{1.384275in}}{\pgfqpoint{5.997415in}{1.390099in}}%
\pgfpathcurveto{\pgfqpoint{5.991591in}{1.395923in}}{\pgfqpoint{5.983691in}{1.399195in}}{\pgfqpoint{5.975454in}{1.399195in}}%
\pgfpathcurveto{\pgfqpoint{5.967218in}{1.399195in}}{\pgfqpoint{5.959318in}{1.395923in}}{\pgfqpoint{5.953494in}{1.390099in}}%
\pgfpathcurveto{\pgfqpoint{5.947670in}{1.384275in}}{\pgfqpoint{5.944398in}{1.376375in}}{\pgfqpoint{5.944398in}{1.368138in}}%
\pgfpathcurveto{\pgfqpoint{5.944398in}{1.359902in}}{\pgfqpoint{5.947670in}{1.352002in}}{\pgfqpoint{5.953494in}{1.346178in}}%
\pgfpathcurveto{\pgfqpoint{5.959318in}{1.340354in}}{\pgfqpoint{5.967218in}{1.337082in}}{\pgfqpoint{5.975454in}{1.337082in}}%
\pgfpathclose%
\pgfusepath{stroke,fill}%
\end{pgfscope}%
\begin{pgfscope}%
\pgfpathrectangle{\pgfqpoint{3.793912in}{0.557870in}}{\pgfqpoint{2.446088in}{1.684734in}}%
\pgfusepath{clip}%
\pgfsetbuttcap%
\pgfsetroundjoin%
\definecolor{currentfill}{rgb}{0.298039,0.447059,0.690196}%
\pgfsetfillcolor{currentfill}%
\pgfsetlinewidth{1.003750pt}%
\definecolor{currentstroke}{rgb}{0.298039,0.447059,0.690196}%
\pgfsetstrokecolor{currentstroke}%
\pgfsetdash{}{0pt}%
\pgfpathmoveto{\pgfqpoint{5.975454in}{1.318740in}}%
\pgfpathcurveto{\pgfqpoint{5.983691in}{1.318740in}}{\pgfqpoint{5.991591in}{1.322012in}}{\pgfqpoint{5.997415in}{1.327836in}}%
\pgfpathcurveto{\pgfqpoint{6.003239in}{1.333660in}}{\pgfqpoint{6.006511in}{1.341560in}}{\pgfqpoint{6.006511in}{1.349796in}}%
\pgfpathcurveto{\pgfqpoint{6.006511in}{1.358032in}}{\pgfqpoint{6.003239in}{1.365932in}}{\pgfqpoint{5.997415in}{1.371756in}}%
\pgfpathcurveto{\pgfqpoint{5.991591in}{1.377580in}}{\pgfqpoint{5.983691in}{1.380853in}}{\pgfqpoint{5.975454in}{1.380853in}}%
\pgfpathcurveto{\pgfqpoint{5.967218in}{1.380853in}}{\pgfqpoint{5.959318in}{1.377580in}}{\pgfqpoint{5.953494in}{1.371756in}}%
\pgfpathcurveto{\pgfqpoint{5.947670in}{1.365932in}}{\pgfqpoint{5.944398in}{1.358032in}}{\pgfqpoint{5.944398in}{1.349796in}}%
\pgfpathcurveto{\pgfqpoint{5.944398in}{1.341560in}}{\pgfqpoint{5.947670in}{1.333660in}}{\pgfqpoint{5.953494in}{1.327836in}}%
\pgfpathcurveto{\pgfqpoint{5.959318in}{1.322012in}}{\pgfqpoint{5.967218in}{1.318740in}}{\pgfqpoint{5.975454in}{1.318740in}}%
\pgfpathclose%
\pgfusepath{stroke,fill}%
\end{pgfscope}%
\begin{pgfscope}%
\pgfpathrectangle{\pgfqpoint{3.793912in}{0.557870in}}{\pgfqpoint{2.446088in}{1.684734in}}%
\pgfusepath{clip}%
\pgfsetbuttcap%
\pgfsetroundjoin%
\definecolor{currentfill}{rgb}{0.298039,0.447059,0.690196}%
\pgfsetfillcolor{currentfill}%
\pgfsetlinewidth{1.003750pt}%
\definecolor{currentstroke}{rgb}{0.298039,0.447059,0.690196}%
\pgfsetstrokecolor{currentstroke}%
\pgfsetdash{}{0pt}%
\pgfpathmoveto{\pgfqpoint{5.975454in}{1.355424in}}%
\pgfpathcurveto{\pgfqpoint{5.983691in}{1.355424in}}{\pgfqpoint{5.991591in}{1.358696in}}{\pgfqpoint{5.997415in}{1.364520in}}%
\pgfpathcurveto{\pgfqpoint{6.003239in}{1.370344in}}{\pgfqpoint{6.006511in}{1.378244in}}{\pgfqpoint{6.006511in}{1.386481in}}%
\pgfpathcurveto{\pgfqpoint{6.006511in}{1.394717in}}{\pgfqpoint{6.003239in}{1.402617in}}{\pgfqpoint{5.997415in}{1.408441in}}%
\pgfpathcurveto{\pgfqpoint{5.991591in}{1.414265in}}{\pgfqpoint{5.983691in}{1.417537in}}{\pgfqpoint{5.975454in}{1.417537in}}%
\pgfpathcurveto{\pgfqpoint{5.967218in}{1.417537in}}{\pgfqpoint{5.959318in}{1.414265in}}{\pgfqpoint{5.953494in}{1.408441in}}%
\pgfpathcurveto{\pgfqpoint{5.947670in}{1.402617in}}{\pgfqpoint{5.944398in}{1.394717in}}{\pgfqpoint{5.944398in}{1.386481in}}%
\pgfpathcurveto{\pgfqpoint{5.944398in}{1.378244in}}{\pgfqpoint{5.947670in}{1.370344in}}{\pgfqpoint{5.953494in}{1.364520in}}%
\pgfpathcurveto{\pgfqpoint{5.959318in}{1.358696in}}{\pgfqpoint{5.967218in}{1.355424in}}{\pgfqpoint{5.975454in}{1.355424in}}%
\pgfpathclose%
\pgfusepath{stroke,fill}%
\end{pgfscope}%
\begin{pgfscope}%
\pgfpathrectangle{\pgfqpoint{3.793912in}{0.557870in}}{\pgfqpoint{2.446088in}{1.684734in}}%
\pgfusepath{clip}%
\pgfsetbuttcap%
\pgfsetroundjoin%
\definecolor{currentfill}{rgb}{0.298039,0.447059,0.690196}%
\pgfsetfillcolor{currentfill}%
\pgfsetlinewidth{1.003750pt}%
\definecolor{currentstroke}{rgb}{0.298039,0.447059,0.690196}%
\pgfsetstrokecolor{currentstroke}%
\pgfsetdash{}{0pt}%
\pgfpathmoveto{\pgfqpoint{5.975454in}{1.373766in}}%
\pgfpathcurveto{\pgfqpoint{5.983691in}{1.373766in}}{\pgfqpoint{5.991591in}{1.377039in}}{\pgfqpoint{5.997415in}{1.382863in}}%
\pgfpathcurveto{\pgfqpoint{6.003239in}{1.388686in}}{\pgfqpoint{6.006511in}{1.396586in}}{\pgfqpoint{6.006511in}{1.404823in}}%
\pgfpathcurveto{\pgfqpoint{6.006511in}{1.413059in}}{\pgfqpoint{6.003239in}{1.420959in}}{\pgfqpoint{5.997415in}{1.426783in}}%
\pgfpathcurveto{\pgfqpoint{5.991591in}{1.432607in}}{\pgfqpoint{5.983691in}{1.435879in}}{\pgfqpoint{5.975454in}{1.435879in}}%
\pgfpathcurveto{\pgfqpoint{5.967218in}{1.435879in}}{\pgfqpoint{5.959318in}{1.432607in}}{\pgfqpoint{5.953494in}{1.426783in}}%
\pgfpathcurveto{\pgfqpoint{5.947670in}{1.420959in}}{\pgfqpoint{5.944398in}{1.413059in}}{\pgfqpoint{5.944398in}{1.404823in}}%
\pgfpathcurveto{\pgfqpoint{5.944398in}{1.396586in}}{\pgfqpoint{5.947670in}{1.388686in}}{\pgfqpoint{5.953494in}{1.382863in}}%
\pgfpathcurveto{\pgfqpoint{5.959318in}{1.377039in}}{\pgfqpoint{5.967218in}{1.373766in}}{\pgfqpoint{5.975454in}{1.373766in}}%
\pgfpathclose%
\pgfusepath{stroke,fill}%
\end{pgfscope}%
\begin{pgfscope}%
\pgfpathrectangle{\pgfqpoint{3.793912in}{0.557870in}}{\pgfqpoint{2.446088in}{1.684734in}}%
\pgfusepath{clip}%
\pgfsetbuttcap%
\pgfsetroundjoin%
\definecolor{currentfill}{rgb}{0.298039,0.447059,0.690196}%
\pgfsetfillcolor{currentfill}%
\pgfsetlinewidth{1.003750pt}%
\definecolor{currentstroke}{rgb}{0.298039,0.447059,0.690196}%
\pgfsetstrokecolor{currentstroke}%
\pgfsetdash{}{0pt}%
\pgfpathmoveto{\pgfqpoint{3.905098in}{2.125798in}}%
\pgfpathcurveto{\pgfqpoint{3.913334in}{2.125798in}}{\pgfqpoint{3.921234in}{2.129070in}}{\pgfqpoint{3.927058in}{2.134894in}}%
\pgfpathcurveto{\pgfqpoint{3.932882in}{2.140718in}}{\pgfqpoint{3.936155in}{2.148618in}}{\pgfqpoint{3.936155in}{2.156854in}}%
\pgfpathcurveto{\pgfqpoint{3.936155in}{2.165091in}}{\pgfqpoint{3.932882in}{2.172991in}}{\pgfqpoint{3.927058in}{2.178814in}}%
\pgfpathcurveto{\pgfqpoint{3.921234in}{2.184638in}}{\pgfqpoint{3.913334in}{2.187911in}}{\pgfqpoint{3.905098in}{2.187911in}}%
\pgfpathcurveto{\pgfqpoint{3.896862in}{2.187911in}}{\pgfqpoint{3.888962in}{2.184638in}}{\pgfqpoint{3.883138in}{2.178814in}}%
\pgfpathcurveto{\pgfqpoint{3.877314in}{2.172991in}}{\pgfqpoint{3.874042in}{2.165091in}}{\pgfqpoint{3.874042in}{2.156854in}}%
\pgfpathcurveto{\pgfqpoint{3.874042in}{2.148618in}}{\pgfqpoint{3.877314in}{2.140718in}}{\pgfqpoint{3.883138in}{2.134894in}}%
\pgfpathcurveto{\pgfqpoint{3.888962in}{2.129070in}}{\pgfqpoint{3.896862in}{2.125798in}}{\pgfqpoint{3.905098in}{2.125798in}}%
\pgfpathclose%
\pgfusepath{stroke,fill}%
\end{pgfscope}%
\begin{pgfscope}%
\pgfpathrectangle{\pgfqpoint{3.793912in}{0.557870in}}{\pgfqpoint{2.446088in}{1.684734in}}%
\pgfusepath{clip}%
\pgfsetbuttcap%
\pgfsetroundjoin%
\definecolor{currentfill}{rgb}{0.298039,0.447059,0.690196}%
\pgfsetfillcolor{currentfill}%
\pgfsetlinewidth{1.003750pt}%
\definecolor{currentstroke}{rgb}{0.298039,0.447059,0.690196}%
\pgfsetstrokecolor{currentstroke}%
\pgfsetdash{}{0pt}%
\pgfpathmoveto{\pgfqpoint{4.058458in}{1.575531in}}%
\pgfpathcurveto{\pgfqpoint{4.066694in}{1.575531in}}{\pgfqpoint{4.074594in}{1.578803in}}{\pgfqpoint{4.080418in}{1.584627in}}%
\pgfpathcurveto{\pgfqpoint{4.086242in}{1.590451in}}{\pgfqpoint{4.089514in}{1.598351in}}{\pgfqpoint{4.089514in}{1.606587in}}%
\pgfpathcurveto{\pgfqpoint{4.089514in}{1.614824in}}{\pgfqpoint{4.086242in}{1.622724in}}{\pgfqpoint{4.080418in}{1.628548in}}%
\pgfpathcurveto{\pgfqpoint{4.074594in}{1.634372in}}{\pgfqpoint{4.066694in}{1.637644in}}{\pgfqpoint{4.058458in}{1.637644in}}%
\pgfpathcurveto{\pgfqpoint{4.050221in}{1.637644in}}{\pgfqpoint{4.042321in}{1.634372in}}{\pgfqpoint{4.036498in}{1.628548in}}%
\pgfpathcurveto{\pgfqpoint{4.030674in}{1.622724in}}{\pgfqpoint{4.027401in}{1.614824in}}{\pgfqpoint{4.027401in}{1.606587in}}%
\pgfpathcurveto{\pgfqpoint{4.027401in}{1.598351in}}{\pgfqpoint{4.030674in}{1.590451in}}{\pgfqpoint{4.036498in}{1.584627in}}%
\pgfpathcurveto{\pgfqpoint{4.042321in}{1.578803in}}{\pgfqpoint{4.050221in}{1.575531in}}{\pgfqpoint{4.058458in}{1.575531in}}%
\pgfpathclose%
\pgfusepath{stroke,fill}%
\end{pgfscope}%
\begin{pgfscope}%
\pgfpathrectangle{\pgfqpoint{3.793912in}{0.557870in}}{\pgfqpoint{2.446088in}{1.684734in}}%
\pgfusepath{clip}%
\pgfsetbuttcap%
\pgfsetroundjoin%
\definecolor{currentfill}{rgb}{0.298039,0.447059,0.690196}%
\pgfsetfillcolor{currentfill}%
\pgfsetlinewidth{1.003750pt}%
\definecolor{currentstroke}{rgb}{0.298039,0.447059,0.690196}%
\pgfsetstrokecolor{currentstroke}%
\pgfsetdash{}{0pt}%
\pgfpathmoveto{\pgfqpoint{3.905098in}{2.125798in}}%
\pgfpathcurveto{\pgfqpoint{3.913334in}{2.125798in}}{\pgfqpoint{3.921234in}{2.129070in}}{\pgfqpoint{3.927058in}{2.134894in}}%
\pgfpathcurveto{\pgfqpoint{3.932882in}{2.140718in}}{\pgfqpoint{3.936155in}{2.148618in}}{\pgfqpoint{3.936155in}{2.156854in}}%
\pgfpathcurveto{\pgfqpoint{3.936155in}{2.165091in}}{\pgfqpoint{3.932882in}{2.172991in}}{\pgfqpoint{3.927058in}{2.178814in}}%
\pgfpathcurveto{\pgfqpoint{3.921234in}{2.184638in}}{\pgfqpoint{3.913334in}{2.187911in}}{\pgfqpoint{3.905098in}{2.187911in}}%
\pgfpathcurveto{\pgfqpoint{3.896862in}{2.187911in}}{\pgfqpoint{3.888962in}{2.184638in}}{\pgfqpoint{3.883138in}{2.178814in}}%
\pgfpathcurveto{\pgfqpoint{3.877314in}{2.172991in}}{\pgfqpoint{3.874042in}{2.165091in}}{\pgfqpoint{3.874042in}{2.156854in}}%
\pgfpathcurveto{\pgfqpoint{3.874042in}{2.148618in}}{\pgfqpoint{3.877314in}{2.140718in}}{\pgfqpoint{3.883138in}{2.134894in}}%
\pgfpathcurveto{\pgfqpoint{3.888962in}{2.129070in}}{\pgfqpoint{3.896862in}{2.125798in}}{\pgfqpoint{3.905098in}{2.125798in}}%
\pgfpathclose%
\pgfusepath{stroke,fill}%
\end{pgfscope}%
\begin{pgfscope}%
\pgfpathrectangle{\pgfqpoint{3.793912in}{0.557870in}}{\pgfqpoint{2.446088in}{1.684734in}}%
\pgfusepath{clip}%
\pgfsetbuttcap%
\pgfsetroundjoin%
\definecolor{currentfill}{rgb}{0.298039,0.447059,0.690196}%
\pgfsetfillcolor{currentfill}%
\pgfsetlinewidth{1.003750pt}%
\definecolor{currentstroke}{rgb}{0.298039,0.447059,0.690196}%
\pgfsetstrokecolor{currentstroke}%
\pgfsetdash{}{0pt}%
\pgfpathmoveto{\pgfqpoint{4.058458in}{1.905691in}}%
\pgfpathcurveto{\pgfqpoint{4.066694in}{1.905691in}}{\pgfqpoint{4.074594in}{1.908963in}}{\pgfqpoint{4.080418in}{1.914787in}}%
\pgfpathcurveto{\pgfqpoint{4.086242in}{1.920611in}}{\pgfqpoint{4.089514in}{1.928511in}}{\pgfqpoint{4.089514in}{1.936747in}}%
\pgfpathcurveto{\pgfqpoint{4.089514in}{1.944984in}}{\pgfqpoint{4.086242in}{1.952884in}}{\pgfqpoint{4.080418in}{1.958708in}}%
\pgfpathcurveto{\pgfqpoint{4.074594in}{1.964532in}}{\pgfqpoint{4.066694in}{1.967804in}}{\pgfqpoint{4.058458in}{1.967804in}}%
\pgfpathcurveto{\pgfqpoint{4.050221in}{1.967804in}}{\pgfqpoint{4.042321in}{1.964532in}}{\pgfqpoint{4.036498in}{1.958708in}}%
\pgfpathcurveto{\pgfqpoint{4.030674in}{1.952884in}}{\pgfqpoint{4.027401in}{1.944984in}}{\pgfqpoint{4.027401in}{1.936747in}}%
\pgfpathcurveto{\pgfqpoint{4.027401in}{1.928511in}}{\pgfqpoint{4.030674in}{1.920611in}}{\pgfqpoint{4.036498in}{1.914787in}}%
\pgfpathcurveto{\pgfqpoint{4.042321in}{1.908963in}}{\pgfqpoint{4.050221in}{1.905691in}}{\pgfqpoint{4.058458in}{1.905691in}}%
\pgfpathclose%
\pgfusepath{stroke,fill}%
\end{pgfscope}%
\begin{pgfscope}%
\pgfpathrectangle{\pgfqpoint{3.793912in}{0.557870in}}{\pgfqpoint{2.446088in}{1.684734in}}%
\pgfusepath{clip}%
\pgfsetbuttcap%
\pgfsetroundjoin%
\definecolor{currentfill}{rgb}{0.298039,0.447059,0.690196}%
\pgfsetfillcolor{currentfill}%
\pgfsetlinewidth{1.003750pt}%
\definecolor{currentstroke}{rgb}{0.298039,0.447059,0.690196}%
\pgfsetstrokecolor{currentstroke}%
\pgfsetdash{}{0pt}%
\pgfpathmoveto{\pgfqpoint{3.905098in}{2.116627in}}%
\pgfpathcurveto{\pgfqpoint{3.913334in}{2.116627in}}{\pgfqpoint{3.921234in}{2.119899in}}{\pgfqpoint{3.927058in}{2.125723in}}%
\pgfpathcurveto{\pgfqpoint{3.932882in}{2.131547in}}{\pgfqpoint{3.936155in}{2.139447in}}{\pgfqpoint{3.936155in}{2.147683in}}%
\pgfpathcurveto{\pgfqpoint{3.936155in}{2.155919in}}{\pgfqpoint{3.932882in}{2.163819in}}{\pgfqpoint{3.927058in}{2.169643in}}%
\pgfpathcurveto{\pgfqpoint{3.921234in}{2.175467in}}{\pgfqpoint{3.913334in}{2.178740in}}{\pgfqpoint{3.905098in}{2.178740in}}%
\pgfpathcurveto{\pgfqpoint{3.896862in}{2.178740in}}{\pgfqpoint{3.888962in}{2.175467in}}{\pgfqpoint{3.883138in}{2.169643in}}%
\pgfpathcurveto{\pgfqpoint{3.877314in}{2.163819in}}{\pgfqpoint{3.874042in}{2.155919in}}{\pgfqpoint{3.874042in}{2.147683in}}%
\pgfpathcurveto{\pgfqpoint{3.874042in}{2.139447in}}{\pgfqpoint{3.877314in}{2.131547in}}{\pgfqpoint{3.883138in}{2.125723in}}%
\pgfpathcurveto{\pgfqpoint{3.888962in}{2.119899in}}{\pgfqpoint{3.896862in}{2.116627in}}{\pgfqpoint{3.905098in}{2.116627in}}%
\pgfpathclose%
\pgfusepath{stroke,fill}%
\end{pgfscope}%
\begin{pgfscope}%
\pgfpathrectangle{\pgfqpoint{3.793912in}{0.557870in}}{\pgfqpoint{2.446088in}{1.684734in}}%
\pgfusepath{clip}%
\pgfsetbuttcap%
\pgfsetroundjoin%
\definecolor{currentfill}{rgb}{0.298039,0.447059,0.690196}%
\pgfsetfillcolor{currentfill}%
\pgfsetlinewidth{1.003750pt}%
\definecolor{currentstroke}{rgb}{0.298039,0.447059,0.690196}%
\pgfsetstrokecolor{currentstroke}%
\pgfsetdash{}{0pt}%
\pgfpathmoveto{\pgfqpoint{3.905098in}{1.777295in}}%
\pgfpathcurveto{\pgfqpoint{3.913334in}{1.777295in}}{\pgfqpoint{3.921234in}{1.780568in}}{\pgfqpoint{3.927058in}{1.786392in}}%
\pgfpathcurveto{\pgfqpoint{3.932882in}{1.792216in}}{\pgfqpoint{3.936155in}{1.800116in}}{\pgfqpoint{3.936155in}{1.808352in}}%
\pgfpathcurveto{\pgfqpoint{3.936155in}{1.816588in}}{\pgfqpoint{3.932882in}{1.824488in}}{\pgfqpoint{3.927058in}{1.830312in}}%
\pgfpathcurveto{\pgfqpoint{3.921234in}{1.836136in}}{\pgfqpoint{3.913334in}{1.839408in}}{\pgfqpoint{3.905098in}{1.839408in}}%
\pgfpathcurveto{\pgfqpoint{3.896862in}{1.839408in}}{\pgfqpoint{3.888962in}{1.836136in}}{\pgfqpoint{3.883138in}{1.830312in}}%
\pgfpathcurveto{\pgfqpoint{3.877314in}{1.824488in}}{\pgfqpoint{3.874042in}{1.816588in}}{\pgfqpoint{3.874042in}{1.808352in}}%
\pgfpathcurveto{\pgfqpoint{3.874042in}{1.800116in}}{\pgfqpoint{3.877314in}{1.792216in}}{\pgfqpoint{3.883138in}{1.786392in}}%
\pgfpathcurveto{\pgfqpoint{3.888962in}{1.780568in}}{\pgfqpoint{3.896862in}{1.777295in}}{\pgfqpoint{3.905098in}{1.777295in}}%
\pgfpathclose%
\pgfusepath{stroke,fill}%
\end{pgfscope}%
\begin{pgfscope}%
\pgfpathrectangle{\pgfqpoint{3.793912in}{0.557870in}}{\pgfqpoint{2.446088in}{1.684734in}}%
\pgfusepath{clip}%
\pgfsetbuttcap%
\pgfsetroundjoin%
\definecolor{currentfill}{rgb}{0.298039,0.447059,0.690196}%
\pgfsetfillcolor{currentfill}%
\pgfsetlinewidth{1.003750pt}%
\definecolor{currentstroke}{rgb}{0.298039,0.447059,0.690196}%
\pgfsetstrokecolor{currentstroke}%
\pgfsetdash{}{0pt}%
\pgfpathmoveto{\pgfqpoint{3.905098in}{2.125798in}}%
\pgfpathcurveto{\pgfqpoint{3.913334in}{2.125798in}}{\pgfqpoint{3.921234in}{2.129070in}}{\pgfqpoint{3.927058in}{2.134894in}}%
\pgfpathcurveto{\pgfqpoint{3.932882in}{2.140718in}}{\pgfqpoint{3.936155in}{2.148618in}}{\pgfqpoint{3.936155in}{2.156854in}}%
\pgfpathcurveto{\pgfqpoint{3.936155in}{2.165091in}}{\pgfqpoint{3.932882in}{2.172991in}}{\pgfqpoint{3.927058in}{2.178814in}}%
\pgfpathcurveto{\pgfqpoint{3.921234in}{2.184638in}}{\pgfqpoint{3.913334in}{2.187911in}}{\pgfqpoint{3.905098in}{2.187911in}}%
\pgfpathcurveto{\pgfqpoint{3.896862in}{2.187911in}}{\pgfqpoint{3.888962in}{2.184638in}}{\pgfqpoint{3.883138in}{2.178814in}}%
\pgfpathcurveto{\pgfqpoint{3.877314in}{2.172991in}}{\pgfqpoint{3.874042in}{2.165091in}}{\pgfqpoint{3.874042in}{2.156854in}}%
\pgfpathcurveto{\pgfqpoint{3.874042in}{2.148618in}}{\pgfqpoint{3.877314in}{2.140718in}}{\pgfqpoint{3.883138in}{2.134894in}}%
\pgfpathcurveto{\pgfqpoint{3.888962in}{2.129070in}}{\pgfqpoint{3.896862in}{2.125798in}}{\pgfqpoint{3.905098in}{2.125798in}}%
\pgfpathclose%
\pgfusepath{stroke,fill}%
\end{pgfscope}%
\begin{pgfscope}%
\pgfpathrectangle{\pgfqpoint{3.793912in}{0.557870in}}{\pgfqpoint{2.446088in}{1.684734in}}%
\pgfusepath{clip}%
\pgfsetbuttcap%
\pgfsetroundjoin%
\definecolor{currentfill}{rgb}{0.298039,0.447059,0.690196}%
\pgfsetfillcolor{currentfill}%
\pgfsetlinewidth{1.003750pt}%
\definecolor{currentstroke}{rgb}{0.298039,0.447059,0.690196}%
\pgfsetstrokecolor{currentstroke}%
\pgfsetdash{}{0pt}%
\pgfpathmoveto{\pgfqpoint{3.905098in}{2.125798in}}%
\pgfpathcurveto{\pgfqpoint{3.913334in}{2.125798in}}{\pgfqpoint{3.921234in}{2.129070in}}{\pgfqpoint{3.927058in}{2.134894in}}%
\pgfpathcurveto{\pgfqpoint{3.932882in}{2.140718in}}{\pgfqpoint{3.936155in}{2.148618in}}{\pgfqpoint{3.936155in}{2.156854in}}%
\pgfpathcurveto{\pgfqpoint{3.936155in}{2.165091in}}{\pgfqpoint{3.932882in}{2.172991in}}{\pgfqpoint{3.927058in}{2.178814in}}%
\pgfpathcurveto{\pgfqpoint{3.921234in}{2.184638in}}{\pgfqpoint{3.913334in}{2.187911in}}{\pgfqpoint{3.905098in}{2.187911in}}%
\pgfpathcurveto{\pgfqpoint{3.896862in}{2.187911in}}{\pgfqpoint{3.888962in}{2.184638in}}{\pgfqpoint{3.883138in}{2.178814in}}%
\pgfpathcurveto{\pgfqpoint{3.877314in}{2.172991in}}{\pgfqpoint{3.874042in}{2.165091in}}{\pgfqpoint{3.874042in}{2.156854in}}%
\pgfpathcurveto{\pgfqpoint{3.874042in}{2.148618in}}{\pgfqpoint{3.877314in}{2.140718in}}{\pgfqpoint{3.883138in}{2.134894in}}%
\pgfpathcurveto{\pgfqpoint{3.888962in}{2.129070in}}{\pgfqpoint{3.896862in}{2.125798in}}{\pgfqpoint{3.905098in}{2.125798in}}%
\pgfpathclose%
\pgfusepath{stroke,fill}%
\end{pgfscope}%
\begin{pgfscope}%
\pgfpathrectangle{\pgfqpoint{3.793912in}{0.557870in}}{\pgfqpoint{2.446088in}{1.684734in}}%
\pgfusepath{clip}%
\pgfsetbuttcap%
\pgfsetroundjoin%
\definecolor{currentfill}{rgb}{0.298039,0.447059,0.690196}%
\pgfsetfillcolor{currentfill}%
\pgfsetlinewidth{1.003750pt}%
\definecolor{currentstroke}{rgb}{0.298039,0.447059,0.690196}%
\pgfsetstrokecolor{currentstroke}%
\pgfsetdash{}{0pt}%
\pgfpathmoveto{\pgfqpoint{3.905098in}{2.125798in}}%
\pgfpathcurveto{\pgfqpoint{3.913334in}{2.125798in}}{\pgfqpoint{3.921234in}{2.129070in}}{\pgfqpoint{3.927058in}{2.134894in}}%
\pgfpathcurveto{\pgfqpoint{3.932882in}{2.140718in}}{\pgfqpoint{3.936155in}{2.148618in}}{\pgfqpoint{3.936155in}{2.156854in}}%
\pgfpathcurveto{\pgfqpoint{3.936155in}{2.165091in}}{\pgfqpoint{3.932882in}{2.172991in}}{\pgfqpoint{3.927058in}{2.178814in}}%
\pgfpathcurveto{\pgfqpoint{3.921234in}{2.184638in}}{\pgfqpoint{3.913334in}{2.187911in}}{\pgfqpoint{3.905098in}{2.187911in}}%
\pgfpathcurveto{\pgfqpoint{3.896862in}{2.187911in}}{\pgfqpoint{3.888962in}{2.184638in}}{\pgfqpoint{3.883138in}{2.178814in}}%
\pgfpathcurveto{\pgfqpoint{3.877314in}{2.172991in}}{\pgfqpoint{3.874042in}{2.165091in}}{\pgfqpoint{3.874042in}{2.156854in}}%
\pgfpathcurveto{\pgfqpoint{3.874042in}{2.148618in}}{\pgfqpoint{3.877314in}{2.140718in}}{\pgfqpoint{3.883138in}{2.134894in}}%
\pgfpathcurveto{\pgfqpoint{3.888962in}{2.129070in}}{\pgfqpoint{3.896862in}{2.125798in}}{\pgfqpoint{3.905098in}{2.125798in}}%
\pgfpathclose%
\pgfusepath{stroke,fill}%
\end{pgfscope}%
\begin{pgfscope}%
\pgfpathrectangle{\pgfqpoint{3.793912in}{0.557870in}}{\pgfqpoint{2.446088in}{1.684734in}}%
\pgfusepath{clip}%
\pgfsetbuttcap%
\pgfsetroundjoin%
\definecolor{currentfill}{rgb}{0.298039,0.447059,0.690196}%
\pgfsetfillcolor{currentfill}%
\pgfsetlinewidth{1.003750pt}%
\definecolor{currentstroke}{rgb}{0.298039,0.447059,0.690196}%
\pgfsetstrokecolor{currentstroke}%
\pgfsetdash{}{0pt}%
\pgfpathmoveto{\pgfqpoint{3.905098in}{2.125798in}}%
\pgfpathcurveto{\pgfqpoint{3.913334in}{2.125798in}}{\pgfqpoint{3.921234in}{2.129070in}}{\pgfqpoint{3.927058in}{2.134894in}}%
\pgfpathcurveto{\pgfqpoint{3.932882in}{2.140718in}}{\pgfqpoint{3.936155in}{2.148618in}}{\pgfqpoint{3.936155in}{2.156854in}}%
\pgfpathcurveto{\pgfqpoint{3.936155in}{2.165091in}}{\pgfqpoint{3.932882in}{2.172991in}}{\pgfqpoint{3.927058in}{2.178814in}}%
\pgfpathcurveto{\pgfqpoint{3.921234in}{2.184638in}}{\pgfqpoint{3.913334in}{2.187911in}}{\pgfqpoint{3.905098in}{2.187911in}}%
\pgfpathcurveto{\pgfqpoint{3.896862in}{2.187911in}}{\pgfqpoint{3.888962in}{2.184638in}}{\pgfqpoint{3.883138in}{2.178814in}}%
\pgfpathcurveto{\pgfqpoint{3.877314in}{2.172991in}}{\pgfqpoint{3.874042in}{2.165091in}}{\pgfqpoint{3.874042in}{2.156854in}}%
\pgfpathcurveto{\pgfqpoint{3.874042in}{2.148618in}}{\pgfqpoint{3.877314in}{2.140718in}}{\pgfqpoint{3.883138in}{2.134894in}}%
\pgfpathcurveto{\pgfqpoint{3.888962in}{2.129070in}}{\pgfqpoint{3.896862in}{2.125798in}}{\pgfqpoint{3.905098in}{2.125798in}}%
\pgfpathclose%
\pgfusepath{stroke,fill}%
\end{pgfscope}%
\begin{pgfscope}%
\pgfpathrectangle{\pgfqpoint{3.793912in}{0.557870in}}{\pgfqpoint{2.446088in}{1.684734in}}%
\pgfusepath{clip}%
\pgfsetbuttcap%
\pgfsetroundjoin%
\definecolor{currentfill}{rgb}{0.298039,0.447059,0.690196}%
\pgfsetfillcolor{currentfill}%
\pgfsetlinewidth{1.003750pt}%
\definecolor{currentstroke}{rgb}{0.298039,0.447059,0.690196}%
\pgfsetstrokecolor{currentstroke}%
\pgfsetdash{}{0pt}%
\pgfpathmoveto{\pgfqpoint{4.671897in}{1.428793in}}%
\pgfpathcurveto{\pgfqpoint{4.680133in}{1.428793in}}{\pgfqpoint{4.688033in}{1.432065in}}{\pgfqpoint{4.693857in}{1.437889in}}%
\pgfpathcurveto{\pgfqpoint{4.699681in}{1.443713in}}{\pgfqpoint{4.702953in}{1.451613in}}{\pgfqpoint{4.702953in}{1.459849in}}%
\pgfpathcurveto{\pgfqpoint{4.702953in}{1.468086in}}{\pgfqpoint{4.699681in}{1.475986in}}{\pgfqpoint{4.693857in}{1.481810in}}%
\pgfpathcurveto{\pgfqpoint{4.688033in}{1.487634in}}{\pgfqpoint{4.680133in}{1.490906in}}{\pgfqpoint{4.671897in}{1.490906in}}%
\pgfpathcurveto{\pgfqpoint{4.663660in}{1.490906in}}{\pgfqpoint{4.655760in}{1.487634in}}{\pgfqpoint{4.649936in}{1.481810in}}%
\pgfpathcurveto{\pgfqpoint{4.644113in}{1.475986in}}{\pgfqpoint{4.640840in}{1.468086in}}{\pgfqpoint{4.640840in}{1.459849in}}%
\pgfpathcurveto{\pgfqpoint{4.640840in}{1.451613in}}{\pgfqpoint{4.644113in}{1.443713in}}{\pgfqpoint{4.649936in}{1.437889in}}%
\pgfpathcurveto{\pgfqpoint{4.655760in}{1.432065in}}{\pgfqpoint{4.663660in}{1.428793in}}{\pgfqpoint{4.671897in}{1.428793in}}%
\pgfpathclose%
\pgfusepath{stroke,fill}%
\end{pgfscope}%
\begin{pgfscope}%
\pgfpathrectangle{\pgfqpoint{3.793912in}{0.557870in}}{\pgfqpoint{2.446088in}{1.684734in}}%
\pgfusepath{clip}%
\pgfsetbuttcap%
\pgfsetroundjoin%
\definecolor{currentfill}{rgb}{0.298039,0.447059,0.690196}%
\pgfsetfillcolor{currentfill}%
\pgfsetlinewidth{1.003750pt}%
\definecolor{currentstroke}{rgb}{0.298039,0.447059,0.690196}%
\pgfsetstrokecolor{currentstroke}%
\pgfsetdash{}{0pt}%
\pgfpathmoveto{\pgfqpoint{3.905098in}{2.116627in}}%
\pgfpathcurveto{\pgfqpoint{3.913334in}{2.116627in}}{\pgfqpoint{3.921234in}{2.119899in}}{\pgfqpoint{3.927058in}{2.125723in}}%
\pgfpathcurveto{\pgfqpoint{3.932882in}{2.131547in}}{\pgfqpoint{3.936155in}{2.139447in}}{\pgfqpoint{3.936155in}{2.147683in}}%
\pgfpathcurveto{\pgfqpoint{3.936155in}{2.155919in}}{\pgfqpoint{3.932882in}{2.163819in}}{\pgfqpoint{3.927058in}{2.169643in}}%
\pgfpathcurveto{\pgfqpoint{3.921234in}{2.175467in}}{\pgfqpoint{3.913334in}{2.178740in}}{\pgfqpoint{3.905098in}{2.178740in}}%
\pgfpathcurveto{\pgfqpoint{3.896862in}{2.178740in}}{\pgfqpoint{3.888962in}{2.175467in}}{\pgfqpoint{3.883138in}{2.169643in}}%
\pgfpathcurveto{\pgfqpoint{3.877314in}{2.163819in}}{\pgfqpoint{3.874042in}{2.155919in}}{\pgfqpoint{3.874042in}{2.147683in}}%
\pgfpathcurveto{\pgfqpoint{3.874042in}{2.139447in}}{\pgfqpoint{3.877314in}{2.131547in}}{\pgfqpoint{3.883138in}{2.125723in}}%
\pgfpathcurveto{\pgfqpoint{3.888962in}{2.119899in}}{\pgfqpoint{3.896862in}{2.116627in}}{\pgfqpoint{3.905098in}{2.116627in}}%
\pgfpathclose%
\pgfusepath{stroke,fill}%
\end{pgfscope}%
\begin{pgfscope}%
\pgfpathrectangle{\pgfqpoint{3.793912in}{0.557870in}}{\pgfqpoint{2.446088in}{1.684734in}}%
\pgfusepath{clip}%
\pgfsetbuttcap%
\pgfsetroundjoin%
\definecolor{currentfill}{rgb}{0.298039,0.447059,0.690196}%
\pgfsetfillcolor{currentfill}%
\pgfsetlinewidth{1.003750pt}%
\definecolor{currentstroke}{rgb}{0.298039,0.447059,0.690196}%
\pgfsetstrokecolor{currentstroke}%
\pgfsetdash{}{0pt}%
\pgfpathmoveto{\pgfqpoint{5.055296in}{1.456306in}}%
\pgfpathcurveto{\pgfqpoint{5.063532in}{1.456306in}}{\pgfqpoint{5.071432in}{1.459579in}}{\pgfqpoint{5.077256in}{1.465403in}}%
\pgfpathcurveto{\pgfqpoint{5.083080in}{1.471226in}}{\pgfqpoint{5.086353in}{1.479127in}}{\pgfqpoint{5.086353in}{1.487363in}}%
\pgfpathcurveto{\pgfqpoint{5.086353in}{1.495599in}}{\pgfqpoint{5.083080in}{1.503499in}}{\pgfqpoint{5.077256in}{1.509323in}}%
\pgfpathcurveto{\pgfqpoint{5.071432in}{1.515147in}}{\pgfqpoint{5.063532in}{1.518419in}}{\pgfqpoint{5.055296in}{1.518419in}}%
\pgfpathcurveto{\pgfqpoint{5.047060in}{1.518419in}}{\pgfqpoint{5.039160in}{1.515147in}}{\pgfqpoint{5.033336in}{1.509323in}}%
\pgfpathcurveto{\pgfqpoint{5.027512in}{1.503499in}}{\pgfqpoint{5.024240in}{1.495599in}}{\pgfqpoint{5.024240in}{1.487363in}}%
\pgfpathcurveto{\pgfqpoint{5.024240in}{1.479127in}}{\pgfqpoint{5.027512in}{1.471226in}}{\pgfqpoint{5.033336in}{1.465403in}}%
\pgfpathcurveto{\pgfqpoint{5.039160in}{1.459579in}}{\pgfqpoint{5.047060in}{1.456306in}}{\pgfqpoint{5.055296in}{1.456306in}}%
\pgfpathclose%
\pgfusepath{stroke,fill}%
\end{pgfscope}%
\begin{pgfscope}%
\pgfpathrectangle{\pgfqpoint{3.793912in}{0.557870in}}{\pgfqpoint{2.446088in}{1.684734in}}%
\pgfusepath{clip}%
\pgfsetbuttcap%
\pgfsetroundjoin%
\definecolor{currentfill}{rgb}{0.298039,0.447059,0.690196}%
\pgfsetfillcolor{currentfill}%
\pgfsetlinewidth{1.003750pt}%
\definecolor{currentstroke}{rgb}{0.298039,0.447059,0.690196}%
\pgfsetstrokecolor{currentstroke}%
\pgfsetdash{}{0pt}%
\pgfpathmoveto{\pgfqpoint{3.905098in}{2.125798in}}%
\pgfpathcurveto{\pgfqpoint{3.913334in}{2.125798in}}{\pgfqpoint{3.921234in}{2.129070in}}{\pgfqpoint{3.927058in}{2.134894in}}%
\pgfpathcurveto{\pgfqpoint{3.932882in}{2.140718in}}{\pgfqpoint{3.936155in}{2.148618in}}{\pgfqpoint{3.936155in}{2.156854in}}%
\pgfpathcurveto{\pgfqpoint{3.936155in}{2.165091in}}{\pgfqpoint{3.932882in}{2.172991in}}{\pgfqpoint{3.927058in}{2.178814in}}%
\pgfpathcurveto{\pgfqpoint{3.921234in}{2.184638in}}{\pgfqpoint{3.913334in}{2.187911in}}{\pgfqpoint{3.905098in}{2.187911in}}%
\pgfpathcurveto{\pgfqpoint{3.896862in}{2.187911in}}{\pgfqpoint{3.888962in}{2.184638in}}{\pgfqpoint{3.883138in}{2.178814in}}%
\pgfpathcurveto{\pgfqpoint{3.877314in}{2.172991in}}{\pgfqpoint{3.874042in}{2.165091in}}{\pgfqpoint{3.874042in}{2.156854in}}%
\pgfpathcurveto{\pgfqpoint{3.874042in}{2.148618in}}{\pgfqpoint{3.877314in}{2.140718in}}{\pgfqpoint{3.883138in}{2.134894in}}%
\pgfpathcurveto{\pgfqpoint{3.888962in}{2.129070in}}{\pgfqpoint{3.896862in}{2.125798in}}{\pgfqpoint{3.905098in}{2.125798in}}%
\pgfpathclose%
\pgfusepath{stroke,fill}%
\end{pgfscope}%
\begin{pgfscope}%
\pgfpathrectangle{\pgfqpoint{3.793912in}{0.557870in}}{\pgfqpoint{2.446088in}{1.684734in}}%
\pgfusepath{clip}%
\pgfsetbuttcap%
\pgfsetroundjoin%
\definecolor{currentfill}{rgb}{0.298039,0.447059,0.690196}%
\pgfsetfillcolor{currentfill}%
\pgfsetlinewidth{1.003750pt}%
\definecolor{currentstroke}{rgb}{0.298039,0.447059,0.690196}%
\pgfsetstrokecolor{currentstroke}%
\pgfsetdash{}{0pt}%
\pgfpathmoveto{\pgfqpoint{3.905098in}{2.116627in}}%
\pgfpathcurveto{\pgfqpoint{3.913334in}{2.116627in}}{\pgfqpoint{3.921234in}{2.119899in}}{\pgfqpoint{3.927058in}{2.125723in}}%
\pgfpathcurveto{\pgfqpoint{3.932882in}{2.131547in}}{\pgfqpoint{3.936155in}{2.139447in}}{\pgfqpoint{3.936155in}{2.147683in}}%
\pgfpathcurveto{\pgfqpoint{3.936155in}{2.155919in}}{\pgfqpoint{3.932882in}{2.163819in}}{\pgfqpoint{3.927058in}{2.169643in}}%
\pgfpathcurveto{\pgfqpoint{3.921234in}{2.175467in}}{\pgfqpoint{3.913334in}{2.178740in}}{\pgfqpoint{3.905098in}{2.178740in}}%
\pgfpathcurveto{\pgfqpoint{3.896862in}{2.178740in}}{\pgfqpoint{3.888962in}{2.175467in}}{\pgfqpoint{3.883138in}{2.169643in}}%
\pgfpathcurveto{\pgfqpoint{3.877314in}{2.163819in}}{\pgfqpoint{3.874042in}{2.155919in}}{\pgfqpoint{3.874042in}{2.147683in}}%
\pgfpathcurveto{\pgfqpoint{3.874042in}{2.139447in}}{\pgfqpoint{3.877314in}{2.131547in}}{\pgfqpoint{3.883138in}{2.125723in}}%
\pgfpathcurveto{\pgfqpoint{3.888962in}{2.119899in}}{\pgfqpoint{3.896862in}{2.116627in}}{\pgfqpoint{3.905098in}{2.116627in}}%
\pgfpathclose%
\pgfusepath{stroke,fill}%
\end{pgfscope}%
\begin{pgfscope}%
\pgfpathrectangle{\pgfqpoint{3.793912in}{0.557870in}}{\pgfqpoint{2.446088in}{1.684734in}}%
\pgfusepath{clip}%
\pgfsetbuttcap%
\pgfsetroundjoin%
\definecolor{currentfill}{rgb}{0.298039,0.447059,0.690196}%
\pgfsetfillcolor{currentfill}%
\pgfsetlinewidth{1.003750pt}%
\definecolor{currentstroke}{rgb}{0.298039,0.447059,0.690196}%
\pgfsetstrokecolor{currentstroke}%
\pgfsetdash{}{0pt}%
\pgfpathmoveto{\pgfqpoint{3.905098in}{2.125798in}}%
\pgfpathcurveto{\pgfqpoint{3.913334in}{2.125798in}}{\pgfqpoint{3.921234in}{2.129070in}}{\pgfqpoint{3.927058in}{2.134894in}}%
\pgfpathcurveto{\pgfqpoint{3.932882in}{2.140718in}}{\pgfqpoint{3.936155in}{2.148618in}}{\pgfqpoint{3.936155in}{2.156854in}}%
\pgfpathcurveto{\pgfqpoint{3.936155in}{2.165091in}}{\pgfqpoint{3.932882in}{2.172991in}}{\pgfqpoint{3.927058in}{2.178814in}}%
\pgfpathcurveto{\pgfqpoint{3.921234in}{2.184638in}}{\pgfqpoint{3.913334in}{2.187911in}}{\pgfqpoint{3.905098in}{2.187911in}}%
\pgfpathcurveto{\pgfqpoint{3.896862in}{2.187911in}}{\pgfqpoint{3.888962in}{2.184638in}}{\pgfqpoint{3.883138in}{2.178814in}}%
\pgfpathcurveto{\pgfqpoint{3.877314in}{2.172991in}}{\pgfqpoint{3.874042in}{2.165091in}}{\pgfqpoint{3.874042in}{2.156854in}}%
\pgfpathcurveto{\pgfqpoint{3.874042in}{2.148618in}}{\pgfqpoint{3.877314in}{2.140718in}}{\pgfqpoint{3.883138in}{2.134894in}}%
\pgfpathcurveto{\pgfqpoint{3.888962in}{2.129070in}}{\pgfqpoint{3.896862in}{2.125798in}}{\pgfqpoint{3.905098in}{2.125798in}}%
\pgfpathclose%
\pgfusepath{stroke,fill}%
\end{pgfscope}%
\begin{pgfscope}%
\pgfpathrectangle{\pgfqpoint{3.793912in}{0.557870in}}{\pgfqpoint{2.446088in}{1.684734in}}%
\pgfusepath{clip}%
\pgfsetbuttcap%
\pgfsetroundjoin%
\definecolor{currentfill}{rgb}{0.298039,0.447059,0.690196}%
\pgfsetfillcolor{currentfill}%
\pgfsetlinewidth{1.003750pt}%
\definecolor{currentstroke}{rgb}{0.298039,0.447059,0.690196}%
\pgfsetstrokecolor{currentstroke}%
\pgfsetdash{}{0pt}%
\pgfpathmoveto{\pgfqpoint{5.975454in}{1.263713in}}%
\pgfpathcurveto{\pgfqpoint{5.983691in}{1.263713in}}{\pgfqpoint{5.991591in}{1.266985in}}{\pgfqpoint{5.997415in}{1.272809in}}%
\pgfpathcurveto{\pgfqpoint{6.003239in}{1.278633in}}{\pgfqpoint{6.006511in}{1.286533in}}{\pgfqpoint{6.006511in}{1.294769in}}%
\pgfpathcurveto{\pgfqpoint{6.006511in}{1.303006in}}{\pgfqpoint{6.003239in}{1.310906in}}{\pgfqpoint{5.997415in}{1.316730in}}%
\pgfpathcurveto{\pgfqpoint{5.991591in}{1.322554in}}{\pgfqpoint{5.983691in}{1.325826in}}{\pgfqpoint{5.975454in}{1.325826in}}%
\pgfpathcurveto{\pgfqpoint{5.967218in}{1.325826in}}{\pgfqpoint{5.959318in}{1.322554in}}{\pgfqpoint{5.953494in}{1.316730in}}%
\pgfpathcurveto{\pgfqpoint{5.947670in}{1.310906in}}{\pgfqpoint{5.944398in}{1.303006in}}{\pgfqpoint{5.944398in}{1.294769in}}%
\pgfpathcurveto{\pgfqpoint{5.944398in}{1.286533in}}{\pgfqpoint{5.947670in}{1.278633in}}{\pgfqpoint{5.953494in}{1.272809in}}%
\pgfpathcurveto{\pgfqpoint{5.959318in}{1.266985in}}{\pgfqpoint{5.967218in}{1.263713in}}{\pgfqpoint{5.975454in}{1.263713in}}%
\pgfpathclose%
\pgfusepath{stroke,fill}%
\end{pgfscope}%
\begin{pgfscope}%
\pgfpathrectangle{\pgfqpoint{3.793912in}{0.557870in}}{\pgfqpoint{2.446088in}{1.684734in}}%
\pgfusepath{clip}%
\pgfsetbuttcap%
\pgfsetroundjoin%
\definecolor{currentfill}{rgb}{0.298039,0.447059,0.690196}%
\pgfsetfillcolor{currentfill}%
\pgfsetlinewidth{1.003750pt}%
\definecolor{currentstroke}{rgb}{0.298039,0.447059,0.690196}%
\pgfsetstrokecolor{currentstroke}%
\pgfsetdash{}{0pt}%
\pgfpathmoveto{\pgfqpoint{5.975454in}{1.346253in}}%
\pgfpathcurveto{\pgfqpoint{5.983691in}{1.346253in}}{\pgfqpoint{5.991591in}{1.349525in}}{\pgfqpoint{5.997415in}{1.355349in}}%
\pgfpathcurveto{\pgfqpoint{6.003239in}{1.361173in}}{\pgfqpoint{6.006511in}{1.369073in}}{\pgfqpoint{6.006511in}{1.377309in}}%
\pgfpathcurveto{\pgfqpoint{6.006511in}{1.385546in}}{\pgfqpoint{6.003239in}{1.393446in}}{\pgfqpoint{5.997415in}{1.399270in}}%
\pgfpathcurveto{\pgfqpoint{5.991591in}{1.405094in}}{\pgfqpoint{5.983691in}{1.408366in}}{\pgfqpoint{5.975454in}{1.408366in}}%
\pgfpathcurveto{\pgfqpoint{5.967218in}{1.408366in}}{\pgfqpoint{5.959318in}{1.405094in}}{\pgfqpoint{5.953494in}{1.399270in}}%
\pgfpathcurveto{\pgfqpoint{5.947670in}{1.393446in}}{\pgfqpoint{5.944398in}{1.385546in}}{\pgfqpoint{5.944398in}{1.377309in}}%
\pgfpathcurveto{\pgfqpoint{5.944398in}{1.369073in}}{\pgfqpoint{5.947670in}{1.361173in}}{\pgfqpoint{5.953494in}{1.355349in}}%
\pgfpathcurveto{\pgfqpoint{5.959318in}{1.349525in}}{\pgfqpoint{5.967218in}{1.346253in}}{\pgfqpoint{5.975454in}{1.346253in}}%
\pgfpathclose%
\pgfusepath{stroke,fill}%
\end{pgfscope}%
\begin{pgfscope}%
\pgfpathrectangle{\pgfqpoint{3.793912in}{0.557870in}}{\pgfqpoint{2.446088in}{1.684734in}}%
\pgfusepath{clip}%
\pgfsetbuttcap%
\pgfsetroundjoin%
\definecolor{currentfill}{rgb}{0.298039,0.447059,0.690196}%
\pgfsetfillcolor{currentfill}%
\pgfsetlinewidth{1.003750pt}%
\definecolor{currentstroke}{rgb}{0.298039,0.447059,0.690196}%
\pgfsetstrokecolor{currentstroke}%
\pgfsetdash{}{0pt}%
\pgfpathmoveto{\pgfqpoint{5.975454in}{1.282055in}}%
\pgfpathcurveto{\pgfqpoint{5.983691in}{1.282055in}}{\pgfqpoint{5.991591in}{1.285327in}}{\pgfqpoint{5.997415in}{1.291151in}}%
\pgfpathcurveto{\pgfqpoint{6.003239in}{1.296975in}}{\pgfqpoint{6.006511in}{1.304875in}}{\pgfqpoint{6.006511in}{1.313112in}}%
\pgfpathcurveto{\pgfqpoint{6.006511in}{1.321348in}}{\pgfqpoint{6.003239in}{1.329248in}}{\pgfqpoint{5.997415in}{1.335072in}}%
\pgfpathcurveto{\pgfqpoint{5.991591in}{1.340896in}}{\pgfqpoint{5.983691in}{1.344168in}}{\pgfqpoint{5.975454in}{1.344168in}}%
\pgfpathcurveto{\pgfqpoint{5.967218in}{1.344168in}}{\pgfqpoint{5.959318in}{1.340896in}}{\pgfqpoint{5.953494in}{1.335072in}}%
\pgfpathcurveto{\pgfqpoint{5.947670in}{1.329248in}}{\pgfqpoint{5.944398in}{1.321348in}}{\pgfqpoint{5.944398in}{1.313112in}}%
\pgfpathcurveto{\pgfqpoint{5.944398in}{1.304875in}}{\pgfqpoint{5.947670in}{1.296975in}}{\pgfqpoint{5.953494in}{1.291151in}}%
\pgfpathcurveto{\pgfqpoint{5.959318in}{1.285327in}}{\pgfqpoint{5.967218in}{1.282055in}}{\pgfqpoint{5.975454in}{1.282055in}}%
\pgfpathclose%
\pgfusepath{stroke,fill}%
\end{pgfscope}%
\begin{pgfscope}%
\pgfpathrectangle{\pgfqpoint{3.793912in}{0.557870in}}{\pgfqpoint{2.446088in}{1.684734in}}%
\pgfusepath{clip}%
\pgfsetbuttcap%
\pgfsetroundjoin%
\definecolor{currentfill}{rgb}{0.298039,0.447059,0.690196}%
\pgfsetfillcolor{currentfill}%
\pgfsetlinewidth{1.003750pt}%
\definecolor{currentstroke}{rgb}{0.298039,0.447059,0.690196}%
\pgfsetstrokecolor{currentstroke}%
\pgfsetdash{}{0pt}%
\pgfpathmoveto{\pgfqpoint{5.975454in}{1.364595in}}%
\pgfpathcurveto{\pgfqpoint{5.983691in}{1.364595in}}{\pgfqpoint{5.991591in}{1.367867in}}{\pgfqpoint{5.997415in}{1.373691in}}%
\pgfpathcurveto{\pgfqpoint{6.003239in}{1.379515in}}{\pgfqpoint{6.006511in}{1.387415in}}{\pgfqpoint{6.006511in}{1.395652in}}%
\pgfpathcurveto{\pgfqpoint{6.006511in}{1.403888in}}{\pgfqpoint{6.003239in}{1.411788in}}{\pgfqpoint{5.997415in}{1.417612in}}%
\pgfpathcurveto{\pgfqpoint{5.991591in}{1.423436in}}{\pgfqpoint{5.983691in}{1.426708in}}{\pgfqpoint{5.975454in}{1.426708in}}%
\pgfpathcurveto{\pgfqpoint{5.967218in}{1.426708in}}{\pgfqpoint{5.959318in}{1.423436in}}{\pgfqpoint{5.953494in}{1.417612in}}%
\pgfpathcurveto{\pgfqpoint{5.947670in}{1.411788in}}{\pgfqpoint{5.944398in}{1.403888in}}{\pgfqpoint{5.944398in}{1.395652in}}%
\pgfpathcurveto{\pgfqpoint{5.944398in}{1.387415in}}{\pgfqpoint{5.947670in}{1.379515in}}{\pgfqpoint{5.953494in}{1.373691in}}%
\pgfpathcurveto{\pgfqpoint{5.959318in}{1.367867in}}{\pgfqpoint{5.967218in}{1.364595in}}{\pgfqpoint{5.975454in}{1.364595in}}%
\pgfpathclose%
\pgfusepath{stroke,fill}%
\end{pgfscope}%
\begin{pgfscope}%
\pgfpathrectangle{\pgfqpoint{3.793912in}{0.557870in}}{\pgfqpoint{2.446088in}{1.684734in}}%
\pgfusepath{clip}%
\pgfsetbuttcap%
\pgfsetroundjoin%
\definecolor{currentfill}{rgb}{0.298039,0.447059,0.690196}%
\pgfsetfillcolor{currentfill}%
\pgfsetlinewidth{1.003750pt}%
\definecolor{currentstroke}{rgb}{0.298039,0.447059,0.690196}%
\pgfsetstrokecolor{currentstroke}%
\pgfsetdash{}{0pt}%
\pgfpathmoveto{\pgfqpoint{3.905098in}{2.125798in}}%
\pgfpathcurveto{\pgfqpoint{3.913334in}{2.125798in}}{\pgfqpoint{3.921234in}{2.129070in}}{\pgfqpoint{3.927058in}{2.134894in}}%
\pgfpathcurveto{\pgfqpoint{3.932882in}{2.140718in}}{\pgfqpoint{3.936155in}{2.148618in}}{\pgfqpoint{3.936155in}{2.156854in}}%
\pgfpathcurveto{\pgfqpoint{3.936155in}{2.165091in}}{\pgfqpoint{3.932882in}{2.172991in}}{\pgfqpoint{3.927058in}{2.178814in}}%
\pgfpathcurveto{\pgfqpoint{3.921234in}{2.184638in}}{\pgfqpoint{3.913334in}{2.187911in}}{\pgfqpoint{3.905098in}{2.187911in}}%
\pgfpathcurveto{\pgfqpoint{3.896862in}{2.187911in}}{\pgfqpoint{3.888962in}{2.184638in}}{\pgfqpoint{3.883138in}{2.178814in}}%
\pgfpathcurveto{\pgfqpoint{3.877314in}{2.172991in}}{\pgfqpoint{3.874042in}{2.165091in}}{\pgfqpoint{3.874042in}{2.156854in}}%
\pgfpathcurveto{\pgfqpoint{3.874042in}{2.148618in}}{\pgfqpoint{3.877314in}{2.140718in}}{\pgfqpoint{3.883138in}{2.134894in}}%
\pgfpathcurveto{\pgfqpoint{3.888962in}{2.129070in}}{\pgfqpoint{3.896862in}{2.125798in}}{\pgfqpoint{3.905098in}{2.125798in}}%
\pgfpathclose%
\pgfusepath{stroke,fill}%
\end{pgfscope}%
\begin{pgfscope}%
\pgfpathrectangle{\pgfqpoint{3.793912in}{0.557870in}}{\pgfqpoint{2.446088in}{1.684734in}}%
\pgfusepath{clip}%
\pgfsetbuttcap%
\pgfsetroundjoin%
\definecolor{currentfill}{rgb}{0.298039,0.447059,0.690196}%
\pgfsetfillcolor{currentfill}%
\pgfsetlinewidth{1.003750pt}%
\definecolor{currentstroke}{rgb}{0.298039,0.447059,0.690196}%
\pgfsetstrokecolor{currentstroke}%
\pgfsetdash{}{0pt}%
\pgfpathmoveto{\pgfqpoint{5.975454in}{1.337082in}}%
\pgfpathcurveto{\pgfqpoint{5.983691in}{1.337082in}}{\pgfqpoint{5.991591in}{1.340354in}}{\pgfqpoint{5.997415in}{1.346178in}}%
\pgfpathcurveto{\pgfqpoint{6.003239in}{1.352002in}}{\pgfqpoint{6.006511in}{1.359902in}}{\pgfqpoint{6.006511in}{1.368138in}}%
\pgfpathcurveto{\pgfqpoint{6.006511in}{1.376375in}}{\pgfqpoint{6.003239in}{1.384275in}}{\pgfqpoint{5.997415in}{1.390099in}}%
\pgfpathcurveto{\pgfqpoint{5.991591in}{1.395923in}}{\pgfqpoint{5.983691in}{1.399195in}}{\pgfqpoint{5.975454in}{1.399195in}}%
\pgfpathcurveto{\pgfqpoint{5.967218in}{1.399195in}}{\pgfqpoint{5.959318in}{1.395923in}}{\pgfqpoint{5.953494in}{1.390099in}}%
\pgfpathcurveto{\pgfqpoint{5.947670in}{1.384275in}}{\pgfqpoint{5.944398in}{1.376375in}}{\pgfqpoint{5.944398in}{1.368138in}}%
\pgfpathcurveto{\pgfqpoint{5.944398in}{1.359902in}}{\pgfqpoint{5.947670in}{1.352002in}}{\pgfqpoint{5.953494in}{1.346178in}}%
\pgfpathcurveto{\pgfqpoint{5.959318in}{1.340354in}}{\pgfqpoint{5.967218in}{1.337082in}}{\pgfqpoint{5.975454in}{1.337082in}}%
\pgfpathclose%
\pgfusepath{stroke,fill}%
\end{pgfscope}%
\begin{pgfscope}%
\pgfpathrectangle{\pgfqpoint{3.793912in}{0.557870in}}{\pgfqpoint{2.446088in}{1.684734in}}%
\pgfusepath{clip}%
\pgfsetbuttcap%
\pgfsetroundjoin%
\definecolor{currentfill}{rgb}{0.298039,0.447059,0.690196}%
\pgfsetfillcolor{currentfill}%
\pgfsetlinewidth{1.003750pt}%
\definecolor{currentstroke}{rgb}{0.298039,0.447059,0.690196}%
\pgfsetstrokecolor{currentstroke}%
\pgfsetdash{}{0pt}%
\pgfpathmoveto{\pgfqpoint{3.905098in}{2.125798in}}%
\pgfpathcurveto{\pgfqpoint{3.913334in}{2.125798in}}{\pgfqpoint{3.921234in}{2.129070in}}{\pgfqpoint{3.927058in}{2.134894in}}%
\pgfpathcurveto{\pgfqpoint{3.932882in}{2.140718in}}{\pgfqpoint{3.936155in}{2.148618in}}{\pgfqpoint{3.936155in}{2.156854in}}%
\pgfpathcurveto{\pgfqpoint{3.936155in}{2.165091in}}{\pgfqpoint{3.932882in}{2.172991in}}{\pgfqpoint{3.927058in}{2.178814in}}%
\pgfpathcurveto{\pgfqpoint{3.921234in}{2.184638in}}{\pgfqpoint{3.913334in}{2.187911in}}{\pgfqpoint{3.905098in}{2.187911in}}%
\pgfpathcurveto{\pgfqpoint{3.896862in}{2.187911in}}{\pgfqpoint{3.888962in}{2.184638in}}{\pgfqpoint{3.883138in}{2.178814in}}%
\pgfpathcurveto{\pgfqpoint{3.877314in}{2.172991in}}{\pgfqpoint{3.874042in}{2.165091in}}{\pgfqpoint{3.874042in}{2.156854in}}%
\pgfpathcurveto{\pgfqpoint{3.874042in}{2.148618in}}{\pgfqpoint{3.877314in}{2.140718in}}{\pgfqpoint{3.883138in}{2.134894in}}%
\pgfpathcurveto{\pgfqpoint{3.888962in}{2.129070in}}{\pgfqpoint{3.896862in}{2.125798in}}{\pgfqpoint{3.905098in}{2.125798in}}%
\pgfpathclose%
\pgfusepath{stroke,fill}%
\end{pgfscope}%
\begin{pgfscope}%
\pgfpathrectangle{\pgfqpoint{3.793912in}{0.557870in}}{\pgfqpoint{2.446088in}{1.684734in}}%
\pgfusepath{clip}%
\pgfsetbuttcap%
\pgfsetroundjoin%
\definecolor{currentfill}{rgb}{0.298039,0.447059,0.690196}%
\pgfsetfillcolor{currentfill}%
\pgfsetlinewidth{1.003750pt}%
\definecolor{currentstroke}{rgb}{0.298039,0.447059,0.690196}%
\pgfsetstrokecolor{currentstroke}%
\pgfsetdash{}{0pt}%
\pgfpathmoveto{\pgfqpoint{5.438695in}{1.933204in}}%
\pgfpathcurveto{\pgfqpoint{5.446932in}{1.933204in}}{\pgfqpoint{5.454832in}{1.936477in}}{\pgfqpoint{5.460656in}{1.942301in}}%
\pgfpathcurveto{\pgfqpoint{5.466480in}{1.948124in}}{\pgfqpoint{5.469752in}{1.956025in}}{\pgfqpoint{5.469752in}{1.964261in}}%
\pgfpathcurveto{\pgfqpoint{5.469752in}{1.972497in}}{\pgfqpoint{5.466480in}{1.980397in}}{\pgfqpoint{5.460656in}{1.986221in}}%
\pgfpathcurveto{\pgfqpoint{5.454832in}{1.992045in}}{\pgfqpoint{5.446932in}{1.995317in}}{\pgfqpoint{5.438695in}{1.995317in}}%
\pgfpathcurveto{\pgfqpoint{5.430459in}{1.995317in}}{\pgfqpoint{5.422559in}{1.992045in}}{\pgfqpoint{5.416735in}{1.986221in}}%
\pgfpathcurveto{\pgfqpoint{5.410911in}{1.980397in}}{\pgfqpoint{5.407639in}{1.972497in}}{\pgfqpoint{5.407639in}{1.964261in}}%
\pgfpathcurveto{\pgfqpoint{5.407639in}{1.956025in}}{\pgfqpoint{5.410911in}{1.948124in}}{\pgfqpoint{5.416735in}{1.942301in}}%
\pgfpathcurveto{\pgfqpoint{5.422559in}{1.936477in}}{\pgfqpoint{5.430459in}{1.933204in}}{\pgfqpoint{5.438695in}{1.933204in}}%
\pgfpathclose%
\pgfusepath{stroke,fill}%
\end{pgfscope}%
\begin{pgfscope}%
\pgfpathrectangle{\pgfqpoint{3.793912in}{0.557870in}}{\pgfqpoint{2.446088in}{1.684734in}}%
\pgfusepath{clip}%
\pgfsetbuttcap%
\pgfsetroundjoin%
\definecolor{currentfill}{rgb}{0.298039,0.447059,0.690196}%
\pgfsetfillcolor{currentfill}%
\pgfsetlinewidth{1.003750pt}%
\definecolor{currentstroke}{rgb}{0.298039,0.447059,0.690196}%
\pgfsetstrokecolor{currentstroke}%
\pgfsetdash{}{0pt}%
\pgfpathmoveto{\pgfqpoint{5.975454in}{1.272884in}}%
\pgfpathcurveto{\pgfqpoint{5.983691in}{1.272884in}}{\pgfqpoint{5.991591in}{1.276156in}}{\pgfqpoint{5.997415in}{1.281980in}}%
\pgfpathcurveto{\pgfqpoint{6.003239in}{1.287804in}}{\pgfqpoint{6.006511in}{1.295704in}}{\pgfqpoint{6.006511in}{1.303941in}}%
\pgfpathcurveto{\pgfqpoint{6.006511in}{1.312177in}}{\pgfqpoint{6.003239in}{1.320077in}}{\pgfqpoint{5.997415in}{1.325901in}}%
\pgfpathcurveto{\pgfqpoint{5.991591in}{1.331725in}}{\pgfqpoint{5.983691in}{1.334997in}}{\pgfqpoint{5.975454in}{1.334997in}}%
\pgfpathcurveto{\pgfqpoint{5.967218in}{1.334997in}}{\pgfqpoint{5.959318in}{1.331725in}}{\pgfqpoint{5.953494in}{1.325901in}}%
\pgfpathcurveto{\pgfqpoint{5.947670in}{1.320077in}}{\pgfqpoint{5.944398in}{1.312177in}}{\pgfqpoint{5.944398in}{1.303941in}}%
\pgfpathcurveto{\pgfqpoint{5.944398in}{1.295704in}}{\pgfqpoint{5.947670in}{1.287804in}}{\pgfqpoint{5.953494in}{1.281980in}}%
\pgfpathcurveto{\pgfqpoint{5.959318in}{1.276156in}}{\pgfqpoint{5.967218in}{1.272884in}}{\pgfqpoint{5.975454in}{1.272884in}}%
\pgfpathclose%
\pgfusepath{stroke,fill}%
\end{pgfscope}%
\begin{pgfscope}%
\pgfpathrectangle{\pgfqpoint{3.793912in}{0.557870in}}{\pgfqpoint{2.446088in}{1.684734in}}%
\pgfusepath{clip}%
\pgfsetbuttcap%
\pgfsetroundjoin%
\definecolor{currentfill}{rgb}{0.298039,0.447059,0.690196}%
\pgfsetfillcolor{currentfill}%
\pgfsetlinewidth{1.003750pt}%
\definecolor{currentstroke}{rgb}{0.298039,0.447059,0.690196}%
\pgfsetstrokecolor{currentstroke}%
\pgfsetdash{}{0pt}%
\pgfpathmoveto{\pgfqpoint{5.975454in}{1.272884in}}%
\pgfpathcurveto{\pgfqpoint{5.983691in}{1.272884in}}{\pgfqpoint{5.991591in}{1.276156in}}{\pgfqpoint{5.997415in}{1.281980in}}%
\pgfpathcurveto{\pgfqpoint{6.003239in}{1.287804in}}{\pgfqpoint{6.006511in}{1.295704in}}{\pgfqpoint{6.006511in}{1.303941in}}%
\pgfpathcurveto{\pgfqpoint{6.006511in}{1.312177in}}{\pgfqpoint{6.003239in}{1.320077in}}{\pgfqpoint{5.997415in}{1.325901in}}%
\pgfpathcurveto{\pgfqpoint{5.991591in}{1.331725in}}{\pgfqpoint{5.983691in}{1.334997in}}{\pgfqpoint{5.975454in}{1.334997in}}%
\pgfpathcurveto{\pgfqpoint{5.967218in}{1.334997in}}{\pgfqpoint{5.959318in}{1.331725in}}{\pgfqpoint{5.953494in}{1.325901in}}%
\pgfpathcurveto{\pgfqpoint{5.947670in}{1.320077in}}{\pgfqpoint{5.944398in}{1.312177in}}{\pgfqpoint{5.944398in}{1.303941in}}%
\pgfpathcurveto{\pgfqpoint{5.944398in}{1.295704in}}{\pgfqpoint{5.947670in}{1.287804in}}{\pgfqpoint{5.953494in}{1.281980in}}%
\pgfpathcurveto{\pgfqpoint{5.959318in}{1.276156in}}{\pgfqpoint{5.967218in}{1.272884in}}{\pgfqpoint{5.975454in}{1.272884in}}%
\pgfpathclose%
\pgfusepath{stroke,fill}%
\end{pgfscope}%
\begin{pgfscope}%
\pgfpathrectangle{\pgfqpoint{3.793912in}{0.557870in}}{\pgfqpoint{2.446088in}{1.684734in}}%
\pgfusepath{clip}%
\pgfsetbuttcap%
\pgfsetroundjoin%
\definecolor{currentfill}{rgb}{0.298039,0.447059,0.690196}%
\pgfsetfillcolor{currentfill}%
\pgfsetlinewidth{1.003750pt}%
\definecolor{currentstroke}{rgb}{0.298039,0.447059,0.690196}%
\pgfsetstrokecolor{currentstroke}%
\pgfsetdash{}{0pt}%
\pgfpathmoveto{\pgfqpoint{4.288497in}{2.107456in}}%
\pgfpathcurveto{\pgfqpoint{4.296734in}{2.107456in}}{\pgfqpoint{4.304634in}{2.110728in}}{\pgfqpoint{4.310458in}{2.116552in}}%
\pgfpathcurveto{\pgfqpoint{4.316282in}{2.122376in}}{\pgfqpoint{4.319554in}{2.130276in}}{\pgfqpoint{4.319554in}{2.138512in}}%
\pgfpathcurveto{\pgfqpoint{4.319554in}{2.146748in}}{\pgfqpoint{4.316282in}{2.154648in}}{\pgfqpoint{4.310458in}{2.160472in}}%
\pgfpathcurveto{\pgfqpoint{4.304634in}{2.166296in}}{\pgfqpoint{4.296734in}{2.169569in}}{\pgfqpoint{4.288497in}{2.169569in}}%
\pgfpathcurveto{\pgfqpoint{4.280261in}{2.169569in}}{\pgfqpoint{4.272361in}{2.166296in}}{\pgfqpoint{4.266537in}{2.160472in}}%
\pgfpathcurveto{\pgfqpoint{4.260713in}{2.154648in}}{\pgfqpoint{4.257441in}{2.146748in}}{\pgfqpoint{4.257441in}{2.138512in}}%
\pgfpathcurveto{\pgfqpoint{4.257441in}{2.130276in}}{\pgfqpoint{4.260713in}{2.122376in}}{\pgfqpoint{4.266537in}{2.116552in}}%
\pgfpathcurveto{\pgfqpoint{4.272361in}{2.110728in}}{\pgfqpoint{4.280261in}{2.107456in}}{\pgfqpoint{4.288497in}{2.107456in}}%
\pgfpathclose%
\pgfusepath{stroke,fill}%
\end{pgfscope}%
\begin{pgfscope}%
\pgfpathrectangle{\pgfqpoint{3.793912in}{0.557870in}}{\pgfqpoint{2.446088in}{1.684734in}}%
\pgfusepath{clip}%
\pgfsetbuttcap%
\pgfsetroundjoin%
\definecolor{currentfill}{rgb}{0.298039,0.447059,0.690196}%
\pgfsetfillcolor{currentfill}%
\pgfsetlinewidth{1.003750pt}%
\definecolor{currentstroke}{rgb}{0.298039,0.447059,0.690196}%
\pgfsetstrokecolor{currentstroke}%
\pgfsetdash{}{0pt}%
\pgfpathmoveto{\pgfqpoint{3.905098in}{2.125798in}}%
\pgfpathcurveto{\pgfqpoint{3.913334in}{2.125798in}}{\pgfqpoint{3.921234in}{2.129070in}}{\pgfqpoint{3.927058in}{2.134894in}}%
\pgfpathcurveto{\pgfqpoint{3.932882in}{2.140718in}}{\pgfqpoint{3.936155in}{2.148618in}}{\pgfqpoint{3.936155in}{2.156854in}}%
\pgfpathcurveto{\pgfqpoint{3.936155in}{2.165091in}}{\pgfqpoint{3.932882in}{2.172991in}}{\pgfqpoint{3.927058in}{2.178814in}}%
\pgfpathcurveto{\pgfqpoint{3.921234in}{2.184638in}}{\pgfqpoint{3.913334in}{2.187911in}}{\pgfqpoint{3.905098in}{2.187911in}}%
\pgfpathcurveto{\pgfqpoint{3.896862in}{2.187911in}}{\pgfqpoint{3.888962in}{2.184638in}}{\pgfqpoint{3.883138in}{2.178814in}}%
\pgfpathcurveto{\pgfqpoint{3.877314in}{2.172991in}}{\pgfqpoint{3.874042in}{2.165091in}}{\pgfqpoint{3.874042in}{2.156854in}}%
\pgfpathcurveto{\pgfqpoint{3.874042in}{2.148618in}}{\pgfqpoint{3.877314in}{2.140718in}}{\pgfqpoint{3.883138in}{2.134894in}}%
\pgfpathcurveto{\pgfqpoint{3.888962in}{2.129070in}}{\pgfqpoint{3.896862in}{2.125798in}}{\pgfqpoint{3.905098in}{2.125798in}}%
\pgfpathclose%
\pgfusepath{stroke,fill}%
\end{pgfscope}%
\begin{pgfscope}%
\pgfpathrectangle{\pgfqpoint{3.793912in}{0.557870in}}{\pgfqpoint{2.446088in}{1.684734in}}%
\pgfusepath{clip}%
\pgfsetbuttcap%
\pgfsetroundjoin%
\definecolor{currentfill}{rgb}{0.298039,0.447059,0.690196}%
\pgfsetfillcolor{currentfill}%
\pgfsetlinewidth{1.003750pt}%
\definecolor{currentstroke}{rgb}{0.298039,0.447059,0.690196}%
\pgfsetstrokecolor{currentstroke}%
\pgfsetdash{}{0pt}%
\pgfpathmoveto{\pgfqpoint{4.978616in}{1.667242in}}%
\pgfpathcurveto{\pgfqpoint{4.986852in}{1.667242in}}{\pgfqpoint{4.994753in}{1.670514in}}{\pgfqpoint{5.000576in}{1.676338in}}%
\pgfpathcurveto{\pgfqpoint{5.006400in}{1.682162in}}{\pgfqpoint{5.009673in}{1.690062in}}{\pgfqpoint{5.009673in}{1.698298in}}%
\pgfpathcurveto{\pgfqpoint{5.009673in}{1.706535in}}{\pgfqpoint{5.006400in}{1.714435in}}{\pgfqpoint{5.000576in}{1.720259in}}%
\pgfpathcurveto{\pgfqpoint{4.994753in}{1.726083in}}{\pgfqpoint{4.986852in}{1.729355in}}{\pgfqpoint{4.978616in}{1.729355in}}%
\pgfpathcurveto{\pgfqpoint{4.970380in}{1.729355in}}{\pgfqpoint{4.962480in}{1.726083in}}{\pgfqpoint{4.956656in}{1.720259in}}%
\pgfpathcurveto{\pgfqpoint{4.950832in}{1.714435in}}{\pgfqpoint{4.947560in}{1.706535in}}{\pgfqpoint{4.947560in}{1.698298in}}%
\pgfpathcurveto{\pgfqpoint{4.947560in}{1.690062in}}{\pgfqpoint{4.950832in}{1.682162in}}{\pgfqpoint{4.956656in}{1.676338in}}%
\pgfpathcurveto{\pgfqpoint{4.962480in}{1.670514in}}{\pgfqpoint{4.970380in}{1.667242in}}{\pgfqpoint{4.978616in}{1.667242in}}%
\pgfpathclose%
\pgfusepath{stroke,fill}%
\end{pgfscope}%
\begin{pgfscope}%
\pgfpathrectangle{\pgfqpoint{3.793912in}{0.557870in}}{\pgfqpoint{2.446088in}{1.684734in}}%
\pgfusepath{clip}%
\pgfsetbuttcap%
\pgfsetroundjoin%
\definecolor{currentfill}{rgb}{0.298039,0.447059,0.690196}%
\pgfsetfillcolor{currentfill}%
\pgfsetlinewidth{1.003750pt}%
\definecolor{currentstroke}{rgb}{0.298039,0.447059,0.690196}%
\pgfsetstrokecolor{currentstroke}%
\pgfsetdash{}{0pt}%
\pgfpathmoveto{\pgfqpoint{3.905098in}{2.125798in}}%
\pgfpathcurveto{\pgfqpoint{3.913334in}{2.125798in}}{\pgfqpoint{3.921234in}{2.129070in}}{\pgfqpoint{3.927058in}{2.134894in}}%
\pgfpathcurveto{\pgfqpoint{3.932882in}{2.140718in}}{\pgfqpoint{3.936155in}{2.148618in}}{\pgfqpoint{3.936155in}{2.156854in}}%
\pgfpathcurveto{\pgfqpoint{3.936155in}{2.165091in}}{\pgfqpoint{3.932882in}{2.172991in}}{\pgfqpoint{3.927058in}{2.178814in}}%
\pgfpathcurveto{\pgfqpoint{3.921234in}{2.184638in}}{\pgfqpoint{3.913334in}{2.187911in}}{\pgfqpoint{3.905098in}{2.187911in}}%
\pgfpathcurveto{\pgfqpoint{3.896862in}{2.187911in}}{\pgfqpoint{3.888962in}{2.184638in}}{\pgfqpoint{3.883138in}{2.178814in}}%
\pgfpathcurveto{\pgfqpoint{3.877314in}{2.172991in}}{\pgfqpoint{3.874042in}{2.165091in}}{\pgfqpoint{3.874042in}{2.156854in}}%
\pgfpathcurveto{\pgfqpoint{3.874042in}{2.148618in}}{\pgfqpoint{3.877314in}{2.140718in}}{\pgfqpoint{3.883138in}{2.134894in}}%
\pgfpathcurveto{\pgfqpoint{3.888962in}{2.129070in}}{\pgfqpoint{3.896862in}{2.125798in}}{\pgfqpoint{3.905098in}{2.125798in}}%
\pgfpathclose%
\pgfusepath{stroke,fill}%
\end{pgfscope}%
\begin{pgfscope}%
\pgfpathrectangle{\pgfqpoint{3.793912in}{0.557870in}}{\pgfqpoint{2.446088in}{1.684734in}}%
\pgfusepath{clip}%
\pgfsetbuttcap%
\pgfsetroundjoin%
\definecolor{currentfill}{rgb}{0.298039,0.447059,0.690196}%
\pgfsetfillcolor{currentfill}%
\pgfsetlinewidth{1.003750pt}%
\definecolor{currentstroke}{rgb}{0.298039,0.447059,0.690196}%
\pgfsetstrokecolor{currentstroke}%
\pgfsetdash{}{0pt}%
\pgfpathmoveto{\pgfqpoint{4.901936in}{1.474649in}}%
\pgfpathcurveto{\pgfqpoint{4.910173in}{1.474649in}}{\pgfqpoint{4.918073in}{1.477921in}}{\pgfqpoint{4.923897in}{1.483745in}}%
\pgfpathcurveto{\pgfqpoint{4.929721in}{1.489569in}}{\pgfqpoint{4.932993in}{1.497469in}}{\pgfqpoint{4.932993in}{1.505705in}}%
\pgfpathcurveto{\pgfqpoint{4.932993in}{1.513941in}}{\pgfqpoint{4.929721in}{1.521841in}}{\pgfqpoint{4.923897in}{1.527665in}}%
\pgfpathcurveto{\pgfqpoint{4.918073in}{1.533489in}}{\pgfqpoint{4.910173in}{1.536762in}}{\pgfqpoint{4.901936in}{1.536762in}}%
\pgfpathcurveto{\pgfqpoint{4.893700in}{1.536762in}}{\pgfqpoint{4.885800in}{1.533489in}}{\pgfqpoint{4.879976in}{1.527665in}}%
\pgfpathcurveto{\pgfqpoint{4.874152in}{1.521841in}}{\pgfqpoint{4.870880in}{1.513941in}}{\pgfqpoint{4.870880in}{1.505705in}}%
\pgfpathcurveto{\pgfqpoint{4.870880in}{1.497469in}}{\pgfqpoint{4.874152in}{1.489569in}}{\pgfqpoint{4.879976in}{1.483745in}}%
\pgfpathcurveto{\pgfqpoint{4.885800in}{1.477921in}}{\pgfqpoint{4.893700in}{1.474649in}}{\pgfqpoint{4.901936in}{1.474649in}}%
\pgfpathclose%
\pgfusepath{stroke,fill}%
\end{pgfscope}%
\begin{pgfscope}%
\pgfpathrectangle{\pgfqpoint{3.793912in}{0.557870in}}{\pgfqpoint{2.446088in}{1.684734in}}%
\pgfusepath{clip}%
\pgfsetbuttcap%
\pgfsetroundjoin%
\definecolor{currentfill}{rgb}{0.298039,0.447059,0.690196}%
\pgfsetfillcolor{currentfill}%
\pgfsetlinewidth{1.003750pt}%
\definecolor{currentstroke}{rgb}{0.298039,0.447059,0.690196}%
\pgfsetstrokecolor{currentstroke}%
\pgfsetdash{}{0pt}%
\pgfpathmoveto{\pgfqpoint{3.905098in}{2.125798in}}%
\pgfpathcurveto{\pgfqpoint{3.913334in}{2.125798in}}{\pgfqpoint{3.921234in}{2.129070in}}{\pgfqpoint{3.927058in}{2.134894in}}%
\pgfpathcurveto{\pgfqpoint{3.932882in}{2.140718in}}{\pgfqpoint{3.936155in}{2.148618in}}{\pgfqpoint{3.936155in}{2.156854in}}%
\pgfpathcurveto{\pgfqpoint{3.936155in}{2.165091in}}{\pgfqpoint{3.932882in}{2.172991in}}{\pgfqpoint{3.927058in}{2.178814in}}%
\pgfpathcurveto{\pgfqpoint{3.921234in}{2.184638in}}{\pgfqpoint{3.913334in}{2.187911in}}{\pgfqpoint{3.905098in}{2.187911in}}%
\pgfpathcurveto{\pgfqpoint{3.896862in}{2.187911in}}{\pgfqpoint{3.888962in}{2.184638in}}{\pgfqpoint{3.883138in}{2.178814in}}%
\pgfpathcurveto{\pgfqpoint{3.877314in}{2.172991in}}{\pgfqpoint{3.874042in}{2.165091in}}{\pgfqpoint{3.874042in}{2.156854in}}%
\pgfpathcurveto{\pgfqpoint{3.874042in}{2.148618in}}{\pgfqpoint{3.877314in}{2.140718in}}{\pgfqpoint{3.883138in}{2.134894in}}%
\pgfpathcurveto{\pgfqpoint{3.888962in}{2.129070in}}{\pgfqpoint{3.896862in}{2.125798in}}{\pgfqpoint{3.905098in}{2.125798in}}%
\pgfpathclose%
\pgfusepath{stroke,fill}%
\end{pgfscope}%
\begin{pgfscope}%
\pgfpathrectangle{\pgfqpoint{3.793912in}{0.557870in}}{\pgfqpoint{2.446088in}{1.684734in}}%
\pgfusepath{clip}%
\pgfsetbuttcap%
\pgfsetroundjoin%
\definecolor{currentfill}{rgb}{0.298039,0.447059,0.690196}%
\pgfsetfillcolor{currentfill}%
\pgfsetlinewidth{1.003750pt}%
\definecolor{currentstroke}{rgb}{0.298039,0.447059,0.690196}%
\pgfsetstrokecolor{currentstroke}%
\pgfsetdash{}{0pt}%
\pgfpathmoveto{\pgfqpoint{3.905098in}{2.125798in}}%
\pgfpathcurveto{\pgfqpoint{3.913334in}{2.125798in}}{\pgfqpoint{3.921234in}{2.129070in}}{\pgfqpoint{3.927058in}{2.134894in}}%
\pgfpathcurveto{\pgfqpoint{3.932882in}{2.140718in}}{\pgfqpoint{3.936155in}{2.148618in}}{\pgfqpoint{3.936155in}{2.156854in}}%
\pgfpathcurveto{\pgfqpoint{3.936155in}{2.165091in}}{\pgfqpoint{3.932882in}{2.172991in}}{\pgfqpoint{3.927058in}{2.178814in}}%
\pgfpathcurveto{\pgfqpoint{3.921234in}{2.184638in}}{\pgfqpoint{3.913334in}{2.187911in}}{\pgfqpoint{3.905098in}{2.187911in}}%
\pgfpathcurveto{\pgfqpoint{3.896862in}{2.187911in}}{\pgfqpoint{3.888962in}{2.184638in}}{\pgfqpoint{3.883138in}{2.178814in}}%
\pgfpathcurveto{\pgfqpoint{3.877314in}{2.172991in}}{\pgfqpoint{3.874042in}{2.165091in}}{\pgfqpoint{3.874042in}{2.156854in}}%
\pgfpathcurveto{\pgfqpoint{3.874042in}{2.148618in}}{\pgfqpoint{3.877314in}{2.140718in}}{\pgfqpoint{3.883138in}{2.134894in}}%
\pgfpathcurveto{\pgfqpoint{3.888962in}{2.129070in}}{\pgfqpoint{3.896862in}{2.125798in}}{\pgfqpoint{3.905098in}{2.125798in}}%
\pgfpathclose%
\pgfusepath{stroke,fill}%
\end{pgfscope}%
\begin{pgfscope}%
\pgfpathrectangle{\pgfqpoint{3.793912in}{0.557870in}}{\pgfqpoint{2.446088in}{1.684734in}}%
\pgfusepath{clip}%
\pgfsetbuttcap%
\pgfsetroundjoin%
\definecolor{currentfill}{rgb}{0.298039,0.447059,0.690196}%
\pgfsetfillcolor{currentfill}%
\pgfsetlinewidth{1.003750pt}%
\definecolor{currentstroke}{rgb}{0.298039,0.447059,0.690196}%
\pgfsetstrokecolor{currentstroke}%
\pgfsetdash{}{0pt}%
\pgfpathmoveto{\pgfqpoint{3.905098in}{2.125798in}}%
\pgfpathcurveto{\pgfqpoint{3.913334in}{2.125798in}}{\pgfqpoint{3.921234in}{2.129070in}}{\pgfqpoint{3.927058in}{2.134894in}}%
\pgfpathcurveto{\pgfqpoint{3.932882in}{2.140718in}}{\pgfqpoint{3.936155in}{2.148618in}}{\pgfqpoint{3.936155in}{2.156854in}}%
\pgfpathcurveto{\pgfqpoint{3.936155in}{2.165091in}}{\pgfqpoint{3.932882in}{2.172991in}}{\pgfqpoint{3.927058in}{2.178814in}}%
\pgfpathcurveto{\pgfqpoint{3.921234in}{2.184638in}}{\pgfqpoint{3.913334in}{2.187911in}}{\pgfqpoint{3.905098in}{2.187911in}}%
\pgfpathcurveto{\pgfqpoint{3.896862in}{2.187911in}}{\pgfqpoint{3.888962in}{2.184638in}}{\pgfqpoint{3.883138in}{2.178814in}}%
\pgfpathcurveto{\pgfqpoint{3.877314in}{2.172991in}}{\pgfqpoint{3.874042in}{2.165091in}}{\pgfqpoint{3.874042in}{2.156854in}}%
\pgfpathcurveto{\pgfqpoint{3.874042in}{2.148618in}}{\pgfqpoint{3.877314in}{2.140718in}}{\pgfqpoint{3.883138in}{2.134894in}}%
\pgfpathcurveto{\pgfqpoint{3.888962in}{2.129070in}}{\pgfqpoint{3.896862in}{2.125798in}}{\pgfqpoint{3.905098in}{2.125798in}}%
\pgfpathclose%
\pgfusepath{stroke,fill}%
\end{pgfscope}%
\begin{pgfscope}%
\pgfpathrectangle{\pgfqpoint{3.793912in}{0.557870in}}{\pgfqpoint{2.446088in}{1.684734in}}%
\pgfusepath{clip}%
\pgfsetbuttcap%
\pgfsetroundjoin%
\definecolor{currentfill}{rgb}{0.298039,0.447059,0.690196}%
\pgfsetfillcolor{currentfill}%
\pgfsetlinewidth{1.003750pt}%
\definecolor{currentstroke}{rgb}{0.298039,0.447059,0.690196}%
\pgfsetstrokecolor{currentstroke}%
\pgfsetdash{}{0pt}%
\pgfpathmoveto{\pgfqpoint{3.905098in}{2.125798in}}%
\pgfpathcurveto{\pgfqpoint{3.913334in}{2.125798in}}{\pgfqpoint{3.921234in}{2.129070in}}{\pgfqpoint{3.927058in}{2.134894in}}%
\pgfpathcurveto{\pgfqpoint{3.932882in}{2.140718in}}{\pgfqpoint{3.936155in}{2.148618in}}{\pgfqpoint{3.936155in}{2.156854in}}%
\pgfpathcurveto{\pgfqpoint{3.936155in}{2.165091in}}{\pgfqpoint{3.932882in}{2.172991in}}{\pgfqpoint{3.927058in}{2.178814in}}%
\pgfpathcurveto{\pgfqpoint{3.921234in}{2.184638in}}{\pgfqpoint{3.913334in}{2.187911in}}{\pgfqpoint{3.905098in}{2.187911in}}%
\pgfpathcurveto{\pgfqpoint{3.896862in}{2.187911in}}{\pgfqpoint{3.888962in}{2.184638in}}{\pgfqpoint{3.883138in}{2.178814in}}%
\pgfpathcurveto{\pgfqpoint{3.877314in}{2.172991in}}{\pgfqpoint{3.874042in}{2.165091in}}{\pgfqpoint{3.874042in}{2.156854in}}%
\pgfpathcurveto{\pgfqpoint{3.874042in}{2.148618in}}{\pgfqpoint{3.877314in}{2.140718in}}{\pgfqpoint{3.883138in}{2.134894in}}%
\pgfpathcurveto{\pgfqpoint{3.888962in}{2.129070in}}{\pgfqpoint{3.896862in}{2.125798in}}{\pgfqpoint{3.905098in}{2.125798in}}%
\pgfpathclose%
\pgfusepath{stroke,fill}%
\end{pgfscope}%
\begin{pgfscope}%
\pgfpathrectangle{\pgfqpoint{3.793912in}{0.557870in}}{\pgfqpoint{2.446088in}{1.684734in}}%
\pgfusepath{clip}%
\pgfsetbuttcap%
\pgfsetroundjoin%
\definecolor{currentfill}{rgb}{0.298039,0.447059,0.690196}%
\pgfsetfillcolor{currentfill}%
\pgfsetlinewidth{1.003750pt}%
\definecolor{currentstroke}{rgb}{0.298039,0.447059,0.690196}%
\pgfsetstrokecolor{currentstroke}%
\pgfsetdash{}{0pt}%
\pgfpathmoveto{\pgfqpoint{3.905098in}{2.125798in}}%
\pgfpathcurveto{\pgfqpoint{3.913334in}{2.125798in}}{\pgfqpoint{3.921234in}{2.129070in}}{\pgfqpoint{3.927058in}{2.134894in}}%
\pgfpathcurveto{\pgfqpoint{3.932882in}{2.140718in}}{\pgfqpoint{3.936155in}{2.148618in}}{\pgfqpoint{3.936155in}{2.156854in}}%
\pgfpathcurveto{\pgfqpoint{3.936155in}{2.165091in}}{\pgfqpoint{3.932882in}{2.172991in}}{\pgfqpoint{3.927058in}{2.178814in}}%
\pgfpathcurveto{\pgfqpoint{3.921234in}{2.184638in}}{\pgfqpoint{3.913334in}{2.187911in}}{\pgfqpoint{3.905098in}{2.187911in}}%
\pgfpathcurveto{\pgfqpoint{3.896862in}{2.187911in}}{\pgfqpoint{3.888962in}{2.184638in}}{\pgfqpoint{3.883138in}{2.178814in}}%
\pgfpathcurveto{\pgfqpoint{3.877314in}{2.172991in}}{\pgfqpoint{3.874042in}{2.165091in}}{\pgfqpoint{3.874042in}{2.156854in}}%
\pgfpathcurveto{\pgfqpoint{3.874042in}{2.148618in}}{\pgfqpoint{3.877314in}{2.140718in}}{\pgfqpoint{3.883138in}{2.134894in}}%
\pgfpathcurveto{\pgfqpoint{3.888962in}{2.129070in}}{\pgfqpoint{3.896862in}{2.125798in}}{\pgfqpoint{3.905098in}{2.125798in}}%
\pgfpathclose%
\pgfusepath{stroke,fill}%
\end{pgfscope}%
\begin{pgfscope}%
\pgfpathrectangle{\pgfqpoint{3.793912in}{0.557870in}}{\pgfqpoint{2.446088in}{1.684734in}}%
\pgfusepath{clip}%
\pgfsetbuttcap%
\pgfsetroundjoin%
\definecolor{currentfill}{rgb}{0.298039,0.447059,0.690196}%
\pgfsetfillcolor{currentfill}%
\pgfsetlinewidth{1.003750pt}%
\definecolor{currentstroke}{rgb}{0.298039,0.447059,0.690196}%
\pgfsetstrokecolor{currentstroke}%
\pgfsetdash{}{0pt}%
\pgfpathmoveto{\pgfqpoint{3.905098in}{2.125798in}}%
\pgfpathcurveto{\pgfqpoint{3.913334in}{2.125798in}}{\pgfqpoint{3.921234in}{2.129070in}}{\pgfqpoint{3.927058in}{2.134894in}}%
\pgfpathcurveto{\pgfqpoint{3.932882in}{2.140718in}}{\pgfqpoint{3.936155in}{2.148618in}}{\pgfqpoint{3.936155in}{2.156854in}}%
\pgfpathcurveto{\pgfqpoint{3.936155in}{2.165091in}}{\pgfqpoint{3.932882in}{2.172991in}}{\pgfqpoint{3.927058in}{2.178814in}}%
\pgfpathcurveto{\pgfqpoint{3.921234in}{2.184638in}}{\pgfqpoint{3.913334in}{2.187911in}}{\pgfqpoint{3.905098in}{2.187911in}}%
\pgfpathcurveto{\pgfqpoint{3.896862in}{2.187911in}}{\pgfqpoint{3.888962in}{2.184638in}}{\pgfqpoint{3.883138in}{2.178814in}}%
\pgfpathcurveto{\pgfqpoint{3.877314in}{2.172991in}}{\pgfqpoint{3.874042in}{2.165091in}}{\pgfqpoint{3.874042in}{2.156854in}}%
\pgfpathcurveto{\pgfqpoint{3.874042in}{2.148618in}}{\pgfqpoint{3.877314in}{2.140718in}}{\pgfqpoint{3.883138in}{2.134894in}}%
\pgfpathcurveto{\pgfqpoint{3.888962in}{2.129070in}}{\pgfqpoint{3.896862in}{2.125798in}}{\pgfqpoint{3.905098in}{2.125798in}}%
\pgfpathclose%
\pgfusepath{stroke,fill}%
\end{pgfscope}%
\begin{pgfscope}%
\pgfpathrectangle{\pgfqpoint{3.793912in}{0.557870in}}{\pgfqpoint{2.446088in}{1.684734in}}%
\pgfusepath{clip}%
\pgfsetbuttcap%
\pgfsetroundjoin%
\definecolor{currentfill}{rgb}{0.298039,0.447059,0.690196}%
\pgfsetfillcolor{currentfill}%
\pgfsetlinewidth{1.003750pt}%
\definecolor{currentstroke}{rgb}{0.298039,0.447059,0.690196}%
\pgfsetstrokecolor{currentstroke}%
\pgfsetdash{}{0pt}%
\pgfpathmoveto{\pgfqpoint{5.055296in}{1.364595in}}%
\pgfpathcurveto{\pgfqpoint{5.063532in}{1.364595in}}{\pgfqpoint{5.071432in}{1.367867in}}{\pgfqpoint{5.077256in}{1.373691in}}%
\pgfpathcurveto{\pgfqpoint{5.083080in}{1.379515in}}{\pgfqpoint{5.086353in}{1.387415in}}{\pgfqpoint{5.086353in}{1.395652in}}%
\pgfpathcurveto{\pgfqpoint{5.086353in}{1.403888in}}{\pgfqpoint{5.083080in}{1.411788in}}{\pgfqpoint{5.077256in}{1.417612in}}%
\pgfpathcurveto{\pgfqpoint{5.071432in}{1.423436in}}{\pgfqpoint{5.063532in}{1.426708in}}{\pgfqpoint{5.055296in}{1.426708in}}%
\pgfpathcurveto{\pgfqpoint{5.047060in}{1.426708in}}{\pgfqpoint{5.039160in}{1.423436in}}{\pgfqpoint{5.033336in}{1.417612in}}%
\pgfpathcurveto{\pgfqpoint{5.027512in}{1.411788in}}{\pgfqpoint{5.024240in}{1.403888in}}{\pgfqpoint{5.024240in}{1.395652in}}%
\pgfpathcurveto{\pgfqpoint{5.024240in}{1.387415in}}{\pgfqpoint{5.027512in}{1.379515in}}{\pgfqpoint{5.033336in}{1.373691in}}%
\pgfpathcurveto{\pgfqpoint{5.039160in}{1.367867in}}{\pgfqpoint{5.047060in}{1.364595in}}{\pgfqpoint{5.055296in}{1.364595in}}%
\pgfpathclose%
\pgfusepath{stroke,fill}%
\end{pgfscope}%
\begin{pgfscope}%
\pgfpathrectangle{\pgfqpoint{3.793912in}{0.557870in}}{\pgfqpoint{2.446088in}{1.684734in}}%
\pgfusepath{clip}%
\pgfsetbuttcap%
\pgfsetroundjoin%
\definecolor{currentfill}{rgb}{0.298039,0.447059,0.690196}%
\pgfsetfillcolor{currentfill}%
\pgfsetlinewidth{1.003750pt}%
\definecolor{currentstroke}{rgb}{0.298039,0.447059,0.690196}%
\pgfsetstrokecolor{currentstroke}%
\pgfsetdash{}{0pt}%
\pgfpathmoveto{\pgfqpoint{3.905098in}{2.125798in}}%
\pgfpathcurveto{\pgfqpoint{3.913334in}{2.125798in}}{\pgfqpoint{3.921234in}{2.129070in}}{\pgfqpoint{3.927058in}{2.134894in}}%
\pgfpathcurveto{\pgfqpoint{3.932882in}{2.140718in}}{\pgfqpoint{3.936155in}{2.148618in}}{\pgfqpoint{3.936155in}{2.156854in}}%
\pgfpathcurveto{\pgfqpoint{3.936155in}{2.165091in}}{\pgfqpoint{3.932882in}{2.172991in}}{\pgfqpoint{3.927058in}{2.178814in}}%
\pgfpathcurveto{\pgfqpoint{3.921234in}{2.184638in}}{\pgfqpoint{3.913334in}{2.187911in}}{\pgfqpoint{3.905098in}{2.187911in}}%
\pgfpathcurveto{\pgfqpoint{3.896862in}{2.187911in}}{\pgfqpoint{3.888962in}{2.184638in}}{\pgfqpoint{3.883138in}{2.178814in}}%
\pgfpathcurveto{\pgfqpoint{3.877314in}{2.172991in}}{\pgfqpoint{3.874042in}{2.165091in}}{\pgfqpoint{3.874042in}{2.156854in}}%
\pgfpathcurveto{\pgfqpoint{3.874042in}{2.148618in}}{\pgfqpoint{3.877314in}{2.140718in}}{\pgfqpoint{3.883138in}{2.134894in}}%
\pgfpathcurveto{\pgfqpoint{3.888962in}{2.129070in}}{\pgfqpoint{3.896862in}{2.125798in}}{\pgfqpoint{3.905098in}{2.125798in}}%
\pgfpathclose%
\pgfusepath{stroke,fill}%
\end{pgfscope}%
\begin{pgfscope}%
\pgfpathrectangle{\pgfqpoint{3.793912in}{0.557870in}}{\pgfqpoint{2.446088in}{1.684734in}}%
\pgfusepath{clip}%
\pgfsetbuttcap%
\pgfsetroundjoin%
\definecolor{currentfill}{rgb}{0.298039,0.447059,0.690196}%
\pgfsetfillcolor{currentfill}%
\pgfsetlinewidth{1.003750pt}%
\definecolor{currentstroke}{rgb}{0.298039,0.447059,0.690196}%
\pgfsetstrokecolor{currentstroke}%
\pgfsetdash{}{0pt}%
\pgfpathmoveto{\pgfqpoint{3.905098in}{2.125798in}}%
\pgfpathcurveto{\pgfqpoint{3.913334in}{2.125798in}}{\pgfqpoint{3.921234in}{2.129070in}}{\pgfqpoint{3.927058in}{2.134894in}}%
\pgfpathcurveto{\pgfqpoint{3.932882in}{2.140718in}}{\pgfqpoint{3.936155in}{2.148618in}}{\pgfqpoint{3.936155in}{2.156854in}}%
\pgfpathcurveto{\pgfqpoint{3.936155in}{2.165091in}}{\pgfqpoint{3.932882in}{2.172991in}}{\pgfqpoint{3.927058in}{2.178814in}}%
\pgfpathcurveto{\pgfqpoint{3.921234in}{2.184638in}}{\pgfqpoint{3.913334in}{2.187911in}}{\pgfqpoint{3.905098in}{2.187911in}}%
\pgfpathcurveto{\pgfqpoint{3.896862in}{2.187911in}}{\pgfqpoint{3.888962in}{2.184638in}}{\pgfqpoint{3.883138in}{2.178814in}}%
\pgfpathcurveto{\pgfqpoint{3.877314in}{2.172991in}}{\pgfqpoint{3.874042in}{2.165091in}}{\pgfqpoint{3.874042in}{2.156854in}}%
\pgfpathcurveto{\pgfqpoint{3.874042in}{2.148618in}}{\pgfqpoint{3.877314in}{2.140718in}}{\pgfqpoint{3.883138in}{2.134894in}}%
\pgfpathcurveto{\pgfqpoint{3.888962in}{2.129070in}}{\pgfqpoint{3.896862in}{2.125798in}}{\pgfqpoint{3.905098in}{2.125798in}}%
\pgfpathclose%
\pgfusepath{stroke,fill}%
\end{pgfscope}%
\begin{pgfscope}%
\pgfpathrectangle{\pgfqpoint{3.793912in}{0.557870in}}{\pgfqpoint{2.446088in}{1.684734in}}%
\pgfusepath{clip}%
\pgfsetbuttcap%
\pgfsetroundjoin%
\definecolor{currentfill}{rgb}{0.298039,0.447059,0.690196}%
\pgfsetfillcolor{currentfill}%
\pgfsetlinewidth{1.003750pt}%
\definecolor{currentstroke}{rgb}{0.298039,0.447059,0.690196}%
\pgfsetstrokecolor{currentstroke}%
\pgfsetdash{}{0pt}%
\pgfpathmoveto{\pgfqpoint{3.905098in}{2.125798in}}%
\pgfpathcurveto{\pgfqpoint{3.913334in}{2.125798in}}{\pgfqpoint{3.921234in}{2.129070in}}{\pgfqpoint{3.927058in}{2.134894in}}%
\pgfpathcurveto{\pgfqpoint{3.932882in}{2.140718in}}{\pgfqpoint{3.936155in}{2.148618in}}{\pgfqpoint{3.936155in}{2.156854in}}%
\pgfpathcurveto{\pgfqpoint{3.936155in}{2.165091in}}{\pgfqpoint{3.932882in}{2.172991in}}{\pgfqpoint{3.927058in}{2.178814in}}%
\pgfpathcurveto{\pgfqpoint{3.921234in}{2.184638in}}{\pgfqpoint{3.913334in}{2.187911in}}{\pgfqpoint{3.905098in}{2.187911in}}%
\pgfpathcurveto{\pgfqpoint{3.896862in}{2.187911in}}{\pgfqpoint{3.888962in}{2.184638in}}{\pgfqpoint{3.883138in}{2.178814in}}%
\pgfpathcurveto{\pgfqpoint{3.877314in}{2.172991in}}{\pgfqpoint{3.874042in}{2.165091in}}{\pgfqpoint{3.874042in}{2.156854in}}%
\pgfpathcurveto{\pgfqpoint{3.874042in}{2.148618in}}{\pgfqpoint{3.877314in}{2.140718in}}{\pgfqpoint{3.883138in}{2.134894in}}%
\pgfpathcurveto{\pgfqpoint{3.888962in}{2.129070in}}{\pgfqpoint{3.896862in}{2.125798in}}{\pgfqpoint{3.905098in}{2.125798in}}%
\pgfpathclose%
\pgfusepath{stroke,fill}%
\end{pgfscope}%
\begin{pgfscope}%
\pgfpathrectangle{\pgfqpoint{3.793912in}{0.557870in}}{\pgfqpoint{2.446088in}{1.684734in}}%
\pgfusepath{clip}%
\pgfsetbuttcap%
\pgfsetroundjoin%
\definecolor{currentfill}{rgb}{0.298039,0.447059,0.690196}%
\pgfsetfillcolor{currentfill}%
\pgfsetlinewidth{1.003750pt}%
\definecolor{currentstroke}{rgb}{0.298039,0.447059,0.690196}%
\pgfsetstrokecolor{currentstroke}%
\pgfsetdash{}{0pt}%
\pgfpathmoveto{\pgfqpoint{3.905098in}{2.125798in}}%
\pgfpathcurveto{\pgfqpoint{3.913334in}{2.125798in}}{\pgfqpoint{3.921234in}{2.129070in}}{\pgfqpoint{3.927058in}{2.134894in}}%
\pgfpathcurveto{\pgfqpoint{3.932882in}{2.140718in}}{\pgfqpoint{3.936155in}{2.148618in}}{\pgfqpoint{3.936155in}{2.156854in}}%
\pgfpathcurveto{\pgfqpoint{3.936155in}{2.165091in}}{\pgfqpoint{3.932882in}{2.172991in}}{\pgfqpoint{3.927058in}{2.178814in}}%
\pgfpathcurveto{\pgfqpoint{3.921234in}{2.184638in}}{\pgfqpoint{3.913334in}{2.187911in}}{\pgfqpoint{3.905098in}{2.187911in}}%
\pgfpathcurveto{\pgfqpoint{3.896862in}{2.187911in}}{\pgfqpoint{3.888962in}{2.184638in}}{\pgfqpoint{3.883138in}{2.178814in}}%
\pgfpathcurveto{\pgfqpoint{3.877314in}{2.172991in}}{\pgfqpoint{3.874042in}{2.165091in}}{\pgfqpoint{3.874042in}{2.156854in}}%
\pgfpathcurveto{\pgfqpoint{3.874042in}{2.148618in}}{\pgfqpoint{3.877314in}{2.140718in}}{\pgfqpoint{3.883138in}{2.134894in}}%
\pgfpathcurveto{\pgfqpoint{3.888962in}{2.129070in}}{\pgfqpoint{3.896862in}{2.125798in}}{\pgfqpoint{3.905098in}{2.125798in}}%
\pgfpathclose%
\pgfusepath{stroke,fill}%
\end{pgfscope}%
\begin{pgfscope}%
\pgfpathrectangle{\pgfqpoint{3.793912in}{0.557870in}}{\pgfqpoint{2.446088in}{1.684734in}}%
\pgfusepath{clip}%
\pgfsetbuttcap%
\pgfsetroundjoin%
\definecolor{currentfill}{rgb}{0.298039,0.447059,0.690196}%
\pgfsetfillcolor{currentfill}%
\pgfsetlinewidth{1.003750pt}%
\definecolor{currentstroke}{rgb}{0.298039,0.447059,0.690196}%
\pgfsetstrokecolor{currentstroke}%
\pgfsetdash{}{0pt}%
\pgfpathmoveto{\pgfqpoint{5.975454in}{1.263713in}}%
\pgfpathcurveto{\pgfqpoint{5.983691in}{1.263713in}}{\pgfqpoint{5.991591in}{1.266985in}}{\pgfqpoint{5.997415in}{1.272809in}}%
\pgfpathcurveto{\pgfqpoint{6.003239in}{1.278633in}}{\pgfqpoint{6.006511in}{1.286533in}}{\pgfqpoint{6.006511in}{1.294769in}}%
\pgfpathcurveto{\pgfqpoint{6.006511in}{1.303006in}}{\pgfqpoint{6.003239in}{1.310906in}}{\pgfqpoint{5.997415in}{1.316730in}}%
\pgfpathcurveto{\pgfqpoint{5.991591in}{1.322554in}}{\pgfqpoint{5.983691in}{1.325826in}}{\pgfqpoint{5.975454in}{1.325826in}}%
\pgfpathcurveto{\pgfqpoint{5.967218in}{1.325826in}}{\pgfqpoint{5.959318in}{1.322554in}}{\pgfqpoint{5.953494in}{1.316730in}}%
\pgfpathcurveto{\pgfqpoint{5.947670in}{1.310906in}}{\pgfqpoint{5.944398in}{1.303006in}}{\pgfqpoint{5.944398in}{1.294769in}}%
\pgfpathcurveto{\pgfqpoint{5.944398in}{1.286533in}}{\pgfqpoint{5.947670in}{1.278633in}}{\pgfqpoint{5.953494in}{1.272809in}}%
\pgfpathcurveto{\pgfqpoint{5.959318in}{1.266985in}}{\pgfqpoint{5.967218in}{1.263713in}}{\pgfqpoint{5.975454in}{1.263713in}}%
\pgfpathclose%
\pgfusepath{stroke,fill}%
\end{pgfscope}%
\begin{pgfscope}%
\pgfpathrectangle{\pgfqpoint{3.793912in}{0.557870in}}{\pgfqpoint{2.446088in}{1.684734in}}%
\pgfusepath{clip}%
\pgfsetbuttcap%
\pgfsetroundjoin%
\definecolor{currentfill}{rgb}{0.298039,0.447059,0.690196}%
\pgfsetfillcolor{currentfill}%
\pgfsetlinewidth{1.003750pt}%
\definecolor{currentstroke}{rgb}{0.298039,0.447059,0.690196}%
\pgfsetstrokecolor{currentstroke}%
\pgfsetdash{}{0pt}%
\pgfpathmoveto{\pgfqpoint{5.975454in}{1.337082in}}%
\pgfpathcurveto{\pgfqpoint{5.983691in}{1.337082in}}{\pgfqpoint{5.991591in}{1.340354in}}{\pgfqpoint{5.997415in}{1.346178in}}%
\pgfpathcurveto{\pgfqpoint{6.003239in}{1.352002in}}{\pgfqpoint{6.006511in}{1.359902in}}{\pgfqpoint{6.006511in}{1.368138in}}%
\pgfpathcurveto{\pgfqpoint{6.006511in}{1.376375in}}{\pgfqpoint{6.003239in}{1.384275in}}{\pgfqpoint{5.997415in}{1.390099in}}%
\pgfpathcurveto{\pgfqpoint{5.991591in}{1.395923in}}{\pgfqpoint{5.983691in}{1.399195in}}{\pgfqpoint{5.975454in}{1.399195in}}%
\pgfpathcurveto{\pgfqpoint{5.967218in}{1.399195in}}{\pgfqpoint{5.959318in}{1.395923in}}{\pgfqpoint{5.953494in}{1.390099in}}%
\pgfpathcurveto{\pgfqpoint{5.947670in}{1.384275in}}{\pgfqpoint{5.944398in}{1.376375in}}{\pgfqpoint{5.944398in}{1.368138in}}%
\pgfpathcurveto{\pgfqpoint{5.944398in}{1.359902in}}{\pgfqpoint{5.947670in}{1.352002in}}{\pgfqpoint{5.953494in}{1.346178in}}%
\pgfpathcurveto{\pgfqpoint{5.959318in}{1.340354in}}{\pgfqpoint{5.967218in}{1.337082in}}{\pgfqpoint{5.975454in}{1.337082in}}%
\pgfpathclose%
\pgfusepath{stroke,fill}%
\end{pgfscope}%
\begin{pgfscope}%
\pgfpathrectangle{\pgfqpoint{3.793912in}{0.557870in}}{\pgfqpoint{2.446088in}{1.684734in}}%
\pgfusepath{clip}%
\pgfsetbuttcap%
\pgfsetroundjoin%
\definecolor{currentfill}{rgb}{0.298039,0.447059,0.690196}%
\pgfsetfillcolor{currentfill}%
\pgfsetlinewidth{1.003750pt}%
\definecolor{currentstroke}{rgb}{0.298039,0.447059,0.690196}%
\pgfsetstrokecolor{currentstroke}%
\pgfsetdash{}{0pt}%
\pgfpathmoveto{\pgfqpoint{5.975454in}{1.291226in}}%
\pgfpathcurveto{\pgfqpoint{5.983691in}{1.291226in}}{\pgfqpoint{5.991591in}{1.294499in}}{\pgfqpoint{5.997415in}{1.300322in}}%
\pgfpathcurveto{\pgfqpoint{6.003239in}{1.306146in}}{\pgfqpoint{6.006511in}{1.314046in}}{\pgfqpoint{6.006511in}{1.322283in}}%
\pgfpathcurveto{\pgfqpoint{6.006511in}{1.330519in}}{\pgfqpoint{6.003239in}{1.338419in}}{\pgfqpoint{5.997415in}{1.344243in}}%
\pgfpathcurveto{\pgfqpoint{5.991591in}{1.350067in}}{\pgfqpoint{5.983691in}{1.353339in}}{\pgfqpoint{5.975454in}{1.353339in}}%
\pgfpathcurveto{\pgfqpoint{5.967218in}{1.353339in}}{\pgfqpoint{5.959318in}{1.350067in}}{\pgfqpoint{5.953494in}{1.344243in}}%
\pgfpathcurveto{\pgfqpoint{5.947670in}{1.338419in}}{\pgfqpoint{5.944398in}{1.330519in}}{\pgfqpoint{5.944398in}{1.322283in}}%
\pgfpathcurveto{\pgfqpoint{5.944398in}{1.314046in}}{\pgfqpoint{5.947670in}{1.306146in}}{\pgfqpoint{5.953494in}{1.300322in}}%
\pgfpathcurveto{\pgfqpoint{5.959318in}{1.294499in}}{\pgfqpoint{5.967218in}{1.291226in}}{\pgfqpoint{5.975454in}{1.291226in}}%
\pgfpathclose%
\pgfusepath{stroke,fill}%
\end{pgfscope}%
\begin{pgfscope}%
\pgfpathrectangle{\pgfqpoint{3.793912in}{0.557870in}}{\pgfqpoint{2.446088in}{1.684734in}}%
\pgfusepath{clip}%
\pgfsetbuttcap%
\pgfsetroundjoin%
\definecolor{currentfill}{rgb}{0.298039,0.447059,0.690196}%
\pgfsetfillcolor{currentfill}%
\pgfsetlinewidth{1.003750pt}%
\definecolor{currentstroke}{rgb}{0.298039,0.447059,0.690196}%
\pgfsetstrokecolor{currentstroke}%
\pgfsetdash{}{0pt}%
\pgfpathmoveto{\pgfqpoint{4.365177in}{2.107456in}}%
\pgfpathcurveto{\pgfqpoint{4.373414in}{2.107456in}}{\pgfqpoint{4.381314in}{2.110728in}}{\pgfqpoint{4.387137in}{2.116552in}}%
\pgfpathcurveto{\pgfqpoint{4.392961in}{2.122376in}}{\pgfqpoint{4.396234in}{2.130276in}}{\pgfqpoint{4.396234in}{2.138512in}}%
\pgfpathcurveto{\pgfqpoint{4.396234in}{2.146748in}}{\pgfqpoint{4.392961in}{2.154648in}}{\pgfqpoint{4.387137in}{2.160472in}}%
\pgfpathcurveto{\pgfqpoint{4.381314in}{2.166296in}}{\pgfqpoint{4.373414in}{2.169569in}}{\pgfqpoint{4.365177in}{2.169569in}}%
\pgfpathcurveto{\pgfqpoint{4.356941in}{2.169569in}}{\pgfqpoint{4.349041in}{2.166296in}}{\pgfqpoint{4.343217in}{2.160472in}}%
\pgfpathcurveto{\pgfqpoint{4.337393in}{2.154648in}}{\pgfqpoint{4.334121in}{2.146748in}}{\pgfqpoint{4.334121in}{2.138512in}}%
\pgfpathcurveto{\pgfqpoint{4.334121in}{2.130276in}}{\pgfqpoint{4.337393in}{2.122376in}}{\pgfqpoint{4.343217in}{2.116552in}}%
\pgfpathcurveto{\pgfqpoint{4.349041in}{2.110728in}}{\pgfqpoint{4.356941in}{2.107456in}}{\pgfqpoint{4.365177in}{2.107456in}}%
\pgfpathclose%
\pgfusepath{stroke,fill}%
\end{pgfscope}%
\begin{pgfscope}%
\pgfpathrectangle{\pgfqpoint{3.793912in}{0.557870in}}{\pgfqpoint{2.446088in}{1.684734in}}%
\pgfusepath{clip}%
\pgfsetbuttcap%
\pgfsetroundjoin%
\definecolor{currentfill}{rgb}{0.298039,0.447059,0.690196}%
\pgfsetfillcolor{currentfill}%
\pgfsetlinewidth{1.003750pt}%
\definecolor{currentstroke}{rgb}{0.298039,0.447059,0.690196}%
\pgfsetstrokecolor{currentstroke}%
\pgfsetdash{}{0pt}%
\pgfpathmoveto{\pgfqpoint{5.745415in}{1.795638in}}%
\pgfpathcurveto{\pgfqpoint{5.753651in}{1.795638in}}{\pgfqpoint{5.761551in}{1.798910in}}{\pgfqpoint{5.767375in}{1.804734in}}%
\pgfpathcurveto{\pgfqpoint{5.773199in}{1.810558in}}{\pgfqpoint{5.776471in}{1.818458in}}{\pgfqpoint{5.776471in}{1.826694in}}%
\pgfpathcurveto{\pgfqpoint{5.776471in}{1.834930in}}{\pgfqpoint{5.773199in}{1.842830in}}{\pgfqpoint{5.767375in}{1.848654in}}%
\pgfpathcurveto{\pgfqpoint{5.761551in}{1.854478in}}{\pgfqpoint{5.753651in}{1.857751in}}{\pgfqpoint{5.745415in}{1.857751in}}%
\pgfpathcurveto{\pgfqpoint{5.737179in}{1.857751in}}{\pgfqpoint{5.729279in}{1.854478in}}{\pgfqpoint{5.723455in}{1.848654in}}%
\pgfpathcurveto{\pgfqpoint{5.717631in}{1.842830in}}{\pgfqpoint{5.714358in}{1.834930in}}{\pgfqpoint{5.714358in}{1.826694in}}%
\pgfpathcurveto{\pgfqpoint{5.714358in}{1.818458in}}{\pgfqpoint{5.717631in}{1.810558in}}{\pgfqpoint{5.723455in}{1.804734in}}%
\pgfpathcurveto{\pgfqpoint{5.729279in}{1.798910in}}{\pgfqpoint{5.737179in}{1.795638in}}{\pgfqpoint{5.745415in}{1.795638in}}%
\pgfpathclose%
\pgfusepath{stroke,fill}%
\end{pgfscope}%
\begin{pgfscope}%
\pgfpathrectangle{\pgfqpoint{3.793912in}{0.557870in}}{\pgfqpoint{2.446088in}{1.684734in}}%
\pgfusepath{clip}%
\pgfsetbuttcap%
\pgfsetroundjoin%
\definecolor{currentfill}{rgb}{0.298039,0.447059,0.690196}%
\pgfsetfillcolor{currentfill}%
\pgfsetlinewidth{1.003750pt}%
\definecolor{currentstroke}{rgb}{0.298039,0.447059,0.690196}%
\pgfsetstrokecolor{currentstroke}%
\pgfsetdash{}{0pt}%
\pgfpathmoveto{\pgfqpoint{3.905098in}{2.125798in}}%
\pgfpathcurveto{\pgfqpoint{3.913334in}{2.125798in}}{\pgfqpoint{3.921234in}{2.129070in}}{\pgfqpoint{3.927058in}{2.134894in}}%
\pgfpathcurveto{\pgfqpoint{3.932882in}{2.140718in}}{\pgfqpoint{3.936155in}{2.148618in}}{\pgfqpoint{3.936155in}{2.156854in}}%
\pgfpathcurveto{\pgfqpoint{3.936155in}{2.165091in}}{\pgfqpoint{3.932882in}{2.172991in}}{\pgfqpoint{3.927058in}{2.178814in}}%
\pgfpathcurveto{\pgfqpoint{3.921234in}{2.184638in}}{\pgfqpoint{3.913334in}{2.187911in}}{\pgfqpoint{3.905098in}{2.187911in}}%
\pgfpathcurveto{\pgfqpoint{3.896862in}{2.187911in}}{\pgfqpoint{3.888962in}{2.184638in}}{\pgfqpoint{3.883138in}{2.178814in}}%
\pgfpathcurveto{\pgfqpoint{3.877314in}{2.172991in}}{\pgfqpoint{3.874042in}{2.165091in}}{\pgfqpoint{3.874042in}{2.156854in}}%
\pgfpathcurveto{\pgfqpoint{3.874042in}{2.148618in}}{\pgfqpoint{3.877314in}{2.140718in}}{\pgfqpoint{3.883138in}{2.134894in}}%
\pgfpathcurveto{\pgfqpoint{3.888962in}{2.129070in}}{\pgfqpoint{3.896862in}{2.125798in}}{\pgfqpoint{3.905098in}{2.125798in}}%
\pgfpathclose%
\pgfusepath{stroke,fill}%
\end{pgfscope}%
\begin{pgfscope}%
\pgfpathrectangle{\pgfqpoint{3.793912in}{0.557870in}}{\pgfqpoint{2.446088in}{1.684734in}}%
\pgfusepath{clip}%
\pgfsetbuttcap%
\pgfsetroundjoin%
\definecolor{currentfill}{rgb}{0.298039,0.447059,0.690196}%
\pgfsetfillcolor{currentfill}%
\pgfsetlinewidth{1.003750pt}%
\definecolor{currentstroke}{rgb}{0.298039,0.447059,0.690196}%
\pgfsetstrokecolor{currentstroke}%
\pgfsetdash{}{0pt}%
\pgfpathmoveto{\pgfqpoint{5.975454in}{1.520504in}}%
\pgfpathcurveto{\pgfqpoint{5.983691in}{1.520504in}}{\pgfqpoint{5.991591in}{1.523776in}}{\pgfqpoint{5.997415in}{1.529600in}}%
\pgfpathcurveto{\pgfqpoint{6.003239in}{1.535424in}}{\pgfqpoint{6.006511in}{1.543324in}}{\pgfqpoint{6.006511in}{1.551561in}}%
\pgfpathcurveto{\pgfqpoint{6.006511in}{1.559797in}}{\pgfqpoint{6.003239in}{1.567697in}}{\pgfqpoint{5.997415in}{1.573521in}}%
\pgfpathcurveto{\pgfqpoint{5.991591in}{1.579345in}}{\pgfqpoint{5.983691in}{1.582617in}}{\pgfqpoint{5.975454in}{1.582617in}}%
\pgfpathcurveto{\pgfqpoint{5.967218in}{1.582617in}}{\pgfqpoint{5.959318in}{1.579345in}}{\pgfqpoint{5.953494in}{1.573521in}}%
\pgfpathcurveto{\pgfqpoint{5.947670in}{1.567697in}}{\pgfqpoint{5.944398in}{1.559797in}}{\pgfqpoint{5.944398in}{1.551561in}}%
\pgfpathcurveto{\pgfqpoint{5.944398in}{1.543324in}}{\pgfqpoint{5.947670in}{1.535424in}}{\pgfqpoint{5.953494in}{1.529600in}}%
\pgfpathcurveto{\pgfqpoint{5.959318in}{1.523776in}}{\pgfqpoint{5.967218in}{1.520504in}}{\pgfqpoint{5.975454in}{1.520504in}}%
\pgfpathclose%
\pgfusepath{stroke,fill}%
\end{pgfscope}%
\begin{pgfscope}%
\pgfpathrectangle{\pgfqpoint{3.793912in}{0.557870in}}{\pgfqpoint{2.446088in}{1.684734in}}%
\pgfusepath{clip}%
\pgfsetbuttcap%
\pgfsetroundjoin%
\definecolor{currentfill}{rgb}{0.298039,0.447059,0.690196}%
\pgfsetfillcolor{currentfill}%
\pgfsetlinewidth{1.003750pt}%
\definecolor{currentstroke}{rgb}{0.298039,0.447059,0.690196}%
\pgfsetstrokecolor{currentstroke}%
\pgfsetdash{}{0pt}%
\pgfpathmoveto{\pgfqpoint{3.905098in}{2.125798in}}%
\pgfpathcurveto{\pgfqpoint{3.913334in}{2.125798in}}{\pgfqpoint{3.921234in}{2.129070in}}{\pgfqpoint{3.927058in}{2.134894in}}%
\pgfpathcurveto{\pgfqpoint{3.932882in}{2.140718in}}{\pgfqpoint{3.936155in}{2.148618in}}{\pgfqpoint{3.936155in}{2.156854in}}%
\pgfpathcurveto{\pgfqpoint{3.936155in}{2.165091in}}{\pgfqpoint{3.932882in}{2.172991in}}{\pgfqpoint{3.927058in}{2.178814in}}%
\pgfpathcurveto{\pgfqpoint{3.921234in}{2.184638in}}{\pgfqpoint{3.913334in}{2.187911in}}{\pgfqpoint{3.905098in}{2.187911in}}%
\pgfpathcurveto{\pgfqpoint{3.896862in}{2.187911in}}{\pgfqpoint{3.888962in}{2.184638in}}{\pgfqpoint{3.883138in}{2.178814in}}%
\pgfpathcurveto{\pgfqpoint{3.877314in}{2.172991in}}{\pgfqpoint{3.874042in}{2.165091in}}{\pgfqpoint{3.874042in}{2.156854in}}%
\pgfpathcurveto{\pgfqpoint{3.874042in}{2.148618in}}{\pgfqpoint{3.877314in}{2.140718in}}{\pgfqpoint{3.883138in}{2.134894in}}%
\pgfpathcurveto{\pgfqpoint{3.888962in}{2.129070in}}{\pgfqpoint{3.896862in}{2.125798in}}{\pgfqpoint{3.905098in}{2.125798in}}%
\pgfpathclose%
\pgfusepath{stroke,fill}%
\end{pgfscope}%
\begin{pgfscope}%
\pgfpathrectangle{\pgfqpoint{3.793912in}{0.557870in}}{\pgfqpoint{2.446088in}{1.684734in}}%
\pgfusepath{clip}%
\pgfsetbuttcap%
\pgfsetroundjoin%
\definecolor{currentfill}{rgb}{0.298039,0.447059,0.690196}%
\pgfsetfillcolor{currentfill}%
\pgfsetlinewidth{1.003750pt}%
\definecolor{currentstroke}{rgb}{0.298039,0.447059,0.690196}%
\pgfsetstrokecolor{currentstroke}%
\pgfsetdash{}{0pt}%
\pgfpathmoveto{\pgfqpoint{5.975454in}{1.621386in}}%
\pgfpathcurveto{\pgfqpoint{5.983691in}{1.621386in}}{\pgfqpoint{5.991591in}{1.624659in}}{\pgfqpoint{5.997415in}{1.630483in}}%
\pgfpathcurveto{\pgfqpoint{6.003239in}{1.636307in}}{\pgfqpoint{6.006511in}{1.644207in}}{\pgfqpoint{6.006511in}{1.652443in}}%
\pgfpathcurveto{\pgfqpoint{6.006511in}{1.660679in}}{\pgfqpoint{6.003239in}{1.668579in}}{\pgfqpoint{5.997415in}{1.674403in}}%
\pgfpathcurveto{\pgfqpoint{5.991591in}{1.680227in}}{\pgfqpoint{5.983691in}{1.683499in}}{\pgfqpoint{5.975454in}{1.683499in}}%
\pgfpathcurveto{\pgfqpoint{5.967218in}{1.683499in}}{\pgfqpoint{5.959318in}{1.680227in}}{\pgfqpoint{5.953494in}{1.674403in}}%
\pgfpathcurveto{\pgfqpoint{5.947670in}{1.668579in}}{\pgfqpoint{5.944398in}{1.660679in}}{\pgfqpoint{5.944398in}{1.652443in}}%
\pgfpathcurveto{\pgfqpoint{5.944398in}{1.644207in}}{\pgfqpoint{5.947670in}{1.636307in}}{\pgfqpoint{5.953494in}{1.630483in}}%
\pgfpathcurveto{\pgfqpoint{5.959318in}{1.624659in}}{\pgfqpoint{5.967218in}{1.621386in}}{\pgfqpoint{5.975454in}{1.621386in}}%
\pgfpathclose%
\pgfusepath{stroke,fill}%
\end{pgfscope}%
\begin{pgfscope}%
\pgfpathrectangle{\pgfqpoint{3.793912in}{0.557870in}}{\pgfqpoint{2.446088in}{1.684734in}}%
\pgfusepath{clip}%
\pgfsetbuttcap%
\pgfsetroundjoin%
\definecolor{currentfill}{rgb}{0.298039,0.447059,0.690196}%
\pgfsetfillcolor{currentfill}%
\pgfsetlinewidth{1.003750pt}%
\definecolor{currentstroke}{rgb}{0.298039,0.447059,0.690196}%
\pgfsetstrokecolor{currentstroke}%
\pgfsetdash{}{0pt}%
\pgfpathmoveto{\pgfqpoint{3.905098in}{2.125798in}}%
\pgfpathcurveto{\pgfqpoint{3.913334in}{2.125798in}}{\pgfqpoint{3.921234in}{2.129070in}}{\pgfqpoint{3.927058in}{2.134894in}}%
\pgfpathcurveto{\pgfqpoint{3.932882in}{2.140718in}}{\pgfqpoint{3.936155in}{2.148618in}}{\pgfqpoint{3.936155in}{2.156854in}}%
\pgfpathcurveto{\pgfqpoint{3.936155in}{2.165091in}}{\pgfqpoint{3.932882in}{2.172991in}}{\pgfqpoint{3.927058in}{2.178814in}}%
\pgfpathcurveto{\pgfqpoint{3.921234in}{2.184638in}}{\pgfqpoint{3.913334in}{2.187911in}}{\pgfqpoint{3.905098in}{2.187911in}}%
\pgfpathcurveto{\pgfqpoint{3.896862in}{2.187911in}}{\pgfqpoint{3.888962in}{2.184638in}}{\pgfqpoint{3.883138in}{2.178814in}}%
\pgfpathcurveto{\pgfqpoint{3.877314in}{2.172991in}}{\pgfqpoint{3.874042in}{2.165091in}}{\pgfqpoint{3.874042in}{2.156854in}}%
\pgfpathcurveto{\pgfqpoint{3.874042in}{2.148618in}}{\pgfqpoint{3.877314in}{2.140718in}}{\pgfqpoint{3.883138in}{2.134894in}}%
\pgfpathcurveto{\pgfqpoint{3.888962in}{2.129070in}}{\pgfqpoint{3.896862in}{2.125798in}}{\pgfqpoint{3.905098in}{2.125798in}}%
\pgfpathclose%
\pgfusepath{stroke,fill}%
\end{pgfscope}%
\begin{pgfscope}%
\pgfpathrectangle{\pgfqpoint{3.793912in}{0.557870in}}{\pgfqpoint{2.446088in}{1.684734in}}%
\pgfusepath{clip}%
\pgfsetbuttcap%
\pgfsetroundjoin%
\definecolor{currentfill}{rgb}{0.298039,0.447059,0.690196}%
\pgfsetfillcolor{currentfill}%
\pgfsetlinewidth{1.003750pt}%
\definecolor{currentstroke}{rgb}{0.298039,0.447059,0.690196}%
\pgfsetstrokecolor{currentstroke}%
\pgfsetdash{}{0pt}%
\pgfpathmoveto{\pgfqpoint{3.905098in}{2.125798in}}%
\pgfpathcurveto{\pgfqpoint{3.913334in}{2.125798in}}{\pgfqpoint{3.921234in}{2.129070in}}{\pgfqpoint{3.927058in}{2.134894in}}%
\pgfpathcurveto{\pgfqpoint{3.932882in}{2.140718in}}{\pgfqpoint{3.936155in}{2.148618in}}{\pgfqpoint{3.936155in}{2.156854in}}%
\pgfpathcurveto{\pgfqpoint{3.936155in}{2.165091in}}{\pgfqpoint{3.932882in}{2.172991in}}{\pgfqpoint{3.927058in}{2.178814in}}%
\pgfpathcurveto{\pgfqpoint{3.921234in}{2.184638in}}{\pgfqpoint{3.913334in}{2.187911in}}{\pgfqpoint{3.905098in}{2.187911in}}%
\pgfpathcurveto{\pgfqpoint{3.896862in}{2.187911in}}{\pgfqpoint{3.888962in}{2.184638in}}{\pgfqpoint{3.883138in}{2.178814in}}%
\pgfpathcurveto{\pgfqpoint{3.877314in}{2.172991in}}{\pgfqpoint{3.874042in}{2.165091in}}{\pgfqpoint{3.874042in}{2.156854in}}%
\pgfpathcurveto{\pgfqpoint{3.874042in}{2.148618in}}{\pgfqpoint{3.877314in}{2.140718in}}{\pgfqpoint{3.883138in}{2.134894in}}%
\pgfpathcurveto{\pgfqpoint{3.888962in}{2.129070in}}{\pgfqpoint{3.896862in}{2.125798in}}{\pgfqpoint{3.905098in}{2.125798in}}%
\pgfpathclose%
\pgfusepath{stroke,fill}%
\end{pgfscope}%
\begin{pgfscope}%
\pgfpathrectangle{\pgfqpoint{3.793912in}{0.557870in}}{\pgfqpoint{2.446088in}{1.684734in}}%
\pgfusepath{clip}%
\pgfsetbuttcap%
\pgfsetroundjoin%
\definecolor{currentfill}{rgb}{0.298039,0.447059,0.690196}%
\pgfsetfillcolor{currentfill}%
\pgfsetlinewidth{1.003750pt}%
\definecolor{currentstroke}{rgb}{0.298039,0.447059,0.690196}%
\pgfsetstrokecolor{currentstroke}%
\pgfsetdash{}{0pt}%
\pgfpathmoveto{\pgfqpoint{3.905098in}{2.125798in}}%
\pgfpathcurveto{\pgfqpoint{3.913334in}{2.125798in}}{\pgfqpoint{3.921234in}{2.129070in}}{\pgfqpoint{3.927058in}{2.134894in}}%
\pgfpathcurveto{\pgfqpoint{3.932882in}{2.140718in}}{\pgfqpoint{3.936155in}{2.148618in}}{\pgfqpoint{3.936155in}{2.156854in}}%
\pgfpathcurveto{\pgfqpoint{3.936155in}{2.165091in}}{\pgfqpoint{3.932882in}{2.172991in}}{\pgfqpoint{3.927058in}{2.178814in}}%
\pgfpathcurveto{\pgfqpoint{3.921234in}{2.184638in}}{\pgfqpoint{3.913334in}{2.187911in}}{\pgfqpoint{3.905098in}{2.187911in}}%
\pgfpathcurveto{\pgfqpoint{3.896862in}{2.187911in}}{\pgfqpoint{3.888962in}{2.184638in}}{\pgfqpoint{3.883138in}{2.178814in}}%
\pgfpathcurveto{\pgfqpoint{3.877314in}{2.172991in}}{\pgfqpoint{3.874042in}{2.165091in}}{\pgfqpoint{3.874042in}{2.156854in}}%
\pgfpathcurveto{\pgfqpoint{3.874042in}{2.148618in}}{\pgfqpoint{3.877314in}{2.140718in}}{\pgfqpoint{3.883138in}{2.134894in}}%
\pgfpathcurveto{\pgfqpoint{3.888962in}{2.129070in}}{\pgfqpoint{3.896862in}{2.125798in}}{\pgfqpoint{3.905098in}{2.125798in}}%
\pgfpathclose%
\pgfusepath{stroke,fill}%
\end{pgfscope}%
\begin{pgfscope}%
\pgfpathrectangle{\pgfqpoint{3.793912in}{0.557870in}}{\pgfqpoint{2.446088in}{1.684734in}}%
\pgfusepath{clip}%
\pgfsetbuttcap%
\pgfsetroundjoin%
\definecolor{currentfill}{rgb}{0.298039,0.447059,0.690196}%
\pgfsetfillcolor{currentfill}%
\pgfsetlinewidth{1.003750pt}%
\definecolor{currentstroke}{rgb}{0.298039,0.447059,0.690196}%
\pgfsetstrokecolor{currentstroke}%
\pgfsetdash{}{0pt}%
\pgfpathmoveto{\pgfqpoint{3.905098in}{2.125798in}}%
\pgfpathcurveto{\pgfqpoint{3.913334in}{2.125798in}}{\pgfqpoint{3.921234in}{2.129070in}}{\pgfqpoint{3.927058in}{2.134894in}}%
\pgfpathcurveto{\pgfqpoint{3.932882in}{2.140718in}}{\pgfqpoint{3.936155in}{2.148618in}}{\pgfqpoint{3.936155in}{2.156854in}}%
\pgfpathcurveto{\pgfqpoint{3.936155in}{2.165091in}}{\pgfqpoint{3.932882in}{2.172991in}}{\pgfqpoint{3.927058in}{2.178814in}}%
\pgfpathcurveto{\pgfqpoint{3.921234in}{2.184638in}}{\pgfqpoint{3.913334in}{2.187911in}}{\pgfqpoint{3.905098in}{2.187911in}}%
\pgfpathcurveto{\pgfqpoint{3.896862in}{2.187911in}}{\pgfqpoint{3.888962in}{2.184638in}}{\pgfqpoint{3.883138in}{2.178814in}}%
\pgfpathcurveto{\pgfqpoint{3.877314in}{2.172991in}}{\pgfqpoint{3.874042in}{2.165091in}}{\pgfqpoint{3.874042in}{2.156854in}}%
\pgfpathcurveto{\pgfqpoint{3.874042in}{2.148618in}}{\pgfqpoint{3.877314in}{2.140718in}}{\pgfqpoint{3.883138in}{2.134894in}}%
\pgfpathcurveto{\pgfqpoint{3.888962in}{2.129070in}}{\pgfqpoint{3.896862in}{2.125798in}}{\pgfqpoint{3.905098in}{2.125798in}}%
\pgfpathclose%
\pgfusepath{stroke,fill}%
\end{pgfscope}%
\begin{pgfscope}%
\pgfpathrectangle{\pgfqpoint{3.793912in}{0.557870in}}{\pgfqpoint{2.446088in}{1.684734in}}%
\pgfusepath{clip}%
\pgfsetbuttcap%
\pgfsetroundjoin%
\definecolor{currentfill}{rgb}{0.298039,0.447059,0.690196}%
\pgfsetfillcolor{currentfill}%
\pgfsetlinewidth{1.003750pt}%
\definecolor{currentstroke}{rgb}{0.298039,0.447059,0.690196}%
\pgfsetstrokecolor{currentstroke}%
\pgfsetdash{}{0pt}%
\pgfpathmoveto{\pgfqpoint{5.055296in}{1.703926in}}%
\pgfpathcurveto{\pgfqpoint{5.063532in}{1.703926in}}{\pgfqpoint{5.071432in}{1.707199in}}{\pgfqpoint{5.077256in}{1.713023in}}%
\pgfpathcurveto{\pgfqpoint{5.083080in}{1.718847in}}{\pgfqpoint{5.086353in}{1.726747in}}{\pgfqpoint{5.086353in}{1.734983in}}%
\pgfpathcurveto{\pgfqpoint{5.086353in}{1.743219in}}{\pgfqpoint{5.083080in}{1.751119in}}{\pgfqpoint{5.077256in}{1.756943in}}%
\pgfpathcurveto{\pgfqpoint{5.071432in}{1.762767in}}{\pgfqpoint{5.063532in}{1.766039in}}{\pgfqpoint{5.055296in}{1.766039in}}%
\pgfpathcurveto{\pgfqpoint{5.047060in}{1.766039in}}{\pgfqpoint{5.039160in}{1.762767in}}{\pgfqpoint{5.033336in}{1.756943in}}%
\pgfpathcurveto{\pgfqpoint{5.027512in}{1.751119in}}{\pgfqpoint{5.024240in}{1.743219in}}{\pgfqpoint{5.024240in}{1.734983in}}%
\pgfpathcurveto{\pgfqpoint{5.024240in}{1.726747in}}{\pgfqpoint{5.027512in}{1.718847in}}{\pgfqpoint{5.033336in}{1.713023in}}%
\pgfpathcurveto{\pgfqpoint{5.039160in}{1.707199in}}{\pgfqpoint{5.047060in}{1.703926in}}{\pgfqpoint{5.055296in}{1.703926in}}%
\pgfpathclose%
\pgfusepath{stroke,fill}%
\end{pgfscope}%
\begin{pgfscope}%
\pgfpathrectangle{\pgfqpoint{3.793912in}{0.557870in}}{\pgfqpoint{2.446088in}{1.684734in}}%
\pgfusepath{clip}%
\pgfsetbuttcap%
\pgfsetroundjoin%
\definecolor{currentfill}{rgb}{0.298039,0.447059,0.690196}%
\pgfsetfillcolor{currentfill}%
\pgfsetlinewidth{1.003750pt}%
\definecolor{currentstroke}{rgb}{0.298039,0.447059,0.690196}%
\pgfsetstrokecolor{currentstroke}%
\pgfsetdash{}{0pt}%
\pgfpathmoveto{\pgfqpoint{3.905098in}{2.125798in}}%
\pgfpathcurveto{\pgfqpoint{3.913334in}{2.125798in}}{\pgfqpoint{3.921234in}{2.129070in}}{\pgfqpoint{3.927058in}{2.134894in}}%
\pgfpathcurveto{\pgfqpoint{3.932882in}{2.140718in}}{\pgfqpoint{3.936155in}{2.148618in}}{\pgfqpoint{3.936155in}{2.156854in}}%
\pgfpathcurveto{\pgfqpoint{3.936155in}{2.165091in}}{\pgfqpoint{3.932882in}{2.172991in}}{\pgfqpoint{3.927058in}{2.178814in}}%
\pgfpathcurveto{\pgfqpoint{3.921234in}{2.184638in}}{\pgfqpoint{3.913334in}{2.187911in}}{\pgfqpoint{3.905098in}{2.187911in}}%
\pgfpathcurveto{\pgfqpoint{3.896862in}{2.187911in}}{\pgfqpoint{3.888962in}{2.184638in}}{\pgfqpoint{3.883138in}{2.178814in}}%
\pgfpathcurveto{\pgfqpoint{3.877314in}{2.172991in}}{\pgfqpoint{3.874042in}{2.165091in}}{\pgfqpoint{3.874042in}{2.156854in}}%
\pgfpathcurveto{\pgfqpoint{3.874042in}{2.148618in}}{\pgfqpoint{3.877314in}{2.140718in}}{\pgfqpoint{3.883138in}{2.134894in}}%
\pgfpathcurveto{\pgfqpoint{3.888962in}{2.129070in}}{\pgfqpoint{3.896862in}{2.125798in}}{\pgfqpoint{3.905098in}{2.125798in}}%
\pgfpathclose%
\pgfusepath{stroke,fill}%
\end{pgfscope}%
\begin{pgfscope}%
\pgfpathrectangle{\pgfqpoint{3.793912in}{0.557870in}}{\pgfqpoint{2.446088in}{1.684734in}}%
\pgfusepath{clip}%
\pgfsetbuttcap%
\pgfsetroundjoin%
\definecolor{currentfill}{rgb}{0.298039,0.447059,0.690196}%
\pgfsetfillcolor{currentfill}%
\pgfsetlinewidth{1.003750pt}%
\definecolor{currentstroke}{rgb}{0.298039,0.447059,0.690196}%
\pgfsetstrokecolor{currentstroke}%
\pgfsetdash{}{0pt}%
\pgfpathmoveto{\pgfqpoint{5.208656in}{1.621386in}}%
\pgfpathcurveto{\pgfqpoint{5.216892in}{1.621386in}}{\pgfqpoint{5.224792in}{1.624659in}}{\pgfqpoint{5.230616in}{1.630483in}}%
\pgfpathcurveto{\pgfqpoint{5.236440in}{1.636307in}}{\pgfqpoint{5.239712in}{1.644207in}}{\pgfqpoint{5.239712in}{1.652443in}}%
\pgfpathcurveto{\pgfqpoint{5.239712in}{1.660679in}}{\pgfqpoint{5.236440in}{1.668579in}}{\pgfqpoint{5.230616in}{1.674403in}}%
\pgfpathcurveto{\pgfqpoint{5.224792in}{1.680227in}}{\pgfqpoint{5.216892in}{1.683499in}}{\pgfqpoint{5.208656in}{1.683499in}}%
\pgfpathcurveto{\pgfqpoint{5.200420in}{1.683499in}}{\pgfqpoint{5.192519in}{1.680227in}}{\pgfqpoint{5.186696in}{1.674403in}}%
\pgfpathcurveto{\pgfqpoint{5.180872in}{1.668579in}}{\pgfqpoint{5.177599in}{1.660679in}}{\pgfqpoint{5.177599in}{1.652443in}}%
\pgfpathcurveto{\pgfqpoint{5.177599in}{1.644207in}}{\pgfqpoint{5.180872in}{1.636307in}}{\pgfqpoint{5.186696in}{1.630483in}}%
\pgfpathcurveto{\pgfqpoint{5.192519in}{1.624659in}}{\pgfqpoint{5.200420in}{1.621386in}}{\pgfqpoint{5.208656in}{1.621386in}}%
\pgfpathclose%
\pgfusepath{stroke,fill}%
\end{pgfscope}%
\begin{pgfscope}%
\pgfpathrectangle{\pgfqpoint{3.793912in}{0.557870in}}{\pgfqpoint{2.446088in}{1.684734in}}%
\pgfusepath{clip}%
\pgfsetbuttcap%
\pgfsetroundjoin%
\definecolor{currentfill}{rgb}{0.298039,0.447059,0.690196}%
\pgfsetfillcolor{currentfill}%
\pgfsetlinewidth{1.003750pt}%
\definecolor{currentstroke}{rgb}{0.298039,0.447059,0.690196}%
\pgfsetstrokecolor{currentstroke}%
\pgfsetdash{}{0pt}%
\pgfpathmoveto{\pgfqpoint{3.905098in}{2.125798in}}%
\pgfpathcurveto{\pgfqpoint{3.913334in}{2.125798in}}{\pgfqpoint{3.921234in}{2.129070in}}{\pgfqpoint{3.927058in}{2.134894in}}%
\pgfpathcurveto{\pgfqpoint{3.932882in}{2.140718in}}{\pgfqpoint{3.936155in}{2.148618in}}{\pgfqpoint{3.936155in}{2.156854in}}%
\pgfpathcurveto{\pgfqpoint{3.936155in}{2.165091in}}{\pgfqpoint{3.932882in}{2.172991in}}{\pgfqpoint{3.927058in}{2.178814in}}%
\pgfpathcurveto{\pgfqpoint{3.921234in}{2.184638in}}{\pgfqpoint{3.913334in}{2.187911in}}{\pgfqpoint{3.905098in}{2.187911in}}%
\pgfpathcurveto{\pgfqpoint{3.896862in}{2.187911in}}{\pgfqpoint{3.888962in}{2.184638in}}{\pgfqpoint{3.883138in}{2.178814in}}%
\pgfpathcurveto{\pgfqpoint{3.877314in}{2.172991in}}{\pgfqpoint{3.874042in}{2.165091in}}{\pgfqpoint{3.874042in}{2.156854in}}%
\pgfpathcurveto{\pgfqpoint{3.874042in}{2.148618in}}{\pgfqpoint{3.877314in}{2.140718in}}{\pgfqpoint{3.883138in}{2.134894in}}%
\pgfpathcurveto{\pgfqpoint{3.888962in}{2.129070in}}{\pgfqpoint{3.896862in}{2.125798in}}{\pgfqpoint{3.905098in}{2.125798in}}%
\pgfpathclose%
\pgfusepath{stroke,fill}%
\end{pgfscope}%
\begin{pgfscope}%
\pgfpathrectangle{\pgfqpoint{3.793912in}{0.557870in}}{\pgfqpoint{2.446088in}{1.684734in}}%
\pgfusepath{clip}%
\pgfsetbuttcap%
\pgfsetroundjoin%
\definecolor{currentfill}{rgb}{0.298039,0.447059,0.690196}%
\pgfsetfillcolor{currentfill}%
\pgfsetlinewidth{1.003750pt}%
\definecolor{currentstroke}{rgb}{0.298039,0.447059,0.690196}%
\pgfsetstrokecolor{currentstroke}%
\pgfsetdash{}{0pt}%
\pgfpathmoveto{\pgfqpoint{3.905098in}{2.125798in}}%
\pgfpathcurveto{\pgfqpoint{3.913334in}{2.125798in}}{\pgfqpoint{3.921234in}{2.129070in}}{\pgfqpoint{3.927058in}{2.134894in}}%
\pgfpathcurveto{\pgfqpoint{3.932882in}{2.140718in}}{\pgfqpoint{3.936155in}{2.148618in}}{\pgfqpoint{3.936155in}{2.156854in}}%
\pgfpathcurveto{\pgfqpoint{3.936155in}{2.165091in}}{\pgfqpoint{3.932882in}{2.172991in}}{\pgfqpoint{3.927058in}{2.178814in}}%
\pgfpathcurveto{\pgfqpoint{3.921234in}{2.184638in}}{\pgfqpoint{3.913334in}{2.187911in}}{\pgfqpoint{3.905098in}{2.187911in}}%
\pgfpathcurveto{\pgfqpoint{3.896862in}{2.187911in}}{\pgfqpoint{3.888962in}{2.184638in}}{\pgfqpoint{3.883138in}{2.178814in}}%
\pgfpathcurveto{\pgfqpoint{3.877314in}{2.172991in}}{\pgfqpoint{3.874042in}{2.165091in}}{\pgfqpoint{3.874042in}{2.156854in}}%
\pgfpathcurveto{\pgfqpoint{3.874042in}{2.148618in}}{\pgfqpoint{3.877314in}{2.140718in}}{\pgfqpoint{3.883138in}{2.134894in}}%
\pgfpathcurveto{\pgfqpoint{3.888962in}{2.129070in}}{\pgfqpoint{3.896862in}{2.125798in}}{\pgfqpoint{3.905098in}{2.125798in}}%
\pgfpathclose%
\pgfusepath{stroke,fill}%
\end{pgfscope}%
\begin{pgfscope}%
\pgfpathrectangle{\pgfqpoint{3.793912in}{0.557870in}}{\pgfqpoint{2.446088in}{1.684734in}}%
\pgfusepath{clip}%
\pgfsetbuttcap%
\pgfsetroundjoin%
\definecolor{currentfill}{rgb}{0.298039,0.447059,0.690196}%
\pgfsetfillcolor{currentfill}%
\pgfsetlinewidth{1.003750pt}%
\definecolor{currentstroke}{rgb}{0.298039,0.447059,0.690196}%
\pgfsetstrokecolor{currentstroke}%
\pgfsetdash{}{0pt}%
\pgfpathmoveto{\pgfqpoint{3.905098in}{2.125798in}}%
\pgfpathcurveto{\pgfqpoint{3.913334in}{2.125798in}}{\pgfqpoint{3.921234in}{2.129070in}}{\pgfqpoint{3.927058in}{2.134894in}}%
\pgfpathcurveto{\pgfqpoint{3.932882in}{2.140718in}}{\pgfqpoint{3.936155in}{2.148618in}}{\pgfqpoint{3.936155in}{2.156854in}}%
\pgfpathcurveto{\pgfqpoint{3.936155in}{2.165091in}}{\pgfqpoint{3.932882in}{2.172991in}}{\pgfqpoint{3.927058in}{2.178814in}}%
\pgfpathcurveto{\pgfqpoint{3.921234in}{2.184638in}}{\pgfqpoint{3.913334in}{2.187911in}}{\pgfqpoint{3.905098in}{2.187911in}}%
\pgfpathcurveto{\pgfqpoint{3.896862in}{2.187911in}}{\pgfqpoint{3.888962in}{2.184638in}}{\pgfqpoint{3.883138in}{2.178814in}}%
\pgfpathcurveto{\pgfqpoint{3.877314in}{2.172991in}}{\pgfqpoint{3.874042in}{2.165091in}}{\pgfqpoint{3.874042in}{2.156854in}}%
\pgfpathcurveto{\pgfqpoint{3.874042in}{2.148618in}}{\pgfqpoint{3.877314in}{2.140718in}}{\pgfqpoint{3.883138in}{2.134894in}}%
\pgfpathcurveto{\pgfqpoint{3.888962in}{2.129070in}}{\pgfqpoint{3.896862in}{2.125798in}}{\pgfqpoint{3.905098in}{2.125798in}}%
\pgfpathclose%
\pgfusepath{stroke,fill}%
\end{pgfscope}%
\begin{pgfscope}%
\pgfpathrectangle{\pgfqpoint{3.793912in}{0.557870in}}{\pgfqpoint{2.446088in}{1.684734in}}%
\pgfusepath{clip}%
\pgfsetbuttcap%
\pgfsetroundjoin%
\definecolor{currentfill}{rgb}{0.298039,0.447059,0.690196}%
\pgfsetfillcolor{currentfill}%
\pgfsetlinewidth{1.003750pt}%
\definecolor{currentstroke}{rgb}{0.298039,0.447059,0.690196}%
\pgfsetstrokecolor{currentstroke}%
\pgfsetdash{}{0pt}%
\pgfpathmoveto{\pgfqpoint{4.058458in}{2.125798in}}%
\pgfpathcurveto{\pgfqpoint{4.066694in}{2.125798in}}{\pgfqpoint{4.074594in}{2.129070in}}{\pgfqpoint{4.080418in}{2.134894in}}%
\pgfpathcurveto{\pgfqpoint{4.086242in}{2.140718in}}{\pgfqpoint{4.089514in}{2.148618in}}{\pgfqpoint{4.089514in}{2.156854in}}%
\pgfpathcurveto{\pgfqpoint{4.089514in}{2.165091in}}{\pgfqpoint{4.086242in}{2.172991in}}{\pgfqpoint{4.080418in}{2.178814in}}%
\pgfpathcurveto{\pgfqpoint{4.074594in}{2.184638in}}{\pgfqpoint{4.066694in}{2.187911in}}{\pgfqpoint{4.058458in}{2.187911in}}%
\pgfpathcurveto{\pgfqpoint{4.050221in}{2.187911in}}{\pgfqpoint{4.042321in}{2.184638in}}{\pgfqpoint{4.036498in}{2.178814in}}%
\pgfpathcurveto{\pgfqpoint{4.030674in}{2.172991in}}{\pgfqpoint{4.027401in}{2.165091in}}{\pgfqpoint{4.027401in}{2.156854in}}%
\pgfpathcurveto{\pgfqpoint{4.027401in}{2.148618in}}{\pgfqpoint{4.030674in}{2.140718in}}{\pgfqpoint{4.036498in}{2.134894in}}%
\pgfpathcurveto{\pgfqpoint{4.042321in}{2.129070in}}{\pgfqpoint{4.050221in}{2.125798in}}{\pgfqpoint{4.058458in}{2.125798in}}%
\pgfpathclose%
\pgfusepath{stroke,fill}%
\end{pgfscope}%
\begin{pgfscope}%
\pgfpathrectangle{\pgfqpoint{3.793912in}{0.557870in}}{\pgfqpoint{2.446088in}{1.684734in}}%
\pgfusepath{clip}%
\pgfsetbuttcap%
\pgfsetroundjoin%
\definecolor{currentfill}{rgb}{0.298039,0.447059,0.690196}%
\pgfsetfillcolor{currentfill}%
\pgfsetlinewidth{1.003750pt}%
\definecolor{currentstroke}{rgb}{0.298039,0.447059,0.690196}%
\pgfsetstrokecolor{currentstroke}%
\pgfsetdash{}{0pt}%
\pgfpathmoveto{\pgfqpoint{3.905098in}{2.125798in}}%
\pgfpathcurveto{\pgfqpoint{3.913334in}{2.125798in}}{\pgfqpoint{3.921234in}{2.129070in}}{\pgfqpoint{3.927058in}{2.134894in}}%
\pgfpathcurveto{\pgfqpoint{3.932882in}{2.140718in}}{\pgfqpoint{3.936155in}{2.148618in}}{\pgfqpoint{3.936155in}{2.156854in}}%
\pgfpathcurveto{\pgfqpoint{3.936155in}{2.165091in}}{\pgfqpoint{3.932882in}{2.172991in}}{\pgfqpoint{3.927058in}{2.178814in}}%
\pgfpathcurveto{\pgfqpoint{3.921234in}{2.184638in}}{\pgfqpoint{3.913334in}{2.187911in}}{\pgfqpoint{3.905098in}{2.187911in}}%
\pgfpathcurveto{\pgfqpoint{3.896862in}{2.187911in}}{\pgfqpoint{3.888962in}{2.184638in}}{\pgfqpoint{3.883138in}{2.178814in}}%
\pgfpathcurveto{\pgfqpoint{3.877314in}{2.172991in}}{\pgfqpoint{3.874042in}{2.165091in}}{\pgfqpoint{3.874042in}{2.156854in}}%
\pgfpathcurveto{\pgfqpoint{3.874042in}{2.148618in}}{\pgfqpoint{3.877314in}{2.140718in}}{\pgfqpoint{3.883138in}{2.134894in}}%
\pgfpathcurveto{\pgfqpoint{3.888962in}{2.129070in}}{\pgfqpoint{3.896862in}{2.125798in}}{\pgfqpoint{3.905098in}{2.125798in}}%
\pgfpathclose%
\pgfusepath{stroke,fill}%
\end{pgfscope}%
\begin{pgfscope}%
\pgfpathrectangle{\pgfqpoint{3.793912in}{0.557870in}}{\pgfqpoint{2.446088in}{1.684734in}}%
\pgfusepath{clip}%
\pgfsetbuttcap%
\pgfsetroundjoin%
\definecolor{currentfill}{rgb}{0.298039,0.447059,0.690196}%
\pgfsetfillcolor{currentfill}%
\pgfsetlinewidth{1.003750pt}%
\definecolor{currentstroke}{rgb}{0.298039,0.447059,0.690196}%
\pgfsetstrokecolor{currentstroke}%
\pgfsetdash{}{0pt}%
\pgfpathmoveto{\pgfqpoint{4.671897in}{1.703926in}}%
\pgfpathcurveto{\pgfqpoint{4.680133in}{1.703926in}}{\pgfqpoint{4.688033in}{1.707199in}}{\pgfqpoint{4.693857in}{1.713023in}}%
\pgfpathcurveto{\pgfqpoint{4.699681in}{1.718847in}}{\pgfqpoint{4.702953in}{1.726747in}}{\pgfqpoint{4.702953in}{1.734983in}}%
\pgfpathcurveto{\pgfqpoint{4.702953in}{1.743219in}}{\pgfqpoint{4.699681in}{1.751119in}}{\pgfqpoint{4.693857in}{1.756943in}}%
\pgfpathcurveto{\pgfqpoint{4.688033in}{1.762767in}}{\pgfqpoint{4.680133in}{1.766039in}}{\pgfqpoint{4.671897in}{1.766039in}}%
\pgfpathcurveto{\pgfqpoint{4.663660in}{1.766039in}}{\pgfqpoint{4.655760in}{1.762767in}}{\pgfqpoint{4.649936in}{1.756943in}}%
\pgfpathcurveto{\pgfqpoint{4.644113in}{1.751119in}}{\pgfqpoint{4.640840in}{1.743219in}}{\pgfqpoint{4.640840in}{1.734983in}}%
\pgfpathcurveto{\pgfqpoint{4.640840in}{1.726747in}}{\pgfqpoint{4.644113in}{1.718847in}}{\pgfqpoint{4.649936in}{1.713023in}}%
\pgfpathcurveto{\pgfqpoint{4.655760in}{1.707199in}}{\pgfqpoint{4.663660in}{1.703926in}}{\pgfqpoint{4.671897in}{1.703926in}}%
\pgfpathclose%
\pgfusepath{stroke,fill}%
\end{pgfscope}%
\begin{pgfscope}%
\pgfpathrectangle{\pgfqpoint{3.793912in}{0.557870in}}{\pgfqpoint{2.446088in}{1.684734in}}%
\pgfusepath{clip}%
\pgfsetbuttcap%
\pgfsetroundjoin%
\definecolor{currentfill}{rgb}{0.298039,0.447059,0.690196}%
\pgfsetfillcolor{currentfill}%
\pgfsetlinewidth{1.003750pt}%
\definecolor{currentstroke}{rgb}{0.298039,0.447059,0.690196}%
\pgfsetstrokecolor{currentstroke}%
\pgfsetdash{}{0pt}%
\pgfpathmoveto{\pgfqpoint{3.905098in}{2.125798in}}%
\pgfpathcurveto{\pgfqpoint{3.913334in}{2.125798in}}{\pgfqpoint{3.921234in}{2.129070in}}{\pgfqpoint{3.927058in}{2.134894in}}%
\pgfpathcurveto{\pgfqpoint{3.932882in}{2.140718in}}{\pgfqpoint{3.936155in}{2.148618in}}{\pgfqpoint{3.936155in}{2.156854in}}%
\pgfpathcurveto{\pgfqpoint{3.936155in}{2.165091in}}{\pgfqpoint{3.932882in}{2.172991in}}{\pgfqpoint{3.927058in}{2.178814in}}%
\pgfpathcurveto{\pgfqpoint{3.921234in}{2.184638in}}{\pgfqpoint{3.913334in}{2.187911in}}{\pgfqpoint{3.905098in}{2.187911in}}%
\pgfpathcurveto{\pgfqpoint{3.896862in}{2.187911in}}{\pgfqpoint{3.888962in}{2.184638in}}{\pgfqpoint{3.883138in}{2.178814in}}%
\pgfpathcurveto{\pgfqpoint{3.877314in}{2.172991in}}{\pgfqpoint{3.874042in}{2.165091in}}{\pgfqpoint{3.874042in}{2.156854in}}%
\pgfpathcurveto{\pgfqpoint{3.874042in}{2.148618in}}{\pgfqpoint{3.877314in}{2.140718in}}{\pgfqpoint{3.883138in}{2.134894in}}%
\pgfpathcurveto{\pgfqpoint{3.888962in}{2.129070in}}{\pgfqpoint{3.896862in}{2.125798in}}{\pgfqpoint{3.905098in}{2.125798in}}%
\pgfpathclose%
\pgfusepath{stroke,fill}%
\end{pgfscope}%
\begin{pgfscope}%
\pgfpathrectangle{\pgfqpoint{3.793912in}{0.557870in}}{\pgfqpoint{2.446088in}{1.684734in}}%
\pgfusepath{clip}%
\pgfsetbuttcap%
\pgfsetroundjoin%
\definecolor{currentfill}{rgb}{0.298039,0.447059,0.690196}%
\pgfsetfillcolor{currentfill}%
\pgfsetlinewidth{1.003750pt}%
\definecolor{currentstroke}{rgb}{0.298039,0.447059,0.690196}%
\pgfsetstrokecolor{currentstroke}%
\pgfsetdash{}{0pt}%
\pgfpathmoveto{\pgfqpoint{4.058458in}{2.116627in}}%
\pgfpathcurveto{\pgfqpoint{4.066694in}{2.116627in}}{\pgfqpoint{4.074594in}{2.119899in}}{\pgfqpoint{4.080418in}{2.125723in}}%
\pgfpathcurveto{\pgfqpoint{4.086242in}{2.131547in}}{\pgfqpoint{4.089514in}{2.139447in}}{\pgfqpoint{4.089514in}{2.147683in}}%
\pgfpathcurveto{\pgfqpoint{4.089514in}{2.155919in}}{\pgfqpoint{4.086242in}{2.163819in}}{\pgfqpoint{4.080418in}{2.169643in}}%
\pgfpathcurveto{\pgfqpoint{4.074594in}{2.175467in}}{\pgfqpoint{4.066694in}{2.178740in}}{\pgfqpoint{4.058458in}{2.178740in}}%
\pgfpathcurveto{\pgfqpoint{4.050221in}{2.178740in}}{\pgfqpoint{4.042321in}{2.175467in}}{\pgfqpoint{4.036498in}{2.169643in}}%
\pgfpathcurveto{\pgfqpoint{4.030674in}{2.163819in}}{\pgfqpoint{4.027401in}{2.155919in}}{\pgfqpoint{4.027401in}{2.147683in}}%
\pgfpathcurveto{\pgfqpoint{4.027401in}{2.139447in}}{\pgfqpoint{4.030674in}{2.131547in}}{\pgfqpoint{4.036498in}{2.125723in}}%
\pgfpathcurveto{\pgfqpoint{4.042321in}{2.119899in}}{\pgfqpoint{4.050221in}{2.116627in}}{\pgfqpoint{4.058458in}{2.116627in}}%
\pgfpathclose%
\pgfusepath{stroke,fill}%
\end{pgfscope}%
\begin{pgfscope}%
\pgfpathrectangle{\pgfqpoint{3.793912in}{0.557870in}}{\pgfqpoint{2.446088in}{1.684734in}}%
\pgfusepath{clip}%
\pgfsetbuttcap%
\pgfsetroundjoin%
\definecolor{currentfill}{rgb}{0.298039,0.447059,0.690196}%
\pgfsetfillcolor{currentfill}%
\pgfsetlinewidth{1.003750pt}%
\definecolor{currentstroke}{rgb}{0.298039,0.447059,0.690196}%
\pgfsetstrokecolor{currentstroke}%
\pgfsetdash{}{0pt}%
\pgfpathmoveto{\pgfqpoint{3.905098in}{2.125798in}}%
\pgfpathcurveto{\pgfqpoint{3.913334in}{2.125798in}}{\pgfqpoint{3.921234in}{2.129070in}}{\pgfqpoint{3.927058in}{2.134894in}}%
\pgfpathcurveto{\pgfqpoint{3.932882in}{2.140718in}}{\pgfqpoint{3.936155in}{2.148618in}}{\pgfqpoint{3.936155in}{2.156854in}}%
\pgfpathcurveto{\pgfqpoint{3.936155in}{2.165091in}}{\pgfqpoint{3.932882in}{2.172991in}}{\pgfqpoint{3.927058in}{2.178814in}}%
\pgfpathcurveto{\pgfqpoint{3.921234in}{2.184638in}}{\pgfqpoint{3.913334in}{2.187911in}}{\pgfqpoint{3.905098in}{2.187911in}}%
\pgfpathcurveto{\pgfqpoint{3.896862in}{2.187911in}}{\pgfqpoint{3.888962in}{2.184638in}}{\pgfqpoint{3.883138in}{2.178814in}}%
\pgfpathcurveto{\pgfqpoint{3.877314in}{2.172991in}}{\pgfqpoint{3.874042in}{2.165091in}}{\pgfqpoint{3.874042in}{2.156854in}}%
\pgfpathcurveto{\pgfqpoint{3.874042in}{2.148618in}}{\pgfqpoint{3.877314in}{2.140718in}}{\pgfqpoint{3.883138in}{2.134894in}}%
\pgfpathcurveto{\pgfqpoint{3.888962in}{2.129070in}}{\pgfqpoint{3.896862in}{2.125798in}}{\pgfqpoint{3.905098in}{2.125798in}}%
\pgfpathclose%
\pgfusepath{stroke,fill}%
\end{pgfscope}%
\begin{pgfscope}%
\pgfpathrectangle{\pgfqpoint{3.793912in}{0.557870in}}{\pgfqpoint{2.446088in}{1.684734in}}%
\pgfusepath{clip}%
\pgfsetbuttcap%
\pgfsetroundjoin%
\definecolor{currentfill}{rgb}{0.298039,0.447059,0.690196}%
\pgfsetfillcolor{currentfill}%
\pgfsetlinewidth{1.003750pt}%
\definecolor{currentstroke}{rgb}{0.298039,0.447059,0.690196}%
\pgfsetstrokecolor{currentstroke}%
\pgfsetdash{}{0pt}%
\pgfpathmoveto{\pgfqpoint{5.975454in}{1.511333in}}%
\pgfpathcurveto{\pgfqpoint{5.983691in}{1.511333in}}{\pgfqpoint{5.991591in}{1.514605in}}{\pgfqpoint{5.997415in}{1.520429in}}%
\pgfpathcurveto{\pgfqpoint{6.003239in}{1.526253in}}{\pgfqpoint{6.006511in}{1.534153in}}{\pgfqpoint{6.006511in}{1.542390in}}%
\pgfpathcurveto{\pgfqpoint{6.006511in}{1.550626in}}{\pgfqpoint{6.003239in}{1.558526in}}{\pgfqpoint{5.997415in}{1.564350in}}%
\pgfpathcurveto{\pgfqpoint{5.991591in}{1.570174in}}{\pgfqpoint{5.983691in}{1.573446in}}{\pgfqpoint{5.975454in}{1.573446in}}%
\pgfpathcurveto{\pgfqpoint{5.967218in}{1.573446in}}{\pgfqpoint{5.959318in}{1.570174in}}{\pgfqpoint{5.953494in}{1.564350in}}%
\pgfpathcurveto{\pgfqpoint{5.947670in}{1.558526in}}{\pgfqpoint{5.944398in}{1.550626in}}{\pgfqpoint{5.944398in}{1.542390in}}%
\pgfpathcurveto{\pgfqpoint{5.944398in}{1.534153in}}{\pgfqpoint{5.947670in}{1.526253in}}{\pgfqpoint{5.953494in}{1.520429in}}%
\pgfpathcurveto{\pgfqpoint{5.959318in}{1.514605in}}{\pgfqpoint{5.967218in}{1.511333in}}{\pgfqpoint{5.975454in}{1.511333in}}%
\pgfpathclose%
\pgfusepath{stroke,fill}%
\end{pgfscope}%
\begin{pgfscope}%
\pgfpathrectangle{\pgfqpoint{3.793912in}{0.557870in}}{\pgfqpoint{2.446088in}{1.684734in}}%
\pgfusepath{clip}%
\pgfsetbuttcap%
\pgfsetroundjoin%
\definecolor{currentfill}{rgb}{0.298039,0.447059,0.690196}%
\pgfsetfillcolor{currentfill}%
\pgfsetlinewidth{1.003750pt}%
\definecolor{currentstroke}{rgb}{0.298039,0.447059,0.690196}%
\pgfsetstrokecolor{currentstroke}%
\pgfsetdash{}{0pt}%
\pgfpathmoveto{\pgfqpoint{3.905098in}{2.125798in}}%
\pgfpathcurveto{\pgfqpoint{3.913334in}{2.125798in}}{\pgfqpoint{3.921234in}{2.129070in}}{\pgfqpoint{3.927058in}{2.134894in}}%
\pgfpathcurveto{\pgfqpoint{3.932882in}{2.140718in}}{\pgfqpoint{3.936155in}{2.148618in}}{\pgfqpoint{3.936155in}{2.156854in}}%
\pgfpathcurveto{\pgfqpoint{3.936155in}{2.165091in}}{\pgfqpoint{3.932882in}{2.172991in}}{\pgfqpoint{3.927058in}{2.178814in}}%
\pgfpathcurveto{\pgfqpoint{3.921234in}{2.184638in}}{\pgfqpoint{3.913334in}{2.187911in}}{\pgfqpoint{3.905098in}{2.187911in}}%
\pgfpathcurveto{\pgfqpoint{3.896862in}{2.187911in}}{\pgfqpoint{3.888962in}{2.184638in}}{\pgfqpoint{3.883138in}{2.178814in}}%
\pgfpathcurveto{\pgfqpoint{3.877314in}{2.172991in}}{\pgfqpoint{3.874042in}{2.165091in}}{\pgfqpoint{3.874042in}{2.156854in}}%
\pgfpathcurveto{\pgfqpoint{3.874042in}{2.148618in}}{\pgfqpoint{3.877314in}{2.140718in}}{\pgfqpoint{3.883138in}{2.134894in}}%
\pgfpathcurveto{\pgfqpoint{3.888962in}{2.129070in}}{\pgfqpoint{3.896862in}{2.125798in}}{\pgfqpoint{3.905098in}{2.125798in}}%
\pgfpathclose%
\pgfusepath{stroke,fill}%
\end{pgfscope}%
\begin{pgfscope}%
\pgfpathrectangle{\pgfqpoint{3.793912in}{0.557870in}}{\pgfqpoint{2.446088in}{1.684734in}}%
\pgfusepath{clip}%
\pgfsetbuttcap%
\pgfsetroundjoin%
\definecolor{currentfill}{rgb}{0.298039,0.447059,0.690196}%
\pgfsetfillcolor{currentfill}%
\pgfsetlinewidth{1.003750pt}%
\definecolor{currentstroke}{rgb}{0.298039,0.447059,0.690196}%
\pgfsetstrokecolor{currentstroke}%
\pgfsetdash{}{0pt}%
\pgfpathmoveto{\pgfqpoint{5.975454in}{1.621386in}}%
\pgfpathcurveto{\pgfqpoint{5.983691in}{1.621386in}}{\pgfqpoint{5.991591in}{1.624659in}}{\pgfqpoint{5.997415in}{1.630483in}}%
\pgfpathcurveto{\pgfqpoint{6.003239in}{1.636307in}}{\pgfqpoint{6.006511in}{1.644207in}}{\pgfqpoint{6.006511in}{1.652443in}}%
\pgfpathcurveto{\pgfqpoint{6.006511in}{1.660679in}}{\pgfqpoint{6.003239in}{1.668579in}}{\pgfqpoint{5.997415in}{1.674403in}}%
\pgfpathcurveto{\pgfqpoint{5.991591in}{1.680227in}}{\pgfqpoint{5.983691in}{1.683499in}}{\pgfqpoint{5.975454in}{1.683499in}}%
\pgfpathcurveto{\pgfqpoint{5.967218in}{1.683499in}}{\pgfqpoint{5.959318in}{1.680227in}}{\pgfqpoint{5.953494in}{1.674403in}}%
\pgfpathcurveto{\pgfqpoint{5.947670in}{1.668579in}}{\pgfqpoint{5.944398in}{1.660679in}}{\pgfqpoint{5.944398in}{1.652443in}}%
\pgfpathcurveto{\pgfqpoint{5.944398in}{1.644207in}}{\pgfqpoint{5.947670in}{1.636307in}}{\pgfqpoint{5.953494in}{1.630483in}}%
\pgfpathcurveto{\pgfqpoint{5.959318in}{1.624659in}}{\pgfqpoint{5.967218in}{1.621386in}}{\pgfqpoint{5.975454in}{1.621386in}}%
\pgfpathclose%
\pgfusepath{stroke,fill}%
\end{pgfscope}%
\begin{pgfscope}%
\pgfpathrectangle{\pgfqpoint{3.793912in}{0.557870in}}{\pgfqpoint{2.446088in}{1.684734in}}%
\pgfusepath{clip}%
\pgfsetbuttcap%
\pgfsetroundjoin%
\definecolor{currentfill}{rgb}{0.298039,0.447059,0.690196}%
\pgfsetfillcolor{currentfill}%
\pgfsetlinewidth{1.003750pt}%
\definecolor{currentstroke}{rgb}{0.298039,0.447059,0.690196}%
\pgfsetstrokecolor{currentstroke}%
\pgfsetdash{}{0pt}%
\pgfpathmoveto{\pgfqpoint{3.905098in}{2.125798in}}%
\pgfpathcurveto{\pgfqpoint{3.913334in}{2.125798in}}{\pgfqpoint{3.921234in}{2.129070in}}{\pgfqpoint{3.927058in}{2.134894in}}%
\pgfpathcurveto{\pgfqpoint{3.932882in}{2.140718in}}{\pgfqpoint{3.936155in}{2.148618in}}{\pgfqpoint{3.936155in}{2.156854in}}%
\pgfpathcurveto{\pgfqpoint{3.936155in}{2.165091in}}{\pgfqpoint{3.932882in}{2.172991in}}{\pgfqpoint{3.927058in}{2.178814in}}%
\pgfpathcurveto{\pgfqpoint{3.921234in}{2.184638in}}{\pgfqpoint{3.913334in}{2.187911in}}{\pgfqpoint{3.905098in}{2.187911in}}%
\pgfpathcurveto{\pgfqpoint{3.896862in}{2.187911in}}{\pgfqpoint{3.888962in}{2.184638in}}{\pgfqpoint{3.883138in}{2.178814in}}%
\pgfpathcurveto{\pgfqpoint{3.877314in}{2.172991in}}{\pgfqpoint{3.874042in}{2.165091in}}{\pgfqpoint{3.874042in}{2.156854in}}%
\pgfpathcurveto{\pgfqpoint{3.874042in}{2.148618in}}{\pgfqpoint{3.877314in}{2.140718in}}{\pgfqpoint{3.883138in}{2.134894in}}%
\pgfpathcurveto{\pgfqpoint{3.888962in}{2.129070in}}{\pgfqpoint{3.896862in}{2.125798in}}{\pgfqpoint{3.905098in}{2.125798in}}%
\pgfpathclose%
\pgfusepath{stroke,fill}%
\end{pgfscope}%
\begin{pgfscope}%
\pgfpathrectangle{\pgfqpoint{3.793912in}{0.557870in}}{\pgfqpoint{2.446088in}{1.684734in}}%
\pgfusepath{clip}%
\pgfsetbuttcap%
\pgfsetroundjoin%
\definecolor{currentfill}{rgb}{0.298039,0.447059,0.690196}%
\pgfsetfillcolor{currentfill}%
\pgfsetlinewidth{1.003750pt}%
\definecolor{currentstroke}{rgb}{0.298039,0.447059,0.690196}%
\pgfsetstrokecolor{currentstroke}%
\pgfsetdash{}{0pt}%
\pgfpathmoveto{\pgfqpoint{3.905098in}{2.125798in}}%
\pgfpathcurveto{\pgfqpoint{3.913334in}{2.125798in}}{\pgfqpoint{3.921234in}{2.129070in}}{\pgfqpoint{3.927058in}{2.134894in}}%
\pgfpathcurveto{\pgfqpoint{3.932882in}{2.140718in}}{\pgfqpoint{3.936155in}{2.148618in}}{\pgfqpoint{3.936155in}{2.156854in}}%
\pgfpathcurveto{\pgfqpoint{3.936155in}{2.165091in}}{\pgfqpoint{3.932882in}{2.172991in}}{\pgfqpoint{3.927058in}{2.178814in}}%
\pgfpathcurveto{\pgfqpoint{3.921234in}{2.184638in}}{\pgfqpoint{3.913334in}{2.187911in}}{\pgfqpoint{3.905098in}{2.187911in}}%
\pgfpathcurveto{\pgfqpoint{3.896862in}{2.187911in}}{\pgfqpoint{3.888962in}{2.184638in}}{\pgfqpoint{3.883138in}{2.178814in}}%
\pgfpathcurveto{\pgfqpoint{3.877314in}{2.172991in}}{\pgfqpoint{3.874042in}{2.165091in}}{\pgfqpoint{3.874042in}{2.156854in}}%
\pgfpathcurveto{\pgfqpoint{3.874042in}{2.148618in}}{\pgfqpoint{3.877314in}{2.140718in}}{\pgfqpoint{3.883138in}{2.134894in}}%
\pgfpathcurveto{\pgfqpoint{3.888962in}{2.129070in}}{\pgfqpoint{3.896862in}{2.125798in}}{\pgfqpoint{3.905098in}{2.125798in}}%
\pgfpathclose%
\pgfusepath{stroke,fill}%
\end{pgfscope}%
\begin{pgfscope}%
\pgfpathrectangle{\pgfqpoint{3.793912in}{0.557870in}}{\pgfqpoint{2.446088in}{1.684734in}}%
\pgfusepath{clip}%
\pgfsetbuttcap%
\pgfsetroundjoin%
\definecolor{currentfill}{rgb}{0.298039,0.447059,0.690196}%
\pgfsetfillcolor{currentfill}%
\pgfsetlinewidth{1.003750pt}%
\definecolor{currentstroke}{rgb}{0.298039,0.447059,0.690196}%
\pgfsetstrokecolor{currentstroke}%
\pgfsetdash{}{0pt}%
\pgfpathmoveto{\pgfqpoint{3.905098in}{2.125798in}}%
\pgfpathcurveto{\pgfqpoint{3.913334in}{2.125798in}}{\pgfqpoint{3.921234in}{2.129070in}}{\pgfqpoint{3.927058in}{2.134894in}}%
\pgfpathcurveto{\pgfqpoint{3.932882in}{2.140718in}}{\pgfqpoint{3.936155in}{2.148618in}}{\pgfqpoint{3.936155in}{2.156854in}}%
\pgfpathcurveto{\pgfqpoint{3.936155in}{2.165091in}}{\pgfqpoint{3.932882in}{2.172991in}}{\pgfqpoint{3.927058in}{2.178814in}}%
\pgfpathcurveto{\pgfqpoint{3.921234in}{2.184638in}}{\pgfqpoint{3.913334in}{2.187911in}}{\pgfqpoint{3.905098in}{2.187911in}}%
\pgfpathcurveto{\pgfqpoint{3.896862in}{2.187911in}}{\pgfqpoint{3.888962in}{2.184638in}}{\pgfqpoint{3.883138in}{2.178814in}}%
\pgfpathcurveto{\pgfqpoint{3.877314in}{2.172991in}}{\pgfqpoint{3.874042in}{2.165091in}}{\pgfqpoint{3.874042in}{2.156854in}}%
\pgfpathcurveto{\pgfqpoint{3.874042in}{2.148618in}}{\pgfqpoint{3.877314in}{2.140718in}}{\pgfqpoint{3.883138in}{2.134894in}}%
\pgfpathcurveto{\pgfqpoint{3.888962in}{2.129070in}}{\pgfqpoint{3.896862in}{2.125798in}}{\pgfqpoint{3.905098in}{2.125798in}}%
\pgfpathclose%
\pgfusepath{stroke,fill}%
\end{pgfscope}%
\begin{pgfscope}%
\pgfpathrectangle{\pgfqpoint{3.793912in}{0.557870in}}{\pgfqpoint{2.446088in}{1.684734in}}%
\pgfusepath{clip}%
\pgfsetbuttcap%
\pgfsetroundjoin%
\definecolor{currentfill}{rgb}{0.298039,0.447059,0.690196}%
\pgfsetfillcolor{currentfill}%
\pgfsetlinewidth{1.003750pt}%
\definecolor{currentstroke}{rgb}{0.298039,0.447059,0.690196}%
\pgfsetstrokecolor{currentstroke}%
\pgfsetdash{}{0pt}%
\pgfpathmoveto{\pgfqpoint{5.745415in}{1.584702in}}%
\pgfpathcurveto{\pgfqpoint{5.753651in}{1.584702in}}{\pgfqpoint{5.761551in}{1.587974in}}{\pgfqpoint{5.767375in}{1.593798in}}%
\pgfpathcurveto{\pgfqpoint{5.773199in}{1.599622in}}{\pgfqpoint{5.776471in}{1.607522in}}{\pgfqpoint{5.776471in}{1.615758in}}%
\pgfpathcurveto{\pgfqpoint{5.776471in}{1.623995in}}{\pgfqpoint{5.773199in}{1.631895in}}{\pgfqpoint{5.767375in}{1.637719in}}%
\pgfpathcurveto{\pgfqpoint{5.761551in}{1.643543in}}{\pgfqpoint{5.753651in}{1.646815in}}{\pgfqpoint{5.745415in}{1.646815in}}%
\pgfpathcurveto{\pgfqpoint{5.737179in}{1.646815in}}{\pgfqpoint{5.729279in}{1.643543in}}{\pgfqpoint{5.723455in}{1.637719in}}%
\pgfpathcurveto{\pgfqpoint{5.717631in}{1.631895in}}{\pgfqpoint{5.714358in}{1.623995in}}{\pgfqpoint{5.714358in}{1.615758in}}%
\pgfpathcurveto{\pgfqpoint{5.714358in}{1.607522in}}{\pgfqpoint{5.717631in}{1.599622in}}{\pgfqpoint{5.723455in}{1.593798in}}%
\pgfpathcurveto{\pgfqpoint{5.729279in}{1.587974in}}{\pgfqpoint{5.737179in}{1.584702in}}{\pgfqpoint{5.745415in}{1.584702in}}%
\pgfpathclose%
\pgfusepath{stroke,fill}%
\end{pgfscope}%
\begin{pgfscope}%
\pgfpathrectangle{\pgfqpoint{3.793912in}{0.557870in}}{\pgfqpoint{2.446088in}{1.684734in}}%
\pgfusepath{clip}%
\pgfsetbuttcap%
\pgfsetroundjoin%
\definecolor{currentfill}{rgb}{0.298039,0.447059,0.690196}%
\pgfsetfillcolor{currentfill}%
\pgfsetlinewidth{1.003750pt}%
\definecolor{currentstroke}{rgb}{0.298039,0.447059,0.690196}%
\pgfsetstrokecolor{currentstroke}%
\pgfsetdash{}{0pt}%
\pgfpathmoveto{\pgfqpoint{5.975454in}{1.318740in}}%
\pgfpathcurveto{\pgfqpoint{5.983691in}{1.318740in}}{\pgfqpoint{5.991591in}{1.322012in}}{\pgfqpoint{5.997415in}{1.327836in}}%
\pgfpathcurveto{\pgfqpoint{6.003239in}{1.333660in}}{\pgfqpoint{6.006511in}{1.341560in}}{\pgfqpoint{6.006511in}{1.349796in}}%
\pgfpathcurveto{\pgfqpoint{6.006511in}{1.358032in}}{\pgfqpoint{6.003239in}{1.365932in}}{\pgfqpoint{5.997415in}{1.371756in}}%
\pgfpathcurveto{\pgfqpoint{5.991591in}{1.377580in}}{\pgfqpoint{5.983691in}{1.380853in}}{\pgfqpoint{5.975454in}{1.380853in}}%
\pgfpathcurveto{\pgfqpoint{5.967218in}{1.380853in}}{\pgfqpoint{5.959318in}{1.377580in}}{\pgfqpoint{5.953494in}{1.371756in}}%
\pgfpathcurveto{\pgfqpoint{5.947670in}{1.365932in}}{\pgfqpoint{5.944398in}{1.358032in}}{\pgfqpoint{5.944398in}{1.349796in}}%
\pgfpathcurveto{\pgfqpoint{5.944398in}{1.341560in}}{\pgfqpoint{5.947670in}{1.333660in}}{\pgfqpoint{5.953494in}{1.327836in}}%
\pgfpathcurveto{\pgfqpoint{5.959318in}{1.322012in}}{\pgfqpoint{5.967218in}{1.318740in}}{\pgfqpoint{5.975454in}{1.318740in}}%
\pgfpathclose%
\pgfusepath{stroke,fill}%
\end{pgfscope}%
\begin{pgfscope}%
\pgfpathrectangle{\pgfqpoint{3.793912in}{0.557870in}}{\pgfqpoint{2.446088in}{1.684734in}}%
\pgfusepath{clip}%
\pgfsetbuttcap%
\pgfsetroundjoin%
\definecolor{currentfill}{rgb}{0.298039,0.447059,0.690196}%
\pgfsetfillcolor{currentfill}%
\pgfsetlinewidth{1.003750pt}%
\definecolor{currentstroke}{rgb}{0.298039,0.447059,0.690196}%
\pgfsetstrokecolor{currentstroke}%
\pgfsetdash{}{0pt}%
\pgfpathmoveto{\pgfqpoint{5.975454in}{1.282055in}}%
\pgfpathcurveto{\pgfqpoint{5.983691in}{1.282055in}}{\pgfqpoint{5.991591in}{1.285327in}}{\pgfqpoint{5.997415in}{1.291151in}}%
\pgfpathcurveto{\pgfqpoint{6.003239in}{1.296975in}}{\pgfqpoint{6.006511in}{1.304875in}}{\pgfqpoint{6.006511in}{1.313112in}}%
\pgfpathcurveto{\pgfqpoint{6.006511in}{1.321348in}}{\pgfqpoint{6.003239in}{1.329248in}}{\pgfqpoint{5.997415in}{1.335072in}}%
\pgfpathcurveto{\pgfqpoint{5.991591in}{1.340896in}}{\pgfqpoint{5.983691in}{1.344168in}}{\pgfqpoint{5.975454in}{1.344168in}}%
\pgfpathcurveto{\pgfqpoint{5.967218in}{1.344168in}}{\pgfqpoint{5.959318in}{1.340896in}}{\pgfqpoint{5.953494in}{1.335072in}}%
\pgfpathcurveto{\pgfqpoint{5.947670in}{1.329248in}}{\pgfqpoint{5.944398in}{1.321348in}}{\pgfqpoint{5.944398in}{1.313112in}}%
\pgfpathcurveto{\pgfqpoint{5.944398in}{1.304875in}}{\pgfqpoint{5.947670in}{1.296975in}}{\pgfqpoint{5.953494in}{1.291151in}}%
\pgfpathcurveto{\pgfqpoint{5.959318in}{1.285327in}}{\pgfqpoint{5.967218in}{1.282055in}}{\pgfqpoint{5.975454in}{1.282055in}}%
\pgfpathclose%
\pgfusepath{stroke,fill}%
\end{pgfscope}%
\begin{pgfscope}%
\pgfpathrectangle{\pgfqpoint{3.793912in}{0.557870in}}{\pgfqpoint{2.446088in}{1.684734in}}%
\pgfusepath{clip}%
\pgfsetbuttcap%
\pgfsetroundjoin%
\definecolor{currentfill}{rgb}{0.298039,0.447059,0.690196}%
\pgfsetfillcolor{currentfill}%
\pgfsetlinewidth{1.003750pt}%
\definecolor{currentstroke}{rgb}{0.298039,0.447059,0.690196}%
\pgfsetstrokecolor{currentstroke}%
\pgfsetdash{}{0pt}%
\pgfpathmoveto{\pgfqpoint{4.058458in}{1.511333in}}%
\pgfpathcurveto{\pgfqpoint{4.066694in}{1.511333in}}{\pgfqpoint{4.074594in}{1.514605in}}{\pgfqpoint{4.080418in}{1.520429in}}%
\pgfpathcurveto{\pgfqpoint{4.086242in}{1.526253in}}{\pgfqpoint{4.089514in}{1.534153in}}{\pgfqpoint{4.089514in}{1.542390in}}%
\pgfpathcurveto{\pgfqpoint{4.089514in}{1.550626in}}{\pgfqpoint{4.086242in}{1.558526in}}{\pgfqpoint{4.080418in}{1.564350in}}%
\pgfpathcurveto{\pgfqpoint{4.074594in}{1.570174in}}{\pgfqpoint{4.066694in}{1.573446in}}{\pgfqpoint{4.058458in}{1.573446in}}%
\pgfpathcurveto{\pgfqpoint{4.050221in}{1.573446in}}{\pgfqpoint{4.042321in}{1.570174in}}{\pgfqpoint{4.036498in}{1.564350in}}%
\pgfpathcurveto{\pgfqpoint{4.030674in}{1.558526in}}{\pgfqpoint{4.027401in}{1.550626in}}{\pgfqpoint{4.027401in}{1.542390in}}%
\pgfpathcurveto{\pgfqpoint{4.027401in}{1.534153in}}{\pgfqpoint{4.030674in}{1.526253in}}{\pgfqpoint{4.036498in}{1.520429in}}%
\pgfpathcurveto{\pgfqpoint{4.042321in}{1.514605in}}{\pgfqpoint{4.050221in}{1.511333in}}{\pgfqpoint{4.058458in}{1.511333in}}%
\pgfpathclose%
\pgfusepath{stroke,fill}%
\end{pgfscope}%
\begin{pgfscope}%
\pgfpathrectangle{\pgfqpoint{3.793912in}{0.557870in}}{\pgfqpoint{2.446088in}{1.684734in}}%
\pgfusepath{clip}%
\pgfsetbuttcap%
\pgfsetroundjoin%
\definecolor{currentfill}{rgb}{0.298039,0.447059,0.690196}%
\pgfsetfillcolor{currentfill}%
\pgfsetlinewidth{1.003750pt}%
\definecolor{currentstroke}{rgb}{0.298039,0.447059,0.690196}%
\pgfsetstrokecolor{currentstroke}%
\pgfsetdash{}{0pt}%
\pgfpathmoveto{\pgfqpoint{3.981778in}{1.869007in}}%
\pgfpathcurveto{\pgfqpoint{3.990014in}{1.869007in}}{\pgfqpoint{3.997914in}{1.872279in}}{\pgfqpoint{4.003738in}{1.878103in}}%
\pgfpathcurveto{\pgfqpoint{4.009562in}{1.883927in}}{\pgfqpoint{4.012834in}{1.891827in}}{\pgfqpoint{4.012834in}{1.900063in}}%
\pgfpathcurveto{\pgfqpoint{4.012834in}{1.908299in}}{\pgfqpoint{4.009562in}{1.916199in}}{\pgfqpoint{4.003738in}{1.922023in}}%
\pgfpathcurveto{\pgfqpoint{3.997914in}{1.927847in}}{\pgfqpoint{3.990014in}{1.931120in}}{\pgfqpoint{3.981778in}{1.931120in}}%
\pgfpathcurveto{\pgfqpoint{3.973542in}{1.931120in}}{\pgfqpoint{3.965642in}{1.927847in}}{\pgfqpoint{3.959818in}{1.922023in}}%
\pgfpathcurveto{\pgfqpoint{3.953994in}{1.916199in}}{\pgfqpoint{3.950721in}{1.908299in}}{\pgfqpoint{3.950721in}{1.900063in}}%
\pgfpathcurveto{\pgfqpoint{3.950721in}{1.891827in}}{\pgfqpoint{3.953994in}{1.883927in}}{\pgfqpoint{3.959818in}{1.878103in}}%
\pgfpathcurveto{\pgfqpoint{3.965642in}{1.872279in}}{\pgfqpoint{3.973542in}{1.869007in}}{\pgfqpoint{3.981778in}{1.869007in}}%
\pgfpathclose%
\pgfusepath{stroke,fill}%
\end{pgfscope}%
\begin{pgfscope}%
\pgfpathrectangle{\pgfqpoint{3.793912in}{0.557870in}}{\pgfqpoint{2.446088in}{1.684734in}}%
\pgfusepath{clip}%
\pgfsetbuttcap%
\pgfsetroundjoin%
\definecolor{currentfill}{rgb}{0.298039,0.447059,0.690196}%
\pgfsetfillcolor{currentfill}%
\pgfsetlinewidth{1.003750pt}%
\definecolor{currentstroke}{rgb}{0.298039,0.447059,0.690196}%
\pgfsetstrokecolor{currentstroke}%
\pgfsetdash{}{0pt}%
\pgfpathmoveto{\pgfqpoint{3.981778in}{1.777295in}}%
\pgfpathcurveto{\pgfqpoint{3.990014in}{1.777295in}}{\pgfqpoint{3.997914in}{1.780568in}}{\pgfqpoint{4.003738in}{1.786392in}}%
\pgfpathcurveto{\pgfqpoint{4.009562in}{1.792216in}}{\pgfqpoint{4.012834in}{1.800116in}}{\pgfqpoint{4.012834in}{1.808352in}}%
\pgfpathcurveto{\pgfqpoint{4.012834in}{1.816588in}}{\pgfqpoint{4.009562in}{1.824488in}}{\pgfqpoint{4.003738in}{1.830312in}}%
\pgfpathcurveto{\pgfqpoint{3.997914in}{1.836136in}}{\pgfqpoint{3.990014in}{1.839408in}}{\pgfqpoint{3.981778in}{1.839408in}}%
\pgfpathcurveto{\pgfqpoint{3.973542in}{1.839408in}}{\pgfqpoint{3.965642in}{1.836136in}}{\pgfqpoint{3.959818in}{1.830312in}}%
\pgfpathcurveto{\pgfqpoint{3.953994in}{1.824488in}}{\pgfqpoint{3.950721in}{1.816588in}}{\pgfqpoint{3.950721in}{1.808352in}}%
\pgfpathcurveto{\pgfqpoint{3.950721in}{1.800116in}}{\pgfqpoint{3.953994in}{1.792216in}}{\pgfqpoint{3.959818in}{1.786392in}}%
\pgfpathcurveto{\pgfqpoint{3.965642in}{1.780568in}}{\pgfqpoint{3.973542in}{1.777295in}}{\pgfqpoint{3.981778in}{1.777295in}}%
\pgfpathclose%
\pgfusepath{stroke,fill}%
\end{pgfscope}%
\begin{pgfscope}%
\pgfpathrectangle{\pgfqpoint{3.793912in}{0.557870in}}{\pgfqpoint{2.446088in}{1.684734in}}%
\pgfusepath{clip}%
\pgfsetbuttcap%
\pgfsetroundjoin%
\definecolor{currentfill}{rgb}{0.298039,0.447059,0.690196}%
\pgfsetfillcolor{currentfill}%
\pgfsetlinewidth{1.003750pt}%
\definecolor{currentstroke}{rgb}{0.298039,0.447059,0.690196}%
\pgfsetstrokecolor{currentstroke}%
\pgfsetdash{}{0pt}%
\pgfpathmoveto{\pgfqpoint{3.905098in}{2.125798in}}%
\pgfpathcurveto{\pgfqpoint{3.913334in}{2.125798in}}{\pgfqpoint{3.921234in}{2.129070in}}{\pgfqpoint{3.927058in}{2.134894in}}%
\pgfpathcurveto{\pgfqpoint{3.932882in}{2.140718in}}{\pgfqpoint{3.936155in}{2.148618in}}{\pgfqpoint{3.936155in}{2.156854in}}%
\pgfpathcurveto{\pgfqpoint{3.936155in}{2.165091in}}{\pgfqpoint{3.932882in}{2.172991in}}{\pgfqpoint{3.927058in}{2.178814in}}%
\pgfpathcurveto{\pgfqpoint{3.921234in}{2.184638in}}{\pgfqpoint{3.913334in}{2.187911in}}{\pgfqpoint{3.905098in}{2.187911in}}%
\pgfpathcurveto{\pgfqpoint{3.896862in}{2.187911in}}{\pgfqpoint{3.888962in}{2.184638in}}{\pgfqpoint{3.883138in}{2.178814in}}%
\pgfpathcurveto{\pgfqpoint{3.877314in}{2.172991in}}{\pgfqpoint{3.874042in}{2.165091in}}{\pgfqpoint{3.874042in}{2.156854in}}%
\pgfpathcurveto{\pgfqpoint{3.874042in}{2.148618in}}{\pgfqpoint{3.877314in}{2.140718in}}{\pgfqpoint{3.883138in}{2.134894in}}%
\pgfpathcurveto{\pgfqpoint{3.888962in}{2.129070in}}{\pgfqpoint{3.896862in}{2.125798in}}{\pgfqpoint{3.905098in}{2.125798in}}%
\pgfpathclose%
\pgfusepath{stroke,fill}%
\end{pgfscope}%
\begin{pgfscope}%
\pgfpathrectangle{\pgfqpoint{3.793912in}{0.557870in}}{\pgfqpoint{2.446088in}{1.684734in}}%
\pgfusepath{clip}%
\pgfsetbuttcap%
\pgfsetroundjoin%
\definecolor{currentfill}{rgb}{0.298039,0.447059,0.690196}%
\pgfsetfillcolor{currentfill}%
\pgfsetlinewidth{1.003750pt}%
\definecolor{currentstroke}{rgb}{0.298039,0.447059,0.690196}%
\pgfsetstrokecolor{currentstroke}%
\pgfsetdash{}{0pt}%
\pgfpathmoveto{\pgfqpoint{4.365177in}{1.841493in}}%
\pgfpathcurveto{\pgfqpoint{4.373414in}{1.841493in}}{\pgfqpoint{4.381314in}{1.844765in}}{\pgfqpoint{4.387137in}{1.850589in}}%
\pgfpathcurveto{\pgfqpoint{4.392961in}{1.856413in}}{\pgfqpoint{4.396234in}{1.864313in}}{\pgfqpoint{4.396234in}{1.872550in}}%
\pgfpathcurveto{\pgfqpoint{4.396234in}{1.880786in}}{\pgfqpoint{4.392961in}{1.888686in}}{\pgfqpoint{4.387137in}{1.894510in}}%
\pgfpathcurveto{\pgfqpoint{4.381314in}{1.900334in}}{\pgfqpoint{4.373414in}{1.903606in}}{\pgfqpoint{4.365177in}{1.903606in}}%
\pgfpathcurveto{\pgfqpoint{4.356941in}{1.903606in}}{\pgfqpoint{4.349041in}{1.900334in}}{\pgfqpoint{4.343217in}{1.894510in}}%
\pgfpathcurveto{\pgfqpoint{4.337393in}{1.888686in}}{\pgfqpoint{4.334121in}{1.880786in}}{\pgfqpoint{4.334121in}{1.872550in}}%
\pgfpathcurveto{\pgfqpoint{4.334121in}{1.864313in}}{\pgfqpoint{4.337393in}{1.856413in}}{\pgfqpoint{4.343217in}{1.850589in}}%
\pgfpathcurveto{\pgfqpoint{4.349041in}{1.844765in}}{\pgfqpoint{4.356941in}{1.841493in}}{\pgfqpoint{4.365177in}{1.841493in}}%
\pgfpathclose%
\pgfusepath{stroke,fill}%
\end{pgfscope}%
\begin{pgfscope}%
\pgfpathrectangle{\pgfqpoint{3.793912in}{0.557870in}}{\pgfqpoint{2.446088in}{1.684734in}}%
\pgfusepath{clip}%
\pgfsetbuttcap%
\pgfsetroundjoin%
\definecolor{currentfill}{rgb}{0.298039,0.447059,0.690196}%
\pgfsetfillcolor{currentfill}%
\pgfsetlinewidth{1.003750pt}%
\definecolor{currentstroke}{rgb}{0.298039,0.447059,0.690196}%
\pgfsetstrokecolor{currentstroke}%
\pgfsetdash{}{0pt}%
\pgfpathmoveto{\pgfqpoint{3.905098in}{2.125798in}}%
\pgfpathcurveto{\pgfqpoint{3.913334in}{2.125798in}}{\pgfqpoint{3.921234in}{2.129070in}}{\pgfqpoint{3.927058in}{2.134894in}}%
\pgfpathcurveto{\pgfqpoint{3.932882in}{2.140718in}}{\pgfqpoint{3.936155in}{2.148618in}}{\pgfqpoint{3.936155in}{2.156854in}}%
\pgfpathcurveto{\pgfqpoint{3.936155in}{2.165091in}}{\pgfqpoint{3.932882in}{2.172991in}}{\pgfqpoint{3.927058in}{2.178814in}}%
\pgfpathcurveto{\pgfqpoint{3.921234in}{2.184638in}}{\pgfqpoint{3.913334in}{2.187911in}}{\pgfqpoint{3.905098in}{2.187911in}}%
\pgfpathcurveto{\pgfqpoint{3.896862in}{2.187911in}}{\pgfqpoint{3.888962in}{2.184638in}}{\pgfqpoint{3.883138in}{2.178814in}}%
\pgfpathcurveto{\pgfqpoint{3.877314in}{2.172991in}}{\pgfqpoint{3.874042in}{2.165091in}}{\pgfqpoint{3.874042in}{2.156854in}}%
\pgfpathcurveto{\pgfqpoint{3.874042in}{2.148618in}}{\pgfqpoint{3.877314in}{2.140718in}}{\pgfqpoint{3.883138in}{2.134894in}}%
\pgfpathcurveto{\pgfqpoint{3.888962in}{2.129070in}}{\pgfqpoint{3.896862in}{2.125798in}}{\pgfqpoint{3.905098in}{2.125798in}}%
\pgfpathclose%
\pgfusepath{stroke,fill}%
\end{pgfscope}%
\begin{pgfscope}%
\pgfpathrectangle{\pgfqpoint{3.793912in}{0.557870in}}{\pgfqpoint{2.446088in}{1.684734in}}%
\pgfusepath{clip}%
\pgfsetbuttcap%
\pgfsetroundjoin%
\definecolor{currentfill}{rgb}{0.298039,0.447059,0.690196}%
\pgfsetfillcolor{currentfill}%
\pgfsetlinewidth{1.003750pt}%
\definecolor{currentstroke}{rgb}{0.298039,0.447059,0.690196}%
\pgfsetstrokecolor{currentstroke}%
\pgfsetdash{}{0pt}%
\pgfpathmoveto{\pgfqpoint{4.365177in}{1.813980in}}%
\pgfpathcurveto{\pgfqpoint{4.373414in}{1.813980in}}{\pgfqpoint{4.381314in}{1.817252in}}{\pgfqpoint{4.387137in}{1.823076in}}%
\pgfpathcurveto{\pgfqpoint{4.392961in}{1.828900in}}{\pgfqpoint{4.396234in}{1.836800in}}{\pgfqpoint{4.396234in}{1.845036in}}%
\pgfpathcurveto{\pgfqpoint{4.396234in}{1.853273in}}{\pgfqpoint{4.392961in}{1.861173in}}{\pgfqpoint{4.387137in}{1.866997in}}%
\pgfpathcurveto{\pgfqpoint{4.381314in}{1.872821in}}{\pgfqpoint{4.373414in}{1.876093in}}{\pgfqpoint{4.365177in}{1.876093in}}%
\pgfpathcurveto{\pgfqpoint{4.356941in}{1.876093in}}{\pgfqpoint{4.349041in}{1.872821in}}{\pgfqpoint{4.343217in}{1.866997in}}%
\pgfpathcurveto{\pgfqpoint{4.337393in}{1.861173in}}{\pgfqpoint{4.334121in}{1.853273in}}{\pgfqpoint{4.334121in}{1.845036in}}%
\pgfpathcurveto{\pgfqpoint{4.334121in}{1.836800in}}{\pgfqpoint{4.337393in}{1.828900in}}{\pgfqpoint{4.343217in}{1.823076in}}%
\pgfpathcurveto{\pgfqpoint{4.349041in}{1.817252in}}{\pgfqpoint{4.356941in}{1.813980in}}{\pgfqpoint{4.365177in}{1.813980in}}%
\pgfpathclose%
\pgfusepath{stroke,fill}%
\end{pgfscope}%
\begin{pgfscope}%
\pgfpathrectangle{\pgfqpoint{3.793912in}{0.557870in}}{\pgfqpoint{2.446088in}{1.684734in}}%
\pgfusepath{clip}%
\pgfsetbuttcap%
\pgfsetroundjoin%
\definecolor{currentfill}{rgb}{0.298039,0.447059,0.690196}%
\pgfsetfillcolor{currentfill}%
\pgfsetlinewidth{1.003750pt}%
\definecolor{currentstroke}{rgb}{0.298039,0.447059,0.690196}%
\pgfsetstrokecolor{currentstroke}%
\pgfsetdash{}{0pt}%
\pgfpathmoveto{\pgfqpoint{3.905098in}{2.125798in}}%
\pgfpathcurveto{\pgfqpoint{3.913334in}{2.125798in}}{\pgfqpoint{3.921234in}{2.129070in}}{\pgfqpoint{3.927058in}{2.134894in}}%
\pgfpathcurveto{\pgfqpoint{3.932882in}{2.140718in}}{\pgfqpoint{3.936155in}{2.148618in}}{\pgfqpoint{3.936155in}{2.156854in}}%
\pgfpathcurveto{\pgfqpoint{3.936155in}{2.165091in}}{\pgfqpoint{3.932882in}{2.172991in}}{\pgfqpoint{3.927058in}{2.178814in}}%
\pgfpathcurveto{\pgfqpoint{3.921234in}{2.184638in}}{\pgfqpoint{3.913334in}{2.187911in}}{\pgfqpoint{3.905098in}{2.187911in}}%
\pgfpathcurveto{\pgfqpoint{3.896862in}{2.187911in}}{\pgfqpoint{3.888962in}{2.184638in}}{\pgfqpoint{3.883138in}{2.178814in}}%
\pgfpathcurveto{\pgfqpoint{3.877314in}{2.172991in}}{\pgfqpoint{3.874042in}{2.165091in}}{\pgfqpoint{3.874042in}{2.156854in}}%
\pgfpathcurveto{\pgfqpoint{3.874042in}{2.148618in}}{\pgfqpoint{3.877314in}{2.140718in}}{\pgfqpoint{3.883138in}{2.134894in}}%
\pgfpathcurveto{\pgfqpoint{3.888962in}{2.129070in}}{\pgfqpoint{3.896862in}{2.125798in}}{\pgfqpoint{3.905098in}{2.125798in}}%
\pgfpathclose%
\pgfusepath{stroke,fill}%
\end{pgfscope}%
\begin{pgfscope}%
\pgfpathrectangle{\pgfqpoint{3.793912in}{0.557870in}}{\pgfqpoint{2.446088in}{1.684734in}}%
\pgfusepath{clip}%
\pgfsetbuttcap%
\pgfsetroundjoin%
\definecolor{currentfill}{rgb}{0.298039,0.447059,0.690196}%
\pgfsetfillcolor{currentfill}%
\pgfsetlinewidth{1.003750pt}%
\definecolor{currentstroke}{rgb}{0.298039,0.447059,0.690196}%
\pgfsetstrokecolor{currentstroke}%
\pgfsetdash{}{0pt}%
\pgfpathmoveto{\pgfqpoint{3.905098in}{2.125798in}}%
\pgfpathcurveto{\pgfqpoint{3.913334in}{2.125798in}}{\pgfqpoint{3.921234in}{2.129070in}}{\pgfqpoint{3.927058in}{2.134894in}}%
\pgfpathcurveto{\pgfqpoint{3.932882in}{2.140718in}}{\pgfqpoint{3.936155in}{2.148618in}}{\pgfqpoint{3.936155in}{2.156854in}}%
\pgfpathcurveto{\pgfqpoint{3.936155in}{2.165091in}}{\pgfqpoint{3.932882in}{2.172991in}}{\pgfqpoint{3.927058in}{2.178814in}}%
\pgfpathcurveto{\pgfqpoint{3.921234in}{2.184638in}}{\pgfqpoint{3.913334in}{2.187911in}}{\pgfqpoint{3.905098in}{2.187911in}}%
\pgfpathcurveto{\pgfqpoint{3.896862in}{2.187911in}}{\pgfqpoint{3.888962in}{2.184638in}}{\pgfqpoint{3.883138in}{2.178814in}}%
\pgfpathcurveto{\pgfqpoint{3.877314in}{2.172991in}}{\pgfqpoint{3.874042in}{2.165091in}}{\pgfqpoint{3.874042in}{2.156854in}}%
\pgfpathcurveto{\pgfqpoint{3.874042in}{2.148618in}}{\pgfqpoint{3.877314in}{2.140718in}}{\pgfqpoint{3.883138in}{2.134894in}}%
\pgfpathcurveto{\pgfqpoint{3.888962in}{2.129070in}}{\pgfqpoint{3.896862in}{2.125798in}}{\pgfqpoint{3.905098in}{2.125798in}}%
\pgfpathclose%
\pgfusepath{stroke,fill}%
\end{pgfscope}%
\begin{pgfscope}%
\pgfpathrectangle{\pgfqpoint{3.793912in}{0.557870in}}{\pgfqpoint{2.446088in}{1.684734in}}%
\pgfusepath{clip}%
\pgfsetbuttcap%
\pgfsetroundjoin%
\definecolor{currentfill}{rgb}{0.298039,0.447059,0.690196}%
\pgfsetfillcolor{currentfill}%
\pgfsetlinewidth{1.003750pt}%
\definecolor{currentstroke}{rgb}{0.298039,0.447059,0.690196}%
\pgfsetstrokecolor{currentstroke}%
\pgfsetdash{}{0pt}%
\pgfpathmoveto{\pgfqpoint{4.825256in}{1.850664in}}%
\pgfpathcurveto{\pgfqpoint{4.833493in}{1.850664in}}{\pgfqpoint{4.841393in}{1.853937in}}{\pgfqpoint{4.847217in}{1.859761in}}%
\pgfpathcurveto{\pgfqpoint{4.853041in}{1.865584in}}{\pgfqpoint{4.856313in}{1.873484in}}{\pgfqpoint{4.856313in}{1.881721in}}%
\pgfpathcurveto{\pgfqpoint{4.856313in}{1.889957in}}{\pgfqpoint{4.853041in}{1.897857in}}{\pgfqpoint{4.847217in}{1.903681in}}%
\pgfpathcurveto{\pgfqpoint{4.841393in}{1.909505in}}{\pgfqpoint{4.833493in}{1.912777in}}{\pgfqpoint{4.825256in}{1.912777in}}%
\pgfpathcurveto{\pgfqpoint{4.817020in}{1.912777in}}{\pgfqpoint{4.809120in}{1.909505in}}{\pgfqpoint{4.803296in}{1.903681in}}%
\pgfpathcurveto{\pgfqpoint{4.797472in}{1.897857in}}{\pgfqpoint{4.794200in}{1.889957in}}{\pgfqpoint{4.794200in}{1.881721in}}%
\pgfpathcurveto{\pgfqpoint{4.794200in}{1.873484in}}{\pgfqpoint{4.797472in}{1.865584in}}{\pgfqpoint{4.803296in}{1.859761in}}%
\pgfpathcurveto{\pgfqpoint{4.809120in}{1.853937in}}{\pgfqpoint{4.817020in}{1.850664in}}{\pgfqpoint{4.825256in}{1.850664in}}%
\pgfpathclose%
\pgfusepath{stroke,fill}%
\end{pgfscope}%
\begin{pgfscope}%
\pgfpathrectangle{\pgfqpoint{3.793912in}{0.557870in}}{\pgfqpoint{2.446088in}{1.684734in}}%
\pgfusepath{clip}%
\pgfsetbuttcap%
\pgfsetroundjoin%
\definecolor{currentfill}{rgb}{0.298039,0.447059,0.690196}%
\pgfsetfillcolor{currentfill}%
\pgfsetlinewidth{1.003750pt}%
\definecolor{currentstroke}{rgb}{0.298039,0.447059,0.690196}%
\pgfsetstrokecolor{currentstroke}%
\pgfsetdash{}{0pt}%
\pgfpathmoveto{\pgfqpoint{3.905098in}{2.125798in}}%
\pgfpathcurveto{\pgfqpoint{3.913334in}{2.125798in}}{\pgfqpoint{3.921234in}{2.129070in}}{\pgfqpoint{3.927058in}{2.134894in}}%
\pgfpathcurveto{\pgfqpoint{3.932882in}{2.140718in}}{\pgfqpoint{3.936155in}{2.148618in}}{\pgfqpoint{3.936155in}{2.156854in}}%
\pgfpathcurveto{\pgfqpoint{3.936155in}{2.165091in}}{\pgfqpoint{3.932882in}{2.172991in}}{\pgfqpoint{3.927058in}{2.178814in}}%
\pgfpathcurveto{\pgfqpoint{3.921234in}{2.184638in}}{\pgfqpoint{3.913334in}{2.187911in}}{\pgfqpoint{3.905098in}{2.187911in}}%
\pgfpathcurveto{\pgfqpoint{3.896862in}{2.187911in}}{\pgfqpoint{3.888962in}{2.184638in}}{\pgfqpoint{3.883138in}{2.178814in}}%
\pgfpathcurveto{\pgfqpoint{3.877314in}{2.172991in}}{\pgfqpoint{3.874042in}{2.165091in}}{\pgfqpoint{3.874042in}{2.156854in}}%
\pgfpathcurveto{\pgfqpoint{3.874042in}{2.148618in}}{\pgfqpoint{3.877314in}{2.140718in}}{\pgfqpoint{3.883138in}{2.134894in}}%
\pgfpathcurveto{\pgfqpoint{3.888962in}{2.129070in}}{\pgfqpoint{3.896862in}{2.125798in}}{\pgfqpoint{3.905098in}{2.125798in}}%
\pgfpathclose%
\pgfusepath{stroke,fill}%
\end{pgfscope}%
\begin{pgfscope}%
\pgfpathrectangle{\pgfqpoint{3.793912in}{0.557870in}}{\pgfqpoint{2.446088in}{1.684734in}}%
\pgfusepath{clip}%
\pgfsetbuttcap%
\pgfsetroundjoin%
\definecolor{currentfill}{rgb}{0.298039,0.447059,0.690196}%
\pgfsetfillcolor{currentfill}%
\pgfsetlinewidth{1.003750pt}%
\definecolor{currentstroke}{rgb}{0.298039,0.447059,0.690196}%
\pgfsetstrokecolor{currentstroke}%
\pgfsetdash{}{0pt}%
\pgfpathmoveto{\pgfqpoint{4.595217in}{1.520504in}}%
\pgfpathcurveto{\pgfqpoint{4.603453in}{1.520504in}}{\pgfqpoint{4.611353in}{1.523776in}}{\pgfqpoint{4.617177in}{1.529600in}}%
\pgfpathcurveto{\pgfqpoint{4.623001in}{1.535424in}}{\pgfqpoint{4.626273in}{1.543324in}}{\pgfqpoint{4.626273in}{1.551561in}}%
\pgfpathcurveto{\pgfqpoint{4.626273in}{1.559797in}}{\pgfqpoint{4.623001in}{1.567697in}}{\pgfqpoint{4.617177in}{1.573521in}}%
\pgfpathcurveto{\pgfqpoint{4.611353in}{1.579345in}}{\pgfqpoint{4.603453in}{1.582617in}}{\pgfqpoint{4.595217in}{1.582617in}}%
\pgfpathcurveto{\pgfqpoint{4.586981in}{1.582617in}}{\pgfqpoint{4.579081in}{1.579345in}}{\pgfqpoint{4.573257in}{1.573521in}}%
\pgfpathcurveto{\pgfqpoint{4.567433in}{1.567697in}}{\pgfqpoint{4.564160in}{1.559797in}}{\pgfqpoint{4.564160in}{1.551561in}}%
\pgfpathcurveto{\pgfqpoint{4.564160in}{1.543324in}}{\pgfqpoint{4.567433in}{1.535424in}}{\pgfqpoint{4.573257in}{1.529600in}}%
\pgfpathcurveto{\pgfqpoint{4.579081in}{1.523776in}}{\pgfqpoint{4.586981in}{1.520504in}}{\pgfqpoint{4.595217in}{1.520504in}}%
\pgfpathclose%
\pgfusepath{stroke,fill}%
\end{pgfscope}%
\begin{pgfscope}%
\pgfpathrectangle{\pgfqpoint{3.793912in}{0.557870in}}{\pgfqpoint{2.446088in}{1.684734in}}%
\pgfusepath{clip}%
\pgfsetbuttcap%
\pgfsetroundjoin%
\definecolor{currentfill}{rgb}{0.298039,0.447059,0.690196}%
\pgfsetfillcolor{currentfill}%
\pgfsetlinewidth{1.003750pt}%
\definecolor{currentstroke}{rgb}{0.298039,0.447059,0.690196}%
\pgfsetstrokecolor{currentstroke}%
\pgfsetdash{}{0pt}%
\pgfpathmoveto{\pgfqpoint{3.905098in}{2.116627in}}%
\pgfpathcurveto{\pgfqpoint{3.913334in}{2.116627in}}{\pgfqpoint{3.921234in}{2.119899in}}{\pgfqpoint{3.927058in}{2.125723in}}%
\pgfpathcurveto{\pgfqpoint{3.932882in}{2.131547in}}{\pgfqpoint{3.936155in}{2.139447in}}{\pgfqpoint{3.936155in}{2.147683in}}%
\pgfpathcurveto{\pgfqpoint{3.936155in}{2.155919in}}{\pgfqpoint{3.932882in}{2.163819in}}{\pgfqpoint{3.927058in}{2.169643in}}%
\pgfpathcurveto{\pgfqpoint{3.921234in}{2.175467in}}{\pgfqpoint{3.913334in}{2.178740in}}{\pgfqpoint{3.905098in}{2.178740in}}%
\pgfpathcurveto{\pgfqpoint{3.896862in}{2.178740in}}{\pgfqpoint{3.888962in}{2.175467in}}{\pgfqpoint{3.883138in}{2.169643in}}%
\pgfpathcurveto{\pgfqpoint{3.877314in}{2.163819in}}{\pgfqpoint{3.874042in}{2.155919in}}{\pgfqpoint{3.874042in}{2.147683in}}%
\pgfpathcurveto{\pgfqpoint{3.874042in}{2.139447in}}{\pgfqpoint{3.877314in}{2.131547in}}{\pgfqpoint{3.883138in}{2.125723in}}%
\pgfpathcurveto{\pgfqpoint{3.888962in}{2.119899in}}{\pgfqpoint{3.896862in}{2.116627in}}{\pgfqpoint{3.905098in}{2.116627in}}%
\pgfpathclose%
\pgfusepath{stroke,fill}%
\end{pgfscope}%
\begin{pgfscope}%
\pgfpathrectangle{\pgfqpoint{3.793912in}{0.557870in}}{\pgfqpoint{2.446088in}{1.684734in}}%
\pgfusepath{clip}%
\pgfsetbuttcap%
\pgfsetroundjoin%
\definecolor{currentfill}{rgb}{0.298039,0.447059,0.690196}%
\pgfsetfillcolor{currentfill}%
\pgfsetlinewidth{1.003750pt}%
\definecolor{currentstroke}{rgb}{0.298039,0.447059,0.690196}%
\pgfsetstrokecolor{currentstroke}%
\pgfsetdash{}{0pt}%
\pgfpathmoveto{\pgfqpoint{4.978616in}{1.548017in}}%
\pgfpathcurveto{\pgfqpoint{4.986852in}{1.548017in}}{\pgfqpoint{4.994753in}{1.551290in}}{\pgfqpoint{5.000576in}{1.557114in}}%
\pgfpathcurveto{\pgfqpoint{5.006400in}{1.562938in}}{\pgfqpoint{5.009673in}{1.570838in}}{\pgfqpoint{5.009673in}{1.579074in}}%
\pgfpathcurveto{\pgfqpoint{5.009673in}{1.587310in}}{\pgfqpoint{5.006400in}{1.595210in}}{\pgfqpoint{5.000576in}{1.601034in}}%
\pgfpathcurveto{\pgfqpoint{4.994753in}{1.606858in}}{\pgfqpoint{4.986852in}{1.610130in}}{\pgfqpoint{4.978616in}{1.610130in}}%
\pgfpathcurveto{\pgfqpoint{4.970380in}{1.610130in}}{\pgfqpoint{4.962480in}{1.606858in}}{\pgfqpoint{4.956656in}{1.601034in}}%
\pgfpathcurveto{\pgfqpoint{4.950832in}{1.595210in}}{\pgfqpoint{4.947560in}{1.587310in}}{\pgfqpoint{4.947560in}{1.579074in}}%
\pgfpathcurveto{\pgfqpoint{4.947560in}{1.570838in}}{\pgfqpoint{4.950832in}{1.562938in}}{\pgfqpoint{4.956656in}{1.557114in}}%
\pgfpathcurveto{\pgfqpoint{4.962480in}{1.551290in}}{\pgfqpoint{4.970380in}{1.548017in}}{\pgfqpoint{4.978616in}{1.548017in}}%
\pgfpathclose%
\pgfusepath{stroke,fill}%
\end{pgfscope}%
\begin{pgfscope}%
\pgfpathrectangle{\pgfqpoint{3.793912in}{0.557870in}}{\pgfqpoint{2.446088in}{1.684734in}}%
\pgfusepath{clip}%
\pgfsetbuttcap%
\pgfsetroundjoin%
\definecolor{currentfill}{rgb}{0.298039,0.447059,0.690196}%
\pgfsetfillcolor{currentfill}%
\pgfsetlinewidth{1.003750pt}%
\definecolor{currentstroke}{rgb}{0.298039,0.447059,0.690196}%
\pgfsetstrokecolor{currentstroke}%
\pgfsetdash{}{0pt}%
\pgfpathmoveto{\pgfqpoint{3.905098in}{2.125798in}}%
\pgfpathcurveto{\pgfqpoint{3.913334in}{2.125798in}}{\pgfqpoint{3.921234in}{2.129070in}}{\pgfqpoint{3.927058in}{2.134894in}}%
\pgfpathcurveto{\pgfqpoint{3.932882in}{2.140718in}}{\pgfqpoint{3.936155in}{2.148618in}}{\pgfqpoint{3.936155in}{2.156854in}}%
\pgfpathcurveto{\pgfqpoint{3.936155in}{2.165091in}}{\pgfqpoint{3.932882in}{2.172991in}}{\pgfqpoint{3.927058in}{2.178814in}}%
\pgfpathcurveto{\pgfqpoint{3.921234in}{2.184638in}}{\pgfqpoint{3.913334in}{2.187911in}}{\pgfqpoint{3.905098in}{2.187911in}}%
\pgfpathcurveto{\pgfqpoint{3.896862in}{2.187911in}}{\pgfqpoint{3.888962in}{2.184638in}}{\pgfqpoint{3.883138in}{2.178814in}}%
\pgfpathcurveto{\pgfqpoint{3.877314in}{2.172991in}}{\pgfqpoint{3.874042in}{2.165091in}}{\pgfqpoint{3.874042in}{2.156854in}}%
\pgfpathcurveto{\pgfqpoint{3.874042in}{2.148618in}}{\pgfqpoint{3.877314in}{2.140718in}}{\pgfqpoint{3.883138in}{2.134894in}}%
\pgfpathcurveto{\pgfqpoint{3.888962in}{2.129070in}}{\pgfqpoint{3.896862in}{2.125798in}}{\pgfqpoint{3.905098in}{2.125798in}}%
\pgfpathclose%
\pgfusepath{stroke,fill}%
\end{pgfscope}%
\begin{pgfscope}%
\pgfpathrectangle{\pgfqpoint{3.793912in}{0.557870in}}{\pgfqpoint{2.446088in}{1.684734in}}%
\pgfusepath{clip}%
\pgfsetbuttcap%
\pgfsetroundjoin%
\definecolor{currentfill}{rgb}{0.298039,0.447059,0.690196}%
\pgfsetfillcolor{currentfill}%
\pgfsetlinewidth{1.003750pt}%
\definecolor{currentstroke}{rgb}{0.298039,0.447059,0.690196}%
\pgfsetstrokecolor{currentstroke}%
\pgfsetdash{}{0pt}%
\pgfpathmoveto{\pgfqpoint{3.905098in}{2.125798in}}%
\pgfpathcurveto{\pgfqpoint{3.913334in}{2.125798in}}{\pgfqpoint{3.921234in}{2.129070in}}{\pgfqpoint{3.927058in}{2.134894in}}%
\pgfpathcurveto{\pgfqpoint{3.932882in}{2.140718in}}{\pgfqpoint{3.936155in}{2.148618in}}{\pgfqpoint{3.936155in}{2.156854in}}%
\pgfpathcurveto{\pgfqpoint{3.936155in}{2.165091in}}{\pgfqpoint{3.932882in}{2.172991in}}{\pgfqpoint{3.927058in}{2.178814in}}%
\pgfpathcurveto{\pgfqpoint{3.921234in}{2.184638in}}{\pgfqpoint{3.913334in}{2.187911in}}{\pgfqpoint{3.905098in}{2.187911in}}%
\pgfpathcurveto{\pgfqpoint{3.896862in}{2.187911in}}{\pgfqpoint{3.888962in}{2.184638in}}{\pgfqpoint{3.883138in}{2.178814in}}%
\pgfpathcurveto{\pgfqpoint{3.877314in}{2.172991in}}{\pgfqpoint{3.874042in}{2.165091in}}{\pgfqpoint{3.874042in}{2.156854in}}%
\pgfpathcurveto{\pgfqpoint{3.874042in}{2.148618in}}{\pgfqpoint{3.877314in}{2.140718in}}{\pgfqpoint{3.883138in}{2.134894in}}%
\pgfpathcurveto{\pgfqpoint{3.888962in}{2.129070in}}{\pgfqpoint{3.896862in}{2.125798in}}{\pgfqpoint{3.905098in}{2.125798in}}%
\pgfpathclose%
\pgfusepath{stroke,fill}%
\end{pgfscope}%
\begin{pgfscope}%
\pgfpathrectangle{\pgfqpoint{3.793912in}{0.557870in}}{\pgfqpoint{2.446088in}{1.684734in}}%
\pgfusepath{clip}%
\pgfsetbuttcap%
\pgfsetroundjoin%
\definecolor{currentfill}{rgb}{0.298039,0.447059,0.690196}%
\pgfsetfillcolor{currentfill}%
\pgfsetlinewidth{1.003750pt}%
\definecolor{currentstroke}{rgb}{0.298039,0.447059,0.690196}%
\pgfsetstrokecolor{currentstroke}%
\pgfsetdash{}{0pt}%
\pgfpathmoveto{\pgfqpoint{3.905098in}{2.125798in}}%
\pgfpathcurveto{\pgfqpoint{3.913334in}{2.125798in}}{\pgfqpoint{3.921234in}{2.129070in}}{\pgfqpoint{3.927058in}{2.134894in}}%
\pgfpathcurveto{\pgfqpoint{3.932882in}{2.140718in}}{\pgfqpoint{3.936155in}{2.148618in}}{\pgfqpoint{3.936155in}{2.156854in}}%
\pgfpathcurveto{\pgfqpoint{3.936155in}{2.165091in}}{\pgfqpoint{3.932882in}{2.172991in}}{\pgfqpoint{3.927058in}{2.178814in}}%
\pgfpathcurveto{\pgfqpoint{3.921234in}{2.184638in}}{\pgfqpoint{3.913334in}{2.187911in}}{\pgfqpoint{3.905098in}{2.187911in}}%
\pgfpathcurveto{\pgfqpoint{3.896862in}{2.187911in}}{\pgfqpoint{3.888962in}{2.184638in}}{\pgfqpoint{3.883138in}{2.178814in}}%
\pgfpathcurveto{\pgfqpoint{3.877314in}{2.172991in}}{\pgfqpoint{3.874042in}{2.165091in}}{\pgfqpoint{3.874042in}{2.156854in}}%
\pgfpathcurveto{\pgfqpoint{3.874042in}{2.148618in}}{\pgfqpoint{3.877314in}{2.140718in}}{\pgfqpoint{3.883138in}{2.134894in}}%
\pgfpathcurveto{\pgfqpoint{3.888962in}{2.129070in}}{\pgfqpoint{3.896862in}{2.125798in}}{\pgfqpoint{3.905098in}{2.125798in}}%
\pgfpathclose%
\pgfusepath{stroke,fill}%
\end{pgfscope}%
\begin{pgfscope}%
\pgfpathrectangle{\pgfqpoint{3.793912in}{0.557870in}}{\pgfqpoint{2.446088in}{1.684734in}}%
\pgfusepath{clip}%
\pgfsetbuttcap%
\pgfsetroundjoin%
\definecolor{currentfill}{rgb}{0.298039,0.447059,0.690196}%
\pgfsetfillcolor{currentfill}%
\pgfsetlinewidth{1.003750pt}%
\definecolor{currentstroke}{rgb}{0.298039,0.447059,0.690196}%
\pgfsetstrokecolor{currentstroke}%
\pgfsetdash{}{0pt}%
\pgfpathmoveto{\pgfqpoint{4.058458in}{1.667242in}}%
\pgfpathcurveto{\pgfqpoint{4.066694in}{1.667242in}}{\pgfqpoint{4.074594in}{1.670514in}}{\pgfqpoint{4.080418in}{1.676338in}}%
\pgfpathcurveto{\pgfqpoint{4.086242in}{1.682162in}}{\pgfqpoint{4.089514in}{1.690062in}}{\pgfqpoint{4.089514in}{1.698298in}}%
\pgfpathcurveto{\pgfqpoint{4.089514in}{1.706535in}}{\pgfqpoint{4.086242in}{1.714435in}}{\pgfqpoint{4.080418in}{1.720259in}}%
\pgfpathcurveto{\pgfqpoint{4.074594in}{1.726083in}}{\pgfqpoint{4.066694in}{1.729355in}}{\pgfqpoint{4.058458in}{1.729355in}}%
\pgfpathcurveto{\pgfqpoint{4.050221in}{1.729355in}}{\pgfqpoint{4.042321in}{1.726083in}}{\pgfqpoint{4.036498in}{1.720259in}}%
\pgfpathcurveto{\pgfqpoint{4.030674in}{1.714435in}}{\pgfqpoint{4.027401in}{1.706535in}}{\pgfqpoint{4.027401in}{1.698298in}}%
\pgfpathcurveto{\pgfqpoint{4.027401in}{1.690062in}}{\pgfqpoint{4.030674in}{1.682162in}}{\pgfqpoint{4.036498in}{1.676338in}}%
\pgfpathcurveto{\pgfqpoint{4.042321in}{1.670514in}}{\pgfqpoint{4.050221in}{1.667242in}}{\pgfqpoint{4.058458in}{1.667242in}}%
\pgfpathclose%
\pgfusepath{stroke,fill}%
\end{pgfscope}%
\begin{pgfscope}%
\pgfpathrectangle{\pgfqpoint{3.793912in}{0.557870in}}{\pgfqpoint{2.446088in}{1.684734in}}%
\pgfusepath{clip}%
\pgfsetbuttcap%
\pgfsetroundjoin%
\definecolor{currentfill}{rgb}{0.298039,0.447059,0.690196}%
\pgfsetfillcolor{currentfill}%
\pgfsetlinewidth{1.003750pt}%
\definecolor{currentstroke}{rgb}{0.298039,0.447059,0.690196}%
\pgfsetstrokecolor{currentstroke}%
\pgfsetdash{}{0pt}%
\pgfpathmoveto{\pgfqpoint{5.822095in}{1.676413in}}%
\pgfpathcurveto{\pgfqpoint{5.830331in}{1.676413in}}{\pgfqpoint{5.838231in}{1.679685in}}{\pgfqpoint{5.844055in}{1.685509in}}%
\pgfpathcurveto{\pgfqpoint{5.849879in}{1.691333in}}{\pgfqpoint{5.853151in}{1.699233in}}{\pgfqpoint{5.853151in}{1.707470in}}%
\pgfpathcurveto{\pgfqpoint{5.853151in}{1.715706in}}{\pgfqpoint{5.849879in}{1.723606in}}{\pgfqpoint{5.844055in}{1.729430in}}%
\pgfpathcurveto{\pgfqpoint{5.838231in}{1.735254in}}{\pgfqpoint{5.830331in}{1.738526in}}{\pgfqpoint{5.822095in}{1.738526in}}%
\pgfpathcurveto{\pgfqpoint{5.813858in}{1.738526in}}{\pgfqpoint{5.805958in}{1.735254in}}{\pgfqpoint{5.800134in}{1.729430in}}%
\pgfpathcurveto{\pgfqpoint{5.794311in}{1.723606in}}{\pgfqpoint{5.791038in}{1.715706in}}{\pgfqpoint{5.791038in}{1.707470in}}%
\pgfpathcurveto{\pgfqpoint{5.791038in}{1.699233in}}{\pgfqpoint{5.794311in}{1.691333in}}{\pgfqpoint{5.800134in}{1.685509in}}%
\pgfpathcurveto{\pgfqpoint{5.805958in}{1.679685in}}{\pgfqpoint{5.813858in}{1.676413in}}{\pgfqpoint{5.822095in}{1.676413in}}%
\pgfpathclose%
\pgfusepath{stroke,fill}%
\end{pgfscope}%
\begin{pgfscope}%
\pgfpathrectangle{\pgfqpoint{3.793912in}{0.557870in}}{\pgfqpoint{2.446088in}{1.684734in}}%
\pgfusepath{clip}%
\pgfsetbuttcap%
\pgfsetroundjoin%
\definecolor{currentfill}{rgb}{0.298039,0.447059,0.690196}%
\pgfsetfillcolor{currentfill}%
\pgfsetlinewidth{1.003750pt}%
\definecolor{currentstroke}{rgb}{0.298039,0.447059,0.690196}%
\pgfsetstrokecolor{currentstroke}%
\pgfsetdash{}{0pt}%
\pgfpathmoveto{\pgfqpoint{5.975454in}{1.263713in}}%
\pgfpathcurveto{\pgfqpoint{5.983691in}{1.263713in}}{\pgfqpoint{5.991591in}{1.266985in}}{\pgfqpoint{5.997415in}{1.272809in}}%
\pgfpathcurveto{\pgfqpoint{6.003239in}{1.278633in}}{\pgfqpoint{6.006511in}{1.286533in}}{\pgfqpoint{6.006511in}{1.294769in}}%
\pgfpathcurveto{\pgfqpoint{6.006511in}{1.303006in}}{\pgfqpoint{6.003239in}{1.310906in}}{\pgfqpoint{5.997415in}{1.316730in}}%
\pgfpathcurveto{\pgfqpoint{5.991591in}{1.322554in}}{\pgfqpoint{5.983691in}{1.325826in}}{\pgfqpoint{5.975454in}{1.325826in}}%
\pgfpathcurveto{\pgfqpoint{5.967218in}{1.325826in}}{\pgfqpoint{5.959318in}{1.322554in}}{\pgfqpoint{5.953494in}{1.316730in}}%
\pgfpathcurveto{\pgfqpoint{5.947670in}{1.310906in}}{\pgfqpoint{5.944398in}{1.303006in}}{\pgfqpoint{5.944398in}{1.294769in}}%
\pgfpathcurveto{\pgfqpoint{5.944398in}{1.286533in}}{\pgfqpoint{5.947670in}{1.278633in}}{\pgfqpoint{5.953494in}{1.272809in}}%
\pgfpathcurveto{\pgfqpoint{5.959318in}{1.266985in}}{\pgfqpoint{5.967218in}{1.263713in}}{\pgfqpoint{5.975454in}{1.263713in}}%
\pgfpathclose%
\pgfusepath{stroke,fill}%
\end{pgfscope}%
\begin{pgfscope}%
\pgfpathrectangle{\pgfqpoint{3.793912in}{0.557870in}}{\pgfqpoint{2.446088in}{1.684734in}}%
\pgfusepath{clip}%
\pgfsetbuttcap%
\pgfsetroundjoin%
\definecolor{currentfill}{rgb}{0.298039,0.447059,0.690196}%
\pgfsetfillcolor{currentfill}%
\pgfsetlinewidth{1.003750pt}%
\definecolor{currentstroke}{rgb}{0.298039,0.447059,0.690196}%
\pgfsetstrokecolor{currentstroke}%
\pgfsetdash{}{0pt}%
\pgfpathmoveto{\pgfqpoint{5.975454in}{1.309568in}}%
\pgfpathcurveto{\pgfqpoint{5.983691in}{1.309568in}}{\pgfqpoint{5.991591in}{1.312841in}}{\pgfqpoint{5.997415in}{1.318665in}}%
\pgfpathcurveto{\pgfqpoint{6.003239in}{1.324489in}}{\pgfqpoint{6.006511in}{1.332389in}}{\pgfqpoint{6.006511in}{1.340625in}}%
\pgfpathcurveto{\pgfqpoint{6.006511in}{1.348861in}}{\pgfqpoint{6.003239in}{1.356761in}}{\pgfqpoint{5.997415in}{1.362585in}}%
\pgfpathcurveto{\pgfqpoint{5.991591in}{1.368409in}}{\pgfqpoint{5.983691in}{1.371681in}}{\pgfqpoint{5.975454in}{1.371681in}}%
\pgfpathcurveto{\pgfqpoint{5.967218in}{1.371681in}}{\pgfqpoint{5.959318in}{1.368409in}}{\pgfqpoint{5.953494in}{1.362585in}}%
\pgfpathcurveto{\pgfqpoint{5.947670in}{1.356761in}}{\pgfqpoint{5.944398in}{1.348861in}}{\pgfqpoint{5.944398in}{1.340625in}}%
\pgfpathcurveto{\pgfqpoint{5.944398in}{1.332389in}}{\pgfqpoint{5.947670in}{1.324489in}}{\pgfqpoint{5.953494in}{1.318665in}}%
\pgfpathcurveto{\pgfqpoint{5.959318in}{1.312841in}}{\pgfqpoint{5.967218in}{1.309568in}}{\pgfqpoint{5.975454in}{1.309568in}}%
\pgfpathclose%
\pgfusepath{stroke,fill}%
\end{pgfscope}%
\begin{pgfscope}%
\pgfpathrectangle{\pgfqpoint{3.793912in}{0.557870in}}{\pgfqpoint{2.446088in}{1.684734in}}%
\pgfusepath{clip}%
\pgfsetbuttcap%
\pgfsetroundjoin%
\definecolor{currentfill}{rgb}{0.298039,0.447059,0.690196}%
\pgfsetfillcolor{currentfill}%
\pgfsetlinewidth{1.003750pt}%
\definecolor{currentstroke}{rgb}{0.298039,0.447059,0.690196}%
\pgfsetstrokecolor{currentstroke}%
\pgfsetdash{}{0pt}%
\pgfpathmoveto{\pgfqpoint{3.905098in}{2.125798in}}%
\pgfpathcurveto{\pgfqpoint{3.913334in}{2.125798in}}{\pgfqpoint{3.921234in}{2.129070in}}{\pgfqpoint{3.927058in}{2.134894in}}%
\pgfpathcurveto{\pgfqpoint{3.932882in}{2.140718in}}{\pgfqpoint{3.936155in}{2.148618in}}{\pgfqpoint{3.936155in}{2.156854in}}%
\pgfpathcurveto{\pgfqpoint{3.936155in}{2.165091in}}{\pgfqpoint{3.932882in}{2.172991in}}{\pgfqpoint{3.927058in}{2.178814in}}%
\pgfpathcurveto{\pgfqpoint{3.921234in}{2.184638in}}{\pgfqpoint{3.913334in}{2.187911in}}{\pgfqpoint{3.905098in}{2.187911in}}%
\pgfpathcurveto{\pgfqpoint{3.896862in}{2.187911in}}{\pgfqpoint{3.888962in}{2.184638in}}{\pgfqpoint{3.883138in}{2.178814in}}%
\pgfpathcurveto{\pgfqpoint{3.877314in}{2.172991in}}{\pgfqpoint{3.874042in}{2.165091in}}{\pgfqpoint{3.874042in}{2.156854in}}%
\pgfpathcurveto{\pgfqpoint{3.874042in}{2.148618in}}{\pgfqpoint{3.877314in}{2.140718in}}{\pgfqpoint{3.883138in}{2.134894in}}%
\pgfpathcurveto{\pgfqpoint{3.888962in}{2.129070in}}{\pgfqpoint{3.896862in}{2.125798in}}{\pgfqpoint{3.905098in}{2.125798in}}%
\pgfpathclose%
\pgfusepath{stroke,fill}%
\end{pgfscope}%
\begin{pgfscope}%
\pgfpathrectangle{\pgfqpoint{3.793912in}{0.557870in}}{\pgfqpoint{2.446088in}{1.684734in}}%
\pgfusepath{clip}%
\pgfsetbuttcap%
\pgfsetroundjoin%
\definecolor{currentfill}{rgb}{0.298039,0.447059,0.690196}%
\pgfsetfillcolor{currentfill}%
\pgfsetlinewidth{1.003750pt}%
\definecolor{currentstroke}{rgb}{0.298039,0.447059,0.690196}%
\pgfsetstrokecolor{currentstroke}%
\pgfsetdash{}{0pt}%
\pgfpathmoveto{\pgfqpoint{5.515375in}{1.850664in}}%
\pgfpathcurveto{\pgfqpoint{5.523612in}{1.850664in}}{\pgfqpoint{5.531512in}{1.853937in}}{\pgfqpoint{5.537336in}{1.859761in}}%
\pgfpathcurveto{\pgfqpoint{5.543159in}{1.865584in}}{\pgfqpoint{5.546432in}{1.873484in}}{\pgfqpoint{5.546432in}{1.881721in}}%
\pgfpathcurveto{\pgfqpoint{5.546432in}{1.889957in}}{\pgfqpoint{5.543159in}{1.897857in}}{\pgfqpoint{5.537336in}{1.903681in}}%
\pgfpathcurveto{\pgfqpoint{5.531512in}{1.909505in}}{\pgfqpoint{5.523612in}{1.912777in}}{\pgfqpoint{5.515375in}{1.912777in}}%
\pgfpathcurveto{\pgfqpoint{5.507139in}{1.912777in}}{\pgfqpoint{5.499239in}{1.909505in}}{\pgfqpoint{5.493415in}{1.903681in}}%
\pgfpathcurveto{\pgfqpoint{5.487591in}{1.897857in}}{\pgfqpoint{5.484319in}{1.889957in}}{\pgfqpoint{5.484319in}{1.881721in}}%
\pgfpathcurveto{\pgfqpoint{5.484319in}{1.873484in}}{\pgfqpoint{5.487591in}{1.865584in}}{\pgfqpoint{5.493415in}{1.859761in}}%
\pgfpathcurveto{\pgfqpoint{5.499239in}{1.853937in}}{\pgfqpoint{5.507139in}{1.850664in}}{\pgfqpoint{5.515375in}{1.850664in}}%
\pgfpathclose%
\pgfusepath{stroke,fill}%
\end{pgfscope}%
\begin{pgfscope}%
\pgfsetrectcap%
\pgfsetmiterjoin%
\pgfsetlinewidth{1.254687pt}%
\definecolor{currentstroke}{rgb}{1.000000,1.000000,1.000000}%
\pgfsetstrokecolor{currentstroke}%
\pgfsetdash{}{0pt}%
\pgfpathmoveto{\pgfqpoint{3.793912in}{0.557870in}}%
\pgfpathlineto{\pgfqpoint{3.793912in}{2.242604in}}%
\pgfusepath{stroke}%
\end{pgfscope}%
\begin{pgfscope}%
\pgfsetrectcap%
\pgfsetmiterjoin%
\pgfsetlinewidth{1.254687pt}%
\definecolor{currentstroke}{rgb}{1.000000,1.000000,1.000000}%
\pgfsetstrokecolor{currentstroke}%
\pgfsetdash{}{0pt}%
\pgfpathmoveto{\pgfqpoint{6.240000in}{0.557870in}}%
\pgfpathlineto{\pgfqpoint{6.240000in}{2.242604in}}%
\pgfusepath{stroke}%
\end{pgfscope}%
\begin{pgfscope}%
\pgfsetrectcap%
\pgfsetmiterjoin%
\pgfsetlinewidth{1.254687pt}%
\definecolor{currentstroke}{rgb}{1.000000,1.000000,1.000000}%
\pgfsetstrokecolor{currentstroke}%
\pgfsetdash{}{0pt}%
\pgfpathmoveto{\pgfqpoint{3.793912in}{0.557870in}}%
\pgfpathlineto{\pgfqpoint{6.240000in}{0.557870in}}%
\pgfusepath{stroke}%
\end{pgfscope}%
\begin{pgfscope}%
\pgfsetrectcap%
\pgfsetmiterjoin%
\pgfsetlinewidth{1.254687pt}%
\definecolor{currentstroke}{rgb}{1.000000,1.000000,1.000000}%
\pgfsetstrokecolor{currentstroke}%
\pgfsetdash{}{0pt}%
\pgfpathmoveto{\pgfqpoint{3.793912in}{2.242604in}}%
\pgfpathlineto{\pgfqpoint{6.240000in}{2.242604in}}%
\pgfusepath{stroke}%
\end{pgfscope}%
\begin{pgfscope}%
\definecolor{textcolor}{rgb}{0.150000,0.150000,0.150000}%
\pgfsetstrokecolor{textcolor}%
\pgfsetfillcolor{textcolor}%
\pgftext[x=5.016956in,y=2.325938in,,base]{\color{textcolor}\sffamily\fontsize{11.000000}{13.200000}\selectfont (b)}%
\end{pgfscope}%
\end{pgfpicture}%
\makeatother%
\endgroup%

    \caption{Distribution of DOR, sensitivity and specificity for the different TSC methods when classifying patient diagnosis.}
    \label{fig:tsc_ind_dor_sens_spec_dist}
\end{figure}

\begin{comment}
[0] \textbf{Comment on spread of DOR.}
    * From the distribution plot in figure \ref{fig:tsc_ind_dor_sens_spec_dist}a one can see that the majority of DORs are close to zero, but there are some methods that acheive a DOR above 30.
[0] \textbf{Comment on spread of sensitivity and specificity.}
    * In the scatter plot in figure \ref{fig:tsc_ind_dor_sens_spec_dist}b one can see that the specificity of the methods and range from $0.5$ to 1, 
      and the sensitivity scores range from 0 to $0.93$. 
    * 
[0] \textbf{Comment on common traits in the high performing methods.} Here you can refer to raw performance results in appendix.
    * As with heart failure, the TSC methods that perform best in terms of DOR use data from a single view. 
      The 2CH view, and GLS curves are the only view and curve that are used among the methods that achieve the five highest DORs.
    * From the table of all the method results in the appendix \ref{tab:tsc_ind_raw_results} one can see that the highest performing method in terms of DOR
      to use a dataset other than GLS curves alone is \textit{gls-rls/2CH/scaled/ward/2} and it achieves a DOR of $26.76$.
    * One can also note that the highest performing method in terms of DOR that uses a view other than only 2CH is \textit{rls/all-views/normalized/weighted/2}
      which achieves a DOR of 25.56.
    * The TSC methods that achieve the highest DOR scores all use no preprocessing, or scaling. 
[0] \textbf{Comment on common traits in the low performing methods.} Here you can refer to raw performance results in appendix.
[0] \textbf{Select one - three methods that are good contendors for being the best method/model in the group and comment on their traits}
    * From table \ref{tab:tsc_ind_dor_sens_spec_dist} one can see that the TSC methods that acheive the highest DOR scores are \textit{gls/2CH/regular/centroid/2}, 
      and \textit{gls/2CH/scaled/centroid/2} which are the same two methods that achieve the highest DORs in the heart failure case study.
\textbf{IF NOT CLUSTERING METHOD}
[NA] \textbf{Make arguments for and against the top three methods in terms of accuracy, sensitivity, specificity, and DOR, and make an informed choice.}
\end{comment}

\begin{table*}
    \centering
    \ra{1.3}
    \begin{tabular}{lrrrr}
        \toprule
        Dataset-Method             &  Accuracy &  Sensitivity &  Specificity &   DOR \\
        \midrule
        gls/2CH/regular/centroid/2 &      0.74 &         0.71 &         0.93 & 33.47 \\
        gls/2CH/scaled/centroid/2  &      0.74 &         0.71 &         0.93 & 33.47 \\
        gls/2CH/scaled/average/2   &      0.73 &         0.69 &         0.93 & 30.71 \\
        gls/2CH/regular/average/2  &      0.73 &         0.69 &         0.93 & 30.71 \\
        gls/2CH/scaled/ward/2      &      0.71 &         0.67 &         0.93 & 27.49 \\
        \bottomrule
    \end{tabular}
    \caption{The accuracy, DOR, sensitivity and specicity scores of the five best performing two-cluster-center TSC methods in terms of DOR, at detecting patient diagnoses.
             The \textbf{Dataset-Method} column indicates 
             \textit{Dataset used}$/$\textit{View used}$/$\textit{Type of preprocessing used}$/$\textit{Linkage criteria of method}$/$\textit{Number of cluster centers}.}
    \label{tab:tsc_ind_dor_sens_spec_dist}
\end{table*}

\begin{table*}
    \centering
    \ra{1.3}
    \begin{tabular}{lr}
        \toprule
        Dataset-Method                   &  ARI \\
        \midrule
        gls-rls/4CH/regular/complete/2   & 0.36 \\
        gls/all-views/regular/weighted/2 & 0.34 \\
        gls/all-views/scaled/weighted/4  & 0.33 \\
        gls/all-views/scaled/weighted/3  & 0.33 \\
        gls/APLAX/regular/single/10      & 0.32 \\
        \bottomrule
    \end{tabular}
    \caption{The five highest ARI scores attained when applying TSC for detecting patient diagnoses.
             The \textbf{Dataset-Method} column indicates \textit{Dataset used}$/$\textit{View used}$/$\textit{Linkage criteria of method}$/$\textit{Number of cluster centers}.}
    \label{tab:tsc_ind_ari}
\end{table*}

\begin{comment}
\textbf{ARI PARAGRAPH. ONLY FOR CLUSTERING METHODS}.
[0] \textbf{Comment on the spread of ARI scores. Be specific since the distribution plots are ommitted}
    * The majority of the ARI scorer for all the TSC methods evaluated at two to nine cluster centers are centered around zero.
[0] \textbf{Comment on the general trends of high performing methods in terms of ARI - are they the same trends as scores performing high in terms of DOR?}
    * As with the TSC methods attaining the highest DORs the methods using no preprocessing or scaling acheive the highest ARI indices when used to identify patient diagnoses.
    * In addition, the GLS curves are also most often part of the dataset for the TSC methods receiving the highest ARI when used to identify patient diagnoses.
[0] \textbf{Comment on whether the methods in the top 5 ARIs are the same methods with the highest DOR. If not, mention it.}
    * From table \ref{tab:tsc_ind_ari} one can see that the TSC methods receiving the five highest ARI scores, are not among the TSC methods that receive the highest DOR scores. 
    * The TSC method \textit{gls-rls/4CH/regular/complete/2} attains the highest ARI score when applied to identify patient diagnoses, and achieves an 
      accuracy of $0.84$, a sensitivity of $0.87$ a specificity of $0.69$ and a DOR $14.65$. 
    * The TSC method \textit{gls/all-views/regular/weighted/2} achieves the second highest ARI when applied to identify patient diagnoses, and achieves an
      accuracy of $0.82$, a sensitivity of $0.81$ a specificity of $0.83$ and a DOR $21.06$. 
    * What should also be noted is that the TSC methods achieving the two highest ARIs when applied to identify patient diagnoses are methods evaluated at two cluster centers,
      which means that none of the TSC methods evaluated at cluster centers between three and nine can perform better than the ones evaluated at two cluster centers.
[NA] \textbf{If the top 1 or 2 ARIs are also top in DOR no further discussion is needed. You can then plot some of the cluster realizations to see what they look like.}
[ ] \textbf{If NOT, why do they differ? Is the method with the highest ARI evaluated at a higher cluster number that 2? Attempt to visualize it, if it is not too difficult.}
    * It may seem strange that the ordered lists of DORs, and ARIs are so different. 
    * The reason for this is not because DOR inherently values sensitivity higher than specificity, but stems from how the DOR is defined. 
    * Recall that $\mathrm{DOR = ( TP \times TN )/ (FP \times FN)}$, since the patient diagnoses dataset is skewed in favour of positives TP has the potential of being as high as 170 
      while TN can be as high as 30. 
    * Therefore the DOR will be higher for methods with a high sensitivity than for methods with an equally high sensitivity.
[ ] \textbf{Make arguments for and against the top three methods in terms of accuracy, sensitivity, specificity, DOR, ARI and potentially the plots, and make an informed choice.}
    * The TSC method that is chosen as the best method for identifying patient diagnoses is \textit{gls/all-views/regular/weighted/2}, because it achieves the second highest ARI, and
      because it's sensitivity and specificity are more balanced than the method attaining the highest ARI and the methods that achieve higher DORs.
[ ] \textbf{Plot some visualizations of the clustering, and comment on them.}
    * 
\end{comment}

\clearpage

\begin{figure}[ht]
    \centering
    %% Creator: Matplotlib, PGF backend
%%
%% To include the figure in your LaTeX document, write
%%   \input{<filename>.pgf}
%%
%% Make sure the required packages are loaded in your preamble
%%   \usepackage{pgf}
%%
%% Figures using additional raster images can only be included by \input if
%% they are in the same directory as the main LaTeX file. For loading figures
%% from other directories you can use the `import` package
%%   \usepackage{import}
%% and then include the figures with
%%   \import{<path to file>}{<filename>.pgf}
%%
%% Matplotlib used the following preamble
%%
\begingroup%
\makeatletter%
\begin{pgfpicture}%
\pgfpathrectangle{\pgfpointorigin}{\pgfqpoint{6.340000in}{8.840000in}}%
\pgfusepath{use as bounding box, clip}%
\begin{pgfscope}%
\pgfsetbuttcap%
\pgfsetmiterjoin%
\definecolor{currentfill}{rgb}{1.000000,1.000000,1.000000}%
\pgfsetfillcolor{currentfill}%
\pgfsetlinewidth{0.000000pt}%
\definecolor{currentstroke}{rgb}{1.000000,1.000000,1.000000}%
\pgfsetstrokecolor{currentstroke}%
\pgfsetdash{}{0pt}%
\pgfpathmoveto{\pgfqpoint{0.000000in}{-0.000000in}}%
\pgfpathlineto{\pgfqpoint{6.340000in}{-0.000000in}}%
\pgfpathlineto{\pgfqpoint{6.340000in}{8.840000in}}%
\pgfpathlineto{\pgfqpoint{0.000000in}{8.840000in}}%
\pgfpathclose%
\pgfusepath{fill}%
\end{pgfscope}%
\begin{pgfscope}%
\pgfsetbuttcap%
\pgfsetmiterjoin%
\definecolor{currentfill}{rgb}{0.917647,0.917647,0.949020}%
\pgfsetfillcolor{currentfill}%
\pgfsetlinewidth{0.000000pt}%
\definecolor{currentstroke}{rgb}{0.000000,0.000000,0.000000}%
\pgfsetstrokecolor{currentstroke}%
\pgfsetstrokeopacity{0.000000}%
\pgfsetdash{}{0pt}%
\pgfpathmoveto{\pgfqpoint{0.828241in}{6.184745in}}%
\pgfpathlineto{\pgfqpoint{3.242963in}{6.184745in}}%
\pgfpathlineto{\pgfqpoint{3.242963in}{8.542604in}}%
\pgfpathlineto{\pgfqpoint{0.828241in}{8.542604in}}%
\pgfpathclose%
\pgfusepath{fill}%
\end{pgfscope}%
\begin{pgfscope}%
\pgfpathrectangle{\pgfqpoint{0.828241in}{6.184745in}}{\pgfqpoint{2.414722in}{2.357859in}}%
\pgfusepath{clip}%
\pgfsetroundcap%
\pgfsetroundjoin%
\pgfsetlinewidth{1.003750pt}%
\definecolor{currentstroke}{rgb}{1.000000,1.000000,1.000000}%
\pgfsetstrokecolor{currentstroke}%
\pgfsetdash{}{0pt}%
\pgfpathmoveto{\pgfqpoint{0.938001in}{6.184745in}}%
\pgfpathlineto{\pgfqpoint{0.938001in}{8.542604in}}%
\pgfusepath{stroke}%
\end{pgfscope}%
\begin{pgfscope}%
\definecolor{textcolor}{rgb}{0.150000,0.150000,0.150000}%
\pgfsetstrokecolor{textcolor}%
\pgfsetfillcolor{textcolor}%
\pgftext[x=0.938001in,y=6.052801in,,top]{\color{textcolor}\sffamily\fontsize{11.000000}{13.200000}\selectfont \(\displaystyle 0.0\)}%
\end{pgfscope}%
\begin{pgfscope}%
\pgfpathrectangle{\pgfqpoint{0.828241in}{6.184745in}}{\pgfqpoint{2.414722in}{2.357859in}}%
\pgfusepath{clip}%
\pgfsetroundcap%
\pgfsetroundjoin%
\pgfsetlinewidth{1.003750pt}%
\definecolor{currentstroke}{rgb}{1.000000,1.000000,1.000000}%
\pgfsetstrokecolor{currentstroke}%
\pgfsetdash{}{0pt}%
\pgfpathmoveto{\pgfqpoint{1.787335in}{6.184745in}}%
\pgfpathlineto{\pgfqpoint{1.787335in}{8.542604in}}%
\pgfusepath{stroke}%
\end{pgfscope}%
\begin{pgfscope}%
\definecolor{textcolor}{rgb}{0.150000,0.150000,0.150000}%
\pgfsetstrokecolor{textcolor}%
\pgfsetfillcolor{textcolor}%
\pgftext[x=1.787335in,y=6.052801in,,top]{\color{textcolor}\sffamily\fontsize{11.000000}{13.200000}\selectfont \(\displaystyle 0.5\)}%
\end{pgfscope}%
\begin{pgfscope}%
\pgfpathrectangle{\pgfqpoint{0.828241in}{6.184745in}}{\pgfqpoint{2.414722in}{2.357859in}}%
\pgfusepath{clip}%
\pgfsetroundcap%
\pgfsetroundjoin%
\pgfsetlinewidth{1.003750pt}%
\definecolor{currentstroke}{rgb}{1.000000,1.000000,1.000000}%
\pgfsetstrokecolor{currentstroke}%
\pgfsetdash{}{0pt}%
\pgfpathmoveto{\pgfqpoint{2.636669in}{6.184745in}}%
\pgfpathlineto{\pgfqpoint{2.636669in}{8.542604in}}%
\pgfusepath{stroke}%
\end{pgfscope}%
\begin{pgfscope}%
\definecolor{textcolor}{rgb}{0.150000,0.150000,0.150000}%
\pgfsetstrokecolor{textcolor}%
\pgfsetfillcolor{textcolor}%
\pgftext[x=2.636669in,y=6.052801in,,top]{\color{textcolor}\sffamily\fontsize{11.000000}{13.200000}\selectfont \(\displaystyle 1.0\)}%
\end{pgfscope}%
\begin{pgfscope}%
\pgfpathrectangle{\pgfqpoint{0.828241in}{6.184745in}}{\pgfqpoint{2.414722in}{2.357859in}}%
\pgfusepath{clip}%
\pgfsetroundcap%
\pgfsetroundjoin%
\pgfsetlinewidth{1.003750pt}%
\definecolor{currentstroke}{rgb}{1.000000,1.000000,1.000000}%
\pgfsetstrokecolor{currentstroke}%
\pgfsetdash{}{0pt}%
\pgfpathmoveto{\pgfqpoint{0.828241in}{6.271864in}}%
\pgfpathlineto{\pgfqpoint{3.242963in}{6.271864in}}%
\pgfusepath{stroke}%
\end{pgfscope}%
\begin{pgfscope}%
\definecolor{textcolor}{rgb}{0.150000,0.150000,0.150000}%
\pgfsetstrokecolor{textcolor}%
\pgfsetfillcolor{textcolor}%
\pgftext[x=0.425926in,y=6.219057in,left,base]{\color{textcolor}\sffamily\fontsize{11.000000}{13.200000}\selectfont \(\displaystyle -12\)}%
\end{pgfscope}%
\begin{pgfscope}%
\pgfpathrectangle{\pgfqpoint{0.828241in}{6.184745in}}{\pgfqpoint{2.414722in}{2.357859in}}%
\pgfusepath{clip}%
\pgfsetroundcap%
\pgfsetroundjoin%
\pgfsetlinewidth{1.003750pt}%
\definecolor{currentstroke}{rgb}{1.000000,1.000000,1.000000}%
\pgfsetstrokecolor{currentstroke}%
\pgfsetdash{}{0pt}%
\pgfpathmoveto{\pgfqpoint{0.828241in}{6.620737in}}%
\pgfpathlineto{\pgfqpoint{3.242963in}{6.620737in}}%
\pgfusepath{stroke}%
\end{pgfscope}%
\begin{pgfscope}%
\definecolor{textcolor}{rgb}{0.150000,0.150000,0.150000}%
\pgfsetstrokecolor{textcolor}%
\pgfsetfillcolor{textcolor}%
\pgftext[x=0.425926in,y=6.567931in,left,base]{\color{textcolor}\sffamily\fontsize{11.000000}{13.200000}\selectfont \(\displaystyle -10\)}%
\end{pgfscope}%
\begin{pgfscope}%
\pgfpathrectangle{\pgfqpoint{0.828241in}{6.184745in}}{\pgfqpoint{2.414722in}{2.357859in}}%
\pgfusepath{clip}%
\pgfsetroundcap%
\pgfsetroundjoin%
\pgfsetlinewidth{1.003750pt}%
\definecolor{currentstroke}{rgb}{1.000000,1.000000,1.000000}%
\pgfsetstrokecolor{currentstroke}%
\pgfsetdash{}{0pt}%
\pgfpathmoveto{\pgfqpoint{0.828241in}{6.969611in}}%
\pgfpathlineto{\pgfqpoint{3.242963in}{6.969611in}}%
\pgfusepath{stroke}%
\end{pgfscope}%
\begin{pgfscope}%
\definecolor{textcolor}{rgb}{0.150000,0.150000,0.150000}%
\pgfsetstrokecolor{textcolor}%
\pgfsetfillcolor{textcolor}%
\pgftext[x=0.501968in,y=6.916804in,left,base]{\color{textcolor}\sffamily\fontsize{11.000000}{13.200000}\selectfont \(\displaystyle -8\)}%
\end{pgfscope}%
\begin{pgfscope}%
\pgfpathrectangle{\pgfqpoint{0.828241in}{6.184745in}}{\pgfqpoint{2.414722in}{2.357859in}}%
\pgfusepath{clip}%
\pgfsetroundcap%
\pgfsetroundjoin%
\pgfsetlinewidth{1.003750pt}%
\definecolor{currentstroke}{rgb}{1.000000,1.000000,1.000000}%
\pgfsetstrokecolor{currentstroke}%
\pgfsetdash{}{0pt}%
\pgfpathmoveto{\pgfqpoint{0.828241in}{7.318484in}}%
\pgfpathlineto{\pgfqpoint{3.242963in}{7.318484in}}%
\pgfusepath{stroke}%
\end{pgfscope}%
\begin{pgfscope}%
\definecolor{textcolor}{rgb}{0.150000,0.150000,0.150000}%
\pgfsetstrokecolor{textcolor}%
\pgfsetfillcolor{textcolor}%
\pgftext[x=0.501968in,y=7.265677in,left,base]{\color{textcolor}\sffamily\fontsize{11.000000}{13.200000}\selectfont \(\displaystyle -6\)}%
\end{pgfscope}%
\begin{pgfscope}%
\pgfpathrectangle{\pgfqpoint{0.828241in}{6.184745in}}{\pgfqpoint{2.414722in}{2.357859in}}%
\pgfusepath{clip}%
\pgfsetroundcap%
\pgfsetroundjoin%
\pgfsetlinewidth{1.003750pt}%
\definecolor{currentstroke}{rgb}{1.000000,1.000000,1.000000}%
\pgfsetstrokecolor{currentstroke}%
\pgfsetdash{}{0pt}%
\pgfpathmoveto{\pgfqpoint{0.828241in}{7.667357in}}%
\pgfpathlineto{\pgfqpoint{3.242963in}{7.667357in}}%
\pgfusepath{stroke}%
\end{pgfscope}%
\begin{pgfscope}%
\definecolor{textcolor}{rgb}{0.150000,0.150000,0.150000}%
\pgfsetstrokecolor{textcolor}%
\pgfsetfillcolor{textcolor}%
\pgftext[x=0.501968in,y=7.614551in,left,base]{\color{textcolor}\sffamily\fontsize{11.000000}{13.200000}\selectfont \(\displaystyle -4\)}%
\end{pgfscope}%
\begin{pgfscope}%
\pgfpathrectangle{\pgfqpoint{0.828241in}{6.184745in}}{\pgfqpoint{2.414722in}{2.357859in}}%
\pgfusepath{clip}%
\pgfsetroundcap%
\pgfsetroundjoin%
\pgfsetlinewidth{1.003750pt}%
\definecolor{currentstroke}{rgb}{1.000000,1.000000,1.000000}%
\pgfsetstrokecolor{currentstroke}%
\pgfsetdash{}{0pt}%
\pgfpathmoveto{\pgfqpoint{0.828241in}{8.016231in}}%
\pgfpathlineto{\pgfqpoint{3.242963in}{8.016231in}}%
\pgfusepath{stroke}%
\end{pgfscope}%
\begin{pgfscope}%
\definecolor{textcolor}{rgb}{0.150000,0.150000,0.150000}%
\pgfsetstrokecolor{textcolor}%
\pgfsetfillcolor{textcolor}%
\pgftext[x=0.501968in,y=7.963424in,left,base]{\color{textcolor}\sffamily\fontsize{11.000000}{13.200000}\selectfont \(\displaystyle -2\)}%
\end{pgfscope}%
\begin{pgfscope}%
\pgfpathrectangle{\pgfqpoint{0.828241in}{6.184745in}}{\pgfqpoint{2.414722in}{2.357859in}}%
\pgfusepath{clip}%
\pgfsetroundcap%
\pgfsetroundjoin%
\pgfsetlinewidth{1.003750pt}%
\definecolor{currentstroke}{rgb}{1.000000,1.000000,1.000000}%
\pgfsetstrokecolor{currentstroke}%
\pgfsetdash{}{0pt}%
\pgfpathmoveto{\pgfqpoint{0.828241in}{8.365104in}}%
\pgfpathlineto{\pgfqpoint{3.242963in}{8.365104in}}%
\pgfusepath{stroke}%
\end{pgfscope}%
\begin{pgfscope}%
\definecolor{textcolor}{rgb}{0.150000,0.150000,0.150000}%
\pgfsetstrokecolor{textcolor}%
\pgfsetfillcolor{textcolor}%
\pgftext[x=0.620255in,y=8.312297in,left,base]{\color{textcolor}\sffamily\fontsize{11.000000}{13.200000}\selectfont \(\displaystyle 0\)}%
\end{pgfscope}%
\begin{pgfscope}%
\definecolor{textcolor}{rgb}{0.150000,0.150000,0.150000}%
\pgfsetstrokecolor{textcolor}%
\pgfsetfillcolor{textcolor}%
\pgftext[x=0.370370in,y=7.363675in,,bottom,rotate=90.000000]{\color{textcolor}\sffamily\fontsize{11.000000}{13.200000}\selectfont 4CH/gls}%
\end{pgfscope}%
\begin{pgfscope}%
\pgfpathrectangle{\pgfqpoint{0.828241in}{6.184745in}}{\pgfqpoint{2.414722in}{2.357859in}}%
\pgfusepath{clip}%
\pgfsetroundcap%
\pgfsetroundjoin%
\pgfsetlinewidth{1.505625pt}%
\definecolor{currentstroke}{rgb}{0.298039,0.447059,0.690196}%
\pgfsetstrokecolor{currentstroke}%
\pgfsetdash{}{0pt}%
\pgfpathmoveto{\pgfqpoint{0.938001in}{7.401988in}}%
\pgfpathlineto{\pgfqpoint{0.965848in}{7.405310in}}%
\pgfpathlineto{\pgfqpoint{0.993695in}{7.427198in}}%
\pgfpathlineto{\pgfqpoint{1.021542in}{7.489106in}}%
\pgfpathlineto{\pgfqpoint{1.049389in}{7.607691in}}%
\pgfpathlineto{\pgfqpoint{1.077236in}{7.781760in}}%
\pgfpathlineto{\pgfqpoint{1.105083in}{7.982943in}}%
\pgfpathlineto{\pgfqpoint{1.132930in}{8.165701in}}%
\pgfpathlineto{\pgfqpoint{1.160777in}{8.293740in}}%
\pgfpathlineto{\pgfqpoint{1.188624in}{8.355952in}}%
\pgfpathlineto{\pgfqpoint{1.216471in}{8.365104in}}%
\pgfpathlineto{\pgfqpoint{1.244318in}{8.351383in}}%
\pgfpathlineto{\pgfqpoint{1.272166in}{8.343586in}}%
\pgfpathlineto{\pgfqpoint{1.300013in}{8.349267in}}%
\pgfpathlineto{\pgfqpoint{1.327860in}{8.351905in}}%
\pgfpathlineto{\pgfqpoint{1.355707in}{8.325741in}}%
\pgfpathlineto{\pgfqpoint{1.383554in}{8.251649in}}%
\pgfpathlineto{\pgfqpoint{1.411401in}{8.125814in}}%
\pgfpathlineto{\pgfqpoint{1.439248in}{7.960147in}}%
\pgfpathlineto{\pgfqpoint{1.467095in}{7.774275in}}%
\pgfpathlineto{\pgfqpoint{1.494942in}{7.583480in}}%
\pgfpathlineto{\pgfqpoint{1.522789in}{7.394911in}}%
\pgfpathlineto{\pgfqpoint{1.550636in}{7.215071in}}%
\pgfpathlineto{\pgfqpoint{1.578483in}{7.053001in}}%
\pgfpathlineto{\pgfqpoint{1.606330in}{6.913124in}}%
\pgfpathlineto{\pgfqpoint{1.634177in}{6.791128in}}%
\pgfpathlineto{\pgfqpoint{1.662024in}{6.679900in}}%
\pgfpathlineto{\pgfqpoint{1.689871in}{6.577519in}}%
\pgfpathlineto{\pgfqpoint{1.717718in}{6.488328in}}%
\pgfpathlineto{\pgfqpoint{1.745565in}{6.420646in}}%
\pgfpathlineto{\pgfqpoint{1.773412in}{6.379673in}}%
\pgfpathlineto{\pgfqpoint{1.801259in}{6.361350in}}%
\pgfpathlineto{\pgfqpoint{1.829106in}{6.352944in}}%
\pgfpathlineto{\pgfqpoint{1.856953in}{6.341419in}}%
\pgfpathlineto{\pgfqpoint{1.884800in}{6.322285in}}%
\pgfpathlineto{\pgfqpoint{1.912647in}{6.301919in}}%
\pgfpathlineto{\pgfqpoint{1.940494in}{6.291921in}}%
\pgfpathlineto{\pgfqpoint{1.968341in}{6.298354in}}%
\pgfpathlineto{\pgfqpoint{1.996188in}{6.325437in}}%
\pgfpathlineto{\pgfqpoint{2.024035in}{6.379544in}}%
\pgfpathlineto{\pgfqpoint{2.051882in}{6.464187in}}%
\pgfpathlineto{\pgfqpoint{2.079729in}{6.578834in}}%
\pgfpathlineto{\pgfqpoint{2.107576in}{6.715174in}}%
\pgfpathlineto{\pgfqpoint{2.135423in}{6.856110in}}%
\pgfpathlineto{\pgfqpoint{2.163270in}{6.983739in}}%
\pgfpathlineto{\pgfqpoint{2.191117in}{7.088658in}}%
\pgfpathlineto{\pgfqpoint{2.218964in}{7.170488in}}%
\pgfpathlineto{\pgfqpoint{2.246811in}{7.232193in}}%
\pgfpathlineto{\pgfqpoint{2.274658in}{7.275026in}}%
\pgfpathlineto{\pgfqpoint{2.302505in}{7.298524in}}%
\pgfpathlineto{\pgfqpoint{2.330352in}{7.305196in}}%
\pgfpathlineto{\pgfqpoint{2.358199in}{7.306276in}}%
\pgfpathlineto{\pgfqpoint{2.386046in}{7.324094in}}%
\pgfpathlineto{\pgfqpoint{2.413893in}{7.389664in}}%
\pgfpathlineto{\pgfqpoint{2.441740in}{7.531601in}}%
\pgfpathlineto{\pgfqpoint{2.469587in}{7.752841in}}%
\pgfpathlineto{\pgfqpoint{2.497434in}{8.011594in}}%
\pgfpathlineto{\pgfqpoint{2.525281in}{8.238520in}}%
\pgfpathlineto{\pgfqpoint{2.553128in}{8.381983in}}%
\pgfpathlineto{\pgfqpoint{2.580975in}{8.431680in}}%
\pgfpathlineto{\pgfqpoint{2.608822in}{8.411936in}}%
\pgfpathlineto{\pgfqpoint{2.636669in}{8.365104in}}%
\pgfpathlineto{\pgfqpoint{2.664516in}{8.331193in}}%
\pgfpathlineto{\pgfqpoint{2.692363in}{8.327780in}}%
\pgfpathlineto{\pgfqpoint{2.720210in}{8.341182in}}%
\pgfpathlineto{\pgfqpoint{2.748057in}{8.332819in}}%
\pgfpathlineto{\pgfqpoint{2.775904in}{8.265223in}}%
\pgfpathlineto{\pgfqpoint{2.803751in}{8.130158in}}%
\pgfpathlineto{\pgfqpoint{2.831598in}{7.948294in}}%
\pgfpathlineto{\pgfqpoint{2.859445in}{7.748114in}}%
\pgfpathlineto{\pgfqpoint{2.887292in}{7.551765in}}%
\pgfpathlineto{\pgfqpoint{2.915139in}{7.365921in}}%
\pgfusepath{stroke}%
\end{pgfscope}%
\begin{pgfscope}%
\pgfpathrectangle{\pgfqpoint{0.828241in}{6.184745in}}{\pgfqpoint{2.414722in}{2.357859in}}%
\pgfusepath{clip}%
\pgfsetroundcap%
\pgfsetroundjoin%
\pgfsetlinewidth{1.505625pt}%
\definecolor{currentstroke}{rgb}{0.866667,0.517647,0.321569}%
\pgfsetstrokecolor{currentstroke}%
\pgfsetdash{}{0pt}%
\pgfpathmoveto{\pgfqpoint{0.938001in}{8.092148in}}%
\pgfpathlineto{\pgfqpoint{0.965848in}{8.099020in}}%
\pgfpathlineto{\pgfqpoint{0.993695in}{8.115012in}}%
\pgfpathlineto{\pgfqpoint{1.021542in}{8.146761in}}%
\pgfpathlineto{\pgfqpoint{1.049389in}{8.194984in}}%
\pgfpathlineto{\pgfqpoint{1.077236in}{8.251747in}}%
\pgfpathlineto{\pgfqpoint{1.105083in}{8.305108in}}%
\pgfpathlineto{\pgfqpoint{1.132930in}{8.346833in}}%
\pgfpathlineto{\pgfqpoint{1.160777in}{8.374256in}}%
\pgfpathlineto{\pgfqpoint{1.188624in}{8.386212in}}%
\pgfpathlineto{\pgfqpoint{1.216471in}{8.382345in}}%
\pgfpathlineto{\pgfqpoint{1.244318in}{8.365104in}}%
\pgfpathlineto{\pgfqpoint{1.272166in}{8.340411in}}%
\pgfpathlineto{\pgfqpoint{1.300013in}{8.316978in}}%
\pgfpathlineto{\pgfqpoint{1.327860in}{8.302966in}}%
\pgfpathlineto{\pgfqpoint{1.355707in}{8.301494in}}%
\pgfpathlineto{\pgfqpoint{1.383554in}{8.307907in}}%
\pgfpathlineto{\pgfqpoint{1.411401in}{8.312072in}}%
\pgfpathlineto{\pgfqpoint{1.439248in}{8.305407in}}%
\pgfpathlineto{\pgfqpoint{1.467095in}{8.283771in}}%
\pgfpathlineto{\pgfqpoint{1.494942in}{8.246704in}}%
\pgfpathlineto{\pgfqpoint{1.522789in}{8.195753in}}%
\pgfpathlineto{\pgfqpoint{1.550636in}{8.133219in}}%
\pgfpathlineto{\pgfqpoint{1.578483in}{8.063442in}}%
\pgfpathlineto{\pgfqpoint{1.606330in}{7.990366in}}%
\pgfpathlineto{\pgfqpoint{1.634177in}{7.913154in}}%
\pgfpathlineto{\pgfqpoint{1.662024in}{7.828970in}}%
\pgfpathlineto{\pgfqpoint{1.689871in}{7.742079in}}%
\pgfpathlineto{\pgfqpoint{1.717718in}{7.661210in}}%
\pgfpathlineto{\pgfqpoint{1.745565in}{7.591759in}}%
\pgfpathlineto{\pgfqpoint{1.773412in}{7.536960in}}%
\pgfpathlineto{\pgfqpoint{1.801259in}{7.500650in}}%
\pgfpathlineto{\pgfqpoint{1.829106in}{7.484523in}}%
\pgfpathlineto{\pgfqpoint{1.856953in}{7.483551in}}%
\pgfpathlineto{\pgfqpoint{1.884800in}{7.486658in}}%
\pgfpathlineto{\pgfqpoint{1.912647in}{7.482245in}}%
\pgfpathlineto{\pgfqpoint{1.940494in}{7.465906in}}%
\pgfpathlineto{\pgfqpoint{1.968341in}{7.447207in}}%
\pgfpathlineto{\pgfqpoint{1.996188in}{7.449483in}}%
\pgfpathlineto{\pgfqpoint{2.024035in}{7.494394in}}%
\pgfpathlineto{\pgfqpoint{2.051882in}{7.582419in}}%
\pgfpathlineto{\pgfqpoint{2.079729in}{7.694915in}}%
\pgfpathlineto{\pgfqpoint{2.107576in}{7.809695in}}%
\pgfpathlineto{\pgfqpoint{2.135423in}{7.908010in}}%
\pgfpathlineto{\pgfqpoint{2.163270in}{7.979173in}}%
\pgfpathlineto{\pgfqpoint{2.191117in}{8.023955in}}%
\pgfpathlineto{\pgfqpoint{2.218964in}{8.049685in}}%
\pgfpathlineto{\pgfqpoint{2.246811in}{8.062850in}}%
\pgfpathlineto{\pgfqpoint{2.274658in}{8.068335in}}%
\pgfpathlineto{\pgfqpoint{2.302505in}{8.071923in}}%
\pgfpathlineto{\pgfqpoint{2.330352in}{8.078664in}}%
\pgfpathlineto{\pgfqpoint{2.358199in}{8.089163in}}%
\pgfpathlineto{\pgfqpoint{2.386046in}{8.101721in}}%
\pgfpathlineto{\pgfqpoint{2.413893in}{8.116258in}}%
\pgfpathlineto{\pgfqpoint{2.441740in}{8.132768in}}%
\pgfpathlineto{\pgfqpoint{2.469587in}{8.148344in}}%
\pgfpathlineto{\pgfqpoint{2.497434in}{8.159355in}}%
\pgfpathlineto{\pgfqpoint{2.525281in}{8.167045in}}%
\pgfpathlineto{\pgfqpoint{2.553128in}{8.179082in}}%
\pgfpathlineto{\pgfqpoint{2.580975in}{8.205474in}}%
\pgfpathlineto{\pgfqpoint{2.608822in}{8.250734in}}%
\pgfpathlineto{\pgfqpoint{2.636669in}{8.308106in}}%
\pgfpathlineto{\pgfqpoint{2.664516in}{8.361185in}}%
\pgfpathlineto{\pgfqpoint{2.692363in}{8.394904in}}%
\pgfpathlineto{\pgfqpoint{2.720210in}{8.406299in}}%
\pgfpathlineto{\pgfqpoint{2.748057in}{8.402181in}}%
\pgfpathlineto{\pgfqpoint{2.775904in}{8.388216in}}%
\pgfpathlineto{\pgfqpoint{2.803751in}{8.365104in}}%
\pgfpathlineto{\pgfqpoint{2.831598in}{8.333851in}}%
\pgfpathlineto{\pgfqpoint{2.859445in}{8.300077in}}%
\pgfpathlineto{\pgfqpoint{2.887292in}{8.270396in}}%
\pgfpathlineto{\pgfqpoint{2.915139in}{8.246705in}}%
\pgfpathlineto{\pgfqpoint{2.942986in}{8.225612in}}%
\pgfpathlineto{\pgfqpoint{2.970833in}{8.200809in}}%
\pgfpathlineto{\pgfqpoint{2.998680in}{8.166436in}}%
\pgfpathlineto{\pgfqpoint{3.026527in}{8.121134in}}%
\pgfpathlineto{\pgfqpoint{3.054374in}{8.066494in}}%
\pgfpathlineto{\pgfqpoint{3.082221in}{8.002725in}}%
\pgfpathlineto{\pgfqpoint{3.110068in}{7.931126in}}%
\pgfusepath{stroke}%
\end{pgfscope}%
\begin{pgfscope}%
\pgfpathrectangle{\pgfqpoint{0.828241in}{6.184745in}}{\pgfqpoint{2.414722in}{2.357859in}}%
\pgfusepath{clip}%
\pgfsetroundcap%
\pgfsetroundjoin%
\pgfsetlinewidth{1.505625pt}%
\definecolor{currentstroke}{rgb}{0.333333,0.658824,0.407843}%
\pgfsetstrokecolor{currentstroke}%
\pgfsetdash{}{0pt}%
\pgfpathmoveto{\pgfqpoint{0.938001in}{7.456894in}}%
\pgfpathlineto{\pgfqpoint{0.964543in}{7.460789in}}%
\pgfpathlineto{\pgfqpoint{0.991085in}{7.474047in}}%
\pgfpathlineto{\pgfqpoint{1.017626in}{7.516596in}}%
\pgfpathlineto{\pgfqpoint{1.044168in}{7.611053in}}%
\pgfpathlineto{\pgfqpoint{1.070710in}{7.767908in}}%
\pgfpathlineto{\pgfqpoint{1.097251in}{7.966149in}}%
\pgfpathlineto{\pgfqpoint{1.123793in}{8.159193in}}%
\pgfpathlineto{\pgfqpoint{1.150335in}{8.299615in}}%
\pgfpathlineto{\pgfqpoint{1.176877in}{8.365104in}}%
\pgfpathlineto{\pgfqpoint{1.203418in}{8.369691in}}%
\pgfpathlineto{\pgfqpoint{1.229960in}{8.342707in}}%
\pgfpathlineto{\pgfqpoint{1.256502in}{8.297593in}}%
\pgfpathlineto{\pgfqpoint{1.283043in}{8.226331in}}%
\pgfpathlineto{\pgfqpoint{1.309585in}{8.118036in}}%
\pgfpathlineto{\pgfqpoint{1.336127in}{7.970268in}}%
\pgfpathlineto{\pgfqpoint{1.362668in}{7.788188in}}%
\pgfpathlineto{\pgfqpoint{1.389210in}{7.587810in}}%
\pgfpathlineto{\pgfqpoint{1.415752in}{7.395655in}}%
\pgfpathlineto{\pgfqpoint{1.442293in}{7.234381in}}%
\pgfpathlineto{\pgfqpoint{1.468835in}{7.109710in}}%
\pgfpathlineto{\pgfqpoint{1.495377in}{7.017117in}}%
\pgfpathlineto{\pgfqpoint{1.521918in}{6.941865in}}%
\pgfpathlineto{\pgfqpoint{1.548460in}{6.862155in}}%
\pgfpathlineto{\pgfqpoint{1.575002in}{6.767257in}}%
\pgfpathlineto{\pgfqpoint{1.601543in}{6.672579in}}%
\pgfpathlineto{\pgfqpoint{1.628085in}{6.600394in}}%
\pgfpathlineto{\pgfqpoint{1.654627in}{6.563246in}}%
\pgfpathlineto{\pgfqpoint{1.681169in}{6.556904in}}%
\pgfpathlineto{\pgfqpoint{1.707710in}{6.564279in}}%
\pgfpathlineto{\pgfqpoint{1.734252in}{6.566465in}}%
\pgfpathlineto{\pgfqpoint{1.760794in}{6.553319in}}%
\pgfpathlineto{\pgfqpoint{1.787335in}{6.526179in}}%
\pgfpathlineto{\pgfqpoint{1.813877in}{6.496568in}}%
\pgfpathlineto{\pgfqpoint{1.840419in}{6.482051in}}%
\pgfpathlineto{\pgfqpoint{1.866960in}{6.495671in}}%
\pgfpathlineto{\pgfqpoint{1.893502in}{6.537004in}}%
\pgfpathlineto{\pgfqpoint{1.920044in}{6.600904in}}%
\pgfpathlineto{\pgfqpoint{1.946585in}{6.686219in}}%
\pgfpathlineto{\pgfqpoint{1.973127in}{6.785927in}}%
\pgfpathlineto{\pgfqpoint{1.999669in}{6.887842in}}%
\pgfpathlineto{\pgfqpoint{2.026210in}{6.982515in}}%
\pgfpathlineto{\pgfqpoint{2.052752in}{7.064889in}}%
\pgfpathlineto{\pgfqpoint{2.079294in}{7.134615in}}%
\pgfpathlineto{\pgfqpoint{2.105836in}{7.195011in}}%
\pgfpathlineto{\pgfqpoint{2.132377in}{7.245854in}}%
\pgfpathlineto{\pgfqpoint{2.158919in}{7.284149in}}%
\pgfpathlineto{\pgfqpoint{2.185461in}{7.310663in}}%
\pgfpathlineto{\pgfqpoint{2.212002in}{7.327712in}}%
\pgfpathlineto{\pgfqpoint{2.238544in}{7.336517in}}%
\pgfpathlineto{\pgfqpoint{2.265086in}{7.340850in}}%
\pgfpathlineto{\pgfqpoint{2.291627in}{7.352820in}}%
\pgfpathlineto{\pgfqpoint{2.318169in}{7.399096in}}%
\pgfpathlineto{\pgfqpoint{2.344711in}{7.514283in}}%
\pgfpathlineto{\pgfqpoint{2.371252in}{7.710643in}}%
\pgfpathlineto{\pgfqpoint{2.397794in}{7.954919in}}%
\pgfpathlineto{\pgfqpoint{2.424336in}{8.180209in}}%
\pgfpathlineto{\pgfqpoint{2.450877in}{8.325970in}}%
\pgfpathlineto{\pgfqpoint{2.477419in}{8.365104in}}%
\pgfpathlineto{\pgfqpoint{2.503961in}{8.323290in}}%
\pgfusepath{stroke}%
\end{pgfscope}%
\begin{pgfscope}%
\pgfpathrectangle{\pgfqpoint{0.828241in}{6.184745in}}{\pgfqpoint{2.414722in}{2.357859in}}%
\pgfusepath{clip}%
\pgfsetroundcap%
\pgfsetroundjoin%
\pgfsetlinewidth{1.505625pt}%
\definecolor{currentstroke}{rgb}{0.768627,0.305882,0.321569}%
\pgfsetstrokecolor{currentstroke}%
\pgfsetdash{}{0pt}%
\pgfpathmoveto{\pgfqpoint{0.938001in}{8.120946in}}%
\pgfpathlineto{\pgfqpoint{0.964135in}{8.136158in}}%
\pgfpathlineto{\pgfqpoint{0.990268in}{8.160371in}}%
\pgfpathlineto{\pgfqpoint{1.016401in}{8.195572in}}%
\pgfpathlineto{\pgfqpoint{1.042535in}{8.237495in}}%
\pgfpathlineto{\pgfqpoint{1.068668in}{8.280473in}}%
\pgfpathlineto{\pgfqpoint{1.094801in}{8.320898in}}%
\pgfpathlineto{\pgfqpoint{1.120935in}{8.356999in}}%
\pgfpathlineto{\pgfqpoint{1.147068in}{8.386221in}}%
\pgfpathlineto{\pgfqpoint{1.173201in}{8.402965in}}%
\pgfpathlineto{\pgfqpoint{1.199335in}{8.402954in}}%
\pgfpathlineto{\pgfqpoint{1.225468in}{8.387940in}}%
\pgfpathlineto{\pgfqpoint{1.251602in}{8.365104in}}%
\pgfpathlineto{\pgfqpoint{1.277735in}{8.342808in}}%
\pgfpathlineto{\pgfqpoint{1.303868in}{8.327303in}}%
\pgfpathlineto{\pgfqpoint{1.330002in}{8.319722in}}%
\pgfpathlineto{\pgfqpoint{1.356135in}{8.316856in}}%
\pgfpathlineto{\pgfqpoint{1.382268in}{8.313449in}}%
\pgfpathlineto{\pgfqpoint{1.408402in}{8.303224in}}%
\pgfpathlineto{\pgfqpoint{1.434535in}{8.279321in}}%
\pgfpathlineto{\pgfqpoint{1.460668in}{8.236368in}}%
\pgfpathlineto{\pgfqpoint{1.486802in}{8.175324in}}%
\pgfpathlineto{\pgfqpoint{1.512935in}{8.106205in}}%
\pgfpathlineto{\pgfqpoint{1.539068in}{8.040991in}}%
\pgfpathlineto{\pgfqpoint{1.565202in}{7.983038in}}%
\pgfpathlineto{\pgfqpoint{1.591335in}{7.926190in}}%
\pgfpathlineto{\pgfqpoint{1.617468in}{7.862483in}}%
\pgfpathlineto{\pgfqpoint{1.643602in}{7.788081in}}%
\pgfpathlineto{\pgfqpoint{1.669735in}{7.705617in}}%
\pgfpathlineto{\pgfqpoint{1.695869in}{7.625446in}}%
\pgfpathlineto{\pgfqpoint{1.722002in}{7.558022in}}%
\pgfpathlineto{\pgfqpoint{1.748135in}{7.509151in}}%
\pgfpathlineto{\pgfqpoint{1.774269in}{7.482131in}}%
\pgfpathlineto{\pgfqpoint{1.800402in}{7.480510in}}%
\pgfpathlineto{\pgfqpoint{1.826535in}{7.502246in}}%
\pgfpathlineto{\pgfqpoint{1.852669in}{7.533496in}}%
\pgfpathlineto{\pgfqpoint{1.878802in}{7.556804in}}%
\pgfpathlineto{\pgfqpoint{1.904935in}{7.564120in}}%
\pgfpathlineto{\pgfqpoint{1.931069in}{7.556462in}}%
\pgfpathlineto{\pgfqpoint{1.957202in}{7.530828in}}%
\pgfpathlineto{\pgfqpoint{1.983335in}{7.482788in}}%
\pgfpathlineto{\pgfqpoint{2.009469in}{7.424902in}}%
\pgfpathlineto{\pgfqpoint{2.035602in}{7.390495in}}%
\pgfpathlineto{\pgfqpoint{2.061735in}{7.412200in}}%
\pgfpathlineto{\pgfqpoint{2.087869in}{7.493663in}}%
\pgfpathlineto{\pgfqpoint{2.114002in}{7.612543in}}%
\pgfpathlineto{\pgfqpoint{2.140136in}{7.737979in}}%
\pgfpathlineto{\pgfqpoint{2.166269in}{7.847975in}}%
\pgfpathlineto{\pgfqpoint{2.192402in}{7.936788in}}%
\pgfpathlineto{\pgfqpoint{2.218536in}{8.006847in}}%
\pgfpathlineto{\pgfqpoint{2.244669in}{8.057832in}}%
\pgfpathlineto{\pgfqpoint{2.270802in}{8.088381in}}%
\pgfpathlineto{\pgfqpoint{2.296936in}{8.101817in}}%
\pgfpathlineto{\pgfqpoint{2.323069in}{8.106608in}}%
\pgfpathlineto{\pgfqpoint{2.349202in}{8.111943in}}%
\pgfpathlineto{\pgfqpoint{2.375336in}{8.123827in}}%
\pgfpathlineto{\pgfqpoint{2.401469in}{8.143022in}}%
\pgfpathlineto{\pgfqpoint{2.427602in}{8.163244in}}%
\pgfpathlineto{\pgfqpoint{2.453736in}{8.175995in}}%
\pgfpathlineto{\pgfqpoint{2.479869in}{8.179388in}}%
\pgfpathlineto{\pgfqpoint{2.506002in}{8.179562in}}%
\pgfpathlineto{\pgfqpoint{2.532136in}{8.182958in}}%
\pgfpathlineto{\pgfqpoint{2.558269in}{8.192769in}}%
\pgfpathlineto{\pgfqpoint{2.584403in}{8.213305in}}%
\pgfpathlineto{\pgfqpoint{2.610536in}{8.250242in}}%
\pgfpathlineto{\pgfqpoint{2.636669in}{8.302029in}}%
\pgfpathlineto{\pgfqpoint{2.662803in}{8.357131in}}%
\pgfpathlineto{\pgfqpoint{2.688936in}{8.401560in}}%
\pgfpathlineto{\pgfqpoint{2.715069in}{8.427658in}}%
\pgfpathlineto{\pgfqpoint{2.741203in}{8.435429in}}%
\pgfpathlineto{\pgfqpoint{2.767336in}{8.426565in}}%
\pgfpathlineto{\pgfqpoint{2.793469in}{8.402143in}}%
\pgfpathlineto{\pgfqpoint{2.819603in}{8.365104in}}%
\pgfpathlineto{\pgfqpoint{2.845736in}{8.321941in}}%
\pgfpathlineto{\pgfqpoint{2.871869in}{8.281239in}}%
\pgfpathlineto{\pgfqpoint{2.898003in}{8.248477in}}%
\pgfpathlineto{\pgfqpoint{2.924136in}{8.223303in}}%
\pgfpathlineto{\pgfqpoint{2.950269in}{8.198239in}}%
\pgfpathlineto{\pgfqpoint{2.976403in}{8.163598in}}%
\pgfpathlineto{\pgfqpoint{3.002536in}{8.113996in}}%
\pgfpathlineto{\pgfqpoint{3.028669in}{8.050668in}}%
\pgfpathlineto{\pgfqpoint{3.054803in}{7.977800in}}%
\pgfpathlineto{\pgfqpoint{3.080936in}{7.899503in}}%
\pgfpathlineto{\pgfqpoint{3.107070in}{7.817566in}}%
\pgfpathlineto{\pgfqpoint{3.133203in}{7.732011in}}%
\pgfusepath{stroke}%
\end{pgfscope}%
\begin{pgfscope}%
\pgfpathrectangle{\pgfqpoint{0.828241in}{6.184745in}}{\pgfqpoint{2.414722in}{2.357859in}}%
\pgfusepath{clip}%
\pgfsetroundcap%
\pgfsetroundjoin%
\pgfsetlinewidth{1.505625pt}%
\definecolor{currentstroke}{rgb}{0.505882,0.447059,0.701961}%
\pgfsetstrokecolor{currentstroke}%
\pgfsetdash{}{0pt}%
\pgfpathmoveto{\pgfqpoint{0.938001in}{7.401988in}}%
\pgfpathlineto{\pgfqpoint{0.965848in}{7.405310in}}%
\pgfpathlineto{\pgfqpoint{0.993695in}{7.427198in}}%
\pgfpathlineto{\pgfqpoint{1.021542in}{7.489106in}}%
\pgfpathlineto{\pgfqpoint{1.049389in}{7.607691in}}%
\pgfpathlineto{\pgfqpoint{1.077236in}{7.781760in}}%
\pgfpathlineto{\pgfqpoint{1.105083in}{7.982943in}}%
\pgfpathlineto{\pgfqpoint{1.132930in}{8.165701in}}%
\pgfpathlineto{\pgfqpoint{1.160777in}{8.293740in}}%
\pgfpathlineto{\pgfqpoint{1.188624in}{8.355952in}}%
\pgfpathlineto{\pgfqpoint{1.216471in}{8.365104in}}%
\pgfpathlineto{\pgfqpoint{1.244318in}{8.351383in}}%
\pgfpathlineto{\pgfqpoint{1.272166in}{8.343586in}}%
\pgfpathlineto{\pgfqpoint{1.300013in}{8.349267in}}%
\pgfpathlineto{\pgfqpoint{1.327860in}{8.351905in}}%
\pgfpathlineto{\pgfqpoint{1.355707in}{8.325741in}}%
\pgfpathlineto{\pgfqpoint{1.383554in}{8.251649in}}%
\pgfpathlineto{\pgfqpoint{1.411401in}{8.125814in}}%
\pgfpathlineto{\pgfqpoint{1.439248in}{7.960147in}}%
\pgfpathlineto{\pgfqpoint{1.467095in}{7.774275in}}%
\pgfpathlineto{\pgfqpoint{1.494942in}{7.583480in}}%
\pgfpathlineto{\pgfqpoint{1.522789in}{7.394911in}}%
\pgfpathlineto{\pgfqpoint{1.550636in}{7.215071in}}%
\pgfpathlineto{\pgfqpoint{1.578483in}{7.053001in}}%
\pgfpathlineto{\pgfqpoint{1.606330in}{6.913124in}}%
\pgfpathlineto{\pgfqpoint{1.634177in}{6.791128in}}%
\pgfpathlineto{\pgfqpoint{1.662024in}{6.679900in}}%
\pgfpathlineto{\pgfqpoint{1.689871in}{6.577519in}}%
\pgfpathlineto{\pgfqpoint{1.717718in}{6.488328in}}%
\pgfpathlineto{\pgfqpoint{1.745565in}{6.420646in}}%
\pgfpathlineto{\pgfqpoint{1.773412in}{6.379673in}}%
\pgfpathlineto{\pgfqpoint{1.801259in}{6.361350in}}%
\pgfpathlineto{\pgfqpoint{1.829106in}{6.352944in}}%
\pgfpathlineto{\pgfqpoint{1.856953in}{6.341419in}}%
\pgfpathlineto{\pgfqpoint{1.884800in}{6.322285in}}%
\pgfpathlineto{\pgfqpoint{1.912647in}{6.301919in}}%
\pgfpathlineto{\pgfqpoint{1.940494in}{6.291921in}}%
\pgfpathlineto{\pgfqpoint{1.968341in}{6.298354in}}%
\pgfpathlineto{\pgfqpoint{1.996188in}{6.325437in}}%
\pgfpathlineto{\pgfqpoint{2.024035in}{6.379544in}}%
\pgfpathlineto{\pgfqpoint{2.051882in}{6.464187in}}%
\pgfpathlineto{\pgfqpoint{2.079729in}{6.578834in}}%
\pgfpathlineto{\pgfqpoint{2.107576in}{6.715174in}}%
\pgfpathlineto{\pgfqpoint{2.135423in}{6.856110in}}%
\pgfpathlineto{\pgfqpoint{2.163270in}{6.983739in}}%
\pgfpathlineto{\pgfqpoint{2.191117in}{7.088658in}}%
\pgfpathlineto{\pgfqpoint{2.218964in}{7.170488in}}%
\pgfpathlineto{\pgfqpoint{2.246811in}{7.232193in}}%
\pgfpathlineto{\pgfqpoint{2.274658in}{7.275026in}}%
\pgfpathlineto{\pgfqpoint{2.302505in}{7.298524in}}%
\pgfpathlineto{\pgfqpoint{2.330352in}{7.305196in}}%
\pgfpathlineto{\pgfqpoint{2.358199in}{7.306276in}}%
\pgfpathlineto{\pgfqpoint{2.386046in}{7.324094in}}%
\pgfpathlineto{\pgfqpoint{2.413893in}{7.389664in}}%
\pgfpathlineto{\pgfqpoint{2.441740in}{7.531601in}}%
\pgfpathlineto{\pgfqpoint{2.469587in}{7.752841in}}%
\pgfpathlineto{\pgfqpoint{2.497434in}{8.011594in}}%
\pgfpathlineto{\pgfqpoint{2.525281in}{8.238520in}}%
\pgfpathlineto{\pgfqpoint{2.553128in}{8.381983in}}%
\pgfpathlineto{\pgfqpoint{2.580975in}{8.431680in}}%
\pgfpathlineto{\pgfqpoint{2.608822in}{8.411936in}}%
\pgfpathlineto{\pgfqpoint{2.636669in}{8.365104in}}%
\pgfpathlineto{\pgfqpoint{2.664516in}{8.331193in}}%
\pgfpathlineto{\pgfqpoint{2.692363in}{8.327780in}}%
\pgfpathlineto{\pgfqpoint{2.720210in}{8.341182in}}%
\pgfpathlineto{\pgfqpoint{2.748057in}{8.332819in}}%
\pgfpathlineto{\pgfqpoint{2.775904in}{8.265223in}}%
\pgfpathlineto{\pgfqpoint{2.803751in}{8.130158in}}%
\pgfpathlineto{\pgfqpoint{2.831598in}{7.948294in}}%
\pgfpathlineto{\pgfqpoint{2.859445in}{7.748114in}}%
\pgfpathlineto{\pgfqpoint{2.887292in}{7.551765in}}%
\pgfpathlineto{\pgfqpoint{2.915139in}{7.365921in}}%
\pgfusepath{stroke}%
\end{pgfscope}%
\begin{pgfscope}%
\pgfsetrectcap%
\pgfsetmiterjoin%
\pgfsetlinewidth{1.254687pt}%
\definecolor{currentstroke}{rgb}{1.000000,1.000000,1.000000}%
\pgfsetstrokecolor{currentstroke}%
\pgfsetdash{}{0pt}%
\pgfpathmoveto{\pgfqpoint{0.828241in}{6.184745in}}%
\pgfpathlineto{\pgfqpoint{0.828241in}{8.542604in}}%
\pgfusepath{stroke}%
\end{pgfscope}%
\begin{pgfscope}%
\pgfsetrectcap%
\pgfsetmiterjoin%
\pgfsetlinewidth{1.254687pt}%
\definecolor{currentstroke}{rgb}{1.000000,1.000000,1.000000}%
\pgfsetstrokecolor{currentstroke}%
\pgfsetdash{}{0pt}%
\pgfpathmoveto{\pgfqpoint{3.242963in}{6.184745in}}%
\pgfpathlineto{\pgfqpoint{3.242963in}{8.542604in}}%
\pgfusepath{stroke}%
\end{pgfscope}%
\begin{pgfscope}%
\pgfsetrectcap%
\pgfsetmiterjoin%
\pgfsetlinewidth{1.254687pt}%
\definecolor{currentstroke}{rgb}{1.000000,1.000000,1.000000}%
\pgfsetstrokecolor{currentstroke}%
\pgfsetdash{}{0pt}%
\pgfpathmoveto{\pgfqpoint{0.828241in}{6.184745in}}%
\pgfpathlineto{\pgfqpoint{3.242963in}{6.184745in}}%
\pgfusepath{stroke}%
\end{pgfscope}%
\begin{pgfscope}%
\pgfsetrectcap%
\pgfsetmiterjoin%
\pgfsetlinewidth{1.254687pt}%
\definecolor{currentstroke}{rgb}{1.000000,1.000000,1.000000}%
\pgfsetstrokecolor{currentstroke}%
\pgfsetdash{}{0pt}%
\pgfpathmoveto{\pgfqpoint{0.828241in}{8.542604in}}%
\pgfpathlineto{\pgfqpoint{3.242963in}{8.542604in}}%
\pgfusepath{stroke}%
\end{pgfscope}%
\begin{pgfscope}%
\definecolor{textcolor}{rgb}{0.150000,0.150000,0.150000}%
\pgfsetstrokecolor{textcolor}%
\pgfsetfillcolor{textcolor}%
\pgftext[x=2.035602in,y=8.625938in,,base]{\color{textcolor}\sffamily\fontsize{11.000000}{13.200000}\selectfont (a) Cluster 1 members}%
\end{pgfscope}%
\begin{pgfscope}%
\pgfsetbuttcap%
\pgfsetmiterjoin%
\definecolor{currentfill}{rgb}{0.917647,0.917647,0.949020}%
\pgfsetfillcolor{currentfill}%
\pgfsetlinewidth{0.000000pt}%
\definecolor{currentstroke}{rgb}{0.000000,0.000000,0.000000}%
\pgfsetstrokecolor{currentstroke}%
\pgfsetstrokeopacity{0.000000}%
\pgfsetdash{}{0pt}%
\pgfpathmoveto{\pgfqpoint{3.825278in}{6.184745in}}%
\pgfpathlineto{\pgfqpoint{6.240000in}{6.184745in}}%
\pgfpathlineto{\pgfqpoint{6.240000in}{8.542604in}}%
\pgfpathlineto{\pgfqpoint{3.825278in}{8.542604in}}%
\pgfpathclose%
\pgfusepath{fill}%
\end{pgfscope}%
\begin{pgfscope}%
\pgfpathrectangle{\pgfqpoint{3.825278in}{6.184745in}}{\pgfqpoint{2.414722in}{2.357859in}}%
\pgfusepath{clip}%
\pgfsetroundcap%
\pgfsetroundjoin%
\pgfsetlinewidth{1.003750pt}%
\definecolor{currentstroke}{rgb}{1.000000,1.000000,1.000000}%
\pgfsetstrokecolor{currentstroke}%
\pgfsetdash{}{0pt}%
\pgfpathmoveto{\pgfqpoint{3.935038in}{6.184745in}}%
\pgfpathlineto{\pgfqpoint{3.935038in}{8.542604in}}%
\pgfusepath{stroke}%
\end{pgfscope}%
\begin{pgfscope}%
\definecolor{textcolor}{rgb}{0.150000,0.150000,0.150000}%
\pgfsetstrokecolor{textcolor}%
\pgfsetfillcolor{textcolor}%
\pgftext[x=3.935038in,y=6.052801in,,top]{\color{textcolor}\sffamily\fontsize{11.000000}{13.200000}\selectfont \(\displaystyle 0.0\)}%
\end{pgfscope}%
\begin{pgfscope}%
\pgfpathrectangle{\pgfqpoint{3.825278in}{6.184745in}}{\pgfqpoint{2.414722in}{2.357859in}}%
\pgfusepath{clip}%
\pgfsetroundcap%
\pgfsetroundjoin%
\pgfsetlinewidth{1.003750pt}%
\definecolor{currentstroke}{rgb}{1.000000,1.000000,1.000000}%
\pgfsetstrokecolor{currentstroke}%
\pgfsetdash{}{0pt}%
\pgfpathmoveto{\pgfqpoint{4.566990in}{6.184745in}}%
\pgfpathlineto{\pgfqpoint{4.566990in}{8.542604in}}%
\pgfusepath{stroke}%
\end{pgfscope}%
\begin{pgfscope}%
\definecolor{textcolor}{rgb}{0.150000,0.150000,0.150000}%
\pgfsetstrokecolor{textcolor}%
\pgfsetfillcolor{textcolor}%
\pgftext[x=4.566990in,y=6.052801in,,top]{\color{textcolor}\sffamily\fontsize{11.000000}{13.200000}\selectfont \(\displaystyle 0.5\)}%
\end{pgfscope}%
\begin{pgfscope}%
\pgfpathrectangle{\pgfqpoint{3.825278in}{6.184745in}}{\pgfqpoint{2.414722in}{2.357859in}}%
\pgfusepath{clip}%
\pgfsetroundcap%
\pgfsetroundjoin%
\pgfsetlinewidth{1.003750pt}%
\definecolor{currentstroke}{rgb}{1.000000,1.000000,1.000000}%
\pgfsetstrokecolor{currentstroke}%
\pgfsetdash{}{0pt}%
\pgfpathmoveto{\pgfqpoint{5.198942in}{6.184745in}}%
\pgfpathlineto{\pgfqpoint{5.198942in}{8.542604in}}%
\pgfusepath{stroke}%
\end{pgfscope}%
\begin{pgfscope}%
\definecolor{textcolor}{rgb}{0.150000,0.150000,0.150000}%
\pgfsetstrokecolor{textcolor}%
\pgfsetfillcolor{textcolor}%
\pgftext[x=5.198942in,y=6.052801in,,top]{\color{textcolor}\sffamily\fontsize{11.000000}{13.200000}\selectfont \(\displaystyle 1.0\)}%
\end{pgfscope}%
\begin{pgfscope}%
\pgfpathrectangle{\pgfqpoint{3.825278in}{6.184745in}}{\pgfqpoint{2.414722in}{2.357859in}}%
\pgfusepath{clip}%
\pgfsetroundcap%
\pgfsetroundjoin%
\pgfsetlinewidth{1.003750pt}%
\definecolor{currentstroke}{rgb}{1.000000,1.000000,1.000000}%
\pgfsetstrokecolor{currentstroke}%
\pgfsetdash{}{0pt}%
\pgfpathmoveto{\pgfqpoint{5.830894in}{6.184745in}}%
\pgfpathlineto{\pgfqpoint{5.830894in}{8.542604in}}%
\pgfusepath{stroke}%
\end{pgfscope}%
\begin{pgfscope}%
\definecolor{textcolor}{rgb}{0.150000,0.150000,0.150000}%
\pgfsetstrokecolor{textcolor}%
\pgfsetfillcolor{textcolor}%
\pgftext[x=5.830894in,y=6.052801in,,top]{\color{textcolor}\sffamily\fontsize{11.000000}{13.200000}\selectfont \(\displaystyle 1.5\)}%
\end{pgfscope}%
\begin{pgfscope}%
\pgfpathrectangle{\pgfqpoint{3.825278in}{6.184745in}}{\pgfqpoint{2.414722in}{2.357859in}}%
\pgfusepath{clip}%
\pgfsetroundcap%
\pgfsetroundjoin%
\pgfsetlinewidth{1.003750pt}%
\definecolor{currentstroke}{rgb}{1.000000,1.000000,1.000000}%
\pgfsetstrokecolor{currentstroke}%
\pgfsetdash{}{0pt}%
\pgfpathmoveto{\pgfqpoint{3.825278in}{6.529809in}}%
\pgfpathlineto{\pgfqpoint{6.240000in}{6.529809in}}%
\pgfusepath{stroke}%
\end{pgfscope}%
\begin{pgfscope}%
\definecolor{textcolor}{rgb}{0.150000,0.150000,0.150000}%
\pgfsetstrokecolor{textcolor}%
\pgfsetfillcolor{textcolor}%
\pgftext[x=3.422963in,y=6.477002in,left,base]{\color{textcolor}\sffamily\fontsize{11.000000}{13.200000}\selectfont \(\displaystyle -20\)}%
\end{pgfscope}%
\begin{pgfscope}%
\pgfpathrectangle{\pgfqpoint{3.825278in}{6.184745in}}{\pgfqpoint{2.414722in}{2.357859in}}%
\pgfusepath{clip}%
\pgfsetroundcap%
\pgfsetroundjoin%
\pgfsetlinewidth{1.003750pt}%
\definecolor{currentstroke}{rgb}{1.000000,1.000000,1.000000}%
\pgfsetstrokecolor{currentstroke}%
\pgfsetdash{}{0pt}%
\pgfpathmoveto{\pgfqpoint{3.825278in}{6.989780in}}%
\pgfpathlineto{\pgfqpoint{6.240000in}{6.989780in}}%
\pgfusepath{stroke}%
\end{pgfscope}%
\begin{pgfscope}%
\definecolor{textcolor}{rgb}{0.150000,0.150000,0.150000}%
\pgfsetstrokecolor{textcolor}%
\pgfsetfillcolor{textcolor}%
\pgftext[x=3.422963in,y=6.936973in,left,base]{\color{textcolor}\sffamily\fontsize{11.000000}{13.200000}\selectfont \(\displaystyle -15\)}%
\end{pgfscope}%
\begin{pgfscope}%
\pgfpathrectangle{\pgfqpoint{3.825278in}{6.184745in}}{\pgfqpoint{2.414722in}{2.357859in}}%
\pgfusepath{clip}%
\pgfsetroundcap%
\pgfsetroundjoin%
\pgfsetlinewidth{1.003750pt}%
\definecolor{currentstroke}{rgb}{1.000000,1.000000,1.000000}%
\pgfsetstrokecolor{currentstroke}%
\pgfsetdash{}{0pt}%
\pgfpathmoveto{\pgfqpoint{3.825278in}{7.449750in}}%
\pgfpathlineto{\pgfqpoint{6.240000in}{7.449750in}}%
\pgfusepath{stroke}%
\end{pgfscope}%
\begin{pgfscope}%
\definecolor{textcolor}{rgb}{0.150000,0.150000,0.150000}%
\pgfsetstrokecolor{textcolor}%
\pgfsetfillcolor{textcolor}%
\pgftext[x=3.422963in,y=7.396943in,left,base]{\color{textcolor}\sffamily\fontsize{11.000000}{13.200000}\selectfont \(\displaystyle -10\)}%
\end{pgfscope}%
\begin{pgfscope}%
\pgfpathrectangle{\pgfqpoint{3.825278in}{6.184745in}}{\pgfqpoint{2.414722in}{2.357859in}}%
\pgfusepath{clip}%
\pgfsetroundcap%
\pgfsetroundjoin%
\pgfsetlinewidth{1.003750pt}%
\definecolor{currentstroke}{rgb}{1.000000,1.000000,1.000000}%
\pgfsetstrokecolor{currentstroke}%
\pgfsetdash{}{0pt}%
\pgfpathmoveto{\pgfqpoint{3.825278in}{7.909721in}}%
\pgfpathlineto{\pgfqpoint{6.240000in}{7.909721in}}%
\pgfusepath{stroke}%
\end{pgfscope}%
\begin{pgfscope}%
\definecolor{textcolor}{rgb}{0.150000,0.150000,0.150000}%
\pgfsetstrokecolor{textcolor}%
\pgfsetfillcolor{textcolor}%
\pgftext[x=3.499005in,y=7.856914in,left,base]{\color{textcolor}\sffamily\fontsize{11.000000}{13.200000}\selectfont \(\displaystyle -5\)}%
\end{pgfscope}%
\begin{pgfscope}%
\pgfpathrectangle{\pgfqpoint{3.825278in}{6.184745in}}{\pgfqpoint{2.414722in}{2.357859in}}%
\pgfusepath{clip}%
\pgfsetroundcap%
\pgfsetroundjoin%
\pgfsetlinewidth{1.003750pt}%
\definecolor{currentstroke}{rgb}{1.000000,1.000000,1.000000}%
\pgfsetstrokecolor{currentstroke}%
\pgfsetdash{}{0pt}%
\pgfpathmoveto{\pgfqpoint{3.825278in}{8.369691in}}%
\pgfpathlineto{\pgfqpoint{6.240000in}{8.369691in}}%
\pgfusepath{stroke}%
\end{pgfscope}%
\begin{pgfscope}%
\definecolor{textcolor}{rgb}{0.150000,0.150000,0.150000}%
\pgfsetstrokecolor{textcolor}%
\pgfsetfillcolor{textcolor}%
\pgftext[x=3.617292in,y=8.316884in,left,base]{\color{textcolor}\sffamily\fontsize{11.000000}{13.200000}\selectfont \(\displaystyle 0\)}%
\end{pgfscope}%
\begin{pgfscope}%
\pgfpathrectangle{\pgfqpoint{3.825278in}{6.184745in}}{\pgfqpoint{2.414722in}{2.357859in}}%
\pgfusepath{clip}%
\pgfsetroundcap%
\pgfsetroundjoin%
\pgfsetlinewidth{1.505625pt}%
\definecolor{currentstroke}{rgb}{0.298039,0.447059,0.690196}%
\pgfsetstrokecolor{currentstroke}%
\pgfsetdash{}{0pt}%
\pgfpathmoveto{\pgfqpoint{3.935038in}{7.754719in}}%
\pgfpathlineto{\pgfqpoint{3.950452in}{7.754811in}}%
\pgfpathlineto{\pgfqpoint{3.965865in}{7.762000in}}%
\pgfpathlineto{\pgfqpoint{3.981279in}{7.784571in}}%
\pgfpathlineto{\pgfqpoint{3.996692in}{7.828940in}}%
\pgfpathlineto{\pgfqpoint{4.012106in}{7.895076in}}%
\pgfpathlineto{\pgfqpoint{4.027519in}{7.977349in}}%
\pgfpathlineto{\pgfqpoint{4.042933in}{8.069678in}}%
\pgfpathlineto{\pgfqpoint{4.058346in}{8.166785in}}%
\pgfpathlineto{\pgfqpoint{4.073760in}{8.258876in}}%
\pgfpathlineto{\pgfqpoint{4.089173in}{8.330618in}}%
\pgfpathlineto{\pgfqpoint{4.104586in}{8.372267in}}%
\pgfpathlineto{\pgfqpoint{4.120000in}{8.383853in}}%
\pgfpathlineto{\pgfqpoint{4.135413in}{8.369691in}}%
\pgfpathlineto{\pgfqpoint{4.150827in}{8.337296in}}%
\pgfpathlineto{\pgfqpoint{4.166240in}{8.300681in}}%
\pgfpathlineto{\pgfqpoint{4.181654in}{8.273148in}}%
\pgfpathlineto{\pgfqpoint{4.197067in}{8.254667in}}%
\pgfpathlineto{\pgfqpoint{4.212481in}{8.235041in}}%
\pgfpathlineto{\pgfqpoint{4.227894in}{8.201936in}}%
\pgfpathlineto{\pgfqpoint{4.243308in}{8.145225in}}%
\pgfpathlineto{\pgfqpoint{4.258721in}{8.061968in}}%
\pgfpathlineto{\pgfqpoint{4.274135in}{7.959206in}}%
\pgfpathlineto{\pgfqpoint{4.289548in}{7.852098in}}%
\pgfpathlineto{\pgfqpoint{4.304961in}{7.754387in}}%
\pgfpathlineto{\pgfqpoint{4.320375in}{7.669762in}}%
\pgfpathlineto{\pgfqpoint{4.335788in}{7.593504in}}%
\pgfpathlineto{\pgfqpoint{4.351202in}{7.518503in}}%
\pgfpathlineto{\pgfqpoint{4.366615in}{7.441444in}}%
\pgfpathlineto{\pgfqpoint{4.382029in}{7.364626in}}%
\pgfpathlineto{\pgfqpoint{4.397442in}{7.290862in}}%
\pgfpathlineto{\pgfqpoint{4.412856in}{7.220627in}}%
\pgfpathlineto{\pgfqpoint{4.428269in}{7.150516in}}%
\pgfpathlineto{\pgfqpoint{4.443683in}{7.075931in}}%
\pgfpathlineto{\pgfqpoint{4.459096in}{7.002069in}}%
\pgfpathlineto{\pgfqpoint{4.474510in}{6.933378in}}%
\pgfpathlineto{\pgfqpoint{4.489923in}{6.870360in}}%
\pgfpathlineto{\pgfqpoint{4.505336in}{6.813215in}}%
\pgfpathlineto{\pgfqpoint{4.520750in}{6.763321in}}%
\pgfpathlineto{\pgfqpoint{4.536163in}{6.723980in}}%
\pgfpathlineto{\pgfqpoint{4.551577in}{6.699656in}}%
\pgfpathlineto{\pgfqpoint{4.566990in}{6.694854in}}%
\pgfpathlineto{\pgfqpoint{4.582404in}{6.711801in}}%
\pgfpathlineto{\pgfqpoint{4.597817in}{6.749985in}}%
\pgfpathlineto{\pgfqpoint{4.613231in}{6.807068in}}%
\pgfpathlineto{\pgfqpoint{4.628644in}{6.877043in}}%
\pgfpathlineto{\pgfqpoint{4.644058in}{6.948071in}}%
\pgfpathlineto{\pgfqpoint{4.659471in}{7.007655in}}%
\pgfpathlineto{\pgfqpoint{4.674885in}{7.050225in}}%
\pgfpathlineto{\pgfqpoint{4.690298in}{7.076570in}}%
\pgfpathlineto{\pgfqpoint{4.705711in}{7.094618in}}%
\pgfpathlineto{\pgfqpoint{4.721125in}{7.117730in}}%
\pgfpathlineto{\pgfqpoint{4.736538in}{7.154772in}}%
\pgfpathlineto{\pgfqpoint{4.751952in}{7.205718in}}%
\pgfpathlineto{\pgfqpoint{4.767365in}{7.265510in}}%
\pgfpathlineto{\pgfqpoint{4.782779in}{7.331122in}}%
\pgfpathlineto{\pgfqpoint{4.798192in}{7.401702in}}%
\pgfpathlineto{\pgfqpoint{4.813606in}{7.475684in}}%
\pgfpathlineto{\pgfqpoint{4.829019in}{7.548040in}}%
\pgfpathlineto{\pgfqpoint{4.844433in}{7.610699in}}%
\pgfpathlineto{\pgfqpoint{4.859846in}{7.655743in}}%
\pgfpathlineto{\pgfqpoint{4.875260in}{7.681523in}}%
\pgfpathlineto{\pgfqpoint{4.890673in}{7.693281in}}%
\pgfpathlineto{\pgfqpoint{4.906086in}{7.697432in}}%
\pgfpathlineto{\pgfqpoint{4.921500in}{7.698424in}}%
\pgfpathlineto{\pgfqpoint{4.936913in}{7.699347in}}%
\pgfpathlineto{\pgfqpoint{4.952327in}{7.702057in}}%
\pgfpathlineto{\pgfqpoint{4.967740in}{7.706967in}}%
\pgfpathlineto{\pgfqpoint{4.983154in}{7.714952in}}%
\pgfpathlineto{\pgfqpoint{4.998567in}{7.729900in}}%
\pgfpathlineto{\pgfqpoint{5.013981in}{7.758852in}}%
\pgfpathlineto{\pgfqpoint{5.029394in}{7.808909in}}%
\pgfpathlineto{\pgfqpoint{5.044808in}{7.882297in}}%
\pgfpathlineto{\pgfqpoint{5.060221in}{7.976182in}}%
\pgfpathlineto{\pgfqpoint{5.075635in}{8.084303in}}%
\pgfpathlineto{\pgfqpoint{5.091048in}{8.198207in}}%
\pgfpathlineto{\pgfqpoint{5.106462in}{8.307294in}}%
\pgfpathlineto{\pgfqpoint{5.121875in}{8.393775in}}%
\pgfpathlineto{\pgfqpoint{5.137288in}{8.435429in}}%
\pgfpathlineto{\pgfqpoint{5.152702in}{8.423333in}}%
\pgfpathlineto{\pgfqpoint{5.168115in}{8.369691in}}%
\pgfpathlineto{\pgfqpoint{5.183529in}{8.301621in}}%
\pgfpathlineto{\pgfqpoint{5.198942in}{8.248728in}}%
\pgfpathlineto{\pgfqpoint{5.214356in}{8.225459in}}%
\pgfpathlineto{\pgfqpoint{5.229769in}{8.220226in}}%
\pgfpathlineto{\pgfqpoint{5.245183in}{8.208926in}}%
\pgfpathlineto{\pgfqpoint{5.260596in}{8.174003in}}%
\pgfpathlineto{\pgfqpoint{5.276010in}{8.110881in}}%
\pgfpathlineto{\pgfqpoint{5.291423in}{8.025517in}}%
\pgfpathlineto{\pgfqpoint{5.306837in}{7.927863in}}%
\pgfpathlineto{\pgfqpoint{5.322250in}{7.825535in}}%
\pgfpathlineto{\pgfqpoint{5.337663in}{7.726734in}}%
\pgfpathlineto{\pgfqpoint{5.353077in}{7.639264in}}%
\pgfpathlineto{\pgfqpoint{5.368490in}{7.562475in}}%
\pgfusepath{stroke}%
\end{pgfscope}%
\begin{pgfscope}%
\pgfpathrectangle{\pgfqpoint{3.825278in}{6.184745in}}{\pgfqpoint{2.414722in}{2.357859in}}%
\pgfusepath{clip}%
\pgfsetroundcap%
\pgfsetroundjoin%
\pgfsetlinewidth{1.505625pt}%
\definecolor{currentstroke}{rgb}{0.866667,0.517647,0.321569}%
\pgfsetstrokecolor{currentstroke}%
\pgfsetdash{}{0pt}%
\pgfpathmoveto{\pgfqpoint{3.935038in}{7.493373in}}%
\pgfpathlineto{\pgfqpoint{3.950452in}{7.473102in}}%
\pgfpathlineto{\pgfqpoint{3.965865in}{7.466264in}}%
\pgfpathlineto{\pgfqpoint{3.981279in}{7.482355in}}%
\pgfpathlineto{\pgfqpoint{3.996692in}{7.527519in}}%
\pgfpathlineto{\pgfqpoint{4.012106in}{7.606506in}}%
\pgfpathlineto{\pgfqpoint{4.027519in}{7.721322in}}%
\pgfpathlineto{\pgfqpoint{4.042933in}{7.867145in}}%
\pgfpathlineto{\pgfqpoint{4.058346in}{8.028476in}}%
\pgfpathlineto{\pgfqpoint{4.073760in}{8.181115in}}%
\pgfpathlineto{\pgfqpoint{4.089173in}{8.298275in}}%
\pgfpathlineto{\pgfqpoint{4.104586in}{8.362642in}}%
\pgfpathlineto{\pgfqpoint{4.120000in}{8.379677in}}%
\pgfpathlineto{\pgfqpoint{4.135413in}{8.369691in}}%
\pgfpathlineto{\pgfqpoint{4.150827in}{8.353447in}}%
\pgfpathlineto{\pgfqpoint{4.166240in}{8.344358in}}%
\pgfpathlineto{\pgfqpoint{4.181654in}{8.346355in}}%
\pgfpathlineto{\pgfqpoint{4.197067in}{8.352026in}}%
\pgfpathlineto{\pgfqpoint{4.212481in}{8.344546in}}%
\pgfpathlineto{\pgfqpoint{4.227894in}{8.305161in}}%
\pgfpathlineto{\pgfqpoint{4.243308in}{8.223595in}}%
\pgfpathlineto{\pgfqpoint{4.258721in}{8.105522in}}%
\pgfpathlineto{\pgfqpoint{4.274135in}{7.969599in}}%
\pgfpathlineto{\pgfqpoint{4.289548in}{7.834921in}}%
\pgfpathlineto{\pgfqpoint{4.304961in}{7.712488in}}%
\pgfpathlineto{\pgfqpoint{4.320375in}{7.602268in}}%
\pgfpathlineto{\pgfqpoint{4.335788in}{7.499315in}}%
\pgfpathlineto{\pgfqpoint{4.351202in}{7.401676in}}%
\pgfpathlineto{\pgfqpoint{4.366615in}{7.309431in}}%
\pgfpathlineto{\pgfqpoint{4.382029in}{7.220392in}}%
\pgfpathlineto{\pgfqpoint{4.397442in}{7.129261in}}%
\pgfpathlineto{\pgfqpoint{4.412856in}{7.032897in}}%
\pgfpathlineto{\pgfqpoint{4.428269in}{6.932738in}}%
\pgfpathlineto{\pgfqpoint{4.443683in}{6.830944in}}%
\pgfpathlineto{\pgfqpoint{4.459096in}{6.731854in}}%
\pgfpathlineto{\pgfqpoint{4.474510in}{6.639933in}}%
\pgfpathlineto{\pgfqpoint{4.489923in}{6.556534in}}%
\pgfpathlineto{\pgfqpoint{4.505336in}{6.481848in}}%
\pgfpathlineto{\pgfqpoint{4.520750in}{6.416509in}}%
\pgfpathlineto{\pgfqpoint{4.536163in}{6.361613in}}%
\pgfpathlineto{\pgfqpoint{4.551577in}{6.319006in}}%
\pgfpathlineto{\pgfqpoint{4.566990in}{6.293861in}}%
\pgfpathlineto{\pgfqpoint{4.582404in}{6.291921in}}%
\pgfpathlineto{\pgfqpoint{4.597817in}{6.312722in}}%
\pgfpathlineto{\pgfqpoint{4.613231in}{6.347578in}}%
\pgfpathlineto{\pgfqpoint{4.628644in}{6.385717in}}%
\pgfpathlineto{\pgfqpoint{4.644058in}{6.421910in}}%
\pgfpathlineto{\pgfqpoint{4.659471in}{6.457071in}}%
\pgfpathlineto{\pgfqpoint{4.674885in}{6.493784in}}%
\pgfpathlineto{\pgfqpoint{4.690298in}{6.532976in}}%
\pgfpathlineto{\pgfqpoint{4.705711in}{6.574333in}}%
\pgfpathlineto{\pgfqpoint{4.721125in}{6.620040in}}%
\pgfpathlineto{\pgfqpoint{4.736538in}{6.676207in}}%
\pgfpathlineto{\pgfqpoint{4.751952in}{6.749546in}}%
\pgfpathlineto{\pgfqpoint{4.767365in}{6.843564in}}%
\pgfpathlineto{\pgfqpoint{4.782779in}{6.956250in}}%
\pgfpathlineto{\pgfqpoint{4.798192in}{7.078191in}}%
\pgfpathlineto{\pgfqpoint{4.813606in}{7.197679in}}%
\pgfpathlineto{\pgfqpoint{4.829019in}{7.309680in}}%
\pgfpathlineto{\pgfqpoint{4.844433in}{7.410902in}}%
\pgfpathlineto{\pgfqpoint{4.859846in}{7.494870in}}%
\pgfpathlineto{\pgfqpoint{4.875260in}{7.555785in}}%
\pgfpathlineto{\pgfqpoint{4.890673in}{7.592599in}}%
\pgfpathlineto{\pgfqpoint{4.906086in}{7.610631in}}%
\pgfpathlineto{\pgfqpoint{4.921500in}{7.617472in}}%
\pgfpathlineto{\pgfqpoint{4.936913in}{7.617612in}}%
\pgfpathlineto{\pgfqpoint{4.952327in}{7.611452in}}%
\pgfpathlineto{\pgfqpoint{4.967740in}{7.600180in}}%
\pgfpathlineto{\pgfqpoint{4.983154in}{7.588380in}}%
\pgfpathlineto{\pgfqpoint{4.998567in}{7.583869in}}%
\pgfpathlineto{\pgfqpoint{5.013981in}{7.597978in}}%
\pgfpathlineto{\pgfqpoint{5.029394in}{7.642334in}}%
\pgfpathlineto{\pgfqpoint{5.044808in}{7.722168in}}%
\pgfpathlineto{\pgfqpoint{5.060221in}{7.831928in}}%
\pgfpathlineto{\pgfqpoint{5.075635in}{7.956914in}}%
\pgfpathlineto{\pgfqpoint{5.091048in}{8.081744in}}%
\pgfpathlineto{\pgfqpoint{5.106462in}{8.196579in}}%
\pgfpathlineto{\pgfqpoint{5.121875in}{8.292657in}}%
\pgfpathlineto{\pgfqpoint{5.137288in}{8.357634in}}%
\pgfpathlineto{\pgfqpoint{5.152702in}{8.382859in}}%
\pgfpathlineto{\pgfqpoint{5.168115in}{8.369691in}}%
\pgfpathlineto{\pgfqpoint{5.183529in}{8.332002in}}%
\pgfpathlineto{\pgfqpoint{5.198942in}{8.293518in}}%
\pgfpathlineto{\pgfqpoint{5.214356in}{8.273105in}}%
\pgfpathlineto{\pgfqpoint{5.229769in}{8.270037in}}%
\pgfpathlineto{\pgfqpoint{5.245183in}{8.266118in}}%
\pgfpathlineto{\pgfqpoint{5.260596in}{8.239400in}}%
\pgfpathlineto{\pgfqpoint{5.276010in}{8.177179in}}%
\pgfpathlineto{\pgfqpoint{5.291423in}{8.079788in}}%
\pgfpathlineto{\pgfqpoint{5.306837in}{7.958755in}}%
\pgfpathlineto{\pgfqpoint{5.322250in}{7.830418in}}%
\pgfpathlineto{\pgfqpoint{5.337663in}{7.709284in}}%
\pgfpathlineto{\pgfqpoint{5.353077in}{7.600032in}}%
\pgfpathlineto{\pgfqpoint{5.368490in}{7.497404in}}%
\pgfusepath{stroke}%
\end{pgfscope}%
\begin{pgfscope}%
\pgfpathrectangle{\pgfqpoint{3.825278in}{6.184745in}}{\pgfqpoint{2.414722in}{2.357859in}}%
\pgfusepath{clip}%
\pgfsetroundcap%
\pgfsetroundjoin%
\pgfsetlinewidth{1.505625pt}%
\definecolor{currentstroke}{rgb}{0.333333,0.658824,0.407843}%
\pgfsetstrokecolor{currentstroke}%
\pgfsetdash{}{0pt}%
\pgfpathmoveto{\pgfqpoint{3.935038in}{7.789594in}}%
\pgfpathlineto{\pgfqpoint{3.957212in}{7.801809in}}%
\pgfpathlineto{\pgfqpoint{3.979386in}{7.807614in}}%
\pgfpathlineto{\pgfqpoint{4.001560in}{7.803827in}}%
\pgfpathlineto{\pgfqpoint{4.023733in}{7.791346in}}%
\pgfpathlineto{\pgfqpoint{4.045907in}{7.776003in}}%
\pgfpathlineto{\pgfqpoint{4.068081in}{7.768499in}}%
\pgfpathlineto{\pgfqpoint{4.090255in}{7.780042in}}%
\pgfpathlineto{\pgfqpoint{4.112428in}{7.817804in}}%
\pgfpathlineto{\pgfqpoint{4.134602in}{7.883035in}}%
\pgfpathlineto{\pgfqpoint{4.156776in}{7.973068in}}%
\pgfpathlineto{\pgfqpoint{4.178950in}{8.082116in}}%
\pgfpathlineto{\pgfqpoint{4.201123in}{8.197713in}}%
\pgfpathlineto{\pgfqpoint{4.223297in}{8.300081in}}%
\pgfpathlineto{\pgfqpoint{4.245471in}{8.369691in}}%
\pgfpathlineto{\pgfqpoint{4.267645in}{8.396530in}}%
\pgfpathlineto{\pgfqpoint{4.289818in}{8.385327in}}%
\pgfpathlineto{\pgfqpoint{4.311992in}{8.353050in}}%
\pgfpathlineto{\pgfqpoint{4.334166in}{8.319215in}}%
\pgfpathlineto{\pgfqpoint{4.356340in}{8.295989in}}%
\pgfpathlineto{\pgfqpoint{4.378513in}{8.284179in}}%
\pgfpathlineto{\pgfqpoint{4.400687in}{8.277007in}}%
\pgfpathlineto{\pgfqpoint{4.422861in}{8.265522in}}%
\pgfpathlineto{\pgfqpoint{4.445035in}{8.241599in}}%
\pgfpathlineto{\pgfqpoint{4.467208in}{8.202431in}}%
\pgfpathlineto{\pgfqpoint{4.489382in}{8.151534in}}%
\pgfpathlineto{\pgfqpoint{4.511556in}{8.096397in}}%
\pgfpathlineto{\pgfqpoint{4.533730in}{8.042941in}}%
\pgfpathlineto{\pgfqpoint{4.555903in}{7.992983in}}%
\pgfpathlineto{\pgfqpoint{4.578077in}{7.945162in}}%
\pgfpathlineto{\pgfqpoint{4.600251in}{7.898137in}}%
\pgfpathlineto{\pgfqpoint{4.622425in}{7.852013in}}%
\pgfpathlineto{\pgfqpoint{4.644598in}{7.808625in}}%
\pgfpathlineto{\pgfqpoint{4.666772in}{7.771660in}}%
\pgfpathlineto{\pgfqpoint{4.688946in}{7.746702in}}%
\pgfpathlineto{\pgfqpoint{4.711120in}{7.739274in}}%
\pgfpathlineto{\pgfqpoint{4.733293in}{7.751218in}}%
\pgfpathlineto{\pgfqpoint{4.755467in}{7.778322in}}%
\pgfpathlineto{\pgfqpoint{4.777641in}{7.811921in}}%
\pgfpathlineto{\pgfqpoint{4.799815in}{7.843863in}}%
\pgfpathlineto{\pgfqpoint{4.821988in}{7.870033in}}%
\pgfpathlineto{\pgfqpoint{4.844162in}{7.889213in}}%
\pgfpathlineto{\pgfqpoint{4.866336in}{7.900572in}}%
\pgfpathlineto{\pgfqpoint{4.888510in}{7.902025in}}%
\pgfpathlineto{\pgfqpoint{4.910683in}{7.890349in}}%
\pgfpathlineto{\pgfqpoint{4.932857in}{7.866547in}}%
\pgfpathlineto{\pgfqpoint{4.955031in}{7.842015in}}%
\pgfpathlineto{\pgfqpoint{4.977205in}{7.831220in}}%
\pgfpathlineto{\pgfqpoint{4.999379in}{7.836307in}}%
\pgfpathlineto{\pgfqpoint{5.021552in}{7.847766in}}%
\pgfpathlineto{\pgfqpoint{5.043726in}{7.855836in}}%
\pgfpathlineto{\pgfqpoint{5.065900in}{7.857941in}}%
\pgfpathlineto{\pgfqpoint{5.088074in}{7.857182in}}%
\pgfpathlineto{\pgfqpoint{5.110247in}{7.857889in}}%
\pgfpathlineto{\pgfqpoint{5.132421in}{7.864101in}}%
\pgfpathlineto{\pgfqpoint{5.154595in}{7.877841in}}%
\pgfpathlineto{\pgfqpoint{5.176769in}{7.897629in}}%
\pgfpathlineto{\pgfqpoint{5.198942in}{7.920491in}}%
\pgfpathlineto{\pgfqpoint{5.221116in}{7.944609in}}%
\pgfpathlineto{\pgfqpoint{5.243290in}{7.969352in}}%
\pgfpathlineto{\pgfqpoint{5.265464in}{7.994423in}}%
\pgfpathlineto{\pgfqpoint{5.287637in}{8.018489in}}%
\pgfpathlineto{\pgfqpoint{5.309811in}{8.040020in}}%
\pgfpathlineto{\pgfqpoint{5.331985in}{8.057659in}}%
\pgfpathlineto{\pgfqpoint{5.354159in}{8.070934in}}%
\pgfpathlineto{\pgfqpoint{5.376332in}{8.080781in}}%
\pgfpathlineto{\pgfqpoint{5.398506in}{8.088063in}}%
\pgfpathlineto{\pgfqpoint{5.420680in}{8.091971in}}%
\pgfpathlineto{\pgfqpoint{5.442854in}{8.091723in}}%
\pgfpathlineto{\pgfqpoint{5.465027in}{8.088725in}}%
\pgfpathlineto{\pgfqpoint{5.487201in}{8.084561in}}%
\pgfpathlineto{\pgfqpoint{5.509375in}{8.079856in}}%
\pgfpathlineto{\pgfqpoint{5.531549in}{8.075237in}}%
\pgfpathlineto{\pgfqpoint{5.553722in}{8.071135in}}%
\pgfpathlineto{\pgfqpoint{5.575896in}{8.067281in}}%
\pgfpathlineto{\pgfqpoint{5.598070in}{8.063411in}}%
\pgfpathlineto{\pgfqpoint{5.620244in}{8.060756in}}%
\pgfpathlineto{\pgfqpoint{5.642417in}{8.065259in}}%
\pgfpathlineto{\pgfqpoint{5.664591in}{8.087262in}}%
\pgfpathlineto{\pgfqpoint{5.686765in}{8.133516in}}%
\pgfpathlineto{\pgfqpoint{5.708939in}{8.199673in}}%
\pgfpathlineto{\pgfqpoint{5.731112in}{8.270625in}}%
\pgfpathlineto{\pgfqpoint{5.753286in}{8.329061in}}%
\pgfpathlineto{\pgfqpoint{5.775460in}{8.364508in}}%
\pgfpathlineto{\pgfqpoint{5.797634in}{8.376262in}}%
\pgfpathlineto{\pgfqpoint{5.819807in}{8.369691in}}%
\pgfpathlineto{\pgfqpoint{5.841981in}{8.350635in}}%
\pgfpathlineto{\pgfqpoint{5.864155in}{8.322883in}}%
\pgfpathlineto{\pgfqpoint{5.886329in}{8.288085in}}%
\pgfpathlineto{\pgfqpoint{5.908502in}{8.245631in}}%
\pgfpathlineto{\pgfqpoint{5.930676in}{8.193859in}}%
\pgfpathlineto{\pgfqpoint{5.952850in}{8.132823in}}%
\pgfpathlineto{\pgfqpoint{5.975024in}{8.065431in}}%
\pgfpathlineto{\pgfqpoint{5.997197in}{7.995428in}}%
\pgfpathlineto{\pgfqpoint{6.019371in}{7.924366in}}%
\pgfpathlineto{\pgfqpoint{6.041545in}{7.853031in}}%
\pgfpathlineto{\pgfqpoint{6.063719in}{7.783709in}}%
\pgfpathlineto{\pgfqpoint{6.085892in}{7.719232in}}%
\pgfpathlineto{\pgfqpoint{6.108066in}{7.659252in}}%
\pgfpathlineto{\pgfqpoint{6.130240in}{7.601103in}}%
\pgfusepath{stroke}%
\end{pgfscope}%
\begin{pgfscope}%
\pgfpathrectangle{\pgfqpoint{3.825278in}{6.184745in}}{\pgfqpoint{2.414722in}{2.357859in}}%
\pgfusepath{clip}%
\pgfsetroundcap%
\pgfsetroundjoin%
\pgfsetlinewidth{1.505625pt}%
\definecolor{currentstroke}{rgb}{0.768627,0.305882,0.321569}%
\pgfsetstrokecolor{currentstroke}%
\pgfsetdash{}{0pt}%
\pgfpathmoveto{\pgfqpoint{3.935038in}{8.108027in}}%
\pgfpathlineto{\pgfqpoint{3.954787in}{8.107245in}}%
\pgfpathlineto{\pgfqpoint{3.974535in}{8.106251in}}%
\pgfpathlineto{\pgfqpoint{3.994284in}{8.105444in}}%
\pgfpathlineto{\pgfqpoint{4.014032in}{8.106507in}}%
\pgfpathlineto{\pgfqpoint{4.033781in}{8.113017in}}%
\pgfpathlineto{\pgfqpoint{4.053529in}{8.130826in}}%
\pgfpathlineto{\pgfqpoint{4.073278in}{8.165705in}}%
\pgfpathlineto{\pgfqpoint{4.093026in}{8.216613in}}%
\pgfpathlineto{\pgfqpoint{4.112775in}{8.272241in}}%
\pgfpathlineto{\pgfqpoint{4.132523in}{8.317911in}}%
\pgfpathlineto{\pgfqpoint{4.152272in}{8.346634in}}%
\pgfpathlineto{\pgfqpoint{4.172020in}{8.361465in}}%
\pgfpathlineto{\pgfqpoint{4.191769in}{8.368610in}}%
\pgfpathlineto{\pgfqpoint{4.211517in}{8.369691in}}%
\pgfpathlineto{\pgfqpoint{4.231266in}{8.360774in}}%
\pgfpathlineto{\pgfqpoint{4.251014in}{8.337426in}}%
\pgfpathlineto{\pgfqpoint{4.270763in}{8.299447in}}%
\pgfpathlineto{\pgfqpoint{4.290511in}{8.249125in}}%
\pgfpathlineto{\pgfqpoint{4.310260in}{8.188115in}}%
\pgfpathlineto{\pgfqpoint{4.330008in}{8.115011in}}%
\pgfpathlineto{\pgfqpoint{4.349757in}{8.029554in}}%
\pgfpathlineto{\pgfqpoint{4.369505in}{7.934406in}}%
\pgfpathlineto{\pgfqpoint{4.389254in}{7.833676in}}%
\pgfpathlineto{\pgfqpoint{4.409002in}{7.729792in}}%
\pgfpathlineto{\pgfqpoint{4.428751in}{7.625960in}}%
\pgfpathlineto{\pgfqpoint{4.448499in}{7.529171in}}%
\pgfpathlineto{\pgfqpoint{4.468248in}{7.446033in}}%
\pgfpathlineto{\pgfqpoint{4.487996in}{7.377200in}}%
\pgfpathlineto{\pgfqpoint{4.507745in}{7.318589in}}%
\pgfpathlineto{\pgfqpoint{4.527493in}{7.266676in}}%
\pgfpathlineto{\pgfqpoint{4.547242in}{7.221462in}}%
\pgfpathlineto{\pgfqpoint{4.566990in}{7.186164in}}%
\pgfpathlineto{\pgfqpoint{4.586739in}{7.164195in}}%
\pgfpathlineto{\pgfqpoint{4.606487in}{7.155546in}}%
\pgfpathlineto{\pgfqpoint{4.626236in}{7.156865in}}%
\pgfpathlineto{\pgfqpoint{4.645984in}{7.163081in}}%
\pgfpathlineto{\pgfqpoint{4.665733in}{7.169836in}}%
\pgfpathlineto{\pgfqpoint{4.685481in}{7.175636in}}%
\pgfpathlineto{\pgfqpoint{4.705230in}{7.181334in}}%
\pgfpathlineto{\pgfqpoint{4.724978in}{7.187306in}}%
\pgfpathlineto{\pgfqpoint{4.744727in}{7.192329in}}%
\pgfpathlineto{\pgfqpoint{4.764475in}{7.195956in}}%
\pgfpathlineto{\pgfqpoint{4.784224in}{7.202767in}}%
\pgfpathlineto{\pgfqpoint{4.803972in}{7.224520in}}%
\pgfpathlineto{\pgfqpoint{4.823721in}{7.275375in}}%
\pgfpathlineto{\pgfqpoint{4.843469in}{7.362268in}}%
\pgfpathlineto{\pgfqpoint{4.863218in}{7.479816in}}%
\pgfpathlineto{\pgfqpoint{4.882966in}{7.611468in}}%
\pgfpathlineto{\pgfqpoint{4.902715in}{7.736438in}}%
\pgfpathlineto{\pgfqpoint{4.922463in}{7.839379in}}%
\pgfpathlineto{\pgfqpoint{4.942212in}{7.915645in}}%
\pgfpathlineto{\pgfqpoint{4.961960in}{7.969909in}}%
\pgfpathlineto{\pgfqpoint{4.981709in}{8.010624in}}%
\pgfpathlineto{\pgfqpoint{5.001457in}{8.043701in}}%
\pgfpathlineto{\pgfqpoint{5.021206in}{8.070660in}}%
\pgfpathlineto{\pgfqpoint{5.040954in}{8.092585in}}%
\pgfpathlineto{\pgfqpoint{5.060703in}{8.111615in}}%
\pgfpathlineto{\pgfqpoint{5.080451in}{8.129150in}}%
\pgfpathlineto{\pgfqpoint{5.100200in}{8.144545in}}%
\pgfpathlineto{\pgfqpoint{5.119948in}{8.156513in}}%
\pgfpathlineto{\pgfqpoint{5.139697in}{8.165800in}}%
\pgfpathlineto{\pgfqpoint{5.159445in}{8.175295in}}%
\pgfpathlineto{\pgfqpoint{5.179194in}{8.187024in}}%
\pgfpathlineto{\pgfqpoint{5.198942in}{8.199328in}}%
\pgfpathlineto{\pgfqpoint{5.218691in}{8.208629in}}%
\pgfpathlineto{\pgfqpoint{5.238439in}{8.212222in}}%
\pgfpathlineto{\pgfqpoint{5.258188in}{8.209348in}}%
\pgfpathlineto{\pgfqpoint{5.277936in}{8.201643in}}%
\pgfpathlineto{\pgfqpoint{5.297685in}{8.191727in}}%
\pgfpathlineto{\pgfqpoint{5.317433in}{8.181903in}}%
\pgfpathlineto{\pgfqpoint{5.337182in}{8.174623in}}%
\pgfpathlineto{\pgfqpoint{5.356930in}{8.172272in}}%
\pgfpathlineto{\pgfqpoint{5.376679in}{8.174286in}}%
\pgfpathlineto{\pgfqpoint{5.396427in}{8.178160in}}%
\pgfpathlineto{\pgfqpoint{5.416176in}{8.182498in}}%
\pgfpathlineto{\pgfqpoint{5.435924in}{8.188892in}}%
\pgfpathlineto{\pgfqpoint{5.455673in}{8.201212in}}%
\pgfpathlineto{\pgfqpoint{5.475421in}{8.222950in}}%
\pgfpathlineto{\pgfqpoint{5.495170in}{8.254010in}}%
\pgfpathlineto{\pgfqpoint{5.514918in}{8.289957in}}%
\pgfpathlineto{\pgfqpoint{5.534667in}{8.324646in}}%
\pgfpathlineto{\pgfqpoint{5.554415in}{8.352822in}}%
\pgfpathlineto{\pgfqpoint{5.574164in}{8.372162in}}%
\pgfpathlineto{\pgfqpoint{5.593912in}{8.382203in}}%
\pgfpathlineto{\pgfqpoint{5.613661in}{8.382148in}}%
\pgfpathlineto{\pgfqpoint{5.633409in}{8.369691in}}%
\pgfpathlineto{\pgfqpoint{5.653158in}{8.341716in}}%
\pgfpathlineto{\pgfqpoint{5.672906in}{8.295683in}}%
\pgfpathlineto{\pgfqpoint{5.692655in}{8.230707in}}%
\pgfpathlineto{\pgfqpoint{5.712403in}{8.149436in}}%
\pgfpathlineto{\pgfqpoint{5.732152in}{8.057868in}}%
\pgfpathlineto{\pgfqpoint{5.751900in}{7.961700in}}%
\pgfpathlineto{\pgfqpoint{5.771649in}{7.862369in}}%
\pgfpathlineto{\pgfqpoint{5.791397in}{7.759388in}}%
\pgfpathlineto{\pgfqpoint{5.811146in}{7.652730in}}%
\pgfpathlineto{\pgfqpoint{5.830894in}{7.543065in}}%
\pgfpathlineto{\pgfqpoint{5.850643in}{7.432200in}}%
\pgfpathlineto{\pgfqpoint{5.870391in}{7.325490in}}%
\pgfpathlineto{\pgfqpoint{5.890140in}{7.229618in}}%
\pgfpathlineto{\pgfqpoint{5.909888in}{7.144169in}}%
\pgfusepath{stroke}%
\end{pgfscope}%
\begin{pgfscope}%
\pgfpathrectangle{\pgfqpoint{3.825278in}{6.184745in}}{\pgfqpoint{2.414722in}{2.357859in}}%
\pgfusepath{clip}%
\pgfsetroundcap%
\pgfsetroundjoin%
\pgfsetlinewidth{1.505625pt}%
\definecolor{currentstroke}{rgb}{0.505882,0.447059,0.701961}%
\pgfsetstrokecolor{currentstroke}%
\pgfsetdash{}{0pt}%
\pgfpathmoveto{\pgfqpoint{3.935038in}{7.860150in}}%
\pgfpathlineto{\pgfqpoint{3.955758in}{8.004046in}}%
\pgfpathlineto{\pgfqpoint{3.976478in}{8.142676in}}%
\pgfpathlineto{\pgfqpoint{3.997198in}{8.260877in}}%
\pgfpathlineto{\pgfqpoint{4.017917in}{8.341613in}}%
\pgfpathlineto{\pgfqpoint{4.038637in}{8.379869in}}%
\pgfpathlineto{\pgfqpoint{4.059357in}{8.384325in}}%
\pgfpathlineto{\pgfqpoint{4.080076in}{8.369691in}}%
\pgfpathlineto{\pgfqpoint{4.100796in}{8.349514in}}%
\pgfpathlineto{\pgfqpoint{4.121516in}{8.331136in}}%
\pgfpathlineto{\pgfqpoint{4.142236in}{8.311919in}}%
\pgfpathlineto{\pgfqpoint{4.162955in}{8.279957in}}%
\pgfpathlineto{\pgfqpoint{4.183675in}{8.224419in}}%
\pgfpathlineto{\pgfqpoint{4.204395in}{8.143623in}}%
\pgfpathlineto{\pgfqpoint{4.225115in}{8.044080in}}%
\pgfpathlineto{\pgfqpoint{4.245834in}{7.934895in}}%
\pgfpathlineto{\pgfqpoint{4.266554in}{7.823699in}}%
\pgfpathlineto{\pgfqpoint{4.287274in}{7.716348in}}%
\pgfpathlineto{\pgfqpoint{4.307994in}{7.616118in}}%
\pgfpathlineto{\pgfqpoint{4.328713in}{7.523205in}}%
\pgfpathlineto{\pgfqpoint{4.349433in}{7.436205in}}%
\pgfpathlineto{\pgfqpoint{4.370153in}{7.352414in}}%
\pgfpathlineto{\pgfqpoint{4.390873in}{7.271036in}}%
\pgfpathlineto{\pgfqpoint{4.411592in}{7.195756in}}%
\pgfpathlineto{\pgfqpoint{4.432312in}{7.127916in}}%
\pgfpathlineto{\pgfqpoint{4.453032in}{7.069014in}}%
\pgfpathlineto{\pgfqpoint{4.473751in}{7.021449in}}%
\pgfpathlineto{\pgfqpoint{4.494471in}{6.985908in}}%
\pgfpathlineto{\pgfqpoint{4.515191in}{6.959517in}}%
\pgfpathlineto{\pgfqpoint{4.535911in}{6.937199in}}%
\pgfpathlineto{\pgfqpoint{4.556630in}{6.915679in}}%
\pgfpathlineto{\pgfqpoint{4.577350in}{6.895272in}}%
\pgfpathlineto{\pgfqpoint{4.598070in}{6.878656in}}%
\pgfpathlineto{\pgfqpoint{4.618790in}{6.872690in}}%
\pgfpathlineto{\pgfqpoint{4.639509in}{6.888419in}}%
\pgfpathlineto{\pgfqpoint{4.660229in}{6.938355in}}%
\pgfpathlineto{\pgfqpoint{4.680949in}{7.029716in}}%
\pgfpathlineto{\pgfqpoint{4.701669in}{7.157101in}}%
\pgfpathlineto{\pgfqpoint{4.722388in}{7.304040in}}%
\pgfpathlineto{\pgfqpoint{4.743108in}{7.450970in}}%
\pgfpathlineto{\pgfqpoint{4.763828in}{7.582646in}}%
\pgfpathlineto{\pgfqpoint{4.784548in}{7.689307in}}%
\pgfpathlineto{\pgfqpoint{4.805267in}{7.764756in}}%
\pgfpathlineto{\pgfqpoint{4.825987in}{7.807850in}}%
\pgfpathlineto{\pgfqpoint{4.846707in}{7.823032in}}%
\pgfpathlineto{\pgfqpoint{4.867426in}{7.817970in}}%
\pgfpathlineto{\pgfqpoint{4.888146in}{7.801618in}}%
\pgfpathlineto{\pgfqpoint{4.908866in}{7.782142in}}%
\pgfpathlineto{\pgfqpoint{4.929586in}{7.765361in}}%
\pgfpathlineto{\pgfqpoint{4.950305in}{7.754261in}}%
\pgfpathlineto{\pgfqpoint{4.971025in}{7.750986in}}%
\pgfpathlineto{\pgfqpoint{4.991745in}{7.760991in}}%
\pgfpathlineto{\pgfqpoint{5.012465in}{7.795401in}}%
\pgfpathlineto{\pgfqpoint{5.033184in}{7.865280in}}%
\pgfpathlineto{\pgfqpoint{5.053904in}{7.971098in}}%
\pgfpathlineto{\pgfqpoint{5.074624in}{8.099382in}}%
\pgfpathlineto{\pgfqpoint{5.095344in}{8.227409in}}%
\pgfpathlineto{\pgfqpoint{5.116063in}{8.330975in}}%
\pgfpathlineto{\pgfqpoint{5.136783in}{8.393845in}}%
\pgfpathlineto{\pgfqpoint{5.157503in}{8.413752in}}%
\pgfpathlineto{\pgfqpoint{5.178223in}{8.400687in}}%
\pgfpathlineto{\pgfqpoint{5.198942in}{8.369691in}}%
\pgfpathlineto{\pgfqpoint{5.219662in}{8.333751in}}%
\pgfpathlineto{\pgfqpoint{5.240382in}{8.298974in}}%
\pgfpathlineto{\pgfqpoint{5.261101in}{8.262846in}}%
\pgfpathlineto{\pgfqpoint{5.281821in}{8.216805in}}%
\pgfpathlineto{\pgfqpoint{5.302541in}{8.154059in}}%
\pgfpathlineto{\pgfqpoint{5.323261in}{8.077817in}}%
\pgfusepath{stroke}%
\end{pgfscope}%
\begin{pgfscope}%
\pgfsetrectcap%
\pgfsetmiterjoin%
\pgfsetlinewidth{1.254687pt}%
\definecolor{currentstroke}{rgb}{1.000000,1.000000,1.000000}%
\pgfsetstrokecolor{currentstroke}%
\pgfsetdash{}{0pt}%
\pgfpathmoveto{\pgfqpoint{3.825278in}{6.184745in}}%
\pgfpathlineto{\pgfqpoint{3.825278in}{8.542604in}}%
\pgfusepath{stroke}%
\end{pgfscope}%
\begin{pgfscope}%
\pgfsetrectcap%
\pgfsetmiterjoin%
\pgfsetlinewidth{1.254687pt}%
\definecolor{currentstroke}{rgb}{1.000000,1.000000,1.000000}%
\pgfsetstrokecolor{currentstroke}%
\pgfsetdash{}{0pt}%
\pgfpathmoveto{\pgfqpoint{6.240000in}{6.184745in}}%
\pgfpathlineto{\pgfqpoint{6.240000in}{8.542604in}}%
\pgfusepath{stroke}%
\end{pgfscope}%
\begin{pgfscope}%
\pgfsetrectcap%
\pgfsetmiterjoin%
\pgfsetlinewidth{1.254687pt}%
\definecolor{currentstroke}{rgb}{1.000000,1.000000,1.000000}%
\pgfsetstrokecolor{currentstroke}%
\pgfsetdash{}{0pt}%
\pgfpathmoveto{\pgfqpoint{3.825278in}{6.184745in}}%
\pgfpathlineto{\pgfqpoint{6.240000in}{6.184745in}}%
\pgfusepath{stroke}%
\end{pgfscope}%
\begin{pgfscope}%
\pgfsetrectcap%
\pgfsetmiterjoin%
\pgfsetlinewidth{1.254687pt}%
\definecolor{currentstroke}{rgb}{1.000000,1.000000,1.000000}%
\pgfsetstrokecolor{currentstroke}%
\pgfsetdash{}{0pt}%
\pgfpathmoveto{\pgfqpoint{3.825278in}{8.542604in}}%
\pgfpathlineto{\pgfqpoint{6.240000in}{8.542604in}}%
\pgfusepath{stroke}%
\end{pgfscope}%
\begin{pgfscope}%
\definecolor{textcolor}{rgb}{0.150000,0.150000,0.150000}%
\pgfsetstrokecolor{textcolor}%
\pgfsetfillcolor{textcolor}%
\pgftext[x=5.032639in,y=8.625938in,,base]{\color{textcolor}\sffamily\fontsize{11.000000}{13.200000}\selectfont (b) Cluster 2 members}%
\end{pgfscope}%
\begin{pgfscope}%
\pgfsetbuttcap%
\pgfsetmiterjoin%
\definecolor{currentfill}{rgb}{0.917647,0.917647,0.949020}%
\pgfsetfillcolor{currentfill}%
\pgfsetlinewidth{0.000000pt}%
\definecolor{currentstroke}{rgb}{0.000000,0.000000,0.000000}%
\pgfsetstrokecolor{currentstroke}%
\pgfsetstrokeopacity{0.000000}%
\pgfsetdash{}{0pt}%
\pgfpathmoveto{\pgfqpoint{0.828241in}{3.379757in}}%
\pgfpathlineto{\pgfqpoint{3.242963in}{3.379757in}}%
\pgfpathlineto{\pgfqpoint{3.242963in}{5.737616in}}%
\pgfpathlineto{\pgfqpoint{0.828241in}{5.737616in}}%
\pgfpathclose%
\pgfusepath{fill}%
\end{pgfscope}%
\begin{pgfscope}%
\pgfpathrectangle{\pgfqpoint{0.828241in}{3.379757in}}{\pgfqpoint{2.414722in}{2.357859in}}%
\pgfusepath{clip}%
\pgfsetroundcap%
\pgfsetroundjoin%
\pgfsetlinewidth{1.003750pt}%
\definecolor{currentstroke}{rgb}{1.000000,1.000000,1.000000}%
\pgfsetstrokecolor{currentstroke}%
\pgfsetdash{}{0pt}%
\pgfpathmoveto{\pgfqpoint{0.938001in}{3.379757in}}%
\pgfpathlineto{\pgfqpoint{0.938001in}{5.737616in}}%
\pgfusepath{stroke}%
\end{pgfscope}%
\begin{pgfscope}%
\definecolor{textcolor}{rgb}{0.150000,0.150000,0.150000}%
\pgfsetstrokecolor{textcolor}%
\pgfsetfillcolor{textcolor}%
\pgftext[x=0.938001in,y=3.247812in,,top]{\color{textcolor}\sffamily\fontsize{11.000000}{13.200000}\selectfont \(\displaystyle 0.0\)}%
\end{pgfscope}%
\begin{pgfscope}%
\pgfpathrectangle{\pgfqpoint{0.828241in}{3.379757in}}{\pgfqpoint{2.414722in}{2.357859in}}%
\pgfusepath{clip}%
\pgfsetroundcap%
\pgfsetroundjoin%
\pgfsetlinewidth{1.003750pt}%
\definecolor{currentstroke}{rgb}{1.000000,1.000000,1.000000}%
\pgfsetstrokecolor{currentstroke}%
\pgfsetdash{}{0pt}%
\pgfpathmoveto{\pgfqpoint{1.787335in}{3.379757in}}%
\pgfpathlineto{\pgfqpoint{1.787335in}{5.737616in}}%
\pgfusepath{stroke}%
\end{pgfscope}%
\begin{pgfscope}%
\definecolor{textcolor}{rgb}{0.150000,0.150000,0.150000}%
\pgfsetstrokecolor{textcolor}%
\pgfsetfillcolor{textcolor}%
\pgftext[x=1.787335in,y=3.247812in,,top]{\color{textcolor}\sffamily\fontsize{11.000000}{13.200000}\selectfont \(\displaystyle 0.5\)}%
\end{pgfscope}%
\begin{pgfscope}%
\pgfpathrectangle{\pgfqpoint{0.828241in}{3.379757in}}{\pgfqpoint{2.414722in}{2.357859in}}%
\pgfusepath{clip}%
\pgfsetroundcap%
\pgfsetroundjoin%
\pgfsetlinewidth{1.003750pt}%
\definecolor{currentstroke}{rgb}{1.000000,1.000000,1.000000}%
\pgfsetstrokecolor{currentstroke}%
\pgfsetdash{}{0pt}%
\pgfpathmoveto{\pgfqpoint{2.636669in}{3.379757in}}%
\pgfpathlineto{\pgfqpoint{2.636669in}{5.737616in}}%
\pgfusepath{stroke}%
\end{pgfscope}%
\begin{pgfscope}%
\definecolor{textcolor}{rgb}{0.150000,0.150000,0.150000}%
\pgfsetstrokecolor{textcolor}%
\pgfsetfillcolor{textcolor}%
\pgftext[x=2.636669in,y=3.247812in,,top]{\color{textcolor}\sffamily\fontsize{11.000000}{13.200000}\selectfont \(\displaystyle 1.0\)}%
\end{pgfscope}%
\begin{pgfscope}%
\pgfpathrectangle{\pgfqpoint{0.828241in}{3.379757in}}{\pgfqpoint{2.414722in}{2.357859in}}%
\pgfusepath{clip}%
\pgfsetroundcap%
\pgfsetroundjoin%
\pgfsetlinewidth{1.003750pt}%
\definecolor{currentstroke}{rgb}{1.000000,1.000000,1.000000}%
\pgfsetstrokecolor{currentstroke}%
\pgfsetdash{}{0pt}%
\pgfpathmoveto{\pgfqpoint{0.828241in}{3.603407in}}%
\pgfpathlineto{\pgfqpoint{3.242963in}{3.603407in}}%
\pgfusepath{stroke}%
\end{pgfscope}%
\begin{pgfscope}%
\definecolor{textcolor}{rgb}{0.150000,0.150000,0.150000}%
\pgfsetstrokecolor{textcolor}%
\pgfsetfillcolor{textcolor}%
\pgftext[x=0.425926in,y=3.550601in,left,base]{\color{textcolor}\sffamily\fontsize{11.000000}{13.200000}\selectfont \(\displaystyle -10\)}%
\end{pgfscope}%
\begin{pgfscope}%
\pgfpathrectangle{\pgfqpoint{0.828241in}{3.379757in}}{\pgfqpoint{2.414722in}{2.357859in}}%
\pgfusepath{clip}%
\pgfsetroundcap%
\pgfsetroundjoin%
\pgfsetlinewidth{1.003750pt}%
\definecolor{currentstroke}{rgb}{1.000000,1.000000,1.000000}%
\pgfsetstrokecolor{currentstroke}%
\pgfsetdash{}{0pt}%
\pgfpathmoveto{\pgfqpoint{0.828241in}{3.995976in}}%
\pgfpathlineto{\pgfqpoint{3.242963in}{3.995976in}}%
\pgfusepath{stroke}%
\end{pgfscope}%
\begin{pgfscope}%
\definecolor{textcolor}{rgb}{0.150000,0.150000,0.150000}%
\pgfsetstrokecolor{textcolor}%
\pgfsetfillcolor{textcolor}%
\pgftext[x=0.501968in,y=3.943169in,left,base]{\color{textcolor}\sffamily\fontsize{11.000000}{13.200000}\selectfont \(\displaystyle -8\)}%
\end{pgfscope}%
\begin{pgfscope}%
\pgfpathrectangle{\pgfqpoint{0.828241in}{3.379757in}}{\pgfqpoint{2.414722in}{2.357859in}}%
\pgfusepath{clip}%
\pgfsetroundcap%
\pgfsetroundjoin%
\pgfsetlinewidth{1.003750pt}%
\definecolor{currentstroke}{rgb}{1.000000,1.000000,1.000000}%
\pgfsetstrokecolor{currentstroke}%
\pgfsetdash{}{0pt}%
\pgfpathmoveto{\pgfqpoint{0.828241in}{4.388545in}}%
\pgfpathlineto{\pgfqpoint{3.242963in}{4.388545in}}%
\pgfusepath{stroke}%
\end{pgfscope}%
\begin{pgfscope}%
\definecolor{textcolor}{rgb}{0.150000,0.150000,0.150000}%
\pgfsetstrokecolor{textcolor}%
\pgfsetfillcolor{textcolor}%
\pgftext[x=0.501968in,y=4.335738in,left,base]{\color{textcolor}\sffamily\fontsize{11.000000}{13.200000}\selectfont \(\displaystyle -6\)}%
\end{pgfscope}%
\begin{pgfscope}%
\pgfpathrectangle{\pgfqpoint{0.828241in}{3.379757in}}{\pgfqpoint{2.414722in}{2.357859in}}%
\pgfusepath{clip}%
\pgfsetroundcap%
\pgfsetroundjoin%
\pgfsetlinewidth{1.003750pt}%
\definecolor{currentstroke}{rgb}{1.000000,1.000000,1.000000}%
\pgfsetstrokecolor{currentstroke}%
\pgfsetdash{}{0pt}%
\pgfpathmoveto{\pgfqpoint{0.828241in}{4.781114in}}%
\pgfpathlineto{\pgfqpoint{3.242963in}{4.781114in}}%
\pgfusepath{stroke}%
\end{pgfscope}%
\begin{pgfscope}%
\definecolor{textcolor}{rgb}{0.150000,0.150000,0.150000}%
\pgfsetstrokecolor{textcolor}%
\pgfsetfillcolor{textcolor}%
\pgftext[x=0.501968in,y=4.728307in,left,base]{\color{textcolor}\sffamily\fontsize{11.000000}{13.200000}\selectfont \(\displaystyle -4\)}%
\end{pgfscope}%
\begin{pgfscope}%
\pgfpathrectangle{\pgfqpoint{0.828241in}{3.379757in}}{\pgfqpoint{2.414722in}{2.357859in}}%
\pgfusepath{clip}%
\pgfsetroundcap%
\pgfsetroundjoin%
\pgfsetlinewidth{1.003750pt}%
\definecolor{currentstroke}{rgb}{1.000000,1.000000,1.000000}%
\pgfsetstrokecolor{currentstroke}%
\pgfsetdash{}{0pt}%
\pgfpathmoveto{\pgfqpoint{0.828241in}{5.173682in}}%
\pgfpathlineto{\pgfqpoint{3.242963in}{5.173682in}}%
\pgfusepath{stroke}%
\end{pgfscope}%
\begin{pgfscope}%
\definecolor{textcolor}{rgb}{0.150000,0.150000,0.150000}%
\pgfsetstrokecolor{textcolor}%
\pgfsetfillcolor{textcolor}%
\pgftext[x=0.501968in,y=5.120876in,left,base]{\color{textcolor}\sffamily\fontsize{11.000000}{13.200000}\selectfont \(\displaystyle -2\)}%
\end{pgfscope}%
\begin{pgfscope}%
\pgfpathrectangle{\pgfqpoint{0.828241in}{3.379757in}}{\pgfqpoint{2.414722in}{2.357859in}}%
\pgfusepath{clip}%
\pgfsetroundcap%
\pgfsetroundjoin%
\pgfsetlinewidth{1.003750pt}%
\definecolor{currentstroke}{rgb}{1.000000,1.000000,1.000000}%
\pgfsetstrokecolor{currentstroke}%
\pgfsetdash{}{0pt}%
\pgfpathmoveto{\pgfqpoint{0.828241in}{5.566251in}}%
\pgfpathlineto{\pgfqpoint{3.242963in}{5.566251in}}%
\pgfusepath{stroke}%
\end{pgfscope}%
\begin{pgfscope}%
\definecolor{textcolor}{rgb}{0.150000,0.150000,0.150000}%
\pgfsetstrokecolor{textcolor}%
\pgfsetfillcolor{textcolor}%
\pgftext[x=0.620255in,y=5.513445in,left,base]{\color{textcolor}\sffamily\fontsize{11.000000}{13.200000}\selectfont \(\displaystyle 0\)}%
\end{pgfscope}%
\begin{pgfscope}%
\definecolor{textcolor}{rgb}{0.150000,0.150000,0.150000}%
\pgfsetstrokecolor{textcolor}%
\pgfsetfillcolor{textcolor}%
\pgftext[x=0.370370in,y=4.558686in,,bottom,rotate=90.000000]{\color{textcolor}\sffamily\fontsize{11.000000}{13.200000}\selectfont 2CH/gls}%
\end{pgfscope}%
\begin{pgfscope}%
\pgfpathrectangle{\pgfqpoint{0.828241in}{3.379757in}}{\pgfqpoint{2.414722in}{2.357859in}}%
\pgfusepath{clip}%
\pgfsetroundcap%
\pgfsetroundjoin%
\pgfsetlinewidth{1.505625pt}%
\definecolor{currentstroke}{rgb}{0.298039,0.447059,0.690196}%
\pgfsetstrokecolor{currentstroke}%
\pgfsetdash{}{0pt}%
\pgfpathmoveto{\pgfqpoint{0.938001in}{4.326237in}}%
\pgfpathlineto{\pgfqpoint{0.965848in}{4.369539in}}%
\pgfpathlineto{\pgfqpoint{0.993695in}{4.451137in}}%
\pgfpathlineto{\pgfqpoint{1.021542in}{4.605988in}}%
\pgfpathlineto{\pgfqpoint{1.049389in}{4.843846in}}%
\pgfpathlineto{\pgfqpoint{1.077236in}{5.126689in}}%
\pgfpathlineto{\pgfqpoint{1.105083in}{5.381522in}}%
\pgfpathlineto{\pgfqpoint{1.132930in}{5.548284in}}%
\pgfpathlineto{\pgfqpoint{1.160777in}{5.614621in}}%
\pgfpathlineto{\pgfqpoint{1.188624in}{5.607360in}}%
\pgfpathlineto{\pgfqpoint{1.216471in}{5.566251in}}%
\pgfpathlineto{\pgfqpoint{1.244318in}{5.525192in}}%
\pgfpathlineto{\pgfqpoint{1.272166in}{5.503824in}}%
\pgfpathlineto{\pgfqpoint{1.300013in}{5.501200in}}%
\pgfpathlineto{\pgfqpoint{1.327860in}{5.494199in}}%
\pgfpathlineto{\pgfqpoint{1.355707in}{5.451112in}}%
\pgfpathlineto{\pgfqpoint{1.383554in}{5.352215in}}%
\pgfpathlineto{\pgfqpoint{1.411401in}{5.198709in}}%
\pgfpathlineto{\pgfqpoint{1.439248in}{5.007689in}}%
\pgfpathlineto{\pgfqpoint{1.467095in}{4.802632in}}%
\pgfpathlineto{\pgfqpoint{1.494942in}{4.606888in}}%
\pgfpathlineto{\pgfqpoint{1.522789in}{4.435897in}}%
\pgfpathlineto{\pgfqpoint{1.550636in}{4.289741in}}%
\pgfpathlineto{\pgfqpoint{1.578483in}{4.160036in}}%
\pgfpathlineto{\pgfqpoint{1.606330in}{4.038761in}}%
\pgfpathlineto{\pgfqpoint{1.634177in}{3.921869in}}%
\pgfpathlineto{\pgfqpoint{1.662024in}{3.811252in}}%
\pgfpathlineto{\pgfqpoint{1.689871in}{3.712418in}}%
\pgfpathlineto{\pgfqpoint{1.717718in}{3.631457in}}%
\pgfpathlineto{\pgfqpoint{1.745565in}{3.573325in}}%
\pgfpathlineto{\pgfqpoint{1.773412in}{3.540357in}}%
\pgfpathlineto{\pgfqpoint{1.801259in}{3.530694in}}%
\pgfpathlineto{\pgfqpoint{1.829106in}{3.536296in}}%
\pgfpathlineto{\pgfqpoint{1.856953in}{3.543183in}}%
\pgfpathlineto{\pgfqpoint{1.884800in}{3.538107in}}%
\pgfpathlineto{\pgfqpoint{1.912647in}{3.519111in}}%
\pgfpathlineto{\pgfqpoint{1.940494in}{3.496701in}}%
\pgfpathlineto{\pgfqpoint{1.968341in}{3.486932in}}%
\pgfpathlineto{\pgfqpoint{1.996188in}{3.504659in}}%
\pgfpathlineto{\pgfqpoint{2.024035in}{3.557932in}}%
\pgfpathlineto{\pgfqpoint{2.051882in}{3.645250in}}%
\pgfpathlineto{\pgfqpoint{2.079729in}{3.756442in}}%
\pgfpathlineto{\pgfqpoint{2.107576in}{3.876526in}}%
\pgfpathlineto{\pgfqpoint{2.135423in}{3.991261in}}%
\pgfpathlineto{\pgfqpoint{2.163270in}{4.091667in}}%
\pgfpathlineto{\pgfqpoint{2.191117in}{4.174050in}}%
\pgfpathlineto{\pgfqpoint{2.218964in}{4.237206in}}%
\pgfpathlineto{\pgfqpoint{2.246811in}{4.280091in}}%
\pgfpathlineto{\pgfqpoint{2.274658in}{4.302571in}}%
\pgfpathlineto{\pgfqpoint{2.302505in}{4.306580in}}%
\pgfpathlineto{\pgfqpoint{2.330352in}{4.300049in}}%
\pgfpathlineto{\pgfqpoint{2.358199in}{4.302329in}}%
\pgfpathlineto{\pgfqpoint{2.386046in}{4.345656in}}%
\pgfpathlineto{\pgfqpoint{2.413893in}{4.465567in}}%
\pgfpathlineto{\pgfqpoint{2.441740in}{4.681371in}}%
\pgfpathlineto{\pgfqpoint{2.469587in}{4.973792in}}%
\pgfpathlineto{\pgfqpoint{2.497434in}{5.274464in}}%
\pgfpathlineto{\pgfqpoint{2.525281in}{5.500132in}}%
\pgfpathlineto{\pgfqpoint{2.553128in}{5.608489in}}%
\pgfpathlineto{\pgfqpoint{2.580975in}{5.614928in}}%
\pgfpathlineto{\pgfqpoint{2.608822in}{5.566251in}}%
\pgfpathlineto{\pgfqpoint{2.636669in}{5.507284in}}%
\pgfpathlineto{\pgfqpoint{2.664516in}{5.464301in}}%
\pgfpathlineto{\pgfqpoint{2.692363in}{5.442439in}}%
\pgfpathlineto{\pgfqpoint{2.720210in}{5.424445in}}%
\pgfpathlineto{\pgfqpoint{2.748057in}{5.379901in}}%
\pgfpathlineto{\pgfqpoint{2.775904in}{5.290642in}}%
\pgfpathlineto{\pgfqpoint{2.803751in}{5.161881in}}%
\pgfpathlineto{\pgfqpoint{2.831598in}{5.008032in}}%
\pgfpathlineto{\pgfqpoint{2.859445in}{4.840497in}}%
\pgfpathlineto{\pgfqpoint{2.887292in}{4.666988in}}%
\pgfusepath{stroke}%
\end{pgfscope}%
\begin{pgfscope}%
\pgfpathrectangle{\pgfqpoint{0.828241in}{3.379757in}}{\pgfqpoint{2.414722in}{2.357859in}}%
\pgfusepath{clip}%
\pgfsetroundcap%
\pgfsetroundjoin%
\pgfsetlinewidth{1.505625pt}%
\definecolor{currentstroke}{rgb}{0.866667,0.517647,0.321569}%
\pgfsetstrokecolor{currentstroke}%
\pgfsetdash{}{0pt}%
\pgfpathmoveto{\pgfqpoint{0.938001in}{5.132251in}}%
\pgfpathlineto{\pgfqpoint{0.965848in}{5.197168in}}%
\pgfpathlineto{\pgfqpoint{0.993695in}{5.271548in}}%
\pgfpathlineto{\pgfqpoint{1.021542in}{5.355872in}}%
\pgfpathlineto{\pgfqpoint{1.049389in}{5.439626in}}%
\pgfpathlineto{\pgfqpoint{1.077236in}{5.510358in}}%
\pgfpathlineto{\pgfqpoint{1.105083in}{5.561908in}}%
\pgfpathlineto{\pgfqpoint{1.132930in}{5.592332in}}%
\pgfpathlineto{\pgfqpoint{1.160777in}{5.601076in}}%
\pgfpathlineto{\pgfqpoint{1.188624in}{5.589893in}}%
\pgfpathlineto{\pgfqpoint{1.216471in}{5.566251in}}%
\pgfpathlineto{\pgfqpoint{1.244318in}{5.542583in}}%
\pgfpathlineto{\pgfqpoint{1.272166in}{5.529712in}}%
\pgfpathlineto{\pgfqpoint{1.300013in}{5.530076in}}%
\pgfpathlineto{\pgfqpoint{1.327860in}{5.536651in}}%
\pgfpathlineto{\pgfqpoint{1.355707in}{5.538147in}}%
\pgfpathlineto{\pgfqpoint{1.383554in}{5.527219in}}%
\pgfpathlineto{\pgfqpoint{1.411401in}{5.502925in}}%
\pgfpathlineto{\pgfqpoint{1.439248in}{5.465686in}}%
\pgfpathlineto{\pgfqpoint{1.467095in}{5.412208in}}%
\pgfpathlineto{\pgfqpoint{1.494942in}{5.340694in}}%
\pgfpathlineto{\pgfqpoint{1.522789in}{5.257088in}}%
\pgfpathlineto{\pgfqpoint{1.550636in}{5.171816in}}%
\pgfpathlineto{\pgfqpoint{1.578483in}{5.090947in}}%
\pgfpathlineto{\pgfqpoint{1.606330in}{5.013363in}}%
\pgfpathlineto{\pgfqpoint{1.634177in}{4.934970in}}%
\pgfpathlineto{\pgfqpoint{1.662024in}{4.856020in}}%
\pgfpathlineto{\pgfqpoint{1.689871in}{4.783348in}}%
\pgfpathlineto{\pgfqpoint{1.717718in}{4.727410in}}%
\pgfpathlineto{\pgfqpoint{1.745565in}{4.695639in}}%
\pgfpathlineto{\pgfqpoint{1.773412in}{4.687081in}}%
\pgfpathlineto{\pgfqpoint{1.801259in}{4.691856in}}%
\pgfpathlineto{\pgfqpoint{1.829106in}{4.694085in}}%
\pgfpathlineto{\pgfqpoint{1.856953in}{4.679525in}}%
\pgfpathlineto{\pgfqpoint{1.884800in}{4.642133in}}%
\pgfpathlineto{\pgfqpoint{1.912647in}{4.588097in}}%
\pgfpathlineto{\pgfqpoint{1.940494in}{4.534818in}}%
\pgfpathlineto{\pgfqpoint{1.968341in}{4.507534in}}%
\pgfpathlineto{\pgfqpoint{1.996188in}{4.528946in}}%
\pgfpathlineto{\pgfqpoint{2.024035in}{4.606300in}}%
\pgfpathlineto{\pgfqpoint{2.051882in}{4.724379in}}%
\pgfpathlineto{\pgfqpoint{2.079729in}{4.850236in}}%
\pgfpathlineto{\pgfqpoint{2.107576in}{4.952317in}}%
\pgfpathlineto{\pgfqpoint{2.135423in}{5.018653in}}%
\pgfpathlineto{\pgfqpoint{2.163270in}{5.056147in}}%
\pgfpathlineto{\pgfqpoint{2.191117in}{5.078440in}}%
\pgfpathlineto{\pgfqpoint{2.218964in}{5.097727in}}%
\pgfpathlineto{\pgfqpoint{2.246811in}{5.120448in}}%
\pgfpathlineto{\pgfqpoint{2.274658in}{5.145971in}}%
\pgfpathlineto{\pgfqpoint{2.302505in}{5.169823in}}%
\pgfpathlineto{\pgfqpoint{2.330352in}{5.189998in}}%
\pgfpathlineto{\pgfqpoint{2.358199in}{5.210630in}}%
\pgfpathlineto{\pgfqpoint{2.386046in}{5.241262in}}%
\pgfpathlineto{\pgfqpoint{2.413893in}{5.292074in}}%
\pgfpathlineto{\pgfqpoint{2.441740in}{5.365659in}}%
\pgfpathlineto{\pgfqpoint{2.469587in}{5.451396in}}%
\pgfpathlineto{\pgfqpoint{2.497434in}{5.530578in}}%
\pgfpathlineto{\pgfqpoint{2.525281in}{5.588436in}}%
\pgfpathlineto{\pgfqpoint{2.553128in}{5.621002in}}%
\pgfpathlineto{\pgfqpoint{2.580975in}{5.630440in}}%
\pgfpathlineto{\pgfqpoint{2.608822in}{5.620153in}}%
\pgfpathlineto{\pgfqpoint{2.636669in}{5.595907in}}%
\pgfpathlineto{\pgfqpoint{2.664516in}{5.566251in}}%
\pgfpathlineto{\pgfqpoint{2.692363in}{5.539961in}}%
\pgfpathlineto{\pgfqpoint{2.720210in}{5.522548in}}%
\pgfpathlineto{\pgfqpoint{2.748057in}{5.513010in}}%
\pgfpathlineto{\pgfqpoint{2.775904in}{5.504944in}}%
\pgfpathlineto{\pgfqpoint{2.803751in}{5.490977in}}%
\pgfpathlineto{\pgfqpoint{2.831598in}{5.466470in}}%
\pgfpathlineto{\pgfqpoint{2.859445in}{5.429807in}}%
\pgfpathlineto{\pgfqpoint{2.887292in}{5.380275in}}%
\pgfpathlineto{\pgfqpoint{2.915139in}{5.319120in}}%
\pgfpathlineto{\pgfqpoint{2.942986in}{5.251013in}}%
\pgfusepath{stroke}%
\end{pgfscope}%
\begin{pgfscope}%
\pgfpathrectangle{\pgfqpoint{0.828241in}{3.379757in}}{\pgfqpoint{2.414722in}{2.357859in}}%
\pgfusepath{clip}%
\pgfsetroundcap%
\pgfsetroundjoin%
\pgfsetlinewidth{1.505625pt}%
\definecolor{currentstroke}{rgb}{0.333333,0.658824,0.407843}%
\pgfsetstrokecolor{currentstroke}%
\pgfsetdash{}{0pt}%
\pgfpathmoveto{\pgfqpoint{0.938001in}{4.116143in}}%
\pgfpathlineto{\pgfqpoint{0.964543in}{4.104850in}}%
\pgfpathlineto{\pgfqpoint{0.991085in}{4.133874in}}%
\pgfpathlineto{\pgfqpoint{1.017626in}{4.239177in}}%
\pgfpathlineto{\pgfqpoint{1.044168in}{4.444148in}}%
\pgfpathlineto{\pgfqpoint{1.070710in}{4.740144in}}%
\pgfpathlineto{\pgfqpoint{1.097251in}{5.065483in}}%
\pgfpathlineto{\pgfqpoint{1.123793in}{5.335218in}}%
\pgfpathlineto{\pgfqpoint{1.150335in}{5.498269in}}%
\pgfpathlineto{\pgfqpoint{1.176877in}{5.566251in}}%
\pgfpathlineto{\pgfqpoint{1.203418in}{5.579383in}}%
\pgfpathlineto{\pgfqpoint{1.229960in}{5.572161in}}%
\pgfpathlineto{\pgfqpoint{1.256502in}{5.556129in}}%
\pgfpathlineto{\pgfqpoint{1.283043in}{5.521996in}}%
\pgfpathlineto{\pgfqpoint{1.309585in}{5.453054in}}%
\pgfpathlineto{\pgfqpoint{1.336127in}{5.339167in}}%
\pgfpathlineto{\pgfqpoint{1.362668in}{5.187955in}}%
\pgfpathlineto{\pgfqpoint{1.389210in}{5.018942in}}%
\pgfpathlineto{\pgfqpoint{1.415752in}{4.850118in}}%
\pgfpathlineto{\pgfqpoint{1.442293in}{4.688961in}}%
\pgfpathlineto{\pgfqpoint{1.468835in}{4.535579in}}%
\pgfpathlineto{\pgfqpoint{1.495377in}{4.392793in}}%
\pgfpathlineto{\pgfqpoint{1.521918in}{4.259838in}}%
\pgfpathlineto{\pgfqpoint{1.548460in}{4.138501in}}%
\pgfpathlineto{\pgfqpoint{1.575002in}{4.037369in}}%
\pgfpathlineto{\pgfqpoint{1.601543in}{3.967844in}}%
\pgfpathlineto{\pgfqpoint{1.628085in}{3.932622in}}%
\pgfpathlineto{\pgfqpoint{1.654627in}{3.924473in}}%
\pgfpathlineto{\pgfqpoint{1.681169in}{3.933262in}}%
\pgfpathlineto{\pgfqpoint{1.707710in}{3.947238in}}%
\pgfpathlineto{\pgfqpoint{1.734252in}{3.956118in}}%
\pgfpathlineto{\pgfqpoint{1.760794in}{3.953645in}}%
\pgfpathlineto{\pgfqpoint{1.787335in}{3.940267in}}%
\pgfpathlineto{\pgfqpoint{1.813877in}{3.926910in}}%
\pgfpathlineto{\pgfqpoint{1.840419in}{3.925895in}}%
\pgfpathlineto{\pgfqpoint{1.866960in}{3.940871in}}%
\pgfpathlineto{\pgfqpoint{1.893502in}{3.974104in}}%
\pgfpathlineto{\pgfqpoint{1.920044in}{4.026953in}}%
\pgfpathlineto{\pgfqpoint{1.946585in}{4.099498in}}%
\pgfpathlineto{\pgfqpoint{1.973127in}{4.188583in}}%
\pgfpathlineto{\pgfqpoint{1.999669in}{4.286459in}}%
\pgfpathlineto{\pgfqpoint{2.026210in}{4.381519in}}%
\pgfpathlineto{\pgfqpoint{2.052752in}{4.463486in}}%
\pgfpathlineto{\pgfqpoint{2.079294in}{4.526652in}}%
\pgfpathlineto{\pgfqpoint{2.105836in}{4.569228in}}%
\pgfpathlineto{\pgfqpoint{2.132377in}{4.592066in}}%
\pgfpathlineto{\pgfqpoint{2.158919in}{4.598195in}}%
\pgfpathlineto{\pgfqpoint{2.185461in}{4.590633in}}%
\pgfpathlineto{\pgfqpoint{2.212002in}{4.579267in}}%
\pgfpathlineto{\pgfqpoint{2.238544in}{4.589568in}}%
\pgfpathlineto{\pgfqpoint{2.265086in}{4.652719in}}%
\pgfpathlineto{\pgfqpoint{2.291627in}{4.784208in}}%
\pgfpathlineto{\pgfqpoint{2.318169in}{4.972325in}}%
\pgfpathlineto{\pgfqpoint{2.344711in}{5.182503in}}%
\pgfpathlineto{\pgfqpoint{2.371252in}{5.369967in}}%
\pgfpathlineto{\pgfqpoint{2.397794in}{5.501093in}}%
\pgfpathlineto{\pgfqpoint{2.424336in}{5.566251in}}%
\pgfpathlineto{\pgfqpoint{2.450877in}{5.581018in}}%
\pgfpathlineto{\pgfqpoint{2.477419in}{5.571812in}}%
\pgfpathlineto{\pgfqpoint{2.503961in}{5.557039in}}%
\pgfusepath{stroke}%
\end{pgfscope}%
\begin{pgfscope}%
\pgfpathrectangle{\pgfqpoint{0.828241in}{3.379757in}}{\pgfqpoint{2.414722in}{2.357859in}}%
\pgfusepath{clip}%
\pgfsetroundcap%
\pgfsetroundjoin%
\pgfsetlinewidth{1.505625pt}%
\definecolor{currentstroke}{rgb}{0.768627,0.305882,0.321569}%
\pgfsetstrokecolor{currentstroke}%
\pgfsetdash{}{0pt}%
\pgfpathmoveto{\pgfqpoint{0.938001in}{4.978878in}}%
\pgfpathlineto{\pgfqpoint{0.964135in}{4.973810in}}%
\pgfpathlineto{\pgfqpoint{0.990268in}{4.982167in}}%
\pgfpathlineto{\pgfqpoint{1.016401in}{5.017661in}}%
\pgfpathlineto{\pgfqpoint{1.042535in}{5.089092in}}%
\pgfpathlineto{\pgfqpoint{1.068668in}{5.193192in}}%
\pgfpathlineto{\pgfqpoint{1.094801in}{5.313084in}}%
\pgfpathlineto{\pgfqpoint{1.120935in}{5.427014in}}%
\pgfpathlineto{\pgfqpoint{1.147068in}{5.517364in}}%
\pgfpathlineto{\pgfqpoint{1.173201in}{5.574369in}}%
\pgfpathlineto{\pgfqpoint{1.199335in}{5.597103in}}%
\pgfpathlineto{\pgfqpoint{1.225468in}{5.591215in}}%
\pgfpathlineto{\pgfqpoint{1.251602in}{5.566251in}}%
\pgfpathlineto{\pgfqpoint{1.277735in}{5.535784in}}%
\pgfpathlineto{\pgfqpoint{1.303868in}{5.516432in}}%
\pgfpathlineto{\pgfqpoint{1.330002in}{5.519463in}}%
\pgfpathlineto{\pgfqpoint{1.356135in}{5.540466in}}%
\pgfpathlineto{\pgfqpoint{1.382268in}{5.561099in}}%
\pgfpathlineto{\pgfqpoint{1.408402in}{5.563089in}}%
\pgfpathlineto{\pgfqpoint{1.434535in}{5.538543in}}%
\pgfpathlineto{\pgfqpoint{1.460668in}{5.489500in}}%
\pgfpathlineto{\pgfqpoint{1.486802in}{5.422321in}}%
\pgfpathlineto{\pgfqpoint{1.512935in}{5.344566in}}%
\pgfpathlineto{\pgfqpoint{1.539068in}{5.262422in}}%
\pgfpathlineto{\pgfqpoint{1.565202in}{5.178119in}}%
\pgfpathlineto{\pgfqpoint{1.591335in}{5.091850in}}%
\pgfpathlineto{\pgfqpoint{1.617468in}{5.005359in}}%
\pgfpathlineto{\pgfqpoint{1.643602in}{4.920545in}}%
\pgfpathlineto{\pgfqpoint{1.669735in}{4.836998in}}%
\pgfpathlineto{\pgfqpoint{1.695869in}{4.755394in}}%
\pgfpathlineto{\pgfqpoint{1.722002in}{4.680099in}}%
\pgfpathlineto{\pgfqpoint{1.748135in}{4.615389in}}%
\pgfpathlineto{\pgfqpoint{1.774269in}{4.565140in}}%
\pgfpathlineto{\pgfqpoint{1.800402in}{4.532810in}}%
\pgfpathlineto{\pgfqpoint{1.826535in}{4.520413in}}%
\pgfpathlineto{\pgfqpoint{1.852669in}{4.525624in}}%
\pgfpathlineto{\pgfqpoint{1.878802in}{4.539624in}}%
\pgfpathlineto{\pgfqpoint{1.904935in}{4.549474in}}%
\pgfpathlineto{\pgfqpoint{1.931069in}{4.539721in}}%
\pgfpathlineto{\pgfqpoint{1.957202in}{4.495975in}}%
\pgfpathlineto{\pgfqpoint{1.983335in}{4.414773in}}%
\pgfpathlineto{\pgfqpoint{2.009469in}{4.314328in}}%
\pgfpathlineto{\pgfqpoint{2.035602in}{4.230877in}}%
\pgfpathlineto{\pgfqpoint{2.061735in}{4.196762in}}%
\pgfpathlineto{\pgfqpoint{2.087869in}{4.224943in}}%
\pgfpathlineto{\pgfqpoint{2.114002in}{4.309675in}}%
\pgfpathlineto{\pgfqpoint{2.140136in}{4.434044in}}%
\pgfpathlineto{\pgfqpoint{2.166269in}{4.574879in}}%
\pgfpathlineto{\pgfqpoint{2.192402in}{4.705789in}}%
\pgfpathlineto{\pgfqpoint{2.218536in}{4.809416in}}%
\pgfpathlineto{\pgfqpoint{2.244669in}{4.885140in}}%
\pgfpathlineto{\pgfqpoint{2.270802in}{4.940834in}}%
\pgfpathlineto{\pgfqpoint{2.296936in}{4.984792in}}%
\pgfpathlineto{\pgfqpoint{2.323069in}{5.024726in}}%
\pgfpathlineto{\pgfqpoint{2.349202in}{5.066165in}}%
\pgfpathlineto{\pgfqpoint{2.375336in}{5.108123in}}%
\pgfpathlineto{\pgfqpoint{2.401469in}{5.145095in}}%
\pgfpathlineto{\pgfqpoint{2.427602in}{5.172677in}}%
\pgfpathlineto{\pgfqpoint{2.453736in}{5.189692in}}%
\pgfpathlineto{\pgfqpoint{2.479869in}{5.198874in}}%
\pgfpathlineto{\pgfqpoint{2.506002in}{5.205291in}}%
\pgfpathlineto{\pgfqpoint{2.532136in}{5.215016in}}%
\pgfpathlineto{\pgfqpoint{2.558269in}{5.233304in}}%
\pgfpathlineto{\pgfqpoint{2.584403in}{5.262270in}}%
\pgfpathlineto{\pgfqpoint{2.610536in}{5.303297in}}%
\pgfpathlineto{\pgfqpoint{2.636669in}{5.357575in}}%
\pgfpathlineto{\pgfqpoint{2.662803in}{5.422085in}}%
\pgfpathlineto{\pgfqpoint{2.688936in}{5.486795in}}%
\pgfpathlineto{\pgfqpoint{2.715069in}{5.539744in}}%
\pgfpathlineto{\pgfqpoint{2.741203in}{5.575200in}}%
\pgfpathlineto{\pgfqpoint{2.767336in}{5.593378in}}%
\pgfpathlineto{\pgfqpoint{2.793469in}{5.596272in}}%
\pgfpathlineto{\pgfqpoint{2.819603in}{5.585815in}}%
\pgfpathlineto{\pgfqpoint{2.845736in}{5.566251in}}%
\pgfpathlineto{\pgfqpoint{2.871869in}{5.546553in}}%
\pgfpathlineto{\pgfqpoint{2.898003in}{5.538070in}}%
\pgfpathlineto{\pgfqpoint{2.924136in}{5.546131in}}%
\pgfpathlineto{\pgfqpoint{2.950269in}{5.564099in}}%
\pgfpathlineto{\pgfqpoint{2.976403in}{5.575967in}}%
\pgfpathlineto{\pgfqpoint{3.002536in}{5.567644in}}%
\pgfpathlineto{\pgfqpoint{3.028669in}{5.534145in}}%
\pgfpathlineto{\pgfqpoint{3.054803in}{5.478970in}}%
\pgfpathlineto{\pgfqpoint{3.080936in}{5.409403in}}%
\pgfpathlineto{\pgfqpoint{3.107070in}{5.329839in}}%
\pgfpathlineto{\pgfqpoint{3.133203in}{5.240455in}}%
\pgfusepath{stroke}%
\end{pgfscope}%
\begin{pgfscope}%
\pgfpathrectangle{\pgfqpoint{0.828241in}{3.379757in}}{\pgfqpoint{2.414722in}{2.357859in}}%
\pgfusepath{clip}%
\pgfsetroundcap%
\pgfsetroundjoin%
\pgfsetlinewidth{1.505625pt}%
\definecolor{currentstroke}{rgb}{0.505882,0.447059,0.701961}%
\pgfsetstrokecolor{currentstroke}%
\pgfsetdash{}{0pt}%
\pgfpathmoveto{\pgfqpoint{0.938001in}{4.326237in}}%
\pgfpathlineto{\pgfqpoint{0.965848in}{4.369539in}}%
\pgfpathlineto{\pgfqpoint{0.993695in}{4.451137in}}%
\pgfpathlineto{\pgfqpoint{1.021542in}{4.605988in}}%
\pgfpathlineto{\pgfqpoint{1.049389in}{4.843846in}}%
\pgfpathlineto{\pgfqpoint{1.077236in}{5.126689in}}%
\pgfpathlineto{\pgfqpoint{1.105083in}{5.381522in}}%
\pgfpathlineto{\pgfqpoint{1.132930in}{5.548284in}}%
\pgfpathlineto{\pgfqpoint{1.160777in}{5.614621in}}%
\pgfpathlineto{\pgfqpoint{1.188624in}{5.607360in}}%
\pgfpathlineto{\pgfqpoint{1.216471in}{5.566251in}}%
\pgfpathlineto{\pgfqpoint{1.244318in}{5.525192in}}%
\pgfpathlineto{\pgfqpoint{1.272166in}{5.503824in}}%
\pgfpathlineto{\pgfqpoint{1.300013in}{5.501200in}}%
\pgfpathlineto{\pgfqpoint{1.327860in}{5.494199in}}%
\pgfpathlineto{\pgfqpoint{1.355707in}{5.451112in}}%
\pgfpathlineto{\pgfqpoint{1.383554in}{5.352215in}}%
\pgfpathlineto{\pgfqpoint{1.411401in}{5.198709in}}%
\pgfpathlineto{\pgfqpoint{1.439248in}{5.007689in}}%
\pgfpathlineto{\pgfqpoint{1.467095in}{4.802632in}}%
\pgfpathlineto{\pgfqpoint{1.494942in}{4.606888in}}%
\pgfpathlineto{\pgfqpoint{1.522789in}{4.435897in}}%
\pgfpathlineto{\pgfqpoint{1.550636in}{4.289741in}}%
\pgfpathlineto{\pgfqpoint{1.578483in}{4.160036in}}%
\pgfpathlineto{\pgfqpoint{1.606330in}{4.038761in}}%
\pgfpathlineto{\pgfqpoint{1.634177in}{3.921869in}}%
\pgfpathlineto{\pgfqpoint{1.662024in}{3.811252in}}%
\pgfpathlineto{\pgfqpoint{1.689871in}{3.712418in}}%
\pgfpathlineto{\pgfqpoint{1.717718in}{3.631457in}}%
\pgfpathlineto{\pgfqpoint{1.745565in}{3.573325in}}%
\pgfpathlineto{\pgfqpoint{1.773412in}{3.540357in}}%
\pgfpathlineto{\pgfqpoint{1.801259in}{3.530694in}}%
\pgfpathlineto{\pgfqpoint{1.829106in}{3.536296in}}%
\pgfpathlineto{\pgfqpoint{1.856953in}{3.543183in}}%
\pgfpathlineto{\pgfqpoint{1.884800in}{3.538107in}}%
\pgfpathlineto{\pgfqpoint{1.912647in}{3.519111in}}%
\pgfpathlineto{\pgfqpoint{1.940494in}{3.496701in}}%
\pgfpathlineto{\pgfqpoint{1.968341in}{3.486932in}}%
\pgfpathlineto{\pgfqpoint{1.996188in}{3.504659in}}%
\pgfpathlineto{\pgfqpoint{2.024035in}{3.557932in}}%
\pgfpathlineto{\pgfqpoint{2.051882in}{3.645250in}}%
\pgfpathlineto{\pgfqpoint{2.079729in}{3.756442in}}%
\pgfpathlineto{\pgfqpoint{2.107576in}{3.876526in}}%
\pgfpathlineto{\pgfqpoint{2.135423in}{3.991261in}}%
\pgfpathlineto{\pgfqpoint{2.163270in}{4.091667in}}%
\pgfpathlineto{\pgfqpoint{2.191117in}{4.174050in}}%
\pgfpathlineto{\pgfqpoint{2.218964in}{4.237206in}}%
\pgfpathlineto{\pgfqpoint{2.246811in}{4.280091in}}%
\pgfpathlineto{\pgfqpoint{2.274658in}{4.302571in}}%
\pgfpathlineto{\pgfqpoint{2.302505in}{4.306580in}}%
\pgfpathlineto{\pgfqpoint{2.330352in}{4.300049in}}%
\pgfpathlineto{\pgfqpoint{2.358199in}{4.302329in}}%
\pgfpathlineto{\pgfqpoint{2.386046in}{4.345656in}}%
\pgfpathlineto{\pgfqpoint{2.413893in}{4.465567in}}%
\pgfpathlineto{\pgfqpoint{2.441740in}{4.681371in}}%
\pgfpathlineto{\pgfqpoint{2.469587in}{4.973792in}}%
\pgfpathlineto{\pgfqpoint{2.497434in}{5.274464in}}%
\pgfpathlineto{\pgfqpoint{2.525281in}{5.500132in}}%
\pgfpathlineto{\pgfqpoint{2.553128in}{5.608489in}}%
\pgfpathlineto{\pgfqpoint{2.580975in}{5.614928in}}%
\pgfpathlineto{\pgfqpoint{2.608822in}{5.566251in}}%
\pgfpathlineto{\pgfqpoint{2.636669in}{5.507284in}}%
\pgfpathlineto{\pgfqpoint{2.664516in}{5.464301in}}%
\pgfpathlineto{\pgfqpoint{2.692363in}{5.442439in}}%
\pgfpathlineto{\pgfqpoint{2.720210in}{5.424445in}}%
\pgfpathlineto{\pgfqpoint{2.748057in}{5.379901in}}%
\pgfpathlineto{\pgfqpoint{2.775904in}{5.290642in}}%
\pgfpathlineto{\pgfqpoint{2.803751in}{5.161881in}}%
\pgfpathlineto{\pgfqpoint{2.831598in}{5.008032in}}%
\pgfpathlineto{\pgfqpoint{2.859445in}{4.840497in}}%
\pgfpathlineto{\pgfqpoint{2.887292in}{4.666988in}}%
\pgfusepath{stroke}%
\end{pgfscope}%
\begin{pgfscope}%
\pgfsetrectcap%
\pgfsetmiterjoin%
\pgfsetlinewidth{1.254687pt}%
\definecolor{currentstroke}{rgb}{1.000000,1.000000,1.000000}%
\pgfsetstrokecolor{currentstroke}%
\pgfsetdash{}{0pt}%
\pgfpathmoveto{\pgfqpoint{0.828241in}{3.379757in}}%
\pgfpathlineto{\pgfqpoint{0.828241in}{5.737616in}}%
\pgfusepath{stroke}%
\end{pgfscope}%
\begin{pgfscope}%
\pgfsetrectcap%
\pgfsetmiterjoin%
\pgfsetlinewidth{1.254687pt}%
\definecolor{currentstroke}{rgb}{1.000000,1.000000,1.000000}%
\pgfsetstrokecolor{currentstroke}%
\pgfsetdash{}{0pt}%
\pgfpathmoveto{\pgfqpoint{3.242963in}{3.379757in}}%
\pgfpathlineto{\pgfqpoint{3.242963in}{5.737616in}}%
\pgfusepath{stroke}%
\end{pgfscope}%
\begin{pgfscope}%
\pgfsetrectcap%
\pgfsetmiterjoin%
\pgfsetlinewidth{1.254687pt}%
\definecolor{currentstroke}{rgb}{1.000000,1.000000,1.000000}%
\pgfsetstrokecolor{currentstroke}%
\pgfsetdash{}{0pt}%
\pgfpathmoveto{\pgfqpoint{0.828241in}{3.379757in}}%
\pgfpathlineto{\pgfqpoint{3.242963in}{3.379757in}}%
\pgfusepath{stroke}%
\end{pgfscope}%
\begin{pgfscope}%
\pgfsetrectcap%
\pgfsetmiterjoin%
\pgfsetlinewidth{1.254687pt}%
\definecolor{currentstroke}{rgb}{1.000000,1.000000,1.000000}%
\pgfsetstrokecolor{currentstroke}%
\pgfsetdash{}{0pt}%
\pgfpathmoveto{\pgfqpoint{0.828241in}{5.737616in}}%
\pgfpathlineto{\pgfqpoint{3.242963in}{5.737616in}}%
\pgfusepath{stroke}%
\end{pgfscope}%
\begin{pgfscope}%
\pgfsetbuttcap%
\pgfsetmiterjoin%
\definecolor{currentfill}{rgb}{0.917647,0.917647,0.949020}%
\pgfsetfillcolor{currentfill}%
\pgfsetlinewidth{0.000000pt}%
\definecolor{currentstroke}{rgb}{0.000000,0.000000,0.000000}%
\pgfsetstrokecolor{currentstroke}%
\pgfsetstrokeopacity{0.000000}%
\pgfsetdash{}{0pt}%
\pgfpathmoveto{\pgfqpoint{3.825278in}{3.379757in}}%
\pgfpathlineto{\pgfqpoint{6.240000in}{3.379757in}}%
\pgfpathlineto{\pgfqpoint{6.240000in}{5.737616in}}%
\pgfpathlineto{\pgfqpoint{3.825278in}{5.737616in}}%
\pgfpathclose%
\pgfusepath{fill}%
\end{pgfscope}%
\begin{pgfscope}%
\pgfpathrectangle{\pgfqpoint{3.825278in}{3.379757in}}{\pgfqpoint{2.414722in}{2.357859in}}%
\pgfusepath{clip}%
\pgfsetroundcap%
\pgfsetroundjoin%
\pgfsetlinewidth{1.003750pt}%
\definecolor{currentstroke}{rgb}{1.000000,1.000000,1.000000}%
\pgfsetstrokecolor{currentstroke}%
\pgfsetdash{}{0pt}%
\pgfpathmoveto{\pgfqpoint{3.935038in}{3.379757in}}%
\pgfpathlineto{\pgfqpoint{3.935038in}{5.737616in}}%
\pgfusepath{stroke}%
\end{pgfscope}%
\begin{pgfscope}%
\definecolor{textcolor}{rgb}{0.150000,0.150000,0.150000}%
\pgfsetstrokecolor{textcolor}%
\pgfsetfillcolor{textcolor}%
\pgftext[x=3.935038in,y=3.247812in,,top]{\color{textcolor}\sffamily\fontsize{11.000000}{13.200000}\selectfont \(\displaystyle 0.0\)}%
\end{pgfscope}%
\begin{pgfscope}%
\pgfpathrectangle{\pgfqpoint{3.825278in}{3.379757in}}{\pgfqpoint{2.414722in}{2.357859in}}%
\pgfusepath{clip}%
\pgfsetroundcap%
\pgfsetroundjoin%
\pgfsetlinewidth{1.003750pt}%
\definecolor{currentstroke}{rgb}{1.000000,1.000000,1.000000}%
\pgfsetstrokecolor{currentstroke}%
\pgfsetdash{}{0pt}%
\pgfpathmoveto{\pgfqpoint{4.637503in}{3.379757in}}%
\pgfpathlineto{\pgfqpoint{4.637503in}{5.737616in}}%
\pgfusepath{stroke}%
\end{pgfscope}%
\begin{pgfscope}%
\definecolor{textcolor}{rgb}{0.150000,0.150000,0.150000}%
\pgfsetstrokecolor{textcolor}%
\pgfsetfillcolor{textcolor}%
\pgftext[x=4.637503in,y=3.247812in,,top]{\color{textcolor}\sffamily\fontsize{11.000000}{13.200000}\selectfont \(\displaystyle 0.5\)}%
\end{pgfscope}%
\begin{pgfscope}%
\pgfpathrectangle{\pgfqpoint{3.825278in}{3.379757in}}{\pgfqpoint{2.414722in}{2.357859in}}%
\pgfusepath{clip}%
\pgfsetroundcap%
\pgfsetroundjoin%
\pgfsetlinewidth{1.003750pt}%
\definecolor{currentstroke}{rgb}{1.000000,1.000000,1.000000}%
\pgfsetstrokecolor{currentstroke}%
\pgfsetdash{}{0pt}%
\pgfpathmoveto{\pgfqpoint{5.339967in}{3.379757in}}%
\pgfpathlineto{\pgfqpoint{5.339967in}{5.737616in}}%
\pgfusepath{stroke}%
\end{pgfscope}%
\begin{pgfscope}%
\definecolor{textcolor}{rgb}{0.150000,0.150000,0.150000}%
\pgfsetstrokecolor{textcolor}%
\pgfsetfillcolor{textcolor}%
\pgftext[x=5.339967in,y=3.247812in,,top]{\color{textcolor}\sffamily\fontsize{11.000000}{13.200000}\selectfont \(\displaystyle 1.0\)}%
\end{pgfscope}%
\begin{pgfscope}%
\pgfpathrectangle{\pgfqpoint{3.825278in}{3.379757in}}{\pgfqpoint{2.414722in}{2.357859in}}%
\pgfusepath{clip}%
\pgfsetroundcap%
\pgfsetroundjoin%
\pgfsetlinewidth{1.003750pt}%
\definecolor{currentstroke}{rgb}{1.000000,1.000000,1.000000}%
\pgfsetstrokecolor{currentstroke}%
\pgfsetdash{}{0pt}%
\pgfpathmoveto{\pgfqpoint{6.042432in}{3.379757in}}%
\pgfpathlineto{\pgfqpoint{6.042432in}{5.737616in}}%
\pgfusepath{stroke}%
\end{pgfscope}%
\begin{pgfscope}%
\definecolor{textcolor}{rgb}{0.150000,0.150000,0.150000}%
\pgfsetstrokecolor{textcolor}%
\pgfsetfillcolor{textcolor}%
\pgftext[x=6.042432in,y=3.247812in,,top]{\color{textcolor}\sffamily\fontsize{11.000000}{13.200000}\selectfont \(\displaystyle 1.5\)}%
\end{pgfscope}%
\begin{pgfscope}%
\pgfpathrectangle{\pgfqpoint{3.825278in}{3.379757in}}{\pgfqpoint{2.414722in}{2.357859in}}%
\pgfusepath{clip}%
\pgfsetroundcap%
\pgfsetroundjoin%
\pgfsetlinewidth{1.003750pt}%
\definecolor{currentstroke}{rgb}{1.000000,1.000000,1.000000}%
\pgfsetstrokecolor{currentstroke}%
\pgfsetdash{}{0pt}%
\pgfpathmoveto{\pgfqpoint{3.825278in}{3.544338in}}%
\pgfpathlineto{\pgfqpoint{6.240000in}{3.544338in}}%
\pgfusepath{stroke}%
\end{pgfscope}%
\begin{pgfscope}%
\definecolor{textcolor}{rgb}{0.150000,0.150000,0.150000}%
\pgfsetstrokecolor{textcolor}%
\pgfsetfillcolor{textcolor}%
\pgftext[x=3.422963in,y=3.491531in,left,base]{\color{textcolor}\sffamily\fontsize{11.000000}{13.200000}\selectfont \(\displaystyle -20\)}%
\end{pgfscope}%
\begin{pgfscope}%
\pgfpathrectangle{\pgfqpoint{3.825278in}{3.379757in}}{\pgfqpoint{2.414722in}{2.357859in}}%
\pgfusepath{clip}%
\pgfsetroundcap%
\pgfsetroundjoin%
\pgfsetlinewidth{1.003750pt}%
\definecolor{currentstroke}{rgb}{1.000000,1.000000,1.000000}%
\pgfsetstrokecolor{currentstroke}%
\pgfsetdash{}{0pt}%
\pgfpathmoveto{\pgfqpoint{3.825278in}{4.046733in}}%
\pgfpathlineto{\pgfqpoint{6.240000in}{4.046733in}}%
\pgfusepath{stroke}%
\end{pgfscope}%
\begin{pgfscope}%
\definecolor{textcolor}{rgb}{0.150000,0.150000,0.150000}%
\pgfsetstrokecolor{textcolor}%
\pgfsetfillcolor{textcolor}%
\pgftext[x=3.422963in,y=3.993927in,left,base]{\color{textcolor}\sffamily\fontsize{11.000000}{13.200000}\selectfont \(\displaystyle -15\)}%
\end{pgfscope}%
\begin{pgfscope}%
\pgfpathrectangle{\pgfqpoint{3.825278in}{3.379757in}}{\pgfqpoint{2.414722in}{2.357859in}}%
\pgfusepath{clip}%
\pgfsetroundcap%
\pgfsetroundjoin%
\pgfsetlinewidth{1.003750pt}%
\definecolor{currentstroke}{rgb}{1.000000,1.000000,1.000000}%
\pgfsetstrokecolor{currentstroke}%
\pgfsetdash{}{0pt}%
\pgfpathmoveto{\pgfqpoint{3.825278in}{4.549129in}}%
\pgfpathlineto{\pgfqpoint{6.240000in}{4.549129in}}%
\pgfusepath{stroke}%
\end{pgfscope}%
\begin{pgfscope}%
\definecolor{textcolor}{rgb}{0.150000,0.150000,0.150000}%
\pgfsetstrokecolor{textcolor}%
\pgfsetfillcolor{textcolor}%
\pgftext[x=3.422963in,y=4.496322in,left,base]{\color{textcolor}\sffamily\fontsize{11.000000}{13.200000}\selectfont \(\displaystyle -10\)}%
\end{pgfscope}%
\begin{pgfscope}%
\pgfpathrectangle{\pgfqpoint{3.825278in}{3.379757in}}{\pgfqpoint{2.414722in}{2.357859in}}%
\pgfusepath{clip}%
\pgfsetroundcap%
\pgfsetroundjoin%
\pgfsetlinewidth{1.003750pt}%
\definecolor{currentstroke}{rgb}{1.000000,1.000000,1.000000}%
\pgfsetstrokecolor{currentstroke}%
\pgfsetdash{}{0pt}%
\pgfpathmoveto{\pgfqpoint{3.825278in}{5.051524in}}%
\pgfpathlineto{\pgfqpoint{6.240000in}{5.051524in}}%
\pgfusepath{stroke}%
\end{pgfscope}%
\begin{pgfscope}%
\definecolor{textcolor}{rgb}{0.150000,0.150000,0.150000}%
\pgfsetstrokecolor{textcolor}%
\pgfsetfillcolor{textcolor}%
\pgftext[x=3.499005in,y=4.998718in,left,base]{\color{textcolor}\sffamily\fontsize{11.000000}{13.200000}\selectfont \(\displaystyle -5\)}%
\end{pgfscope}%
\begin{pgfscope}%
\pgfpathrectangle{\pgfqpoint{3.825278in}{3.379757in}}{\pgfqpoint{2.414722in}{2.357859in}}%
\pgfusepath{clip}%
\pgfsetroundcap%
\pgfsetroundjoin%
\pgfsetlinewidth{1.003750pt}%
\definecolor{currentstroke}{rgb}{1.000000,1.000000,1.000000}%
\pgfsetstrokecolor{currentstroke}%
\pgfsetdash{}{0pt}%
\pgfpathmoveto{\pgfqpoint{3.825278in}{5.553920in}}%
\pgfpathlineto{\pgfqpoint{6.240000in}{5.553920in}}%
\pgfusepath{stroke}%
\end{pgfscope}%
\begin{pgfscope}%
\definecolor{textcolor}{rgb}{0.150000,0.150000,0.150000}%
\pgfsetstrokecolor{textcolor}%
\pgfsetfillcolor{textcolor}%
\pgftext[x=3.617292in,y=5.501113in,left,base]{\color{textcolor}\sffamily\fontsize{11.000000}{13.200000}\selectfont \(\displaystyle 0\)}%
\end{pgfscope}%
\begin{pgfscope}%
\pgfpathrectangle{\pgfqpoint{3.825278in}{3.379757in}}{\pgfqpoint{2.414722in}{2.357859in}}%
\pgfusepath{clip}%
\pgfsetroundcap%
\pgfsetroundjoin%
\pgfsetlinewidth{1.505625pt}%
\definecolor{currentstroke}{rgb}{0.298039,0.447059,0.690196}%
\pgfsetstrokecolor{currentstroke}%
\pgfsetdash{}{0pt}%
\pgfpathmoveto{\pgfqpoint{3.935038in}{5.536127in}}%
\pgfpathlineto{\pgfqpoint{3.952172in}{5.556864in}}%
\pgfpathlineto{\pgfqpoint{3.969305in}{5.565171in}}%
\pgfpathlineto{\pgfqpoint{3.986438in}{5.553920in}}%
\pgfpathlineto{\pgfqpoint{4.003571in}{5.523636in}}%
\pgfpathlineto{\pgfqpoint{4.020705in}{5.481916in}}%
\pgfpathlineto{\pgfqpoint{4.037838in}{5.440040in}}%
\pgfpathlineto{\pgfqpoint{4.054971in}{5.404547in}}%
\pgfpathlineto{\pgfqpoint{4.072105in}{5.369020in}}%
\pgfpathlineto{\pgfqpoint{4.089238in}{5.318923in}}%
\pgfpathlineto{\pgfqpoint{4.106371in}{5.243963in}}%
\pgfpathlineto{\pgfqpoint{4.123504in}{5.143460in}}%
\pgfpathlineto{\pgfqpoint{4.140638in}{5.024856in}}%
\pgfpathlineto{\pgfqpoint{4.157771in}{4.897738in}}%
\pgfpathlineto{\pgfqpoint{4.174904in}{4.771482in}}%
\pgfpathlineto{\pgfqpoint{4.192038in}{4.654407in}}%
\pgfpathlineto{\pgfqpoint{4.209171in}{4.549189in}}%
\pgfpathlineto{\pgfqpoint{4.226304in}{4.452705in}}%
\pgfpathlineto{\pgfqpoint{4.243437in}{4.360609in}}%
\pgfpathlineto{\pgfqpoint{4.260571in}{4.270559in}}%
\pgfpathlineto{\pgfqpoint{4.277704in}{4.183268in}}%
\pgfpathlineto{\pgfqpoint{4.294837in}{4.099390in}}%
\pgfpathlineto{\pgfqpoint{4.311971in}{4.018420in}}%
\pgfpathlineto{\pgfqpoint{4.329104in}{3.941387in}}%
\pgfpathlineto{\pgfqpoint{4.346237in}{3.869777in}}%
\pgfpathlineto{\pgfqpoint{4.363370in}{3.803857in}}%
\pgfpathlineto{\pgfqpoint{4.380504in}{3.743605in}}%
\pgfpathlineto{\pgfqpoint{4.397637in}{3.687061in}}%
\pgfpathlineto{\pgfqpoint{4.414770in}{3.633185in}}%
\pgfpathlineto{\pgfqpoint{4.431903in}{3.585623in}}%
\pgfpathlineto{\pgfqpoint{4.449037in}{3.550533in}}%
\pgfpathlineto{\pgfqpoint{4.466170in}{3.534509in}}%
\pgfpathlineto{\pgfqpoint{4.483303in}{3.543815in}}%
\pgfpathlineto{\pgfqpoint{4.500437in}{3.580409in}}%
\pgfpathlineto{\pgfqpoint{4.517570in}{3.640435in}}%
\pgfpathlineto{\pgfqpoint{4.534703in}{3.715786in}}%
\pgfpathlineto{\pgfqpoint{4.551836in}{3.793659in}}%
\pgfpathlineto{\pgfqpoint{4.568970in}{3.859590in}}%
\pgfpathlineto{\pgfqpoint{4.586103in}{3.908103in}}%
\pgfpathlineto{\pgfqpoint{4.603236in}{3.946946in}}%
\pgfpathlineto{\pgfqpoint{4.620370in}{3.985510in}}%
\pgfpathlineto{\pgfqpoint{4.637503in}{4.024382in}}%
\pgfpathlineto{\pgfqpoint{4.654636in}{4.059922in}}%
\pgfpathlineto{\pgfqpoint{4.671769in}{4.092873in}}%
\pgfpathlineto{\pgfqpoint{4.688903in}{4.127181in}}%
\pgfpathlineto{\pgfqpoint{4.706036in}{4.165120in}}%
\pgfpathlineto{\pgfqpoint{4.723169in}{4.209366in}}%
\pgfpathlineto{\pgfqpoint{4.740303in}{4.260775in}}%
\pgfpathlineto{\pgfqpoint{4.757436in}{4.316631in}}%
\pgfpathlineto{\pgfqpoint{4.774569in}{4.371066in}}%
\pgfpathlineto{\pgfqpoint{4.791702in}{4.418899in}}%
\pgfpathlineto{\pgfqpoint{4.808836in}{4.459606in}}%
\pgfpathlineto{\pgfqpoint{4.825969in}{4.496861in}}%
\pgfpathlineto{\pgfqpoint{4.843102in}{4.536893in}}%
\pgfpathlineto{\pgfqpoint{4.860235in}{4.587763in}}%
\pgfpathlineto{\pgfqpoint{4.877369in}{4.656841in}}%
\pgfpathlineto{\pgfqpoint{4.894502in}{4.749331in}}%
\pgfpathlineto{\pgfqpoint{4.911635in}{4.869417in}}%
\pgfpathlineto{\pgfqpoint{4.928769in}{5.017172in}}%
\pgfpathlineto{\pgfqpoint{4.945902in}{5.182051in}}%
\pgfpathlineto{\pgfqpoint{4.963035in}{5.340210in}}%
\pgfpathlineto{\pgfqpoint{4.980168in}{5.465082in}}%
\pgfpathlineto{\pgfqpoint{4.997302in}{5.541395in}}%
\pgfpathlineto{\pgfqpoint{5.014435in}{5.568193in}}%
\pgfpathlineto{\pgfqpoint{5.031568in}{5.553920in}}%
\pgfpathlineto{\pgfqpoint{5.048702in}{5.509139in}}%
\pgfpathlineto{\pgfqpoint{5.065835in}{5.446333in}}%
\pgfusepath{stroke}%
\end{pgfscope}%
\begin{pgfscope}%
\pgfpathrectangle{\pgfqpoint{3.825278in}{3.379757in}}{\pgfqpoint{2.414722in}{2.357859in}}%
\pgfusepath{clip}%
\pgfsetroundcap%
\pgfsetroundjoin%
\pgfsetlinewidth{1.505625pt}%
\definecolor{currentstroke}{rgb}{0.866667,0.517647,0.321569}%
\pgfsetstrokecolor{currentstroke}%
\pgfsetdash{}{0pt}%
\pgfpathmoveto{\pgfqpoint{3.935038in}{4.660074in}}%
\pgfpathlineto{\pgfqpoint{3.952172in}{4.673089in}}%
\pgfpathlineto{\pgfqpoint{3.969305in}{4.701194in}}%
\pgfpathlineto{\pgfqpoint{3.986438in}{4.756481in}}%
\pgfpathlineto{\pgfqpoint{4.003571in}{4.841609in}}%
\pgfpathlineto{\pgfqpoint{4.020705in}{4.951587in}}%
\pgfpathlineto{\pgfqpoint{4.037838in}{5.078289in}}%
\pgfpathlineto{\pgfqpoint{4.054971in}{5.209579in}}%
\pgfpathlineto{\pgfqpoint{4.072105in}{5.329944in}}%
\pgfpathlineto{\pgfqpoint{4.089238in}{5.428892in}}%
\pgfpathlineto{\pgfqpoint{4.106371in}{5.501693in}}%
\pgfpathlineto{\pgfqpoint{4.123504in}{5.545782in}}%
\pgfpathlineto{\pgfqpoint{4.140638in}{5.560983in}}%
\pgfpathlineto{\pgfqpoint{4.157771in}{5.553920in}}%
\pgfpathlineto{\pgfqpoint{4.174904in}{5.538466in}}%
\pgfpathlineto{\pgfqpoint{4.192038in}{5.527677in}}%
\pgfpathlineto{\pgfqpoint{4.209171in}{5.523829in}}%
\pgfpathlineto{\pgfqpoint{4.226304in}{5.516579in}}%
\pgfpathlineto{\pgfqpoint{4.243437in}{5.491164in}}%
\pgfpathlineto{\pgfqpoint{4.260571in}{5.438601in}}%
\pgfpathlineto{\pgfqpoint{4.277704in}{5.357794in}}%
\pgfpathlineto{\pgfqpoint{4.294837in}{5.254661in}}%
\pgfpathlineto{\pgfqpoint{4.311971in}{5.140151in}}%
\pgfpathlineto{\pgfqpoint{4.329104in}{5.025899in}}%
\pgfpathlineto{\pgfqpoint{4.346237in}{4.920457in}}%
\pgfpathlineto{\pgfqpoint{4.363370in}{4.825738in}}%
\pgfpathlineto{\pgfqpoint{4.380504in}{4.736709in}}%
\pgfpathlineto{\pgfqpoint{4.397637in}{4.647208in}}%
\pgfpathlineto{\pgfqpoint{4.414770in}{4.553392in}}%
\pgfpathlineto{\pgfqpoint{4.431903in}{4.456709in}}%
\pgfpathlineto{\pgfqpoint{4.449037in}{4.363246in}}%
\pgfpathlineto{\pgfqpoint{4.466170in}{4.275397in}}%
\pgfpathlineto{\pgfqpoint{4.483303in}{4.186772in}}%
\pgfpathlineto{\pgfqpoint{4.500437in}{4.089933in}}%
\pgfpathlineto{\pgfqpoint{4.517570in}{3.990533in}}%
\pgfpathlineto{\pgfqpoint{4.534703in}{3.898356in}}%
\pgfpathlineto{\pgfqpoint{4.551836in}{3.818172in}}%
\pgfpathlineto{\pgfqpoint{4.568970in}{3.747952in}}%
\pgfpathlineto{\pgfqpoint{4.586103in}{3.684301in}}%
\pgfpathlineto{\pgfqpoint{4.603236in}{3.626911in}}%
\pgfpathlineto{\pgfqpoint{4.620370in}{3.575846in}}%
\pgfpathlineto{\pgfqpoint{4.637503in}{3.531074in}}%
\pgfpathlineto{\pgfqpoint{4.654636in}{3.498378in}}%
\pgfpathlineto{\pgfqpoint{4.671769in}{3.486932in}}%
\pgfpathlineto{\pgfqpoint{4.688903in}{3.497993in}}%
\pgfpathlineto{\pgfqpoint{4.706036in}{3.524401in}}%
\pgfpathlineto{\pgfqpoint{4.723169in}{3.559537in}}%
\pgfpathlineto{\pgfqpoint{4.740303in}{3.599921in}}%
\pgfpathlineto{\pgfqpoint{4.757436in}{3.642783in}}%
\pgfpathlineto{\pgfqpoint{4.774569in}{3.683807in}}%
\pgfpathlineto{\pgfqpoint{4.791702in}{3.720252in}}%
\pgfpathlineto{\pgfqpoint{4.808836in}{3.755045in}}%
\pgfpathlineto{\pgfqpoint{4.825969in}{3.795054in}}%
\pgfpathlineto{\pgfqpoint{4.843102in}{3.846366in}}%
\pgfpathlineto{\pgfqpoint{4.860235in}{3.911528in}}%
\pgfpathlineto{\pgfqpoint{4.877369in}{3.992085in}}%
\pgfpathlineto{\pgfqpoint{4.894502in}{4.089049in}}%
\pgfpathlineto{\pgfqpoint{4.911635in}{4.199986in}}%
\pgfpathlineto{\pgfqpoint{4.928769in}{4.317833in}}%
\pgfpathlineto{\pgfqpoint{4.945902in}{4.433057in}}%
\pgfpathlineto{\pgfqpoint{4.963035in}{4.539178in}}%
\pgfpathlineto{\pgfqpoint{4.980168in}{4.632587in}}%
\pgfpathlineto{\pgfqpoint{4.997302in}{4.710183in}}%
\pgfpathlineto{\pgfqpoint{5.014435in}{4.769755in}}%
\pgfpathlineto{\pgfqpoint{5.031568in}{4.811553in}}%
\pgfpathlineto{\pgfqpoint{5.048702in}{4.837212in}}%
\pgfpathlineto{\pgfqpoint{5.065835in}{4.848974in}}%
\pgfpathlineto{\pgfqpoint{5.082968in}{4.850231in}}%
\pgfpathlineto{\pgfqpoint{5.100101in}{4.846803in}}%
\pgfpathlineto{\pgfqpoint{5.117235in}{4.847805in}}%
\pgfpathlineto{\pgfqpoint{5.134368in}{4.865014in}}%
\pgfpathlineto{\pgfqpoint{5.151501in}{4.907696in}}%
\pgfpathlineto{\pgfqpoint{5.168635in}{4.977796in}}%
\pgfpathlineto{\pgfqpoint{5.185768in}{5.070028in}}%
\pgfpathlineto{\pgfqpoint{5.202901in}{5.175592in}}%
\pgfpathlineto{\pgfqpoint{5.220034in}{5.285694in}}%
\pgfpathlineto{\pgfqpoint{5.237168in}{5.392355in}}%
\pgfpathlineto{\pgfqpoint{5.254301in}{5.487647in}}%
\pgfpathlineto{\pgfqpoint{5.271434in}{5.562394in}}%
\pgfpathlineto{\pgfqpoint{5.288568in}{5.606853in}}%
\pgfpathlineto{\pgfqpoint{5.305701in}{5.615743in}}%
\pgfpathlineto{\pgfqpoint{5.322834in}{5.593074in}}%
\pgfpathlineto{\pgfqpoint{5.339967in}{5.553920in}}%
\pgfpathlineto{\pgfqpoint{5.357101in}{5.519437in}}%
\pgfpathlineto{\pgfqpoint{5.374234in}{5.502868in}}%
\pgfpathlineto{\pgfqpoint{5.391367in}{5.497691in}}%
\pgfpathlineto{\pgfqpoint{5.408500in}{5.483926in}}%
\pgfpathlineto{\pgfqpoint{5.425634in}{5.443157in}}%
\pgfpathlineto{\pgfqpoint{5.442767in}{5.367141in}}%
\pgfpathlineto{\pgfqpoint{5.459900in}{5.259248in}}%
\pgfpathlineto{\pgfqpoint{5.477034in}{5.134619in}}%
\pgfpathlineto{\pgfqpoint{5.494167in}{5.012221in}}%
\pgfpathlineto{\pgfqpoint{5.511300in}{4.903546in}}%
\pgfpathlineto{\pgfqpoint{5.528433in}{4.808860in}}%
\pgfusepath{stroke}%
\end{pgfscope}%
\begin{pgfscope}%
\pgfpathrectangle{\pgfqpoint{3.825278in}{3.379757in}}{\pgfqpoint{2.414722in}{2.357859in}}%
\pgfusepath{clip}%
\pgfsetroundcap%
\pgfsetroundjoin%
\pgfsetlinewidth{1.505625pt}%
\definecolor{currentstroke}{rgb}{0.333333,0.658824,0.407843}%
\pgfsetstrokecolor{currentstroke}%
\pgfsetdash{}{0pt}%
\pgfpathmoveto{\pgfqpoint{3.935038in}{4.892509in}}%
\pgfpathlineto{\pgfqpoint{3.959686in}{4.890869in}}%
\pgfpathlineto{\pgfqpoint{3.984334in}{4.889118in}}%
\pgfpathlineto{\pgfqpoint{4.008982in}{4.892492in}}%
\pgfpathlineto{\pgfqpoint{4.033630in}{4.913624in}}%
\pgfpathlineto{\pgfqpoint{4.058278in}{4.969060in}}%
\pgfpathlineto{\pgfqpoint{4.082926in}{5.069017in}}%
\pgfpathlineto{\pgfqpoint{4.107573in}{5.204350in}}%
\pgfpathlineto{\pgfqpoint{4.132221in}{5.346249in}}%
\pgfpathlineto{\pgfqpoint{4.156869in}{5.462092in}}%
\pgfpathlineto{\pgfqpoint{4.181517in}{5.532861in}}%
\pgfpathlineto{\pgfqpoint{4.206165in}{5.559078in}}%
\pgfpathlineto{\pgfqpoint{4.230813in}{5.553920in}}%
\pgfpathlineto{\pgfqpoint{4.255461in}{5.532506in}}%
\pgfpathlineto{\pgfqpoint{4.280109in}{5.505518in}}%
\pgfpathlineto{\pgfqpoint{4.304757in}{5.474525in}}%
\pgfpathlineto{\pgfqpoint{4.329404in}{5.432030in}}%
\pgfpathlineto{\pgfqpoint{4.354052in}{5.368111in}}%
\pgfpathlineto{\pgfqpoint{4.378700in}{5.277366in}}%
\pgfpathlineto{\pgfqpoint{4.403348in}{5.163006in}}%
\pgfpathlineto{\pgfqpoint{4.427996in}{5.038865in}}%
\pgfpathlineto{\pgfqpoint{4.452644in}{4.923116in}}%
\pgfpathlineto{\pgfqpoint{4.477292in}{4.825937in}}%
\pgfpathlineto{\pgfqpoint{4.501940in}{4.743895in}}%
\pgfpathlineto{\pgfqpoint{4.526587in}{4.667103in}}%
\pgfpathlineto{\pgfqpoint{4.551235in}{4.592025in}}%
\pgfpathlineto{\pgfqpoint{4.575883in}{4.520664in}}%
\pgfpathlineto{\pgfqpoint{4.600531in}{4.454638in}}%
\pgfpathlineto{\pgfqpoint{4.625179in}{4.393286in}}%
\pgfpathlineto{\pgfqpoint{4.649827in}{4.336229in}}%
\pgfpathlineto{\pgfqpoint{4.674475in}{4.286937in}}%
\pgfpathlineto{\pgfqpoint{4.699123in}{4.253056in}}%
\pgfpathlineto{\pgfqpoint{4.723770in}{4.241894in}}%
\pgfpathlineto{\pgfqpoint{4.748418in}{4.256041in}}%
\pgfpathlineto{\pgfqpoint{4.773066in}{4.291211in}}%
\pgfpathlineto{\pgfqpoint{4.797714in}{4.336676in}}%
\pgfpathlineto{\pgfqpoint{4.822362in}{4.380273in}}%
\pgfpathlineto{\pgfqpoint{4.847010in}{4.411664in}}%
\pgfpathlineto{\pgfqpoint{4.871658in}{4.423083in}}%
\pgfpathlineto{\pgfqpoint{4.896306in}{4.409516in}}%
\pgfpathlineto{\pgfqpoint{4.920953in}{4.371882in}}%
\pgfpathlineto{\pgfqpoint{4.945601in}{4.320085in}}%
\pgfpathlineto{\pgfqpoint{4.970249in}{4.271444in}}%
\pgfpathlineto{\pgfqpoint{4.994897in}{4.241623in}}%
\pgfpathlineto{\pgfqpoint{5.019545in}{4.236762in}}%
\pgfpathlineto{\pgfqpoint{5.044193in}{4.255769in}}%
\pgfpathlineto{\pgfqpoint{5.068841in}{4.296190in}}%
\pgfpathlineto{\pgfqpoint{5.093489in}{4.349146in}}%
\pgfpathlineto{\pgfqpoint{5.118136in}{4.401208in}}%
\pgfpathlineto{\pgfqpoint{5.142784in}{4.443561in}}%
\pgfpathlineto{\pgfqpoint{5.167432in}{4.475445in}}%
\pgfpathlineto{\pgfqpoint{5.192080in}{4.500387in}}%
\pgfpathlineto{\pgfqpoint{5.216728in}{4.523184in}}%
\pgfpathlineto{\pgfqpoint{5.241376in}{4.548563in}}%
\pgfpathlineto{\pgfqpoint{5.266024in}{4.578677in}}%
\pgfpathlineto{\pgfqpoint{5.290672in}{4.611387in}}%
\pgfpathlineto{\pgfqpoint{5.315319in}{4.643673in}}%
\pgfpathlineto{\pgfqpoint{5.339967in}{4.673973in}}%
\pgfpathlineto{\pgfqpoint{5.364615in}{4.703094in}}%
\pgfpathlineto{\pgfqpoint{5.389263in}{4.730521in}}%
\pgfpathlineto{\pgfqpoint{5.413911in}{4.753742in}}%
\pgfpathlineto{\pgfqpoint{5.438559in}{4.772397in}}%
\pgfpathlineto{\pgfqpoint{5.463207in}{4.789125in}}%
\pgfpathlineto{\pgfqpoint{5.487855in}{4.806029in}}%
\pgfpathlineto{\pgfqpoint{5.512502in}{4.822924in}}%
\pgfpathlineto{\pgfqpoint{5.537150in}{4.837955in}}%
\pgfpathlineto{\pgfqpoint{5.561798in}{4.850764in}}%
\pgfpathlineto{\pgfqpoint{5.586446in}{4.865700in}}%
\pgfpathlineto{\pgfqpoint{5.611094in}{4.892733in}}%
\pgfpathlineto{\pgfqpoint{5.635742in}{4.946715in}}%
\pgfpathlineto{\pgfqpoint{5.660390in}{5.040259in}}%
\pgfpathlineto{\pgfqpoint{5.685038in}{5.170817in}}%
\pgfpathlineto{\pgfqpoint{5.709686in}{5.317549in}}%
\pgfpathlineto{\pgfqpoint{5.734333in}{5.449558in}}%
\pgfpathlineto{\pgfqpoint{5.758981in}{5.538209in}}%
\pgfpathlineto{\pgfqpoint{5.783629in}{5.569981in}}%
\pgfpathlineto{\pgfqpoint{5.808277in}{5.553920in}}%
\pgfpathlineto{\pgfqpoint{5.832925in}{5.513746in}}%
\pgfpathlineto{\pgfqpoint{5.857573in}{5.472288in}}%
\pgfpathlineto{\pgfqpoint{5.882221in}{5.441054in}}%
\pgfpathlineto{\pgfqpoint{5.906869in}{5.417764in}}%
\pgfpathlineto{\pgfqpoint{5.931516in}{5.389044in}}%
\pgfpathlineto{\pgfqpoint{5.956164in}{5.338893in}}%
\pgfpathlineto{\pgfqpoint{5.980812in}{5.260557in}}%
\pgfpathlineto{\pgfqpoint{6.005460in}{5.161388in}}%
\pgfpathlineto{\pgfqpoint{6.030108in}{5.057004in}}%
\pgfpathlineto{\pgfqpoint{6.054756in}{4.962456in}}%
\pgfpathlineto{\pgfqpoint{6.079404in}{4.885717in}}%
\pgfpathlineto{\pgfqpoint{6.104052in}{4.822341in}}%
\pgfusepath{stroke}%
\end{pgfscope}%
\begin{pgfscope}%
\pgfpathrectangle{\pgfqpoint{3.825278in}{3.379757in}}{\pgfqpoint{2.414722in}{2.357859in}}%
\pgfusepath{clip}%
\pgfsetroundcap%
\pgfsetroundjoin%
\pgfsetlinewidth{1.505625pt}%
\definecolor{currentstroke}{rgb}{0.768627,0.305882,0.321569}%
\pgfsetstrokecolor{currentstroke}%
\pgfsetdash{}{0pt}%
\pgfpathmoveto{\pgfqpoint{3.935038in}{5.240198in}}%
\pgfpathlineto{\pgfqpoint{3.956990in}{5.256071in}}%
\pgfpathlineto{\pgfqpoint{3.978942in}{5.271620in}}%
\pgfpathlineto{\pgfqpoint{4.000894in}{5.284999in}}%
\pgfpathlineto{\pgfqpoint{4.022846in}{5.294633in}}%
\pgfpathlineto{\pgfqpoint{4.044798in}{5.302481in}}%
\pgfpathlineto{\pgfqpoint{4.066750in}{5.315328in}}%
\pgfpathlineto{\pgfqpoint{4.088702in}{5.342864in}}%
\pgfpathlineto{\pgfqpoint{4.110654in}{5.390588in}}%
\pgfpathlineto{\pgfqpoint{4.132606in}{5.451704in}}%
\pgfpathlineto{\pgfqpoint{4.154559in}{5.509770in}}%
\pgfpathlineto{\pgfqpoint{4.176511in}{5.551214in}}%
\pgfpathlineto{\pgfqpoint{4.198463in}{5.572295in}}%
\pgfpathlineto{\pgfqpoint{4.220415in}{5.576005in}}%
\pgfpathlineto{\pgfqpoint{4.242367in}{5.567826in}}%
\pgfpathlineto{\pgfqpoint{4.264319in}{5.553920in}}%
\pgfpathlineto{\pgfqpoint{4.286271in}{5.537024in}}%
\pgfpathlineto{\pgfqpoint{4.308223in}{5.514316in}}%
\pgfpathlineto{\pgfqpoint{4.330175in}{5.480045in}}%
\pgfpathlineto{\pgfqpoint{4.352127in}{5.428113in}}%
\pgfpathlineto{\pgfqpoint{4.374079in}{5.355264in}}%
\pgfpathlineto{\pgfqpoint{4.396031in}{5.263222in}}%
\pgfpathlineto{\pgfqpoint{4.417983in}{5.158360in}}%
\pgfpathlineto{\pgfqpoint{4.439935in}{5.048567in}}%
\pgfpathlineto{\pgfqpoint{4.461887in}{4.940092in}}%
\pgfpathlineto{\pgfqpoint{4.483839in}{4.836242in}}%
\pgfpathlineto{\pgfqpoint{4.505791in}{4.738910in}}%
\pgfpathlineto{\pgfqpoint{4.527743in}{4.649953in}}%
\pgfpathlineto{\pgfqpoint{4.549695in}{4.569925in}}%
\pgfpathlineto{\pgfqpoint{4.571647in}{4.497828in}}%
\pgfpathlineto{\pgfqpoint{4.593599in}{4.433895in}}%
\pgfpathlineto{\pgfqpoint{4.615551in}{4.380462in}}%
\pgfpathlineto{\pgfqpoint{4.637503in}{4.339817in}}%
\pgfpathlineto{\pgfqpoint{4.659455in}{4.312690in}}%
\pgfpathlineto{\pgfqpoint{4.681407in}{4.298287in}}%
\pgfpathlineto{\pgfqpoint{4.703359in}{4.294916in}}%
\pgfpathlineto{\pgfqpoint{4.725311in}{4.300147in}}%
\pgfpathlineto{\pgfqpoint{4.747263in}{4.310658in}}%
\pgfpathlineto{\pgfqpoint{4.769215in}{4.323026in}}%
\pgfpathlineto{\pgfqpoint{4.791167in}{4.334896in}}%
\pgfpathlineto{\pgfqpoint{4.813119in}{4.344455in}}%
\pgfpathlineto{\pgfqpoint{4.835071in}{4.350387in}}%
\pgfpathlineto{\pgfqpoint{4.857023in}{4.354301in}}%
\pgfpathlineto{\pgfqpoint{4.878975in}{4.362801in}}%
\pgfpathlineto{\pgfqpoint{4.900927in}{4.384645in}}%
\pgfpathlineto{\pgfqpoint{4.922879in}{4.425162in}}%
\pgfpathlineto{\pgfqpoint{4.944831in}{4.484435in}}%
\pgfpathlineto{\pgfqpoint{4.966783in}{4.559592in}}%
\pgfpathlineto{\pgfqpoint{4.988735in}{4.647665in}}%
\pgfpathlineto{\pgfqpoint{5.010687in}{4.745230in}}%
\pgfpathlineto{\pgfqpoint{5.032639in}{4.847080in}}%
\pgfpathlineto{\pgfqpoint{5.054591in}{4.945770in}}%
\pgfpathlineto{\pgfqpoint{5.076543in}{5.032950in}}%
\pgfpathlineto{\pgfqpoint{5.098495in}{5.101158in}}%
\pgfpathlineto{\pgfqpoint{5.120447in}{5.146720in}}%
\pgfpathlineto{\pgfqpoint{5.142399in}{5.172055in}}%
\pgfpathlineto{\pgfqpoint{5.164351in}{5.183399in}}%
\pgfpathlineto{\pgfqpoint{5.186303in}{5.187330in}}%
\pgfpathlineto{\pgfqpoint{5.208255in}{5.189716in}}%
\pgfpathlineto{\pgfqpoint{5.230207in}{5.194394in}}%
\pgfpathlineto{\pgfqpoint{5.252159in}{5.202357in}}%
\pgfpathlineto{\pgfqpoint{5.274111in}{5.212738in}}%
\pgfpathlineto{\pgfqpoint{5.296063in}{5.223964in}}%
\pgfpathlineto{\pgfqpoint{5.318015in}{5.233759in}}%
\pgfpathlineto{\pgfqpoint{5.339967in}{5.240588in}}%
\pgfpathlineto{\pgfqpoint{5.361919in}{5.244935in}}%
\pgfpathlineto{\pgfqpoint{5.383871in}{5.249086in}}%
\pgfpathlineto{\pgfqpoint{5.405823in}{5.255327in}}%
\pgfpathlineto{\pgfqpoint{5.427775in}{5.263965in}}%
\pgfpathlineto{\pgfqpoint{5.449727in}{5.273084in}}%
\pgfpathlineto{\pgfqpoint{5.471679in}{5.280371in}}%
\pgfpathlineto{\pgfqpoint{5.493631in}{5.284656in}}%
\pgfpathlineto{\pgfqpoint{5.515583in}{5.285357in}}%
\pgfpathlineto{\pgfqpoint{5.537535in}{5.282755in}}%
\pgfpathlineto{\pgfqpoint{5.559488in}{5.278635in}}%
\pgfpathlineto{\pgfqpoint{5.581440in}{5.275571in}}%
\pgfpathlineto{\pgfqpoint{5.603392in}{5.275343in}}%
\pgfpathlineto{\pgfqpoint{5.625344in}{5.277576in}}%
\pgfpathlineto{\pgfqpoint{5.647296in}{5.280615in}}%
\pgfpathlineto{\pgfqpoint{5.669248in}{5.283356in}}%
\pgfpathlineto{\pgfqpoint{5.691200in}{5.287631in}}%
\pgfpathlineto{\pgfqpoint{5.713152in}{5.300438in}}%
\pgfpathlineto{\pgfqpoint{5.735104in}{5.332755in}}%
\pgfpathlineto{\pgfqpoint{5.757056in}{5.389916in}}%
\pgfpathlineto{\pgfqpoint{5.779008in}{5.462341in}}%
\pgfpathlineto{\pgfqpoint{5.800960in}{5.529257in}}%
\pgfpathlineto{\pgfqpoint{5.822912in}{5.573919in}}%
\pgfpathlineto{\pgfqpoint{5.844864in}{5.592957in}}%
\pgfpathlineto{\pgfqpoint{5.866816in}{5.591996in}}%
\pgfpathlineto{\pgfqpoint{5.888768in}{5.577604in}}%
\pgfpathlineto{\pgfqpoint{5.910720in}{5.553920in}}%
\pgfpathlineto{\pgfqpoint{5.932672in}{5.522527in}}%
\pgfpathlineto{\pgfqpoint{5.954624in}{5.481925in}}%
\pgfpathlineto{\pgfqpoint{5.976576in}{5.426886in}}%
\pgfpathlineto{\pgfqpoint{5.998528in}{5.352150in}}%
\pgfpathlineto{\pgfqpoint{6.020480in}{5.257609in}}%
\pgfpathlineto{\pgfqpoint{6.042432in}{5.148363in}}%
\pgfpathlineto{\pgfqpoint{6.064384in}{5.029894in}}%
\pgfpathlineto{\pgfqpoint{6.086336in}{4.906595in}}%
\pgfpathlineto{\pgfqpoint{6.108288in}{4.784311in}}%
\pgfpathlineto{\pgfqpoint{6.130240in}{4.668144in}}%
\pgfusepath{stroke}%
\end{pgfscope}%
\begin{pgfscope}%
\pgfpathrectangle{\pgfqpoint{3.825278in}{3.379757in}}{\pgfqpoint{2.414722in}{2.357859in}}%
\pgfusepath{clip}%
\pgfsetroundcap%
\pgfsetroundjoin%
\pgfsetlinewidth{1.505625pt}%
\definecolor{currentstroke}{rgb}{0.505882,0.447059,0.701961}%
\pgfsetstrokecolor{currentstroke}%
\pgfsetdash{}{0pt}%
\pgfpathmoveto{\pgfqpoint{3.935038in}{5.124921in}}%
\pgfpathlineto{\pgfqpoint{3.958070in}{5.261241in}}%
\pgfpathlineto{\pgfqpoint{3.981102in}{5.385164in}}%
\pgfpathlineto{\pgfqpoint{4.004133in}{5.485540in}}%
\pgfpathlineto{\pgfqpoint{4.027165in}{5.554950in}}%
\pgfpathlineto{\pgfqpoint{4.050196in}{5.587236in}}%
\pgfpathlineto{\pgfqpoint{4.073228in}{5.582887in}}%
\pgfpathlineto{\pgfqpoint{4.096260in}{5.553920in}}%
\pgfpathlineto{\pgfqpoint{4.119291in}{5.515888in}}%
\pgfpathlineto{\pgfqpoint{4.142323in}{5.475979in}}%
\pgfpathlineto{\pgfqpoint{4.165355in}{5.430991in}}%
\pgfpathlineto{\pgfqpoint{4.188386in}{5.371405in}}%
\pgfpathlineto{\pgfqpoint{4.211418in}{5.289771in}}%
\pgfpathlineto{\pgfqpoint{4.234449in}{5.188448in}}%
\pgfpathlineto{\pgfqpoint{4.257481in}{5.077252in}}%
\pgfpathlineto{\pgfqpoint{4.280513in}{4.962618in}}%
\pgfpathlineto{\pgfqpoint{4.303544in}{4.846535in}}%
\pgfpathlineto{\pgfqpoint{4.326576in}{4.730048in}}%
\pgfpathlineto{\pgfqpoint{4.349608in}{4.614602in}}%
\pgfpathlineto{\pgfqpoint{4.372639in}{4.500673in}}%
\pgfpathlineto{\pgfqpoint{4.395671in}{4.390224in}}%
\pgfpathlineto{\pgfqpoint{4.418702in}{4.287907in}}%
\pgfpathlineto{\pgfqpoint{4.441734in}{4.194255in}}%
\pgfpathlineto{\pgfqpoint{4.464766in}{4.110484in}}%
\pgfpathlineto{\pgfqpoint{4.487797in}{4.037814in}}%
\pgfpathlineto{\pgfqpoint{4.510829in}{3.978703in}}%
\pgfpathlineto{\pgfqpoint{4.533861in}{3.938273in}}%
\pgfpathlineto{\pgfqpoint{4.556892in}{3.921490in}}%
\pgfpathlineto{\pgfqpoint{4.579924in}{3.927105in}}%
\pgfpathlineto{\pgfqpoint{4.602955in}{3.944963in}}%
\pgfpathlineto{\pgfqpoint{4.625987in}{3.963006in}}%
\pgfpathlineto{\pgfqpoint{4.649019in}{3.977397in}}%
\pgfpathlineto{\pgfqpoint{4.672050in}{3.993323in}}%
\pgfpathlineto{\pgfqpoint{4.695082in}{4.017545in}}%
\pgfpathlineto{\pgfqpoint{4.718114in}{4.054217in}}%
\pgfpathlineto{\pgfqpoint{4.741145in}{4.111002in}}%
\pgfpathlineto{\pgfqpoint{4.764177in}{4.195412in}}%
\pgfpathlineto{\pgfqpoint{4.787208in}{4.305741in}}%
\pgfpathlineto{\pgfqpoint{4.810240in}{4.432815in}}%
\pgfpathlineto{\pgfqpoint{4.833272in}{4.565843in}}%
\pgfpathlineto{\pgfqpoint{4.856303in}{4.693422in}}%
\pgfpathlineto{\pgfqpoint{4.879335in}{4.805420in}}%
\pgfpathlineto{\pgfqpoint{4.902367in}{4.898275in}}%
\pgfpathlineto{\pgfqpoint{4.925398in}{4.974327in}}%
\pgfpathlineto{\pgfqpoint{4.948430in}{5.036365in}}%
\pgfpathlineto{\pgfqpoint{4.971461in}{5.083110in}}%
\pgfpathlineto{\pgfqpoint{4.994493in}{5.110857in}}%
\pgfpathlineto{\pgfqpoint{5.017525in}{5.119235in}}%
\pgfpathlineto{\pgfqpoint{5.040556in}{5.113195in}}%
\pgfpathlineto{\pgfqpoint{5.063588in}{5.100023in}}%
\pgfpathlineto{\pgfqpoint{5.086619in}{5.086894in}}%
\pgfpathlineto{\pgfqpoint{5.109651in}{5.083561in}}%
\pgfpathlineto{\pgfqpoint{5.132683in}{5.106118in}}%
\pgfpathlineto{\pgfqpoint{5.155714in}{5.170963in}}%
\pgfpathlineto{\pgfqpoint{5.178746in}{5.279191in}}%
\pgfpathlineto{\pgfqpoint{5.201778in}{5.407407in}}%
\pgfpathlineto{\pgfqpoint{5.224809in}{5.521594in}}%
\pgfpathlineto{\pgfqpoint{5.247841in}{5.597742in}}%
\pgfpathlineto{\pgfqpoint{5.270872in}{5.630440in}}%
\pgfpathlineto{\pgfqpoint{5.293904in}{5.626564in}}%
\pgfpathlineto{\pgfqpoint{5.316936in}{5.597493in}}%
\pgfpathlineto{\pgfqpoint{5.339967in}{5.553920in}}%
\pgfpathlineto{\pgfqpoint{5.362999in}{5.502431in}}%
\pgfpathlineto{\pgfqpoint{5.386031in}{5.444201in}}%
\pgfpathlineto{\pgfqpoint{5.409062in}{5.375777in}}%
\pgfpathlineto{\pgfqpoint{5.432094in}{5.292837in}}%
\pgfpathlineto{\pgfqpoint{5.455125in}{5.196184in}}%
\pgfpathlineto{\pgfqpoint{5.478157in}{5.092391in}}%
\pgfusepath{stroke}%
\end{pgfscope}%
\begin{pgfscope}%
\pgfsetrectcap%
\pgfsetmiterjoin%
\pgfsetlinewidth{1.254687pt}%
\definecolor{currentstroke}{rgb}{1.000000,1.000000,1.000000}%
\pgfsetstrokecolor{currentstroke}%
\pgfsetdash{}{0pt}%
\pgfpathmoveto{\pgfqpoint{3.825278in}{3.379757in}}%
\pgfpathlineto{\pgfqpoint{3.825278in}{5.737616in}}%
\pgfusepath{stroke}%
\end{pgfscope}%
\begin{pgfscope}%
\pgfsetrectcap%
\pgfsetmiterjoin%
\pgfsetlinewidth{1.254687pt}%
\definecolor{currentstroke}{rgb}{1.000000,1.000000,1.000000}%
\pgfsetstrokecolor{currentstroke}%
\pgfsetdash{}{0pt}%
\pgfpathmoveto{\pgfqpoint{6.240000in}{3.379757in}}%
\pgfpathlineto{\pgfqpoint{6.240000in}{5.737616in}}%
\pgfusepath{stroke}%
\end{pgfscope}%
\begin{pgfscope}%
\pgfsetrectcap%
\pgfsetmiterjoin%
\pgfsetlinewidth{1.254687pt}%
\definecolor{currentstroke}{rgb}{1.000000,1.000000,1.000000}%
\pgfsetstrokecolor{currentstroke}%
\pgfsetdash{}{0pt}%
\pgfpathmoveto{\pgfqpoint{3.825278in}{3.379757in}}%
\pgfpathlineto{\pgfqpoint{6.240000in}{3.379757in}}%
\pgfusepath{stroke}%
\end{pgfscope}%
\begin{pgfscope}%
\pgfsetrectcap%
\pgfsetmiterjoin%
\pgfsetlinewidth{1.254687pt}%
\definecolor{currentstroke}{rgb}{1.000000,1.000000,1.000000}%
\pgfsetstrokecolor{currentstroke}%
\pgfsetdash{}{0pt}%
\pgfpathmoveto{\pgfqpoint{3.825278in}{5.737616in}}%
\pgfpathlineto{\pgfqpoint{6.240000in}{5.737616in}}%
\pgfusepath{stroke}%
\end{pgfscope}%
\begin{pgfscope}%
\pgfsetbuttcap%
\pgfsetmiterjoin%
\definecolor{currentfill}{rgb}{0.917647,0.917647,0.949020}%
\pgfsetfillcolor{currentfill}%
\pgfsetlinewidth{0.000000pt}%
\definecolor{currentstroke}{rgb}{0.000000,0.000000,0.000000}%
\pgfsetstrokecolor{currentstroke}%
\pgfsetstrokeopacity{0.000000}%
\pgfsetdash{}{0pt}%
\pgfpathmoveto{\pgfqpoint{0.828241in}{0.574768in}}%
\pgfpathlineto{\pgfqpoint{3.242963in}{0.574768in}}%
\pgfpathlineto{\pgfqpoint{3.242963in}{2.932627in}}%
\pgfpathlineto{\pgfqpoint{0.828241in}{2.932627in}}%
\pgfpathclose%
\pgfusepath{fill}%
\end{pgfscope}%
\begin{pgfscope}%
\pgfpathrectangle{\pgfqpoint{0.828241in}{0.574768in}}{\pgfqpoint{2.414722in}{2.357859in}}%
\pgfusepath{clip}%
\pgfsetroundcap%
\pgfsetroundjoin%
\pgfsetlinewidth{1.003750pt}%
\definecolor{currentstroke}{rgb}{1.000000,1.000000,1.000000}%
\pgfsetstrokecolor{currentstroke}%
\pgfsetdash{}{0pt}%
\pgfpathmoveto{\pgfqpoint{0.938001in}{0.574768in}}%
\pgfpathlineto{\pgfqpoint{0.938001in}{2.932627in}}%
\pgfusepath{stroke}%
\end{pgfscope}%
\begin{pgfscope}%
\definecolor{textcolor}{rgb}{0.150000,0.150000,0.150000}%
\pgfsetstrokecolor{textcolor}%
\pgfsetfillcolor{textcolor}%
\pgftext[x=0.938001in,y=0.442824in,,top]{\color{textcolor}\sffamily\fontsize{11.000000}{13.200000}\selectfont \(\displaystyle 0.0\)}%
\end{pgfscope}%
\begin{pgfscope}%
\pgfpathrectangle{\pgfqpoint{0.828241in}{0.574768in}}{\pgfqpoint{2.414722in}{2.357859in}}%
\pgfusepath{clip}%
\pgfsetroundcap%
\pgfsetroundjoin%
\pgfsetlinewidth{1.003750pt}%
\definecolor{currentstroke}{rgb}{1.000000,1.000000,1.000000}%
\pgfsetstrokecolor{currentstroke}%
\pgfsetdash{}{0pt}%
\pgfpathmoveto{\pgfqpoint{1.787335in}{0.574768in}}%
\pgfpathlineto{\pgfqpoint{1.787335in}{2.932627in}}%
\pgfusepath{stroke}%
\end{pgfscope}%
\begin{pgfscope}%
\definecolor{textcolor}{rgb}{0.150000,0.150000,0.150000}%
\pgfsetstrokecolor{textcolor}%
\pgfsetfillcolor{textcolor}%
\pgftext[x=1.787335in,y=0.442824in,,top]{\color{textcolor}\sffamily\fontsize{11.000000}{13.200000}\selectfont \(\displaystyle 0.5\)}%
\end{pgfscope}%
\begin{pgfscope}%
\pgfpathrectangle{\pgfqpoint{0.828241in}{0.574768in}}{\pgfqpoint{2.414722in}{2.357859in}}%
\pgfusepath{clip}%
\pgfsetroundcap%
\pgfsetroundjoin%
\pgfsetlinewidth{1.003750pt}%
\definecolor{currentstroke}{rgb}{1.000000,1.000000,1.000000}%
\pgfsetstrokecolor{currentstroke}%
\pgfsetdash{}{0pt}%
\pgfpathmoveto{\pgfqpoint{2.636669in}{0.574768in}}%
\pgfpathlineto{\pgfqpoint{2.636669in}{2.932627in}}%
\pgfusepath{stroke}%
\end{pgfscope}%
\begin{pgfscope}%
\definecolor{textcolor}{rgb}{0.150000,0.150000,0.150000}%
\pgfsetstrokecolor{textcolor}%
\pgfsetfillcolor{textcolor}%
\pgftext[x=2.636669in,y=0.442824in,,top]{\color{textcolor}\sffamily\fontsize{11.000000}{13.200000}\selectfont \(\displaystyle 1.0\)}%
\end{pgfscope}%
\begin{pgfscope}%
\definecolor{textcolor}{rgb}{0.150000,0.150000,0.150000}%
\pgfsetstrokecolor{textcolor}%
\pgfsetfillcolor{textcolor}%
\pgftext[x=2.035602in,y=0.252083in,,top]{\color{textcolor}\sffamily\fontsize{11.000000}{13.200000}\selectfont Time [s]}%
\end{pgfscope}%
\begin{pgfscope}%
\pgfpathrectangle{\pgfqpoint{0.828241in}{0.574768in}}{\pgfqpoint{2.414722in}{2.357859in}}%
\pgfusepath{clip}%
\pgfsetroundcap%
\pgfsetroundjoin%
\pgfsetlinewidth{1.003750pt}%
\definecolor{currentstroke}{rgb}{1.000000,1.000000,1.000000}%
\pgfsetstrokecolor{currentstroke}%
\pgfsetdash{}{0pt}%
\pgfpathmoveto{\pgfqpoint{0.828241in}{0.591647in}}%
\pgfpathlineto{\pgfqpoint{3.242963in}{0.591647in}}%
\pgfusepath{stroke}%
\end{pgfscope}%
\begin{pgfscope}%
\definecolor{textcolor}{rgb}{0.150000,0.150000,0.150000}%
\pgfsetstrokecolor{textcolor}%
\pgfsetfillcolor{textcolor}%
\pgftext[x=0.307639in,y=0.538841in,left,base]{\color{textcolor}\sffamily\fontsize{11.000000}{13.200000}\selectfont \(\displaystyle -15.0\)}%
\end{pgfscope}%
\begin{pgfscope}%
\pgfpathrectangle{\pgfqpoint{0.828241in}{0.574768in}}{\pgfqpoint{2.414722in}{2.357859in}}%
\pgfusepath{clip}%
\pgfsetroundcap%
\pgfsetroundjoin%
\pgfsetlinewidth{1.003750pt}%
\definecolor{currentstroke}{rgb}{1.000000,1.000000,1.000000}%
\pgfsetstrokecolor{currentstroke}%
\pgfsetdash{}{0pt}%
\pgfpathmoveto{\pgfqpoint{0.828241in}{0.949308in}}%
\pgfpathlineto{\pgfqpoint{3.242963in}{0.949308in}}%
\pgfusepath{stroke}%
\end{pgfscope}%
\begin{pgfscope}%
\definecolor{textcolor}{rgb}{0.150000,0.150000,0.150000}%
\pgfsetstrokecolor{textcolor}%
\pgfsetfillcolor{textcolor}%
\pgftext[x=0.307639in,y=0.896502in,left,base]{\color{textcolor}\sffamily\fontsize{11.000000}{13.200000}\selectfont \(\displaystyle -12.5\)}%
\end{pgfscope}%
\begin{pgfscope}%
\pgfpathrectangle{\pgfqpoint{0.828241in}{0.574768in}}{\pgfqpoint{2.414722in}{2.357859in}}%
\pgfusepath{clip}%
\pgfsetroundcap%
\pgfsetroundjoin%
\pgfsetlinewidth{1.003750pt}%
\definecolor{currentstroke}{rgb}{1.000000,1.000000,1.000000}%
\pgfsetstrokecolor{currentstroke}%
\pgfsetdash{}{0pt}%
\pgfpathmoveto{\pgfqpoint{0.828241in}{1.306969in}}%
\pgfpathlineto{\pgfqpoint{3.242963in}{1.306969in}}%
\pgfusepath{stroke}%
\end{pgfscope}%
\begin{pgfscope}%
\definecolor{textcolor}{rgb}{0.150000,0.150000,0.150000}%
\pgfsetstrokecolor{textcolor}%
\pgfsetfillcolor{textcolor}%
\pgftext[x=0.307639in,y=1.254163in,left,base]{\color{textcolor}\sffamily\fontsize{11.000000}{13.200000}\selectfont \(\displaystyle -10.0\)}%
\end{pgfscope}%
\begin{pgfscope}%
\pgfpathrectangle{\pgfqpoint{0.828241in}{0.574768in}}{\pgfqpoint{2.414722in}{2.357859in}}%
\pgfusepath{clip}%
\pgfsetroundcap%
\pgfsetroundjoin%
\pgfsetlinewidth{1.003750pt}%
\definecolor{currentstroke}{rgb}{1.000000,1.000000,1.000000}%
\pgfsetstrokecolor{currentstroke}%
\pgfsetdash{}{0pt}%
\pgfpathmoveto{\pgfqpoint{0.828241in}{1.664630in}}%
\pgfpathlineto{\pgfqpoint{3.242963in}{1.664630in}}%
\pgfusepath{stroke}%
\end{pgfscope}%
\begin{pgfscope}%
\definecolor{textcolor}{rgb}{0.150000,0.150000,0.150000}%
\pgfsetstrokecolor{textcolor}%
\pgfsetfillcolor{textcolor}%
\pgftext[x=0.383680in,y=1.611824in,left,base]{\color{textcolor}\sffamily\fontsize{11.000000}{13.200000}\selectfont \(\displaystyle -7.5\)}%
\end{pgfscope}%
\begin{pgfscope}%
\pgfpathrectangle{\pgfqpoint{0.828241in}{0.574768in}}{\pgfqpoint{2.414722in}{2.357859in}}%
\pgfusepath{clip}%
\pgfsetroundcap%
\pgfsetroundjoin%
\pgfsetlinewidth{1.003750pt}%
\definecolor{currentstroke}{rgb}{1.000000,1.000000,1.000000}%
\pgfsetstrokecolor{currentstroke}%
\pgfsetdash{}{0pt}%
\pgfpathmoveto{\pgfqpoint{0.828241in}{2.022291in}}%
\pgfpathlineto{\pgfqpoint{3.242963in}{2.022291in}}%
\pgfusepath{stroke}%
\end{pgfscope}%
\begin{pgfscope}%
\definecolor{textcolor}{rgb}{0.150000,0.150000,0.150000}%
\pgfsetstrokecolor{textcolor}%
\pgfsetfillcolor{textcolor}%
\pgftext[x=0.383680in,y=1.969485in,left,base]{\color{textcolor}\sffamily\fontsize{11.000000}{13.200000}\selectfont \(\displaystyle -5.0\)}%
\end{pgfscope}%
\begin{pgfscope}%
\pgfpathrectangle{\pgfqpoint{0.828241in}{0.574768in}}{\pgfqpoint{2.414722in}{2.357859in}}%
\pgfusepath{clip}%
\pgfsetroundcap%
\pgfsetroundjoin%
\pgfsetlinewidth{1.003750pt}%
\definecolor{currentstroke}{rgb}{1.000000,1.000000,1.000000}%
\pgfsetstrokecolor{currentstroke}%
\pgfsetdash{}{0pt}%
\pgfpathmoveto{\pgfqpoint{0.828241in}{2.379952in}}%
\pgfpathlineto{\pgfqpoint{3.242963in}{2.379952in}}%
\pgfusepath{stroke}%
\end{pgfscope}%
\begin{pgfscope}%
\definecolor{textcolor}{rgb}{0.150000,0.150000,0.150000}%
\pgfsetstrokecolor{textcolor}%
\pgfsetfillcolor{textcolor}%
\pgftext[x=0.383680in,y=2.327146in,left,base]{\color{textcolor}\sffamily\fontsize{11.000000}{13.200000}\selectfont \(\displaystyle -2.5\)}%
\end{pgfscope}%
\begin{pgfscope}%
\pgfpathrectangle{\pgfqpoint{0.828241in}{0.574768in}}{\pgfqpoint{2.414722in}{2.357859in}}%
\pgfusepath{clip}%
\pgfsetroundcap%
\pgfsetroundjoin%
\pgfsetlinewidth{1.003750pt}%
\definecolor{currentstroke}{rgb}{1.000000,1.000000,1.000000}%
\pgfsetstrokecolor{currentstroke}%
\pgfsetdash{}{0pt}%
\pgfpathmoveto{\pgfqpoint{0.828241in}{2.737613in}}%
\pgfpathlineto{\pgfqpoint{3.242963in}{2.737613in}}%
\pgfusepath{stroke}%
\end{pgfscope}%
\begin{pgfscope}%
\definecolor{textcolor}{rgb}{0.150000,0.150000,0.150000}%
\pgfsetstrokecolor{textcolor}%
\pgfsetfillcolor{textcolor}%
\pgftext[x=0.501968in,y=2.684807in,left,base]{\color{textcolor}\sffamily\fontsize{11.000000}{13.200000}\selectfont \(\displaystyle 0.0\)}%
\end{pgfscope}%
\begin{pgfscope}%
\definecolor{textcolor}{rgb}{0.150000,0.150000,0.150000}%
\pgfsetstrokecolor{textcolor}%
\pgfsetfillcolor{textcolor}%
\pgftext[x=0.252083in,y=1.753698in,,bottom,rotate=90.000000]{\color{textcolor}\sffamily\fontsize{11.000000}{13.200000}\selectfont APLAX/gls}%
\end{pgfscope}%
\begin{pgfscope}%
\pgfpathrectangle{\pgfqpoint{0.828241in}{0.574768in}}{\pgfqpoint{2.414722in}{2.357859in}}%
\pgfusepath{clip}%
\pgfsetroundcap%
\pgfsetroundjoin%
\pgfsetlinewidth{1.505625pt}%
\definecolor{currentstroke}{rgb}{0.298039,0.447059,0.690196}%
\pgfsetstrokecolor{currentstroke}%
\pgfsetdash{}{0pt}%
\pgfpathmoveto{\pgfqpoint{0.938001in}{1.848014in}}%
\pgfpathlineto{\pgfqpoint{0.964135in}{1.887173in}}%
\pgfpathlineto{\pgfqpoint{0.990268in}{1.963865in}}%
\pgfpathlineto{\pgfqpoint{1.016401in}{2.100659in}}%
\pgfpathlineto{\pgfqpoint{1.042535in}{2.284777in}}%
\pgfpathlineto{\pgfqpoint{1.068668in}{2.471666in}}%
\pgfpathlineto{\pgfqpoint{1.094801in}{2.613499in}}%
\pgfpathlineto{\pgfqpoint{1.120935in}{2.691464in}}%
\pgfpathlineto{\pgfqpoint{1.147068in}{2.720469in}}%
\pgfpathlineto{\pgfqpoint{1.173201in}{2.729155in}}%
\pgfpathlineto{\pgfqpoint{1.199335in}{2.737613in}}%
\pgfpathlineto{\pgfqpoint{1.225468in}{2.749788in}}%
\pgfpathlineto{\pgfqpoint{1.251602in}{2.759411in}}%
\pgfpathlineto{\pgfqpoint{1.277735in}{2.758272in}}%
\pgfpathlineto{\pgfqpoint{1.303868in}{2.737804in}}%
\pgfpathlineto{\pgfqpoint{1.330002in}{2.688670in}}%
\pgfpathlineto{\pgfqpoint{1.356135in}{2.604964in}}%
\pgfpathlineto{\pgfqpoint{1.382268in}{2.488723in}}%
\pgfpathlineto{\pgfqpoint{1.408402in}{2.348875in}}%
\pgfpathlineto{\pgfqpoint{1.434535in}{2.196372in}}%
\pgfpathlineto{\pgfqpoint{1.460668in}{2.037786in}}%
\pgfpathlineto{\pgfqpoint{1.486802in}{1.873633in}}%
\pgfpathlineto{\pgfqpoint{1.512935in}{1.703132in}}%
\pgfpathlineto{\pgfqpoint{1.539068in}{1.529329in}}%
\pgfpathlineto{\pgfqpoint{1.565202in}{1.358930in}}%
\pgfpathlineto{\pgfqpoint{1.591335in}{1.198362in}}%
\pgfpathlineto{\pgfqpoint{1.617468in}{1.052423in}}%
\pgfpathlineto{\pgfqpoint{1.643602in}{0.925447in}}%
\pgfpathlineto{\pgfqpoint{1.669735in}{0.820194in}}%
\pgfpathlineto{\pgfqpoint{1.695869in}{0.740318in}}%
\pgfpathlineto{\pgfqpoint{1.722002in}{0.692294in}}%
\pgfpathlineto{\pgfqpoint{1.748135in}{0.681944in}}%
\pgfpathlineto{\pgfqpoint{1.774269in}{0.707189in}}%
\pgfpathlineto{\pgfqpoint{1.800402in}{0.755123in}}%
\pgfpathlineto{\pgfqpoint{1.826535in}{0.807765in}}%
\pgfpathlineto{\pgfqpoint{1.852669in}{0.849046in}}%
\pgfpathlineto{\pgfqpoint{1.878802in}{0.870351in}}%
\pgfpathlineto{\pgfqpoint{1.904935in}{0.876511in}}%
\pgfpathlineto{\pgfqpoint{1.931069in}{0.886417in}}%
\pgfpathlineto{\pgfqpoint{1.957202in}{0.920951in}}%
\pgfpathlineto{\pgfqpoint{1.983335in}{0.992813in}}%
\pgfpathlineto{\pgfqpoint{2.009469in}{1.099796in}}%
\pgfpathlineto{\pgfqpoint{2.035602in}{1.226671in}}%
\pgfpathlineto{\pgfqpoint{2.061735in}{1.353648in}}%
\pgfpathlineto{\pgfqpoint{2.087869in}{1.466534in}}%
\pgfpathlineto{\pgfqpoint{2.114002in}{1.560280in}}%
\pgfpathlineto{\pgfqpoint{2.140136in}{1.636343in}}%
\pgfpathlineto{\pgfqpoint{2.166269in}{1.698248in}}%
\pgfpathlineto{\pgfqpoint{2.192402in}{1.748317in}}%
\pgfpathlineto{\pgfqpoint{2.218536in}{1.787080in}}%
\pgfpathlineto{\pgfqpoint{2.244669in}{1.813467in}}%
\pgfpathlineto{\pgfqpoint{2.270802in}{1.827636in}}%
\pgfpathlineto{\pgfqpoint{2.296936in}{1.837170in}}%
\pgfpathlineto{\pgfqpoint{2.323069in}{1.861296in}}%
\pgfpathlineto{\pgfqpoint{2.349202in}{1.928443in}}%
\pgfpathlineto{\pgfqpoint{2.375336in}{2.060477in}}%
\pgfpathlineto{\pgfqpoint{2.401469in}{2.248211in}}%
\pgfpathlineto{\pgfqpoint{2.427602in}{2.445947in}}%
\pgfpathlineto{\pgfqpoint{2.453736in}{2.600793in}}%
\pgfpathlineto{\pgfqpoint{2.479869in}{2.688672in}}%
\pgfpathlineto{\pgfqpoint{2.506002in}{2.722733in}}%
\pgfpathlineto{\pgfqpoint{2.532136in}{2.731696in}}%
\pgfpathlineto{\pgfqpoint{2.558269in}{2.737613in}}%
\pgfpathlineto{\pgfqpoint{2.584403in}{2.749858in}}%
\pgfpathlineto{\pgfqpoint{2.610536in}{2.765426in}}%
\pgfpathlineto{\pgfqpoint{2.636669in}{2.771576in}}%
\pgfpathlineto{\pgfqpoint{2.662803in}{2.752409in}}%
\pgfpathlineto{\pgfqpoint{2.688936in}{2.696724in}}%
\pgfpathlineto{\pgfqpoint{2.715069in}{2.605924in}}%
\pgfpathlineto{\pgfqpoint{2.741203in}{2.490240in}}%
\pgfpathlineto{\pgfqpoint{2.767336in}{2.358659in}}%
\pgfpathlineto{\pgfqpoint{2.793469in}{2.218516in}}%
\pgfusepath{stroke}%
\end{pgfscope}%
\begin{pgfscope}%
\pgfpathrectangle{\pgfqpoint{0.828241in}{0.574768in}}{\pgfqpoint{2.414722in}{2.357859in}}%
\pgfusepath{clip}%
\pgfsetroundcap%
\pgfsetroundjoin%
\pgfsetlinewidth{1.505625pt}%
\definecolor{currentstroke}{rgb}{0.866667,0.517647,0.321569}%
\pgfsetstrokecolor{currentstroke}%
\pgfsetdash{}{0pt}%
\pgfpathmoveto{\pgfqpoint{0.938001in}{2.294479in}}%
\pgfpathlineto{\pgfqpoint{0.965848in}{2.307680in}}%
\pgfpathlineto{\pgfqpoint{0.993695in}{2.324701in}}%
\pgfpathlineto{\pgfqpoint{1.021542in}{2.353736in}}%
\pgfpathlineto{\pgfqpoint{1.049389in}{2.402050in}}%
\pgfpathlineto{\pgfqpoint{1.077236in}{2.467995in}}%
\pgfpathlineto{\pgfqpoint{1.105083in}{2.541025in}}%
\pgfpathlineto{\pgfqpoint{1.132930in}{2.609569in}}%
\pgfpathlineto{\pgfqpoint{1.160777in}{2.665703in}}%
\pgfpathlineto{\pgfqpoint{1.188624in}{2.705936in}}%
\pgfpathlineto{\pgfqpoint{1.216471in}{2.729661in}}%
\pgfpathlineto{\pgfqpoint{1.244318in}{2.737613in}}%
\pgfpathlineto{\pgfqpoint{1.272166in}{2.732897in}}%
\pgfpathlineto{\pgfqpoint{1.300013in}{2.723949in}}%
\pgfpathlineto{\pgfqpoint{1.327860in}{2.722918in}}%
\pgfpathlineto{\pgfqpoint{1.355707in}{2.736927in}}%
\pgfpathlineto{\pgfqpoint{1.383554in}{2.762861in}}%
\pgfpathlineto{\pgfqpoint{1.411401in}{2.791690in}}%
\pgfpathlineto{\pgfqpoint{1.439248in}{2.814851in}}%
\pgfpathlineto{\pgfqpoint{1.467095in}{2.825452in}}%
\pgfpathlineto{\pgfqpoint{1.494942in}{2.817354in}}%
\pgfpathlineto{\pgfqpoint{1.522789in}{2.787462in}}%
\pgfpathlineto{\pgfqpoint{1.550636in}{2.737760in}}%
\pgfpathlineto{\pgfqpoint{1.578483in}{2.674467in}}%
\pgfpathlineto{\pgfqpoint{1.606330in}{2.603438in}}%
\pgfpathlineto{\pgfqpoint{1.634177in}{2.529585in}}%
\pgfpathlineto{\pgfqpoint{1.662024in}{2.458884in}}%
\pgfpathlineto{\pgfqpoint{1.689871in}{2.398033in}}%
\pgfpathlineto{\pgfqpoint{1.717718in}{2.349091in}}%
\pgfpathlineto{\pgfqpoint{1.745565in}{2.310822in}}%
\pgfpathlineto{\pgfqpoint{1.773412in}{2.285176in}}%
\pgfpathlineto{\pgfqpoint{1.801259in}{2.278017in}}%
\pgfpathlineto{\pgfqpoint{1.829106in}{2.290633in}}%
\pgfpathlineto{\pgfqpoint{1.856953in}{2.313165in}}%
\pgfpathlineto{\pgfqpoint{1.884800in}{2.329022in}}%
\pgfpathlineto{\pgfqpoint{1.912647in}{2.322496in}}%
\pgfpathlineto{\pgfqpoint{1.940494in}{2.283410in}}%
\pgfpathlineto{\pgfqpoint{1.968341in}{2.212489in}}%
\pgfpathlineto{\pgfqpoint{1.996188in}{2.125126in}}%
\pgfpathlineto{\pgfqpoint{2.024035in}{2.047460in}}%
\pgfpathlineto{\pgfqpoint{2.051882in}{2.005180in}}%
\pgfpathlineto{\pgfqpoint{2.079729in}{2.010960in}}%
\pgfpathlineto{\pgfqpoint{2.107576in}{2.059791in}}%
\pgfpathlineto{\pgfqpoint{2.135423in}{2.133631in}}%
\pgfpathlineto{\pgfqpoint{2.163270in}{2.211171in}}%
\pgfpathlineto{\pgfqpoint{2.191117in}{2.275211in}}%
\pgfpathlineto{\pgfqpoint{2.218964in}{2.316898in}}%
\pgfpathlineto{\pgfqpoint{2.246811in}{2.336796in}}%
\pgfpathlineto{\pgfqpoint{2.274658in}{2.342576in}}%
\pgfpathlineto{\pgfqpoint{2.302505in}{2.343203in}}%
\pgfpathlineto{\pgfqpoint{2.330352in}{2.343115in}}%
\pgfpathlineto{\pgfqpoint{2.358199in}{2.341872in}}%
\pgfpathlineto{\pgfqpoint{2.386046in}{2.339037in}}%
\pgfpathlineto{\pgfqpoint{2.413893in}{2.336463in}}%
\pgfpathlineto{\pgfqpoint{2.441740in}{2.336416in}}%
\pgfpathlineto{\pgfqpoint{2.469587in}{2.339822in}}%
\pgfpathlineto{\pgfqpoint{2.497434in}{2.348039in}}%
\pgfpathlineto{\pgfqpoint{2.525281in}{2.364973in}}%
\pgfpathlineto{\pgfqpoint{2.553128in}{2.400229in}}%
\pgfpathlineto{\pgfqpoint{2.580975in}{2.462378in}}%
\pgfpathlineto{\pgfqpoint{2.608822in}{2.545224in}}%
\pgfpathlineto{\pgfqpoint{2.636669in}{2.627719in}}%
\pgfpathlineto{\pgfqpoint{2.664516in}{2.690920in}}%
\pgfpathlineto{\pgfqpoint{2.692363in}{2.729928in}}%
\pgfpathlineto{\pgfqpoint{2.720210in}{2.748322in}}%
\pgfpathlineto{\pgfqpoint{2.748057in}{2.750089in}}%
\pgfpathlineto{\pgfqpoint{2.775904in}{2.737613in}}%
\pgfpathlineto{\pgfqpoint{2.803751in}{2.716486in}}%
\pgfpathlineto{\pgfqpoint{2.831598in}{2.697868in}}%
\pgfpathlineto{\pgfqpoint{2.859445in}{2.693203in}}%
\pgfpathlineto{\pgfqpoint{2.887292in}{2.707293in}}%
\pgfpathlineto{\pgfqpoint{2.915139in}{2.737471in}}%
\pgfpathlineto{\pgfqpoint{2.942986in}{2.774755in}}%
\pgfpathlineto{\pgfqpoint{2.970833in}{2.805809in}}%
\pgfpathlineto{\pgfqpoint{2.998680in}{2.816277in}}%
\pgfpathlineto{\pgfqpoint{3.026527in}{2.795592in}}%
\pgfpathlineto{\pgfqpoint{3.054374in}{2.743979in}}%
\pgfpathlineto{\pgfqpoint{3.082221in}{2.675001in}}%
\pgfusepath{stroke}%
\end{pgfscope}%
\begin{pgfscope}%
\pgfpathrectangle{\pgfqpoint{0.828241in}{0.574768in}}{\pgfqpoint{2.414722in}{2.357859in}}%
\pgfusepath{clip}%
\pgfsetroundcap%
\pgfsetroundjoin%
\pgfsetlinewidth{1.505625pt}%
\definecolor{currentstroke}{rgb}{0.333333,0.658824,0.407843}%
\pgfsetstrokecolor{currentstroke}%
\pgfsetdash{}{0pt}%
\pgfpathmoveto{\pgfqpoint{0.938001in}{2.136340in}}%
\pgfpathlineto{\pgfqpoint{0.964543in}{2.106410in}}%
\pgfpathlineto{\pgfqpoint{0.991085in}{2.077615in}}%
\pgfpathlineto{\pgfqpoint{1.017626in}{2.070933in}}%
\pgfpathlineto{\pgfqpoint{1.044168in}{2.117754in}}%
\pgfpathlineto{\pgfqpoint{1.070710in}{2.240004in}}%
\pgfpathlineto{\pgfqpoint{1.097251in}{2.423792in}}%
\pgfpathlineto{\pgfqpoint{1.123793in}{2.609910in}}%
\pgfpathlineto{\pgfqpoint{1.150335in}{2.737613in}}%
\pgfpathlineto{\pgfqpoint{1.176877in}{2.783258in}}%
\pgfpathlineto{\pgfqpoint{1.203418in}{2.762976in}}%
\pgfpathlineto{\pgfqpoint{1.229960in}{2.719507in}}%
\pgfpathlineto{\pgfqpoint{1.256502in}{2.691332in}}%
\pgfpathlineto{\pgfqpoint{1.283043in}{2.684642in}}%
\pgfpathlineto{\pgfqpoint{1.309585in}{2.674415in}}%
\pgfpathlineto{\pgfqpoint{1.336127in}{2.631961in}}%
\pgfpathlineto{\pgfqpoint{1.362668in}{2.551332in}}%
\pgfpathlineto{\pgfqpoint{1.389210in}{2.452009in}}%
\pgfpathlineto{\pgfqpoint{1.415752in}{2.355172in}}%
\pgfpathlineto{\pgfqpoint{1.442293in}{2.265838in}}%
\pgfpathlineto{\pgfqpoint{1.468835in}{2.179884in}}%
\pgfpathlineto{\pgfqpoint{1.495377in}{2.096896in}}%
\pgfpathlineto{\pgfqpoint{1.521918in}{2.021371in}}%
\pgfpathlineto{\pgfqpoint{1.548460in}{1.957853in}}%
\pgfpathlineto{\pgfqpoint{1.575002in}{1.911397in}}%
\pgfpathlineto{\pgfqpoint{1.601543in}{1.884504in}}%
\pgfpathlineto{\pgfqpoint{1.628085in}{1.873265in}}%
\pgfpathlineto{\pgfqpoint{1.654627in}{1.865673in}}%
\pgfpathlineto{\pgfqpoint{1.681169in}{1.848346in}}%
\pgfpathlineto{\pgfqpoint{1.707710in}{1.819545in}}%
\pgfpathlineto{\pgfqpoint{1.734252in}{1.791223in}}%
\pgfpathlineto{\pgfqpoint{1.760794in}{1.773897in}}%
\pgfpathlineto{\pgfqpoint{1.787335in}{1.762400in}}%
\pgfpathlineto{\pgfqpoint{1.813877in}{1.749633in}}%
\pgfpathlineto{\pgfqpoint{1.840419in}{1.742674in}}%
\pgfpathlineto{\pgfqpoint{1.866960in}{1.760650in}}%
\pgfpathlineto{\pgfqpoint{1.893502in}{1.820801in}}%
\pgfpathlineto{\pgfqpoint{1.920044in}{1.916947in}}%
\pgfpathlineto{\pgfqpoint{1.946585in}{2.016103in}}%
\pgfpathlineto{\pgfqpoint{1.973127in}{2.089337in}}%
\pgfpathlineto{\pgfqpoint{1.999669in}{2.131171in}}%
\pgfpathlineto{\pgfqpoint{2.026210in}{2.152805in}}%
\pgfpathlineto{\pgfqpoint{2.052752in}{2.165241in}}%
\pgfpathlineto{\pgfqpoint{2.079294in}{2.170336in}}%
\pgfpathlineto{\pgfqpoint{2.105836in}{2.165958in}}%
\pgfpathlineto{\pgfqpoint{2.132377in}{2.151207in}}%
\pgfpathlineto{\pgfqpoint{2.158919in}{2.136303in}}%
\pgfpathlineto{\pgfqpoint{2.185461in}{2.148750in}}%
\pgfpathlineto{\pgfqpoint{2.212002in}{2.219176in}}%
\pgfpathlineto{\pgfqpoint{2.238544in}{2.353923in}}%
\pgfpathlineto{\pgfqpoint{2.265086in}{2.521078in}}%
\pgfpathlineto{\pgfqpoint{2.291627in}{2.664905in}}%
\pgfpathlineto{\pgfqpoint{2.318169in}{2.737613in}}%
\pgfpathlineto{\pgfqpoint{2.344711in}{2.726446in}}%
\pgfpathlineto{\pgfqpoint{2.371252in}{2.665450in}}%
\pgfpathlineto{\pgfqpoint{2.397794in}{2.606091in}}%
\pgfpathlineto{\pgfqpoint{2.424336in}{2.575167in}}%
\pgfpathlineto{\pgfqpoint{2.450877in}{2.563231in}}%
\pgfpathlineto{\pgfqpoint{2.477419in}{2.543465in}}%
\pgfpathlineto{\pgfqpoint{2.503961in}{2.497334in}}%
\pgfusepath{stroke}%
\end{pgfscope}%
\begin{pgfscope}%
\pgfpathrectangle{\pgfqpoint{0.828241in}{0.574768in}}{\pgfqpoint{2.414722in}{2.357859in}}%
\pgfusepath{clip}%
\pgfsetroundcap%
\pgfsetroundjoin%
\pgfsetlinewidth{1.505625pt}%
\definecolor{currentstroke}{rgb}{0.768627,0.305882,0.321569}%
\pgfsetstrokecolor{currentstroke}%
\pgfsetdash{}{0pt}%
\pgfpathmoveto{\pgfqpoint{0.938001in}{2.337728in}}%
\pgfpathlineto{\pgfqpoint{0.964135in}{2.339310in}}%
\pgfpathlineto{\pgfqpoint{0.990268in}{2.342933in}}%
\pgfpathlineto{\pgfqpoint{1.016401in}{2.350773in}}%
\pgfpathlineto{\pgfqpoint{1.042535in}{2.367237in}}%
\pgfpathlineto{\pgfqpoint{1.068668in}{2.399273in}}%
\pgfpathlineto{\pgfqpoint{1.094801in}{2.453830in}}%
\pgfpathlineto{\pgfqpoint{1.120935in}{2.529314in}}%
\pgfpathlineto{\pgfqpoint{1.147068in}{2.609417in}}%
\pgfpathlineto{\pgfqpoint{1.173201in}{2.673000in}}%
\pgfpathlineto{\pgfqpoint{1.199335in}{2.709809in}}%
\pgfpathlineto{\pgfqpoint{1.225468in}{2.725255in}}%
\pgfpathlineto{\pgfqpoint{1.251602in}{2.730862in}}%
\pgfpathlineto{\pgfqpoint{1.277735in}{2.734462in}}%
\pgfpathlineto{\pgfqpoint{1.303868in}{2.737613in}}%
\pgfpathlineto{\pgfqpoint{1.330002in}{2.740198in}}%
\pgfpathlineto{\pgfqpoint{1.356135in}{2.745777in}}%
\pgfpathlineto{\pgfqpoint{1.382268in}{2.758966in}}%
\pgfpathlineto{\pgfqpoint{1.408402in}{2.777840in}}%
\pgfpathlineto{\pgfqpoint{1.434535in}{2.793569in}}%
\pgfpathlineto{\pgfqpoint{1.460668in}{2.794159in}}%
\pgfpathlineto{\pgfqpoint{1.486802in}{2.769962in}}%
\pgfpathlineto{\pgfqpoint{1.512935in}{2.720982in}}%
\pgfpathlineto{\pgfqpoint{1.539068in}{2.657320in}}%
\pgfpathlineto{\pgfqpoint{1.565202in}{2.588590in}}%
\pgfpathlineto{\pgfqpoint{1.591335in}{2.518051in}}%
\pgfpathlineto{\pgfqpoint{1.617468in}{2.445018in}}%
\pgfpathlineto{\pgfqpoint{1.643602in}{2.368887in}}%
\pgfpathlineto{\pgfqpoint{1.669735in}{2.291009in}}%
\pgfpathlineto{\pgfqpoint{1.695869in}{2.213615in}}%
\pgfpathlineto{\pgfqpoint{1.722002in}{2.137682in}}%
\pgfpathlineto{\pgfqpoint{1.748135in}{2.063654in}}%
\pgfpathlineto{\pgfqpoint{1.774269in}{1.994411in}}%
\pgfpathlineto{\pgfqpoint{1.800402in}{1.933384in}}%
\pgfpathlineto{\pgfqpoint{1.826535in}{1.884065in}}%
\pgfpathlineto{\pgfqpoint{1.852669in}{1.849312in}}%
\pgfpathlineto{\pgfqpoint{1.878802in}{1.829684in}}%
\pgfpathlineto{\pgfqpoint{1.904935in}{1.822008in}}%
\pgfpathlineto{\pgfqpoint{1.931069in}{1.820786in}}%
\pgfpathlineto{\pgfqpoint{1.957202in}{1.821566in}}%
\pgfpathlineto{\pgfqpoint{1.983335in}{1.821258in}}%
\pgfpathlineto{\pgfqpoint{2.009469in}{1.816727in}}%
\pgfpathlineto{\pgfqpoint{2.035602in}{1.804539in}}%
\pgfpathlineto{\pgfqpoint{2.061735in}{1.785784in}}%
\pgfpathlineto{\pgfqpoint{2.087869in}{1.769216in}}%
\pgfpathlineto{\pgfqpoint{2.114002in}{1.764785in}}%
\pgfpathlineto{\pgfqpoint{2.140136in}{1.775805in}}%
\pgfpathlineto{\pgfqpoint{2.166269in}{1.799204in}}%
\pgfpathlineto{\pgfqpoint{2.192402in}{1.830926in}}%
\pgfpathlineto{\pgfqpoint{2.218536in}{1.870244in}}%
\pgfpathlineto{\pgfqpoint{2.244669in}{1.920246in}}%
\pgfpathlineto{\pgfqpoint{2.270802in}{1.982637in}}%
\pgfpathlineto{\pgfqpoint{2.296936in}{2.053343in}}%
\pgfpathlineto{\pgfqpoint{2.323069in}{2.123295in}}%
\pgfpathlineto{\pgfqpoint{2.349202in}{2.183237in}}%
\pgfpathlineto{\pgfqpoint{2.375336in}{2.230024in}}%
\pgfpathlineto{\pgfqpoint{2.401469in}{2.266714in}}%
\pgfpathlineto{\pgfqpoint{2.427602in}{2.296208in}}%
\pgfpathlineto{\pgfqpoint{2.453736in}{2.318240in}}%
\pgfpathlineto{\pgfqpoint{2.479869in}{2.332936in}}%
\pgfpathlineto{\pgfqpoint{2.506002in}{2.343754in}}%
\pgfpathlineto{\pgfqpoint{2.532136in}{2.355440in}}%
\pgfpathlineto{\pgfqpoint{2.558269in}{2.369036in}}%
\pgfpathlineto{\pgfqpoint{2.584403in}{2.381173in}}%
\pgfpathlineto{\pgfqpoint{2.610536in}{2.387973in}}%
\pgfpathlineto{\pgfqpoint{2.636669in}{2.388936in}}%
\pgfpathlineto{\pgfqpoint{2.662803in}{2.386984in}}%
\pgfpathlineto{\pgfqpoint{2.688936in}{2.385673in}}%
\pgfpathlineto{\pgfqpoint{2.715069in}{2.386836in}}%
\pgfpathlineto{\pgfqpoint{2.741203in}{2.390170in}}%
\pgfpathlineto{\pgfqpoint{2.767336in}{2.394581in}}%
\pgfpathlineto{\pgfqpoint{2.793469in}{2.399391in}}%
\pgfpathlineto{\pgfqpoint{2.819603in}{2.405526in}}%
\pgfpathlineto{\pgfqpoint{2.845736in}{2.415674in}}%
\pgfpathlineto{\pgfqpoint{2.871869in}{2.434341in}}%
\pgfpathlineto{\pgfqpoint{2.898003in}{2.467210in}}%
\pgfpathlineto{\pgfqpoint{2.924136in}{2.516562in}}%
\pgfpathlineto{\pgfqpoint{2.950269in}{2.577014in}}%
\pgfpathlineto{\pgfqpoint{2.976403in}{2.637830in}}%
\pgfpathlineto{\pgfqpoint{3.002536in}{2.688798in}}%
\pgfpathlineto{\pgfqpoint{3.028669in}{2.723724in}}%
\pgfpathlineto{\pgfqpoint{3.054803in}{2.742125in}}%
\pgfpathlineto{\pgfqpoint{3.080936in}{2.747232in}}%
\pgfpathlineto{\pgfqpoint{3.107070in}{2.744029in}}%
\pgfpathlineto{\pgfqpoint{3.133203in}{2.737613in}}%
\pgfusepath{stroke}%
\end{pgfscope}%
\begin{pgfscope}%
\pgfpathrectangle{\pgfqpoint{0.828241in}{0.574768in}}{\pgfqpoint{2.414722in}{2.357859in}}%
\pgfusepath{clip}%
\pgfsetroundcap%
\pgfsetroundjoin%
\pgfsetlinewidth{1.505625pt}%
\definecolor{currentstroke}{rgb}{0.505882,0.447059,0.701961}%
\pgfsetstrokecolor{currentstroke}%
\pgfsetdash{}{0pt}%
\pgfpathmoveto{\pgfqpoint{0.938001in}{1.848014in}}%
\pgfpathlineto{\pgfqpoint{0.964135in}{1.887173in}}%
\pgfpathlineto{\pgfqpoint{0.990268in}{1.963865in}}%
\pgfpathlineto{\pgfqpoint{1.016401in}{2.100659in}}%
\pgfpathlineto{\pgfqpoint{1.042535in}{2.284777in}}%
\pgfpathlineto{\pgfqpoint{1.068668in}{2.471666in}}%
\pgfpathlineto{\pgfqpoint{1.094801in}{2.613499in}}%
\pgfpathlineto{\pgfqpoint{1.120935in}{2.691464in}}%
\pgfpathlineto{\pgfqpoint{1.147068in}{2.720469in}}%
\pgfpathlineto{\pgfqpoint{1.173201in}{2.729155in}}%
\pgfpathlineto{\pgfqpoint{1.199335in}{2.737613in}}%
\pgfpathlineto{\pgfqpoint{1.225468in}{2.749788in}}%
\pgfpathlineto{\pgfqpoint{1.251602in}{2.759411in}}%
\pgfpathlineto{\pgfqpoint{1.277735in}{2.758272in}}%
\pgfpathlineto{\pgfqpoint{1.303868in}{2.737804in}}%
\pgfpathlineto{\pgfqpoint{1.330002in}{2.688670in}}%
\pgfpathlineto{\pgfqpoint{1.356135in}{2.604964in}}%
\pgfpathlineto{\pgfqpoint{1.382268in}{2.488723in}}%
\pgfpathlineto{\pgfqpoint{1.408402in}{2.348875in}}%
\pgfpathlineto{\pgfqpoint{1.434535in}{2.196372in}}%
\pgfpathlineto{\pgfqpoint{1.460668in}{2.037786in}}%
\pgfpathlineto{\pgfqpoint{1.486802in}{1.873633in}}%
\pgfpathlineto{\pgfqpoint{1.512935in}{1.703132in}}%
\pgfpathlineto{\pgfqpoint{1.539068in}{1.529329in}}%
\pgfpathlineto{\pgfqpoint{1.565202in}{1.358930in}}%
\pgfpathlineto{\pgfqpoint{1.591335in}{1.198362in}}%
\pgfpathlineto{\pgfqpoint{1.617468in}{1.052423in}}%
\pgfpathlineto{\pgfqpoint{1.643602in}{0.925447in}}%
\pgfpathlineto{\pgfqpoint{1.669735in}{0.820194in}}%
\pgfpathlineto{\pgfqpoint{1.695869in}{0.740318in}}%
\pgfpathlineto{\pgfqpoint{1.722002in}{0.692294in}}%
\pgfpathlineto{\pgfqpoint{1.748135in}{0.681944in}}%
\pgfpathlineto{\pgfqpoint{1.774269in}{0.707189in}}%
\pgfpathlineto{\pgfqpoint{1.800402in}{0.755123in}}%
\pgfpathlineto{\pgfqpoint{1.826535in}{0.807765in}}%
\pgfpathlineto{\pgfqpoint{1.852669in}{0.849046in}}%
\pgfpathlineto{\pgfqpoint{1.878802in}{0.870351in}}%
\pgfpathlineto{\pgfqpoint{1.904935in}{0.876511in}}%
\pgfpathlineto{\pgfqpoint{1.931069in}{0.886417in}}%
\pgfpathlineto{\pgfqpoint{1.957202in}{0.920951in}}%
\pgfpathlineto{\pgfqpoint{1.983335in}{0.992813in}}%
\pgfpathlineto{\pgfqpoint{2.009469in}{1.099796in}}%
\pgfpathlineto{\pgfqpoint{2.035602in}{1.226671in}}%
\pgfpathlineto{\pgfqpoint{2.061735in}{1.353648in}}%
\pgfpathlineto{\pgfqpoint{2.087869in}{1.466534in}}%
\pgfpathlineto{\pgfqpoint{2.114002in}{1.560280in}}%
\pgfpathlineto{\pgfqpoint{2.140136in}{1.636343in}}%
\pgfpathlineto{\pgfqpoint{2.166269in}{1.698248in}}%
\pgfpathlineto{\pgfqpoint{2.192402in}{1.748317in}}%
\pgfpathlineto{\pgfqpoint{2.218536in}{1.787080in}}%
\pgfpathlineto{\pgfqpoint{2.244669in}{1.813467in}}%
\pgfpathlineto{\pgfqpoint{2.270802in}{1.827636in}}%
\pgfpathlineto{\pgfqpoint{2.296936in}{1.837170in}}%
\pgfpathlineto{\pgfqpoint{2.323069in}{1.861296in}}%
\pgfpathlineto{\pgfqpoint{2.349202in}{1.928443in}}%
\pgfpathlineto{\pgfqpoint{2.375336in}{2.060477in}}%
\pgfpathlineto{\pgfqpoint{2.401469in}{2.248211in}}%
\pgfpathlineto{\pgfqpoint{2.427602in}{2.445947in}}%
\pgfpathlineto{\pgfqpoint{2.453736in}{2.600793in}}%
\pgfpathlineto{\pgfqpoint{2.479869in}{2.688672in}}%
\pgfpathlineto{\pgfqpoint{2.506002in}{2.722733in}}%
\pgfpathlineto{\pgfqpoint{2.532136in}{2.731696in}}%
\pgfpathlineto{\pgfqpoint{2.558269in}{2.737613in}}%
\pgfpathlineto{\pgfqpoint{2.584403in}{2.749858in}}%
\pgfpathlineto{\pgfqpoint{2.610536in}{2.765426in}}%
\pgfpathlineto{\pgfqpoint{2.636669in}{2.771576in}}%
\pgfpathlineto{\pgfqpoint{2.662803in}{2.752409in}}%
\pgfpathlineto{\pgfqpoint{2.688936in}{2.696724in}}%
\pgfpathlineto{\pgfqpoint{2.715069in}{2.605924in}}%
\pgfpathlineto{\pgfqpoint{2.741203in}{2.490240in}}%
\pgfpathlineto{\pgfqpoint{2.767336in}{2.358659in}}%
\pgfpathlineto{\pgfqpoint{2.793469in}{2.218516in}}%
\pgfusepath{stroke}%
\end{pgfscope}%
\begin{pgfscope}%
\pgfsetrectcap%
\pgfsetmiterjoin%
\pgfsetlinewidth{1.254687pt}%
\definecolor{currentstroke}{rgb}{1.000000,1.000000,1.000000}%
\pgfsetstrokecolor{currentstroke}%
\pgfsetdash{}{0pt}%
\pgfpathmoveto{\pgfqpoint{0.828241in}{0.574768in}}%
\pgfpathlineto{\pgfqpoint{0.828241in}{2.932627in}}%
\pgfusepath{stroke}%
\end{pgfscope}%
\begin{pgfscope}%
\pgfsetrectcap%
\pgfsetmiterjoin%
\pgfsetlinewidth{1.254687pt}%
\definecolor{currentstroke}{rgb}{1.000000,1.000000,1.000000}%
\pgfsetstrokecolor{currentstroke}%
\pgfsetdash{}{0pt}%
\pgfpathmoveto{\pgfqpoint{3.242963in}{0.574768in}}%
\pgfpathlineto{\pgfqpoint{3.242963in}{2.932627in}}%
\pgfusepath{stroke}%
\end{pgfscope}%
\begin{pgfscope}%
\pgfsetrectcap%
\pgfsetmiterjoin%
\pgfsetlinewidth{1.254687pt}%
\definecolor{currentstroke}{rgb}{1.000000,1.000000,1.000000}%
\pgfsetstrokecolor{currentstroke}%
\pgfsetdash{}{0pt}%
\pgfpathmoveto{\pgfqpoint{0.828241in}{0.574768in}}%
\pgfpathlineto{\pgfqpoint{3.242963in}{0.574768in}}%
\pgfusepath{stroke}%
\end{pgfscope}%
\begin{pgfscope}%
\pgfsetrectcap%
\pgfsetmiterjoin%
\pgfsetlinewidth{1.254687pt}%
\definecolor{currentstroke}{rgb}{1.000000,1.000000,1.000000}%
\pgfsetstrokecolor{currentstroke}%
\pgfsetdash{}{0pt}%
\pgfpathmoveto{\pgfqpoint{0.828241in}{2.932627in}}%
\pgfpathlineto{\pgfqpoint{3.242963in}{2.932627in}}%
\pgfusepath{stroke}%
\end{pgfscope}%
\begin{pgfscope}%
\pgfsetbuttcap%
\pgfsetmiterjoin%
\definecolor{currentfill}{rgb}{0.917647,0.917647,0.949020}%
\pgfsetfillcolor{currentfill}%
\pgfsetlinewidth{0.000000pt}%
\definecolor{currentstroke}{rgb}{0.000000,0.000000,0.000000}%
\pgfsetstrokecolor{currentstroke}%
\pgfsetstrokeopacity{0.000000}%
\pgfsetdash{}{0pt}%
\pgfpathmoveto{\pgfqpoint{3.825278in}{0.574768in}}%
\pgfpathlineto{\pgfqpoint{6.240000in}{0.574768in}}%
\pgfpathlineto{\pgfqpoint{6.240000in}{2.932627in}}%
\pgfpathlineto{\pgfqpoint{3.825278in}{2.932627in}}%
\pgfpathclose%
\pgfusepath{fill}%
\end{pgfscope}%
\begin{pgfscope}%
\pgfpathrectangle{\pgfqpoint{3.825278in}{0.574768in}}{\pgfqpoint{2.414722in}{2.357859in}}%
\pgfusepath{clip}%
\pgfsetroundcap%
\pgfsetroundjoin%
\pgfsetlinewidth{1.003750pt}%
\definecolor{currentstroke}{rgb}{1.000000,1.000000,1.000000}%
\pgfsetstrokecolor{currentstroke}%
\pgfsetdash{}{0pt}%
\pgfpathmoveto{\pgfqpoint{3.935038in}{0.574768in}}%
\pgfpathlineto{\pgfqpoint{3.935038in}{2.932627in}}%
\pgfusepath{stroke}%
\end{pgfscope}%
\begin{pgfscope}%
\definecolor{textcolor}{rgb}{0.150000,0.150000,0.150000}%
\pgfsetstrokecolor{textcolor}%
\pgfsetfillcolor{textcolor}%
\pgftext[x=3.935038in,y=0.442824in,,top]{\color{textcolor}\sffamily\fontsize{11.000000}{13.200000}\selectfont \(\displaystyle 0.0\)}%
\end{pgfscope}%
\begin{pgfscope}%
\pgfpathrectangle{\pgfqpoint{3.825278in}{0.574768in}}{\pgfqpoint{2.414722in}{2.357859in}}%
\pgfusepath{clip}%
\pgfsetroundcap%
\pgfsetroundjoin%
\pgfsetlinewidth{1.003750pt}%
\definecolor{currentstroke}{rgb}{1.000000,1.000000,1.000000}%
\pgfsetstrokecolor{currentstroke}%
\pgfsetdash{}{0pt}%
\pgfpathmoveto{\pgfqpoint{4.637503in}{0.574768in}}%
\pgfpathlineto{\pgfqpoint{4.637503in}{2.932627in}}%
\pgfusepath{stroke}%
\end{pgfscope}%
\begin{pgfscope}%
\definecolor{textcolor}{rgb}{0.150000,0.150000,0.150000}%
\pgfsetstrokecolor{textcolor}%
\pgfsetfillcolor{textcolor}%
\pgftext[x=4.637503in,y=0.442824in,,top]{\color{textcolor}\sffamily\fontsize{11.000000}{13.200000}\selectfont \(\displaystyle 0.5\)}%
\end{pgfscope}%
\begin{pgfscope}%
\pgfpathrectangle{\pgfqpoint{3.825278in}{0.574768in}}{\pgfqpoint{2.414722in}{2.357859in}}%
\pgfusepath{clip}%
\pgfsetroundcap%
\pgfsetroundjoin%
\pgfsetlinewidth{1.003750pt}%
\definecolor{currentstroke}{rgb}{1.000000,1.000000,1.000000}%
\pgfsetstrokecolor{currentstroke}%
\pgfsetdash{}{0pt}%
\pgfpathmoveto{\pgfqpoint{5.339967in}{0.574768in}}%
\pgfpathlineto{\pgfqpoint{5.339967in}{2.932627in}}%
\pgfusepath{stroke}%
\end{pgfscope}%
\begin{pgfscope}%
\definecolor{textcolor}{rgb}{0.150000,0.150000,0.150000}%
\pgfsetstrokecolor{textcolor}%
\pgfsetfillcolor{textcolor}%
\pgftext[x=5.339967in,y=0.442824in,,top]{\color{textcolor}\sffamily\fontsize{11.000000}{13.200000}\selectfont \(\displaystyle 1.0\)}%
\end{pgfscope}%
\begin{pgfscope}%
\pgfpathrectangle{\pgfqpoint{3.825278in}{0.574768in}}{\pgfqpoint{2.414722in}{2.357859in}}%
\pgfusepath{clip}%
\pgfsetroundcap%
\pgfsetroundjoin%
\pgfsetlinewidth{1.003750pt}%
\definecolor{currentstroke}{rgb}{1.000000,1.000000,1.000000}%
\pgfsetstrokecolor{currentstroke}%
\pgfsetdash{}{0pt}%
\pgfpathmoveto{\pgfqpoint{6.042432in}{0.574768in}}%
\pgfpathlineto{\pgfqpoint{6.042432in}{2.932627in}}%
\pgfusepath{stroke}%
\end{pgfscope}%
\begin{pgfscope}%
\definecolor{textcolor}{rgb}{0.150000,0.150000,0.150000}%
\pgfsetstrokecolor{textcolor}%
\pgfsetfillcolor{textcolor}%
\pgftext[x=6.042432in,y=0.442824in,,top]{\color{textcolor}\sffamily\fontsize{11.000000}{13.200000}\selectfont \(\displaystyle 1.5\)}%
\end{pgfscope}%
\begin{pgfscope}%
\definecolor{textcolor}{rgb}{0.150000,0.150000,0.150000}%
\pgfsetstrokecolor{textcolor}%
\pgfsetfillcolor{textcolor}%
\pgftext[x=5.032639in,y=0.252083in,,top]{\color{textcolor}\sffamily\fontsize{11.000000}{13.200000}\selectfont Time [s]}%
\end{pgfscope}%
\begin{pgfscope}%
\pgfpathrectangle{\pgfqpoint{3.825278in}{0.574768in}}{\pgfqpoint{2.414722in}{2.357859in}}%
\pgfusepath{clip}%
\pgfsetroundcap%
\pgfsetroundjoin%
\pgfsetlinewidth{1.003750pt}%
\definecolor{currentstroke}{rgb}{1.000000,1.000000,1.000000}%
\pgfsetstrokecolor{currentstroke}%
\pgfsetdash{}{0pt}%
\pgfpathmoveto{\pgfqpoint{3.825278in}{0.884193in}}%
\pgfpathlineto{\pgfqpoint{6.240000in}{0.884193in}}%
\pgfusepath{stroke}%
\end{pgfscope}%
\begin{pgfscope}%
\definecolor{textcolor}{rgb}{0.150000,0.150000,0.150000}%
\pgfsetstrokecolor{textcolor}%
\pgfsetfillcolor{textcolor}%
\pgftext[x=3.422963in,y=0.831386in,left,base]{\color{textcolor}\sffamily\fontsize{11.000000}{13.200000}\selectfont \(\displaystyle -20\)}%
\end{pgfscope}%
\begin{pgfscope}%
\pgfpathrectangle{\pgfqpoint{3.825278in}{0.574768in}}{\pgfqpoint{2.414722in}{2.357859in}}%
\pgfusepath{clip}%
\pgfsetroundcap%
\pgfsetroundjoin%
\pgfsetlinewidth{1.003750pt}%
\definecolor{currentstroke}{rgb}{1.000000,1.000000,1.000000}%
\pgfsetstrokecolor{currentstroke}%
\pgfsetdash{}{0pt}%
\pgfpathmoveto{\pgfqpoint{3.825278in}{1.351024in}}%
\pgfpathlineto{\pgfqpoint{6.240000in}{1.351024in}}%
\pgfusepath{stroke}%
\end{pgfscope}%
\begin{pgfscope}%
\definecolor{textcolor}{rgb}{0.150000,0.150000,0.150000}%
\pgfsetstrokecolor{textcolor}%
\pgfsetfillcolor{textcolor}%
\pgftext[x=3.422963in,y=1.298218in,left,base]{\color{textcolor}\sffamily\fontsize{11.000000}{13.200000}\selectfont \(\displaystyle -15\)}%
\end{pgfscope}%
\begin{pgfscope}%
\pgfpathrectangle{\pgfqpoint{3.825278in}{0.574768in}}{\pgfqpoint{2.414722in}{2.357859in}}%
\pgfusepath{clip}%
\pgfsetroundcap%
\pgfsetroundjoin%
\pgfsetlinewidth{1.003750pt}%
\definecolor{currentstroke}{rgb}{1.000000,1.000000,1.000000}%
\pgfsetstrokecolor{currentstroke}%
\pgfsetdash{}{0pt}%
\pgfpathmoveto{\pgfqpoint{3.825278in}{1.817856in}}%
\pgfpathlineto{\pgfqpoint{6.240000in}{1.817856in}}%
\pgfusepath{stroke}%
\end{pgfscope}%
\begin{pgfscope}%
\definecolor{textcolor}{rgb}{0.150000,0.150000,0.150000}%
\pgfsetstrokecolor{textcolor}%
\pgfsetfillcolor{textcolor}%
\pgftext[x=3.422963in,y=1.765049in,left,base]{\color{textcolor}\sffamily\fontsize{11.000000}{13.200000}\selectfont \(\displaystyle -10\)}%
\end{pgfscope}%
\begin{pgfscope}%
\pgfpathrectangle{\pgfqpoint{3.825278in}{0.574768in}}{\pgfqpoint{2.414722in}{2.357859in}}%
\pgfusepath{clip}%
\pgfsetroundcap%
\pgfsetroundjoin%
\pgfsetlinewidth{1.003750pt}%
\definecolor{currentstroke}{rgb}{1.000000,1.000000,1.000000}%
\pgfsetstrokecolor{currentstroke}%
\pgfsetdash{}{0pt}%
\pgfpathmoveto{\pgfqpoint{3.825278in}{2.284688in}}%
\pgfpathlineto{\pgfqpoint{6.240000in}{2.284688in}}%
\pgfusepath{stroke}%
\end{pgfscope}%
\begin{pgfscope}%
\definecolor{textcolor}{rgb}{0.150000,0.150000,0.150000}%
\pgfsetstrokecolor{textcolor}%
\pgfsetfillcolor{textcolor}%
\pgftext[x=3.499005in,y=2.231881in,left,base]{\color{textcolor}\sffamily\fontsize{11.000000}{13.200000}\selectfont \(\displaystyle -5\)}%
\end{pgfscope}%
\begin{pgfscope}%
\pgfpathrectangle{\pgfqpoint{3.825278in}{0.574768in}}{\pgfqpoint{2.414722in}{2.357859in}}%
\pgfusepath{clip}%
\pgfsetroundcap%
\pgfsetroundjoin%
\pgfsetlinewidth{1.003750pt}%
\definecolor{currentstroke}{rgb}{1.000000,1.000000,1.000000}%
\pgfsetstrokecolor{currentstroke}%
\pgfsetdash{}{0pt}%
\pgfpathmoveto{\pgfqpoint{3.825278in}{2.751519in}}%
\pgfpathlineto{\pgfqpoint{6.240000in}{2.751519in}}%
\pgfusepath{stroke}%
\end{pgfscope}%
\begin{pgfscope}%
\definecolor{textcolor}{rgb}{0.150000,0.150000,0.150000}%
\pgfsetstrokecolor{textcolor}%
\pgfsetfillcolor{textcolor}%
\pgftext[x=3.617292in,y=2.698713in,left,base]{\color{textcolor}\sffamily\fontsize{11.000000}{13.200000}\selectfont \(\displaystyle 0\)}%
\end{pgfscope}%
\begin{pgfscope}%
\pgfpathrectangle{\pgfqpoint{3.825278in}{0.574768in}}{\pgfqpoint{2.414722in}{2.357859in}}%
\pgfusepath{clip}%
\pgfsetroundcap%
\pgfsetroundjoin%
\pgfsetlinewidth{1.505625pt}%
\definecolor{currentstroke}{rgb}{0.298039,0.447059,0.690196}%
\pgfsetstrokecolor{currentstroke}%
\pgfsetdash{}{0pt}%
\pgfpathmoveto{\pgfqpoint{3.935038in}{2.076589in}}%
\pgfpathlineto{\pgfqpoint{3.952172in}{2.080485in}}%
\pgfpathlineto{\pgfqpoint{3.969305in}{2.093272in}}%
\pgfpathlineto{\pgfqpoint{3.986438in}{2.125760in}}%
\pgfpathlineto{\pgfqpoint{4.003571in}{2.185376in}}%
\pgfpathlineto{\pgfqpoint{4.020705in}{2.273360in}}%
\pgfpathlineto{\pgfqpoint{4.037838in}{2.383437in}}%
\pgfpathlineto{\pgfqpoint{4.054971in}{2.502446in}}%
\pgfpathlineto{\pgfqpoint{4.072105in}{2.614148in}}%
\pgfpathlineto{\pgfqpoint{4.089238in}{2.703788in}}%
\pgfpathlineto{\pgfqpoint{4.106371in}{2.761291in}}%
\pgfpathlineto{\pgfqpoint{4.123504in}{2.783072in}}%
\pgfpathlineto{\pgfqpoint{4.140638in}{2.774959in}}%
\pgfpathlineto{\pgfqpoint{4.157771in}{2.751519in}}%
\pgfpathlineto{\pgfqpoint{4.174904in}{2.726551in}}%
\pgfpathlineto{\pgfqpoint{4.192038in}{2.704275in}}%
\pgfpathlineto{\pgfqpoint{4.209171in}{2.680379in}}%
\pgfpathlineto{\pgfqpoint{4.226304in}{2.648804in}}%
\pgfpathlineto{\pgfqpoint{4.243437in}{2.603806in}}%
\pgfpathlineto{\pgfqpoint{4.260571in}{2.541990in}}%
\pgfpathlineto{\pgfqpoint{4.277704in}{2.463009in}}%
\pgfpathlineto{\pgfqpoint{4.294837in}{2.371394in}}%
\pgfpathlineto{\pgfqpoint{4.311971in}{2.276680in}}%
\pgfpathlineto{\pgfqpoint{4.329104in}{2.185947in}}%
\pgfpathlineto{\pgfqpoint{4.346237in}{2.102211in}}%
\pgfpathlineto{\pgfqpoint{4.363370in}{2.028814in}}%
\pgfpathlineto{\pgfqpoint{4.380504in}{1.966517in}}%
\pgfpathlineto{\pgfqpoint{4.397637in}{1.909380in}}%
\pgfpathlineto{\pgfqpoint{4.414770in}{1.848893in}}%
\pgfpathlineto{\pgfqpoint{4.431903in}{1.779820in}}%
\pgfpathlineto{\pgfqpoint{4.449037in}{1.702455in}}%
\pgfpathlineto{\pgfqpoint{4.466170in}{1.621276in}}%
\pgfpathlineto{\pgfqpoint{4.483303in}{1.540449in}}%
\pgfpathlineto{\pgfqpoint{4.500437in}{1.461117in}}%
\pgfpathlineto{\pgfqpoint{4.517570in}{1.383193in}}%
\pgfpathlineto{\pgfqpoint{4.534703in}{1.307168in}}%
\pgfpathlineto{\pgfqpoint{4.551836in}{1.234773in}}%
\pgfpathlineto{\pgfqpoint{4.568970in}{1.168137in}}%
\pgfpathlineto{\pgfqpoint{4.586103in}{1.110071in}}%
\pgfpathlineto{\pgfqpoint{4.603236in}{1.063278in}}%
\pgfpathlineto{\pgfqpoint{4.620370in}{1.030277in}}%
\pgfpathlineto{\pgfqpoint{4.637503in}{1.015437in}}%
\pgfpathlineto{\pgfqpoint{4.654636in}{1.025158in}}%
\pgfpathlineto{\pgfqpoint{4.671769in}{1.062650in}}%
\pgfpathlineto{\pgfqpoint{4.688903in}{1.124628in}}%
\pgfpathlineto{\pgfqpoint{4.706036in}{1.203159in}}%
\pgfpathlineto{\pgfqpoint{4.723169in}{1.287600in}}%
\pgfpathlineto{\pgfqpoint{4.740303in}{1.366700in}}%
\pgfpathlineto{\pgfqpoint{4.757436in}{1.431115in}}%
\pgfpathlineto{\pgfqpoint{4.774569in}{1.474838in}}%
\pgfpathlineto{\pgfqpoint{4.791702in}{1.501526in}}%
\pgfpathlineto{\pgfqpoint{4.808836in}{1.525817in}}%
\pgfpathlineto{\pgfqpoint{4.825969in}{1.561283in}}%
\pgfpathlineto{\pgfqpoint{4.843102in}{1.612199in}}%
\pgfpathlineto{\pgfqpoint{4.860235in}{1.675711in}}%
\pgfpathlineto{\pgfqpoint{4.877369in}{1.744863in}}%
\pgfpathlineto{\pgfqpoint{4.894502in}{1.811629in}}%
\pgfpathlineto{\pgfqpoint{4.911635in}{1.871237in}}%
\pgfpathlineto{\pgfqpoint{4.928769in}{1.924051in}}%
\pgfpathlineto{\pgfqpoint{4.945902in}{1.972415in}}%
\pgfpathlineto{\pgfqpoint{4.963035in}{2.016833in}}%
\pgfpathlineto{\pgfqpoint{4.980168in}{2.055239in}}%
\pgfpathlineto{\pgfqpoint{4.997302in}{2.085140in}}%
\pgfpathlineto{\pgfqpoint{5.014435in}{2.106300in}}%
\pgfpathlineto{\pgfqpoint{5.031568in}{2.120487in}}%
\pgfpathlineto{\pgfqpoint{5.048702in}{2.129251in}}%
\pgfpathlineto{\pgfqpoint{5.065835in}{2.134034in}}%
\pgfpathlineto{\pgfqpoint{5.082968in}{2.136183in}}%
\pgfpathlineto{\pgfqpoint{5.100101in}{2.136182in}}%
\pgfpathlineto{\pgfqpoint{5.117235in}{2.134741in}}%
\pgfpathlineto{\pgfqpoint{5.134368in}{2.135269in}}%
\pgfpathlineto{\pgfqpoint{5.151501in}{2.143325in}}%
\pgfpathlineto{\pgfqpoint{5.168635in}{2.165401in}}%
\pgfpathlineto{\pgfqpoint{5.185768in}{2.208578in}}%
\pgfpathlineto{\pgfqpoint{5.202901in}{2.279344in}}%
\pgfpathlineto{\pgfqpoint{5.220034in}{2.378369in}}%
\pgfpathlineto{\pgfqpoint{5.237168in}{2.496352in}}%
\pgfpathlineto{\pgfqpoint{5.254301in}{2.615833in}}%
\pgfpathlineto{\pgfqpoint{5.271434in}{2.718975in}}%
\pgfpathlineto{\pgfqpoint{5.288568in}{2.792699in}}%
\pgfpathlineto{\pgfqpoint{5.305701in}{2.825452in}}%
\pgfpathlineto{\pgfqpoint{5.322834in}{2.809675in}}%
\pgfpathlineto{\pgfqpoint{5.339967in}{2.751519in}}%
\pgfpathlineto{\pgfqpoint{5.357101in}{2.673683in}}%
\pgfpathlineto{\pgfqpoint{5.374234in}{2.601788in}}%
\pgfpathlineto{\pgfqpoint{5.391367in}{2.547077in}}%
\pgfpathlineto{\pgfqpoint{5.408500in}{2.501470in}}%
\pgfpathlineto{\pgfqpoint{5.425634in}{2.449583in}}%
\pgfpathlineto{\pgfqpoint{5.442767in}{2.380792in}}%
\pgfpathlineto{\pgfqpoint{5.459900in}{2.293968in}}%
\pgfpathlineto{\pgfqpoint{5.477034in}{2.194756in}}%
\pgfpathlineto{\pgfqpoint{5.494167in}{2.090785in}}%
\pgfpathlineto{\pgfqpoint{5.511300in}{1.986779in}}%
\pgfpathlineto{\pgfqpoint{5.528433in}{1.885874in}}%
\pgfusepath{stroke}%
\end{pgfscope}%
\begin{pgfscope}%
\pgfpathrectangle{\pgfqpoint{3.825278in}{0.574768in}}{\pgfqpoint{2.414722in}{2.357859in}}%
\pgfusepath{clip}%
\pgfsetroundcap%
\pgfsetroundjoin%
\pgfsetlinewidth{1.505625pt}%
\definecolor{currentstroke}{rgb}{0.866667,0.517647,0.321569}%
\pgfsetstrokecolor{currentstroke}%
\pgfsetdash{}{0pt}%
\pgfpathmoveto{\pgfqpoint{3.935038in}{1.859189in}}%
\pgfpathlineto{\pgfqpoint{3.952172in}{1.857411in}}%
\pgfpathlineto{\pgfqpoint{3.969305in}{1.868257in}}%
\pgfpathlineto{\pgfqpoint{3.986438in}{1.902656in}}%
\pgfpathlineto{\pgfqpoint{4.003571in}{1.967432in}}%
\pgfpathlineto{\pgfqpoint{4.020705in}{2.061718in}}%
\pgfpathlineto{\pgfqpoint{4.037838in}{2.176694in}}%
\pgfpathlineto{\pgfqpoint{4.054971in}{2.301116in}}%
\pgfpathlineto{\pgfqpoint{4.072105in}{2.425389in}}%
\pgfpathlineto{\pgfqpoint{4.089238in}{2.536972in}}%
\pgfpathlineto{\pgfqpoint{4.106371in}{2.625042in}}%
\pgfpathlineto{\pgfqpoint{4.123504in}{2.688136in}}%
\pgfpathlineto{\pgfqpoint{4.140638in}{2.730028in}}%
\pgfpathlineto{\pgfqpoint{4.157771in}{2.751519in}}%
\pgfpathlineto{\pgfqpoint{4.174904in}{2.755517in}}%
\pgfpathlineto{\pgfqpoint{4.192038in}{2.752088in}}%
\pgfpathlineto{\pgfqpoint{4.209171in}{2.749837in}}%
\pgfpathlineto{\pgfqpoint{4.226304in}{2.745358in}}%
\pgfpathlineto{\pgfqpoint{4.243437in}{2.727829in}}%
\pgfpathlineto{\pgfqpoint{4.260571in}{2.686389in}}%
\pgfpathlineto{\pgfqpoint{4.277704in}{2.615482in}}%
\pgfpathlineto{\pgfqpoint{4.294837in}{2.521330in}}%
\pgfpathlineto{\pgfqpoint{4.311971in}{2.420421in}}%
\pgfpathlineto{\pgfqpoint{4.329104in}{2.325998in}}%
\pgfpathlineto{\pgfqpoint{4.346237in}{2.238367in}}%
\pgfpathlineto{\pgfqpoint{4.363370in}{2.150806in}}%
\pgfpathlineto{\pgfqpoint{4.380504in}{2.057443in}}%
\pgfpathlineto{\pgfqpoint{4.397637in}{1.955893in}}%
\pgfpathlineto{\pgfqpoint{4.414770in}{1.847525in}}%
\pgfpathlineto{\pgfqpoint{4.431903in}{1.735625in}}%
\pgfpathlineto{\pgfqpoint{4.449037in}{1.622956in}}%
\pgfpathlineto{\pgfqpoint{4.466170in}{1.510213in}}%
\pgfpathlineto{\pgfqpoint{4.483303in}{1.396087in}}%
\pgfpathlineto{\pgfqpoint{4.500437in}{1.282070in}}%
\pgfpathlineto{\pgfqpoint{4.517570in}{1.173223in}}%
\pgfpathlineto{\pgfqpoint{4.534703in}{1.073113in}}%
\pgfpathlineto{\pgfqpoint{4.551836in}{0.982586in}}%
\pgfpathlineto{\pgfqpoint{4.568970in}{0.901177in}}%
\pgfpathlineto{\pgfqpoint{4.586103in}{0.829964in}}%
\pgfpathlineto{\pgfqpoint{4.603236in}{0.770749in}}%
\pgfpathlineto{\pgfqpoint{4.620370in}{0.724097in}}%
\pgfpathlineto{\pgfqpoint{4.637503in}{0.692157in}}%
\pgfpathlineto{\pgfqpoint{4.654636in}{0.681944in}}%
\pgfpathlineto{\pgfqpoint{4.671769in}{0.699470in}}%
\pgfpathlineto{\pgfqpoint{4.688903in}{0.738327in}}%
\pgfpathlineto{\pgfqpoint{4.706036in}{0.780881in}}%
\pgfpathlineto{\pgfqpoint{4.723169in}{0.814150in}}%
\pgfpathlineto{\pgfqpoint{4.740303in}{0.838815in}}%
\pgfpathlineto{\pgfqpoint{4.757436in}{0.863786in}}%
\pgfpathlineto{\pgfqpoint{4.774569in}{0.896919in}}%
\pgfpathlineto{\pgfqpoint{4.791702in}{0.941732in}}%
\pgfpathlineto{\pgfqpoint{4.808836in}{1.000436in}}%
\pgfpathlineto{\pgfqpoint{4.825969in}{1.077246in}}%
\pgfpathlineto{\pgfqpoint{4.843102in}{1.176133in}}%
\pgfpathlineto{\pgfqpoint{4.860235in}{1.296079in}}%
\pgfpathlineto{\pgfqpoint{4.877369in}{1.428280in}}%
\pgfpathlineto{\pgfqpoint{4.894502in}{1.556165in}}%
\pgfpathlineto{\pgfqpoint{4.911635in}{1.664461in}}%
\pgfpathlineto{\pgfqpoint{4.928769in}{1.751943in}}%
\pgfpathlineto{\pgfqpoint{4.945902in}{1.829061in}}%
\pgfpathlineto{\pgfqpoint{4.963035in}{1.900970in}}%
\pgfpathlineto{\pgfqpoint{4.980168in}{1.961885in}}%
\pgfpathlineto{\pgfqpoint{4.997302in}{2.004566in}}%
\pgfpathlineto{\pgfqpoint{5.014435in}{2.028545in}}%
\pgfpathlineto{\pgfqpoint{5.031568in}{2.037806in}}%
\pgfpathlineto{\pgfqpoint{5.048702in}{2.036110in}}%
\pgfpathlineto{\pgfqpoint{5.065835in}{2.025233in}}%
\pgfpathlineto{\pgfqpoint{5.082968in}{2.006952in}}%
\pgfpathlineto{\pgfqpoint{5.100101in}{1.986174in}}%
\pgfpathlineto{\pgfqpoint{5.117235in}{1.970467in}}%
\pgfpathlineto{\pgfqpoint{5.134368in}{1.967595in}}%
\pgfpathlineto{\pgfqpoint{5.151501in}{1.984889in}}%
\pgfpathlineto{\pgfqpoint{5.168635in}{2.027530in}}%
\pgfpathlineto{\pgfqpoint{5.185768in}{2.096229in}}%
\pgfpathlineto{\pgfqpoint{5.202901in}{2.187454in}}%
\pgfpathlineto{\pgfqpoint{5.220034in}{2.295476in}}%
\pgfpathlineto{\pgfqpoint{5.237168in}{2.413374in}}%
\pgfpathlineto{\pgfqpoint{5.254301in}{2.529256in}}%
\pgfpathlineto{\pgfqpoint{5.271434in}{2.628574in}}%
\pgfpathlineto{\pgfqpoint{5.288568in}{2.700918in}}%
\pgfpathlineto{\pgfqpoint{5.305701in}{2.743321in}}%
\pgfpathlineto{\pgfqpoint{5.322834in}{2.758278in}}%
\pgfpathlineto{\pgfqpoint{5.339967in}{2.751519in}}%
\pgfpathlineto{\pgfqpoint{5.357101in}{2.734369in}}%
\pgfpathlineto{\pgfqpoint{5.374234in}{2.720356in}}%
\pgfpathlineto{\pgfqpoint{5.391367in}{2.713280in}}%
\pgfpathlineto{\pgfqpoint{5.408500in}{2.703798in}}%
\pgfpathlineto{\pgfqpoint{5.425634in}{2.678867in}}%
\pgfpathlineto{\pgfqpoint{5.442767in}{2.630985in}}%
\pgfpathlineto{\pgfqpoint{5.459900in}{2.560366in}}%
\pgfpathlineto{\pgfqpoint{5.477034in}{2.473014in}}%
\pgfpathlineto{\pgfqpoint{5.494167in}{2.378766in}}%
\pgfpathlineto{\pgfqpoint{5.511300in}{2.285208in}}%
\pgfpathlineto{\pgfqpoint{5.528433in}{2.193811in}}%
\pgfusepath{stroke}%
\end{pgfscope}%
\begin{pgfscope}%
\pgfpathrectangle{\pgfqpoint{3.825278in}{0.574768in}}{\pgfqpoint{2.414722in}{2.357859in}}%
\pgfusepath{clip}%
\pgfsetroundcap%
\pgfsetroundjoin%
\pgfsetlinewidth{1.505625pt}%
\definecolor{currentstroke}{rgb}{0.333333,0.658824,0.407843}%
\pgfsetstrokecolor{currentstroke}%
\pgfsetdash{}{0pt}%
\pgfpathmoveto{\pgfqpoint{3.935038in}{2.447114in}}%
\pgfpathlineto{\pgfqpoint{3.959686in}{2.517043in}}%
\pgfpathlineto{\pgfqpoint{3.984334in}{2.586475in}}%
\pgfpathlineto{\pgfqpoint{4.008982in}{2.650422in}}%
\pgfpathlineto{\pgfqpoint{4.033630in}{2.700845in}}%
\pgfpathlineto{\pgfqpoint{4.058278in}{2.733793in}}%
\pgfpathlineto{\pgfqpoint{4.082926in}{2.751519in}}%
\pgfpathlineto{\pgfqpoint{4.107573in}{2.757849in}}%
\pgfpathlineto{\pgfqpoint{4.132221in}{2.753249in}}%
\pgfpathlineto{\pgfqpoint{4.156869in}{2.735284in}}%
\pgfpathlineto{\pgfqpoint{4.181517in}{2.703230in}}%
\pgfpathlineto{\pgfqpoint{4.206165in}{2.658845in}}%
\pgfpathlineto{\pgfqpoint{4.230813in}{2.603344in}}%
\pgfpathlineto{\pgfqpoint{4.255461in}{2.537377in}}%
\pgfpathlineto{\pgfqpoint{4.280109in}{2.462450in}}%
\pgfpathlineto{\pgfqpoint{4.304757in}{2.381233in}}%
\pgfpathlineto{\pgfqpoint{4.329404in}{2.298722in}}%
\pgfpathlineto{\pgfqpoint{4.354052in}{2.220600in}}%
\pgfpathlineto{\pgfqpoint{4.378700in}{2.148395in}}%
\pgfpathlineto{\pgfqpoint{4.403348in}{2.080080in}}%
\pgfpathlineto{\pgfqpoint{4.427996in}{2.012454in}}%
\pgfpathlineto{\pgfqpoint{4.452644in}{1.943070in}}%
\pgfpathlineto{\pgfqpoint{4.477292in}{1.873951in}}%
\pgfpathlineto{\pgfqpoint{4.501940in}{1.811469in}}%
\pgfpathlineto{\pgfqpoint{4.526587in}{1.760993in}}%
\pgfpathlineto{\pgfqpoint{4.551235in}{1.724786in}}%
\pgfpathlineto{\pgfqpoint{4.575883in}{1.704124in}}%
\pgfpathlineto{\pgfqpoint{4.600531in}{1.700404in}}%
\pgfpathlineto{\pgfqpoint{4.625179in}{1.713905in}}%
\pgfpathlineto{\pgfqpoint{4.649827in}{1.741578in}}%
\pgfpathlineto{\pgfqpoint{4.674475in}{1.777123in}}%
\pgfpathlineto{\pgfqpoint{4.699123in}{1.814381in}}%
\pgfpathlineto{\pgfqpoint{4.723770in}{1.847998in}}%
\pgfpathlineto{\pgfqpoint{4.748418in}{1.872625in}}%
\pgfpathlineto{\pgfqpoint{4.773066in}{1.884861in}}%
\pgfpathlineto{\pgfqpoint{4.797714in}{1.883192in}}%
\pgfpathlineto{\pgfqpoint{4.822362in}{1.865219in}}%
\pgfpathlineto{\pgfqpoint{4.847010in}{1.830106in}}%
\pgfpathlineto{\pgfqpoint{4.871658in}{1.784449in}}%
\pgfpathlineto{\pgfqpoint{4.896306in}{1.743092in}}%
\pgfpathlineto{\pgfqpoint{4.920953in}{1.718529in}}%
\pgfpathlineto{\pgfqpoint{4.945601in}{1.713863in}}%
\pgfpathlineto{\pgfqpoint{4.970249in}{1.726596in}}%
\pgfpathlineto{\pgfqpoint{4.994897in}{1.754070in}}%
\pgfpathlineto{\pgfqpoint{5.019545in}{1.792943in}}%
\pgfpathlineto{\pgfqpoint{5.044193in}{1.838004in}}%
\pgfpathlineto{\pgfqpoint{5.068841in}{1.884665in}}%
\pgfpathlineto{\pgfqpoint{5.093489in}{1.931110in}}%
\pgfpathlineto{\pgfqpoint{5.118136in}{1.977328in}}%
\pgfpathlineto{\pgfqpoint{5.142784in}{2.022596in}}%
\pgfpathlineto{\pgfqpoint{5.167432in}{2.065043in}}%
\pgfpathlineto{\pgfqpoint{5.192080in}{2.102914in}}%
\pgfpathlineto{\pgfqpoint{5.216728in}{2.134761in}}%
\pgfpathlineto{\pgfqpoint{5.241376in}{2.159698in}}%
\pgfpathlineto{\pgfqpoint{5.266024in}{2.179086in}}%
\pgfpathlineto{\pgfqpoint{5.290672in}{2.196543in}}%
\pgfpathlineto{\pgfqpoint{5.315319in}{2.214131in}}%
\pgfpathlineto{\pgfqpoint{5.339967in}{2.230864in}}%
\pgfpathlineto{\pgfqpoint{5.364615in}{2.245530in}}%
\pgfpathlineto{\pgfqpoint{5.389263in}{2.258471in}}%
\pgfpathlineto{\pgfqpoint{5.413911in}{2.272183in}}%
\pgfpathlineto{\pgfqpoint{5.438559in}{2.291483in}}%
\pgfpathlineto{\pgfqpoint{5.463207in}{2.323178in}}%
\pgfpathlineto{\pgfqpoint{5.487855in}{2.374730in}}%
\pgfpathlineto{\pgfqpoint{5.512502in}{2.449076in}}%
\pgfpathlineto{\pgfqpoint{5.537150in}{2.537412in}}%
\pgfpathlineto{\pgfqpoint{5.561798in}{2.620431in}}%
\pgfpathlineto{\pgfqpoint{5.586446in}{2.683336in}}%
\pgfpathlineto{\pgfqpoint{5.611094in}{2.725189in}}%
\pgfpathlineto{\pgfqpoint{5.635742in}{2.751519in}}%
\pgfpathlineto{\pgfqpoint{5.660390in}{2.764160in}}%
\pgfpathlineto{\pgfqpoint{5.685038in}{2.761299in}}%
\pgfpathlineto{\pgfqpoint{5.709686in}{2.743916in}}%
\pgfpathlineto{\pgfqpoint{5.734333in}{2.717201in}}%
\pgfpathlineto{\pgfqpoint{5.758981in}{2.683511in}}%
\pgfpathlineto{\pgfqpoint{5.783629in}{2.640440in}}%
\pgfpathlineto{\pgfqpoint{5.808277in}{2.584854in}}%
\pgfpathlineto{\pgfqpoint{5.832925in}{2.517851in}}%
\pgfpathlineto{\pgfqpoint{5.857573in}{2.444553in}}%
\pgfpathlineto{\pgfqpoint{5.882221in}{2.369667in}}%
\pgfpathlineto{\pgfqpoint{5.906869in}{2.295928in}}%
\pgfpathlineto{\pgfqpoint{5.931516in}{2.223900in}}%
\pgfusepath{stroke}%
\end{pgfscope}%
\begin{pgfscope}%
\pgfpathrectangle{\pgfqpoint{3.825278in}{0.574768in}}{\pgfqpoint{2.414722in}{2.357859in}}%
\pgfusepath{clip}%
\pgfsetroundcap%
\pgfsetroundjoin%
\pgfsetlinewidth{1.505625pt}%
\definecolor{currentstroke}{rgb}{0.768627,0.305882,0.321569}%
\pgfsetstrokecolor{currentstroke}%
\pgfsetdash{}{0pt}%
\pgfpathmoveto{\pgfqpoint{3.935038in}{2.568674in}}%
\pgfpathlineto{\pgfqpoint{3.956990in}{2.569399in}}%
\pgfpathlineto{\pgfqpoint{3.978942in}{2.570556in}}%
\pgfpathlineto{\pgfqpoint{4.000894in}{2.572615in}}%
\pgfpathlineto{\pgfqpoint{4.022846in}{2.576272in}}%
\pgfpathlineto{\pgfqpoint{4.044798in}{2.583417in}}%
\pgfpathlineto{\pgfqpoint{4.066750in}{2.597158in}}%
\pgfpathlineto{\pgfqpoint{4.088702in}{2.619670in}}%
\pgfpathlineto{\pgfqpoint{4.110654in}{2.649029in}}%
\pgfpathlineto{\pgfqpoint{4.132606in}{2.679535in}}%
\pgfpathlineto{\pgfqpoint{4.154559in}{2.705422in}}%
\pgfpathlineto{\pgfqpoint{4.176511in}{2.723560in}}%
\pgfpathlineto{\pgfqpoint{4.198463in}{2.735285in}}%
\pgfpathlineto{\pgfqpoint{4.220415in}{2.744153in}}%
\pgfpathlineto{\pgfqpoint{4.242367in}{2.751519in}}%
\pgfpathlineto{\pgfqpoint{4.264319in}{2.754282in}}%
\pgfpathlineto{\pgfqpoint{4.286271in}{2.744861in}}%
\pgfpathlineto{\pgfqpoint{4.308223in}{2.712721in}}%
\pgfpathlineto{\pgfqpoint{4.330175in}{2.649547in}}%
\pgfpathlineto{\pgfqpoint{4.352127in}{2.557554in}}%
\pgfpathlineto{\pgfqpoint{4.374079in}{2.450333in}}%
\pgfpathlineto{\pgfqpoint{4.396031in}{2.343346in}}%
\pgfpathlineto{\pgfqpoint{4.417983in}{2.243852in}}%
\pgfpathlineto{\pgfqpoint{4.439935in}{2.149422in}}%
\pgfpathlineto{\pgfqpoint{4.461887in}{2.053921in}}%
\pgfpathlineto{\pgfqpoint{4.483839in}{1.954821in}}%
\pgfpathlineto{\pgfqpoint{4.505791in}{1.855808in}}%
\pgfpathlineto{\pgfqpoint{4.527743in}{1.764366in}}%
\pgfpathlineto{\pgfqpoint{4.549695in}{1.686430in}}%
\pgfpathlineto{\pgfqpoint{4.571647in}{1.623217in}}%
\pgfpathlineto{\pgfqpoint{4.593599in}{1.572158in}}%
\pgfpathlineto{\pgfqpoint{4.615551in}{1.531338in}}%
\pgfpathlineto{\pgfqpoint{4.637503in}{1.503108in}}%
\pgfpathlineto{\pgfqpoint{4.659455in}{1.490910in}}%
\pgfpathlineto{\pgfqpoint{4.681407in}{1.495663in}}%
\pgfpathlineto{\pgfqpoint{4.703359in}{1.514316in}}%
\pgfpathlineto{\pgfqpoint{4.725311in}{1.540677in}}%
\pgfpathlineto{\pgfqpoint{4.747263in}{1.567862in}}%
\pgfpathlineto{\pgfqpoint{4.769215in}{1.589643in}}%
\pgfpathlineto{\pgfqpoint{4.791167in}{1.600564in}}%
\pgfpathlineto{\pgfqpoint{4.813119in}{1.596995in}}%
\pgfpathlineto{\pgfqpoint{4.835071in}{1.579640in}}%
\pgfpathlineto{\pgfqpoint{4.857023in}{1.556384in}}%
\pgfpathlineto{\pgfqpoint{4.878975in}{1.540926in}}%
\pgfpathlineto{\pgfqpoint{4.900927in}{1.550011in}}%
\pgfpathlineto{\pgfqpoint{4.922879in}{1.599925in}}%
\pgfpathlineto{\pgfqpoint{4.944831in}{1.698232in}}%
\pgfpathlineto{\pgfqpoint{4.966783in}{1.836945in}}%
\pgfpathlineto{\pgfqpoint{4.988735in}{1.994427in}}%
\pgfpathlineto{\pgfqpoint{5.010687in}{2.143447in}}%
\pgfpathlineto{\pgfqpoint{5.032639in}{2.262524in}}%
\pgfpathlineto{\pgfqpoint{5.054591in}{2.342597in}}%
\pgfpathlineto{\pgfqpoint{5.076543in}{2.386648in}}%
\pgfpathlineto{\pgfqpoint{5.098495in}{2.405372in}}%
\pgfpathlineto{\pgfqpoint{5.120447in}{2.413104in}}%
\pgfpathlineto{\pgfqpoint{5.142399in}{2.423463in}}%
\pgfpathlineto{\pgfqpoint{5.164351in}{2.443406in}}%
\pgfpathlineto{\pgfqpoint{5.186303in}{2.471630in}}%
\pgfpathlineto{\pgfqpoint{5.208255in}{2.501932in}}%
\pgfpathlineto{\pgfqpoint{5.230207in}{2.527878in}}%
\pgfpathlineto{\pgfqpoint{5.252159in}{2.546501in}}%
\pgfpathlineto{\pgfqpoint{5.274111in}{2.558714in}}%
\pgfpathlineto{\pgfqpoint{5.296063in}{2.566817in}}%
\pgfpathlineto{\pgfqpoint{5.318015in}{2.572209in}}%
\pgfpathlineto{\pgfqpoint{5.339967in}{2.575261in}}%
\pgfpathlineto{\pgfqpoint{5.361919in}{2.575966in}}%
\pgfpathlineto{\pgfqpoint{5.383871in}{2.574616in}}%
\pgfpathlineto{\pgfqpoint{5.405823in}{2.572415in}}%
\pgfpathlineto{\pgfqpoint{5.427775in}{2.570652in}}%
\pgfpathlineto{\pgfqpoint{5.449727in}{2.569495in}}%
\pgfpathlineto{\pgfqpoint{5.471679in}{2.568116in}}%
\pgfpathlineto{\pgfqpoint{5.493631in}{2.566578in}}%
\pgfpathlineto{\pgfqpoint{5.515583in}{2.565888in}}%
\pgfpathlineto{\pgfqpoint{5.537535in}{2.566547in}}%
\pgfpathlineto{\pgfqpoint{5.559488in}{2.567798in}}%
\pgfpathlineto{\pgfqpoint{5.581440in}{2.568471in}}%
\pgfpathlineto{\pgfqpoint{5.603392in}{2.567797in}}%
\pgfpathlineto{\pgfqpoint{5.625344in}{2.567151in}}%
\pgfpathlineto{\pgfqpoint{5.647296in}{2.571622in}}%
\pgfpathlineto{\pgfqpoint{5.669248in}{2.588017in}}%
\pgfpathlineto{\pgfqpoint{5.691200in}{2.619633in}}%
\pgfpathlineto{\pgfqpoint{5.713152in}{2.662083in}}%
\pgfpathlineto{\pgfqpoint{5.735104in}{2.704505in}}%
\pgfpathlineto{\pgfqpoint{5.757056in}{2.737152in}}%
\pgfpathlineto{\pgfqpoint{5.779008in}{2.755789in}}%
\pgfpathlineto{\pgfqpoint{5.800960in}{2.762352in}}%
\pgfpathlineto{\pgfqpoint{5.822912in}{2.762246in}}%
\pgfpathlineto{\pgfqpoint{5.844864in}{2.759277in}}%
\pgfpathlineto{\pgfqpoint{5.866816in}{2.751519in}}%
\pgfpathlineto{\pgfqpoint{5.888768in}{2.731499in}}%
\pgfpathlineto{\pgfqpoint{5.910720in}{2.690481in}}%
\pgfpathlineto{\pgfqpoint{5.932672in}{2.622016in}}%
\pgfpathlineto{\pgfqpoint{5.954624in}{2.525799in}}%
\pgfpathlineto{\pgfqpoint{5.976576in}{2.410707in}}%
\pgfpathlineto{\pgfqpoint{5.998528in}{2.288944in}}%
\pgfpathlineto{\pgfqpoint{6.020480in}{2.168785in}}%
\pgfpathlineto{\pgfqpoint{6.042432in}{2.051633in}}%
\pgfpathlineto{\pgfqpoint{6.064384in}{1.933678in}}%
\pgfpathlineto{\pgfqpoint{6.086336in}{1.812983in}}%
\pgfpathlineto{\pgfqpoint{6.108288in}{1.693631in}}%
\pgfpathlineto{\pgfqpoint{6.130240in}{1.583627in}}%
\pgfusepath{stroke}%
\end{pgfscope}%
\begin{pgfscope}%
\pgfpathrectangle{\pgfqpoint{3.825278in}{0.574768in}}{\pgfqpoint{2.414722in}{2.357859in}}%
\pgfusepath{clip}%
\pgfsetroundcap%
\pgfsetroundjoin%
\pgfsetlinewidth{1.505625pt}%
\definecolor{currentstroke}{rgb}{0.505882,0.447059,0.701961}%
\pgfsetstrokecolor{currentstroke}%
\pgfsetdash{}{0pt}%
\pgfpathmoveto{\pgfqpoint{3.935038in}{2.023519in}}%
\pgfpathlineto{\pgfqpoint{3.958070in}{2.199541in}}%
\pgfpathlineto{\pgfqpoint{3.981102in}{2.363775in}}%
\pgfpathlineto{\pgfqpoint{4.004133in}{2.508114in}}%
\pgfpathlineto{\pgfqpoint{4.027165in}{2.624299in}}%
\pgfpathlineto{\pgfqpoint{4.050196in}{2.701648in}}%
\pgfpathlineto{\pgfqpoint{4.073228in}{2.739628in}}%
\pgfpathlineto{\pgfqpoint{4.096260in}{2.751519in}}%
\pgfpathlineto{\pgfqpoint{4.119291in}{2.751167in}}%
\pgfpathlineto{\pgfqpoint{4.142323in}{2.742228in}}%
\pgfpathlineto{\pgfqpoint{4.165355in}{2.720359in}}%
\pgfpathlineto{\pgfqpoint{4.188386in}{2.678692in}}%
\pgfpathlineto{\pgfqpoint{4.211418in}{2.611869in}}%
\pgfpathlineto{\pgfqpoint{4.234449in}{2.519467in}}%
\pgfpathlineto{\pgfqpoint{4.257481in}{2.407006in}}%
\pgfpathlineto{\pgfqpoint{4.280513in}{2.285007in}}%
\pgfpathlineto{\pgfqpoint{4.303544in}{2.163607in}}%
\pgfpathlineto{\pgfqpoint{4.326576in}{2.045673in}}%
\pgfpathlineto{\pgfqpoint{4.349608in}{1.929450in}}%
\pgfpathlineto{\pgfqpoint{4.372639in}{1.815676in}}%
\pgfpathlineto{\pgfqpoint{4.395671in}{1.706514in}}%
\pgfpathlineto{\pgfqpoint{4.418702in}{1.603297in}}%
\pgfpathlineto{\pgfqpoint{4.441734in}{1.506902in}}%
\pgfpathlineto{\pgfqpoint{4.464766in}{1.419090in}}%
\pgfpathlineto{\pgfqpoint{4.487797in}{1.341149in}}%
\pgfpathlineto{\pgfqpoint{4.510829in}{1.274104in}}%
\pgfpathlineto{\pgfqpoint{4.533861in}{1.220140in}}%
\pgfpathlineto{\pgfqpoint{4.556892in}{1.182460in}}%
\pgfpathlineto{\pgfqpoint{4.579924in}{1.162366in}}%
\pgfpathlineto{\pgfqpoint{4.602955in}{1.156061in}}%
\pgfpathlineto{\pgfqpoint{4.625987in}{1.156847in}}%
\pgfpathlineto{\pgfqpoint{4.649019in}{1.163758in}}%
\pgfpathlineto{\pgfqpoint{4.672050in}{1.187128in}}%
\pgfpathlineto{\pgfqpoint{4.695082in}{1.238202in}}%
\pgfpathlineto{\pgfqpoint{4.718114in}{1.314504in}}%
\pgfpathlineto{\pgfqpoint{4.741145in}{1.400221in}}%
\pgfpathlineto{\pgfqpoint{4.764177in}{1.480961in}}%
\pgfpathlineto{\pgfqpoint{4.787208in}{1.555021in}}%
\pgfpathlineto{\pgfqpoint{4.810240in}{1.628837in}}%
\pgfpathlineto{\pgfqpoint{4.833272in}{1.708225in}}%
\pgfpathlineto{\pgfqpoint{4.856303in}{1.793225in}}%
\pgfpathlineto{\pgfqpoint{4.879335in}{1.880576in}}%
\pgfpathlineto{\pgfqpoint{4.902367in}{1.965158in}}%
\pgfpathlineto{\pgfqpoint{4.925398in}{2.039800in}}%
\pgfpathlineto{\pgfqpoint{4.948430in}{2.098301in}}%
\pgfpathlineto{\pgfqpoint{4.971461in}{2.137726in}}%
\pgfpathlineto{\pgfqpoint{4.994493in}{2.158978in}}%
\pgfpathlineto{\pgfqpoint{5.017525in}{2.165003in}}%
\pgfpathlineto{\pgfqpoint{5.040556in}{2.159739in}}%
\pgfpathlineto{\pgfqpoint{5.063588in}{2.149778in}}%
\pgfpathlineto{\pgfqpoint{5.086619in}{2.147060in}}%
\pgfpathlineto{\pgfqpoint{5.109651in}{2.168311in}}%
\pgfpathlineto{\pgfqpoint{5.132683in}{2.230488in}}%
\pgfpathlineto{\pgfqpoint{5.155714in}{2.336753in}}%
\pgfpathlineto{\pgfqpoint{5.178746in}{2.468883in}}%
\pgfpathlineto{\pgfqpoint{5.201778in}{2.597971in}}%
\pgfpathlineto{\pgfqpoint{5.224809in}{2.697572in}}%
\pgfpathlineto{\pgfqpoint{5.247841in}{2.750258in}}%
\pgfpathlineto{\pgfqpoint{5.270872in}{2.760806in}}%
\pgfpathlineto{\pgfqpoint{5.293904in}{2.751519in}}%
\pgfpathlineto{\pgfqpoint{5.316936in}{2.742824in}}%
\pgfpathlineto{\pgfqpoint{5.339967in}{2.738214in}}%
\pgfpathlineto{\pgfqpoint{5.362999in}{2.725837in}}%
\pgfpathlineto{\pgfqpoint{5.386031in}{2.689762in}}%
\pgfpathlineto{\pgfqpoint{5.409062in}{2.621128in}}%
\pgfpathlineto{\pgfqpoint{5.432094in}{2.521586in}}%
\pgfpathlineto{\pgfqpoint{5.455125in}{2.401467in}}%
\pgfusepath{stroke}%
\end{pgfscope}%
\begin{pgfscope}%
\pgfsetrectcap%
\pgfsetmiterjoin%
\pgfsetlinewidth{1.254687pt}%
\definecolor{currentstroke}{rgb}{1.000000,1.000000,1.000000}%
\pgfsetstrokecolor{currentstroke}%
\pgfsetdash{}{0pt}%
\pgfpathmoveto{\pgfqpoint{3.825278in}{0.574768in}}%
\pgfpathlineto{\pgfqpoint{3.825278in}{2.932627in}}%
\pgfusepath{stroke}%
\end{pgfscope}%
\begin{pgfscope}%
\pgfsetrectcap%
\pgfsetmiterjoin%
\pgfsetlinewidth{1.254687pt}%
\definecolor{currentstroke}{rgb}{1.000000,1.000000,1.000000}%
\pgfsetstrokecolor{currentstroke}%
\pgfsetdash{}{0pt}%
\pgfpathmoveto{\pgfqpoint{6.240000in}{0.574768in}}%
\pgfpathlineto{\pgfqpoint{6.240000in}{2.932627in}}%
\pgfusepath{stroke}%
\end{pgfscope}%
\begin{pgfscope}%
\pgfsetrectcap%
\pgfsetmiterjoin%
\pgfsetlinewidth{1.254687pt}%
\definecolor{currentstroke}{rgb}{1.000000,1.000000,1.000000}%
\pgfsetstrokecolor{currentstroke}%
\pgfsetdash{}{0pt}%
\pgfpathmoveto{\pgfqpoint{3.825278in}{0.574768in}}%
\pgfpathlineto{\pgfqpoint{6.240000in}{0.574768in}}%
\pgfusepath{stroke}%
\end{pgfscope}%
\begin{pgfscope}%
\pgfsetrectcap%
\pgfsetmiterjoin%
\pgfsetlinewidth{1.254687pt}%
\definecolor{currentstroke}{rgb}{1.000000,1.000000,1.000000}%
\pgfsetstrokecolor{currentstroke}%
\pgfsetdash{}{0pt}%
\pgfpathmoveto{\pgfqpoint{3.825278in}{2.932627in}}%
\pgfpathlineto{\pgfqpoint{6.240000in}{2.932627in}}%
\pgfusepath{stroke}%
\end{pgfscope}%
\end{pgfpicture}%
\makeatother%
\endgroup%

    \caption{Here the curves of five random cluster members assigned by the \textit{gls/all-views/regular/weighted/2} method.
             Each row represents one of the seven possible strain curves in the 4CH view. Coloumn (a) and (b) represent cluster 1 and 2 respectively.
             To make it easier to visually separate the curves, only five random members from cluster 1 and 2 are included in the figure.}
    \label{fig:five_members_gls_rls_4CH_regular_complete_two}
\end{figure}

\clearpage

\subsection{Peak-value Clustering}

\begin{figure}[H]
    \centering
    % \includegraphics[width=\textwidth]{results/pvc_ind_dor_sens_spec_dist.png}
    %% Creator: Matplotlib, PGF backend
%%
%% To include the figure in your LaTeX document, write
%%   \input{<filename>.pgf}
%%
%% Make sure the required packages are loaded in your preamble
%%   \usepackage{pgf}
%%
%% Figures using additional raster images can only be included by \input if
%% they are in the same directory as the main LaTeX file. For loading figures
%% from other directories you can use the `import` package
%%   \usepackage{import}
%% and then include the figures with
%%   \import{<path to file>}{<filename>.pgf}
%%
%% Matplotlib used the following preamble
%%
\begingroup%
\makeatletter%
\begin{pgfpicture}%
\pgfpathrectangle{\pgfpointorigin}{\pgfqpoint{6.360708in}{2.540000in}}%
\pgfusepath{use as bounding box, clip}%
\begin{pgfscope}%
\pgfsetbuttcap%
\pgfsetmiterjoin%
\definecolor{currentfill}{rgb}{1.000000,1.000000,1.000000}%
\pgfsetfillcolor{currentfill}%
\pgfsetlinewidth{0.000000pt}%
\definecolor{currentstroke}{rgb}{1.000000,1.000000,1.000000}%
\pgfsetstrokecolor{currentstroke}%
\pgfsetdash{}{0pt}%
\pgfpathmoveto{\pgfqpoint{0.000000in}{0.000000in}}%
\pgfpathlineto{\pgfqpoint{6.360708in}{0.000000in}}%
\pgfpathlineto{\pgfqpoint{6.360708in}{2.540000in}}%
\pgfpathlineto{\pgfqpoint{0.000000in}{2.540000in}}%
\pgfpathclose%
\pgfusepath{fill}%
\end{pgfscope}%
\begin{pgfscope}%
\pgfsetbuttcap%
\pgfsetmiterjoin%
\definecolor{currentfill}{rgb}{0.917647,0.917647,0.949020}%
\pgfsetfillcolor{currentfill}%
\pgfsetlinewidth{0.000000pt}%
\definecolor{currentstroke}{rgb}{0.000000,0.000000,0.000000}%
\pgfsetstrokecolor{currentstroke}%
\pgfsetstrokeopacity{0.000000}%
\pgfsetdash{}{0pt}%
\pgfpathmoveto{\pgfqpoint{0.498727in}{0.557870in}}%
\pgfpathlineto{\pgfqpoint{3.017235in}{0.557870in}}%
\pgfpathlineto{\pgfqpoint{3.017235in}{2.242604in}}%
\pgfpathlineto{\pgfqpoint{0.498727in}{2.242604in}}%
\pgfpathclose%
\pgfusepath{fill}%
\end{pgfscope}%
\begin{pgfscope}%
\pgfpathrectangle{\pgfqpoint{0.498727in}{0.557870in}}{\pgfqpoint{2.518508in}{1.684734in}}%
\pgfusepath{clip}%
\pgfsetroundcap%
\pgfsetroundjoin%
\pgfsetlinewidth{1.003750pt}%
\definecolor{currentstroke}{rgb}{1.000000,1.000000,1.000000}%
\pgfsetstrokecolor{currentstroke}%
\pgfsetdash{}{0pt}%
\pgfpathmoveto{\pgfqpoint{0.613204in}{0.557870in}}%
\pgfpathlineto{\pgfqpoint{0.613204in}{2.242604in}}%
\pgfusepath{stroke}%
\end{pgfscope}%
\begin{pgfscope}%
\definecolor{textcolor}{rgb}{0.150000,0.150000,0.150000}%
\pgfsetstrokecolor{textcolor}%
\pgfsetfillcolor{textcolor}%
\pgftext[x=0.613204in,y=0.425926in,,top]{\color{textcolor}\sffamily\fontsize{11.000000}{13.200000}\selectfont \(\displaystyle 0\)}%
\end{pgfscope}%
\begin{pgfscope}%
\pgfpathrectangle{\pgfqpoint{0.498727in}{0.557870in}}{\pgfqpoint{2.518508in}{1.684734in}}%
\pgfusepath{clip}%
\pgfsetroundcap%
\pgfsetroundjoin%
\pgfsetlinewidth{1.003750pt}%
\definecolor{currentstroke}{rgb}{1.000000,1.000000,1.000000}%
\pgfsetstrokecolor{currentstroke}%
\pgfsetdash{}{0pt}%
\pgfpathmoveto{\pgfqpoint{1.195294in}{0.557870in}}%
\pgfpathlineto{\pgfqpoint{1.195294in}{2.242604in}}%
\pgfusepath{stroke}%
\end{pgfscope}%
\begin{pgfscope}%
\definecolor{textcolor}{rgb}{0.150000,0.150000,0.150000}%
\pgfsetstrokecolor{textcolor}%
\pgfsetfillcolor{textcolor}%
\pgftext[x=1.195294in,y=0.425926in,,top]{\color{textcolor}\sffamily\fontsize{11.000000}{13.200000}\selectfont \(\displaystyle 10\)}%
\end{pgfscope}%
\begin{pgfscope}%
\pgfpathrectangle{\pgfqpoint{0.498727in}{0.557870in}}{\pgfqpoint{2.518508in}{1.684734in}}%
\pgfusepath{clip}%
\pgfsetroundcap%
\pgfsetroundjoin%
\pgfsetlinewidth{1.003750pt}%
\definecolor{currentstroke}{rgb}{1.000000,1.000000,1.000000}%
\pgfsetstrokecolor{currentstroke}%
\pgfsetdash{}{0pt}%
\pgfpathmoveto{\pgfqpoint{1.777384in}{0.557870in}}%
\pgfpathlineto{\pgfqpoint{1.777384in}{2.242604in}}%
\pgfusepath{stroke}%
\end{pgfscope}%
\begin{pgfscope}%
\definecolor{textcolor}{rgb}{0.150000,0.150000,0.150000}%
\pgfsetstrokecolor{textcolor}%
\pgfsetfillcolor{textcolor}%
\pgftext[x=1.777384in,y=0.425926in,,top]{\color{textcolor}\sffamily\fontsize{11.000000}{13.200000}\selectfont \(\displaystyle 20\)}%
\end{pgfscope}%
\begin{pgfscope}%
\pgfpathrectangle{\pgfqpoint{0.498727in}{0.557870in}}{\pgfqpoint{2.518508in}{1.684734in}}%
\pgfusepath{clip}%
\pgfsetroundcap%
\pgfsetroundjoin%
\pgfsetlinewidth{1.003750pt}%
\definecolor{currentstroke}{rgb}{1.000000,1.000000,1.000000}%
\pgfsetstrokecolor{currentstroke}%
\pgfsetdash{}{0pt}%
\pgfpathmoveto{\pgfqpoint{2.359474in}{0.557870in}}%
\pgfpathlineto{\pgfqpoint{2.359474in}{2.242604in}}%
\pgfusepath{stroke}%
\end{pgfscope}%
\begin{pgfscope}%
\definecolor{textcolor}{rgb}{0.150000,0.150000,0.150000}%
\pgfsetstrokecolor{textcolor}%
\pgfsetfillcolor{textcolor}%
\pgftext[x=2.359474in,y=0.425926in,,top]{\color{textcolor}\sffamily\fontsize{11.000000}{13.200000}\selectfont \(\displaystyle 30\)}%
\end{pgfscope}%
\begin{pgfscope}%
\pgfpathrectangle{\pgfqpoint{0.498727in}{0.557870in}}{\pgfqpoint{2.518508in}{1.684734in}}%
\pgfusepath{clip}%
\pgfsetroundcap%
\pgfsetroundjoin%
\pgfsetlinewidth{1.003750pt}%
\definecolor{currentstroke}{rgb}{1.000000,1.000000,1.000000}%
\pgfsetstrokecolor{currentstroke}%
\pgfsetdash{}{0pt}%
\pgfpathmoveto{\pgfqpoint{2.941563in}{0.557870in}}%
\pgfpathlineto{\pgfqpoint{2.941563in}{2.242604in}}%
\pgfusepath{stroke}%
\end{pgfscope}%
\begin{pgfscope}%
\definecolor{textcolor}{rgb}{0.150000,0.150000,0.150000}%
\pgfsetstrokecolor{textcolor}%
\pgfsetfillcolor{textcolor}%
\pgftext[x=2.941563in,y=0.425926in,,top]{\color{textcolor}\sffamily\fontsize{11.000000}{13.200000}\selectfont \(\displaystyle 40\)}%
\end{pgfscope}%
\begin{pgfscope}%
\definecolor{textcolor}{rgb}{0.150000,0.150000,0.150000}%
\pgfsetstrokecolor{textcolor}%
\pgfsetfillcolor{textcolor}%
\pgftext[x=1.757981in,y=0.235185in,,top]{\color{textcolor}\sffamily\fontsize{11.000000}{13.200000}\selectfont DOR}%
\end{pgfscope}%
\begin{pgfscope}%
\pgfpathrectangle{\pgfqpoint{0.498727in}{0.557870in}}{\pgfqpoint{2.518508in}{1.684734in}}%
\pgfusepath{clip}%
\pgfsetroundcap%
\pgfsetroundjoin%
\pgfsetlinewidth{1.003750pt}%
\definecolor{currentstroke}{rgb}{1.000000,1.000000,1.000000}%
\pgfsetstrokecolor{currentstroke}%
\pgfsetdash{}{0pt}%
\pgfpathmoveto{\pgfqpoint{0.498727in}{0.557870in}}%
\pgfpathlineto{\pgfqpoint{3.017235in}{0.557870in}}%
\pgfusepath{stroke}%
\end{pgfscope}%
\begin{pgfscope}%
\definecolor{textcolor}{rgb}{0.150000,0.150000,0.150000}%
\pgfsetstrokecolor{textcolor}%
\pgfsetfillcolor{textcolor}%
\pgftext[x=0.290741in,y=0.505064in,left,base]{\color{textcolor}\sffamily\fontsize{11.000000}{13.200000}\selectfont \(\displaystyle 0\)}%
\end{pgfscope}%
\begin{pgfscope}%
\pgfpathrectangle{\pgfqpoint{0.498727in}{0.557870in}}{\pgfqpoint{2.518508in}{1.684734in}}%
\pgfusepath{clip}%
\pgfsetroundcap%
\pgfsetroundjoin%
\pgfsetlinewidth{1.003750pt}%
\definecolor{currentstroke}{rgb}{1.000000,1.000000,1.000000}%
\pgfsetstrokecolor{currentstroke}%
\pgfsetdash{}{0pt}%
\pgfpathmoveto{\pgfqpoint{0.498727in}{0.958997in}}%
\pgfpathlineto{\pgfqpoint{3.017235in}{0.958997in}}%
\pgfusepath{stroke}%
\end{pgfscope}%
\begin{pgfscope}%
\definecolor{textcolor}{rgb}{0.150000,0.150000,0.150000}%
\pgfsetstrokecolor{textcolor}%
\pgfsetfillcolor{textcolor}%
\pgftext[x=0.290741in,y=0.906191in,left,base]{\color{textcolor}\sffamily\fontsize{11.000000}{13.200000}\selectfont \(\displaystyle 2\)}%
\end{pgfscope}%
\begin{pgfscope}%
\pgfpathrectangle{\pgfqpoint{0.498727in}{0.557870in}}{\pgfqpoint{2.518508in}{1.684734in}}%
\pgfusepath{clip}%
\pgfsetroundcap%
\pgfsetroundjoin%
\pgfsetlinewidth{1.003750pt}%
\definecolor{currentstroke}{rgb}{1.000000,1.000000,1.000000}%
\pgfsetstrokecolor{currentstroke}%
\pgfsetdash{}{0pt}%
\pgfpathmoveto{\pgfqpoint{0.498727in}{1.360125in}}%
\pgfpathlineto{\pgfqpoint{3.017235in}{1.360125in}}%
\pgfusepath{stroke}%
\end{pgfscope}%
\begin{pgfscope}%
\definecolor{textcolor}{rgb}{0.150000,0.150000,0.150000}%
\pgfsetstrokecolor{textcolor}%
\pgfsetfillcolor{textcolor}%
\pgftext[x=0.290741in,y=1.307318in,left,base]{\color{textcolor}\sffamily\fontsize{11.000000}{13.200000}\selectfont \(\displaystyle 4\)}%
\end{pgfscope}%
\begin{pgfscope}%
\pgfpathrectangle{\pgfqpoint{0.498727in}{0.557870in}}{\pgfqpoint{2.518508in}{1.684734in}}%
\pgfusepath{clip}%
\pgfsetroundcap%
\pgfsetroundjoin%
\pgfsetlinewidth{1.003750pt}%
\definecolor{currentstroke}{rgb}{1.000000,1.000000,1.000000}%
\pgfsetstrokecolor{currentstroke}%
\pgfsetdash{}{0pt}%
\pgfpathmoveto{\pgfqpoint{0.498727in}{1.761252in}}%
\pgfpathlineto{\pgfqpoint{3.017235in}{1.761252in}}%
\pgfusepath{stroke}%
\end{pgfscope}%
\begin{pgfscope}%
\definecolor{textcolor}{rgb}{0.150000,0.150000,0.150000}%
\pgfsetstrokecolor{textcolor}%
\pgfsetfillcolor{textcolor}%
\pgftext[x=0.290741in,y=1.708445in,left,base]{\color{textcolor}\sffamily\fontsize{11.000000}{13.200000}\selectfont \(\displaystyle 6\)}%
\end{pgfscope}%
\begin{pgfscope}%
\pgfpathrectangle{\pgfqpoint{0.498727in}{0.557870in}}{\pgfqpoint{2.518508in}{1.684734in}}%
\pgfusepath{clip}%
\pgfsetroundcap%
\pgfsetroundjoin%
\pgfsetlinewidth{1.003750pt}%
\definecolor{currentstroke}{rgb}{1.000000,1.000000,1.000000}%
\pgfsetstrokecolor{currentstroke}%
\pgfsetdash{}{0pt}%
\pgfpathmoveto{\pgfqpoint{0.498727in}{2.162379in}}%
\pgfpathlineto{\pgfqpoint{3.017235in}{2.162379in}}%
\pgfusepath{stroke}%
\end{pgfscope}%
\begin{pgfscope}%
\definecolor{textcolor}{rgb}{0.150000,0.150000,0.150000}%
\pgfsetstrokecolor{textcolor}%
\pgfsetfillcolor{textcolor}%
\pgftext[x=0.290741in,y=2.109572in,left,base]{\color{textcolor}\sffamily\fontsize{11.000000}{13.200000}\selectfont \(\displaystyle 8\)}%
\end{pgfscope}%
\begin{pgfscope}%
\definecolor{textcolor}{rgb}{0.150000,0.150000,0.150000}%
\pgfsetstrokecolor{textcolor}%
\pgfsetfillcolor{textcolor}%
\pgftext[x=0.235185in,y=1.400237in,,bottom,rotate=90.000000]{\color{textcolor}\sffamily\fontsize{11.000000}{13.200000}\selectfont Occurance}%
\end{pgfscope}%
\begin{pgfscope}%
\pgfpathrectangle{\pgfqpoint{0.498727in}{0.557870in}}{\pgfqpoint{2.518508in}{1.684734in}}%
\pgfusepath{clip}%
\pgfsetbuttcap%
\pgfsetmiterjoin%
\definecolor{currentfill}{rgb}{0.298039,0.447059,0.690196}%
\pgfsetfillcolor{currentfill}%
\pgfsetfillopacity{0.400000}%
\pgfsetlinewidth{1.003750pt}%
\definecolor{currentstroke}{rgb}{1.000000,1.000000,1.000000}%
\pgfsetstrokecolor{currentstroke}%
\pgfsetstrokeopacity{0.400000}%
\pgfsetdash{}{0pt}%
\pgfpathmoveto{\pgfqpoint{0.613204in}{0.557870in}}%
\pgfpathlineto{\pgfqpoint{0.842160in}{0.557870in}}%
\pgfpathlineto{\pgfqpoint{0.842160in}{2.162379in}}%
\pgfpathlineto{\pgfqpoint{0.613204in}{2.162379in}}%
\pgfpathclose%
\pgfusepath{stroke,fill}%
\end{pgfscope}%
\begin{pgfscope}%
\pgfpathrectangle{\pgfqpoint{0.498727in}{0.557870in}}{\pgfqpoint{2.518508in}{1.684734in}}%
\pgfusepath{clip}%
\pgfsetbuttcap%
\pgfsetmiterjoin%
\definecolor{currentfill}{rgb}{0.298039,0.447059,0.690196}%
\pgfsetfillcolor{currentfill}%
\pgfsetfillopacity{0.400000}%
\pgfsetlinewidth{1.003750pt}%
\definecolor{currentstroke}{rgb}{1.000000,1.000000,1.000000}%
\pgfsetstrokecolor{currentstroke}%
\pgfsetstrokeopacity{0.400000}%
\pgfsetdash{}{0pt}%
\pgfpathmoveto{\pgfqpoint{0.842160in}{0.557870in}}%
\pgfpathlineto{\pgfqpoint{1.071115in}{0.557870in}}%
\pgfpathlineto{\pgfqpoint{1.071115in}{0.758434in}}%
\pgfpathlineto{\pgfqpoint{0.842160in}{0.758434in}}%
\pgfpathclose%
\pgfusepath{stroke,fill}%
\end{pgfscope}%
\begin{pgfscope}%
\pgfpathrectangle{\pgfqpoint{0.498727in}{0.557870in}}{\pgfqpoint{2.518508in}{1.684734in}}%
\pgfusepath{clip}%
\pgfsetbuttcap%
\pgfsetmiterjoin%
\definecolor{currentfill}{rgb}{0.298039,0.447059,0.690196}%
\pgfsetfillcolor{currentfill}%
\pgfsetfillopacity{0.400000}%
\pgfsetlinewidth{1.003750pt}%
\definecolor{currentstroke}{rgb}{1.000000,1.000000,1.000000}%
\pgfsetstrokecolor{currentstroke}%
\pgfsetstrokeopacity{0.400000}%
\pgfsetdash{}{0pt}%
\pgfpathmoveto{\pgfqpoint{1.071115in}{0.557870in}}%
\pgfpathlineto{\pgfqpoint{1.300070in}{0.557870in}}%
\pgfpathlineto{\pgfqpoint{1.300070in}{0.958997in}}%
\pgfpathlineto{\pgfqpoint{1.071115in}{0.958997in}}%
\pgfpathclose%
\pgfusepath{stroke,fill}%
\end{pgfscope}%
\begin{pgfscope}%
\pgfpathrectangle{\pgfqpoint{0.498727in}{0.557870in}}{\pgfqpoint{2.518508in}{1.684734in}}%
\pgfusepath{clip}%
\pgfsetbuttcap%
\pgfsetmiterjoin%
\definecolor{currentfill}{rgb}{0.298039,0.447059,0.690196}%
\pgfsetfillcolor{currentfill}%
\pgfsetfillopacity{0.400000}%
\pgfsetlinewidth{1.003750pt}%
\definecolor{currentstroke}{rgb}{1.000000,1.000000,1.000000}%
\pgfsetstrokecolor{currentstroke}%
\pgfsetstrokeopacity{0.400000}%
\pgfsetdash{}{0pt}%
\pgfpathmoveto{\pgfqpoint{1.300070in}{0.557870in}}%
\pgfpathlineto{\pgfqpoint{1.529026in}{0.557870in}}%
\pgfpathlineto{\pgfqpoint{1.529026in}{1.360125in}}%
\pgfpathlineto{\pgfqpoint{1.300070in}{1.360125in}}%
\pgfpathclose%
\pgfusepath{stroke,fill}%
\end{pgfscope}%
\begin{pgfscope}%
\pgfpathrectangle{\pgfqpoint{0.498727in}{0.557870in}}{\pgfqpoint{2.518508in}{1.684734in}}%
\pgfusepath{clip}%
\pgfsetbuttcap%
\pgfsetmiterjoin%
\definecolor{currentfill}{rgb}{0.298039,0.447059,0.690196}%
\pgfsetfillcolor{currentfill}%
\pgfsetfillopacity{0.400000}%
\pgfsetlinewidth{1.003750pt}%
\definecolor{currentstroke}{rgb}{1.000000,1.000000,1.000000}%
\pgfsetstrokecolor{currentstroke}%
\pgfsetstrokeopacity{0.400000}%
\pgfsetdash{}{0pt}%
\pgfpathmoveto{\pgfqpoint{1.529026in}{0.557870in}}%
\pgfpathlineto{\pgfqpoint{1.757981in}{0.557870in}}%
\pgfpathlineto{\pgfqpoint{1.757981in}{0.557870in}}%
\pgfpathlineto{\pgfqpoint{1.529026in}{0.557870in}}%
\pgfpathclose%
\pgfusepath{stroke,fill}%
\end{pgfscope}%
\begin{pgfscope}%
\pgfpathrectangle{\pgfqpoint{0.498727in}{0.557870in}}{\pgfqpoint{2.518508in}{1.684734in}}%
\pgfusepath{clip}%
\pgfsetbuttcap%
\pgfsetmiterjoin%
\definecolor{currentfill}{rgb}{0.298039,0.447059,0.690196}%
\pgfsetfillcolor{currentfill}%
\pgfsetfillopacity{0.400000}%
\pgfsetlinewidth{1.003750pt}%
\definecolor{currentstroke}{rgb}{1.000000,1.000000,1.000000}%
\pgfsetstrokecolor{currentstroke}%
\pgfsetstrokeopacity{0.400000}%
\pgfsetdash{}{0pt}%
\pgfpathmoveto{\pgfqpoint{1.757981in}{0.557870in}}%
\pgfpathlineto{\pgfqpoint{1.986936in}{0.557870in}}%
\pgfpathlineto{\pgfqpoint{1.986936in}{0.758434in}}%
\pgfpathlineto{\pgfqpoint{1.757981in}{0.758434in}}%
\pgfpathclose%
\pgfusepath{stroke,fill}%
\end{pgfscope}%
\begin{pgfscope}%
\pgfpathrectangle{\pgfqpoint{0.498727in}{0.557870in}}{\pgfqpoint{2.518508in}{1.684734in}}%
\pgfusepath{clip}%
\pgfsetbuttcap%
\pgfsetmiterjoin%
\definecolor{currentfill}{rgb}{0.298039,0.447059,0.690196}%
\pgfsetfillcolor{currentfill}%
\pgfsetfillopacity{0.400000}%
\pgfsetlinewidth{1.003750pt}%
\definecolor{currentstroke}{rgb}{1.000000,1.000000,1.000000}%
\pgfsetstrokecolor{currentstroke}%
\pgfsetstrokeopacity{0.400000}%
\pgfsetdash{}{0pt}%
\pgfpathmoveto{\pgfqpoint{1.986936in}{0.557870in}}%
\pgfpathlineto{\pgfqpoint{2.215892in}{0.557870in}}%
\pgfpathlineto{\pgfqpoint{2.215892in}{0.758434in}}%
\pgfpathlineto{\pgfqpoint{1.986936in}{0.758434in}}%
\pgfpathclose%
\pgfusepath{stroke,fill}%
\end{pgfscope}%
\begin{pgfscope}%
\pgfpathrectangle{\pgfqpoint{0.498727in}{0.557870in}}{\pgfqpoint{2.518508in}{1.684734in}}%
\pgfusepath{clip}%
\pgfsetbuttcap%
\pgfsetmiterjoin%
\definecolor{currentfill}{rgb}{0.298039,0.447059,0.690196}%
\pgfsetfillcolor{currentfill}%
\pgfsetfillopacity{0.400000}%
\pgfsetlinewidth{1.003750pt}%
\definecolor{currentstroke}{rgb}{1.000000,1.000000,1.000000}%
\pgfsetstrokecolor{currentstroke}%
\pgfsetstrokeopacity{0.400000}%
\pgfsetdash{}{0pt}%
\pgfpathmoveto{\pgfqpoint{2.215892in}{0.557870in}}%
\pgfpathlineto{\pgfqpoint{2.444847in}{0.557870in}}%
\pgfpathlineto{\pgfqpoint{2.444847in}{0.557870in}}%
\pgfpathlineto{\pgfqpoint{2.215892in}{0.557870in}}%
\pgfpathclose%
\pgfusepath{stroke,fill}%
\end{pgfscope}%
\begin{pgfscope}%
\pgfpathrectangle{\pgfqpoint{0.498727in}{0.557870in}}{\pgfqpoint{2.518508in}{1.684734in}}%
\pgfusepath{clip}%
\pgfsetbuttcap%
\pgfsetmiterjoin%
\definecolor{currentfill}{rgb}{0.298039,0.447059,0.690196}%
\pgfsetfillcolor{currentfill}%
\pgfsetfillopacity{0.400000}%
\pgfsetlinewidth{1.003750pt}%
\definecolor{currentstroke}{rgb}{1.000000,1.000000,1.000000}%
\pgfsetstrokecolor{currentstroke}%
\pgfsetstrokeopacity{0.400000}%
\pgfsetdash{}{0pt}%
\pgfpathmoveto{\pgfqpoint{2.444847in}{0.557870in}}%
\pgfpathlineto{\pgfqpoint{2.673802in}{0.557870in}}%
\pgfpathlineto{\pgfqpoint{2.673802in}{0.958997in}}%
\pgfpathlineto{\pgfqpoint{2.444847in}{0.958997in}}%
\pgfpathclose%
\pgfusepath{stroke,fill}%
\end{pgfscope}%
\begin{pgfscope}%
\pgfpathrectangle{\pgfqpoint{0.498727in}{0.557870in}}{\pgfqpoint{2.518508in}{1.684734in}}%
\pgfusepath{clip}%
\pgfsetbuttcap%
\pgfsetmiterjoin%
\definecolor{currentfill}{rgb}{0.298039,0.447059,0.690196}%
\pgfsetfillcolor{currentfill}%
\pgfsetfillopacity{0.400000}%
\pgfsetlinewidth{1.003750pt}%
\definecolor{currentstroke}{rgb}{1.000000,1.000000,1.000000}%
\pgfsetstrokecolor{currentstroke}%
\pgfsetstrokeopacity{0.400000}%
\pgfsetdash{}{0pt}%
\pgfpathmoveto{\pgfqpoint{2.673802in}{0.557870in}}%
\pgfpathlineto{\pgfqpoint{2.902757in}{0.557870in}}%
\pgfpathlineto{\pgfqpoint{2.902757in}{0.958997in}}%
\pgfpathlineto{\pgfqpoint{2.673802in}{0.958997in}}%
\pgfpathclose%
\pgfusepath{stroke,fill}%
\end{pgfscope}%
\begin{pgfscope}%
\pgfsetrectcap%
\pgfsetmiterjoin%
\pgfsetlinewidth{1.254687pt}%
\definecolor{currentstroke}{rgb}{1.000000,1.000000,1.000000}%
\pgfsetstrokecolor{currentstroke}%
\pgfsetdash{}{0pt}%
\pgfpathmoveto{\pgfqpoint{0.498727in}{0.557870in}}%
\pgfpathlineto{\pgfqpoint{0.498727in}{2.242604in}}%
\pgfusepath{stroke}%
\end{pgfscope}%
\begin{pgfscope}%
\pgfsetrectcap%
\pgfsetmiterjoin%
\pgfsetlinewidth{1.254687pt}%
\definecolor{currentstroke}{rgb}{1.000000,1.000000,1.000000}%
\pgfsetstrokecolor{currentstroke}%
\pgfsetdash{}{0pt}%
\pgfpathmoveto{\pgfqpoint{3.017235in}{0.557870in}}%
\pgfpathlineto{\pgfqpoint{3.017235in}{2.242604in}}%
\pgfusepath{stroke}%
\end{pgfscope}%
\begin{pgfscope}%
\pgfsetrectcap%
\pgfsetmiterjoin%
\pgfsetlinewidth{1.254687pt}%
\definecolor{currentstroke}{rgb}{1.000000,1.000000,1.000000}%
\pgfsetstrokecolor{currentstroke}%
\pgfsetdash{}{0pt}%
\pgfpathmoveto{\pgfqpoint{0.498727in}{0.557870in}}%
\pgfpathlineto{\pgfqpoint{3.017235in}{0.557870in}}%
\pgfusepath{stroke}%
\end{pgfscope}%
\begin{pgfscope}%
\pgfsetrectcap%
\pgfsetmiterjoin%
\pgfsetlinewidth{1.254687pt}%
\definecolor{currentstroke}{rgb}{1.000000,1.000000,1.000000}%
\pgfsetstrokecolor{currentstroke}%
\pgfsetdash{}{0pt}%
\pgfpathmoveto{\pgfqpoint{0.498727in}{2.242604in}}%
\pgfpathlineto{\pgfqpoint{3.017235in}{2.242604in}}%
\pgfusepath{stroke}%
\end{pgfscope}%
\begin{pgfscope}%
\definecolor{textcolor}{rgb}{0.150000,0.150000,0.150000}%
\pgfsetstrokecolor{textcolor}%
\pgfsetfillcolor{textcolor}%
\pgftext[x=1.757981in,y=2.325938in,,base]{\color{textcolor}\sffamily\fontsize{11.000000}{13.200000}\selectfont (a)}%
\end{pgfscope}%
\begin{pgfscope}%
\pgfsetbuttcap%
\pgfsetmiterjoin%
\definecolor{currentfill}{rgb}{0.917647,0.917647,0.949020}%
\pgfsetfillcolor{currentfill}%
\pgfsetlinewidth{0.000000pt}%
\definecolor{currentstroke}{rgb}{0.000000,0.000000,0.000000}%
\pgfsetstrokecolor{currentstroke}%
\pgfsetstrokeopacity{0.000000}%
\pgfsetdash{}{0pt}%
\pgfpathmoveto{\pgfqpoint{3.721492in}{0.557870in}}%
\pgfpathlineto{\pgfqpoint{6.240000in}{0.557870in}}%
\pgfpathlineto{\pgfqpoint{6.240000in}{2.242604in}}%
\pgfpathlineto{\pgfqpoint{3.721492in}{2.242604in}}%
\pgfpathclose%
\pgfusepath{fill}%
\end{pgfscope}%
\begin{pgfscope}%
\pgfpathrectangle{\pgfqpoint{3.721492in}{0.557870in}}{\pgfqpoint{2.518508in}{1.684734in}}%
\pgfusepath{clip}%
\pgfsetroundcap%
\pgfsetroundjoin%
\pgfsetlinewidth{1.003750pt}%
\definecolor{currentstroke}{rgb}{1.000000,1.000000,1.000000}%
\pgfsetstrokecolor{currentstroke}%
\pgfsetdash{}{0pt}%
\pgfpathmoveto{\pgfqpoint{3.835969in}{0.557870in}}%
\pgfpathlineto{\pgfqpoint{3.835969in}{2.242604in}}%
\pgfusepath{stroke}%
\end{pgfscope}%
\begin{pgfscope}%
\definecolor{textcolor}{rgb}{0.150000,0.150000,0.150000}%
\pgfsetstrokecolor{textcolor}%
\pgfsetfillcolor{textcolor}%
\pgftext[x=3.835969in,y=0.425926in,,top]{\color{textcolor}\sffamily\fontsize{11.000000}{13.200000}\selectfont \(\displaystyle 0.00\)}%
\end{pgfscope}%
\begin{pgfscope}%
\pgfpathrectangle{\pgfqpoint{3.721492in}{0.557870in}}{\pgfqpoint{2.518508in}{1.684734in}}%
\pgfusepath{clip}%
\pgfsetroundcap%
\pgfsetroundjoin%
\pgfsetlinewidth{1.003750pt}%
\definecolor{currentstroke}{rgb}{1.000000,1.000000,1.000000}%
\pgfsetstrokecolor{currentstroke}%
\pgfsetdash{}{0pt}%
\pgfpathmoveto{\pgfqpoint{4.408358in}{0.557870in}}%
\pgfpathlineto{\pgfqpoint{4.408358in}{2.242604in}}%
\pgfusepath{stroke}%
\end{pgfscope}%
\begin{pgfscope}%
\definecolor{textcolor}{rgb}{0.150000,0.150000,0.150000}%
\pgfsetstrokecolor{textcolor}%
\pgfsetfillcolor{textcolor}%
\pgftext[x=4.408358in,y=0.425926in,,top]{\color{textcolor}\sffamily\fontsize{11.000000}{13.200000}\selectfont \(\displaystyle 0.25\)}%
\end{pgfscope}%
\begin{pgfscope}%
\pgfpathrectangle{\pgfqpoint{3.721492in}{0.557870in}}{\pgfqpoint{2.518508in}{1.684734in}}%
\pgfusepath{clip}%
\pgfsetroundcap%
\pgfsetroundjoin%
\pgfsetlinewidth{1.003750pt}%
\definecolor{currentstroke}{rgb}{1.000000,1.000000,1.000000}%
\pgfsetstrokecolor{currentstroke}%
\pgfsetdash{}{0pt}%
\pgfpathmoveto{\pgfqpoint{4.980746in}{0.557870in}}%
\pgfpathlineto{\pgfqpoint{4.980746in}{2.242604in}}%
\pgfusepath{stroke}%
\end{pgfscope}%
\begin{pgfscope}%
\definecolor{textcolor}{rgb}{0.150000,0.150000,0.150000}%
\pgfsetstrokecolor{textcolor}%
\pgfsetfillcolor{textcolor}%
\pgftext[x=4.980746in,y=0.425926in,,top]{\color{textcolor}\sffamily\fontsize{11.000000}{13.200000}\selectfont \(\displaystyle 0.50\)}%
\end{pgfscope}%
\begin{pgfscope}%
\pgfpathrectangle{\pgfqpoint{3.721492in}{0.557870in}}{\pgfqpoint{2.518508in}{1.684734in}}%
\pgfusepath{clip}%
\pgfsetroundcap%
\pgfsetroundjoin%
\pgfsetlinewidth{1.003750pt}%
\definecolor{currentstroke}{rgb}{1.000000,1.000000,1.000000}%
\pgfsetstrokecolor{currentstroke}%
\pgfsetdash{}{0pt}%
\pgfpathmoveto{\pgfqpoint{5.553134in}{0.557870in}}%
\pgfpathlineto{\pgfqpoint{5.553134in}{2.242604in}}%
\pgfusepath{stroke}%
\end{pgfscope}%
\begin{pgfscope}%
\definecolor{textcolor}{rgb}{0.150000,0.150000,0.150000}%
\pgfsetstrokecolor{textcolor}%
\pgfsetfillcolor{textcolor}%
\pgftext[x=5.553134in,y=0.425926in,,top]{\color{textcolor}\sffamily\fontsize{11.000000}{13.200000}\selectfont \(\displaystyle 0.75\)}%
\end{pgfscope}%
\begin{pgfscope}%
\pgfpathrectangle{\pgfqpoint{3.721492in}{0.557870in}}{\pgfqpoint{2.518508in}{1.684734in}}%
\pgfusepath{clip}%
\pgfsetroundcap%
\pgfsetroundjoin%
\pgfsetlinewidth{1.003750pt}%
\definecolor{currentstroke}{rgb}{1.000000,1.000000,1.000000}%
\pgfsetstrokecolor{currentstroke}%
\pgfsetdash{}{0pt}%
\pgfpathmoveto{\pgfqpoint{6.125522in}{0.557870in}}%
\pgfpathlineto{\pgfqpoint{6.125522in}{2.242604in}}%
\pgfusepath{stroke}%
\end{pgfscope}%
\begin{pgfscope}%
\definecolor{textcolor}{rgb}{0.150000,0.150000,0.150000}%
\pgfsetstrokecolor{textcolor}%
\pgfsetfillcolor{textcolor}%
\pgftext[x=6.125522in,y=0.425926in,,top]{\color{textcolor}\sffamily\fontsize{11.000000}{13.200000}\selectfont \(\displaystyle 1.00\)}%
\end{pgfscope}%
\begin{pgfscope}%
\definecolor{textcolor}{rgb}{0.150000,0.150000,0.150000}%
\pgfsetstrokecolor{textcolor}%
\pgfsetfillcolor{textcolor}%
\pgftext[x=4.980746in,y=0.235185in,,top]{\color{textcolor}\sffamily\fontsize{11.000000}{13.200000}\selectfont Specificity}%
\end{pgfscope}%
\begin{pgfscope}%
\pgfpathrectangle{\pgfqpoint{3.721492in}{0.557870in}}{\pgfqpoint{2.518508in}{1.684734in}}%
\pgfusepath{clip}%
\pgfsetroundcap%
\pgfsetroundjoin%
\pgfsetlinewidth{1.003750pt}%
\definecolor{currentstroke}{rgb}{1.000000,1.000000,1.000000}%
\pgfsetstrokecolor{currentstroke}%
\pgfsetdash{}{0pt}%
\pgfpathmoveto{\pgfqpoint{3.721492in}{0.634449in}}%
\pgfpathlineto{\pgfqpoint{6.240000in}{0.634449in}}%
\pgfusepath{stroke}%
\end{pgfscope}%
\begin{pgfscope}%
\definecolor{textcolor}{rgb}{0.150000,0.150000,0.150000}%
\pgfsetstrokecolor{textcolor}%
\pgfsetfillcolor{textcolor}%
\pgftext[x=3.319177in,y=0.581642in,left,base]{\color{textcolor}\sffamily\fontsize{11.000000}{13.200000}\selectfont \(\displaystyle 0.00\)}%
\end{pgfscope}%
\begin{pgfscope}%
\pgfpathrectangle{\pgfqpoint{3.721492in}{0.557870in}}{\pgfqpoint{2.518508in}{1.684734in}}%
\pgfusepath{clip}%
\pgfsetroundcap%
\pgfsetroundjoin%
\pgfsetlinewidth{1.003750pt}%
\definecolor{currentstroke}{rgb}{1.000000,1.000000,1.000000}%
\pgfsetstrokecolor{currentstroke}%
\pgfsetdash{}{0pt}%
\pgfpathmoveto{\pgfqpoint{3.721492in}{1.017343in}}%
\pgfpathlineto{\pgfqpoint{6.240000in}{1.017343in}}%
\pgfusepath{stroke}%
\end{pgfscope}%
\begin{pgfscope}%
\definecolor{textcolor}{rgb}{0.150000,0.150000,0.150000}%
\pgfsetstrokecolor{textcolor}%
\pgfsetfillcolor{textcolor}%
\pgftext[x=3.319177in,y=0.964536in,left,base]{\color{textcolor}\sffamily\fontsize{11.000000}{13.200000}\selectfont \(\displaystyle 0.25\)}%
\end{pgfscope}%
\begin{pgfscope}%
\pgfpathrectangle{\pgfqpoint{3.721492in}{0.557870in}}{\pgfqpoint{2.518508in}{1.684734in}}%
\pgfusepath{clip}%
\pgfsetroundcap%
\pgfsetroundjoin%
\pgfsetlinewidth{1.003750pt}%
\definecolor{currentstroke}{rgb}{1.000000,1.000000,1.000000}%
\pgfsetstrokecolor{currentstroke}%
\pgfsetdash{}{0pt}%
\pgfpathmoveto{\pgfqpoint{3.721492in}{1.400237in}}%
\pgfpathlineto{\pgfqpoint{6.240000in}{1.400237in}}%
\pgfusepath{stroke}%
\end{pgfscope}%
\begin{pgfscope}%
\definecolor{textcolor}{rgb}{0.150000,0.150000,0.150000}%
\pgfsetstrokecolor{textcolor}%
\pgfsetfillcolor{textcolor}%
\pgftext[x=3.319177in,y=1.347431in,left,base]{\color{textcolor}\sffamily\fontsize{11.000000}{13.200000}\selectfont \(\displaystyle 0.50\)}%
\end{pgfscope}%
\begin{pgfscope}%
\pgfpathrectangle{\pgfqpoint{3.721492in}{0.557870in}}{\pgfqpoint{2.518508in}{1.684734in}}%
\pgfusepath{clip}%
\pgfsetroundcap%
\pgfsetroundjoin%
\pgfsetlinewidth{1.003750pt}%
\definecolor{currentstroke}{rgb}{1.000000,1.000000,1.000000}%
\pgfsetstrokecolor{currentstroke}%
\pgfsetdash{}{0pt}%
\pgfpathmoveto{\pgfqpoint{3.721492in}{1.783131in}}%
\pgfpathlineto{\pgfqpoint{6.240000in}{1.783131in}}%
\pgfusepath{stroke}%
\end{pgfscope}%
\begin{pgfscope}%
\definecolor{textcolor}{rgb}{0.150000,0.150000,0.150000}%
\pgfsetstrokecolor{textcolor}%
\pgfsetfillcolor{textcolor}%
\pgftext[x=3.319177in,y=1.730325in,left,base]{\color{textcolor}\sffamily\fontsize{11.000000}{13.200000}\selectfont \(\displaystyle 0.75\)}%
\end{pgfscope}%
\begin{pgfscope}%
\pgfpathrectangle{\pgfqpoint{3.721492in}{0.557870in}}{\pgfqpoint{2.518508in}{1.684734in}}%
\pgfusepath{clip}%
\pgfsetroundcap%
\pgfsetroundjoin%
\pgfsetlinewidth{1.003750pt}%
\definecolor{currentstroke}{rgb}{1.000000,1.000000,1.000000}%
\pgfsetstrokecolor{currentstroke}%
\pgfsetdash{}{0pt}%
\pgfpathmoveto{\pgfqpoint{3.721492in}{2.166025in}}%
\pgfpathlineto{\pgfqpoint{6.240000in}{2.166025in}}%
\pgfusepath{stroke}%
\end{pgfscope}%
\begin{pgfscope}%
\definecolor{textcolor}{rgb}{0.150000,0.150000,0.150000}%
\pgfsetstrokecolor{textcolor}%
\pgfsetfillcolor{textcolor}%
\pgftext[x=3.319177in,y=2.113219in,left,base]{\color{textcolor}\sffamily\fontsize{11.000000}{13.200000}\selectfont \(\displaystyle 1.00\)}%
\end{pgfscope}%
\begin{pgfscope}%
\definecolor{textcolor}{rgb}{0.150000,0.150000,0.150000}%
\pgfsetstrokecolor{textcolor}%
\pgfsetfillcolor{textcolor}%
\pgftext[x=3.263621in,y=1.400237in,,bottom,rotate=90.000000]{\color{textcolor}\sffamily\fontsize{11.000000}{13.200000}\selectfont Sensitivity}%
\end{pgfscope}%
\begin{pgfscope}%
\pgfpathrectangle{\pgfqpoint{3.721492in}{0.557870in}}{\pgfqpoint{2.518508in}{1.684734in}}%
\pgfusepath{clip}%
\pgfsetbuttcap%
\pgfsetroundjoin%
\definecolor{currentfill}{rgb}{0.298039,0.447059,0.690196}%
\pgfsetfillcolor{currentfill}%
\pgfsetlinewidth{1.003750pt}%
\definecolor{currentstroke}{rgb}{0.298039,0.447059,0.690196}%
\pgfsetstrokecolor{currentstroke}%
\pgfsetdash{}{0pt}%
\pgfpathmoveto{\pgfqpoint{4.336809in}{1.890668in}}%
\pgfpathcurveto{\pgfqpoint{4.345045in}{1.890668in}}{\pgfqpoint{4.352945in}{1.893941in}}{\pgfqpoint{4.358769in}{1.899765in}}%
\pgfpathcurveto{\pgfqpoint{4.364593in}{1.905589in}}{\pgfqpoint{4.367866in}{1.913489in}}{\pgfqpoint{4.367866in}{1.921725in}}%
\pgfpathcurveto{\pgfqpoint{4.367866in}{1.929961in}}{\pgfqpoint{4.364593in}{1.937861in}}{\pgfqpoint{4.358769in}{1.943685in}}%
\pgfpathcurveto{\pgfqpoint{4.352945in}{1.949509in}}{\pgfqpoint{4.345045in}{1.952781in}}{\pgfqpoint{4.336809in}{1.952781in}}%
\pgfpathcurveto{\pgfqpoint{4.328573in}{1.952781in}}{\pgfqpoint{4.320673in}{1.949509in}}{\pgfqpoint{4.314849in}{1.943685in}}%
\pgfpathcurveto{\pgfqpoint{4.309025in}{1.937861in}}{\pgfqpoint{4.305753in}{1.929961in}}{\pgfqpoint{4.305753in}{1.921725in}}%
\pgfpathcurveto{\pgfqpoint{4.305753in}{1.913489in}}{\pgfqpoint{4.309025in}{1.905589in}}{\pgfqpoint{4.314849in}{1.899765in}}%
\pgfpathcurveto{\pgfqpoint{4.320673in}{1.893941in}}{\pgfqpoint{4.328573in}{1.890668in}}{\pgfqpoint{4.336809in}{1.890668in}}%
\pgfpathclose%
\pgfusepath{stroke,fill}%
\end{pgfscope}%
\begin{pgfscope}%
\pgfpathrectangle{\pgfqpoint{3.721492in}{0.557870in}}{\pgfqpoint{2.518508in}{1.684734in}}%
\pgfusepath{clip}%
\pgfsetbuttcap%
\pgfsetroundjoin%
\definecolor{currentfill}{rgb}{0.298039,0.447059,0.690196}%
\pgfsetfillcolor{currentfill}%
\pgfsetlinewidth{1.003750pt}%
\definecolor{currentstroke}{rgb}{0.298039,0.447059,0.690196}%
\pgfsetstrokecolor{currentstroke}%
\pgfsetdash{}{0pt}%
\pgfpathmoveto{\pgfqpoint{5.910877in}{1.373879in}}%
\pgfpathcurveto{\pgfqpoint{5.919113in}{1.373879in}}{\pgfqpoint{5.927013in}{1.377151in}}{\pgfqpoint{5.932837in}{1.382975in}}%
\pgfpathcurveto{\pgfqpoint{5.938661in}{1.388799in}}{\pgfqpoint{5.941933in}{1.396699in}}{\pgfqpoint{5.941933in}{1.404935in}}%
\pgfpathcurveto{\pgfqpoint{5.941933in}{1.413172in}}{\pgfqpoint{5.938661in}{1.421072in}}{\pgfqpoint{5.932837in}{1.426896in}}%
\pgfpathcurveto{\pgfqpoint{5.927013in}{1.432719in}}{\pgfqpoint{5.919113in}{1.435992in}}{\pgfqpoint{5.910877in}{1.435992in}}%
\pgfpathcurveto{\pgfqpoint{5.902640in}{1.435992in}}{\pgfqpoint{5.894740in}{1.432719in}}{\pgfqpoint{5.888917in}{1.426896in}}%
\pgfpathcurveto{\pgfqpoint{5.883093in}{1.421072in}}{\pgfqpoint{5.879820in}{1.413172in}}{\pgfqpoint{5.879820in}{1.404935in}}%
\pgfpathcurveto{\pgfqpoint{5.879820in}{1.396699in}}{\pgfqpoint{5.883093in}{1.388799in}}{\pgfqpoint{5.888917in}{1.382975in}}%
\pgfpathcurveto{\pgfqpoint{5.894740in}{1.377151in}}{\pgfqpoint{5.902640in}{1.373879in}}{\pgfqpoint{5.910877in}{1.373879in}}%
\pgfpathclose%
\pgfusepath{stroke,fill}%
\end{pgfscope}%
\begin{pgfscope}%
\pgfpathrectangle{\pgfqpoint{3.721492in}{0.557870in}}{\pgfqpoint{2.518508in}{1.684734in}}%
\pgfusepath{clip}%
\pgfsetbuttcap%
\pgfsetroundjoin%
\definecolor{currentfill}{rgb}{0.298039,0.447059,0.690196}%
\pgfsetfillcolor{currentfill}%
\pgfsetlinewidth{1.003750pt}%
\definecolor{currentstroke}{rgb}{0.298039,0.447059,0.690196}%
\pgfsetstrokecolor{currentstroke}%
\pgfsetdash{}{0pt}%
\pgfpathmoveto{\pgfqpoint{3.835969in}{2.125336in}}%
\pgfpathcurveto{\pgfqpoint{3.844206in}{2.125336in}}{\pgfqpoint{3.852106in}{2.128609in}}{\pgfqpoint{3.857930in}{2.134433in}}%
\pgfpathcurveto{\pgfqpoint{3.863754in}{2.140256in}}{\pgfqpoint{3.867026in}{2.148157in}}{\pgfqpoint{3.867026in}{2.156393in}}%
\pgfpathcurveto{\pgfqpoint{3.867026in}{2.164629in}}{\pgfqpoint{3.863754in}{2.172529in}}{\pgfqpoint{3.857930in}{2.178353in}}%
\pgfpathcurveto{\pgfqpoint{3.852106in}{2.184177in}}{\pgfqpoint{3.844206in}{2.187449in}}{\pgfqpoint{3.835969in}{2.187449in}}%
\pgfpathcurveto{\pgfqpoint{3.827733in}{2.187449in}}{\pgfqpoint{3.819833in}{2.184177in}}{\pgfqpoint{3.814009in}{2.178353in}}%
\pgfpathcurveto{\pgfqpoint{3.808185in}{2.172529in}}{\pgfqpoint{3.804913in}{2.164629in}}{\pgfqpoint{3.804913in}{2.156393in}}%
\pgfpathcurveto{\pgfqpoint{3.804913in}{2.148157in}}{\pgfqpoint{3.808185in}{2.140256in}}{\pgfqpoint{3.814009in}{2.134433in}}%
\pgfpathcurveto{\pgfqpoint{3.819833in}{2.128609in}}{\pgfqpoint{3.827733in}{2.125336in}}{\pgfqpoint{3.835969in}{2.125336in}}%
\pgfpathclose%
\pgfusepath{stroke,fill}%
\end{pgfscope}%
\begin{pgfscope}%
\pgfpathrectangle{\pgfqpoint{3.721492in}{0.557870in}}{\pgfqpoint{2.518508in}{1.684734in}}%
\pgfusepath{clip}%
\pgfsetbuttcap%
\pgfsetroundjoin%
\definecolor{currentfill}{rgb}{0.298039,0.447059,0.690196}%
\pgfsetfillcolor{currentfill}%
\pgfsetlinewidth{1.003750pt}%
\definecolor{currentstroke}{rgb}{0.298039,0.447059,0.690196}%
\pgfsetstrokecolor{currentstroke}%
\pgfsetdash{}{0pt}%
\pgfpathmoveto{\pgfqpoint{5.967622in}{1.383630in}}%
\pgfpathcurveto{\pgfqpoint{5.975858in}{1.383630in}}{\pgfqpoint{5.983758in}{1.386902in}}{\pgfqpoint{5.989582in}{1.392726in}}%
\pgfpathcurveto{\pgfqpoint{5.995406in}{1.398550in}}{\pgfqpoint{5.998679in}{1.406450in}}{\pgfqpoint{5.998679in}{1.414686in}}%
\pgfpathcurveto{\pgfqpoint{5.998679in}{1.422922in}}{\pgfqpoint{5.995406in}{1.430822in}}{\pgfqpoint{5.989582in}{1.436646in}}%
\pgfpathcurveto{\pgfqpoint{5.983758in}{1.442470in}}{\pgfqpoint{5.975858in}{1.445743in}}{\pgfqpoint{5.967622in}{1.445743in}}%
\pgfpathcurveto{\pgfqpoint{5.959386in}{1.445743in}}{\pgfqpoint{5.951486in}{1.442470in}}{\pgfqpoint{5.945662in}{1.436646in}}%
\pgfpathcurveto{\pgfqpoint{5.939838in}{1.430822in}}{\pgfqpoint{5.936566in}{1.422922in}}{\pgfqpoint{5.936566in}{1.414686in}}%
\pgfpathcurveto{\pgfqpoint{5.936566in}{1.406450in}}{\pgfqpoint{5.939838in}{1.398550in}}{\pgfqpoint{5.945662in}{1.392726in}}%
\pgfpathcurveto{\pgfqpoint{5.951486in}{1.386902in}}{\pgfqpoint{5.959386in}{1.383630in}}{\pgfqpoint{5.967622in}{1.383630in}}%
\pgfpathclose%
\pgfusepath{stroke,fill}%
\end{pgfscope}%
\begin{pgfscope}%
\pgfpathrectangle{\pgfqpoint{3.721492in}{0.557870in}}{\pgfqpoint{2.518508in}{1.684734in}}%
\pgfusepath{clip}%
\pgfsetbuttcap%
\pgfsetroundjoin%
\definecolor{currentfill}{rgb}{0.298039,0.447059,0.690196}%
\pgfsetfillcolor{currentfill}%
\pgfsetlinewidth{1.003750pt}%
\definecolor{currentstroke}{rgb}{0.298039,0.447059,0.690196}%
\pgfsetstrokecolor{currentstroke}%
\pgfsetdash{}{0pt}%
\pgfpathmoveto{\pgfqpoint{5.967622in}{1.354732in}}%
\pgfpathcurveto{\pgfqpoint{5.975858in}{1.354732in}}{\pgfqpoint{5.983758in}{1.358004in}}{\pgfqpoint{5.989582in}{1.363828in}}%
\pgfpathcurveto{\pgfqpoint{5.995406in}{1.369652in}}{\pgfqpoint{5.998679in}{1.377552in}}{\pgfqpoint{5.998679in}{1.385788in}}%
\pgfpathcurveto{\pgfqpoint{5.998679in}{1.394025in}}{\pgfqpoint{5.995406in}{1.401925in}}{\pgfqpoint{5.989582in}{1.407749in}}%
\pgfpathcurveto{\pgfqpoint{5.983758in}{1.413573in}}{\pgfqpoint{5.975858in}{1.416845in}}{\pgfqpoint{5.967622in}{1.416845in}}%
\pgfpathcurveto{\pgfqpoint{5.959386in}{1.416845in}}{\pgfqpoint{5.951486in}{1.413573in}}{\pgfqpoint{5.945662in}{1.407749in}}%
\pgfpathcurveto{\pgfqpoint{5.939838in}{1.401925in}}{\pgfqpoint{5.936566in}{1.394025in}}{\pgfqpoint{5.936566in}{1.385788in}}%
\pgfpathcurveto{\pgfqpoint{5.936566in}{1.377552in}}{\pgfqpoint{5.939838in}{1.369652in}}{\pgfqpoint{5.945662in}{1.363828in}}%
\pgfpathcurveto{\pgfqpoint{5.951486in}{1.358004in}}{\pgfqpoint{5.959386in}{1.354732in}}{\pgfqpoint{5.967622in}{1.354732in}}%
\pgfpathclose%
\pgfusepath{stroke,fill}%
\end{pgfscope}%
\begin{pgfscope}%
\pgfpathrectangle{\pgfqpoint{3.721492in}{0.557870in}}{\pgfqpoint{2.518508in}{1.684734in}}%
\pgfusepath{clip}%
\pgfsetbuttcap%
\pgfsetroundjoin%
\definecolor{currentfill}{rgb}{0.298039,0.447059,0.690196}%
\pgfsetfillcolor{currentfill}%
\pgfsetlinewidth{1.003750pt}%
\definecolor{currentstroke}{rgb}{0.298039,0.447059,0.690196}%
\pgfsetstrokecolor{currentstroke}%
\pgfsetdash{}{0pt}%
\pgfpathmoveto{\pgfqpoint{5.967622in}{1.325834in}}%
\pgfpathcurveto{\pgfqpoint{5.975858in}{1.325834in}}{\pgfqpoint{5.983758in}{1.329107in}}{\pgfqpoint{5.989582in}{1.334930in}}%
\pgfpathcurveto{\pgfqpoint{5.995406in}{1.340754in}}{\pgfqpoint{5.998679in}{1.348654in}}{\pgfqpoint{5.998679in}{1.356891in}}%
\pgfpathcurveto{\pgfqpoint{5.998679in}{1.365127in}}{\pgfqpoint{5.995406in}{1.373027in}}{\pgfqpoint{5.989582in}{1.378851in}}%
\pgfpathcurveto{\pgfqpoint{5.983758in}{1.384675in}}{\pgfqpoint{5.975858in}{1.387947in}}{\pgfqpoint{5.967622in}{1.387947in}}%
\pgfpathcurveto{\pgfqpoint{5.959386in}{1.387947in}}{\pgfqpoint{5.951486in}{1.384675in}}{\pgfqpoint{5.945662in}{1.378851in}}%
\pgfpathcurveto{\pgfqpoint{5.939838in}{1.373027in}}{\pgfqpoint{5.936566in}{1.365127in}}{\pgfqpoint{5.936566in}{1.356891in}}%
\pgfpathcurveto{\pgfqpoint{5.936566in}{1.348654in}}{\pgfqpoint{5.939838in}{1.340754in}}{\pgfqpoint{5.945662in}{1.334930in}}%
\pgfpathcurveto{\pgfqpoint{5.951486in}{1.329107in}}{\pgfqpoint{5.959386in}{1.325834in}}{\pgfqpoint{5.967622in}{1.325834in}}%
\pgfpathclose%
\pgfusepath{stroke,fill}%
\end{pgfscope}%
\begin{pgfscope}%
\pgfpathrectangle{\pgfqpoint{3.721492in}{0.557870in}}{\pgfqpoint{2.518508in}{1.684734in}}%
\pgfusepath{clip}%
\pgfsetbuttcap%
\pgfsetroundjoin%
\definecolor{currentfill}{rgb}{0.298039,0.447059,0.690196}%
\pgfsetfillcolor{currentfill}%
\pgfsetlinewidth{1.003750pt}%
\definecolor{currentstroke}{rgb}{0.298039,0.447059,0.690196}%
\pgfsetstrokecolor{currentstroke}%
\pgfsetdash{}{0pt}%
\pgfpathmoveto{\pgfqpoint{3.835969in}{2.085563in}}%
\pgfpathcurveto{\pgfqpoint{3.844206in}{2.085563in}}{\pgfqpoint{3.852106in}{2.088835in}}{\pgfqpoint{3.857930in}{2.094659in}}%
\pgfpathcurveto{\pgfqpoint{3.863754in}{2.100483in}}{\pgfqpoint{3.867026in}{2.108383in}}{\pgfqpoint{3.867026in}{2.116620in}}%
\pgfpathcurveto{\pgfqpoint{3.867026in}{2.124856in}}{\pgfqpoint{3.863754in}{2.132756in}}{\pgfqpoint{3.857930in}{2.138580in}}%
\pgfpathcurveto{\pgfqpoint{3.852106in}{2.144404in}}{\pgfqpoint{3.844206in}{2.147676in}}{\pgfqpoint{3.835969in}{2.147676in}}%
\pgfpathcurveto{\pgfqpoint{3.827733in}{2.147676in}}{\pgfqpoint{3.819833in}{2.144404in}}{\pgfqpoint{3.814009in}{2.138580in}}%
\pgfpathcurveto{\pgfqpoint{3.808185in}{2.132756in}}{\pgfqpoint{3.804913in}{2.124856in}}{\pgfqpoint{3.804913in}{2.116620in}}%
\pgfpathcurveto{\pgfqpoint{3.804913in}{2.108383in}}{\pgfqpoint{3.808185in}{2.100483in}}{\pgfqpoint{3.814009in}{2.094659in}}%
\pgfpathcurveto{\pgfqpoint{3.819833in}{2.088835in}}{\pgfqpoint{3.827733in}{2.085563in}}{\pgfqpoint{3.835969in}{2.085563in}}%
\pgfpathclose%
\pgfusepath{stroke,fill}%
\end{pgfscope}%
\begin{pgfscope}%
\pgfpathrectangle{\pgfqpoint{3.721492in}{0.557870in}}{\pgfqpoint{2.518508in}{1.684734in}}%
\pgfusepath{clip}%
\pgfsetbuttcap%
\pgfsetroundjoin%
\definecolor{currentfill}{rgb}{0.298039,0.447059,0.690196}%
\pgfsetfillcolor{currentfill}%
\pgfsetlinewidth{1.003750pt}%
\definecolor{currentstroke}{rgb}{0.298039,0.447059,0.690196}%
\pgfsetstrokecolor{currentstroke}%
\pgfsetdash{}{0pt}%
\pgfpathmoveto{\pgfqpoint{3.993870in}{2.115207in}}%
\pgfpathcurveto{\pgfqpoint{4.002106in}{2.115207in}}{\pgfqpoint{4.010006in}{2.118479in}}{\pgfqpoint{4.015830in}{2.124303in}}%
\pgfpathcurveto{\pgfqpoint{4.021654in}{2.130127in}}{\pgfqpoint{4.024926in}{2.138027in}}{\pgfqpoint{4.024926in}{2.146263in}}%
\pgfpathcurveto{\pgfqpoint{4.024926in}{2.154499in}}{\pgfqpoint{4.021654in}{2.162399in}}{\pgfqpoint{4.015830in}{2.168223in}}%
\pgfpathcurveto{\pgfqpoint{4.010006in}{2.174047in}}{\pgfqpoint{4.002106in}{2.177320in}}{\pgfqpoint{3.993870in}{2.177320in}}%
\pgfpathcurveto{\pgfqpoint{3.985633in}{2.177320in}}{\pgfqpoint{3.977733in}{2.174047in}}{\pgfqpoint{3.971909in}{2.168223in}}%
\pgfpathcurveto{\pgfqpoint{3.966085in}{2.162399in}}{\pgfqpoint{3.962813in}{2.154499in}}{\pgfqpoint{3.962813in}{2.146263in}}%
\pgfpathcurveto{\pgfqpoint{3.962813in}{2.138027in}}{\pgfqpoint{3.966085in}{2.130127in}}{\pgfqpoint{3.971909in}{2.124303in}}%
\pgfpathcurveto{\pgfqpoint{3.977733in}{2.118479in}}{\pgfqpoint{3.985633in}{2.115207in}}{\pgfqpoint{3.993870in}{2.115207in}}%
\pgfpathclose%
\pgfusepath{stroke,fill}%
\end{pgfscope}%
\begin{pgfscope}%
\pgfpathrectangle{\pgfqpoint{3.721492in}{0.557870in}}{\pgfqpoint{2.518508in}{1.684734in}}%
\pgfusepath{clip}%
\pgfsetbuttcap%
\pgfsetroundjoin%
\definecolor{currentfill}{rgb}{0.298039,0.447059,0.690196}%
\pgfsetfillcolor{currentfill}%
\pgfsetlinewidth{1.003750pt}%
\definecolor{currentstroke}{rgb}{0.298039,0.447059,0.690196}%
\pgfsetstrokecolor{currentstroke}%
\pgfsetdash{}{0pt}%
\pgfpathmoveto{\pgfqpoint{5.967622in}{1.334597in}}%
\pgfpathcurveto{\pgfqpoint{5.975858in}{1.334597in}}{\pgfqpoint{5.983758in}{1.337869in}}{\pgfqpoint{5.989582in}{1.343693in}}%
\pgfpathcurveto{\pgfqpoint{5.995406in}{1.349517in}}{\pgfqpoint{5.998679in}{1.357417in}}{\pgfqpoint{5.998679in}{1.365653in}}%
\pgfpathcurveto{\pgfqpoint{5.998679in}{1.373890in}}{\pgfqpoint{5.995406in}{1.381790in}}{\pgfqpoint{5.989582in}{1.387614in}}%
\pgfpathcurveto{\pgfqpoint{5.983758in}{1.393437in}}{\pgfqpoint{5.975858in}{1.396710in}}{\pgfqpoint{5.967622in}{1.396710in}}%
\pgfpathcurveto{\pgfqpoint{5.959386in}{1.396710in}}{\pgfqpoint{5.951486in}{1.393437in}}{\pgfqpoint{5.945662in}{1.387614in}}%
\pgfpathcurveto{\pgfqpoint{5.939838in}{1.381790in}}{\pgfqpoint{5.936566in}{1.373890in}}{\pgfqpoint{5.936566in}{1.365653in}}%
\pgfpathcurveto{\pgfqpoint{5.936566in}{1.357417in}}{\pgfqpoint{5.939838in}{1.349517in}}{\pgfqpoint{5.945662in}{1.343693in}}%
\pgfpathcurveto{\pgfqpoint{5.951486in}{1.337869in}}{\pgfqpoint{5.959386in}{1.334597in}}{\pgfqpoint{5.967622in}{1.334597in}}%
\pgfpathclose%
\pgfusepath{stroke,fill}%
\end{pgfscope}%
\begin{pgfscope}%
\pgfpathrectangle{\pgfqpoint{3.721492in}{0.557870in}}{\pgfqpoint{2.518508in}{1.684734in}}%
\pgfusepath{clip}%
\pgfsetbuttcap%
\pgfsetroundjoin%
\definecolor{currentfill}{rgb}{0.298039,0.447059,0.690196}%
\pgfsetfillcolor{currentfill}%
\pgfsetlinewidth{1.003750pt}%
\definecolor{currentstroke}{rgb}{0.298039,0.447059,0.690196}%
\pgfsetstrokecolor{currentstroke}%
\pgfsetdash{}{0pt}%
\pgfpathmoveto{\pgfqpoint{3.835969in}{2.125573in}}%
\pgfpathcurveto{\pgfqpoint{3.844206in}{2.125573in}}{\pgfqpoint{3.852106in}{2.128845in}}{\pgfqpoint{3.857930in}{2.134669in}}%
\pgfpathcurveto{\pgfqpoint{3.863754in}{2.140493in}}{\pgfqpoint{3.867026in}{2.148393in}}{\pgfqpoint{3.867026in}{2.156629in}}%
\pgfpathcurveto{\pgfqpoint{3.867026in}{2.164865in}}{\pgfqpoint{3.863754in}{2.172766in}}{\pgfqpoint{3.857930in}{2.178589in}}%
\pgfpathcurveto{\pgfqpoint{3.852106in}{2.184413in}}{\pgfqpoint{3.844206in}{2.187686in}}{\pgfqpoint{3.835969in}{2.187686in}}%
\pgfpathcurveto{\pgfqpoint{3.827733in}{2.187686in}}{\pgfqpoint{3.819833in}{2.184413in}}{\pgfqpoint{3.814009in}{2.178589in}}%
\pgfpathcurveto{\pgfqpoint{3.808185in}{2.172766in}}{\pgfqpoint{3.804913in}{2.164865in}}{\pgfqpoint{3.804913in}{2.156629in}}%
\pgfpathcurveto{\pgfqpoint{3.804913in}{2.148393in}}{\pgfqpoint{3.808185in}{2.140493in}}{\pgfqpoint{3.814009in}{2.134669in}}%
\pgfpathcurveto{\pgfqpoint{3.819833in}{2.128845in}}{\pgfqpoint{3.827733in}{2.125573in}}{\pgfqpoint{3.835969in}{2.125573in}}%
\pgfpathclose%
\pgfusepath{stroke,fill}%
\end{pgfscope}%
\begin{pgfscope}%
\pgfpathrectangle{\pgfqpoint{3.721492in}{0.557870in}}{\pgfqpoint{2.518508in}{1.684734in}}%
\pgfusepath{clip}%
\pgfsetbuttcap%
\pgfsetroundjoin%
\definecolor{currentfill}{rgb}{0.298039,0.447059,0.690196}%
\pgfsetfillcolor{currentfill}%
\pgfsetlinewidth{1.003750pt}%
\definecolor{currentstroke}{rgb}{0.298039,0.447059,0.690196}%
\pgfsetstrokecolor{currentstroke}%
\pgfsetdash{}{0pt}%
\pgfpathmoveto{\pgfqpoint{5.982425in}{1.571198in}}%
\pgfpathcurveto{\pgfqpoint{5.990662in}{1.571198in}}{\pgfqpoint{5.998562in}{1.574471in}}{\pgfqpoint{6.004386in}{1.580295in}}%
\pgfpathcurveto{\pgfqpoint{6.010209in}{1.586119in}}{\pgfqpoint{6.013482in}{1.594019in}}{\pgfqpoint{6.013482in}{1.602255in}}%
\pgfpathcurveto{\pgfqpoint{6.013482in}{1.610491in}}{\pgfqpoint{6.010209in}{1.618391in}}{\pgfqpoint{6.004386in}{1.624215in}}%
\pgfpathcurveto{\pgfqpoint{5.998562in}{1.630039in}}{\pgfqpoint{5.990662in}{1.633311in}}{\pgfqpoint{5.982425in}{1.633311in}}%
\pgfpathcurveto{\pgfqpoint{5.974189in}{1.633311in}}{\pgfqpoint{5.966289in}{1.630039in}}{\pgfqpoint{5.960465in}{1.624215in}}%
\pgfpathcurveto{\pgfqpoint{5.954641in}{1.618391in}}{\pgfqpoint{5.951369in}{1.610491in}}{\pgfqpoint{5.951369in}{1.602255in}}%
\pgfpathcurveto{\pgfqpoint{5.951369in}{1.594019in}}{\pgfqpoint{5.954641in}{1.586119in}}{\pgfqpoint{5.960465in}{1.580295in}}%
\pgfpathcurveto{\pgfqpoint{5.966289in}{1.574471in}}{\pgfqpoint{5.974189in}{1.571198in}}{\pgfqpoint{5.982425in}{1.571198in}}%
\pgfpathclose%
\pgfusepath{stroke,fill}%
\end{pgfscope}%
\begin{pgfscope}%
\pgfpathrectangle{\pgfqpoint{3.721492in}{0.557870in}}{\pgfqpoint{2.518508in}{1.684734in}}%
\pgfusepath{clip}%
\pgfsetbuttcap%
\pgfsetroundjoin%
\definecolor{currentfill}{rgb}{0.298039,0.447059,0.690196}%
\pgfsetfillcolor{currentfill}%
\pgfsetlinewidth{1.003750pt}%
\definecolor{currentstroke}{rgb}{0.298039,0.447059,0.690196}%
\pgfsetstrokecolor{currentstroke}%
\pgfsetdash{}{0pt}%
\pgfpathmoveto{\pgfqpoint{5.982425in}{1.674556in}}%
\pgfpathcurveto{\pgfqpoint{5.990662in}{1.674556in}}{\pgfqpoint{5.998562in}{1.677829in}}{\pgfqpoint{6.004386in}{1.683653in}}%
\pgfpathcurveto{\pgfqpoint{6.010209in}{1.689477in}}{\pgfqpoint{6.013482in}{1.697377in}}{\pgfqpoint{6.013482in}{1.705613in}}%
\pgfpathcurveto{\pgfqpoint{6.013482in}{1.713849in}}{\pgfqpoint{6.010209in}{1.721749in}}{\pgfqpoint{6.004386in}{1.727573in}}%
\pgfpathcurveto{\pgfqpoint{5.998562in}{1.733397in}}{\pgfqpoint{5.990662in}{1.736669in}}{\pgfqpoint{5.982425in}{1.736669in}}%
\pgfpathcurveto{\pgfqpoint{5.974189in}{1.736669in}}{\pgfqpoint{5.966289in}{1.733397in}}{\pgfqpoint{5.960465in}{1.727573in}}%
\pgfpathcurveto{\pgfqpoint{5.954641in}{1.721749in}}{\pgfqpoint{5.951369in}{1.713849in}}{\pgfqpoint{5.951369in}{1.705613in}}%
\pgfpathcurveto{\pgfqpoint{5.951369in}{1.697377in}}{\pgfqpoint{5.954641in}{1.689477in}}{\pgfqpoint{5.960465in}{1.683653in}}%
\pgfpathcurveto{\pgfqpoint{5.966289in}{1.677829in}}{\pgfqpoint{5.974189in}{1.674556in}}{\pgfqpoint{5.982425in}{1.674556in}}%
\pgfpathclose%
\pgfusepath{stroke,fill}%
\end{pgfscope}%
\begin{pgfscope}%
\pgfpathrectangle{\pgfqpoint{3.721492in}{0.557870in}}{\pgfqpoint{2.518508in}{1.684734in}}%
\pgfusepath{clip}%
\pgfsetbuttcap%
\pgfsetroundjoin%
\definecolor{currentfill}{rgb}{0.298039,0.447059,0.690196}%
\pgfsetfillcolor{currentfill}%
\pgfsetlinewidth{1.003750pt}%
\definecolor{currentstroke}{rgb}{0.298039,0.447059,0.690196}%
\pgfsetstrokecolor{currentstroke}%
\pgfsetdash{}{0pt}%
\pgfpathmoveto{\pgfqpoint{5.982425in}{1.712141in}}%
\pgfpathcurveto{\pgfqpoint{5.990662in}{1.712141in}}{\pgfqpoint{5.998562in}{1.715413in}}{\pgfqpoint{6.004386in}{1.721237in}}%
\pgfpathcurveto{\pgfqpoint{6.010209in}{1.727061in}}{\pgfqpoint{6.013482in}{1.734961in}}{\pgfqpoint{6.013482in}{1.743198in}}%
\pgfpathcurveto{\pgfqpoint{6.013482in}{1.751434in}}{\pgfqpoint{6.010209in}{1.759334in}}{\pgfqpoint{6.004386in}{1.765158in}}%
\pgfpathcurveto{\pgfqpoint{5.998562in}{1.770982in}}{\pgfqpoint{5.990662in}{1.774254in}}{\pgfqpoint{5.982425in}{1.774254in}}%
\pgfpathcurveto{\pgfqpoint{5.974189in}{1.774254in}}{\pgfqpoint{5.966289in}{1.770982in}}{\pgfqpoint{5.960465in}{1.765158in}}%
\pgfpathcurveto{\pgfqpoint{5.954641in}{1.759334in}}{\pgfqpoint{5.951369in}{1.751434in}}{\pgfqpoint{5.951369in}{1.743198in}}%
\pgfpathcurveto{\pgfqpoint{5.951369in}{1.734961in}}{\pgfqpoint{5.954641in}{1.727061in}}{\pgfqpoint{5.960465in}{1.721237in}}%
\pgfpathcurveto{\pgfqpoint{5.966289in}{1.715413in}}{\pgfqpoint{5.974189in}{1.712141in}}{\pgfqpoint{5.982425in}{1.712141in}}%
\pgfpathclose%
\pgfusepath{stroke,fill}%
\end{pgfscope}%
\begin{pgfscope}%
\pgfpathrectangle{\pgfqpoint{3.721492in}{0.557870in}}{\pgfqpoint{2.518508in}{1.684734in}}%
\pgfusepath{clip}%
\pgfsetbuttcap%
\pgfsetroundjoin%
\definecolor{currentfill}{rgb}{0.298039,0.447059,0.690196}%
\pgfsetfillcolor{currentfill}%
\pgfsetlinewidth{1.003750pt}%
\definecolor{currentstroke}{rgb}{0.298039,0.447059,0.690196}%
\pgfsetstrokecolor{currentstroke}%
\pgfsetdash{}{0pt}%
\pgfpathmoveto{\pgfqpoint{3.835969in}{2.125336in}}%
\pgfpathcurveto{\pgfqpoint{3.844206in}{2.125336in}}{\pgfqpoint{3.852106in}{2.128609in}}{\pgfqpoint{3.857930in}{2.134433in}}%
\pgfpathcurveto{\pgfqpoint{3.863754in}{2.140256in}}{\pgfqpoint{3.867026in}{2.148157in}}{\pgfqpoint{3.867026in}{2.156393in}}%
\pgfpathcurveto{\pgfqpoint{3.867026in}{2.164629in}}{\pgfqpoint{3.863754in}{2.172529in}}{\pgfqpoint{3.857930in}{2.178353in}}%
\pgfpathcurveto{\pgfqpoint{3.852106in}{2.184177in}}{\pgfqpoint{3.844206in}{2.187449in}}{\pgfqpoint{3.835969in}{2.187449in}}%
\pgfpathcurveto{\pgfqpoint{3.827733in}{2.187449in}}{\pgfqpoint{3.819833in}{2.184177in}}{\pgfqpoint{3.814009in}{2.178353in}}%
\pgfpathcurveto{\pgfqpoint{3.808185in}{2.172529in}}{\pgfqpoint{3.804913in}{2.164629in}}{\pgfqpoint{3.804913in}{2.156393in}}%
\pgfpathcurveto{\pgfqpoint{3.804913in}{2.148157in}}{\pgfqpoint{3.808185in}{2.140256in}}{\pgfqpoint{3.814009in}{2.134433in}}%
\pgfpathcurveto{\pgfqpoint{3.819833in}{2.128609in}}{\pgfqpoint{3.827733in}{2.125336in}}{\pgfqpoint{3.835969in}{2.125336in}}%
\pgfpathclose%
\pgfusepath{stroke,fill}%
\end{pgfscope}%
\begin{pgfscope}%
\pgfpathrectangle{\pgfqpoint{3.721492in}{0.557870in}}{\pgfqpoint{2.518508in}{1.684734in}}%
\pgfusepath{clip}%
\pgfsetbuttcap%
\pgfsetroundjoin%
\definecolor{currentfill}{rgb}{0.298039,0.447059,0.690196}%
\pgfsetfillcolor{currentfill}%
\pgfsetlinewidth{1.003750pt}%
\definecolor{currentstroke}{rgb}{0.298039,0.447059,0.690196}%
\pgfsetstrokecolor{currentstroke}%
\pgfsetdash{}{0pt}%
\pgfpathmoveto{\pgfqpoint{5.967622in}{1.730402in}}%
\pgfpathcurveto{\pgfqpoint{5.975858in}{1.730402in}}{\pgfqpoint{5.983758in}{1.733674in}}{\pgfqpoint{5.989582in}{1.739498in}}%
\pgfpathcurveto{\pgfqpoint{5.995406in}{1.745322in}}{\pgfqpoint{5.998679in}{1.753222in}}{\pgfqpoint{5.998679in}{1.761458in}}%
\pgfpathcurveto{\pgfqpoint{5.998679in}{1.769694in}}{\pgfqpoint{5.995406in}{1.777594in}}{\pgfqpoint{5.989582in}{1.783418in}}%
\pgfpathcurveto{\pgfqpoint{5.983758in}{1.789242in}}{\pgfqpoint{5.975858in}{1.792515in}}{\pgfqpoint{5.967622in}{1.792515in}}%
\pgfpathcurveto{\pgfqpoint{5.959386in}{1.792515in}}{\pgfqpoint{5.951486in}{1.789242in}}{\pgfqpoint{5.945662in}{1.783418in}}%
\pgfpathcurveto{\pgfqpoint{5.939838in}{1.777594in}}{\pgfqpoint{5.936566in}{1.769694in}}{\pgfqpoint{5.936566in}{1.761458in}}%
\pgfpathcurveto{\pgfqpoint{5.936566in}{1.753222in}}{\pgfqpoint{5.939838in}{1.745322in}}{\pgfqpoint{5.945662in}{1.739498in}}%
\pgfpathcurveto{\pgfqpoint{5.951486in}{1.733674in}}{\pgfqpoint{5.959386in}{1.730402in}}{\pgfqpoint{5.967622in}{1.730402in}}%
\pgfpathclose%
\pgfusepath{stroke,fill}%
\end{pgfscope}%
\begin{pgfscope}%
\pgfpathrectangle{\pgfqpoint{3.721492in}{0.557870in}}{\pgfqpoint{2.518508in}{1.684734in}}%
\pgfusepath{clip}%
\pgfsetbuttcap%
\pgfsetroundjoin%
\definecolor{currentfill}{rgb}{0.298039,0.447059,0.690196}%
\pgfsetfillcolor{currentfill}%
\pgfsetlinewidth{1.003750pt}%
\definecolor{currentstroke}{rgb}{0.298039,0.447059,0.690196}%
\pgfsetstrokecolor{currentstroke}%
\pgfsetdash{}{0pt}%
\pgfpathmoveto{\pgfqpoint{3.835969in}{2.125336in}}%
\pgfpathcurveto{\pgfqpoint{3.844206in}{2.125336in}}{\pgfqpoint{3.852106in}{2.128609in}}{\pgfqpoint{3.857930in}{2.134433in}}%
\pgfpathcurveto{\pgfqpoint{3.863754in}{2.140256in}}{\pgfqpoint{3.867026in}{2.148157in}}{\pgfqpoint{3.867026in}{2.156393in}}%
\pgfpathcurveto{\pgfqpoint{3.867026in}{2.164629in}}{\pgfqpoint{3.863754in}{2.172529in}}{\pgfqpoint{3.857930in}{2.178353in}}%
\pgfpathcurveto{\pgfqpoint{3.852106in}{2.184177in}}{\pgfqpoint{3.844206in}{2.187449in}}{\pgfqpoint{3.835969in}{2.187449in}}%
\pgfpathcurveto{\pgfqpoint{3.827733in}{2.187449in}}{\pgfqpoint{3.819833in}{2.184177in}}{\pgfqpoint{3.814009in}{2.178353in}}%
\pgfpathcurveto{\pgfqpoint{3.808185in}{2.172529in}}{\pgfqpoint{3.804913in}{2.164629in}}{\pgfqpoint{3.804913in}{2.156393in}}%
\pgfpathcurveto{\pgfqpoint{3.804913in}{2.148157in}}{\pgfqpoint{3.808185in}{2.140256in}}{\pgfqpoint{3.814009in}{2.134433in}}%
\pgfpathcurveto{\pgfqpoint{3.819833in}{2.128609in}}{\pgfqpoint{3.827733in}{2.125336in}}{\pgfqpoint{3.835969in}{2.125336in}}%
\pgfpathclose%
\pgfusepath{stroke,fill}%
\end{pgfscope}%
\begin{pgfscope}%
\pgfpathrectangle{\pgfqpoint{3.721492in}{0.557870in}}{\pgfqpoint{2.518508in}{1.684734in}}%
\pgfusepath{clip}%
\pgfsetbuttcap%
\pgfsetroundjoin%
\definecolor{currentfill}{rgb}{0.298039,0.447059,0.690196}%
\pgfsetfillcolor{currentfill}%
\pgfsetlinewidth{1.003750pt}%
\definecolor{currentstroke}{rgb}{0.298039,0.447059,0.690196}%
\pgfsetstrokecolor{currentstroke}%
\pgfsetdash{}{0pt}%
\pgfpathmoveto{\pgfqpoint{5.967622in}{1.306569in}}%
\pgfpathcurveto{\pgfqpoint{5.975858in}{1.306569in}}{\pgfqpoint{5.983758in}{1.309841in}}{\pgfqpoint{5.989582in}{1.315665in}}%
\pgfpathcurveto{\pgfqpoint{5.995406in}{1.321489in}}{\pgfqpoint{5.998679in}{1.329389in}}{\pgfqpoint{5.998679in}{1.337626in}}%
\pgfpathcurveto{\pgfqpoint{5.998679in}{1.345862in}}{\pgfqpoint{5.995406in}{1.353762in}}{\pgfqpoint{5.989582in}{1.359586in}}%
\pgfpathcurveto{\pgfqpoint{5.983758in}{1.365410in}}{\pgfqpoint{5.975858in}{1.368682in}}{\pgfqpoint{5.967622in}{1.368682in}}%
\pgfpathcurveto{\pgfqpoint{5.959386in}{1.368682in}}{\pgfqpoint{5.951486in}{1.365410in}}{\pgfqpoint{5.945662in}{1.359586in}}%
\pgfpathcurveto{\pgfqpoint{5.939838in}{1.353762in}}{\pgfqpoint{5.936566in}{1.345862in}}{\pgfqpoint{5.936566in}{1.337626in}}%
\pgfpathcurveto{\pgfqpoint{5.936566in}{1.329389in}}{\pgfqpoint{5.939838in}{1.321489in}}{\pgfqpoint{5.945662in}{1.315665in}}%
\pgfpathcurveto{\pgfqpoint{5.951486in}{1.309841in}}{\pgfqpoint{5.959386in}{1.306569in}}{\pgfqpoint{5.967622in}{1.306569in}}%
\pgfpathclose%
\pgfusepath{stroke,fill}%
\end{pgfscope}%
\begin{pgfscope}%
\pgfpathrectangle{\pgfqpoint{3.721492in}{0.557870in}}{\pgfqpoint{2.518508in}{1.684734in}}%
\pgfusepath{clip}%
\pgfsetbuttcap%
\pgfsetroundjoin%
\definecolor{currentfill}{rgb}{0.298039,0.447059,0.690196}%
\pgfsetfillcolor{currentfill}%
\pgfsetlinewidth{1.003750pt}%
\definecolor{currentstroke}{rgb}{0.298039,0.447059,0.690196}%
\pgfsetstrokecolor{currentstroke}%
\pgfsetdash{}{0pt}%
\pgfpathmoveto{\pgfqpoint{3.835969in}{2.125088in}}%
\pgfpathcurveto{\pgfqpoint{3.844206in}{2.125088in}}{\pgfqpoint{3.852106in}{2.128360in}}{\pgfqpoint{3.857930in}{2.134184in}}%
\pgfpathcurveto{\pgfqpoint{3.863754in}{2.140008in}}{\pgfqpoint{3.867026in}{2.147908in}}{\pgfqpoint{3.867026in}{2.156144in}}%
\pgfpathcurveto{\pgfqpoint{3.867026in}{2.164380in}}{\pgfqpoint{3.863754in}{2.172281in}}{\pgfqpoint{3.857930in}{2.178104in}}%
\pgfpathcurveto{\pgfqpoint{3.852106in}{2.183928in}}{\pgfqpoint{3.844206in}{2.187201in}}{\pgfqpoint{3.835969in}{2.187201in}}%
\pgfpathcurveto{\pgfqpoint{3.827733in}{2.187201in}}{\pgfqpoint{3.819833in}{2.183928in}}{\pgfqpoint{3.814009in}{2.178104in}}%
\pgfpathcurveto{\pgfqpoint{3.808185in}{2.172281in}}{\pgfqpoint{3.804913in}{2.164380in}}{\pgfqpoint{3.804913in}{2.156144in}}%
\pgfpathcurveto{\pgfqpoint{3.804913in}{2.147908in}}{\pgfqpoint{3.808185in}{2.140008in}}{\pgfqpoint{3.814009in}{2.134184in}}%
\pgfpathcurveto{\pgfqpoint{3.819833in}{2.128360in}}{\pgfqpoint{3.827733in}{2.125088in}}{\pgfqpoint{3.835969in}{2.125088in}}%
\pgfpathclose%
\pgfusepath{stroke,fill}%
\end{pgfscope}%
\begin{pgfscope}%
\pgfpathrectangle{\pgfqpoint{3.721492in}{0.557870in}}{\pgfqpoint{2.518508in}{1.684734in}}%
\pgfusepath{clip}%
\pgfsetbuttcap%
\pgfsetroundjoin%
\definecolor{currentfill}{rgb}{0.298039,0.447059,0.690196}%
\pgfsetfillcolor{currentfill}%
\pgfsetlinewidth{1.003750pt}%
\definecolor{currentstroke}{rgb}{0.298039,0.447059,0.690196}%
\pgfsetstrokecolor{currentstroke}%
\pgfsetdash{}{0pt}%
\pgfpathmoveto{\pgfqpoint{5.967622in}{1.561863in}}%
\pgfpathcurveto{\pgfqpoint{5.975858in}{1.561863in}}{\pgfqpoint{5.983758in}{1.565135in}}{\pgfqpoint{5.989582in}{1.570959in}}%
\pgfpathcurveto{\pgfqpoint{5.995406in}{1.576783in}}{\pgfqpoint{5.998679in}{1.584683in}}{\pgfqpoint{5.998679in}{1.592919in}}%
\pgfpathcurveto{\pgfqpoint{5.998679in}{1.601156in}}{\pgfqpoint{5.995406in}{1.609056in}}{\pgfqpoint{5.989582in}{1.614880in}}%
\pgfpathcurveto{\pgfqpoint{5.983758in}{1.620704in}}{\pgfqpoint{5.975858in}{1.623976in}}{\pgfqpoint{5.967622in}{1.623976in}}%
\pgfpathcurveto{\pgfqpoint{5.959386in}{1.623976in}}{\pgfqpoint{5.951486in}{1.620704in}}{\pgfqpoint{5.945662in}{1.614880in}}%
\pgfpathcurveto{\pgfqpoint{5.939838in}{1.609056in}}{\pgfqpoint{5.936566in}{1.601156in}}{\pgfqpoint{5.936566in}{1.592919in}}%
\pgfpathcurveto{\pgfqpoint{5.936566in}{1.584683in}}{\pgfqpoint{5.939838in}{1.576783in}}{\pgfqpoint{5.945662in}{1.570959in}}%
\pgfpathcurveto{\pgfqpoint{5.951486in}{1.565135in}}{\pgfqpoint{5.959386in}{1.561863in}}{\pgfqpoint{5.967622in}{1.561863in}}%
\pgfpathclose%
\pgfusepath{stroke,fill}%
\end{pgfscope}%
\begin{pgfscope}%
\pgfpathrectangle{\pgfqpoint{3.721492in}{0.557870in}}{\pgfqpoint{2.518508in}{1.684734in}}%
\pgfusepath{clip}%
\pgfsetbuttcap%
\pgfsetroundjoin%
\definecolor{currentfill}{rgb}{0.298039,0.447059,0.690196}%
\pgfsetfillcolor{currentfill}%
\pgfsetlinewidth{1.003750pt}%
\definecolor{currentstroke}{rgb}{0.298039,0.447059,0.690196}%
\pgfsetstrokecolor{currentstroke}%
\pgfsetdash{}{0pt}%
\pgfpathmoveto{\pgfqpoint{3.835969in}{2.125088in}}%
\pgfpathcurveto{\pgfqpoint{3.844206in}{2.125088in}}{\pgfqpoint{3.852106in}{2.128360in}}{\pgfqpoint{3.857930in}{2.134184in}}%
\pgfpathcurveto{\pgfqpoint{3.863754in}{2.140008in}}{\pgfqpoint{3.867026in}{2.147908in}}{\pgfqpoint{3.867026in}{2.156144in}}%
\pgfpathcurveto{\pgfqpoint{3.867026in}{2.164380in}}{\pgfqpoint{3.863754in}{2.172281in}}{\pgfqpoint{3.857930in}{2.178104in}}%
\pgfpathcurveto{\pgfqpoint{3.852106in}{2.183928in}}{\pgfqpoint{3.844206in}{2.187201in}}{\pgfqpoint{3.835969in}{2.187201in}}%
\pgfpathcurveto{\pgfqpoint{3.827733in}{2.187201in}}{\pgfqpoint{3.819833in}{2.183928in}}{\pgfqpoint{3.814009in}{2.178104in}}%
\pgfpathcurveto{\pgfqpoint{3.808185in}{2.172281in}}{\pgfqpoint{3.804913in}{2.164380in}}{\pgfqpoint{3.804913in}{2.156144in}}%
\pgfpathcurveto{\pgfqpoint{3.804913in}{2.147908in}}{\pgfqpoint{3.808185in}{2.140008in}}{\pgfqpoint{3.814009in}{2.134184in}}%
\pgfpathcurveto{\pgfqpoint{3.819833in}{2.128360in}}{\pgfqpoint{3.827733in}{2.125088in}}{\pgfqpoint{3.835969in}{2.125088in}}%
\pgfpathclose%
\pgfusepath{stroke,fill}%
\end{pgfscope}%
\begin{pgfscope}%
\pgfpathrectangle{\pgfqpoint{3.721492in}{0.557870in}}{\pgfqpoint{2.518508in}{1.684734in}}%
\pgfusepath{clip}%
\pgfsetbuttcap%
\pgfsetroundjoin%
\definecolor{currentfill}{rgb}{0.298039,0.447059,0.690196}%
\pgfsetfillcolor{currentfill}%
\pgfsetlinewidth{1.003750pt}%
\definecolor{currentstroke}{rgb}{0.298039,0.447059,0.690196}%
\pgfsetstrokecolor{currentstroke}%
\pgfsetdash{}{0pt}%
\pgfpathmoveto{\pgfqpoint{5.967622in}{1.710080in}}%
\pgfpathcurveto{\pgfqpoint{5.975858in}{1.710080in}}{\pgfqpoint{5.983758in}{1.713352in}}{\pgfqpoint{5.989582in}{1.719176in}}%
\pgfpathcurveto{\pgfqpoint{5.995406in}{1.725000in}}{\pgfqpoint{5.998679in}{1.732900in}}{\pgfqpoint{5.998679in}{1.741136in}}%
\pgfpathcurveto{\pgfqpoint{5.998679in}{1.749373in}}{\pgfqpoint{5.995406in}{1.757273in}}{\pgfqpoint{5.989582in}{1.763097in}}%
\pgfpathcurveto{\pgfqpoint{5.983758in}{1.768921in}}{\pgfqpoint{5.975858in}{1.772193in}}{\pgfqpoint{5.967622in}{1.772193in}}%
\pgfpathcurveto{\pgfqpoint{5.959386in}{1.772193in}}{\pgfqpoint{5.951486in}{1.768921in}}{\pgfqpoint{5.945662in}{1.763097in}}%
\pgfpathcurveto{\pgfqpoint{5.939838in}{1.757273in}}{\pgfqpoint{5.936566in}{1.749373in}}{\pgfqpoint{5.936566in}{1.741136in}}%
\pgfpathcurveto{\pgfqpoint{5.936566in}{1.732900in}}{\pgfqpoint{5.939838in}{1.725000in}}{\pgfqpoint{5.945662in}{1.719176in}}%
\pgfpathcurveto{\pgfqpoint{5.951486in}{1.713352in}}{\pgfqpoint{5.959386in}{1.710080in}}{\pgfqpoint{5.967622in}{1.710080in}}%
\pgfpathclose%
\pgfusepath{stroke,fill}%
\end{pgfscope}%
\begin{pgfscope}%
\pgfsetrectcap%
\pgfsetmiterjoin%
\pgfsetlinewidth{1.254687pt}%
\definecolor{currentstroke}{rgb}{1.000000,1.000000,1.000000}%
\pgfsetstrokecolor{currentstroke}%
\pgfsetdash{}{0pt}%
\pgfpathmoveto{\pgfqpoint{3.721492in}{0.557870in}}%
\pgfpathlineto{\pgfqpoint{3.721492in}{2.242604in}}%
\pgfusepath{stroke}%
\end{pgfscope}%
\begin{pgfscope}%
\pgfsetrectcap%
\pgfsetmiterjoin%
\pgfsetlinewidth{1.254687pt}%
\definecolor{currentstroke}{rgb}{1.000000,1.000000,1.000000}%
\pgfsetstrokecolor{currentstroke}%
\pgfsetdash{}{0pt}%
\pgfpathmoveto{\pgfqpoint{6.240000in}{0.557870in}}%
\pgfpathlineto{\pgfqpoint{6.240000in}{2.242604in}}%
\pgfusepath{stroke}%
\end{pgfscope}%
\begin{pgfscope}%
\pgfsetrectcap%
\pgfsetmiterjoin%
\pgfsetlinewidth{1.254687pt}%
\definecolor{currentstroke}{rgb}{1.000000,1.000000,1.000000}%
\pgfsetstrokecolor{currentstroke}%
\pgfsetdash{}{0pt}%
\pgfpathmoveto{\pgfqpoint{3.721492in}{0.557870in}}%
\pgfpathlineto{\pgfqpoint{6.240000in}{0.557870in}}%
\pgfusepath{stroke}%
\end{pgfscope}%
\begin{pgfscope}%
\pgfsetrectcap%
\pgfsetmiterjoin%
\pgfsetlinewidth{1.254687pt}%
\definecolor{currentstroke}{rgb}{1.000000,1.000000,1.000000}%
\pgfsetstrokecolor{currentstroke}%
\pgfsetdash{}{0pt}%
\pgfpathmoveto{\pgfqpoint{3.721492in}{2.242604in}}%
\pgfpathlineto{\pgfqpoint{6.240000in}{2.242604in}}%
\pgfusepath{stroke}%
\end{pgfscope}%
\begin{pgfscope}%
\definecolor{textcolor}{rgb}{0.150000,0.150000,0.150000}%
\pgfsetstrokecolor{textcolor}%
\pgfsetfillcolor{textcolor}%
\pgftext[x=4.980746in,y=2.325938in,,base]{\color{textcolor}\sffamily\fontsize{11.000000}{13.200000}\selectfont (b)}%
\end{pgfscope}%
\end{pgfpicture}%
\makeatother%
\endgroup%

    \caption{Distribution of DOR, sensitivity and specificity for the different PVC methods when classifying patient diagnosis.}
    \label{fig:pvc_ind_dor_sens_spec_dist}
\end{figure}

\begin{comment}
[ ] \textbf{Comment on spread of DOR.}
    * From the distribution plot in figure \ref{fig:pvc_ind_dor_sens_spec_dist}a one can see that the majority of the PVC methods get DORs close to zero,
      but there are a few methods that attain DORs above 30, and close to 40. 
[ ] \textbf{Comment on spread of sensitivity and specificity.}
    * From the scatterplot in \ref{fig:pvc_ind_dor_sens_spec_dist}b one can see that almost all the sensitivity scores are above $0.5$, while the specificity scores are concentrated
      in the areas $0$ to $0.25$ and $0.95$.
[ ] \textbf{Comment on common traits in the high performing methods.} Here you can refer to raw performance results in appendix.
    * As with the heart failure case study the PVC methods that perform high in terms of DOR use a dataset that is a combination of peak systolic strain values and EF.
[ ] \textbf{Comment on common traits in the low performing methods.} Here you can refer to raw performance results in appendix.
[ ] \textbf{Select one - three methods that are good contendors for being the best method/model in the group and comment on their traits}
    * From table \ref{tab:pvc_ind_dor_sens_spec_dist} one can see that \textit{gls-EF/ward/2} and \textit{rls-EF/complete/2} are the two top performers in terms of DOR. 
    * \textit{gls-EF/ward/2} achieves a slightly higher specificity score, where as \textit{rls-EF/complete/2} attains a slightly higher specificity score.
\textbf{IF NOT CLUSTERING METHOD}
[NA] \textbf{Make arguments for and against the top three methods in terms of accuracy, sensitivity, specificity, and DOR, and make an informed choice.}
\end{comment}

\begin{table*}
    \centering
    \ra{1.3}
    \begin{tabular}{lrrrr}
        \toprule
        Dataset-Method    &  Accuracy &  Sensitivity &  Specificity &   DOR \\
        \midrule
        gls-EF/ward/2     &      0.76 &         0.72 &         0.94 & 39.33 \\
        rls-EF/complete/2 &      0.77 &         0.74 &         0.93 & 37.61 \\
        gls-rls-EF/ward/2 &      0.76 &         0.72 &         0.93 & 35.16 \\
        gls-EF/average/2  &      0.74 &         0.70 &         0.94 & 34.90 \\
        gls-EF/complete/2 &      0.68 &         0.63 &         0.94 & 25.75 \\
        \bottomrule
    \end{tabular}
    \caption{The accuracy, DOR, sensitivity and specicity scores of the five best performing two-cluster-center PVC methods in terms of DOR, at detecting patient diagnoses.
             The \textbf{Dataset-Method} column indicates \textit{Dataset used}$/$\textit{Linkage criteria of method}$/$\textit{Number of cluster centers}.}
    \label{tab:pvc_ind_dor_sens_spec_dist}
\end{table*}

\begin{table*}
    \centering
    \ra{1.3}
    \begin{tabular}{lr}
        \toprule
        Dataset-Method     &  ARI \\
        \midrule
        gls/average/6      & 0.29 \\
        gls/average/7      & 0.29 \\
        gls-rls/complete/3 & 0.28 \\
        rls-EF/complete/2  & 0.26 \\
        gls-EF/ward/2      & 0.25 \\
        \bottomrule
    \end{tabular}
    \caption{The five highest ARI scores attained when applying PVC for detecting patient diagnoses.
             The \textbf{Dataset-Method} column indicates \textit{Dataset used}$/$\textit{Linkage criteria of method}$/$\textit{Number of cluster centers}.}
    \label{tab:pvc_ind_ari}
\end{table*}

\begin{figure}[H]
    \centering
    \begin{subfigure}[b]{0.49\textwidth}
        \centering
        \includegraphics[width=0.99\textwidth]{results/pd/scatter_gls_indication_bin.png}
        \caption{Patient Diagnosis.}
        \label{fig:scatter_gls_ef_hf}
    \end{subfigure}
    \begin{subfigure}[b]{0.49\textwidth}
        \centering
        \includegraphics[width=0.99\textwidth]{results/pd/scatter_gls_EF_ward2.png}
        \caption{\textit{GLS-EF Ward/2} cluster assignments.}
        \label{fig:scatter_gls_ef_ward2}
    \end{subfigure}\\
    \begin{subfigure}[b]{0.49\textwidth}
        \centering
        \includegraphics[width=0.99\textwidth]{results/pd/scatter_gls_average6.png}
        \caption{\textit{GLS Average/6} cluster assignments.}
        \label{fig:scatter_gls_ef_complete2}
    \end{subfigure}
    \begin{subfigure}[b]{0.49\textwidth}
        \centering
        \includegraphics[width=0.99\textwidth]{results/pd/scatter_gls_average7.png}
        \caption{\textit{GLS Average/7} cluster assignments.}
        \label{fig:scatter_gls_ef_average2}
    \end{subfigure}
    \caption{Scatterplot of peak GLS values in each view. Colors in the of the different dots are given by heart failure diagnosis, and cluster assignments of 
             \textit{gls-EF/ward/2}, \textit{average/6} and \textit{average/7} methods. Numbers are not included on the axes because the point of the figure is to illustrate the separability 
             of clusters, and patient diagnosis.}
             \label{fig:scatter_gls_ef_hf_cluster_assignments}
\end{figure}

\begin{comment}
\textbf{ARI PARAGRAPH. ONLY FOR CLUSTERING METHODS}.
[ ] \textbf{Comment on the spread of ARI scores. Be specific since the distribution plots are ommitted}
[ ] \textbf{Comment on the general trends of high performing methods in terms of ARI - are they the same trends as scores performing high in terms of DOR?}
[ ] \textbf{Comment on whether the methods in the top 5 ARIs are the same methods with the highest DOR. If not, mention it.}
[ ] \textbf{If the top 1 or 2 ARIs are also top in DOR no further discussion is needed. You can then plot some of the cluster realizations to see what they look like.}
[ ] \textbf{If NOT, why do they differ? Is the method with the highest ARI evaluated at a higher cluster number that 2? Attempt to visualize it, if it is not too difficult.}
[ ] \textbf{Plot some visualizations of the clustering, and comment on them.}
[ ] \textbf{Make arguments for and against the top three methods in terms of accuracy, sensitivity, specificity, DOR, ARI and potentially the plots, and make an informed choice.}
\end{comment}

\newpage

\subsection{Deep Neural Network}

\begin{figure}[H]
    \centering
    % \includegraphics[width=\textwidth]{results/dl_ind_dor_sens_spec_dist.png}
    %% Creator: Matplotlib, PGF backend
%%
%% To include the figure in your LaTeX document, write
%%   \input{<filename>.pgf}
%%
%% Make sure the required packages are loaded in your preamble
%%   \usepackage{pgf}
%%
%% Figures using additional raster images can only be included by \input if
%% they are in the same directory as the main LaTeX file. For loading figures
%% from other directories you can use the `import` package
%%   \usepackage{import}
%% and then include the figures with
%%   \import{<path to file>}{<filename>.pgf}
%%
%% Matplotlib used the following preamble
%%
\begingroup%
\makeatletter%
\begin{pgfpicture}%
\pgfpathrectangle{\pgfpointorigin}{\pgfqpoint{6.439273in}{2.540000in}}%
\pgfusepath{use as bounding box, clip}%
\begin{pgfscope}%
\pgfsetbuttcap%
\pgfsetmiterjoin%
\definecolor{currentfill}{rgb}{1.000000,1.000000,1.000000}%
\pgfsetfillcolor{currentfill}%
\pgfsetlinewidth{0.000000pt}%
\definecolor{currentstroke}{rgb}{1.000000,1.000000,1.000000}%
\pgfsetstrokecolor{currentstroke}%
\pgfsetdash{}{0pt}%
\pgfpathmoveto{\pgfqpoint{0.000000in}{0.000000in}}%
\pgfpathlineto{\pgfqpoint{6.439273in}{0.000000in}}%
\pgfpathlineto{\pgfqpoint{6.439273in}{2.540000in}}%
\pgfpathlineto{\pgfqpoint{0.000000in}{2.540000in}}%
\pgfpathclose%
\pgfusepath{fill}%
\end{pgfscope}%
\begin{pgfscope}%
\pgfsetbuttcap%
\pgfsetmiterjoin%
\definecolor{currentfill}{rgb}{0.917647,0.917647,0.949020}%
\pgfsetfillcolor{currentfill}%
\pgfsetlinewidth{0.000000pt}%
\definecolor{currentstroke}{rgb}{0.000000,0.000000,0.000000}%
\pgfsetstrokecolor{currentstroke}%
\pgfsetstrokeopacity{0.000000}%
\pgfsetdash{}{0pt}%
\pgfpathmoveto{\pgfqpoint{0.693056in}{0.557870in}}%
\pgfpathlineto{\pgfqpoint{3.156042in}{0.557870in}}%
\pgfpathlineto{\pgfqpoint{3.156042in}{2.242604in}}%
\pgfpathlineto{\pgfqpoint{0.693056in}{2.242604in}}%
\pgfpathclose%
\pgfusepath{fill}%
\end{pgfscope}%
\begin{pgfscope}%
\pgfpathrectangle{\pgfqpoint{0.693056in}{0.557870in}}{\pgfqpoint{2.462986in}{1.684734in}}%
\pgfusepath{clip}%
\pgfsetroundcap%
\pgfsetroundjoin%
\pgfsetlinewidth{1.003750pt}%
\definecolor{currentstroke}{rgb}{1.000000,1.000000,1.000000}%
\pgfsetstrokecolor{currentstroke}%
\pgfsetdash{}{0pt}%
\pgfpathmoveto{\pgfqpoint{0.805010in}{0.557870in}}%
\pgfpathlineto{\pgfqpoint{0.805010in}{2.242604in}}%
\pgfusepath{stroke}%
\end{pgfscope}%
\begin{pgfscope}%
\definecolor{textcolor}{rgb}{0.150000,0.150000,0.150000}%
\pgfsetstrokecolor{textcolor}%
\pgfsetfillcolor{textcolor}%
\pgftext[x=0.805010in,y=0.425926in,,top]{\color{textcolor}\sffamily\fontsize{11.000000}{13.200000}\selectfont \(\displaystyle -0.50\)}%
\end{pgfscope}%
\begin{pgfscope}%
\pgfpathrectangle{\pgfqpoint{0.693056in}{0.557870in}}{\pgfqpoint{2.462986in}{1.684734in}}%
\pgfusepath{clip}%
\pgfsetroundcap%
\pgfsetroundjoin%
\pgfsetlinewidth{1.003750pt}%
\definecolor{currentstroke}{rgb}{1.000000,1.000000,1.000000}%
\pgfsetstrokecolor{currentstroke}%
\pgfsetdash{}{0pt}%
\pgfpathmoveto{\pgfqpoint{1.364779in}{0.557870in}}%
\pgfpathlineto{\pgfqpoint{1.364779in}{2.242604in}}%
\pgfusepath{stroke}%
\end{pgfscope}%
\begin{pgfscope}%
\definecolor{textcolor}{rgb}{0.150000,0.150000,0.150000}%
\pgfsetstrokecolor{textcolor}%
\pgfsetfillcolor{textcolor}%
\pgftext[x=1.364779in,y=0.425926in,,top]{\color{textcolor}\sffamily\fontsize{11.000000}{13.200000}\selectfont \(\displaystyle -0.25\)}%
\end{pgfscope}%
\begin{pgfscope}%
\pgfpathrectangle{\pgfqpoint{0.693056in}{0.557870in}}{\pgfqpoint{2.462986in}{1.684734in}}%
\pgfusepath{clip}%
\pgfsetroundcap%
\pgfsetroundjoin%
\pgfsetlinewidth{1.003750pt}%
\definecolor{currentstroke}{rgb}{1.000000,1.000000,1.000000}%
\pgfsetstrokecolor{currentstroke}%
\pgfsetdash{}{0pt}%
\pgfpathmoveto{\pgfqpoint{1.924549in}{0.557870in}}%
\pgfpathlineto{\pgfqpoint{1.924549in}{2.242604in}}%
\pgfusepath{stroke}%
\end{pgfscope}%
\begin{pgfscope}%
\definecolor{textcolor}{rgb}{0.150000,0.150000,0.150000}%
\pgfsetstrokecolor{textcolor}%
\pgfsetfillcolor{textcolor}%
\pgftext[x=1.924549in,y=0.425926in,,top]{\color{textcolor}\sffamily\fontsize{11.000000}{13.200000}\selectfont \(\displaystyle 0.00\)}%
\end{pgfscope}%
\begin{pgfscope}%
\pgfpathrectangle{\pgfqpoint{0.693056in}{0.557870in}}{\pgfqpoint{2.462986in}{1.684734in}}%
\pgfusepath{clip}%
\pgfsetroundcap%
\pgfsetroundjoin%
\pgfsetlinewidth{1.003750pt}%
\definecolor{currentstroke}{rgb}{1.000000,1.000000,1.000000}%
\pgfsetstrokecolor{currentstroke}%
\pgfsetdash{}{0pt}%
\pgfpathmoveto{\pgfqpoint{2.484318in}{0.557870in}}%
\pgfpathlineto{\pgfqpoint{2.484318in}{2.242604in}}%
\pgfusepath{stroke}%
\end{pgfscope}%
\begin{pgfscope}%
\definecolor{textcolor}{rgb}{0.150000,0.150000,0.150000}%
\pgfsetstrokecolor{textcolor}%
\pgfsetfillcolor{textcolor}%
\pgftext[x=2.484318in,y=0.425926in,,top]{\color{textcolor}\sffamily\fontsize{11.000000}{13.200000}\selectfont \(\displaystyle 0.25\)}%
\end{pgfscope}%
\begin{pgfscope}%
\pgfpathrectangle{\pgfqpoint{0.693056in}{0.557870in}}{\pgfqpoint{2.462986in}{1.684734in}}%
\pgfusepath{clip}%
\pgfsetroundcap%
\pgfsetroundjoin%
\pgfsetlinewidth{1.003750pt}%
\definecolor{currentstroke}{rgb}{1.000000,1.000000,1.000000}%
\pgfsetstrokecolor{currentstroke}%
\pgfsetdash{}{0pt}%
\pgfpathmoveto{\pgfqpoint{3.044088in}{0.557870in}}%
\pgfpathlineto{\pgfqpoint{3.044088in}{2.242604in}}%
\pgfusepath{stroke}%
\end{pgfscope}%
\begin{pgfscope}%
\definecolor{textcolor}{rgb}{0.150000,0.150000,0.150000}%
\pgfsetstrokecolor{textcolor}%
\pgfsetfillcolor{textcolor}%
\pgftext[x=3.044088in,y=0.425926in,,top]{\color{textcolor}\sffamily\fontsize{11.000000}{13.200000}\selectfont \(\displaystyle 0.50\)}%
\end{pgfscope}%
\begin{pgfscope}%
\definecolor{textcolor}{rgb}{0.150000,0.150000,0.150000}%
\pgfsetstrokecolor{textcolor}%
\pgfsetfillcolor{textcolor}%
\pgftext[x=1.924549in,y=0.235185in,,top]{\color{textcolor}\sffamily\fontsize{11.000000}{13.200000}\selectfont DOR}%
\end{pgfscope}%
\begin{pgfscope}%
\pgfpathrectangle{\pgfqpoint{0.693056in}{0.557870in}}{\pgfqpoint{2.462986in}{1.684734in}}%
\pgfusepath{clip}%
\pgfsetroundcap%
\pgfsetroundjoin%
\pgfsetlinewidth{1.003750pt}%
\definecolor{currentstroke}{rgb}{1.000000,1.000000,1.000000}%
\pgfsetstrokecolor{currentstroke}%
\pgfsetdash{}{0pt}%
\pgfpathmoveto{\pgfqpoint{0.693056in}{0.557870in}}%
\pgfpathlineto{\pgfqpoint{3.156042in}{0.557870in}}%
\pgfusepath{stroke}%
\end{pgfscope}%
\begin{pgfscope}%
\definecolor{textcolor}{rgb}{0.150000,0.150000,0.150000}%
\pgfsetstrokecolor{textcolor}%
\pgfsetfillcolor{textcolor}%
\pgftext[x=0.290741in,y=0.505064in,left,base]{\color{textcolor}\sffamily\fontsize{11.000000}{13.200000}\selectfont \(\displaystyle 0.00\)}%
\end{pgfscope}%
\begin{pgfscope}%
\pgfpathrectangle{\pgfqpoint{0.693056in}{0.557870in}}{\pgfqpoint{2.462986in}{1.684734in}}%
\pgfusepath{clip}%
\pgfsetroundcap%
\pgfsetroundjoin%
\pgfsetlinewidth{1.003750pt}%
\definecolor{currentstroke}{rgb}{1.000000,1.000000,1.000000}%
\pgfsetstrokecolor{currentstroke}%
\pgfsetdash{}{0pt}%
\pgfpathmoveto{\pgfqpoint{0.693056in}{0.958997in}}%
\pgfpathlineto{\pgfqpoint{3.156042in}{0.958997in}}%
\pgfusepath{stroke}%
\end{pgfscope}%
\begin{pgfscope}%
\definecolor{textcolor}{rgb}{0.150000,0.150000,0.150000}%
\pgfsetstrokecolor{textcolor}%
\pgfsetfillcolor{textcolor}%
\pgftext[x=0.290741in,y=0.906191in,left,base]{\color{textcolor}\sffamily\fontsize{11.000000}{13.200000}\selectfont \(\displaystyle 0.25\)}%
\end{pgfscope}%
\begin{pgfscope}%
\pgfpathrectangle{\pgfqpoint{0.693056in}{0.557870in}}{\pgfqpoint{2.462986in}{1.684734in}}%
\pgfusepath{clip}%
\pgfsetroundcap%
\pgfsetroundjoin%
\pgfsetlinewidth{1.003750pt}%
\definecolor{currentstroke}{rgb}{1.000000,1.000000,1.000000}%
\pgfsetstrokecolor{currentstroke}%
\pgfsetdash{}{0pt}%
\pgfpathmoveto{\pgfqpoint{0.693056in}{1.360125in}}%
\pgfpathlineto{\pgfqpoint{3.156042in}{1.360125in}}%
\pgfusepath{stroke}%
\end{pgfscope}%
\begin{pgfscope}%
\definecolor{textcolor}{rgb}{0.150000,0.150000,0.150000}%
\pgfsetstrokecolor{textcolor}%
\pgfsetfillcolor{textcolor}%
\pgftext[x=0.290741in,y=1.307318in,left,base]{\color{textcolor}\sffamily\fontsize{11.000000}{13.200000}\selectfont \(\displaystyle 0.50\)}%
\end{pgfscope}%
\begin{pgfscope}%
\pgfpathrectangle{\pgfqpoint{0.693056in}{0.557870in}}{\pgfqpoint{2.462986in}{1.684734in}}%
\pgfusepath{clip}%
\pgfsetroundcap%
\pgfsetroundjoin%
\pgfsetlinewidth{1.003750pt}%
\definecolor{currentstroke}{rgb}{1.000000,1.000000,1.000000}%
\pgfsetstrokecolor{currentstroke}%
\pgfsetdash{}{0pt}%
\pgfpathmoveto{\pgfqpoint{0.693056in}{1.761252in}}%
\pgfpathlineto{\pgfqpoint{3.156042in}{1.761252in}}%
\pgfusepath{stroke}%
\end{pgfscope}%
\begin{pgfscope}%
\definecolor{textcolor}{rgb}{0.150000,0.150000,0.150000}%
\pgfsetstrokecolor{textcolor}%
\pgfsetfillcolor{textcolor}%
\pgftext[x=0.290741in,y=1.708445in,left,base]{\color{textcolor}\sffamily\fontsize{11.000000}{13.200000}\selectfont \(\displaystyle 0.75\)}%
\end{pgfscope}%
\begin{pgfscope}%
\pgfpathrectangle{\pgfqpoint{0.693056in}{0.557870in}}{\pgfqpoint{2.462986in}{1.684734in}}%
\pgfusepath{clip}%
\pgfsetroundcap%
\pgfsetroundjoin%
\pgfsetlinewidth{1.003750pt}%
\definecolor{currentstroke}{rgb}{1.000000,1.000000,1.000000}%
\pgfsetstrokecolor{currentstroke}%
\pgfsetdash{}{0pt}%
\pgfpathmoveto{\pgfqpoint{0.693056in}{2.162379in}}%
\pgfpathlineto{\pgfqpoint{3.156042in}{2.162379in}}%
\pgfusepath{stroke}%
\end{pgfscope}%
\begin{pgfscope}%
\definecolor{textcolor}{rgb}{0.150000,0.150000,0.150000}%
\pgfsetstrokecolor{textcolor}%
\pgfsetfillcolor{textcolor}%
\pgftext[x=0.290741in,y=2.109572in,left,base]{\color{textcolor}\sffamily\fontsize{11.000000}{13.200000}\selectfont \(\displaystyle 1.00\)}%
\end{pgfscope}%
\begin{pgfscope}%
\definecolor{textcolor}{rgb}{0.150000,0.150000,0.150000}%
\pgfsetstrokecolor{textcolor}%
\pgfsetfillcolor{textcolor}%
\pgftext[x=0.235185in,y=1.400237in,,bottom,rotate=90.000000]{\color{textcolor}\sffamily\fontsize{11.000000}{13.200000}\selectfont Occurance}%
\end{pgfscope}%
\begin{pgfscope}%
\pgfpathrectangle{\pgfqpoint{0.693056in}{0.557870in}}{\pgfqpoint{2.462986in}{1.684734in}}%
\pgfusepath{clip}%
\pgfsetbuttcap%
\pgfsetmiterjoin%
\definecolor{currentfill}{rgb}{0.298039,0.447059,0.690196}%
\pgfsetfillcolor{currentfill}%
\pgfsetfillopacity{0.400000}%
\pgfsetlinewidth{1.003750pt}%
\definecolor{currentstroke}{rgb}{1.000000,1.000000,1.000000}%
\pgfsetstrokecolor{currentstroke}%
\pgfsetstrokeopacity{0.400000}%
\pgfsetdash{}{0pt}%
\pgfpathmoveto{\pgfqpoint{0.805010in}{0.557870in}}%
\pgfpathlineto{\pgfqpoint{1.028918in}{0.557870in}}%
\pgfpathlineto{\pgfqpoint{1.028918in}{0.557870in}}%
\pgfpathlineto{\pgfqpoint{0.805010in}{0.557870in}}%
\pgfpathclose%
\pgfusepath{stroke,fill}%
\end{pgfscope}%
\begin{pgfscope}%
\pgfpathrectangle{\pgfqpoint{0.693056in}{0.557870in}}{\pgfqpoint{2.462986in}{1.684734in}}%
\pgfusepath{clip}%
\pgfsetbuttcap%
\pgfsetmiterjoin%
\definecolor{currentfill}{rgb}{0.298039,0.447059,0.690196}%
\pgfsetfillcolor{currentfill}%
\pgfsetfillopacity{0.400000}%
\pgfsetlinewidth{1.003750pt}%
\definecolor{currentstroke}{rgb}{1.000000,1.000000,1.000000}%
\pgfsetstrokecolor{currentstroke}%
\pgfsetstrokeopacity{0.400000}%
\pgfsetdash{}{0pt}%
\pgfpathmoveto{\pgfqpoint{1.028918in}{0.557870in}}%
\pgfpathlineto{\pgfqpoint{1.252825in}{0.557870in}}%
\pgfpathlineto{\pgfqpoint{1.252825in}{0.557870in}}%
\pgfpathlineto{\pgfqpoint{1.028918in}{0.557870in}}%
\pgfpathclose%
\pgfusepath{stroke,fill}%
\end{pgfscope}%
\begin{pgfscope}%
\pgfpathrectangle{\pgfqpoint{0.693056in}{0.557870in}}{\pgfqpoint{2.462986in}{1.684734in}}%
\pgfusepath{clip}%
\pgfsetbuttcap%
\pgfsetmiterjoin%
\definecolor{currentfill}{rgb}{0.298039,0.447059,0.690196}%
\pgfsetfillcolor{currentfill}%
\pgfsetfillopacity{0.400000}%
\pgfsetlinewidth{1.003750pt}%
\definecolor{currentstroke}{rgb}{1.000000,1.000000,1.000000}%
\pgfsetstrokecolor{currentstroke}%
\pgfsetstrokeopacity{0.400000}%
\pgfsetdash{}{0pt}%
\pgfpathmoveto{\pgfqpoint{1.252825in}{0.557870in}}%
\pgfpathlineto{\pgfqpoint{1.476733in}{0.557870in}}%
\pgfpathlineto{\pgfqpoint{1.476733in}{0.557870in}}%
\pgfpathlineto{\pgfqpoint{1.252825in}{0.557870in}}%
\pgfpathclose%
\pgfusepath{stroke,fill}%
\end{pgfscope}%
\begin{pgfscope}%
\pgfpathrectangle{\pgfqpoint{0.693056in}{0.557870in}}{\pgfqpoint{2.462986in}{1.684734in}}%
\pgfusepath{clip}%
\pgfsetbuttcap%
\pgfsetmiterjoin%
\definecolor{currentfill}{rgb}{0.298039,0.447059,0.690196}%
\pgfsetfillcolor{currentfill}%
\pgfsetfillopacity{0.400000}%
\pgfsetlinewidth{1.003750pt}%
\definecolor{currentstroke}{rgb}{1.000000,1.000000,1.000000}%
\pgfsetstrokecolor{currentstroke}%
\pgfsetstrokeopacity{0.400000}%
\pgfsetdash{}{0pt}%
\pgfpathmoveto{\pgfqpoint{1.476733in}{0.557870in}}%
\pgfpathlineto{\pgfqpoint{1.700641in}{0.557870in}}%
\pgfpathlineto{\pgfqpoint{1.700641in}{0.557870in}}%
\pgfpathlineto{\pgfqpoint{1.476733in}{0.557870in}}%
\pgfpathclose%
\pgfusepath{stroke,fill}%
\end{pgfscope}%
\begin{pgfscope}%
\pgfpathrectangle{\pgfqpoint{0.693056in}{0.557870in}}{\pgfqpoint{2.462986in}{1.684734in}}%
\pgfusepath{clip}%
\pgfsetbuttcap%
\pgfsetmiterjoin%
\definecolor{currentfill}{rgb}{0.298039,0.447059,0.690196}%
\pgfsetfillcolor{currentfill}%
\pgfsetfillopacity{0.400000}%
\pgfsetlinewidth{1.003750pt}%
\definecolor{currentstroke}{rgb}{1.000000,1.000000,1.000000}%
\pgfsetstrokecolor{currentstroke}%
\pgfsetstrokeopacity{0.400000}%
\pgfsetdash{}{0pt}%
\pgfpathmoveto{\pgfqpoint{1.700641in}{0.557870in}}%
\pgfpathlineto{\pgfqpoint{1.924549in}{0.557870in}}%
\pgfpathlineto{\pgfqpoint{1.924549in}{0.557870in}}%
\pgfpathlineto{\pgfqpoint{1.700641in}{0.557870in}}%
\pgfpathclose%
\pgfusepath{stroke,fill}%
\end{pgfscope}%
\begin{pgfscope}%
\pgfpathrectangle{\pgfqpoint{0.693056in}{0.557870in}}{\pgfqpoint{2.462986in}{1.684734in}}%
\pgfusepath{clip}%
\pgfsetbuttcap%
\pgfsetmiterjoin%
\definecolor{currentfill}{rgb}{0.298039,0.447059,0.690196}%
\pgfsetfillcolor{currentfill}%
\pgfsetfillopacity{0.400000}%
\pgfsetlinewidth{1.003750pt}%
\definecolor{currentstroke}{rgb}{1.000000,1.000000,1.000000}%
\pgfsetstrokecolor{currentstroke}%
\pgfsetstrokeopacity{0.400000}%
\pgfsetdash{}{0pt}%
\pgfpathmoveto{\pgfqpoint{1.924549in}{0.557870in}}%
\pgfpathlineto{\pgfqpoint{2.148457in}{0.557870in}}%
\pgfpathlineto{\pgfqpoint{2.148457in}{2.162379in}}%
\pgfpathlineto{\pgfqpoint{1.924549in}{2.162379in}}%
\pgfpathclose%
\pgfusepath{stroke,fill}%
\end{pgfscope}%
\begin{pgfscope}%
\pgfpathrectangle{\pgfqpoint{0.693056in}{0.557870in}}{\pgfqpoint{2.462986in}{1.684734in}}%
\pgfusepath{clip}%
\pgfsetbuttcap%
\pgfsetmiterjoin%
\definecolor{currentfill}{rgb}{0.298039,0.447059,0.690196}%
\pgfsetfillcolor{currentfill}%
\pgfsetfillopacity{0.400000}%
\pgfsetlinewidth{1.003750pt}%
\definecolor{currentstroke}{rgb}{1.000000,1.000000,1.000000}%
\pgfsetstrokecolor{currentstroke}%
\pgfsetstrokeopacity{0.400000}%
\pgfsetdash{}{0pt}%
\pgfpathmoveto{\pgfqpoint{2.148457in}{0.557870in}}%
\pgfpathlineto{\pgfqpoint{2.372364in}{0.557870in}}%
\pgfpathlineto{\pgfqpoint{2.372364in}{0.557870in}}%
\pgfpathlineto{\pgfqpoint{2.148457in}{0.557870in}}%
\pgfpathclose%
\pgfusepath{stroke,fill}%
\end{pgfscope}%
\begin{pgfscope}%
\pgfpathrectangle{\pgfqpoint{0.693056in}{0.557870in}}{\pgfqpoint{2.462986in}{1.684734in}}%
\pgfusepath{clip}%
\pgfsetbuttcap%
\pgfsetmiterjoin%
\definecolor{currentfill}{rgb}{0.298039,0.447059,0.690196}%
\pgfsetfillcolor{currentfill}%
\pgfsetfillopacity{0.400000}%
\pgfsetlinewidth{1.003750pt}%
\definecolor{currentstroke}{rgb}{1.000000,1.000000,1.000000}%
\pgfsetstrokecolor{currentstroke}%
\pgfsetstrokeopacity{0.400000}%
\pgfsetdash{}{0pt}%
\pgfpathmoveto{\pgfqpoint{2.372364in}{0.557870in}}%
\pgfpathlineto{\pgfqpoint{2.596272in}{0.557870in}}%
\pgfpathlineto{\pgfqpoint{2.596272in}{0.557870in}}%
\pgfpathlineto{\pgfqpoint{2.372364in}{0.557870in}}%
\pgfpathclose%
\pgfusepath{stroke,fill}%
\end{pgfscope}%
\begin{pgfscope}%
\pgfpathrectangle{\pgfqpoint{0.693056in}{0.557870in}}{\pgfqpoint{2.462986in}{1.684734in}}%
\pgfusepath{clip}%
\pgfsetbuttcap%
\pgfsetmiterjoin%
\definecolor{currentfill}{rgb}{0.298039,0.447059,0.690196}%
\pgfsetfillcolor{currentfill}%
\pgfsetfillopacity{0.400000}%
\pgfsetlinewidth{1.003750pt}%
\definecolor{currentstroke}{rgb}{1.000000,1.000000,1.000000}%
\pgfsetstrokecolor{currentstroke}%
\pgfsetstrokeopacity{0.400000}%
\pgfsetdash{}{0pt}%
\pgfpathmoveto{\pgfqpoint{2.596272in}{0.557870in}}%
\pgfpathlineto{\pgfqpoint{2.820180in}{0.557870in}}%
\pgfpathlineto{\pgfqpoint{2.820180in}{0.557870in}}%
\pgfpathlineto{\pgfqpoint{2.596272in}{0.557870in}}%
\pgfpathclose%
\pgfusepath{stroke,fill}%
\end{pgfscope}%
\begin{pgfscope}%
\pgfpathrectangle{\pgfqpoint{0.693056in}{0.557870in}}{\pgfqpoint{2.462986in}{1.684734in}}%
\pgfusepath{clip}%
\pgfsetbuttcap%
\pgfsetmiterjoin%
\definecolor{currentfill}{rgb}{0.298039,0.447059,0.690196}%
\pgfsetfillcolor{currentfill}%
\pgfsetfillopacity{0.400000}%
\pgfsetlinewidth{1.003750pt}%
\definecolor{currentstroke}{rgb}{1.000000,1.000000,1.000000}%
\pgfsetstrokecolor{currentstroke}%
\pgfsetstrokeopacity{0.400000}%
\pgfsetdash{}{0pt}%
\pgfpathmoveto{\pgfqpoint{2.820180in}{0.557870in}}%
\pgfpathlineto{\pgfqpoint{3.044088in}{0.557870in}}%
\pgfpathlineto{\pgfqpoint{3.044088in}{0.557870in}}%
\pgfpathlineto{\pgfqpoint{2.820180in}{0.557870in}}%
\pgfpathclose%
\pgfusepath{stroke,fill}%
\end{pgfscope}%
\begin{pgfscope}%
\pgfsetrectcap%
\pgfsetmiterjoin%
\pgfsetlinewidth{1.254687pt}%
\definecolor{currentstroke}{rgb}{1.000000,1.000000,1.000000}%
\pgfsetstrokecolor{currentstroke}%
\pgfsetdash{}{0pt}%
\pgfpathmoveto{\pgfqpoint{0.693056in}{0.557870in}}%
\pgfpathlineto{\pgfqpoint{0.693056in}{2.242604in}}%
\pgfusepath{stroke}%
\end{pgfscope}%
\begin{pgfscope}%
\pgfsetrectcap%
\pgfsetmiterjoin%
\pgfsetlinewidth{1.254687pt}%
\definecolor{currentstroke}{rgb}{1.000000,1.000000,1.000000}%
\pgfsetstrokecolor{currentstroke}%
\pgfsetdash{}{0pt}%
\pgfpathmoveto{\pgfqpoint{3.156042in}{0.557870in}}%
\pgfpathlineto{\pgfqpoint{3.156042in}{2.242604in}}%
\pgfusepath{stroke}%
\end{pgfscope}%
\begin{pgfscope}%
\pgfsetrectcap%
\pgfsetmiterjoin%
\pgfsetlinewidth{1.254687pt}%
\definecolor{currentstroke}{rgb}{1.000000,1.000000,1.000000}%
\pgfsetstrokecolor{currentstroke}%
\pgfsetdash{}{0pt}%
\pgfpathmoveto{\pgfqpoint{0.693056in}{0.557870in}}%
\pgfpathlineto{\pgfqpoint{3.156042in}{0.557870in}}%
\pgfusepath{stroke}%
\end{pgfscope}%
\begin{pgfscope}%
\pgfsetrectcap%
\pgfsetmiterjoin%
\pgfsetlinewidth{1.254687pt}%
\definecolor{currentstroke}{rgb}{1.000000,1.000000,1.000000}%
\pgfsetstrokecolor{currentstroke}%
\pgfsetdash{}{0pt}%
\pgfpathmoveto{\pgfqpoint{0.693056in}{2.242604in}}%
\pgfpathlineto{\pgfqpoint{3.156042in}{2.242604in}}%
\pgfusepath{stroke}%
\end{pgfscope}%
\begin{pgfscope}%
\definecolor{textcolor}{rgb}{0.150000,0.150000,0.150000}%
\pgfsetstrokecolor{textcolor}%
\pgfsetfillcolor{textcolor}%
\pgftext[x=1.924549in,y=2.325938in,,base]{\color{textcolor}\sffamily\fontsize{11.000000}{13.200000}\selectfont (a)}%
\end{pgfscope}%
\begin{pgfscope}%
\pgfsetbuttcap%
\pgfsetmiterjoin%
\definecolor{currentfill}{rgb}{0.917647,0.917647,0.949020}%
\pgfsetfillcolor{currentfill}%
\pgfsetlinewidth{0.000000pt}%
\definecolor{currentstroke}{rgb}{0.000000,0.000000,0.000000}%
\pgfsetstrokecolor{currentstroke}%
\pgfsetstrokeopacity{0.000000}%
\pgfsetdash{}{0pt}%
\pgfpathmoveto{\pgfqpoint{3.853056in}{0.557870in}}%
\pgfpathlineto{\pgfqpoint{6.316042in}{0.557870in}}%
\pgfpathlineto{\pgfqpoint{6.316042in}{2.242604in}}%
\pgfpathlineto{\pgfqpoint{3.853056in}{2.242604in}}%
\pgfpathclose%
\pgfusepath{fill}%
\end{pgfscope}%
\begin{pgfscope}%
\pgfpathrectangle{\pgfqpoint{3.853056in}{0.557870in}}{\pgfqpoint{2.462986in}{1.684734in}}%
\pgfusepath{clip}%
\pgfsetroundcap%
\pgfsetroundjoin%
\pgfsetlinewidth{1.003750pt}%
\definecolor{currentstroke}{rgb}{1.000000,1.000000,1.000000}%
\pgfsetstrokecolor{currentstroke}%
\pgfsetdash{}{0pt}%
\pgfpathmoveto{\pgfqpoint{3.965010in}{0.557870in}}%
\pgfpathlineto{\pgfqpoint{3.965010in}{2.242604in}}%
\pgfusepath{stroke}%
\end{pgfscope}%
\begin{pgfscope}%
\definecolor{textcolor}{rgb}{0.150000,0.150000,0.150000}%
\pgfsetstrokecolor{textcolor}%
\pgfsetfillcolor{textcolor}%
\pgftext[x=3.965010in,y=0.425926in,,top]{\color{textcolor}\sffamily\fontsize{11.000000}{13.200000}\selectfont \(\displaystyle 0.00\)}%
\end{pgfscope}%
\begin{pgfscope}%
\pgfpathrectangle{\pgfqpoint{3.853056in}{0.557870in}}{\pgfqpoint{2.462986in}{1.684734in}}%
\pgfusepath{clip}%
\pgfsetroundcap%
\pgfsetroundjoin%
\pgfsetlinewidth{1.003750pt}%
\definecolor{currentstroke}{rgb}{1.000000,1.000000,1.000000}%
\pgfsetstrokecolor{currentstroke}%
\pgfsetdash{}{0pt}%
\pgfpathmoveto{\pgfqpoint{4.524779in}{0.557870in}}%
\pgfpathlineto{\pgfqpoint{4.524779in}{2.242604in}}%
\pgfusepath{stroke}%
\end{pgfscope}%
\begin{pgfscope}%
\definecolor{textcolor}{rgb}{0.150000,0.150000,0.150000}%
\pgfsetstrokecolor{textcolor}%
\pgfsetfillcolor{textcolor}%
\pgftext[x=4.524779in,y=0.425926in,,top]{\color{textcolor}\sffamily\fontsize{11.000000}{13.200000}\selectfont \(\displaystyle 0.25\)}%
\end{pgfscope}%
\begin{pgfscope}%
\pgfpathrectangle{\pgfqpoint{3.853056in}{0.557870in}}{\pgfqpoint{2.462986in}{1.684734in}}%
\pgfusepath{clip}%
\pgfsetroundcap%
\pgfsetroundjoin%
\pgfsetlinewidth{1.003750pt}%
\definecolor{currentstroke}{rgb}{1.000000,1.000000,1.000000}%
\pgfsetstrokecolor{currentstroke}%
\pgfsetdash{}{0pt}%
\pgfpathmoveto{\pgfqpoint{5.084549in}{0.557870in}}%
\pgfpathlineto{\pgfqpoint{5.084549in}{2.242604in}}%
\pgfusepath{stroke}%
\end{pgfscope}%
\begin{pgfscope}%
\definecolor{textcolor}{rgb}{0.150000,0.150000,0.150000}%
\pgfsetstrokecolor{textcolor}%
\pgfsetfillcolor{textcolor}%
\pgftext[x=5.084549in,y=0.425926in,,top]{\color{textcolor}\sffamily\fontsize{11.000000}{13.200000}\selectfont \(\displaystyle 0.50\)}%
\end{pgfscope}%
\begin{pgfscope}%
\pgfpathrectangle{\pgfqpoint{3.853056in}{0.557870in}}{\pgfqpoint{2.462986in}{1.684734in}}%
\pgfusepath{clip}%
\pgfsetroundcap%
\pgfsetroundjoin%
\pgfsetlinewidth{1.003750pt}%
\definecolor{currentstroke}{rgb}{1.000000,1.000000,1.000000}%
\pgfsetstrokecolor{currentstroke}%
\pgfsetdash{}{0pt}%
\pgfpathmoveto{\pgfqpoint{5.644318in}{0.557870in}}%
\pgfpathlineto{\pgfqpoint{5.644318in}{2.242604in}}%
\pgfusepath{stroke}%
\end{pgfscope}%
\begin{pgfscope}%
\definecolor{textcolor}{rgb}{0.150000,0.150000,0.150000}%
\pgfsetstrokecolor{textcolor}%
\pgfsetfillcolor{textcolor}%
\pgftext[x=5.644318in,y=0.425926in,,top]{\color{textcolor}\sffamily\fontsize{11.000000}{13.200000}\selectfont \(\displaystyle 0.75\)}%
\end{pgfscope}%
\begin{pgfscope}%
\pgfpathrectangle{\pgfqpoint{3.853056in}{0.557870in}}{\pgfqpoint{2.462986in}{1.684734in}}%
\pgfusepath{clip}%
\pgfsetroundcap%
\pgfsetroundjoin%
\pgfsetlinewidth{1.003750pt}%
\definecolor{currentstroke}{rgb}{1.000000,1.000000,1.000000}%
\pgfsetstrokecolor{currentstroke}%
\pgfsetdash{}{0pt}%
\pgfpathmoveto{\pgfqpoint{6.204088in}{0.557870in}}%
\pgfpathlineto{\pgfqpoint{6.204088in}{2.242604in}}%
\pgfusepath{stroke}%
\end{pgfscope}%
\begin{pgfscope}%
\definecolor{textcolor}{rgb}{0.150000,0.150000,0.150000}%
\pgfsetstrokecolor{textcolor}%
\pgfsetfillcolor{textcolor}%
\pgftext[x=6.204088in,y=0.425926in,,top]{\color{textcolor}\sffamily\fontsize{11.000000}{13.200000}\selectfont \(\displaystyle 1.00\)}%
\end{pgfscope}%
\begin{pgfscope}%
\definecolor{textcolor}{rgb}{0.150000,0.150000,0.150000}%
\pgfsetstrokecolor{textcolor}%
\pgfsetfillcolor{textcolor}%
\pgftext[x=5.084549in,y=0.235185in,,top]{\color{textcolor}\sffamily\fontsize{11.000000}{13.200000}\selectfont Specificity}%
\end{pgfscope}%
\begin{pgfscope}%
\pgfpathrectangle{\pgfqpoint{3.853056in}{0.557870in}}{\pgfqpoint{2.462986in}{1.684734in}}%
\pgfusepath{clip}%
\pgfsetroundcap%
\pgfsetroundjoin%
\pgfsetlinewidth{1.003750pt}%
\definecolor{currentstroke}{rgb}{1.000000,1.000000,1.000000}%
\pgfsetstrokecolor{currentstroke}%
\pgfsetdash{}{0pt}%
\pgfpathmoveto{\pgfqpoint{3.853056in}{0.634449in}}%
\pgfpathlineto{\pgfqpoint{6.316042in}{0.634449in}}%
\pgfusepath{stroke}%
\end{pgfscope}%
\begin{pgfscope}%
\definecolor{textcolor}{rgb}{0.150000,0.150000,0.150000}%
\pgfsetstrokecolor{textcolor}%
\pgfsetfillcolor{textcolor}%
\pgftext[x=3.450741in,y=0.581642in,left,base]{\color{textcolor}\sffamily\fontsize{11.000000}{13.200000}\selectfont \(\displaystyle 0.00\)}%
\end{pgfscope}%
\begin{pgfscope}%
\pgfpathrectangle{\pgfqpoint{3.853056in}{0.557870in}}{\pgfqpoint{2.462986in}{1.684734in}}%
\pgfusepath{clip}%
\pgfsetroundcap%
\pgfsetroundjoin%
\pgfsetlinewidth{1.003750pt}%
\definecolor{currentstroke}{rgb}{1.000000,1.000000,1.000000}%
\pgfsetstrokecolor{currentstroke}%
\pgfsetdash{}{0pt}%
\pgfpathmoveto{\pgfqpoint{3.853056in}{1.017343in}}%
\pgfpathlineto{\pgfqpoint{6.316042in}{1.017343in}}%
\pgfusepath{stroke}%
\end{pgfscope}%
\begin{pgfscope}%
\definecolor{textcolor}{rgb}{0.150000,0.150000,0.150000}%
\pgfsetstrokecolor{textcolor}%
\pgfsetfillcolor{textcolor}%
\pgftext[x=3.450741in,y=0.964536in,left,base]{\color{textcolor}\sffamily\fontsize{11.000000}{13.200000}\selectfont \(\displaystyle 0.25\)}%
\end{pgfscope}%
\begin{pgfscope}%
\pgfpathrectangle{\pgfqpoint{3.853056in}{0.557870in}}{\pgfqpoint{2.462986in}{1.684734in}}%
\pgfusepath{clip}%
\pgfsetroundcap%
\pgfsetroundjoin%
\pgfsetlinewidth{1.003750pt}%
\definecolor{currentstroke}{rgb}{1.000000,1.000000,1.000000}%
\pgfsetstrokecolor{currentstroke}%
\pgfsetdash{}{0pt}%
\pgfpathmoveto{\pgfqpoint{3.853056in}{1.400237in}}%
\pgfpathlineto{\pgfqpoint{6.316042in}{1.400237in}}%
\pgfusepath{stroke}%
\end{pgfscope}%
\begin{pgfscope}%
\definecolor{textcolor}{rgb}{0.150000,0.150000,0.150000}%
\pgfsetstrokecolor{textcolor}%
\pgfsetfillcolor{textcolor}%
\pgftext[x=3.450741in,y=1.347431in,left,base]{\color{textcolor}\sffamily\fontsize{11.000000}{13.200000}\selectfont \(\displaystyle 0.50\)}%
\end{pgfscope}%
\begin{pgfscope}%
\pgfpathrectangle{\pgfqpoint{3.853056in}{0.557870in}}{\pgfqpoint{2.462986in}{1.684734in}}%
\pgfusepath{clip}%
\pgfsetroundcap%
\pgfsetroundjoin%
\pgfsetlinewidth{1.003750pt}%
\definecolor{currentstroke}{rgb}{1.000000,1.000000,1.000000}%
\pgfsetstrokecolor{currentstroke}%
\pgfsetdash{}{0pt}%
\pgfpathmoveto{\pgfqpoint{3.853056in}{1.783131in}}%
\pgfpathlineto{\pgfqpoint{6.316042in}{1.783131in}}%
\pgfusepath{stroke}%
\end{pgfscope}%
\begin{pgfscope}%
\definecolor{textcolor}{rgb}{0.150000,0.150000,0.150000}%
\pgfsetstrokecolor{textcolor}%
\pgfsetfillcolor{textcolor}%
\pgftext[x=3.450741in,y=1.730325in,left,base]{\color{textcolor}\sffamily\fontsize{11.000000}{13.200000}\selectfont \(\displaystyle 0.75\)}%
\end{pgfscope}%
\begin{pgfscope}%
\pgfpathrectangle{\pgfqpoint{3.853056in}{0.557870in}}{\pgfqpoint{2.462986in}{1.684734in}}%
\pgfusepath{clip}%
\pgfsetroundcap%
\pgfsetroundjoin%
\pgfsetlinewidth{1.003750pt}%
\definecolor{currentstroke}{rgb}{1.000000,1.000000,1.000000}%
\pgfsetstrokecolor{currentstroke}%
\pgfsetdash{}{0pt}%
\pgfpathmoveto{\pgfqpoint{3.853056in}{2.166025in}}%
\pgfpathlineto{\pgfqpoint{6.316042in}{2.166025in}}%
\pgfusepath{stroke}%
\end{pgfscope}%
\begin{pgfscope}%
\definecolor{textcolor}{rgb}{0.150000,0.150000,0.150000}%
\pgfsetstrokecolor{textcolor}%
\pgfsetfillcolor{textcolor}%
\pgftext[x=3.450741in,y=2.113219in,left,base]{\color{textcolor}\sffamily\fontsize{11.000000}{13.200000}\selectfont \(\displaystyle 1.00\)}%
\end{pgfscope}%
\begin{pgfscope}%
\definecolor{textcolor}{rgb}{0.150000,0.150000,0.150000}%
\pgfsetstrokecolor{textcolor}%
\pgfsetfillcolor{textcolor}%
\pgftext[x=3.395185in,y=1.400237in,,bottom,rotate=90.000000]{\color{textcolor}\sffamily\fontsize{11.000000}{13.200000}\selectfont Sensitivity}%
\end{pgfscope}%
\begin{pgfscope}%
\pgfpathrectangle{\pgfqpoint{3.853056in}{0.557870in}}{\pgfqpoint{2.462986in}{1.684734in}}%
\pgfusepath{clip}%
\pgfsetbuttcap%
\pgfsetroundjoin%
\definecolor{currentfill}{rgb}{0.298039,0.447059,0.690196}%
\pgfsetfillcolor{currentfill}%
\pgfsetlinewidth{1.003750pt}%
\definecolor{currentstroke}{rgb}{0.298039,0.447059,0.690196}%
\pgfsetstrokecolor{currentstroke}%
\pgfsetdash{}{0pt}%
\pgfpathmoveto{\pgfqpoint{3.965010in}{2.125798in}}%
\pgfpathcurveto{\pgfqpoint{3.973246in}{2.125798in}}{\pgfqpoint{3.981146in}{2.129070in}}{\pgfqpoint{3.986970in}{2.134894in}}%
\pgfpathcurveto{\pgfqpoint{3.992794in}{2.140718in}}{\pgfqpoint{3.996066in}{2.148618in}}{\pgfqpoint{3.996066in}{2.156854in}}%
\pgfpathcurveto{\pgfqpoint{3.996066in}{2.165091in}}{\pgfqpoint{3.992794in}{2.172991in}}{\pgfqpoint{3.986970in}{2.178814in}}%
\pgfpathcurveto{\pgfqpoint{3.981146in}{2.184638in}}{\pgfqpoint{3.973246in}{2.187911in}}{\pgfqpoint{3.965010in}{2.187911in}}%
\pgfpathcurveto{\pgfqpoint{3.956773in}{2.187911in}}{\pgfqpoint{3.948873in}{2.184638in}}{\pgfqpoint{3.943049in}{2.178814in}}%
\pgfpathcurveto{\pgfqpoint{3.937226in}{2.172991in}}{\pgfqpoint{3.933953in}{2.165091in}}{\pgfqpoint{3.933953in}{2.156854in}}%
\pgfpathcurveto{\pgfqpoint{3.933953in}{2.148618in}}{\pgfqpoint{3.937226in}{2.140718in}}{\pgfqpoint{3.943049in}{2.134894in}}%
\pgfpathcurveto{\pgfqpoint{3.948873in}{2.129070in}}{\pgfqpoint{3.956773in}{2.125798in}}{\pgfqpoint{3.965010in}{2.125798in}}%
\pgfpathclose%
\pgfusepath{stroke,fill}%
\end{pgfscope}%
\begin{pgfscope}%
\pgfsetrectcap%
\pgfsetmiterjoin%
\pgfsetlinewidth{1.254687pt}%
\definecolor{currentstroke}{rgb}{1.000000,1.000000,1.000000}%
\pgfsetstrokecolor{currentstroke}%
\pgfsetdash{}{0pt}%
\pgfpathmoveto{\pgfqpoint{3.853056in}{0.557870in}}%
\pgfpathlineto{\pgfqpoint{3.853056in}{2.242604in}}%
\pgfusepath{stroke}%
\end{pgfscope}%
\begin{pgfscope}%
\pgfsetrectcap%
\pgfsetmiterjoin%
\pgfsetlinewidth{1.254687pt}%
\definecolor{currentstroke}{rgb}{1.000000,1.000000,1.000000}%
\pgfsetstrokecolor{currentstroke}%
\pgfsetdash{}{0pt}%
\pgfpathmoveto{\pgfqpoint{6.316042in}{0.557870in}}%
\pgfpathlineto{\pgfqpoint{6.316042in}{2.242604in}}%
\pgfusepath{stroke}%
\end{pgfscope}%
\begin{pgfscope}%
\pgfsetrectcap%
\pgfsetmiterjoin%
\pgfsetlinewidth{1.254687pt}%
\definecolor{currentstroke}{rgb}{1.000000,1.000000,1.000000}%
\pgfsetstrokecolor{currentstroke}%
\pgfsetdash{}{0pt}%
\pgfpathmoveto{\pgfqpoint{3.853056in}{0.557870in}}%
\pgfpathlineto{\pgfqpoint{6.316042in}{0.557870in}}%
\pgfusepath{stroke}%
\end{pgfscope}%
\begin{pgfscope}%
\pgfsetrectcap%
\pgfsetmiterjoin%
\pgfsetlinewidth{1.254687pt}%
\definecolor{currentstroke}{rgb}{1.000000,1.000000,1.000000}%
\pgfsetstrokecolor{currentstroke}%
\pgfsetdash{}{0pt}%
\pgfpathmoveto{\pgfqpoint{3.853056in}{2.242604in}}%
\pgfpathlineto{\pgfqpoint{6.316042in}{2.242604in}}%
\pgfusepath{stroke}%
\end{pgfscope}%
\begin{pgfscope}%
\definecolor{textcolor}{rgb}{0.150000,0.150000,0.150000}%
\pgfsetstrokecolor{textcolor}%
\pgfsetfillcolor{textcolor}%
\pgftext[x=5.084549in,y=2.325938in,,base]{\color{textcolor}\sffamily\fontsize{11.000000}{13.200000}\selectfont (b)}%
\end{pgfscope}%
\end{pgfpicture}%
\makeatother%
\endgroup%

    \caption{Distribution of DOR, sensitivity and specificity for the NN-variations trained to predict patient diagnosis.}
    \label{fig:dl_ind_dor_sens_spec_dist}
\end{figure}

\begin{table*}
    \centering
    \ra{1.3}
    \begin{tabular}{lrrrr}
        \toprule
        Dataset-Model              &  Accuracy &  Sensitivity &  Specificity &  DOR \\
        \midrule
        gls/2CH/regular            &      0.85 &         1.00 &         0.00 &  NaN \\
        rls/2CH/regular            &      0.85 &         1.00 &         0.00 &  NaN \\
        all-strain/2CH/regular     &      0.85 &         1.00 &         0.00 &  NaN \\
        all-strain/2CH/downsampled &      0.85 &         1.00 &         0.00 &  NaN \\
        all-strain/2CH/upsampled   &      0.85 &         1.00 &         0.00 &  NaN \\
        \bottomrule
    \end{tabular}
    \caption{The accuracy, DOR, sensitivity and specicity scores of the five best performing variations of the NN in terms of DOR, when trained to predict patient diagnoses.
             The \textbf{Dataset-Model} column indicates \textit{Dataset used}$/$\textit{View used}$/$\textit{Whether curve has been upsampled, downsampled or is regular}.}
    \label{tab:dl_hf_dor_sens_spec_dist}
\end{table*}

\begin{comment}
[ ] \textbf{Comment on spread of DOR.}
[ ] \textbf{Comment on spread of sensitivity and specificity.}
[ ] \textbf{Comment on common traits in the high performing methods.} Here you can refer to raw performance results in appendix.
[ ] \textbf{Comment on common traits in the low performing methods.} Here you can refer to raw performance results in appendix.
[ ] \textbf{Select one - three methods that are good contendors for being the best method/model in the group and comment on their traits}
\textbf{IF NOT CLUSTERING METHOD}
[ ] \textbf{Make arguments for and against the top three methods in terms of accuracy, sensitivity, specificity, and DOR, and make an informed choice.}
\end{comment}

\newpage

\subsection{Peak-value Classifiers}

\begin{figure}[H]
    \centering
    % \includegraphics[width=\textwidth]{results/pvmlc_ind_dor_sens_spec_dist.png}
    %% Creator: Matplotlib, PGF backend
%%
%% To include the figure in your LaTeX document, write
%%   \input{<filename>.pgf}
%%
%% Make sure the required packages are loaded in your preamble
%%   \usepackage{pgf}
%%
%% Figures using additional raster images can only be included by \input if
%% they are in the same directory as the main LaTeX file. For loading figures
%% from other directories you can use the `import` package
%%   \usepackage{import}
%% and then include the figures with
%%   \import{<path to file>}{<filename>.pgf}
%%
%% Matplotlib used the following preamble
%%
\begingroup%
\makeatletter%
\begin{pgfpicture}%
\pgfpathrectangle{\pgfpointorigin}{\pgfqpoint{6.362271in}{2.540000in}}%
\pgfusepath{use as bounding box, clip}%
\begin{pgfscope}%
\pgfsetbuttcap%
\pgfsetmiterjoin%
\definecolor{currentfill}{rgb}{1.000000,1.000000,1.000000}%
\pgfsetfillcolor{currentfill}%
\pgfsetlinewidth{0.000000pt}%
\definecolor{currentstroke}{rgb}{1.000000,1.000000,1.000000}%
\pgfsetstrokecolor{currentstroke}%
\pgfsetdash{}{0pt}%
\pgfpathmoveto{\pgfqpoint{0.000000in}{0.000000in}}%
\pgfpathlineto{\pgfqpoint{6.362271in}{0.000000in}}%
\pgfpathlineto{\pgfqpoint{6.362271in}{2.540000in}}%
\pgfpathlineto{\pgfqpoint{0.000000in}{2.540000in}}%
\pgfpathclose%
\pgfusepath{fill}%
\end{pgfscope}%
\begin{pgfscope}%
\pgfsetbuttcap%
\pgfsetmiterjoin%
\definecolor{currentfill}{rgb}{0.917647,0.917647,0.949020}%
\pgfsetfillcolor{currentfill}%
\pgfsetlinewidth{0.000000pt}%
\definecolor{currentstroke}{rgb}{0.000000,0.000000,0.000000}%
\pgfsetstrokecolor{currentstroke}%
\pgfsetstrokeopacity{0.000000}%
\pgfsetdash{}{0pt}%
\pgfpathmoveto{\pgfqpoint{0.574769in}{0.557870in}}%
\pgfpathlineto{\pgfqpoint{3.058877in}{0.557870in}}%
\pgfpathlineto{\pgfqpoint{3.058877in}{2.242604in}}%
\pgfpathlineto{\pgfqpoint{0.574769in}{2.242604in}}%
\pgfpathclose%
\pgfusepath{fill}%
\end{pgfscope}%
\begin{pgfscope}%
\pgfpathrectangle{\pgfqpoint{0.574769in}{0.557870in}}{\pgfqpoint{2.484109in}{1.684734in}}%
\pgfusepath{clip}%
\pgfsetroundcap%
\pgfsetroundjoin%
\pgfsetlinewidth{1.003750pt}%
\definecolor{currentstroke}{rgb}{1.000000,1.000000,1.000000}%
\pgfsetstrokecolor{currentstroke}%
\pgfsetdash{}{0pt}%
\pgfpathmoveto{\pgfqpoint{0.626035in}{0.557870in}}%
\pgfpathlineto{\pgfqpoint{0.626035in}{2.242604in}}%
\pgfusepath{stroke}%
\end{pgfscope}%
\begin{pgfscope}%
\definecolor{textcolor}{rgb}{0.150000,0.150000,0.150000}%
\pgfsetstrokecolor{textcolor}%
\pgfsetfillcolor{textcolor}%
\pgftext[x=0.626035in,y=0.425926in,,top]{\color{textcolor}\sffamily\fontsize{11.000000}{13.200000}\selectfont \(\displaystyle 0\)}%
\end{pgfscope}%
\begin{pgfscope}%
\pgfpathrectangle{\pgfqpoint{0.574769in}{0.557870in}}{\pgfqpoint{2.484109in}{1.684734in}}%
\pgfusepath{clip}%
\pgfsetroundcap%
\pgfsetroundjoin%
\pgfsetlinewidth{1.003750pt}%
\definecolor{currentstroke}{rgb}{1.000000,1.000000,1.000000}%
\pgfsetstrokecolor{currentstroke}%
\pgfsetdash{}{0pt}%
\pgfpathmoveto{\pgfqpoint{1.464058in}{0.557870in}}%
\pgfpathlineto{\pgfqpoint{1.464058in}{2.242604in}}%
\pgfusepath{stroke}%
\end{pgfscope}%
\begin{pgfscope}%
\definecolor{textcolor}{rgb}{0.150000,0.150000,0.150000}%
\pgfsetstrokecolor{textcolor}%
\pgfsetfillcolor{textcolor}%
\pgftext[x=1.464058in,y=0.425926in,,top]{\color{textcolor}\sffamily\fontsize{11.000000}{13.200000}\selectfont \(\displaystyle 50\)}%
\end{pgfscope}%
\begin{pgfscope}%
\pgfpathrectangle{\pgfqpoint{0.574769in}{0.557870in}}{\pgfqpoint{2.484109in}{1.684734in}}%
\pgfusepath{clip}%
\pgfsetroundcap%
\pgfsetroundjoin%
\pgfsetlinewidth{1.003750pt}%
\definecolor{currentstroke}{rgb}{1.000000,1.000000,1.000000}%
\pgfsetstrokecolor{currentstroke}%
\pgfsetdash{}{0pt}%
\pgfpathmoveto{\pgfqpoint{2.302082in}{0.557870in}}%
\pgfpathlineto{\pgfqpoint{2.302082in}{2.242604in}}%
\pgfusepath{stroke}%
\end{pgfscope}%
\begin{pgfscope}%
\definecolor{textcolor}{rgb}{0.150000,0.150000,0.150000}%
\pgfsetstrokecolor{textcolor}%
\pgfsetfillcolor{textcolor}%
\pgftext[x=2.302082in,y=0.425926in,,top]{\color{textcolor}\sffamily\fontsize{11.000000}{13.200000}\selectfont \(\displaystyle 100\)}%
\end{pgfscope}%
\begin{pgfscope}%
\definecolor{textcolor}{rgb}{0.150000,0.150000,0.150000}%
\pgfsetstrokecolor{textcolor}%
\pgfsetfillcolor{textcolor}%
\pgftext[x=1.816823in,y=0.235185in,,top]{\color{textcolor}\sffamily\fontsize{11.000000}{13.200000}\selectfont DOR}%
\end{pgfscope}%
\begin{pgfscope}%
\pgfpathrectangle{\pgfqpoint{0.574769in}{0.557870in}}{\pgfqpoint{2.484109in}{1.684734in}}%
\pgfusepath{clip}%
\pgfsetroundcap%
\pgfsetroundjoin%
\pgfsetlinewidth{1.003750pt}%
\definecolor{currentstroke}{rgb}{1.000000,1.000000,1.000000}%
\pgfsetstrokecolor{currentstroke}%
\pgfsetdash{}{0pt}%
\pgfpathmoveto{\pgfqpoint{0.574769in}{0.557870in}}%
\pgfpathlineto{\pgfqpoint{3.058877in}{0.557870in}}%
\pgfusepath{stroke}%
\end{pgfscope}%
\begin{pgfscope}%
\definecolor{textcolor}{rgb}{0.150000,0.150000,0.150000}%
\pgfsetstrokecolor{textcolor}%
\pgfsetfillcolor{textcolor}%
\pgftext[x=0.366783in,y=0.505064in,left,base]{\color{textcolor}\sffamily\fontsize{11.000000}{13.200000}\selectfont \(\displaystyle 0\)}%
\end{pgfscope}%
\begin{pgfscope}%
\pgfpathrectangle{\pgfqpoint{0.574769in}{0.557870in}}{\pgfqpoint{2.484109in}{1.684734in}}%
\pgfusepath{clip}%
\pgfsetroundcap%
\pgfsetroundjoin%
\pgfsetlinewidth{1.003750pt}%
\definecolor{currentstroke}{rgb}{1.000000,1.000000,1.000000}%
\pgfsetstrokecolor{currentstroke}%
\pgfsetdash{}{0pt}%
\pgfpathmoveto{\pgfqpoint{0.574769in}{0.922531in}}%
\pgfpathlineto{\pgfqpoint{3.058877in}{0.922531in}}%
\pgfusepath{stroke}%
\end{pgfscope}%
\begin{pgfscope}%
\definecolor{textcolor}{rgb}{0.150000,0.150000,0.150000}%
\pgfsetstrokecolor{textcolor}%
\pgfsetfillcolor{textcolor}%
\pgftext[x=0.366783in,y=0.869725in,left,base]{\color{textcolor}\sffamily\fontsize{11.000000}{13.200000}\selectfont \(\displaystyle 5\)}%
\end{pgfscope}%
\begin{pgfscope}%
\pgfpathrectangle{\pgfqpoint{0.574769in}{0.557870in}}{\pgfqpoint{2.484109in}{1.684734in}}%
\pgfusepath{clip}%
\pgfsetroundcap%
\pgfsetroundjoin%
\pgfsetlinewidth{1.003750pt}%
\definecolor{currentstroke}{rgb}{1.000000,1.000000,1.000000}%
\pgfsetstrokecolor{currentstroke}%
\pgfsetdash{}{0pt}%
\pgfpathmoveto{\pgfqpoint{0.574769in}{1.287192in}}%
\pgfpathlineto{\pgfqpoint{3.058877in}{1.287192in}}%
\pgfusepath{stroke}%
\end{pgfscope}%
\begin{pgfscope}%
\definecolor{textcolor}{rgb}{0.150000,0.150000,0.150000}%
\pgfsetstrokecolor{textcolor}%
\pgfsetfillcolor{textcolor}%
\pgftext[x=0.290741in,y=1.234386in,left,base]{\color{textcolor}\sffamily\fontsize{11.000000}{13.200000}\selectfont \(\displaystyle 10\)}%
\end{pgfscope}%
\begin{pgfscope}%
\pgfpathrectangle{\pgfqpoint{0.574769in}{0.557870in}}{\pgfqpoint{2.484109in}{1.684734in}}%
\pgfusepath{clip}%
\pgfsetroundcap%
\pgfsetroundjoin%
\pgfsetlinewidth{1.003750pt}%
\definecolor{currentstroke}{rgb}{1.000000,1.000000,1.000000}%
\pgfsetstrokecolor{currentstroke}%
\pgfsetdash{}{0pt}%
\pgfpathmoveto{\pgfqpoint{0.574769in}{1.651853in}}%
\pgfpathlineto{\pgfqpoint{3.058877in}{1.651853in}}%
\pgfusepath{stroke}%
\end{pgfscope}%
\begin{pgfscope}%
\definecolor{textcolor}{rgb}{0.150000,0.150000,0.150000}%
\pgfsetstrokecolor{textcolor}%
\pgfsetfillcolor{textcolor}%
\pgftext[x=0.290741in,y=1.599047in,left,base]{\color{textcolor}\sffamily\fontsize{11.000000}{13.200000}\selectfont \(\displaystyle 15\)}%
\end{pgfscope}%
\begin{pgfscope}%
\pgfpathrectangle{\pgfqpoint{0.574769in}{0.557870in}}{\pgfqpoint{2.484109in}{1.684734in}}%
\pgfusepath{clip}%
\pgfsetroundcap%
\pgfsetroundjoin%
\pgfsetlinewidth{1.003750pt}%
\definecolor{currentstroke}{rgb}{1.000000,1.000000,1.000000}%
\pgfsetstrokecolor{currentstroke}%
\pgfsetdash{}{0pt}%
\pgfpathmoveto{\pgfqpoint{0.574769in}{2.016514in}}%
\pgfpathlineto{\pgfqpoint{3.058877in}{2.016514in}}%
\pgfusepath{stroke}%
\end{pgfscope}%
\begin{pgfscope}%
\definecolor{textcolor}{rgb}{0.150000,0.150000,0.150000}%
\pgfsetstrokecolor{textcolor}%
\pgfsetfillcolor{textcolor}%
\pgftext[x=0.290741in,y=1.963708in,left,base]{\color{textcolor}\sffamily\fontsize{11.000000}{13.200000}\selectfont \(\displaystyle 20\)}%
\end{pgfscope}%
\begin{pgfscope}%
\definecolor{textcolor}{rgb}{0.150000,0.150000,0.150000}%
\pgfsetstrokecolor{textcolor}%
\pgfsetfillcolor{textcolor}%
\pgftext[x=0.235185in,y=1.400237in,,bottom,rotate=90.000000]{\color{textcolor}\sffamily\fontsize{11.000000}{13.200000}\selectfont Occurance}%
\end{pgfscope}%
\begin{pgfscope}%
\pgfpathrectangle{\pgfqpoint{0.574769in}{0.557870in}}{\pgfqpoint{2.484109in}{1.684734in}}%
\pgfusepath{clip}%
\pgfsetbuttcap%
\pgfsetmiterjoin%
\definecolor{currentfill}{rgb}{0.298039,0.447059,0.690196}%
\pgfsetfillcolor{currentfill}%
\pgfsetfillopacity{0.400000}%
\pgfsetlinewidth{1.003750pt}%
\definecolor{currentstroke}{rgb}{1.000000,1.000000,1.000000}%
\pgfsetstrokecolor{currentstroke}%
\pgfsetstrokeopacity{0.400000}%
\pgfsetdash{}{0pt}%
\pgfpathmoveto{\pgfqpoint{0.687683in}{0.557870in}}%
\pgfpathlineto{\pgfqpoint{0.913511in}{0.557870in}}%
\pgfpathlineto{\pgfqpoint{0.913511in}{2.162379in}}%
\pgfpathlineto{\pgfqpoint{0.687683in}{2.162379in}}%
\pgfpathclose%
\pgfusepath{stroke,fill}%
\end{pgfscope}%
\begin{pgfscope}%
\pgfpathrectangle{\pgfqpoint{0.574769in}{0.557870in}}{\pgfqpoint{2.484109in}{1.684734in}}%
\pgfusepath{clip}%
\pgfsetbuttcap%
\pgfsetmiterjoin%
\definecolor{currentfill}{rgb}{0.298039,0.447059,0.690196}%
\pgfsetfillcolor{currentfill}%
\pgfsetfillopacity{0.400000}%
\pgfsetlinewidth{1.003750pt}%
\definecolor{currentstroke}{rgb}{1.000000,1.000000,1.000000}%
\pgfsetstrokecolor{currentstroke}%
\pgfsetstrokeopacity{0.400000}%
\pgfsetdash{}{0pt}%
\pgfpathmoveto{\pgfqpoint{0.913511in}{0.557870in}}%
\pgfpathlineto{\pgfqpoint{1.139339in}{0.557870in}}%
\pgfpathlineto{\pgfqpoint{1.139339in}{1.651853in}}%
\pgfpathlineto{\pgfqpoint{0.913511in}{1.651853in}}%
\pgfpathclose%
\pgfusepath{stroke,fill}%
\end{pgfscope}%
\begin{pgfscope}%
\pgfpathrectangle{\pgfqpoint{0.574769in}{0.557870in}}{\pgfqpoint{2.484109in}{1.684734in}}%
\pgfusepath{clip}%
\pgfsetbuttcap%
\pgfsetmiterjoin%
\definecolor{currentfill}{rgb}{0.298039,0.447059,0.690196}%
\pgfsetfillcolor{currentfill}%
\pgfsetfillopacity{0.400000}%
\pgfsetlinewidth{1.003750pt}%
\definecolor{currentstroke}{rgb}{1.000000,1.000000,1.000000}%
\pgfsetstrokecolor{currentstroke}%
\pgfsetstrokeopacity{0.400000}%
\pgfsetdash{}{0pt}%
\pgfpathmoveto{\pgfqpoint{1.139339in}{0.557870in}}%
\pgfpathlineto{\pgfqpoint{1.365167in}{0.557870in}}%
\pgfpathlineto{\pgfqpoint{1.365167in}{0.776667in}}%
\pgfpathlineto{\pgfqpoint{1.139339in}{0.776667in}}%
\pgfpathclose%
\pgfusepath{stroke,fill}%
\end{pgfscope}%
\begin{pgfscope}%
\pgfpathrectangle{\pgfqpoint{0.574769in}{0.557870in}}{\pgfqpoint{2.484109in}{1.684734in}}%
\pgfusepath{clip}%
\pgfsetbuttcap%
\pgfsetmiterjoin%
\definecolor{currentfill}{rgb}{0.298039,0.447059,0.690196}%
\pgfsetfillcolor{currentfill}%
\pgfsetfillopacity{0.400000}%
\pgfsetlinewidth{1.003750pt}%
\definecolor{currentstroke}{rgb}{1.000000,1.000000,1.000000}%
\pgfsetstrokecolor{currentstroke}%
\pgfsetstrokeopacity{0.400000}%
\pgfsetdash{}{0pt}%
\pgfpathmoveto{\pgfqpoint{1.365167in}{0.557870in}}%
\pgfpathlineto{\pgfqpoint{1.590995in}{0.557870in}}%
\pgfpathlineto{\pgfqpoint{1.590995in}{1.068396in}}%
\pgfpathlineto{\pgfqpoint{1.365167in}{1.068396in}}%
\pgfpathclose%
\pgfusepath{stroke,fill}%
\end{pgfscope}%
\begin{pgfscope}%
\pgfpathrectangle{\pgfqpoint{0.574769in}{0.557870in}}{\pgfqpoint{2.484109in}{1.684734in}}%
\pgfusepath{clip}%
\pgfsetbuttcap%
\pgfsetmiterjoin%
\definecolor{currentfill}{rgb}{0.298039,0.447059,0.690196}%
\pgfsetfillcolor{currentfill}%
\pgfsetfillopacity{0.400000}%
\pgfsetlinewidth{1.003750pt}%
\definecolor{currentstroke}{rgb}{1.000000,1.000000,1.000000}%
\pgfsetstrokecolor{currentstroke}%
\pgfsetstrokeopacity{0.400000}%
\pgfsetdash{}{0pt}%
\pgfpathmoveto{\pgfqpoint{1.590995in}{0.557870in}}%
\pgfpathlineto{\pgfqpoint{1.816823in}{0.557870in}}%
\pgfpathlineto{\pgfqpoint{1.816823in}{0.922531in}}%
\pgfpathlineto{\pgfqpoint{1.590995in}{0.922531in}}%
\pgfpathclose%
\pgfusepath{stroke,fill}%
\end{pgfscope}%
\begin{pgfscope}%
\pgfpathrectangle{\pgfqpoint{0.574769in}{0.557870in}}{\pgfqpoint{2.484109in}{1.684734in}}%
\pgfusepath{clip}%
\pgfsetbuttcap%
\pgfsetmiterjoin%
\definecolor{currentfill}{rgb}{0.298039,0.447059,0.690196}%
\pgfsetfillcolor{currentfill}%
\pgfsetfillopacity{0.400000}%
\pgfsetlinewidth{1.003750pt}%
\definecolor{currentstroke}{rgb}{1.000000,1.000000,1.000000}%
\pgfsetstrokecolor{currentstroke}%
\pgfsetstrokeopacity{0.400000}%
\pgfsetdash{}{0pt}%
\pgfpathmoveto{\pgfqpoint{1.816823in}{0.557870in}}%
\pgfpathlineto{\pgfqpoint{2.042651in}{0.557870in}}%
\pgfpathlineto{\pgfqpoint{2.042651in}{0.849599in}}%
\pgfpathlineto{\pgfqpoint{1.816823in}{0.849599in}}%
\pgfpathclose%
\pgfusepath{stroke,fill}%
\end{pgfscope}%
\begin{pgfscope}%
\pgfpathrectangle{\pgfqpoint{0.574769in}{0.557870in}}{\pgfqpoint{2.484109in}{1.684734in}}%
\pgfusepath{clip}%
\pgfsetbuttcap%
\pgfsetmiterjoin%
\definecolor{currentfill}{rgb}{0.298039,0.447059,0.690196}%
\pgfsetfillcolor{currentfill}%
\pgfsetfillopacity{0.400000}%
\pgfsetlinewidth{1.003750pt}%
\definecolor{currentstroke}{rgb}{1.000000,1.000000,1.000000}%
\pgfsetstrokecolor{currentstroke}%
\pgfsetstrokeopacity{0.400000}%
\pgfsetdash{}{0pt}%
\pgfpathmoveto{\pgfqpoint{2.042651in}{0.557870in}}%
\pgfpathlineto{\pgfqpoint{2.268479in}{0.557870in}}%
\pgfpathlineto{\pgfqpoint{2.268479in}{0.630802in}}%
\pgfpathlineto{\pgfqpoint{2.042651in}{0.630802in}}%
\pgfpathclose%
\pgfusepath{stroke,fill}%
\end{pgfscope}%
\begin{pgfscope}%
\pgfpathrectangle{\pgfqpoint{0.574769in}{0.557870in}}{\pgfqpoint{2.484109in}{1.684734in}}%
\pgfusepath{clip}%
\pgfsetbuttcap%
\pgfsetmiterjoin%
\definecolor{currentfill}{rgb}{0.298039,0.447059,0.690196}%
\pgfsetfillcolor{currentfill}%
\pgfsetfillopacity{0.400000}%
\pgfsetlinewidth{1.003750pt}%
\definecolor{currentstroke}{rgb}{1.000000,1.000000,1.000000}%
\pgfsetstrokecolor{currentstroke}%
\pgfsetstrokeopacity{0.400000}%
\pgfsetdash{}{0pt}%
\pgfpathmoveto{\pgfqpoint{2.268479in}{0.557870in}}%
\pgfpathlineto{\pgfqpoint{2.494307in}{0.557870in}}%
\pgfpathlineto{\pgfqpoint{2.494307in}{0.557870in}}%
\pgfpathlineto{\pgfqpoint{2.268479in}{0.557870in}}%
\pgfpathclose%
\pgfusepath{stroke,fill}%
\end{pgfscope}%
\begin{pgfscope}%
\pgfpathrectangle{\pgfqpoint{0.574769in}{0.557870in}}{\pgfqpoint{2.484109in}{1.684734in}}%
\pgfusepath{clip}%
\pgfsetbuttcap%
\pgfsetmiterjoin%
\definecolor{currentfill}{rgb}{0.298039,0.447059,0.690196}%
\pgfsetfillcolor{currentfill}%
\pgfsetfillopacity{0.400000}%
\pgfsetlinewidth{1.003750pt}%
\definecolor{currentstroke}{rgb}{1.000000,1.000000,1.000000}%
\pgfsetstrokecolor{currentstroke}%
\pgfsetstrokeopacity{0.400000}%
\pgfsetdash{}{0pt}%
\pgfpathmoveto{\pgfqpoint{2.494307in}{0.557870in}}%
\pgfpathlineto{\pgfqpoint{2.720135in}{0.557870in}}%
\pgfpathlineto{\pgfqpoint{2.720135in}{0.557870in}}%
\pgfpathlineto{\pgfqpoint{2.494307in}{0.557870in}}%
\pgfpathclose%
\pgfusepath{stroke,fill}%
\end{pgfscope}%
\begin{pgfscope}%
\pgfpathrectangle{\pgfqpoint{0.574769in}{0.557870in}}{\pgfqpoint{2.484109in}{1.684734in}}%
\pgfusepath{clip}%
\pgfsetbuttcap%
\pgfsetmiterjoin%
\definecolor{currentfill}{rgb}{0.298039,0.447059,0.690196}%
\pgfsetfillcolor{currentfill}%
\pgfsetfillopacity{0.400000}%
\pgfsetlinewidth{1.003750pt}%
\definecolor{currentstroke}{rgb}{1.000000,1.000000,1.000000}%
\pgfsetstrokecolor{currentstroke}%
\pgfsetstrokeopacity{0.400000}%
\pgfsetdash{}{0pt}%
\pgfpathmoveto{\pgfqpoint{2.720135in}{0.557870in}}%
\pgfpathlineto{\pgfqpoint{2.945963in}{0.557870in}}%
\pgfpathlineto{\pgfqpoint{2.945963in}{0.630802in}}%
\pgfpathlineto{\pgfqpoint{2.720135in}{0.630802in}}%
\pgfpathclose%
\pgfusepath{stroke,fill}%
\end{pgfscope}%
\begin{pgfscope}%
\pgfsetrectcap%
\pgfsetmiterjoin%
\pgfsetlinewidth{1.254687pt}%
\definecolor{currentstroke}{rgb}{1.000000,1.000000,1.000000}%
\pgfsetstrokecolor{currentstroke}%
\pgfsetdash{}{0pt}%
\pgfpathmoveto{\pgfqpoint{0.574769in}{0.557870in}}%
\pgfpathlineto{\pgfqpoint{0.574769in}{2.242604in}}%
\pgfusepath{stroke}%
\end{pgfscope}%
\begin{pgfscope}%
\pgfsetrectcap%
\pgfsetmiterjoin%
\pgfsetlinewidth{1.254687pt}%
\definecolor{currentstroke}{rgb}{1.000000,1.000000,1.000000}%
\pgfsetstrokecolor{currentstroke}%
\pgfsetdash{}{0pt}%
\pgfpathmoveto{\pgfqpoint{3.058877in}{0.557870in}}%
\pgfpathlineto{\pgfqpoint{3.058877in}{2.242604in}}%
\pgfusepath{stroke}%
\end{pgfscope}%
\begin{pgfscope}%
\pgfsetrectcap%
\pgfsetmiterjoin%
\pgfsetlinewidth{1.254687pt}%
\definecolor{currentstroke}{rgb}{1.000000,1.000000,1.000000}%
\pgfsetstrokecolor{currentstroke}%
\pgfsetdash{}{0pt}%
\pgfpathmoveto{\pgfqpoint{0.574769in}{0.557870in}}%
\pgfpathlineto{\pgfqpoint{3.058877in}{0.557870in}}%
\pgfusepath{stroke}%
\end{pgfscope}%
\begin{pgfscope}%
\pgfsetrectcap%
\pgfsetmiterjoin%
\pgfsetlinewidth{1.254687pt}%
\definecolor{currentstroke}{rgb}{1.000000,1.000000,1.000000}%
\pgfsetstrokecolor{currentstroke}%
\pgfsetdash{}{0pt}%
\pgfpathmoveto{\pgfqpoint{0.574769in}{2.242604in}}%
\pgfpathlineto{\pgfqpoint{3.058877in}{2.242604in}}%
\pgfusepath{stroke}%
\end{pgfscope}%
\begin{pgfscope}%
\definecolor{textcolor}{rgb}{0.150000,0.150000,0.150000}%
\pgfsetstrokecolor{textcolor}%
\pgfsetfillcolor{textcolor}%
\pgftext[x=1.816823in,y=2.325938in,,base]{\color{textcolor}\sffamily\fontsize{11.000000}{13.200000}\selectfont (a)}%
\end{pgfscope}%
\begin{pgfscope}%
\pgfsetbuttcap%
\pgfsetmiterjoin%
\definecolor{currentfill}{rgb}{0.917647,0.917647,0.949020}%
\pgfsetfillcolor{currentfill}%
\pgfsetlinewidth{0.000000pt}%
\definecolor{currentstroke}{rgb}{0.000000,0.000000,0.000000}%
\pgfsetstrokecolor{currentstroke}%
\pgfsetstrokeopacity{0.000000}%
\pgfsetdash{}{0pt}%
\pgfpathmoveto{\pgfqpoint{3.755891in}{0.557870in}}%
\pgfpathlineto{\pgfqpoint{6.240000in}{0.557870in}}%
\pgfpathlineto{\pgfqpoint{6.240000in}{2.242604in}}%
\pgfpathlineto{\pgfqpoint{3.755891in}{2.242604in}}%
\pgfpathclose%
\pgfusepath{fill}%
\end{pgfscope}%
\begin{pgfscope}%
\pgfpathrectangle{\pgfqpoint{3.755891in}{0.557870in}}{\pgfqpoint{2.484109in}{1.684734in}}%
\pgfusepath{clip}%
\pgfsetroundcap%
\pgfsetroundjoin%
\pgfsetlinewidth{1.003750pt}%
\definecolor{currentstroke}{rgb}{1.000000,1.000000,1.000000}%
\pgfsetstrokecolor{currentstroke}%
\pgfsetdash{}{0pt}%
\pgfpathmoveto{\pgfqpoint{3.868805in}{0.557870in}}%
\pgfpathlineto{\pgfqpoint{3.868805in}{2.242604in}}%
\pgfusepath{stroke}%
\end{pgfscope}%
\begin{pgfscope}%
\definecolor{textcolor}{rgb}{0.150000,0.150000,0.150000}%
\pgfsetstrokecolor{textcolor}%
\pgfsetfillcolor{textcolor}%
\pgftext[x=3.868805in,y=0.425926in,,top]{\color{textcolor}\sffamily\fontsize{11.000000}{13.200000}\selectfont \(\displaystyle 0.00\)}%
\end{pgfscope}%
\begin{pgfscope}%
\pgfpathrectangle{\pgfqpoint{3.755891in}{0.557870in}}{\pgfqpoint{2.484109in}{1.684734in}}%
\pgfusepath{clip}%
\pgfsetroundcap%
\pgfsetroundjoin%
\pgfsetlinewidth{1.003750pt}%
\definecolor{currentstroke}{rgb}{1.000000,1.000000,1.000000}%
\pgfsetstrokecolor{currentstroke}%
\pgfsetdash{}{0pt}%
\pgfpathmoveto{\pgfqpoint{4.433376in}{0.557870in}}%
\pgfpathlineto{\pgfqpoint{4.433376in}{2.242604in}}%
\pgfusepath{stroke}%
\end{pgfscope}%
\begin{pgfscope}%
\definecolor{textcolor}{rgb}{0.150000,0.150000,0.150000}%
\pgfsetstrokecolor{textcolor}%
\pgfsetfillcolor{textcolor}%
\pgftext[x=4.433376in,y=0.425926in,,top]{\color{textcolor}\sffamily\fontsize{11.000000}{13.200000}\selectfont \(\displaystyle 0.25\)}%
\end{pgfscope}%
\begin{pgfscope}%
\pgfpathrectangle{\pgfqpoint{3.755891in}{0.557870in}}{\pgfqpoint{2.484109in}{1.684734in}}%
\pgfusepath{clip}%
\pgfsetroundcap%
\pgfsetroundjoin%
\pgfsetlinewidth{1.003750pt}%
\definecolor{currentstroke}{rgb}{1.000000,1.000000,1.000000}%
\pgfsetstrokecolor{currentstroke}%
\pgfsetdash{}{0pt}%
\pgfpathmoveto{\pgfqpoint{4.997946in}{0.557870in}}%
\pgfpathlineto{\pgfqpoint{4.997946in}{2.242604in}}%
\pgfusepath{stroke}%
\end{pgfscope}%
\begin{pgfscope}%
\definecolor{textcolor}{rgb}{0.150000,0.150000,0.150000}%
\pgfsetstrokecolor{textcolor}%
\pgfsetfillcolor{textcolor}%
\pgftext[x=4.997946in,y=0.425926in,,top]{\color{textcolor}\sffamily\fontsize{11.000000}{13.200000}\selectfont \(\displaystyle 0.50\)}%
\end{pgfscope}%
\begin{pgfscope}%
\pgfpathrectangle{\pgfqpoint{3.755891in}{0.557870in}}{\pgfqpoint{2.484109in}{1.684734in}}%
\pgfusepath{clip}%
\pgfsetroundcap%
\pgfsetroundjoin%
\pgfsetlinewidth{1.003750pt}%
\definecolor{currentstroke}{rgb}{1.000000,1.000000,1.000000}%
\pgfsetstrokecolor{currentstroke}%
\pgfsetdash{}{0pt}%
\pgfpathmoveto{\pgfqpoint{5.562516in}{0.557870in}}%
\pgfpathlineto{\pgfqpoint{5.562516in}{2.242604in}}%
\pgfusepath{stroke}%
\end{pgfscope}%
\begin{pgfscope}%
\definecolor{textcolor}{rgb}{0.150000,0.150000,0.150000}%
\pgfsetstrokecolor{textcolor}%
\pgfsetfillcolor{textcolor}%
\pgftext[x=5.562516in,y=0.425926in,,top]{\color{textcolor}\sffamily\fontsize{11.000000}{13.200000}\selectfont \(\displaystyle 0.75\)}%
\end{pgfscope}%
\begin{pgfscope}%
\pgfpathrectangle{\pgfqpoint{3.755891in}{0.557870in}}{\pgfqpoint{2.484109in}{1.684734in}}%
\pgfusepath{clip}%
\pgfsetroundcap%
\pgfsetroundjoin%
\pgfsetlinewidth{1.003750pt}%
\definecolor{currentstroke}{rgb}{1.000000,1.000000,1.000000}%
\pgfsetstrokecolor{currentstroke}%
\pgfsetdash{}{0pt}%
\pgfpathmoveto{\pgfqpoint{6.127086in}{0.557870in}}%
\pgfpathlineto{\pgfqpoint{6.127086in}{2.242604in}}%
\pgfusepath{stroke}%
\end{pgfscope}%
\begin{pgfscope}%
\definecolor{textcolor}{rgb}{0.150000,0.150000,0.150000}%
\pgfsetstrokecolor{textcolor}%
\pgfsetfillcolor{textcolor}%
\pgftext[x=6.127086in,y=0.425926in,,top]{\color{textcolor}\sffamily\fontsize{11.000000}{13.200000}\selectfont \(\displaystyle 1.00\)}%
\end{pgfscope}%
\begin{pgfscope}%
\definecolor{textcolor}{rgb}{0.150000,0.150000,0.150000}%
\pgfsetstrokecolor{textcolor}%
\pgfsetfillcolor{textcolor}%
\pgftext[x=4.997946in,y=0.235185in,,top]{\color{textcolor}\sffamily\fontsize{11.000000}{13.200000}\selectfont Specificity}%
\end{pgfscope}%
\begin{pgfscope}%
\pgfpathrectangle{\pgfqpoint{3.755891in}{0.557870in}}{\pgfqpoint{2.484109in}{1.684734in}}%
\pgfusepath{clip}%
\pgfsetroundcap%
\pgfsetroundjoin%
\pgfsetlinewidth{1.003750pt}%
\definecolor{currentstroke}{rgb}{1.000000,1.000000,1.000000}%
\pgfsetstrokecolor{currentstroke}%
\pgfsetdash{}{0pt}%
\pgfpathmoveto{\pgfqpoint{3.755891in}{0.634449in}}%
\pgfpathlineto{\pgfqpoint{6.240000in}{0.634449in}}%
\pgfusepath{stroke}%
\end{pgfscope}%
\begin{pgfscope}%
\definecolor{textcolor}{rgb}{0.150000,0.150000,0.150000}%
\pgfsetstrokecolor{textcolor}%
\pgfsetfillcolor{textcolor}%
\pgftext[x=3.353576in,y=0.581642in,left,base]{\color{textcolor}\sffamily\fontsize{11.000000}{13.200000}\selectfont \(\displaystyle 0.00\)}%
\end{pgfscope}%
\begin{pgfscope}%
\pgfpathrectangle{\pgfqpoint{3.755891in}{0.557870in}}{\pgfqpoint{2.484109in}{1.684734in}}%
\pgfusepath{clip}%
\pgfsetroundcap%
\pgfsetroundjoin%
\pgfsetlinewidth{1.003750pt}%
\definecolor{currentstroke}{rgb}{1.000000,1.000000,1.000000}%
\pgfsetstrokecolor{currentstroke}%
\pgfsetdash{}{0pt}%
\pgfpathmoveto{\pgfqpoint{3.755891in}{1.017343in}}%
\pgfpathlineto{\pgfqpoint{6.240000in}{1.017343in}}%
\pgfusepath{stroke}%
\end{pgfscope}%
\begin{pgfscope}%
\definecolor{textcolor}{rgb}{0.150000,0.150000,0.150000}%
\pgfsetstrokecolor{textcolor}%
\pgfsetfillcolor{textcolor}%
\pgftext[x=3.353576in,y=0.964536in,left,base]{\color{textcolor}\sffamily\fontsize{11.000000}{13.200000}\selectfont \(\displaystyle 0.25\)}%
\end{pgfscope}%
\begin{pgfscope}%
\pgfpathrectangle{\pgfqpoint{3.755891in}{0.557870in}}{\pgfqpoint{2.484109in}{1.684734in}}%
\pgfusepath{clip}%
\pgfsetroundcap%
\pgfsetroundjoin%
\pgfsetlinewidth{1.003750pt}%
\definecolor{currentstroke}{rgb}{1.000000,1.000000,1.000000}%
\pgfsetstrokecolor{currentstroke}%
\pgfsetdash{}{0pt}%
\pgfpathmoveto{\pgfqpoint{3.755891in}{1.400237in}}%
\pgfpathlineto{\pgfqpoint{6.240000in}{1.400237in}}%
\pgfusepath{stroke}%
\end{pgfscope}%
\begin{pgfscope}%
\definecolor{textcolor}{rgb}{0.150000,0.150000,0.150000}%
\pgfsetstrokecolor{textcolor}%
\pgfsetfillcolor{textcolor}%
\pgftext[x=3.353576in,y=1.347431in,left,base]{\color{textcolor}\sffamily\fontsize{11.000000}{13.200000}\selectfont \(\displaystyle 0.50\)}%
\end{pgfscope}%
\begin{pgfscope}%
\pgfpathrectangle{\pgfqpoint{3.755891in}{0.557870in}}{\pgfqpoint{2.484109in}{1.684734in}}%
\pgfusepath{clip}%
\pgfsetroundcap%
\pgfsetroundjoin%
\pgfsetlinewidth{1.003750pt}%
\definecolor{currentstroke}{rgb}{1.000000,1.000000,1.000000}%
\pgfsetstrokecolor{currentstroke}%
\pgfsetdash{}{0pt}%
\pgfpathmoveto{\pgfqpoint{3.755891in}{1.783131in}}%
\pgfpathlineto{\pgfqpoint{6.240000in}{1.783131in}}%
\pgfusepath{stroke}%
\end{pgfscope}%
\begin{pgfscope}%
\definecolor{textcolor}{rgb}{0.150000,0.150000,0.150000}%
\pgfsetstrokecolor{textcolor}%
\pgfsetfillcolor{textcolor}%
\pgftext[x=3.353576in,y=1.730325in,left,base]{\color{textcolor}\sffamily\fontsize{11.000000}{13.200000}\selectfont \(\displaystyle 0.75\)}%
\end{pgfscope}%
\begin{pgfscope}%
\pgfpathrectangle{\pgfqpoint{3.755891in}{0.557870in}}{\pgfqpoint{2.484109in}{1.684734in}}%
\pgfusepath{clip}%
\pgfsetroundcap%
\pgfsetroundjoin%
\pgfsetlinewidth{1.003750pt}%
\definecolor{currentstroke}{rgb}{1.000000,1.000000,1.000000}%
\pgfsetstrokecolor{currentstroke}%
\pgfsetdash{}{0pt}%
\pgfpathmoveto{\pgfqpoint{3.755891in}{2.166025in}}%
\pgfpathlineto{\pgfqpoint{6.240000in}{2.166025in}}%
\pgfusepath{stroke}%
\end{pgfscope}%
\begin{pgfscope}%
\definecolor{textcolor}{rgb}{0.150000,0.150000,0.150000}%
\pgfsetstrokecolor{textcolor}%
\pgfsetfillcolor{textcolor}%
\pgftext[x=3.353576in,y=2.113219in,left,base]{\color{textcolor}\sffamily\fontsize{11.000000}{13.200000}\selectfont \(\displaystyle 1.00\)}%
\end{pgfscope}%
\begin{pgfscope}%
\definecolor{textcolor}{rgb}{0.150000,0.150000,0.150000}%
\pgfsetstrokecolor{textcolor}%
\pgfsetfillcolor{textcolor}%
\pgftext[x=3.298021in,y=1.400237in,,bottom,rotate=90.000000]{\color{textcolor}\sffamily\fontsize{11.000000}{13.200000}\selectfont Sensitivity}%
\end{pgfscope}%
\begin{pgfscope}%
\pgfpathrectangle{\pgfqpoint{3.755891in}{0.557870in}}{\pgfqpoint{2.484109in}{1.684734in}}%
\pgfusepath{clip}%
\pgfsetbuttcap%
\pgfsetroundjoin%
\definecolor{currentfill}{rgb}{0.298039,0.447059,0.690196}%
\pgfsetfillcolor{currentfill}%
\pgfsetlinewidth{1.003750pt}%
\definecolor{currentstroke}{rgb}{0.298039,0.447059,0.690196}%
\pgfsetstrokecolor{currentstroke}%
\pgfsetdash{}{0pt}%
\pgfpathmoveto{\pgfqpoint{4.160196in}{2.106780in}}%
\pgfpathcurveto{\pgfqpoint{4.168433in}{2.106780in}}{\pgfqpoint{4.176333in}{2.110053in}}{\pgfqpoint{4.182157in}{2.115877in}}%
\pgfpathcurveto{\pgfqpoint{4.187981in}{2.121701in}}{\pgfqpoint{4.191253in}{2.129601in}}{\pgfqpoint{4.191253in}{2.137837in}}%
\pgfpathcurveto{\pgfqpoint{4.191253in}{2.146073in}}{\pgfqpoint{4.187981in}{2.153973in}}{\pgfqpoint{4.182157in}{2.159797in}}%
\pgfpathcurveto{\pgfqpoint{4.176333in}{2.165621in}}{\pgfqpoint{4.168433in}{2.168893in}}{\pgfqpoint{4.160196in}{2.168893in}}%
\pgfpathcurveto{\pgfqpoint{4.151960in}{2.168893in}}{\pgfqpoint{4.144060in}{2.165621in}}{\pgfqpoint{4.138236in}{2.159797in}}%
\pgfpathcurveto{\pgfqpoint{4.132412in}{2.153973in}}{\pgfqpoint{4.129140in}{2.146073in}}{\pgfqpoint{4.129140in}{2.137837in}}%
\pgfpathcurveto{\pgfqpoint{4.129140in}{2.129601in}}{\pgfqpoint{4.132412in}{2.121701in}}{\pgfqpoint{4.138236in}{2.115877in}}%
\pgfpathcurveto{\pgfqpoint{4.144060in}{2.110053in}}{\pgfqpoint{4.151960in}{2.106780in}}{\pgfqpoint{4.160196in}{2.106780in}}%
\pgfpathclose%
\pgfusepath{stroke,fill}%
\end{pgfscope}%
\begin{pgfscope}%
\pgfpathrectangle{\pgfqpoint{3.755891in}{0.557870in}}{\pgfqpoint{2.484109in}{1.684734in}}%
\pgfusepath{clip}%
\pgfsetbuttcap%
\pgfsetroundjoin%
\definecolor{currentfill}{rgb}{0.298039,0.447059,0.690196}%
\pgfsetfillcolor{currentfill}%
\pgfsetlinewidth{1.003750pt}%
\definecolor{currentstroke}{rgb}{0.298039,0.447059,0.690196}%
\pgfsetstrokecolor{currentstroke}%
\pgfsetdash{}{0pt}%
\pgfpathmoveto{\pgfqpoint{4.815826in}{2.031611in}}%
\pgfpathcurveto{\pgfqpoint{4.824063in}{2.031611in}}{\pgfqpoint{4.831963in}{2.034883in}}{\pgfqpoint{4.837787in}{2.040707in}}%
\pgfpathcurveto{\pgfqpoint{4.843610in}{2.046531in}}{\pgfqpoint{4.846883in}{2.054431in}}{\pgfqpoint{4.846883in}{2.062667in}}%
\pgfpathcurveto{\pgfqpoint{4.846883in}{2.070904in}}{\pgfqpoint{4.843610in}{2.078804in}}{\pgfqpoint{4.837787in}{2.084628in}}%
\pgfpathcurveto{\pgfqpoint{4.831963in}{2.090452in}}{\pgfqpoint{4.824063in}{2.093724in}}{\pgfqpoint{4.815826in}{2.093724in}}%
\pgfpathcurveto{\pgfqpoint{4.807590in}{2.093724in}}{\pgfqpoint{4.799690in}{2.090452in}}{\pgfqpoint{4.793866in}{2.084628in}}%
\pgfpathcurveto{\pgfqpoint{4.788042in}{2.078804in}}{\pgfqpoint{4.784770in}{2.070904in}}{\pgfqpoint{4.784770in}{2.062667in}}%
\pgfpathcurveto{\pgfqpoint{4.784770in}{2.054431in}}{\pgfqpoint{4.788042in}{2.046531in}}{\pgfqpoint{4.793866in}{2.040707in}}%
\pgfpathcurveto{\pgfqpoint{4.799690in}{2.034883in}}{\pgfqpoint{4.807590in}{2.031611in}}{\pgfqpoint{4.815826in}{2.031611in}}%
\pgfpathclose%
\pgfusepath{stroke,fill}%
\end{pgfscope}%
\begin{pgfscope}%
\pgfpathrectangle{\pgfqpoint{3.755891in}{0.557870in}}{\pgfqpoint{2.484109in}{1.684734in}}%
\pgfusepath{clip}%
\pgfsetbuttcap%
\pgfsetroundjoin%
\definecolor{currentfill}{rgb}{0.298039,0.447059,0.690196}%
\pgfsetfillcolor{currentfill}%
\pgfsetlinewidth{1.003750pt}%
\definecolor{currentstroke}{rgb}{0.298039,0.447059,0.690196}%
\pgfsetstrokecolor{currentstroke}%
\pgfsetdash{}{0pt}%
\pgfpathmoveto{\pgfqpoint{4.014501in}{2.116177in}}%
\pgfpathcurveto{\pgfqpoint{4.022737in}{2.116177in}}{\pgfqpoint{4.030637in}{2.119449in}}{\pgfqpoint{4.036461in}{2.125273in}}%
\pgfpathcurveto{\pgfqpoint{4.042285in}{2.131097in}}{\pgfqpoint{4.045557in}{2.138997in}}{\pgfqpoint{4.045557in}{2.147233in}}%
\pgfpathcurveto{\pgfqpoint{4.045557in}{2.155469in}}{\pgfqpoint{4.042285in}{2.163369in}}{\pgfqpoint{4.036461in}{2.169193in}}%
\pgfpathcurveto{\pgfqpoint{4.030637in}{2.175017in}}{\pgfqpoint{4.022737in}{2.178290in}}{\pgfqpoint{4.014501in}{2.178290in}}%
\pgfpathcurveto{\pgfqpoint{4.006265in}{2.178290in}}{\pgfqpoint{3.998365in}{2.175017in}}{\pgfqpoint{3.992541in}{2.169193in}}%
\pgfpathcurveto{\pgfqpoint{3.986717in}{2.163369in}}{\pgfqpoint{3.983444in}{2.155469in}}{\pgfqpoint{3.983444in}{2.147233in}}%
\pgfpathcurveto{\pgfqpoint{3.983444in}{2.138997in}}{\pgfqpoint{3.986717in}{2.131097in}}{\pgfqpoint{3.992541in}{2.125273in}}%
\pgfpathcurveto{\pgfqpoint{3.998365in}{2.119449in}}{\pgfqpoint{4.006265in}{2.116177in}}{\pgfqpoint{4.014501in}{2.116177in}}%
\pgfpathclose%
\pgfusepath{stroke,fill}%
\end{pgfscope}%
\begin{pgfscope}%
\pgfpathrectangle{\pgfqpoint{3.755891in}{0.557870in}}{\pgfqpoint{2.484109in}{1.684734in}}%
\pgfusepath{clip}%
\pgfsetbuttcap%
\pgfsetroundjoin%
\definecolor{currentfill}{rgb}{0.298039,0.447059,0.690196}%
\pgfsetfillcolor{currentfill}%
\pgfsetlinewidth{1.003750pt}%
\definecolor{currentstroke}{rgb}{0.298039,0.447059,0.690196}%
\pgfsetstrokecolor{currentstroke}%
\pgfsetdash{}{0pt}%
\pgfpathmoveto{\pgfqpoint{4.014501in}{2.106780in}}%
\pgfpathcurveto{\pgfqpoint{4.022737in}{2.106780in}}{\pgfqpoint{4.030637in}{2.110053in}}{\pgfqpoint{4.036461in}{2.115877in}}%
\pgfpathcurveto{\pgfqpoint{4.042285in}{2.121701in}}{\pgfqpoint{4.045557in}{2.129601in}}{\pgfqpoint{4.045557in}{2.137837in}}%
\pgfpathcurveto{\pgfqpoint{4.045557in}{2.146073in}}{\pgfqpoint{4.042285in}{2.153973in}}{\pgfqpoint{4.036461in}{2.159797in}}%
\pgfpathcurveto{\pgfqpoint{4.030637in}{2.165621in}}{\pgfqpoint{4.022737in}{2.168893in}}{\pgfqpoint{4.014501in}{2.168893in}}%
\pgfpathcurveto{\pgfqpoint{4.006265in}{2.168893in}}{\pgfqpoint{3.998365in}{2.165621in}}{\pgfqpoint{3.992541in}{2.159797in}}%
\pgfpathcurveto{\pgfqpoint{3.986717in}{2.153973in}}{\pgfqpoint{3.983444in}{2.146073in}}{\pgfqpoint{3.983444in}{2.137837in}}%
\pgfpathcurveto{\pgfqpoint{3.983444in}{2.129601in}}{\pgfqpoint{3.986717in}{2.121701in}}{\pgfqpoint{3.992541in}{2.115877in}}%
\pgfpathcurveto{\pgfqpoint{3.998365in}{2.110053in}}{\pgfqpoint{4.006265in}{2.106780in}}{\pgfqpoint{4.014501in}{2.106780in}}%
\pgfpathclose%
\pgfusepath{stroke,fill}%
\end{pgfscope}%
\begin{pgfscope}%
\pgfpathrectangle{\pgfqpoint{3.755891in}{0.557870in}}{\pgfqpoint{2.484109in}{1.684734in}}%
\pgfusepath{clip}%
\pgfsetbuttcap%
\pgfsetroundjoin%
\definecolor{currentfill}{rgb}{0.298039,0.447059,0.690196}%
\pgfsetfillcolor{currentfill}%
\pgfsetlinewidth{1.003750pt}%
\definecolor{currentstroke}{rgb}{0.298039,0.447059,0.690196}%
\pgfsetstrokecolor{currentstroke}%
\pgfsetdash{}{0pt}%
\pgfpathmoveto{\pgfqpoint{4.888674in}{1.984630in}}%
\pgfpathcurveto{\pgfqpoint{4.896910in}{1.984630in}}{\pgfqpoint{4.904810in}{1.987902in}}{\pgfqpoint{4.910634in}{1.993726in}}%
\pgfpathcurveto{\pgfqpoint{4.916458in}{1.999550in}}{\pgfqpoint{4.919731in}{2.007450in}}{\pgfqpoint{4.919731in}{2.015687in}}%
\pgfpathcurveto{\pgfqpoint{4.919731in}{2.023923in}}{\pgfqpoint{4.916458in}{2.031823in}}{\pgfqpoint{4.910634in}{2.037647in}}%
\pgfpathcurveto{\pgfqpoint{4.904810in}{2.043471in}}{\pgfqpoint{4.896910in}{2.046743in}}{\pgfqpoint{4.888674in}{2.046743in}}%
\pgfpathcurveto{\pgfqpoint{4.880438in}{2.046743in}}{\pgfqpoint{4.872538in}{2.043471in}}{\pgfqpoint{4.866714in}{2.037647in}}%
\pgfpathcurveto{\pgfqpoint{4.860890in}{2.031823in}}{\pgfqpoint{4.857618in}{2.023923in}}{\pgfqpoint{4.857618in}{2.015687in}}%
\pgfpathcurveto{\pgfqpoint{4.857618in}{2.007450in}}{\pgfqpoint{4.860890in}{1.999550in}}{\pgfqpoint{4.866714in}{1.993726in}}%
\pgfpathcurveto{\pgfqpoint{4.872538in}{1.987902in}}{\pgfqpoint{4.880438in}{1.984630in}}{\pgfqpoint{4.888674in}{1.984630in}}%
\pgfpathclose%
\pgfusepath{stroke,fill}%
\end{pgfscope}%
\begin{pgfscope}%
\pgfpathrectangle{\pgfqpoint{3.755891in}{0.557870in}}{\pgfqpoint{2.484109in}{1.684734in}}%
\pgfusepath{clip}%
\pgfsetbuttcap%
\pgfsetroundjoin%
\definecolor{currentfill}{rgb}{0.298039,0.447059,0.690196}%
\pgfsetfillcolor{currentfill}%
\pgfsetlinewidth{1.003750pt}%
\definecolor{currentstroke}{rgb}{0.298039,0.447059,0.690196}%
\pgfsetstrokecolor{currentstroke}%
\pgfsetdash{}{0pt}%
\pgfpathmoveto{\pgfqpoint{4.815826in}{1.984630in}}%
\pgfpathcurveto{\pgfqpoint{4.824063in}{1.984630in}}{\pgfqpoint{4.831963in}{1.987902in}}{\pgfqpoint{4.837787in}{1.993726in}}%
\pgfpathcurveto{\pgfqpoint{4.843610in}{1.999550in}}{\pgfqpoint{4.846883in}{2.007450in}}{\pgfqpoint{4.846883in}{2.015687in}}%
\pgfpathcurveto{\pgfqpoint{4.846883in}{2.023923in}}{\pgfqpoint{4.843610in}{2.031823in}}{\pgfqpoint{4.837787in}{2.037647in}}%
\pgfpathcurveto{\pgfqpoint{4.831963in}{2.043471in}}{\pgfqpoint{4.824063in}{2.046743in}}{\pgfqpoint{4.815826in}{2.046743in}}%
\pgfpathcurveto{\pgfqpoint{4.807590in}{2.046743in}}{\pgfqpoint{4.799690in}{2.043471in}}{\pgfqpoint{4.793866in}{2.037647in}}%
\pgfpathcurveto{\pgfqpoint{4.788042in}{2.031823in}}{\pgfqpoint{4.784770in}{2.023923in}}{\pgfqpoint{4.784770in}{2.015687in}}%
\pgfpathcurveto{\pgfqpoint{4.784770in}{2.007450in}}{\pgfqpoint{4.788042in}{1.999550in}}{\pgfqpoint{4.793866in}{1.993726in}}%
\pgfpathcurveto{\pgfqpoint{4.799690in}{1.987902in}}{\pgfqpoint{4.807590in}{1.984630in}}{\pgfqpoint{4.815826in}{1.984630in}}%
\pgfpathclose%
\pgfusepath{stroke,fill}%
\end{pgfscope}%
\begin{pgfscope}%
\pgfpathrectangle{\pgfqpoint{3.755891in}{0.557870in}}{\pgfqpoint{2.484109in}{1.684734in}}%
\pgfusepath{clip}%
\pgfsetbuttcap%
\pgfsetroundjoin%
\definecolor{currentfill}{rgb}{0.298039,0.447059,0.690196}%
\pgfsetfillcolor{currentfill}%
\pgfsetlinewidth{1.003750pt}%
\definecolor{currentstroke}{rgb}{0.298039,0.447059,0.690196}%
\pgfsetstrokecolor{currentstroke}%
\pgfsetdash{}{0pt}%
\pgfpathmoveto{\pgfqpoint{4.815826in}{2.041007in}}%
\pgfpathcurveto{\pgfqpoint{4.824063in}{2.041007in}}{\pgfqpoint{4.831963in}{2.044279in}}{\pgfqpoint{4.837787in}{2.050103in}}%
\pgfpathcurveto{\pgfqpoint{4.843610in}{2.055927in}}{\pgfqpoint{4.846883in}{2.063827in}}{\pgfqpoint{4.846883in}{2.072064in}}%
\pgfpathcurveto{\pgfqpoint{4.846883in}{2.080300in}}{\pgfqpoint{4.843610in}{2.088200in}}{\pgfqpoint{4.837787in}{2.094024in}}%
\pgfpathcurveto{\pgfqpoint{4.831963in}{2.099848in}}{\pgfqpoint{4.824063in}{2.103120in}}{\pgfqpoint{4.815826in}{2.103120in}}%
\pgfpathcurveto{\pgfqpoint{4.807590in}{2.103120in}}{\pgfqpoint{4.799690in}{2.099848in}}{\pgfqpoint{4.793866in}{2.094024in}}%
\pgfpathcurveto{\pgfqpoint{4.788042in}{2.088200in}}{\pgfqpoint{4.784770in}{2.080300in}}{\pgfqpoint{4.784770in}{2.072064in}}%
\pgfpathcurveto{\pgfqpoint{4.784770in}{2.063827in}}{\pgfqpoint{4.788042in}{2.055927in}}{\pgfqpoint{4.793866in}{2.050103in}}%
\pgfpathcurveto{\pgfqpoint{4.799690in}{2.044279in}}{\pgfqpoint{4.807590in}{2.041007in}}{\pgfqpoint{4.815826in}{2.041007in}}%
\pgfpathclose%
\pgfusepath{stroke,fill}%
\end{pgfscope}%
\begin{pgfscope}%
\pgfpathrectangle{\pgfqpoint{3.755891in}{0.557870in}}{\pgfqpoint{2.484109in}{1.684734in}}%
\pgfusepath{clip}%
\pgfsetbuttcap%
\pgfsetroundjoin%
\definecolor{currentfill}{rgb}{0.298039,0.447059,0.690196}%
\pgfsetfillcolor{currentfill}%
\pgfsetlinewidth{1.003750pt}%
\definecolor{currentstroke}{rgb}{0.298039,0.447059,0.690196}%
\pgfsetstrokecolor{currentstroke}%
\pgfsetdash{}{0pt}%
\pgfpathmoveto{\pgfqpoint{4.888674in}{2.031611in}}%
\pgfpathcurveto{\pgfqpoint{4.896910in}{2.031611in}}{\pgfqpoint{4.904810in}{2.034883in}}{\pgfqpoint{4.910634in}{2.040707in}}%
\pgfpathcurveto{\pgfqpoint{4.916458in}{2.046531in}}{\pgfqpoint{4.919731in}{2.054431in}}{\pgfqpoint{4.919731in}{2.062667in}}%
\pgfpathcurveto{\pgfqpoint{4.919731in}{2.070904in}}{\pgfqpoint{4.916458in}{2.078804in}}{\pgfqpoint{4.910634in}{2.084628in}}%
\pgfpathcurveto{\pgfqpoint{4.904810in}{2.090452in}}{\pgfqpoint{4.896910in}{2.093724in}}{\pgfqpoint{4.888674in}{2.093724in}}%
\pgfpathcurveto{\pgfqpoint{4.880438in}{2.093724in}}{\pgfqpoint{4.872538in}{2.090452in}}{\pgfqpoint{4.866714in}{2.084628in}}%
\pgfpathcurveto{\pgfqpoint{4.860890in}{2.078804in}}{\pgfqpoint{4.857618in}{2.070904in}}{\pgfqpoint{4.857618in}{2.062667in}}%
\pgfpathcurveto{\pgfqpoint{4.857618in}{2.054431in}}{\pgfqpoint{4.860890in}{2.046531in}}{\pgfqpoint{4.866714in}{2.040707in}}%
\pgfpathcurveto{\pgfqpoint{4.872538in}{2.034883in}}{\pgfqpoint{4.880438in}{2.031611in}}{\pgfqpoint{4.888674in}{2.031611in}}%
\pgfpathclose%
\pgfusepath{stroke,fill}%
\end{pgfscope}%
\begin{pgfscope}%
\pgfpathrectangle{\pgfqpoint{3.755891in}{0.557870in}}{\pgfqpoint{2.484109in}{1.684734in}}%
\pgfusepath{clip}%
\pgfsetbuttcap%
\pgfsetroundjoin%
\definecolor{currentfill}{rgb}{0.298039,0.447059,0.690196}%
\pgfsetfillcolor{currentfill}%
\pgfsetlinewidth{1.003750pt}%
\definecolor{currentstroke}{rgb}{0.298039,0.447059,0.690196}%
\pgfsetstrokecolor{currentstroke}%
\pgfsetdash{}{0pt}%
\pgfpathmoveto{\pgfqpoint{5.689999in}{1.890668in}}%
\pgfpathcurveto{\pgfqpoint{5.698236in}{1.890668in}}{\pgfqpoint{5.706136in}{1.893941in}}{\pgfqpoint{5.711960in}{1.899765in}}%
\pgfpathcurveto{\pgfqpoint{5.717784in}{1.905589in}}{\pgfqpoint{5.721056in}{1.913489in}}{\pgfqpoint{5.721056in}{1.921725in}}%
\pgfpathcurveto{\pgfqpoint{5.721056in}{1.929961in}}{\pgfqpoint{5.717784in}{1.937861in}}{\pgfqpoint{5.711960in}{1.943685in}}%
\pgfpathcurveto{\pgfqpoint{5.706136in}{1.949509in}}{\pgfqpoint{5.698236in}{1.952781in}}{\pgfqpoint{5.689999in}{1.952781in}}%
\pgfpathcurveto{\pgfqpoint{5.681763in}{1.952781in}}{\pgfqpoint{5.673863in}{1.949509in}}{\pgfqpoint{5.668039in}{1.943685in}}%
\pgfpathcurveto{\pgfqpoint{5.662215in}{1.937861in}}{\pgfqpoint{5.658943in}{1.929961in}}{\pgfqpoint{5.658943in}{1.921725in}}%
\pgfpathcurveto{\pgfqpoint{5.658943in}{1.913489in}}{\pgfqpoint{5.662215in}{1.905589in}}{\pgfqpoint{5.668039in}{1.899765in}}%
\pgfpathcurveto{\pgfqpoint{5.673863in}{1.893941in}}{\pgfqpoint{5.681763in}{1.890668in}}{\pgfqpoint{5.689999in}{1.890668in}}%
\pgfpathclose%
\pgfusepath{stroke,fill}%
\end{pgfscope}%
\begin{pgfscope}%
\pgfpathrectangle{\pgfqpoint{3.755891in}{0.557870in}}{\pgfqpoint{2.484109in}{1.684734in}}%
\pgfusepath{clip}%
\pgfsetbuttcap%
\pgfsetroundjoin%
\definecolor{currentfill}{rgb}{0.298039,0.447059,0.690196}%
\pgfsetfillcolor{currentfill}%
\pgfsetlinewidth{1.003750pt}%
\definecolor{currentstroke}{rgb}{0.298039,0.447059,0.690196}%
\pgfsetstrokecolor{currentstroke}%
\pgfsetdash{}{0pt}%
\pgfpathmoveto{\pgfqpoint{4.597283in}{2.041007in}}%
\pgfpathcurveto{\pgfqpoint{4.605519in}{2.041007in}}{\pgfqpoint{4.613419in}{2.044279in}}{\pgfqpoint{4.619243in}{2.050103in}}%
\pgfpathcurveto{\pgfqpoint{4.625067in}{2.055927in}}{\pgfqpoint{4.628339in}{2.063827in}}{\pgfqpoint{4.628339in}{2.072064in}}%
\pgfpathcurveto{\pgfqpoint{4.628339in}{2.080300in}}{\pgfqpoint{4.625067in}{2.088200in}}{\pgfqpoint{4.619243in}{2.094024in}}%
\pgfpathcurveto{\pgfqpoint{4.613419in}{2.099848in}}{\pgfqpoint{4.605519in}{2.103120in}}{\pgfqpoint{4.597283in}{2.103120in}}%
\pgfpathcurveto{\pgfqpoint{4.589047in}{2.103120in}}{\pgfqpoint{4.581147in}{2.099848in}}{\pgfqpoint{4.575323in}{2.094024in}}%
\pgfpathcurveto{\pgfqpoint{4.569499in}{2.088200in}}{\pgfqpoint{4.566226in}{2.080300in}}{\pgfqpoint{4.566226in}{2.072064in}}%
\pgfpathcurveto{\pgfqpoint{4.566226in}{2.063827in}}{\pgfqpoint{4.569499in}{2.055927in}}{\pgfqpoint{4.575323in}{2.050103in}}%
\pgfpathcurveto{\pgfqpoint{4.581147in}{2.044279in}}{\pgfqpoint{4.589047in}{2.041007in}}{\pgfqpoint{4.597283in}{2.041007in}}%
\pgfpathclose%
\pgfusepath{stroke,fill}%
\end{pgfscope}%
\begin{pgfscope}%
\pgfpathrectangle{\pgfqpoint{3.755891in}{0.557870in}}{\pgfqpoint{2.484109in}{1.684734in}}%
\pgfusepath{clip}%
\pgfsetbuttcap%
\pgfsetroundjoin%
\definecolor{currentfill}{rgb}{0.298039,0.447059,0.690196}%
\pgfsetfillcolor{currentfill}%
\pgfsetlinewidth{1.003750pt}%
\definecolor{currentstroke}{rgb}{0.298039,0.447059,0.690196}%
\pgfsetstrokecolor{currentstroke}%
\pgfsetdash{}{0pt}%
\pgfpathmoveto{\pgfqpoint{4.755987in}{2.029011in}}%
\pgfpathcurveto{\pgfqpoint{4.764223in}{2.029011in}}{\pgfqpoint{4.772123in}{2.032283in}}{\pgfqpoint{4.777947in}{2.038107in}}%
\pgfpathcurveto{\pgfqpoint{4.783771in}{2.043931in}}{\pgfqpoint{4.787044in}{2.051831in}}{\pgfqpoint{4.787044in}{2.060067in}}%
\pgfpathcurveto{\pgfqpoint{4.787044in}{2.068304in}}{\pgfqpoint{4.783771in}{2.076204in}}{\pgfqpoint{4.777947in}{2.082028in}}%
\pgfpathcurveto{\pgfqpoint{4.772123in}{2.087851in}}{\pgfqpoint{4.764223in}{2.091124in}}{\pgfqpoint{4.755987in}{2.091124in}}%
\pgfpathcurveto{\pgfqpoint{4.747751in}{2.091124in}}{\pgfqpoint{4.739851in}{2.087851in}}{\pgfqpoint{4.734027in}{2.082028in}}%
\pgfpathcurveto{\pgfqpoint{4.728203in}{2.076204in}}{\pgfqpoint{4.724931in}{2.068304in}}{\pgfqpoint{4.724931in}{2.060067in}}%
\pgfpathcurveto{\pgfqpoint{4.724931in}{2.051831in}}{\pgfqpoint{4.728203in}{2.043931in}}{\pgfqpoint{4.734027in}{2.038107in}}%
\pgfpathcurveto{\pgfqpoint{4.739851in}{2.032283in}}{\pgfqpoint{4.747751in}{2.029011in}}{\pgfqpoint{4.755987in}{2.029011in}}%
\pgfpathclose%
\pgfusepath{stroke,fill}%
\end{pgfscope}%
\begin{pgfscope}%
\pgfpathrectangle{\pgfqpoint{3.755891in}{0.557870in}}{\pgfqpoint{2.484109in}{1.684734in}}%
\pgfusepath{clip}%
\pgfsetbuttcap%
\pgfsetroundjoin%
\definecolor{currentfill}{rgb}{0.298039,0.447059,0.690196}%
\pgfsetfillcolor{currentfill}%
\pgfsetlinewidth{1.003750pt}%
\definecolor{currentstroke}{rgb}{0.298039,0.447059,0.690196}%
\pgfsetstrokecolor{currentstroke}%
\pgfsetdash{}{0pt}%
\pgfpathmoveto{\pgfqpoint{5.481863in}{2.057908in}}%
\pgfpathcurveto{\pgfqpoint{5.490099in}{2.057908in}}{\pgfqpoint{5.497999in}{2.061181in}}{\pgfqpoint{5.503823in}{2.067005in}}%
\pgfpathcurveto{\pgfqpoint{5.509647in}{2.072829in}}{\pgfqpoint{5.512919in}{2.080729in}}{\pgfqpoint{5.512919in}{2.088965in}}%
\pgfpathcurveto{\pgfqpoint{5.512919in}{2.097201in}}{\pgfqpoint{5.509647in}{2.105101in}}{\pgfqpoint{5.503823in}{2.110925in}}%
\pgfpathcurveto{\pgfqpoint{5.497999in}{2.116749in}}{\pgfqpoint{5.490099in}{2.120021in}}{\pgfqpoint{5.481863in}{2.120021in}}%
\pgfpathcurveto{\pgfqpoint{5.473627in}{2.120021in}}{\pgfqpoint{5.465727in}{2.116749in}}{\pgfqpoint{5.459903in}{2.110925in}}%
\pgfpathcurveto{\pgfqpoint{5.454079in}{2.105101in}}{\pgfqpoint{5.450806in}{2.097201in}}{\pgfqpoint{5.450806in}{2.088965in}}%
\pgfpathcurveto{\pgfqpoint{5.450806in}{2.080729in}}{\pgfqpoint{5.454079in}{2.072829in}}{\pgfqpoint{5.459903in}{2.067005in}}%
\pgfpathcurveto{\pgfqpoint{5.465727in}{2.061181in}}{\pgfqpoint{5.473627in}{2.057908in}}{\pgfqpoint{5.481863in}{2.057908in}}%
\pgfpathclose%
\pgfusepath{stroke,fill}%
\end{pgfscope}%
\begin{pgfscope}%
\pgfpathrectangle{\pgfqpoint{3.755891in}{0.557870in}}{\pgfqpoint{2.484109in}{1.684734in}}%
\pgfusepath{clip}%
\pgfsetbuttcap%
\pgfsetroundjoin%
\definecolor{currentfill}{rgb}{0.298039,0.447059,0.690196}%
\pgfsetfillcolor{currentfill}%
\pgfsetlinewidth{1.003750pt}%
\definecolor{currentstroke}{rgb}{0.298039,0.447059,0.690196}%
\pgfsetstrokecolor{currentstroke}%
\pgfsetdash{}{0pt}%
\pgfpathmoveto{\pgfqpoint{5.320557in}{2.096439in}}%
\pgfpathcurveto{\pgfqpoint{5.328793in}{2.096439in}}{\pgfqpoint{5.336694in}{2.099711in}}{\pgfqpoint{5.342517in}{2.105535in}}%
\pgfpathcurveto{\pgfqpoint{5.348341in}{2.111359in}}{\pgfqpoint{5.351614in}{2.119259in}}{\pgfqpoint{5.351614in}{2.127495in}}%
\pgfpathcurveto{\pgfqpoint{5.351614in}{2.135731in}}{\pgfqpoint{5.348341in}{2.143631in}}{\pgfqpoint{5.342517in}{2.149455in}}%
\pgfpathcurveto{\pgfqpoint{5.336694in}{2.155279in}}{\pgfqpoint{5.328793in}{2.158552in}}{\pgfqpoint{5.320557in}{2.158552in}}%
\pgfpathcurveto{\pgfqpoint{5.312321in}{2.158552in}}{\pgfqpoint{5.304421in}{2.155279in}}{\pgfqpoint{5.298597in}{2.149455in}}%
\pgfpathcurveto{\pgfqpoint{5.292773in}{2.143631in}}{\pgfqpoint{5.289501in}{2.135731in}}{\pgfqpoint{5.289501in}{2.127495in}}%
\pgfpathcurveto{\pgfqpoint{5.289501in}{2.119259in}}{\pgfqpoint{5.292773in}{2.111359in}}{\pgfqpoint{5.298597in}{2.105535in}}%
\pgfpathcurveto{\pgfqpoint{5.304421in}{2.099711in}}{\pgfqpoint{5.312321in}{2.096439in}}{\pgfqpoint{5.320557in}{2.096439in}}%
\pgfpathclose%
\pgfusepath{stroke,fill}%
\end{pgfscope}%
\begin{pgfscope}%
\pgfpathrectangle{\pgfqpoint{3.755891in}{0.557870in}}{\pgfqpoint{2.484109in}{1.684734in}}%
\pgfusepath{clip}%
\pgfsetbuttcap%
\pgfsetroundjoin%
\definecolor{currentfill}{rgb}{0.298039,0.447059,0.690196}%
\pgfsetfillcolor{currentfill}%
\pgfsetlinewidth{1.003750pt}%
\definecolor{currentstroke}{rgb}{0.298039,0.447059,0.690196}%
\pgfsetstrokecolor{currentstroke}%
\pgfsetdash{}{0pt}%
\pgfpathmoveto{\pgfqpoint{5.481863in}{2.019378in}}%
\pgfpathcurveto{\pgfqpoint{5.490099in}{2.019378in}}{\pgfqpoint{5.497999in}{2.022651in}}{\pgfqpoint{5.503823in}{2.028474in}}%
\pgfpathcurveto{\pgfqpoint{5.509647in}{2.034298in}}{\pgfqpoint{5.512919in}{2.042198in}}{\pgfqpoint{5.512919in}{2.050435in}}%
\pgfpathcurveto{\pgfqpoint{5.512919in}{2.058671in}}{\pgfqpoint{5.509647in}{2.066571in}}{\pgfqpoint{5.503823in}{2.072395in}}%
\pgfpathcurveto{\pgfqpoint{5.497999in}{2.078219in}}{\pgfqpoint{5.490099in}{2.081491in}}{\pgfqpoint{5.481863in}{2.081491in}}%
\pgfpathcurveto{\pgfqpoint{5.473627in}{2.081491in}}{\pgfqpoint{5.465727in}{2.078219in}}{\pgfqpoint{5.459903in}{2.072395in}}%
\pgfpathcurveto{\pgfqpoint{5.454079in}{2.066571in}}{\pgfqpoint{5.450806in}{2.058671in}}{\pgfqpoint{5.450806in}{2.050435in}}%
\pgfpathcurveto{\pgfqpoint{5.450806in}{2.042198in}}{\pgfqpoint{5.454079in}{2.034298in}}{\pgfqpoint{5.459903in}{2.028474in}}%
\pgfpathcurveto{\pgfqpoint{5.465727in}{2.022651in}}{\pgfqpoint{5.473627in}{2.019378in}}{\pgfqpoint{5.481863in}{2.019378in}}%
\pgfpathclose%
\pgfusepath{stroke,fill}%
\end{pgfscope}%
\begin{pgfscope}%
\pgfpathrectangle{\pgfqpoint{3.755891in}{0.557870in}}{\pgfqpoint{2.484109in}{1.684734in}}%
\pgfusepath{clip}%
\pgfsetbuttcap%
\pgfsetroundjoin%
\definecolor{currentfill}{rgb}{0.298039,0.447059,0.690196}%
\pgfsetfillcolor{currentfill}%
\pgfsetlinewidth{1.003750pt}%
\definecolor{currentstroke}{rgb}{0.298039,0.447059,0.690196}%
\pgfsetstrokecolor{currentstroke}%
\pgfsetdash{}{0pt}%
\pgfpathmoveto{\pgfqpoint{5.078599in}{2.038643in}}%
\pgfpathcurveto{\pgfqpoint{5.086835in}{2.038643in}}{\pgfqpoint{5.094735in}{2.041916in}}{\pgfqpoint{5.100559in}{2.047740in}}%
\pgfpathcurveto{\pgfqpoint{5.106383in}{2.053563in}}{\pgfqpoint{5.109655in}{2.061464in}}{\pgfqpoint{5.109655in}{2.069700in}}%
\pgfpathcurveto{\pgfqpoint{5.109655in}{2.077936in}}{\pgfqpoint{5.106383in}{2.085836in}}{\pgfqpoint{5.100559in}{2.091660in}}%
\pgfpathcurveto{\pgfqpoint{5.094735in}{2.097484in}}{\pgfqpoint{5.086835in}{2.100756in}}{\pgfqpoint{5.078599in}{2.100756in}}%
\pgfpathcurveto{\pgfqpoint{5.070362in}{2.100756in}}{\pgfqpoint{5.062462in}{2.097484in}}{\pgfqpoint{5.056638in}{2.091660in}}%
\pgfpathcurveto{\pgfqpoint{5.050814in}{2.085836in}}{\pgfqpoint{5.047542in}{2.077936in}}{\pgfqpoint{5.047542in}{2.069700in}}%
\pgfpathcurveto{\pgfqpoint{5.047542in}{2.061464in}}{\pgfqpoint{5.050814in}{2.053563in}}{\pgfqpoint{5.056638in}{2.047740in}}%
\pgfpathcurveto{\pgfqpoint{5.062462in}{2.041916in}}{\pgfqpoint{5.070362in}{2.038643in}}{\pgfqpoint{5.078599in}{2.038643in}}%
\pgfpathclose%
\pgfusepath{stroke,fill}%
\end{pgfscope}%
\begin{pgfscope}%
\pgfpathrectangle{\pgfqpoint{3.755891in}{0.557870in}}{\pgfqpoint{2.484109in}{1.684734in}}%
\pgfusepath{clip}%
\pgfsetbuttcap%
\pgfsetroundjoin%
\definecolor{currentfill}{rgb}{0.298039,0.447059,0.690196}%
\pgfsetfillcolor{currentfill}%
\pgfsetlinewidth{1.003750pt}%
\definecolor{currentstroke}{rgb}{0.298039,0.447059,0.690196}%
\pgfsetstrokecolor{currentstroke}%
\pgfsetdash{}{0pt}%
\pgfpathmoveto{\pgfqpoint{5.320557in}{2.057908in}}%
\pgfpathcurveto{\pgfqpoint{5.328793in}{2.057908in}}{\pgfqpoint{5.336694in}{2.061181in}}{\pgfqpoint{5.342517in}{2.067005in}}%
\pgfpathcurveto{\pgfqpoint{5.348341in}{2.072829in}}{\pgfqpoint{5.351614in}{2.080729in}}{\pgfqpoint{5.351614in}{2.088965in}}%
\pgfpathcurveto{\pgfqpoint{5.351614in}{2.097201in}}{\pgfqpoint{5.348341in}{2.105101in}}{\pgfqpoint{5.342517in}{2.110925in}}%
\pgfpathcurveto{\pgfqpoint{5.336694in}{2.116749in}}{\pgfqpoint{5.328793in}{2.120021in}}{\pgfqpoint{5.320557in}{2.120021in}}%
\pgfpathcurveto{\pgfqpoint{5.312321in}{2.120021in}}{\pgfqpoint{5.304421in}{2.116749in}}{\pgfqpoint{5.298597in}{2.110925in}}%
\pgfpathcurveto{\pgfqpoint{5.292773in}{2.105101in}}{\pgfqpoint{5.289501in}{2.097201in}}{\pgfqpoint{5.289501in}{2.088965in}}%
\pgfpathcurveto{\pgfqpoint{5.289501in}{2.080729in}}{\pgfqpoint{5.292773in}{2.072829in}}{\pgfqpoint{5.298597in}{2.067005in}}%
\pgfpathcurveto{\pgfqpoint{5.304421in}{2.061181in}}{\pgfqpoint{5.312321in}{2.057908in}}{\pgfqpoint{5.320557in}{2.057908in}}%
\pgfpathclose%
\pgfusepath{stroke,fill}%
\end{pgfscope}%
\begin{pgfscope}%
\pgfpathrectangle{\pgfqpoint{3.755891in}{0.557870in}}{\pgfqpoint{2.484109in}{1.684734in}}%
\pgfusepath{clip}%
\pgfsetbuttcap%
\pgfsetroundjoin%
\definecolor{currentfill}{rgb}{0.298039,0.447059,0.690196}%
\pgfsetfillcolor{currentfill}%
\pgfsetlinewidth{1.003750pt}%
\definecolor{currentstroke}{rgb}{0.298039,0.447059,0.690196}%
\pgfsetstrokecolor{currentstroke}%
\pgfsetdash{}{0pt}%
\pgfpathmoveto{\pgfqpoint{5.078599in}{2.086806in}}%
\pgfpathcurveto{\pgfqpoint{5.086835in}{2.086806in}}{\pgfqpoint{5.094735in}{2.090078in}}{\pgfqpoint{5.100559in}{2.095902in}}%
\pgfpathcurveto{\pgfqpoint{5.106383in}{2.101726in}}{\pgfqpoint{5.109655in}{2.109626in}}{\pgfqpoint{5.109655in}{2.117863in}}%
\pgfpathcurveto{\pgfqpoint{5.109655in}{2.126099in}}{\pgfqpoint{5.106383in}{2.133999in}}{\pgfqpoint{5.100559in}{2.139823in}}%
\pgfpathcurveto{\pgfqpoint{5.094735in}{2.145647in}}{\pgfqpoint{5.086835in}{2.148919in}}{\pgfqpoint{5.078599in}{2.148919in}}%
\pgfpathcurveto{\pgfqpoint{5.070362in}{2.148919in}}{\pgfqpoint{5.062462in}{2.145647in}}{\pgfqpoint{5.056638in}{2.139823in}}%
\pgfpathcurveto{\pgfqpoint{5.050814in}{2.133999in}}{\pgfqpoint{5.047542in}{2.126099in}}{\pgfqpoint{5.047542in}{2.117863in}}%
\pgfpathcurveto{\pgfqpoint{5.047542in}{2.109626in}}{\pgfqpoint{5.050814in}{2.101726in}}{\pgfqpoint{5.056638in}{2.095902in}}%
\pgfpathcurveto{\pgfqpoint{5.062462in}{2.090078in}}{\pgfqpoint{5.070362in}{2.086806in}}{\pgfqpoint{5.078599in}{2.086806in}}%
\pgfpathclose%
\pgfusepath{stroke,fill}%
\end{pgfscope}%
\begin{pgfscope}%
\pgfpathrectangle{\pgfqpoint{3.755891in}{0.557870in}}{\pgfqpoint{2.484109in}{1.684734in}}%
\pgfusepath{clip}%
\pgfsetbuttcap%
\pgfsetroundjoin%
\definecolor{currentfill}{rgb}{0.298039,0.447059,0.690196}%
\pgfsetfillcolor{currentfill}%
\pgfsetlinewidth{1.003750pt}%
\definecolor{currentstroke}{rgb}{0.298039,0.447059,0.690196}%
\pgfsetstrokecolor{currentstroke}%
\pgfsetdash{}{0pt}%
\pgfpathmoveto{\pgfqpoint{5.401210in}{2.086806in}}%
\pgfpathcurveto{\pgfqpoint{5.409446in}{2.086806in}}{\pgfqpoint{5.417346in}{2.090078in}}{\pgfqpoint{5.423170in}{2.095902in}}%
\pgfpathcurveto{\pgfqpoint{5.428994in}{2.101726in}}{\pgfqpoint{5.432267in}{2.109626in}}{\pgfqpoint{5.432267in}{2.117863in}}%
\pgfpathcurveto{\pgfqpoint{5.432267in}{2.126099in}}{\pgfqpoint{5.428994in}{2.133999in}}{\pgfqpoint{5.423170in}{2.139823in}}%
\pgfpathcurveto{\pgfqpoint{5.417346in}{2.145647in}}{\pgfqpoint{5.409446in}{2.148919in}}{\pgfqpoint{5.401210in}{2.148919in}}%
\pgfpathcurveto{\pgfqpoint{5.392974in}{2.148919in}}{\pgfqpoint{5.385074in}{2.145647in}}{\pgfqpoint{5.379250in}{2.139823in}}%
\pgfpathcurveto{\pgfqpoint{5.373426in}{2.133999in}}{\pgfqpoint{5.370154in}{2.126099in}}{\pgfqpoint{5.370154in}{2.117863in}}%
\pgfpathcurveto{\pgfqpoint{5.370154in}{2.109626in}}{\pgfqpoint{5.373426in}{2.101726in}}{\pgfqpoint{5.379250in}{2.095902in}}%
\pgfpathcurveto{\pgfqpoint{5.385074in}{2.090078in}}{\pgfqpoint{5.392974in}{2.086806in}}{\pgfqpoint{5.401210in}{2.086806in}}%
\pgfpathclose%
\pgfusepath{stroke,fill}%
\end{pgfscope}%
\begin{pgfscope}%
\pgfpathrectangle{\pgfqpoint{3.755891in}{0.557870in}}{\pgfqpoint{2.484109in}{1.684734in}}%
\pgfusepath{clip}%
\pgfsetbuttcap%
\pgfsetroundjoin%
\definecolor{currentfill}{rgb}{0.298039,0.447059,0.690196}%
\pgfsetfillcolor{currentfill}%
\pgfsetlinewidth{1.003750pt}%
\definecolor{currentstroke}{rgb}{0.298039,0.447059,0.690196}%
\pgfsetstrokecolor{currentstroke}%
\pgfsetdash{}{0pt}%
\pgfpathmoveto{\pgfqpoint{5.885127in}{1.768932in}}%
\pgfpathcurveto{\pgfqpoint{5.893364in}{1.768932in}}{\pgfqpoint{5.901264in}{1.772204in}}{\pgfqpoint{5.907088in}{1.778028in}}%
\pgfpathcurveto{\pgfqpoint{5.912912in}{1.783852in}}{\pgfqpoint{5.916184in}{1.791752in}}{\pgfqpoint{5.916184in}{1.799988in}}%
\pgfpathcurveto{\pgfqpoint{5.916184in}{1.808225in}}{\pgfqpoint{5.912912in}{1.816125in}}{\pgfqpoint{5.907088in}{1.821949in}}%
\pgfpathcurveto{\pgfqpoint{5.901264in}{1.827772in}}{\pgfqpoint{5.893364in}{1.831045in}}{\pgfqpoint{5.885127in}{1.831045in}}%
\pgfpathcurveto{\pgfqpoint{5.876891in}{1.831045in}}{\pgfqpoint{5.868991in}{1.827772in}}{\pgfqpoint{5.863167in}{1.821949in}}%
\pgfpathcurveto{\pgfqpoint{5.857343in}{1.816125in}}{\pgfqpoint{5.854071in}{1.808225in}}{\pgfqpoint{5.854071in}{1.799988in}}%
\pgfpathcurveto{\pgfqpoint{5.854071in}{1.791752in}}{\pgfqpoint{5.857343in}{1.783852in}}{\pgfqpoint{5.863167in}{1.778028in}}%
\pgfpathcurveto{\pgfqpoint{5.868991in}{1.772204in}}{\pgfqpoint{5.876891in}{1.768932in}}{\pgfqpoint{5.885127in}{1.768932in}}%
\pgfpathclose%
\pgfusepath{stroke,fill}%
\end{pgfscope}%
\begin{pgfscope}%
\pgfpathrectangle{\pgfqpoint{3.755891in}{0.557870in}}{\pgfqpoint{2.484109in}{1.684734in}}%
\pgfusepath{clip}%
\pgfsetbuttcap%
\pgfsetroundjoin%
\definecolor{currentfill}{rgb}{0.298039,0.447059,0.690196}%
\pgfsetfillcolor{currentfill}%
\pgfsetlinewidth{1.003750pt}%
\definecolor{currentstroke}{rgb}{0.298039,0.447059,0.690196}%
\pgfsetstrokecolor{currentstroke}%
\pgfsetdash{}{0pt}%
\pgfpathmoveto{\pgfqpoint{4.110764in}{2.115704in}}%
\pgfpathcurveto{\pgfqpoint{4.119000in}{2.115704in}}{\pgfqpoint{4.126900in}{2.118976in}}{\pgfqpoint{4.132724in}{2.124800in}}%
\pgfpathcurveto{\pgfqpoint{4.138548in}{2.130624in}}{\pgfqpoint{4.141820in}{2.138524in}}{\pgfqpoint{4.141820in}{2.146760in}}%
\pgfpathcurveto{\pgfqpoint{4.141820in}{2.154997in}}{\pgfqpoint{4.138548in}{2.162897in}}{\pgfqpoint{4.132724in}{2.168721in}}%
\pgfpathcurveto{\pgfqpoint{4.126900in}{2.174544in}}{\pgfqpoint{4.119000in}{2.177817in}}{\pgfqpoint{4.110764in}{2.177817in}}%
\pgfpathcurveto{\pgfqpoint{4.102528in}{2.177817in}}{\pgfqpoint{4.094628in}{2.174544in}}{\pgfqpoint{4.088804in}{2.168721in}}%
\pgfpathcurveto{\pgfqpoint{4.082980in}{2.162897in}}{\pgfqpoint{4.079707in}{2.154997in}}{\pgfqpoint{4.079707in}{2.146760in}}%
\pgfpathcurveto{\pgfqpoint{4.079707in}{2.138524in}}{\pgfqpoint{4.082980in}{2.130624in}}{\pgfqpoint{4.088804in}{2.124800in}}%
\pgfpathcurveto{\pgfqpoint{4.094628in}{2.118976in}}{\pgfqpoint{4.102528in}{2.115704in}}{\pgfqpoint{4.110764in}{2.115704in}}%
\pgfpathclose%
\pgfusepath{stroke,fill}%
\end{pgfscope}%
\begin{pgfscope}%
\pgfpathrectangle{\pgfqpoint{3.755891in}{0.557870in}}{\pgfqpoint{2.484109in}{1.684734in}}%
\pgfusepath{clip}%
\pgfsetbuttcap%
\pgfsetroundjoin%
\definecolor{currentfill}{rgb}{0.298039,0.447059,0.690196}%
\pgfsetfillcolor{currentfill}%
\pgfsetlinewidth{1.003750pt}%
\definecolor{currentstroke}{rgb}{0.298039,0.447059,0.690196}%
\pgfsetstrokecolor{currentstroke}%
\pgfsetdash{}{0pt}%
\pgfpathmoveto{\pgfqpoint{5.159251in}{2.036157in}}%
\pgfpathcurveto{\pgfqpoint{5.167488in}{2.036157in}}{\pgfqpoint{5.175388in}{2.039430in}}{\pgfqpoint{5.181212in}{2.045254in}}%
\pgfpathcurveto{\pgfqpoint{5.187036in}{2.051078in}}{\pgfqpoint{5.190308in}{2.058978in}}{\pgfqpoint{5.190308in}{2.067214in}}%
\pgfpathcurveto{\pgfqpoint{5.190308in}{2.075450in}}{\pgfqpoint{5.187036in}{2.083350in}}{\pgfqpoint{5.181212in}{2.089174in}}%
\pgfpathcurveto{\pgfqpoint{5.175388in}{2.094998in}}{\pgfqpoint{5.167488in}{2.098270in}}{\pgfqpoint{5.159251in}{2.098270in}}%
\pgfpathcurveto{\pgfqpoint{5.151015in}{2.098270in}}{\pgfqpoint{5.143115in}{2.094998in}}{\pgfqpoint{5.137291in}{2.089174in}}%
\pgfpathcurveto{\pgfqpoint{5.131467in}{2.083350in}}{\pgfqpoint{5.128195in}{2.075450in}}{\pgfqpoint{5.128195in}{2.067214in}}%
\pgfpathcurveto{\pgfqpoint{5.128195in}{2.058978in}}{\pgfqpoint{5.131467in}{2.051078in}}{\pgfqpoint{5.137291in}{2.045254in}}%
\pgfpathcurveto{\pgfqpoint{5.143115in}{2.039430in}}{\pgfqpoint{5.151015in}{2.036157in}}{\pgfqpoint{5.159251in}{2.036157in}}%
\pgfpathclose%
\pgfusepath{stroke,fill}%
\end{pgfscope}%
\begin{pgfscope}%
\pgfpathrectangle{\pgfqpoint{3.755891in}{0.557870in}}{\pgfqpoint{2.484109in}{1.684734in}}%
\pgfusepath{clip}%
\pgfsetbuttcap%
\pgfsetroundjoin%
\definecolor{currentfill}{rgb}{0.298039,0.447059,0.690196}%
\pgfsetfillcolor{currentfill}%
\pgfsetlinewidth{1.003750pt}%
\definecolor{currentstroke}{rgb}{0.298039,0.447059,0.690196}%
\pgfsetstrokecolor{currentstroke}%
\pgfsetdash{}{0pt}%
\pgfpathmoveto{\pgfqpoint{5.723822in}{2.055920in}}%
\pgfpathcurveto{\pgfqpoint{5.732058in}{2.055920in}}{\pgfqpoint{5.739958in}{2.059192in}}{\pgfqpoint{5.745782in}{2.065016in}}%
\pgfpathcurveto{\pgfqpoint{5.751606in}{2.070840in}}{\pgfqpoint{5.754878in}{2.078740in}}{\pgfqpoint{5.754878in}{2.086976in}}%
\pgfpathcurveto{\pgfqpoint{5.754878in}{2.095213in}}{\pgfqpoint{5.751606in}{2.103113in}}{\pgfqpoint{5.745782in}{2.108937in}}%
\pgfpathcurveto{\pgfqpoint{5.739958in}{2.114760in}}{\pgfqpoint{5.732058in}{2.118033in}}{\pgfqpoint{5.723822in}{2.118033in}}%
\pgfpathcurveto{\pgfqpoint{5.715585in}{2.118033in}}{\pgfqpoint{5.707685in}{2.114760in}}{\pgfqpoint{5.701861in}{2.108937in}}%
\pgfpathcurveto{\pgfqpoint{5.696037in}{2.103113in}}{\pgfqpoint{5.692765in}{2.095213in}}{\pgfqpoint{5.692765in}{2.086976in}}%
\pgfpathcurveto{\pgfqpoint{5.692765in}{2.078740in}}{\pgfqpoint{5.696037in}{2.070840in}}{\pgfqpoint{5.701861in}{2.065016in}}%
\pgfpathcurveto{\pgfqpoint{5.707685in}{2.059192in}}{\pgfqpoint{5.715585in}{2.055920in}}{\pgfqpoint{5.723822in}{2.055920in}}%
\pgfpathclose%
\pgfusepath{stroke,fill}%
\end{pgfscope}%
\begin{pgfscope}%
\pgfpathrectangle{\pgfqpoint{3.755891in}{0.557870in}}{\pgfqpoint{2.484109in}{1.684734in}}%
\pgfusepath{clip}%
\pgfsetbuttcap%
\pgfsetroundjoin%
\definecolor{currentfill}{rgb}{0.298039,0.447059,0.690196}%
\pgfsetfillcolor{currentfill}%
\pgfsetlinewidth{1.003750pt}%
\definecolor{currentstroke}{rgb}{0.298039,0.447059,0.690196}%
\pgfsetstrokecolor{currentstroke}%
\pgfsetdash{}{0pt}%
\pgfpathmoveto{\pgfqpoint{5.320557in}{2.075682in}}%
\pgfpathcurveto{\pgfqpoint{5.328793in}{2.075682in}}{\pgfqpoint{5.336694in}{2.078954in}}{\pgfqpoint{5.342517in}{2.084778in}}%
\pgfpathcurveto{\pgfqpoint{5.348341in}{2.090602in}}{\pgfqpoint{5.351614in}{2.098502in}}{\pgfqpoint{5.351614in}{2.106739in}}%
\pgfpathcurveto{\pgfqpoint{5.351614in}{2.114975in}}{\pgfqpoint{5.348341in}{2.122875in}}{\pgfqpoint{5.342517in}{2.128699in}}%
\pgfpathcurveto{\pgfqpoint{5.336694in}{2.134523in}}{\pgfqpoint{5.328793in}{2.137795in}}{\pgfqpoint{5.320557in}{2.137795in}}%
\pgfpathcurveto{\pgfqpoint{5.312321in}{2.137795in}}{\pgfqpoint{5.304421in}{2.134523in}}{\pgfqpoint{5.298597in}{2.128699in}}%
\pgfpathcurveto{\pgfqpoint{5.292773in}{2.122875in}}{\pgfqpoint{5.289501in}{2.114975in}}{\pgfqpoint{5.289501in}{2.106739in}}%
\pgfpathcurveto{\pgfqpoint{5.289501in}{2.098502in}}{\pgfqpoint{5.292773in}{2.090602in}}{\pgfqpoint{5.298597in}{2.084778in}}%
\pgfpathcurveto{\pgfqpoint{5.304421in}{2.078954in}}{\pgfqpoint{5.312321in}{2.075682in}}{\pgfqpoint{5.320557in}{2.075682in}}%
\pgfpathclose%
\pgfusepath{stroke,fill}%
\end{pgfscope}%
\begin{pgfscope}%
\pgfpathrectangle{\pgfqpoint{3.755891in}{0.557870in}}{\pgfqpoint{2.484109in}{1.684734in}}%
\pgfusepath{clip}%
\pgfsetbuttcap%
\pgfsetroundjoin%
\definecolor{currentfill}{rgb}{0.298039,0.447059,0.690196}%
\pgfsetfillcolor{currentfill}%
\pgfsetlinewidth{1.003750pt}%
\definecolor{currentstroke}{rgb}{0.298039,0.447059,0.690196}%
\pgfsetstrokecolor{currentstroke}%
\pgfsetdash{}{0pt}%
\pgfpathmoveto{\pgfqpoint{5.320557in}{2.006514in}}%
\pgfpathcurveto{\pgfqpoint{5.328793in}{2.006514in}}{\pgfqpoint{5.336694in}{2.009786in}}{\pgfqpoint{5.342517in}{2.015610in}}%
\pgfpathcurveto{\pgfqpoint{5.348341in}{2.021434in}}{\pgfqpoint{5.351614in}{2.029334in}}{\pgfqpoint{5.351614in}{2.037571in}}%
\pgfpathcurveto{\pgfqpoint{5.351614in}{2.045807in}}{\pgfqpoint{5.348341in}{2.053707in}}{\pgfqpoint{5.342517in}{2.059531in}}%
\pgfpathcurveto{\pgfqpoint{5.336694in}{2.065355in}}{\pgfqpoint{5.328793in}{2.068627in}}{\pgfqpoint{5.320557in}{2.068627in}}%
\pgfpathcurveto{\pgfqpoint{5.312321in}{2.068627in}}{\pgfqpoint{5.304421in}{2.065355in}}{\pgfqpoint{5.298597in}{2.059531in}}%
\pgfpathcurveto{\pgfqpoint{5.292773in}{2.053707in}}{\pgfqpoint{5.289501in}{2.045807in}}{\pgfqpoint{5.289501in}{2.037571in}}%
\pgfpathcurveto{\pgfqpoint{5.289501in}{2.029334in}}{\pgfqpoint{5.292773in}{2.021434in}}{\pgfqpoint{5.298597in}{2.015610in}}%
\pgfpathcurveto{\pgfqpoint{5.304421in}{2.009786in}}{\pgfqpoint{5.312321in}{2.006514in}}{\pgfqpoint{5.320557in}{2.006514in}}%
\pgfpathclose%
\pgfusepath{stroke,fill}%
\end{pgfscope}%
\begin{pgfscope}%
\pgfpathrectangle{\pgfqpoint{3.755891in}{0.557870in}}{\pgfqpoint{2.484109in}{1.684734in}}%
\pgfusepath{clip}%
\pgfsetbuttcap%
\pgfsetroundjoin%
\definecolor{currentfill}{rgb}{0.298039,0.447059,0.690196}%
\pgfsetfillcolor{currentfill}%
\pgfsetlinewidth{1.003750pt}%
\definecolor{currentstroke}{rgb}{0.298039,0.447059,0.690196}%
\pgfsetstrokecolor{currentstroke}%
\pgfsetdash{}{0pt}%
\pgfpathmoveto{\pgfqpoint{5.481863in}{2.055920in}}%
\pgfpathcurveto{\pgfqpoint{5.490099in}{2.055920in}}{\pgfqpoint{5.497999in}{2.059192in}}{\pgfqpoint{5.503823in}{2.065016in}}%
\pgfpathcurveto{\pgfqpoint{5.509647in}{2.070840in}}{\pgfqpoint{5.512919in}{2.078740in}}{\pgfqpoint{5.512919in}{2.086976in}}%
\pgfpathcurveto{\pgfqpoint{5.512919in}{2.095213in}}{\pgfqpoint{5.509647in}{2.103113in}}{\pgfqpoint{5.503823in}{2.108937in}}%
\pgfpathcurveto{\pgfqpoint{5.497999in}{2.114760in}}{\pgfqpoint{5.490099in}{2.118033in}}{\pgfqpoint{5.481863in}{2.118033in}}%
\pgfpathcurveto{\pgfqpoint{5.473627in}{2.118033in}}{\pgfqpoint{5.465727in}{2.114760in}}{\pgfqpoint{5.459903in}{2.108937in}}%
\pgfpathcurveto{\pgfqpoint{5.454079in}{2.103113in}}{\pgfqpoint{5.450806in}{2.095213in}}{\pgfqpoint{5.450806in}{2.086976in}}%
\pgfpathcurveto{\pgfqpoint{5.450806in}{2.078740in}}{\pgfqpoint{5.454079in}{2.070840in}}{\pgfqpoint{5.459903in}{2.065016in}}%
\pgfpathcurveto{\pgfqpoint{5.465727in}{2.059192in}}{\pgfqpoint{5.473627in}{2.055920in}}{\pgfqpoint{5.481863in}{2.055920in}}%
\pgfpathclose%
\pgfusepath{stroke,fill}%
\end{pgfscope}%
\begin{pgfscope}%
\pgfpathrectangle{\pgfqpoint{3.755891in}{0.557870in}}{\pgfqpoint{2.484109in}{1.684734in}}%
\pgfusepath{clip}%
\pgfsetbuttcap%
\pgfsetroundjoin%
\definecolor{currentfill}{rgb}{0.298039,0.447059,0.690196}%
\pgfsetfillcolor{currentfill}%
\pgfsetlinewidth{1.003750pt}%
\definecolor{currentstroke}{rgb}{0.298039,0.447059,0.690196}%
\pgfsetstrokecolor{currentstroke}%
\pgfsetdash{}{0pt}%
\pgfpathmoveto{\pgfqpoint{5.078599in}{2.075682in}}%
\pgfpathcurveto{\pgfqpoint{5.086835in}{2.075682in}}{\pgfqpoint{5.094735in}{2.078954in}}{\pgfqpoint{5.100559in}{2.084778in}}%
\pgfpathcurveto{\pgfqpoint{5.106383in}{2.090602in}}{\pgfqpoint{5.109655in}{2.098502in}}{\pgfqpoint{5.109655in}{2.106739in}}%
\pgfpathcurveto{\pgfqpoint{5.109655in}{2.114975in}}{\pgfqpoint{5.106383in}{2.122875in}}{\pgfqpoint{5.100559in}{2.128699in}}%
\pgfpathcurveto{\pgfqpoint{5.094735in}{2.134523in}}{\pgfqpoint{5.086835in}{2.137795in}}{\pgfqpoint{5.078599in}{2.137795in}}%
\pgfpathcurveto{\pgfqpoint{5.070362in}{2.137795in}}{\pgfqpoint{5.062462in}{2.134523in}}{\pgfqpoint{5.056638in}{2.128699in}}%
\pgfpathcurveto{\pgfqpoint{5.050814in}{2.122875in}}{\pgfqpoint{5.047542in}{2.114975in}}{\pgfqpoint{5.047542in}{2.106739in}}%
\pgfpathcurveto{\pgfqpoint{5.047542in}{2.098502in}}{\pgfqpoint{5.050814in}{2.090602in}}{\pgfqpoint{5.056638in}{2.084778in}}%
\pgfpathcurveto{\pgfqpoint{5.062462in}{2.078954in}}{\pgfqpoint{5.070362in}{2.075682in}}{\pgfqpoint{5.078599in}{2.075682in}}%
\pgfpathclose%
\pgfusepath{stroke,fill}%
\end{pgfscope}%
\begin{pgfscope}%
\pgfpathrectangle{\pgfqpoint{3.755891in}{0.557870in}}{\pgfqpoint{2.484109in}{1.684734in}}%
\pgfusepath{clip}%
\pgfsetbuttcap%
\pgfsetroundjoin%
\definecolor{currentfill}{rgb}{0.298039,0.447059,0.690196}%
\pgfsetfillcolor{currentfill}%
\pgfsetlinewidth{1.003750pt}%
\definecolor{currentstroke}{rgb}{0.298039,0.447059,0.690196}%
\pgfsetstrokecolor{currentstroke}%
\pgfsetdash{}{0pt}%
\pgfpathmoveto{\pgfqpoint{5.320557in}{2.065801in}}%
\pgfpathcurveto{\pgfqpoint{5.328793in}{2.065801in}}{\pgfqpoint{5.336694in}{2.069073in}}{\pgfqpoint{5.342517in}{2.074897in}}%
\pgfpathcurveto{\pgfqpoint{5.348341in}{2.080721in}}{\pgfqpoint{5.351614in}{2.088621in}}{\pgfqpoint{5.351614in}{2.096857in}}%
\pgfpathcurveto{\pgfqpoint{5.351614in}{2.105094in}}{\pgfqpoint{5.348341in}{2.112994in}}{\pgfqpoint{5.342517in}{2.118818in}}%
\pgfpathcurveto{\pgfqpoint{5.336694in}{2.124642in}}{\pgfqpoint{5.328793in}{2.127914in}}{\pgfqpoint{5.320557in}{2.127914in}}%
\pgfpathcurveto{\pgfqpoint{5.312321in}{2.127914in}}{\pgfqpoint{5.304421in}{2.124642in}}{\pgfqpoint{5.298597in}{2.118818in}}%
\pgfpathcurveto{\pgfqpoint{5.292773in}{2.112994in}}{\pgfqpoint{5.289501in}{2.105094in}}{\pgfqpoint{5.289501in}{2.096857in}}%
\pgfpathcurveto{\pgfqpoint{5.289501in}{2.088621in}}{\pgfqpoint{5.292773in}{2.080721in}}{\pgfqpoint{5.298597in}{2.074897in}}%
\pgfpathcurveto{\pgfqpoint{5.304421in}{2.069073in}}{\pgfqpoint{5.312321in}{2.065801in}}{\pgfqpoint{5.320557in}{2.065801in}}%
\pgfpathclose%
\pgfusepath{stroke,fill}%
\end{pgfscope}%
\begin{pgfscope}%
\pgfpathrectangle{\pgfqpoint{3.755891in}{0.557870in}}{\pgfqpoint{2.484109in}{1.684734in}}%
\pgfusepath{clip}%
\pgfsetbuttcap%
\pgfsetroundjoin%
\definecolor{currentfill}{rgb}{0.298039,0.447059,0.690196}%
\pgfsetfillcolor{currentfill}%
\pgfsetlinewidth{1.003750pt}%
\definecolor{currentstroke}{rgb}{0.298039,0.447059,0.690196}%
\pgfsetstrokecolor{currentstroke}%
\pgfsetdash{}{0pt}%
\pgfpathmoveto{\pgfqpoint{5.481863in}{2.085563in}}%
\pgfpathcurveto{\pgfqpoint{5.490099in}{2.085563in}}{\pgfqpoint{5.497999in}{2.088835in}}{\pgfqpoint{5.503823in}{2.094659in}}%
\pgfpathcurveto{\pgfqpoint{5.509647in}{2.100483in}}{\pgfqpoint{5.512919in}{2.108383in}}{\pgfqpoint{5.512919in}{2.116620in}}%
\pgfpathcurveto{\pgfqpoint{5.512919in}{2.124856in}}{\pgfqpoint{5.509647in}{2.132756in}}{\pgfqpoint{5.503823in}{2.138580in}}%
\pgfpathcurveto{\pgfqpoint{5.497999in}{2.144404in}}{\pgfqpoint{5.490099in}{2.147676in}}{\pgfqpoint{5.481863in}{2.147676in}}%
\pgfpathcurveto{\pgfqpoint{5.473627in}{2.147676in}}{\pgfqpoint{5.465727in}{2.144404in}}{\pgfqpoint{5.459903in}{2.138580in}}%
\pgfpathcurveto{\pgfqpoint{5.454079in}{2.132756in}}{\pgfqpoint{5.450806in}{2.124856in}}{\pgfqpoint{5.450806in}{2.116620in}}%
\pgfpathcurveto{\pgfqpoint{5.450806in}{2.108383in}}{\pgfqpoint{5.454079in}{2.100483in}}{\pgfqpoint{5.459903in}{2.094659in}}%
\pgfpathcurveto{\pgfqpoint{5.465727in}{2.088835in}}{\pgfqpoint{5.473627in}{2.085563in}}{\pgfqpoint{5.481863in}{2.085563in}}%
\pgfpathclose%
\pgfusepath{stroke,fill}%
\end{pgfscope}%
\begin{pgfscope}%
\pgfpathrectangle{\pgfqpoint{3.755891in}{0.557870in}}{\pgfqpoint{2.484109in}{1.684734in}}%
\pgfusepath{clip}%
\pgfsetbuttcap%
\pgfsetroundjoin%
\definecolor{currentfill}{rgb}{0.298039,0.447059,0.690196}%
\pgfsetfillcolor{currentfill}%
\pgfsetlinewidth{1.003750pt}%
\definecolor{currentstroke}{rgb}{0.298039,0.447059,0.690196}%
\pgfsetstrokecolor{currentstroke}%
\pgfsetdash{}{0pt}%
\pgfpathmoveto{\pgfqpoint{5.885127in}{1.759486in}}%
\pgfpathcurveto{\pgfqpoint{5.893364in}{1.759486in}}{\pgfqpoint{5.901264in}{1.762758in}}{\pgfqpoint{5.907088in}{1.768582in}}%
\pgfpathcurveto{\pgfqpoint{5.912912in}{1.774406in}}{\pgfqpoint{5.916184in}{1.782306in}}{\pgfqpoint{5.916184in}{1.790542in}}%
\pgfpathcurveto{\pgfqpoint{5.916184in}{1.798778in}}{\pgfqpoint{5.912912in}{1.806678in}}{\pgfqpoint{5.907088in}{1.812502in}}%
\pgfpathcurveto{\pgfqpoint{5.901264in}{1.818326in}}{\pgfqpoint{5.893364in}{1.821599in}}{\pgfqpoint{5.885127in}{1.821599in}}%
\pgfpathcurveto{\pgfqpoint{5.876891in}{1.821599in}}{\pgfqpoint{5.868991in}{1.818326in}}{\pgfqpoint{5.863167in}{1.812502in}}%
\pgfpathcurveto{\pgfqpoint{5.857343in}{1.806678in}}{\pgfqpoint{5.854071in}{1.798778in}}{\pgfqpoint{5.854071in}{1.790542in}}%
\pgfpathcurveto{\pgfqpoint{5.854071in}{1.782306in}}{\pgfqpoint{5.857343in}{1.774406in}}{\pgfqpoint{5.863167in}{1.768582in}}%
\pgfpathcurveto{\pgfqpoint{5.868991in}{1.762758in}}{\pgfqpoint{5.876891in}{1.759486in}}{\pgfqpoint{5.885127in}{1.759486in}}%
\pgfpathclose%
\pgfusepath{stroke,fill}%
\end{pgfscope}%
\begin{pgfscope}%
\pgfpathrectangle{\pgfqpoint{3.755891in}{0.557870in}}{\pgfqpoint{2.484109in}{1.684734in}}%
\pgfusepath{clip}%
\pgfsetbuttcap%
\pgfsetroundjoin%
\definecolor{currentfill}{rgb}{0.298039,0.447059,0.690196}%
\pgfsetfillcolor{currentfill}%
\pgfsetlinewidth{1.003750pt}%
\definecolor{currentstroke}{rgb}{0.298039,0.447059,0.690196}%
\pgfsetstrokecolor{currentstroke}%
\pgfsetdash{}{0pt}%
\pgfpathmoveto{\pgfqpoint{4.233044in}{2.087988in}}%
\pgfpathcurveto{\pgfqpoint{4.241280in}{2.087988in}}{\pgfqpoint{4.249180in}{2.091260in}}{\pgfqpoint{4.255004in}{2.097084in}}%
\pgfpathcurveto{\pgfqpoint{4.260828in}{2.102908in}}{\pgfqpoint{4.264101in}{2.110808in}}{\pgfqpoint{4.264101in}{2.119044in}}%
\pgfpathcurveto{\pgfqpoint{4.264101in}{2.127281in}}{\pgfqpoint{4.260828in}{2.135181in}}{\pgfqpoint{4.255004in}{2.141005in}}%
\pgfpathcurveto{\pgfqpoint{4.249180in}{2.146829in}}{\pgfqpoint{4.241280in}{2.150101in}}{\pgfqpoint{4.233044in}{2.150101in}}%
\pgfpathcurveto{\pgfqpoint{4.224808in}{2.150101in}}{\pgfqpoint{4.216908in}{2.146829in}}{\pgfqpoint{4.211084in}{2.141005in}}%
\pgfpathcurveto{\pgfqpoint{4.205260in}{2.135181in}}{\pgfqpoint{4.201988in}{2.127281in}}{\pgfqpoint{4.201988in}{2.119044in}}%
\pgfpathcurveto{\pgfqpoint{4.201988in}{2.110808in}}{\pgfqpoint{4.205260in}{2.102908in}}{\pgfqpoint{4.211084in}{2.097084in}}%
\pgfpathcurveto{\pgfqpoint{4.216908in}{2.091260in}}{\pgfqpoint{4.224808in}{2.087988in}}{\pgfqpoint{4.233044in}{2.087988in}}%
\pgfpathclose%
\pgfusepath{stroke,fill}%
\end{pgfscope}%
\begin{pgfscope}%
\pgfpathrectangle{\pgfqpoint{3.755891in}{0.557870in}}{\pgfqpoint{2.484109in}{1.684734in}}%
\pgfusepath{clip}%
\pgfsetbuttcap%
\pgfsetroundjoin%
\definecolor{currentfill}{rgb}{0.298039,0.447059,0.690196}%
\pgfsetfillcolor{currentfill}%
\pgfsetlinewidth{1.003750pt}%
\definecolor{currentstroke}{rgb}{0.298039,0.447059,0.690196}%
\pgfsetstrokecolor{currentstroke}%
\pgfsetdash{}{0pt}%
\pgfpathmoveto{\pgfqpoint{5.180065in}{1.994026in}}%
\pgfpathcurveto{\pgfqpoint{5.188301in}{1.994026in}}{\pgfqpoint{5.196201in}{1.997299in}}{\pgfqpoint{5.202025in}{2.003122in}}%
\pgfpathcurveto{\pgfqpoint{5.207849in}{2.008946in}}{\pgfqpoint{5.211122in}{2.016846in}}{\pgfqpoint{5.211122in}{2.025083in}}%
\pgfpathcurveto{\pgfqpoint{5.211122in}{2.033319in}}{\pgfqpoint{5.207849in}{2.041219in}}{\pgfqpoint{5.202025in}{2.047043in}}%
\pgfpathcurveto{\pgfqpoint{5.196201in}{2.052867in}}{\pgfqpoint{5.188301in}{2.056139in}}{\pgfqpoint{5.180065in}{2.056139in}}%
\pgfpathcurveto{\pgfqpoint{5.171829in}{2.056139in}}{\pgfqpoint{5.163929in}{2.052867in}}{\pgfqpoint{5.158105in}{2.047043in}}%
\pgfpathcurveto{\pgfqpoint{5.152281in}{2.041219in}}{\pgfqpoint{5.149009in}{2.033319in}}{\pgfqpoint{5.149009in}{2.025083in}}%
\pgfpathcurveto{\pgfqpoint{5.149009in}{2.016846in}}{\pgfqpoint{5.152281in}{2.008946in}}{\pgfqpoint{5.158105in}{2.003122in}}%
\pgfpathcurveto{\pgfqpoint{5.163929in}{1.997299in}}{\pgfqpoint{5.171829in}{1.994026in}}{\pgfqpoint{5.180065in}{1.994026in}}%
\pgfpathclose%
\pgfusepath{stroke,fill}%
\end{pgfscope}%
\begin{pgfscope}%
\pgfpathrectangle{\pgfqpoint{3.755891in}{0.557870in}}{\pgfqpoint{2.484109in}{1.684734in}}%
\pgfusepath{clip}%
\pgfsetbuttcap%
\pgfsetroundjoin%
\definecolor{currentfill}{rgb}{0.298039,0.447059,0.690196}%
\pgfsetfillcolor{currentfill}%
\pgfsetlinewidth{1.003750pt}%
\definecolor{currentstroke}{rgb}{0.298039,0.447059,0.690196}%
\pgfsetstrokecolor{currentstroke}%
\pgfsetdash{}{0pt}%
\pgfpathmoveto{\pgfqpoint{4.014501in}{2.116177in}}%
\pgfpathcurveto{\pgfqpoint{4.022737in}{2.116177in}}{\pgfqpoint{4.030637in}{2.119449in}}{\pgfqpoint{4.036461in}{2.125273in}}%
\pgfpathcurveto{\pgfqpoint{4.042285in}{2.131097in}}{\pgfqpoint{4.045557in}{2.138997in}}{\pgfqpoint{4.045557in}{2.147233in}}%
\pgfpathcurveto{\pgfqpoint{4.045557in}{2.155469in}}{\pgfqpoint{4.042285in}{2.163369in}}{\pgfqpoint{4.036461in}{2.169193in}}%
\pgfpathcurveto{\pgfqpoint{4.030637in}{2.175017in}}{\pgfqpoint{4.022737in}{2.178290in}}{\pgfqpoint{4.014501in}{2.178290in}}%
\pgfpathcurveto{\pgfqpoint{4.006265in}{2.178290in}}{\pgfqpoint{3.998365in}{2.175017in}}{\pgfqpoint{3.992541in}{2.169193in}}%
\pgfpathcurveto{\pgfqpoint{3.986717in}{2.163369in}}{\pgfqpoint{3.983444in}{2.155469in}}{\pgfqpoint{3.983444in}{2.147233in}}%
\pgfpathcurveto{\pgfqpoint{3.983444in}{2.138997in}}{\pgfqpoint{3.986717in}{2.131097in}}{\pgfqpoint{3.992541in}{2.125273in}}%
\pgfpathcurveto{\pgfqpoint{3.998365in}{2.119449in}}{\pgfqpoint{4.006265in}{2.116177in}}{\pgfqpoint{4.014501in}{2.116177in}}%
\pgfpathclose%
\pgfusepath{stroke,fill}%
\end{pgfscope}%
\begin{pgfscope}%
\pgfpathrectangle{\pgfqpoint{3.755891in}{0.557870in}}{\pgfqpoint{2.484109in}{1.684734in}}%
\pgfusepath{clip}%
\pgfsetbuttcap%
\pgfsetroundjoin%
\definecolor{currentfill}{rgb}{0.298039,0.447059,0.690196}%
\pgfsetfillcolor{currentfill}%
\pgfsetlinewidth{1.003750pt}%
\definecolor{currentstroke}{rgb}{0.298039,0.447059,0.690196}%
\pgfsetstrokecolor{currentstroke}%
\pgfsetdash{}{0pt}%
\pgfpathmoveto{\pgfqpoint{4.014501in}{2.125573in}}%
\pgfpathcurveto{\pgfqpoint{4.022737in}{2.125573in}}{\pgfqpoint{4.030637in}{2.128845in}}{\pgfqpoint{4.036461in}{2.134669in}}%
\pgfpathcurveto{\pgfqpoint{4.042285in}{2.140493in}}{\pgfqpoint{4.045557in}{2.148393in}}{\pgfqpoint{4.045557in}{2.156629in}}%
\pgfpathcurveto{\pgfqpoint{4.045557in}{2.164865in}}{\pgfqpoint{4.042285in}{2.172766in}}{\pgfqpoint{4.036461in}{2.178589in}}%
\pgfpathcurveto{\pgfqpoint{4.030637in}{2.184413in}}{\pgfqpoint{4.022737in}{2.187686in}}{\pgfqpoint{4.014501in}{2.187686in}}%
\pgfpathcurveto{\pgfqpoint{4.006265in}{2.187686in}}{\pgfqpoint{3.998365in}{2.184413in}}{\pgfqpoint{3.992541in}{2.178589in}}%
\pgfpathcurveto{\pgfqpoint{3.986717in}{2.172766in}}{\pgfqpoint{3.983444in}{2.164865in}}{\pgfqpoint{3.983444in}{2.156629in}}%
\pgfpathcurveto{\pgfqpoint{3.983444in}{2.148393in}}{\pgfqpoint{3.986717in}{2.140493in}}{\pgfqpoint{3.992541in}{2.134669in}}%
\pgfpathcurveto{\pgfqpoint{3.998365in}{2.128845in}}{\pgfqpoint{4.006265in}{2.125573in}}{\pgfqpoint{4.014501in}{2.125573in}}%
\pgfpathclose%
\pgfusepath{stroke,fill}%
\end{pgfscope}%
\begin{pgfscope}%
\pgfpathrectangle{\pgfqpoint{3.755891in}{0.557870in}}{\pgfqpoint{2.484109in}{1.684734in}}%
\pgfusepath{clip}%
\pgfsetbuttcap%
\pgfsetroundjoin%
\definecolor{currentfill}{rgb}{0.298039,0.447059,0.690196}%
\pgfsetfillcolor{currentfill}%
\pgfsetlinewidth{1.003750pt}%
\definecolor{currentstroke}{rgb}{0.298039,0.447059,0.690196}%
\pgfsetstrokecolor{currentstroke}%
\pgfsetdash{}{0pt}%
\pgfpathmoveto{\pgfqpoint{5.107217in}{1.984630in}}%
\pgfpathcurveto{\pgfqpoint{5.115454in}{1.984630in}}{\pgfqpoint{5.123354in}{1.987902in}}{\pgfqpoint{5.129178in}{1.993726in}}%
\pgfpathcurveto{\pgfqpoint{5.135001in}{1.999550in}}{\pgfqpoint{5.138274in}{2.007450in}}{\pgfqpoint{5.138274in}{2.015687in}}%
\pgfpathcurveto{\pgfqpoint{5.138274in}{2.023923in}}{\pgfqpoint{5.135001in}{2.031823in}}{\pgfqpoint{5.129178in}{2.037647in}}%
\pgfpathcurveto{\pgfqpoint{5.123354in}{2.043471in}}{\pgfqpoint{5.115454in}{2.046743in}}{\pgfqpoint{5.107217in}{2.046743in}}%
\pgfpathcurveto{\pgfqpoint{5.098981in}{2.046743in}}{\pgfqpoint{5.091081in}{2.043471in}}{\pgfqpoint{5.085257in}{2.037647in}}%
\pgfpathcurveto{\pgfqpoint{5.079433in}{2.031823in}}{\pgfqpoint{5.076161in}{2.023923in}}{\pgfqpoint{5.076161in}{2.015687in}}%
\pgfpathcurveto{\pgfqpoint{5.076161in}{2.007450in}}{\pgfqpoint{5.079433in}{1.999550in}}{\pgfqpoint{5.085257in}{1.993726in}}%
\pgfpathcurveto{\pgfqpoint{5.091081in}{1.987902in}}{\pgfqpoint{5.098981in}{1.984630in}}{\pgfqpoint{5.107217in}{1.984630in}}%
\pgfpathclose%
\pgfusepath{stroke,fill}%
\end{pgfscope}%
\begin{pgfscope}%
\pgfpathrectangle{\pgfqpoint{3.755891in}{0.557870in}}{\pgfqpoint{2.484109in}{1.684734in}}%
\pgfusepath{clip}%
\pgfsetbuttcap%
\pgfsetroundjoin%
\definecolor{currentfill}{rgb}{0.298039,0.447059,0.690196}%
\pgfsetfillcolor{currentfill}%
\pgfsetlinewidth{1.003750pt}%
\definecolor{currentstroke}{rgb}{0.298039,0.447059,0.690196}%
\pgfsetstrokecolor{currentstroke}%
\pgfsetdash{}{0pt}%
\pgfpathmoveto{\pgfqpoint{4.888674in}{1.984630in}}%
\pgfpathcurveto{\pgfqpoint{4.896910in}{1.984630in}}{\pgfqpoint{4.904810in}{1.987902in}}{\pgfqpoint{4.910634in}{1.993726in}}%
\pgfpathcurveto{\pgfqpoint{4.916458in}{1.999550in}}{\pgfqpoint{4.919731in}{2.007450in}}{\pgfqpoint{4.919731in}{2.015687in}}%
\pgfpathcurveto{\pgfqpoint{4.919731in}{2.023923in}}{\pgfqpoint{4.916458in}{2.031823in}}{\pgfqpoint{4.910634in}{2.037647in}}%
\pgfpathcurveto{\pgfqpoint{4.904810in}{2.043471in}}{\pgfqpoint{4.896910in}{2.046743in}}{\pgfqpoint{4.888674in}{2.046743in}}%
\pgfpathcurveto{\pgfqpoint{4.880438in}{2.046743in}}{\pgfqpoint{4.872538in}{2.043471in}}{\pgfqpoint{4.866714in}{2.037647in}}%
\pgfpathcurveto{\pgfqpoint{4.860890in}{2.031823in}}{\pgfqpoint{4.857618in}{2.023923in}}{\pgfqpoint{4.857618in}{2.015687in}}%
\pgfpathcurveto{\pgfqpoint{4.857618in}{2.007450in}}{\pgfqpoint{4.860890in}{1.999550in}}{\pgfqpoint{4.866714in}{1.993726in}}%
\pgfpathcurveto{\pgfqpoint{4.872538in}{1.987902in}}{\pgfqpoint{4.880438in}{1.984630in}}{\pgfqpoint{4.888674in}{1.984630in}}%
\pgfpathclose%
\pgfusepath{stroke,fill}%
\end{pgfscope}%
\begin{pgfscope}%
\pgfpathrectangle{\pgfqpoint{3.755891in}{0.557870in}}{\pgfqpoint{2.484109in}{1.684734in}}%
\pgfusepath{clip}%
\pgfsetbuttcap%
\pgfsetroundjoin%
\definecolor{currentfill}{rgb}{0.298039,0.447059,0.690196}%
\pgfsetfillcolor{currentfill}%
\pgfsetlinewidth{1.003750pt}%
\definecolor{currentstroke}{rgb}{0.298039,0.447059,0.690196}%
\pgfsetstrokecolor{currentstroke}%
\pgfsetdash{}{0pt}%
\pgfpathmoveto{\pgfqpoint{5.034370in}{2.031611in}}%
\pgfpathcurveto{\pgfqpoint{5.042606in}{2.031611in}}{\pgfqpoint{5.050506in}{2.034883in}}{\pgfqpoint{5.056330in}{2.040707in}}%
\pgfpathcurveto{\pgfqpoint{5.062154in}{2.046531in}}{\pgfqpoint{5.065426in}{2.054431in}}{\pgfqpoint{5.065426in}{2.062667in}}%
\pgfpathcurveto{\pgfqpoint{5.065426in}{2.070904in}}{\pgfqpoint{5.062154in}{2.078804in}}{\pgfqpoint{5.056330in}{2.084628in}}%
\pgfpathcurveto{\pgfqpoint{5.050506in}{2.090452in}}{\pgfqpoint{5.042606in}{2.093724in}}{\pgfqpoint{5.034370in}{2.093724in}}%
\pgfpathcurveto{\pgfqpoint{5.026133in}{2.093724in}}{\pgfqpoint{5.018233in}{2.090452in}}{\pgfqpoint{5.012409in}{2.084628in}}%
\pgfpathcurveto{\pgfqpoint{5.006585in}{2.078804in}}{\pgfqpoint{5.003313in}{2.070904in}}{\pgfqpoint{5.003313in}{2.062667in}}%
\pgfpathcurveto{\pgfqpoint{5.003313in}{2.054431in}}{\pgfqpoint{5.006585in}{2.046531in}}{\pgfqpoint{5.012409in}{2.040707in}}%
\pgfpathcurveto{\pgfqpoint{5.018233in}{2.034883in}}{\pgfqpoint{5.026133in}{2.031611in}}{\pgfqpoint{5.034370in}{2.031611in}}%
\pgfpathclose%
\pgfusepath{stroke,fill}%
\end{pgfscope}%
\begin{pgfscope}%
\pgfpathrectangle{\pgfqpoint{3.755891in}{0.557870in}}{\pgfqpoint{2.484109in}{1.684734in}}%
\pgfusepath{clip}%
\pgfsetbuttcap%
\pgfsetroundjoin%
\definecolor{currentfill}{rgb}{0.298039,0.447059,0.690196}%
\pgfsetfillcolor{currentfill}%
\pgfsetlinewidth{1.003750pt}%
\definecolor{currentstroke}{rgb}{0.298039,0.447059,0.690196}%
\pgfsetstrokecolor{currentstroke}%
\pgfsetdash{}{0pt}%
\pgfpathmoveto{\pgfqpoint{5.180065in}{2.050403in}}%
\pgfpathcurveto{\pgfqpoint{5.188301in}{2.050403in}}{\pgfqpoint{5.196201in}{2.053676in}}{\pgfqpoint{5.202025in}{2.059500in}}%
\pgfpathcurveto{\pgfqpoint{5.207849in}{2.065323in}}{\pgfqpoint{5.211122in}{2.073224in}}{\pgfqpoint{5.211122in}{2.081460in}}%
\pgfpathcurveto{\pgfqpoint{5.211122in}{2.089696in}}{\pgfqpoint{5.207849in}{2.097596in}}{\pgfqpoint{5.202025in}{2.103420in}}%
\pgfpathcurveto{\pgfqpoint{5.196201in}{2.109244in}}{\pgfqpoint{5.188301in}{2.112516in}}{\pgfqpoint{5.180065in}{2.112516in}}%
\pgfpathcurveto{\pgfqpoint{5.171829in}{2.112516in}}{\pgfqpoint{5.163929in}{2.109244in}}{\pgfqpoint{5.158105in}{2.103420in}}%
\pgfpathcurveto{\pgfqpoint{5.152281in}{2.097596in}}{\pgfqpoint{5.149009in}{2.089696in}}{\pgfqpoint{5.149009in}{2.081460in}}%
\pgfpathcurveto{\pgfqpoint{5.149009in}{2.073224in}}{\pgfqpoint{5.152281in}{2.065323in}}{\pgfqpoint{5.158105in}{2.059500in}}%
\pgfpathcurveto{\pgfqpoint{5.163929in}{2.053676in}}{\pgfqpoint{5.171829in}{2.050403in}}{\pgfqpoint{5.180065in}{2.050403in}}%
\pgfpathclose%
\pgfusepath{stroke,fill}%
\end{pgfscope}%
\begin{pgfscope}%
\pgfpathrectangle{\pgfqpoint{3.755891in}{0.557870in}}{\pgfqpoint{2.484109in}{1.684734in}}%
\pgfusepath{clip}%
\pgfsetbuttcap%
\pgfsetroundjoin%
\definecolor{currentfill}{rgb}{0.298039,0.447059,0.690196}%
\pgfsetfillcolor{currentfill}%
\pgfsetlinewidth{1.003750pt}%
\definecolor{currentstroke}{rgb}{0.298039,0.447059,0.690196}%
\pgfsetstrokecolor{currentstroke}%
\pgfsetdash{}{0pt}%
\pgfpathmoveto{\pgfqpoint{5.762847in}{1.843687in}}%
\pgfpathcurveto{\pgfqpoint{5.771083in}{1.843687in}}{\pgfqpoint{5.778983in}{1.846960in}}{\pgfqpoint{5.784807in}{1.852784in}}%
\pgfpathcurveto{\pgfqpoint{5.790631in}{1.858608in}}{\pgfqpoint{5.793904in}{1.866508in}}{\pgfqpoint{5.793904in}{1.874744in}}%
\pgfpathcurveto{\pgfqpoint{5.793904in}{1.882980in}}{\pgfqpoint{5.790631in}{1.890880in}}{\pgfqpoint{5.784807in}{1.896704in}}%
\pgfpathcurveto{\pgfqpoint{5.778983in}{1.902528in}}{\pgfqpoint{5.771083in}{1.905800in}}{\pgfqpoint{5.762847in}{1.905800in}}%
\pgfpathcurveto{\pgfqpoint{5.754611in}{1.905800in}}{\pgfqpoint{5.746711in}{1.902528in}}{\pgfqpoint{5.740887in}{1.896704in}}%
\pgfpathcurveto{\pgfqpoint{5.735063in}{1.890880in}}{\pgfqpoint{5.731791in}{1.882980in}}{\pgfqpoint{5.731791in}{1.874744in}}%
\pgfpathcurveto{\pgfqpoint{5.731791in}{1.866508in}}{\pgfqpoint{5.735063in}{1.858608in}}{\pgfqpoint{5.740887in}{1.852784in}}%
\pgfpathcurveto{\pgfqpoint{5.746711in}{1.846960in}}{\pgfqpoint{5.754611in}{1.843687in}}{\pgfqpoint{5.762847in}{1.843687in}}%
\pgfpathclose%
\pgfusepath{stroke,fill}%
\end{pgfscope}%
\begin{pgfscope}%
\pgfpathrectangle{\pgfqpoint{3.755891in}{0.557870in}}{\pgfqpoint{2.484109in}{1.684734in}}%
\pgfusepath{clip}%
\pgfsetbuttcap%
\pgfsetroundjoin%
\definecolor{currentfill}{rgb}{0.298039,0.447059,0.690196}%
\pgfsetfillcolor{currentfill}%
\pgfsetlinewidth{1.003750pt}%
\definecolor{currentstroke}{rgb}{0.298039,0.447059,0.690196}%
\pgfsetstrokecolor{currentstroke}%
\pgfsetdash{}{0pt}%
\pgfpathmoveto{\pgfqpoint{4.597283in}{2.041007in}}%
\pgfpathcurveto{\pgfqpoint{4.605519in}{2.041007in}}{\pgfqpoint{4.613419in}{2.044279in}}{\pgfqpoint{4.619243in}{2.050103in}}%
\pgfpathcurveto{\pgfqpoint{4.625067in}{2.055927in}}{\pgfqpoint{4.628339in}{2.063827in}}{\pgfqpoint{4.628339in}{2.072064in}}%
\pgfpathcurveto{\pgfqpoint{4.628339in}{2.080300in}}{\pgfqpoint{4.625067in}{2.088200in}}{\pgfqpoint{4.619243in}{2.094024in}}%
\pgfpathcurveto{\pgfqpoint{4.613419in}{2.099848in}}{\pgfqpoint{4.605519in}{2.103120in}}{\pgfqpoint{4.597283in}{2.103120in}}%
\pgfpathcurveto{\pgfqpoint{4.589047in}{2.103120in}}{\pgfqpoint{4.581147in}{2.099848in}}{\pgfqpoint{4.575323in}{2.094024in}}%
\pgfpathcurveto{\pgfqpoint{4.569499in}{2.088200in}}{\pgfqpoint{4.566226in}{2.080300in}}{\pgfqpoint{4.566226in}{2.072064in}}%
\pgfpathcurveto{\pgfqpoint{4.566226in}{2.063827in}}{\pgfqpoint{4.569499in}{2.055927in}}{\pgfqpoint{4.575323in}{2.050103in}}%
\pgfpathcurveto{\pgfqpoint{4.581147in}{2.044279in}}{\pgfqpoint{4.589047in}{2.041007in}}{\pgfqpoint{4.597283in}{2.041007in}}%
\pgfpathclose%
\pgfusepath{stroke,fill}%
\end{pgfscope}%
\begin{pgfscope}%
\pgfpathrectangle{\pgfqpoint{3.755891in}{0.557870in}}{\pgfqpoint{2.484109in}{1.684734in}}%
\pgfusepath{clip}%
\pgfsetbuttcap%
\pgfsetroundjoin%
\definecolor{currentfill}{rgb}{0.298039,0.447059,0.690196}%
\pgfsetfillcolor{currentfill}%
\pgfsetlinewidth{1.003750pt}%
\definecolor{currentstroke}{rgb}{0.298039,0.447059,0.690196}%
\pgfsetstrokecolor{currentstroke}%
\pgfsetdash{}{0pt}%
\pgfpathmoveto{\pgfqpoint{4.755987in}{2.057908in}}%
\pgfpathcurveto{\pgfqpoint{4.764223in}{2.057908in}}{\pgfqpoint{4.772123in}{2.061181in}}{\pgfqpoint{4.777947in}{2.067005in}}%
\pgfpathcurveto{\pgfqpoint{4.783771in}{2.072829in}}{\pgfqpoint{4.787044in}{2.080729in}}{\pgfqpoint{4.787044in}{2.088965in}}%
\pgfpathcurveto{\pgfqpoint{4.787044in}{2.097201in}}{\pgfqpoint{4.783771in}{2.105101in}}{\pgfqpoint{4.777947in}{2.110925in}}%
\pgfpathcurveto{\pgfqpoint{4.772123in}{2.116749in}}{\pgfqpoint{4.764223in}{2.120021in}}{\pgfqpoint{4.755987in}{2.120021in}}%
\pgfpathcurveto{\pgfqpoint{4.747751in}{2.120021in}}{\pgfqpoint{4.739851in}{2.116749in}}{\pgfqpoint{4.734027in}{2.110925in}}%
\pgfpathcurveto{\pgfqpoint{4.728203in}{2.105101in}}{\pgfqpoint{4.724931in}{2.097201in}}{\pgfqpoint{4.724931in}{2.088965in}}%
\pgfpathcurveto{\pgfqpoint{4.724931in}{2.080729in}}{\pgfqpoint{4.728203in}{2.072829in}}{\pgfqpoint{4.734027in}{2.067005in}}%
\pgfpathcurveto{\pgfqpoint{4.739851in}{2.061181in}}{\pgfqpoint{4.747751in}{2.057908in}}{\pgfqpoint{4.755987in}{2.057908in}}%
\pgfpathclose%
\pgfusepath{stroke,fill}%
\end{pgfscope}%
\begin{pgfscope}%
\pgfpathrectangle{\pgfqpoint{3.755891in}{0.557870in}}{\pgfqpoint{2.484109in}{1.684734in}}%
\pgfusepath{clip}%
\pgfsetbuttcap%
\pgfsetroundjoin%
\definecolor{currentfill}{rgb}{0.298039,0.447059,0.690196}%
\pgfsetfillcolor{currentfill}%
\pgfsetlinewidth{1.003750pt}%
\definecolor{currentstroke}{rgb}{0.298039,0.447059,0.690196}%
\pgfsetstrokecolor{currentstroke}%
\pgfsetdash{}{0pt}%
\pgfpathmoveto{\pgfqpoint{5.643169in}{2.038643in}}%
\pgfpathcurveto{\pgfqpoint{5.651405in}{2.038643in}}{\pgfqpoint{5.659305in}{2.041916in}}{\pgfqpoint{5.665129in}{2.047740in}}%
\pgfpathcurveto{\pgfqpoint{5.670953in}{2.053563in}}{\pgfqpoint{5.674225in}{2.061464in}}{\pgfqpoint{5.674225in}{2.069700in}}%
\pgfpathcurveto{\pgfqpoint{5.674225in}{2.077936in}}{\pgfqpoint{5.670953in}{2.085836in}}{\pgfqpoint{5.665129in}{2.091660in}}%
\pgfpathcurveto{\pgfqpoint{5.659305in}{2.097484in}}{\pgfqpoint{5.651405in}{2.100756in}}{\pgfqpoint{5.643169in}{2.100756in}}%
\pgfpathcurveto{\pgfqpoint{5.634932in}{2.100756in}}{\pgfqpoint{5.627032in}{2.097484in}}{\pgfqpoint{5.621208in}{2.091660in}}%
\pgfpathcurveto{\pgfqpoint{5.615385in}{2.085836in}}{\pgfqpoint{5.612112in}{2.077936in}}{\pgfqpoint{5.612112in}{2.069700in}}%
\pgfpathcurveto{\pgfqpoint{5.612112in}{2.061464in}}{\pgfqpoint{5.615385in}{2.053563in}}{\pgfqpoint{5.621208in}{2.047740in}}%
\pgfpathcurveto{\pgfqpoint{5.627032in}{2.041916in}}{\pgfqpoint{5.634932in}{2.038643in}}{\pgfqpoint{5.643169in}{2.038643in}}%
\pgfpathclose%
\pgfusepath{stroke,fill}%
\end{pgfscope}%
\begin{pgfscope}%
\pgfpathrectangle{\pgfqpoint{3.755891in}{0.557870in}}{\pgfqpoint{2.484109in}{1.684734in}}%
\pgfusepath{clip}%
\pgfsetbuttcap%
\pgfsetroundjoin%
\definecolor{currentfill}{rgb}{0.298039,0.447059,0.690196}%
\pgfsetfillcolor{currentfill}%
\pgfsetlinewidth{1.003750pt}%
\definecolor{currentstroke}{rgb}{0.298039,0.447059,0.690196}%
\pgfsetstrokecolor{currentstroke}%
\pgfsetdash{}{0pt}%
\pgfpathmoveto{\pgfqpoint{5.320557in}{2.086806in}}%
\pgfpathcurveto{\pgfqpoint{5.328793in}{2.086806in}}{\pgfqpoint{5.336694in}{2.090078in}}{\pgfqpoint{5.342517in}{2.095902in}}%
\pgfpathcurveto{\pgfqpoint{5.348341in}{2.101726in}}{\pgfqpoint{5.351614in}{2.109626in}}{\pgfqpoint{5.351614in}{2.117863in}}%
\pgfpathcurveto{\pgfqpoint{5.351614in}{2.126099in}}{\pgfqpoint{5.348341in}{2.133999in}}{\pgfqpoint{5.342517in}{2.139823in}}%
\pgfpathcurveto{\pgfqpoint{5.336694in}{2.145647in}}{\pgfqpoint{5.328793in}{2.148919in}}{\pgfqpoint{5.320557in}{2.148919in}}%
\pgfpathcurveto{\pgfqpoint{5.312321in}{2.148919in}}{\pgfqpoint{5.304421in}{2.145647in}}{\pgfqpoint{5.298597in}{2.139823in}}%
\pgfpathcurveto{\pgfqpoint{5.292773in}{2.133999in}}{\pgfqpoint{5.289501in}{2.126099in}}{\pgfqpoint{5.289501in}{2.117863in}}%
\pgfpathcurveto{\pgfqpoint{5.289501in}{2.109626in}}{\pgfqpoint{5.292773in}{2.101726in}}{\pgfqpoint{5.298597in}{2.095902in}}%
\pgfpathcurveto{\pgfqpoint{5.304421in}{2.090078in}}{\pgfqpoint{5.312321in}{2.086806in}}{\pgfqpoint{5.320557in}{2.086806in}}%
\pgfpathclose%
\pgfusepath{stroke,fill}%
\end{pgfscope}%
\begin{pgfscope}%
\pgfpathrectangle{\pgfqpoint{3.755891in}{0.557870in}}{\pgfqpoint{2.484109in}{1.684734in}}%
\pgfusepath{clip}%
\pgfsetbuttcap%
\pgfsetroundjoin%
\definecolor{currentfill}{rgb}{0.298039,0.447059,0.690196}%
\pgfsetfillcolor{currentfill}%
\pgfsetlinewidth{1.003750pt}%
\definecolor{currentstroke}{rgb}{0.298039,0.447059,0.690196}%
\pgfsetstrokecolor{currentstroke}%
\pgfsetdash{}{0pt}%
\pgfpathmoveto{\pgfqpoint{5.643169in}{2.048276in}}%
\pgfpathcurveto{\pgfqpoint{5.651405in}{2.048276in}}{\pgfqpoint{5.659305in}{2.051548in}}{\pgfqpoint{5.665129in}{2.057372in}}%
\pgfpathcurveto{\pgfqpoint{5.670953in}{2.063196in}}{\pgfqpoint{5.674225in}{2.071096in}}{\pgfqpoint{5.674225in}{2.079332in}}%
\pgfpathcurveto{\pgfqpoint{5.674225in}{2.087569in}}{\pgfqpoint{5.670953in}{2.095469in}}{\pgfqpoint{5.665129in}{2.101293in}}%
\pgfpathcurveto{\pgfqpoint{5.659305in}{2.107117in}}{\pgfqpoint{5.651405in}{2.110389in}}{\pgfqpoint{5.643169in}{2.110389in}}%
\pgfpathcurveto{\pgfqpoint{5.634932in}{2.110389in}}{\pgfqpoint{5.627032in}{2.107117in}}{\pgfqpoint{5.621208in}{2.101293in}}%
\pgfpathcurveto{\pgfqpoint{5.615385in}{2.095469in}}{\pgfqpoint{5.612112in}{2.087569in}}{\pgfqpoint{5.612112in}{2.079332in}}%
\pgfpathcurveto{\pgfqpoint{5.612112in}{2.071096in}}{\pgfqpoint{5.615385in}{2.063196in}}{\pgfqpoint{5.621208in}{2.057372in}}%
\pgfpathcurveto{\pgfqpoint{5.627032in}{2.051548in}}{\pgfqpoint{5.634932in}{2.048276in}}{\pgfqpoint{5.643169in}{2.048276in}}%
\pgfpathclose%
\pgfusepath{stroke,fill}%
\end{pgfscope}%
\begin{pgfscope}%
\pgfpathrectangle{\pgfqpoint{3.755891in}{0.557870in}}{\pgfqpoint{2.484109in}{1.684734in}}%
\pgfusepath{clip}%
\pgfsetbuttcap%
\pgfsetroundjoin%
\definecolor{currentfill}{rgb}{0.298039,0.447059,0.690196}%
\pgfsetfillcolor{currentfill}%
\pgfsetlinewidth{1.003750pt}%
\definecolor{currentstroke}{rgb}{0.298039,0.447059,0.690196}%
\pgfsetstrokecolor{currentstroke}%
\pgfsetdash{}{0pt}%
\pgfpathmoveto{\pgfqpoint{5.078599in}{2.057908in}}%
\pgfpathcurveto{\pgfqpoint{5.086835in}{2.057908in}}{\pgfqpoint{5.094735in}{2.061181in}}{\pgfqpoint{5.100559in}{2.067005in}}%
\pgfpathcurveto{\pgfqpoint{5.106383in}{2.072829in}}{\pgfqpoint{5.109655in}{2.080729in}}{\pgfqpoint{5.109655in}{2.088965in}}%
\pgfpathcurveto{\pgfqpoint{5.109655in}{2.097201in}}{\pgfqpoint{5.106383in}{2.105101in}}{\pgfqpoint{5.100559in}{2.110925in}}%
\pgfpathcurveto{\pgfqpoint{5.094735in}{2.116749in}}{\pgfqpoint{5.086835in}{2.120021in}}{\pgfqpoint{5.078599in}{2.120021in}}%
\pgfpathcurveto{\pgfqpoint{5.070362in}{2.120021in}}{\pgfqpoint{5.062462in}{2.116749in}}{\pgfqpoint{5.056638in}{2.110925in}}%
\pgfpathcurveto{\pgfqpoint{5.050814in}{2.105101in}}{\pgfqpoint{5.047542in}{2.097201in}}{\pgfqpoint{5.047542in}{2.088965in}}%
\pgfpathcurveto{\pgfqpoint{5.047542in}{2.080729in}}{\pgfqpoint{5.050814in}{2.072829in}}{\pgfqpoint{5.056638in}{2.067005in}}%
\pgfpathcurveto{\pgfqpoint{5.062462in}{2.061181in}}{\pgfqpoint{5.070362in}{2.057908in}}{\pgfqpoint{5.078599in}{2.057908in}}%
\pgfpathclose%
\pgfusepath{stroke,fill}%
\end{pgfscope}%
\begin{pgfscope}%
\pgfpathrectangle{\pgfqpoint{3.755891in}{0.557870in}}{\pgfqpoint{2.484109in}{1.684734in}}%
\pgfusepath{clip}%
\pgfsetbuttcap%
\pgfsetroundjoin%
\definecolor{currentfill}{rgb}{0.298039,0.447059,0.690196}%
\pgfsetfillcolor{currentfill}%
\pgfsetlinewidth{1.003750pt}%
\definecolor{currentstroke}{rgb}{0.298039,0.447059,0.690196}%
\pgfsetstrokecolor{currentstroke}%
\pgfsetdash{}{0pt}%
\pgfpathmoveto{\pgfqpoint{5.401210in}{2.077174in}}%
\pgfpathcurveto{\pgfqpoint{5.409446in}{2.077174in}}{\pgfqpoint{5.417346in}{2.080446in}}{\pgfqpoint{5.423170in}{2.086270in}}%
\pgfpathcurveto{\pgfqpoint{5.428994in}{2.092094in}}{\pgfqpoint{5.432267in}{2.099994in}}{\pgfqpoint{5.432267in}{2.108230in}}%
\pgfpathcurveto{\pgfqpoint{5.432267in}{2.116466in}}{\pgfqpoint{5.428994in}{2.124366in}}{\pgfqpoint{5.423170in}{2.130190in}}%
\pgfpathcurveto{\pgfqpoint{5.417346in}{2.136014in}}{\pgfqpoint{5.409446in}{2.139287in}}{\pgfqpoint{5.401210in}{2.139287in}}%
\pgfpathcurveto{\pgfqpoint{5.392974in}{2.139287in}}{\pgfqpoint{5.385074in}{2.136014in}}{\pgfqpoint{5.379250in}{2.130190in}}%
\pgfpathcurveto{\pgfqpoint{5.373426in}{2.124366in}}{\pgfqpoint{5.370154in}{2.116466in}}{\pgfqpoint{5.370154in}{2.108230in}}%
\pgfpathcurveto{\pgfqpoint{5.370154in}{2.099994in}}{\pgfqpoint{5.373426in}{2.092094in}}{\pgfqpoint{5.379250in}{2.086270in}}%
\pgfpathcurveto{\pgfqpoint{5.385074in}{2.080446in}}{\pgfqpoint{5.392974in}{2.077174in}}{\pgfqpoint{5.401210in}{2.077174in}}%
\pgfpathclose%
\pgfusepath{stroke,fill}%
\end{pgfscope}%
\begin{pgfscope}%
\pgfpathrectangle{\pgfqpoint{3.755891in}{0.557870in}}{\pgfqpoint{2.484109in}{1.684734in}}%
\pgfusepath{clip}%
\pgfsetbuttcap%
\pgfsetroundjoin%
\definecolor{currentfill}{rgb}{0.298039,0.447059,0.690196}%
\pgfsetfillcolor{currentfill}%
\pgfsetlinewidth{1.003750pt}%
\definecolor{currentstroke}{rgb}{0.298039,0.447059,0.690196}%
\pgfsetstrokecolor{currentstroke}%
\pgfsetdash{}{0pt}%
\pgfpathmoveto{\pgfqpoint{5.320557in}{2.096439in}}%
\pgfpathcurveto{\pgfqpoint{5.328793in}{2.096439in}}{\pgfqpoint{5.336694in}{2.099711in}}{\pgfqpoint{5.342517in}{2.105535in}}%
\pgfpathcurveto{\pgfqpoint{5.348341in}{2.111359in}}{\pgfqpoint{5.351614in}{2.119259in}}{\pgfqpoint{5.351614in}{2.127495in}}%
\pgfpathcurveto{\pgfqpoint{5.351614in}{2.135731in}}{\pgfqpoint{5.348341in}{2.143631in}}{\pgfqpoint{5.342517in}{2.149455in}}%
\pgfpathcurveto{\pgfqpoint{5.336694in}{2.155279in}}{\pgfqpoint{5.328793in}{2.158552in}}{\pgfqpoint{5.320557in}{2.158552in}}%
\pgfpathcurveto{\pgfqpoint{5.312321in}{2.158552in}}{\pgfqpoint{5.304421in}{2.155279in}}{\pgfqpoint{5.298597in}{2.149455in}}%
\pgfpathcurveto{\pgfqpoint{5.292773in}{2.143631in}}{\pgfqpoint{5.289501in}{2.135731in}}{\pgfqpoint{5.289501in}{2.127495in}}%
\pgfpathcurveto{\pgfqpoint{5.289501in}{2.119259in}}{\pgfqpoint{5.292773in}{2.111359in}}{\pgfqpoint{5.298597in}{2.105535in}}%
\pgfpathcurveto{\pgfqpoint{5.304421in}{2.099711in}}{\pgfqpoint{5.312321in}{2.096439in}}{\pgfqpoint{5.320557in}{2.096439in}}%
\pgfpathclose%
\pgfusepath{stroke,fill}%
\end{pgfscope}%
\begin{pgfscope}%
\pgfpathrectangle{\pgfqpoint{3.755891in}{0.557870in}}{\pgfqpoint{2.484109in}{1.684734in}}%
\pgfusepath{clip}%
\pgfsetbuttcap%
\pgfsetroundjoin%
\definecolor{currentfill}{rgb}{0.298039,0.447059,0.690196}%
\pgfsetfillcolor{currentfill}%
\pgfsetlinewidth{1.003750pt}%
\definecolor{currentstroke}{rgb}{0.298039,0.447059,0.690196}%
\pgfsetstrokecolor{currentstroke}%
\pgfsetdash{}{0pt}%
\pgfpathmoveto{\pgfqpoint{5.562516in}{2.077174in}}%
\pgfpathcurveto{\pgfqpoint{5.570752in}{2.077174in}}{\pgfqpoint{5.578652in}{2.080446in}}{\pgfqpoint{5.584476in}{2.086270in}}%
\pgfpathcurveto{\pgfqpoint{5.590300in}{2.092094in}}{\pgfqpoint{5.593572in}{2.099994in}}{\pgfqpoint{5.593572in}{2.108230in}}%
\pgfpathcurveto{\pgfqpoint{5.593572in}{2.116466in}}{\pgfqpoint{5.590300in}{2.124366in}}{\pgfqpoint{5.584476in}{2.130190in}}%
\pgfpathcurveto{\pgfqpoint{5.578652in}{2.136014in}}{\pgfqpoint{5.570752in}{2.139287in}}{\pgfqpoint{5.562516in}{2.139287in}}%
\pgfpathcurveto{\pgfqpoint{5.554280in}{2.139287in}}{\pgfqpoint{5.546379in}{2.136014in}}{\pgfqpoint{5.540556in}{2.130190in}}%
\pgfpathcurveto{\pgfqpoint{5.534732in}{2.124366in}}{\pgfqpoint{5.531459in}{2.116466in}}{\pgfqpoint{5.531459in}{2.108230in}}%
\pgfpathcurveto{\pgfqpoint{5.531459in}{2.099994in}}{\pgfqpoint{5.534732in}{2.092094in}}{\pgfqpoint{5.540556in}{2.086270in}}%
\pgfpathcurveto{\pgfqpoint{5.546379in}{2.080446in}}{\pgfqpoint{5.554280in}{2.077174in}}{\pgfqpoint{5.562516in}{2.077174in}}%
\pgfpathclose%
\pgfusepath{stroke,fill}%
\end{pgfscope}%
\begin{pgfscope}%
\pgfpathrectangle{\pgfqpoint{3.755891in}{0.557870in}}{\pgfqpoint{2.484109in}{1.684734in}}%
\pgfusepath{clip}%
\pgfsetbuttcap%
\pgfsetroundjoin%
\definecolor{currentfill}{rgb}{0.298039,0.447059,0.690196}%
\pgfsetfillcolor{currentfill}%
\pgfsetlinewidth{1.003750pt}%
\definecolor{currentstroke}{rgb}{0.298039,0.447059,0.690196}%
\pgfsetstrokecolor{currentstroke}%
\pgfsetdash{}{0pt}%
\pgfpathmoveto{\pgfqpoint{5.885127in}{1.759299in}}%
\pgfpathcurveto{\pgfqpoint{5.893364in}{1.759299in}}{\pgfqpoint{5.901264in}{1.762572in}}{\pgfqpoint{5.907088in}{1.768395in}}%
\pgfpathcurveto{\pgfqpoint{5.912912in}{1.774219in}}{\pgfqpoint{5.916184in}{1.782119in}}{\pgfqpoint{5.916184in}{1.790356in}}%
\pgfpathcurveto{\pgfqpoint{5.916184in}{1.798592in}}{\pgfqpoint{5.912912in}{1.806492in}}{\pgfqpoint{5.907088in}{1.812316in}}%
\pgfpathcurveto{\pgfqpoint{5.901264in}{1.818140in}}{\pgfqpoint{5.893364in}{1.821412in}}{\pgfqpoint{5.885127in}{1.821412in}}%
\pgfpathcurveto{\pgfqpoint{5.876891in}{1.821412in}}{\pgfqpoint{5.868991in}{1.818140in}}{\pgfqpoint{5.863167in}{1.812316in}}%
\pgfpathcurveto{\pgfqpoint{5.857343in}{1.806492in}}{\pgfqpoint{5.854071in}{1.798592in}}{\pgfqpoint{5.854071in}{1.790356in}}%
\pgfpathcurveto{\pgfqpoint{5.854071in}{1.782119in}}{\pgfqpoint{5.857343in}{1.774219in}}{\pgfqpoint{5.863167in}{1.768395in}}%
\pgfpathcurveto{\pgfqpoint{5.868991in}{1.762572in}}{\pgfqpoint{5.876891in}{1.759299in}}{\pgfqpoint{5.885127in}{1.759299in}}%
\pgfpathclose%
\pgfusepath{stroke,fill}%
\end{pgfscope}%
\begin{pgfscope}%
\pgfpathrectangle{\pgfqpoint{3.755891in}{0.557870in}}{\pgfqpoint{2.484109in}{1.684734in}}%
\pgfusepath{clip}%
\pgfsetbuttcap%
\pgfsetroundjoin%
\definecolor{currentfill}{rgb}{0.298039,0.447059,0.690196}%
\pgfsetfillcolor{currentfill}%
\pgfsetlinewidth{1.003750pt}%
\definecolor{currentstroke}{rgb}{0.298039,0.447059,0.690196}%
\pgfsetstrokecolor{currentstroke}%
\pgfsetdash{}{0pt}%
\pgfpathmoveto{\pgfqpoint{4.110764in}{2.115704in}}%
\pgfpathcurveto{\pgfqpoint{4.119000in}{2.115704in}}{\pgfqpoint{4.126900in}{2.118976in}}{\pgfqpoint{4.132724in}{2.124800in}}%
\pgfpathcurveto{\pgfqpoint{4.138548in}{2.130624in}}{\pgfqpoint{4.141820in}{2.138524in}}{\pgfqpoint{4.141820in}{2.146760in}}%
\pgfpathcurveto{\pgfqpoint{4.141820in}{2.154997in}}{\pgfqpoint{4.138548in}{2.162897in}}{\pgfqpoint{4.132724in}{2.168721in}}%
\pgfpathcurveto{\pgfqpoint{4.126900in}{2.174544in}}{\pgfqpoint{4.119000in}{2.177817in}}{\pgfqpoint{4.110764in}{2.177817in}}%
\pgfpathcurveto{\pgfqpoint{4.102528in}{2.177817in}}{\pgfqpoint{4.094628in}{2.174544in}}{\pgfqpoint{4.088804in}{2.168721in}}%
\pgfpathcurveto{\pgfqpoint{4.082980in}{2.162897in}}{\pgfqpoint{4.079707in}{2.154997in}}{\pgfqpoint{4.079707in}{2.146760in}}%
\pgfpathcurveto{\pgfqpoint{4.079707in}{2.138524in}}{\pgfqpoint{4.082980in}{2.130624in}}{\pgfqpoint{4.088804in}{2.124800in}}%
\pgfpathcurveto{\pgfqpoint{4.094628in}{2.118976in}}{\pgfqpoint{4.102528in}{2.115704in}}{\pgfqpoint{4.110764in}{2.115704in}}%
\pgfpathclose%
\pgfusepath{stroke,fill}%
\end{pgfscope}%
\begin{pgfscope}%
\pgfpathrectangle{\pgfqpoint{3.755891in}{0.557870in}}{\pgfqpoint{2.484109in}{1.684734in}}%
\pgfusepath{clip}%
\pgfsetbuttcap%
\pgfsetroundjoin%
\definecolor{currentfill}{rgb}{0.298039,0.447059,0.690196}%
\pgfsetfillcolor{currentfill}%
\pgfsetlinewidth{1.003750pt}%
\definecolor{currentstroke}{rgb}{0.298039,0.447059,0.690196}%
\pgfsetstrokecolor{currentstroke}%
\pgfsetdash{}{0pt}%
\pgfpathmoveto{\pgfqpoint{4.675334in}{2.046039in}}%
\pgfpathcurveto{\pgfqpoint{4.683570in}{2.046039in}}{\pgfqpoint{4.691470in}{2.049311in}}{\pgfqpoint{4.697294in}{2.055135in}}%
\pgfpathcurveto{\pgfqpoint{4.703118in}{2.060959in}}{\pgfqpoint{4.706391in}{2.068859in}}{\pgfqpoint{4.706391in}{2.077095in}}%
\pgfpathcurveto{\pgfqpoint{4.706391in}{2.085331in}}{\pgfqpoint{4.703118in}{2.093231in}}{\pgfqpoint{4.697294in}{2.099055in}}%
\pgfpathcurveto{\pgfqpoint{4.691470in}{2.104879in}}{\pgfqpoint{4.683570in}{2.108152in}}{\pgfqpoint{4.675334in}{2.108152in}}%
\pgfpathcurveto{\pgfqpoint{4.667098in}{2.108152in}}{\pgfqpoint{4.659198in}{2.104879in}}{\pgfqpoint{4.653374in}{2.099055in}}%
\pgfpathcurveto{\pgfqpoint{4.647550in}{2.093231in}}{\pgfqpoint{4.644278in}{2.085331in}}{\pgfqpoint{4.644278in}{2.077095in}}%
\pgfpathcurveto{\pgfqpoint{4.644278in}{2.068859in}}{\pgfqpoint{4.647550in}{2.060959in}}{\pgfqpoint{4.653374in}{2.055135in}}%
\pgfpathcurveto{\pgfqpoint{4.659198in}{2.049311in}}{\pgfqpoint{4.667098in}{2.046039in}}{\pgfqpoint{4.675334in}{2.046039in}}%
\pgfpathclose%
\pgfusepath{stroke,fill}%
\end{pgfscope}%
\begin{pgfscope}%
\pgfpathrectangle{\pgfqpoint{3.755891in}{0.557870in}}{\pgfqpoint{2.484109in}{1.684734in}}%
\pgfusepath{clip}%
\pgfsetbuttcap%
\pgfsetroundjoin%
\definecolor{currentfill}{rgb}{0.298039,0.447059,0.690196}%
\pgfsetfillcolor{currentfill}%
\pgfsetlinewidth{1.003750pt}%
\definecolor{currentstroke}{rgb}{0.298039,0.447059,0.690196}%
\pgfsetstrokecolor{currentstroke}%
\pgfsetdash{}{0pt}%
\pgfpathmoveto{\pgfqpoint{5.723822in}{2.046039in}}%
\pgfpathcurveto{\pgfqpoint{5.732058in}{2.046039in}}{\pgfqpoint{5.739958in}{2.049311in}}{\pgfqpoint{5.745782in}{2.055135in}}%
\pgfpathcurveto{\pgfqpoint{5.751606in}{2.060959in}}{\pgfqpoint{5.754878in}{2.068859in}}{\pgfqpoint{5.754878in}{2.077095in}}%
\pgfpathcurveto{\pgfqpoint{5.754878in}{2.085331in}}{\pgfqpoint{5.751606in}{2.093231in}}{\pgfqpoint{5.745782in}{2.099055in}}%
\pgfpathcurveto{\pgfqpoint{5.739958in}{2.104879in}}{\pgfqpoint{5.732058in}{2.108152in}}{\pgfqpoint{5.723822in}{2.108152in}}%
\pgfpathcurveto{\pgfqpoint{5.715585in}{2.108152in}}{\pgfqpoint{5.707685in}{2.104879in}}{\pgfqpoint{5.701861in}{2.099055in}}%
\pgfpathcurveto{\pgfqpoint{5.696037in}{2.093231in}}{\pgfqpoint{5.692765in}{2.085331in}}{\pgfqpoint{5.692765in}{2.077095in}}%
\pgfpathcurveto{\pgfqpoint{5.692765in}{2.068859in}}{\pgfqpoint{5.696037in}{2.060959in}}{\pgfqpoint{5.701861in}{2.055135in}}%
\pgfpathcurveto{\pgfqpoint{5.707685in}{2.049311in}}{\pgfqpoint{5.715585in}{2.046039in}}{\pgfqpoint{5.723822in}{2.046039in}}%
\pgfpathclose%
\pgfusepath{stroke,fill}%
\end{pgfscope}%
\begin{pgfscope}%
\pgfpathrectangle{\pgfqpoint{3.755891in}{0.557870in}}{\pgfqpoint{2.484109in}{1.684734in}}%
\pgfusepath{clip}%
\pgfsetbuttcap%
\pgfsetroundjoin%
\definecolor{currentfill}{rgb}{0.298039,0.447059,0.690196}%
\pgfsetfillcolor{currentfill}%
\pgfsetlinewidth{1.003750pt}%
\definecolor{currentstroke}{rgb}{0.298039,0.447059,0.690196}%
\pgfsetstrokecolor{currentstroke}%
\pgfsetdash{}{0pt}%
\pgfpathmoveto{\pgfqpoint{5.401210in}{2.085563in}}%
\pgfpathcurveto{\pgfqpoint{5.409446in}{2.085563in}}{\pgfqpoint{5.417346in}{2.088835in}}{\pgfqpoint{5.423170in}{2.094659in}}%
\pgfpathcurveto{\pgfqpoint{5.428994in}{2.100483in}}{\pgfqpoint{5.432267in}{2.108383in}}{\pgfqpoint{5.432267in}{2.116620in}}%
\pgfpathcurveto{\pgfqpoint{5.432267in}{2.124856in}}{\pgfqpoint{5.428994in}{2.132756in}}{\pgfqpoint{5.423170in}{2.138580in}}%
\pgfpathcurveto{\pgfqpoint{5.417346in}{2.144404in}}{\pgfqpoint{5.409446in}{2.147676in}}{\pgfqpoint{5.401210in}{2.147676in}}%
\pgfpathcurveto{\pgfqpoint{5.392974in}{2.147676in}}{\pgfqpoint{5.385074in}{2.144404in}}{\pgfqpoint{5.379250in}{2.138580in}}%
\pgfpathcurveto{\pgfqpoint{5.373426in}{2.132756in}}{\pgfqpoint{5.370154in}{2.124856in}}{\pgfqpoint{5.370154in}{2.116620in}}%
\pgfpathcurveto{\pgfqpoint{5.370154in}{2.108383in}}{\pgfqpoint{5.373426in}{2.100483in}}{\pgfqpoint{5.379250in}{2.094659in}}%
\pgfpathcurveto{\pgfqpoint{5.385074in}{2.088835in}}{\pgfqpoint{5.392974in}{2.085563in}}{\pgfqpoint{5.401210in}{2.085563in}}%
\pgfpathclose%
\pgfusepath{stroke,fill}%
\end{pgfscope}%
\begin{pgfscope}%
\pgfpathrectangle{\pgfqpoint{3.755891in}{0.557870in}}{\pgfqpoint{2.484109in}{1.684734in}}%
\pgfusepath{clip}%
\pgfsetbuttcap%
\pgfsetroundjoin%
\definecolor{currentfill}{rgb}{0.298039,0.447059,0.690196}%
\pgfsetfillcolor{currentfill}%
\pgfsetlinewidth{1.003750pt}%
\definecolor{currentstroke}{rgb}{0.298039,0.447059,0.690196}%
\pgfsetstrokecolor{currentstroke}%
\pgfsetdash{}{0pt}%
\pgfpathmoveto{\pgfqpoint{5.643169in}{2.026276in}}%
\pgfpathcurveto{\pgfqpoint{5.651405in}{2.026276in}}{\pgfqpoint{5.659305in}{2.029549in}}{\pgfqpoint{5.665129in}{2.035373in}}%
\pgfpathcurveto{\pgfqpoint{5.670953in}{2.041197in}}{\pgfqpoint{5.674225in}{2.049097in}}{\pgfqpoint{5.674225in}{2.057333in}}%
\pgfpathcurveto{\pgfqpoint{5.674225in}{2.065569in}}{\pgfqpoint{5.670953in}{2.073469in}}{\pgfqpoint{5.665129in}{2.079293in}}%
\pgfpathcurveto{\pgfqpoint{5.659305in}{2.085117in}}{\pgfqpoint{5.651405in}{2.088389in}}{\pgfqpoint{5.643169in}{2.088389in}}%
\pgfpathcurveto{\pgfqpoint{5.634932in}{2.088389in}}{\pgfqpoint{5.627032in}{2.085117in}}{\pgfqpoint{5.621208in}{2.079293in}}%
\pgfpathcurveto{\pgfqpoint{5.615385in}{2.073469in}}{\pgfqpoint{5.612112in}{2.065569in}}{\pgfqpoint{5.612112in}{2.057333in}}%
\pgfpathcurveto{\pgfqpoint{5.612112in}{2.049097in}}{\pgfqpoint{5.615385in}{2.041197in}}{\pgfqpoint{5.621208in}{2.035373in}}%
\pgfpathcurveto{\pgfqpoint{5.627032in}{2.029549in}}{\pgfqpoint{5.634932in}{2.026276in}}{\pgfqpoint{5.643169in}{2.026276in}}%
\pgfpathclose%
\pgfusepath{stroke,fill}%
\end{pgfscope}%
\begin{pgfscope}%
\pgfpathrectangle{\pgfqpoint{3.755891in}{0.557870in}}{\pgfqpoint{2.484109in}{1.684734in}}%
\pgfusepath{clip}%
\pgfsetbuttcap%
\pgfsetroundjoin%
\definecolor{currentfill}{rgb}{0.298039,0.447059,0.690196}%
\pgfsetfillcolor{currentfill}%
\pgfsetlinewidth{1.003750pt}%
\definecolor{currentstroke}{rgb}{0.298039,0.447059,0.690196}%
\pgfsetstrokecolor{currentstroke}%
\pgfsetdash{}{0pt}%
\pgfpathmoveto{\pgfqpoint{5.481863in}{2.016395in}}%
\pgfpathcurveto{\pgfqpoint{5.490099in}{2.016395in}}{\pgfqpoint{5.497999in}{2.019668in}}{\pgfqpoint{5.503823in}{2.025491in}}%
\pgfpathcurveto{\pgfqpoint{5.509647in}{2.031315in}}{\pgfqpoint{5.512919in}{2.039215in}}{\pgfqpoint{5.512919in}{2.047452in}}%
\pgfpathcurveto{\pgfqpoint{5.512919in}{2.055688in}}{\pgfqpoint{5.509647in}{2.063588in}}{\pgfqpoint{5.503823in}{2.069412in}}%
\pgfpathcurveto{\pgfqpoint{5.497999in}{2.075236in}}{\pgfqpoint{5.490099in}{2.078508in}}{\pgfqpoint{5.481863in}{2.078508in}}%
\pgfpathcurveto{\pgfqpoint{5.473627in}{2.078508in}}{\pgfqpoint{5.465727in}{2.075236in}}{\pgfqpoint{5.459903in}{2.069412in}}%
\pgfpathcurveto{\pgfqpoint{5.454079in}{2.063588in}}{\pgfqpoint{5.450806in}{2.055688in}}{\pgfqpoint{5.450806in}{2.047452in}}%
\pgfpathcurveto{\pgfqpoint{5.450806in}{2.039215in}}{\pgfqpoint{5.454079in}{2.031315in}}{\pgfqpoint{5.459903in}{2.025491in}}%
\pgfpathcurveto{\pgfqpoint{5.465727in}{2.019668in}}{\pgfqpoint{5.473627in}{2.016395in}}{\pgfqpoint{5.481863in}{2.016395in}}%
\pgfpathclose%
\pgfusepath{stroke,fill}%
\end{pgfscope}%
\begin{pgfscope}%
\pgfpathrectangle{\pgfqpoint{3.755891in}{0.557870in}}{\pgfqpoint{2.484109in}{1.684734in}}%
\pgfusepath{clip}%
\pgfsetbuttcap%
\pgfsetroundjoin%
\definecolor{currentfill}{rgb}{0.298039,0.447059,0.690196}%
\pgfsetfillcolor{currentfill}%
\pgfsetlinewidth{1.003750pt}%
\definecolor{currentstroke}{rgb}{0.298039,0.447059,0.690196}%
\pgfsetstrokecolor{currentstroke}%
\pgfsetdash{}{0pt}%
\pgfpathmoveto{\pgfqpoint{5.643169in}{2.095444in}}%
\pgfpathcurveto{\pgfqpoint{5.651405in}{2.095444in}}{\pgfqpoint{5.659305in}{2.098717in}}{\pgfqpoint{5.665129in}{2.104541in}}%
\pgfpathcurveto{\pgfqpoint{5.670953in}{2.110364in}}{\pgfqpoint{5.674225in}{2.118265in}}{\pgfqpoint{5.674225in}{2.126501in}}%
\pgfpathcurveto{\pgfqpoint{5.674225in}{2.134737in}}{\pgfqpoint{5.670953in}{2.142637in}}{\pgfqpoint{5.665129in}{2.148461in}}%
\pgfpathcurveto{\pgfqpoint{5.659305in}{2.154285in}}{\pgfqpoint{5.651405in}{2.157557in}}{\pgfqpoint{5.643169in}{2.157557in}}%
\pgfpathcurveto{\pgfqpoint{5.634932in}{2.157557in}}{\pgfqpoint{5.627032in}{2.154285in}}{\pgfqpoint{5.621208in}{2.148461in}}%
\pgfpathcurveto{\pgfqpoint{5.615385in}{2.142637in}}{\pgfqpoint{5.612112in}{2.134737in}}{\pgfqpoint{5.612112in}{2.126501in}}%
\pgfpathcurveto{\pgfqpoint{5.612112in}{2.118265in}}{\pgfqpoint{5.615385in}{2.110364in}}{\pgfqpoint{5.621208in}{2.104541in}}%
\pgfpathcurveto{\pgfqpoint{5.627032in}{2.098717in}}{\pgfqpoint{5.634932in}{2.095444in}}{\pgfqpoint{5.643169in}{2.095444in}}%
\pgfpathclose%
\pgfusepath{stroke,fill}%
\end{pgfscope}%
\begin{pgfscope}%
\pgfpathrectangle{\pgfqpoint{3.755891in}{0.557870in}}{\pgfqpoint{2.484109in}{1.684734in}}%
\pgfusepath{clip}%
\pgfsetbuttcap%
\pgfsetroundjoin%
\definecolor{currentfill}{rgb}{0.298039,0.447059,0.690196}%
\pgfsetfillcolor{currentfill}%
\pgfsetlinewidth{1.003750pt}%
\definecolor{currentstroke}{rgb}{0.298039,0.447059,0.690196}%
\pgfsetstrokecolor{currentstroke}%
\pgfsetdash{}{0pt}%
\pgfpathmoveto{\pgfqpoint{5.078599in}{2.055920in}}%
\pgfpathcurveto{\pgfqpoint{5.086835in}{2.055920in}}{\pgfqpoint{5.094735in}{2.059192in}}{\pgfqpoint{5.100559in}{2.065016in}}%
\pgfpathcurveto{\pgfqpoint{5.106383in}{2.070840in}}{\pgfqpoint{5.109655in}{2.078740in}}{\pgfqpoint{5.109655in}{2.086976in}}%
\pgfpathcurveto{\pgfqpoint{5.109655in}{2.095213in}}{\pgfqpoint{5.106383in}{2.103113in}}{\pgfqpoint{5.100559in}{2.108937in}}%
\pgfpathcurveto{\pgfqpoint{5.094735in}{2.114760in}}{\pgfqpoint{5.086835in}{2.118033in}}{\pgfqpoint{5.078599in}{2.118033in}}%
\pgfpathcurveto{\pgfqpoint{5.070362in}{2.118033in}}{\pgfqpoint{5.062462in}{2.114760in}}{\pgfqpoint{5.056638in}{2.108937in}}%
\pgfpathcurveto{\pgfqpoint{5.050814in}{2.103113in}}{\pgfqpoint{5.047542in}{2.095213in}}{\pgfqpoint{5.047542in}{2.086976in}}%
\pgfpathcurveto{\pgfqpoint{5.047542in}{2.078740in}}{\pgfqpoint{5.050814in}{2.070840in}}{\pgfqpoint{5.056638in}{2.065016in}}%
\pgfpathcurveto{\pgfqpoint{5.062462in}{2.059192in}}{\pgfqpoint{5.070362in}{2.055920in}}{\pgfqpoint{5.078599in}{2.055920in}}%
\pgfpathclose%
\pgfusepath{stroke,fill}%
\end{pgfscope}%
\begin{pgfscope}%
\pgfpathrectangle{\pgfqpoint{3.755891in}{0.557870in}}{\pgfqpoint{2.484109in}{1.684734in}}%
\pgfusepath{clip}%
\pgfsetbuttcap%
\pgfsetroundjoin%
\definecolor{currentfill}{rgb}{0.298039,0.447059,0.690196}%
\pgfsetfillcolor{currentfill}%
\pgfsetlinewidth{1.003750pt}%
\definecolor{currentstroke}{rgb}{0.298039,0.447059,0.690196}%
\pgfsetstrokecolor{currentstroke}%
\pgfsetdash{}{0pt}%
\pgfpathmoveto{\pgfqpoint{5.481863in}{2.085563in}}%
\pgfpathcurveto{\pgfqpoint{5.490099in}{2.085563in}}{\pgfqpoint{5.497999in}{2.088835in}}{\pgfqpoint{5.503823in}{2.094659in}}%
\pgfpathcurveto{\pgfqpoint{5.509647in}{2.100483in}}{\pgfqpoint{5.512919in}{2.108383in}}{\pgfqpoint{5.512919in}{2.116620in}}%
\pgfpathcurveto{\pgfqpoint{5.512919in}{2.124856in}}{\pgfqpoint{5.509647in}{2.132756in}}{\pgfqpoint{5.503823in}{2.138580in}}%
\pgfpathcurveto{\pgfqpoint{5.497999in}{2.144404in}}{\pgfqpoint{5.490099in}{2.147676in}}{\pgfqpoint{5.481863in}{2.147676in}}%
\pgfpathcurveto{\pgfqpoint{5.473627in}{2.147676in}}{\pgfqpoint{5.465727in}{2.144404in}}{\pgfqpoint{5.459903in}{2.138580in}}%
\pgfpathcurveto{\pgfqpoint{5.454079in}{2.132756in}}{\pgfqpoint{5.450806in}{2.124856in}}{\pgfqpoint{5.450806in}{2.116620in}}%
\pgfpathcurveto{\pgfqpoint{5.450806in}{2.108383in}}{\pgfqpoint{5.454079in}{2.100483in}}{\pgfqpoint{5.459903in}{2.094659in}}%
\pgfpathcurveto{\pgfqpoint{5.465727in}{2.088835in}}{\pgfqpoint{5.473627in}{2.085563in}}{\pgfqpoint{5.481863in}{2.085563in}}%
\pgfpathclose%
\pgfusepath{stroke,fill}%
\end{pgfscope}%
\begin{pgfscope}%
\pgfpathrectangle{\pgfqpoint{3.755891in}{0.557870in}}{\pgfqpoint{2.484109in}{1.684734in}}%
\pgfusepath{clip}%
\pgfsetbuttcap%
\pgfsetroundjoin%
\definecolor{currentfill}{rgb}{0.298039,0.447059,0.690196}%
\pgfsetfillcolor{currentfill}%
\pgfsetlinewidth{1.003750pt}%
\definecolor{currentstroke}{rgb}{0.298039,0.447059,0.690196}%
\pgfsetstrokecolor{currentstroke}%
\pgfsetdash{}{0pt}%
\pgfpathmoveto{\pgfqpoint{5.885127in}{1.749605in}}%
\pgfpathcurveto{\pgfqpoint{5.893364in}{1.749605in}}{\pgfqpoint{5.901264in}{1.752877in}}{\pgfqpoint{5.907088in}{1.758701in}}%
\pgfpathcurveto{\pgfqpoint{5.912912in}{1.764525in}}{\pgfqpoint{5.916184in}{1.772425in}}{\pgfqpoint{5.916184in}{1.780661in}}%
\pgfpathcurveto{\pgfqpoint{5.916184in}{1.788897in}}{\pgfqpoint{5.912912in}{1.796797in}}{\pgfqpoint{5.907088in}{1.802621in}}%
\pgfpathcurveto{\pgfqpoint{5.901264in}{1.808445in}}{\pgfqpoint{5.893364in}{1.811718in}}{\pgfqpoint{5.885127in}{1.811718in}}%
\pgfpathcurveto{\pgfqpoint{5.876891in}{1.811718in}}{\pgfqpoint{5.868991in}{1.808445in}}{\pgfqpoint{5.863167in}{1.802621in}}%
\pgfpathcurveto{\pgfqpoint{5.857343in}{1.796797in}}{\pgfqpoint{5.854071in}{1.788897in}}{\pgfqpoint{5.854071in}{1.780661in}}%
\pgfpathcurveto{\pgfqpoint{5.854071in}{1.772425in}}{\pgfqpoint{5.857343in}{1.764525in}}{\pgfqpoint{5.863167in}{1.758701in}}%
\pgfpathcurveto{\pgfqpoint{5.868991in}{1.752877in}}{\pgfqpoint{5.876891in}{1.749605in}}{\pgfqpoint{5.885127in}{1.749605in}}%
\pgfpathclose%
\pgfusepath{stroke,fill}%
\end{pgfscope}%
\begin{pgfscope}%
\pgfsetrectcap%
\pgfsetmiterjoin%
\pgfsetlinewidth{1.254687pt}%
\definecolor{currentstroke}{rgb}{1.000000,1.000000,1.000000}%
\pgfsetstrokecolor{currentstroke}%
\pgfsetdash{}{0pt}%
\pgfpathmoveto{\pgfqpoint{3.755891in}{0.557870in}}%
\pgfpathlineto{\pgfqpoint{3.755891in}{2.242604in}}%
\pgfusepath{stroke}%
\end{pgfscope}%
\begin{pgfscope}%
\pgfsetrectcap%
\pgfsetmiterjoin%
\pgfsetlinewidth{1.254687pt}%
\definecolor{currentstroke}{rgb}{1.000000,1.000000,1.000000}%
\pgfsetstrokecolor{currentstroke}%
\pgfsetdash{}{0pt}%
\pgfpathmoveto{\pgfqpoint{6.240000in}{0.557870in}}%
\pgfpathlineto{\pgfqpoint{6.240000in}{2.242604in}}%
\pgfusepath{stroke}%
\end{pgfscope}%
\begin{pgfscope}%
\pgfsetrectcap%
\pgfsetmiterjoin%
\pgfsetlinewidth{1.254687pt}%
\definecolor{currentstroke}{rgb}{1.000000,1.000000,1.000000}%
\pgfsetstrokecolor{currentstroke}%
\pgfsetdash{}{0pt}%
\pgfpathmoveto{\pgfqpoint{3.755891in}{0.557870in}}%
\pgfpathlineto{\pgfqpoint{6.240000in}{0.557870in}}%
\pgfusepath{stroke}%
\end{pgfscope}%
\begin{pgfscope}%
\pgfsetrectcap%
\pgfsetmiterjoin%
\pgfsetlinewidth{1.254687pt}%
\definecolor{currentstroke}{rgb}{1.000000,1.000000,1.000000}%
\pgfsetstrokecolor{currentstroke}%
\pgfsetdash{}{0pt}%
\pgfpathmoveto{\pgfqpoint{3.755891in}{2.242604in}}%
\pgfpathlineto{\pgfqpoint{6.240000in}{2.242604in}}%
\pgfusepath{stroke}%
\end{pgfscope}%
\begin{pgfscope}%
\definecolor{textcolor}{rgb}{0.150000,0.150000,0.150000}%
\pgfsetstrokecolor{textcolor}%
\pgfsetfillcolor{textcolor}%
\pgftext[x=4.997946in,y=2.325938in,,base]{\color{textcolor}\sffamily\fontsize{11.000000}{13.200000}\selectfont (b)}%
\end{pgfscope}%
\end{pgfpicture}%
\makeatother%
\endgroup%

    \caption{Distribution of DOR, sensitivity and specificity for the different peak-value classifiers trained to predict patient diagnosis.}
    \label{fig:pvmlc_ind_dor_sens_spec_dist}
\end{figure}

\begin{table*}
    \centering
    \ra{1.3}
    \begin{tabular}{lrrrr}
        \toprule
        Dataset-Model              &  Accuracy &  Sensitivity &  Specificity &  DOR \\
        \midrule
        gls-rls-EF/Ada-Boost   &      0.95 &         0.97 &         0.79 & 138.42 \\
        gls-rls/KNN            &      0.93 &         0.95 &         0.82 &  84.53 \\
        rls-EF/Extra-Trees     &      0.93 &         0.96 &         0.75 &  76.50 \\
        gls-rls-EF/Extra-Trees &      0.93 &         0.97 &         0.71 &  75.00 \\
        gls-rls/Extra-Trees    &      0.93 &         0.97 &         0.71 &  75.00 \\
        \bottomrule
    \end{tabular}
    \caption{The accuracy, DOR, sensitivity and specicity scores of the five best performing PVSC models in terms of DOR, when trained to predict patient diagnosis.
             The \textbf{Dataset-Model} column indicates \textit{Dataset used}$/$\textit{Specific machine learning model used}.}
    \label{tab:dl_hf_dor_sens_spec_dist}
\end{table*}

\begin{comment}
[ ] \textbf{Comment on spread of DOR.}
[ ] \textbf{Comment on spread of sensitivity and specificity.}
[ ] \textbf{Comment on common traits in the high performing methods.} Here you can refer to raw performance results in appendix.
[ ] \textbf{Comment on common traits in the low performing methods.} Here you can refer to raw performance results in appendix.
[ ] \textbf{Select one - three methods that are good contendors for being the best method/model in the group and comment on their traits}
\textbf{IF NOT CLUSTERING METHOD}
[ ] \textbf{Make arguments for and against the top three methods in terms of accuracy, sensitivity, specificity, and DOR, and make an informed choice.}
\end{comment}

\newpage

\subsection{Comparisons}

\newpage

