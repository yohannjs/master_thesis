\section{Case Study: Segment Indication}

\subsection{Time-series Clustering}

\begin{figure}[htb]
    \centering
    %% Creator: Matplotlib, PGF backend
%%
%% To include the figure in your LaTeX document, write
%%   \input{<filename>.pgf}
%%
%% Make sure the required packages are loaded in your preamble
%%   \usepackage{pgf}
%%
%% Figures using additional raster images can only be included by \input if
%% they are in the same directory as the main LaTeX file. For loading figures
%% from other directories you can use the `import` package
%%   \usepackage{import}
%% and then include the figures with
%%   \import{<path to file>}{<filename>.pgf}
%%
%% Matplotlib used the following preamble
%%
\begingroup%
\makeatletter%
\begin{pgfpicture}%
\pgfpathrectangle{\pgfpointorigin}{\pgfqpoint{6.480568in}{2.540000in}}%
\pgfusepath{use as bounding box, clip}%
\begin{pgfscope}%
\pgfsetbuttcap%
\pgfsetmiterjoin%
\definecolor{currentfill}{rgb}{1.000000,1.000000,1.000000}%
\pgfsetfillcolor{currentfill}%
\pgfsetlinewidth{0.000000pt}%
\definecolor{currentstroke}{rgb}{1.000000,1.000000,1.000000}%
\pgfsetstrokecolor{currentstroke}%
\pgfsetdash{}{0pt}%
\pgfpathmoveto{\pgfqpoint{0.000000in}{0.000000in}}%
\pgfpathlineto{\pgfqpoint{6.480568in}{0.000000in}}%
\pgfpathlineto{\pgfqpoint{6.480568in}{2.540000in}}%
\pgfpathlineto{\pgfqpoint{0.000000in}{2.540000in}}%
\pgfpathclose%
\pgfusepath{fill}%
\end{pgfscope}%
\begin{pgfscope}%
\pgfsetbuttcap%
\pgfsetmiterjoin%
\definecolor{currentfill}{rgb}{0.917647,0.917647,0.949020}%
\pgfsetfillcolor{currentfill}%
\pgfsetlinewidth{0.000000pt}%
\definecolor{currentstroke}{rgb}{0.000000,0.000000,0.000000}%
\pgfsetstrokecolor{currentstroke}%
\pgfsetstrokeopacity{0.000000}%
\pgfsetdash{}{0pt}%
\pgfpathmoveto{\pgfqpoint{0.693056in}{0.557870in}}%
\pgfpathlineto{\pgfqpoint{3.176962in}{0.557870in}}%
\pgfpathlineto{\pgfqpoint{3.176962in}{2.242604in}}%
\pgfpathlineto{\pgfqpoint{0.693056in}{2.242604in}}%
\pgfpathclose%
\pgfusepath{fill}%
\end{pgfscope}%
\begin{pgfscope}%
\pgfpathrectangle{\pgfqpoint{0.693056in}{0.557870in}}{\pgfqpoint{2.483906in}{1.684734in}}%
\pgfusepath{clip}%
\pgfsetroundcap%
\pgfsetroundjoin%
\pgfsetlinewidth{1.003750pt}%
\definecolor{currentstroke}{rgb}{1.000000,1.000000,1.000000}%
\pgfsetstrokecolor{currentstroke}%
\pgfsetdash{}{0pt}%
\pgfpathmoveto{\pgfqpoint{0.805961in}{0.557870in}}%
\pgfpathlineto{\pgfqpoint{0.805961in}{2.242604in}}%
\pgfusepath{stroke}%
\end{pgfscope}%
\begin{pgfscope}%
\definecolor{textcolor}{rgb}{0.150000,0.150000,0.150000}%
\pgfsetstrokecolor{textcolor}%
\pgfsetfillcolor{textcolor}%
\pgftext[x=0.805961in,y=0.425926in,,top]{\color{textcolor}\sffamily\fontsize{11.000000}{13.200000}\selectfont \(\displaystyle 0\)}%
\end{pgfscope}%
\begin{pgfscope}%
\pgfpathrectangle{\pgfqpoint{0.693056in}{0.557870in}}{\pgfqpoint{2.483906in}{1.684734in}}%
\pgfusepath{clip}%
\pgfsetroundcap%
\pgfsetroundjoin%
\pgfsetlinewidth{1.003750pt}%
\definecolor{currentstroke}{rgb}{1.000000,1.000000,1.000000}%
\pgfsetstrokecolor{currentstroke}%
\pgfsetdash{}{0pt}%
\pgfpathmoveto{\pgfqpoint{1.568945in}{0.557870in}}%
\pgfpathlineto{\pgfqpoint{1.568945in}{2.242604in}}%
\pgfusepath{stroke}%
\end{pgfscope}%
\begin{pgfscope}%
\definecolor{textcolor}{rgb}{0.150000,0.150000,0.150000}%
\pgfsetstrokecolor{textcolor}%
\pgfsetfillcolor{textcolor}%
\pgftext[x=1.568945in,y=0.425926in,,top]{\color{textcolor}\sffamily\fontsize{11.000000}{13.200000}\selectfont \(\displaystyle 5\)}%
\end{pgfscope}%
\begin{pgfscope}%
\pgfpathrectangle{\pgfqpoint{0.693056in}{0.557870in}}{\pgfqpoint{2.483906in}{1.684734in}}%
\pgfusepath{clip}%
\pgfsetroundcap%
\pgfsetroundjoin%
\pgfsetlinewidth{1.003750pt}%
\definecolor{currentstroke}{rgb}{1.000000,1.000000,1.000000}%
\pgfsetstrokecolor{currentstroke}%
\pgfsetdash{}{0pt}%
\pgfpathmoveto{\pgfqpoint{2.331929in}{0.557870in}}%
\pgfpathlineto{\pgfqpoint{2.331929in}{2.242604in}}%
\pgfusepath{stroke}%
\end{pgfscope}%
\begin{pgfscope}%
\definecolor{textcolor}{rgb}{0.150000,0.150000,0.150000}%
\pgfsetstrokecolor{textcolor}%
\pgfsetfillcolor{textcolor}%
\pgftext[x=2.331929in,y=0.425926in,,top]{\color{textcolor}\sffamily\fontsize{11.000000}{13.200000}\selectfont \(\displaystyle 10\)}%
\end{pgfscope}%
\begin{pgfscope}%
\pgfpathrectangle{\pgfqpoint{0.693056in}{0.557870in}}{\pgfqpoint{2.483906in}{1.684734in}}%
\pgfusepath{clip}%
\pgfsetroundcap%
\pgfsetroundjoin%
\pgfsetlinewidth{1.003750pt}%
\definecolor{currentstroke}{rgb}{1.000000,1.000000,1.000000}%
\pgfsetstrokecolor{currentstroke}%
\pgfsetdash{}{0pt}%
\pgfpathmoveto{\pgfqpoint{3.094913in}{0.557870in}}%
\pgfpathlineto{\pgfqpoint{3.094913in}{2.242604in}}%
\pgfusepath{stroke}%
\end{pgfscope}%
\begin{pgfscope}%
\definecolor{textcolor}{rgb}{0.150000,0.150000,0.150000}%
\pgfsetstrokecolor{textcolor}%
\pgfsetfillcolor{textcolor}%
\pgftext[x=3.094913in,y=0.425926in,,top]{\color{textcolor}\sffamily\fontsize{11.000000}{13.200000}\selectfont \(\displaystyle 15\)}%
\end{pgfscope}%
\begin{pgfscope}%
\definecolor{textcolor}{rgb}{0.150000,0.150000,0.150000}%
\pgfsetstrokecolor{textcolor}%
\pgfsetfillcolor{textcolor}%
\pgftext[x=1.935009in,y=0.235185in,,top]{\color{textcolor}\sffamily\fontsize{11.000000}{13.200000}\selectfont DOR}%
\end{pgfscope}%
\begin{pgfscope}%
\pgfpathrectangle{\pgfqpoint{0.693056in}{0.557870in}}{\pgfqpoint{2.483906in}{1.684734in}}%
\pgfusepath{clip}%
\pgfsetroundcap%
\pgfsetroundjoin%
\pgfsetlinewidth{1.003750pt}%
\definecolor{currentstroke}{rgb}{1.000000,1.000000,1.000000}%
\pgfsetstrokecolor{currentstroke}%
\pgfsetdash{}{0pt}%
\pgfpathmoveto{\pgfqpoint{0.693056in}{0.557870in}}%
\pgfpathlineto{\pgfqpoint{3.176962in}{0.557870in}}%
\pgfusepath{stroke}%
\end{pgfscope}%
\begin{pgfscope}%
\definecolor{textcolor}{rgb}{0.150000,0.150000,0.150000}%
\pgfsetstrokecolor{textcolor}%
\pgfsetfillcolor{textcolor}%
\pgftext[x=0.366783in,y=0.505064in,left,base]{\color{textcolor}\sffamily\fontsize{11.000000}{13.200000}\selectfont \(\displaystyle 0.0\)}%
\end{pgfscope}%
\begin{pgfscope}%
\pgfpathrectangle{\pgfqpoint{0.693056in}{0.557870in}}{\pgfqpoint{2.483906in}{1.684734in}}%
\pgfusepath{clip}%
\pgfsetroundcap%
\pgfsetroundjoin%
\pgfsetlinewidth{1.003750pt}%
\definecolor{currentstroke}{rgb}{1.000000,1.000000,1.000000}%
\pgfsetstrokecolor{currentstroke}%
\pgfsetdash{}{0pt}%
\pgfpathmoveto{\pgfqpoint{0.693056in}{0.958997in}}%
\pgfpathlineto{\pgfqpoint{3.176962in}{0.958997in}}%
\pgfusepath{stroke}%
\end{pgfscope}%
\begin{pgfscope}%
\definecolor{textcolor}{rgb}{0.150000,0.150000,0.150000}%
\pgfsetstrokecolor{textcolor}%
\pgfsetfillcolor{textcolor}%
\pgftext[x=0.366783in,y=0.906191in,left,base]{\color{textcolor}\sffamily\fontsize{11.000000}{13.200000}\selectfont \(\displaystyle 2.5\)}%
\end{pgfscope}%
\begin{pgfscope}%
\pgfpathrectangle{\pgfqpoint{0.693056in}{0.557870in}}{\pgfqpoint{2.483906in}{1.684734in}}%
\pgfusepath{clip}%
\pgfsetroundcap%
\pgfsetroundjoin%
\pgfsetlinewidth{1.003750pt}%
\definecolor{currentstroke}{rgb}{1.000000,1.000000,1.000000}%
\pgfsetstrokecolor{currentstroke}%
\pgfsetdash{}{0pt}%
\pgfpathmoveto{\pgfqpoint{0.693056in}{1.360125in}}%
\pgfpathlineto{\pgfqpoint{3.176962in}{1.360125in}}%
\pgfusepath{stroke}%
\end{pgfscope}%
\begin{pgfscope}%
\definecolor{textcolor}{rgb}{0.150000,0.150000,0.150000}%
\pgfsetstrokecolor{textcolor}%
\pgfsetfillcolor{textcolor}%
\pgftext[x=0.366783in,y=1.307318in,left,base]{\color{textcolor}\sffamily\fontsize{11.000000}{13.200000}\selectfont \(\displaystyle 5.0\)}%
\end{pgfscope}%
\begin{pgfscope}%
\pgfpathrectangle{\pgfqpoint{0.693056in}{0.557870in}}{\pgfqpoint{2.483906in}{1.684734in}}%
\pgfusepath{clip}%
\pgfsetroundcap%
\pgfsetroundjoin%
\pgfsetlinewidth{1.003750pt}%
\definecolor{currentstroke}{rgb}{1.000000,1.000000,1.000000}%
\pgfsetstrokecolor{currentstroke}%
\pgfsetdash{}{0pt}%
\pgfpathmoveto{\pgfqpoint{0.693056in}{1.761252in}}%
\pgfpathlineto{\pgfqpoint{3.176962in}{1.761252in}}%
\pgfusepath{stroke}%
\end{pgfscope}%
\begin{pgfscope}%
\definecolor{textcolor}{rgb}{0.150000,0.150000,0.150000}%
\pgfsetstrokecolor{textcolor}%
\pgfsetfillcolor{textcolor}%
\pgftext[x=0.366783in,y=1.708445in,left,base]{\color{textcolor}\sffamily\fontsize{11.000000}{13.200000}\selectfont \(\displaystyle 7.5\)}%
\end{pgfscope}%
\begin{pgfscope}%
\pgfpathrectangle{\pgfqpoint{0.693056in}{0.557870in}}{\pgfqpoint{2.483906in}{1.684734in}}%
\pgfusepath{clip}%
\pgfsetroundcap%
\pgfsetroundjoin%
\pgfsetlinewidth{1.003750pt}%
\definecolor{currentstroke}{rgb}{1.000000,1.000000,1.000000}%
\pgfsetstrokecolor{currentstroke}%
\pgfsetdash{}{0pt}%
\pgfpathmoveto{\pgfqpoint{0.693056in}{2.162379in}}%
\pgfpathlineto{\pgfqpoint{3.176962in}{2.162379in}}%
\pgfusepath{stroke}%
\end{pgfscope}%
\begin{pgfscope}%
\definecolor{textcolor}{rgb}{0.150000,0.150000,0.150000}%
\pgfsetstrokecolor{textcolor}%
\pgfsetfillcolor{textcolor}%
\pgftext[x=0.290741in,y=2.109572in,left,base]{\color{textcolor}\sffamily\fontsize{11.000000}{13.200000}\selectfont \(\displaystyle 10.0\)}%
\end{pgfscope}%
\begin{pgfscope}%
\definecolor{textcolor}{rgb}{0.150000,0.150000,0.150000}%
\pgfsetstrokecolor{textcolor}%
\pgfsetfillcolor{textcolor}%
\pgftext[x=0.235185in,y=1.400237in,,bottom,rotate=90.000000]{\color{textcolor}\sffamily\fontsize{11.000000}{13.200000}\selectfont Occurance}%
\end{pgfscope}%
\begin{pgfscope}%
\pgfpathrectangle{\pgfqpoint{0.693056in}{0.557870in}}{\pgfqpoint{2.483906in}{1.684734in}}%
\pgfusepath{clip}%
\pgfsetbuttcap%
\pgfsetmiterjoin%
\definecolor{currentfill}{rgb}{0.298039,0.447059,0.690196}%
\pgfsetfillcolor{currentfill}%
\pgfsetfillopacity{0.400000}%
\pgfsetlinewidth{1.003750pt}%
\definecolor{currentstroke}{rgb}{1.000000,1.000000,1.000000}%
\pgfsetstrokecolor{currentstroke}%
\pgfsetstrokeopacity{0.400000}%
\pgfsetdash{}{0pt}%
\pgfpathmoveto{\pgfqpoint{0.805961in}{0.557870in}}%
\pgfpathlineto{\pgfqpoint{1.031770in}{0.557870in}}%
\pgfpathlineto{\pgfqpoint{1.031770in}{2.162379in}}%
\pgfpathlineto{\pgfqpoint{0.805961in}{2.162379in}}%
\pgfpathclose%
\pgfusepath{stroke,fill}%
\end{pgfscope}%
\begin{pgfscope}%
\pgfpathrectangle{\pgfqpoint{0.693056in}{0.557870in}}{\pgfqpoint{2.483906in}{1.684734in}}%
\pgfusepath{clip}%
\pgfsetbuttcap%
\pgfsetmiterjoin%
\definecolor{currentfill}{rgb}{0.298039,0.447059,0.690196}%
\pgfsetfillcolor{currentfill}%
\pgfsetfillopacity{0.400000}%
\pgfsetlinewidth{1.003750pt}%
\definecolor{currentstroke}{rgb}{1.000000,1.000000,1.000000}%
\pgfsetstrokecolor{currentstroke}%
\pgfsetstrokeopacity{0.400000}%
\pgfsetdash{}{0pt}%
\pgfpathmoveto{\pgfqpoint{1.031770in}{0.557870in}}%
\pgfpathlineto{\pgfqpoint{1.257580in}{0.557870in}}%
\pgfpathlineto{\pgfqpoint{1.257580in}{0.718321in}}%
\pgfpathlineto{\pgfqpoint{1.031770in}{0.718321in}}%
\pgfpathclose%
\pgfusepath{stroke,fill}%
\end{pgfscope}%
\begin{pgfscope}%
\pgfpathrectangle{\pgfqpoint{0.693056in}{0.557870in}}{\pgfqpoint{2.483906in}{1.684734in}}%
\pgfusepath{clip}%
\pgfsetbuttcap%
\pgfsetmiterjoin%
\definecolor{currentfill}{rgb}{0.298039,0.447059,0.690196}%
\pgfsetfillcolor{currentfill}%
\pgfsetfillopacity{0.400000}%
\pgfsetlinewidth{1.003750pt}%
\definecolor{currentstroke}{rgb}{1.000000,1.000000,1.000000}%
\pgfsetstrokecolor{currentstroke}%
\pgfsetstrokeopacity{0.400000}%
\pgfsetdash{}{0pt}%
\pgfpathmoveto{\pgfqpoint{1.257580in}{0.557870in}}%
\pgfpathlineto{\pgfqpoint{1.483390in}{0.557870in}}%
\pgfpathlineto{\pgfqpoint{1.483390in}{0.557870in}}%
\pgfpathlineto{\pgfqpoint{1.257580in}{0.557870in}}%
\pgfpathclose%
\pgfusepath{stroke,fill}%
\end{pgfscope}%
\begin{pgfscope}%
\pgfpathrectangle{\pgfqpoint{0.693056in}{0.557870in}}{\pgfqpoint{2.483906in}{1.684734in}}%
\pgfusepath{clip}%
\pgfsetbuttcap%
\pgfsetmiterjoin%
\definecolor{currentfill}{rgb}{0.298039,0.447059,0.690196}%
\pgfsetfillcolor{currentfill}%
\pgfsetfillopacity{0.400000}%
\pgfsetlinewidth{1.003750pt}%
\definecolor{currentstroke}{rgb}{1.000000,1.000000,1.000000}%
\pgfsetstrokecolor{currentstroke}%
\pgfsetstrokeopacity{0.400000}%
\pgfsetdash{}{0pt}%
\pgfpathmoveto{\pgfqpoint{1.483390in}{0.557870in}}%
\pgfpathlineto{\pgfqpoint{1.709199in}{0.557870in}}%
\pgfpathlineto{\pgfqpoint{1.709199in}{0.718321in}}%
\pgfpathlineto{\pgfqpoint{1.483390in}{0.718321in}}%
\pgfpathclose%
\pgfusepath{stroke,fill}%
\end{pgfscope}%
\begin{pgfscope}%
\pgfpathrectangle{\pgfqpoint{0.693056in}{0.557870in}}{\pgfqpoint{2.483906in}{1.684734in}}%
\pgfusepath{clip}%
\pgfsetbuttcap%
\pgfsetmiterjoin%
\definecolor{currentfill}{rgb}{0.298039,0.447059,0.690196}%
\pgfsetfillcolor{currentfill}%
\pgfsetfillopacity{0.400000}%
\pgfsetlinewidth{1.003750pt}%
\definecolor{currentstroke}{rgb}{1.000000,1.000000,1.000000}%
\pgfsetstrokecolor{currentstroke}%
\pgfsetstrokeopacity{0.400000}%
\pgfsetdash{}{0pt}%
\pgfpathmoveto{\pgfqpoint{1.709199in}{0.557870in}}%
\pgfpathlineto{\pgfqpoint{1.935009in}{0.557870in}}%
\pgfpathlineto{\pgfqpoint{1.935009in}{0.718321in}}%
\pgfpathlineto{\pgfqpoint{1.709199in}{0.718321in}}%
\pgfpathclose%
\pgfusepath{stroke,fill}%
\end{pgfscope}%
\begin{pgfscope}%
\pgfpathrectangle{\pgfqpoint{0.693056in}{0.557870in}}{\pgfqpoint{2.483906in}{1.684734in}}%
\pgfusepath{clip}%
\pgfsetbuttcap%
\pgfsetmiterjoin%
\definecolor{currentfill}{rgb}{0.298039,0.447059,0.690196}%
\pgfsetfillcolor{currentfill}%
\pgfsetfillopacity{0.400000}%
\pgfsetlinewidth{1.003750pt}%
\definecolor{currentstroke}{rgb}{1.000000,1.000000,1.000000}%
\pgfsetstrokecolor{currentstroke}%
\pgfsetstrokeopacity{0.400000}%
\pgfsetdash{}{0pt}%
\pgfpathmoveto{\pgfqpoint{1.935009in}{0.557870in}}%
\pgfpathlineto{\pgfqpoint{2.160818in}{0.557870in}}%
\pgfpathlineto{\pgfqpoint{2.160818in}{0.557870in}}%
\pgfpathlineto{\pgfqpoint{1.935009in}{0.557870in}}%
\pgfpathclose%
\pgfusepath{stroke,fill}%
\end{pgfscope}%
\begin{pgfscope}%
\pgfpathrectangle{\pgfqpoint{0.693056in}{0.557870in}}{\pgfqpoint{2.483906in}{1.684734in}}%
\pgfusepath{clip}%
\pgfsetbuttcap%
\pgfsetmiterjoin%
\definecolor{currentfill}{rgb}{0.298039,0.447059,0.690196}%
\pgfsetfillcolor{currentfill}%
\pgfsetfillopacity{0.400000}%
\pgfsetlinewidth{1.003750pt}%
\definecolor{currentstroke}{rgb}{1.000000,1.000000,1.000000}%
\pgfsetstrokecolor{currentstroke}%
\pgfsetstrokeopacity{0.400000}%
\pgfsetdash{}{0pt}%
\pgfpathmoveto{\pgfqpoint{2.160818in}{0.557870in}}%
\pgfpathlineto{\pgfqpoint{2.386628in}{0.557870in}}%
\pgfpathlineto{\pgfqpoint{2.386628in}{0.557870in}}%
\pgfpathlineto{\pgfqpoint{2.160818in}{0.557870in}}%
\pgfpathclose%
\pgfusepath{stroke,fill}%
\end{pgfscope}%
\begin{pgfscope}%
\pgfpathrectangle{\pgfqpoint{0.693056in}{0.557870in}}{\pgfqpoint{2.483906in}{1.684734in}}%
\pgfusepath{clip}%
\pgfsetbuttcap%
\pgfsetmiterjoin%
\definecolor{currentfill}{rgb}{0.298039,0.447059,0.690196}%
\pgfsetfillcolor{currentfill}%
\pgfsetfillopacity{0.400000}%
\pgfsetlinewidth{1.003750pt}%
\definecolor{currentstroke}{rgb}{1.000000,1.000000,1.000000}%
\pgfsetstrokecolor{currentstroke}%
\pgfsetstrokeopacity{0.400000}%
\pgfsetdash{}{0pt}%
\pgfpathmoveto{\pgfqpoint{2.386628in}{0.557870in}}%
\pgfpathlineto{\pgfqpoint{2.612438in}{0.557870in}}%
\pgfpathlineto{\pgfqpoint{2.612438in}{0.557870in}}%
\pgfpathlineto{\pgfqpoint{2.386628in}{0.557870in}}%
\pgfpathclose%
\pgfusepath{stroke,fill}%
\end{pgfscope}%
\begin{pgfscope}%
\pgfpathrectangle{\pgfqpoint{0.693056in}{0.557870in}}{\pgfqpoint{2.483906in}{1.684734in}}%
\pgfusepath{clip}%
\pgfsetbuttcap%
\pgfsetmiterjoin%
\definecolor{currentfill}{rgb}{0.298039,0.447059,0.690196}%
\pgfsetfillcolor{currentfill}%
\pgfsetfillopacity{0.400000}%
\pgfsetlinewidth{1.003750pt}%
\definecolor{currentstroke}{rgb}{1.000000,1.000000,1.000000}%
\pgfsetstrokecolor{currentstroke}%
\pgfsetstrokeopacity{0.400000}%
\pgfsetdash{}{0pt}%
\pgfpathmoveto{\pgfqpoint{2.612438in}{0.557870in}}%
\pgfpathlineto{\pgfqpoint{2.838247in}{0.557870in}}%
\pgfpathlineto{\pgfqpoint{2.838247in}{1.199674in}}%
\pgfpathlineto{\pgfqpoint{2.612438in}{1.199674in}}%
\pgfpathclose%
\pgfusepath{stroke,fill}%
\end{pgfscope}%
\begin{pgfscope}%
\pgfpathrectangle{\pgfqpoint{0.693056in}{0.557870in}}{\pgfqpoint{2.483906in}{1.684734in}}%
\pgfusepath{clip}%
\pgfsetbuttcap%
\pgfsetmiterjoin%
\definecolor{currentfill}{rgb}{0.298039,0.447059,0.690196}%
\pgfsetfillcolor{currentfill}%
\pgfsetfillopacity{0.400000}%
\pgfsetlinewidth{1.003750pt}%
\definecolor{currentstroke}{rgb}{1.000000,1.000000,1.000000}%
\pgfsetstrokecolor{currentstroke}%
\pgfsetstrokeopacity{0.400000}%
\pgfsetdash{}{0pt}%
\pgfpathmoveto{\pgfqpoint{2.838247in}{0.557870in}}%
\pgfpathlineto{\pgfqpoint{3.064057in}{0.557870in}}%
\pgfpathlineto{\pgfqpoint{3.064057in}{0.878772in}}%
\pgfpathlineto{\pgfqpoint{2.838247in}{0.878772in}}%
\pgfpathclose%
\pgfusepath{stroke,fill}%
\end{pgfscope}%
\begin{pgfscope}%
\pgfsetrectcap%
\pgfsetmiterjoin%
\pgfsetlinewidth{1.254687pt}%
\definecolor{currentstroke}{rgb}{1.000000,1.000000,1.000000}%
\pgfsetstrokecolor{currentstroke}%
\pgfsetdash{}{0pt}%
\pgfpathmoveto{\pgfqpoint{0.693056in}{0.557870in}}%
\pgfpathlineto{\pgfqpoint{0.693056in}{2.242604in}}%
\pgfusepath{stroke}%
\end{pgfscope}%
\begin{pgfscope}%
\pgfsetrectcap%
\pgfsetmiterjoin%
\pgfsetlinewidth{1.254687pt}%
\definecolor{currentstroke}{rgb}{1.000000,1.000000,1.000000}%
\pgfsetstrokecolor{currentstroke}%
\pgfsetdash{}{0pt}%
\pgfpathmoveto{\pgfqpoint{3.176962in}{0.557870in}}%
\pgfpathlineto{\pgfqpoint{3.176962in}{2.242604in}}%
\pgfusepath{stroke}%
\end{pgfscope}%
\begin{pgfscope}%
\pgfsetrectcap%
\pgfsetmiterjoin%
\pgfsetlinewidth{1.254687pt}%
\definecolor{currentstroke}{rgb}{1.000000,1.000000,1.000000}%
\pgfsetstrokecolor{currentstroke}%
\pgfsetdash{}{0pt}%
\pgfpathmoveto{\pgfqpoint{0.693056in}{0.557870in}}%
\pgfpathlineto{\pgfqpoint{3.176962in}{0.557870in}}%
\pgfusepath{stroke}%
\end{pgfscope}%
\begin{pgfscope}%
\pgfsetrectcap%
\pgfsetmiterjoin%
\pgfsetlinewidth{1.254687pt}%
\definecolor{currentstroke}{rgb}{1.000000,1.000000,1.000000}%
\pgfsetstrokecolor{currentstroke}%
\pgfsetdash{}{0pt}%
\pgfpathmoveto{\pgfqpoint{0.693056in}{2.242604in}}%
\pgfpathlineto{\pgfqpoint{3.176962in}{2.242604in}}%
\pgfusepath{stroke}%
\end{pgfscope}%
\begin{pgfscope}%
\definecolor{textcolor}{rgb}{0.150000,0.150000,0.150000}%
\pgfsetstrokecolor{textcolor}%
\pgfsetfillcolor{textcolor}%
\pgftext[x=1.935009in,y=2.325938in,,base]{\color{textcolor}\sffamily\fontsize{11.000000}{13.200000}\selectfont (a)}%
\end{pgfscope}%
\begin{pgfscope}%
\pgfsetbuttcap%
\pgfsetmiterjoin%
\definecolor{currentfill}{rgb}{0.917647,0.917647,0.949020}%
\pgfsetfillcolor{currentfill}%
\pgfsetlinewidth{0.000000pt}%
\definecolor{currentstroke}{rgb}{0.000000,0.000000,0.000000}%
\pgfsetstrokecolor{currentstroke}%
\pgfsetstrokeopacity{0.000000}%
\pgfsetdash{}{0pt}%
\pgfpathmoveto{\pgfqpoint{3.874381in}{0.557870in}}%
\pgfpathlineto{\pgfqpoint{6.358287in}{0.557870in}}%
\pgfpathlineto{\pgfqpoint{6.358287in}{2.242604in}}%
\pgfpathlineto{\pgfqpoint{3.874381in}{2.242604in}}%
\pgfpathclose%
\pgfusepath{fill}%
\end{pgfscope}%
\begin{pgfscope}%
\pgfpathrectangle{\pgfqpoint{3.874381in}{0.557870in}}{\pgfqpoint{2.483906in}{1.684734in}}%
\pgfusepath{clip}%
\pgfsetroundcap%
\pgfsetroundjoin%
\pgfsetlinewidth{1.003750pt}%
\definecolor{currentstroke}{rgb}{1.000000,1.000000,1.000000}%
\pgfsetstrokecolor{currentstroke}%
\pgfsetdash{}{0pt}%
\pgfpathmoveto{\pgfqpoint{3.987286in}{0.557870in}}%
\pgfpathlineto{\pgfqpoint{3.987286in}{2.242604in}}%
\pgfusepath{stroke}%
\end{pgfscope}%
\begin{pgfscope}%
\definecolor{textcolor}{rgb}{0.150000,0.150000,0.150000}%
\pgfsetstrokecolor{textcolor}%
\pgfsetfillcolor{textcolor}%
\pgftext[x=3.987286in,y=0.425926in,,top]{\color{textcolor}\sffamily\fontsize{11.000000}{13.200000}\selectfont \(\displaystyle 0.00\)}%
\end{pgfscope}%
\begin{pgfscope}%
\pgfpathrectangle{\pgfqpoint{3.874381in}{0.557870in}}{\pgfqpoint{2.483906in}{1.684734in}}%
\pgfusepath{clip}%
\pgfsetroundcap%
\pgfsetroundjoin%
\pgfsetlinewidth{1.003750pt}%
\definecolor{currentstroke}{rgb}{1.000000,1.000000,1.000000}%
\pgfsetstrokecolor{currentstroke}%
\pgfsetdash{}{0pt}%
\pgfpathmoveto{\pgfqpoint{4.551810in}{0.557870in}}%
\pgfpathlineto{\pgfqpoint{4.551810in}{2.242604in}}%
\pgfusepath{stroke}%
\end{pgfscope}%
\begin{pgfscope}%
\definecolor{textcolor}{rgb}{0.150000,0.150000,0.150000}%
\pgfsetstrokecolor{textcolor}%
\pgfsetfillcolor{textcolor}%
\pgftext[x=4.551810in,y=0.425926in,,top]{\color{textcolor}\sffamily\fontsize{11.000000}{13.200000}\selectfont \(\displaystyle 0.25\)}%
\end{pgfscope}%
\begin{pgfscope}%
\pgfpathrectangle{\pgfqpoint{3.874381in}{0.557870in}}{\pgfqpoint{2.483906in}{1.684734in}}%
\pgfusepath{clip}%
\pgfsetroundcap%
\pgfsetroundjoin%
\pgfsetlinewidth{1.003750pt}%
\definecolor{currentstroke}{rgb}{1.000000,1.000000,1.000000}%
\pgfsetstrokecolor{currentstroke}%
\pgfsetdash{}{0pt}%
\pgfpathmoveto{\pgfqpoint{5.116334in}{0.557870in}}%
\pgfpathlineto{\pgfqpoint{5.116334in}{2.242604in}}%
\pgfusepath{stroke}%
\end{pgfscope}%
\begin{pgfscope}%
\definecolor{textcolor}{rgb}{0.150000,0.150000,0.150000}%
\pgfsetstrokecolor{textcolor}%
\pgfsetfillcolor{textcolor}%
\pgftext[x=5.116334in,y=0.425926in,,top]{\color{textcolor}\sffamily\fontsize{11.000000}{13.200000}\selectfont \(\displaystyle 0.50\)}%
\end{pgfscope}%
\begin{pgfscope}%
\pgfpathrectangle{\pgfqpoint{3.874381in}{0.557870in}}{\pgfqpoint{2.483906in}{1.684734in}}%
\pgfusepath{clip}%
\pgfsetroundcap%
\pgfsetroundjoin%
\pgfsetlinewidth{1.003750pt}%
\definecolor{currentstroke}{rgb}{1.000000,1.000000,1.000000}%
\pgfsetstrokecolor{currentstroke}%
\pgfsetdash{}{0pt}%
\pgfpathmoveto{\pgfqpoint{5.680858in}{0.557870in}}%
\pgfpathlineto{\pgfqpoint{5.680858in}{2.242604in}}%
\pgfusepath{stroke}%
\end{pgfscope}%
\begin{pgfscope}%
\definecolor{textcolor}{rgb}{0.150000,0.150000,0.150000}%
\pgfsetstrokecolor{textcolor}%
\pgfsetfillcolor{textcolor}%
\pgftext[x=5.680858in,y=0.425926in,,top]{\color{textcolor}\sffamily\fontsize{11.000000}{13.200000}\selectfont \(\displaystyle 0.75\)}%
\end{pgfscope}%
\begin{pgfscope}%
\pgfpathrectangle{\pgfqpoint{3.874381in}{0.557870in}}{\pgfqpoint{2.483906in}{1.684734in}}%
\pgfusepath{clip}%
\pgfsetroundcap%
\pgfsetroundjoin%
\pgfsetlinewidth{1.003750pt}%
\definecolor{currentstroke}{rgb}{1.000000,1.000000,1.000000}%
\pgfsetstrokecolor{currentstroke}%
\pgfsetdash{}{0pt}%
\pgfpathmoveto{\pgfqpoint{6.245382in}{0.557870in}}%
\pgfpathlineto{\pgfqpoint{6.245382in}{2.242604in}}%
\pgfusepath{stroke}%
\end{pgfscope}%
\begin{pgfscope}%
\definecolor{textcolor}{rgb}{0.150000,0.150000,0.150000}%
\pgfsetstrokecolor{textcolor}%
\pgfsetfillcolor{textcolor}%
\pgftext[x=6.245382in,y=0.425926in,,top]{\color{textcolor}\sffamily\fontsize{11.000000}{13.200000}\selectfont \(\displaystyle 1.00\)}%
\end{pgfscope}%
\begin{pgfscope}%
\definecolor{textcolor}{rgb}{0.150000,0.150000,0.150000}%
\pgfsetstrokecolor{textcolor}%
\pgfsetfillcolor{textcolor}%
\pgftext[x=5.116334in,y=0.235185in,,top]{\color{textcolor}\sffamily\fontsize{11.000000}{13.200000}\selectfont Specificity}%
\end{pgfscope}%
\begin{pgfscope}%
\pgfpathrectangle{\pgfqpoint{3.874381in}{0.557870in}}{\pgfqpoint{2.483906in}{1.684734in}}%
\pgfusepath{clip}%
\pgfsetroundcap%
\pgfsetroundjoin%
\pgfsetlinewidth{1.003750pt}%
\definecolor{currentstroke}{rgb}{1.000000,1.000000,1.000000}%
\pgfsetstrokecolor{currentstroke}%
\pgfsetdash{}{0pt}%
\pgfpathmoveto{\pgfqpoint{3.874381in}{0.634449in}}%
\pgfpathlineto{\pgfqpoint{6.358287in}{0.634449in}}%
\pgfusepath{stroke}%
\end{pgfscope}%
\begin{pgfscope}%
\definecolor{textcolor}{rgb}{0.150000,0.150000,0.150000}%
\pgfsetstrokecolor{textcolor}%
\pgfsetfillcolor{textcolor}%
\pgftext[x=3.472066in,y=0.581642in,left,base]{\color{textcolor}\sffamily\fontsize{11.000000}{13.200000}\selectfont \(\displaystyle 0.00\)}%
\end{pgfscope}%
\begin{pgfscope}%
\pgfpathrectangle{\pgfqpoint{3.874381in}{0.557870in}}{\pgfqpoint{2.483906in}{1.684734in}}%
\pgfusepath{clip}%
\pgfsetroundcap%
\pgfsetroundjoin%
\pgfsetlinewidth{1.003750pt}%
\definecolor{currentstroke}{rgb}{1.000000,1.000000,1.000000}%
\pgfsetstrokecolor{currentstroke}%
\pgfsetdash{}{0pt}%
\pgfpathmoveto{\pgfqpoint{3.874381in}{1.017343in}}%
\pgfpathlineto{\pgfqpoint{6.358287in}{1.017343in}}%
\pgfusepath{stroke}%
\end{pgfscope}%
\begin{pgfscope}%
\definecolor{textcolor}{rgb}{0.150000,0.150000,0.150000}%
\pgfsetstrokecolor{textcolor}%
\pgfsetfillcolor{textcolor}%
\pgftext[x=3.472066in,y=0.964536in,left,base]{\color{textcolor}\sffamily\fontsize{11.000000}{13.200000}\selectfont \(\displaystyle 0.25\)}%
\end{pgfscope}%
\begin{pgfscope}%
\pgfpathrectangle{\pgfqpoint{3.874381in}{0.557870in}}{\pgfqpoint{2.483906in}{1.684734in}}%
\pgfusepath{clip}%
\pgfsetroundcap%
\pgfsetroundjoin%
\pgfsetlinewidth{1.003750pt}%
\definecolor{currentstroke}{rgb}{1.000000,1.000000,1.000000}%
\pgfsetstrokecolor{currentstroke}%
\pgfsetdash{}{0pt}%
\pgfpathmoveto{\pgfqpoint{3.874381in}{1.400237in}}%
\pgfpathlineto{\pgfqpoint{6.358287in}{1.400237in}}%
\pgfusepath{stroke}%
\end{pgfscope}%
\begin{pgfscope}%
\definecolor{textcolor}{rgb}{0.150000,0.150000,0.150000}%
\pgfsetstrokecolor{textcolor}%
\pgfsetfillcolor{textcolor}%
\pgftext[x=3.472066in,y=1.347431in,left,base]{\color{textcolor}\sffamily\fontsize{11.000000}{13.200000}\selectfont \(\displaystyle 0.50\)}%
\end{pgfscope}%
\begin{pgfscope}%
\pgfpathrectangle{\pgfqpoint{3.874381in}{0.557870in}}{\pgfqpoint{2.483906in}{1.684734in}}%
\pgfusepath{clip}%
\pgfsetroundcap%
\pgfsetroundjoin%
\pgfsetlinewidth{1.003750pt}%
\definecolor{currentstroke}{rgb}{1.000000,1.000000,1.000000}%
\pgfsetstrokecolor{currentstroke}%
\pgfsetdash{}{0pt}%
\pgfpathmoveto{\pgfqpoint{3.874381in}{1.783131in}}%
\pgfpathlineto{\pgfqpoint{6.358287in}{1.783131in}}%
\pgfusepath{stroke}%
\end{pgfscope}%
\begin{pgfscope}%
\definecolor{textcolor}{rgb}{0.150000,0.150000,0.150000}%
\pgfsetstrokecolor{textcolor}%
\pgfsetfillcolor{textcolor}%
\pgftext[x=3.472066in,y=1.730325in,left,base]{\color{textcolor}\sffamily\fontsize{11.000000}{13.200000}\selectfont \(\displaystyle 0.75\)}%
\end{pgfscope}%
\begin{pgfscope}%
\pgfpathrectangle{\pgfqpoint{3.874381in}{0.557870in}}{\pgfqpoint{2.483906in}{1.684734in}}%
\pgfusepath{clip}%
\pgfsetroundcap%
\pgfsetroundjoin%
\pgfsetlinewidth{1.003750pt}%
\definecolor{currentstroke}{rgb}{1.000000,1.000000,1.000000}%
\pgfsetstrokecolor{currentstroke}%
\pgfsetdash{}{0pt}%
\pgfpathmoveto{\pgfqpoint{3.874381in}{2.166025in}}%
\pgfpathlineto{\pgfqpoint{6.358287in}{2.166025in}}%
\pgfusepath{stroke}%
\end{pgfscope}%
\begin{pgfscope}%
\definecolor{textcolor}{rgb}{0.150000,0.150000,0.150000}%
\pgfsetstrokecolor{textcolor}%
\pgfsetfillcolor{textcolor}%
\pgftext[x=3.472066in,y=2.113219in,left,base]{\color{textcolor}\sffamily\fontsize{11.000000}{13.200000}\selectfont \(\displaystyle 1.00\)}%
\end{pgfscope}%
\begin{pgfscope}%
\definecolor{textcolor}{rgb}{0.150000,0.150000,0.150000}%
\pgfsetstrokecolor{textcolor}%
\pgfsetfillcolor{textcolor}%
\pgftext[x=3.416511in,y=1.400237in,,bottom,rotate=90.000000]{\color{textcolor}\sffamily\fontsize{11.000000}{13.200000}\selectfont Sensitivity}%
\end{pgfscope}%
\begin{pgfscope}%
\pgfpathrectangle{\pgfqpoint{3.874381in}{0.557870in}}{\pgfqpoint{2.483906in}{1.684734in}}%
\pgfusepath{clip}%
\pgfsetbuttcap%
\pgfsetroundjoin%
\definecolor{currentfill}{rgb}{0.298039,0.447059,0.690196}%
\pgfsetfillcolor{currentfill}%
\pgfsetlinewidth{1.003750pt}%
\definecolor{currentstroke}{rgb}{0.298039,0.447059,0.690196}%
\pgfsetstrokecolor{currentstroke}%
\pgfsetdash{}{0pt}%
\pgfpathmoveto{\pgfqpoint{6.132145in}{1.275137in}}%
\pgfpathcurveto{\pgfqpoint{6.140381in}{1.275137in}}{\pgfqpoint{6.148281in}{1.278409in}}{\pgfqpoint{6.154105in}{1.284233in}}%
\pgfpathcurveto{\pgfqpoint{6.159929in}{1.290057in}}{\pgfqpoint{6.163201in}{1.297957in}}{\pgfqpoint{6.163201in}{1.306193in}}%
\pgfpathcurveto{\pgfqpoint{6.163201in}{1.314429in}}{\pgfqpoint{6.159929in}{1.322329in}}{\pgfqpoint{6.154105in}{1.328153in}}%
\pgfpathcurveto{\pgfqpoint{6.148281in}{1.333977in}}{\pgfqpoint{6.140381in}{1.337250in}}{\pgfqpoint{6.132145in}{1.337250in}}%
\pgfpathcurveto{\pgfqpoint{6.123908in}{1.337250in}}{\pgfqpoint{6.116008in}{1.333977in}}{\pgfqpoint{6.110184in}{1.328153in}}%
\pgfpathcurveto{\pgfqpoint{6.104360in}{1.322329in}}{\pgfqpoint{6.101088in}{1.314429in}}{\pgfqpoint{6.101088in}{1.306193in}}%
\pgfpathcurveto{\pgfqpoint{6.101088in}{1.297957in}}{\pgfqpoint{6.104360in}{1.290057in}}{\pgfqpoint{6.110184in}{1.284233in}}%
\pgfpathcurveto{\pgfqpoint{6.116008in}{1.278409in}}{\pgfqpoint{6.123908in}{1.275137in}}{\pgfqpoint{6.132145in}{1.275137in}}%
\pgfpathclose%
\pgfusepath{stroke,fill}%
\end{pgfscope}%
\begin{pgfscope}%
\pgfpathrectangle{\pgfqpoint{3.874381in}{0.557870in}}{\pgfqpoint{2.483906in}{1.684734in}}%
\pgfusepath{clip}%
\pgfsetbuttcap%
\pgfsetroundjoin%
\definecolor{currentfill}{rgb}{0.298039,0.447059,0.690196}%
\pgfsetfillcolor{currentfill}%
\pgfsetlinewidth{1.003750pt}%
\definecolor{currentstroke}{rgb}{0.298039,0.447059,0.690196}%
\pgfsetstrokecolor{currentstroke}%
\pgfsetdash{}{0pt}%
\pgfpathmoveto{\pgfqpoint{6.132145in}{1.275137in}}%
\pgfpathcurveto{\pgfqpoint{6.140381in}{1.275137in}}{\pgfqpoint{6.148281in}{1.278409in}}{\pgfqpoint{6.154105in}{1.284233in}}%
\pgfpathcurveto{\pgfqpoint{6.159929in}{1.290057in}}{\pgfqpoint{6.163201in}{1.297957in}}{\pgfqpoint{6.163201in}{1.306193in}}%
\pgfpathcurveto{\pgfqpoint{6.163201in}{1.314429in}}{\pgfqpoint{6.159929in}{1.322329in}}{\pgfqpoint{6.154105in}{1.328153in}}%
\pgfpathcurveto{\pgfqpoint{6.148281in}{1.333977in}}{\pgfqpoint{6.140381in}{1.337250in}}{\pgfqpoint{6.132145in}{1.337250in}}%
\pgfpathcurveto{\pgfqpoint{6.123908in}{1.337250in}}{\pgfqpoint{6.116008in}{1.333977in}}{\pgfqpoint{6.110184in}{1.328153in}}%
\pgfpathcurveto{\pgfqpoint{6.104360in}{1.322329in}}{\pgfqpoint{6.101088in}{1.314429in}}{\pgfqpoint{6.101088in}{1.306193in}}%
\pgfpathcurveto{\pgfqpoint{6.101088in}{1.297957in}}{\pgfqpoint{6.104360in}{1.290057in}}{\pgfqpoint{6.110184in}{1.284233in}}%
\pgfpathcurveto{\pgfqpoint{6.116008in}{1.278409in}}{\pgfqpoint{6.123908in}{1.275137in}}{\pgfqpoint{6.132145in}{1.275137in}}%
\pgfpathclose%
\pgfusepath{stroke,fill}%
\end{pgfscope}%
\begin{pgfscope}%
\pgfpathrectangle{\pgfqpoint{3.874381in}{0.557870in}}{\pgfqpoint{2.483906in}{1.684734in}}%
\pgfusepath{clip}%
\pgfsetbuttcap%
\pgfsetroundjoin%
\definecolor{currentfill}{rgb}{0.298039,0.447059,0.690196}%
\pgfsetfillcolor{currentfill}%
\pgfsetlinewidth{1.003750pt}%
\definecolor{currentstroke}{rgb}{0.298039,0.447059,0.690196}%
\pgfsetstrokecolor{currentstroke}%
\pgfsetdash{}{0pt}%
\pgfpathmoveto{\pgfqpoint{5.973612in}{1.587803in}}%
\pgfpathcurveto{\pgfqpoint{5.981848in}{1.587803in}}{\pgfqpoint{5.989748in}{1.591075in}}{\pgfqpoint{5.995572in}{1.596899in}}%
\pgfpathcurveto{\pgfqpoint{6.001396in}{1.602723in}}{\pgfqpoint{6.004668in}{1.610623in}}{\pgfqpoint{6.004668in}{1.618859in}}%
\pgfpathcurveto{\pgfqpoint{6.004668in}{1.627096in}}{\pgfqpoint{6.001396in}{1.634996in}}{\pgfqpoint{5.995572in}{1.640820in}}%
\pgfpathcurveto{\pgfqpoint{5.989748in}{1.646644in}}{\pgfqpoint{5.981848in}{1.649916in}}{\pgfqpoint{5.973612in}{1.649916in}}%
\pgfpathcurveto{\pgfqpoint{5.965375in}{1.649916in}}{\pgfqpoint{5.957475in}{1.646644in}}{\pgfqpoint{5.951651in}{1.640820in}}%
\pgfpathcurveto{\pgfqpoint{5.945827in}{1.634996in}}{\pgfqpoint{5.942555in}{1.627096in}}{\pgfqpoint{5.942555in}{1.618859in}}%
\pgfpathcurveto{\pgfqpoint{5.942555in}{1.610623in}}{\pgfqpoint{5.945827in}{1.602723in}}{\pgfqpoint{5.951651in}{1.596899in}}%
\pgfpathcurveto{\pgfqpoint{5.957475in}{1.591075in}}{\pgfqpoint{5.965375in}{1.587803in}}{\pgfqpoint{5.973612in}{1.587803in}}%
\pgfpathclose%
\pgfusepath{stroke,fill}%
\end{pgfscope}%
\begin{pgfscope}%
\pgfpathrectangle{\pgfqpoint{3.874381in}{0.557870in}}{\pgfqpoint{2.483906in}{1.684734in}}%
\pgfusepath{clip}%
\pgfsetbuttcap%
\pgfsetroundjoin%
\definecolor{currentfill}{rgb}{0.298039,0.447059,0.690196}%
\pgfsetfillcolor{currentfill}%
\pgfsetlinewidth{1.003750pt}%
\definecolor{currentstroke}{rgb}{0.298039,0.447059,0.690196}%
\pgfsetstrokecolor{currentstroke}%
\pgfsetdash{}{0pt}%
\pgfpathmoveto{\pgfqpoint{5.973612in}{1.587803in}}%
\pgfpathcurveto{\pgfqpoint{5.981848in}{1.587803in}}{\pgfqpoint{5.989748in}{1.591075in}}{\pgfqpoint{5.995572in}{1.596899in}}%
\pgfpathcurveto{\pgfqpoint{6.001396in}{1.602723in}}{\pgfqpoint{6.004668in}{1.610623in}}{\pgfqpoint{6.004668in}{1.618859in}}%
\pgfpathcurveto{\pgfqpoint{6.004668in}{1.627096in}}{\pgfqpoint{6.001396in}{1.634996in}}{\pgfqpoint{5.995572in}{1.640820in}}%
\pgfpathcurveto{\pgfqpoint{5.989748in}{1.646644in}}{\pgfqpoint{5.981848in}{1.649916in}}{\pgfqpoint{5.973612in}{1.649916in}}%
\pgfpathcurveto{\pgfqpoint{5.965375in}{1.649916in}}{\pgfqpoint{5.957475in}{1.646644in}}{\pgfqpoint{5.951651in}{1.640820in}}%
\pgfpathcurveto{\pgfqpoint{5.945827in}{1.634996in}}{\pgfqpoint{5.942555in}{1.627096in}}{\pgfqpoint{5.942555in}{1.618859in}}%
\pgfpathcurveto{\pgfqpoint{5.942555in}{1.610623in}}{\pgfqpoint{5.945827in}{1.602723in}}{\pgfqpoint{5.951651in}{1.596899in}}%
\pgfpathcurveto{\pgfqpoint{5.957475in}{1.591075in}}{\pgfqpoint{5.965375in}{1.587803in}}{\pgfqpoint{5.973612in}{1.587803in}}%
\pgfpathclose%
\pgfusepath{stroke,fill}%
\end{pgfscope}%
\begin{pgfscope}%
\pgfpathrectangle{\pgfqpoint{3.874381in}{0.557870in}}{\pgfqpoint{2.483906in}{1.684734in}}%
\pgfusepath{clip}%
\pgfsetbuttcap%
\pgfsetroundjoin%
\definecolor{currentfill}{rgb}{0.298039,0.447059,0.690196}%
\pgfsetfillcolor{currentfill}%
\pgfsetlinewidth{1.003750pt}%
\definecolor{currentstroke}{rgb}{0.298039,0.447059,0.690196}%
\pgfsetstrokecolor{currentstroke}%
\pgfsetdash{}{0pt}%
\pgfpathmoveto{\pgfqpoint{6.005585in}{1.529992in}}%
\pgfpathcurveto{\pgfqpoint{6.013821in}{1.529992in}}{\pgfqpoint{6.021721in}{1.533264in}}{\pgfqpoint{6.027545in}{1.539088in}}%
\pgfpathcurveto{\pgfqpoint{6.033369in}{1.544912in}}{\pgfqpoint{6.036641in}{1.552812in}}{\pgfqpoint{6.036641in}{1.561049in}}%
\pgfpathcurveto{\pgfqpoint{6.036641in}{1.569285in}}{\pgfqpoint{6.033369in}{1.577185in}}{\pgfqpoint{6.027545in}{1.583009in}}%
\pgfpathcurveto{\pgfqpoint{6.021721in}{1.588833in}}{\pgfqpoint{6.013821in}{1.592105in}}{\pgfqpoint{6.005585in}{1.592105in}}%
\pgfpathcurveto{\pgfqpoint{5.997348in}{1.592105in}}{\pgfqpoint{5.989448in}{1.588833in}}{\pgfqpoint{5.983624in}{1.583009in}}%
\pgfpathcurveto{\pgfqpoint{5.977800in}{1.577185in}}{\pgfqpoint{5.974528in}{1.569285in}}{\pgfqpoint{5.974528in}{1.561049in}}%
\pgfpathcurveto{\pgfqpoint{5.974528in}{1.552812in}}{\pgfqpoint{5.977800in}{1.544912in}}{\pgfqpoint{5.983624in}{1.539088in}}%
\pgfpathcurveto{\pgfqpoint{5.989448in}{1.533264in}}{\pgfqpoint{5.997348in}{1.529992in}}{\pgfqpoint{6.005585in}{1.529992in}}%
\pgfpathclose%
\pgfusepath{stroke,fill}%
\end{pgfscope}%
\begin{pgfscope}%
\pgfpathrectangle{\pgfqpoint{3.874381in}{0.557870in}}{\pgfqpoint{2.483906in}{1.684734in}}%
\pgfusepath{clip}%
\pgfsetbuttcap%
\pgfsetroundjoin%
\definecolor{currentfill}{rgb}{0.298039,0.447059,0.690196}%
\pgfsetfillcolor{currentfill}%
\pgfsetlinewidth{1.003750pt}%
\definecolor{currentstroke}{rgb}{0.298039,0.447059,0.690196}%
\pgfsetstrokecolor{currentstroke}%
\pgfsetdash{}{0pt}%
\pgfpathmoveto{\pgfqpoint{6.005585in}{1.529992in}}%
\pgfpathcurveto{\pgfqpoint{6.013821in}{1.529992in}}{\pgfqpoint{6.021721in}{1.533264in}}{\pgfqpoint{6.027545in}{1.539088in}}%
\pgfpathcurveto{\pgfqpoint{6.033369in}{1.544912in}}{\pgfqpoint{6.036641in}{1.552812in}}{\pgfqpoint{6.036641in}{1.561049in}}%
\pgfpathcurveto{\pgfqpoint{6.036641in}{1.569285in}}{\pgfqpoint{6.033369in}{1.577185in}}{\pgfqpoint{6.027545in}{1.583009in}}%
\pgfpathcurveto{\pgfqpoint{6.021721in}{1.588833in}}{\pgfqpoint{6.013821in}{1.592105in}}{\pgfqpoint{6.005585in}{1.592105in}}%
\pgfpathcurveto{\pgfqpoint{5.997348in}{1.592105in}}{\pgfqpoint{5.989448in}{1.588833in}}{\pgfqpoint{5.983624in}{1.583009in}}%
\pgfpathcurveto{\pgfqpoint{5.977800in}{1.577185in}}{\pgfqpoint{5.974528in}{1.569285in}}{\pgfqpoint{5.974528in}{1.561049in}}%
\pgfpathcurveto{\pgfqpoint{5.974528in}{1.552812in}}{\pgfqpoint{5.977800in}{1.544912in}}{\pgfqpoint{5.983624in}{1.539088in}}%
\pgfpathcurveto{\pgfqpoint{5.989448in}{1.533264in}}{\pgfqpoint{5.997348in}{1.529992in}}{\pgfqpoint{6.005585in}{1.529992in}}%
\pgfpathclose%
\pgfusepath{stroke,fill}%
\end{pgfscope}%
\begin{pgfscope}%
\pgfpathrectangle{\pgfqpoint{3.874381in}{0.557870in}}{\pgfqpoint{2.483906in}{1.684734in}}%
\pgfusepath{clip}%
\pgfsetbuttcap%
\pgfsetroundjoin%
\definecolor{currentfill}{rgb}{0.298039,0.447059,0.690196}%
\pgfsetfillcolor{currentfill}%
\pgfsetlinewidth{1.003750pt}%
\definecolor{currentstroke}{rgb}{0.298039,0.447059,0.690196}%
\pgfsetstrokecolor{currentstroke}%
\pgfsetdash{}{0pt}%
\pgfpathmoveto{\pgfqpoint{6.124151in}{0.988526in}}%
\pgfpathcurveto{\pgfqpoint{6.132388in}{0.988526in}}{\pgfqpoint{6.140288in}{0.991798in}}{\pgfqpoint{6.146112in}{0.997622in}}%
\pgfpathcurveto{\pgfqpoint{6.151935in}{1.003446in}}{\pgfqpoint{6.155208in}{1.011346in}}{\pgfqpoint{6.155208in}{1.019582in}}%
\pgfpathcurveto{\pgfqpoint{6.155208in}{1.027819in}}{\pgfqpoint{6.151935in}{1.035719in}}{\pgfqpoint{6.146112in}{1.041543in}}%
\pgfpathcurveto{\pgfqpoint{6.140288in}{1.047366in}}{\pgfqpoint{6.132388in}{1.050639in}}{\pgfqpoint{6.124151in}{1.050639in}}%
\pgfpathcurveto{\pgfqpoint{6.115915in}{1.050639in}}{\pgfqpoint{6.108015in}{1.047366in}}{\pgfqpoint{6.102191in}{1.041543in}}%
\pgfpathcurveto{\pgfqpoint{6.096367in}{1.035719in}}{\pgfqpoint{6.093095in}{1.027819in}}{\pgfqpoint{6.093095in}{1.019582in}}%
\pgfpathcurveto{\pgfqpoint{6.093095in}{1.011346in}}{\pgfqpoint{6.096367in}{1.003446in}}{\pgfqpoint{6.102191in}{0.997622in}}%
\pgfpathcurveto{\pgfqpoint{6.108015in}{0.991798in}}{\pgfqpoint{6.115915in}{0.988526in}}{\pgfqpoint{6.124151in}{0.988526in}}%
\pgfpathclose%
\pgfusepath{stroke,fill}%
\end{pgfscope}%
\begin{pgfscope}%
\pgfpathrectangle{\pgfqpoint{3.874381in}{0.557870in}}{\pgfqpoint{2.483906in}{1.684734in}}%
\pgfusepath{clip}%
\pgfsetbuttcap%
\pgfsetroundjoin%
\definecolor{currentfill}{rgb}{0.298039,0.447059,0.690196}%
\pgfsetfillcolor{currentfill}%
\pgfsetlinewidth{1.003750pt}%
\definecolor{currentstroke}{rgb}{0.298039,0.447059,0.690196}%
\pgfsetstrokecolor{currentstroke}%
\pgfsetdash{}{0pt}%
\pgfpathmoveto{\pgfqpoint{6.056209in}{1.082163in}}%
\pgfpathcurveto{\pgfqpoint{6.064445in}{1.082163in}}{\pgfqpoint{6.072345in}{1.085435in}}{\pgfqpoint{6.078169in}{1.091259in}}%
\pgfpathcurveto{\pgfqpoint{6.083993in}{1.097083in}}{\pgfqpoint{6.087265in}{1.104983in}}{\pgfqpoint{6.087265in}{1.113219in}}%
\pgfpathcurveto{\pgfqpoint{6.087265in}{1.121456in}}{\pgfqpoint{6.083993in}{1.129356in}}{\pgfqpoint{6.078169in}{1.135180in}}%
\pgfpathcurveto{\pgfqpoint{6.072345in}{1.141004in}}{\pgfqpoint{6.064445in}{1.144276in}}{\pgfqpoint{6.056209in}{1.144276in}}%
\pgfpathcurveto{\pgfqpoint{6.047972in}{1.144276in}}{\pgfqpoint{6.040072in}{1.141004in}}{\pgfqpoint{6.034248in}{1.135180in}}%
\pgfpathcurveto{\pgfqpoint{6.028424in}{1.129356in}}{\pgfqpoint{6.025152in}{1.121456in}}{\pgfqpoint{6.025152in}{1.113219in}}%
\pgfpathcurveto{\pgfqpoint{6.025152in}{1.104983in}}{\pgfqpoint{6.028424in}{1.097083in}}{\pgfqpoint{6.034248in}{1.091259in}}%
\pgfpathcurveto{\pgfqpoint{6.040072in}{1.085435in}}{\pgfqpoint{6.047972in}{1.082163in}}{\pgfqpoint{6.056209in}{1.082163in}}%
\pgfpathclose%
\pgfusepath{stroke,fill}%
\end{pgfscope}%
\begin{pgfscope}%
\pgfpathrectangle{\pgfqpoint{3.874381in}{0.557870in}}{\pgfqpoint{2.483906in}{1.684734in}}%
\pgfusepath{clip}%
\pgfsetbuttcap%
\pgfsetroundjoin%
\definecolor{currentfill}{rgb}{0.298039,0.447059,0.690196}%
\pgfsetfillcolor{currentfill}%
\pgfsetlinewidth{1.003750pt}%
\definecolor{currentstroke}{rgb}{0.298039,0.447059,0.690196}%
\pgfsetstrokecolor{currentstroke}%
\pgfsetdash{}{0pt}%
\pgfpathmoveto{\pgfqpoint{5.384775in}{1.353303in}}%
\pgfpathcurveto{\pgfqpoint{5.393011in}{1.353303in}}{\pgfqpoint{5.400911in}{1.356575in}}{\pgfqpoint{5.406735in}{1.362399in}}%
\pgfpathcurveto{\pgfqpoint{5.412559in}{1.368223in}}{\pgfqpoint{5.415831in}{1.376123in}}{\pgfqpoint{5.415831in}{1.384360in}}%
\pgfpathcurveto{\pgfqpoint{5.415831in}{1.392596in}}{\pgfqpoint{5.412559in}{1.400496in}}{\pgfqpoint{5.406735in}{1.406320in}}%
\pgfpathcurveto{\pgfqpoint{5.400911in}{1.412144in}}{\pgfqpoint{5.393011in}{1.415416in}}{\pgfqpoint{5.384775in}{1.415416in}}%
\pgfpathcurveto{\pgfqpoint{5.376538in}{1.415416in}}{\pgfqpoint{5.368638in}{1.412144in}}{\pgfqpoint{5.362814in}{1.406320in}}%
\pgfpathcurveto{\pgfqpoint{5.356990in}{1.400496in}}{\pgfqpoint{5.353718in}{1.392596in}}{\pgfqpoint{5.353718in}{1.384360in}}%
\pgfpathcurveto{\pgfqpoint{5.353718in}{1.376123in}}{\pgfqpoint{5.356990in}{1.368223in}}{\pgfqpoint{5.362814in}{1.362399in}}%
\pgfpathcurveto{\pgfqpoint{5.368638in}{1.356575in}}{\pgfqpoint{5.376538in}{1.353303in}}{\pgfqpoint{5.384775in}{1.353303in}}%
\pgfpathclose%
\pgfusepath{stroke,fill}%
\end{pgfscope}%
\begin{pgfscope}%
\pgfpathrectangle{\pgfqpoint{3.874381in}{0.557870in}}{\pgfqpoint{2.483906in}{1.684734in}}%
\pgfusepath{clip}%
\pgfsetbuttcap%
\pgfsetroundjoin%
\definecolor{currentfill}{rgb}{0.298039,0.447059,0.690196}%
\pgfsetfillcolor{currentfill}%
\pgfsetlinewidth{1.003750pt}%
\definecolor{currentstroke}{rgb}{0.298039,0.447059,0.690196}%
\pgfsetstrokecolor{currentstroke}%
\pgfsetdash{}{0pt}%
\pgfpathmoveto{\pgfqpoint{5.113004in}{1.514522in}}%
\pgfpathcurveto{\pgfqpoint{5.121240in}{1.514522in}}{\pgfqpoint{5.129140in}{1.517794in}}{\pgfqpoint{5.134964in}{1.523618in}}%
\pgfpathcurveto{\pgfqpoint{5.140788in}{1.529442in}}{\pgfqpoint{5.144060in}{1.537342in}}{\pgfqpoint{5.144060in}{1.545578in}}%
\pgfpathcurveto{\pgfqpoint{5.144060in}{1.553814in}}{\pgfqpoint{5.140788in}{1.561715in}}{\pgfqpoint{5.134964in}{1.567538in}}%
\pgfpathcurveto{\pgfqpoint{5.129140in}{1.573362in}}{\pgfqpoint{5.121240in}{1.576635in}}{\pgfqpoint{5.113004in}{1.576635in}}%
\pgfpathcurveto{\pgfqpoint{5.104767in}{1.576635in}}{\pgfqpoint{5.096867in}{1.573362in}}{\pgfqpoint{5.091043in}{1.567538in}}%
\pgfpathcurveto{\pgfqpoint{5.085220in}{1.561715in}}{\pgfqpoint{5.081947in}{1.553814in}}{\pgfqpoint{5.081947in}{1.545578in}}%
\pgfpathcurveto{\pgfqpoint{5.081947in}{1.537342in}}{\pgfqpoint{5.085220in}{1.529442in}}{\pgfqpoint{5.091043in}{1.523618in}}%
\pgfpathcurveto{\pgfqpoint{5.096867in}{1.517794in}}{\pgfqpoint{5.104767in}{1.514522in}}{\pgfqpoint{5.113004in}{1.514522in}}%
\pgfpathclose%
\pgfusepath{stroke,fill}%
\end{pgfscope}%
\begin{pgfscope}%
\pgfpathrectangle{\pgfqpoint{3.874381in}{0.557870in}}{\pgfqpoint{2.483906in}{1.684734in}}%
\pgfusepath{clip}%
\pgfsetbuttcap%
\pgfsetroundjoin%
\definecolor{currentfill}{rgb}{0.298039,0.447059,0.690196}%
\pgfsetfillcolor{currentfill}%
\pgfsetlinewidth{1.003750pt}%
\definecolor{currentstroke}{rgb}{0.298039,0.447059,0.690196}%
\pgfsetstrokecolor{currentstroke}%
\pgfsetdash{}{0pt}%
\pgfpathmoveto{\pgfqpoint{4.064554in}{2.036446in}}%
\pgfpathcurveto{\pgfqpoint{4.072791in}{2.036446in}}{\pgfqpoint{4.080691in}{2.039719in}}{\pgfqpoint{4.086515in}{2.045543in}}%
\pgfpathcurveto{\pgfqpoint{4.092339in}{2.051367in}}{\pgfqpoint{4.095611in}{2.059267in}}{\pgfqpoint{4.095611in}{2.067503in}}%
\pgfpathcurveto{\pgfqpoint{4.095611in}{2.075739in}}{\pgfqpoint{4.092339in}{2.083639in}}{\pgfqpoint{4.086515in}{2.089463in}}%
\pgfpathcurveto{\pgfqpoint{4.080691in}{2.095287in}}{\pgfqpoint{4.072791in}{2.098559in}}{\pgfqpoint{4.064554in}{2.098559in}}%
\pgfpathcurveto{\pgfqpoint{4.056318in}{2.098559in}}{\pgfqpoint{4.048418in}{2.095287in}}{\pgfqpoint{4.042594in}{2.089463in}}%
\pgfpathcurveto{\pgfqpoint{4.036770in}{2.083639in}}{\pgfqpoint{4.033498in}{2.075739in}}{\pgfqpoint{4.033498in}{2.067503in}}%
\pgfpathcurveto{\pgfqpoint{4.033498in}{2.059267in}}{\pgfqpoint{4.036770in}{2.051367in}}{\pgfqpoint{4.042594in}{2.045543in}}%
\pgfpathcurveto{\pgfqpoint{4.048418in}{2.039719in}}{\pgfqpoint{4.056318in}{2.036446in}}{\pgfqpoint{4.064554in}{2.036446in}}%
\pgfpathclose%
\pgfusepath{stroke,fill}%
\end{pgfscope}%
\begin{pgfscope}%
\pgfpathrectangle{\pgfqpoint{3.874381in}{0.557870in}}{\pgfqpoint{2.483906in}{1.684734in}}%
\pgfusepath{clip}%
\pgfsetbuttcap%
\pgfsetroundjoin%
\definecolor{currentfill}{rgb}{0.298039,0.447059,0.690196}%
\pgfsetfillcolor{currentfill}%
\pgfsetlinewidth{1.003750pt}%
\definecolor{currentstroke}{rgb}{0.298039,0.447059,0.690196}%
\pgfsetstrokecolor{currentstroke}%
\pgfsetdash{}{0pt}%
\pgfpathmoveto{\pgfqpoint{3.992615in}{2.084486in}}%
\pgfpathcurveto{\pgfqpoint{4.000851in}{2.084486in}}{\pgfqpoint{4.008751in}{2.087759in}}{\pgfqpoint{4.014575in}{2.093583in}}%
\pgfpathcurveto{\pgfqpoint{4.020399in}{2.099406in}}{\pgfqpoint{4.023671in}{2.107306in}}{\pgfqpoint{4.023671in}{2.115543in}}%
\pgfpathcurveto{\pgfqpoint{4.023671in}{2.123779in}}{\pgfqpoint{4.020399in}{2.131679in}}{\pgfqpoint{4.014575in}{2.137503in}}%
\pgfpathcurveto{\pgfqpoint{4.008751in}{2.143327in}}{\pgfqpoint{4.000851in}{2.146599in}}{\pgfqpoint{3.992615in}{2.146599in}}%
\pgfpathcurveto{\pgfqpoint{3.984379in}{2.146599in}}{\pgfqpoint{3.976479in}{2.143327in}}{\pgfqpoint{3.970655in}{2.137503in}}%
\pgfpathcurveto{\pgfqpoint{3.964831in}{2.131679in}}{\pgfqpoint{3.961558in}{2.123779in}}{\pgfqpoint{3.961558in}{2.115543in}}%
\pgfpathcurveto{\pgfqpoint{3.961558in}{2.107306in}}{\pgfqpoint{3.964831in}{2.099406in}}{\pgfqpoint{3.970655in}{2.093583in}}%
\pgfpathcurveto{\pgfqpoint{3.976479in}{2.087759in}}{\pgfqpoint{3.984379in}{2.084486in}}{\pgfqpoint{3.992615in}{2.084486in}}%
\pgfpathclose%
\pgfusepath{stroke,fill}%
\end{pgfscope}%
\begin{pgfscope}%
\pgfpathrectangle{\pgfqpoint{3.874381in}{0.557870in}}{\pgfqpoint{2.483906in}{1.684734in}}%
\pgfusepath{clip}%
\pgfsetbuttcap%
\pgfsetroundjoin%
\definecolor{currentfill}{rgb}{0.298039,0.447059,0.690196}%
\pgfsetfillcolor{currentfill}%
\pgfsetlinewidth{1.003750pt}%
\definecolor{currentstroke}{rgb}{0.298039,0.447059,0.690196}%
\pgfsetstrokecolor{currentstroke}%
\pgfsetdash{}{0pt}%
\pgfpathmoveto{\pgfqpoint{3.987286in}{2.133340in}}%
\pgfpathcurveto{\pgfqpoint{3.995522in}{2.133340in}}{\pgfqpoint{4.003422in}{2.136613in}}{\pgfqpoint{4.009246in}{2.142437in}}%
\pgfpathcurveto{\pgfqpoint{4.015070in}{2.148261in}}{\pgfqpoint{4.018343in}{2.156161in}}{\pgfqpoint{4.018343in}{2.164397in}}%
\pgfpathcurveto{\pgfqpoint{4.018343in}{2.172633in}}{\pgfqpoint{4.015070in}{2.180533in}}{\pgfqpoint{4.009246in}{2.186357in}}%
\pgfpathcurveto{\pgfqpoint{4.003422in}{2.192181in}}{\pgfqpoint{3.995522in}{2.195453in}}{\pgfqpoint{3.987286in}{2.195453in}}%
\pgfpathcurveto{\pgfqpoint{3.979050in}{2.195453in}}{\pgfqpoint{3.971150in}{2.192181in}}{\pgfqpoint{3.965326in}{2.186357in}}%
\pgfpathcurveto{\pgfqpoint{3.959502in}{2.180533in}}{\pgfqpoint{3.956230in}{2.172633in}}{\pgfqpoint{3.956230in}{2.164397in}}%
\pgfpathcurveto{\pgfqpoint{3.956230in}{2.156161in}}{\pgfqpoint{3.959502in}{2.148261in}}{\pgfqpoint{3.965326in}{2.142437in}}%
\pgfpathcurveto{\pgfqpoint{3.971150in}{2.136613in}}{\pgfqpoint{3.979050in}{2.133340in}}{\pgfqpoint{3.987286in}{2.133340in}}%
\pgfpathclose%
\pgfusepath{stroke,fill}%
\end{pgfscope}%
\begin{pgfscope}%
\pgfpathrectangle{\pgfqpoint{3.874381in}{0.557870in}}{\pgfqpoint{2.483906in}{1.684734in}}%
\pgfusepath{clip}%
\pgfsetbuttcap%
\pgfsetroundjoin%
\definecolor{currentfill}{rgb}{0.298039,0.447059,0.690196}%
\pgfsetfillcolor{currentfill}%
\pgfsetlinewidth{1.003750pt}%
\definecolor{currentstroke}{rgb}{0.298039,0.447059,0.690196}%
\pgfsetstrokecolor{currentstroke}%
\pgfsetdash{}{0pt}%
\pgfpathmoveto{\pgfqpoint{3.987286in}{2.133340in}}%
\pgfpathcurveto{\pgfqpoint{3.995522in}{2.133340in}}{\pgfqpoint{4.003422in}{2.136613in}}{\pgfqpoint{4.009246in}{2.142437in}}%
\pgfpathcurveto{\pgfqpoint{4.015070in}{2.148261in}}{\pgfqpoint{4.018343in}{2.156161in}}{\pgfqpoint{4.018343in}{2.164397in}}%
\pgfpathcurveto{\pgfqpoint{4.018343in}{2.172633in}}{\pgfqpoint{4.015070in}{2.180533in}}{\pgfqpoint{4.009246in}{2.186357in}}%
\pgfpathcurveto{\pgfqpoint{4.003422in}{2.192181in}}{\pgfqpoint{3.995522in}{2.195453in}}{\pgfqpoint{3.987286in}{2.195453in}}%
\pgfpathcurveto{\pgfqpoint{3.979050in}{2.195453in}}{\pgfqpoint{3.971150in}{2.192181in}}{\pgfqpoint{3.965326in}{2.186357in}}%
\pgfpathcurveto{\pgfqpoint{3.959502in}{2.180533in}}{\pgfqpoint{3.956230in}{2.172633in}}{\pgfqpoint{3.956230in}{2.164397in}}%
\pgfpathcurveto{\pgfqpoint{3.956230in}{2.156161in}}{\pgfqpoint{3.959502in}{2.148261in}}{\pgfqpoint{3.965326in}{2.142437in}}%
\pgfpathcurveto{\pgfqpoint{3.971150in}{2.136613in}}{\pgfqpoint{3.979050in}{2.133340in}}{\pgfqpoint{3.987286in}{2.133340in}}%
\pgfpathclose%
\pgfusepath{stroke,fill}%
\end{pgfscope}%
\begin{pgfscope}%
\pgfpathrectangle{\pgfqpoint{3.874381in}{0.557870in}}{\pgfqpoint{2.483906in}{1.684734in}}%
\pgfusepath{clip}%
\pgfsetbuttcap%
\pgfsetroundjoin%
\definecolor{currentfill}{rgb}{0.298039,0.447059,0.690196}%
\pgfsetfillcolor{currentfill}%
\pgfsetlinewidth{1.003750pt}%
\definecolor{currentstroke}{rgb}{0.298039,0.447059,0.690196}%
\pgfsetstrokecolor{currentstroke}%
\pgfsetdash{}{0pt}%
\pgfpathmoveto{\pgfqpoint{3.987286in}{2.132526in}}%
\pgfpathcurveto{\pgfqpoint{3.995522in}{2.132526in}}{\pgfqpoint{4.003422in}{2.135798in}}{\pgfqpoint{4.009246in}{2.141622in}}%
\pgfpathcurveto{\pgfqpoint{4.015070in}{2.147446in}}{\pgfqpoint{4.018343in}{2.155346in}}{\pgfqpoint{4.018343in}{2.163583in}}%
\pgfpathcurveto{\pgfqpoint{4.018343in}{2.171819in}}{\pgfqpoint{4.015070in}{2.179719in}}{\pgfqpoint{4.009246in}{2.185543in}}%
\pgfpathcurveto{\pgfqpoint{4.003422in}{2.191367in}}{\pgfqpoint{3.995522in}{2.194639in}}{\pgfqpoint{3.987286in}{2.194639in}}%
\pgfpathcurveto{\pgfqpoint{3.979050in}{2.194639in}}{\pgfqpoint{3.971150in}{2.191367in}}{\pgfqpoint{3.965326in}{2.185543in}}%
\pgfpathcurveto{\pgfqpoint{3.959502in}{2.179719in}}{\pgfqpoint{3.956230in}{2.171819in}}{\pgfqpoint{3.956230in}{2.163583in}}%
\pgfpathcurveto{\pgfqpoint{3.956230in}{2.155346in}}{\pgfqpoint{3.959502in}{2.147446in}}{\pgfqpoint{3.965326in}{2.141622in}}%
\pgfpathcurveto{\pgfqpoint{3.971150in}{2.135798in}}{\pgfqpoint{3.979050in}{2.132526in}}{\pgfqpoint{3.987286in}{2.132526in}}%
\pgfpathclose%
\pgfusepath{stroke,fill}%
\end{pgfscope}%
\begin{pgfscope}%
\pgfpathrectangle{\pgfqpoint{3.874381in}{0.557870in}}{\pgfqpoint{2.483906in}{1.684734in}}%
\pgfusepath{clip}%
\pgfsetbuttcap%
\pgfsetroundjoin%
\definecolor{currentfill}{rgb}{0.298039,0.447059,0.690196}%
\pgfsetfillcolor{currentfill}%
\pgfsetlinewidth{1.003750pt}%
\definecolor{currentstroke}{rgb}{0.298039,0.447059,0.690196}%
\pgfsetstrokecolor{currentstroke}%
\pgfsetdash{}{0pt}%
\pgfpathmoveto{\pgfqpoint{3.987286in}{2.132526in}}%
\pgfpathcurveto{\pgfqpoint{3.995522in}{2.132526in}}{\pgfqpoint{4.003422in}{2.135798in}}{\pgfqpoint{4.009246in}{2.141622in}}%
\pgfpathcurveto{\pgfqpoint{4.015070in}{2.147446in}}{\pgfqpoint{4.018343in}{2.155346in}}{\pgfqpoint{4.018343in}{2.163583in}}%
\pgfpathcurveto{\pgfqpoint{4.018343in}{2.171819in}}{\pgfqpoint{4.015070in}{2.179719in}}{\pgfqpoint{4.009246in}{2.185543in}}%
\pgfpathcurveto{\pgfqpoint{4.003422in}{2.191367in}}{\pgfqpoint{3.995522in}{2.194639in}}{\pgfqpoint{3.987286in}{2.194639in}}%
\pgfpathcurveto{\pgfqpoint{3.979050in}{2.194639in}}{\pgfqpoint{3.971150in}{2.191367in}}{\pgfqpoint{3.965326in}{2.185543in}}%
\pgfpathcurveto{\pgfqpoint{3.959502in}{2.179719in}}{\pgfqpoint{3.956230in}{2.171819in}}{\pgfqpoint{3.956230in}{2.163583in}}%
\pgfpathcurveto{\pgfqpoint{3.956230in}{2.155346in}}{\pgfqpoint{3.959502in}{2.147446in}}{\pgfqpoint{3.965326in}{2.141622in}}%
\pgfpathcurveto{\pgfqpoint{3.971150in}{2.135798in}}{\pgfqpoint{3.979050in}{2.132526in}}{\pgfqpoint{3.987286in}{2.132526in}}%
\pgfpathclose%
\pgfusepath{stroke,fill}%
\end{pgfscope}%
\begin{pgfscope}%
\pgfpathrectangle{\pgfqpoint{3.874381in}{0.557870in}}{\pgfqpoint{2.483906in}{1.684734in}}%
\pgfusepath{clip}%
\pgfsetbuttcap%
\pgfsetroundjoin%
\definecolor{currentfill}{rgb}{0.298039,0.447059,0.690196}%
\pgfsetfillcolor{currentfill}%
\pgfsetlinewidth{1.003750pt}%
\definecolor{currentstroke}{rgb}{0.298039,0.447059,0.690196}%
\pgfsetstrokecolor{currentstroke}%
\pgfsetdash{}{0pt}%
\pgfpathmoveto{\pgfqpoint{3.987286in}{2.131712in}}%
\pgfpathcurveto{\pgfqpoint{3.995522in}{2.131712in}}{\pgfqpoint{4.003422in}{2.134984in}}{\pgfqpoint{4.009246in}{2.140808in}}%
\pgfpathcurveto{\pgfqpoint{4.015070in}{2.146632in}}{\pgfqpoint{4.018343in}{2.154532in}}{\pgfqpoint{4.018343in}{2.162768in}}%
\pgfpathcurveto{\pgfqpoint{4.018343in}{2.171005in}}{\pgfqpoint{4.015070in}{2.178905in}}{\pgfqpoint{4.009246in}{2.184729in}}%
\pgfpathcurveto{\pgfqpoint{4.003422in}{2.190553in}}{\pgfqpoint{3.995522in}{2.193825in}}{\pgfqpoint{3.987286in}{2.193825in}}%
\pgfpathcurveto{\pgfqpoint{3.979050in}{2.193825in}}{\pgfqpoint{3.971150in}{2.190553in}}{\pgfqpoint{3.965326in}{2.184729in}}%
\pgfpathcurveto{\pgfqpoint{3.959502in}{2.178905in}}{\pgfqpoint{3.956230in}{2.171005in}}{\pgfqpoint{3.956230in}{2.162768in}}%
\pgfpathcurveto{\pgfqpoint{3.956230in}{2.154532in}}{\pgfqpoint{3.959502in}{2.146632in}}{\pgfqpoint{3.965326in}{2.140808in}}%
\pgfpathcurveto{\pgfqpoint{3.971150in}{2.134984in}}{\pgfqpoint{3.979050in}{2.131712in}}{\pgfqpoint{3.987286in}{2.131712in}}%
\pgfpathclose%
\pgfusepath{stroke,fill}%
\end{pgfscope}%
\begin{pgfscope}%
\pgfpathrectangle{\pgfqpoint{3.874381in}{0.557870in}}{\pgfqpoint{2.483906in}{1.684734in}}%
\pgfusepath{clip}%
\pgfsetbuttcap%
\pgfsetroundjoin%
\definecolor{currentfill}{rgb}{0.298039,0.447059,0.690196}%
\pgfsetfillcolor{currentfill}%
\pgfsetlinewidth{1.003750pt}%
\definecolor{currentstroke}{rgb}{0.298039,0.447059,0.690196}%
\pgfsetstrokecolor{currentstroke}%
\pgfsetdash{}{0pt}%
\pgfpathmoveto{\pgfqpoint{3.987286in}{2.131712in}}%
\pgfpathcurveto{\pgfqpoint{3.995522in}{2.131712in}}{\pgfqpoint{4.003422in}{2.134984in}}{\pgfqpoint{4.009246in}{2.140808in}}%
\pgfpathcurveto{\pgfqpoint{4.015070in}{2.146632in}}{\pgfqpoint{4.018343in}{2.154532in}}{\pgfqpoint{4.018343in}{2.162768in}}%
\pgfpathcurveto{\pgfqpoint{4.018343in}{2.171005in}}{\pgfqpoint{4.015070in}{2.178905in}}{\pgfqpoint{4.009246in}{2.184729in}}%
\pgfpathcurveto{\pgfqpoint{4.003422in}{2.190553in}}{\pgfqpoint{3.995522in}{2.193825in}}{\pgfqpoint{3.987286in}{2.193825in}}%
\pgfpathcurveto{\pgfqpoint{3.979050in}{2.193825in}}{\pgfqpoint{3.971150in}{2.190553in}}{\pgfqpoint{3.965326in}{2.184729in}}%
\pgfpathcurveto{\pgfqpoint{3.959502in}{2.178905in}}{\pgfqpoint{3.956230in}{2.171005in}}{\pgfqpoint{3.956230in}{2.162768in}}%
\pgfpathcurveto{\pgfqpoint{3.956230in}{2.154532in}}{\pgfqpoint{3.959502in}{2.146632in}}{\pgfqpoint{3.965326in}{2.140808in}}%
\pgfpathcurveto{\pgfqpoint{3.971150in}{2.134984in}}{\pgfqpoint{3.979050in}{2.131712in}}{\pgfqpoint{3.987286in}{2.131712in}}%
\pgfpathclose%
\pgfusepath{stroke,fill}%
\end{pgfscope}%
\begin{pgfscope}%
\pgfpathrectangle{\pgfqpoint{3.874381in}{0.557870in}}{\pgfqpoint{2.483906in}{1.684734in}}%
\pgfusepath{clip}%
\pgfsetbuttcap%
\pgfsetroundjoin%
\definecolor{currentfill}{rgb}{0.298039,0.447059,0.690196}%
\pgfsetfillcolor{currentfill}%
\pgfsetlinewidth{1.003750pt}%
\definecolor{currentstroke}{rgb}{0.298039,0.447059,0.690196}%
\pgfsetstrokecolor{currentstroke}%
\pgfsetdash{}{0pt}%
\pgfpathmoveto{\pgfqpoint{3.987286in}{2.127641in}}%
\pgfpathcurveto{\pgfqpoint{3.995522in}{2.127641in}}{\pgfqpoint{4.003422in}{2.130913in}}{\pgfqpoint{4.009246in}{2.136737in}}%
\pgfpathcurveto{\pgfqpoint{4.015070in}{2.142561in}}{\pgfqpoint{4.018343in}{2.150461in}}{\pgfqpoint{4.018343in}{2.158697in}}%
\pgfpathcurveto{\pgfqpoint{4.018343in}{2.166934in}}{\pgfqpoint{4.015070in}{2.174834in}}{\pgfqpoint{4.009246in}{2.180657in}}%
\pgfpathcurveto{\pgfqpoint{4.003422in}{2.186481in}}{\pgfqpoint{3.995522in}{2.189754in}}{\pgfqpoint{3.987286in}{2.189754in}}%
\pgfpathcurveto{\pgfqpoint{3.979050in}{2.189754in}}{\pgfqpoint{3.971150in}{2.186481in}}{\pgfqpoint{3.965326in}{2.180657in}}%
\pgfpathcurveto{\pgfqpoint{3.959502in}{2.174834in}}{\pgfqpoint{3.956230in}{2.166934in}}{\pgfqpoint{3.956230in}{2.158697in}}%
\pgfpathcurveto{\pgfqpoint{3.956230in}{2.150461in}}{\pgfqpoint{3.959502in}{2.142561in}}{\pgfqpoint{3.965326in}{2.136737in}}%
\pgfpathcurveto{\pgfqpoint{3.971150in}{2.130913in}}{\pgfqpoint{3.979050in}{2.127641in}}{\pgfqpoint{3.987286in}{2.127641in}}%
\pgfpathclose%
\pgfusepath{stroke,fill}%
\end{pgfscope}%
\begin{pgfscope}%
\pgfsetrectcap%
\pgfsetmiterjoin%
\pgfsetlinewidth{1.254687pt}%
\definecolor{currentstroke}{rgb}{1.000000,1.000000,1.000000}%
\pgfsetstrokecolor{currentstroke}%
\pgfsetdash{}{0pt}%
\pgfpathmoveto{\pgfqpoint{3.874381in}{0.557870in}}%
\pgfpathlineto{\pgfqpoint{3.874381in}{2.242604in}}%
\pgfusepath{stroke}%
\end{pgfscope}%
\begin{pgfscope}%
\pgfsetrectcap%
\pgfsetmiterjoin%
\pgfsetlinewidth{1.254687pt}%
\definecolor{currentstroke}{rgb}{1.000000,1.000000,1.000000}%
\pgfsetstrokecolor{currentstroke}%
\pgfsetdash{}{0pt}%
\pgfpathmoveto{\pgfqpoint{6.358287in}{0.557870in}}%
\pgfpathlineto{\pgfqpoint{6.358287in}{2.242604in}}%
\pgfusepath{stroke}%
\end{pgfscope}%
\begin{pgfscope}%
\pgfsetrectcap%
\pgfsetmiterjoin%
\pgfsetlinewidth{1.254687pt}%
\definecolor{currentstroke}{rgb}{1.000000,1.000000,1.000000}%
\pgfsetstrokecolor{currentstroke}%
\pgfsetdash{}{0pt}%
\pgfpathmoveto{\pgfqpoint{3.874381in}{0.557870in}}%
\pgfpathlineto{\pgfqpoint{6.358287in}{0.557870in}}%
\pgfusepath{stroke}%
\end{pgfscope}%
\begin{pgfscope}%
\pgfsetrectcap%
\pgfsetmiterjoin%
\pgfsetlinewidth{1.254687pt}%
\definecolor{currentstroke}{rgb}{1.000000,1.000000,1.000000}%
\pgfsetstrokecolor{currentstroke}%
\pgfsetdash{}{0pt}%
\pgfpathmoveto{\pgfqpoint{3.874381in}{2.242604in}}%
\pgfpathlineto{\pgfqpoint{6.358287in}{2.242604in}}%
\pgfusepath{stroke}%
\end{pgfscope}%
\begin{pgfscope}%
\definecolor{textcolor}{rgb}{0.150000,0.150000,0.150000}%
\pgfsetstrokecolor{textcolor}%
\pgfsetfillcolor{textcolor}%
\pgftext[x=5.116334in,y=2.325938in,,base]{\color{textcolor}\sffamily\fontsize{11.000000}{13.200000}\selectfont (b)}%
\end{pgfscope}%
\end{pgfpicture}%
\makeatother%
\endgroup%

    % \includegraphics[width=\textwidth]{results/tsc_segm_ind_dor_sens_spec_dist.png}
    \caption{Distribution of \acrshort{dor}, sensitivity and specificity for the different \acrshort{tsc} models when classifying left ventricle segment indication.}
    \label{fig:tsc_segm_ind_dor_sens_spec_dist}
\end{figure}

\begin{table*}[htb]
    \centering
    \ra{1.3}
    \begin{tabular}{lrrrr}
        \toprule
        Dataset-model     &  Accuracy &  Sensitivity &  Specificity &  \acrshort{dor} \\
        \midrule
        regular/weighted/2 &      0.69 &         0.45 &         0.95 & 15.63 \\
        scaled/weighted/2  &      0.69 &         0.45 &         0.95 & 15.63 \\
        regular/ward/2     &      0.77 &         0.66 &         0.88 & 14.26 \\
        scaled/ward/2      &      0.77 &         0.66 &         0.88 & 14.26 \\
        regular/complete/2 &      0.75 &         0.62 &         0.89 & 13.92 \\
        \bottomrule
    \end{tabular}
    \caption{The accuracy, \acrshort{dor}, sensitivity and specicity scores of the five best performing two-cluster-center \acrshort{tsc} models in terms of \acrshort{dor}, at detecting segment indication.
             The \textbf{Dataset-model} column indicates \textit{Type of preprocessing used}$/$\textit{Linkage criteria of model}$/$\textit{Number of cluster centers}.}
    \label{tab:tsc_segm_ind_dor_sens_spec_dist}
\end{table*}

From the distribution plot in figure \ref{fig:tsc_segm_ind_dor_sens_spec_dist}a one can see that the majority of the \acrshort{dor} are close to zero, but a few models are able to achieve \acrshort{dor} above 12, and some models attain a \acrshort{dor} close to 15 when applied to identify segment indication. From the scatter plot in figure \ref{fig:tsc_segm_ind_dor_sens_spec_dist}b one can see that the sensitivity of the \acrshort{tsc} models range from $0.25$ to 1, and the specificity of the \acrshort{tsc} models range from 0 to approximately 1. The spread in both sensitivity and specificity is quite large, and there are very few models that are able to a attain a high sensitivity while at the same time attaining a high specificity, and vice versa. Common to the high performing \acrshort{tsc} models in terms of \acrshort{dor} is that they all use either no preprocessing at all, or scaling. \textit{z-norm/complete/2} is the seventh best \acrshort{tsc} model in terms of \acrshort{dor}, and attains a \acrshort{dor} of 5.92 when applied to identify segment indication. \textit{norm/ward/2} is the ninth best models in terms of \acrshort{dor}, and attains a \acrshort{dor} of 1.56, when applied to identify segment indication. This can be comfirmed from table \ref{tab:tsc_segm_ind_raw_results}. The two \acrshort{tsc} models attaining the highest \acrshort{dor} \textit{regular/weighted/2}, and \textit{scaled/weighted/2} differ only in type of preprocessing used. From table \ref{tab:tsc_segm_ind_dor_sens_spec_dist} and table \ref{tab:tsc_segm_ind_raw_results} one can see that the two models attain the same scores in all metrics, this is because they yield the exact same cluster assignments to the individual segment strain curves. The same goes for the next two \acrshort{tsc} models in line \textit{regular/ward/2} \textit{scaled/ward/2}, these two models are also the models that attain the highest accarcy of all the \acrshort{tsc} models. Of the two \acrshort{tsc} models \textit{regular/weighted/2}, and \textit{regular/ward/2} the latter is preferred for predicting segment indication because \textit{regular/ward/2} has a more persistent performance in both sensitivity and specificity, where as \textit{regular/weighted/2} has a high specificity, but a very low sensitivity.

\begin{table*}
    \centering
    \ra{1.3}
    \begin{tabular}{lr}
        \toprule
        Dataset-model     &  \acrshort{ari} \\
        \midrule
        scaled/centroid/5  & 0.286 \\
        regular/centroid/5 & 0.286 \\
        regular/ward/2     & 0.284 \\
        scaled/ward/2      & 0.284 \\
        scaled/centroid/6  & 0.271 \\
        \bottomrule
    \end{tabular}
    \caption{The five highest \acrshort{ari} scores attained when applying \acrshort{tsc} for detecting segmend indication.
             The \textbf{Dataset-model} column indicates \textit{Type of preprocessing used}$/$\textit{Linkage criteria of model}$/$\textit{Number of cluster centers}.}
    \label{tab:tsc_segm_ind_ari}
\end{table*}

The majority of the \acrshort{ari} of \acrshort{tsc} models applied to identify segment indication, but as one can see from table \ref{tab:tsc_segm_ind_ari} some models are able to attain \acrshort{ari} above 25. As with the other case studies, the \acrshort{tsc} models that attain the highest \acrshort{ari} are models that use either no preprocessing at all or scaling. Puzzlingly enough the top two \acrshort{tsc} models for classifying segment indication in terms of \acrshort{ari}, are models evaluated at five cluster centers, not two. TSC models \textit{scaled/centroid/5}, and \textit{regular/centroid/5} differ only in type of preprocessing used, and they yield the exact same cluster assignments, and evaluations scores. The next two models in order of \acrshort{ari} \textit{regular/ward/2}, and \textit{scaled/ward/2} are familiar from the list of \acrshort{tsc} models attaining the highest \acrshort{dor} when applied to identify segment indication. From table \ref{tab:tsc_segm_ind_ari} one can also see that the difference in \acrshort{ari} between \textit{regular/centroid/5}, and \textit{regular/ward/2} is only 0.002 Since the \textit{regular/ward/2} model will be considered the best of the \acrshort{tsc} models at classifying segment indication. It attains the third highest \acrshort{ari} of all the \acrshort{tsc} models applied to identify segment indication, and is the preferred model among the \acrshort{tsc} models evaluated at two cluster centers.

\newpage

\subsection{Artificial Neural Network}

\begin{table*}[htb]
    \centering
    \ra{1.3}
    \begin{tabular}{lrrrr}
        \toprule
        model      &  Accuracy &  Sensitivity &  Specificity &  \acrshort{dor} \\
        \midrule
        regular     &      0.74 &         0.80 &         0.68 & 8.65 \\
        downsampled &      0.74 &         0.74 &         0.75 & 8.38 \\
        upsampled   &      0.65 &         0.55 &         0.73 & 3.36 \\
        \bottomrule
    \end{tabular}
    \caption{Evaluation metrics of the \acrshort{ann} for classifying the binary indication of individual segments in the left ventricle.}
    \label{tab:ANN_segm_ind_perf}
\end{table*}

Of the three variations of the \acrshort{ann} model, the one that uses no resampling, and the one that downsamples all signals to the lowest sample rate achieve relatively similar \acrshort{dor} scores. The variation that upsamples the sample rate of all the curves to the highest sample rate performs significantly worse than the other two in terms of \acrshort{dor} and sensitivity. Of the three variations the model that uses downsampling is the preferred model of the three since its sensitivity and specificity are more balanced than the model that uses no resampling, and accuracy is higher than the model that uses upsampling.

\subsection{Comparisons}

\begin{table*}
    \centering
    \ra{1.3}
    \begin{tabular}{lcccc}
        \toprule
        Dataset-model               & Accuracy & Sensitivity & Specificity & \acrshort{dor} \\
        \midrule
        \textbf{TSC}-regular/ward/2 &     0.76 &        0.64 &        0.88 & 13.15 \\
        \textbf{ANN}-downsampled    &     0.74 &        0.74 &        0.75 & 8.38 \\
        \midrule
        Dataset-model               &  TP  &  TN  &  FP  &  FN \\
        \midrule
        \textbf{TSC}-regular/ward/2 & 1202 & 1491 &  204 &  616 \\
        \textbf{ANN}-downsampled    & 1255 & 1390 &  473 &  440 \\
        \bottomrule
    \end{tabular}
    \caption{A table comparing the best contenders within each model group for predicting segment indication. 
             The top table compare the models by their accuracy, sensitivity, specificity and \acrshort{dor}, 
             and the bottom table shows the number of TPs, TNs, FPs and FNs that the different models attain.}
    \label{tab:segm_ind_compare}
\end{table*}

From table \ref{tab:segm_ind_compare} one can see that the performances of the \acrshort{ann}, and \acrshort{tsc} models are quite close in terms of accuracy, but differ significantly in the other metrics. The \acrshort{tsc} model \textit{regular/ward/2} attains a higher accuracy, specificity and \acrshort{dor} than the \acrshort{ann} model \textit{downsampled}. This can also be confirmed by the fact that the \acrshort{tsc} model attains more TN, and fewer FP than the \acrshort{ann} model.  The \acrshort{ann} model attains the highest sensitivity, which can be confirmed by the fact that it attains more TP and fewer FN than the \acrshort{tsc} model. The \acrshort{ann} model is also the model that attains the most balanced scores of sensitivity and specificity. Therefore the \acrshort{ann} model is chosen as the best performer at predicting the segment indication. 