\chapter{Results}

In this chapter the results will be presented in the form of three case studies. 
Each case study will focus on a single target variable, and aims to find which model group performs best at predicting the target variable in question.
Recall that the three target variables that will be considered in this thesis are: Heart failure, patient diagnosis, and the indication of individual left ventricle segments.
As mentioned earlier in the chapter, four model groups will be tested. 
The case studies will first deal with each model group individually, where variants of the models with different hypermarameters will be tested on the different datasets. 
Then, the best performing model within each model group will be used to compare the four model groups.
The supervised models will be assessed with the metrics: accuracy, sensitivity, specificity and DOR.
The clustering methods evaluated at two cluster centers will be assessed with the same methods as the supervised models. 
The clustering methods evaluated at two to nine cluster centers will also be assessed with ARI to determine 
whether the models evaluated at a higher number of cluster centers could fit the data better.

\section{Case Study: Heart Failure}

\subsection{Time-series Clustering}

\begin{figure}[htb]
    \centering
    % \includegraphics[width=\textwidth]{results/tsc_hf_dor_sens_spec_dist.png}
    %% Creator: Matplotlib, PGF backend
%%
%% To include the figure in your LaTeX document, write
%%   \input{<filename>.pgf}
%%
%% Make sure the required packages are loaded in your preamble
%%   \usepackage{pgf}
%%
%% Figures using additional raster images can only be included by \input if
%% they are in the same directory as the main LaTeX file. For loading figures
%% from other directories you can use the `import` package
%%   \usepackage{import}
%% and then include the figures with
%%   \import{<path to file>}{<filename>.pgf}
%%
%% Matplotlib used the following preamble
%%
\begingroup%
\makeatletter%
\begin{pgfpicture}%
\pgfpathrectangle{\pgfpointorigin}{\pgfqpoint{6.364000in}{2.540000in}}%
\pgfusepath{use as bounding box, clip}%
\begin{pgfscope}%
\pgfsetbuttcap%
\pgfsetmiterjoin%
\definecolor{currentfill}{rgb}{1.000000,1.000000,1.000000}%
\pgfsetfillcolor{currentfill}%
\pgfsetlinewidth{0.000000pt}%
\definecolor{currentstroke}{rgb}{1.000000,1.000000,1.000000}%
\pgfsetstrokecolor{currentstroke}%
\pgfsetdash{}{0pt}%
\pgfpathmoveto{\pgfqpoint{0.000000in}{0.000000in}}%
\pgfpathlineto{\pgfqpoint{6.364000in}{0.000000in}}%
\pgfpathlineto{\pgfqpoint{6.364000in}{2.540000in}}%
\pgfpathlineto{\pgfqpoint{0.000000in}{2.540000in}}%
\pgfpathclose%
\pgfusepath{fill}%
\end{pgfscope}%
\begin{pgfscope}%
\pgfsetbuttcap%
\pgfsetmiterjoin%
\definecolor{currentfill}{rgb}{0.917647,0.917647,0.949020}%
\pgfsetfillcolor{currentfill}%
\pgfsetlinewidth{0.000000pt}%
\definecolor{currentstroke}{rgb}{0.000000,0.000000,0.000000}%
\pgfsetstrokecolor{currentstroke}%
\pgfsetstrokeopacity{0.000000}%
\pgfsetdash{}{0pt}%
\pgfpathmoveto{\pgfqpoint{0.650810in}{0.557870in}}%
\pgfpathlineto{\pgfqpoint{3.096898in}{0.557870in}}%
\pgfpathlineto{\pgfqpoint{3.096898in}{2.242604in}}%
\pgfpathlineto{\pgfqpoint{0.650810in}{2.242604in}}%
\pgfpathclose%
\pgfusepath{fill}%
\end{pgfscope}%
\begin{pgfscope}%
\pgfpathrectangle{\pgfqpoint{0.650810in}{0.557870in}}{\pgfqpoint{2.446088in}{1.684734in}}%
\pgfusepath{clip}%
\pgfsetroundcap%
\pgfsetroundjoin%
\pgfsetlinewidth{1.003750pt}%
\definecolor{currentstroke}{rgb}{1.000000,1.000000,1.000000}%
\pgfsetstrokecolor{currentstroke}%
\pgfsetdash{}{0pt}%
\pgfpathmoveto{\pgfqpoint{0.761996in}{0.557870in}}%
\pgfpathlineto{\pgfqpoint{0.761996in}{2.242604in}}%
\pgfusepath{stroke}%
\end{pgfscope}%
\begin{pgfscope}%
\definecolor{textcolor}{rgb}{0.150000,0.150000,0.150000}%
\pgfsetstrokecolor{textcolor}%
\pgfsetfillcolor{textcolor}%
\pgftext[x=0.761996in,y=0.425926in,,top]{\color{textcolor}\sffamily\fontsize{11.000000}{13.200000}\selectfont \(\displaystyle 0\)}%
\end{pgfscope}%
\begin{pgfscope}%
\pgfpathrectangle{\pgfqpoint{0.650810in}{0.557870in}}{\pgfqpoint{2.446088in}{1.684734in}}%
\pgfusepath{clip}%
\pgfsetroundcap%
\pgfsetroundjoin%
\pgfsetlinewidth{1.003750pt}%
\definecolor{currentstroke}{rgb}{1.000000,1.000000,1.000000}%
\pgfsetstrokecolor{currentstroke}%
\pgfsetdash{}{0pt}%
\pgfpathmoveto{\pgfqpoint{1.710787in}{0.557870in}}%
\pgfpathlineto{\pgfqpoint{1.710787in}{2.242604in}}%
\pgfusepath{stroke}%
\end{pgfscope}%
\begin{pgfscope}%
\definecolor{textcolor}{rgb}{0.150000,0.150000,0.150000}%
\pgfsetstrokecolor{textcolor}%
\pgfsetfillcolor{textcolor}%
\pgftext[x=1.710787in,y=0.425926in,,top]{\color{textcolor}\sffamily\fontsize{11.000000}{13.200000}\selectfont \(\displaystyle 5\)}%
\end{pgfscope}%
\begin{pgfscope}%
\pgfpathrectangle{\pgfqpoint{0.650810in}{0.557870in}}{\pgfqpoint{2.446088in}{1.684734in}}%
\pgfusepath{clip}%
\pgfsetroundcap%
\pgfsetroundjoin%
\pgfsetlinewidth{1.003750pt}%
\definecolor{currentstroke}{rgb}{1.000000,1.000000,1.000000}%
\pgfsetstrokecolor{currentstroke}%
\pgfsetdash{}{0pt}%
\pgfpathmoveto{\pgfqpoint{2.659578in}{0.557870in}}%
\pgfpathlineto{\pgfqpoint{2.659578in}{2.242604in}}%
\pgfusepath{stroke}%
\end{pgfscope}%
\begin{pgfscope}%
\definecolor{textcolor}{rgb}{0.150000,0.150000,0.150000}%
\pgfsetstrokecolor{textcolor}%
\pgfsetfillcolor{textcolor}%
\pgftext[x=2.659578in,y=0.425926in,,top]{\color{textcolor}\sffamily\fontsize{11.000000}{13.200000}\selectfont \(\displaystyle 10\)}%
\end{pgfscope}%
\begin{pgfscope}%
\definecolor{textcolor}{rgb}{0.150000,0.150000,0.150000}%
\pgfsetstrokecolor{textcolor}%
\pgfsetfillcolor{textcolor}%
\pgftext[x=1.873854in,y=0.235185in,,top]{\color{textcolor}\sffamily\fontsize{11.000000}{13.200000}\selectfont DOR}%
\end{pgfscope}%
\begin{pgfscope}%
\pgfpathrectangle{\pgfqpoint{0.650810in}{0.557870in}}{\pgfqpoint{2.446088in}{1.684734in}}%
\pgfusepath{clip}%
\pgfsetroundcap%
\pgfsetroundjoin%
\pgfsetlinewidth{1.003750pt}%
\definecolor{currentstroke}{rgb}{1.000000,1.000000,1.000000}%
\pgfsetstrokecolor{currentstroke}%
\pgfsetdash{}{0pt}%
\pgfpathmoveto{\pgfqpoint{0.650810in}{0.557870in}}%
\pgfpathlineto{\pgfqpoint{3.096898in}{0.557870in}}%
\pgfusepath{stroke}%
\end{pgfscope}%
\begin{pgfscope}%
\definecolor{textcolor}{rgb}{0.150000,0.150000,0.150000}%
\pgfsetstrokecolor{textcolor}%
\pgfsetfillcolor{textcolor}%
\pgftext[x=0.442824in,y=0.505064in,left,base]{\color{textcolor}\sffamily\fontsize{11.000000}{13.200000}\selectfont \(\displaystyle 0\)}%
\end{pgfscope}%
\begin{pgfscope}%
\pgfpathrectangle{\pgfqpoint{0.650810in}{0.557870in}}{\pgfqpoint{2.446088in}{1.684734in}}%
\pgfusepath{clip}%
\pgfsetroundcap%
\pgfsetroundjoin%
\pgfsetlinewidth{1.003750pt}%
\definecolor{currentstroke}{rgb}{1.000000,1.000000,1.000000}%
\pgfsetstrokecolor{currentstroke}%
\pgfsetdash{}{0pt}%
\pgfpathmoveto{\pgfqpoint{0.650810in}{1.135032in}}%
\pgfpathlineto{\pgfqpoint{3.096898in}{1.135032in}}%
\pgfusepath{stroke}%
\end{pgfscope}%
\begin{pgfscope}%
\definecolor{textcolor}{rgb}{0.150000,0.150000,0.150000}%
\pgfsetstrokecolor{textcolor}%
\pgfsetfillcolor{textcolor}%
\pgftext[x=0.366783in,y=1.082225in,left,base]{\color{textcolor}\sffamily\fontsize{11.000000}{13.200000}\selectfont \(\displaystyle 50\)}%
\end{pgfscope}%
\begin{pgfscope}%
\pgfpathrectangle{\pgfqpoint{0.650810in}{0.557870in}}{\pgfqpoint{2.446088in}{1.684734in}}%
\pgfusepath{clip}%
\pgfsetroundcap%
\pgfsetroundjoin%
\pgfsetlinewidth{1.003750pt}%
\definecolor{currentstroke}{rgb}{1.000000,1.000000,1.000000}%
\pgfsetstrokecolor{currentstroke}%
\pgfsetdash{}{0pt}%
\pgfpathmoveto{\pgfqpoint{0.650810in}{1.712193in}}%
\pgfpathlineto{\pgfqpoint{3.096898in}{1.712193in}}%
\pgfusepath{stroke}%
\end{pgfscope}%
\begin{pgfscope}%
\definecolor{textcolor}{rgb}{0.150000,0.150000,0.150000}%
\pgfsetstrokecolor{textcolor}%
\pgfsetfillcolor{textcolor}%
\pgftext[x=0.290741in,y=1.659386in,left,base]{\color{textcolor}\sffamily\fontsize{11.000000}{13.200000}\selectfont \(\displaystyle 100\)}%
\end{pgfscope}%
\begin{pgfscope}%
\definecolor{textcolor}{rgb}{0.150000,0.150000,0.150000}%
\pgfsetstrokecolor{textcolor}%
\pgfsetfillcolor{textcolor}%
\pgftext[x=0.235185in,y=1.400237in,,bottom,rotate=90.000000]{\color{textcolor}\sffamily\fontsize{11.000000}{13.200000}\selectfont Occurance}%
\end{pgfscope}%
\begin{pgfscope}%
\pgfpathrectangle{\pgfqpoint{0.650810in}{0.557870in}}{\pgfqpoint{2.446088in}{1.684734in}}%
\pgfusepath{clip}%
\pgfsetbuttcap%
\pgfsetmiterjoin%
\definecolor{currentfill}{rgb}{0.298039,0.447059,0.690196}%
\pgfsetfillcolor{currentfill}%
\pgfsetfillopacity{0.400000}%
\pgfsetlinewidth{1.003750pt}%
\definecolor{currentstroke}{rgb}{1.000000,1.000000,1.000000}%
\pgfsetstrokecolor{currentstroke}%
\pgfsetstrokeopacity{0.400000}%
\pgfsetdash{}{0pt}%
\pgfpathmoveto{\pgfqpoint{0.761996in}{0.557870in}}%
\pgfpathlineto{\pgfqpoint{0.984368in}{0.557870in}}%
\pgfpathlineto{\pgfqpoint{0.984368in}{2.162379in}}%
\pgfpathlineto{\pgfqpoint{0.761996in}{2.162379in}}%
\pgfpathclose%
\pgfusepath{stroke,fill}%
\end{pgfscope}%
\begin{pgfscope}%
\pgfpathrectangle{\pgfqpoint{0.650810in}{0.557870in}}{\pgfqpoint{2.446088in}{1.684734in}}%
\pgfusepath{clip}%
\pgfsetbuttcap%
\pgfsetmiterjoin%
\definecolor{currentfill}{rgb}{0.298039,0.447059,0.690196}%
\pgfsetfillcolor{currentfill}%
\pgfsetfillopacity{0.400000}%
\pgfsetlinewidth{1.003750pt}%
\definecolor{currentstroke}{rgb}{1.000000,1.000000,1.000000}%
\pgfsetstrokecolor{currentstroke}%
\pgfsetstrokeopacity{0.400000}%
\pgfsetdash{}{0pt}%
\pgfpathmoveto{\pgfqpoint{0.984368in}{0.557870in}}%
\pgfpathlineto{\pgfqpoint{1.206739in}{0.557870in}}%
\pgfpathlineto{\pgfqpoint{1.206739in}{0.996513in}}%
\pgfpathlineto{\pgfqpoint{0.984368in}{0.996513in}}%
\pgfpathclose%
\pgfusepath{stroke,fill}%
\end{pgfscope}%
\begin{pgfscope}%
\pgfpathrectangle{\pgfqpoint{0.650810in}{0.557870in}}{\pgfqpoint{2.446088in}{1.684734in}}%
\pgfusepath{clip}%
\pgfsetbuttcap%
\pgfsetmiterjoin%
\definecolor{currentfill}{rgb}{0.298039,0.447059,0.690196}%
\pgfsetfillcolor{currentfill}%
\pgfsetfillopacity{0.400000}%
\pgfsetlinewidth{1.003750pt}%
\definecolor{currentstroke}{rgb}{1.000000,1.000000,1.000000}%
\pgfsetstrokecolor{currentstroke}%
\pgfsetstrokeopacity{0.400000}%
\pgfsetdash{}{0pt}%
\pgfpathmoveto{\pgfqpoint{1.206739in}{0.557870in}}%
\pgfpathlineto{\pgfqpoint{1.429111in}{0.557870in}}%
\pgfpathlineto{\pgfqpoint{1.429111in}{0.604043in}}%
\pgfpathlineto{\pgfqpoint{1.206739in}{0.604043in}}%
\pgfpathclose%
\pgfusepath{stroke,fill}%
\end{pgfscope}%
\begin{pgfscope}%
\pgfpathrectangle{\pgfqpoint{0.650810in}{0.557870in}}{\pgfqpoint{2.446088in}{1.684734in}}%
\pgfusepath{clip}%
\pgfsetbuttcap%
\pgfsetmiterjoin%
\definecolor{currentfill}{rgb}{0.298039,0.447059,0.690196}%
\pgfsetfillcolor{currentfill}%
\pgfsetfillopacity{0.400000}%
\pgfsetlinewidth{1.003750pt}%
\definecolor{currentstroke}{rgb}{1.000000,1.000000,1.000000}%
\pgfsetstrokecolor{currentstroke}%
\pgfsetstrokeopacity{0.400000}%
\pgfsetdash{}{0pt}%
\pgfpathmoveto{\pgfqpoint{1.429111in}{0.557870in}}%
\pgfpathlineto{\pgfqpoint{1.651483in}{0.557870in}}%
\pgfpathlineto{\pgfqpoint{1.651483in}{0.927254in}}%
\pgfpathlineto{\pgfqpoint{1.429111in}{0.927254in}}%
\pgfpathclose%
\pgfusepath{stroke,fill}%
\end{pgfscope}%
\begin{pgfscope}%
\pgfpathrectangle{\pgfqpoint{0.650810in}{0.557870in}}{\pgfqpoint{2.446088in}{1.684734in}}%
\pgfusepath{clip}%
\pgfsetbuttcap%
\pgfsetmiterjoin%
\definecolor{currentfill}{rgb}{0.298039,0.447059,0.690196}%
\pgfsetfillcolor{currentfill}%
\pgfsetfillopacity{0.400000}%
\pgfsetlinewidth{1.003750pt}%
\definecolor{currentstroke}{rgb}{1.000000,1.000000,1.000000}%
\pgfsetstrokecolor{currentstroke}%
\pgfsetstrokeopacity{0.400000}%
\pgfsetdash{}{0pt}%
\pgfpathmoveto{\pgfqpoint{1.651483in}{0.557870in}}%
\pgfpathlineto{\pgfqpoint{1.873854in}{0.557870in}}%
\pgfpathlineto{\pgfqpoint{1.873854in}{1.008056in}}%
\pgfpathlineto{\pgfqpoint{1.651483in}{1.008056in}}%
\pgfpathclose%
\pgfusepath{stroke,fill}%
\end{pgfscope}%
\begin{pgfscope}%
\pgfpathrectangle{\pgfqpoint{0.650810in}{0.557870in}}{\pgfqpoint{2.446088in}{1.684734in}}%
\pgfusepath{clip}%
\pgfsetbuttcap%
\pgfsetmiterjoin%
\definecolor{currentfill}{rgb}{0.298039,0.447059,0.690196}%
\pgfsetfillcolor{currentfill}%
\pgfsetfillopacity{0.400000}%
\pgfsetlinewidth{1.003750pt}%
\definecolor{currentstroke}{rgb}{1.000000,1.000000,1.000000}%
\pgfsetstrokecolor{currentstroke}%
\pgfsetstrokeopacity{0.400000}%
\pgfsetdash{}{0pt}%
\pgfpathmoveto{\pgfqpoint{1.873854in}{0.557870in}}%
\pgfpathlineto{\pgfqpoint{2.096226in}{0.557870in}}%
\pgfpathlineto{\pgfqpoint{2.096226in}{0.684846in}}%
\pgfpathlineto{\pgfqpoint{1.873854in}{0.684846in}}%
\pgfpathclose%
\pgfusepath{stroke,fill}%
\end{pgfscope}%
\begin{pgfscope}%
\pgfpathrectangle{\pgfqpoint{0.650810in}{0.557870in}}{\pgfqpoint{2.446088in}{1.684734in}}%
\pgfusepath{clip}%
\pgfsetbuttcap%
\pgfsetmiterjoin%
\definecolor{currentfill}{rgb}{0.298039,0.447059,0.690196}%
\pgfsetfillcolor{currentfill}%
\pgfsetfillopacity{0.400000}%
\pgfsetlinewidth{1.003750pt}%
\definecolor{currentstroke}{rgb}{1.000000,1.000000,1.000000}%
\pgfsetstrokecolor{currentstroke}%
\pgfsetstrokeopacity{0.400000}%
\pgfsetdash{}{0pt}%
\pgfpathmoveto{\pgfqpoint{2.096226in}{0.557870in}}%
\pgfpathlineto{\pgfqpoint{2.318598in}{0.557870in}}%
\pgfpathlineto{\pgfqpoint{2.318598in}{0.661759in}}%
\pgfpathlineto{\pgfqpoint{2.096226in}{0.661759in}}%
\pgfpathclose%
\pgfusepath{stroke,fill}%
\end{pgfscope}%
\begin{pgfscope}%
\pgfpathrectangle{\pgfqpoint{0.650810in}{0.557870in}}{\pgfqpoint{2.446088in}{1.684734in}}%
\pgfusepath{clip}%
\pgfsetbuttcap%
\pgfsetmiterjoin%
\definecolor{currentfill}{rgb}{0.298039,0.447059,0.690196}%
\pgfsetfillcolor{currentfill}%
\pgfsetfillopacity{0.400000}%
\pgfsetlinewidth{1.003750pt}%
\definecolor{currentstroke}{rgb}{1.000000,1.000000,1.000000}%
\pgfsetstrokecolor{currentstroke}%
\pgfsetstrokeopacity{0.400000}%
\pgfsetdash{}{0pt}%
\pgfpathmoveto{\pgfqpoint{2.318598in}{0.557870in}}%
\pgfpathlineto{\pgfqpoint{2.540969in}{0.557870in}}%
\pgfpathlineto{\pgfqpoint{2.540969in}{0.592500in}}%
\pgfpathlineto{\pgfqpoint{2.318598in}{0.592500in}}%
\pgfpathclose%
\pgfusepath{stroke,fill}%
\end{pgfscope}%
\begin{pgfscope}%
\pgfpathrectangle{\pgfqpoint{0.650810in}{0.557870in}}{\pgfqpoint{2.446088in}{1.684734in}}%
\pgfusepath{clip}%
\pgfsetbuttcap%
\pgfsetmiterjoin%
\definecolor{currentfill}{rgb}{0.298039,0.447059,0.690196}%
\pgfsetfillcolor{currentfill}%
\pgfsetfillopacity{0.400000}%
\pgfsetlinewidth{1.003750pt}%
\definecolor{currentstroke}{rgb}{1.000000,1.000000,1.000000}%
\pgfsetstrokecolor{currentstroke}%
\pgfsetstrokeopacity{0.400000}%
\pgfsetdash{}{0pt}%
\pgfpathmoveto{\pgfqpoint{2.540969in}{0.557870in}}%
\pgfpathlineto{\pgfqpoint{2.763341in}{0.557870in}}%
\pgfpathlineto{\pgfqpoint{2.763341in}{0.580957in}}%
\pgfpathlineto{\pgfqpoint{2.540969in}{0.580957in}}%
\pgfpathclose%
\pgfusepath{stroke,fill}%
\end{pgfscope}%
\begin{pgfscope}%
\pgfpathrectangle{\pgfqpoint{0.650810in}{0.557870in}}{\pgfqpoint{2.446088in}{1.684734in}}%
\pgfusepath{clip}%
\pgfsetbuttcap%
\pgfsetmiterjoin%
\definecolor{currentfill}{rgb}{0.298039,0.447059,0.690196}%
\pgfsetfillcolor{currentfill}%
\pgfsetfillopacity{0.400000}%
\pgfsetlinewidth{1.003750pt}%
\definecolor{currentstroke}{rgb}{1.000000,1.000000,1.000000}%
\pgfsetstrokecolor{currentstroke}%
\pgfsetstrokeopacity{0.400000}%
\pgfsetdash{}{0pt}%
\pgfpathmoveto{\pgfqpoint{2.763341in}{0.557870in}}%
\pgfpathlineto{\pgfqpoint{2.985712in}{0.557870in}}%
\pgfpathlineto{\pgfqpoint{2.985712in}{0.580957in}}%
\pgfpathlineto{\pgfqpoint{2.763341in}{0.580957in}}%
\pgfpathclose%
\pgfusepath{stroke,fill}%
\end{pgfscope}%
\begin{pgfscope}%
\pgfsetrectcap%
\pgfsetmiterjoin%
\pgfsetlinewidth{1.254687pt}%
\definecolor{currentstroke}{rgb}{1.000000,1.000000,1.000000}%
\pgfsetstrokecolor{currentstroke}%
\pgfsetdash{}{0pt}%
\pgfpathmoveto{\pgfqpoint{0.650810in}{0.557870in}}%
\pgfpathlineto{\pgfqpoint{0.650810in}{2.242604in}}%
\pgfusepath{stroke}%
\end{pgfscope}%
\begin{pgfscope}%
\pgfsetrectcap%
\pgfsetmiterjoin%
\pgfsetlinewidth{1.254687pt}%
\definecolor{currentstroke}{rgb}{1.000000,1.000000,1.000000}%
\pgfsetstrokecolor{currentstroke}%
\pgfsetdash{}{0pt}%
\pgfpathmoveto{\pgfqpoint{3.096898in}{0.557870in}}%
\pgfpathlineto{\pgfqpoint{3.096898in}{2.242604in}}%
\pgfusepath{stroke}%
\end{pgfscope}%
\begin{pgfscope}%
\pgfsetrectcap%
\pgfsetmiterjoin%
\pgfsetlinewidth{1.254687pt}%
\definecolor{currentstroke}{rgb}{1.000000,1.000000,1.000000}%
\pgfsetstrokecolor{currentstroke}%
\pgfsetdash{}{0pt}%
\pgfpathmoveto{\pgfqpoint{0.650810in}{0.557870in}}%
\pgfpathlineto{\pgfqpoint{3.096898in}{0.557870in}}%
\pgfusepath{stroke}%
\end{pgfscope}%
\begin{pgfscope}%
\pgfsetrectcap%
\pgfsetmiterjoin%
\pgfsetlinewidth{1.254687pt}%
\definecolor{currentstroke}{rgb}{1.000000,1.000000,1.000000}%
\pgfsetstrokecolor{currentstroke}%
\pgfsetdash{}{0pt}%
\pgfpathmoveto{\pgfqpoint{0.650810in}{2.242604in}}%
\pgfpathlineto{\pgfqpoint{3.096898in}{2.242604in}}%
\pgfusepath{stroke}%
\end{pgfscope}%
\begin{pgfscope}%
\definecolor{textcolor}{rgb}{0.150000,0.150000,0.150000}%
\pgfsetstrokecolor{textcolor}%
\pgfsetfillcolor{textcolor}%
\pgftext[x=1.873854in,y=2.325938in,,base]{\color{textcolor}\sffamily\fontsize{11.000000}{13.200000}\selectfont (a)}%
\end{pgfscope}%
\begin{pgfscope}%
\pgfsetbuttcap%
\pgfsetmiterjoin%
\definecolor{currentfill}{rgb}{0.917647,0.917647,0.949020}%
\pgfsetfillcolor{currentfill}%
\pgfsetlinewidth{0.000000pt}%
\definecolor{currentstroke}{rgb}{0.000000,0.000000,0.000000}%
\pgfsetstrokecolor{currentstroke}%
\pgfsetstrokeopacity{0.000000}%
\pgfsetdash{}{0pt}%
\pgfpathmoveto{\pgfqpoint{3.793912in}{0.557870in}}%
\pgfpathlineto{\pgfqpoint{6.240000in}{0.557870in}}%
\pgfpathlineto{\pgfqpoint{6.240000in}{2.242604in}}%
\pgfpathlineto{\pgfqpoint{3.793912in}{2.242604in}}%
\pgfpathclose%
\pgfusepath{fill}%
\end{pgfscope}%
\begin{pgfscope}%
\pgfpathrectangle{\pgfqpoint{3.793912in}{0.557870in}}{\pgfqpoint{2.446088in}{1.684734in}}%
\pgfusepath{clip}%
\pgfsetroundcap%
\pgfsetroundjoin%
\pgfsetlinewidth{1.003750pt}%
\definecolor{currentstroke}{rgb}{1.000000,1.000000,1.000000}%
\pgfsetstrokecolor{currentstroke}%
\pgfsetdash{}{0pt}%
\pgfpathmoveto{\pgfqpoint{3.905098in}{0.557870in}}%
\pgfpathlineto{\pgfqpoint{3.905098in}{2.242604in}}%
\pgfusepath{stroke}%
\end{pgfscope}%
\begin{pgfscope}%
\definecolor{textcolor}{rgb}{0.150000,0.150000,0.150000}%
\pgfsetstrokecolor{textcolor}%
\pgfsetfillcolor{textcolor}%
\pgftext[x=3.905098in,y=0.425926in,,top]{\color{textcolor}\sffamily\fontsize{11.000000}{13.200000}\selectfont \(\displaystyle 0.00\)}%
\end{pgfscope}%
\begin{pgfscope}%
\pgfpathrectangle{\pgfqpoint{3.793912in}{0.557870in}}{\pgfqpoint{2.446088in}{1.684734in}}%
\pgfusepath{clip}%
\pgfsetroundcap%
\pgfsetroundjoin%
\pgfsetlinewidth{1.003750pt}%
\definecolor{currentstroke}{rgb}{1.000000,1.000000,1.000000}%
\pgfsetstrokecolor{currentstroke}%
\pgfsetdash{}{0pt}%
\pgfpathmoveto{\pgfqpoint{4.461027in}{0.557870in}}%
\pgfpathlineto{\pgfqpoint{4.461027in}{2.242604in}}%
\pgfusepath{stroke}%
\end{pgfscope}%
\begin{pgfscope}%
\definecolor{textcolor}{rgb}{0.150000,0.150000,0.150000}%
\pgfsetstrokecolor{textcolor}%
\pgfsetfillcolor{textcolor}%
\pgftext[x=4.461027in,y=0.425926in,,top]{\color{textcolor}\sffamily\fontsize{11.000000}{13.200000}\selectfont \(\displaystyle 0.25\)}%
\end{pgfscope}%
\begin{pgfscope}%
\pgfpathrectangle{\pgfqpoint{3.793912in}{0.557870in}}{\pgfqpoint{2.446088in}{1.684734in}}%
\pgfusepath{clip}%
\pgfsetroundcap%
\pgfsetroundjoin%
\pgfsetlinewidth{1.003750pt}%
\definecolor{currentstroke}{rgb}{1.000000,1.000000,1.000000}%
\pgfsetstrokecolor{currentstroke}%
\pgfsetdash{}{0pt}%
\pgfpathmoveto{\pgfqpoint{5.016956in}{0.557870in}}%
\pgfpathlineto{\pgfqpoint{5.016956in}{2.242604in}}%
\pgfusepath{stroke}%
\end{pgfscope}%
\begin{pgfscope}%
\definecolor{textcolor}{rgb}{0.150000,0.150000,0.150000}%
\pgfsetstrokecolor{textcolor}%
\pgfsetfillcolor{textcolor}%
\pgftext[x=5.016956in,y=0.425926in,,top]{\color{textcolor}\sffamily\fontsize{11.000000}{13.200000}\selectfont \(\displaystyle 0.50\)}%
\end{pgfscope}%
\begin{pgfscope}%
\pgfpathrectangle{\pgfqpoint{3.793912in}{0.557870in}}{\pgfqpoint{2.446088in}{1.684734in}}%
\pgfusepath{clip}%
\pgfsetroundcap%
\pgfsetroundjoin%
\pgfsetlinewidth{1.003750pt}%
\definecolor{currentstroke}{rgb}{1.000000,1.000000,1.000000}%
\pgfsetstrokecolor{currentstroke}%
\pgfsetdash{}{0pt}%
\pgfpathmoveto{\pgfqpoint{5.572885in}{0.557870in}}%
\pgfpathlineto{\pgfqpoint{5.572885in}{2.242604in}}%
\pgfusepath{stroke}%
\end{pgfscope}%
\begin{pgfscope}%
\definecolor{textcolor}{rgb}{0.150000,0.150000,0.150000}%
\pgfsetstrokecolor{textcolor}%
\pgfsetfillcolor{textcolor}%
\pgftext[x=5.572885in,y=0.425926in,,top]{\color{textcolor}\sffamily\fontsize{11.000000}{13.200000}\selectfont \(\displaystyle 0.75\)}%
\end{pgfscope}%
\begin{pgfscope}%
\pgfpathrectangle{\pgfqpoint{3.793912in}{0.557870in}}{\pgfqpoint{2.446088in}{1.684734in}}%
\pgfusepath{clip}%
\pgfsetroundcap%
\pgfsetroundjoin%
\pgfsetlinewidth{1.003750pt}%
\definecolor{currentstroke}{rgb}{1.000000,1.000000,1.000000}%
\pgfsetstrokecolor{currentstroke}%
\pgfsetdash{}{0pt}%
\pgfpathmoveto{\pgfqpoint{6.128814in}{0.557870in}}%
\pgfpathlineto{\pgfqpoint{6.128814in}{2.242604in}}%
\pgfusepath{stroke}%
\end{pgfscope}%
\begin{pgfscope}%
\definecolor{textcolor}{rgb}{0.150000,0.150000,0.150000}%
\pgfsetstrokecolor{textcolor}%
\pgfsetfillcolor{textcolor}%
\pgftext[x=6.128814in,y=0.425926in,,top]{\color{textcolor}\sffamily\fontsize{11.000000}{13.200000}\selectfont \(\displaystyle 1.00\)}%
\end{pgfscope}%
\begin{pgfscope}%
\definecolor{textcolor}{rgb}{0.150000,0.150000,0.150000}%
\pgfsetstrokecolor{textcolor}%
\pgfsetfillcolor{textcolor}%
\pgftext[x=5.016956in,y=0.235185in,,top]{\color{textcolor}\sffamily\fontsize{11.000000}{13.200000}\selectfont Specificity}%
\end{pgfscope}%
\begin{pgfscope}%
\pgfpathrectangle{\pgfqpoint{3.793912in}{0.557870in}}{\pgfqpoint{2.446088in}{1.684734in}}%
\pgfusepath{clip}%
\pgfsetroundcap%
\pgfsetroundjoin%
\pgfsetlinewidth{1.003750pt}%
\definecolor{currentstroke}{rgb}{1.000000,1.000000,1.000000}%
\pgfsetstrokecolor{currentstroke}%
\pgfsetdash{}{0pt}%
\pgfpathmoveto{\pgfqpoint{3.793912in}{0.634449in}}%
\pgfpathlineto{\pgfqpoint{6.240000in}{0.634449in}}%
\pgfusepath{stroke}%
\end{pgfscope}%
\begin{pgfscope}%
\definecolor{textcolor}{rgb}{0.150000,0.150000,0.150000}%
\pgfsetstrokecolor{textcolor}%
\pgfsetfillcolor{textcolor}%
\pgftext[x=3.391597in,y=0.581642in,left,base]{\color{textcolor}\sffamily\fontsize{11.000000}{13.200000}\selectfont \(\displaystyle 0.00\)}%
\end{pgfscope}%
\begin{pgfscope}%
\pgfpathrectangle{\pgfqpoint{3.793912in}{0.557870in}}{\pgfqpoint{2.446088in}{1.684734in}}%
\pgfusepath{clip}%
\pgfsetroundcap%
\pgfsetroundjoin%
\pgfsetlinewidth{1.003750pt}%
\definecolor{currentstroke}{rgb}{1.000000,1.000000,1.000000}%
\pgfsetstrokecolor{currentstroke}%
\pgfsetdash{}{0pt}%
\pgfpathmoveto{\pgfqpoint{3.793912in}{1.017343in}}%
\pgfpathlineto{\pgfqpoint{6.240000in}{1.017343in}}%
\pgfusepath{stroke}%
\end{pgfscope}%
\begin{pgfscope}%
\definecolor{textcolor}{rgb}{0.150000,0.150000,0.150000}%
\pgfsetstrokecolor{textcolor}%
\pgfsetfillcolor{textcolor}%
\pgftext[x=3.391597in,y=0.964536in,left,base]{\color{textcolor}\sffamily\fontsize{11.000000}{13.200000}\selectfont \(\displaystyle 0.25\)}%
\end{pgfscope}%
\begin{pgfscope}%
\pgfpathrectangle{\pgfqpoint{3.793912in}{0.557870in}}{\pgfqpoint{2.446088in}{1.684734in}}%
\pgfusepath{clip}%
\pgfsetroundcap%
\pgfsetroundjoin%
\pgfsetlinewidth{1.003750pt}%
\definecolor{currentstroke}{rgb}{1.000000,1.000000,1.000000}%
\pgfsetstrokecolor{currentstroke}%
\pgfsetdash{}{0pt}%
\pgfpathmoveto{\pgfqpoint{3.793912in}{1.400237in}}%
\pgfpathlineto{\pgfqpoint{6.240000in}{1.400237in}}%
\pgfusepath{stroke}%
\end{pgfscope}%
\begin{pgfscope}%
\definecolor{textcolor}{rgb}{0.150000,0.150000,0.150000}%
\pgfsetstrokecolor{textcolor}%
\pgfsetfillcolor{textcolor}%
\pgftext[x=3.391597in,y=1.347431in,left,base]{\color{textcolor}\sffamily\fontsize{11.000000}{13.200000}\selectfont \(\displaystyle 0.50\)}%
\end{pgfscope}%
\begin{pgfscope}%
\pgfpathrectangle{\pgfqpoint{3.793912in}{0.557870in}}{\pgfqpoint{2.446088in}{1.684734in}}%
\pgfusepath{clip}%
\pgfsetroundcap%
\pgfsetroundjoin%
\pgfsetlinewidth{1.003750pt}%
\definecolor{currentstroke}{rgb}{1.000000,1.000000,1.000000}%
\pgfsetstrokecolor{currentstroke}%
\pgfsetdash{}{0pt}%
\pgfpathmoveto{\pgfqpoint{3.793912in}{1.783131in}}%
\pgfpathlineto{\pgfqpoint{6.240000in}{1.783131in}}%
\pgfusepath{stroke}%
\end{pgfscope}%
\begin{pgfscope}%
\definecolor{textcolor}{rgb}{0.150000,0.150000,0.150000}%
\pgfsetstrokecolor{textcolor}%
\pgfsetfillcolor{textcolor}%
\pgftext[x=3.391597in,y=1.730325in,left,base]{\color{textcolor}\sffamily\fontsize{11.000000}{13.200000}\selectfont \(\displaystyle 0.75\)}%
\end{pgfscope}%
\begin{pgfscope}%
\pgfpathrectangle{\pgfqpoint{3.793912in}{0.557870in}}{\pgfqpoint{2.446088in}{1.684734in}}%
\pgfusepath{clip}%
\pgfsetroundcap%
\pgfsetroundjoin%
\pgfsetlinewidth{1.003750pt}%
\definecolor{currentstroke}{rgb}{1.000000,1.000000,1.000000}%
\pgfsetstrokecolor{currentstroke}%
\pgfsetdash{}{0pt}%
\pgfpathmoveto{\pgfqpoint{3.793912in}{2.166025in}}%
\pgfpathlineto{\pgfqpoint{6.240000in}{2.166025in}}%
\pgfusepath{stroke}%
\end{pgfscope}%
\begin{pgfscope}%
\definecolor{textcolor}{rgb}{0.150000,0.150000,0.150000}%
\pgfsetstrokecolor{textcolor}%
\pgfsetfillcolor{textcolor}%
\pgftext[x=3.391597in,y=2.113219in,left,base]{\color{textcolor}\sffamily\fontsize{11.000000}{13.200000}\selectfont \(\displaystyle 1.00\)}%
\end{pgfscope}%
\begin{pgfscope}%
\definecolor{textcolor}{rgb}{0.150000,0.150000,0.150000}%
\pgfsetstrokecolor{textcolor}%
\pgfsetfillcolor{textcolor}%
\pgftext[x=3.336042in,y=1.400237in,,bottom,rotate=90.000000]{\color{textcolor}\sffamily\fontsize{11.000000}{13.200000}\selectfont Sensitivity}%
\end{pgfscope}%
\begin{pgfscope}%
\pgfpathrectangle{\pgfqpoint{3.793912in}{0.557870in}}{\pgfqpoint{2.446088in}{1.684734in}}%
\pgfusepath{clip}%
\pgfsetbuttcap%
\pgfsetroundjoin%
\definecolor{currentfill}{rgb}{0.298039,0.447059,0.690196}%
\pgfsetfillcolor{currentfill}%
\pgfsetlinewidth{1.003750pt}%
\definecolor{currentstroke}{rgb}{0.298039,0.447059,0.690196}%
\pgfsetstrokecolor{currentstroke}%
\pgfsetdash{}{0pt}%
\pgfpathmoveto{\pgfqpoint{5.601541in}{1.562562in}}%
\pgfpathcurveto{\pgfqpoint{5.609778in}{1.562562in}}{\pgfqpoint{5.617678in}{1.565834in}}{\pgfqpoint{5.623502in}{1.571658in}}%
\pgfpathcurveto{\pgfqpoint{5.629325in}{1.577482in}}{\pgfqpoint{5.632598in}{1.585382in}}{\pgfqpoint{5.632598in}{1.593618in}}%
\pgfpathcurveto{\pgfqpoint{5.632598in}{1.601854in}}{\pgfqpoint{5.629325in}{1.609754in}}{\pgfqpoint{5.623502in}{1.615578in}}%
\pgfpathcurveto{\pgfqpoint{5.617678in}{1.621402in}}{\pgfqpoint{5.609778in}{1.624675in}}{\pgfqpoint{5.601541in}{1.624675in}}%
\pgfpathcurveto{\pgfqpoint{5.593305in}{1.624675in}}{\pgfqpoint{5.585405in}{1.621402in}}{\pgfqpoint{5.579581in}{1.615578in}}%
\pgfpathcurveto{\pgfqpoint{5.573757in}{1.609754in}}{\pgfqpoint{5.570485in}{1.601854in}}{\pgfqpoint{5.570485in}{1.593618in}}%
\pgfpathcurveto{\pgfqpoint{5.570485in}{1.585382in}}{\pgfqpoint{5.573757in}{1.577482in}}{\pgfqpoint{5.579581in}{1.571658in}}%
\pgfpathcurveto{\pgfqpoint{5.585405in}{1.565834in}}{\pgfqpoint{5.593305in}{1.562562in}}{\pgfqpoint{5.601541in}{1.562562in}}%
\pgfpathclose%
\pgfusepath{stroke,fill}%
\end{pgfscope}%
\begin{pgfscope}%
\pgfpathrectangle{\pgfqpoint{3.793912in}{0.557870in}}{\pgfqpoint{2.446088in}{1.684734in}}%
\pgfusepath{clip}%
\pgfsetbuttcap%
\pgfsetroundjoin%
\definecolor{currentfill}{rgb}{0.298039,0.447059,0.690196}%
\pgfsetfillcolor{currentfill}%
\pgfsetlinewidth{1.003750pt}%
\definecolor{currentstroke}{rgb}{0.298039,0.447059,0.690196}%
\pgfsetstrokecolor{currentstroke}%
\pgfsetdash{}{0pt}%
\pgfpathmoveto{\pgfqpoint{5.601541in}{1.562562in}}%
\pgfpathcurveto{\pgfqpoint{5.609778in}{1.562562in}}{\pgfqpoint{5.617678in}{1.565834in}}{\pgfqpoint{5.623502in}{1.571658in}}%
\pgfpathcurveto{\pgfqpoint{5.629325in}{1.577482in}}{\pgfqpoint{5.632598in}{1.585382in}}{\pgfqpoint{5.632598in}{1.593618in}}%
\pgfpathcurveto{\pgfqpoint{5.632598in}{1.601854in}}{\pgfqpoint{5.629325in}{1.609754in}}{\pgfqpoint{5.623502in}{1.615578in}}%
\pgfpathcurveto{\pgfqpoint{5.617678in}{1.621402in}}{\pgfqpoint{5.609778in}{1.624675in}}{\pgfqpoint{5.601541in}{1.624675in}}%
\pgfpathcurveto{\pgfqpoint{5.593305in}{1.624675in}}{\pgfqpoint{5.585405in}{1.621402in}}{\pgfqpoint{5.579581in}{1.615578in}}%
\pgfpathcurveto{\pgfqpoint{5.573757in}{1.609754in}}{\pgfqpoint{5.570485in}{1.601854in}}{\pgfqpoint{5.570485in}{1.593618in}}%
\pgfpathcurveto{\pgfqpoint{5.570485in}{1.585382in}}{\pgfqpoint{5.573757in}{1.577482in}}{\pgfqpoint{5.579581in}{1.571658in}}%
\pgfpathcurveto{\pgfqpoint{5.585405in}{1.565834in}}{\pgfqpoint{5.593305in}{1.562562in}}{\pgfqpoint{5.601541in}{1.562562in}}%
\pgfpathclose%
\pgfusepath{stroke,fill}%
\end{pgfscope}%
\begin{pgfscope}%
\pgfpathrectangle{\pgfqpoint{3.793912in}{0.557870in}}{\pgfqpoint{2.446088in}{1.684734in}}%
\pgfusepath{clip}%
\pgfsetbuttcap%
\pgfsetroundjoin%
\definecolor{currentfill}{rgb}{0.298039,0.447059,0.690196}%
\pgfsetfillcolor{currentfill}%
\pgfsetlinewidth{1.003750pt}%
\definecolor{currentstroke}{rgb}{0.298039,0.447059,0.690196}%
\pgfsetstrokecolor{currentstroke}%
\pgfsetdash{}{0pt}%
\pgfpathmoveto{\pgfqpoint{5.532767in}{1.639914in}}%
\pgfpathcurveto{\pgfqpoint{5.541003in}{1.639914in}}{\pgfqpoint{5.548903in}{1.643186in}}{\pgfqpoint{5.554727in}{1.649010in}}%
\pgfpathcurveto{\pgfqpoint{5.560551in}{1.654834in}}{\pgfqpoint{5.563823in}{1.662734in}}{\pgfqpoint{5.563823in}{1.670970in}}%
\pgfpathcurveto{\pgfqpoint{5.563823in}{1.679207in}}{\pgfqpoint{5.560551in}{1.687107in}}{\pgfqpoint{5.554727in}{1.692931in}}%
\pgfpathcurveto{\pgfqpoint{5.548903in}{1.698755in}}{\pgfqpoint{5.541003in}{1.702027in}}{\pgfqpoint{5.532767in}{1.702027in}}%
\pgfpathcurveto{\pgfqpoint{5.524530in}{1.702027in}}{\pgfqpoint{5.516630in}{1.698755in}}{\pgfqpoint{5.510806in}{1.692931in}}%
\pgfpathcurveto{\pgfqpoint{5.504982in}{1.687107in}}{\pgfqpoint{5.501710in}{1.679207in}}{\pgfqpoint{5.501710in}{1.670970in}}%
\pgfpathcurveto{\pgfqpoint{5.501710in}{1.662734in}}{\pgfqpoint{5.504982in}{1.654834in}}{\pgfqpoint{5.510806in}{1.649010in}}%
\pgfpathcurveto{\pgfqpoint{5.516630in}{1.643186in}}{\pgfqpoint{5.524530in}{1.639914in}}{\pgfqpoint{5.532767in}{1.639914in}}%
\pgfpathclose%
\pgfusepath{stroke,fill}%
\end{pgfscope}%
\begin{pgfscope}%
\pgfpathrectangle{\pgfqpoint{3.793912in}{0.557870in}}{\pgfqpoint{2.446088in}{1.684734in}}%
\pgfusepath{clip}%
\pgfsetbuttcap%
\pgfsetroundjoin%
\definecolor{currentfill}{rgb}{0.298039,0.447059,0.690196}%
\pgfsetfillcolor{currentfill}%
\pgfsetlinewidth{1.003750pt}%
\definecolor{currentstroke}{rgb}{0.298039,0.447059,0.690196}%
\pgfsetstrokecolor{currentstroke}%
\pgfsetdash{}{0pt}%
\pgfpathmoveto{\pgfqpoint{5.441067in}{1.748207in}}%
\pgfpathcurveto{\pgfqpoint{5.449303in}{1.748207in}}{\pgfqpoint{5.457203in}{1.751479in}}{\pgfqpoint{5.463027in}{1.757303in}}%
\pgfpathcurveto{\pgfqpoint{5.468851in}{1.763127in}}{\pgfqpoint{5.472123in}{1.771027in}}{\pgfqpoint{5.472123in}{1.779264in}}%
\pgfpathcurveto{\pgfqpoint{5.472123in}{1.787500in}}{\pgfqpoint{5.468851in}{1.795400in}}{\pgfqpoint{5.463027in}{1.801224in}}%
\pgfpathcurveto{\pgfqpoint{5.457203in}{1.807048in}}{\pgfqpoint{5.449303in}{1.810320in}}{\pgfqpoint{5.441067in}{1.810320in}}%
\pgfpathcurveto{\pgfqpoint{5.432831in}{1.810320in}}{\pgfqpoint{5.424931in}{1.807048in}}{\pgfqpoint{5.419107in}{1.801224in}}%
\pgfpathcurveto{\pgfqpoint{5.413283in}{1.795400in}}{\pgfqpoint{5.410010in}{1.787500in}}{\pgfqpoint{5.410010in}{1.779264in}}%
\pgfpathcurveto{\pgfqpoint{5.410010in}{1.771027in}}{\pgfqpoint{5.413283in}{1.763127in}}{\pgfqpoint{5.419107in}{1.757303in}}%
\pgfpathcurveto{\pgfqpoint{5.424931in}{1.751479in}}{\pgfqpoint{5.432831in}{1.748207in}}{\pgfqpoint{5.441067in}{1.748207in}}%
\pgfpathclose%
\pgfusepath{stroke,fill}%
\end{pgfscope}%
\begin{pgfscope}%
\pgfpathrectangle{\pgfqpoint{3.793912in}{0.557870in}}{\pgfqpoint{2.446088in}{1.684734in}}%
\pgfusepath{clip}%
\pgfsetbuttcap%
\pgfsetroundjoin%
\definecolor{currentfill}{rgb}{0.298039,0.447059,0.690196}%
\pgfsetfillcolor{currentfill}%
\pgfsetlinewidth{1.003750pt}%
\definecolor{currentstroke}{rgb}{0.298039,0.447059,0.690196}%
\pgfsetstrokecolor{currentstroke}%
\pgfsetdash{}{0pt}%
\pgfpathmoveto{\pgfqpoint{6.037115in}{0.928272in}}%
\pgfpathcurveto{\pgfqpoint{6.045351in}{0.928272in}}{\pgfqpoint{6.053251in}{0.931545in}}{\pgfqpoint{6.059075in}{0.937369in}}%
\pgfpathcurveto{\pgfqpoint{6.064899in}{0.943193in}}{\pgfqpoint{6.068171in}{0.951093in}}{\pgfqpoint{6.068171in}{0.959329in}}%
\pgfpathcurveto{\pgfqpoint{6.068171in}{0.967565in}}{\pgfqpoint{6.064899in}{0.975465in}}{\pgfqpoint{6.059075in}{0.981289in}}%
\pgfpathcurveto{\pgfqpoint{6.053251in}{0.987113in}}{\pgfqpoint{6.045351in}{0.990385in}}{\pgfqpoint{6.037115in}{0.990385in}}%
\pgfpathcurveto{\pgfqpoint{6.028878in}{0.990385in}}{\pgfqpoint{6.020978in}{0.987113in}}{\pgfqpoint{6.015154in}{0.981289in}}%
\pgfpathcurveto{\pgfqpoint{6.009330in}{0.975465in}}{\pgfqpoint{6.006058in}{0.967565in}}{\pgfqpoint{6.006058in}{0.959329in}}%
\pgfpathcurveto{\pgfqpoint{6.006058in}{0.951093in}}{\pgfqpoint{6.009330in}{0.943193in}}{\pgfqpoint{6.015154in}{0.937369in}}%
\pgfpathcurveto{\pgfqpoint{6.020978in}{0.931545in}}{\pgfqpoint{6.028878in}{0.928272in}}{\pgfqpoint{6.037115in}{0.928272in}}%
\pgfpathclose%
\pgfusepath{stroke,fill}%
\end{pgfscope}%
\begin{pgfscope}%
\pgfpathrectangle{\pgfqpoint{3.793912in}{0.557870in}}{\pgfqpoint{2.446088in}{1.684734in}}%
\pgfusepath{clip}%
\pgfsetbuttcap%
\pgfsetroundjoin%
\definecolor{currentfill}{rgb}{0.298039,0.447059,0.690196}%
\pgfsetfillcolor{currentfill}%
\pgfsetlinewidth{1.003750pt}%
\definecolor{currentstroke}{rgb}{0.298039,0.447059,0.690196}%
\pgfsetstrokecolor{currentstroke}%
\pgfsetdash{}{0pt}%
\pgfpathmoveto{\pgfqpoint{4.913794in}{1.964794in}}%
\pgfpathcurveto{\pgfqpoint{4.922030in}{1.964794in}}{\pgfqpoint{4.929930in}{1.968066in}}{\pgfqpoint{4.935754in}{1.973890in}}%
\pgfpathcurveto{\pgfqpoint{4.941578in}{1.979714in}}{\pgfqpoint{4.944851in}{1.987614in}}{\pgfqpoint{4.944851in}{1.995850in}}%
\pgfpathcurveto{\pgfqpoint{4.944851in}{2.004086in}}{\pgfqpoint{4.941578in}{2.011987in}}{\pgfqpoint{4.935754in}{2.017810in}}%
\pgfpathcurveto{\pgfqpoint{4.929930in}{2.023634in}}{\pgfqpoint{4.922030in}{2.026907in}}{\pgfqpoint{4.913794in}{2.026907in}}%
\pgfpathcurveto{\pgfqpoint{4.905558in}{2.026907in}}{\pgfqpoint{4.897658in}{2.023634in}}{\pgfqpoint{4.891834in}{2.017810in}}%
\pgfpathcurveto{\pgfqpoint{4.886010in}{2.011987in}}{\pgfqpoint{4.882738in}{2.004086in}}{\pgfqpoint{4.882738in}{1.995850in}}%
\pgfpathcurveto{\pgfqpoint{4.882738in}{1.987614in}}{\pgfqpoint{4.886010in}{1.979714in}}{\pgfqpoint{4.891834in}{1.973890in}}%
\pgfpathcurveto{\pgfqpoint{4.897658in}{1.968066in}}{\pgfqpoint{4.905558in}{1.964794in}}{\pgfqpoint{4.913794in}{1.964794in}}%
\pgfpathclose%
\pgfusepath{stroke,fill}%
\end{pgfscope}%
\begin{pgfscope}%
\pgfpathrectangle{\pgfqpoint{3.793912in}{0.557870in}}{\pgfqpoint{2.446088in}{1.684734in}}%
\pgfusepath{clip}%
\pgfsetbuttcap%
\pgfsetroundjoin%
\definecolor{currentfill}{rgb}{0.298039,0.447059,0.690196}%
\pgfsetfillcolor{currentfill}%
\pgfsetlinewidth{1.003750pt}%
\definecolor{currentstroke}{rgb}{0.298039,0.447059,0.690196}%
\pgfsetstrokecolor{currentstroke}%
\pgfsetdash{}{0pt}%
\pgfpathmoveto{\pgfqpoint{3.905098in}{2.119498in}}%
\pgfpathcurveto{\pgfqpoint{3.913334in}{2.119498in}}{\pgfqpoint{3.921234in}{2.122771in}}{\pgfqpoint{3.927058in}{2.128595in}}%
\pgfpathcurveto{\pgfqpoint{3.932882in}{2.134419in}}{\pgfqpoint{3.936155in}{2.142319in}}{\pgfqpoint{3.936155in}{2.150555in}}%
\pgfpathcurveto{\pgfqpoint{3.936155in}{2.158791in}}{\pgfqpoint{3.932882in}{2.166691in}}{\pgfqpoint{3.927058in}{2.172515in}}%
\pgfpathcurveto{\pgfqpoint{3.921234in}{2.178339in}}{\pgfqpoint{3.913334in}{2.181611in}}{\pgfqpoint{3.905098in}{2.181611in}}%
\pgfpathcurveto{\pgfqpoint{3.896862in}{2.181611in}}{\pgfqpoint{3.888962in}{2.178339in}}{\pgfqpoint{3.883138in}{2.172515in}}%
\pgfpathcurveto{\pgfqpoint{3.877314in}{2.166691in}}{\pgfqpoint{3.874042in}{2.158791in}}{\pgfqpoint{3.874042in}{2.150555in}}%
\pgfpathcurveto{\pgfqpoint{3.874042in}{2.142319in}}{\pgfqpoint{3.877314in}{2.134419in}}{\pgfqpoint{3.883138in}{2.128595in}}%
\pgfpathcurveto{\pgfqpoint{3.888962in}{2.122771in}}{\pgfqpoint{3.896862in}{2.119498in}}{\pgfqpoint{3.905098in}{2.119498in}}%
\pgfpathclose%
\pgfusepath{stroke,fill}%
\end{pgfscope}%
\begin{pgfscope}%
\pgfpathrectangle{\pgfqpoint{3.793912in}{0.557870in}}{\pgfqpoint{2.446088in}{1.684734in}}%
\pgfusepath{clip}%
\pgfsetbuttcap%
\pgfsetroundjoin%
\definecolor{currentfill}{rgb}{0.298039,0.447059,0.690196}%
\pgfsetfillcolor{currentfill}%
\pgfsetlinewidth{1.003750pt}%
\definecolor{currentstroke}{rgb}{0.298039,0.447059,0.690196}%
\pgfsetstrokecolor{currentstroke}%
\pgfsetdash{}{0pt}%
\pgfpathmoveto{\pgfqpoint{4.248972in}{1.902912in}}%
\pgfpathcurveto{\pgfqpoint{4.257208in}{1.902912in}}{\pgfqpoint{4.265108in}{1.906184in}}{\pgfqpoint{4.270932in}{1.912008in}}%
\pgfpathcurveto{\pgfqpoint{4.276756in}{1.917832in}}{\pgfqpoint{4.280028in}{1.925732in}}{\pgfqpoint{4.280028in}{1.933968in}}%
\pgfpathcurveto{\pgfqpoint{4.280028in}{1.942205in}}{\pgfqpoint{4.276756in}{1.950105in}}{\pgfqpoint{4.270932in}{1.955929in}}%
\pgfpathcurveto{\pgfqpoint{4.265108in}{1.961753in}}{\pgfqpoint{4.257208in}{1.965025in}}{\pgfqpoint{4.248972in}{1.965025in}}%
\pgfpathcurveto{\pgfqpoint{4.240735in}{1.965025in}}{\pgfqpoint{4.232835in}{1.961753in}}{\pgfqpoint{4.227011in}{1.955929in}}%
\pgfpathcurveto{\pgfqpoint{4.221187in}{1.950105in}}{\pgfqpoint{4.217915in}{1.942205in}}{\pgfqpoint{4.217915in}{1.933968in}}%
\pgfpathcurveto{\pgfqpoint{4.217915in}{1.925732in}}{\pgfqpoint{4.221187in}{1.917832in}}{\pgfqpoint{4.227011in}{1.912008in}}%
\pgfpathcurveto{\pgfqpoint{4.232835in}{1.906184in}}{\pgfqpoint{4.240735in}{1.902912in}}{\pgfqpoint{4.248972in}{1.902912in}}%
\pgfpathclose%
\pgfusepath{stroke,fill}%
\end{pgfscope}%
\begin{pgfscope}%
\pgfpathrectangle{\pgfqpoint{3.793912in}{0.557870in}}{\pgfqpoint{2.446088in}{1.684734in}}%
\pgfusepath{clip}%
\pgfsetbuttcap%
\pgfsetroundjoin%
\definecolor{currentfill}{rgb}{0.298039,0.447059,0.690196}%
\pgfsetfillcolor{currentfill}%
\pgfsetlinewidth{1.003750pt}%
\definecolor{currentstroke}{rgb}{0.298039,0.447059,0.690196}%
\pgfsetstrokecolor{currentstroke}%
\pgfsetdash{}{0pt}%
\pgfpathmoveto{\pgfqpoint{3.905098in}{2.119498in}}%
\pgfpathcurveto{\pgfqpoint{3.913334in}{2.119498in}}{\pgfqpoint{3.921234in}{2.122771in}}{\pgfqpoint{3.927058in}{2.128595in}}%
\pgfpathcurveto{\pgfqpoint{3.932882in}{2.134419in}}{\pgfqpoint{3.936155in}{2.142319in}}{\pgfqpoint{3.936155in}{2.150555in}}%
\pgfpathcurveto{\pgfqpoint{3.936155in}{2.158791in}}{\pgfqpoint{3.932882in}{2.166691in}}{\pgfqpoint{3.927058in}{2.172515in}}%
\pgfpathcurveto{\pgfqpoint{3.921234in}{2.178339in}}{\pgfqpoint{3.913334in}{2.181611in}}{\pgfqpoint{3.905098in}{2.181611in}}%
\pgfpathcurveto{\pgfqpoint{3.896862in}{2.181611in}}{\pgfqpoint{3.888962in}{2.178339in}}{\pgfqpoint{3.883138in}{2.172515in}}%
\pgfpathcurveto{\pgfqpoint{3.877314in}{2.166691in}}{\pgfqpoint{3.874042in}{2.158791in}}{\pgfqpoint{3.874042in}{2.150555in}}%
\pgfpathcurveto{\pgfqpoint{3.874042in}{2.142319in}}{\pgfqpoint{3.877314in}{2.134419in}}{\pgfqpoint{3.883138in}{2.128595in}}%
\pgfpathcurveto{\pgfqpoint{3.888962in}{2.122771in}}{\pgfqpoint{3.896862in}{2.119498in}}{\pgfqpoint{3.905098in}{2.119498in}}%
\pgfpathclose%
\pgfusepath{stroke,fill}%
\end{pgfscope}%
\begin{pgfscope}%
\pgfpathrectangle{\pgfqpoint{3.793912in}{0.557870in}}{\pgfqpoint{2.446088in}{1.684734in}}%
\pgfusepath{clip}%
\pgfsetbuttcap%
\pgfsetroundjoin%
\definecolor{currentfill}{rgb}{0.298039,0.447059,0.690196}%
\pgfsetfillcolor{currentfill}%
\pgfsetlinewidth{1.003750pt}%
\definecolor{currentstroke}{rgb}{0.298039,0.447059,0.690196}%
\pgfsetstrokecolor{currentstroke}%
\pgfsetdash{}{0pt}%
\pgfpathmoveto{\pgfqpoint{5.441067in}{1.175800in}}%
\pgfpathcurveto{\pgfqpoint{5.449303in}{1.175800in}}{\pgfqpoint{5.457203in}{1.179072in}}{\pgfqpoint{5.463027in}{1.184896in}}%
\pgfpathcurveto{\pgfqpoint{5.468851in}{1.190720in}}{\pgfqpoint{5.472123in}{1.198620in}}{\pgfqpoint{5.472123in}{1.206856in}}%
\pgfpathcurveto{\pgfqpoint{5.472123in}{1.215093in}}{\pgfqpoint{5.468851in}{1.222993in}}{\pgfqpoint{5.463027in}{1.228817in}}%
\pgfpathcurveto{\pgfqpoint{5.457203in}{1.234641in}}{\pgfqpoint{5.449303in}{1.237913in}}{\pgfqpoint{5.441067in}{1.237913in}}%
\pgfpathcurveto{\pgfqpoint{5.432831in}{1.237913in}}{\pgfqpoint{5.424931in}{1.234641in}}{\pgfqpoint{5.419107in}{1.228817in}}%
\pgfpathcurveto{\pgfqpoint{5.413283in}{1.222993in}}{\pgfqpoint{5.410010in}{1.215093in}}{\pgfqpoint{5.410010in}{1.206856in}}%
\pgfpathcurveto{\pgfqpoint{5.410010in}{1.198620in}}{\pgfqpoint{5.413283in}{1.190720in}}{\pgfqpoint{5.419107in}{1.184896in}}%
\pgfpathcurveto{\pgfqpoint{5.424931in}{1.179072in}}{\pgfqpoint{5.432831in}{1.175800in}}{\pgfqpoint{5.441067in}{1.175800in}}%
\pgfpathclose%
\pgfusepath{stroke,fill}%
\end{pgfscope}%
\begin{pgfscope}%
\pgfpathrectangle{\pgfqpoint{3.793912in}{0.557870in}}{\pgfqpoint{2.446088in}{1.684734in}}%
\pgfusepath{clip}%
\pgfsetbuttcap%
\pgfsetroundjoin%
\definecolor{currentfill}{rgb}{0.298039,0.447059,0.690196}%
\pgfsetfillcolor{currentfill}%
\pgfsetlinewidth{1.003750pt}%
\definecolor{currentstroke}{rgb}{0.298039,0.447059,0.690196}%
\pgfsetstrokecolor{currentstroke}%
\pgfsetdash{}{0pt}%
\pgfpathmoveto{\pgfqpoint{3.905098in}{2.119498in}}%
\pgfpathcurveto{\pgfqpoint{3.913334in}{2.119498in}}{\pgfqpoint{3.921234in}{2.122771in}}{\pgfqpoint{3.927058in}{2.128595in}}%
\pgfpathcurveto{\pgfqpoint{3.932882in}{2.134419in}}{\pgfqpoint{3.936155in}{2.142319in}}{\pgfqpoint{3.936155in}{2.150555in}}%
\pgfpathcurveto{\pgfqpoint{3.936155in}{2.158791in}}{\pgfqpoint{3.932882in}{2.166691in}}{\pgfqpoint{3.927058in}{2.172515in}}%
\pgfpathcurveto{\pgfqpoint{3.921234in}{2.178339in}}{\pgfqpoint{3.913334in}{2.181611in}}{\pgfqpoint{3.905098in}{2.181611in}}%
\pgfpathcurveto{\pgfqpoint{3.896862in}{2.181611in}}{\pgfqpoint{3.888962in}{2.178339in}}{\pgfqpoint{3.883138in}{2.172515in}}%
\pgfpathcurveto{\pgfqpoint{3.877314in}{2.166691in}}{\pgfqpoint{3.874042in}{2.158791in}}{\pgfqpoint{3.874042in}{2.150555in}}%
\pgfpathcurveto{\pgfqpoint{3.874042in}{2.142319in}}{\pgfqpoint{3.877314in}{2.134419in}}{\pgfqpoint{3.883138in}{2.128595in}}%
\pgfpathcurveto{\pgfqpoint{3.888962in}{2.122771in}}{\pgfqpoint{3.896862in}{2.119498in}}{\pgfqpoint{3.905098in}{2.119498in}}%
\pgfpathclose%
\pgfusepath{stroke,fill}%
\end{pgfscope}%
\begin{pgfscope}%
\pgfpathrectangle{\pgfqpoint{3.793912in}{0.557870in}}{\pgfqpoint{2.446088in}{1.684734in}}%
\pgfusepath{clip}%
\pgfsetbuttcap%
\pgfsetroundjoin%
\definecolor{currentfill}{rgb}{0.298039,0.447059,0.690196}%
\pgfsetfillcolor{currentfill}%
\pgfsetlinewidth{1.003750pt}%
\definecolor{currentstroke}{rgb}{0.298039,0.447059,0.690196}%
\pgfsetstrokecolor{currentstroke}%
\pgfsetdash{}{0pt}%
\pgfpathmoveto{\pgfqpoint{3.905098in}{2.119498in}}%
\pgfpathcurveto{\pgfqpoint{3.913334in}{2.119498in}}{\pgfqpoint{3.921234in}{2.122771in}}{\pgfqpoint{3.927058in}{2.128595in}}%
\pgfpathcurveto{\pgfqpoint{3.932882in}{2.134419in}}{\pgfqpoint{3.936155in}{2.142319in}}{\pgfqpoint{3.936155in}{2.150555in}}%
\pgfpathcurveto{\pgfqpoint{3.936155in}{2.158791in}}{\pgfqpoint{3.932882in}{2.166691in}}{\pgfqpoint{3.927058in}{2.172515in}}%
\pgfpathcurveto{\pgfqpoint{3.921234in}{2.178339in}}{\pgfqpoint{3.913334in}{2.181611in}}{\pgfqpoint{3.905098in}{2.181611in}}%
\pgfpathcurveto{\pgfqpoint{3.896862in}{2.181611in}}{\pgfqpoint{3.888962in}{2.178339in}}{\pgfqpoint{3.883138in}{2.172515in}}%
\pgfpathcurveto{\pgfqpoint{3.877314in}{2.166691in}}{\pgfqpoint{3.874042in}{2.158791in}}{\pgfqpoint{3.874042in}{2.150555in}}%
\pgfpathcurveto{\pgfqpoint{3.874042in}{2.142319in}}{\pgfqpoint{3.877314in}{2.134419in}}{\pgfqpoint{3.883138in}{2.128595in}}%
\pgfpathcurveto{\pgfqpoint{3.888962in}{2.122771in}}{\pgfqpoint{3.896862in}{2.119498in}}{\pgfqpoint{3.905098in}{2.119498in}}%
\pgfpathclose%
\pgfusepath{stroke,fill}%
\end{pgfscope}%
\begin{pgfscope}%
\pgfpathrectangle{\pgfqpoint{3.793912in}{0.557870in}}{\pgfqpoint{2.446088in}{1.684734in}}%
\pgfusepath{clip}%
\pgfsetbuttcap%
\pgfsetroundjoin%
\definecolor{currentfill}{rgb}{0.298039,0.447059,0.690196}%
\pgfsetfillcolor{currentfill}%
\pgfsetlinewidth{1.003750pt}%
\definecolor{currentstroke}{rgb}{0.298039,0.447059,0.690196}%
\pgfsetstrokecolor{currentstroke}%
\pgfsetdash{}{0pt}%
\pgfpathmoveto{\pgfqpoint{3.905098in}{2.119498in}}%
\pgfpathcurveto{\pgfqpoint{3.913334in}{2.119498in}}{\pgfqpoint{3.921234in}{2.122771in}}{\pgfqpoint{3.927058in}{2.128595in}}%
\pgfpathcurveto{\pgfqpoint{3.932882in}{2.134419in}}{\pgfqpoint{3.936155in}{2.142319in}}{\pgfqpoint{3.936155in}{2.150555in}}%
\pgfpathcurveto{\pgfqpoint{3.936155in}{2.158791in}}{\pgfqpoint{3.932882in}{2.166691in}}{\pgfqpoint{3.927058in}{2.172515in}}%
\pgfpathcurveto{\pgfqpoint{3.921234in}{2.178339in}}{\pgfqpoint{3.913334in}{2.181611in}}{\pgfqpoint{3.905098in}{2.181611in}}%
\pgfpathcurveto{\pgfqpoint{3.896862in}{2.181611in}}{\pgfqpoint{3.888962in}{2.178339in}}{\pgfqpoint{3.883138in}{2.172515in}}%
\pgfpathcurveto{\pgfqpoint{3.877314in}{2.166691in}}{\pgfqpoint{3.874042in}{2.158791in}}{\pgfqpoint{3.874042in}{2.150555in}}%
\pgfpathcurveto{\pgfqpoint{3.874042in}{2.142319in}}{\pgfqpoint{3.877314in}{2.134419in}}{\pgfqpoint{3.883138in}{2.128595in}}%
\pgfpathcurveto{\pgfqpoint{3.888962in}{2.122771in}}{\pgfqpoint{3.896862in}{2.119498in}}{\pgfqpoint{3.905098in}{2.119498in}}%
\pgfpathclose%
\pgfusepath{stroke,fill}%
\end{pgfscope}%
\begin{pgfscope}%
\pgfpathrectangle{\pgfqpoint{3.793912in}{0.557870in}}{\pgfqpoint{2.446088in}{1.684734in}}%
\pgfusepath{clip}%
\pgfsetbuttcap%
\pgfsetroundjoin%
\definecolor{currentfill}{rgb}{0.298039,0.447059,0.690196}%
\pgfsetfillcolor{currentfill}%
\pgfsetlinewidth{1.003750pt}%
\definecolor{currentstroke}{rgb}{0.298039,0.447059,0.690196}%
\pgfsetstrokecolor{currentstroke}%
\pgfsetdash{}{0pt}%
\pgfpathmoveto{\pgfqpoint{3.905098in}{2.119498in}}%
\pgfpathcurveto{\pgfqpoint{3.913334in}{2.119498in}}{\pgfqpoint{3.921234in}{2.122771in}}{\pgfqpoint{3.927058in}{2.128595in}}%
\pgfpathcurveto{\pgfqpoint{3.932882in}{2.134419in}}{\pgfqpoint{3.936155in}{2.142319in}}{\pgfqpoint{3.936155in}{2.150555in}}%
\pgfpathcurveto{\pgfqpoint{3.936155in}{2.158791in}}{\pgfqpoint{3.932882in}{2.166691in}}{\pgfqpoint{3.927058in}{2.172515in}}%
\pgfpathcurveto{\pgfqpoint{3.921234in}{2.178339in}}{\pgfqpoint{3.913334in}{2.181611in}}{\pgfqpoint{3.905098in}{2.181611in}}%
\pgfpathcurveto{\pgfqpoint{3.896862in}{2.181611in}}{\pgfqpoint{3.888962in}{2.178339in}}{\pgfqpoint{3.883138in}{2.172515in}}%
\pgfpathcurveto{\pgfqpoint{3.877314in}{2.166691in}}{\pgfqpoint{3.874042in}{2.158791in}}{\pgfqpoint{3.874042in}{2.150555in}}%
\pgfpathcurveto{\pgfqpoint{3.874042in}{2.142319in}}{\pgfqpoint{3.877314in}{2.134419in}}{\pgfqpoint{3.883138in}{2.128595in}}%
\pgfpathcurveto{\pgfqpoint{3.888962in}{2.122771in}}{\pgfqpoint{3.896862in}{2.119498in}}{\pgfqpoint{3.905098in}{2.119498in}}%
\pgfpathclose%
\pgfusepath{stroke,fill}%
\end{pgfscope}%
\begin{pgfscope}%
\pgfpathrectangle{\pgfqpoint{3.793912in}{0.557870in}}{\pgfqpoint{2.446088in}{1.684734in}}%
\pgfusepath{clip}%
\pgfsetbuttcap%
\pgfsetroundjoin%
\definecolor{currentfill}{rgb}{0.298039,0.447059,0.690196}%
\pgfsetfillcolor{currentfill}%
\pgfsetlinewidth{1.003750pt}%
\definecolor{currentstroke}{rgb}{0.298039,0.447059,0.690196}%
\pgfsetstrokecolor{currentstroke}%
\pgfsetdash{}{0pt}%
\pgfpathmoveto{\pgfqpoint{4.707470in}{1.593502in}}%
\pgfpathcurveto{\pgfqpoint{4.715706in}{1.593502in}}{\pgfqpoint{4.723606in}{1.596775in}}{\pgfqpoint{4.729430in}{1.602599in}}%
\pgfpathcurveto{\pgfqpoint{4.735254in}{1.608423in}}{\pgfqpoint{4.738526in}{1.616323in}}{\pgfqpoint{4.738526in}{1.624559in}}%
\pgfpathcurveto{\pgfqpoint{4.738526in}{1.632795in}}{\pgfqpoint{4.735254in}{1.640695in}}{\pgfqpoint{4.729430in}{1.646519in}}%
\pgfpathcurveto{\pgfqpoint{4.723606in}{1.652343in}}{\pgfqpoint{4.715706in}{1.655615in}}{\pgfqpoint{4.707470in}{1.655615in}}%
\pgfpathcurveto{\pgfqpoint{4.699234in}{1.655615in}}{\pgfqpoint{4.691334in}{1.652343in}}{\pgfqpoint{4.685510in}{1.646519in}}%
\pgfpathcurveto{\pgfqpoint{4.679686in}{1.640695in}}{\pgfqpoint{4.676413in}{1.632795in}}{\pgfqpoint{4.676413in}{1.624559in}}%
\pgfpathcurveto{\pgfqpoint{4.676413in}{1.616323in}}{\pgfqpoint{4.679686in}{1.608423in}}{\pgfqpoint{4.685510in}{1.602599in}}%
\pgfpathcurveto{\pgfqpoint{4.691334in}{1.596775in}}{\pgfqpoint{4.699234in}{1.593502in}}{\pgfqpoint{4.707470in}{1.593502in}}%
\pgfpathclose%
\pgfusepath{stroke,fill}%
\end{pgfscope}%
\begin{pgfscope}%
\pgfpathrectangle{\pgfqpoint{3.793912in}{0.557870in}}{\pgfqpoint{2.446088in}{1.684734in}}%
\pgfusepath{clip}%
\pgfsetbuttcap%
\pgfsetroundjoin%
\definecolor{currentfill}{rgb}{0.298039,0.447059,0.690196}%
\pgfsetfillcolor{currentfill}%
\pgfsetlinewidth{1.003750pt}%
\definecolor{currentstroke}{rgb}{0.298039,0.447059,0.690196}%
\pgfsetstrokecolor{currentstroke}%
\pgfsetdash{}{0pt}%
\pgfpathmoveto{\pgfqpoint{3.905098in}{2.119498in}}%
\pgfpathcurveto{\pgfqpoint{3.913334in}{2.119498in}}{\pgfqpoint{3.921234in}{2.122771in}}{\pgfqpoint{3.927058in}{2.128595in}}%
\pgfpathcurveto{\pgfqpoint{3.932882in}{2.134419in}}{\pgfqpoint{3.936155in}{2.142319in}}{\pgfqpoint{3.936155in}{2.150555in}}%
\pgfpathcurveto{\pgfqpoint{3.936155in}{2.158791in}}{\pgfqpoint{3.932882in}{2.166691in}}{\pgfqpoint{3.927058in}{2.172515in}}%
\pgfpathcurveto{\pgfqpoint{3.921234in}{2.178339in}}{\pgfqpoint{3.913334in}{2.181611in}}{\pgfqpoint{3.905098in}{2.181611in}}%
\pgfpathcurveto{\pgfqpoint{3.896862in}{2.181611in}}{\pgfqpoint{3.888962in}{2.178339in}}{\pgfqpoint{3.883138in}{2.172515in}}%
\pgfpathcurveto{\pgfqpoint{3.877314in}{2.166691in}}{\pgfqpoint{3.874042in}{2.158791in}}{\pgfqpoint{3.874042in}{2.150555in}}%
\pgfpathcurveto{\pgfqpoint{3.874042in}{2.142319in}}{\pgfqpoint{3.877314in}{2.134419in}}{\pgfqpoint{3.883138in}{2.128595in}}%
\pgfpathcurveto{\pgfqpoint{3.888962in}{2.122771in}}{\pgfqpoint{3.896862in}{2.119498in}}{\pgfqpoint{3.905098in}{2.119498in}}%
\pgfpathclose%
\pgfusepath{stroke,fill}%
\end{pgfscope}%
\begin{pgfscope}%
\pgfpathrectangle{\pgfqpoint{3.793912in}{0.557870in}}{\pgfqpoint{2.446088in}{1.684734in}}%
\pgfusepath{clip}%
\pgfsetbuttcap%
\pgfsetroundjoin%
\definecolor{currentfill}{rgb}{0.298039,0.447059,0.690196}%
\pgfsetfillcolor{currentfill}%
\pgfsetlinewidth{1.003750pt}%
\definecolor{currentstroke}{rgb}{0.298039,0.447059,0.690196}%
\pgfsetstrokecolor{currentstroke}%
\pgfsetdash{}{0pt}%
\pgfpathmoveto{\pgfqpoint{5.165968in}{1.438798in}}%
\pgfpathcurveto{\pgfqpoint{5.174204in}{1.438798in}}{\pgfqpoint{5.182104in}{1.442070in}}{\pgfqpoint{5.187928in}{1.447894in}}%
\pgfpathcurveto{\pgfqpoint{5.193752in}{1.453718in}}{\pgfqpoint{5.197025in}{1.461618in}}{\pgfqpoint{5.197025in}{1.469854in}}%
\pgfpathcurveto{\pgfqpoint{5.197025in}{1.478091in}}{\pgfqpoint{5.193752in}{1.485991in}}{\pgfqpoint{5.187928in}{1.491815in}}%
\pgfpathcurveto{\pgfqpoint{5.182104in}{1.497639in}}{\pgfqpoint{5.174204in}{1.500911in}}{\pgfqpoint{5.165968in}{1.500911in}}%
\pgfpathcurveto{\pgfqpoint{5.157732in}{1.500911in}}{\pgfqpoint{5.149832in}{1.497639in}}{\pgfqpoint{5.144008in}{1.491815in}}%
\pgfpathcurveto{\pgfqpoint{5.138184in}{1.485991in}}{\pgfqpoint{5.134912in}{1.478091in}}{\pgfqpoint{5.134912in}{1.469854in}}%
\pgfpathcurveto{\pgfqpoint{5.134912in}{1.461618in}}{\pgfqpoint{5.138184in}{1.453718in}}{\pgfqpoint{5.144008in}{1.447894in}}%
\pgfpathcurveto{\pgfqpoint{5.149832in}{1.442070in}}{\pgfqpoint{5.157732in}{1.438798in}}{\pgfqpoint{5.165968in}{1.438798in}}%
\pgfpathclose%
\pgfusepath{stroke,fill}%
\end{pgfscope}%
\begin{pgfscope}%
\pgfpathrectangle{\pgfqpoint{3.793912in}{0.557870in}}{\pgfqpoint{2.446088in}{1.684734in}}%
\pgfusepath{clip}%
\pgfsetbuttcap%
\pgfsetroundjoin%
\definecolor{currentfill}{rgb}{0.298039,0.447059,0.690196}%
\pgfsetfillcolor{currentfill}%
\pgfsetlinewidth{1.003750pt}%
\definecolor{currentstroke}{rgb}{0.298039,0.447059,0.690196}%
\pgfsetstrokecolor{currentstroke}%
\pgfsetdash{}{0pt}%
\pgfpathmoveto{\pgfqpoint{3.905098in}{2.119498in}}%
\pgfpathcurveto{\pgfqpoint{3.913334in}{2.119498in}}{\pgfqpoint{3.921234in}{2.122771in}}{\pgfqpoint{3.927058in}{2.128595in}}%
\pgfpathcurveto{\pgfqpoint{3.932882in}{2.134419in}}{\pgfqpoint{3.936155in}{2.142319in}}{\pgfqpoint{3.936155in}{2.150555in}}%
\pgfpathcurveto{\pgfqpoint{3.936155in}{2.158791in}}{\pgfqpoint{3.932882in}{2.166691in}}{\pgfqpoint{3.927058in}{2.172515in}}%
\pgfpathcurveto{\pgfqpoint{3.921234in}{2.178339in}}{\pgfqpoint{3.913334in}{2.181611in}}{\pgfqpoint{3.905098in}{2.181611in}}%
\pgfpathcurveto{\pgfqpoint{3.896862in}{2.181611in}}{\pgfqpoint{3.888962in}{2.178339in}}{\pgfqpoint{3.883138in}{2.172515in}}%
\pgfpathcurveto{\pgfqpoint{3.877314in}{2.166691in}}{\pgfqpoint{3.874042in}{2.158791in}}{\pgfqpoint{3.874042in}{2.150555in}}%
\pgfpathcurveto{\pgfqpoint{3.874042in}{2.142319in}}{\pgfqpoint{3.877314in}{2.134419in}}{\pgfqpoint{3.883138in}{2.128595in}}%
\pgfpathcurveto{\pgfqpoint{3.888962in}{2.122771in}}{\pgfqpoint{3.896862in}{2.119498in}}{\pgfqpoint{3.905098in}{2.119498in}}%
\pgfpathclose%
\pgfusepath{stroke,fill}%
\end{pgfscope}%
\begin{pgfscope}%
\pgfpathrectangle{\pgfqpoint{3.793912in}{0.557870in}}{\pgfqpoint{2.446088in}{1.684734in}}%
\pgfusepath{clip}%
\pgfsetbuttcap%
\pgfsetroundjoin%
\definecolor{currentfill}{rgb}{0.298039,0.447059,0.690196}%
\pgfsetfillcolor{currentfill}%
\pgfsetlinewidth{1.003750pt}%
\definecolor{currentstroke}{rgb}{0.298039,0.447059,0.690196}%
\pgfsetstrokecolor{currentstroke}%
\pgfsetdash{}{0pt}%
\pgfpathmoveto{\pgfqpoint{3.905098in}{2.119498in}}%
\pgfpathcurveto{\pgfqpoint{3.913334in}{2.119498in}}{\pgfqpoint{3.921234in}{2.122771in}}{\pgfqpoint{3.927058in}{2.128595in}}%
\pgfpathcurveto{\pgfqpoint{3.932882in}{2.134419in}}{\pgfqpoint{3.936155in}{2.142319in}}{\pgfqpoint{3.936155in}{2.150555in}}%
\pgfpathcurveto{\pgfqpoint{3.936155in}{2.158791in}}{\pgfqpoint{3.932882in}{2.166691in}}{\pgfqpoint{3.927058in}{2.172515in}}%
\pgfpathcurveto{\pgfqpoint{3.921234in}{2.178339in}}{\pgfqpoint{3.913334in}{2.181611in}}{\pgfqpoint{3.905098in}{2.181611in}}%
\pgfpathcurveto{\pgfqpoint{3.896862in}{2.181611in}}{\pgfqpoint{3.888962in}{2.178339in}}{\pgfqpoint{3.883138in}{2.172515in}}%
\pgfpathcurveto{\pgfqpoint{3.877314in}{2.166691in}}{\pgfqpoint{3.874042in}{2.158791in}}{\pgfqpoint{3.874042in}{2.150555in}}%
\pgfpathcurveto{\pgfqpoint{3.874042in}{2.142319in}}{\pgfqpoint{3.877314in}{2.134419in}}{\pgfqpoint{3.883138in}{2.128595in}}%
\pgfpathcurveto{\pgfqpoint{3.888962in}{2.122771in}}{\pgfqpoint{3.896862in}{2.119498in}}{\pgfqpoint{3.905098in}{2.119498in}}%
\pgfpathclose%
\pgfusepath{stroke,fill}%
\end{pgfscope}%
\begin{pgfscope}%
\pgfpathrectangle{\pgfqpoint{3.793912in}{0.557870in}}{\pgfqpoint{2.446088in}{1.684734in}}%
\pgfusepath{clip}%
\pgfsetbuttcap%
\pgfsetroundjoin%
\definecolor{currentfill}{rgb}{0.298039,0.447059,0.690196}%
\pgfsetfillcolor{currentfill}%
\pgfsetlinewidth{1.003750pt}%
\definecolor{currentstroke}{rgb}{0.298039,0.447059,0.690196}%
\pgfsetstrokecolor{currentstroke}%
\pgfsetdash{}{0pt}%
\pgfpathmoveto{\pgfqpoint{3.905098in}{2.119498in}}%
\pgfpathcurveto{\pgfqpoint{3.913334in}{2.119498in}}{\pgfqpoint{3.921234in}{2.122771in}}{\pgfqpoint{3.927058in}{2.128595in}}%
\pgfpathcurveto{\pgfqpoint{3.932882in}{2.134419in}}{\pgfqpoint{3.936155in}{2.142319in}}{\pgfqpoint{3.936155in}{2.150555in}}%
\pgfpathcurveto{\pgfqpoint{3.936155in}{2.158791in}}{\pgfqpoint{3.932882in}{2.166691in}}{\pgfqpoint{3.927058in}{2.172515in}}%
\pgfpathcurveto{\pgfqpoint{3.921234in}{2.178339in}}{\pgfqpoint{3.913334in}{2.181611in}}{\pgfqpoint{3.905098in}{2.181611in}}%
\pgfpathcurveto{\pgfqpoint{3.896862in}{2.181611in}}{\pgfqpoint{3.888962in}{2.178339in}}{\pgfqpoint{3.883138in}{2.172515in}}%
\pgfpathcurveto{\pgfqpoint{3.877314in}{2.166691in}}{\pgfqpoint{3.874042in}{2.158791in}}{\pgfqpoint{3.874042in}{2.150555in}}%
\pgfpathcurveto{\pgfqpoint{3.874042in}{2.142319in}}{\pgfqpoint{3.877314in}{2.134419in}}{\pgfqpoint{3.883138in}{2.128595in}}%
\pgfpathcurveto{\pgfqpoint{3.888962in}{2.122771in}}{\pgfqpoint{3.896862in}{2.119498in}}{\pgfqpoint{3.905098in}{2.119498in}}%
\pgfpathclose%
\pgfusepath{stroke,fill}%
\end{pgfscope}%
\begin{pgfscope}%
\pgfpathrectangle{\pgfqpoint{3.793912in}{0.557870in}}{\pgfqpoint{2.446088in}{1.684734in}}%
\pgfusepath{clip}%
\pgfsetbuttcap%
\pgfsetroundjoin%
\definecolor{currentfill}{rgb}{0.298039,0.447059,0.690196}%
\pgfsetfillcolor{currentfill}%
\pgfsetlinewidth{1.003750pt}%
\definecolor{currentstroke}{rgb}{0.298039,0.447059,0.690196}%
\pgfsetstrokecolor{currentstroke}%
\pgfsetdash{}{0pt}%
\pgfpathmoveto{\pgfqpoint{5.532767in}{1.624443in}}%
\pgfpathcurveto{\pgfqpoint{5.541003in}{1.624443in}}{\pgfqpoint{5.548903in}{1.627716in}}{\pgfqpoint{5.554727in}{1.633540in}}%
\pgfpathcurveto{\pgfqpoint{5.560551in}{1.639364in}}{\pgfqpoint{5.563823in}{1.647264in}}{\pgfqpoint{5.563823in}{1.655500in}}%
\pgfpathcurveto{\pgfqpoint{5.563823in}{1.663736in}}{\pgfqpoint{5.560551in}{1.671636in}}{\pgfqpoint{5.554727in}{1.677460in}}%
\pgfpathcurveto{\pgfqpoint{5.548903in}{1.683284in}}{\pgfqpoint{5.541003in}{1.686556in}}{\pgfqpoint{5.532767in}{1.686556in}}%
\pgfpathcurveto{\pgfqpoint{5.524530in}{1.686556in}}{\pgfqpoint{5.516630in}{1.683284in}}{\pgfqpoint{5.510806in}{1.677460in}}%
\pgfpathcurveto{\pgfqpoint{5.504982in}{1.671636in}}{\pgfqpoint{5.501710in}{1.663736in}}{\pgfqpoint{5.501710in}{1.655500in}}%
\pgfpathcurveto{\pgfqpoint{5.501710in}{1.647264in}}{\pgfqpoint{5.504982in}{1.639364in}}{\pgfqpoint{5.510806in}{1.633540in}}%
\pgfpathcurveto{\pgfqpoint{5.516630in}{1.627716in}}{\pgfqpoint{5.524530in}{1.624443in}}{\pgfqpoint{5.532767in}{1.624443in}}%
\pgfpathclose%
\pgfusepath{stroke,fill}%
\end{pgfscope}%
\begin{pgfscope}%
\pgfpathrectangle{\pgfqpoint{3.793912in}{0.557870in}}{\pgfqpoint{2.446088in}{1.684734in}}%
\pgfusepath{clip}%
\pgfsetbuttcap%
\pgfsetroundjoin%
\definecolor{currentfill}{rgb}{0.298039,0.447059,0.690196}%
\pgfsetfillcolor{currentfill}%
\pgfsetlinewidth{1.003750pt}%
\definecolor{currentstroke}{rgb}{0.298039,0.447059,0.690196}%
\pgfsetstrokecolor{currentstroke}%
\pgfsetdash{}{0pt}%
\pgfpathmoveto{\pgfqpoint{5.441067in}{1.732737in}}%
\pgfpathcurveto{\pgfqpoint{5.449303in}{1.732737in}}{\pgfqpoint{5.457203in}{1.736009in}}{\pgfqpoint{5.463027in}{1.741833in}}%
\pgfpathcurveto{\pgfqpoint{5.468851in}{1.747657in}}{\pgfqpoint{5.472123in}{1.755557in}}{\pgfqpoint{5.472123in}{1.763793in}}%
\pgfpathcurveto{\pgfqpoint{5.472123in}{1.772029in}}{\pgfqpoint{5.468851in}{1.779930in}}{\pgfqpoint{5.463027in}{1.785753in}}%
\pgfpathcurveto{\pgfqpoint{5.457203in}{1.791577in}}{\pgfqpoint{5.449303in}{1.794850in}}{\pgfqpoint{5.441067in}{1.794850in}}%
\pgfpathcurveto{\pgfqpoint{5.432831in}{1.794850in}}{\pgfqpoint{5.424931in}{1.791577in}}{\pgfqpoint{5.419107in}{1.785753in}}%
\pgfpathcurveto{\pgfqpoint{5.413283in}{1.779930in}}{\pgfqpoint{5.410010in}{1.772029in}}{\pgfqpoint{5.410010in}{1.763793in}}%
\pgfpathcurveto{\pgfqpoint{5.410010in}{1.755557in}}{\pgfqpoint{5.413283in}{1.747657in}}{\pgfqpoint{5.419107in}{1.741833in}}%
\pgfpathcurveto{\pgfqpoint{5.424931in}{1.736009in}}{\pgfqpoint{5.432831in}{1.732737in}}{\pgfqpoint{5.441067in}{1.732737in}}%
\pgfpathclose%
\pgfusepath{stroke,fill}%
\end{pgfscope}%
\begin{pgfscope}%
\pgfpathrectangle{\pgfqpoint{3.793912in}{0.557870in}}{\pgfqpoint{2.446088in}{1.684734in}}%
\pgfusepath{clip}%
\pgfsetbuttcap%
\pgfsetroundjoin%
\definecolor{currentfill}{rgb}{0.298039,0.447059,0.690196}%
\pgfsetfillcolor{currentfill}%
\pgfsetlinewidth{1.003750pt}%
\definecolor{currentstroke}{rgb}{0.298039,0.447059,0.690196}%
\pgfsetstrokecolor{currentstroke}%
\pgfsetdash{}{0pt}%
\pgfpathmoveto{\pgfqpoint{5.532767in}{1.639914in}}%
\pgfpathcurveto{\pgfqpoint{5.541003in}{1.639914in}}{\pgfqpoint{5.548903in}{1.643186in}}{\pgfqpoint{5.554727in}{1.649010in}}%
\pgfpathcurveto{\pgfqpoint{5.560551in}{1.654834in}}{\pgfqpoint{5.563823in}{1.662734in}}{\pgfqpoint{5.563823in}{1.670970in}}%
\pgfpathcurveto{\pgfqpoint{5.563823in}{1.679207in}}{\pgfqpoint{5.560551in}{1.687107in}}{\pgfqpoint{5.554727in}{1.692931in}}%
\pgfpathcurveto{\pgfqpoint{5.548903in}{1.698755in}}{\pgfqpoint{5.541003in}{1.702027in}}{\pgfqpoint{5.532767in}{1.702027in}}%
\pgfpathcurveto{\pgfqpoint{5.524530in}{1.702027in}}{\pgfqpoint{5.516630in}{1.698755in}}{\pgfqpoint{5.510806in}{1.692931in}}%
\pgfpathcurveto{\pgfqpoint{5.504982in}{1.687107in}}{\pgfqpoint{5.501710in}{1.679207in}}{\pgfqpoint{5.501710in}{1.670970in}}%
\pgfpathcurveto{\pgfqpoint{5.501710in}{1.662734in}}{\pgfqpoint{5.504982in}{1.654834in}}{\pgfqpoint{5.510806in}{1.649010in}}%
\pgfpathcurveto{\pgfqpoint{5.516630in}{1.643186in}}{\pgfqpoint{5.524530in}{1.639914in}}{\pgfqpoint{5.532767in}{1.639914in}}%
\pgfpathclose%
\pgfusepath{stroke,fill}%
\end{pgfscope}%
\begin{pgfscope}%
\pgfpathrectangle{\pgfqpoint{3.793912in}{0.557870in}}{\pgfqpoint{2.446088in}{1.684734in}}%
\pgfusepath{clip}%
\pgfsetbuttcap%
\pgfsetroundjoin%
\definecolor{currentfill}{rgb}{0.298039,0.447059,0.690196}%
\pgfsetfillcolor{currentfill}%
\pgfsetlinewidth{1.003750pt}%
\definecolor{currentstroke}{rgb}{0.298039,0.447059,0.690196}%
\pgfsetstrokecolor{currentstroke}%
\pgfsetdash{}{0pt}%
\pgfpathmoveto{\pgfqpoint{5.578616in}{1.547091in}}%
\pgfpathcurveto{\pgfqpoint{5.586853in}{1.547091in}}{\pgfqpoint{5.594753in}{1.550363in}}{\pgfqpoint{5.600577in}{1.556187in}}%
\pgfpathcurveto{\pgfqpoint{5.606401in}{1.562011in}}{\pgfqpoint{5.609673in}{1.569911in}}{\pgfqpoint{5.609673in}{1.578148in}}%
\pgfpathcurveto{\pgfqpoint{5.609673in}{1.586384in}}{\pgfqpoint{5.606401in}{1.594284in}}{\pgfqpoint{5.600577in}{1.600108in}}%
\pgfpathcurveto{\pgfqpoint{5.594753in}{1.605932in}}{\pgfqpoint{5.586853in}{1.609204in}}{\pgfqpoint{5.578616in}{1.609204in}}%
\pgfpathcurveto{\pgfqpoint{5.570380in}{1.609204in}}{\pgfqpoint{5.562480in}{1.605932in}}{\pgfqpoint{5.556656in}{1.600108in}}%
\pgfpathcurveto{\pgfqpoint{5.550832in}{1.594284in}}{\pgfqpoint{5.547560in}{1.586384in}}{\pgfqpoint{5.547560in}{1.578148in}}%
\pgfpathcurveto{\pgfqpoint{5.547560in}{1.569911in}}{\pgfqpoint{5.550832in}{1.562011in}}{\pgfqpoint{5.556656in}{1.556187in}}%
\pgfpathcurveto{\pgfqpoint{5.562480in}{1.550363in}}{\pgfqpoint{5.570380in}{1.547091in}}{\pgfqpoint{5.578616in}{1.547091in}}%
\pgfpathclose%
\pgfusepath{stroke,fill}%
\end{pgfscope}%
\begin{pgfscope}%
\pgfpathrectangle{\pgfqpoint{3.793912in}{0.557870in}}{\pgfqpoint{2.446088in}{1.684734in}}%
\pgfusepath{clip}%
\pgfsetbuttcap%
\pgfsetroundjoin%
\definecolor{currentfill}{rgb}{0.298039,0.447059,0.690196}%
\pgfsetfillcolor{currentfill}%
\pgfsetlinewidth{1.003750pt}%
\definecolor{currentstroke}{rgb}{0.298039,0.447059,0.690196}%
\pgfsetstrokecolor{currentstroke}%
\pgfsetdash{}{0pt}%
\pgfpathmoveto{\pgfqpoint{5.395217in}{1.608973in}}%
\pgfpathcurveto{\pgfqpoint{5.403453in}{1.608973in}}{\pgfqpoint{5.411353in}{1.612245in}}{\pgfqpoint{5.417177in}{1.618069in}}%
\pgfpathcurveto{\pgfqpoint{5.423001in}{1.623893in}}{\pgfqpoint{5.426274in}{1.631793in}}{\pgfqpoint{5.426274in}{1.640029in}}%
\pgfpathcurveto{\pgfqpoint{5.426274in}{1.648266in}}{\pgfqpoint{5.423001in}{1.656166in}}{\pgfqpoint{5.417177in}{1.661990in}}%
\pgfpathcurveto{\pgfqpoint{5.411353in}{1.667814in}}{\pgfqpoint{5.403453in}{1.671086in}}{\pgfqpoint{5.395217in}{1.671086in}}%
\pgfpathcurveto{\pgfqpoint{5.386981in}{1.671086in}}{\pgfqpoint{5.379081in}{1.667814in}}{\pgfqpoint{5.373257in}{1.661990in}}%
\pgfpathcurveto{\pgfqpoint{5.367433in}{1.656166in}}{\pgfqpoint{5.364161in}{1.648266in}}{\pgfqpoint{5.364161in}{1.640029in}}%
\pgfpathcurveto{\pgfqpoint{5.364161in}{1.631793in}}{\pgfqpoint{5.367433in}{1.623893in}}{\pgfqpoint{5.373257in}{1.618069in}}%
\pgfpathcurveto{\pgfqpoint{5.379081in}{1.612245in}}{\pgfqpoint{5.386981in}{1.608973in}}{\pgfqpoint{5.395217in}{1.608973in}}%
\pgfpathclose%
\pgfusepath{stroke,fill}%
\end{pgfscope}%
\begin{pgfscope}%
\pgfpathrectangle{\pgfqpoint{3.793912in}{0.557870in}}{\pgfqpoint{2.446088in}{1.684734in}}%
\pgfusepath{clip}%
\pgfsetbuttcap%
\pgfsetroundjoin%
\definecolor{currentfill}{rgb}{0.298039,0.447059,0.690196}%
\pgfsetfillcolor{currentfill}%
\pgfsetlinewidth{1.003750pt}%
\definecolor{currentstroke}{rgb}{0.298039,0.447059,0.690196}%
\pgfsetstrokecolor{currentstroke}%
\pgfsetdash{}{0pt}%
\pgfpathmoveto{\pgfqpoint{6.060039in}{0.804509in}}%
\pgfpathcurveto{\pgfqpoint{6.068276in}{0.804509in}}{\pgfqpoint{6.076176in}{0.807781in}}{\pgfqpoint{6.082000in}{0.813605in}}%
\pgfpathcurveto{\pgfqpoint{6.087824in}{0.819429in}}{\pgfqpoint{6.091096in}{0.827329in}}{\pgfqpoint{6.091096in}{0.835565in}}%
\pgfpathcurveto{\pgfqpoint{6.091096in}{0.843801in}}{\pgfqpoint{6.087824in}{0.851701in}}{\pgfqpoint{6.082000in}{0.857525in}}%
\pgfpathcurveto{\pgfqpoint{6.076176in}{0.863349in}}{\pgfqpoint{6.068276in}{0.866622in}}{\pgfqpoint{6.060039in}{0.866622in}}%
\pgfpathcurveto{\pgfqpoint{6.051803in}{0.866622in}}{\pgfqpoint{6.043903in}{0.863349in}}{\pgfqpoint{6.038079in}{0.857525in}}%
\pgfpathcurveto{\pgfqpoint{6.032255in}{0.851701in}}{\pgfqpoint{6.028983in}{0.843801in}}{\pgfqpoint{6.028983in}{0.835565in}}%
\pgfpathcurveto{\pgfqpoint{6.028983in}{0.827329in}}{\pgfqpoint{6.032255in}{0.819429in}}{\pgfqpoint{6.038079in}{0.813605in}}%
\pgfpathcurveto{\pgfqpoint{6.043903in}{0.807781in}}{\pgfqpoint{6.051803in}{0.804509in}}{\pgfqpoint{6.060039in}{0.804509in}}%
\pgfpathclose%
\pgfusepath{stroke,fill}%
\end{pgfscope}%
\begin{pgfscope}%
\pgfpathrectangle{\pgfqpoint{3.793912in}{0.557870in}}{\pgfqpoint{2.446088in}{1.684734in}}%
\pgfusepath{clip}%
\pgfsetbuttcap%
\pgfsetroundjoin%
\definecolor{currentfill}{rgb}{0.298039,0.447059,0.690196}%
\pgfsetfillcolor{currentfill}%
\pgfsetlinewidth{1.003750pt}%
\definecolor{currentstroke}{rgb}{0.298039,0.447059,0.690196}%
\pgfsetstrokecolor{currentstroke}%
\pgfsetdash{}{0pt}%
\pgfpathmoveto{\pgfqpoint{5.372292in}{1.748207in}}%
\pgfpathcurveto{\pgfqpoint{5.380528in}{1.748207in}}{\pgfqpoint{5.388429in}{1.751479in}}{\pgfqpoint{5.394252in}{1.757303in}}%
\pgfpathcurveto{\pgfqpoint{5.400076in}{1.763127in}}{\pgfqpoint{5.403349in}{1.771027in}}{\pgfqpoint{5.403349in}{1.779264in}}%
\pgfpathcurveto{\pgfqpoint{5.403349in}{1.787500in}}{\pgfqpoint{5.400076in}{1.795400in}}{\pgfqpoint{5.394252in}{1.801224in}}%
\pgfpathcurveto{\pgfqpoint{5.388429in}{1.807048in}}{\pgfqpoint{5.380528in}{1.810320in}}{\pgfqpoint{5.372292in}{1.810320in}}%
\pgfpathcurveto{\pgfqpoint{5.364056in}{1.810320in}}{\pgfqpoint{5.356156in}{1.807048in}}{\pgfqpoint{5.350332in}{1.801224in}}%
\pgfpathcurveto{\pgfqpoint{5.344508in}{1.795400in}}{\pgfqpoint{5.341236in}{1.787500in}}{\pgfqpoint{5.341236in}{1.779264in}}%
\pgfpathcurveto{\pgfqpoint{5.341236in}{1.771027in}}{\pgfqpoint{5.344508in}{1.763127in}}{\pgfqpoint{5.350332in}{1.757303in}}%
\pgfpathcurveto{\pgfqpoint{5.356156in}{1.751479in}}{\pgfqpoint{5.364056in}{1.748207in}}{\pgfqpoint{5.372292in}{1.748207in}}%
\pgfpathclose%
\pgfusepath{stroke,fill}%
\end{pgfscope}%
\begin{pgfscope}%
\pgfpathrectangle{\pgfqpoint{3.793912in}{0.557870in}}{\pgfqpoint{2.446088in}{1.684734in}}%
\pgfusepath{clip}%
\pgfsetbuttcap%
\pgfsetroundjoin%
\definecolor{currentfill}{rgb}{0.298039,0.447059,0.690196}%
\pgfsetfillcolor{currentfill}%
\pgfsetlinewidth{1.003750pt}%
\definecolor{currentstroke}{rgb}{0.298039,0.447059,0.690196}%
\pgfsetstrokecolor{currentstroke}%
\pgfsetdash{}{0pt}%
\pgfpathmoveto{\pgfqpoint{5.784941in}{1.268623in}}%
\pgfpathcurveto{\pgfqpoint{5.793177in}{1.268623in}}{\pgfqpoint{5.801077in}{1.271895in}}{\pgfqpoint{5.806901in}{1.277719in}}%
\pgfpathcurveto{\pgfqpoint{5.812725in}{1.283543in}}{\pgfqpoint{5.815997in}{1.291443in}}{\pgfqpoint{5.815997in}{1.299679in}}%
\pgfpathcurveto{\pgfqpoint{5.815997in}{1.307915in}}{\pgfqpoint{5.812725in}{1.315816in}}{\pgfqpoint{5.806901in}{1.321639in}}%
\pgfpathcurveto{\pgfqpoint{5.801077in}{1.327463in}}{\pgfqpoint{5.793177in}{1.330736in}}{\pgfqpoint{5.784941in}{1.330736in}}%
\pgfpathcurveto{\pgfqpoint{5.776704in}{1.330736in}}{\pgfqpoint{5.768804in}{1.327463in}}{\pgfqpoint{5.762980in}{1.321639in}}%
\pgfpathcurveto{\pgfqpoint{5.757156in}{1.315816in}}{\pgfqpoint{5.753884in}{1.307915in}}{\pgfqpoint{5.753884in}{1.299679in}}%
\pgfpathcurveto{\pgfqpoint{5.753884in}{1.291443in}}{\pgfqpoint{5.757156in}{1.283543in}}{\pgfqpoint{5.762980in}{1.277719in}}%
\pgfpathcurveto{\pgfqpoint{5.768804in}{1.271895in}}{\pgfqpoint{5.776704in}{1.268623in}}{\pgfqpoint{5.784941in}{1.268623in}}%
\pgfpathclose%
\pgfusepath{stroke,fill}%
\end{pgfscope}%
\begin{pgfscope}%
\pgfpathrectangle{\pgfqpoint{3.793912in}{0.557870in}}{\pgfqpoint{2.446088in}{1.684734in}}%
\pgfusepath{clip}%
\pgfsetbuttcap%
\pgfsetroundjoin%
\definecolor{currentfill}{rgb}{0.298039,0.447059,0.690196}%
\pgfsetfillcolor{currentfill}%
\pgfsetlinewidth{1.003750pt}%
\definecolor{currentstroke}{rgb}{0.298039,0.447059,0.690196}%
\pgfsetstrokecolor{currentstroke}%
\pgfsetdash{}{0pt}%
\pgfpathmoveto{\pgfqpoint{5.372292in}{1.794619in}}%
\pgfpathcurveto{\pgfqpoint{5.380528in}{1.794619in}}{\pgfqpoint{5.388429in}{1.797891in}}{\pgfqpoint{5.394252in}{1.803715in}}%
\pgfpathcurveto{\pgfqpoint{5.400076in}{1.809539in}}{\pgfqpoint{5.403349in}{1.817439in}}{\pgfqpoint{5.403349in}{1.825675in}}%
\pgfpathcurveto{\pgfqpoint{5.403349in}{1.833911in}}{\pgfqpoint{5.400076in}{1.841811in}}{\pgfqpoint{5.394252in}{1.847635in}}%
\pgfpathcurveto{\pgfqpoint{5.388429in}{1.853459in}}{\pgfqpoint{5.380528in}{1.856732in}}{\pgfqpoint{5.372292in}{1.856732in}}%
\pgfpathcurveto{\pgfqpoint{5.364056in}{1.856732in}}{\pgfqpoint{5.356156in}{1.853459in}}{\pgfqpoint{5.350332in}{1.847635in}}%
\pgfpathcurveto{\pgfqpoint{5.344508in}{1.841811in}}{\pgfqpoint{5.341236in}{1.833911in}}{\pgfqpoint{5.341236in}{1.825675in}}%
\pgfpathcurveto{\pgfqpoint{5.341236in}{1.817439in}}{\pgfqpoint{5.344508in}{1.809539in}}{\pgfqpoint{5.350332in}{1.803715in}}%
\pgfpathcurveto{\pgfqpoint{5.356156in}{1.797891in}}{\pgfqpoint{5.364056in}{1.794619in}}{\pgfqpoint{5.372292in}{1.794619in}}%
\pgfpathclose%
\pgfusepath{stroke,fill}%
\end{pgfscope}%
\begin{pgfscope}%
\pgfpathrectangle{\pgfqpoint{3.793912in}{0.557870in}}{\pgfqpoint{2.446088in}{1.684734in}}%
\pgfusepath{clip}%
\pgfsetbuttcap%
\pgfsetroundjoin%
\definecolor{currentfill}{rgb}{0.298039,0.447059,0.690196}%
\pgfsetfillcolor{currentfill}%
\pgfsetlinewidth{1.003750pt}%
\definecolor{currentstroke}{rgb}{0.298039,0.447059,0.690196}%
\pgfsetstrokecolor{currentstroke}%
\pgfsetdash{}{0pt}%
\pgfpathmoveto{\pgfqpoint{5.991265in}{0.974684in}}%
\pgfpathcurveto{\pgfqpoint{5.999501in}{0.974684in}}{\pgfqpoint{6.007401in}{0.977956in}}{\pgfqpoint{6.013225in}{0.983780in}}%
\pgfpathcurveto{\pgfqpoint{6.019049in}{0.989604in}}{\pgfqpoint{6.022321in}{0.997504in}}{\pgfqpoint{6.022321in}{1.005740in}}%
\pgfpathcurveto{\pgfqpoint{6.022321in}{1.013977in}}{\pgfqpoint{6.019049in}{1.021877in}}{\pgfqpoint{6.013225in}{1.027701in}}%
\pgfpathcurveto{\pgfqpoint{6.007401in}{1.033524in}}{\pgfqpoint{5.999501in}{1.036797in}}{\pgfqpoint{5.991265in}{1.036797in}}%
\pgfpathcurveto{\pgfqpoint{5.983028in}{1.036797in}}{\pgfqpoint{5.975128in}{1.033524in}}{\pgfqpoint{5.969304in}{1.027701in}}%
\pgfpathcurveto{\pgfqpoint{5.963481in}{1.021877in}}{\pgfqpoint{5.960208in}{1.013977in}}{\pgfqpoint{5.960208in}{1.005740in}}%
\pgfpathcurveto{\pgfqpoint{5.960208in}{0.997504in}}{\pgfqpoint{5.963481in}{0.989604in}}{\pgfqpoint{5.969304in}{0.983780in}}%
\pgfpathcurveto{\pgfqpoint{5.975128in}{0.977956in}}{\pgfqpoint{5.983028in}{0.974684in}}{\pgfqpoint{5.991265in}{0.974684in}}%
\pgfpathclose%
\pgfusepath{stroke,fill}%
\end{pgfscope}%
\begin{pgfscope}%
\pgfpathrectangle{\pgfqpoint{3.793912in}{0.557870in}}{\pgfqpoint{2.446088in}{1.684734in}}%
\pgfusepath{clip}%
\pgfsetbuttcap%
\pgfsetroundjoin%
\definecolor{currentfill}{rgb}{0.298039,0.447059,0.690196}%
\pgfsetfillcolor{currentfill}%
\pgfsetlinewidth{1.003750pt}%
\definecolor{currentstroke}{rgb}{0.298039,0.447059,0.690196}%
\pgfsetstrokecolor{currentstroke}%
\pgfsetdash{}{0pt}%
\pgfpathmoveto{\pgfqpoint{5.991265in}{0.974684in}}%
\pgfpathcurveto{\pgfqpoint{5.999501in}{0.974684in}}{\pgfqpoint{6.007401in}{0.977956in}}{\pgfqpoint{6.013225in}{0.983780in}}%
\pgfpathcurveto{\pgfqpoint{6.019049in}{0.989604in}}{\pgfqpoint{6.022321in}{0.997504in}}{\pgfqpoint{6.022321in}{1.005740in}}%
\pgfpathcurveto{\pgfqpoint{6.022321in}{1.013977in}}{\pgfqpoint{6.019049in}{1.021877in}}{\pgfqpoint{6.013225in}{1.027701in}}%
\pgfpathcurveto{\pgfqpoint{6.007401in}{1.033524in}}{\pgfqpoint{5.999501in}{1.036797in}}{\pgfqpoint{5.991265in}{1.036797in}}%
\pgfpathcurveto{\pgfqpoint{5.983028in}{1.036797in}}{\pgfqpoint{5.975128in}{1.033524in}}{\pgfqpoint{5.969304in}{1.027701in}}%
\pgfpathcurveto{\pgfqpoint{5.963481in}{1.021877in}}{\pgfqpoint{5.960208in}{1.013977in}}{\pgfqpoint{5.960208in}{1.005740in}}%
\pgfpathcurveto{\pgfqpoint{5.960208in}{0.997504in}}{\pgfqpoint{5.963481in}{0.989604in}}{\pgfqpoint{5.969304in}{0.983780in}}%
\pgfpathcurveto{\pgfqpoint{5.975128in}{0.977956in}}{\pgfqpoint{5.983028in}{0.974684in}}{\pgfqpoint{5.991265in}{0.974684in}}%
\pgfpathclose%
\pgfusepath{stroke,fill}%
\end{pgfscope}%
\begin{pgfscope}%
\pgfpathrectangle{\pgfqpoint{3.793912in}{0.557870in}}{\pgfqpoint{2.446088in}{1.684734in}}%
\pgfusepath{clip}%
\pgfsetbuttcap%
\pgfsetroundjoin%
\definecolor{currentfill}{rgb}{0.298039,0.447059,0.690196}%
\pgfsetfillcolor{currentfill}%
\pgfsetlinewidth{1.003750pt}%
\definecolor{currentstroke}{rgb}{0.298039,0.447059,0.690196}%
\pgfsetstrokecolor{currentstroke}%
\pgfsetdash{}{0pt}%
\pgfpathmoveto{\pgfqpoint{3.905098in}{2.119498in}}%
\pgfpathcurveto{\pgfqpoint{3.913334in}{2.119498in}}{\pgfqpoint{3.921234in}{2.122771in}}{\pgfqpoint{3.927058in}{2.128595in}}%
\pgfpathcurveto{\pgfqpoint{3.932882in}{2.134419in}}{\pgfqpoint{3.936155in}{2.142319in}}{\pgfqpoint{3.936155in}{2.150555in}}%
\pgfpathcurveto{\pgfqpoint{3.936155in}{2.158791in}}{\pgfqpoint{3.932882in}{2.166691in}}{\pgfqpoint{3.927058in}{2.172515in}}%
\pgfpathcurveto{\pgfqpoint{3.921234in}{2.178339in}}{\pgfqpoint{3.913334in}{2.181611in}}{\pgfqpoint{3.905098in}{2.181611in}}%
\pgfpathcurveto{\pgfqpoint{3.896862in}{2.181611in}}{\pgfqpoint{3.888962in}{2.178339in}}{\pgfqpoint{3.883138in}{2.172515in}}%
\pgfpathcurveto{\pgfqpoint{3.877314in}{2.166691in}}{\pgfqpoint{3.874042in}{2.158791in}}{\pgfqpoint{3.874042in}{2.150555in}}%
\pgfpathcurveto{\pgfqpoint{3.874042in}{2.142319in}}{\pgfqpoint{3.877314in}{2.134419in}}{\pgfqpoint{3.883138in}{2.128595in}}%
\pgfpathcurveto{\pgfqpoint{3.888962in}{2.122771in}}{\pgfqpoint{3.896862in}{2.119498in}}{\pgfqpoint{3.905098in}{2.119498in}}%
\pgfpathclose%
\pgfusepath{stroke,fill}%
\end{pgfscope}%
\begin{pgfscope}%
\pgfpathrectangle{\pgfqpoint{3.793912in}{0.557870in}}{\pgfqpoint{2.446088in}{1.684734in}}%
\pgfusepath{clip}%
\pgfsetbuttcap%
\pgfsetroundjoin%
\definecolor{currentfill}{rgb}{0.298039,0.447059,0.690196}%
\pgfsetfillcolor{currentfill}%
\pgfsetlinewidth{1.003750pt}%
\definecolor{currentstroke}{rgb}{0.298039,0.447059,0.690196}%
\pgfsetstrokecolor{currentstroke}%
\pgfsetdash{}{0pt}%
\pgfpathmoveto{\pgfqpoint{3.905098in}{2.119498in}}%
\pgfpathcurveto{\pgfqpoint{3.913334in}{2.119498in}}{\pgfqpoint{3.921234in}{2.122771in}}{\pgfqpoint{3.927058in}{2.128595in}}%
\pgfpathcurveto{\pgfqpoint{3.932882in}{2.134419in}}{\pgfqpoint{3.936155in}{2.142319in}}{\pgfqpoint{3.936155in}{2.150555in}}%
\pgfpathcurveto{\pgfqpoint{3.936155in}{2.158791in}}{\pgfqpoint{3.932882in}{2.166691in}}{\pgfqpoint{3.927058in}{2.172515in}}%
\pgfpathcurveto{\pgfqpoint{3.921234in}{2.178339in}}{\pgfqpoint{3.913334in}{2.181611in}}{\pgfqpoint{3.905098in}{2.181611in}}%
\pgfpathcurveto{\pgfqpoint{3.896862in}{2.181611in}}{\pgfqpoint{3.888962in}{2.178339in}}{\pgfqpoint{3.883138in}{2.172515in}}%
\pgfpathcurveto{\pgfqpoint{3.877314in}{2.166691in}}{\pgfqpoint{3.874042in}{2.158791in}}{\pgfqpoint{3.874042in}{2.150555in}}%
\pgfpathcurveto{\pgfqpoint{3.874042in}{2.142319in}}{\pgfqpoint{3.877314in}{2.134419in}}{\pgfqpoint{3.883138in}{2.128595in}}%
\pgfpathcurveto{\pgfqpoint{3.888962in}{2.122771in}}{\pgfqpoint{3.896862in}{2.119498in}}{\pgfqpoint{3.905098in}{2.119498in}}%
\pgfpathclose%
\pgfusepath{stroke,fill}%
\end{pgfscope}%
\begin{pgfscope}%
\pgfpathrectangle{\pgfqpoint{3.793912in}{0.557870in}}{\pgfqpoint{2.446088in}{1.684734in}}%
\pgfusepath{clip}%
\pgfsetbuttcap%
\pgfsetroundjoin%
\definecolor{currentfill}{rgb}{0.298039,0.447059,0.690196}%
\pgfsetfillcolor{currentfill}%
\pgfsetlinewidth{1.003750pt}%
\definecolor{currentstroke}{rgb}{0.298039,0.447059,0.690196}%
\pgfsetstrokecolor{currentstroke}%
\pgfsetdash{}{0pt}%
\pgfpathmoveto{\pgfqpoint{3.905098in}{2.119498in}}%
\pgfpathcurveto{\pgfqpoint{3.913334in}{2.119498in}}{\pgfqpoint{3.921234in}{2.122771in}}{\pgfqpoint{3.927058in}{2.128595in}}%
\pgfpathcurveto{\pgfqpoint{3.932882in}{2.134419in}}{\pgfqpoint{3.936155in}{2.142319in}}{\pgfqpoint{3.936155in}{2.150555in}}%
\pgfpathcurveto{\pgfqpoint{3.936155in}{2.158791in}}{\pgfqpoint{3.932882in}{2.166691in}}{\pgfqpoint{3.927058in}{2.172515in}}%
\pgfpathcurveto{\pgfqpoint{3.921234in}{2.178339in}}{\pgfqpoint{3.913334in}{2.181611in}}{\pgfqpoint{3.905098in}{2.181611in}}%
\pgfpathcurveto{\pgfqpoint{3.896862in}{2.181611in}}{\pgfqpoint{3.888962in}{2.178339in}}{\pgfqpoint{3.883138in}{2.172515in}}%
\pgfpathcurveto{\pgfqpoint{3.877314in}{2.166691in}}{\pgfqpoint{3.874042in}{2.158791in}}{\pgfqpoint{3.874042in}{2.150555in}}%
\pgfpathcurveto{\pgfqpoint{3.874042in}{2.142319in}}{\pgfqpoint{3.877314in}{2.134419in}}{\pgfqpoint{3.883138in}{2.128595in}}%
\pgfpathcurveto{\pgfqpoint{3.888962in}{2.122771in}}{\pgfqpoint{3.896862in}{2.119498in}}{\pgfqpoint{3.905098in}{2.119498in}}%
\pgfpathclose%
\pgfusepath{stroke,fill}%
\end{pgfscope}%
\begin{pgfscope}%
\pgfpathrectangle{\pgfqpoint{3.793912in}{0.557870in}}{\pgfqpoint{2.446088in}{1.684734in}}%
\pgfusepath{clip}%
\pgfsetbuttcap%
\pgfsetroundjoin%
\definecolor{currentfill}{rgb}{0.298039,0.447059,0.690196}%
\pgfsetfillcolor{currentfill}%
\pgfsetlinewidth{1.003750pt}%
\definecolor{currentstroke}{rgb}{0.298039,0.447059,0.690196}%
\pgfsetstrokecolor{currentstroke}%
\pgfsetdash{}{0pt}%
\pgfpathmoveto{\pgfqpoint{5.441067in}{1.206741in}}%
\pgfpathcurveto{\pgfqpoint{5.449303in}{1.206741in}}{\pgfqpoint{5.457203in}{1.210013in}}{\pgfqpoint{5.463027in}{1.215837in}}%
\pgfpathcurveto{\pgfqpoint{5.468851in}{1.221661in}}{\pgfqpoint{5.472123in}{1.229561in}}{\pgfqpoint{5.472123in}{1.237797in}}%
\pgfpathcurveto{\pgfqpoint{5.472123in}{1.246034in}}{\pgfqpoint{5.468851in}{1.253934in}}{\pgfqpoint{5.463027in}{1.259758in}}%
\pgfpathcurveto{\pgfqpoint{5.457203in}{1.265582in}}{\pgfqpoint{5.449303in}{1.268854in}}{\pgfqpoint{5.441067in}{1.268854in}}%
\pgfpathcurveto{\pgfqpoint{5.432831in}{1.268854in}}{\pgfqpoint{5.424931in}{1.265582in}}{\pgfqpoint{5.419107in}{1.259758in}}%
\pgfpathcurveto{\pgfqpoint{5.413283in}{1.253934in}}{\pgfqpoint{5.410010in}{1.246034in}}{\pgfqpoint{5.410010in}{1.237797in}}%
\pgfpathcurveto{\pgfqpoint{5.410010in}{1.229561in}}{\pgfqpoint{5.413283in}{1.221661in}}{\pgfqpoint{5.419107in}{1.215837in}}%
\pgfpathcurveto{\pgfqpoint{5.424931in}{1.210013in}}{\pgfqpoint{5.432831in}{1.206741in}}{\pgfqpoint{5.441067in}{1.206741in}}%
\pgfpathclose%
\pgfusepath{stroke,fill}%
\end{pgfscope}%
\begin{pgfscope}%
\pgfpathrectangle{\pgfqpoint{3.793912in}{0.557870in}}{\pgfqpoint{2.446088in}{1.684734in}}%
\pgfusepath{clip}%
\pgfsetbuttcap%
\pgfsetroundjoin%
\definecolor{currentfill}{rgb}{0.298039,0.447059,0.690196}%
\pgfsetfillcolor{currentfill}%
\pgfsetlinewidth{1.003750pt}%
\definecolor{currentstroke}{rgb}{0.298039,0.447059,0.690196}%
\pgfsetstrokecolor{currentstroke}%
\pgfsetdash{}{0pt}%
\pgfpathmoveto{\pgfqpoint{3.905098in}{2.119498in}}%
\pgfpathcurveto{\pgfqpoint{3.913334in}{2.119498in}}{\pgfqpoint{3.921234in}{2.122771in}}{\pgfqpoint{3.927058in}{2.128595in}}%
\pgfpathcurveto{\pgfqpoint{3.932882in}{2.134419in}}{\pgfqpoint{3.936155in}{2.142319in}}{\pgfqpoint{3.936155in}{2.150555in}}%
\pgfpathcurveto{\pgfqpoint{3.936155in}{2.158791in}}{\pgfqpoint{3.932882in}{2.166691in}}{\pgfqpoint{3.927058in}{2.172515in}}%
\pgfpathcurveto{\pgfqpoint{3.921234in}{2.178339in}}{\pgfqpoint{3.913334in}{2.181611in}}{\pgfqpoint{3.905098in}{2.181611in}}%
\pgfpathcurveto{\pgfqpoint{3.896862in}{2.181611in}}{\pgfqpoint{3.888962in}{2.178339in}}{\pgfqpoint{3.883138in}{2.172515in}}%
\pgfpathcurveto{\pgfqpoint{3.877314in}{2.166691in}}{\pgfqpoint{3.874042in}{2.158791in}}{\pgfqpoint{3.874042in}{2.150555in}}%
\pgfpathcurveto{\pgfqpoint{3.874042in}{2.142319in}}{\pgfqpoint{3.877314in}{2.134419in}}{\pgfqpoint{3.883138in}{2.128595in}}%
\pgfpathcurveto{\pgfqpoint{3.888962in}{2.122771in}}{\pgfqpoint{3.896862in}{2.119498in}}{\pgfqpoint{3.905098in}{2.119498in}}%
\pgfpathclose%
\pgfusepath{stroke,fill}%
\end{pgfscope}%
\begin{pgfscope}%
\pgfpathrectangle{\pgfqpoint{3.793912in}{0.557870in}}{\pgfqpoint{2.446088in}{1.684734in}}%
\pgfusepath{clip}%
\pgfsetbuttcap%
\pgfsetroundjoin%
\definecolor{currentfill}{rgb}{0.298039,0.447059,0.690196}%
\pgfsetfillcolor{currentfill}%
\pgfsetlinewidth{1.003750pt}%
\definecolor{currentstroke}{rgb}{0.298039,0.447059,0.690196}%
\pgfsetstrokecolor{currentstroke}%
\pgfsetdash{}{0pt}%
\pgfpathmoveto{\pgfqpoint{3.905098in}{2.119498in}}%
\pgfpathcurveto{\pgfqpoint{3.913334in}{2.119498in}}{\pgfqpoint{3.921234in}{2.122771in}}{\pgfqpoint{3.927058in}{2.128595in}}%
\pgfpathcurveto{\pgfqpoint{3.932882in}{2.134419in}}{\pgfqpoint{3.936155in}{2.142319in}}{\pgfqpoint{3.936155in}{2.150555in}}%
\pgfpathcurveto{\pgfqpoint{3.936155in}{2.158791in}}{\pgfqpoint{3.932882in}{2.166691in}}{\pgfqpoint{3.927058in}{2.172515in}}%
\pgfpathcurveto{\pgfqpoint{3.921234in}{2.178339in}}{\pgfqpoint{3.913334in}{2.181611in}}{\pgfqpoint{3.905098in}{2.181611in}}%
\pgfpathcurveto{\pgfqpoint{3.896862in}{2.181611in}}{\pgfqpoint{3.888962in}{2.178339in}}{\pgfqpoint{3.883138in}{2.172515in}}%
\pgfpathcurveto{\pgfqpoint{3.877314in}{2.166691in}}{\pgfqpoint{3.874042in}{2.158791in}}{\pgfqpoint{3.874042in}{2.150555in}}%
\pgfpathcurveto{\pgfqpoint{3.874042in}{2.142319in}}{\pgfqpoint{3.877314in}{2.134419in}}{\pgfqpoint{3.883138in}{2.128595in}}%
\pgfpathcurveto{\pgfqpoint{3.888962in}{2.122771in}}{\pgfqpoint{3.896862in}{2.119498in}}{\pgfqpoint{3.905098in}{2.119498in}}%
\pgfpathclose%
\pgfusepath{stroke,fill}%
\end{pgfscope}%
\begin{pgfscope}%
\pgfpathrectangle{\pgfqpoint{3.793912in}{0.557870in}}{\pgfqpoint{2.446088in}{1.684734in}}%
\pgfusepath{clip}%
\pgfsetbuttcap%
\pgfsetroundjoin%
\definecolor{currentfill}{rgb}{0.298039,0.447059,0.690196}%
\pgfsetfillcolor{currentfill}%
\pgfsetlinewidth{1.003750pt}%
\definecolor{currentstroke}{rgb}{0.298039,0.447059,0.690196}%
\pgfsetstrokecolor{currentstroke}%
\pgfsetdash{}{0pt}%
\pgfpathmoveto{\pgfqpoint{3.905098in}{2.119498in}}%
\pgfpathcurveto{\pgfqpoint{3.913334in}{2.119498in}}{\pgfqpoint{3.921234in}{2.122771in}}{\pgfqpoint{3.927058in}{2.128595in}}%
\pgfpathcurveto{\pgfqpoint{3.932882in}{2.134419in}}{\pgfqpoint{3.936155in}{2.142319in}}{\pgfqpoint{3.936155in}{2.150555in}}%
\pgfpathcurveto{\pgfqpoint{3.936155in}{2.158791in}}{\pgfqpoint{3.932882in}{2.166691in}}{\pgfqpoint{3.927058in}{2.172515in}}%
\pgfpathcurveto{\pgfqpoint{3.921234in}{2.178339in}}{\pgfqpoint{3.913334in}{2.181611in}}{\pgfqpoint{3.905098in}{2.181611in}}%
\pgfpathcurveto{\pgfqpoint{3.896862in}{2.181611in}}{\pgfqpoint{3.888962in}{2.178339in}}{\pgfqpoint{3.883138in}{2.172515in}}%
\pgfpathcurveto{\pgfqpoint{3.877314in}{2.166691in}}{\pgfqpoint{3.874042in}{2.158791in}}{\pgfqpoint{3.874042in}{2.150555in}}%
\pgfpathcurveto{\pgfqpoint{3.874042in}{2.142319in}}{\pgfqpoint{3.877314in}{2.134419in}}{\pgfqpoint{3.883138in}{2.128595in}}%
\pgfpathcurveto{\pgfqpoint{3.888962in}{2.122771in}}{\pgfqpoint{3.896862in}{2.119498in}}{\pgfqpoint{3.905098in}{2.119498in}}%
\pgfpathclose%
\pgfusepath{stroke,fill}%
\end{pgfscope}%
\begin{pgfscope}%
\pgfpathrectangle{\pgfqpoint{3.793912in}{0.557870in}}{\pgfqpoint{2.446088in}{1.684734in}}%
\pgfusepath{clip}%
\pgfsetbuttcap%
\pgfsetroundjoin%
\definecolor{currentfill}{rgb}{0.298039,0.447059,0.690196}%
\pgfsetfillcolor{currentfill}%
\pgfsetlinewidth{1.003750pt}%
\definecolor{currentstroke}{rgb}{0.298039,0.447059,0.690196}%
\pgfsetstrokecolor{currentstroke}%
\pgfsetdash{}{0pt}%
\pgfpathmoveto{\pgfqpoint{3.905098in}{2.119498in}}%
\pgfpathcurveto{\pgfqpoint{3.913334in}{2.119498in}}{\pgfqpoint{3.921234in}{2.122771in}}{\pgfqpoint{3.927058in}{2.128595in}}%
\pgfpathcurveto{\pgfqpoint{3.932882in}{2.134419in}}{\pgfqpoint{3.936155in}{2.142319in}}{\pgfqpoint{3.936155in}{2.150555in}}%
\pgfpathcurveto{\pgfqpoint{3.936155in}{2.158791in}}{\pgfqpoint{3.932882in}{2.166691in}}{\pgfqpoint{3.927058in}{2.172515in}}%
\pgfpathcurveto{\pgfqpoint{3.921234in}{2.178339in}}{\pgfqpoint{3.913334in}{2.181611in}}{\pgfqpoint{3.905098in}{2.181611in}}%
\pgfpathcurveto{\pgfqpoint{3.896862in}{2.181611in}}{\pgfqpoint{3.888962in}{2.178339in}}{\pgfqpoint{3.883138in}{2.172515in}}%
\pgfpathcurveto{\pgfqpoint{3.877314in}{2.166691in}}{\pgfqpoint{3.874042in}{2.158791in}}{\pgfqpoint{3.874042in}{2.150555in}}%
\pgfpathcurveto{\pgfqpoint{3.874042in}{2.142319in}}{\pgfqpoint{3.877314in}{2.134419in}}{\pgfqpoint{3.883138in}{2.128595in}}%
\pgfpathcurveto{\pgfqpoint{3.888962in}{2.122771in}}{\pgfqpoint{3.896862in}{2.119498in}}{\pgfqpoint{3.905098in}{2.119498in}}%
\pgfpathclose%
\pgfusepath{stroke,fill}%
\end{pgfscope}%
\begin{pgfscope}%
\pgfpathrectangle{\pgfqpoint{3.793912in}{0.557870in}}{\pgfqpoint{2.446088in}{1.684734in}}%
\pgfusepath{clip}%
\pgfsetbuttcap%
\pgfsetroundjoin%
\definecolor{currentfill}{rgb}{0.298039,0.447059,0.690196}%
\pgfsetfillcolor{currentfill}%
\pgfsetlinewidth{1.003750pt}%
\definecolor{currentstroke}{rgb}{0.298039,0.447059,0.690196}%
\pgfsetstrokecolor{currentstroke}%
\pgfsetdash{}{0pt}%
\pgfpathmoveto{\pgfqpoint{5.303517in}{1.330505in}}%
\pgfpathcurveto{\pgfqpoint{5.311754in}{1.330505in}}{\pgfqpoint{5.319654in}{1.333777in}}{\pgfqpoint{5.325478in}{1.339601in}}%
\pgfpathcurveto{\pgfqpoint{5.331302in}{1.345425in}}{\pgfqpoint{5.334574in}{1.353325in}}{\pgfqpoint{5.334574in}{1.361561in}}%
\pgfpathcurveto{\pgfqpoint{5.334574in}{1.369797in}}{\pgfqpoint{5.331302in}{1.377697in}}{\pgfqpoint{5.325478in}{1.383521in}}%
\pgfpathcurveto{\pgfqpoint{5.319654in}{1.389345in}}{\pgfqpoint{5.311754in}{1.392618in}}{\pgfqpoint{5.303517in}{1.392618in}}%
\pgfpathcurveto{\pgfqpoint{5.295281in}{1.392618in}}{\pgfqpoint{5.287381in}{1.389345in}}{\pgfqpoint{5.281557in}{1.383521in}}%
\pgfpathcurveto{\pgfqpoint{5.275733in}{1.377697in}}{\pgfqpoint{5.272461in}{1.369797in}}{\pgfqpoint{5.272461in}{1.361561in}}%
\pgfpathcurveto{\pgfqpoint{5.272461in}{1.353325in}}{\pgfqpoint{5.275733in}{1.345425in}}{\pgfqpoint{5.281557in}{1.339601in}}%
\pgfpathcurveto{\pgfqpoint{5.287381in}{1.333777in}}{\pgfqpoint{5.295281in}{1.330505in}}{\pgfqpoint{5.303517in}{1.330505in}}%
\pgfpathclose%
\pgfusepath{stroke,fill}%
\end{pgfscope}%
\begin{pgfscope}%
\pgfpathrectangle{\pgfqpoint{3.793912in}{0.557870in}}{\pgfqpoint{2.446088in}{1.684734in}}%
\pgfusepath{clip}%
\pgfsetbuttcap%
\pgfsetroundjoin%
\definecolor{currentfill}{rgb}{0.298039,0.447059,0.690196}%
\pgfsetfillcolor{currentfill}%
\pgfsetlinewidth{1.003750pt}%
\definecolor{currentstroke}{rgb}{0.298039,0.447059,0.690196}%
\pgfsetstrokecolor{currentstroke}%
\pgfsetdash{}{0pt}%
\pgfpathmoveto{\pgfqpoint{3.905098in}{2.119498in}}%
\pgfpathcurveto{\pgfqpoint{3.913334in}{2.119498in}}{\pgfqpoint{3.921234in}{2.122771in}}{\pgfqpoint{3.927058in}{2.128595in}}%
\pgfpathcurveto{\pgfqpoint{3.932882in}{2.134419in}}{\pgfqpoint{3.936155in}{2.142319in}}{\pgfqpoint{3.936155in}{2.150555in}}%
\pgfpathcurveto{\pgfqpoint{3.936155in}{2.158791in}}{\pgfqpoint{3.932882in}{2.166691in}}{\pgfqpoint{3.927058in}{2.172515in}}%
\pgfpathcurveto{\pgfqpoint{3.921234in}{2.178339in}}{\pgfqpoint{3.913334in}{2.181611in}}{\pgfqpoint{3.905098in}{2.181611in}}%
\pgfpathcurveto{\pgfqpoint{3.896862in}{2.181611in}}{\pgfqpoint{3.888962in}{2.178339in}}{\pgfqpoint{3.883138in}{2.172515in}}%
\pgfpathcurveto{\pgfqpoint{3.877314in}{2.166691in}}{\pgfqpoint{3.874042in}{2.158791in}}{\pgfqpoint{3.874042in}{2.150555in}}%
\pgfpathcurveto{\pgfqpoint{3.874042in}{2.142319in}}{\pgfqpoint{3.877314in}{2.134419in}}{\pgfqpoint{3.883138in}{2.128595in}}%
\pgfpathcurveto{\pgfqpoint{3.888962in}{2.122771in}}{\pgfqpoint{3.896862in}{2.119498in}}{\pgfqpoint{3.905098in}{2.119498in}}%
\pgfpathclose%
\pgfusepath{stroke,fill}%
\end{pgfscope}%
\begin{pgfscope}%
\pgfpathrectangle{\pgfqpoint{3.793912in}{0.557870in}}{\pgfqpoint{2.446088in}{1.684734in}}%
\pgfusepath{clip}%
\pgfsetbuttcap%
\pgfsetroundjoin%
\definecolor{currentfill}{rgb}{0.298039,0.447059,0.690196}%
\pgfsetfillcolor{currentfill}%
\pgfsetlinewidth{1.003750pt}%
\definecolor{currentstroke}{rgb}{0.298039,0.447059,0.690196}%
\pgfsetstrokecolor{currentstroke}%
\pgfsetdash{}{0pt}%
\pgfpathmoveto{\pgfqpoint{5.441067in}{1.113918in}}%
\pgfpathcurveto{\pgfqpoint{5.449303in}{1.113918in}}{\pgfqpoint{5.457203in}{1.117190in}}{\pgfqpoint{5.463027in}{1.123014in}}%
\pgfpathcurveto{\pgfqpoint{5.468851in}{1.128838in}}{\pgfqpoint{5.472123in}{1.136738in}}{\pgfqpoint{5.472123in}{1.144975in}}%
\pgfpathcurveto{\pgfqpoint{5.472123in}{1.153211in}}{\pgfqpoint{5.468851in}{1.161111in}}{\pgfqpoint{5.463027in}{1.166935in}}%
\pgfpathcurveto{\pgfqpoint{5.457203in}{1.172759in}}{\pgfqpoint{5.449303in}{1.176031in}}{\pgfqpoint{5.441067in}{1.176031in}}%
\pgfpathcurveto{\pgfqpoint{5.432831in}{1.176031in}}{\pgfqpoint{5.424931in}{1.172759in}}{\pgfqpoint{5.419107in}{1.166935in}}%
\pgfpathcurveto{\pgfqpoint{5.413283in}{1.161111in}}{\pgfqpoint{5.410010in}{1.153211in}}{\pgfqpoint{5.410010in}{1.144975in}}%
\pgfpathcurveto{\pgfqpoint{5.410010in}{1.136738in}}{\pgfqpoint{5.413283in}{1.128838in}}{\pgfqpoint{5.419107in}{1.123014in}}%
\pgfpathcurveto{\pgfqpoint{5.424931in}{1.117190in}}{\pgfqpoint{5.432831in}{1.113918in}}{\pgfqpoint{5.441067in}{1.113918in}}%
\pgfpathclose%
\pgfusepath{stroke,fill}%
\end{pgfscope}%
\begin{pgfscope}%
\pgfpathrectangle{\pgfqpoint{3.793912in}{0.557870in}}{\pgfqpoint{2.446088in}{1.684734in}}%
\pgfusepath{clip}%
\pgfsetbuttcap%
\pgfsetroundjoin%
\definecolor{currentfill}{rgb}{0.298039,0.447059,0.690196}%
\pgfsetfillcolor{currentfill}%
\pgfsetlinewidth{1.003750pt}%
\definecolor{currentstroke}{rgb}{0.298039,0.447059,0.690196}%
\pgfsetstrokecolor{currentstroke}%
\pgfsetdash{}{0pt}%
\pgfpathmoveto{\pgfqpoint{3.905098in}{2.119498in}}%
\pgfpathcurveto{\pgfqpoint{3.913334in}{2.119498in}}{\pgfqpoint{3.921234in}{2.122771in}}{\pgfqpoint{3.927058in}{2.128595in}}%
\pgfpathcurveto{\pgfqpoint{3.932882in}{2.134419in}}{\pgfqpoint{3.936155in}{2.142319in}}{\pgfqpoint{3.936155in}{2.150555in}}%
\pgfpathcurveto{\pgfqpoint{3.936155in}{2.158791in}}{\pgfqpoint{3.932882in}{2.166691in}}{\pgfqpoint{3.927058in}{2.172515in}}%
\pgfpathcurveto{\pgfqpoint{3.921234in}{2.178339in}}{\pgfqpoint{3.913334in}{2.181611in}}{\pgfqpoint{3.905098in}{2.181611in}}%
\pgfpathcurveto{\pgfqpoint{3.896862in}{2.181611in}}{\pgfqpoint{3.888962in}{2.178339in}}{\pgfqpoint{3.883138in}{2.172515in}}%
\pgfpathcurveto{\pgfqpoint{3.877314in}{2.166691in}}{\pgfqpoint{3.874042in}{2.158791in}}{\pgfqpoint{3.874042in}{2.150555in}}%
\pgfpathcurveto{\pgfqpoint{3.874042in}{2.142319in}}{\pgfqpoint{3.877314in}{2.134419in}}{\pgfqpoint{3.883138in}{2.128595in}}%
\pgfpathcurveto{\pgfqpoint{3.888962in}{2.122771in}}{\pgfqpoint{3.896862in}{2.119498in}}{\pgfqpoint{3.905098in}{2.119498in}}%
\pgfpathclose%
\pgfusepath{stroke,fill}%
\end{pgfscope}%
\begin{pgfscope}%
\pgfpathrectangle{\pgfqpoint{3.793912in}{0.557870in}}{\pgfqpoint{2.446088in}{1.684734in}}%
\pgfusepath{clip}%
\pgfsetbuttcap%
\pgfsetroundjoin%
\definecolor{currentfill}{rgb}{0.298039,0.447059,0.690196}%
\pgfsetfillcolor{currentfill}%
\pgfsetlinewidth{1.003750pt}%
\definecolor{currentstroke}{rgb}{0.298039,0.447059,0.690196}%
\pgfsetstrokecolor{currentstroke}%
\pgfsetdash{}{0pt}%
\pgfpathmoveto{\pgfqpoint{3.905098in}{2.119498in}}%
\pgfpathcurveto{\pgfqpoint{3.913334in}{2.119498in}}{\pgfqpoint{3.921234in}{2.122771in}}{\pgfqpoint{3.927058in}{2.128595in}}%
\pgfpathcurveto{\pgfqpoint{3.932882in}{2.134419in}}{\pgfqpoint{3.936155in}{2.142319in}}{\pgfqpoint{3.936155in}{2.150555in}}%
\pgfpathcurveto{\pgfqpoint{3.936155in}{2.158791in}}{\pgfqpoint{3.932882in}{2.166691in}}{\pgfqpoint{3.927058in}{2.172515in}}%
\pgfpathcurveto{\pgfqpoint{3.921234in}{2.178339in}}{\pgfqpoint{3.913334in}{2.181611in}}{\pgfqpoint{3.905098in}{2.181611in}}%
\pgfpathcurveto{\pgfqpoint{3.896862in}{2.181611in}}{\pgfqpoint{3.888962in}{2.178339in}}{\pgfqpoint{3.883138in}{2.172515in}}%
\pgfpathcurveto{\pgfqpoint{3.877314in}{2.166691in}}{\pgfqpoint{3.874042in}{2.158791in}}{\pgfqpoint{3.874042in}{2.150555in}}%
\pgfpathcurveto{\pgfqpoint{3.874042in}{2.142319in}}{\pgfqpoint{3.877314in}{2.134419in}}{\pgfqpoint{3.883138in}{2.128595in}}%
\pgfpathcurveto{\pgfqpoint{3.888962in}{2.122771in}}{\pgfqpoint{3.896862in}{2.119498in}}{\pgfqpoint{3.905098in}{2.119498in}}%
\pgfpathclose%
\pgfusepath{stroke,fill}%
\end{pgfscope}%
\begin{pgfscope}%
\pgfpathrectangle{\pgfqpoint{3.793912in}{0.557870in}}{\pgfqpoint{2.446088in}{1.684734in}}%
\pgfusepath{clip}%
\pgfsetbuttcap%
\pgfsetroundjoin%
\definecolor{currentfill}{rgb}{0.298039,0.447059,0.690196}%
\pgfsetfillcolor{currentfill}%
\pgfsetlinewidth{1.003750pt}%
\definecolor{currentstroke}{rgb}{0.298039,0.447059,0.690196}%
\pgfsetstrokecolor{currentstroke}%
\pgfsetdash{}{0pt}%
\pgfpathmoveto{\pgfqpoint{3.905098in}{2.119498in}}%
\pgfpathcurveto{\pgfqpoint{3.913334in}{2.119498in}}{\pgfqpoint{3.921234in}{2.122771in}}{\pgfqpoint{3.927058in}{2.128595in}}%
\pgfpathcurveto{\pgfqpoint{3.932882in}{2.134419in}}{\pgfqpoint{3.936155in}{2.142319in}}{\pgfqpoint{3.936155in}{2.150555in}}%
\pgfpathcurveto{\pgfqpoint{3.936155in}{2.158791in}}{\pgfqpoint{3.932882in}{2.166691in}}{\pgfqpoint{3.927058in}{2.172515in}}%
\pgfpathcurveto{\pgfqpoint{3.921234in}{2.178339in}}{\pgfqpoint{3.913334in}{2.181611in}}{\pgfqpoint{3.905098in}{2.181611in}}%
\pgfpathcurveto{\pgfqpoint{3.896862in}{2.181611in}}{\pgfqpoint{3.888962in}{2.178339in}}{\pgfqpoint{3.883138in}{2.172515in}}%
\pgfpathcurveto{\pgfqpoint{3.877314in}{2.166691in}}{\pgfqpoint{3.874042in}{2.158791in}}{\pgfqpoint{3.874042in}{2.150555in}}%
\pgfpathcurveto{\pgfqpoint{3.874042in}{2.142319in}}{\pgfqpoint{3.877314in}{2.134419in}}{\pgfqpoint{3.883138in}{2.128595in}}%
\pgfpathcurveto{\pgfqpoint{3.888962in}{2.122771in}}{\pgfqpoint{3.896862in}{2.119498in}}{\pgfqpoint{3.905098in}{2.119498in}}%
\pgfpathclose%
\pgfusepath{stroke,fill}%
\end{pgfscope}%
\begin{pgfscope}%
\pgfpathrectangle{\pgfqpoint{3.793912in}{0.557870in}}{\pgfqpoint{2.446088in}{1.684734in}}%
\pgfusepath{clip}%
\pgfsetbuttcap%
\pgfsetroundjoin%
\definecolor{currentfill}{rgb}{0.298039,0.447059,0.690196}%
\pgfsetfillcolor{currentfill}%
\pgfsetlinewidth{1.003750pt}%
\definecolor{currentstroke}{rgb}{0.298039,0.447059,0.690196}%
\pgfsetstrokecolor{currentstroke}%
\pgfsetdash{}{0pt}%
\pgfpathmoveto{\pgfqpoint{5.372292in}{1.748207in}}%
\pgfpathcurveto{\pgfqpoint{5.380528in}{1.748207in}}{\pgfqpoint{5.388429in}{1.751479in}}{\pgfqpoint{5.394252in}{1.757303in}}%
\pgfpathcurveto{\pgfqpoint{5.400076in}{1.763127in}}{\pgfqpoint{5.403349in}{1.771027in}}{\pgfqpoint{5.403349in}{1.779264in}}%
\pgfpathcurveto{\pgfqpoint{5.403349in}{1.787500in}}{\pgfqpoint{5.400076in}{1.795400in}}{\pgfqpoint{5.394252in}{1.801224in}}%
\pgfpathcurveto{\pgfqpoint{5.388429in}{1.807048in}}{\pgfqpoint{5.380528in}{1.810320in}}{\pgfqpoint{5.372292in}{1.810320in}}%
\pgfpathcurveto{\pgfqpoint{5.364056in}{1.810320in}}{\pgfqpoint{5.356156in}{1.807048in}}{\pgfqpoint{5.350332in}{1.801224in}}%
\pgfpathcurveto{\pgfqpoint{5.344508in}{1.795400in}}{\pgfqpoint{5.341236in}{1.787500in}}{\pgfqpoint{5.341236in}{1.779264in}}%
\pgfpathcurveto{\pgfqpoint{5.341236in}{1.771027in}}{\pgfqpoint{5.344508in}{1.763127in}}{\pgfqpoint{5.350332in}{1.757303in}}%
\pgfpathcurveto{\pgfqpoint{5.356156in}{1.751479in}}{\pgfqpoint{5.364056in}{1.748207in}}{\pgfqpoint{5.372292in}{1.748207in}}%
\pgfpathclose%
\pgfusepath{stroke,fill}%
\end{pgfscope}%
\begin{pgfscope}%
\pgfpathrectangle{\pgfqpoint{3.793912in}{0.557870in}}{\pgfqpoint{2.446088in}{1.684734in}}%
\pgfusepath{clip}%
\pgfsetbuttcap%
\pgfsetroundjoin%
\definecolor{currentfill}{rgb}{0.298039,0.447059,0.690196}%
\pgfsetfillcolor{currentfill}%
\pgfsetlinewidth{1.003750pt}%
\definecolor{currentstroke}{rgb}{0.298039,0.447059,0.690196}%
\pgfsetstrokecolor{currentstroke}%
\pgfsetdash{}{0pt}%
\pgfpathmoveto{\pgfqpoint{5.784941in}{1.268623in}}%
\pgfpathcurveto{\pgfqpoint{5.793177in}{1.268623in}}{\pgfqpoint{5.801077in}{1.271895in}}{\pgfqpoint{5.806901in}{1.277719in}}%
\pgfpathcurveto{\pgfqpoint{5.812725in}{1.283543in}}{\pgfqpoint{5.815997in}{1.291443in}}{\pgfqpoint{5.815997in}{1.299679in}}%
\pgfpathcurveto{\pgfqpoint{5.815997in}{1.307915in}}{\pgfqpoint{5.812725in}{1.315816in}}{\pgfqpoint{5.806901in}{1.321639in}}%
\pgfpathcurveto{\pgfqpoint{5.801077in}{1.327463in}}{\pgfqpoint{5.793177in}{1.330736in}}{\pgfqpoint{5.784941in}{1.330736in}}%
\pgfpathcurveto{\pgfqpoint{5.776704in}{1.330736in}}{\pgfqpoint{5.768804in}{1.327463in}}{\pgfqpoint{5.762980in}{1.321639in}}%
\pgfpathcurveto{\pgfqpoint{5.757156in}{1.315816in}}{\pgfqpoint{5.753884in}{1.307915in}}{\pgfqpoint{5.753884in}{1.299679in}}%
\pgfpathcurveto{\pgfqpoint{5.753884in}{1.291443in}}{\pgfqpoint{5.757156in}{1.283543in}}{\pgfqpoint{5.762980in}{1.277719in}}%
\pgfpathcurveto{\pgfqpoint{5.768804in}{1.271895in}}{\pgfqpoint{5.776704in}{1.268623in}}{\pgfqpoint{5.784941in}{1.268623in}}%
\pgfpathclose%
\pgfusepath{stroke,fill}%
\end{pgfscope}%
\begin{pgfscope}%
\pgfpathrectangle{\pgfqpoint{3.793912in}{0.557870in}}{\pgfqpoint{2.446088in}{1.684734in}}%
\pgfusepath{clip}%
\pgfsetbuttcap%
\pgfsetroundjoin%
\definecolor{currentfill}{rgb}{0.298039,0.447059,0.690196}%
\pgfsetfillcolor{currentfill}%
\pgfsetlinewidth{1.003750pt}%
\definecolor{currentstroke}{rgb}{0.298039,0.447059,0.690196}%
\pgfsetstrokecolor{currentstroke}%
\pgfsetdash{}{0pt}%
\pgfpathmoveto{\pgfqpoint{5.372292in}{1.794619in}}%
\pgfpathcurveto{\pgfqpoint{5.380528in}{1.794619in}}{\pgfqpoint{5.388429in}{1.797891in}}{\pgfqpoint{5.394252in}{1.803715in}}%
\pgfpathcurveto{\pgfqpoint{5.400076in}{1.809539in}}{\pgfqpoint{5.403349in}{1.817439in}}{\pgfqpoint{5.403349in}{1.825675in}}%
\pgfpathcurveto{\pgfqpoint{5.403349in}{1.833911in}}{\pgfqpoint{5.400076in}{1.841811in}}{\pgfqpoint{5.394252in}{1.847635in}}%
\pgfpathcurveto{\pgfqpoint{5.388429in}{1.853459in}}{\pgfqpoint{5.380528in}{1.856732in}}{\pgfqpoint{5.372292in}{1.856732in}}%
\pgfpathcurveto{\pgfqpoint{5.364056in}{1.856732in}}{\pgfqpoint{5.356156in}{1.853459in}}{\pgfqpoint{5.350332in}{1.847635in}}%
\pgfpathcurveto{\pgfqpoint{5.344508in}{1.841811in}}{\pgfqpoint{5.341236in}{1.833911in}}{\pgfqpoint{5.341236in}{1.825675in}}%
\pgfpathcurveto{\pgfqpoint{5.341236in}{1.817439in}}{\pgfqpoint{5.344508in}{1.809539in}}{\pgfqpoint{5.350332in}{1.803715in}}%
\pgfpathcurveto{\pgfqpoint{5.356156in}{1.797891in}}{\pgfqpoint{5.364056in}{1.794619in}}{\pgfqpoint{5.372292in}{1.794619in}}%
\pgfpathclose%
\pgfusepath{stroke,fill}%
\end{pgfscope}%
\begin{pgfscope}%
\pgfpathrectangle{\pgfqpoint{3.793912in}{0.557870in}}{\pgfqpoint{2.446088in}{1.684734in}}%
\pgfusepath{clip}%
\pgfsetbuttcap%
\pgfsetroundjoin%
\definecolor{currentfill}{rgb}{0.298039,0.447059,0.690196}%
\pgfsetfillcolor{currentfill}%
\pgfsetlinewidth{1.003750pt}%
\definecolor{currentstroke}{rgb}{0.298039,0.447059,0.690196}%
\pgfsetstrokecolor{currentstroke}%
\pgfsetdash{}{0pt}%
\pgfpathmoveto{\pgfqpoint{5.991265in}{0.974684in}}%
\pgfpathcurveto{\pgfqpoint{5.999501in}{0.974684in}}{\pgfqpoint{6.007401in}{0.977956in}}{\pgfqpoint{6.013225in}{0.983780in}}%
\pgfpathcurveto{\pgfqpoint{6.019049in}{0.989604in}}{\pgfqpoint{6.022321in}{0.997504in}}{\pgfqpoint{6.022321in}{1.005740in}}%
\pgfpathcurveto{\pgfqpoint{6.022321in}{1.013977in}}{\pgfqpoint{6.019049in}{1.021877in}}{\pgfqpoint{6.013225in}{1.027701in}}%
\pgfpathcurveto{\pgfqpoint{6.007401in}{1.033524in}}{\pgfqpoint{5.999501in}{1.036797in}}{\pgfqpoint{5.991265in}{1.036797in}}%
\pgfpathcurveto{\pgfqpoint{5.983028in}{1.036797in}}{\pgfqpoint{5.975128in}{1.033524in}}{\pgfqpoint{5.969304in}{1.027701in}}%
\pgfpathcurveto{\pgfqpoint{5.963481in}{1.021877in}}{\pgfqpoint{5.960208in}{1.013977in}}{\pgfqpoint{5.960208in}{1.005740in}}%
\pgfpathcurveto{\pgfqpoint{5.960208in}{0.997504in}}{\pgfqpoint{5.963481in}{0.989604in}}{\pgfqpoint{5.969304in}{0.983780in}}%
\pgfpathcurveto{\pgfqpoint{5.975128in}{0.977956in}}{\pgfqpoint{5.983028in}{0.974684in}}{\pgfqpoint{5.991265in}{0.974684in}}%
\pgfpathclose%
\pgfusepath{stroke,fill}%
\end{pgfscope}%
\begin{pgfscope}%
\pgfpathrectangle{\pgfqpoint{3.793912in}{0.557870in}}{\pgfqpoint{2.446088in}{1.684734in}}%
\pgfusepath{clip}%
\pgfsetbuttcap%
\pgfsetroundjoin%
\definecolor{currentfill}{rgb}{0.298039,0.447059,0.690196}%
\pgfsetfillcolor{currentfill}%
\pgfsetlinewidth{1.003750pt}%
\definecolor{currentstroke}{rgb}{0.298039,0.447059,0.690196}%
\pgfsetstrokecolor{currentstroke}%
\pgfsetdash{}{0pt}%
\pgfpathmoveto{\pgfqpoint{5.991265in}{0.974684in}}%
\pgfpathcurveto{\pgfqpoint{5.999501in}{0.974684in}}{\pgfqpoint{6.007401in}{0.977956in}}{\pgfqpoint{6.013225in}{0.983780in}}%
\pgfpathcurveto{\pgfqpoint{6.019049in}{0.989604in}}{\pgfqpoint{6.022321in}{0.997504in}}{\pgfqpoint{6.022321in}{1.005740in}}%
\pgfpathcurveto{\pgfqpoint{6.022321in}{1.013977in}}{\pgfqpoint{6.019049in}{1.021877in}}{\pgfqpoint{6.013225in}{1.027701in}}%
\pgfpathcurveto{\pgfqpoint{6.007401in}{1.033524in}}{\pgfqpoint{5.999501in}{1.036797in}}{\pgfqpoint{5.991265in}{1.036797in}}%
\pgfpathcurveto{\pgfqpoint{5.983028in}{1.036797in}}{\pgfqpoint{5.975128in}{1.033524in}}{\pgfqpoint{5.969304in}{1.027701in}}%
\pgfpathcurveto{\pgfqpoint{5.963481in}{1.021877in}}{\pgfqpoint{5.960208in}{1.013977in}}{\pgfqpoint{5.960208in}{1.005740in}}%
\pgfpathcurveto{\pgfqpoint{5.960208in}{0.997504in}}{\pgfqpoint{5.963481in}{0.989604in}}{\pgfqpoint{5.969304in}{0.983780in}}%
\pgfpathcurveto{\pgfqpoint{5.975128in}{0.977956in}}{\pgfqpoint{5.983028in}{0.974684in}}{\pgfqpoint{5.991265in}{0.974684in}}%
\pgfpathclose%
\pgfusepath{stroke,fill}%
\end{pgfscope}%
\begin{pgfscope}%
\pgfpathrectangle{\pgfqpoint{3.793912in}{0.557870in}}{\pgfqpoint{2.446088in}{1.684734in}}%
\pgfusepath{clip}%
\pgfsetbuttcap%
\pgfsetroundjoin%
\definecolor{currentfill}{rgb}{0.298039,0.447059,0.690196}%
\pgfsetfillcolor{currentfill}%
\pgfsetlinewidth{1.003750pt}%
\definecolor{currentstroke}{rgb}{0.298039,0.447059,0.690196}%
\pgfsetstrokecolor{currentstroke}%
\pgfsetdash{}{0pt}%
\pgfpathmoveto{\pgfqpoint{5.463992in}{1.763678in}}%
\pgfpathcurveto{\pgfqpoint{5.472228in}{1.763678in}}{\pgfqpoint{5.480128in}{1.766950in}}{\pgfqpoint{5.485952in}{1.772774in}}%
\pgfpathcurveto{\pgfqpoint{5.491776in}{1.778598in}}{\pgfqpoint{5.495048in}{1.786498in}}{\pgfqpoint{5.495048in}{1.794734in}}%
\pgfpathcurveto{\pgfqpoint{5.495048in}{1.802970in}}{\pgfqpoint{5.491776in}{1.810870in}}{\pgfqpoint{5.485952in}{1.816694in}}%
\pgfpathcurveto{\pgfqpoint{5.480128in}{1.822518in}}{\pgfqpoint{5.472228in}{1.825791in}}{\pgfqpoint{5.463992in}{1.825791in}}%
\pgfpathcurveto{\pgfqpoint{5.455756in}{1.825791in}}{\pgfqpoint{5.447856in}{1.822518in}}{\pgfqpoint{5.442032in}{1.816694in}}%
\pgfpathcurveto{\pgfqpoint{5.436208in}{1.810870in}}{\pgfqpoint{5.432935in}{1.802970in}}{\pgfqpoint{5.432935in}{1.794734in}}%
\pgfpathcurveto{\pgfqpoint{5.432935in}{1.786498in}}{\pgfqpoint{5.436208in}{1.778598in}}{\pgfqpoint{5.442032in}{1.772774in}}%
\pgfpathcurveto{\pgfqpoint{5.447856in}{1.766950in}}{\pgfqpoint{5.455756in}{1.763678in}}{\pgfqpoint{5.463992in}{1.763678in}}%
\pgfpathclose%
\pgfusepath{stroke,fill}%
\end{pgfscope}%
\begin{pgfscope}%
\pgfpathrectangle{\pgfqpoint{3.793912in}{0.557870in}}{\pgfqpoint{2.446088in}{1.684734in}}%
\pgfusepath{clip}%
\pgfsetbuttcap%
\pgfsetroundjoin%
\definecolor{currentfill}{rgb}{0.298039,0.447059,0.690196}%
\pgfsetfillcolor{currentfill}%
\pgfsetlinewidth{1.003750pt}%
\definecolor{currentstroke}{rgb}{0.298039,0.447059,0.690196}%
\pgfsetstrokecolor{currentstroke}%
\pgfsetdash{}{0pt}%
\pgfpathmoveto{\pgfqpoint{5.349367in}{1.902912in}}%
\pgfpathcurveto{\pgfqpoint{5.357604in}{1.902912in}}{\pgfqpoint{5.365504in}{1.906184in}}{\pgfqpoint{5.371328in}{1.912008in}}%
\pgfpathcurveto{\pgfqpoint{5.377151in}{1.917832in}}{\pgfqpoint{5.380424in}{1.925732in}}{\pgfqpoint{5.380424in}{1.933968in}}%
\pgfpathcurveto{\pgfqpoint{5.380424in}{1.942205in}}{\pgfqpoint{5.377151in}{1.950105in}}{\pgfqpoint{5.371328in}{1.955929in}}%
\pgfpathcurveto{\pgfqpoint{5.365504in}{1.961753in}}{\pgfqpoint{5.357604in}{1.965025in}}{\pgfqpoint{5.349367in}{1.965025in}}%
\pgfpathcurveto{\pgfqpoint{5.341131in}{1.965025in}}{\pgfqpoint{5.333231in}{1.961753in}}{\pgfqpoint{5.327407in}{1.955929in}}%
\pgfpathcurveto{\pgfqpoint{5.321583in}{1.950105in}}{\pgfqpoint{5.318311in}{1.942205in}}{\pgfqpoint{5.318311in}{1.933968in}}%
\pgfpathcurveto{\pgfqpoint{5.318311in}{1.925732in}}{\pgfqpoint{5.321583in}{1.917832in}}{\pgfqpoint{5.327407in}{1.912008in}}%
\pgfpathcurveto{\pgfqpoint{5.333231in}{1.906184in}}{\pgfqpoint{5.341131in}{1.902912in}}{\pgfqpoint{5.349367in}{1.902912in}}%
\pgfpathclose%
\pgfusepath{stroke,fill}%
\end{pgfscope}%
\begin{pgfscope}%
\pgfpathrectangle{\pgfqpoint{3.793912in}{0.557870in}}{\pgfqpoint{2.446088in}{1.684734in}}%
\pgfusepath{clip}%
\pgfsetbuttcap%
\pgfsetroundjoin%
\definecolor{currentfill}{rgb}{0.298039,0.447059,0.690196}%
\pgfsetfillcolor{currentfill}%
\pgfsetlinewidth{1.003750pt}%
\definecolor{currentstroke}{rgb}{0.298039,0.447059,0.690196}%
\pgfsetstrokecolor{currentstroke}%
\pgfsetdash{}{0pt}%
\pgfpathmoveto{\pgfqpoint{5.349367in}{1.841030in}}%
\pgfpathcurveto{\pgfqpoint{5.357604in}{1.841030in}}{\pgfqpoint{5.365504in}{1.844302in}}{\pgfqpoint{5.371328in}{1.850126in}}%
\pgfpathcurveto{\pgfqpoint{5.377151in}{1.855950in}}{\pgfqpoint{5.380424in}{1.863850in}}{\pgfqpoint{5.380424in}{1.872086in}}%
\pgfpathcurveto{\pgfqpoint{5.380424in}{1.880323in}}{\pgfqpoint{5.377151in}{1.888223in}}{\pgfqpoint{5.371328in}{1.894047in}}%
\pgfpathcurveto{\pgfqpoint{5.365504in}{1.899871in}}{\pgfqpoint{5.357604in}{1.903143in}}{\pgfqpoint{5.349367in}{1.903143in}}%
\pgfpathcurveto{\pgfqpoint{5.341131in}{1.903143in}}{\pgfqpoint{5.333231in}{1.899871in}}{\pgfqpoint{5.327407in}{1.894047in}}%
\pgfpathcurveto{\pgfqpoint{5.321583in}{1.888223in}}{\pgfqpoint{5.318311in}{1.880323in}}{\pgfqpoint{5.318311in}{1.872086in}}%
\pgfpathcurveto{\pgfqpoint{5.318311in}{1.863850in}}{\pgfqpoint{5.321583in}{1.855950in}}{\pgfqpoint{5.327407in}{1.850126in}}%
\pgfpathcurveto{\pgfqpoint{5.333231in}{1.844302in}}{\pgfqpoint{5.341131in}{1.841030in}}{\pgfqpoint{5.349367in}{1.841030in}}%
\pgfpathclose%
\pgfusepath{stroke,fill}%
\end{pgfscope}%
\begin{pgfscope}%
\pgfpathrectangle{\pgfqpoint{3.793912in}{0.557870in}}{\pgfqpoint{2.446088in}{1.684734in}}%
\pgfusepath{clip}%
\pgfsetbuttcap%
\pgfsetroundjoin%
\definecolor{currentfill}{rgb}{0.298039,0.447059,0.690196}%
\pgfsetfillcolor{currentfill}%
\pgfsetlinewidth{1.003750pt}%
\definecolor{currentstroke}{rgb}{0.298039,0.447059,0.690196}%
\pgfsetstrokecolor{currentstroke}%
\pgfsetdash{}{0pt}%
\pgfpathmoveto{\pgfqpoint{5.326442in}{1.933853in}}%
\pgfpathcurveto{\pgfqpoint{5.334679in}{1.933853in}}{\pgfqpoint{5.342579in}{1.937125in}}{\pgfqpoint{5.348403in}{1.942949in}}%
\pgfpathcurveto{\pgfqpoint{5.354227in}{1.948773in}}{\pgfqpoint{5.357499in}{1.956673in}}{\pgfqpoint{5.357499in}{1.964909in}}%
\pgfpathcurveto{\pgfqpoint{5.357499in}{1.973146in}}{\pgfqpoint{5.354227in}{1.981046in}}{\pgfqpoint{5.348403in}{1.986870in}}%
\pgfpathcurveto{\pgfqpoint{5.342579in}{1.992693in}}{\pgfqpoint{5.334679in}{1.995966in}}{\pgfqpoint{5.326442in}{1.995966in}}%
\pgfpathcurveto{\pgfqpoint{5.318206in}{1.995966in}}{\pgfqpoint{5.310306in}{1.992693in}}{\pgfqpoint{5.304482in}{1.986870in}}%
\pgfpathcurveto{\pgfqpoint{5.298658in}{1.981046in}}{\pgfqpoint{5.295386in}{1.973146in}}{\pgfqpoint{5.295386in}{1.964909in}}%
\pgfpathcurveto{\pgfqpoint{5.295386in}{1.956673in}}{\pgfqpoint{5.298658in}{1.948773in}}{\pgfqpoint{5.304482in}{1.942949in}}%
\pgfpathcurveto{\pgfqpoint{5.310306in}{1.937125in}}{\pgfqpoint{5.318206in}{1.933853in}}{\pgfqpoint{5.326442in}{1.933853in}}%
\pgfpathclose%
\pgfusepath{stroke,fill}%
\end{pgfscope}%
\begin{pgfscope}%
\pgfpathrectangle{\pgfqpoint{3.793912in}{0.557870in}}{\pgfqpoint{2.446088in}{1.684734in}}%
\pgfusepath{clip}%
\pgfsetbuttcap%
\pgfsetroundjoin%
\definecolor{currentfill}{rgb}{0.298039,0.447059,0.690196}%
\pgfsetfillcolor{currentfill}%
\pgfsetlinewidth{1.003750pt}%
\definecolor{currentstroke}{rgb}{0.298039,0.447059,0.690196}%
\pgfsetstrokecolor{currentstroke}%
\pgfsetdash{}{0pt}%
\pgfpathmoveto{\pgfqpoint{5.601541in}{1.485209in}}%
\pgfpathcurveto{\pgfqpoint{5.609778in}{1.485209in}}{\pgfqpoint{5.617678in}{1.488482in}}{\pgfqpoint{5.623502in}{1.494305in}}%
\pgfpathcurveto{\pgfqpoint{5.629325in}{1.500129in}}{\pgfqpoint{5.632598in}{1.508029in}}{\pgfqpoint{5.632598in}{1.516266in}}%
\pgfpathcurveto{\pgfqpoint{5.632598in}{1.524502in}}{\pgfqpoint{5.629325in}{1.532402in}}{\pgfqpoint{5.623502in}{1.538226in}}%
\pgfpathcurveto{\pgfqpoint{5.617678in}{1.544050in}}{\pgfqpoint{5.609778in}{1.547322in}}{\pgfqpoint{5.601541in}{1.547322in}}%
\pgfpathcurveto{\pgfqpoint{5.593305in}{1.547322in}}{\pgfqpoint{5.585405in}{1.544050in}}{\pgfqpoint{5.579581in}{1.538226in}}%
\pgfpathcurveto{\pgfqpoint{5.573757in}{1.532402in}}{\pgfqpoint{5.570485in}{1.524502in}}{\pgfqpoint{5.570485in}{1.516266in}}%
\pgfpathcurveto{\pgfqpoint{5.570485in}{1.508029in}}{\pgfqpoint{5.573757in}{1.500129in}}{\pgfqpoint{5.579581in}{1.494305in}}%
\pgfpathcurveto{\pgfqpoint{5.585405in}{1.488482in}}{\pgfqpoint{5.593305in}{1.485209in}}{\pgfqpoint{5.601541in}{1.485209in}}%
\pgfpathclose%
\pgfusepath{stroke,fill}%
\end{pgfscope}%
\begin{pgfscope}%
\pgfpathrectangle{\pgfqpoint{3.793912in}{0.557870in}}{\pgfqpoint{2.446088in}{1.684734in}}%
\pgfusepath{clip}%
\pgfsetbuttcap%
\pgfsetroundjoin%
\definecolor{currentfill}{rgb}{0.298039,0.447059,0.690196}%
\pgfsetfillcolor{currentfill}%
\pgfsetlinewidth{1.003750pt}%
\definecolor{currentstroke}{rgb}{0.298039,0.447059,0.690196}%
\pgfsetstrokecolor{currentstroke}%
\pgfsetdash{}{0pt}%
\pgfpathmoveto{\pgfqpoint{5.601541in}{1.485209in}}%
\pgfpathcurveto{\pgfqpoint{5.609778in}{1.485209in}}{\pgfqpoint{5.617678in}{1.488482in}}{\pgfqpoint{5.623502in}{1.494305in}}%
\pgfpathcurveto{\pgfqpoint{5.629325in}{1.500129in}}{\pgfqpoint{5.632598in}{1.508029in}}{\pgfqpoint{5.632598in}{1.516266in}}%
\pgfpathcurveto{\pgfqpoint{5.632598in}{1.524502in}}{\pgfqpoint{5.629325in}{1.532402in}}{\pgfqpoint{5.623502in}{1.538226in}}%
\pgfpathcurveto{\pgfqpoint{5.617678in}{1.544050in}}{\pgfqpoint{5.609778in}{1.547322in}}{\pgfqpoint{5.601541in}{1.547322in}}%
\pgfpathcurveto{\pgfqpoint{5.593305in}{1.547322in}}{\pgfqpoint{5.585405in}{1.544050in}}{\pgfqpoint{5.579581in}{1.538226in}}%
\pgfpathcurveto{\pgfqpoint{5.573757in}{1.532402in}}{\pgfqpoint{5.570485in}{1.524502in}}{\pgfqpoint{5.570485in}{1.516266in}}%
\pgfpathcurveto{\pgfqpoint{5.570485in}{1.508029in}}{\pgfqpoint{5.573757in}{1.500129in}}{\pgfqpoint{5.579581in}{1.494305in}}%
\pgfpathcurveto{\pgfqpoint{5.585405in}{1.488482in}}{\pgfqpoint{5.593305in}{1.485209in}}{\pgfqpoint{5.601541in}{1.485209in}}%
\pgfpathclose%
\pgfusepath{stroke,fill}%
\end{pgfscope}%
\begin{pgfscope}%
\pgfpathrectangle{\pgfqpoint{3.793912in}{0.557870in}}{\pgfqpoint{2.446088in}{1.684734in}}%
\pgfusepath{clip}%
\pgfsetbuttcap%
\pgfsetroundjoin%
\definecolor{currentfill}{rgb}{0.298039,0.447059,0.690196}%
\pgfsetfillcolor{currentfill}%
\pgfsetlinewidth{1.003750pt}%
\definecolor{currentstroke}{rgb}{0.298039,0.447059,0.690196}%
\pgfsetstrokecolor{currentstroke}%
\pgfsetdash{}{0pt}%
\pgfpathmoveto{\pgfqpoint{3.905098in}{2.119498in}}%
\pgfpathcurveto{\pgfqpoint{3.913334in}{2.119498in}}{\pgfqpoint{3.921234in}{2.122771in}}{\pgfqpoint{3.927058in}{2.128595in}}%
\pgfpathcurveto{\pgfqpoint{3.932882in}{2.134419in}}{\pgfqpoint{3.936155in}{2.142319in}}{\pgfqpoint{3.936155in}{2.150555in}}%
\pgfpathcurveto{\pgfqpoint{3.936155in}{2.158791in}}{\pgfqpoint{3.932882in}{2.166691in}}{\pgfqpoint{3.927058in}{2.172515in}}%
\pgfpathcurveto{\pgfqpoint{3.921234in}{2.178339in}}{\pgfqpoint{3.913334in}{2.181611in}}{\pgfqpoint{3.905098in}{2.181611in}}%
\pgfpathcurveto{\pgfqpoint{3.896862in}{2.181611in}}{\pgfqpoint{3.888962in}{2.178339in}}{\pgfqpoint{3.883138in}{2.172515in}}%
\pgfpathcurveto{\pgfqpoint{3.877314in}{2.166691in}}{\pgfqpoint{3.874042in}{2.158791in}}{\pgfqpoint{3.874042in}{2.150555in}}%
\pgfpathcurveto{\pgfqpoint{3.874042in}{2.142319in}}{\pgfqpoint{3.877314in}{2.134419in}}{\pgfqpoint{3.883138in}{2.128595in}}%
\pgfpathcurveto{\pgfqpoint{3.888962in}{2.122771in}}{\pgfqpoint{3.896862in}{2.119498in}}{\pgfqpoint{3.905098in}{2.119498in}}%
\pgfpathclose%
\pgfusepath{stroke,fill}%
\end{pgfscope}%
\begin{pgfscope}%
\pgfpathrectangle{\pgfqpoint{3.793912in}{0.557870in}}{\pgfqpoint{2.446088in}{1.684734in}}%
\pgfusepath{clip}%
\pgfsetbuttcap%
\pgfsetroundjoin%
\definecolor{currentfill}{rgb}{0.298039,0.447059,0.690196}%
\pgfsetfillcolor{currentfill}%
\pgfsetlinewidth{1.003750pt}%
\definecolor{currentstroke}{rgb}{0.298039,0.447059,0.690196}%
\pgfsetstrokecolor{currentstroke}%
\pgfsetdash{}{0pt}%
\pgfpathmoveto{\pgfqpoint{3.905098in}{2.119498in}}%
\pgfpathcurveto{\pgfqpoint{3.913334in}{2.119498in}}{\pgfqpoint{3.921234in}{2.122771in}}{\pgfqpoint{3.927058in}{2.128595in}}%
\pgfpathcurveto{\pgfqpoint{3.932882in}{2.134419in}}{\pgfqpoint{3.936155in}{2.142319in}}{\pgfqpoint{3.936155in}{2.150555in}}%
\pgfpathcurveto{\pgfqpoint{3.936155in}{2.158791in}}{\pgfqpoint{3.932882in}{2.166691in}}{\pgfqpoint{3.927058in}{2.172515in}}%
\pgfpathcurveto{\pgfqpoint{3.921234in}{2.178339in}}{\pgfqpoint{3.913334in}{2.181611in}}{\pgfqpoint{3.905098in}{2.181611in}}%
\pgfpathcurveto{\pgfqpoint{3.896862in}{2.181611in}}{\pgfqpoint{3.888962in}{2.178339in}}{\pgfqpoint{3.883138in}{2.172515in}}%
\pgfpathcurveto{\pgfqpoint{3.877314in}{2.166691in}}{\pgfqpoint{3.874042in}{2.158791in}}{\pgfqpoint{3.874042in}{2.150555in}}%
\pgfpathcurveto{\pgfqpoint{3.874042in}{2.142319in}}{\pgfqpoint{3.877314in}{2.134419in}}{\pgfqpoint{3.883138in}{2.128595in}}%
\pgfpathcurveto{\pgfqpoint{3.888962in}{2.122771in}}{\pgfqpoint{3.896862in}{2.119498in}}{\pgfqpoint{3.905098in}{2.119498in}}%
\pgfpathclose%
\pgfusepath{stroke,fill}%
\end{pgfscope}%
\begin{pgfscope}%
\pgfpathrectangle{\pgfqpoint{3.793912in}{0.557870in}}{\pgfqpoint{2.446088in}{1.684734in}}%
\pgfusepath{clip}%
\pgfsetbuttcap%
\pgfsetroundjoin%
\definecolor{currentfill}{rgb}{0.298039,0.447059,0.690196}%
\pgfsetfillcolor{currentfill}%
\pgfsetlinewidth{1.003750pt}%
\definecolor{currentstroke}{rgb}{0.298039,0.447059,0.690196}%
\pgfsetstrokecolor{currentstroke}%
\pgfsetdash{}{0pt}%
\pgfpathmoveto{\pgfqpoint{3.905098in}{2.119498in}}%
\pgfpathcurveto{\pgfqpoint{3.913334in}{2.119498in}}{\pgfqpoint{3.921234in}{2.122771in}}{\pgfqpoint{3.927058in}{2.128595in}}%
\pgfpathcurveto{\pgfqpoint{3.932882in}{2.134419in}}{\pgfqpoint{3.936155in}{2.142319in}}{\pgfqpoint{3.936155in}{2.150555in}}%
\pgfpathcurveto{\pgfqpoint{3.936155in}{2.158791in}}{\pgfqpoint{3.932882in}{2.166691in}}{\pgfqpoint{3.927058in}{2.172515in}}%
\pgfpathcurveto{\pgfqpoint{3.921234in}{2.178339in}}{\pgfqpoint{3.913334in}{2.181611in}}{\pgfqpoint{3.905098in}{2.181611in}}%
\pgfpathcurveto{\pgfqpoint{3.896862in}{2.181611in}}{\pgfqpoint{3.888962in}{2.178339in}}{\pgfqpoint{3.883138in}{2.172515in}}%
\pgfpathcurveto{\pgfqpoint{3.877314in}{2.166691in}}{\pgfqpoint{3.874042in}{2.158791in}}{\pgfqpoint{3.874042in}{2.150555in}}%
\pgfpathcurveto{\pgfqpoint{3.874042in}{2.142319in}}{\pgfqpoint{3.877314in}{2.134419in}}{\pgfqpoint{3.883138in}{2.128595in}}%
\pgfpathcurveto{\pgfqpoint{3.888962in}{2.122771in}}{\pgfqpoint{3.896862in}{2.119498in}}{\pgfqpoint{3.905098in}{2.119498in}}%
\pgfpathclose%
\pgfusepath{stroke,fill}%
\end{pgfscope}%
\begin{pgfscope}%
\pgfpathrectangle{\pgfqpoint{3.793912in}{0.557870in}}{\pgfqpoint{2.446088in}{1.684734in}}%
\pgfusepath{clip}%
\pgfsetbuttcap%
\pgfsetroundjoin%
\definecolor{currentfill}{rgb}{0.298039,0.447059,0.690196}%
\pgfsetfillcolor{currentfill}%
\pgfsetlinewidth{1.003750pt}%
\definecolor{currentstroke}{rgb}{0.298039,0.447059,0.690196}%
\pgfsetstrokecolor{currentstroke}%
\pgfsetdash{}{0pt}%
\pgfpathmoveto{\pgfqpoint{4.845019in}{1.562562in}}%
\pgfpathcurveto{\pgfqpoint{4.853256in}{1.562562in}}{\pgfqpoint{4.861156in}{1.565834in}}{\pgfqpoint{4.866980in}{1.571658in}}%
\pgfpathcurveto{\pgfqpoint{4.872803in}{1.577482in}}{\pgfqpoint{4.876076in}{1.585382in}}{\pgfqpoint{4.876076in}{1.593618in}}%
\pgfpathcurveto{\pgfqpoint{4.876076in}{1.601854in}}{\pgfqpoint{4.872803in}{1.609754in}}{\pgfqpoint{4.866980in}{1.615578in}}%
\pgfpathcurveto{\pgfqpoint{4.861156in}{1.621402in}}{\pgfqpoint{4.853256in}{1.624675in}}{\pgfqpoint{4.845019in}{1.624675in}}%
\pgfpathcurveto{\pgfqpoint{4.836783in}{1.624675in}}{\pgfqpoint{4.828883in}{1.621402in}}{\pgfqpoint{4.823059in}{1.615578in}}%
\pgfpathcurveto{\pgfqpoint{4.817235in}{1.609754in}}{\pgfqpoint{4.813963in}{1.601854in}}{\pgfqpoint{4.813963in}{1.593618in}}%
\pgfpathcurveto{\pgfqpoint{4.813963in}{1.585382in}}{\pgfqpoint{4.817235in}{1.577482in}}{\pgfqpoint{4.823059in}{1.571658in}}%
\pgfpathcurveto{\pgfqpoint{4.828883in}{1.565834in}}{\pgfqpoint{4.836783in}{1.562562in}}{\pgfqpoint{4.845019in}{1.562562in}}%
\pgfpathclose%
\pgfusepath{stroke,fill}%
\end{pgfscope}%
\begin{pgfscope}%
\pgfpathrectangle{\pgfqpoint{3.793912in}{0.557870in}}{\pgfqpoint{2.446088in}{1.684734in}}%
\pgfusepath{clip}%
\pgfsetbuttcap%
\pgfsetroundjoin%
\definecolor{currentfill}{rgb}{0.298039,0.447059,0.690196}%
\pgfsetfillcolor{currentfill}%
\pgfsetlinewidth{1.003750pt}%
\definecolor{currentstroke}{rgb}{0.298039,0.447059,0.690196}%
\pgfsetstrokecolor{currentstroke}%
\pgfsetdash{}{0pt}%
\pgfpathmoveto{\pgfqpoint{3.905098in}{2.119498in}}%
\pgfpathcurveto{\pgfqpoint{3.913334in}{2.119498in}}{\pgfqpoint{3.921234in}{2.122771in}}{\pgfqpoint{3.927058in}{2.128595in}}%
\pgfpathcurveto{\pgfqpoint{3.932882in}{2.134419in}}{\pgfqpoint{3.936155in}{2.142319in}}{\pgfqpoint{3.936155in}{2.150555in}}%
\pgfpathcurveto{\pgfqpoint{3.936155in}{2.158791in}}{\pgfqpoint{3.932882in}{2.166691in}}{\pgfqpoint{3.927058in}{2.172515in}}%
\pgfpathcurveto{\pgfqpoint{3.921234in}{2.178339in}}{\pgfqpoint{3.913334in}{2.181611in}}{\pgfqpoint{3.905098in}{2.181611in}}%
\pgfpathcurveto{\pgfqpoint{3.896862in}{2.181611in}}{\pgfqpoint{3.888962in}{2.178339in}}{\pgfqpoint{3.883138in}{2.172515in}}%
\pgfpathcurveto{\pgfqpoint{3.877314in}{2.166691in}}{\pgfqpoint{3.874042in}{2.158791in}}{\pgfqpoint{3.874042in}{2.150555in}}%
\pgfpathcurveto{\pgfqpoint{3.874042in}{2.142319in}}{\pgfqpoint{3.877314in}{2.134419in}}{\pgfqpoint{3.883138in}{2.128595in}}%
\pgfpathcurveto{\pgfqpoint{3.888962in}{2.122771in}}{\pgfqpoint{3.896862in}{2.119498in}}{\pgfqpoint{3.905098in}{2.119498in}}%
\pgfpathclose%
\pgfusepath{stroke,fill}%
\end{pgfscope}%
\begin{pgfscope}%
\pgfpathrectangle{\pgfqpoint{3.793912in}{0.557870in}}{\pgfqpoint{2.446088in}{1.684734in}}%
\pgfusepath{clip}%
\pgfsetbuttcap%
\pgfsetroundjoin%
\definecolor{currentfill}{rgb}{0.298039,0.447059,0.690196}%
\pgfsetfillcolor{currentfill}%
\pgfsetlinewidth{1.003750pt}%
\definecolor{currentstroke}{rgb}{0.298039,0.447059,0.690196}%
\pgfsetstrokecolor{currentstroke}%
\pgfsetdash{}{0pt}%
\pgfpathmoveto{\pgfqpoint{3.905098in}{2.119498in}}%
\pgfpathcurveto{\pgfqpoint{3.913334in}{2.119498in}}{\pgfqpoint{3.921234in}{2.122771in}}{\pgfqpoint{3.927058in}{2.128595in}}%
\pgfpathcurveto{\pgfqpoint{3.932882in}{2.134419in}}{\pgfqpoint{3.936155in}{2.142319in}}{\pgfqpoint{3.936155in}{2.150555in}}%
\pgfpathcurveto{\pgfqpoint{3.936155in}{2.158791in}}{\pgfqpoint{3.932882in}{2.166691in}}{\pgfqpoint{3.927058in}{2.172515in}}%
\pgfpathcurveto{\pgfqpoint{3.921234in}{2.178339in}}{\pgfqpoint{3.913334in}{2.181611in}}{\pgfqpoint{3.905098in}{2.181611in}}%
\pgfpathcurveto{\pgfqpoint{3.896862in}{2.181611in}}{\pgfqpoint{3.888962in}{2.178339in}}{\pgfqpoint{3.883138in}{2.172515in}}%
\pgfpathcurveto{\pgfqpoint{3.877314in}{2.166691in}}{\pgfqpoint{3.874042in}{2.158791in}}{\pgfqpoint{3.874042in}{2.150555in}}%
\pgfpathcurveto{\pgfqpoint{3.874042in}{2.142319in}}{\pgfqpoint{3.877314in}{2.134419in}}{\pgfqpoint{3.883138in}{2.128595in}}%
\pgfpathcurveto{\pgfqpoint{3.888962in}{2.122771in}}{\pgfqpoint{3.896862in}{2.119498in}}{\pgfqpoint{3.905098in}{2.119498in}}%
\pgfpathclose%
\pgfusepath{stroke,fill}%
\end{pgfscope}%
\begin{pgfscope}%
\pgfpathrectangle{\pgfqpoint{3.793912in}{0.557870in}}{\pgfqpoint{2.446088in}{1.684734in}}%
\pgfusepath{clip}%
\pgfsetbuttcap%
\pgfsetroundjoin%
\definecolor{currentfill}{rgb}{0.298039,0.447059,0.690196}%
\pgfsetfillcolor{currentfill}%
\pgfsetlinewidth{1.003750pt}%
\definecolor{currentstroke}{rgb}{0.298039,0.447059,0.690196}%
\pgfsetstrokecolor{currentstroke}%
\pgfsetdash{}{0pt}%
\pgfpathmoveto{\pgfqpoint{3.905098in}{2.119498in}}%
\pgfpathcurveto{\pgfqpoint{3.913334in}{2.119498in}}{\pgfqpoint{3.921234in}{2.122771in}}{\pgfqpoint{3.927058in}{2.128595in}}%
\pgfpathcurveto{\pgfqpoint{3.932882in}{2.134419in}}{\pgfqpoint{3.936155in}{2.142319in}}{\pgfqpoint{3.936155in}{2.150555in}}%
\pgfpathcurveto{\pgfqpoint{3.936155in}{2.158791in}}{\pgfqpoint{3.932882in}{2.166691in}}{\pgfqpoint{3.927058in}{2.172515in}}%
\pgfpathcurveto{\pgfqpoint{3.921234in}{2.178339in}}{\pgfqpoint{3.913334in}{2.181611in}}{\pgfqpoint{3.905098in}{2.181611in}}%
\pgfpathcurveto{\pgfqpoint{3.896862in}{2.181611in}}{\pgfqpoint{3.888962in}{2.178339in}}{\pgfqpoint{3.883138in}{2.172515in}}%
\pgfpathcurveto{\pgfqpoint{3.877314in}{2.166691in}}{\pgfqpoint{3.874042in}{2.158791in}}{\pgfqpoint{3.874042in}{2.150555in}}%
\pgfpathcurveto{\pgfqpoint{3.874042in}{2.142319in}}{\pgfqpoint{3.877314in}{2.134419in}}{\pgfqpoint{3.883138in}{2.128595in}}%
\pgfpathcurveto{\pgfqpoint{3.888962in}{2.122771in}}{\pgfqpoint{3.896862in}{2.119498in}}{\pgfqpoint{3.905098in}{2.119498in}}%
\pgfpathclose%
\pgfusepath{stroke,fill}%
\end{pgfscope}%
\begin{pgfscope}%
\pgfpathrectangle{\pgfqpoint{3.793912in}{0.557870in}}{\pgfqpoint{2.446088in}{1.684734in}}%
\pgfusepath{clip}%
\pgfsetbuttcap%
\pgfsetroundjoin%
\definecolor{currentfill}{rgb}{0.298039,0.447059,0.690196}%
\pgfsetfillcolor{currentfill}%
\pgfsetlinewidth{1.003750pt}%
\definecolor{currentstroke}{rgb}{0.298039,0.447059,0.690196}%
\pgfsetstrokecolor{currentstroke}%
\pgfsetdash{}{0pt}%
\pgfpathmoveto{\pgfqpoint{3.905098in}{2.119498in}}%
\pgfpathcurveto{\pgfqpoint{3.913334in}{2.119498in}}{\pgfqpoint{3.921234in}{2.122771in}}{\pgfqpoint{3.927058in}{2.128595in}}%
\pgfpathcurveto{\pgfqpoint{3.932882in}{2.134419in}}{\pgfqpoint{3.936155in}{2.142319in}}{\pgfqpoint{3.936155in}{2.150555in}}%
\pgfpathcurveto{\pgfqpoint{3.936155in}{2.158791in}}{\pgfqpoint{3.932882in}{2.166691in}}{\pgfqpoint{3.927058in}{2.172515in}}%
\pgfpathcurveto{\pgfqpoint{3.921234in}{2.178339in}}{\pgfqpoint{3.913334in}{2.181611in}}{\pgfqpoint{3.905098in}{2.181611in}}%
\pgfpathcurveto{\pgfqpoint{3.896862in}{2.181611in}}{\pgfqpoint{3.888962in}{2.178339in}}{\pgfqpoint{3.883138in}{2.172515in}}%
\pgfpathcurveto{\pgfqpoint{3.877314in}{2.166691in}}{\pgfqpoint{3.874042in}{2.158791in}}{\pgfqpoint{3.874042in}{2.150555in}}%
\pgfpathcurveto{\pgfqpoint{3.874042in}{2.142319in}}{\pgfqpoint{3.877314in}{2.134419in}}{\pgfqpoint{3.883138in}{2.128595in}}%
\pgfpathcurveto{\pgfqpoint{3.888962in}{2.122771in}}{\pgfqpoint{3.896862in}{2.119498in}}{\pgfqpoint{3.905098in}{2.119498in}}%
\pgfpathclose%
\pgfusepath{stroke,fill}%
\end{pgfscope}%
\begin{pgfscope}%
\pgfpathrectangle{\pgfqpoint{3.793912in}{0.557870in}}{\pgfqpoint{2.446088in}{1.684734in}}%
\pgfusepath{clip}%
\pgfsetbuttcap%
\pgfsetroundjoin%
\definecolor{currentfill}{rgb}{0.298039,0.447059,0.690196}%
\pgfsetfillcolor{currentfill}%
\pgfsetlinewidth{1.003750pt}%
\definecolor{currentstroke}{rgb}{0.298039,0.447059,0.690196}%
\pgfsetstrokecolor{currentstroke}%
\pgfsetdash{}{0pt}%
\pgfpathmoveto{\pgfqpoint{3.905098in}{2.119498in}}%
\pgfpathcurveto{\pgfqpoint{3.913334in}{2.119498in}}{\pgfqpoint{3.921234in}{2.122771in}}{\pgfqpoint{3.927058in}{2.128595in}}%
\pgfpathcurveto{\pgfqpoint{3.932882in}{2.134419in}}{\pgfqpoint{3.936155in}{2.142319in}}{\pgfqpoint{3.936155in}{2.150555in}}%
\pgfpathcurveto{\pgfqpoint{3.936155in}{2.158791in}}{\pgfqpoint{3.932882in}{2.166691in}}{\pgfqpoint{3.927058in}{2.172515in}}%
\pgfpathcurveto{\pgfqpoint{3.921234in}{2.178339in}}{\pgfqpoint{3.913334in}{2.181611in}}{\pgfqpoint{3.905098in}{2.181611in}}%
\pgfpathcurveto{\pgfqpoint{3.896862in}{2.181611in}}{\pgfqpoint{3.888962in}{2.178339in}}{\pgfqpoint{3.883138in}{2.172515in}}%
\pgfpathcurveto{\pgfqpoint{3.877314in}{2.166691in}}{\pgfqpoint{3.874042in}{2.158791in}}{\pgfqpoint{3.874042in}{2.150555in}}%
\pgfpathcurveto{\pgfqpoint{3.874042in}{2.142319in}}{\pgfqpoint{3.877314in}{2.134419in}}{\pgfqpoint{3.883138in}{2.128595in}}%
\pgfpathcurveto{\pgfqpoint{3.888962in}{2.122771in}}{\pgfqpoint{3.896862in}{2.119498in}}{\pgfqpoint{3.905098in}{2.119498in}}%
\pgfpathclose%
\pgfusepath{stroke,fill}%
\end{pgfscope}%
\begin{pgfscope}%
\pgfpathrectangle{\pgfqpoint{3.793912in}{0.557870in}}{\pgfqpoint{2.446088in}{1.684734in}}%
\pgfusepath{clip}%
\pgfsetbuttcap%
\pgfsetroundjoin%
\definecolor{currentfill}{rgb}{0.298039,0.447059,0.690196}%
\pgfsetfillcolor{currentfill}%
\pgfsetlinewidth{1.003750pt}%
\definecolor{currentstroke}{rgb}{0.298039,0.447059,0.690196}%
\pgfsetstrokecolor{currentstroke}%
\pgfsetdash{}{0pt}%
\pgfpathmoveto{\pgfqpoint{3.905098in}{2.119498in}}%
\pgfpathcurveto{\pgfqpoint{3.913334in}{2.119498in}}{\pgfqpoint{3.921234in}{2.122771in}}{\pgfqpoint{3.927058in}{2.128595in}}%
\pgfpathcurveto{\pgfqpoint{3.932882in}{2.134419in}}{\pgfqpoint{3.936155in}{2.142319in}}{\pgfqpoint{3.936155in}{2.150555in}}%
\pgfpathcurveto{\pgfqpoint{3.936155in}{2.158791in}}{\pgfqpoint{3.932882in}{2.166691in}}{\pgfqpoint{3.927058in}{2.172515in}}%
\pgfpathcurveto{\pgfqpoint{3.921234in}{2.178339in}}{\pgfqpoint{3.913334in}{2.181611in}}{\pgfqpoint{3.905098in}{2.181611in}}%
\pgfpathcurveto{\pgfqpoint{3.896862in}{2.181611in}}{\pgfqpoint{3.888962in}{2.178339in}}{\pgfqpoint{3.883138in}{2.172515in}}%
\pgfpathcurveto{\pgfqpoint{3.877314in}{2.166691in}}{\pgfqpoint{3.874042in}{2.158791in}}{\pgfqpoint{3.874042in}{2.150555in}}%
\pgfpathcurveto{\pgfqpoint{3.874042in}{2.142319in}}{\pgfqpoint{3.877314in}{2.134419in}}{\pgfqpoint{3.883138in}{2.128595in}}%
\pgfpathcurveto{\pgfqpoint{3.888962in}{2.122771in}}{\pgfqpoint{3.896862in}{2.119498in}}{\pgfqpoint{3.905098in}{2.119498in}}%
\pgfpathclose%
\pgfusepath{stroke,fill}%
\end{pgfscope}%
\begin{pgfscope}%
\pgfpathrectangle{\pgfqpoint{3.793912in}{0.557870in}}{\pgfqpoint{2.446088in}{1.684734in}}%
\pgfusepath{clip}%
\pgfsetbuttcap%
\pgfsetroundjoin%
\definecolor{currentfill}{rgb}{0.298039,0.447059,0.690196}%
\pgfsetfillcolor{currentfill}%
\pgfsetlinewidth{1.003750pt}%
\definecolor{currentstroke}{rgb}{0.298039,0.447059,0.690196}%
\pgfsetstrokecolor{currentstroke}%
\pgfsetdash{}{0pt}%
\pgfpathmoveto{\pgfqpoint{4.867944in}{1.686325in}}%
\pgfpathcurveto{\pgfqpoint{4.876180in}{1.686325in}}{\pgfqpoint{4.884081in}{1.689598in}}{\pgfqpoint{4.889904in}{1.695422in}}%
\pgfpathcurveto{\pgfqpoint{4.895728in}{1.701245in}}{\pgfqpoint{4.899001in}{1.709146in}}{\pgfqpoint{4.899001in}{1.717382in}}%
\pgfpathcurveto{\pgfqpoint{4.899001in}{1.725618in}}{\pgfqpoint{4.895728in}{1.733518in}}{\pgfqpoint{4.889904in}{1.739342in}}%
\pgfpathcurveto{\pgfqpoint{4.884081in}{1.745166in}}{\pgfqpoint{4.876180in}{1.748438in}}{\pgfqpoint{4.867944in}{1.748438in}}%
\pgfpathcurveto{\pgfqpoint{4.859708in}{1.748438in}}{\pgfqpoint{4.851808in}{1.745166in}}{\pgfqpoint{4.845984in}{1.739342in}}%
\pgfpathcurveto{\pgfqpoint{4.840160in}{1.733518in}}{\pgfqpoint{4.836888in}{1.725618in}}{\pgfqpoint{4.836888in}{1.717382in}}%
\pgfpathcurveto{\pgfqpoint{4.836888in}{1.709146in}}{\pgfqpoint{4.840160in}{1.701245in}}{\pgfqpoint{4.845984in}{1.695422in}}%
\pgfpathcurveto{\pgfqpoint{4.851808in}{1.689598in}}{\pgfqpoint{4.859708in}{1.686325in}}{\pgfqpoint{4.867944in}{1.686325in}}%
\pgfpathclose%
\pgfusepath{stroke,fill}%
\end{pgfscope}%
\begin{pgfscope}%
\pgfpathrectangle{\pgfqpoint{3.793912in}{0.557870in}}{\pgfqpoint{2.446088in}{1.684734in}}%
\pgfusepath{clip}%
\pgfsetbuttcap%
\pgfsetroundjoin%
\definecolor{currentfill}{rgb}{0.298039,0.447059,0.690196}%
\pgfsetfillcolor{currentfill}%
\pgfsetlinewidth{1.003750pt}%
\definecolor{currentstroke}{rgb}{0.298039,0.447059,0.690196}%
\pgfsetstrokecolor{currentstroke}%
\pgfsetdash{}{0pt}%
\pgfpathmoveto{\pgfqpoint{3.905098in}{2.119498in}}%
\pgfpathcurveto{\pgfqpoint{3.913334in}{2.119498in}}{\pgfqpoint{3.921234in}{2.122771in}}{\pgfqpoint{3.927058in}{2.128595in}}%
\pgfpathcurveto{\pgfqpoint{3.932882in}{2.134419in}}{\pgfqpoint{3.936155in}{2.142319in}}{\pgfqpoint{3.936155in}{2.150555in}}%
\pgfpathcurveto{\pgfqpoint{3.936155in}{2.158791in}}{\pgfqpoint{3.932882in}{2.166691in}}{\pgfqpoint{3.927058in}{2.172515in}}%
\pgfpathcurveto{\pgfqpoint{3.921234in}{2.178339in}}{\pgfqpoint{3.913334in}{2.181611in}}{\pgfqpoint{3.905098in}{2.181611in}}%
\pgfpathcurveto{\pgfqpoint{3.896862in}{2.181611in}}{\pgfqpoint{3.888962in}{2.178339in}}{\pgfqpoint{3.883138in}{2.172515in}}%
\pgfpathcurveto{\pgfqpoint{3.877314in}{2.166691in}}{\pgfqpoint{3.874042in}{2.158791in}}{\pgfqpoint{3.874042in}{2.150555in}}%
\pgfpathcurveto{\pgfqpoint{3.874042in}{2.142319in}}{\pgfqpoint{3.877314in}{2.134419in}}{\pgfqpoint{3.883138in}{2.128595in}}%
\pgfpathcurveto{\pgfqpoint{3.888962in}{2.122771in}}{\pgfqpoint{3.896862in}{2.119498in}}{\pgfqpoint{3.905098in}{2.119498in}}%
\pgfpathclose%
\pgfusepath{stroke,fill}%
\end{pgfscope}%
\begin{pgfscope}%
\pgfpathrectangle{\pgfqpoint{3.793912in}{0.557870in}}{\pgfqpoint{2.446088in}{1.684734in}}%
\pgfusepath{clip}%
\pgfsetbuttcap%
\pgfsetroundjoin%
\definecolor{currentfill}{rgb}{0.298039,0.447059,0.690196}%
\pgfsetfillcolor{currentfill}%
\pgfsetlinewidth{1.003750pt}%
\definecolor{currentstroke}{rgb}{0.298039,0.447059,0.690196}%
\pgfsetstrokecolor{currentstroke}%
\pgfsetdash{}{0pt}%
\pgfpathmoveto{\pgfqpoint{3.905098in}{2.119498in}}%
\pgfpathcurveto{\pgfqpoint{3.913334in}{2.119498in}}{\pgfqpoint{3.921234in}{2.122771in}}{\pgfqpoint{3.927058in}{2.128595in}}%
\pgfpathcurveto{\pgfqpoint{3.932882in}{2.134419in}}{\pgfqpoint{3.936155in}{2.142319in}}{\pgfqpoint{3.936155in}{2.150555in}}%
\pgfpathcurveto{\pgfqpoint{3.936155in}{2.158791in}}{\pgfqpoint{3.932882in}{2.166691in}}{\pgfqpoint{3.927058in}{2.172515in}}%
\pgfpathcurveto{\pgfqpoint{3.921234in}{2.178339in}}{\pgfqpoint{3.913334in}{2.181611in}}{\pgfqpoint{3.905098in}{2.181611in}}%
\pgfpathcurveto{\pgfqpoint{3.896862in}{2.181611in}}{\pgfqpoint{3.888962in}{2.178339in}}{\pgfqpoint{3.883138in}{2.172515in}}%
\pgfpathcurveto{\pgfqpoint{3.877314in}{2.166691in}}{\pgfqpoint{3.874042in}{2.158791in}}{\pgfqpoint{3.874042in}{2.150555in}}%
\pgfpathcurveto{\pgfqpoint{3.874042in}{2.142319in}}{\pgfqpoint{3.877314in}{2.134419in}}{\pgfqpoint{3.883138in}{2.128595in}}%
\pgfpathcurveto{\pgfqpoint{3.888962in}{2.122771in}}{\pgfqpoint{3.896862in}{2.119498in}}{\pgfqpoint{3.905098in}{2.119498in}}%
\pgfpathclose%
\pgfusepath{stroke,fill}%
\end{pgfscope}%
\begin{pgfscope}%
\pgfpathrectangle{\pgfqpoint{3.793912in}{0.557870in}}{\pgfqpoint{2.446088in}{1.684734in}}%
\pgfusepath{clip}%
\pgfsetbuttcap%
\pgfsetroundjoin%
\definecolor{currentfill}{rgb}{0.298039,0.447059,0.690196}%
\pgfsetfillcolor{currentfill}%
\pgfsetlinewidth{1.003750pt}%
\definecolor{currentstroke}{rgb}{0.298039,0.447059,0.690196}%
\pgfsetstrokecolor{currentstroke}%
\pgfsetdash{}{0pt}%
\pgfpathmoveto{\pgfqpoint{3.905098in}{2.119498in}}%
\pgfpathcurveto{\pgfqpoint{3.913334in}{2.119498in}}{\pgfqpoint{3.921234in}{2.122771in}}{\pgfqpoint{3.927058in}{2.128595in}}%
\pgfpathcurveto{\pgfqpoint{3.932882in}{2.134419in}}{\pgfqpoint{3.936155in}{2.142319in}}{\pgfqpoint{3.936155in}{2.150555in}}%
\pgfpathcurveto{\pgfqpoint{3.936155in}{2.158791in}}{\pgfqpoint{3.932882in}{2.166691in}}{\pgfqpoint{3.927058in}{2.172515in}}%
\pgfpathcurveto{\pgfqpoint{3.921234in}{2.178339in}}{\pgfqpoint{3.913334in}{2.181611in}}{\pgfqpoint{3.905098in}{2.181611in}}%
\pgfpathcurveto{\pgfqpoint{3.896862in}{2.181611in}}{\pgfqpoint{3.888962in}{2.178339in}}{\pgfqpoint{3.883138in}{2.172515in}}%
\pgfpathcurveto{\pgfqpoint{3.877314in}{2.166691in}}{\pgfqpoint{3.874042in}{2.158791in}}{\pgfqpoint{3.874042in}{2.150555in}}%
\pgfpathcurveto{\pgfqpoint{3.874042in}{2.142319in}}{\pgfqpoint{3.877314in}{2.134419in}}{\pgfqpoint{3.883138in}{2.128595in}}%
\pgfpathcurveto{\pgfqpoint{3.888962in}{2.122771in}}{\pgfqpoint{3.896862in}{2.119498in}}{\pgfqpoint{3.905098in}{2.119498in}}%
\pgfpathclose%
\pgfusepath{stroke,fill}%
\end{pgfscope}%
\begin{pgfscope}%
\pgfpathrectangle{\pgfqpoint{3.793912in}{0.557870in}}{\pgfqpoint{2.446088in}{1.684734in}}%
\pgfusepath{clip}%
\pgfsetbuttcap%
\pgfsetroundjoin%
\definecolor{currentfill}{rgb}{0.298039,0.447059,0.690196}%
\pgfsetfillcolor{currentfill}%
\pgfsetlinewidth{1.003750pt}%
\definecolor{currentstroke}{rgb}{0.298039,0.447059,0.690196}%
\pgfsetstrokecolor{currentstroke}%
\pgfsetdash{}{0pt}%
\pgfpathmoveto{\pgfqpoint{5.463992in}{1.763678in}}%
\pgfpathcurveto{\pgfqpoint{5.472228in}{1.763678in}}{\pgfqpoint{5.480128in}{1.766950in}}{\pgfqpoint{5.485952in}{1.772774in}}%
\pgfpathcurveto{\pgfqpoint{5.491776in}{1.778598in}}{\pgfqpoint{5.495048in}{1.786498in}}{\pgfqpoint{5.495048in}{1.794734in}}%
\pgfpathcurveto{\pgfqpoint{5.495048in}{1.802970in}}{\pgfqpoint{5.491776in}{1.810870in}}{\pgfqpoint{5.485952in}{1.816694in}}%
\pgfpathcurveto{\pgfqpoint{5.480128in}{1.822518in}}{\pgfqpoint{5.472228in}{1.825791in}}{\pgfqpoint{5.463992in}{1.825791in}}%
\pgfpathcurveto{\pgfqpoint{5.455756in}{1.825791in}}{\pgfqpoint{5.447856in}{1.822518in}}{\pgfqpoint{5.442032in}{1.816694in}}%
\pgfpathcurveto{\pgfqpoint{5.436208in}{1.810870in}}{\pgfqpoint{5.432935in}{1.802970in}}{\pgfqpoint{5.432935in}{1.794734in}}%
\pgfpathcurveto{\pgfqpoint{5.432935in}{1.786498in}}{\pgfqpoint{5.436208in}{1.778598in}}{\pgfqpoint{5.442032in}{1.772774in}}%
\pgfpathcurveto{\pgfqpoint{5.447856in}{1.766950in}}{\pgfqpoint{5.455756in}{1.763678in}}{\pgfqpoint{5.463992in}{1.763678in}}%
\pgfpathclose%
\pgfusepath{stroke,fill}%
\end{pgfscope}%
\begin{pgfscope}%
\pgfpathrectangle{\pgfqpoint{3.793912in}{0.557870in}}{\pgfqpoint{2.446088in}{1.684734in}}%
\pgfusepath{clip}%
\pgfsetbuttcap%
\pgfsetroundjoin%
\definecolor{currentfill}{rgb}{0.298039,0.447059,0.690196}%
\pgfsetfillcolor{currentfill}%
\pgfsetlinewidth{1.003750pt}%
\definecolor{currentstroke}{rgb}{0.298039,0.447059,0.690196}%
\pgfsetstrokecolor{currentstroke}%
\pgfsetdash{}{0pt}%
\pgfpathmoveto{\pgfqpoint{5.349367in}{1.902912in}}%
\pgfpathcurveto{\pgfqpoint{5.357604in}{1.902912in}}{\pgfqpoint{5.365504in}{1.906184in}}{\pgfqpoint{5.371328in}{1.912008in}}%
\pgfpathcurveto{\pgfqpoint{5.377151in}{1.917832in}}{\pgfqpoint{5.380424in}{1.925732in}}{\pgfqpoint{5.380424in}{1.933968in}}%
\pgfpathcurveto{\pgfqpoint{5.380424in}{1.942205in}}{\pgfqpoint{5.377151in}{1.950105in}}{\pgfqpoint{5.371328in}{1.955929in}}%
\pgfpathcurveto{\pgfqpoint{5.365504in}{1.961753in}}{\pgfqpoint{5.357604in}{1.965025in}}{\pgfqpoint{5.349367in}{1.965025in}}%
\pgfpathcurveto{\pgfqpoint{5.341131in}{1.965025in}}{\pgfqpoint{5.333231in}{1.961753in}}{\pgfqpoint{5.327407in}{1.955929in}}%
\pgfpathcurveto{\pgfqpoint{5.321583in}{1.950105in}}{\pgfqpoint{5.318311in}{1.942205in}}{\pgfqpoint{5.318311in}{1.933968in}}%
\pgfpathcurveto{\pgfqpoint{5.318311in}{1.925732in}}{\pgfqpoint{5.321583in}{1.917832in}}{\pgfqpoint{5.327407in}{1.912008in}}%
\pgfpathcurveto{\pgfqpoint{5.333231in}{1.906184in}}{\pgfqpoint{5.341131in}{1.902912in}}{\pgfqpoint{5.349367in}{1.902912in}}%
\pgfpathclose%
\pgfusepath{stroke,fill}%
\end{pgfscope}%
\begin{pgfscope}%
\pgfpathrectangle{\pgfqpoint{3.793912in}{0.557870in}}{\pgfqpoint{2.446088in}{1.684734in}}%
\pgfusepath{clip}%
\pgfsetbuttcap%
\pgfsetroundjoin%
\definecolor{currentfill}{rgb}{0.298039,0.447059,0.690196}%
\pgfsetfillcolor{currentfill}%
\pgfsetlinewidth{1.003750pt}%
\definecolor{currentstroke}{rgb}{0.298039,0.447059,0.690196}%
\pgfsetstrokecolor{currentstroke}%
\pgfsetdash{}{0pt}%
\pgfpathmoveto{\pgfqpoint{5.349367in}{1.841030in}}%
\pgfpathcurveto{\pgfqpoint{5.357604in}{1.841030in}}{\pgfqpoint{5.365504in}{1.844302in}}{\pgfqpoint{5.371328in}{1.850126in}}%
\pgfpathcurveto{\pgfqpoint{5.377151in}{1.855950in}}{\pgfqpoint{5.380424in}{1.863850in}}{\pgfqpoint{5.380424in}{1.872086in}}%
\pgfpathcurveto{\pgfqpoint{5.380424in}{1.880323in}}{\pgfqpoint{5.377151in}{1.888223in}}{\pgfqpoint{5.371328in}{1.894047in}}%
\pgfpathcurveto{\pgfqpoint{5.365504in}{1.899871in}}{\pgfqpoint{5.357604in}{1.903143in}}{\pgfqpoint{5.349367in}{1.903143in}}%
\pgfpathcurveto{\pgfqpoint{5.341131in}{1.903143in}}{\pgfqpoint{5.333231in}{1.899871in}}{\pgfqpoint{5.327407in}{1.894047in}}%
\pgfpathcurveto{\pgfqpoint{5.321583in}{1.888223in}}{\pgfqpoint{5.318311in}{1.880323in}}{\pgfqpoint{5.318311in}{1.872086in}}%
\pgfpathcurveto{\pgfqpoint{5.318311in}{1.863850in}}{\pgfqpoint{5.321583in}{1.855950in}}{\pgfqpoint{5.327407in}{1.850126in}}%
\pgfpathcurveto{\pgfqpoint{5.333231in}{1.844302in}}{\pgfqpoint{5.341131in}{1.841030in}}{\pgfqpoint{5.349367in}{1.841030in}}%
\pgfpathclose%
\pgfusepath{stroke,fill}%
\end{pgfscope}%
\begin{pgfscope}%
\pgfpathrectangle{\pgfqpoint{3.793912in}{0.557870in}}{\pgfqpoint{2.446088in}{1.684734in}}%
\pgfusepath{clip}%
\pgfsetbuttcap%
\pgfsetroundjoin%
\definecolor{currentfill}{rgb}{0.298039,0.447059,0.690196}%
\pgfsetfillcolor{currentfill}%
\pgfsetlinewidth{1.003750pt}%
\definecolor{currentstroke}{rgb}{0.298039,0.447059,0.690196}%
\pgfsetstrokecolor{currentstroke}%
\pgfsetdash{}{0pt}%
\pgfpathmoveto{\pgfqpoint{5.326442in}{1.933853in}}%
\pgfpathcurveto{\pgfqpoint{5.334679in}{1.933853in}}{\pgfqpoint{5.342579in}{1.937125in}}{\pgfqpoint{5.348403in}{1.942949in}}%
\pgfpathcurveto{\pgfqpoint{5.354227in}{1.948773in}}{\pgfqpoint{5.357499in}{1.956673in}}{\pgfqpoint{5.357499in}{1.964909in}}%
\pgfpathcurveto{\pgfqpoint{5.357499in}{1.973146in}}{\pgfqpoint{5.354227in}{1.981046in}}{\pgfqpoint{5.348403in}{1.986870in}}%
\pgfpathcurveto{\pgfqpoint{5.342579in}{1.992693in}}{\pgfqpoint{5.334679in}{1.995966in}}{\pgfqpoint{5.326442in}{1.995966in}}%
\pgfpathcurveto{\pgfqpoint{5.318206in}{1.995966in}}{\pgfqpoint{5.310306in}{1.992693in}}{\pgfqpoint{5.304482in}{1.986870in}}%
\pgfpathcurveto{\pgfqpoint{5.298658in}{1.981046in}}{\pgfqpoint{5.295386in}{1.973146in}}{\pgfqpoint{5.295386in}{1.964909in}}%
\pgfpathcurveto{\pgfqpoint{5.295386in}{1.956673in}}{\pgfqpoint{5.298658in}{1.948773in}}{\pgfqpoint{5.304482in}{1.942949in}}%
\pgfpathcurveto{\pgfqpoint{5.310306in}{1.937125in}}{\pgfqpoint{5.318206in}{1.933853in}}{\pgfqpoint{5.326442in}{1.933853in}}%
\pgfpathclose%
\pgfusepath{stroke,fill}%
\end{pgfscope}%
\begin{pgfscope}%
\pgfpathrectangle{\pgfqpoint{3.793912in}{0.557870in}}{\pgfqpoint{2.446088in}{1.684734in}}%
\pgfusepath{clip}%
\pgfsetbuttcap%
\pgfsetroundjoin%
\definecolor{currentfill}{rgb}{0.298039,0.447059,0.690196}%
\pgfsetfillcolor{currentfill}%
\pgfsetlinewidth{1.003750pt}%
\definecolor{currentstroke}{rgb}{0.298039,0.447059,0.690196}%
\pgfsetstrokecolor{currentstroke}%
\pgfsetdash{}{0pt}%
\pgfpathmoveto{\pgfqpoint{5.601541in}{1.485209in}}%
\pgfpathcurveto{\pgfqpoint{5.609778in}{1.485209in}}{\pgfqpoint{5.617678in}{1.488482in}}{\pgfqpoint{5.623502in}{1.494305in}}%
\pgfpathcurveto{\pgfqpoint{5.629325in}{1.500129in}}{\pgfqpoint{5.632598in}{1.508029in}}{\pgfqpoint{5.632598in}{1.516266in}}%
\pgfpathcurveto{\pgfqpoint{5.632598in}{1.524502in}}{\pgfqpoint{5.629325in}{1.532402in}}{\pgfqpoint{5.623502in}{1.538226in}}%
\pgfpathcurveto{\pgfqpoint{5.617678in}{1.544050in}}{\pgfqpoint{5.609778in}{1.547322in}}{\pgfqpoint{5.601541in}{1.547322in}}%
\pgfpathcurveto{\pgfqpoint{5.593305in}{1.547322in}}{\pgfqpoint{5.585405in}{1.544050in}}{\pgfqpoint{5.579581in}{1.538226in}}%
\pgfpathcurveto{\pgfqpoint{5.573757in}{1.532402in}}{\pgfqpoint{5.570485in}{1.524502in}}{\pgfqpoint{5.570485in}{1.516266in}}%
\pgfpathcurveto{\pgfqpoint{5.570485in}{1.508029in}}{\pgfqpoint{5.573757in}{1.500129in}}{\pgfqpoint{5.579581in}{1.494305in}}%
\pgfpathcurveto{\pgfqpoint{5.585405in}{1.488482in}}{\pgfqpoint{5.593305in}{1.485209in}}{\pgfqpoint{5.601541in}{1.485209in}}%
\pgfpathclose%
\pgfusepath{stroke,fill}%
\end{pgfscope}%
\begin{pgfscope}%
\pgfpathrectangle{\pgfqpoint{3.793912in}{0.557870in}}{\pgfqpoint{2.446088in}{1.684734in}}%
\pgfusepath{clip}%
\pgfsetbuttcap%
\pgfsetroundjoin%
\definecolor{currentfill}{rgb}{0.298039,0.447059,0.690196}%
\pgfsetfillcolor{currentfill}%
\pgfsetlinewidth{1.003750pt}%
\definecolor{currentstroke}{rgb}{0.298039,0.447059,0.690196}%
\pgfsetstrokecolor{currentstroke}%
\pgfsetdash{}{0pt}%
\pgfpathmoveto{\pgfqpoint{5.601541in}{1.485209in}}%
\pgfpathcurveto{\pgfqpoint{5.609778in}{1.485209in}}{\pgfqpoint{5.617678in}{1.488482in}}{\pgfqpoint{5.623502in}{1.494305in}}%
\pgfpathcurveto{\pgfqpoint{5.629325in}{1.500129in}}{\pgfqpoint{5.632598in}{1.508029in}}{\pgfqpoint{5.632598in}{1.516266in}}%
\pgfpathcurveto{\pgfqpoint{5.632598in}{1.524502in}}{\pgfqpoint{5.629325in}{1.532402in}}{\pgfqpoint{5.623502in}{1.538226in}}%
\pgfpathcurveto{\pgfqpoint{5.617678in}{1.544050in}}{\pgfqpoint{5.609778in}{1.547322in}}{\pgfqpoint{5.601541in}{1.547322in}}%
\pgfpathcurveto{\pgfqpoint{5.593305in}{1.547322in}}{\pgfqpoint{5.585405in}{1.544050in}}{\pgfqpoint{5.579581in}{1.538226in}}%
\pgfpathcurveto{\pgfqpoint{5.573757in}{1.532402in}}{\pgfqpoint{5.570485in}{1.524502in}}{\pgfqpoint{5.570485in}{1.516266in}}%
\pgfpathcurveto{\pgfqpoint{5.570485in}{1.508029in}}{\pgfqpoint{5.573757in}{1.500129in}}{\pgfqpoint{5.579581in}{1.494305in}}%
\pgfpathcurveto{\pgfqpoint{5.585405in}{1.488482in}}{\pgfqpoint{5.593305in}{1.485209in}}{\pgfqpoint{5.601541in}{1.485209in}}%
\pgfpathclose%
\pgfusepath{stroke,fill}%
\end{pgfscope}%
\begin{pgfscope}%
\pgfpathrectangle{\pgfqpoint{3.793912in}{0.557870in}}{\pgfqpoint{2.446088in}{1.684734in}}%
\pgfusepath{clip}%
\pgfsetbuttcap%
\pgfsetroundjoin%
\definecolor{currentfill}{rgb}{0.298039,0.447059,0.690196}%
\pgfsetfillcolor{currentfill}%
\pgfsetlinewidth{1.003750pt}%
\definecolor{currentstroke}{rgb}{0.298039,0.447059,0.690196}%
\pgfsetstrokecolor{currentstroke}%
\pgfsetdash{}{0pt}%
\pgfpathmoveto{\pgfqpoint{5.784941in}{1.330505in}}%
\pgfpathcurveto{\pgfqpoint{5.793177in}{1.330505in}}{\pgfqpoint{5.801077in}{1.333777in}}{\pgfqpoint{5.806901in}{1.339601in}}%
\pgfpathcurveto{\pgfqpoint{5.812725in}{1.345425in}}{\pgfqpoint{5.815997in}{1.353325in}}{\pgfqpoint{5.815997in}{1.361561in}}%
\pgfpathcurveto{\pgfqpoint{5.815997in}{1.369797in}}{\pgfqpoint{5.812725in}{1.377697in}}{\pgfqpoint{5.806901in}{1.383521in}}%
\pgfpathcurveto{\pgfqpoint{5.801077in}{1.389345in}}{\pgfqpoint{5.793177in}{1.392618in}}{\pgfqpoint{5.784941in}{1.392618in}}%
\pgfpathcurveto{\pgfqpoint{5.776704in}{1.392618in}}{\pgfqpoint{5.768804in}{1.389345in}}{\pgfqpoint{5.762980in}{1.383521in}}%
\pgfpathcurveto{\pgfqpoint{5.757156in}{1.377697in}}{\pgfqpoint{5.753884in}{1.369797in}}{\pgfqpoint{5.753884in}{1.361561in}}%
\pgfpathcurveto{\pgfqpoint{5.753884in}{1.353325in}}{\pgfqpoint{5.757156in}{1.345425in}}{\pgfqpoint{5.762980in}{1.339601in}}%
\pgfpathcurveto{\pgfqpoint{5.768804in}{1.333777in}}{\pgfqpoint{5.776704in}{1.330505in}}{\pgfqpoint{5.784941in}{1.330505in}}%
\pgfpathclose%
\pgfusepath{stroke,fill}%
\end{pgfscope}%
\begin{pgfscope}%
\pgfpathrectangle{\pgfqpoint{3.793912in}{0.557870in}}{\pgfqpoint{2.446088in}{1.684734in}}%
\pgfusepath{clip}%
\pgfsetbuttcap%
\pgfsetroundjoin%
\definecolor{currentfill}{rgb}{0.298039,0.447059,0.690196}%
\pgfsetfillcolor{currentfill}%
\pgfsetlinewidth{1.003750pt}%
\definecolor{currentstroke}{rgb}{0.298039,0.447059,0.690196}%
\pgfsetstrokecolor{currentstroke}%
\pgfsetdash{}{0pt}%
\pgfpathmoveto{\pgfqpoint{5.555691in}{1.578032in}}%
\pgfpathcurveto{\pgfqpoint{5.563928in}{1.578032in}}{\pgfqpoint{5.571828in}{1.581304in}}{\pgfqpoint{5.577652in}{1.587128in}}%
\pgfpathcurveto{\pgfqpoint{5.583476in}{1.592952in}}{\pgfqpoint{5.586748in}{1.600852in}}{\pgfqpoint{5.586748in}{1.609089in}}%
\pgfpathcurveto{\pgfqpoint{5.586748in}{1.617325in}}{\pgfqpoint{5.583476in}{1.625225in}}{\pgfqpoint{5.577652in}{1.631049in}}%
\pgfpathcurveto{\pgfqpoint{5.571828in}{1.636873in}}{\pgfqpoint{5.563928in}{1.640145in}}{\pgfqpoint{5.555691in}{1.640145in}}%
\pgfpathcurveto{\pgfqpoint{5.547455in}{1.640145in}}{\pgfqpoint{5.539555in}{1.636873in}}{\pgfqpoint{5.533731in}{1.631049in}}%
\pgfpathcurveto{\pgfqpoint{5.527907in}{1.625225in}}{\pgfqpoint{5.524635in}{1.617325in}}{\pgfqpoint{5.524635in}{1.609089in}}%
\pgfpathcurveto{\pgfqpoint{5.524635in}{1.600852in}}{\pgfqpoint{5.527907in}{1.592952in}}{\pgfqpoint{5.533731in}{1.587128in}}%
\pgfpathcurveto{\pgfqpoint{5.539555in}{1.581304in}}{\pgfqpoint{5.547455in}{1.578032in}}{\pgfqpoint{5.555691in}{1.578032in}}%
\pgfpathclose%
\pgfusepath{stroke,fill}%
\end{pgfscope}%
\begin{pgfscope}%
\pgfpathrectangle{\pgfqpoint{3.793912in}{0.557870in}}{\pgfqpoint{2.446088in}{1.684734in}}%
\pgfusepath{clip}%
\pgfsetbuttcap%
\pgfsetroundjoin%
\definecolor{currentfill}{rgb}{0.298039,0.447059,0.690196}%
\pgfsetfillcolor{currentfill}%
\pgfsetlinewidth{1.003750pt}%
\definecolor{currentstroke}{rgb}{0.298039,0.447059,0.690196}%
\pgfsetstrokecolor{currentstroke}%
\pgfsetdash{}{0pt}%
\pgfpathmoveto{\pgfqpoint{5.532767in}{1.608973in}}%
\pgfpathcurveto{\pgfqpoint{5.541003in}{1.608973in}}{\pgfqpoint{5.548903in}{1.612245in}}{\pgfqpoint{5.554727in}{1.618069in}}%
\pgfpathcurveto{\pgfqpoint{5.560551in}{1.623893in}}{\pgfqpoint{5.563823in}{1.631793in}}{\pgfqpoint{5.563823in}{1.640029in}}%
\pgfpathcurveto{\pgfqpoint{5.563823in}{1.648266in}}{\pgfqpoint{5.560551in}{1.656166in}}{\pgfqpoint{5.554727in}{1.661990in}}%
\pgfpathcurveto{\pgfqpoint{5.548903in}{1.667814in}}{\pgfqpoint{5.541003in}{1.671086in}}{\pgfqpoint{5.532767in}{1.671086in}}%
\pgfpathcurveto{\pgfqpoint{5.524530in}{1.671086in}}{\pgfqpoint{5.516630in}{1.667814in}}{\pgfqpoint{5.510806in}{1.661990in}}%
\pgfpathcurveto{\pgfqpoint{5.504982in}{1.656166in}}{\pgfqpoint{5.501710in}{1.648266in}}{\pgfqpoint{5.501710in}{1.640029in}}%
\pgfpathcurveto{\pgfqpoint{5.501710in}{1.631793in}}{\pgfqpoint{5.504982in}{1.623893in}}{\pgfqpoint{5.510806in}{1.618069in}}%
\pgfpathcurveto{\pgfqpoint{5.516630in}{1.612245in}}{\pgfqpoint{5.524530in}{1.608973in}}{\pgfqpoint{5.532767in}{1.608973in}}%
\pgfpathclose%
\pgfusepath{stroke,fill}%
\end{pgfscope}%
\begin{pgfscope}%
\pgfpathrectangle{\pgfqpoint{3.793912in}{0.557870in}}{\pgfqpoint{2.446088in}{1.684734in}}%
\pgfusepath{clip}%
\pgfsetbuttcap%
\pgfsetroundjoin%
\definecolor{currentfill}{rgb}{0.298039,0.447059,0.690196}%
\pgfsetfillcolor{currentfill}%
\pgfsetlinewidth{1.003750pt}%
\definecolor{currentstroke}{rgb}{0.298039,0.447059,0.690196}%
\pgfsetstrokecolor{currentstroke}%
\pgfsetdash{}{0pt}%
\pgfpathmoveto{\pgfqpoint{5.509842in}{1.608973in}}%
\pgfpathcurveto{\pgfqpoint{5.518078in}{1.608973in}}{\pgfqpoint{5.525978in}{1.612245in}}{\pgfqpoint{5.531802in}{1.618069in}}%
\pgfpathcurveto{\pgfqpoint{5.537626in}{1.623893in}}{\pgfqpoint{5.540898in}{1.631793in}}{\pgfqpoint{5.540898in}{1.640029in}}%
\pgfpathcurveto{\pgfqpoint{5.540898in}{1.648266in}}{\pgfqpoint{5.537626in}{1.656166in}}{\pgfqpoint{5.531802in}{1.661990in}}%
\pgfpathcurveto{\pgfqpoint{5.525978in}{1.667814in}}{\pgfqpoint{5.518078in}{1.671086in}}{\pgfqpoint{5.509842in}{1.671086in}}%
\pgfpathcurveto{\pgfqpoint{5.501605in}{1.671086in}}{\pgfqpoint{5.493705in}{1.667814in}}{\pgfqpoint{5.487881in}{1.661990in}}%
\pgfpathcurveto{\pgfqpoint{5.482057in}{1.656166in}}{\pgfqpoint{5.478785in}{1.648266in}}{\pgfqpoint{5.478785in}{1.640029in}}%
\pgfpathcurveto{\pgfqpoint{5.478785in}{1.631793in}}{\pgfqpoint{5.482057in}{1.623893in}}{\pgfqpoint{5.487881in}{1.618069in}}%
\pgfpathcurveto{\pgfqpoint{5.493705in}{1.612245in}}{\pgfqpoint{5.501605in}{1.608973in}}{\pgfqpoint{5.509842in}{1.608973in}}%
\pgfpathclose%
\pgfusepath{stroke,fill}%
\end{pgfscope}%
\begin{pgfscope}%
\pgfpathrectangle{\pgfqpoint{3.793912in}{0.557870in}}{\pgfqpoint{2.446088in}{1.684734in}}%
\pgfusepath{clip}%
\pgfsetbuttcap%
\pgfsetroundjoin%
\definecolor{currentfill}{rgb}{0.298039,0.447059,0.690196}%
\pgfsetfillcolor{currentfill}%
\pgfsetlinewidth{1.003750pt}%
\definecolor{currentstroke}{rgb}{0.298039,0.447059,0.690196}%
\pgfsetstrokecolor{currentstroke}%
\pgfsetdash{}{0pt}%
\pgfpathmoveto{\pgfqpoint{5.532767in}{1.608973in}}%
\pgfpathcurveto{\pgfqpoint{5.541003in}{1.608973in}}{\pgfqpoint{5.548903in}{1.612245in}}{\pgfqpoint{5.554727in}{1.618069in}}%
\pgfpathcurveto{\pgfqpoint{5.560551in}{1.623893in}}{\pgfqpoint{5.563823in}{1.631793in}}{\pgfqpoint{5.563823in}{1.640029in}}%
\pgfpathcurveto{\pgfqpoint{5.563823in}{1.648266in}}{\pgfqpoint{5.560551in}{1.656166in}}{\pgfqpoint{5.554727in}{1.661990in}}%
\pgfpathcurveto{\pgfqpoint{5.548903in}{1.667814in}}{\pgfqpoint{5.541003in}{1.671086in}}{\pgfqpoint{5.532767in}{1.671086in}}%
\pgfpathcurveto{\pgfqpoint{5.524530in}{1.671086in}}{\pgfqpoint{5.516630in}{1.667814in}}{\pgfqpoint{5.510806in}{1.661990in}}%
\pgfpathcurveto{\pgfqpoint{5.504982in}{1.656166in}}{\pgfqpoint{5.501710in}{1.648266in}}{\pgfqpoint{5.501710in}{1.640029in}}%
\pgfpathcurveto{\pgfqpoint{5.501710in}{1.631793in}}{\pgfqpoint{5.504982in}{1.623893in}}{\pgfqpoint{5.510806in}{1.618069in}}%
\pgfpathcurveto{\pgfqpoint{5.516630in}{1.612245in}}{\pgfqpoint{5.524530in}{1.608973in}}{\pgfqpoint{5.532767in}{1.608973in}}%
\pgfpathclose%
\pgfusepath{stroke,fill}%
\end{pgfscope}%
\begin{pgfscope}%
\pgfpathrectangle{\pgfqpoint{3.793912in}{0.557870in}}{\pgfqpoint{2.446088in}{1.684734in}}%
\pgfusepath{clip}%
\pgfsetbuttcap%
\pgfsetroundjoin%
\definecolor{currentfill}{rgb}{0.298039,0.447059,0.690196}%
\pgfsetfillcolor{currentfill}%
\pgfsetlinewidth{1.003750pt}%
\definecolor{currentstroke}{rgb}{0.298039,0.447059,0.690196}%
\pgfsetstrokecolor{currentstroke}%
\pgfsetdash{}{0pt}%
\pgfpathmoveto{\pgfqpoint{5.876640in}{1.160329in}}%
\pgfpathcurveto{\pgfqpoint{5.884876in}{1.160329in}}{\pgfqpoint{5.892777in}{1.163602in}}{\pgfqpoint{5.898600in}{1.169426in}}%
\pgfpathcurveto{\pgfqpoint{5.904424in}{1.175250in}}{\pgfqpoint{5.907697in}{1.183150in}}{\pgfqpoint{5.907697in}{1.191386in}}%
\pgfpathcurveto{\pgfqpoint{5.907697in}{1.199622in}}{\pgfqpoint{5.904424in}{1.207522in}}{\pgfqpoint{5.898600in}{1.213346in}}%
\pgfpathcurveto{\pgfqpoint{5.892777in}{1.219170in}}{\pgfqpoint{5.884876in}{1.222442in}}{\pgfqpoint{5.876640in}{1.222442in}}%
\pgfpathcurveto{\pgfqpoint{5.868404in}{1.222442in}}{\pgfqpoint{5.860504in}{1.219170in}}{\pgfqpoint{5.854680in}{1.213346in}}%
\pgfpathcurveto{\pgfqpoint{5.848856in}{1.207522in}}{\pgfqpoint{5.845584in}{1.199622in}}{\pgfqpoint{5.845584in}{1.191386in}}%
\pgfpathcurveto{\pgfqpoint{5.845584in}{1.183150in}}{\pgfqpoint{5.848856in}{1.175250in}}{\pgfqpoint{5.854680in}{1.169426in}}%
\pgfpathcurveto{\pgfqpoint{5.860504in}{1.163602in}}{\pgfqpoint{5.868404in}{1.160329in}}{\pgfqpoint{5.876640in}{1.160329in}}%
\pgfpathclose%
\pgfusepath{stroke,fill}%
\end{pgfscope}%
\begin{pgfscope}%
\pgfpathrectangle{\pgfqpoint{3.793912in}{0.557870in}}{\pgfqpoint{2.446088in}{1.684734in}}%
\pgfusepath{clip}%
\pgfsetbuttcap%
\pgfsetroundjoin%
\definecolor{currentfill}{rgb}{0.298039,0.447059,0.690196}%
\pgfsetfillcolor{currentfill}%
\pgfsetlinewidth{1.003750pt}%
\definecolor{currentstroke}{rgb}{0.298039,0.447059,0.690196}%
\pgfsetstrokecolor{currentstroke}%
\pgfsetdash{}{0pt}%
\pgfpathmoveto{\pgfqpoint{3.905098in}{2.119498in}}%
\pgfpathcurveto{\pgfqpoint{3.913334in}{2.119498in}}{\pgfqpoint{3.921234in}{2.122771in}}{\pgfqpoint{3.927058in}{2.128595in}}%
\pgfpathcurveto{\pgfqpoint{3.932882in}{2.134419in}}{\pgfqpoint{3.936155in}{2.142319in}}{\pgfqpoint{3.936155in}{2.150555in}}%
\pgfpathcurveto{\pgfqpoint{3.936155in}{2.158791in}}{\pgfqpoint{3.932882in}{2.166691in}}{\pgfqpoint{3.927058in}{2.172515in}}%
\pgfpathcurveto{\pgfqpoint{3.921234in}{2.178339in}}{\pgfqpoint{3.913334in}{2.181611in}}{\pgfqpoint{3.905098in}{2.181611in}}%
\pgfpathcurveto{\pgfqpoint{3.896862in}{2.181611in}}{\pgfqpoint{3.888962in}{2.178339in}}{\pgfqpoint{3.883138in}{2.172515in}}%
\pgfpathcurveto{\pgfqpoint{3.877314in}{2.166691in}}{\pgfqpoint{3.874042in}{2.158791in}}{\pgfqpoint{3.874042in}{2.150555in}}%
\pgfpathcurveto{\pgfqpoint{3.874042in}{2.142319in}}{\pgfqpoint{3.877314in}{2.134419in}}{\pgfqpoint{3.883138in}{2.128595in}}%
\pgfpathcurveto{\pgfqpoint{3.888962in}{2.122771in}}{\pgfqpoint{3.896862in}{2.119498in}}{\pgfqpoint{3.905098in}{2.119498in}}%
\pgfpathclose%
\pgfusepath{stroke,fill}%
\end{pgfscope}%
\begin{pgfscope}%
\pgfpathrectangle{\pgfqpoint{3.793912in}{0.557870in}}{\pgfqpoint{2.446088in}{1.684734in}}%
\pgfusepath{clip}%
\pgfsetbuttcap%
\pgfsetroundjoin%
\definecolor{currentfill}{rgb}{0.298039,0.447059,0.690196}%
\pgfsetfillcolor{currentfill}%
\pgfsetlinewidth{1.003750pt}%
\definecolor{currentstroke}{rgb}{0.298039,0.447059,0.690196}%
\pgfsetstrokecolor{currentstroke}%
\pgfsetdash{}{0pt}%
\pgfpathmoveto{\pgfqpoint{5.807865in}{0.943743in}}%
\pgfpathcurveto{\pgfqpoint{5.816102in}{0.943743in}}{\pgfqpoint{5.824002in}{0.947015in}}{\pgfqpoint{5.829826in}{0.952839in}}%
\pgfpathcurveto{\pgfqpoint{5.835650in}{0.958663in}}{\pgfqpoint{5.838922in}{0.966563in}}{\pgfqpoint{5.838922in}{0.974799in}}%
\pgfpathcurveto{\pgfqpoint{5.838922in}{0.983036in}}{\pgfqpoint{5.835650in}{0.990936in}}{\pgfqpoint{5.829826in}{0.996760in}}%
\pgfpathcurveto{\pgfqpoint{5.824002in}{1.002584in}}{\pgfqpoint{5.816102in}{1.005856in}}{\pgfqpoint{5.807865in}{1.005856in}}%
\pgfpathcurveto{\pgfqpoint{5.799629in}{1.005856in}}{\pgfqpoint{5.791729in}{1.002584in}}{\pgfqpoint{5.785905in}{0.996760in}}%
\pgfpathcurveto{\pgfqpoint{5.780081in}{0.990936in}}{\pgfqpoint{5.776809in}{0.983036in}}{\pgfqpoint{5.776809in}{0.974799in}}%
\pgfpathcurveto{\pgfqpoint{5.776809in}{0.966563in}}{\pgfqpoint{5.780081in}{0.958663in}}{\pgfqpoint{5.785905in}{0.952839in}}%
\pgfpathcurveto{\pgfqpoint{5.791729in}{0.947015in}}{\pgfqpoint{5.799629in}{0.943743in}}{\pgfqpoint{5.807865in}{0.943743in}}%
\pgfpathclose%
\pgfusepath{stroke,fill}%
\end{pgfscope}%
\begin{pgfscope}%
\pgfpathrectangle{\pgfqpoint{3.793912in}{0.557870in}}{\pgfqpoint{2.446088in}{1.684734in}}%
\pgfusepath{clip}%
\pgfsetbuttcap%
\pgfsetroundjoin%
\definecolor{currentfill}{rgb}{0.298039,0.447059,0.690196}%
\pgfsetfillcolor{currentfill}%
\pgfsetlinewidth{1.003750pt}%
\definecolor{currentstroke}{rgb}{0.298039,0.447059,0.690196}%
\pgfsetstrokecolor{currentstroke}%
\pgfsetdash{}{0pt}%
\pgfpathmoveto{\pgfqpoint{3.905098in}{2.119498in}}%
\pgfpathcurveto{\pgfqpoint{3.913334in}{2.119498in}}{\pgfqpoint{3.921234in}{2.122771in}}{\pgfqpoint{3.927058in}{2.128595in}}%
\pgfpathcurveto{\pgfqpoint{3.932882in}{2.134419in}}{\pgfqpoint{3.936155in}{2.142319in}}{\pgfqpoint{3.936155in}{2.150555in}}%
\pgfpathcurveto{\pgfqpoint{3.936155in}{2.158791in}}{\pgfqpoint{3.932882in}{2.166691in}}{\pgfqpoint{3.927058in}{2.172515in}}%
\pgfpathcurveto{\pgfqpoint{3.921234in}{2.178339in}}{\pgfqpoint{3.913334in}{2.181611in}}{\pgfqpoint{3.905098in}{2.181611in}}%
\pgfpathcurveto{\pgfqpoint{3.896862in}{2.181611in}}{\pgfqpoint{3.888962in}{2.178339in}}{\pgfqpoint{3.883138in}{2.172515in}}%
\pgfpathcurveto{\pgfqpoint{3.877314in}{2.166691in}}{\pgfqpoint{3.874042in}{2.158791in}}{\pgfqpoint{3.874042in}{2.150555in}}%
\pgfpathcurveto{\pgfqpoint{3.874042in}{2.142319in}}{\pgfqpoint{3.877314in}{2.134419in}}{\pgfqpoint{3.883138in}{2.128595in}}%
\pgfpathcurveto{\pgfqpoint{3.888962in}{2.122771in}}{\pgfqpoint{3.896862in}{2.119498in}}{\pgfqpoint{3.905098in}{2.119498in}}%
\pgfpathclose%
\pgfusepath{stroke,fill}%
\end{pgfscope}%
\begin{pgfscope}%
\pgfpathrectangle{\pgfqpoint{3.793912in}{0.557870in}}{\pgfqpoint{2.446088in}{1.684734in}}%
\pgfusepath{clip}%
\pgfsetbuttcap%
\pgfsetroundjoin%
\definecolor{currentfill}{rgb}{0.298039,0.447059,0.690196}%
\pgfsetfillcolor{currentfill}%
\pgfsetlinewidth{1.003750pt}%
\definecolor{currentstroke}{rgb}{0.298039,0.447059,0.690196}%
\pgfsetstrokecolor{currentstroke}%
\pgfsetdash{}{0pt}%
\pgfpathmoveto{\pgfqpoint{4.455296in}{1.779148in}}%
\pgfpathcurveto{\pgfqpoint{4.463532in}{1.779148in}}{\pgfqpoint{4.471432in}{1.782420in}}{\pgfqpoint{4.477256in}{1.788244in}}%
\pgfpathcurveto{\pgfqpoint{4.483080in}{1.794068in}}{\pgfqpoint{4.486352in}{1.801968in}}{\pgfqpoint{4.486352in}{1.810205in}}%
\pgfpathcurveto{\pgfqpoint{4.486352in}{1.818441in}}{\pgfqpoint{4.483080in}{1.826341in}}{\pgfqpoint{4.477256in}{1.832165in}}%
\pgfpathcurveto{\pgfqpoint{4.471432in}{1.837989in}}{\pgfqpoint{4.463532in}{1.841261in}}{\pgfqpoint{4.455296in}{1.841261in}}%
\pgfpathcurveto{\pgfqpoint{4.447060in}{1.841261in}}{\pgfqpoint{4.439160in}{1.837989in}}{\pgfqpoint{4.433336in}{1.832165in}}%
\pgfpathcurveto{\pgfqpoint{4.427512in}{1.826341in}}{\pgfqpoint{4.424239in}{1.818441in}}{\pgfqpoint{4.424239in}{1.810205in}}%
\pgfpathcurveto{\pgfqpoint{4.424239in}{1.801968in}}{\pgfqpoint{4.427512in}{1.794068in}}{\pgfqpoint{4.433336in}{1.788244in}}%
\pgfpathcurveto{\pgfqpoint{4.439160in}{1.782420in}}{\pgfqpoint{4.447060in}{1.779148in}}{\pgfqpoint{4.455296in}{1.779148in}}%
\pgfpathclose%
\pgfusepath{stroke,fill}%
\end{pgfscope}%
\begin{pgfscope}%
\pgfpathrectangle{\pgfqpoint{3.793912in}{0.557870in}}{\pgfqpoint{2.446088in}{1.684734in}}%
\pgfusepath{clip}%
\pgfsetbuttcap%
\pgfsetroundjoin%
\definecolor{currentfill}{rgb}{0.298039,0.447059,0.690196}%
\pgfsetfillcolor{currentfill}%
\pgfsetlinewidth{1.003750pt}%
\definecolor{currentstroke}{rgb}{0.298039,0.447059,0.690196}%
\pgfsetstrokecolor{currentstroke}%
\pgfsetdash{}{0pt}%
\pgfpathmoveto{\pgfqpoint{3.905098in}{2.119498in}}%
\pgfpathcurveto{\pgfqpoint{3.913334in}{2.119498in}}{\pgfqpoint{3.921234in}{2.122771in}}{\pgfqpoint{3.927058in}{2.128595in}}%
\pgfpathcurveto{\pgfqpoint{3.932882in}{2.134419in}}{\pgfqpoint{3.936155in}{2.142319in}}{\pgfqpoint{3.936155in}{2.150555in}}%
\pgfpathcurveto{\pgfqpoint{3.936155in}{2.158791in}}{\pgfqpoint{3.932882in}{2.166691in}}{\pgfqpoint{3.927058in}{2.172515in}}%
\pgfpathcurveto{\pgfqpoint{3.921234in}{2.178339in}}{\pgfqpoint{3.913334in}{2.181611in}}{\pgfqpoint{3.905098in}{2.181611in}}%
\pgfpathcurveto{\pgfqpoint{3.896862in}{2.181611in}}{\pgfqpoint{3.888962in}{2.178339in}}{\pgfqpoint{3.883138in}{2.172515in}}%
\pgfpathcurveto{\pgfqpoint{3.877314in}{2.166691in}}{\pgfqpoint{3.874042in}{2.158791in}}{\pgfqpoint{3.874042in}{2.150555in}}%
\pgfpathcurveto{\pgfqpoint{3.874042in}{2.142319in}}{\pgfqpoint{3.877314in}{2.134419in}}{\pgfqpoint{3.883138in}{2.128595in}}%
\pgfpathcurveto{\pgfqpoint{3.888962in}{2.122771in}}{\pgfqpoint{3.896862in}{2.119498in}}{\pgfqpoint{3.905098in}{2.119498in}}%
\pgfpathclose%
\pgfusepath{stroke,fill}%
\end{pgfscope}%
\begin{pgfscope}%
\pgfpathrectangle{\pgfqpoint{3.793912in}{0.557870in}}{\pgfqpoint{2.446088in}{1.684734in}}%
\pgfusepath{clip}%
\pgfsetbuttcap%
\pgfsetroundjoin%
\definecolor{currentfill}{rgb}{0.298039,0.447059,0.690196}%
\pgfsetfillcolor{currentfill}%
\pgfsetlinewidth{1.003750pt}%
\definecolor{currentstroke}{rgb}{0.298039,0.447059,0.690196}%
\pgfsetstrokecolor{currentstroke}%
\pgfsetdash{}{0pt}%
\pgfpathmoveto{\pgfqpoint{3.905098in}{2.119498in}}%
\pgfpathcurveto{\pgfqpoint{3.913334in}{2.119498in}}{\pgfqpoint{3.921234in}{2.122771in}}{\pgfqpoint{3.927058in}{2.128595in}}%
\pgfpathcurveto{\pgfqpoint{3.932882in}{2.134419in}}{\pgfqpoint{3.936155in}{2.142319in}}{\pgfqpoint{3.936155in}{2.150555in}}%
\pgfpathcurveto{\pgfqpoint{3.936155in}{2.158791in}}{\pgfqpoint{3.932882in}{2.166691in}}{\pgfqpoint{3.927058in}{2.172515in}}%
\pgfpathcurveto{\pgfqpoint{3.921234in}{2.178339in}}{\pgfqpoint{3.913334in}{2.181611in}}{\pgfqpoint{3.905098in}{2.181611in}}%
\pgfpathcurveto{\pgfqpoint{3.896862in}{2.181611in}}{\pgfqpoint{3.888962in}{2.178339in}}{\pgfqpoint{3.883138in}{2.172515in}}%
\pgfpathcurveto{\pgfqpoint{3.877314in}{2.166691in}}{\pgfqpoint{3.874042in}{2.158791in}}{\pgfqpoint{3.874042in}{2.150555in}}%
\pgfpathcurveto{\pgfqpoint{3.874042in}{2.142319in}}{\pgfqpoint{3.877314in}{2.134419in}}{\pgfqpoint{3.883138in}{2.128595in}}%
\pgfpathcurveto{\pgfqpoint{3.888962in}{2.122771in}}{\pgfqpoint{3.896862in}{2.119498in}}{\pgfqpoint{3.905098in}{2.119498in}}%
\pgfpathclose%
\pgfusepath{stroke,fill}%
\end{pgfscope}%
\begin{pgfscope}%
\pgfpathrectangle{\pgfqpoint{3.793912in}{0.557870in}}{\pgfqpoint{2.446088in}{1.684734in}}%
\pgfusepath{clip}%
\pgfsetbuttcap%
\pgfsetroundjoin%
\definecolor{currentfill}{rgb}{0.298039,0.447059,0.690196}%
\pgfsetfillcolor{currentfill}%
\pgfsetlinewidth{1.003750pt}%
\definecolor{currentstroke}{rgb}{0.298039,0.447059,0.690196}%
\pgfsetstrokecolor{currentstroke}%
\pgfsetdash{}{0pt}%
\pgfpathmoveto{\pgfqpoint{3.905098in}{2.119498in}}%
\pgfpathcurveto{\pgfqpoint{3.913334in}{2.119498in}}{\pgfqpoint{3.921234in}{2.122771in}}{\pgfqpoint{3.927058in}{2.128595in}}%
\pgfpathcurveto{\pgfqpoint{3.932882in}{2.134419in}}{\pgfqpoint{3.936155in}{2.142319in}}{\pgfqpoint{3.936155in}{2.150555in}}%
\pgfpathcurveto{\pgfqpoint{3.936155in}{2.158791in}}{\pgfqpoint{3.932882in}{2.166691in}}{\pgfqpoint{3.927058in}{2.172515in}}%
\pgfpathcurveto{\pgfqpoint{3.921234in}{2.178339in}}{\pgfqpoint{3.913334in}{2.181611in}}{\pgfqpoint{3.905098in}{2.181611in}}%
\pgfpathcurveto{\pgfqpoint{3.896862in}{2.181611in}}{\pgfqpoint{3.888962in}{2.178339in}}{\pgfqpoint{3.883138in}{2.172515in}}%
\pgfpathcurveto{\pgfqpoint{3.877314in}{2.166691in}}{\pgfqpoint{3.874042in}{2.158791in}}{\pgfqpoint{3.874042in}{2.150555in}}%
\pgfpathcurveto{\pgfqpoint{3.874042in}{2.142319in}}{\pgfqpoint{3.877314in}{2.134419in}}{\pgfqpoint{3.883138in}{2.128595in}}%
\pgfpathcurveto{\pgfqpoint{3.888962in}{2.122771in}}{\pgfqpoint{3.896862in}{2.119498in}}{\pgfqpoint{3.905098in}{2.119498in}}%
\pgfpathclose%
\pgfusepath{stroke,fill}%
\end{pgfscope}%
\begin{pgfscope}%
\pgfpathrectangle{\pgfqpoint{3.793912in}{0.557870in}}{\pgfqpoint{2.446088in}{1.684734in}}%
\pgfusepath{clip}%
\pgfsetbuttcap%
\pgfsetroundjoin%
\definecolor{currentfill}{rgb}{0.298039,0.447059,0.690196}%
\pgfsetfillcolor{currentfill}%
\pgfsetlinewidth{1.003750pt}%
\definecolor{currentstroke}{rgb}{0.298039,0.447059,0.690196}%
\pgfsetstrokecolor{currentstroke}%
\pgfsetdash{}{0pt}%
\pgfpathmoveto{\pgfqpoint{3.905098in}{2.119498in}}%
\pgfpathcurveto{\pgfqpoint{3.913334in}{2.119498in}}{\pgfqpoint{3.921234in}{2.122771in}}{\pgfqpoint{3.927058in}{2.128595in}}%
\pgfpathcurveto{\pgfqpoint{3.932882in}{2.134419in}}{\pgfqpoint{3.936155in}{2.142319in}}{\pgfqpoint{3.936155in}{2.150555in}}%
\pgfpathcurveto{\pgfqpoint{3.936155in}{2.158791in}}{\pgfqpoint{3.932882in}{2.166691in}}{\pgfqpoint{3.927058in}{2.172515in}}%
\pgfpathcurveto{\pgfqpoint{3.921234in}{2.178339in}}{\pgfqpoint{3.913334in}{2.181611in}}{\pgfqpoint{3.905098in}{2.181611in}}%
\pgfpathcurveto{\pgfqpoint{3.896862in}{2.181611in}}{\pgfqpoint{3.888962in}{2.178339in}}{\pgfqpoint{3.883138in}{2.172515in}}%
\pgfpathcurveto{\pgfqpoint{3.877314in}{2.166691in}}{\pgfqpoint{3.874042in}{2.158791in}}{\pgfqpoint{3.874042in}{2.150555in}}%
\pgfpathcurveto{\pgfqpoint{3.874042in}{2.142319in}}{\pgfqpoint{3.877314in}{2.134419in}}{\pgfqpoint{3.883138in}{2.128595in}}%
\pgfpathcurveto{\pgfqpoint{3.888962in}{2.122771in}}{\pgfqpoint{3.896862in}{2.119498in}}{\pgfqpoint{3.905098in}{2.119498in}}%
\pgfpathclose%
\pgfusepath{stroke,fill}%
\end{pgfscope}%
\begin{pgfscope}%
\pgfpathrectangle{\pgfqpoint{3.793912in}{0.557870in}}{\pgfqpoint{2.446088in}{1.684734in}}%
\pgfusepath{clip}%
\pgfsetbuttcap%
\pgfsetroundjoin%
\definecolor{currentfill}{rgb}{0.298039,0.447059,0.690196}%
\pgfsetfillcolor{currentfill}%
\pgfsetlinewidth{1.003750pt}%
\definecolor{currentstroke}{rgb}{0.298039,0.447059,0.690196}%
\pgfsetstrokecolor{currentstroke}%
\pgfsetdash{}{0pt}%
\pgfpathmoveto{\pgfqpoint{5.395217in}{1.222211in}}%
\pgfpathcurveto{\pgfqpoint{5.403453in}{1.222211in}}{\pgfqpoint{5.411353in}{1.225484in}}{\pgfqpoint{5.417177in}{1.231308in}}%
\pgfpathcurveto{\pgfqpoint{5.423001in}{1.237131in}}{\pgfqpoint{5.426274in}{1.245032in}}{\pgfqpoint{5.426274in}{1.253268in}}%
\pgfpathcurveto{\pgfqpoint{5.426274in}{1.261504in}}{\pgfqpoint{5.423001in}{1.269404in}}{\pgfqpoint{5.417177in}{1.275228in}}%
\pgfpathcurveto{\pgfqpoint{5.411353in}{1.281052in}}{\pgfqpoint{5.403453in}{1.284324in}}{\pgfqpoint{5.395217in}{1.284324in}}%
\pgfpathcurveto{\pgfqpoint{5.386981in}{1.284324in}}{\pgfqpoint{5.379081in}{1.281052in}}{\pgfqpoint{5.373257in}{1.275228in}}%
\pgfpathcurveto{\pgfqpoint{5.367433in}{1.269404in}}{\pgfqpoint{5.364161in}{1.261504in}}{\pgfqpoint{5.364161in}{1.253268in}}%
\pgfpathcurveto{\pgfqpoint{5.364161in}{1.245032in}}{\pgfqpoint{5.367433in}{1.237131in}}{\pgfqpoint{5.373257in}{1.231308in}}%
\pgfpathcurveto{\pgfqpoint{5.379081in}{1.225484in}}{\pgfqpoint{5.386981in}{1.222211in}}{\pgfqpoint{5.395217in}{1.222211in}}%
\pgfpathclose%
\pgfusepath{stroke,fill}%
\end{pgfscope}%
\begin{pgfscope}%
\pgfpathrectangle{\pgfqpoint{3.793912in}{0.557870in}}{\pgfqpoint{2.446088in}{1.684734in}}%
\pgfusepath{clip}%
\pgfsetbuttcap%
\pgfsetroundjoin%
\definecolor{currentfill}{rgb}{0.298039,0.447059,0.690196}%
\pgfsetfillcolor{currentfill}%
\pgfsetlinewidth{1.003750pt}%
\definecolor{currentstroke}{rgb}{0.298039,0.447059,0.690196}%
\pgfsetstrokecolor{currentstroke}%
\pgfsetdash{}{0pt}%
\pgfpathmoveto{\pgfqpoint{3.905098in}{2.119498in}}%
\pgfpathcurveto{\pgfqpoint{3.913334in}{2.119498in}}{\pgfqpoint{3.921234in}{2.122771in}}{\pgfqpoint{3.927058in}{2.128595in}}%
\pgfpathcurveto{\pgfqpoint{3.932882in}{2.134419in}}{\pgfqpoint{3.936155in}{2.142319in}}{\pgfqpoint{3.936155in}{2.150555in}}%
\pgfpathcurveto{\pgfqpoint{3.936155in}{2.158791in}}{\pgfqpoint{3.932882in}{2.166691in}}{\pgfqpoint{3.927058in}{2.172515in}}%
\pgfpathcurveto{\pgfqpoint{3.921234in}{2.178339in}}{\pgfqpoint{3.913334in}{2.181611in}}{\pgfqpoint{3.905098in}{2.181611in}}%
\pgfpathcurveto{\pgfqpoint{3.896862in}{2.181611in}}{\pgfqpoint{3.888962in}{2.178339in}}{\pgfqpoint{3.883138in}{2.172515in}}%
\pgfpathcurveto{\pgfqpoint{3.877314in}{2.166691in}}{\pgfqpoint{3.874042in}{2.158791in}}{\pgfqpoint{3.874042in}{2.150555in}}%
\pgfpathcurveto{\pgfqpoint{3.874042in}{2.142319in}}{\pgfqpoint{3.877314in}{2.134419in}}{\pgfqpoint{3.883138in}{2.128595in}}%
\pgfpathcurveto{\pgfqpoint{3.888962in}{2.122771in}}{\pgfqpoint{3.896862in}{2.119498in}}{\pgfqpoint{3.905098in}{2.119498in}}%
\pgfpathclose%
\pgfusepath{stroke,fill}%
\end{pgfscope}%
\begin{pgfscope}%
\pgfpathrectangle{\pgfqpoint{3.793912in}{0.557870in}}{\pgfqpoint{2.446088in}{1.684734in}}%
\pgfusepath{clip}%
\pgfsetbuttcap%
\pgfsetroundjoin%
\definecolor{currentfill}{rgb}{0.298039,0.447059,0.690196}%
\pgfsetfillcolor{currentfill}%
\pgfsetlinewidth{1.003750pt}%
\definecolor{currentstroke}{rgb}{0.298039,0.447059,0.690196}%
\pgfsetstrokecolor{currentstroke}%
\pgfsetdash{}{0pt}%
\pgfpathmoveto{\pgfqpoint{5.418142in}{1.175800in}}%
\pgfpathcurveto{\pgfqpoint{5.426378in}{1.175800in}}{\pgfqpoint{5.434278in}{1.179072in}}{\pgfqpoint{5.440102in}{1.184896in}}%
\pgfpathcurveto{\pgfqpoint{5.445926in}{1.190720in}}{\pgfqpoint{5.449199in}{1.198620in}}{\pgfqpoint{5.449199in}{1.206856in}}%
\pgfpathcurveto{\pgfqpoint{5.449199in}{1.215093in}}{\pgfqpoint{5.445926in}{1.222993in}}{\pgfqpoint{5.440102in}{1.228817in}}%
\pgfpathcurveto{\pgfqpoint{5.434278in}{1.234641in}}{\pgfqpoint{5.426378in}{1.237913in}}{\pgfqpoint{5.418142in}{1.237913in}}%
\pgfpathcurveto{\pgfqpoint{5.409906in}{1.237913in}}{\pgfqpoint{5.402006in}{1.234641in}}{\pgfqpoint{5.396182in}{1.228817in}}%
\pgfpathcurveto{\pgfqpoint{5.390358in}{1.222993in}}{\pgfqpoint{5.387086in}{1.215093in}}{\pgfqpoint{5.387086in}{1.206856in}}%
\pgfpathcurveto{\pgfqpoint{5.387086in}{1.198620in}}{\pgfqpoint{5.390358in}{1.190720in}}{\pgfqpoint{5.396182in}{1.184896in}}%
\pgfpathcurveto{\pgfqpoint{5.402006in}{1.179072in}}{\pgfqpoint{5.409906in}{1.175800in}}{\pgfqpoint{5.418142in}{1.175800in}}%
\pgfpathclose%
\pgfusepath{stroke,fill}%
\end{pgfscope}%
\begin{pgfscope}%
\pgfpathrectangle{\pgfqpoint{3.793912in}{0.557870in}}{\pgfqpoint{2.446088in}{1.684734in}}%
\pgfusepath{clip}%
\pgfsetbuttcap%
\pgfsetroundjoin%
\definecolor{currentfill}{rgb}{0.298039,0.447059,0.690196}%
\pgfsetfillcolor{currentfill}%
\pgfsetlinewidth{1.003750pt}%
\definecolor{currentstroke}{rgb}{0.298039,0.447059,0.690196}%
\pgfsetstrokecolor{currentstroke}%
\pgfsetdash{}{0pt}%
\pgfpathmoveto{\pgfqpoint{3.905098in}{2.119498in}}%
\pgfpathcurveto{\pgfqpoint{3.913334in}{2.119498in}}{\pgfqpoint{3.921234in}{2.122771in}}{\pgfqpoint{3.927058in}{2.128595in}}%
\pgfpathcurveto{\pgfqpoint{3.932882in}{2.134419in}}{\pgfqpoint{3.936155in}{2.142319in}}{\pgfqpoint{3.936155in}{2.150555in}}%
\pgfpathcurveto{\pgfqpoint{3.936155in}{2.158791in}}{\pgfqpoint{3.932882in}{2.166691in}}{\pgfqpoint{3.927058in}{2.172515in}}%
\pgfpathcurveto{\pgfqpoint{3.921234in}{2.178339in}}{\pgfqpoint{3.913334in}{2.181611in}}{\pgfqpoint{3.905098in}{2.181611in}}%
\pgfpathcurveto{\pgfqpoint{3.896862in}{2.181611in}}{\pgfqpoint{3.888962in}{2.178339in}}{\pgfqpoint{3.883138in}{2.172515in}}%
\pgfpathcurveto{\pgfqpoint{3.877314in}{2.166691in}}{\pgfqpoint{3.874042in}{2.158791in}}{\pgfqpoint{3.874042in}{2.150555in}}%
\pgfpathcurveto{\pgfqpoint{3.874042in}{2.142319in}}{\pgfqpoint{3.877314in}{2.134419in}}{\pgfqpoint{3.883138in}{2.128595in}}%
\pgfpathcurveto{\pgfqpoint{3.888962in}{2.122771in}}{\pgfqpoint{3.896862in}{2.119498in}}{\pgfqpoint{3.905098in}{2.119498in}}%
\pgfpathclose%
\pgfusepath{stroke,fill}%
\end{pgfscope}%
\begin{pgfscope}%
\pgfpathrectangle{\pgfqpoint{3.793912in}{0.557870in}}{\pgfqpoint{2.446088in}{1.684734in}}%
\pgfusepath{clip}%
\pgfsetbuttcap%
\pgfsetroundjoin%
\definecolor{currentfill}{rgb}{0.298039,0.447059,0.690196}%
\pgfsetfillcolor{currentfill}%
\pgfsetlinewidth{1.003750pt}%
\definecolor{currentstroke}{rgb}{0.298039,0.447059,0.690196}%
\pgfsetstrokecolor{currentstroke}%
\pgfsetdash{}{0pt}%
\pgfpathmoveto{\pgfqpoint{3.905098in}{2.119498in}}%
\pgfpathcurveto{\pgfqpoint{3.913334in}{2.119498in}}{\pgfqpoint{3.921234in}{2.122771in}}{\pgfqpoint{3.927058in}{2.128595in}}%
\pgfpathcurveto{\pgfqpoint{3.932882in}{2.134419in}}{\pgfqpoint{3.936155in}{2.142319in}}{\pgfqpoint{3.936155in}{2.150555in}}%
\pgfpathcurveto{\pgfqpoint{3.936155in}{2.158791in}}{\pgfqpoint{3.932882in}{2.166691in}}{\pgfqpoint{3.927058in}{2.172515in}}%
\pgfpathcurveto{\pgfqpoint{3.921234in}{2.178339in}}{\pgfqpoint{3.913334in}{2.181611in}}{\pgfqpoint{3.905098in}{2.181611in}}%
\pgfpathcurveto{\pgfqpoint{3.896862in}{2.181611in}}{\pgfqpoint{3.888962in}{2.178339in}}{\pgfqpoint{3.883138in}{2.172515in}}%
\pgfpathcurveto{\pgfqpoint{3.877314in}{2.166691in}}{\pgfqpoint{3.874042in}{2.158791in}}{\pgfqpoint{3.874042in}{2.150555in}}%
\pgfpathcurveto{\pgfqpoint{3.874042in}{2.142319in}}{\pgfqpoint{3.877314in}{2.134419in}}{\pgfqpoint{3.883138in}{2.128595in}}%
\pgfpathcurveto{\pgfqpoint{3.888962in}{2.122771in}}{\pgfqpoint{3.896862in}{2.119498in}}{\pgfqpoint{3.905098in}{2.119498in}}%
\pgfpathclose%
\pgfusepath{stroke,fill}%
\end{pgfscope}%
\begin{pgfscope}%
\pgfpathrectangle{\pgfqpoint{3.793912in}{0.557870in}}{\pgfqpoint{2.446088in}{1.684734in}}%
\pgfusepath{clip}%
\pgfsetbuttcap%
\pgfsetroundjoin%
\definecolor{currentfill}{rgb}{0.298039,0.447059,0.690196}%
\pgfsetfillcolor{currentfill}%
\pgfsetlinewidth{1.003750pt}%
\definecolor{currentstroke}{rgb}{0.298039,0.447059,0.690196}%
\pgfsetstrokecolor{currentstroke}%
\pgfsetdash{}{0pt}%
\pgfpathmoveto{\pgfqpoint{5.028419in}{1.933853in}}%
\pgfpathcurveto{\pgfqpoint{5.036655in}{1.933853in}}{\pgfqpoint{5.044555in}{1.937125in}}{\pgfqpoint{5.050379in}{1.942949in}}%
\pgfpathcurveto{\pgfqpoint{5.056203in}{1.948773in}}{\pgfqpoint{5.059475in}{1.956673in}}{\pgfqpoint{5.059475in}{1.964909in}}%
\pgfpathcurveto{\pgfqpoint{5.059475in}{1.973146in}}{\pgfqpoint{5.056203in}{1.981046in}}{\pgfqpoint{5.050379in}{1.986870in}}%
\pgfpathcurveto{\pgfqpoint{5.044555in}{1.992693in}}{\pgfqpoint{5.036655in}{1.995966in}}{\pgfqpoint{5.028419in}{1.995966in}}%
\pgfpathcurveto{\pgfqpoint{5.020182in}{1.995966in}}{\pgfqpoint{5.012282in}{1.992693in}}{\pgfqpoint{5.006458in}{1.986870in}}%
\pgfpathcurveto{\pgfqpoint{5.000634in}{1.981046in}}{\pgfqpoint{4.997362in}{1.973146in}}{\pgfqpoint{4.997362in}{1.964909in}}%
\pgfpathcurveto{\pgfqpoint{4.997362in}{1.956673in}}{\pgfqpoint{5.000634in}{1.948773in}}{\pgfqpoint{5.006458in}{1.942949in}}%
\pgfpathcurveto{\pgfqpoint{5.012282in}{1.937125in}}{\pgfqpoint{5.020182in}{1.933853in}}{\pgfqpoint{5.028419in}{1.933853in}}%
\pgfpathclose%
\pgfusepath{stroke,fill}%
\end{pgfscope}%
\begin{pgfscope}%
\pgfpathrectangle{\pgfqpoint{3.793912in}{0.557870in}}{\pgfqpoint{2.446088in}{1.684734in}}%
\pgfusepath{clip}%
\pgfsetbuttcap%
\pgfsetroundjoin%
\definecolor{currentfill}{rgb}{0.298039,0.447059,0.690196}%
\pgfsetfillcolor{currentfill}%
\pgfsetlinewidth{1.003750pt}%
\definecolor{currentstroke}{rgb}{0.298039,0.447059,0.690196}%
\pgfsetstrokecolor{currentstroke}%
\pgfsetdash{}{0pt}%
\pgfpathmoveto{\pgfqpoint{5.670316in}{1.438798in}}%
\pgfpathcurveto{\pgfqpoint{5.678552in}{1.438798in}}{\pgfqpoint{5.686452in}{1.442070in}}{\pgfqpoint{5.692276in}{1.447894in}}%
\pgfpathcurveto{\pgfqpoint{5.698100in}{1.453718in}}{\pgfqpoint{5.701373in}{1.461618in}}{\pgfqpoint{5.701373in}{1.469854in}}%
\pgfpathcurveto{\pgfqpoint{5.701373in}{1.478091in}}{\pgfqpoint{5.698100in}{1.485991in}}{\pgfqpoint{5.692276in}{1.491815in}}%
\pgfpathcurveto{\pgfqpoint{5.686452in}{1.497639in}}{\pgfqpoint{5.678552in}{1.500911in}}{\pgfqpoint{5.670316in}{1.500911in}}%
\pgfpathcurveto{\pgfqpoint{5.662080in}{1.500911in}}{\pgfqpoint{5.654180in}{1.497639in}}{\pgfqpoint{5.648356in}{1.491815in}}%
\pgfpathcurveto{\pgfqpoint{5.642532in}{1.485991in}}{\pgfqpoint{5.639260in}{1.478091in}}{\pgfqpoint{5.639260in}{1.469854in}}%
\pgfpathcurveto{\pgfqpoint{5.639260in}{1.461618in}}{\pgfqpoint{5.642532in}{1.453718in}}{\pgfqpoint{5.648356in}{1.447894in}}%
\pgfpathcurveto{\pgfqpoint{5.654180in}{1.442070in}}{\pgfqpoint{5.662080in}{1.438798in}}{\pgfqpoint{5.670316in}{1.438798in}}%
\pgfpathclose%
\pgfusepath{stroke,fill}%
\end{pgfscope}%
\begin{pgfscope}%
\pgfpathrectangle{\pgfqpoint{3.793912in}{0.557870in}}{\pgfqpoint{2.446088in}{1.684734in}}%
\pgfusepath{clip}%
\pgfsetbuttcap%
\pgfsetroundjoin%
\definecolor{currentfill}{rgb}{0.298039,0.447059,0.690196}%
\pgfsetfillcolor{currentfill}%
\pgfsetlinewidth{1.003750pt}%
\definecolor{currentstroke}{rgb}{0.298039,0.447059,0.690196}%
\pgfsetstrokecolor{currentstroke}%
\pgfsetdash{}{0pt}%
\pgfpathmoveto{\pgfqpoint{5.624466in}{1.438798in}}%
\pgfpathcurveto{\pgfqpoint{5.632702in}{1.438798in}}{\pgfqpoint{5.640603in}{1.442070in}}{\pgfqpoint{5.646426in}{1.447894in}}%
\pgfpathcurveto{\pgfqpoint{5.652250in}{1.453718in}}{\pgfqpoint{5.655523in}{1.461618in}}{\pgfqpoint{5.655523in}{1.469854in}}%
\pgfpathcurveto{\pgfqpoint{5.655523in}{1.478091in}}{\pgfqpoint{5.652250in}{1.485991in}}{\pgfqpoint{5.646426in}{1.491815in}}%
\pgfpathcurveto{\pgfqpoint{5.640603in}{1.497639in}}{\pgfqpoint{5.632702in}{1.500911in}}{\pgfqpoint{5.624466in}{1.500911in}}%
\pgfpathcurveto{\pgfqpoint{5.616230in}{1.500911in}}{\pgfqpoint{5.608330in}{1.497639in}}{\pgfqpoint{5.602506in}{1.491815in}}%
\pgfpathcurveto{\pgfqpoint{5.596682in}{1.485991in}}{\pgfqpoint{5.593410in}{1.478091in}}{\pgfqpoint{5.593410in}{1.469854in}}%
\pgfpathcurveto{\pgfqpoint{5.593410in}{1.461618in}}{\pgfqpoint{5.596682in}{1.453718in}}{\pgfqpoint{5.602506in}{1.447894in}}%
\pgfpathcurveto{\pgfqpoint{5.608330in}{1.442070in}}{\pgfqpoint{5.616230in}{1.438798in}}{\pgfqpoint{5.624466in}{1.438798in}}%
\pgfpathclose%
\pgfusepath{stroke,fill}%
\end{pgfscope}%
\begin{pgfscope}%
\pgfpathrectangle{\pgfqpoint{3.793912in}{0.557870in}}{\pgfqpoint{2.446088in}{1.684734in}}%
\pgfusepath{clip}%
\pgfsetbuttcap%
\pgfsetroundjoin%
\definecolor{currentfill}{rgb}{0.298039,0.447059,0.690196}%
\pgfsetfillcolor{currentfill}%
\pgfsetlinewidth{1.003750pt}%
\definecolor{currentstroke}{rgb}{0.298039,0.447059,0.690196}%
\pgfsetstrokecolor{currentstroke}%
\pgfsetdash{}{0pt}%
\pgfpathmoveto{\pgfqpoint{4.019723in}{2.057617in}}%
\pgfpathcurveto{\pgfqpoint{4.027959in}{2.057617in}}{\pgfqpoint{4.035859in}{2.060889in}}{\pgfqpoint{4.041683in}{2.066713in}}%
\pgfpathcurveto{\pgfqpoint{4.047507in}{2.072537in}}{\pgfqpoint{4.050779in}{2.080437in}}{\pgfqpoint{4.050779in}{2.088673in}}%
\pgfpathcurveto{\pgfqpoint{4.050779in}{2.096909in}}{\pgfqpoint{4.047507in}{2.104809in}}{\pgfqpoint{4.041683in}{2.110633in}}%
\pgfpathcurveto{\pgfqpoint{4.035859in}{2.116457in}}{\pgfqpoint{4.027959in}{2.119730in}}{\pgfqpoint{4.019723in}{2.119730in}}%
\pgfpathcurveto{\pgfqpoint{4.011486in}{2.119730in}}{\pgfqpoint{4.003586in}{2.116457in}}{\pgfqpoint{3.997762in}{2.110633in}}%
\pgfpathcurveto{\pgfqpoint{3.991938in}{2.104809in}}{\pgfqpoint{3.988666in}{2.096909in}}{\pgfqpoint{3.988666in}{2.088673in}}%
\pgfpathcurveto{\pgfqpoint{3.988666in}{2.080437in}}{\pgfqpoint{3.991938in}{2.072537in}}{\pgfqpoint{3.997762in}{2.066713in}}%
\pgfpathcurveto{\pgfqpoint{4.003586in}{2.060889in}}{\pgfqpoint{4.011486in}{2.057617in}}{\pgfqpoint{4.019723in}{2.057617in}}%
\pgfpathclose%
\pgfusepath{stroke,fill}%
\end{pgfscope}%
\begin{pgfscope}%
\pgfpathrectangle{\pgfqpoint{3.793912in}{0.557870in}}{\pgfqpoint{2.446088in}{1.684734in}}%
\pgfusepath{clip}%
\pgfsetbuttcap%
\pgfsetroundjoin%
\definecolor{currentfill}{rgb}{0.298039,0.447059,0.690196}%
\pgfsetfillcolor{currentfill}%
\pgfsetlinewidth{1.003750pt}%
\definecolor{currentstroke}{rgb}{0.298039,0.447059,0.690196}%
\pgfsetstrokecolor{currentstroke}%
\pgfsetdash{}{0pt}%
\pgfpathmoveto{\pgfqpoint{3.905098in}{2.119498in}}%
\pgfpathcurveto{\pgfqpoint{3.913334in}{2.119498in}}{\pgfqpoint{3.921234in}{2.122771in}}{\pgfqpoint{3.927058in}{2.128595in}}%
\pgfpathcurveto{\pgfqpoint{3.932882in}{2.134419in}}{\pgfqpoint{3.936155in}{2.142319in}}{\pgfqpoint{3.936155in}{2.150555in}}%
\pgfpathcurveto{\pgfqpoint{3.936155in}{2.158791in}}{\pgfqpoint{3.932882in}{2.166691in}}{\pgfqpoint{3.927058in}{2.172515in}}%
\pgfpathcurveto{\pgfqpoint{3.921234in}{2.178339in}}{\pgfqpoint{3.913334in}{2.181611in}}{\pgfqpoint{3.905098in}{2.181611in}}%
\pgfpathcurveto{\pgfqpoint{3.896862in}{2.181611in}}{\pgfqpoint{3.888962in}{2.178339in}}{\pgfqpoint{3.883138in}{2.172515in}}%
\pgfpathcurveto{\pgfqpoint{3.877314in}{2.166691in}}{\pgfqpoint{3.874042in}{2.158791in}}{\pgfqpoint{3.874042in}{2.150555in}}%
\pgfpathcurveto{\pgfqpoint{3.874042in}{2.142319in}}{\pgfqpoint{3.877314in}{2.134419in}}{\pgfqpoint{3.883138in}{2.128595in}}%
\pgfpathcurveto{\pgfqpoint{3.888962in}{2.122771in}}{\pgfqpoint{3.896862in}{2.119498in}}{\pgfqpoint{3.905098in}{2.119498in}}%
\pgfpathclose%
\pgfusepath{stroke,fill}%
\end{pgfscope}%
\begin{pgfscope}%
\pgfpathrectangle{\pgfqpoint{3.793912in}{0.557870in}}{\pgfqpoint{2.446088in}{1.684734in}}%
\pgfusepath{clip}%
\pgfsetbuttcap%
\pgfsetroundjoin%
\definecolor{currentfill}{rgb}{0.298039,0.447059,0.690196}%
\pgfsetfillcolor{currentfill}%
\pgfsetlinewidth{1.003750pt}%
\definecolor{currentstroke}{rgb}{0.298039,0.447059,0.690196}%
\pgfsetstrokecolor{currentstroke}%
\pgfsetdash{}{0pt}%
\pgfpathmoveto{\pgfqpoint{3.905098in}{2.119498in}}%
\pgfpathcurveto{\pgfqpoint{3.913334in}{2.119498in}}{\pgfqpoint{3.921234in}{2.122771in}}{\pgfqpoint{3.927058in}{2.128595in}}%
\pgfpathcurveto{\pgfqpoint{3.932882in}{2.134419in}}{\pgfqpoint{3.936155in}{2.142319in}}{\pgfqpoint{3.936155in}{2.150555in}}%
\pgfpathcurveto{\pgfqpoint{3.936155in}{2.158791in}}{\pgfqpoint{3.932882in}{2.166691in}}{\pgfqpoint{3.927058in}{2.172515in}}%
\pgfpathcurveto{\pgfqpoint{3.921234in}{2.178339in}}{\pgfqpoint{3.913334in}{2.181611in}}{\pgfqpoint{3.905098in}{2.181611in}}%
\pgfpathcurveto{\pgfqpoint{3.896862in}{2.181611in}}{\pgfqpoint{3.888962in}{2.178339in}}{\pgfqpoint{3.883138in}{2.172515in}}%
\pgfpathcurveto{\pgfqpoint{3.877314in}{2.166691in}}{\pgfqpoint{3.874042in}{2.158791in}}{\pgfqpoint{3.874042in}{2.150555in}}%
\pgfpathcurveto{\pgfqpoint{3.874042in}{2.142319in}}{\pgfqpoint{3.877314in}{2.134419in}}{\pgfqpoint{3.883138in}{2.128595in}}%
\pgfpathcurveto{\pgfqpoint{3.888962in}{2.122771in}}{\pgfqpoint{3.896862in}{2.119498in}}{\pgfqpoint{3.905098in}{2.119498in}}%
\pgfpathclose%
\pgfusepath{stroke,fill}%
\end{pgfscope}%
\begin{pgfscope}%
\pgfpathrectangle{\pgfqpoint{3.793912in}{0.557870in}}{\pgfqpoint{2.446088in}{1.684734in}}%
\pgfusepath{clip}%
\pgfsetbuttcap%
\pgfsetroundjoin%
\definecolor{currentfill}{rgb}{0.298039,0.447059,0.690196}%
\pgfsetfillcolor{currentfill}%
\pgfsetlinewidth{1.003750pt}%
\definecolor{currentstroke}{rgb}{0.298039,0.447059,0.690196}%
\pgfsetstrokecolor{currentstroke}%
\pgfsetdash{}{0pt}%
\pgfpathmoveto{\pgfqpoint{5.739091in}{1.144859in}}%
\pgfpathcurveto{\pgfqpoint{5.747327in}{1.144859in}}{\pgfqpoint{5.755227in}{1.148131in}}{\pgfqpoint{5.761051in}{1.153955in}}%
\pgfpathcurveto{\pgfqpoint{5.766875in}{1.159779in}}{\pgfqpoint{5.770147in}{1.167679in}}{\pgfqpoint{5.770147in}{1.175915in}}%
\pgfpathcurveto{\pgfqpoint{5.770147in}{1.184152in}}{\pgfqpoint{5.766875in}{1.192052in}}{\pgfqpoint{5.761051in}{1.197876in}}%
\pgfpathcurveto{\pgfqpoint{5.755227in}{1.203700in}}{\pgfqpoint{5.747327in}{1.206972in}}{\pgfqpoint{5.739091in}{1.206972in}}%
\pgfpathcurveto{\pgfqpoint{5.730854in}{1.206972in}}{\pgfqpoint{5.722954in}{1.203700in}}{\pgfqpoint{5.717130in}{1.197876in}}%
\pgfpathcurveto{\pgfqpoint{5.711307in}{1.192052in}}{\pgfqpoint{5.708034in}{1.184152in}}{\pgfqpoint{5.708034in}{1.175915in}}%
\pgfpathcurveto{\pgfqpoint{5.708034in}{1.167679in}}{\pgfqpoint{5.711307in}{1.159779in}}{\pgfqpoint{5.717130in}{1.153955in}}%
\pgfpathcurveto{\pgfqpoint{5.722954in}{1.148131in}}{\pgfqpoint{5.730854in}{1.144859in}}{\pgfqpoint{5.739091in}{1.144859in}}%
\pgfpathclose%
\pgfusepath{stroke,fill}%
\end{pgfscope}%
\begin{pgfscope}%
\pgfpathrectangle{\pgfqpoint{3.793912in}{0.557870in}}{\pgfqpoint{2.446088in}{1.684734in}}%
\pgfusepath{clip}%
\pgfsetbuttcap%
\pgfsetroundjoin%
\definecolor{currentfill}{rgb}{0.298039,0.447059,0.690196}%
\pgfsetfillcolor{currentfill}%
\pgfsetlinewidth{1.003750pt}%
\definecolor{currentstroke}{rgb}{0.298039,0.447059,0.690196}%
\pgfsetstrokecolor{currentstroke}%
\pgfsetdash{}{0pt}%
\pgfpathmoveto{\pgfqpoint{5.188893in}{1.500680in}}%
\pgfpathcurveto{\pgfqpoint{5.197129in}{1.500680in}}{\pgfqpoint{5.205029in}{1.503952in}}{\pgfqpoint{5.210853in}{1.509776in}}%
\pgfpathcurveto{\pgfqpoint{5.216677in}{1.515600in}}{\pgfqpoint{5.219949in}{1.523500in}}{\pgfqpoint{5.219949in}{1.531736in}}%
\pgfpathcurveto{\pgfqpoint{5.219949in}{1.539972in}}{\pgfqpoint{5.216677in}{1.547873in}}{\pgfqpoint{5.210853in}{1.553696in}}%
\pgfpathcurveto{\pgfqpoint{5.205029in}{1.559520in}}{\pgfqpoint{5.197129in}{1.562793in}}{\pgfqpoint{5.188893in}{1.562793in}}%
\pgfpathcurveto{\pgfqpoint{5.180657in}{1.562793in}}{\pgfqpoint{5.172757in}{1.559520in}}{\pgfqpoint{5.166933in}{1.553696in}}%
\pgfpathcurveto{\pgfqpoint{5.161109in}{1.547873in}}{\pgfqpoint{5.157836in}{1.539972in}}{\pgfqpoint{5.157836in}{1.531736in}}%
\pgfpathcurveto{\pgfqpoint{5.157836in}{1.523500in}}{\pgfqpoint{5.161109in}{1.515600in}}{\pgfqpoint{5.166933in}{1.509776in}}%
\pgfpathcurveto{\pgfqpoint{5.172757in}{1.503952in}}{\pgfqpoint{5.180657in}{1.500680in}}{\pgfqpoint{5.188893in}{1.500680in}}%
\pgfpathclose%
\pgfusepath{stroke,fill}%
\end{pgfscope}%
\begin{pgfscope}%
\pgfpathrectangle{\pgfqpoint{3.793912in}{0.557870in}}{\pgfqpoint{2.446088in}{1.684734in}}%
\pgfusepath{clip}%
\pgfsetbuttcap%
\pgfsetroundjoin%
\definecolor{currentfill}{rgb}{0.298039,0.447059,0.690196}%
\pgfsetfillcolor{currentfill}%
\pgfsetlinewidth{1.003750pt}%
\definecolor{currentstroke}{rgb}{0.298039,0.447059,0.690196}%
\pgfsetstrokecolor{currentstroke}%
\pgfsetdash{}{0pt}%
\pgfpathmoveto{\pgfqpoint{3.905098in}{2.119498in}}%
\pgfpathcurveto{\pgfqpoint{3.913334in}{2.119498in}}{\pgfqpoint{3.921234in}{2.122771in}}{\pgfqpoint{3.927058in}{2.128595in}}%
\pgfpathcurveto{\pgfqpoint{3.932882in}{2.134419in}}{\pgfqpoint{3.936155in}{2.142319in}}{\pgfqpoint{3.936155in}{2.150555in}}%
\pgfpathcurveto{\pgfqpoint{3.936155in}{2.158791in}}{\pgfqpoint{3.932882in}{2.166691in}}{\pgfqpoint{3.927058in}{2.172515in}}%
\pgfpathcurveto{\pgfqpoint{3.921234in}{2.178339in}}{\pgfqpoint{3.913334in}{2.181611in}}{\pgfqpoint{3.905098in}{2.181611in}}%
\pgfpathcurveto{\pgfqpoint{3.896862in}{2.181611in}}{\pgfqpoint{3.888962in}{2.178339in}}{\pgfqpoint{3.883138in}{2.172515in}}%
\pgfpathcurveto{\pgfqpoint{3.877314in}{2.166691in}}{\pgfqpoint{3.874042in}{2.158791in}}{\pgfqpoint{3.874042in}{2.150555in}}%
\pgfpathcurveto{\pgfqpoint{3.874042in}{2.142319in}}{\pgfqpoint{3.877314in}{2.134419in}}{\pgfqpoint{3.883138in}{2.128595in}}%
\pgfpathcurveto{\pgfqpoint{3.888962in}{2.122771in}}{\pgfqpoint{3.896862in}{2.119498in}}{\pgfqpoint{3.905098in}{2.119498in}}%
\pgfpathclose%
\pgfusepath{stroke,fill}%
\end{pgfscope}%
\begin{pgfscope}%
\pgfpathrectangle{\pgfqpoint{3.793912in}{0.557870in}}{\pgfqpoint{2.446088in}{1.684734in}}%
\pgfusepath{clip}%
\pgfsetbuttcap%
\pgfsetroundjoin%
\definecolor{currentfill}{rgb}{0.298039,0.447059,0.690196}%
\pgfsetfillcolor{currentfill}%
\pgfsetlinewidth{1.003750pt}%
\definecolor{currentstroke}{rgb}{0.298039,0.447059,0.690196}%
\pgfsetstrokecolor{currentstroke}%
\pgfsetdash{}{0pt}%
\pgfpathmoveto{\pgfqpoint{3.905098in}{2.119498in}}%
\pgfpathcurveto{\pgfqpoint{3.913334in}{2.119498in}}{\pgfqpoint{3.921234in}{2.122771in}}{\pgfqpoint{3.927058in}{2.128595in}}%
\pgfpathcurveto{\pgfqpoint{3.932882in}{2.134419in}}{\pgfqpoint{3.936155in}{2.142319in}}{\pgfqpoint{3.936155in}{2.150555in}}%
\pgfpathcurveto{\pgfqpoint{3.936155in}{2.158791in}}{\pgfqpoint{3.932882in}{2.166691in}}{\pgfqpoint{3.927058in}{2.172515in}}%
\pgfpathcurveto{\pgfqpoint{3.921234in}{2.178339in}}{\pgfqpoint{3.913334in}{2.181611in}}{\pgfqpoint{3.905098in}{2.181611in}}%
\pgfpathcurveto{\pgfqpoint{3.896862in}{2.181611in}}{\pgfqpoint{3.888962in}{2.178339in}}{\pgfqpoint{3.883138in}{2.172515in}}%
\pgfpathcurveto{\pgfqpoint{3.877314in}{2.166691in}}{\pgfqpoint{3.874042in}{2.158791in}}{\pgfqpoint{3.874042in}{2.150555in}}%
\pgfpathcurveto{\pgfqpoint{3.874042in}{2.142319in}}{\pgfqpoint{3.877314in}{2.134419in}}{\pgfqpoint{3.883138in}{2.128595in}}%
\pgfpathcurveto{\pgfqpoint{3.888962in}{2.122771in}}{\pgfqpoint{3.896862in}{2.119498in}}{\pgfqpoint{3.905098in}{2.119498in}}%
\pgfpathclose%
\pgfusepath{stroke,fill}%
\end{pgfscope}%
\begin{pgfscope}%
\pgfpathrectangle{\pgfqpoint{3.793912in}{0.557870in}}{\pgfqpoint{2.446088in}{1.684734in}}%
\pgfusepath{clip}%
\pgfsetbuttcap%
\pgfsetroundjoin%
\definecolor{currentfill}{rgb}{0.298039,0.447059,0.690196}%
\pgfsetfillcolor{currentfill}%
\pgfsetlinewidth{1.003750pt}%
\definecolor{currentstroke}{rgb}{0.298039,0.447059,0.690196}%
\pgfsetstrokecolor{currentstroke}%
\pgfsetdash{}{0pt}%
\pgfpathmoveto{\pgfqpoint{3.996798in}{2.119498in}}%
\pgfpathcurveto{\pgfqpoint{4.005034in}{2.119498in}}{\pgfqpoint{4.012934in}{2.122771in}}{\pgfqpoint{4.018758in}{2.128595in}}%
\pgfpathcurveto{\pgfqpoint{4.024582in}{2.134419in}}{\pgfqpoint{4.027854in}{2.142319in}}{\pgfqpoint{4.027854in}{2.150555in}}%
\pgfpathcurveto{\pgfqpoint{4.027854in}{2.158791in}}{\pgfqpoint{4.024582in}{2.166691in}}{\pgfqpoint{4.018758in}{2.172515in}}%
\pgfpathcurveto{\pgfqpoint{4.012934in}{2.178339in}}{\pgfqpoint{4.005034in}{2.181611in}}{\pgfqpoint{3.996798in}{2.181611in}}%
\pgfpathcurveto{\pgfqpoint{3.988561in}{2.181611in}}{\pgfqpoint{3.980661in}{2.178339in}}{\pgfqpoint{3.974837in}{2.172515in}}%
\pgfpathcurveto{\pgfqpoint{3.969013in}{2.166691in}}{\pgfqpoint{3.965741in}{2.158791in}}{\pgfqpoint{3.965741in}{2.150555in}}%
\pgfpathcurveto{\pgfqpoint{3.965741in}{2.142319in}}{\pgfqpoint{3.969013in}{2.134419in}}{\pgfqpoint{3.974837in}{2.128595in}}%
\pgfpathcurveto{\pgfqpoint{3.980661in}{2.122771in}}{\pgfqpoint{3.988561in}{2.119498in}}{\pgfqpoint{3.996798in}{2.119498in}}%
\pgfpathclose%
\pgfusepath{stroke,fill}%
\end{pgfscope}%
\begin{pgfscope}%
\pgfpathrectangle{\pgfqpoint{3.793912in}{0.557870in}}{\pgfqpoint{2.446088in}{1.684734in}}%
\pgfusepath{clip}%
\pgfsetbuttcap%
\pgfsetroundjoin%
\definecolor{currentfill}{rgb}{0.298039,0.447059,0.690196}%
\pgfsetfillcolor{currentfill}%
\pgfsetlinewidth{1.003750pt}%
\definecolor{currentstroke}{rgb}{0.298039,0.447059,0.690196}%
\pgfsetstrokecolor{currentstroke}%
\pgfsetdash{}{0pt}%
\pgfpathmoveto{\pgfqpoint{3.905098in}{2.119498in}}%
\pgfpathcurveto{\pgfqpoint{3.913334in}{2.119498in}}{\pgfqpoint{3.921234in}{2.122771in}}{\pgfqpoint{3.927058in}{2.128595in}}%
\pgfpathcurveto{\pgfqpoint{3.932882in}{2.134419in}}{\pgfqpoint{3.936155in}{2.142319in}}{\pgfqpoint{3.936155in}{2.150555in}}%
\pgfpathcurveto{\pgfqpoint{3.936155in}{2.158791in}}{\pgfqpoint{3.932882in}{2.166691in}}{\pgfqpoint{3.927058in}{2.172515in}}%
\pgfpathcurveto{\pgfqpoint{3.921234in}{2.178339in}}{\pgfqpoint{3.913334in}{2.181611in}}{\pgfqpoint{3.905098in}{2.181611in}}%
\pgfpathcurveto{\pgfqpoint{3.896862in}{2.181611in}}{\pgfqpoint{3.888962in}{2.178339in}}{\pgfqpoint{3.883138in}{2.172515in}}%
\pgfpathcurveto{\pgfqpoint{3.877314in}{2.166691in}}{\pgfqpoint{3.874042in}{2.158791in}}{\pgfqpoint{3.874042in}{2.150555in}}%
\pgfpathcurveto{\pgfqpoint{3.874042in}{2.142319in}}{\pgfqpoint{3.877314in}{2.134419in}}{\pgfqpoint{3.883138in}{2.128595in}}%
\pgfpathcurveto{\pgfqpoint{3.888962in}{2.122771in}}{\pgfqpoint{3.896862in}{2.119498in}}{\pgfqpoint{3.905098in}{2.119498in}}%
\pgfpathclose%
\pgfusepath{stroke,fill}%
\end{pgfscope}%
\begin{pgfscope}%
\pgfpathrectangle{\pgfqpoint{3.793912in}{0.557870in}}{\pgfqpoint{2.446088in}{1.684734in}}%
\pgfusepath{clip}%
\pgfsetbuttcap%
\pgfsetroundjoin%
\definecolor{currentfill}{rgb}{0.298039,0.447059,0.690196}%
\pgfsetfillcolor{currentfill}%
\pgfsetlinewidth{1.003750pt}%
\definecolor{currentstroke}{rgb}{0.298039,0.447059,0.690196}%
\pgfsetstrokecolor{currentstroke}%
\pgfsetdash{}{0pt}%
\pgfpathmoveto{\pgfqpoint{5.211818in}{1.345975in}}%
\pgfpathcurveto{\pgfqpoint{5.220054in}{1.345975in}}{\pgfqpoint{5.227954in}{1.349247in}}{\pgfqpoint{5.233778in}{1.355071in}}%
\pgfpathcurveto{\pgfqpoint{5.239602in}{1.360895in}}{\pgfqpoint{5.242874in}{1.368795in}}{\pgfqpoint{5.242874in}{1.377032in}}%
\pgfpathcurveto{\pgfqpoint{5.242874in}{1.385268in}}{\pgfqpoint{5.239602in}{1.393168in}}{\pgfqpoint{5.233778in}{1.398992in}}%
\pgfpathcurveto{\pgfqpoint{5.227954in}{1.404816in}}{\pgfqpoint{5.220054in}{1.408088in}}{\pgfqpoint{5.211818in}{1.408088in}}%
\pgfpathcurveto{\pgfqpoint{5.203582in}{1.408088in}}{\pgfqpoint{5.195682in}{1.404816in}}{\pgfqpoint{5.189858in}{1.398992in}}%
\pgfpathcurveto{\pgfqpoint{5.184034in}{1.393168in}}{\pgfqpoint{5.180761in}{1.385268in}}{\pgfqpoint{5.180761in}{1.377032in}}%
\pgfpathcurveto{\pgfqpoint{5.180761in}{1.368795in}}{\pgfqpoint{5.184034in}{1.360895in}}{\pgfqpoint{5.189858in}{1.355071in}}%
\pgfpathcurveto{\pgfqpoint{5.195682in}{1.349247in}}{\pgfqpoint{5.203582in}{1.345975in}}{\pgfqpoint{5.211818in}{1.345975in}}%
\pgfpathclose%
\pgfusepath{stroke,fill}%
\end{pgfscope}%
\begin{pgfscope}%
\pgfpathrectangle{\pgfqpoint{3.793912in}{0.557870in}}{\pgfqpoint{2.446088in}{1.684734in}}%
\pgfusepath{clip}%
\pgfsetbuttcap%
\pgfsetroundjoin%
\definecolor{currentfill}{rgb}{0.298039,0.447059,0.690196}%
\pgfsetfillcolor{currentfill}%
\pgfsetlinewidth{1.003750pt}%
\definecolor{currentstroke}{rgb}{0.298039,0.447059,0.690196}%
\pgfsetstrokecolor{currentstroke}%
\pgfsetdash{}{0pt}%
\pgfpathmoveto{\pgfqpoint{5.303517in}{1.376916in}}%
\pgfpathcurveto{\pgfqpoint{5.311754in}{1.376916in}}{\pgfqpoint{5.319654in}{1.380188in}}{\pgfqpoint{5.325478in}{1.386012in}}%
\pgfpathcurveto{\pgfqpoint{5.331302in}{1.391836in}}{\pgfqpoint{5.334574in}{1.399736in}}{\pgfqpoint{5.334574in}{1.407972in}}%
\pgfpathcurveto{\pgfqpoint{5.334574in}{1.416209in}}{\pgfqpoint{5.331302in}{1.424109in}}{\pgfqpoint{5.325478in}{1.429933in}}%
\pgfpathcurveto{\pgfqpoint{5.319654in}{1.435757in}}{\pgfqpoint{5.311754in}{1.439029in}}{\pgfqpoint{5.303517in}{1.439029in}}%
\pgfpathcurveto{\pgfqpoint{5.295281in}{1.439029in}}{\pgfqpoint{5.287381in}{1.435757in}}{\pgfqpoint{5.281557in}{1.429933in}}%
\pgfpathcurveto{\pgfqpoint{5.275733in}{1.424109in}}{\pgfqpoint{5.272461in}{1.416209in}}{\pgfqpoint{5.272461in}{1.407972in}}%
\pgfpathcurveto{\pgfqpoint{5.272461in}{1.399736in}}{\pgfqpoint{5.275733in}{1.391836in}}{\pgfqpoint{5.281557in}{1.386012in}}%
\pgfpathcurveto{\pgfqpoint{5.287381in}{1.380188in}}{\pgfqpoint{5.295281in}{1.376916in}}{\pgfqpoint{5.303517in}{1.376916in}}%
\pgfpathclose%
\pgfusepath{stroke,fill}%
\end{pgfscope}%
\begin{pgfscope}%
\pgfpathrectangle{\pgfqpoint{3.793912in}{0.557870in}}{\pgfqpoint{2.446088in}{1.684734in}}%
\pgfusepath{clip}%
\pgfsetbuttcap%
\pgfsetroundjoin%
\definecolor{currentfill}{rgb}{0.298039,0.447059,0.690196}%
\pgfsetfillcolor{currentfill}%
\pgfsetlinewidth{1.003750pt}%
\definecolor{currentstroke}{rgb}{0.298039,0.447059,0.690196}%
\pgfsetstrokecolor{currentstroke}%
\pgfsetdash{}{0pt}%
\pgfpathmoveto{\pgfqpoint{3.905098in}{2.119498in}}%
\pgfpathcurveto{\pgfqpoint{3.913334in}{2.119498in}}{\pgfqpoint{3.921234in}{2.122771in}}{\pgfqpoint{3.927058in}{2.128595in}}%
\pgfpathcurveto{\pgfqpoint{3.932882in}{2.134419in}}{\pgfqpoint{3.936155in}{2.142319in}}{\pgfqpoint{3.936155in}{2.150555in}}%
\pgfpathcurveto{\pgfqpoint{3.936155in}{2.158791in}}{\pgfqpoint{3.932882in}{2.166691in}}{\pgfqpoint{3.927058in}{2.172515in}}%
\pgfpathcurveto{\pgfqpoint{3.921234in}{2.178339in}}{\pgfqpoint{3.913334in}{2.181611in}}{\pgfqpoint{3.905098in}{2.181611in}}%
\pgfpathcurveto{\pgfqpoint{3.896862in}{2.181611in}}{\pgfqpoint{3.888962in}{2.178339in}}{\pgfqpoint{3.883138in}{2.172515in}}%
\pgfpathcurveto{\pgfqpoint{3.877314in}{2.166691in}}{\pgfqpoint{3.874042in}{2.158791in}}{\pgfqpoint{3.874042in}{2.150555in}}%
\pgfpathcurveto{\pgfqpoint{3.874042in}{2.142319in}}{\pgfqpoint{3.877314in}{2.134419in}}{\pgfqpoint{3.883138in}{2.128595in}}%
\pgfpathcurveto{\pgfqpoint{3.888962in}{2.122771in}}{\pgfqpoint{3.896862in}{2.119498in}}{\pgfqpoint{3.905098in}{2.119498in}}%
\pgfpathclose%
\pgfusepath{stroke,fill}%
\end{pgfscope}%
\begin{pgfscope}%
\pgfpathrectangle{\pgfqpoint{3.793912in}{0.557870in}}{\pgfqpoint{2.446088in}{1.684734in}}%
\pgfusepath{clip}%
\pgfsetbuttcap%
\pgfsetroundjoin%
\definecolor{currentfill}{rgb}{0.298039,0.447059,0.690196}%
\pgfsetfillcolor{currentfill}%
\pgfsetlinewidth{1.003750pt}%
\definecolor{currentstroke}{rgb}{0.298039,0.447059,0.690196}%
\pgfsetstrokecolor{currentstroke}%
\pgfsetdash{}{0pt}%
\pgfpathmoveto{\pgfqpoint{3.905098in}{2.119498in}}%
\pgfpathcurveto{\pgfqpoint{3.913334in}{2.119498in}}{\pgfqpoint{3.921234in}{2.122771in}}{\pgfqpoint{3.927058in}{2.128595in}}%
\pgfpathcurveto{\pgfqpoint{3.932882in}{2.134419in}}{\pgfqpoint{3.936155in}{2.142319in}}{\pgfqpoint{3.936155in}{2.150555in}}%
\pgfpathcurveto{\pgfqpoint{3.936155in}{2.158791in}}{\pgfqpoint{3.932882in}{2.166691in}}{\pgfqpoint{3.927058in}{2.172515in}}%
\pgfpathcurveto{\pgfqpoint{3.921234in}{2.178339in}}{\pgfqpoint{3.913334in}{2.181611in}}{\pgfqpoint{3.905098in}{2.181611in}}%
\pgfpathcurveto{\pgfqpoint{3.896862in}{2.181611in}}{\pgfqpoint{3.888962in}{2.178339in}}{\pgfqpoint{3.883138in}{2.172515in}}%
\pgfpathcurveto{\pgfqpoint{3.877314in}{2.166691in}}{\pgfqpoint{3.874042in}{2.158791in}}{\pgfqpoint{3.874042in}{2.150555in}}%
\pgfpathcurveto{\pgfqpoint{3.874042in}{2.142319in}}{\pgfqpoint{3.877314in}{2.134419in}}{\pgfqpoint{3.883138in}{2.128595in}}%
\pgfpathcurveto{\pgfqpoint{3.888962in}{2.122771in}}{\pgfqpoint{3.896862in}{2.119498in}}{\pgfqpoint{3.905098in}{2.119498in}}%
\pgfpathclose%
\pgfusepath{stroke,fill}%
\end{pgfscope}%
\begin{pgfscope}%
\pgfpathrectangle{\pgfqpoint{3.793912in}{0.557870in}}{\pgfqpoint{2.446088in}{1.684734in}}%
\pgfusepath{clip}%
\pgfsetbuttcap%
\pgfsetroundjoin%
\definecolor{currentfill}{rgb}{0.298039,0.447059,0.690196}%
\pgfsetfillcolor{currentfill}%
\pgfsetlinewidth{1.003750pt}%
\definecolor{currentstroke}{rgb}{0.298039,0.447059,0.690196}%
\pgfsetstrokecolor{currentstroke}%
\pgfsetdash{}{0pt}%
\pgfpathmoveto{\pgfqpoint{3.905098in}{2.119498in}}%
\pgfpathcurveto{\pgfqpoint{3.913334in}{2.119498in}}{\pgfqpoint{3.921234in}{2.122771in}}{\pgfqpoint{3.927058in}{2.128595in}}%
\pgfpathcurveto{\pgfqpoint{3.932882in}{2.134419in}}{\pgfqpoint{3.936155in}{2.142319in}}{\pgfqpoint{3.936155in}{2.150555in}}%
\pgfpathcurveto{\pgfqpoint{3.936155in}{2.158791in}}{\pgfqpoint{3.932882in}{2.166691in}}{\pgfqpoint{3.927058in}{2.172515in}}%
\pgfpathcurveto{\pgfqpoint{3.921234in}{2.178339in}}{\pgfqpoint{3.913334in}{2.181611in}}{\pgfqpoint{3.905098in}{2.181611in}}%
\pgfpathcurveto{\pgfqpoint{3.896862in}{2.181611in}}{\pgfqpoint{3.888962in}{2.178339in}}{\pgfqpoint{3.883138in}{2.172515in}}%
\pgfpathcurveto{\pgfqpoint{3.877314in}{2.166691in}}{\pgfqpoint{3.874042in}{2.158791in}}{\pgfqpoint{3.874042in}{2.150555in}}%
\pgfpathcurveto{\pgfqpoint{3.874042in}{2.142319in}}{\pgfqpoint{3.877314in}{2.134419in}}{\pgfqpoint{3.883138in}{2.128595in}}%
\pgfpathcurveto{\pgfqpoint{3.888962in}{2.122771in}}{\pgfqpoint{3.896862in}{2.119498in}}{\pgfqpoint{3.905098in}{2.119498in}}%
\pgfpathclose%
\pgfusepath{stroke,fill}%
\end{pgfscope}%
\begin{pgfscope}%
\pgfpathrectangle{\pgfqpoint{3.793912in}{0.557870in}}{\pgfqpoint{2.446088in}{1.684734in}}%
\pgfusepath{clip}%
\pgfsetbuttcap%
\pgfsetroundjoin%
\definecolor{currentfill}{rgb}{0.298039,0.447059,0.690196}%
\pgfsetfillcolor{currentfill}%
\pgfsetlinewidth{1.003750pt}%
\definecolor{currentstroke}{rgb}{0.298039,0.447059,0.690196}%
\pgfsetstrokecolor{currentstroke}%
\pgfsetdash{}{0pt}%
\pgfpathmoveto{\pgfqpoint{5.624466in}{1.516150in}}%
\pgfpathcurveto{\pgfqpoint{5.632702in}{1.516150in}}{\pgfqpoint{5.640603in}{1.519422in}}{\pgfqpoint{5.646426in}{1.525246in}}%
\pgfpathcurveto{\pgfqpoint{5.652250in}{1.531070in}}{\pgfqpoint{5.655523in}{1.538970in}}{\pgfqpoint{5.655523in}{1.547207in}}%
\pgfpathcurveto{\pgfqpoint{5.655523in}{1.555443in}}{\pgfqpoint{5.652250in}{1.563343in}}{\pgfqpoint{5.646426in}{1.569167in}}%
\pgfpathcurveto{\pgfqpoint{5.640603in}{1.574991in}}{\pgfqpoint{5.632702in}{1.578263in}}{\pgfqpoint{5.624466in}{1.578263in}}%
\pgfpathcurveto{\pgfqpoint{5.616230in}{1.578263in}}{\pgfqpoint{5.608330in}{1.574991in}}{\pgfqpoint{5.602506in}{1.569167in}}%
\pgfpathcurveto{\pgfqpoint{5.596682in}{1.563343in}}{\pgfqpoint{5.593410in}{1.555443in}}{\pgfqpoint{5.593410in}{1.547207in}}%
\pgfpathcurveto{\pgfqpoint{5.593410in}{1.538970in}}{\pgfqpoint{5.596682in}{1.531070in}}{\pgfqpoint{5.602506in}{1.525246in}}%
\pgfpathcurveto{\pgfqpoint{5.608330in}{1.519422in}}{\pgfqpoint{5.616230in}{1.516150in}}{\pgfqpoint{5.624466in}{1.516150in}}%
\pgfpathclose%
\pgfusepath{stroke,fill}%
\end{pgfscope}%
\begin{pgfscope}%
\pgfpathrectangle{\pgfqpoint{3.793912in}{0.557870in}}{\pgfqpoint{2.446088in}{1.684734in}}%
\pgfusepath{clip}%
\pgfsetbuttcap%
\pgfsetroundjoin%
\definecolor{currentfill}{rgb}{0.298039,0.447059,0.690196}%
\pgfsetfillcolor{currentfill}%
\pgfsetlinewidth{1.003750pt}%
\definecolor{currentstroke}{rgb}{0.298039,0.447059,0.690196}%
\pgfsetstrokecolor{currentstroke}%
\pgfsetdash{}{0pt}%
\pgfpathmoveto{\pgfqpoint{5.670316in}{1.438798in}}%
\pgfpathcurveto{\pgfqpoint{5.678552in}{1.438798in}}{\pgfqpoint{5.686452in}{1.442070in}}{\pgfqpoint{5.692276in}{1.447894in}}%
\pgfpathcurveto{\pgfqpoint{5.698100in}{1.453718in}}{\pgfqpoint{5.701373in}{1.461618in}}{\pgfqpoint{5.701373in}{1.469854in}}%
\pgfpathcurveto{\pgfqpoint{5.701373in}{1.478091in}}{\pgfqpoint{5.698100in}{1.485991in}}{\pgfqpoint{5.692276in}{1.491815in}}%
\pgfpathcurveto{\pgfqpoint{5.686452in}{1.497639in}}{\pgfqpoint{5.678552in}{1.500911in}}{\pgfqpoint{5.670316in}{1.500911in}}%
\pgfpathcurveto{\pgfqpoint{5.662080in}{1.500911in}}{\pgfqpoint{5.654180in}{1.497639in}}{\pgfqpoint{5.648356in}{1.491815in}}%
\pgfpathcurveto{\pgfqpoint{5.642532in}{1.485991in}}{\pgfqpoint{5.639260in}{1.478091in}}{\pgfqpoint{5.639260in}{1.469854in}}%
\pgfpathcurveto{\pgfqpoint{5.639260in}{1.461618in}}{\pgfqpoint{5.642532in}{1.453718in}}{\pgfqpoint{5.648356in}{1.447894in}}%
\pgfpathcurveto{\pgfqpoint{5.654180in}{1.442070in}}{\pgfqpoint{5.662080in}{1.438798in}}{\pgfqpoint{5.670316in}{1.438798in}}%
\pgfpathclose%
\pgfusepath{stroke,fill}%
\end{pgfscope}%
\begin{pgfscope}%
\pgfpathrectangle{\pgfqpoint{3.793912in}{0.557870in}}{\pgfqpoint{2.446088in}{1.684734in}}%
\pgfusepath{clip}%
\pgfsetbuttcap%
\pgfsetroundjoin%
\definecolor{currentfill}{rgb}{0.298039,0.447059,0.690196}%
\pgfsetfillcolor{currentfill}%
\pgfsetlinewidth{1.003750pt}%
\definecolor{currentstroke}{rgb}{0.298039,0.447059,0.690196}%
\pgfsetstrokecolor{currentstroke}%
\pgfsetdash{}{0pt}%
\pgfpathmoveto{\pgfqpoint{5.647391in}{1.516150in}}%
\pgfpathcurveto{\pgfqpoint{5.655627in}{1.516150in}}{\pgfqpoint{5.663527in}{1.519422in}}{\pgfqpoint{5.669351in}{1.525246in}}%
\pgfpathcurveto{\pgfqpoint{5.675175in}{1.531070in}}{\pgfqpoint{5.678448in}{1.538970in}}{\pgfqpoint{5.678448in}{1.547207in}}%
\pgfpathcurveto{\pgfqpoint{5.678448in}{1.555443in}}{\pgfqpoint{5.675175in}{1.563343in}}{\pgfqpoint{5.669351in}{1.569167in}}%
\pgfpathcurveto{\pgfqpoint{5.663527in}{1.574991in}}{\pgfqpoint{5.655627in}{1.578263in}}{\pgfqpoint{5.647391in}{1.578263in}}%
\pgfpathcurveto{\pgfqpoint{5.639155in}{1.578263in}}{\pgfqpoint{5.631255in}{1.574991in}}{\pgfqpoint{5.625431in}{1.569167in}}%
\pgfpathcurveto{\pgfqpoint{5.619607in}{1.563343in}}{\pgfqpoint{5.616335in}{1.555443in}}{\pgfqpoint{5.616335in}{1.547207in}}%
\pgfpathcurveto{\pgfqpoint{5.616335in}{1.538970in}}{\pgfqpoint{5.619607in}{1.531070in}}{\pgfqpoint{5.625431in}{1.525246in}}%
\pgfpathcurveto{\pgfqpoint{5.631255in}{1.519422in}}{\pgfqpoint{5.639155in}{1.516150in}}{\pgfqpoint{5.647391in}{1.516150in}}%
\pgfpathclose%
\pgfusepath{stroke,fill}%
\end{pgfscope}%
\begin{pgfscope}%
\pgfpathrectangle{\pgfqpoint{3.793912in}{0.557870in}}{\pgfqpoint{2.446088in}{1.684734in}}%
\pgfusepath{clip}%
\pgfsetbuttcap%
\pgfsetroundjoin%
\definecolor{currentfill}{rgb}{0.298039,0.447059,0.690196}%
\pgfsetfillcolor{currentfill}%
\pgfsetlinewidth{1.003750pt}%
\definecolor{currentstroke}{rgb}{0.298039,0.447059,0.690196}%
\pgfsetstrokecolor{currentstroke}%
\pgfsetdash{}{0pt}%
\pgfpathmoveto{\pgfqpoint{3.950948in}{2.119498in}}%
\pgfpathcurveto{\pgfqpoint{3.959184in}{2.119498in}}{\pgfqpoint{3.967084in}{2.122771in}}{\pgfqpoint{3.972908in}{2.128595in}}%
\pgfpathcurveto{\pgfqpoint{3.978732in}{2.134419in}}{\pgfqpoint{3.982004in}{2.142319in}}{\pgfqpoint{3.982004in}{2.150555in}}%
\pgfpathcurveto{\pgfqpoint{3.982004in}{2.158791in}}{\pgfqpoint{3.978732in}{2.166691in}}{\pgfqpoint{3.972908in}{2.172515in}}%
\pgfpathcurveto{\pgfqpoint{3.967084in}{2.178339in}}{\pgfqpoint{3.959184in}{2.181611in}}{\pgfqpoint{3.950948in}{2.181611in}}%
\pgfpathcurveto{\pgfqpoint{3.942712in}{2.181611in}}{\pgfqpoint{3.934812in}{2.178339in}}{\pgfqpoint{3.928988in}{2.172515in}}%
\pgfpathcurveto{\pgfqpoint{3.923164in}{2.166691in}}{\pgfqpoint{3.919891in}{2.158791in}}{\pgfqpoint{3.919891in}{2.150555in}}%
\pgfpathcurveto{\pgfqpoint{3.919891in}{2.142319in}}{\pgfqpoint{3.923164in}{2.134419in}}{\pgfqpoint{3.928988in}{2.128595in}}%
\pgfpathcurveto{\pgfqpoint{3.934812in}{2.122771in}}{\pgfqpoint{3.942712in}{2.119498in}}{\pgfqpoint{3.950948in}{2.119498in}}%
\pgfpathclose%
\pgfusepath{stroke,fill}%
\end{pgfscope}%
\begin{pgfscope}%
\pgfpathrectangle{\pgfqpoint{3.793912in}{0.557870in}}{\pgfqpoint{2.446088in}{1.684734in}}%
\pgfusepath{clip}%
\pgfsetbuttcap%
\pgfsetroundjoin%
\definecolor{currentfill}{rgb}{0.298039,0.447059,0.690196}%
\pgfsetfillcolor{currentfill}%
\pgfsetlinewidth{1.003750pt}%
\definecolor{currentstroke}{rgb}{0.298039,0.447059,0.690196}%
\pgfsetstrokecolor{currentstroke}%
\pgfsetdash{}{0pt}%
\pgfpathmoveto{\pgfqpoint{5.647391in}{1.500680in}}%
\pgfpathcurveto{\pgfqpoint{5.655627in}{1.500680in}}{\pgfqpoint{5.663527in}{1.503952in}}{\pgfqpoint{5.669351in}{1.509776in}}%
\pgfpathcurveto{\pgfqpoint{5.675175in}{1.515600in}}{\pgfqpoint{5.678448in}{1.523500in}}{\pgfqpoint{5.678448in}{1.531736in}}%
\pgfpathcurveto{\pgfqpoint{5.678448in}{1.539972in}}{\pgfqpoint{5.675175in}{1.547873in}}{\pgfqpoint{5.669351in}{1.553696in}}%
\pgfpathcurveto{\pgfqpoint{5.663527in}{1.559520in}}{\pgfqpoint{5.655627in}{1.562793in}}{\pgfqpoint{5.647391in}{1.562793in}}%
\pgfpathcurveto{\pgfqpoint{5.639155in}{1.562793in}}{\pgfqpoint{5.631255in}{1.559520in}}{\pgfqpoint{5.625431in}{1.553696in}}%
\pgfpathcurveto{\pgfqpoint{5.619607in}{1.547873in}}{\pgfqpoint{5.616335in}{1.539972in}}{\pgfqpoint{5.616335in}{1.531736in}}%
\pgfpathcurveto{\pgfqpoint{5.616335in}{1.523500in}}{\pgfqpoint{5.619607in}{1.515600in}}{\pgfqpoint{5.625431in}{1.509776in}}%
\pgfpathcurveto{\pgfqpoint{5.631255in}{1.503952in}}{\pgfqpoint{5.639155in}{1.500680in}}{\pgfqpoint{5.647391in}{1.500680in}}%
\pgfpathclose%
\pgfusepath{stroke,fill}%
\end{pgfscope}%
\begin{pgfscope}%
\pgfpathrectangle{\pgfqpoint{3.793912in}{0.557870in}}{\pgfqpoint{2.446088in}{1.684734in}}%
\pgfusepath{clip}%
\pgfsetbuttcap%
\pgfsetroundjoin%
\definecolor{currentfill}{rgb}{0.298039,0.447059,0.690196}%
\pgfsetfillcolor{currentfill}%
\pgfsetlinewidth{1.003750pt}%
\definecolor{currentstroke}{rgb}{0.298039,0.447059,0.690196}%
\pgfsetstrokecolor{currentstroke}%
\pgfsetdash{}{0pt}%
\pgfpathmoveto{\pgfqpoint{3.905098in}{2.119498in}}%
\pgfpathcurveto{\pgfqpoint{3.913334in}{2.119498in}}{\pgfqpoint{3.921234in}{2.122771in}}{\pgfqpoint{3.927058in}{2.128595in}}%
\pgfpathcurveto{\pgfqpoint{3.932882in}{2.134419in}}{\pgfqpoint{3.936155in}{2.142319in}}{\pgfqpoint{3.936155in}{2.150555in}}%
\pgfpathcurveto{\pgfqpoint{3.936155in}{2.158791in}}{\pgfqpoint{3.932882in}{2.166691in}}{\pgfqpoint{3.927058in}{2.172515in}}%
\pgfpathcurveto{\pgfqpoint{3.921234in}{2.178339in}}{\pgfqpoint{3.913334in}{2.181611in}}{\pgfqpoint{3.905098in}{2.181611in}}%
\pgfpathcurveto{\pgfqpoint{3.896862in}{2.181611in}}{\pgfqpoint{3.888962in}{2.178339in}}{\pgfqpoint{3.883138in}{2.172515in}}%
\pgfpathcurveto{\pgfqpoint{3.877314in}{2.166691in}}{\pgfqpoint{3.874042in}{2.158791in}}{\pgfqpoint{3.874042in}{2.150555in}}%
\pgfpathcurveto{\pgfqpoint{3.874042in}{2.142319in}}{\pgfqpoint{3.877314in}{2.134419in}}{\pgfqpoint{3.883138in}{2.128595in}}%
\pgfpathcurveto{\pgfqpoint{3.888962in}{2.122771in}}{\pgfqpoint{3.896862in}{2.119498in}}{\pgfqpoint{3.905098in}{2.119498in}}%
\pgfpathclose%
\pgfusepath{stroke,fill}%
\end{pgfscope}%
\begin{pgfscope}%
\pgfpathrectangle{\pgfqpoint{3.793912in}{0.557870in}}{\pgfqpoint{2.446088in}{1.684734in}}%
\pgfusepath{clip}%
\pgfsetbuttcap%
\pgfsetroundjoin%
\definecolor{currentfill}{rgb}{0.298039,0.447059,0.690196}%
\pgfsetfillcolor{currentfill}%
\pgfsetlinewidth{1.003750pt}%
\definecolor{currentstroke}{rgb}{0.298039,0.447059,0.690196}%
\pgfsetstrokecolor{currentstroke}%
\pgfsetdash{}{0pt}%
\pgfpathmoveto{\pgfqpoint{5.486917in}{1.593502in}}%
\pgfpathcurveto{\pgfqpoint{5.495153in}{1.593502in}}{\pgfqpoint{5.503053in}{1.596775in}}{\pgfqpoint{5.508877in}{1.602599in}}%
\pgfpathcurveto{\pgfqpoint{5.514701in}{1.608423in}}{\pgfqpoint{5.517973in}{1.616323in}}{\pgfqpoint{5.517973in}{1.624559in}}%
\pgfpathcurveto{\pgfqpoint{5.517973in}{1.632795in}}{\pgfqpoint{5.514701in}{1.640695in}}{\pgfqpoint{5.508877in}{1.646519in}}%
\pgfpathcurveto{\pgfqpoint{5.503053in}{1.652343in}}{\pgfqpoint{5.495153in}{1.655615in}}{\pgfqpoint{5.486917in}{1.655615in}}%
\pgfpathcurveto{\pgfqpoint{5.478680in}{1.655615in}}{\pgfqpoint{5.470780in}{1.652343in}}{\pgfqpoint{5.464956in}{1.646519in}}%
\pgfpathcurveto{\pgfqpoint{5.459133in}{1.640695in}}{\pgfqpoint{5.455860in}{1.632795in}}{\pgfqpoint{5.455860in}{1.624559in}}%
\pgfpathcurveto{\pgfqpoint{5.455860in}{1.616323in}}{\pgfqpoint{5.459133in}{1.608423in}}{\pgfqpoint{5.464956in}{1.602599in}}%
\pgfpathcurveto{\pgfqpoint{5.470780in}{1.596775in}}{\pgfqpoint{5.478680in}{1.593502in}}{\pgfqpoint{5.486917in}{1.593502in}}%
\pgfpathclose%
\pgfusepath{stroke,fill}%
\end{pgfscope}%
\begin{pgfscope}%
\pgfpathrectangle{\pgfqpoint{3.793912in}{0.557870in}}{\pgfqpoint{2.446088in}{1.684734in}}%
\pgfusepath{clip}%
\pgfsetbuttcap%
\pgfsetroundjoin%
\definecolor{currentfill}{rgb}{0.298039,0.447059,0.690196}%
\pgfsetfillcolor{currentfill}%
\pgfsetlinewidth{1.003750pt}%
\definecolor{currentstroke}{rgb}{0.298039,0.447059,0.690196}%
\pgfsetstrokecolor{currentstroke}%
\pgfsetdash{}{0pt}%
\pgfpathmoveto{\pgfqpoint{3.928023in}{2.119498in}}%
\pgfpathcurveto{\pgfqpoint{3.936259in}{2.119498in}}{\pgfqpoint{3.944159in}{2.122771in}}{\pgfqpoint{3.949983in}{2.128595in}}%
\pgfpathcurveto{\pgfqpoint{3.955807in}{2.134419in}}{\pgfqpoint{3.959079in}{2.142319in}}{\pgfqpoint{3.959079in}{2.150555in}}%
\pgfpathcurveto{\pgfqpoint{3.959079in}{2.158791in}}{\pgfqpoint{3.955807in}{2.166691in}}{\pgfqpoint{3.949983in}{2.172515in}}%
\pgfpathcurveto{\pgfqpoint{3.944159in}{2.178339in}}{\pgfqpoint{3.936259in}{2.181611in}}{\pgfqpoint{3.928023in}{2.181611in}}%
\pgfpathcurveto{\pgfqpoint{3.919787in}{2.181611in}}{\pgfqpoint{3.911887in}{2.178339in}}{\pgfqpoint{3.906063in}{2.172515in}}%
\pgfpathcurveto{\pgfqpoint{3.900239in}{2.166691in}}{\pgfqpoint{3.896966in}{2.158791in}}{\pgfqpoint{3.896966in}{2.150555in}}%
\pgfpathcurveto{\pgfqpoint{3.896966in}{2.142319in}}{\pgfqpoint{3.900239in}{2.134419in}}{\pgfqpoint{3.906063in}{2.128595in}}%
\pgfpathcurveto{\pgfqpoint{3.911887in}{2.122771in}}{\pgfqpoint{3.919787in}{2.119498in}}{\pgfqpoint{3.928023in}{2.119498in}}%
\pgfpathclose%
\pgfusepath{stroke,fill}%
\end{pgfscope}%
\begin{pgfscope}%
\pgfpathrectangle{\pgfqpoint{3.793912in}{0.557870in}}{\pgfqpoint{2.446088in}{1.684734in}}%
\pgfusepath{clip}%
\pgfsetbuttcap%
\pgfsetroundjoin%
\definecolor{currentfill}{rgb}{0.298039,0.447059,0.690196}%
\pgfsetfillcolor{currentfill}%
\pgfsetlinewidth{1.003750pt}%
\definecolor{currentstroke}{rgb}{0.298039,0.447059,0.690196}%
\pgfsetstrokecolor{currentstroke}%
\pgfsetdash{}{0pt}%
\pgfpathmoveto{\pgfqpoint{5.624466in}{1.407857in}}%
\pgfpathcurveto{\pgfqpoint{5.632702in}{1.407857in}}{\pgfqpoint{5.640603in}{1.411129in}}{\pgfqpoint{5.646426in}{1.416953in}}%
\pgfpathcurveto{\pgfqpoint{5.652250in}{1.422777in}}{\pgfqpoint{5.655523in}{1.430677in}}{\pgfqpoint{5.655523in}{1.438913in}}%
\pgfpathcurveto{\pgfqpoint{5.655523in}{1.447150in}}{\pgfqpoint{5.652250in}{1.455050in}}{\pgfqpoint{5.646426in}{1.460874in}}%
\pgfpathcurveto{\pgfqpoint{5.640603in}{1.466698in}}{\pgfqpoint{5.632702in}{1.469970in}}{\pgfqpoint{5.624466in}{1.469970in}}%
\pgfpathcurveto{\pgfqpoint{5.616230in}{1.469970in}}{\pgfqpoint{5.608330in}{1.466698in}}{\pgfqpoint{5.602506in}{1.460874in}}%
\pgfpathcurveto{\pgfqpoint{5.596682in}{1.455050in}}{\pgfqpoint{5.593410in}{1.447150in}}{\pgfqpoint{5.593410in}{1.438913in}}%
\pgfpathcurveto{\pgfqpoint{5.593410in}{1.430677in}}{\pgfqpoint{5.596682in}{1.422777in}}{\pgfqpoint{5.602506in}{1.416953in}}%
\pgfpathcurveto{\pgfqpoint{5.608330in}{1.411129in}}{\pgfqpoint{5.616230in}{1.407857in}}{\pgfqpoint{5.624466in}{1.407857in}}%
\pgfpathclose%
\pgfusepath{stroke,fill}%
\end{pgfscope}%
\begin{pgfscope}%
\pgfpathrectangle{\pgfqpoint{3.793912in}{0.557870in}}{\pgfqpoint{2.446088in}{1.684734in}}%
\pgfusepath{clip}%
\pgfsetbuttcap%
\pgfsetroundjoin%
\definecolor{currentfill}{rgb}{0.298039,0.447059,0.690196}%
\pgfsetfillcolor{currentfill}%
\pgfsetlinewidth{1.003750pt}%
\definecolor{currentstroke}{rgb}{0.298039,0.447059,0.690196}%
\pgfsetstrokecolor{currentstroke}%
\pgfsetdash{}{0pt}%
\pgfpathmoveto{\pgfqpoint{3.905098in}{2.119498in}}%
\pgfpathcurveto{\pgfqpoint{3.913334in}{2.119498in}}{\pgfqpoint{3.921234in}{2.122771in}}{\pgfqpoint{3.927058in}{2.128595in}}%
\pgfpathcurveto{\pgfqpoint{3.932882in}{2.134419in}}{\pgfqpoint{3.936155in}{2.142319in}}{\pgfqpoint{3.936155in}{2.150555in}}%
\pgfpathcurveto{\pgfqpoint{3.936155in}{2.158791in}}{\pgfqpoint{3.932882in}{2.166691in}}{\pgfqpoint{3.927058in}{2.172515in}}%
\pgfpathcurveto{\pgfqpoint{3.921234in}{2.178339in}}{\pgfqpoint{3.913334in}{2.181611in}}{\pgfqpoint{3.905098in}{2.181611in}}%
\pgfpathcurveto{\pgfqpoint{3.896862in}{2.181611in}}{\pgfqpoint{3.888962in}{2.178339in}}{\pgfqpoint{3.883138in}{2.172515in}}%
\pgfpathcurveto{\pgfqpoint{3.877314in}{2.166691in}}{\pgfqpoint{3.874042in}{2.158791in}}{\pgfqpoint{3.874042in}{2.150555in}}%
\pgfpathcurveto{\pgfqpoint{3.874042in}{2.142319in}}{\pgfqpoint{3.877314in}{2.134419in}}{\pgfqpoint{3.883138in}{2.128595in}}%
\pgfpathcurveto{\pgfqpoint{3.888962in}{2.122771in}}{\pgfqpoint{3.896862in}{2.119498in}}{\pgfqpoint{3.905098in}{2.119498in}}%
\pgfpathclose%
\pgfusepath{stroke,fill}%
\end{pgfscope}%
\begin{pgfscope}%
\pgfpathrectangle{\pgfqpoint{3.793912in}{0.557870in}}{\pgfqpoint{2.446088in}{1.684734in}}%
\pgfusepath{clip}%
\pgfsetbuttcap%
\pgfsetroundjoin%
\definecolor{currentfill}{rgb}{0.298039,0.447059,0.690196}%
\pgfsetfillcolor{currentfill}%
\pgfsetlinewidth{1.003750pt}%
\definecolor{currentstroke}{rgb}{0.298039,0.447059,0.690196}%
\pgfsetstrokecolor{currentstroke}%
\pgfsetdash{}{0pt}%
\pgfpathmoveto{\pgfqpoint{4.042647in}{2.104028in}}%
\pgfpathcurveto{\pgfqpoint{4.050884in}{2.104028in}}{\pgfqpoint{4.058784in}{2.107300in}}{\pgfqpoint{4.064608in}{2.113124in}}%
\pgfpathcurveto{\pgfqpoint{4.070432in}{2.118948in}}{\pgfqpoint{4.073704in}{2.126848in}}{\pgfqpoint{4.073704in}{2.135084in}}%
\pgfpathcurveto{\pgfqpoint{4.073704in}{2.143321in}}{\pgfqpoint{4.070432in}{2.151221in}}{\pgfqpoint{4.064608in}{2.157045in}}%
\pgfpathcurveto{\pgfqpoint{4.058784in}{2.162869in}}{\pgfqpoint{4.050884in}{2.166141in}}{\pgfqpoint{4.042647in}{2.166141in}}%
\pgfpathcurveto{\pgfqpoint{4.034411in}{2.166141in}}{\pgfqpoint{4.026511in}{2.162869in}}{\pgfqpoint{4.020687in}{2.157045in}}%
\pgfpathcurveto{\pgfqpoint{4.014863in}{2.151221in}}{\pgfqpoint{4.011591in}{2.143321in}}{\pgfqpoint{4.011591in}{2.135084in}}%
\pgfpathcurveto{\pgfqpoint{4.011591in}{2.126848in}}{\pgfqpoint{4.014863in}{2.118948in}}{\pgfqpoint{4.020687in}{2.113124in}}%
\pgfpathcurveto{\pgfqpoint{4.026511in}{2.107300in}}{\pgfqpoint{4.034411in}{2.104028in}}{\pgfqpoint{4.042647in}{2.104028in}}%
\pgfpathclose%
\pgfusepath{stroke,fill}%
\end{pgfscope}%
\begin{pgfscope}%
\pgfpathrectangle{\pgfqpoint{3.793912in}{0.557870in}}{\pgfqpoint{2.446088in}{1.684734in}}%
\pgfusepath{clip}%
\pgfsetbuttcap%
\pgfsetroundjoin%
\definecolor{currentfill}{rgb}{0.298039,0.447059,0.690196}%
\pgfsetfillcolor{currentfill}%
\pgfsetlinewidth{1.003750pt}%
\definecolor{currentstroke}{rgb}{0.298039,0.447059,0.690196}%
\pgfsetstrokecolor{currentstroke}%
\pgfsetdash{}{0pt}%
\pgfpathmoveto{\pgfqpoint{3.905098in}{2.119498in}}%
\pgfpathcurveto{\pgfqpoint{3.913334in}{2.119498in}}{\pgfqpoint{3.921234in}{2.122771in}}{\pgfqpoint{3.927058in}{2.128595in}}%
\pgfpathcurveto{\pgfqpoint{3.932882in}{2.134419in}}{\pgfqpoint{3.936155in}{2.142319in}}{\pgfqpoint{3.936155in}{2.150555in}}%
\pgfpathcurveto{\pgfqpoint{3.936155in}{2.158791in}}{\pgfqpoint{3.932882in}{2.166691in}}{\pgfqpoint{3.927058in}{2.172515in}}%
\pgfpathcurveto{\pgfqpoint{3.921234in}{2.178339in}}{\pgfqpoint{3.913334in}{2.181611in}}{\pgfqpoint{3.905098in}{2.181611in}}%
\pgfpathcurveto{\pgfqpoint{3.896862in}{2.181611in}}{\pgfqpoint{3.888962in}{2.178339in}}{\pgfqpoint{3.883138in}{2.172515in}}%
\pgfpathcurveto{\pgfqpoint{3.877314in}{2.166691in}}{\pgfqpoint{3.874042in}{2.158791in}}{\pgfqpoint{3.874042in}{2.150555in}}%
\pgfpathcurveto{\pgfqpoint{3.874042in}{2.142319in}}{\pgfqpoint{3.877314in}{2.134419in}}{\pgfqpoint{3.883138in}{2.128595in}}%
\pgfpathcurveto{\pgfqpoint{3.888962in}{2.122771in}}{\pgfqpoint{3.896862in}{2.119498in}}{\pgfqpoint{3.905098in}{2.119498in}}%
\pgfpathclose%
\pgfusepath{stroke,fill}%
\end{pgfscope}%
\begin{pgfscope}%
\pgfpathrectangle{\pgfqpoint{3.793912in}{0.557870in}}{\pgfqpoint{2.446088in}{1.684734in}}%
\pgfusepath{clip}%
\pgfsetbuttcap%
\pgfsetroundjoin%
\definecolor{currentfill}{rgb}{0.298039,0.447059,0.690196}%
\pgfsetfillcolor{currentfill}%
\pgfsetlinewidth{1.003750pt}%
\definecolor{currentstroke}{rgb}{0.298039,0.447059,0.690196}%
\pgfsetstrokecolor{currentstroke}%
\pgfsetdash{}{0pt}%
\pgfpathmoveto{\pgfqpoint{4.546995in}{1.794619in}}%
\pgfpathcurveto{\pgfqpoint{4.555232in}{1.794619in}}{\pgfqpoint{4.563132in}{1.797891in}}{\pgfqpoint{4.568956in}{1.803715in}}%
\pgfpathcurveto{\pgfqpoint{4.574780in}{1.809539in}}{\pgfqpoint{4.578052in}{1.817439in}}{\pgfqpoint{4.578052in}{1.825675in}}%
\pgfpathcurveto{\pgfqpoint{4.578052in}{1.833911in}}{\pgfqpoint{4.574780in}{1.841811in}}{\pgfqpoint{4.568956in}{1.847635in}}%
\pgfpathcurveto{\pgfqpoint{4.563132in}{1.853459in}}{\pgfqpoint{4.555232in}{1.856732in}}{\pgfqpoint{4.546995in}{1.856732in}}%
\pgfpathcurveto{\pgfqpoint{4.538759in}{1.856732in}}{\pgfqpoint{4.530859in}{1.853459in}}{\pgfqpoint{4.525035in}{1.847635in}}%
\pgfpathcurveto{\pgfqpoint{4.519211in}{1.841811in}}{\pgfqpoint{4.515939in}{1.833911in}}{\pgfqpoint{4.515939in}{1.825675in}}%
\pgfpathcurveto{\pgfqpoint{4.515939in}{1.817439in}}{\pgfqpoint{4.519211in}{1.809539in}}{\pgfqpoint{4.525035in}{1.803715in}}%
\pgfpathcurveto{\pgfqpoint{4.530859in}{1.797891in}}{\pgfqpoint{4.538759in}{1.794619in}}{\pgfqpoint{4.546995in}{1.794619in}}%
\pgfpathclose%
\pgfusepath{stroke,fill}%
\end{pgfscope}%
\begin{pgfscope}%
\pgfpathrectangle{\pgfqpoint{3.793912in}{0.557870in}}{\pgfqpoint{2.446088in}{1.684734in}}%
\pgfusepath{clip}%
\pgfsetbuttcap%
\pgfsetroundjoin%
\definecolor{currentfill}{rgb}{0.298039,0.447059,0.690196}%
\pgfsetfillcolor{currentfill}%
\pgfsetlinewidth{1.003750pt}%
\definecolor{currentstroke}{rgb}{0.298039,0.447059,0.690196}%
\pgfsetstrokecolor{currentstroke}%
\pgfsetdash{}{0pt}%
\pgfpathmoveto{\pgfqpoint{3.905098in}{2.119498in}}%
\pgfpathcurveto{\pgfqpoint{3.913334in}{2.119498in}}{\pgfqpoint{3.921234in}{2.122771in}}{\pgfqpoint{3.927058in}{2.128595in}}%
\pgfpathcurveto{\pgfqpoint{3.932882in}{2.134419in}}{\pgfqpoint{3.936155in}{2.142319in}}{\pgfqpoint{3.936155in}{2.150555in}}%
\pgfpathcurveto{\pgfqpoint{3.936155in}{2.158791in}}{\pgfqpoint{3.932882in}{2.166691in}}{\pgfqpoint{3.927058in}{2.172515in}}%
\pgfpathcurveto{\pgfqpoint{3.921234in}{2.178339in}}{\pgfqpoint{3.913334in}{2.181611in}}{\pgfqpoint{3.905098in}{2.181611in}}%
\pgfpathcurveto{\pgfqpoint{3.896862in}{2.181611in}}{\pgfqpoint{3.888962in}{2.178339in}}{\pgfqpoint{3.883138in}{2.172515in}}%
\pgfpathcurveto{\pgfqpoint{3.877314in}{2.166691in}}{\pgfqpoint{3.874042in}{2.158791in}}{\pgfqpoint{3.874042in}{2.150555in}}%
\pgfpathcurveto{\pgfqpoint{3.874042in}{2.142319in}}{\pgfqpoint{3.877314in}{2.134419in}}{\pgfqpoint{3.883138in}{2.128595in}}%
\pgfpathcurveto{\pgfqpoint{3.888962in}{2.122771in}}{\pgfqpoint{3.896862in}{2.119498in}}{\pgfqpoint{3.905098in}{2.119498in}}%
\pgfpathclose%
\pgfusepath{stroke,fill}%
\end{pgfscope}%
\begin{pgfscope}%
\pgfpathrectangle{\pgfqpoint{3.793912in}{0.557870in}}{\pgfqpoint{2.446088in}{1.684734in}}%
\pgfusepath{clip}%
\pgfsetbuttcap%
\pgfsetroundjoin%
\definecolor{currentfill}{rgb}{0.298039,0.447059,0.690196}%
\pgfsetfillcolor{currentfill}%
\pgfsetlinewidth{1.003750pt}%
\definecolor{currentstroke}{rgb}{0.298039,0.447059,0.690196}%
\pgfsetstrokecolor{currentstroke}%
\pgfsetdash{}{0pt}%
\pgfpathmoveto{\pgfqpoint{4.432371in}{1.825560in}}%
\pgfpathcurveto{\pgfqpoint{4.440607in}{1.825560in}}{\pgfqpoint{4.448507in}{1.828832in}}{\pgfqpoint{4.454331in}{1.834656in}}%
\pgfpathcurveto{\pgfqpoint{4.460155in}{1.840480in}}{\pgfqpoint{4.463427in}{1.848380in}}{\pgfqpoint{4.463427in}{1.856616in}}%
\pgfpathcurveto{\pgfqpoint{4.463427in}{1.864852in}}{\pgfqpoint{4.460155in}{1.872752in}}{\pgfqpoint{4.454331in}{1.878576in}}%
\pgfpathcurveto{\pgfqpoint{4.448507in}{1.884400in}}{\pgfqpoint{4.440607in}{1.887673in}}{\pgfqpoint{4.432371in}{1.887673in}}%
\pgfpathcurveto{\pgfqpoint{4.424135in}{1.887673in}}{\pgfqpoint{4.416235in}{1.884400in}}{\pgfqpoint{4.410411in}{1.878576in}}%
\pgfpathcurveto{\pgfqpoint{4.404587in}{1.872752in}}{\pgfqpoint{4.401314in}{1.864852in}}{\pgfqpoint{4.401314in}{1.856616in}}%
\pgfpathcurveto{\pgfqpoint{4.401314in}{1.848380in}}{\pgfqpoint{4.404587in}{1.840480in}}{\pgfqpoint{4.410411in}{1.834656in}}%
\pgfpathcurveto{\pgfqpoint{4.416235in}{1.828832in}}{\pgfqpoint{4.424135in}{1.825560in}}{\pgfqpoint{4.432371in}{1.825560in}}%
\pgfpathclose%
\pgfusepath{stroke,fill}%
\end{pgfscope}%
\begin{pgfscope}%
\pgfpathrectangle{\pgfqpoint{3.793912in}{0.557870in}}{\pgfqpoint{2.446088in}{1.684734in}}%
\pgfusepath{clip}%
\pgfsetbuttcap%
\pgfsetroundjoin%
\definecolor{currentfill}{rgb}{0.298039,0.447059,0.690196}%
\pgfsetfillcolor{currentfill}%
\pgfsetlinewidth{1.003750pt}%
\definecolor{currentstroke}{rgb}{0.298039,0.447059,0.690196}%
\pgfsetstrokecolor{currentstroke}%
\pgfsetdash{}{0pt}%
\pgfpathmoveto{\pgfqpoint{3.905098in}{2.119498in}}%
\pgfpathcurveto{\pgfqpoint{3.913334in}{2.119498in}}{\pgfqpoint{3.921234in}{2.122771in}}{\pgfqpoint{3.927058in}{2.128595in}}%
\pgfpathcurveto{\pgfqpoint{3.932882in}{2.134419in}}{\pgfqpoint{3.936155in}{2.142319in}}{\pgfqpoint{3.936155in}{2.150555in}}%
\pgfpathcurveto{\pgfqpoint{3.936155in}{2.158791in}}{\pgfqpoint{3.932882in}{2.166691in}}{\pgfqpoint{3.927058in}{2.172515in}}%
\pgfpathcurveto{\pgfqpoint{3.921234in}{2.178339in}}{\pgfqpoint{3.913334in}{2.181611in}}{\pgfqpoint{3.905098in}{2.181611in}}%
\pgfpathcurveto{\pgfqpoint{3.896862in}{2.181611in}}{\pgfqpoint{3.888962in}{2.178339in}}{\pgfqpoint{3.883138in}{2.172515in}}%
\pgfpathcurveto{\pgfqpoint{3.877314in}{2.166691in}}{\pgfqpoint{3.874042in}{2.158791in}}{\pgfqpoint{3.874042in}{2.150555in}}%
\pgfpathcurveto{\pgfqpoint{3.874042in}{2.142319in}}{\pgfqpoint{3.877314in}{2.134419in}}{\pgfqpoint{3.883138in}{2.128595in}}%
\pgfpathcurveto{\pgfqpoint{3.888962in}{2.122771in}}{\pgfqpoint{3.896862in}{2.119498in}}{\pgfqpoint{3.905098in}{2.119498in}}%
\pgfpathclose%
\pgfusepath{stroke,fill}%
\end{pgfscope}%
\begin{pgfscope}%
\pgfpathrectangle{\pgfqpoint{3.793912in}{0.557870in}}{\pgfqpoint{2.446088in}{1.684734in}}%
\pgfusepath{clip}%
\pgfsetbuttcap%
\pgfsetroundjoin%
\definecolor{currentfill}{rgb}{0.298039,0.447059,0.690196}%
\pgfsetfillcolor{currentfill}%
\pgfsetlinewidth{1.003750pt}%
\definecolor{currentstroke}{rgb}{0.298039,0.447059,0.690196}%
\pgfsetstrokecolor{currentstroke}%
\pgfsetdash{}{0pt}%
\pgfpathmoveto{\pgfqpoint{3.905098in}{2.119498in}}%
\pgfpathcurveto{\pgfqpoint{3.913334in}{2.119498in}}{\pgfqpoint{3.921234in}{2.122771in}}{\pgfqpoint{3.927058in}{2.128595in}}%
\pgfpathcurveto{\pgfqpoint{3.932882in}{2.134419in}}{\pgfqpoint{3.936155in}{2.142319in}}{\pgfqpoint{3.936155in}{2.150555in}}%
\pgfpathcurveto{\pgfqpoint{3.936155in}{2.158791in}}{\pgfqpoint{3.932882in}{2.166691in}}{\pgfqpoint{3.927058in}{2.172515in}}%
\pgfpathcurveto{\pgfqpoint{3.921234in}{2.178339in}}{\pgfqpoint{3.913334in}{2.181611in}}{\pgfqpoint{3.905098in}{2.181611in}}%
\pgfpathcurveto{\pgfqpoint{3.896862in}{2.181611in}}{\pgfqpoint{3.888962in}{2.178339in}}{\pgfqpoint{3.883138in}{2.172515in}}%
\pgfpathcurveto{\pgfqpoint{3.877314in}{2.166691in}}{\pgfqpoint{3.874042in}{2.158791in}}{\pgfqpoint{3.874042in}{2.150555in}}%
\pgfpathcurveto{\pgfqpoint{3.874042in}{2.142319in}}{\pgfqpoint{3.877314in}{2.134419in}}{\pgfqpoint{3.883138in}{2.128595in}}%
\pgfpathcurveto{\pgfqpoint{3.888962in}{2.122771in}}{\pgfqpoint{3.896862in}{2.119498in}}{\pgfqpoint{3.905098in}{2.119498in}}%
\pgfpathclose%
\pgfusepath{stroke,fill}%
\end{pgfscope}%
\begin{pgfscope}%
\pgfpathrectangle{\pgfqpoint{3.793912in}{0.557870in}}{\pgfqpoint{2.446088in}{1.684734in}}%
\pgfusepath{clip}%
\pgfsetbuttcap%
\pgfsetroundjoin%
\definecolor{currentfill}{rgb}{0.298039,0.447059,0.690196}%
\pgfsetfillcolor{currentfill}%
\pgfsetlinewidth{1.003750pt}%
\definecolor{currentstroke}{rgb}{0.298039,0.447059,0.690196}%
\pgfsetstrokecolor{currentstroke}%
\pgfsetdash{}{0pt}%
\pgfpathmoveto{\pgfqpoint{3.905098in}{2.119498in}}%
\pgfpathcurveto{\pgfqpoint{3.913334in}{2.119498in}}{\pgfqpoint{3.921234in}{2.122771in}}{\pgfqpoint{3.927058in}{2.128595in}}%
\pgfpathcurveto{\pgfqpoint{3.932882in}{2.134419in}}{\pgfqpoint{3.936155in}{2.142319in}}{\pgfqpoint{3.936155in}{2.150555in}}%
\pgfpathcurveto{\pgfqpoint{3.936155in}{2.158791in}}{\pgfqpoint{3.932882in}{2.166691in}}{\pgfqpoint{3.927058in}{2.172515in}}%
\pgfpathcurveto{\pgfqpoint{3.921234in}{2.178339in}}{\pgfqpoint{3.913334in}{2.181611in}}{\pgfqpoint{3.905098in}{2.181611in}}%
\pgfpathcurveto{\pgfqpoint{3.896862in}{2.181611in}}{\pgfqpoint{3.888962in}{2.178339in}}{\pgfqpoint{3.883138in}{2.172515in}}%
\pgfpathcurveto{\pgfqpoint{3.877314in}{2.166691in}}{\pgfqpoint{3.874042in}{2.158791in}}{\pgfqpoint{3.874042in}{2.150555in}}%
\pgfpathcurveto{\pgfqpoint{3.874042in}{2.142319in}}{\pgfqpoint{3.877314in}{2.134419in}}{\pgfqpoint{3.883138in}{2.128595in}}%
\pgfpathcurveto{\pgfqpoint{3.888962in}{2.122771in}}{\pgfqpoint{3.896862in}{2.119498in}}{\pgfqpoint{3.905098in}{2.119498in}}%
\pgfpathclose%
\pgfusepath{stroke,fill}%
\end{pgfscope}%
\begin{pgfscope}%
\pgfpathrectangle{\pgfqpoint{3.793912in}{0.557870in}}{\pgfqpoint{2.446088in}{1.684734in}}%
\pgfusepath{clip}%
\pgfsetbuttcap%
\pgfsetroundjoin%
\definecolor{currentfill}{rgb}{0.298039,0.447059,0.690196}%
\pgfsetfillcolor{currentfill}%
\pgfsetlinewidth{1.003750pt}%
\definecolor{currentstroke}{rgb}{0.298039,0.447059,0.690196}%
\pgfsetstrokecolor{currentstroke}%
\pgfsetdash{}{0pt}%
\pgfpathmoveto{\pgfqpoint{3.905098in}{2.119498in}}%
\pgfpathcurveto{\pgfqpoint{3.913334in}{2.119498in}}{\pgfqpoint{3.921234in}{2.122771in}}{\pgfqpoint{3.927058in}{2.128595in}}%
\pgfpathcurveto{\pgfqpoint{3.932882in}{2.134419in}}{\pgfqpoint{3.936155in}{2.142319in}}{\pgfqpoint{3.936155in}{2.150555in}}%
\pgfpathcurveto{\pgfqpoint{3.936155in}{2.158791in}}{\pgfqpoint{3.932882in}{2.166691in}}{\pgfqpoint{3.927058in}{2.172515in}}%
\pgfpathcurveto{\pgfqpoint{3.921234in}{2.178339in}}{\pgfqpoint{3.913334in}{2.181611in}}{\pgfqpoint{3.905098in}{2.181611in}}%
\pgfpathcurveto{\pgfqpoint{3.896862in}{2.181611in}}{\pgfqpoint{3.888962in}{2.178339in}}{\pgfqpoint{3.883138in}{2.172515in}}%
\pgfpathcurveto{\pgfqpoint{3.877314in}{2.166691in}}{\pgfqpoint{3.874042in}{2.158791in}}{\pgfqpoint{3.874042in}{2.150555in}}%
\pgfpathcurveto{\pgfqpoint{3.874042in}{2.142319in}}{\pgfqpoint{3.877314in}{2.134419in}}{\pgfqpoint{3.883138in}{2.128595in}}%
\pgfpathcurveto{\pgfqpoint{3.888962in}{2.122771in}}{\pgfqpoint{3.896862in}{2.119498in}}{\pgfqpoint{3.905098in}{2.119498in}}%
\pgfpathclose%
\pgfusepath{stroke,fill}%
\end{pgfscope}%
\begin{pgfscope}%
\pgfpathrectangle{\pgfqpoint{3.793912in}{0.557870in}}{\pgfqpoint{2.446088in}{1.684734in}}%
\pgfusepath{clip}%
\pgfsetbuttcap%
\pgfsetroundjoin%
\definecolor{currentfill}{rgb}{0.298039,0.447059,0.690196}%
\pgfsetfillcolor{currentfill}%
\pgfsetlinewidth{1.003750pt}%
\definecolor{currentstroke}{rgb}{0.298039,0.447059,0.690196}%
\pgfsetstrokecolor{currentstroke}%
\pgfsetdash{}{0pt}%
\pgfpathmoveto{\pgfqpoint{3.905098in}{2.119498in}}%
\pgfpathcurveto{\pgfqpoint{3.913334in}{2.119498in}}{\pgfqpoint{3.921234in}{2.122771in}}{\pgfqpoint{3.927058in}{2.128595in}}%
\pgfpathcurveto{\pgfqpoint{3.932882in}{2.134419in}}{\pgfqpoint{3.936155in}{2.142319in}}{\pgfqpoint{3.936155in}{2.150555in}}%
\pgfpathcurveto{\pgfqpoint{3.936155in}{2.158791in}}{\pgfqpoint{3.932882in}{2.166691in}}{\pgfqpoint{3.927058in}{2.172515in}}%
\pgfpathcurveto{\pgfqpoint{3.921234in}{2.178339in}}{\pgfqpoint{3.913334in}{2.181611in}}{\pgfqpoint{3.905098in}{2.181611in}}%
\pgfpathcurveto{\pgfqpoint{3.896862in}{2.181611in}}{\pgfqpoint{3.888962in}{2.178339in}}{\pgfqpoint{3.883138in}{2.172515in}}%
\pgfpathcurveto{\pgfqpoint{3.877314in}{2.166691in}}{\pgfqpoint{3.874042in}{2.158791in}}{\pgfqpoint{3.874042in}{2.150555in}}%
\pgfpathcurveto{\pgfqpoint{3.874042in}{2.142319in}}{\pgfqpoint{3.877314in}{2.134419in}}{\pgfqpoint{3.883138in}{2.128595in}}%
\pgfpathcurveto{\pgfqpoint{3.888962in}{2.122771in}}{\pgfqpoint{3.896862in}{2.119498in}}{\pgfqpoint{3.905098in}{2.119498in}}%
\pgfpathclose%
\pgfusepath{stroke,fill}%
\end{pgfscope}%
\begin{pgfscope}%
\pgfpathrectangle{\pgfqpoint{3.793912in}{0.557870in}}{\pgfqpoint{2.446088in}{1.684734in}}%
\pgfusepath{clip}%
\pgfsetbuttcap%
\pgfsetroundjoin%
\definecolor{currentfill}{rgb}{0.298039,0.447059,0.690196}%
\pgfsetfillcolor{currentfill}%
\pgfsetlinewidth{1.003750pt}%
\definecolor{currentstroke}{rgb}{0.298039,0.447059,0.690196}%
\pgfsetstrokecolor{currentstroke}%
\pgfsetdash{}{0pt}%
\pgfpathmoveto{\pgfqpoint{3.905098in}{2.119498in}}%
\pgfpathcurveto{\pgfqpoint{3.913334in}{2.119498in}}{\pgfqpoint{3.921234in}{2.122771in}}{\pgfqpoint{3.927058in}{2.128595in}}%
\pgfpathcurveto{\pgfqpoint{3.932882in}{2.134419in}}{\pgfqpoint{3.936155in}{2.142319in}}{\pgfqpoint{3.936155in}{2.150555in}}%
\pgfpathcurveto{\pgfqpoint{3.936155in}{2.158791in}}{\pgfqpoint{3.932882in}{2.166691in}}{\pgfqpoint{3.927058in}{2.172515in}}%
\pgfpathcurveto{\pgfqpoint{3.921234in}{2.178339in}}{\pgfqpoint{3.913334in}{2.181611in}}{\pgfqpoint{3.905098in}{2.181611in}}%
\pgfpathcurveto{\pgfqpoint{3.896862in}{2.181611in}}{\pgfqpoint{3.888962in}{2.178339in}}{\pgfqpoint{3.883138in}{2.172515in}}%
\pgfpathcurveto{\pgfqpoint{3.877314in}{2.166691in}}{\pgfqpoint{3.874042in}{2.158791in}}{\pgfqpoint{3.874042in}{2.150555in}}%
\pgfpathcurveto{\pgfqpoint{3.874042in}{2.142319in}}{\pgfqpoint{3.877314in}{2.134419in}}{\pgfqpoint{3.883138in}{2.128595in}}%
\pgfpathcurveto{\pgfqpoint{3.888962in}{2.122771in}}{\pgfqpoint{3.896862in}{2.119498in}}{\pgfqpoint{3.905098in}{2.119498in}}%
\pgfpathclose%
\pgfusepath{stroke,fill}%
\end{pgfscope}%
\begin{pgfscope}%
\pgfpathrectangle{\pgfqpoint{3.793912in}{0.557870in}}{\pgfqpoint{2.446088in}{1.684734in}}%
\pgfusepath{clip}%
\pgfsetbuttcap%
\pgfsetroundjoin%
\definecolor{currentfill}{rgb}{0.298039,0.447059,0.690196}%
\pgfsetfillcolor{currentfill}%
\pgfsetlinewidth{1.003750pt}%
\definecolor{currentstroke}{rgb}{0.298039,0.447059,0.690196}%
\pgfsetstrokecolor{currentstroke}%
\pgfsetdash{}{0pt}%
\pgfpathmoveto{\pgfqpoint{5.188893in}{1.531621in}}%
\pgfpathcurveto{\pgfqpoint{5.197129in}{1.531621in}}{\pgfqpoint{5.205029in}{1.534893in}}{\pgfqpoint{5.210853in}{1.540717in}}%
\pgfpathcurveto{\pgfqpoint{5.216677in}{1.546541in}}{\pgfqpoint{5.219949in}{1.554441in}}{\pgfqpoint{5.219949in}{1.562677in}}%
\pgfpathcurveto{\pgfqpoint{5.219949in}{1.570913in}}{\pgfqpoint{5.216677in}{1.578813in}}{\pgfqpoint{5.210853in}{1.584637in}}%
\pgfpathcurveto{\pgfqpoint{5.205029in}{1.590461in}}{\pgfqpoint{5.197129in}{1.593734in}}{\pgfqpoint{5.188893in}{1.593734in}}%
\pgfpathcurveto{\pgfqpoint{5.180657in}{1.593734in}}{\pgfqpoint{5.172757in}{1.590461in}}{\pgfqpoint{5.166933in}{1.584637in}}%
\pgfpathcurveto{\pgfqpoint{5.161109in}{1.578813in}}{\pgfqpoint{5.157836in}{1.570913in}}{\pgfqpoint{5.157836in}{1.562677in}}%
\pgfpathcurveto{\pgfqpoint{5.157836in}{1.554441in}}{\pgfqpoint{5.161109in}{1.546541in}}{\pgfqpoint{5.166933in}{1.540717in}}%
\pgfpathcurveto{\pgfqpoint{5.172757in}{1.534893in}}{\pgfqpoint{5.180657in}{1.531621in}}{\pgfqpoint{5.188893in}{1.531621in}}%
\pgfpathclose%
\pgfusepath{stroke,fill}%
\end{pgfscope}%
\begin{pgfscope}%
\pgfpathrectangle{\pgfqpoint{3.793912in}{0.557870in}}{\pgfqpoint{2.446088in}{1.684734in}}%
\pgfusepath{clip}%
\pgfsetbuttcap%
\pgfsetroundjoin%
\definecolor{currentfill}{rgb}{0.298039,0.447059,0.690196}%
\pgfsetfillcolor{currentfill}%
\pgfsetlinewidth{1.003750pt}%
\definecolor{currentstroke}{rgb}{0.298039,0.447059,0.690196}%
\pgfsetstrokecolor{currentstroke}%
\pgfsetdash{}{0pt}%
\pgfpathmoveto{\pgfqpoint{3.905098in}{2.119498in}}%
\pgfpathcurveto{\pgfqpoint{3.913334in}{2.119498in}}{\pgfqpoint{3.921234in}{2.122771in}}{\pgfqpoint{3.927058in}{2.128595in}}%
\pgfpathcurveto{\pgfqpoint{3.932882in}{2.134419in}}{\pgfqpoint{3.936155in}{2.142319in}}{\pgfqpoint{3.936155in}{2.150555in}}%
\pgfpathcurveto{\pgfqpoint{3.936155in}{2.158791in}}{\pgfqpoint{3.932882in}{2.166691in}}{\pgfqpoint{3.927058in}{2.172515in}}%
\pgfpathcurveto{\pgfqpoint{3.921234in}{2.178339in}}{\pgfqpoint{3.913334in}{2.181611in}}{\pgfqpoint{3.905098in}{2.181611in}}%
\pgfpathcurveto{\pgfqpoint{3.896862in}{2.181611in}}{\pgfqpoint{3.888962in}{2.178339in}}{\pgfqpoint{3.883138in}{2.172515in}}%
\pgfpathcurveto{\pgfqpoint{3.877314in}{2.166691in}}{\pgfqpoint{3.874042in}{2.158791in}}{\pgfqpoint{3.874042in}{2.150555in}}%
\pgfpathcurveto{\pgfqpoint{3.874042in}{2.142319in}}{\pgfqpoint{3.877314in}{2.134419in}}{\pgfqpoint{3.883138in}{2.128595in}}%
\pgfpathcurveto{\pgfqpoint{3.888962in}{2.122771in}}{\pgfqpoint{3.896862in}{2.119498in}}{\pgfqpoint{3.905098in}{2.119498in}}%
\pgfpathclose%
\pgfusepath{stroke,fill}%
\end{pgfscope}%
\begin{pgfscope}%
\pgfpathrectangle{\pgfqpoint{3.793912in}{0.557870in}}{\pgfqpoint{2.446088in}{1.684734in}}%
\pgfusepath{clip}%
\pgfsetbuttcap%
\pgfsetroundjoin%
\definecolor{currentfill}{rgb}{0.298039,0.447059,0.690196}%
\pgfsetfillcolor{currentfill}%
\pgfsetlinewidth{1.003750pt}%
\definecolor{currentstroke}{rgb}{0.298039,0.447059,0.690196}%
\pgfsetstrokecolor{currentstroke}%
\pgfsetdash{}{0pt}%
\pgfpathmoveto{\pgfqpoint{3.905098in}{2.119498in}}%
\pgfpathcurveto{\pgfqpoint{3.913334in}{2.119498in}}{\pgfqpoint{3.921234in}{2.122771in}}{\pgfqpoint{3.927058in}{2.128595in}}%
\pgfpathcurveto{\pgfqpoint{3.932882in}{2.134419in}}{\pgfqpoint{3.936155in}{2.142319in}}{\pgfqpoint{3.936155in}{2.150555in}}%
\pgfpathcurveto{\pgfqpoint{3.936155in}{2.158791in}}{\pgfqpoint{3.932882in}{2.166691in}}{\pgfqpoint{3.927058in}{2.172515in}}%
\pgfpathcurveto{\pgfqpoint{3.921234in}{2.178339in}}{\pgfqpoint{3.913334in}{2.181611in}}{\pgfqpoint{3.905098in}{2.181611in}}%
\pgfpathcurveto{\pgfqpoint{3.896862in}{2.181611in}}{\pgfqpoint{3.888962in}{2.178339in}}{\pgfqpoint{3.883138in}{2.172515in}}%
\pgfpathcurveto{\pgfqpoint{3.877314in}{2.166691in}}{\pgfqpoint{3.874042in}{2.158791in}}{\pgfqpoint{3.874042in}{2.150555in}}%
\pgfpathcurveto{\pgfqpoint{3.874042in}{2.142319in}}{\pgfqpoint{3.877314in}{2.134419in}}{\pgfqpoint{3.883138in}{2.128595in}}%
\pgfpathcurveto{\pgfqpoint{3.888962in}{2.122771in}}{\pgfqpoint{3.896862in}{2.119498in}}{\pgfqpoint{3.905098in}{2.119498in}}%
\pgfpathclose%
\pgfusepath{stroke,fill}%
\end{pgfscope}%
\begin{pgfscope}%
\pgfpathrectangle{\pgfqpoint{3.793912in}{0.557870in}}{\pgfqpoint{2.446088in}{1.684734in}}%
\pgfusepath{clip}%
\pgfsetbuttcap%
\pgfsetroundjoin%
\definecolor{currentfill}{rgb}{0.298039,0.447059,0.690196}%
\pgfsetfillcolor{currentfill}%
\pgfsetlinewidth{1.003750pt}%
\definecolor{currentstroke}{rgb}{0.298039,0.447059,0.690196}%
\pgfsetstrokecolor{currentstroke}%
\pgfsetdash{}{0pt}%
\pgfpathmoveto{\pgfqpoint{3.905098in}{2.119498in}}%
\pgfpathcurveto{\pgfqpoint{3.913334in}{2.119498in}}{\pgfqpoint{3.921234in}{2.122771in}}{\pgfqpoint{3.927058in}{2.128595in}}%
\pgfpathcurveto{\pgfqpoint{3.932882in}{2.134419in}}{\pgfqpoint{3.936155in}{2.142319in}}{\pgfqpoint{3.936155in}{2.150555in}}%
\pgfpathcurveto{\pgfqpoint{3.936155in}{2.158791in}}{\pgfqpoint{3.932882in}{2.166691in}}{\pgfqpoint{3.927058in}{2.172515in}}%
\pgfpathcurveto{\pgfqpoint{3.921234in}{2.178339in}}{\pgfqpoint{3.913334in}{2.181611in}}{\pgfqpoint{3.905098in}{2.181611in}}%
\pgfpathcurveto{\pgfqpoint{3.896862in}{2.181611in}}{\pgfqpoint{3.888962in}{2.178339in}}{\pgfqpoint{3.883138in}{2.172515in}}%
\pgfpathcurveto{\pgfqpoint{3.877314in}{2.166691in}}{\pgfqpoint{3.874042in}{2.158791in}}{\pgfqpoint{3.874042in}{2.150555in}}%
\pgfpathcurveto{\pgfqpoint{3.874042in}{2.142319in}}{\pgfqpoint{3.877314in}{2.134419in}}{\pgfqpoint{3.883138in}{2.128595in}}%
\pgfpathcurveto{\pgfqpoint{3.888962in}{2.122771in}}{\pgfqpoint{3.896862in}{2.119498in}}{\pgfqpoint{3.905098in}{2.119498in}}%
\pgfpathclose%
\pgfusepath{stroke,fill}%
\end{pgfscope}%
\begin{pgfscope}%
\pgfpathrectangle{\pgfqpoint{3.793912in}{0.557870in}}{\pgfqpoint{2.446088in}{1.684734in}}%
\pgfusepath{clip}%
\pgfsetbuttcap%
\pgfsetroundjoin%
\definecolor{currentfill}{rgb}{0.298039,0.447059,0.690196}%
\pgfsetfillcolor{currentfill}%
\pgfsetlinewidth{1.003750pt}%
\definecolor{currentstroke}{rgb}{0.298039,0.447059,0.690196}%
\pgfsetstrokecolor{currentstroke}%
\pgfsetdash{}{0pt}%
\pgfpathmoveto{\pgfqpoint{3.905098in}{2.119498in}}%
\pgfpathcurveto{\pgfqpoint{3.913334in}{2.119498in}}{\pgfqpoint{3.921234in}{2.122771in}}{\pgfqpoint{3.927058in}{2.128595in}}%
\pgfpathcurveto{\pgfqpoint{3.932882in}{2.134419in}}{\pgfqpoint{3.936155in}{2.142319in}}{\pgfqpoint{3.936155in}{2.150555in}}%
\pgfpathcurveto{\pgfqpoint{3.936155in}{2.158791in}}{\pgfqpoint{3.932882in}{2.166691in}}{\pgfqpoint{3.927058in}{2.172515in}}%
\pgfpathcurveto{\pgfqpoint{3.921234in}{2.178339in}}{\pgfqpoint{3.913334in}{2.181611in}}{\pgfqpoint{3.905098in}{2.181611in}}%
\pgfpathcurveto{\pgfqpoint{3.896862in}{2.181611in}}{\pgfqpoint{3.888962in}{2.178339in}}{\pgfqpoint{3.883138in}{2.172515in}}%
\pgfpathcurveto{\pgfqpoint{3.877314in}{2.166691in}}{\pgfqpoint{3.874042in}{2.158791in}}{\pgfqpoint{3.874042in}{2.150555in}}%
\pgfpathcurveto{\pgfqpoint{3.874042in}{2.142319in}}{\pgfqpoint{3.877314in}{2.134419in}}{\pgfqpoint{3.883138in}{2.128595in}}%
\pgfpathcurveto{\pgfqpoint{3.888962in}{2.122771in}}{\pgfqpoint{3.896862in}{2.119498in}}{\pgfqpoint{3.905098in}{2.119498in}}%
\pgfpathclose%
\pgfusepath{stroke,fill}%
\end{pgfscope}%
\begin{pgfscope}%
\pgfpathrectangle{\pgfqpoint{3.793912in}{0.557870in}}{\pgfqpoint{2.446088in}{1.684734in}}%
\pgfusepath{clip}%
\pgfsetbuttcap%
\pgfsetroundjoin%
\definecolor{currentfill}{rgb}{0.298039,0.447059,0.690196}%
\pgfsetfillcolor{currentfill}%
\pgfsetlinewidth{1.003750pt}%
\definecolor{currentstroke}{rgb}{0.298039,0.447059,0.690196}%
\pgfsetstrokecolor{currentstroke}%
\pgfsetdash{}{0pt}%
\pgfpathmoveto{\pgfqpoint{5.532767in}{1.500680in}}%
\pgfpathcurveto{\pgfqpoint{5.541003in}{1.500680in}}{\pgfqpoint{5.548903in}{1.503952in}}{\pgfqpoint{5.554727in}{1.509776in}}%
\pgfpathcurveto{\pgfqpoint{5.560551in}{1.515600in}}{\pgfqpoint{5.563823in}{1.523500in}}{\pgfqpoint{5.563823in}{1.531736in}}%
\pgfpathcurveto{\pgfqpoint{5.563823in}{1.539972in}}{\pgfqpoint{5.560551in}{1.547873in}}{\pgfqpoint{5.554727in}{1.553696in}}%
\pgfpathcurveto{\pgfqpoint{5.548903in}{1.559520in}}{\pgfqpoint{5.541003in}{1.562793in}}{\pgfqpoint{5.532767in}{1.562793in}}%
\pgfpathcurveto{\pgfqpoint{5.524530in}{1.562793in}}{\pgfqpoint{5.516630in}{1.559520in}}{\pgfqpoint{5.510806in}{1.553696in}}%
\pgfpathcurveto{\pgfqpoint{5.504982in}{1.547873in}}{\pgfqpoint{5.501710in}{1.539972in}}{\pgfqpoint{5.501710in}{1.531736in}}%
\pgfpathcurveto{\pgfqpoint{5.501710in}{1.523500in}}{\pgfqpoint{5.504982in}{1.515600in}}{\pgfqpoint{5.510806in}{1.509776in}}%
\pgfpathcurveto{\pgfqpoint{5.516630in}{1.503952in}}{\pgfqpoint{5.524530in}{1.500680in}}{\pgfqpoint{5.532767in}{1.500680in}}%
\pgfpathclose%
\pgfusepath{stroke,fill}%
\end{pgfscope}%
\begin{pgfscope}%
\pgfpathrectangle{\pgfqpoint{3.793912in}{0.557870in}}{\pgfqpoint{2.446088in}{1.684734in}}%
\pgfusepath{clip}%
\pgfsetbuttcap%
\pgfsetroundjoin%
\definecolor{currentfill}{rgb}{0.298039,0.447059,0.690196}%
\pgfsetfillcolor{currentfill}%
\pgfsetlinewidth{1.003750pt}%
\definecolor{currentstroke}{rgb}{0.298039,0.447059,0.690196}%
\pgfsetstrokecolor{currentstroke}%
\pgfsetdash{}{0pt}%
\pgfpathmoveto{\pgfqpoint{5.647391in}{1.469739in}}%
\pgfpathcurveto{\pgfqpoint{5.655627in}{1.469739in}}{\pgfqpoint{5.663527in}{1.473011in}}{\pgfqpoint{5.669351in}{1.478835in}}%
\pgfpathcurveto{\pgfqpoint{5.675175in}{1.484659in}}{\pgfqpoint{5.678448in}{1.492559in}}{\pgfqpoint{5.678448in}{1.500795in}}%
\pgfpathcurveto{\pgfqpoint{5.678448in}{1.509032in}}{\pgfqpoint{5.675175in}{1.516932in}}{\pgfqpoint{5.669351in}{1.522756in}}%
\pgfpathcurveto{\pgfqpoint{5.663527in}{1.528579in}}{\pgfqpoint{5.655627in}{1.531852in}}{\pgfqpoint{5.647391in}{1.531852in}}%
\pgfpathcurveto{\pgfqpoint{5.639155in}{1.531852in}}{\pgfqpoint{5.631255in}{1.528579in}}{\pgfqpoint{5.625431in}{1.522756in}}%
\pgfpathcurveto{\pgfqpoint{5.619607in}{1.516932in}}{\pgfqpoint{5.616335in}{1.509032in}}{\pgfqpoint{5.616335in}{1.500795in}}%
\pgfpathcurveto{\pgfqpoint{5.616335in}{1.492559in}}{\pgfqpoint{5.619607in}{1.484659in}}{\pgfqpoint{5.625431in}{1.478835in}}%
\pgfpathcurveto{\pgfqpoint{5.631255in}{1.473011in}}{\pgfqpoint{5.639155in}{1.469739in}}{\pgfqpoint{5.647391in}{1.469739in}}%
\pgfpathclose%
\pgfusepath{stroke,fill}%
\end{pgfscope}%
\begin{pgfscope}%
\pgfpathrectangle{\pgfqpoint{3.793912in}{0.557870in}}{\pgfqpoint{2.446088in}{1.684734in}}%
\pgfusepath{clip}%
\pgfsetbuttcap%
\pgfsetroundjoin%
\definecolor{currentfill}{rgb}{0.298039,0.447059,0.690196}%
\pgfsetfillcolor{currentfill}%
\pgfsetlinewidth{1.003750pt}%
\definecolor{currentstroke}{rgb}{0.298039,0.447059,0.690196}%
\pgfsetstrokecolor{currentstroke}%
\pgfsetdash{}{0pt}%
\pgfpathmoveto{\pgfqpoint{5.670316in}{1.407857in}}%
\pgfpathcurveto{\pgfqpoint{5.678552in}{1.407857in}}{\pgfqpoint{5.686452in}{1.411129in}}{\pgfqpoint{5.692276in}{1.416953in}}%
\pgfpathcurveto{\pgfqpoint{5.698100in}{1.422777in}}{\pgfqpoint{5.701373in}{1.430677in}}{\pgfqpoint{5.701373in}{1.438913in}}%
\pgfpathcurveto{\pgfqpoint{5.701373in}{1.447150in}}{\pgfqpoint{5.698100in}{1.455050in}}{\pgfqpoint{5.692276in}{1.460874in}}%
\pgfpathcurveto{\pgfqpoint{5.686452in}{1.466698in}}{\pgfqpoint{5.678552in}{1.469970in}}{\pgfqpoint{5.670316in}{1.469970in}}%
\pgfpathcurveto{\pgfqpoint{5.662080in}{1.469970in}}{\pgfqpoint{5.654180in}{1.466698in}}{\pgfqpoint{5.648356in}{1.460874in}}%
\pgfpathcurveto{\pgfqpoint{5.642532in}{1.455050in}}{\pgfqpoint{5.639260in}{1.447150in}}{\pgfqpoint{5.639260in}{1.438913in}}%
\pgfpathcurveto{\pgfqpoint{5.639260in}{1.430677in}}{\pgfqpoint{5.642532in}{1.422777in}}{\pgfqpoint{5.648356in}{1.416953in}}%
\pgfpathcurveto{\pgfqpoint{5.654180in}{1.411129in}}{\pgfqpoint{5.662080in}{1.407857in}}{\pgfqpoint{5.670316in}{1.407857in}}%
\pgfpathclose%
\pgfusepath{stroke,fill}%
\end{pgfscope}%
\begin{pgfscope}%
\pgfpathrectangle{\pgfqpoint{3.793912in}{0.557870in}}{\pgfqpoint{2.446088in}{1.684734in}}%
\pgfusepath{clip}%
\pgfsetbuttcap%
\pgfsetroundjoin%
\definecolor{currentfill}{rgb}{0.298039,0.447059,0.690196}%
\pgfsetfillcolor{currentfill}%
\pgfsetlinewidth{1.003750pt}%
\definecolor{currentstroke}{rgb}{0.298039,0.447059,0.690196}%
\pgfsetstrokecolor{currentstroke}%
\pgfsetdash{}{0pt}%
\pgfpathmoveto{\pgfqpoint{3.928023in}{2.119498in}}%
\pgfpathcurveto{\pgfqpoint{3.936259in}{2.119498in}}{\pgfqpoint{3.944159in}{2.122771in}}{\pgfqpoint{3.949983in}{2.128595in}}%
\pgfpathcurveto{\pgfqpoint{3.955807in}{2.134419in}}{\pgfqpoint{3.959079in}{2.142319in}}{\pgfqpoint{3.959079in}{2.150555in}}%
\pgfpathcurveto{\pgfqpoint{3.959079in}{2.158791in}}{\pgfqpoint{3.955807in}{2.166691in}}{\pgfqpoint{3.949983in}{2.172515in}}%
\pgfpathcurveto{\pgfqpoint{3.944159in}{2.178339in}}{\pgfqpoint{3.936259in}{2.181611in}}{\pgfqpoint{3.928023in}{2.181611in}}%
\pgfpathcurveto{\pgfqpoint{3.919787in}{2.181611in}}{\pgfqpoint{3.911887in}{2.178339in}}{\pgfqpoint{3.906063in}{2.172515in}}%
\pgfpathcurveto{\pgfqpoint{3.900239in}{2.166691in}}{\pgfqpoint{3.896966in}{2.158791in}}{\pgfqpoint{3.896966in}{2.150555in}}%
\pgfpathcurveto{\pgfqpoint{3.896966in}{2.142319in}}{\pgfqpoint{3.900239in}{2.134419in}}{\pgfqpoint{3.906063in}{2.128595in}}%
\pgfpathcurveto{\pgfqpoint{3.911887in}{2.122771in}}{\pgfqpoint{3.919787in}{2.119498in}}{\pgfqpoint{3.928023in}{2.119498in}}%
\pgfpathclose%
\pgfusepath{stroke,fill}%
\end{pgfscope}%
\begin{pgfscope}%
\pgfpathrectangle{\pgfqpoint{3.793912in}{0.557870in}}{\pgfqpoint{2.446088in}{1.684734in}}%
\pgfusepath{clip}%
\pgfsetbuttcap%
\pgfsetroundjoin%
\definecolor{currentfill}{rgb}{0.298039,0.447059,0.690196}%
\pgfsetfillcolor{currentfill}%
\pgfsetlinewidth{1.003750pt}%
\definecolor{currentstroke}{rgb}{0.298039,0.447059,0.690196}%
\pgfsetstrokecolor{currentstroke}%
\pgfsetdash{}{0pt}%
\pgfpathmoveto{\pgfqpoint{3.905098in}{2.119498in}}%
\pgfpathcurveto{\pgfqpoint{3.913334in}{2.119498in}}{\pgfqpoint{3.921234in}{2.122771in}}{\pgfqpoint{3.927058in}{2.128595in}}%
\pgfpathcurveto{\pgfqpoint{3.932882in}{2.134419in}}{\pgfqpoint{3.936155in}{2.142319in}}{\pgfqpoint{3.936155in}{2.150555in}}%
\pgfpathcurveto{\pgfqpoint{3.936155in}{2.158791in}}{\pgfqpoint{3.932882in}{2.166691in}}{\pgfqpoint{3.927058in}{2.172515in}}%
\pgfpathcurveto{\pgfqpoint{3.921234in}{2.178339in}}{\pgfqpoint{3.913334in}{2.181611in}}{\pgfqpoint{3.905098in}{2.181611in}}%
\pgfpathcurveto{\pgfqpoint{3.896862in}{2.181611in}}{\pgfqpoint{3.888962in}{2.178339in}}{\pgfqpoint{3.883138in}{2.172515in}}%
\pgfpathcurveto{\pgfqpoint{3.877314in}{2.166691in}}{\pgfqpoint{3.874042in}{2.158791in}}{\pgfqpoint{3.874042in}{2.150555in}}%
\pgfpathcurveto{\pgfqpoint{3.874042in}{2.142319in}}{\pgfqpoint{3.877314in}{2.134419in}}{\pgfqpoint{3.883138in}{2.128595in}}%
\pgfpathcurveto{\pgfqpoint{3.888962in}{2.122771in}}{\pgfqpoint{3.896862in}{2.119498in}}{\pgfqpoint{3.905098in}{2.119498in}}%
\pgfpathclose%
\pgfusepath{stroke,fill}%
\end{pgfscope}%
\begin{pgfscope}%
\pgfpathrectangle{\pgfqpoint{3.793912in}{0.557870in}}{\pgfqpoint{2.446088in}{1.684734in}}%
\pgfusepath{clip}%
\pgfsetbuttcap%
\pgfsetroundjoin%
\definecolor{currentfill}{rgb}{0.298039,0.447059,0.690196}%
\pgfsetfillcolor{currentfill}%
\pgfsetlinewidth{1.003750pt}%
\definecolor{currentstroke}{rgb}{0.298039,0.447059,0.690196}%
\pgfsetstrokecolor{currentstroke}%
\pgfsetdash{}{0pt}%
\pgfpathmoveto{\pgfqpoint{5.509842in}{1.639914in}}%
\pgfpathcurveto{\pgfqpoint{5.518078in}{1.639914in}}{\pgfqpoint{5.525978in}{1.643186in}}{\pgfqpoint{5.531802in}{1.649010in}}%
\pgfpathcurveto{\pgfqpoint{5.537626in}{1.654834in}}{\pgfqpoint{5.540898in}{1.662734in}}{\pgfqpoint{5.540898in}{1.670970in}}%
\pgfpathcurveto{\pgfqpoint{5.540898in}{1.679207in}}{\pgfqpoint{5.537626in}{1.687107in}}{\pgfqpoint{5.531802in}{1.692931in}}%
\pgfpathcurveto{\pgfqpoint{5.525978in}{1.698755in}}{\pgfqpoint{5.518078in}{1.702027in}}{\pgfqpoint{5.509842in}{1.702027in}}%
\pgfpathcurveto{\pgfqpoint{5.501605in}{1.702027in}}{\pgfqpoint{5.493705in}{1.698755in}}{\pgfqpoint{5.487881in}{1.692931in}}%
\pgfpathcurveto{\pgfqpoint{5.482057in}{1.687107in}}{\pgfqpoint{5.478785in}{1.679207in}}{\pgfqpoint{5.478785in}{1.670970in}}%
\pgfpathcurveto{\pgfqpoint{5.478785in}{1.662734in}}{\pgfqpoint{5.482057in}{1.654834in}}{\pgfqpoint{5.487881in}{1.649010in}}%
\pgfpathcurveto{\pgfqpoint{5.493705in}{1.643186in}}{\pgfqpoint{5.501605in}{1.639914in}}{\pgfqpoint{5.509842in}{1.639914in}}%
\pgfpathclose%
\pgfusepath{stroke,fill}%
\end{pgfscope}%
\begin{pgfscope}%
\pgfpathrectangle{\pgfqpoint{3.793912in}{0.557870in}}{\pgfqpoint{2.446088in}{1.684734in}}%
\pgfusepath{clip}%
\pgfsetbuttcap%
\pgfsetroundjoin%
\definecolor{currentfill}{rgb}{0.298039,0.447059,0.690196}%
\pgfsetfillcolor{currentfill}%
\pgfsetlinewidth{1.003750pt}%
\definecolor{currentstroke}{rgb}{0.298039,0.447059,0.690196}%
\pgfsetstrokecolor{currentstroke}%
\pgfsetdash{}{0pt}%
\pgfpathmoveto{\pgfqpoint{3.905098in}{2.119498in}}%
\pgfpathcurveto{\pgfqpoint{3.913334in}{2.119498in}}{\pgfqpoint{3.921234in}{2.122771in}}{\pgfqpoint{3.927058in}{2.128595in}}%
\pgfpathcurveto{\pgfqpoint{3.932882in}{2.134419in}}{\pgfqpoint{3.936155in}{2.142319in}}{\pgfqpoint{3.936155in}{2.150555in}}%
\pgfpathcurveto{\pgfqpoint{3.936155in}{2.158791in}}{\pgfqpoint{3.932882in}{2.166691in}}{\pgfqpoint{3.927058in}{2.172515in}}%
\pgfpathcurveto{\pgfqpoint{3.921234in}{2.178339in}}{\pgfqpoint{3.913334in}{2.181611in}}{\pgfqpoint{3.905098in}{2.181611in}}%
\pgfpathcurveto{\pgfqpoint{3.896862in}{2.181611in}}{\pgfqpoint{3.888962in}{2.178339in}}{\pgfqpoint{3.883138in}{2.172515in}}%
\pgfpathcurveto{\pgfqpoint{3.877314in}{2.166691in}}{\pgfqpoint{3.874042in}{2.158791in}}{\pgfqpoint{3.874042in}{2.150555in}}%
\pgfpathcurveto{\pgfqpoint{3.874042in}{2.142319in}}{\pgfqpoint{3.877314in}{2.134419in}}{\pgfqpoint{3.883138in}{2.128595in}}%
\pgfpathcurveto{\pgfqpoint{3.888962in}{2.122771in}}{\pgfqpoint{3.896862in}{2.119498in}}{\pgfqpoint{3.905098in}{2.119498in}}%
\pgfpathclose%
\pgfusepath{stroke,fill}%
\end{pgfscope}%
\begin{pgfscope}%
\pgfpathrectangle{\pgfqpoint{3.793912in}{0.557870in}}{\pgfqpoint{2.446088in}{1.684734in}}%
\pgfusepath{clip}%
\pgfsetbuttcap%
\pgfsetroundjoin%
\definecolor{currentfill}{rgb}{0.298039,0.447059,0.690196}%
\pgfsetfillcolor{currentfill}%
\pgfsetlinewidth{1.003750pt}%
\definecolor{currentstroke}{rgb}{0.298039,0.447059,0.690196}%
\pgfsetstrokecolor{currentstroke}%
\pgfsetdash{}{0pt}%
\pgfpathmoveto{\pgfqpoint{5.716166in}{1.299564in}}%
\pgfpathcurveto{\pgfqpoint{5.724402in}{1.299564in}}{\pgfqpoint{5.732302in}{1.302836in}}{\pgfqpoint{5.738126in}{1.308660in}}%
\pgfpathcurveto{\pgfqpoint{5.743950in}{1.314484in}}{\pgfqpoint{5.747222in}{1.322384in}}{\pgfqpoint{5.747222in}{1.330620in}}%
\pgfpathcurveto{\pgfqpoint{5.747222in}{1.338856in}}{\pgfqpoint{5.743950in}{1.346756in}}{\pgfqpoint{5.738126in}{1.352580in}}%
\pgfpathcurveto{\pgfqpoint{5.732302in}{1.358404in}}{\pgfqpoint{5.724402in}{1.361677in}}{\pgfqpoint{5.716166in}{1.361677in}}%
\pgfpathcurveto{\pgfqpoint{5.707930in}{1.361677in}}{\pgfqpoint{5.700029in}{1.358404in}}{\pgfqpoint{5.694206in}{1.352580in}}%
\pgfpathcurveto{\pgfqpoint{5.688382in}{1.346756in}}{\pgfqpoint{5.685109in}{1.338856in}}{\pgfqpoint{5.685109in}{1.330620in}}%
\pgfpathcurveto{\pgfqpoint{5.685109in}{1.322384in}}{\pgfqpoint{5.688382in}{1.314484in}}{\pgfqpoint{5.694206in}{1.308660in}}%
\pgfpathcurveto{\pgfqpoint{5.700029in}{1.302836in}}{\pgfqpoint{5.707930in}{1.299564in}}{\pgfqpoint{5.716166in}{1.299564in}}%
\pgfpathclose%
\pgfusepath{stroke,fill}%
\end{pgfscope}%
\begin{pgfscope}%
\pgfpathrectangle{\pgfqpoint{3.793912in}{0.557870in}}{\pgfqpoint{2.446088in}{1.684734in}}%
\pgfusepath{clip}%
\pgfsetbuttcap%
\pgfsetroundjoin%
\definecolor{currentfill}{rgb}{0.298039,0.447059,0.690196}%
\pgfsetfillcolor{currentfill}%
\pgfsetlinewidth{1.003750pt}%
\definecolor{currentstroke}{rgb}{0.298039,0.447059,0.690196}%
\pgfsetstrokecolor{currentstroke}%
\pgfsetdash{}{0pt}%
\pgfpathmoveto{\pgfqpoint{3.905098in}{2.119498in}}%
\pgfpathcurveto{\pgfqpoint{3.913334in}{2.119498in}}{\pgfqpoint{3.921234in}{2.122771in}}{\pgfqpoint{3.927058in}{2.128595in}}%
\pgfpathcurveto{\pgfqpoint{3.932882in}{2.134419in}}{\pgfqpoint{3.936155in}{2.142319in}}{\pgfqpoint{3.936155in}{2.150555in}}%
\pgfpathcurveto{\pgfqpoint{3.936155in}{2.158791in}}{\pgfqpoint{3.932882in}{2.166691in}}{\pgfqpoint{3.927058in}{2.172515in}}%
\pgfpathcurveto{\pgfqpoint{3.921234in}{2.178339in}}{\pgfqpoint{3.913334in}{2.181611in}}{\pgfqpoint{3.905098in}{2.181611in}}%
\pgfpathcurveto{\pgfqpoint{3.896862in}{2.181611in}}{\pgfqpoint{3.888962in}{2.178339in}}{\pgfqpoint{3.883138in}{2.172515in}}%
\pgfpathcurveto{\pgfqpoint{3.877314in}{2.166691in}}{\pgfqpoint{3.874042in}{2.158791in}}{\pgfqpoint{3.874042in}{2.150555in}}%
\pgfpathcurveto{\pgfqpoint{3.874042in}{2.142319in}}{\pgfqpoint{3.877314in}{2.134419in}}{\pgfqpoint{3.883138in}{2.128595in}}%
\pgfpathcurveto{\pgfqpoint{3.888962in}{2.122771in}}{\pgfqpoint{3.896862in}{2.119498in}}{\pgfqpoint{3.905098in}{2.119498in}}%
\pgfpathclose%
\pgfusepath{stroke,fill}%
\end{pgfscope}%
\begin{pgfscope}%
\pgfpathrectangle{\pgfqpoint{3.793912in}{0.557870in}}{\pgfqpoint{2.446088in}{1.684734in}}%
\pgfusepath{clip}%
\pgfsetbuttcap%
\pgfsetroundjoin%
\definecolor{currentfill}{rgb}{0.298039,0.447059,0.690196}%
\pgfsetfillcolor{currentfill}%
\pgfsetlinewidth{1.003750pt}%
\definecolor{currentstroke}{rgb}{0.298039,0.447059,0.690196}%
\pgfsetstrokecolor{currentstroke}%
\pgfsetdash{}{0pt}%
\pgfpathmoveto{\pgfqpoint{3.905098in}{2.119498in}}%
\pgfpathcurveto{\pgfqpoint{3.913334in}{2.119498in}}{\pgfqpoint{3.921234in}{2.122771in}}{\pgfqpoint{3.927058in}{2.128595in}}%
\pgfpathcurveto{\pgfqpoint{3.932882in}{2.134419in}}{\pgfqpoint{3.936155in}{2.142319in}}{\pgfqpoint{3.936155in}{2.150555in}}%
\pgfpathcurveto{\pgfqpoint{3.936155in}{2.158791in}}{\pgfqpoint{3.932882in}{2.166691in}}{\pgfqpoint{3.927058in}{2.172515in}}%
\pgfpathcurveto{\pgfqpoint{3.921234in}{2.178339in}}{\pgfqpoint{3.913334in}{2.181611in}}{\pgfqpoint{3.905098in}{2.181611in}}%
\pgfpathcurveto{\pgfqpoint{3.896862in}{2.181611in}}{\pgfqpoint{3.888962in}{2.178339in}}{\pgfqpoint{3.883138in}{2.172515in}}%
\pgfpathcurveto{\pgfqpoint{3.877314in}{2.166691in}}{\pgfqpoint{3.874042in}{2.158791in}}{\pgfqpoint{3.874042in}{2.150555in}}%
\pgfpathcurveto{\pgfqpoint{3.874042in}{2.142319in}}{\pgfqpoint{3.877314in}{2.134419in}}{\pgfqpoint{3.883138in}{2.128595in}}%
\pgfpathcurveto{\pgfqpoint{3.888962in}{2.122771in}}{\pgfqpoint{3.896862in}{2.119498in}}{\pgfqpoint{3.905098in}{2.119498in}}%
\pgfpathclose%
\pgfusepath{stroke,fill}%
\end{pgfscope}%
\begin{pgfscope}%
\pgfpathrectangle{\pgfqpoint{3.793912in}{0.557870in}}{\pgfqpoint{2.446088in}{1.684734in}}%
\pgfusepath{clip}%
\pgfsetbuttcap%
\pgfsetroundjoin%
\definecolor{currentfill}{rgb}{0.298039,0.447059,0.690196}%
\pgfsetfillcolor{currentfill}%
\pgfsetlinewidth{1.003750pt}%
\definecolor{currentstroke}{rgb}{0.298039,0.447059,0.690196}%
\pgfsetstrokecolor{currentstroke}%
\pgfsetdash{}{0pt}%
\pgfpathmoveto{\pgfqpoint{3.905098in}{2.119498in}}%
\pgfpathcurveto{\pgfqpoint{3.913334in}{2.119498in}}{\pgfqpoint{3.921234in}{2.122771in}}{\pgfqpoint{3.927058in}{2.128595in}}%
\pgfpathcurveto{\pgfqpoint{3.932882in}{2.134419in}}{\pgfqpoint{3.936155in}{2.142319in}}{\pgfqpoint{3.936155in}{2.150555in}}%
\pgfpathcurveto{\pgfqpoint{3.936155in}{2.158791in}}{\pgfqpoint{3.932882in}{2.166691in}}{\pgfqpoint{3.927058in}{2.172515in}}%
\pgfpathcurveto{\pgfqpoint{3.921234in}{2.178339in}}{\pgfqpoint{3.913334in}{2.181611in}}{\pgfqpoint{3.905098in}{2.181611in}}%
\pgfpathcurveto{\pgfqpoint{3.896862in}{2.181611in}}{\pgfqpoint{3.888962in}{2.178339in}}{\pgfqpoint{3.883138in}{2.172515in}}%
\pgfpathcurveto{\pgfqpoint{3.877314in}{2.166691in}}{\pgfqpoint{3.874042in}{2.158791in}}{\pgfqpoint{3.874042in}{2.150555in}}%
\pgfpathcurveto{\pgfqpoint{3.874042in}{2.142319in}}{\pgfqpoint{3.877314in}{2.134419in}}{\pgfqpoint{3.883138in}{2.128595in}}%
\pgfpathcurveto{\pgfqpoint{3.888962in}{2.122771in}}{\pgfqpoint{3.896862in}{2.119498in}}{\pgfqpoint{3.905098in}{2.119498in}}%
\pgfpathclose%
\pgfusepath{stroke,fill}%
\end{pgfscope}%
\begin{pgfscope}%
\pgfpathrectangle{\pgfqpoint{3.793912in}{0.557870in}}{\pgfqpoint{2.446088in}{1.684734in}}%
\pgfusepath{clip}%
\pgfsetbuttcap%
\pgfsetroundjoin%
\definecolor{currentfill}{rgb}{0.298039,0.447059,0.690196}%
\pgfsetfillcolor{currentfill}%
\pgfsetlinewidth{1.003750pt}%
\definecolor{currentstroke}{rgb}{0.298039,0.447059,0.690196}%
\pgfsetstrokecolor{currentstroke}%
\pgfsetdash{}{0pt}%
\pgfpathmoveto{\pgfqpoint{4.546995in}{1.794619in}}%
\pgfpathcurveto{\pgfqpoint{4.555232in}{1.794619in}}{\pgfqpoint{4.563132in}{1.797891in}}{\pgfqpoint{4.568956in}{1.803715in}}%
\pgfpathcurveto{\pgfqpoint{4.574780in}{1.809539in}}{\pgfqpoint{4.578052in}{1.817439in}}{\pgfqpoint{4.578052in}{1.825675in}}%
\pgfpathcurveto{\pgfqpoint{4.578052in}{1.833911in}}{\pgfqpoint{4.574780in}{1.841811in}}{\pgfqpoint{4.568956in}{1.847635in}}%
\pgfpathcurveto{\pgfqpoint{4.563132in}{1.853459in}}{\pgfqpoint{4.555232in}{1.856732in}}{\pgfqpoint{4.546995in}{1.856732in}}%
\pgfpathcurveto{\pgfqpoint{4.538759in}{1.856732in}}{\pgfqpoint{4.530859in}{1.853459in}}{\pgfqpoint{4.525035in}{1.847635in}}%
\pgfpathcurveto{\pgfqpoint{4.519211in}{1.841811in}}{\pgfqpoint{4.515939in}{1.833911in}}{\pgfqpoint{4.515939in}{1.825675in}}%
\pgfpathcurveto{\pgfqpoint{4.515939in}{1.817439in}}{\pgfqpoint{4.519211in}{1.809539in}}{\pgfqpoint{4.525035in}{1.803715in}}%
\pgfpathcurveto{\pgfqpoint{4.530859in}{1.797891in}}{\pgfqpoint{4.538759in}{1.794619in}}{\pgfqpoint{4.546995in}{1.794619in}}%
\pgfpathclose%
\pgfusepath{stroke,fill}%
\end{pgfscope}%
\begin{pgfscope}%
\pgfpathrectangle{\pgfqpoint{3.793912in}{0.557870in}}{\pgfqpoint{2.446088in}{1.684734in}}%
\pgfusepath{clip}%
\pgfsetbuttcap%
\pgfsetroundjoin%
\definecolor{currentfill}{rgb}{0.298039,0.447059,0.690196}%
\pgfsetfillcolor{currentfill}%
\pgfsetlinewidth{1.003750pt}%
\definecolor{currentstroke}{rgb}{0.298039,0.447059,0.690196}%
\pgfsetstrokecolor{currentstroke}%
\pgfsetdash{}{0pt}%
\pgfpathmoveto{\pgfqpoint{3.905098in}{2.119498in}}%
\pgfpathcurveto{\pgfqpoint{3.913334in}{2.119498in}}{\pgfqpoint{3.921234in}{2.122771in}}{\pgfqpoint{3.927058in}{2.128595in}}%
\pgfpathcurveto{\pgfqpoint{3.932882in}{2.134419in}}{\pgfqpoint{3.936155in}{2.142319in}}{\pgfqpoint{3.936155in}{2.150555in}}%
\pgfpathcurveto{\pgfqpoint{3.936155in}{2.158791in}}{\pgfqpoint{3.932882in}{2.166691in}}{\pgfqpoint{3.927058in}{2.172515in}}%
\pgfpathcurveto{\pgfqpoint{3.921234in}{2.178339in}}{\pgfqpoint{3.913334in}{2.181611in}}{\pgfqpoint{3.905098in}{2.181611in}}%
\pgfpathcurveto{\pgfqpoint{3.896862in}{2.181611in}}{\pgfqpoint{3.888962in}{2.178339in}}{\pgfqpoint{3.883138in}{2.172515in}}%
\pgfpathcurveto{\pgfqpoint{3.877314in}{2.166691in}}{\pgfqpoint{3.874042in}{2.158791in}}{\pgfqpoint{3.874042in}{2.150555in}}%
\pgfpathcurveto{\pgfqpoint{3.874042in}{2.142319in}}{\pgfqpoint{3.877314in}{2.134419in}}{\pgfqpoint{3.883138in}{2.128595in}}%
\pgfpathcurveto{\pgfqpoint{3.888962in}{2.122771in}}{\pgfqpoint{3.896862in}{2.119498in}}{\pgfqpoint{3.905098in}{2.119498in}}%
\pgfpathclose%
\pgfusepath{stroke,fill}%
\end{pgfscope}%
\begin{pgfscope}%
\pgfpathrectangle{\pgfqpoint{3.793912in}{0.557870in}}{\pgfqpoint{2.446088in}{1.684734in}}%
\pgfusepath{clip}%
\pgfsetbuttcap%
\pgfsetroundjoin%
\definecolor{currentfill}{rgb}{0.298039,0.447059,0.690196}%
\pgfsetfillcolor{currentfill}%
\pgfsetlinewidth{1.003750pt}%
\definecolor{currentstroke}{rgb}{0.298039,0.447059,0.690196}%
\pgfsetstrokecolor{currentstroke}%
\pgfsetdash{}{0pt}%
\pgfpathmoveto{\pgfqpoint{4.638695in}{1.748207in}}%
\pgfpathcurveto{\pgfqpoint{4.646931in}{1.748207in}}{\pgfqpoint{4.654831in}{1.751479in}}{\pgfqpoint{4.660655in}{1.757303in}}%
\pgfpathcurveto{\pgfqpoint{4.666479in}{1.763127in}}{\pgfqpoint{4.669752in}{1.771027in}}{\pgfqpoint{4.669752in}{1.779264in}}%
\pgfpathcurveto{\pgfqpoint{4.669752in}{1.787500in}}{\pgfqpoint{4.666479in}{1.795400in}}{\pgfqpoint{4.660655in}{1.801224in}}%
\pgfpathcurveto{\pgfqpoint{4.654831in}{1.807048in}}{\pgfqpoint{4.646931in}{1.810320in}}{\pgfqpoint{4.638695in}{1.810320in}}%
\pgfpathcurveto{\pgfqpoint{4.630459in}{1.810320in}}{\pgfqpoint{4.622559in}{1.807048in}}{\pgfqpoint{4.616735in}{1.801224in}}%
\pgfpathcurveto{\pgfqpoint{4.610911in}{1.795400in}}{\pgfqpoint{4.607639in}{1.787500in}}{\pgfqpoint{4.607639in}{1.779264in}}%
\pgfpathcurveto{\pgfqpoint{4.607639in}{1.771027in}}{\pgfqpoint{4.610911in}{1.763127in}}{\pgfqpoint{4.616735in}{1.757303in}}%
\pgfpathcurveto{\pgfqpoint{4.622559in}{1.751479in}}{\pgfqpoint{4.630459in}{1.748207in}}{\pgfqpoint{4.638695in}{1.748207in}}%
\pgfpathclose%
\pgfusepath{stroke,fill}%
\end{pgfscope}%
\begin{pgfscope}%
\pgfpathrectangle{\pgfqpoint{3.793912in}{0.557870in}}{\pgfqpoint{2.446088in}{1.684734in}}%
\pgfusepath{clip}%
\pgfsetbuttcap%
\pgfsetroundjoin%
\definecolor{currentfill}{rgb}{0.298039,0.447059,0.690196}%
\pgfsetfillcolor{currentfill}%
\pgfsetlinewidth{1.003750pt}%
\definecolor{currentstroke}{rgb}{0.298039,0.447059,0.690196}%
\pgfsetstrokecolor{currentstroke}%
\pgfsetdash{}{0pt}%
\pgfpathmoveto{\pgfqpoint{3.905098in}{2.119498in}}%
\pgfpathcurveto{\pgfqpoint{3.913334in}{2.119498in}}{\pgfqpoint{3.921234in}{2.122771in}}{\pgfqpoint{3.927058in}{2.128595in}}%
\pgfpathcurveto{\pgfqpoint{3.932882in}{2.134419in}}{\pgfqpoint{3.936155in}{2.142319in}}{\pgfqpoint{3.936155in}{2.150555in}}%
\pgfpathcurveto{\pgfqpoint{3.936155in}{2.158791in}}{\pgfqpoint{3.932882in}{2.166691in}}{\pgfqpoint{3.927058in}{2.172515in}}%
\pgfpathcurveto{\pgfqpoint{3.921234in}{2.178339in}}{\pgfqpoint{3.913334in}{2.181611in}}{\pgfqpoint{3.905098in}{2.181611in}}%
\pgfpathcurveto{\pgfqpoint{3.896862in}{2.181611in}}{\pgfqpoint{3.888962in}{2.178339in}}{\pgfqpoint{3.883138in}{2.172515in}}%
\pgfpathcurveto{\pgfqpoint{3.877314in}{2.166691in}}{\pgfqpoint{3.874042in}{2.158791in}}{\pgfqpoint{3.874042in}{2.150555in}}%
\pgfpathcurveto{\pgfqpoint{3.874042in}{2.142319in}}{\pgfqpoint{3.877314in}{2.134419in}}{\pgfqpoint{3.883138in}{2.128595in}}%
\pgfpathcurveto{\pgfqpoint{3.888962in}{2.122771in}}{\pgfqpoint{3.896862in}{2.119498in}}{\pgfqpoint{3.905098in}{2.119498in}}%
\pgfpathclose%
\pgfusepath{stroke,fill}%
\end{pgfscope}%
\begin{pgfscope}%
\pgfpathrectangle{\pgfqpoint{3.793912in}{0.557870in}}{\pgfqpoint{2.446088in}{1.684734in}}%
\pgfusepath{clip}%
\pgfsetbuttcap%
\pgfsetroundjoin%
\definecolor{currentfill}{rgb}{0.298039,0.447059,0.690196}%
\pgfsetfillcolor{currentfill}%
\pgfsetlinewidth{1.003750pt}%
\definecolor{currentstroke}{rgb}{0.298039,0.447059,0.690196}%
\pgfsetstrokecolor{currentstroke}%
\pgfsetdash{}{0pt}%
\pgfpathmoveto{\pgfqpoint{5.922490in}{1.067507in}}%
\pgfpathcurveto{\pgfqpoint{5.930726in}{1.067507in}}{\pgfqpoint{5.938626in}{1.070779in}}{\pgfqpoint{5.944450in}{1.076603in}}%
\pgfpathcurveto{\pgfqpoint{5.950274in}{1.082427in}}{\pgfqpoint{5.953547in}{1.090327in}}{\pgfqpoint{5.953547in}{1.098563in}}%
\pgfpathcurveto{\pgfqpoint{5.953547in}{1.106799in}}{\pgfqpoint{5.950274in}{1.114699in}}{\pgfqpoint{5.944450in}{1.120523in}}%
\pgfpathcurveto{\pgfqpoint{5.938626in}{1.126347in}}{\pgfqpoint{5.930726in}{1.129620in}}{\pgfqpoint{5.922490in}{1.129620in}}%
\pgfpathcurveto{\pgfqpoint{5.914254in}{1.129620in}}{\pgfqpoint{5.906354in}{1.126347in}}{\pgfqpoint{5.900530in}{1.120523in}}%
\pgfpathcurveto{\pgfqpoint{5.894706in}{1.114699in}}{\pgfqpoint{5.891434in}{1.106799in}}{\pgfqpoint{5.891434in}{1.098563in}}%
\pgfpathcurveto{\pgfqpoint{5.891434in}{1.090327in}}{\pgfqpoint{5.894706in}{1.082427in}}{\pgfqpoint{5.900530in}{1.076603in}}%
\pgfpathcurveto{\pgfqpoint{5.906354in}{1.070779in}}{\pgfqpoint{5.914254in}{1.067507in}}{\pgfqpoint{5.922490in}{1.067507in}}%
\pgfpathclose%
\pgfusepath{stroke,fill}%
\end{pgfscope}%
\begin{pgfscope}%
\pgfpathrectangle{\pgfqpoint{3.793912in}{0.557870in}}{\pgfqpoint{2.446088in}{1.684734in}}%
\pgfusepath{clip}%
\pgfsetbuttcap%
\pgfsetroundjoin%
\definecolor{currentfill}{rgb}{0.298039,0.447059,0.690196}%
\pgfsetfillcolor{currentfill}%
\pgfsetlinewidth{1.003750pt}%
\definecolor{currentstroke}{rgb}{0.298039,0.447059,0.690196}%
\pgfsetstrokecolor{currentstroke}%
\pgfsetdash{}{0pt}%
\pgfpathmoveto{\pgfqpoint{3.905098in}{2.119498in}}%
\pgfpathcurveto{\pgfqpoint{3.913334in}{2.119498in}}{\pgfqpoint{3.921234in}{2.122771in}}{\pgfqpoint{3.927058in}{2.128595in}}%
\pgfpathcurveto{\pgfqpoint{3.932882in}{2.134419in}}{\pgfqpoint{3.936155in}{2.142319in}}{\pgfqpoint{3.936155in}{2.150555in}}%
\pgfpathcurveto{\pgfqpoint{3.936155in}{2.158791in}}{\pgfqpoint{3.932882in}{2.166691in}}{\pgfqpoint{3.927058in}{2.172515in}}%
\pgfpathcurveto{\pgfqpoint{3.921234in}{2.178339in}}{\pgfqpoint{3.913334in}{2.181611in}}{\pgfqpoint{3.905098in}{2.181611in}}%
\pgfpathcurveto{\pgfqpoint{3.896862in}{2.181611in}}{\pgfqpoint{3.888962in}{2.178339in}}{\pgfqpoint{3.883138in}{2.172515in}}%
\pgfpathcurveto{\pgfqpoint{3.877314in}{2.166691in}}{\pgfqpoint{3.874042in}{2.158791in}}{\pgfqpoint{3.874042in}{2.150555in}}%
\pgfpathcurveto{\pgfqpoint{3.874042in}{2.142319in}}{\pgfqpoint{3.877314in}{2.134419in}}{\pgfqpoint{3.883138in}{2.128595in}}%
\pgfpathcurveto{\pgfqpoint{3.888962in}{2.122771in}}{\pgfqpoint{3.896862in}{2.119498in}}{\pgfqpoint{3.905098in}{2.119498in}}%
\pgfpathclose%
\pgfusepath{stroke,fill}%
\end{pgfscope}%
\begin{pgfscope}%
\pgfpathrectangle{\pgfqpoint{3.793912in}{0.557870in}}{\pgfqpoint{2.446088in}{1.684734in}}%
\pgfusepath{clip}%
\pgfsetbuttcap%
\pgfsetroundjoin%
\definecolor{currentfill}{rgb}{0.298039,0.447059,0.690196}%
\pgfsetfillcolor{currentfill}%
\pgfsetlinewidth{1.003750pt}%
\definecolor{currentstroke}{rgb}{0.298039,0.447059,0.690196}%
\pgfsetstrokecolor{currentstroke}%
\pgfsetdash{}{0pt}%
\pgfpathmoveto{\pgfqpoint{3.905098in}{2.119498in}}%
\pgfpathcurveto{\pgfqpoint{3.913334in}{2.119498in}}{\pgfqpoint{3.921234in}{2.122771in}}{\pgfqpoint{3.927058in}{2.128595in}}%
\pgfpathcurveto{\pgfqpoint{3.932882in}{2.134419in}}{\pgfqpoint{3.936155in}{2.142319in}}{\pgfqpoint{3.936155in}{2.150555in}}%
\pgfpathcurveto{\pgfqpoint{3.936155in}{2.158791in}}{\pgfqpoint{3.932882in}{2.166691in}}{\pgfqpoint{3.927058in}{2.172515in}}%
\pgfpathcurveto{\pgfqpoint{3.921234in}{2.178339in}}{\pgfqpoint{3.913334in}{2.181611in}}{\pgfqpoint{3.905098in}{2.181611in}}%
\pgfpathcurveto{\pgfqpoint{3.896862in}{2.181611in}}{\pgfqpoint{3.888962in}{2.178339in}}{\pgfqpoint{3.883138in}{2.172515in}}%
\pgfpathcurveto{\pgfqpoint{3.877314in}{2.166691in}}{\pgfqpoint{3.874042in}{2.158791in}}{\pgfqpoint{3.874042in}{2.150555in}}%
\pgfpathcurveto{\pgfqpoint{3.874042in}{2.142319in}}{\pgfqpoint{3.877314in}{2.134419in}}{\pgfqpoint{3.883138in}{2.128595in}}%
\pgfpathcurveto{\pgfqpoint{3.888962in}{2.122771in}}{\pgfqpoint{3.896862in}{2.119498in}}{\pgfqpoint{3.905098in}{2.119498in}}%
\pgfpathclose%
\pgfusepath{stroke,fill}%
\end{pgfscope}%
\begin{pgfscope}%
\pgfpathrectangle{\pgfqpoint{3.793912in}{0.557870in}}{\pgfqpoint{2.446088in}{1.684734in}}%
\pgfusepath{clip}%
\pgfsetbuttcap%
\pgfsetroundjoin%
\definecolor{currentfill}{rgb}{0.298039,0.447059,0.690196}%
\pgfsetfillcolor{currentfill}%
\pgfsetlinewidth{1.003750pt}%
\definecolor{currentstroke}{rgb}{0.298039,0.447059,0.690196}%
\pgfsetstrokecolor{currentstroke}%
\pgfsetdash{}{0pt}%
\pgfpathmoveto{\pgfqpoint{5.165968in}{1.361445in}}%
\pgfpathcurveto{\pgfqpoint{5.174204in}{1.361445in}}{\pgfqpoint{5.182104in}{1.364718in}}{\pgfqpoint{5.187928in}{1.370542in}}%
\pgfpathcurveto{\pgfqpoint{5.193752in}{1.376366in}}{\pgfqpoint{5.197025in}{1.384266in}}{\pgfqpoint{5.197025in}{1.392502in}}%
\pgfpathcurveto{\pgfqpoint{5.197025in}{1.400738in}}{\pgfqpoint{5.193752in}{1.408638in}}{\pgfqpoint{5.187928in}{1.414462in}}%
\pgfpathcurveto{\pgfqpoint{5.182104in}{1.420286in}}{\pgfqpoint{5.174204in}{1.423558in}}{\pgfqpoint{5.165968in}{1.423558in}}%
\pgfpathcurveto{\pgfqpoint{5.157732in}{1.423558in}}{\pgfqpoint{5.149832in}{1.420286in}}{\pgfqpoint{5.144008in}{1.414462in}}%
\pgfpathcurveto{\pgfqpoint{5.138184in}{1.408638in}}{\pgfqpoint{5.134912in}{1.400738in}}{\pgfqpoint{5.134912in}{1.392502in}}%
\pgfpathcurveto{\pgfqpoint{5.134912in}{1.384266in}}{\pgfqpoint{5.138184in}{1.376366in}}{\pgfqpoint{5.144008in}{1.370542in}}%
\pgfpathcurveto{\pgfqpoint{5.149832in}{1.364718in}}{\pgfqpoint{5.157732in}{1.361445in}}{\pgfqpoint{5.165968in}{1.361445in}}%
\pgfpathclose%
\pgfusepath{stroke,fill}%
\end{pgfscope}%
\begin{pgfscope}%
\pgfpathrectangle{\pgfqpoint{3.793912in}{0.557870in}}{\pgfqpoint{2.446088in}{1.684734in}}%
\pgfusepath{clip}%
\pgfsetbuttcap%
\pgfsetroundjoin%
\definecolor{currentfill}{rgb}{0.298039,0.447059,0.690196}%
\pgfsetfillcolor{currentfill}%
\pgfsetlinewidth{1.003750pt}%
\definecolor{currentstroke}{rgb}{0.298039,0.447059,0.690196}%
\pgfsetstrokecolor{currentstroke}%
\pgfsetdash{}{0pt}%
\pgfpathmoveto{\pgfqpoint{3.905098in}{2.119498in}}%
\pgfpathcurveto{\pgfqpoint{3.913334in}{2.119498in}}{\pgfqpoint{3.921234in}{2.122771in}}{\pgfqpoint{3.927058in}{2.128595in}}%
\pgfpathcurveto{\pgfqpoint{3.932882in}{2.134419in}}{\pgfqpoint{3.936155in}{2.142319in}}{\pgfqpoint{3.936155in}{2.150555in}}%
\pgfpathcurveto{\pgfqpoint{3.936155in}{2.158791in}}{\pgfqpoint{3.932882in}{2.166691in}}{\pgfqpoint{3.927058in}{2.172515in}}%
\pgfpathcurveto{\pgfqpoint{3.921234in}{2.178339in}}{\pgfqpoint{3.913334in}{2.181611in}}{\pgfqpoint{3.905098in}{2.181611in}}%
\pgfpathcurveto{\pgfqpoint{3.896862in}{2.181611in}}{\pgfqpoint{3.888962in}{2.178339in}}{\pgfqpoint{3.883138in}{2.172515in}}%
\pgfpathcurveto{\pgfqpoint{3.877314in}{2.166691in}}{\pgfqpoint{3.874042in}{2.158791in}}{\pgfqpoint{3.874042in}{2.150555in}}%
\pgfpathcurveto{\pgfqpoint{3.874042in}{2.142319in}}{\pgfqpoint{3.877314in}{2.134419in}}{\pgfqpoint{3.883138in}{2.128595in}}%
\pgfpathcurveto{\pgfqpoint{3.888962in}{2.122771in}}{\pgfqpoint{3.896862in}{2.119498in}}{\pgfqpoint{3.905098in}{2.119498in}}%
\pgfpathclose%
\pgfusepath{stroke,fill}%
\end{pgfscope}%
\begin{pgfscope}%
\pgfpathrectangle{\pgfqpoint{3.793912in}{0.557870in}}{\pgfqpoint{2.446088in}{1.684734in}}%
\pgfusepath{clip}%
\pgfsetbuttcap%
\pgfsetroundjoin%
\definecolor{currentfill}{rgb}{0.298039,0.447059,0.690196}%
\pgfsetfillcolor{currentfill}%
\pgfsetlinewidth{1.003750pt}%
\definecolor{currentstroke}{rgb}{0.298039,0.447059,0.690196}%
\pgfsetstrokecolor{currentstroke}%
\pgfsetdash{}{0pt}%
\pgfpathmoveto{\pgfqpoint{3.950948in}{2.119498in}}%
\pgfpathcurveto{\pgfqpoint{3.959184in}{2.119498in}}{\pgfqpoint{3.967084in}{2.122771in}}{\pgfqpoint{3.972908in}{2.128595in}}%
\pgfpathcurveto{\pgfqpoint{3.978732in}{2.134419in}}{\pgfqpoint{3.982004in}{2.142319in}}{\pgfqpoint{3.982004in}{2.150555in}}%
\pgfpathcurveto{\pgfqpoint{3.982004in}{2.158791in}}{\pgfqpoint{3.978732in}{2.166691in}}{\pgfqpoint{3.972908in}{2.172515in}}%
\pgfpathcurveto{\pgfqpoint{3.967084in}{2.178339in}}{\pgfqpoint{3.959184in}{2.181611in}}{\pgfqpoint{3.950948in}{2.181611in}}%
\pgfpathcurveto{\pgfqpoint{3.942712in}{2.181611in}}{\pgfqpoint{3.934812in}{2.178339in}}{\pgfqpoint{3.928988in}{2.172515in}}%
\pgfpathcurveto{\pgfqpoint{3.923164in}{2.166691in}}{\pgfqpoint{3.919891in}{2.158791in}}{\pgfqpoint{3.919891in}{2.150555in}}%
\pgfpathcurveto{\pgfqpoint{3.919891in}{2.142319in}}{\pgfqpoint{3.923164in}{2.134419in}}{\pgfqpoint{3.928988in}{2.128595in}}%
\pgfpathcurveto{\pgfqpoint{3.934812in}{2.122771in}}{\pgfqpoint{3.942712in}{2.119498in}}{\pgfqpoint{3.950948in}{2.119498in}}%
\pgfpathclose%
\pgfusepath{stroke,fill}%
\end{pgfscope}%
\begin{pgfscope}%
\pgfpathrectangle{\pgfqpoint{3.793912in}{0.557870in}}{\pgfqpoint{2.446088in}{1.684734in}}%
\pgfusepath{clip}%
\pgfsetbuttcap%
\pgfsetroundjoin%
\definecolor{currentfill}{rgb}{0.298039,0.447059,0.690196}%
\pgfsetfillcolor{currentfill}%
\pgfsetlinewidth{1.003750pt}%
\definecolor{currentstroke}{rgb}{0.298039,0.447059,0.690196}%
\pgfsetstrokecolor{currentstroke}%
\pgfsetdash{}{0pt}%
\pgfpathmoveto{\pgfqpoint{3.905098in}{2.119498in}}%
\pgfpathcurveto{\pgfqpoint{3.913334in}{2.119498in}}{\pgfqpoint{3.921234in}{2.122771in}}{\pgfqpoint{3.927058in}{2.128595in}}%
\pgfpathcurveto{\pgfqpoint{3.932882in}{2.134419in}}{\pgfqpoint{3.936155in}{2.142319in}}{\pgfqpoint{3.936155in}{2.150555in}}%
\pgfpathcurveto{\pgfqpoint{3.936155in}{2.158791in}}{\pgfqpoint{3.932882in}{2.166691in}}{\pgfqpoint{3.927058in}{2.172515in}}%
\pgfpathcurveto{\pgfqpoint{3.921234in}{2.178339in}}{\pgfqpoint{3.913334in}{2.181611in}}{\pgfqpoint{3.905098in}{2.181611in}}%
\pgfpathcurveto{\pgfqpoint{3.896862in}{2.181611in}}{\pgfqpoint{3.888962in}{2.178339in}}{\pgfqpoint{3.883138in}{2.172515in}}%
\pgfpathcurveto{\pgfqpoint{3.877314in}{2.166691in}}{\pgfqpoint{3.874042in}{2.158791in}}{\pgfqpoint{3.874042in}{2.150555in}}%
\pgfpathcurveto{\pgfqpoint{3.874042in}{2.142319in}}{\pgfqpoint{3.877314in}{2.134419in}}{\pgfqpoint{3.883138in}{2.128595in}}%
\pgfpathcurveto{\pgfqpoint{3.888962in}{2.122771in}}{\pgfqpoint{3.896862in}{2.119498in}}{\pgfqpoint{3.905098in}{2.119498in}}%
\pgfpathclose%
\pgfusepath{stroke,fill}%
\end{pgfscope}%
\begin{pgfscope}%
\pgfpathrectangle{\pgfqpoint{3.793912in}{0.557870in}}{\pgfqpoint{2.446088in}{1.684734in}}%
\pgfusepath{clip}%
\pgfsetbuttcap%
\pgfsetroundjoin%
\definecolor{currentfill}{rgb}{0.298039,0.447059,0.690196}%
\pgfsetfillcolor{currentfill}%
\pgfsetlinewidth{1.003750pt}%
\definecolor{currentstroke}{rgb}{0.298039,0.447059,0.690196}%
\pgfsetstrokecolor{currentstroke}%
\pgfsetdash{}{0pt}%
\pgfpathmoveto{\pgfqpoint{5.532767in}{1.531621in}}%
\pgfpathcurveto{\pgfqpoint{5.541003in}{1.531621in}}{\pgfqpoint{5.548903in}{1.534893in}}{\pgfqpoint{5.554727in}{1.540717in}}%
\pgfpathcurveto{\pgfqpoint{5.560551in}{1.546541in}}{\pgfqpoint{5.563823in}{1.554441in}}{\pgfqpoint{5.563823in}{1.562677in}}%
\pgfpathcurveto{\pgfqpoint{5.563823in}{1.570913in}}{\pgfqpoint{5.560551in}{1.578813in}}{\pgfqpoint{5.554727in}{1.584637in}}%
\pgfpathcurveto{\pgfqpoint{5.548903in}{1.590461in}}{\pgfqpoint{5.541003in}{1.593734in}}{\pgfqpoint{5.532767in}{1.593734in}}%
\pgfpathcurveto{\pgfqpoint{5.524530in}{1.593734in}}{\pgfqpoint{5.516630in}{1.590461in}}{\pgfqpoint{5.510806in}{1.584637in}}%
\pgfpathcurveto{\pgfqpoint{5.504982in}{1.578813in}}{\pgfqpoint{5.501710in}{1.570913in}}{\pgfqpoint{5.501710in}{1.562677in}}%
\pgfpathcurveto{\pgfqpoint{5.501710in}{1.554441in}}{\pgfqpoint{5.504982in}{1.546541in}}{\pgfqpoint{5.510806in}{1.540717in}}%
\pgfpathcurveto{\pgfqpoint{5.516630in}{1.534893in}}{\pgfqpoint{5.524530in}{1.531621in}}{\pgfqpoint{5.532767in}{1.531621in}}%
\pgfpathclose%
\pgfusepath{stroke,fill}%
\end{pgfscope}%
\begin{pgfscope}%
\pgfpathrectangle{\pgfqpoint{3.793912in}{0.557870in}}{\pgfqpoint{2.446088in}{1.684734in}}%
\pgfusepath{clip}%
\pgfsetbuttcap%
\pgfsetroundjoin%
\definecolor{currentfill}{rgb}{0.298039,0.447059,0.690196}%
\pgfsetfillcolor{currentfill}%
\pgfsetlinewidth{1.003750pt}%
\definecolor{currentstroke}{rgb}{0.298039,0.447059,0.690196}%
\pgfsetstrokecolor{currentstroke}%
\pgfsetdash{}{0pt}%
\pgfpathmoveto{\pgfqpoint{3.905098in}{2.119498in}}%
\pgfpathcurveto{\pgfqpoint{3.913334in}{2.119498in}}{\pgfqpoint{3.921234in}{2.122771in}}{\pgfqpoint{3.927058in}{2.128595in}}%
\pgfpathcurveto{\pgfqpoint{3.932882in}{2.134419in}}{\pgfqpoint{3.936155in}{2.142319in}}{\pgfqpoint{3.936155in}{2.150555in}}%
\pgfpathcurveto{\pgfqpoint{3.936155in}{2.158791in}}{\pgfqpoint{3.932882in}{2.166691in}}{\pgfqpoint{3.927058in}{2.172515in}}%
\pgfpathcurveto{\pgfqpoint{3.921234in}{2.178339in}}{\pgfqpoint{3.913334in}{2.181611in}}{\pgfqpoint{3.905098in}{2.181611in}}%
\pgfpathcurveto{\pgfqpoint{3.896862in}{2.181611in}}{\pgfqpoint{3.888962in}{2.178339in}}{\pgfqpoint{3.883138in}{2.172515in}}%
\pgfpathcurveto{\pgfqpoint{3.877314in}{2.166691in}}{\pgfqpoint{3.874042in}{2.158791in}}{\pgfqpoint{3.874042in}{2.150555in}}%
\pgfpathcurveto{\pgfqpoint{3.874042in}{2.142319in}}{\pgfqpoint{3.877314in}{2.134419in}}{\pgfqpoint{3.883138in}{2.128595in}}%
\pgfpathcurveto{\pgfqpoint{3.888962in}{2.122771in}}{\pgfqpoint{3.896862in}{2.119498in}}{\pgfqpoint{3.905098in}{2.119498in}}%
\pgfpathclose%
\pgfusepath{stroke,fill}%
\end{pgfscope}%
\begin{pgfscope}%
\pgfpathrectangle{\pgfqpoint{3.793912in}{0.557870in}}{\pgfqpoint{2.446088in}{1.684734in}}%
\pgfusepath{clip}%
\pgfsetbuttcap%
\pgfsetroundjoin%
\definecolor{currentfill}{rgb}{0.298039,0.447059,0.690196}%
\pgfsetfillcolor{currentfill}%
\pgfsetlinewidth{1.003750pt}%
\definecolor{currentstroke}{rgb}{0.298039,0.447059,0.690196}%
\pgfsetstrokecolor{currentstroke}%
\pgfsetdash{}{0pt}%
\pgfpathmoveto{\pgfqpoint{5.418142in}{1.794619in}}%
\pgfpathcurveto{\pgfqpoint{5.426378in}{1.794619in}}{\pgfqpoint{5.434278in}{1.797891in}}{\pgfqpoint{5.440102in}{1.803715in}}%
\pgfpathcurveto{\pgfqpoint{5.445926in}{1.809539in}}{\pgfqpoint{5.449199in}{1.817439in}}{\pgfqpoint{5.449199in}{1.825675in}}%
\pgfpathcurveto{\pgfqpoint{5.449199in}{1.833911in}}{\pgfqpoint{5.445926in}{1.841811in}}{\pgfqpoint{5.440102in}{1.847635in}}%
\pgfpathcurveto{\pgfqpoint{5.434278in}{1.853459in}}{\pgfqpoint{5.426378in}{1.856732in}}{\pgfqpoint{5.418142in}{1.856732in}}%
\pgfpathcurveto{\pgfqpoint{5.409906in}{1.856732in}}{\pgfqpoint{5.402006in}{1.853459in}}{\pgfqpoint{5.396182in}{1.847635in}}%
\pgfpathcurveto{\pgfqpoint{5.390358in}{1.841811in}}{\pgfqpoint{5.387086in}{1.833911in}}{\pgfqpoint{5.387086in}{1.825675in}}%
\pgfpathcurveto{\pgfqpoint{5.387086in}{1.817439in}}{\pgfqpoint{5.390358in}{1.809539in}}{\pgfqpoint{5.396182in}{1.803715in}}%
\pgfpathcurveto{\pgfqpoint{5.402006in}{1.797891in}}{\pgfqpoint{5.409906in}{1.794619in}}{\pgfqpoint{5.418142in}{1.794619in}}%
\pgfpathclose%
\pgfusepath{stroke,fill}%
\end{pgfscope}%
\begin{pgfscope}%
\pgfpathrectangle{\pgfqpoint{3.793912in}{0.557870in}}{\pgfqpoint{2.446088in}{1.684734in}}%
\pgfusepath{clip}%
\pgfsetbuttcap%
\pgfsetroundjoin%
\definecolor{currentfill}{rgb}{0.298039,0.447059,0.690196}%
\pgfsetfillcolor{currentfill}%
\pgfsetlinewidth{1.003750pt}%
\definecolor{currentstroke}{rgb}{0.298039,0.447059,0.690196}%
\pgfsetstrokecolor{currentstroke}%
\pgfsetdash{}{0pt}%
\pgfpathmoveto{\pgfqpoint{3.905098in}{2.119498in}}%
\pgfpathcurveto{\pgfqpoint{3.913334in}{2.119498in}}{\pgfqpoint{3.921234in}{2.122771in}}{\pgfqpoint{3.927058in}{2.128595in}}%
\pgfpathcurveto{\pgfqpoint{3.932882in}{2.134419in}}{\pgfqpoint{3.936155in}{2.142319in}}{\pgfqpoint{3.936155in}{2.150555in}}%
\pgfpathcurveto{\pgfqpoint{3.936155in}{2.158791in}}{\pgfqpoint{3.932882in}{2.166691in}}{\pgfqpoint{3.927058in}{2.172515in}}%
\pgfpathcurveto{\pgfqpoint{3.921234in}{2.178339in}}{\pgfqpoint{3.913334in}{2.181611in}}{\pgfqpoint{3.905098in}{2.181611in}}%
\pgfpathcurveto{\pgfqpoint{3.896862in}{2.181611in}}{\pgfqpoint{3.888962in}{2.178339in}}{\pgfqpoint{3.883138in}{2.172515in}}%
\pgfpathcurveto{\pgfqpoint{3.877314in}{2.166691in}}{\pgfqpoint{3.874042in}{2.158791in}}{\pgfqpoint{3.874042in}{2.150555in}}%
\pgfpathcurveto{\pgfqpoint{3.874042in}{2.142319in}}{\pgfqpoint{3.877314in}{2.134419in}}{\pgfqpoint{3.883138in}{2.128595in}}%
\pgfpathcurveto{\pgfqpoint{3.888962in}{2.122771in}}{\pgfqpoint{3.896862in}{2.119498in}}{\pgfqpoint{3.905098in}{2.119498in}}%
\pgfpathclose%
\pgfusepath{stroke,fill}%
\end{pgfscope}%
\begin{pgfscope}%
\pgfpathrectangle{\pgfqpoint{3.793912in}{0.557870in}}{\pgfqpoint{2.446088in}{1.684734in}}%
\pgfusepath{clip}%
\pgfsetbuttcap%
\pgfsetroundjoin%
\definecolor{currentfill}{rgb}{0.298039,0.447059,0.690196}%
\pgfsetfillcolor{currentfill}%
\pgfsetlinewidth{1.003750pt}%
\definecolor{currentstroke}{rgb}{0.298039,0.447059,0.690196}%
\pgfsetstrokecolor{currentstroke}%
\pgfsetdash{}{0pt}%
\pgfpathmoveto{\pgfqpoint{3.905098in}{2.119498in}}%
\pgfpathcurveto{\pgfqpoint{3.913334in}{2.119498in}}{\pgfqpoint{3.921234in}{2.122771in}}{\pgfqpoint{3.927058in}{2.128595in}}%
\pgfpathcurveto{\pgfqpoint{3.932882in}{2.134419in}}{\pgfqpoint{3.936155in}{2.142319in}}{\pgfqpoint{3.936155in}{2.150555in}}%
\pgfpathcurveto{\pgfqpoint{3.936155in}{2.158791in}}{\pgfqpoint{3.932882in}{2.166691in}}{\pgfqpoint{3.927058in}{2.172515in}}%
\pgfpathcurveto{\pgfqpoint{3.921234in}{2.178339in}}{\pgfqpoint{3.913334in}{2.181611in}}{\pgfqpoint{3.905098in}{2.181611in}}%
\pgfpathcurveto{\pgfqpoint{3.896862in}{2.181611in}}{\pgfqpoint{3.888962in}{2.178339in}}{\pgfqpoint{3.883138in}{2.172515in}}%
\pgfpathcurveto{\pgfqpoint{3.877314in}{2.166691in}}{\pgfqpoint{3.874042in}{2.158791in}}{\pgfqpoint{3.874042in}{2.150555in}}%
\pgfpathcurveto{\pgfqpoint{3.874042in}{2.142319in}}{\pgfqpoint{3.877314in}{2.134419in}}{\pgfqpoint{3.883138in}{2.128595in}}%
\pgfpathcurveto{\pgfqpoint{3.888962in}{2.122771in}}{\pgfqpoint{3.896862in}{2.119498in}}{\pgfqpoint{3.905098in}{2.119498in}}%
\pgfpathclose%
\pgfusepath{stroke,fill}%
\end{pgfscope}%
\begin{pgfscope}%
\pgfpathrectangle{\pgfqpoint{3.793912in}{0.557870in}}{\pgfqpoint{2.446088in}{1.684734in}}%
\pgfusepath{clip}%
\pgfsetbuttcap%
\pgfsetroundjoin%
\definecolor{currentfill}{rgb}{0.298039,0.447059,0.690196}%
\pgfsetfillcolor{currentfill}%
\pgfsetlinewidth{1.003750pt}%
\definecolor{currentstroke}{rgb}{0.298039,0.447059,0.690196}%
\pgfsetstrokecolor{currentstroke}%
\pgfsetdash{}{0pt}%
\pgfpathmoveto{\pgfqpoint{3.905098in}{2.119498in}}%
\pgfpathcurveto{\pgfqpoint{3.913334in}{2.119498in}}{\pgfqpoint{3.921234in}{2.122771in}}{\pgfqpoint{3.927058in}{2.128595in}}%
\pgfpathcurveto{\pgfqpoint{3.932882in}{2.134419in}}{\pgfqpoint{3.936155in}{2.142319in}}{\pgfqpoint{3.936155in}{2.150555in}}%
\pgfpathcurveto{\pgfqpoint{3.936155in}{2.158791in}}{\pgfqpoint{3.932882in}{2.166691in}}{\pgfqpoint{3.927058in}{2.172515in}}%
\pgfpathcurveto{\pgfqpoint{3.921234in}{2.178339in}}{\pgfqpoint{3.913334in}{2.181611in}}{\pgfqpoint{3.905098in}{2.181611in}}%
\pgfpathcurveto{\pgfqpoint{3.896862in}{2.181611in}}{\pgfqpoint{3.888962in}{2.178339in}}{\pgfqpoint{3.883138in}{2.172515in}}%
\pgfpathcurveto{\pgfqpoint{3.877314in}{2.166691in}}{\pgfqpoint{3.874042in}{2.158791in}}{\pgfqpoint{3.874042in}{2.150555in}}%
\pgfpathcurveto{\pgfqpoint{3.874042in}{2.142319in}}{\pgfqpoint{3.877314in}{2.134419in}}{\pgfqpoint{3.883138in}{2.128595in}}%
\pgfpathcurveto{\pgfqpoint{3.888962in}{2.122771in}}{\pgfqpoint{3.896862in}{2.119498in}}{\pgfqpoint{3.905098in}{2.119498in}}%
\pgfpathclose%
\pgfusepath{stroke,fill}%
\end{pgfscope}%
\begin{pgfscope}%
\pgfpathrectangle{\pgfqpoint{3.793912in}{0.557870in}}{\pgfqpoint{2.446088in}{1.684734in}}%
\pgfusepath{clip}%
\pgfsetbuttcap%
\pgfsetroundjoin%
\definecolor{currentfill}{rgb}{0.298039,0.447059,0.690196}%
\pgfsetfillcolor{currentfill}%
\pgfsetlinewidth{1.003750pt}%
\definecolor{currentstroke}{rgb}{0.298039,0.447059,0.690196}%
\pgfsetstrokecolor{currentstroke}%
\pgfsetdash{}{0pt}%
\pgfpathmoveto{\pgfqpoint{5.647391in}{1.438798in}}%
\pgfpathcurveto{\pgfqpoint{5.655627in}{1.438798in}}{\pgfqpoint{5.663527in}{1.442070in}}{\pgfqpoint{5.669351in}{1.447894in}}%
\pgfpathcurveto{\pgfqpoint{5.675175in}{1.453718in}}{\pgfqpoint{5.678448in}{1.461618in}}{\pgfqpoint{5.678448in}{1.469854in}}%
\pgfpathcurveto{\pgfqpoint{5.678448in}{1.478091in}}{\pgfqpoint{5.675175in}{1.485991in}}{\pgfqpoint{5.669351in}{1.491815in}}%
\pgfpathcurveto{\pgfqpoint{5.663527in}{1.497639in}}{\pgfqpoint{5.655627in}{1.500911in}}{\pgfqpoint{5.647391in}{1.500911in}}%
\pgfpathcurveto{\pgfqpoint{5.639155in}{1.500911in}}{\pgfqpoint{5.631255in}{1.497639in}}{\pgfqpoint{5.625431in}{1.491815in}}%
\pgfpathcurveto{\pgfqpoint{5.619607in}{1.485991in}}{\pgfqpoint{5.616335in}{1.478091in}}{\pgfqpoint{5.616335in}{1.469854in}}%
\pgfpathcurveto{\pgfqpoint{5.616335in}{1.461618in}}{\pgfqpoint{5.619607in}{1.453718in}}{\pgfqpoint{5.625431in}{1.447894in}}%
\pgfpathcurveto{\pgfqpoint{5.631255in}{1.442070in}}{\pgfqpoint{5.639155in}{1.438798in}}{\pgfqpoint{5.647391in}{1.438798in}}%
\pgfpathclose%
\pgfusepath{stroke,fill}%
\end{pgfscope}%
\begin{pgfscope}%
\pgfpathrectangle{\pgfqpoint{3.793912in}{0.557870in}}{\pgfqpoint{2.446088in}{1.684734in}}%
\pgfusepath{clip}%
\pgfsetbuttcap%
\pgfsetroundjoin%
\definecolor{currentfill}{rgb}{0.298039,0.447059,0.690196}%
\pgfsetfillcolor{currentfill}%
\pgfsetlinewidth{1.003750pt}%
\definecolor{currentstroke}{rgb}{0.298039,0.447059,0.690196}%
\pgfsetstrokecolor{currentstroke}%
\pgfsetdash{}{0pt}%
\pgfpathmoveto{\pgfqpoint{5.624466in}{1.500680in}}%
\pgfpathcurveto{\pgfqpoint{5.632702in}{1.500680in}}{\pgfqpoint{5.640603in}{1.503952in}}{\pgfqpoint{5.646426in}{1.509776in}}%
\pgfpathcurveto{\pgfqpoint{5.652250in}{1.515600in}}{\pgfqpoint{5.655523in}{1.523500in}}{\pgfqpoint{5.655523in}{1.531736in}}%
\pgfpathcurveto{\pgfqpoint{5.655523in}{1.539972in}}{\pgfqpoint{5.652250in}{1.547873in}}{\pgfqpoint{5.646426in}{1.553696in}}%
\pgfpathcurveto{\pgfqpoint{5.640603in}{1.559520in}}{\pgfqpoint{5.632702in}{1.562793in}}{\pgfqpoint{5.624466in}{1.562793in}}%
\pgfpathcurveto{\pgfqpoint{5.616230in}{1.562793in}}{\pgfqpoint{5.608330in}{1.559520in}}{\pgfqpoint{5.602506in}{1.553696in}}%
\pgfpathcurveto{\pgfqpoint{5.596682in}{1.547873in}}{\pgfqpoint{5.593410in}{1.539972in}}{\pgfqpoint{5.593410in}{1.531736in}}%
\pgfpathcurveto{\pgfqpoint{5.593410in}{1.523500in}}{\pgfqpoint{5.596682in}{1.515600in}}{\pgfqpoint{5.602506in}{1.509776in}}%
\pgfpathcurveto{\pgfqpoint{5.608330in}{1.503952in}}{\pgfqpoint{5.616230in}{1.500680in}}{\pgfqpoint{5.624466in}{1.500680in}}%
\pgfpathclose%
\pgfusepath{stroke,fill}%
\end{pgfscope}%
\begin{pgfscope}%
\pgfpathrectangle{\pgfqpoint{3.793912in}{0.557870in}}{\pgfqpoint{2.446088in}{1.684734in}}%
\pgfusepath{clip}%
\pgfsetbuttcap%
\pgfsetroundjoin%
\definecolor{currentfill}{rgb}{0.298039,0.447059,0.690196}%
\pgfsetfillcolor{currentfill}%
\pgfsetlinewidth{1.003750pt}%
\definecolor{currentstroke}{rgb}{0.298039,0.447059,0.690196}%
\pgfsetstrokecolor{currentstroke}%
\pgfsetdash{}{0pt}%
\pgfpathmoveto{\pgfqpoint{5.784941in}{1.315034in}}%
\pgfpathcurveto{\pgfqpoint{5.793177in}{1.315034in}}{\pgfqpoint{5.801077in}{1.318306in}}{\pgfqpoint{5.806901in}{1.324130in}}%
\pgfpathcurveto{\pgfqpoint{5.812725in}{1.329954in}}{\pgfqpoint{5.815997in}{1.337854in}}{\pgfqpoint{5.815997in}{1.346091in}}%
\pgfpathcurveto{\pgfqpoint{5.815997in}{1.354327in}}{\pgfqpoint{5.812725in}{1.362227in}}{\pgfqpoint{5.806901in}{1.368051in}}%
\pgfpathcurveto{\pgfqpoint{5.801077in}{1.373875in}}{\pgfqpoint{5.793177in}{1.377147in}}{\pgfqpoint{5.784941in}{1.377147in}}%
\pgfpathcurveto{\pgfqpoint{5.776704in}{1.377147in}}{\pgfqpoint{5.768804in}{1.373875in}}{\pgfqpoint{5.762980in}{1.368051in}}%
\pgfpathcurveto{\pgfqpoint{5.757156in}{1.362227in}}{\pgfqpoint{5.753884in}{1.354327in}}{\pgfqpoint{5.753884in}{1.346091in}}%
\pgfpathcurveto{\pgfqpoint{5.753884in}{1.337854in}}{\pgfqpoint{5.757156in}{1.329954in}}{\pgfqpoint{5.762980in}{1.324130in}}%
\pgfpathcurveto{\pgfqpoint{5.768804in}{1.318306in}}{\pgfqpoint{5.776704in}{1.315034in}}{\pgfqpoint{5.784941in}{1.315034in}}%
\pgfpathclose%
\pgfusepath{stroke,fill}%
\end{pgfscope}%
\begin{pgfscope}%
\pgfpathrectangle{\pgfqpoint{3.793912in}{0.557870in}}{\pgfqpoint{2.446088in}{1.684734in}}%
\pgfusepath{clip}%
\pgfsetbuttcap%
\pgfsetroundjoin%
\definecolor{currentfill}{rgb}{0.298039,0.447059,0.690196}%
\pgfsetfillcolor{currentfill}%
\pgfsetlinewidth{1.003750pt}%
\definecolor{currentstroke}{rgb}{0.298039,0.447059,0.690196}%
\pgfsetstrokecolor{currentstroke}%
\pgfsetdash{}{0pt}%
\pgfpathmoveto{\pgfqpoint{3.905098in}{2.119498in}}%
\pgfpathcurveto{\pgfqpoint{3.913334in}{2.119498in}}{\pgfqpoint{3.921234in}{2.122771in}}{\pgfqpoint{3.927058in}{2.128595in}}%
\pgfpathcurveto{\pgfqpoint{3.932882in}{2.134419in}}{\pgfqpoint{3.936155in}{2.142319in}}{\pgfqpoint{3.936155in}{2.150555in}}%
\pgfpathcurveto{\pgfqpoint{3.936155in}{2.158791in}}{\pgfqpoint{3.932882in}{2.166691in}}{\pgfqpoint{3.927058in}{2.172515in}}%
\pgfpathcurveto{\pgfqpoint{3.921234in}{2.178339in}}{\pgfqpoint{3.913334in}{2.181611in}}{\pgfqpoint{3.905098in}{2.181611in}}%
\pgfpathcurveto{\pgfqpoint{3.896862in}{2.181611in}}{\pgfqpoint{3.888962in}{2.178339in}}{\pgfqpoint{3.883138in}{2.172515in}}%
\pgfpathcurveto{\pgfqpoint{3.877314in}{2.166691in}}{\pgfqpoint{3.874042in}{2.158791in}}{\pgfqpoint{3.874042in}{2.150555in}}%
\pgfpathcurveto{\pgfqpoint{3.874042in}{2.142319in}}{\pgfqpoint{3.877314in}{2.134419in}}{\pgfqpoint{3.883138in}{2.128595in}}%
\pgfpathcurveto{\pgfqpoint{3.888962in}{2.122771in}}{\pgfqpoint{3.896862in}{2.119498in}}{\pgfqpoint{3.905098in}{2.119498in}}%
\pgfpathclose%
\pgfusepath{stroke,fill}%
\end{pgfscope}%
\begin{pgfscope}%
\pgfpathrectangle{\pgfqpoint{3.793912in}{0.557870in}}{\pgfqpoint{2.446088in}{1.684734in}}%
\pgfusepath{clip}%
\pgfsetbuttcap%
\pgfsetroundjoin%
\definecolor{currentfill}{rgb}{0.298039,0.447059,0.690196}%
\pgfsetfillcolor{currentfill}%
\pgfsetlinewidth{1.003750pt}%
\definecolor{currentstroke}{rgb}{0.298039,0.447059,0.690196}%
\pgfsetstrokecolor{currentstroke}%
\pgfsetdash{}{0pt}%
\pgfpathmoveto{\pgfqpoint{5.463992in}{1.129388in}}%
\pgfpathcurveto{\pgfqpoint{5.472228in}{1.129388in}}{\pgfqpoint{5.480128in}{1.132661in}}{\pgfqpoint{5.485952in}{1.138485in}}%
\pgfpathcurveto{\pgfqpoint{5.491776in}{1.144309in}}{\pgfqpoint{5.495048in}{1.152209in}}{\pgfqpoint{5.495048in}{1.160445in}}%
\pgfpathcurveto{\pgfqpoint{5.495048in}{1.168681in}}{\pgfqpoint{5.491776in}{1.176581in}}{\pgfqpoint{5.485952in}{1.182405in}}%
\pgfpathcurveto{\pgfqpoint{5.480128in}{1.188229in}}{\pgfqpoint{5.472228in}{1.191501in}}{\pgfqpoint{5.463992in}{1.191501in}}%
\pgfpathcurveto{\pgfqpoint{5.455756in}{1.191501in}}{\pgfqpoint{5.447856in}{1.188229in}}{\pgfqpoint{5.442032in}{1.182405in}}%
\pgfpathcurveto{\pgfqpoint{5.436208in}{1.176581in}}{\pgfqpoint{5.432935in}{1.168681in}}{\pgfqpoint{5.432935in}{1.160445in}}%
\pgfpathcurveto{\pgfqpoint{5.432935in}{1.152209in}}{\pgfqpoint{5.436208in}{1.144309in}}{\pgfqpoint{5.442032in}{1.138485in}}%
\pgfpathcurveto{\pgfqpoint{5.447856in}{1.132661in}}{\pgfqpoint{5.455756in}{1.129388in}}{\pgfqpoint{5.463992in}{1.129388in}}%
\pgfpathclose%
\pgfusepath{stroke,fill}%
\end{pgfscope}%
\begin{pgfscope}%
\pgfpathrectangle{\pgfqpoint{3.793912in}{0.557870in}}{\pgfqpoint{2.446088in}{1.684734in}}%
\pgfusepath{clip}%
\pgfsetbuttcap%
\pgfsetroundjoin%
\definecolor{currentfill}{rgb}{0.298039,0.447059,0.690196}%
\pgfsetfillcolor{currentfill}%
\pgfsetlinewidth{1.003750pt}%
\definecolor{currentstroke}{rgb}{0.298039,0.447059,0.690196}%
\pgfsetstrokecolor{currentstroke}%
\pgfsetdash{}{0pt}%
\pgfpathmoveto{\pgfqpoint{5.762016in}{0.974684in}}%
\pgfpathcurveto{\pgfqpoint{5.770252in}{0.974684in}}{\pgfqpoint{5.778152in}{0.977956in}}{\pgfqpoint{5.783976in}{0.983780in}}%
\pgfpathcurveto{\pgfqpoint{5.789800in}{0.989604in}}{\pgfqpoint{5.793072in}{0.997504in}}{\pgfqpoint{5.793072in}{1.005740in}}%
\pgfpathcurveto{\pgfqpoint{5.793072in}{1.013977in}}{\pgfqpoint{5.789800in}{1.021877in}}{\pgfqpoint{5.783976in}{1.027701in}}%
\pgfpathcurveto{\pgfqpoint{5.778152in}{1.033524in}}{\pgfqpoint{5.770252in}{1.036797in}}{\pgfqpoint{5.762016in}{1.036797in}}%
\pgfpathcurveto{\pgfqpoint{5.753779in}{1.036797in}}{\pgfqpoint{5.745879in}{1.033524in}}{\pgfqpoint{5.740055in}{1.027701in}}%
\pgfpathcurveto{\pgfqpoint{5.734231in}{1.021877in}}{\pgfqpoint{5.730959in}{1.013977in}}{\pgfqpoint{5.730959in}{1.005740in}}%
\pgfpathcurveto{\pgfqpoint{5.730959in}{0.997504in}}{\pgfqpoint{5.734231in}{0.989604in}}{\pgfqpoint{5.740055in}{0.983780in}}%
\pgfpathcurveto{\pgfqpoint{5.745879in}{0.977956in}}{\pgfqpoint{5.753779in}{0.974684in}}{\pgfqpoint{5.762016in}{0.974684in}}%
\pgfpathclose%
\pgfusepath{stroke,fill}%
\end{pgfscope}%
\begin{pgfscope}%
\pgfpathrectangle{\pgfqpoint{3.793912in}{0.557870in}}{\pgfqpoint{2.446088in}{1.684734in}}%
\pgfusepath{clip}%
\pgfsetbuttcap%
\pgfsetroundjoin%
\definecolor{currentfill}{rgb}{0.298039,0.447059,0.690196}%
\pgfsetfillcolor{currentfill}%
\pgfsetlinewidth{1.003750pt}%
\definecolor{currentstroke}{rgb}{0.298039,0.447059,0.690196}%
\pgfsetstrokecolor{currentstroke}%
\pgfsetdash{}{0pt}%
\pgfpathmoveto{\pgfqpoint{3.905098in}{2.119498in}}%
\pgfpathcurveto{\pgfqpoint{3.913334in}{2.119498in}}{\pgfqpoint{3.921234in}{2.122771in}}{\pgfqpoint{3.927058in}{2.128595in}}%
\pgfpathcurveto{\pgfqpoint{3.932882in}{2.134419in}}{\pgfqpoint{3.936155in}{2.142319in}}{\pgfqpoint{3.936155in}{2.150555in}}%
\pgfpathcurveto{\pgfqpoint{3.936155in}{2.158791in}}{\pgfqpoint{3.932882in}{2.166691in}}{\pgfqpoint{3.927058in}{2.172515in}}%
\pgfpathcurveto{\pgfqpoint{3.921234in}{2.178339in}}{\pgfqpoint{3.913334in}{2.181611in}}{\pgfqpoint{3.905098in}{2.181611in}}%
\pgfpathcurveto{\pgfqpoint{3.896862in}{2.181611in}}{\pgfqpoint{3.888962in}{2.178339in}}{\pgfqpoint{3.883138in}{2.172515in}}%
\pgfpathcurveto{\pgfqpoint{3.877314in}{2.166691in}}{\pgfqpoint{3.874042in}{2.158791in}}{\pgfqpoint{3.874042in}{2.150555in}}%
\pgfpathcurveto{\pgfqpoint{3.874042in}{2.142319in}}{\pgfqpoint{3.877314in}{2.134419in}}{\pgfqpoint{3.883138in}{2.128595in}}%
\pgfpathcurveto{\pgfqpoint{3.888962in}{2.122771in}}{\pgfqpoint{3.896862in}{2.119498in}}{\pgfqpoint{3.905098in}{2.119498in}}%
\pgfpathclose%
\pgfusepath{stroke,fill}%
\end{pgfscope}%
\begin{pgfscope}%
\pgfpathrectangle{\pgfqpoint{3.793912in}{0.557870in}}{\pgfqpoint{2.446088in}{1.684734in}}%
\pgfusepath{clip}%
\pgfsetbuttcap%
\pgfsetroundjoin%
\definecolor{currentfill}{rgb}{0.298039,0.447059,0.690196}%
\pgfsetfillcolor{currentfill}%
\pgfsetlinewidth{1.003750pt}%
\definecolor{currentstroke}{rgb}{0.298039,0.447059,0.690196}%
\pgfsetstrokecolor{currentstroke}%
\pgfsetdash{}{0pt}%
\pgfpathmoveto{\pgfqpoint{3.996798in}{2.119498in}}%
\pgfpathcurveto{\pgfqpoint{4.005034in}{2.119498in}}{\pgfqpoint{4.012934in}{2.122771in}}{\pgfqpoint{4.018758in}{2.128595in}}%
\pgfpathcurveto{\pgfqpoint{4.024582in}{2.134419in}}{\pgfqpoint{4.027854in}{2.142319in}}{\pgfqpoint{4.027854in}{2.150555in}}%
\pgfpathcurveto{\pgfqpoint{4.027854in}{2.158791in}}{\pgfqpoint{4.024582in}{2.166691in}}{\pgfqpoint{4.018758in}{2.172515in}}%
\pgfpathcurveto{\pgfqpoint{4.012934in}{2.178339in}}{\pgfqpoint{4.005034in}{2.181611in}}{\pgfqpoint{3.996798in}{2.181611in}}%
\pgfpathcurveto{\pgfqpoint{3.988561in}{2.181611in}}{\pgfqpoint{3.980661in}{2.178339in}}{\pgfqpoint{3.974837in}{2.172515in}}%
\pgfpathcurveto{\pgfqpoint{3.969013in}{2.166691in}}{\pgfqpoint{3.965741in}{2.158791in}}{\pgfqpoint{3.965741in}{2.150555in}}%
\pgfpathcurveto{\pgfqpoint{3.965741in}{2.142319in}}{\pgfqpoint{3.969013in}{2.134419in}}{\pgfqpoint{3.974837in}{2.128595in}}%
\pgfpathcurveto{\pgfqpoint{3.980661in}{2.122771in}}{\pgfqpoint{3.988561in}{2.119498in}}{\pgfqpoint{3.996798in}{2.119498in}}%
\pgfpathclose%
\pgfusepath{stroke,fill}%
\end{pgfscope}%
\begin{pgfscope}%
\pgfpathrectangle{\pgfqpoint{3.793912in}{0.557870in}}{\pgfqpoint{2.446088in}{1.684734in}}%
\pgfusepath{clip}%
\pgfsetbuttcap%
\pgfsetroundjoin%
\definecolor{currentfill}{rgb}{0.298039,0.447059,0.690196}%
\pgfsetfillcolor{currentfill}%
\pgfsetlinewidth{1.003750pt}%
\definecolor{currentstroke}{rgb}{0.298039,0.447059,0.690196}%
\pgfsetstrokecolor{currentstroke}%
\pgfsetdash{}{0pt}%
\pgfpathmoveto{\pgfqpoint{3.905098in}{2.119498in}}%
\pgfpathcurveto{\pgfqpoint{3.913334in}{2.119498in}}{\pgfqpoint{3.921234in}{2.122771in}}{\pgfqpoint{3.927058in}{2.128595in}}%
\pgfpathcurveto{\pgfqpoint{3.932882in}{2.134419in}}{\pgfqpoint{3.936155in}{2.142319in}}{\pgfqpoint{3.936155in}{2.150555in}}%
\pgfpathcurveto{\pgfqpoint{3.936155in}{2.158791in}}{\pgfqpoint{3.932882in}{2.166691in}}{\pgfqpoint{3.927058in}{2.172515in}}%
\pgfpathcurveto{\pgfqpoint{3.921234in}{2.178339in}}{\pgfqpoint{3.913334in}{2.181611in}}{\pgfqpoint{3.905098in}{2.181611in}}%
\pgfpathcurveto{\pgfqpoint{3.896862in}{2.181611in}}{\pgfqpoint{3.888962in}{2.178339in}}{\pgfqpoint{3.883138in}{2.172515in}}%
\pgfpathcurveto{\pgfqpoint{3.877314in}{2.166691in}}{\pgfqpoint{3.874042in}{2.158791in}}{\pgfqpoint{3.874042in}{2.150555in}}%
\pgfpathcurveto{\pgfqpoint{3.874042in}{2.142319in}}{\pgfqpoint{3.877314in}{2.134419in}}{\pgfqpoint{3.883138in}{2.128595in}}%
\pgfpathcurveto{\pgfqpoint{3.888962in}{2.122771in}}{\pgfqpoint{3.896862in}{2.119498in}}{\pgfqpoint{3.905098in}{2.119498in}}%
\pgfpathclose%
\pgfusepath{stroke,fill}%
\end{pgfscope}%
\begin{pgfscope}%
\pgfpathrectangle{\pgfqpoint{3.793912in}{0.557870in}}{\pgfqpoint{2.446088in}{1.684734in}}%
\pgfusepath{clip}%
\pgfsetbuttcap%
\pgfsetroundjoin%
\definecolor{currentfill}{rgb}{0.298039,0.447059,0.690196}%
\pgfsetfillcolor{currentfill}%
\pgfsetlinewidth{1.003750pt}%
\definecolor{currentstroke}{rgb}{0.298039,0.447059,0.690196}%
\pgfsetstrokecolor{currentstroke}%
\pgfsetdash{}{0pt}%
\pgfpathmoveto{\pgfqpoint{3.905098in}{2.119498in}}%
\pgfpathcurveto{\pgfqpoint{3.913334in}{2.119498in}}{\pgfqpoint{3.921234in}{2.122771in}}{\pgfqpoint{3.927058in}{2.128595in}}%
\pgfpathcurveto{\pgfqpoint{3.932882in}{2.134419in}}{\pgfqpoint{3.936155in}{2.142319in}}{\pgfqpoint{3.936155in}{2.150555in}}%
\pgfpathcurveto{\pgfqpoint{3.936155in}{2.158791in}}{\pgfqpoint{3.932882in}{2.166691in}}{\pgfqpoint{3.927058in}{2.172515in}}%
\pgfpathcurveto{\pgfqpoint{3.921234in}{2.178339in}}{\pgfqpoint{3.913334in}{2.181611in}}{\pgfqpoint{3.905098in}{2.181611in}}%
\pgfpathcurveto{\pgfqpoint{3.896862in}{2.181611in}}{\pgfqpoint{3.888962in}{2.178339in}}{\pgfqpoint{3.883138in}{2.172515in}}%
\pgfpathcurveto{\pgfqpoint{3.877314in}{2.166691in}}{\pgfqpoint{3.874042in}{2.158791in}}{\pgfqpoint{3.874042in}{2.150555in}}%
\pgfpathcurveto{\pgfqpoint{3.874042in}{2.142319in}}{\pgfqpoint{3.877314in}{2.134419in}}{\pgfqpoint{3.883138in}{2.128595in}}%
\pgfpathcurveto{\pgfqpoint{3.888962in}{2.122771in}}{\pgfqpoint{3.896862in}{2.119498in}}{\pgfqpoint{3.905098in}{2.119498in}}%
\pgfpathclose%
\pgfusepath{stroke,fill}%
\end{pgfscope}%
\begin{pgfscope}%
\pgfpathrectangle{\pgfqpoint{3.793912in}{0.557870in}}{\pgfqpoint{2.446088in}{1.684734in}}%
\pgfusepath{clip}%
\pgfsetbuttcap%
\pgfsetroundjoin%
\definecolor{currentfill}{rgb}{0.298039,0.447059,0.690196}%
\pgfsetfillcolor{currentfill}%
\pgfsetlinewidth{1.003750pt}%
\definecolor{currentstroke}{rgb}{0.298039,0.447059,0.690196}%
\pgfsetstrokecolor{currentstroke}%
\pgfsetdash{}{0pt}%
\pgfpathmoveto{\pgfqpoint{4.867944in}{1.531621in}}%
\pgfpathcurveto{\pgfqpoint{4.876180in}{1.531621in}}{\pgfqpoint{4.884081in}{1.534893in}}{\pgfqpoint{4.889904in}{1.540717in}}%
\pgfpathcurveto{\pgfqpoint{4.895728in}{1.546541in}}{\pgfqpoint{4.899001in}{1.554441in}}{\pgfqpoint{4.899001in}{1.562677in}}%
\pgfpathcurveto{\pgfqpoint{4.899001in}{1.570913in}}{\pgfqpoint{4.895728in}{1.578813in}}{\pgfqpoint{4.889904in}{1.584637in}}%
\pgfpathcurveto{\pgfqpoint{4.884081in}{1.590461in}}{\pgfqpoint{4.876180in}{1.593734in}}{\pgfqpoint{4.867944in}{1.593734in}}%
\pgfpathcurveto{\pgfqpoint{4.859708in}{1.593734in}}{\pgfqpoint{4.851808in}{1.590461in}}{\pgfqpoint{4.845984in}{1.584637in}}%
\pgfpathcurveto{\pgfqpoint{4.840160in}{1.578813in}}{\pgfqpoint{4.836888in}{1.570913in}}{\pgfqpoint{4.836888in}{1.562677in}}%
\pgfpathcurveto{\pgfqpoint{4.836888in}{1.554441in}}{\pgfqpoint{4.840160in}{1.546541in}}{\pgfqpoint{4.845984in}{1.540717in}}%
\pgfpathcurveto{\pgfqpoint{4.851808in}{1.534893in}}{\pgfqpoint{4.859708in}{1.531621in}}{\pgfqpoint{4.867944in}{1.531621in}}%
\pgfpathclose%
\pgfusepath{stroke,fill}%
\end{pgfscope}%
\begin{pgfscope}%
\pgfpathrectangle{\pgfqpoint{3.793912in}{0.557870in}}{\pgfqpoint{2.446088in}{1.684734in}}%
\pgfusepath{clip}%
\pgfsetbuttcap%
\pgfsetroundjoin%
\definecolor{currentfill}{rgb}{0.298039,0.447059,0.690196}%
\pgfsetfillcolor{currentfill}%
\pgfsetlinewidth{1.003750pt}%
\definecolor{currentstroke}{rgb}{0.298039,0.447059,0.690196}%
\pgfsetstrokecolor{currentstroke}%
\pgfsetdash{}{0pt}%
\pgfpathmoveto{\pgfqpoint{5.509842in}{1.067507in}}%
\pgfpathcurveto{\pgfqpoint{5.518078in}{1.067507in}}{\pgfqpoint{5.525978in}{1.070779in}}{\pgfqpoint{5.531802in}{1.076603in}}%
\pgfpathcurveto{\pgfqpoint{5.537626in}{1.082427in}}{\pgfqpoint{5.540898in}{1.090327in}}{\pgfqpoint{5.540898in}{1.098563in}}%
\pgfpathcurveto{\pgfqpoint{5.540898in}{1.106799in}}{\pgfqpoint{5.537626in}{1.114699in}}{\pgfqpoint{5.531802in}{1.120523in}}%
\pgfpathcurveto{\pgfqpoint{5.525978in}{1.126347in}}{\pgfqpoint{5.518078in}{1.129620in}}{\pgfqpoint{5.509842in}{1.129620in}}%
\pgfpathcurveto{\pgfqpoint{5.501605in}{1.129620in}}{\pgfqpoint{5.493705in}{1.126347in}}{\pgfqpoint{5.487881in}{1.120523in}}%
\pgfpathcurveto{\pgfqpoint{5.482057in}{1.114699in}}{\pgfqpoint{5.478785in}{1.106799in}}{\pgfqpoint{5.478785in}{1.098563in}}%
\pgfpathcurveto{\pgfqpoint{5.478785in}{1.090327in}}{\pgfqpoint{5.482057in}{1.082427in}}{\pgfqpoint{5.487881in}{1.076603in}}%
\pgfpathcurveto{\pgfqpoint{5.493705in}{1.070779in}}{\pgfqpoint{5.501605in}{1.067507in}}{\pgfqpoint{5.509842in}{1.067507in}}%
\pgfpathclose%
\pgfusepath{stroke,fill}%
\end{pgfscope}%
\begin{pgfscope}%
\pgfpathrectangle{\pgfqpoint{3.793912in}{0.557870in}}{\pgfqpoint{2.446088in}{1.684734in}}%
\pgfusepath{clip}%
\pgfsetbuttcap%
\pgfsetroundjoin%
\definecolor{currentfill}{rgb}{0.298039,0.447059,0.690196}%
\pgfsetfillcolor{currentfill}%
\pgfsetlinewidth{1.003750pt}%
\definecolor{currentstroke}{rgb}{0.298039,0.447059,0.690196}%
\pgfsetstrokecolor{currentstroke}%
\pgfsetdash{}{0pt}%
\pgfpathmoveto{\pgfqpoint{3.905098in}{2.119498in}}%
\pgfpathcurveto{\pgfqpoint{3.913334in}{2.119498in}}{\pgfqpoint{3.921234in}{2.122771in}}{\pgfqpoint{3.927058in}{2.128595in}}%
\pgfpathcurveto{\pgfqpoint{3.932882in}{2.134419in}}{\pgfqpoint{3.936155in}{2.142319in}}{\pgfqpoint{3.936155in}{2.150555in}}%
\pgfpathcurveto{\pgfqpoint{3.936155in}{2.158791in}}{\pgfqpoint{3.932882in}{2.166691in}}{\pgfqpoint{3.927058in}{2.172515in}}%
\pgfpathcurveto{\pgfqpoint{3.921234in}{2.178339in}}{\pgfqpoint{3.913334in}{2.181611in}}{\pgfqpoint{3.905098in}{2.181611in}}%
\pgfpathcurveto{\pgfqpoint{3.896862in}{2.181611in}}{\pgfqpoint{3.888962in}{2.178339in}}{\pgfqpoint{3.883138in}{2.172515in}}%
\pgfpathcurveto{\pgfqpoint{3.877314in}{2.166691in}}{\pgfqpoint{3.874042in}{2.158791in}}{\pgfqpoint{3.874042in}{2.150555in}}%
\pgfpathcurveto{\pgfqpoint{3.874042in}{2.142319in}}{\pgfqpoint{3.877314in}{2.134419in}}{\pgfqpoint{3.883138in}{2.128595in}}%
\pgfpathcurveto{\pgfqpoint{3.888962in}{2.122771in}}{\pgfqpoint{3.896862in}{2.119498in}}{\pgfqpoint{3.905098in}{2.119498in}}%
\pgfpathclose%
\pgfusepath{stroke,fill}%
\end{pgfscope}%
\begin{pgfscope}%
\pgfpathrectangle{\pgfqpoint{3.793912in}{0.557870in}}{\pgfqpoint{2.446088in}{1.684734in}}%
\pgfusepath{clip}%
\pgfsetbuttcap%
\pgfsetroundjoin%
\definecolor{currentfill}{rgb}{0.298039,0.447059,0.690196}%
\pgfsetfillcolor{currentfill}%
\pgfsetlinewidth{1.003750pt}%
\definecolor{currentstroke}{rgb}{0.298039,0.447059,0.690196}%
\pgfsetstrokecolor{currentstroke}%
\pgfsetdash{}{0pt}%
\pgfpathmoveto{\pgfqpoint{3.905098in}{2.119498in}}%
\pgfpathcurveto{\pgfqpoint{3.913334in}{2.119498in}}{\pgfqpoint{3.921234in}{2.122771in}}{\pgfqpoint{3.927058in}{2.128595in}}%
\pgfpathcurveto{\pgfqpoint{3.932882in}{2.134419in}}{\pgfqpoint{3.936155in}{2.142319in}}{\pgfqpoint{3.936155in}{2.150555in}}%
\pgfpathcurveto{\pgfqpoint{3.936155in}{2.158791in}}{\pgfqpoint{3.932882in}{2.166691in}}{\pgfqpoint{3.927058in}{2.172515in}}%
\pgfpathcurveto{\pgfqpoint{3.921234in}{2.178339in}}{\pgfqpoint{3.913334in}{2.181611in}}{\pgfqpoint{3.905098in}{2.181611in}}%
\pgfpathcurveto{\pgfqpoint{3.896862in}{2.181611in}}{\pgfqpoint{3.888962in}{2.178339in}}{\pgfqpoint{3.883138in}{2.172515in}}%
\pgfpathcurveto{\pgfqpoint{3.877314in}{2.166691in}}{\pgfqpoint{3.874042in}{2.158791in}}{\pgfqpoint{3.874042in}{2.150555in}}%
\pgfpathcurveto{\pgfqpoint{3.874042in}{2.142319in}}{\pgfqpoint{3.877314in}{2.134419in}}{\pgfqpoint{3.883138in}{2.128595in}}%
\pgfpathcurveto{\pgfqpoint{3.888962in}{2.122771in}}{\pgfqpoint{3.896862in}{2.119498in}}{\pgfqpoint{3.905098in}{2.119498in}}%
\pgfpathclose%
\pgfusepath{stroke,fill}%
\end{pgfscope}%
\begin{pgfscope}%
\pgfpathrectangle{\pgfqpoint{3.793912in}{0.557870in}}{\pgfqpoint{2.446088in}{1.684734in}}%
\pgfusepath{clip}%
\pgfsetbuttcap%
\pgfsetroundjoin%
\definecolor{currentfill}{rgb}{0.298039,0.447059,0.690196}%
\pgfsetfillcolor{currentfill}%
\pgfsetlinewidth{1.003750pt}%
\definecolor{currentstroke}{rgb}{0.298039,0.447059,0.690196}%
\pgfsetstrokecolor{currentstroke}%
\pgfsetdash{}{0pt}%
\pgfpathmoveto{\pgfqpoint{5.601541in}{1.500680in}}%
\pgfpathcurveto{\pgfqpoint{5.609778in}{1.500680in}}{\pgfqpoint{5.617678in}{1.503952in}}{\pgfqpoint{5.623502in}{1.509776in}}%
\pgfpathcurveto{\pgfqpoint{5.629325in}{1.515600in}}{\pgfqpoint{5.632598in}{1.523500in}}{\pgfqpoint{5.632598in}{1.531736in}}%
\pgfpathcurveto{\pgfqpoint{5.632598in}{1.539972in}}{\pgfqpoint{5.629325in}{1.547873in}}{\pgfqpoint{5.623502in}{1.553696in}}%
\pgfpathcurveto{\pgfqpoint{5.617678in}{1.559520in}}{\pgfqpoint{5.609778in}{1.562793in}}{\pgfqpoint{5.601541in}{1.562793in}}%
\pgfpathcurveto{\pgfqpoint{5.593305in}{1.562793in}}{\pgfqpoint{5.585405in}{1.559520in}}{\pgfqpoint{5.579581in}{1.553696in}}%
\pgfpathcurveto{\pgfqpoint{5.573757in}{1.547873in}}{\pgfqpoint{5.570485in}{1.539972in}}{\pgfqpoint{5.570485in}{1.531736in}}%
\pgfpathcurveto{\pgfqpoint{5.570485in}{1.523500in}}{\pgfqpoint{5.573757in}{1.515600in}}{\pgfqpoint{5.579581in}{1.509776in}}%
\pgfpathcurveto{\pgfqpoint{5.585405in}{1.503952in}}{\pgfqpoint{5.593305in}{1.500680in}}{\pgfqpoint{5.601541in}{1.500680in}}%
\pgfpathclose%
\pgfusepath{stroke,fill}%
\end{pgfscope}%
\begin{pgfscope}%
\pgfpathrectangle{\pgfqpoint{3.793912in}{0.557870in}}{\pgfqpoint{2.446088in}{1.684734in}}%
\pgfusepath{clip}%
\pgfsetbuttcap%
\pgfsetroundjoin%
\definecolor{currentfill}{rgb}{0.298039,0.447059,0.690196}%
\pgfsetfillcolor{currentfill}%
\pgfsetlinewidth{1.003750pt}%
\definecolor{currentstroke}{rgb}{0.298039,0.447059,0.690196}%
\pgfsetstrokecolor{currentstroke}%
\pgfsetdash{}{0pt}%
\pgfpathmoveto{\pgfqpoint{5.280593in}{1.871971in}}%
\pgfpathcurveto{\pgfqpoint{5.288829in}{1.871971in}}{\pgfqpoint{5.296729in}{1.875243in}}{\pgfqpoint{5.302553in}{1.881067in}}%
\pgfpathcurveto{\pgfqpoint{5.308377in}{1.886891in}}{\pgfqpoint{5.311649in}{1.894791in}}{\pgfqpoint{5.311649in}{1.903027in}}%
\pgfpathcurveto{\pgfqpoint{5.311649in}{1.911264in}}{\pgfqpoint{5.308377in}{1.919164in}}{\pgfqpoint{5.302553in}{1.924988in}}%
\pgfpathcurveto{\pgfqpoint{5.296729in}{1.930812in}}{\pgfqpoint{5.288829in}{1.934084in}}{\pgfqpoint{5.280593in}{1.934084in}}%
\pgfpathcurveto{\pgfqpoint{5.272356in}{1.934084in}}{\pgfqpoint{5.264456in}{1.930812in}}{\pgfqpoint{5.258632in}{1.924988in}}%
\pgfpathcurveto{\pgfqpoint{5.252808in}{1.919164in}}{\pgfqpoint{5.249536in}{1.911264in}}{\pgfqpoint{5.249536in}{1.903027in}}%
\pgfpathcurveto{\pgfqpoint{5.249536in}{1.894791in}}{\pgfqpoint{5.252808in}{1.886891in}}{\pgfqpoint{5.258632in}{1.881067in}}%
\pgfpathcurveto{\pgfqpoint{5.264456in}{1.875243in}}{\pgfqpoint{5.272356in}{1.871971in}}{\pgfqpoint{5.280593in}{1.871971in}}%
\pgfpathclose%
\pgfusepath{stroke,fill}%
\end{pgfscope}%
\begin{pgfscope}%
\pgfpathrectangle{\pgfqpoint{3.793912in}{0.557870in}}{\pgfqpoint{2.446088in}{1.684734in}}%
\pgfusepath{clip}%
\pgfsetbuttcap%
\pgfsetroundjoin%
\definecolor{currentfill}{rgb}{0.298039,0.447059,0.690196}%
\pgfsetfillcolor{currentfill}%
\pgfsetlinewidth{1.003750pt}%
\definecolor{currentstroke}{rgb}{0.298039,0.447059,0.690196}%
\pgfsetstrokecolor{currentstroke}%
\pgfsetdash{}{0pt}%
\pgfpathmoveto{\pgfqpoint{5.624466in}{1.516150in}}%
\pgfpathcurveto{\pgfqpoint{5.632702in}{1.516150in}}{\pgfqpoint{5.640603in}{1.519422in}}{\pgfqpoint{5.646426in}{1.525246in}}%
\pgfpathcurveto{\pgfqpoint{5.652250in}{1.531070in}}{\pgfqpoint{5.655523in}{1.538970in}}{\pgfqpoint{5.655523in}{1.547207in}}%
\pgfpathcurveto{\pgfqpoint{5.655523in}{1.555443in}}{\pgfqpoint{5.652250in}{1.563343in}}{\pgfqpoint{5.646426in}{1.569167in}}%
\pgfpathcurveto{\pgfqpoint{5.640603in}{1.574991in}}{\pgfqpoint{5.632702in}{1.578263in}}{\pgfqpoint{5.624466in}{1.578263in}}%
\pgfpathcurveto{\pgfqpoint{5.616230in}{1.578263in}}{\pgfqpoint{5.608330in}{1.574991in}}{\pgfqpoint{5.602506in}{1.569167in}}%
\pgfpathcurveto{\pgfqpoint{5.596682in}{1.563343in}}{\pgfqpoint{5.593410in}{1.555443in}}{\pgfqpoint{5.593410in}{1.547207in}}%
\pgfpathcurveto{\pgfqpoint{5.593410in}{1.538970in}}{\pgfqpoint{5.596682in}{1.531070in}}{\pgfqpoint{5.602506in}{1.525246in}}%
\pgfpathcurveto{\pgfqpoint{5.608330in}{1.519422in}}{\pgfqpoint{5.616230in}{1.516150in}}{\pgfqpoint{5.624466in}{1.516150in}}%
\pgfpathclose%
\pgfusepath{stroke,fill}%
\end{pgfscope}%
\begin{pgfscope}%
\pgfpathrectangle{\pgfqpoint{3.793912in}{0.557870in}}{\pgfqpoint{2.446088in}{1.684734in}}%
\pgfusepath{clip}%
\pgfsetbuttcap%
\pgfsetroundjoin%
\definecolor{currentfill}{rgb}{0.298039,0.447059,0.690196}%
\pgfsetfillcolor{currentfill}%
\pgfsetlinewidth{1.003750pt}%
\definecolor{currentstroke}{rgb}{0.298039,0.447059,0.690196}%
\pgfsetstrokecolor{currentstroke}%
\pgfsetdash{}{0pt}%
\pgfpathmoveto{\pgfqpoint{4.936719in}{1.995735in}}%
\pgfpathcurveto{\pgfqpoint{4.944955in}{1.995735in}}{\pgfqpoint{4.952855in}{1.999007in}}{\pgfqpoint{4.958679in}{2.004831in}}%
\pgfpathcurveto{\pgfqpoint{4.964503in}{2.010655in}}{\pgfqpoint{4.967775in}{2.018555in}}{\pgfqpoint{4.967775in}{2.026791in}}%
\pgfpathcurveto{\pgfqpoint{4.967775in}{2.035027in}}{\pgfqpoint{4.964503in}{2.042927in}}{\pgfqpoint{4.958679in}{2.048751in}}%
\pgfpathcurveto{\pgfqpoint{4.952855in}{2.054575in}}{\pgfqpoint{4.944955in}{2.057848in}}{\pgfqpoint{4.936719in}{2.057848in}}%
\pgfpathcurveto{\pgfqpoint{4.928483in}{2.057848in}}{\pgfqpoint{4.920583in}{2.054575in}}{\pgfqpoint{4.914759in}{2.048751in}}%
\pgfpathcurveto{\pgfqpoint{4.908935in}{2.042927in}}{\pgfqpoint{4.905662in}{2.035027in}}{\pgfqpoint{4.905662in}{2.026791in}}%
\pgfpathcurveto{\pgfqpoint{4.905662in}{2.018555in}}{\pgfqpoint{4.908935in}{2.010655in}}{\pgfqpoint{4.914759in}{2.004831in}}%
\pgfpathcurveto{\pgfqpoint{4.920583in}{1.999007in}}{\pgfqpoint{4.928483in}{1.995735in}}{\pgfqpoint{4.936719in}{1.995735in}}%
\pgfpathclose%
\pgfusepath{stroke,fill}%
\end{pgfscope}%
\begin{pgfscope}%
\pgfpathrectangle{\pgfqpoint{3.793912in}{0.557870in}}{\pgfqpoint{2.446088in}{1.684734in}}%
\pgfusepath{clip}%
\pgfsetbuttcap%
\pgfsetroundjoin%
\definecolor{currentfill}{rgb}{0.298039,0.447059,0.690196}%
\pgfsetfillcolor{currentfill}%
\pgfsetlinewidth{1.003750pt}%
\definecolor{currentstroke}{rgb}{0.298039,0.447059,0.690196}%
\pgfsetstrokecolor{currentstroke}%
\pgfsetdash{}{0pt}%
\pgfpathmoveto{\pgfqpoint{3.905098in}{2.119498in}}%
\pgfpathcurveto{\pgfqpoint{3.913334in}{2.119498in}}{\pgfqpoint{3.921234in}{2.122771in}}{\pgfqpoint{3.927058in}{2.128595in}}%
\pgfpathcurveto{\pgfqpoint{3.932882in}{2.134419in}}{\pgfqpoint{3.936155in}{2.142319in}}{\pgfqpoint{3.936155in}{2.150555in}}%
\pgfpathcurveto{\pgfqpoint{3.936155in}{2.158791in}}{\pgfqpoint{3.932882in}{2.166691in}}{\pgfqpoint{3.927058in}{2.172515in}}%
\pgfpathcurveto{\pgfqpoint{3.921234in}{2.178339in}}{\pgfqpoint{3.913334in}{2.181611in}}{\pgfqpoint{3.905098in}{2.181611in}}%
\pgfpathcurveto{\pgfqpoint{3.896862in}{2.181611in}}{\pgfqpoint{3.888962in}{2.178339in}}{\pgfqpoint{3.883138in}{2.172515in}}%
\pgfpathcurveto{\pgfqpoint{3.877314in}{2.166691in}}{\pgfqpoint{3.874042in}{2.158791in}}{\pgfqpoint{3.874042in}{2.150555in}}%
\pgfpathcurveto{\pgfqpoint{3.874042in}{2.142319in}}{\pgfqpoint{3.877314in}{2.134419in}}{\pgfqpoint{3.883138in}{2.128595in}}%
\pgfpathcurveto{\pgfqpoint{3.888962in}{2.122771in}}{\pgfqpoint{3.896862in}{2.119498in}}{\pgfqpoint{3.905098in}{2.119498in}}%
\pgfpathclose%
\pgfusepath{stroke,fill}%
\end{pgfscope}%
\begin{pgfscope}%
\pgfpathrectangle{\pgfqpoint{3.793912in}{0.557870in}}{\pgfqpoint{2.446088in}{1.684734in}}%
\pgfusepath{clip}%
\pgfsetbuttcap%
\pgfsetroundjoin%
\definecolor{currentfill}{rgb}{0.298039,0.447059,0.690196}%
\pgfsetfillcolor{currentfill}%
\pgfsetlinewidth{1.003750pt}%
\definecolor{currentstroke}{rgb}{0.298039,0.447059,0.690196}%
\pgfsetstrokecolor{currentstroke}%
\pgfsetdash{}{0pt}%
\pgfpathmoveto{\pgfqpoint{5.624466in}{1.531621in}}%
\pgfpathcurveto{\pgfqpoint{5.632702in}{1.531621in}}{\pgfqpoint{5.640603in}{1.534893in}}{\pgfqpoint{5.646426in}{1.540717in}}%
\pgfpathcurveto{\pgfqpoint{5.652250in}{1.546541in}}{\pgfqpoint{5.655523in}{1.554441in}}{\pgfqpoint{5.655523in}{1.562677in}}%
\pgfpathcurveto{\pgfqpoint{5.655523in}{1.570913in}}{\pgfqpoint{5.652250in}{1.578813in}}{\pgfqpoint{5.646426in}{1.584637in}}%
\pgfpathcurveto{\pgfqpoint{5.640603in}{1.590461in}}{\pgfqpoint{5.632702in}{1.593734in}}{\pgfqpoint{5.624466in}{1.593734in}}%
\pgfpathcurveto{\pgfqpoint{5.616230in}{1.593734in}}{\pgfqpoint{5.608330in}{1.590461in}}{\pgfqpoint{5.602506in}{1.584637in}}%
\pgfpathcurveto{\pgfqpoint{5.596682in}{1.578813in}}{\pgfqpoint{5.593410in}{1.570913in}}{\pgfqpoint{5.593410in}{1.562677in}}%
\pgfpathcurveto{\pgfqpoint{5.593410in}{1.554441in}}{\pgfqpoint{5.596682in}{1.546541in}}{\pgfqpoint{5.602506in}{1.540717in}}%
\pgfpathcurveto{\pgfqpoint{5.608330in}{1.534893in}}{\pgfqpoint{5.616230in}{1.531621in}}{\pgfqpoint{5.624466in}{1.531621in}}%
\pgfpathclose%
\pgfusepath{stroke,fill}%
\end{pgfscope}%
\begin{pgfscope}%
\pgfpathrectangle{\pgfqpoint{3.793912in}{0.557870in}}{\pgfqpoint{2.446088in}{1.684734in}}%
\pgfusepath{clip}%
\pgfsetbuttcap%
\pgfsetroundjoin%
\definecolor{currentfill}{rgb}{0.298039,0.447059,0.690196}%
\pgfsetfillcolor{currentfill}%
\pgfsetlinewidth{1.003750pt}%
\definecolor{currentstroke}{rgb}{0.298039,0.447059,0.690196}%
\pgfsetstrokecolor{currentstroke}%
\pgfsetdash{}{0pt}%
\pgfpathmoveto{\pgfqpoint{5.624466in}{1.500680in}}%
\pgfpathcurveto{\pgfqpoint{5.632702in}{1.500680in}}{\pgfqpoint{5.640603in}{1.503952in}}{\pgfqpoint{5.646426in}{1.509776in}}%
\pgfpathcurveto{\pgfqpoint{5.652250in}{1.515600in}}{\pgfqpoint{5.655523in}{1.523500in}}{\pgfqpoint{5.655523in}{1.531736in}}%
\pgfpathcurveto{\pgfqpoint{5.655523in}{1.539972in}}{\pgfqpoint{5.652250in}{1.547873in}}{\pgfqpoint{5.646426in}{1.553696in}}%
\pgfpathcurveto{\pgfqpoint{5.640603in}{1.559520in}}{\pgfqpoint{5.632702in}{1.562793in}}{\pgfqpoint{5.624466in}{1.562793in}}%
\pgfpathcurveto{\pgfqpoint{5.616230in}{1.562793in}}{\pgfqpoint{5.608330in}{1.559520in}}{\pgfqpoint{5.602506in}{1.553696in}}%
\pgfpathcurveto{\pgfqpoint{5.596682in}{1.547873in}}{\pgfqpoint{5.593410in}{1.539972in}}{\pgfqpoint{5.593410in}{1.531736in}}%
\pgfpathcurveto{\pgfqpoint{5.593410in}{1.523500in}}{\pgfqpoint{5.596682in}{1.515600in}}{\pgfqpoint{5.602506in}{1.509776in}}%
\pgfpathcurveto{\pgfqpoint{5.608330in}{1.503952in}}{\pgfqpoint{5.616230in}{1.500680in}}{\pgfqpoint{5.624466in}{1.500680in}}%
\pgfpathclose%
\pgfusepath{stroke,fill}%
\end{pgfscope}%
\begin{pgfscope}%
\pgfpathrectangle{\pgfqpoint{3.793912in}{0.557870in}}{\pgfqpoint{2.446088in}{1.684734in}}%
\pgfusepath{clip}%
\pgfsetbuttcap%
\pgfsetroundjoin%
\definecolor{currentfill}{rgb}{0.298039,0.447059,0.690196}%
\pgfsetfillcolor{currentfill}%
\pgfsetlinewidth{1.003750pt}%
\definecolor{currentstroke}{rgb}{0.298039,0.447059,0.690196}%
\pgfsetstrokecolor{currentstroke}%
\pgfsetdash{}{0pt}%
\pgfpathmoveto{\pgfqpoint{5.601541in}{1.547091in}}%
\pgfpathcurveto{\pgfqpoint{5.609778in}{1.547091in}}{\pgfqpoint{5.617678in}{1.550363in}}{\pgfqpoint{5.623502in}{1.556187in}}%
\pgfpathcurveto{\pgfqpoint{5.629325in}{1.562011in}}{\pgfqpoint{5.632598in}{1.569911in}}{\pgfqpoint{5.632598in}{1.578148in}}%
\pgfpathcurveto{\pgfqpoint{5.632598in}{1.586384in}}{\pgfqpoint{5.629325in}{1.594284in}}{\pgfqpoint{5.623502in}{1.600108in}}%
\pgfpathcurveto{\pgfqpoint{5.617678in}{1.605932in}}{\pgfqpoint{5.609778in}{1.609204in}}{\pgfqpoint{5.601541in}{1.609204in}}%
\pgfpathcurveto{\pgfqpoint{5.593305in}{1.609204in}}{\pgfqpoint{5.585405in}{1.605932in}}{\pgfqpoint{5.579581in}{1.600108in}}%
\pgfpathcurveto{\pgfqpoint{5.573757in}{1.594284in}}{\pgfqpoint{5.570485in}{1.586384in}}{\pgfqpoint{5.570485in}{1.578148in}}%
\pgfpathcurveto{\pgfqpoint{5.570485in}{1.569911in}}{\pgfqpoint{5.573757in}{1.562011in}}{\pgfqpoint{5.579581in}{1.556187in}}%
\pgfpathcurveto{\pgfqpoint{5.585405in}{1.550363in}}{\pgfqpoint{5.593305in}{1.547091in}}{\pgfqpoint{5.601541in}{1.547091in}}%
\pgfpathclose%
\pgfusepath{stroke,fill}%
\end{pgfscope}%
\begin{pgfscope}%
\pgfpathrectangle{\pgfqpoint{3.793912in}{0.557870in}}{\pgfqpoint{2.446088in}{1.684734in}}%
\pgfusepath{clip}%
\pgfsetbuttcap%
\pgfsetroundjoin%
\definecolor{currentfill}{rgb}{0.298039,0.447059,0.690196}%
\pgfsetfillcolor{currentfill}%
\pgfsetlinewidth{1.003750pt}%
\definecolor{currentstroke}{rgb}{0.298039,0.447059,0.690196}%
\pgfsetstrokecolor{currentstroke}%
\pgfsetdash{}{0pt}%
\pgfpathmoveto{\pgfqpoint{5.578616in}{1.562562in}}%
\pgfpathcurveto{\pgfqpoint{5.586853in}{1.562562in}}{\pgfqpoint{5.594753in}{1.565834in}}{\pgfqpoint{5.600577in}{1.571658in}}%
\pgfpathcurveto{\pgfqpoint{5.606401in}{1.577482in}}{\pgfqpoint{5.609673in}{1.585382in}}{\pgfqpoint{5.609673in}{1.593618in}}%
\pgfpathcurveto{\pgfqpoint{5.609673in}{1.601854in}}{\pgfqpoint{5.606401in}{1.609754in}}{\pgfqpoint{5.600577in}{1.615578in}}%
\pgfpathcurveto{\pgfqpoint{5.594753in}{1.621402in}}{\pgfqpoint{5.586853in}{1.624675in}}{\pgfqpoint{5.578616in}{1.624675in}}%
\pgfpathcurveto{\pgfqpoint{5.570380in}{1.624675in}}{\pgfqpoint{5.562480in}{1.621402in}}{\pgfqpoint{5.556656in}{1.615578in}}%
\pgfpathcurveto{\pgfqpoint{5.550832in}{1.609754in}}{\pgfqpoint{5.547560in}{1.601854in}}{\pgfqpoint{5.547560in}{1.593618in}}%
\pgfpathcurveto{\pgfqpoint{5.547560in}{1.585382in}}{\pgfqpoint{5.550832in}{1.577482in}}{\pgfqpoint{5.556656in}{1.571658in}}%
\pgfpathcurveto{\pgfqpoint{5.562480in}{1.565834in}}{\pgfqpoint{5.570380in}{1.562562in}}{\pgfqpoint{5.578616in}{1.562562in}}%
\pgfpathclose%
\pgfusepath{stroke,fill}%
\end{pgfscope}%
\begin{pgfscope}%
\pgfpathrectangle{\pgfqpoint{3.793912in}{0.557870in}}{\pgfqpoint{2.446088in}{1.684734in}}%
\pgfusepath{clip}%
\pgfsetbuttcap%
\pgfsetroundjoin%
\definecolor{currentfill}{rgb}{0.298039,0.447059,0.690196}%
\pgfsetfillcolor{currentfill}%
\pgfsetlinewidth{1.003750pt}%
\definecolor{currentstroke}{rgb}{0.298039,0.447059,0.690196}%
\pgfsetstrokecolor{currentstroke}%
\pgfsetdash{}{0pt}%
\pgfpathmoveto{\pgfqpoint{3.905098in}{2.119498in}}%
\pgfpathcurveto{\pgfqpoint{3.913334in}{2.119498in}}{\pgfqpoint{3.921234in}{2.122771in}}{\pgfqpoint{3.927058in}{2.128595in}}%
\pgfpathcurveto{\pgfqpoint{3.932882in}{2.134419in}}{\pgfqpoint{3.936155in}{2.142319in}}{\pgfqpoint{3.936155in}{2.150555in}}%
\pgfpathcurveto{\pgfqpoint{3.936155in}{2.158791in}}{\pgfqpoint{3.932882in}{2.166691in}}{\pgfqpoint{3.927058in}{2.172515in}}%
\pgfpathcurveto{\pgfqpoint{3.921234in}{2.178339in}}{\pgfqpoint{3.913334in}{2.181611in}}{\pgfqpoint{3.905098in}{2.181611in}}%
\pgfpathcurveto{\pgfqpoint{3.896862in}{2.181611in}}{\pgfqpoint{3.888962in}{2.178339in}}{\pgfqpoint{3.883138in}{2.172515in}}%
\pgfpathcurveto{\pgfqpoint{3.877314in}{2.166691in}}{\pgfqpoint{3.874042in}{2.158791in}}{\pgfqpoint{3.874042in}{2.150555in}}%
\pgfpathcurveto{\pgfqpoint{3.874042in}{2.142319in}}{\pgfqpoint{3.877314in}{2.134419in}}{\pgfqpoint{3.883138in}{2.128595in}}%
\pgfpathcurveto{\pgfqpoint{3.888962in}{2.122771in}}{\pgfqpoint{3.896862in}{2.119498in}}{\pgfqpoint{3.905098in}{2.119498in}}%
\pgfpathclose%
\pgfusepath{stroke,fill}%
\end{pgfscope}%
\begin{pgfscope}%
\pgfpathrectangle{\pgfqpoint{3.793912in}{0.557870in}}{\pgfqpoint{2.446088in}{1.684734in}}%
\pgfusepath{clip}%
\pgfsetbuttcap%
\pgfsetroundjoin%
\definecolor{currentfill}{rgb}{0.298039,0.447059,0.690196}%
\pgfsetfillcolor{currentfill}%
\pgfsetlinewidth{1.003750pt}%
\definecolor{currentstroke}{rgb}{0.298039,0.447059,0.690196}%
\pgfsetstrokecolor{currentstroke}%
\pgfsetdash{}{0pt}%
\pgfpathmoveto{\pgfqpoint{5.739091in}{1.315034in}}%
\pgfpathcurveto{\pgfqpoint{5.747327in}{1.315034in}}{\pgfqpoint{5.755227in}{1.318306in}}{\pgfqpoint{5.761051in}{1.324130in}}%
\pgfpathcurveto{\pgfqpoint{5.766875in}{1.329954in}}{\pgfqpoint{5.770147in}{1.337854in}}{\pgfqpoint{5.770147in}{1.346091in}}%
\pgfpathcurveto{\pgfqpoint{5.770147in}{1.354327in}}{\pgfqpoint{5.766875in}{1.362227in}}{\pgfqpoint{5.761051in}{1.368051in}}%
\pgfpathcurveto{\pgfqpoint{5.755227in}{1.373875in}}{\pgfqpoint{5.747327in}{1.377147in}}{\pgfqpoint{5.739091in}{1.377147in}}%
\pgfpathcurveto{\pgfqpoint{5.730854in}{1.377147in}}{\pgfqpoint{5.722954in}{1.373875in}}{\pgfqpoint{5.717130in}{1.368051in}}%
\pgfpathcurveto{\pgfqpoint{5.711307in}{1.362227in}}{\pgfqpoint{5.708034in}{1.354327in}}{\pgfqpoint{5.708034in}{1.346091in}}%
\pgfpathcurveto{\pgfqpoint{5.708034in}{1.337854in}}{\pgfqpoint{5.711307in}{1.329954in}}{\pgfqpoint{5.717130in}{1.324130in}}%
\pgfpathcurveto{\pgfqpoint{5.722954in}{1.318306in}}{\pgfqpoint{5.730854in}{1.315034in}}{\pgfqpoint{5.739091in}{1.315034in}}%
\pgfpathclose%
\pgfusepath{stroke,fill}%
\end{pgfscope}%
\begin{pgfscope}%
\pgfpathrectangle{\pgfqpoint{3.793912in}{0.557870in}}{\pgfqpoint{2.446088in}{1.684734in}}%
\pgfusepath{clip}%
\pgfsetbuttcap%
\pgfsetroundjoin%
\definecolor{currentfill}{rgb}{0.298039,0.447059,0.690196}%
\pgfsetfillcolor{currentfill}%
\pgfsetlinewidth{1.003750pt}%
\definecolor{currentstroke}{rgb}{0.298039,0.447059,0.690196}%
\pgfsetstrokecolor{currentstroke}%
\pgfsetdash{}{0pt}%
\pgfpathmoveto{\pgfqpoint{3.905098in}{2.119498in}}%
\pgfpathcurveto{\pgfqpoint{3.913334in}{2.119498in}}{\pgfqpoint{3.921234in}{2.122771in}}{\pgfqpoint{3.927058in}{2.128595in}}%
\pgfpathcurveto{\pgfqpoint{3.932882in}{2.134419in}}{\pgfqpoint{3.936155in}{2.142319in}}{\pgfqpoint{3.936155in}{2.150555in}}%
\pgfpathcurveto{\pgfqpoint{3.936155in}{2.158791in}}{\pgfqpoint{3.932882in}{2.166691in}}{\pgfqpoint{3.927058in}{2.172515in}}%
\pgfpathcurveto{\pgfqpoint{3.921234in}{2.178339in}}{\pgfqpoint{3.913334in}{2.181611in}}{\pgfqpoint{3.905098in}{2.181611in}}%
\pgfpathcurveto{\pgfqpoint{3.896862in}{2.181611in}}{\pgfqpoint{3.888962in}{2.178339in}}{\pgfqpoint{3.883138in}{2.172515in}}%
\pgfpathcurveto{\pgfqpoint{3.877314in}{2.166691in}}{\pgfqpoint{3.874042in}{2.158791in}}{\pgfqpoint{3.874042in}{2.150555in}}%
\pgfpathcurveto{\pgfqpoint{3.874042in}{2.142319in}}{\pgfqpoint{3.877314in}{2.134419in}}{\pgfqpoint{3.883138in}{2.128595in}}%
\pgfpathcurveto{\pgfqpoint{3.888962in}{2.122771in}}{\pgfqpoint{3.896862in}{2.119498in}}{\pgfqpoint{3.905098in}{2.119498in}}%
\pgfpathclose%
\pgfusepath{stroke,fill}%
\end{pgfscope}%
\begin{pgfscope}%
\pgfpathrectangle{\pgfqpoint{3.793912in}{0.557870in}}{\pgfqpoint{2.446088in}{1.684734in}}%
\pgfusepath{clip}%
\pgfsetbuttcap%
\pgfsetroundjoin%
\definecolor{currentfill}{rgb}{0.298039,0.447059,0.690196}%
\pgfsetfillcolor{currentfill}%
\pgfsetlinewidth{1.003750pt}%
\definecolor{currentstroke}{rgb}{0.298039,0.447059,0.690196}%
\pgfsetstrokecolor{currentstroke}%
\pgfsetdash{}{0pt}%
\pgfpathmoveto{\pgfqpoint{5.853715in}{0.835450in}}%
\pgfpathcurveto{\pgfqpoint{5.861952in}{0.835450in}}{\pgfqpoint{5.869852in}{0.838722in}}{\pgfqpoint{5.875676in}{0.844546in}}%
\pgfpathcurveto{\pgfqpoint{5.881499in}{0.850370in}}{\pgfqpoint{5.884772in}{0.858270in}}{\pgfqpoint{5.884772in}{0.866506in}}%
\pgfpathcurveto{\pgfqpoint{5.884772in}{0.874742in}}{\pgfqpoint{5.881499in}{0.882642in}}{\pgfqpoint{5.875676in}{0.888466in}}%
\pgfpathcurveto{\pgfqpoint{5.869852in}{0.894290in}}{\pgfqpoint{5.861952in}{0.897563in}}{\pgfqpoint{5.853715in}{0.897563in}}%
\pgfpathcurveto{\pgfqpoint{5.845479in}{0.897563in}}{\pgfqpoint{5.837579in}{0.894290in}}{\pgfqpoint{5.831755in}{0.888466in}}%
\pgfpathcurveto{\pgfqpoint{5.825931in}{0.882642in}}{\pgfqpoint{5.822659in}{0.874742in}}{\pgfqpoint{5.822659in}{0.866506in}}%
\pgfpathcurveto{\pgfqpoint{5.822659in}{0.858270in}}{\pgfqpoint{5.825931in}{0.850370in}}{\pgfqpoint{5.831755in}{0.844546in}}%
\pgfpathcurveto{\pgfqpoint{5.837579in}{0.838722in}}{\pgfqpoint{5.845479in}{0.835450in}}{\pgfqpoint{5.853715in}{0.835450in}}%
\pgfpathclose%
\pgfusepath{stroke,fill}%
\end{pgfscope}%
\begin{pgfscope}%
\pgfpathrectangle{\pgfqpoint{3.793912in}{0.557870in}}{\pgfqpoint{2.446088in}{1.684734in}}%
\pgfusepath{clip}%
\pgfsetbuttcap%
\pgfsetroundjoin%
\definecolor{currentfill}{rgb}{0.298039,0.447059,0.690196}%
\pgfsetfillcolor{currentfill}%
\pgfsetlinewidth{1.003750pt}%
\definecolor{currentstroke}{rgb}{0.298039,0.447059,0.690196}%
\pgfsetstrokecolor{currentstroke}%
\pgfsetdash{}{0pt}%
\pgfpathmoveto{\pgfqpoint{5.853715in}{1.021095in}}%
\pgfpathcurveto{\pgfqpoint{5.861952in}{1.021095in}}{\pgfqpoint{5.869852in}{1.024368in}}{\pgfqpoint{5.875676in}{1.030191in}}%
\pgfpathcurveto{\pgfqpoint{5.881499in}{1.036015in}}{\pgfqpoint{5.884772in}{1.043915in}}{\pgfqpoint{5.884772in}{1.052152in}}%
\pgfpathcurveto{\pgfqpoint{5.884772in}{1.060388in}}{\pgfqpoint{5.881499in}{1.068288in}}{\pgfqpoint{5.875676in}{1.074112in}}%
\pgfpathcurveto{\pgfqpoint{5.869852in}{1.079936in}}{\pgfqpoint{5.861952in}{1.083208in}}{\pgfqpoint{5.853715in}{1.083208in}}%
\pgfpathcurveto{\pgfqpoint{5.845479in}{1.083208in}}{\pgfqpoint{5.837579in}{1.079936in}}{\pgfqpoint{5.831755in}{1.074112in}}%
\pgfpathcurveto{\pgfqpoint{5.825931in}{1.068288in}}{\pgfqpoint{5.822659in}{1.060388in}}{\pgfqpoint{5.822659in}{1.052152in}}%
\pgfpathcurveto{\pgfqpoint{5.822659in}{1.043915in}}{\pgfqpoint{5.825931in}{1.036015in}}{\pgfqpoint{5.831755in}{1.030191in}}%
\pgfpathcurveto{\pgfqpoint{5.837579in}{1.024368in}}{\pgfqpoint{5.845479in}{1.021095in}}{\pgfqpoint{5.853715in}{1.021095in}}%
\pgfpathclose%
\pgfusepath{stroke,fill}%
\end{pgfscope}%
\begin{pgfscope}%
\pgfpathrectangle{\pgfqpoint{3.793912in}{0.557870in}}{\pgfqpoint{2.446088in}{1.684734in}}%
\pgfusepath{clip}%
\pgfsetbuttcap%
\pgfsetroundjoin%
\definecolor{currentfill}{rgb}{0.298039,0.447059,0.690196}%
\pgfsetfillcolor{currentfill}%
\pgfsetlinewidth{1.003750pt}%
\definecolor{currentstroke}{rgb}{0.298039,0.447059,0.690196}%
\pgfsetstrokecolor{currentstroke}%
\pgfsetdash{}{0pt}%
\pgfpathmoveto{\pgfqpoint{3.905098in}{2.119498in}}%
\pgfpathcurveto{\pgfqpoint{3.913334in}{2.119498in}}{\pgfqpoint{3.921234in}{2.122771in}}{\pgfqpoint{3.927058in}{2.128595in}}%
\pgfpathcurveto{\pgfqpoint{3.932882in}{2.134419in}}{\pgfqpoint{3.936155in}{2.142319in}}{\pgfqpoint{3.936155in}{2.150555in}}%
\pgfpathcurveto{\pgfqpoint{3.936155in}{2.158791in}}{\pgfqpoint{3.932882in}{2.166691in}}{\pgfqpoint{3.927058in}{2.172515in}}%
\pgfpathcurveto{\pgfqpoint{3.921234in}{2.178339in}}{\pgfqpoint{3.913334in}{2.181611in}}{\pgfqpoint{3.905098in}{2.181611in}}%
\pgfpathcurveto{\pgfqpoint{3.896862in}{2.181611in}}{\pgfqpoint{3.888962in}{2.178339in}}{\pgfqpoint{3.883138in}{2.172515in}}%
\pgfpathcurveto{\pgfqpoint{3.877314in}{2.166691in}}{\pgfqpoint{3.874042in}{2.158791in}}{\pgfqpoint{3.874042in}{2.150555in}}%
\pgfpathcurveto{\pgfqpoint{3.874042in}{2.142319in}}{\pgfqpoint{3.877314in}{2.134419in}}{\pgfqpoint{3.883138in}{2.128595in}}%
\pgfpathcurveto{\pgfqpoint{3.888962in}{2.122771in}}{\pgfqpoint{3.896862in}{2.119498in}}{\pgfqpoint{3.905098in}{2.119498in}}%
\pgfpathclose%
\pgfusepath{stroke,fill}%
\end{pgfscope}%
\begin{pgfscope}%
\pgfpathrectangle{\pgfqpoint{3.793912in}{0.557870in}}{\pgfqpoint{2.446088in}{1.684734in}}%
\pgfusepath{clip}%
\pgfsetbuttcap%
\pgfsetroundjoin%
\definecolor{currentfill}{rgb}{0.298039,0.447059,0.690196}%
\pgfsetfillcolor{currentfill}%
\pgfsetlinewidth{1.003750pt}%
\definecolor{currentstroke}{rgb}{0.298039,0.447059,0.690196}%
\pgfsetstrokecolor{currentstroke}%
\pgfsetdash{}{0pt}%
\pgfpathmoveto{\pgfqpoint{3.905098in}{2.119498in}}%
\pgfpathcurveto{\pgfqpoint{3.913334in}{2.119498in}}{\pgfqpoint{3.921234in}{2.122771in}}{\pgfqpoint{3.927058in}{2.128595in}}%
\pgfpathcurveto{\pgfqpoint{3.932882in}{2.134419in}}{\pgfqpoint{3.936155in}{2.142319in}}{\pgfqpoint{3.936155in}{2.150555in}}%
\pgfpathcurveto{\pgfqpoint{3.936155in}{2.158791in}}{\pgfqpoint{3.932882in}{2.166691in}}{\pgfqpoint{3.927058in}{2.172515in}}%
\pgfpathcurveto{\pgfqpoint{3.921234in}{2.178339in}}{\pgfqpoint{3.913334in}{2.181611in}}{\pgfqpoint{3.905098in}{2.181611in}}%
\pgfpathcurveto{\pgfqpoint{3.896862in}{2.181611in}}{\pgfqpoint{3.888962in}{2.178339in}}{\pgfqpoint{3.883138in}{2.172515in}}%
\pgfpathcurveto{\pgfqpoint{3.877314in}{2.166691in}}{\pgfqpoint{3.874042in}{2.158791in}}{\pgfqpoint{3.874042in}{2.150555in}}%
\pgfpathcurveto{\pgfqpoint{3.874042in}{2.142319in}}{\pgfqpoint{3.877314in}{2.134419in}}{\pgfqpoint{3.883138in}{2.128595in}}%
\pgfpathcurveto{\pgfqpoint{3.888962in}{2.122771in}}{\pgfqpoint{3.896862in}{2.119498in}}{\pgfqpoint{3.905098in}{2.119498in}}%
\pgfpathclose%
\pgfusepath{stroke,fill}%
\end{pgfscope}%
\begin{pgfscope}%
\pgfpathrectangle{\pgfqpoint{3.793912in}{0.557870in}}{\pgfqpoint{2.446088in}{1.684734in}}%
\pgfusepath{clip}%
\pgfsetbuttcap%
\pgfsetroundjoin%
\definecolor{currentfill}{rgb}{0.298039,0.447059,0.690196}%
\pgfsetfillcolor{currentfill}%
\pgfsetlinewidth{1.003750pt}%
\definecolor{currentstroke}{rgb}{0.298039,0.447059,0.690196}%
\pgfsetstrokecolor{currentstroke}%
\pgfsetdash{}{0pt}%
\pgfpathmoveto{\pgfqpoint{5.257668in}{1.361445in}}%
\pgfpathcurveto{\pgfqpoint{5.265904in}{1.361445in}}{\pgfqpoint{5.273804in}{1.364718in}}{\pgfqpoint{5.279628in}{1.370542in}}%
\pgfpathcurveto{\pgfqpoint{5.285452in}{1.376366in}}{\pgfqpoint{5.288724in}{1.384266in}}{\pgfqpoint{5.288724in}{1.392502in}}%
\pgfpathcurveto{\pgfqpoint{5.288724in}{1.400738in}}{\pgfqpoint{5.285452in}{1.408638in}}{\pgfqpoint{5.279628in}{1.414462in}}%
\pgfpathcurveto{\pgfqpoint{5.273804in}{1.420286in}}{\pgfqpoint{5.265904in}{1.423558in}}{\pgfqpoint{5.257668in}{1.423558in}}%
\pgfpathcurveto{\pgfqpoint{5.249431in}{1.423558in}}{\pgfqpoint{5.241531in}{1.420286in}}{\pgfqpoint{5.235707in}{1.414462in}}%
\pgfpathcurveto{\pgfqpoint{5.229883in}{1.408638in}}{\pgfqpoint{5.226611in}{1.400738in}}{\pgfqpoint{5.226611in}{1.392502in}}%
\pgfpathcurveto{\pgfqpoint{5.226611in}{1.384266in}}{\pgfqpoint{5.229883in}{1.376366in}}{\pgfqpoint{5.235707in}{1.370542in}}%
\pgfpathcurveto{\pgfqpoint{5.241531in}{1.364718in}}{\pgfqpoint{5.249431in}{1.361445in}}{\pgfqpoint{5.257668in}{1.361445in}}%
\pgfpathclose%
\pgfusepath{stroke,fill}%
\end{pgfscope}%
\begin{pgfscope}%
\pgfpathrectangle{\pgfqpoint{3.793912in}{0.557870in}}{\pgfqpoint{2.446088in}{1.684734in}}%
\pgfusepath{clip}%
\pgfsetbuttcap%
\pgfsetroundjoin%
\definecolor{currentfill}{rgb}{0.298039,0.447059,0.690196}%
\pgfsetfillcolor{currentfill}%
\pgfsetlinewidth{1.003750pt}%
\definecolor{currentstroke}{rgb}{0.298039,0.447059,0.690196}%
\pgfsetstrokecolor{currentstroke}%
\pgfsetdash{}{0pt}%
\pgfpathmoveto{\pgfqpoint{5.257668in}{1.392386in}}%
\pgfpathcurveto{\pgfqpoint{5.265904in}{1.392386in}}{\pgfqpoint{5.273804in}{1.395659in}}{\pgfqpoint{5.279628in}{1.401483in}}%
\pgfpathcurveto{\pgfqpoint{5.285452in}{1.407307in}}{\pgfqpoint{5.288724in}{1.415207in}}{\pgfqpoint{5.288724in}{1.423443in}}%
\pgfpathcurveto{\pgfqpoint{5.288724in}{1.431679in}}{\pgfqpoint{5.285452in}{1.439579in}}{\pgfqpoint{5.279628in}{1.445403in}}%
\pgfpathcurveto{\pgfqpoint{5.273804in}{1.451227in}}{\pgfqpoint{5.265904in}{1.454499in}}{\pgfqpoint{5.257668in}{1.454499in}}%
\pgfpathcurveto{\pgfqpoint{5.249431in}{1.454499in}}{\pgfqpoint{5.241531in}{1.451227in}}{\pgfqpoint{5.235707in}{1.445403in}}%
\pgfpathcurveto{\pgfqpoint{5.229883in}{1.439579in}}{\pgfqpoint{5.226611in}{1.431679in}}{\pgfqpoint{5.226611in}{1.423443in}}%
\pgfpathcurveto{\pgfqpoint{5.226611in}{1.415207in}}{\pgfqpoint{5.229883in}{1.407307in}}{\pgfqpoint{5.235707in}{1.401483in}}%
\pgfpathcurveto{\pgfqpoint{5.241531in}{1.395659in}}{\pgfqpoint{5.249431in}{1.392386in}}{\pgfqpoint{5.257668in}{1.392386in}}%
\pgfpathclose%
\pgfusepath{stroke,fill}%
\end{pgfscope}%
\begin{pgfscope}%
\pgfpathrectangle{\pgfqpoint{3.793912in}{0.557870in}}{\pgfqpoint{2.446088in}{1.684734in}}%
\pgfusepath{clip}%
\pgfsetbuttcap%
\pgfsetroundjoin%
\definecolor{currentfill}{rgb}{0.298039,0.447059,0.690196}%
\pgfsetfillcolor{currentfill}%
\pgfsetlinewidth{1.003750pt}%
\definecolor{currentstroke}{rgb}{0.298039,0.447059,0.690196}%
\pgfsetstrokecolor{currentstroke}%
\pgfsetdash{}{0pt}%
\pgfpathmoveto{\pgfqpoint{3.905098in}{2.119498in}}%
\pgfpathcurveto{\pgfqpoint{3.913334in}{2.119498in}}{\pgfqpoint{3.921234in}{2.122771in}}{\pgfqpoint{3.927058in}{2.128595in}}%
\pgfpathcurveto{\pgfqpoint{3.932882in}{2.134419in}}{\pgfqpoint{3.936155in}{2.142319in}}{\pgfqpoint{3.936155in}{2.150555in}}%
\pgfpathcurveto{\pgfqpoint{3.936155in}{2.158791in}}{\pgfqpoint{3.932882in}{2.166691in}}{\pgfqpoint{3.927058in}{2.172515in}}%
\pgfpathcurveto{\pgfqpoint{3.921234in}{2.178339in}}{\pgfqpoint{3.913334in}{2.181611in}}{\pgfqpoint{3.905098in}{2.181611in}}%
\pgfpathcurveto{\pgfqpoint{3.896862in}{2.181611in}}{\pgfqpoint{3.888962in}{2.178339in}}{\pgfqpoint{3.883138in}{2.172515in}}%
\pgfpathcurveto{\pgfqpoint{3.877314in}{2.166691in}}{\pgfqpoint{3.874042in}{2.158791in}}{\pgfqpoint{3.874042in}{2.150555in}}%
\pgfpathcurveto{\pgfqpoint{3.874042in}{2.142319in}}{\pgfqpoint{3.877314in}{2.134419in}}{\pgfqpoint{3.883138in}{2.128595in}}%
\pgfpathcurveto{\pgfqpoint{3.888962in}{2.122771in}}{\pgfqpoint{3.896862in}{2.119498in}}{\pgfqpoint{3.905098in}{2.119498in}}%
\pgfpathclose%
\pgfusepath{stroke,fill}%
\end{pgfscope}%
\begin{pgfscope}%
\pgfpathrectangle{\pgfqpoint{3.793912in}{0.557870in}}{\pgfqpoint{2.446088in}{1.684734in}}%
\pgfusepath{clip}%
\pgfsetbuttcap%
\pgfsetroundjoin%
\definecolor{currentfill}{rgb}{0.298039,0.447059,0.690196}%
\pgfsetfillcolor{currentfill}%
\pgfsetlinewidth{1.003750pt}%
\definecolor{currentstroke}{rgb}{0.298039,0.447059,0.690196}%
\pgfsetstrokecolor{currentstroke}%
\pgfsetdash{}{0pt}%
\pgfpathmoveto{\pgfqpoint{3.905098in}{2.119498in}}%
\pgfpathcurveto{\pgfqpoint{3.913334in}{2.119498in}}{\pgfqpoint{3.921234in}{2.122771in}}{\pgfqpoint{3.927058in}{2.128595in}}%
\pgfpathcurveto{\pgfqpoint{3.932882in}{2.134419in}}{\pgfqpoint{3.936155in}{2.142319in}}{\pgfqpoint{3.936155in}{2.150555in}}%
\pgfpathcurveto{\pgfqpoint{3.936155in}{2.158791in}}{\pgfqpoint{3.932882in}{2.166691in}}{\pgfqpoint{3.927058in}{2.172515in}}%
\pgfpathcurveto{\pgfqpoint{3.921234in}{2.178339in}}{\pgfqpoint{3.913334in}{2.181611in}}{\pgfqpoint{3.905098in}{2.181611in}}%
\pgfpathcurveto{\pgfqpoint{3.896862in}{2.181611in}}{\pgfqpoint{3.888962in}{2.178339in}}{\pgfqpoint{3.883138in}{2.172515in}}%
\pgfpathcurveto{\pgfqpoint{3.877314in}{2.166691in}}{\pgfqpoint{3.874042in}{2.158791in}}{\pgfqpoint{3.874042in}{2.150555in}}%
\pgfpathcurveto{\pgfqpoint{3.874042in}{2.142319in}}{\pgfqpoint{3.877314in}{2.134419in}}{\pgfqpoint{3.883138in}{2.128595in}}%
\pgfpathcurveto{\pgfqpoint{3.888962in}{2.122771in}}{\pgfqpoint{3.896862in}{2.119498in}}{\pgfqpoint{3.905098in}{2.119498in}}%
\pgfpathclose%
\pgfusepath{stroke,fill}%
\end{pgfscope}%
\begin{pgfscope}%
\pgfpathrectangle{\pgfqpoint{3.793912in}{0.557870in}}{\pgfqpoint{2.446088in}{1.684734in}}%
\pgfusepath{clip}%
\pgfsetbuttcap%
\pgfsetroundjoin%
\definecolor{currentfill}{rgb}{0.298039,0.447059,0.690196}%
\pgfsetfillcolor{currentfill}%
\pgfsetlinewidth{1.003750pt}%
\definecolor{currentstroke}{rgb}{0.298039,0.447059,0.690196}%
\pgfsetstrokecolor{currentstroke}%
\pgfsetdash{}{0pt}%
\pgfpathmoveto{\pgfqpoint{5.670316in}{1.438798in}}%
\pgfpathcurveto{\pgfqpoint{5.678552in}{1.438798in}}{\pgfqpoint{5.686452in}{1.442070in}}{\pgfqpoint{5.692276in}{1.447894in}}%
\pgfpathcurveto{\pgfqpoint{5.698100in}{1.453718in}}{\pgfqpoint{5.701373in}{1.461618in}}{\pgfqpoint{5.701373in}{1.469854in}}%
\pgfpathcurveto{\pgfqpoint{5.701373in}{1.478091in}}{\pgfqpoint{5.698100in}{1.485991in}}{\pgfqpoint{5.692276in}{1.491815in}}%
\pgfpathcurveto{\pgfqpoint{5.686452in}{1.497639in}}{\pgfqpoint{5.678552in}{1.500911in}}{\pgfqpoint{5.670316in}{1.500911in}}%
\pgfpathcurveto{\pgfqpoint{5.662080in}{1.500911in}}{\pgfqpoint{5.654180in}{1.497639in}}{\pgfqpoint{5.648356in}{1.491815in}}%
\pgfpathcurveto{\pgfqpoint{5.642532in}{1.485991in}}{\pgfqpoint{5.639260in}{1.478091in}}{\pgfqpoint{5.639260in}{1.469854in}}%
\pgfpathcurveto{\pgfqpoint{5.639260in}{1.461618in}}{\pgfqpoint{5.642532in}{1.453718in}}{\pgfqpoint{5.648356in}{1.447894in}}%
\pgfpathcurveto{\pgfqpoint{5.654180in}{1.442070in}}{\pgfqpoint{5.662080in}{1.438798in}}{\pgfqpoint{5.670316in}{1.438798in}}%
\pgfpathclose%
\pgfusepath{stroke,fill}%
\end{pgfscope}%
\begin{pgfscope}%
\pgfpathrectangle{\pgfqpoint{3.793912in}{0.557870in}}{\pgfqpoint{2.446088in}{1.684734in}}%
\pgfusepath{clip}%
\pgfsetbuttcap%
\pgfsetroundjoin%
\definecolor{currentfill}{rgb}{0.298039,0.447059,0.690196}%
\pgfsetfillcolor{currentfill}%
\pgfsetlinewidth{1.003750pt}%
\definecolor{currentstroke}{rgb}{0.298039,0.447059,0.690196}%
\pgfsetstrokecolor{currentstroke}%
\pgfsetdash{}{0pt}%
\pgfpathmoveto{\pgfqpoint{5.601541in}{1.531621in}}%
\pgfpathcurveto{\pgfqpoint{5.609778in}{1.531621in}}{\pgfqpoint{5.617678in}{1.534893in}}{\pgfqpoint{5.623502in}{1.540717in}}%
\pgfpathcurveto{\pgfqpoint{5.629325in}{1.546541in}}{\pgfqpoint{5.632598in}{1.554441in}}{\pgfqpoint{5.632598in}{1.562677in}}%
\pgfpathcurveto{\pgfqpoint{5.632598in}{1.570913in}}{\pgfqpoint{5.629325in}{1.578813in}}{\pgfqpoint{5.623502in}{1.584637in}}%
\pgfpathcurveto{\pgfqpoint{5.617678in}{1.590461in}}{\pgfqpoint{5.609778in}{1.593734in}}{\pgfqpoint{5.601541in}{1.593734in}}%
\pgfpathcurveto{\pgfqpoint{5.593305in}{1.593734in}}{\pgfqpoint{5.585405in}{1.590461in}}{\pgfqpoint{5.579581in}{1.584637in}}%
\pgfpathcurveto{\pgfqpoint{5.573757in}{1.578813in}}{\pgfqpoint{5.570485in}{1.570913in}}{\pgfqpoint{5.570485in}{1.562677in}}%
\pgfpathcurveto{\pgfqpoint{5.570485in}{1.554441in}}{\pgfqpoint{5.573757in}{1.546541in}}{\pgfqpoint{5.579581in}{1.540717in}}%
\pgfpathcurveto{\pgfqpoint{5.585405in}{1.534893in}}{\pgfqpoint{5.593305in}{1.531621in}}{\pgfqpoint{5.601541in}{1.531621in}}%
\pgfpathclose%
\pgfusepath{stroke,fill}%
\end{pgfscope}%
\begin{pgfscope}%
\pgfpathrectangle{\pgfqpoint{3.793912in}{0.557870in}}{\pgfqpoint{2.446088in}{1.684734in}}%
\pgfusepath{clip}%
\pgfsetbuttcap%
\pgfsetroundjoin%
\definecolor{currentfill}{rgb}{0.298039,0.447059,0.690196}%
\pgfsetfillcolor{currentfill}%
\pgfsetlinewidth{1.003750pt}%
\definecolor{currentstroke}{rgb}{0.298039,0.447059,0.690196}%
\pgfsetstrokecolor{currentstroke}%
\pgfsetdash{}{0pt}%
\pgfpathmoveto{\pgfqpoint{5.647391in}{1.454268in}}%
\pgfpathcurveto{\pgfqpoint{5.655627in}{1.454268in}}{\pgfqpoint{5.663527in}{1.457541in}}{\pgfqpoint{5.669351in}{1.463365in}}%
\pgfpathcurveto{\pgfqpoint{5.675175in}{1.469188in}}{\pgfqpoint{5.678448in}{1.477089in}}{\pgfqpoint{5.678448in}{1.485325in}}%
\pgfpathcurveto{\pgfqpoint{5.678448in}{1.493561in}}{\pgfqpoint{5.675175in}{1.501461in}}{\pgfqpoint{5.669351in}{1.507285in}}%
\pgfpathcurveto{\pgfqpoint{5.663527in}{1.513109in}}{\pgfqpoint{5.655627in}{1.516381in}}{\pgfqpoint{5.647391in}{1.516381in}}%
\pgfpathcurveto{\pgfqpoint{5.639155in}{1.516381in}}{\pgfqpoint{5.631255in}{1.513109in}}{\pgfqpoint{5.625431in}{1.507285in}}%
\pgfpathcurveto{\pgfqpoint{5.619607in}{1.501461in}}{\pgfqpoint{5.616335in}{1.493561in}}{\pgfqpoint{5.616335in}{1.485325in}}%
\pgfpathcurveto{\pgfqpoint{5.616335in}{1.477089in}}{\pgfqpoint{5.619607in}{1.469188in}}{\pgfqpoint{5.625431in}{1.463365in}}%
\pgfpathcurveto{\pgfqpoint{5.631255in}{1.457541in}}{\pgfqpoint{5.639155in}{1.454268in}}{\pgfqpoint{5.647391in}{1.454268in}}%
\pgfpathclose%
\pgfusepath{stroke,fill}%
\end{pgfscope}%
\begin{pgfscope}%
\pgfpathrectangle{\pgfqpoint{3.793912in}{0.557870in}}{\pgfqpoint{2.446088in}{1.684734in}}%
\pgfusepath{clip}%
\pgfsetbuttcap%
\pgfsetroundjoin%
\definecolor{currentfill}{rgb}{0.298039,0.447059,0.690196}%
\pgfsetfillcolor{currentfill}%
\pgfsetlinewidth{1.003750pt}%
\definecolor{currentstroke}{rgb}{0.298039,0.447059,0.690196}%
\pgfsetstrokecolor{currentstroke}%
\pgfsetdash{}{0pt}%
\pgfpathmoveto{\pgfqpoint{5.578616in}{1.547091in}}%
\pgfpathcurveto{\pgfqpoint{5.586853in}{1.547091in}}{\pgfqpoint{5.594753in}{1.550363in}}{\pgfqpoint{5.600577in}{1.556187in}}%
\pgfpathcurveto{\pgfqpoint{5.606401in}{1.562011in}}{\pgfqpoint{5.609673in}{1.569911in}}{\pgfqpoint{5.609673in}{1.578148in}}%
\pgfpathcurveto{\pgfqpoint{5.609673in}{1.586384in}}{\pgfqpoint{5.606401in}{1.594284in}}{\pgfqpoint{5.600577in}{1.600108in}}%
\pgfpathcurveto{\pgfqpoint{5.594753in}{1.605932in}}{\pgfqpoint{5.586853in}{1.609204in}}{\pgfqpoint{5.578616in}{1.609204in}}%
\pgfpathcurveto{\pgfqpoint{5.570380in}{1.609204in}}{\pgfqpoint{5.562480in}{1.605932in}}{\pgfqpoint{5.556656in}{1.600108in}}%
\pgfpathcurveto{\pgfqpoint{5.550832in}{1.594284in}}{\pgfqpoint{5.547560in}{1.586384in}}{\pgfqpoint{5.547560in}{1.578148in}}%
\pgfpathcurveto{\pgfqpoint{5.547560in}{1.569911in}}{\pgfqpoint{5.550832in}{1.562011in}}{\pgfqpoint{5.556656in}{1.556187in}}%
\pgfpathcurveto{\pgfqpoint{5.562480in}{1.550363in}}{\pgfqpoint{5.570380in}{1.547091in}}{\pgfqpoint{5.578616in}{1.547091in}}%
\pgfpathclose%
\pgfusepath{stroke,fill}%
\end{pgfscope}%
\begin{pgfscope}%
\pgfpathrectangle{\pgfqpoint{3.793912in}{0.557870in}}{\pgfqpoint{2.446088in}{1.684734in}}%
\pgfusepath{clip}%
\pgfsetbuttcap%
\pgfsetroundjoin%
\definecolor{currentfill}{rgb}{0.298039,0.447059,0.690196}%
\pgfsetfillcolor{currentfill}%
\pgfsetlinewidth{1.003750pt}%
\definecolor{currentstroke}{rgb}{0.298039,0.447059,0.690196}%
\pgfsetstrokecolor{currentstroke}%
\pgfsetdash{}{0pt}%
\pgfpathmoveto{\pgfqpoint{3.905098in}{2.119498in}}%
\pgfpathcurveto{\pgfqpoint{3.913334in}{2.119498in}}{\pgfqpoint{3.921234in}{2.122771in}}{\pgfqpoint{3.927058in}{2.128595in}}%
\pgfpathcurveto{\pgfqpoint{3.932882in}{2.134419in}}{\pgfqpoint{3.936155in}{2.142319in}}{\pgfqpoint{3.936155in}{2.150555in}}%
\pgfpathcurveto{\pgfqpoint{3.936155in}{2.158791in}}{\pgfqpoint{3.932882in}{2.166691in}}{\pgfqpoint{3.927058in}{2.172515in}}%
\pgfpathcurveto{\pgfqpoint{3.921234in}{2.178339in}}{\pgfqpoint{3.913334in}{2.181611in}}{\pgfqpoint{3.905098in}{2.181611in}}%
\pgfpathcurveto{\pgfqpoint{3.896862in}{2.181611in}}{\pgfqpoint{3.888962in}{2.178339in}}{\pgfqpoint{3.883138in}{2.172515in}}%
\pgfpathcurveto{\pgfqpoint{3.877314in}{2.166691in}}{\pgfqpoint{3.874042in}{2.158791in}}{\pgfqpoint{3.874042in}{2.150555in}}%
\pgfpathcurveto{\pgfqpoint{3.874042in}{2.142319in}}{\pgfqpoint{3.877314in}{2.134419in}}{\pgfqpoint{3.883138in}{2.128595in}}%
\pgfpathcurveto{\pgfqpoint{3.888962in}{2.122771in}}{\pgfqpoint{3.896862in}{2.119498in}}{\pgfqpoint{3.905098in}{2.119498in}}%
\pgfpathclose%
\pgfusepath{stroke,fill}%
\end{pgfscope}%
\begin{pgfscope}%
\pgfpathrectangle{\pgfqpoint{3.793912in}{0.557870in}}{\pgfqpoint{2.446088in}{1.684734in}}%
\pgfusepath{clip}%
\pgfsetbuttcap%
\pgfsetroundjoin%
\definecolor{currentfill}{rgb}{0.298039,0.447059,0.690196}%
\pgfsetfillcolor{currentfill}%
\pgfsetlinewidth{1.003750pt}%
\definecolor{currentstroke}{rgb}{0.298039,0.447059,0.690196}%
\pgfsetstrokecolor{currentstroke}%
\pgfsetdash{}{0pt}%
\pgfpathmoveto{\pgfqpoint{5.624466in}{1.531621in}}%
\pgfpathcurveto{\pgfqpoint{5.632702in}{1.531621in}}{\pgfqpoint{5.640603in}{1.534893in}}{\pgfqpoint{5.646426in}{1.540717in}}%
\pgfpathcurveto{\pgfqpoint{5.652250in}{1.546541in}}{\pgfqpoint{5.655523in}{1.554441in}}{\pgfqpoint{5.655523in}{1.562677in}}%
\pgfpathcurveto{\pgfqpoint{5.655523in}{1.570913in}}{\pgfqpoint{5.652250in}{1.578813in}}{\pgfqpoint{5.646426in}{1.584637in}}%
\pgfpathcurveto{\pgfqpoint{5.640603in}{1.590461in}}{\pgfqpoint{5.632702in}{1.593734in}}{\pgfqpoint{5.624466in}{1.593734in}}%
\pgfpathcurveto{\pgfqpoint{5.616230in}{1.593734in}}{\pgfqpoint{5.608330in}{1.590461in}}{\pgfqpoint{5.602506in}{1.584637in}}%
\pgfpathcurveto{\pgfqpoint{5.596682in}{1.578813in}}{\pgfqpoint{5.593410in}{1.570913in}}{\pgfqpoint{5.593410in}{1.562677in}}%
\pgfpathcurveto{\pgfqpoint{5.593410in}{1.554441in}}{\pgfqpoint{5.596682in}{1.546541in}}{\pgfqpoint{5.602506in}{1.540717in}}%
\pgfpathcurveto{\pgfqpoint{5.608330in}{1.534893in}}{\pgfqpoint{5.616230in}{1.531621in}}{\pgfqpoint{5.624466in}{1.531621in}}%
\pgfpathclose%
\pgfusepath{stroke,fill}%
\end{pgfscope}%
\begin{pgfscope}%
\pgfpathrectangle{\pgfqpoint{3.793912in}{0.557870in}}{\pgfqpoint{2.446088in}{1.684734in}}%
\pgfusepath{clip}%
\pgfsetbuttcap%
\pgfsetroundjoin%
\definecolor{currentfill}{rgb}{0.298039,0.447059,0.690196}%
\pgfsetfillcolor{currentfill}%
\pgfsetlinewidth{1.003750pt}%
\definecolor{currentstroke}{rgb}{0.298039,0.447059,0.690196}%
\pgfsetstrokecolor{currentstroke}%
\pgfsetdash{}{0pt}%
\pgfpathmoveto{\pgfqpoint{3.905098in}{2.119498in}}%
\pgfpathcurveto{\pgfqpoint{3.913334in}{2.119498in}}{\pgfqpoint{3.921234in}{2.122771in}}{\pgfqpoint{3.927058in}{2.128595in}}%
\pgfpathcurveto{\pgfqpoint{3.932882in}{2.134419in}}{\pgfqpoint{3.936155in}{2.142319in}}{\pgfqpoint{3.936155in}{2.150555in}}%
\pgfpathcurveto{\pgfqpoint{3.936155in}{2.158791in}}{\pgfqpoint{3.932882in}{2.166691in}}{\pgfqpoint{3.927058in}{2.172515in}}%
\pgfpathcurveto{\pgfqpoint{3.921234in}{2.178339in}}{\pgfqpoint{3.913334in}{2.181611in}}{\pgfqpoint{3.905098in}{2.181611in}}%
\pgfpathcurveto{\pgfqpoint{3.896862in}{2.181611in}}{\pgfqpoint{3.888962in}{2.178339in}}{\pgfqpoint{3.883138in}{2.172515in}}%
\pgfpathcurveto{\pgfqpoint{3.877314in}{2.166691in}}{\pgfqpoint{3.874042in}{2.158791in}}{\pgfqpoint{3.874042in}{2.150555in}}%
\pgfpathcurveto{\pgfqpoint{3.874042in}{2.142319in}}{\pgfqpoint{3.877314in}{2.134419in}}{\pgfqpoint{3.883138in}{2.128595in}}%
\pgfpathcurveto{\pgfqpoint{3.888962in}{2.122771in}}{\pgfqpoint{3.896862in}{2.119498in}}{\pgfqpoint{3.905098in}{2.119498in}}%
\pgfpathclose%
\pgfusepath{stroke,fill}%
\end{pgfscope}%
\begin{pgfscope}%
\pgfpathrectangle{\pgfqpoint{3.793912in}{0.557870in}}{\pgfqpoint{2.446088in}{1.684734in}}%
\pgfusepath{clip}%
\pgfsetbuttcap%
\pgfsetroundjoin%
\definecolor{currentfill}{rgb}{0.298039,0.447059,0.690196}%
\pgfsetfillcolor{currentfill}%
\pgfsetlinewidth{1.003750pt}%
\definecolor{currentstroke}{rgb}{0.298039,0.447059,0.690196}%
\pgfsetstrokecolor{currentstroke}%
\pgfsetdash{}{0pt}%
\pgfpathmoveto{\pgfqpoint{4.684545in}{2.011205in}}%
\pgfpathcurveto{\pgfqpoint{4.692781in}{2.011205in}}{\pgfqpoint{4.700681in}{2.014477in}}{\pgfqpoint{4.706505in}{2.020301in}}%
\pgfpathcurveto{\pgfqpoint{4.712329in}{2.026125in}}{\pgfqpoint{4.715601in}{2.034025in}}{\pgfqpoint{4.715601in}{2.042262in}}%
\pgfpathcurveto{\pgfqpoint{4.715601in}{2.050498in}}{\pgfqpoint{4.712329in}{2.058398in}}{\pgfqpoint{4.706505in}{2.064222in}}%
\pgfpathcurveto{\pgfqpoint{4.700681in}{2.070046in}}{\pgfqpoint{4.692781in}{2.073318in}}{\pgfqpoint{4.684545in}{2.073318in}}%
\pgfpathcurveto{\pgfqpoint{4.676309in}{2.073318in}}{\pgfqpoint{4.668409in}{2.070046in}}{\pgfqpoint{4.662585in}{2.064222in}}%
\pgfpathcurveto{\pgfqpoint{4.656761in}{2.058398in}}{\pgfqpoint{4.653488in}{2.050498in}}{\pgfqpoint{4.653488in}{2.042262in}}%
\pgfpathcurveto{\pgfqpoint{4.653488in}{2.034025in}}{\pgfqpoint{4.656761in}{2.026125in}}{\pgfqpoint{4.662585in}{2.020301in}}%
\pgfpathcurveto{\pgfqpoint{4.668409in}{2.014477in}}{\pgfqpoint{4.676309in}{2.011205in}}{\pgfqpoint{4.684545in}{2.011205in}}%
\pgfpathclose%
\pgfusepath{stroke,fill}%
\end{pgfscope}%
\begin{pgfscope}%
\pgfpathrectangle{\pgfqpoint{3.793912in}{0.557870in}}{\pgfqpoint{2.446088in}{1.684734in}}%
\pgfusepath{clip}%
\pgfsetbuttcap%
\pgfsetroundjoin%
\definecolor{currentfill}{rgb}{0.298039,0.447059,0.690196}%
\pgfsetfillcolor{currentfill}%
\pgfsetlinewidth{1.003750pt}%
\definecolor{currentstroke}{rgb}{0.298039,0.447059,0.690196}%
\pgfsetstrokecolor{currentstroke}%
\pgfsetdash{}{0pt}%
\pgfpathmoveto{\pgfqpoint{5.647391in}{1.438798in}}%
\pgfpathcurveto{\pgfqpoint{5.655627in}{1.438798in}}{\pgfqpoint{5.663527in}{1.442070in}}{\pgfqpoint{5.669351in}{1.447894in}}%
\pgfpathcurveto{\pgfqpoint{5.675175in}{1.453718in}}{\pgfqpoint{5.678448in}{1.461618in}}{\pgfqpoint{5.678448in}{1.469854in}}%
\pgfpathcurveto{\pgfqpoint{5.678448in}{1.478091in}}{\pgfqpoint{5.675175in}{1.485991in}}{\pgfqpoint{5.669351in}{1.491815in}}%
\pgfpathcurveto{\pgfqpoint{5.663527in}{1.497639in}}{\pgfqpoint{5.655627in}{1.500911in}}{\pgfqpoint{5.647391in}{1.500911in}}%
\pgfpathcurveto{\pgfqpoint{5.639155in}{1.500911in}}{\pgfqpoint{5.631255in}{1.497639in}}{\pgfqpoint{5.625431in}{1.491815in}}%
\pgfpathcurveto{\pgfqpoint{5.619607in}{1.485991in}}{\pgfqpoint{5.616335in}{1.478091in}}{\pgfqpoint{5.616335in}{1.469854in}}%
\pgfpathcurveto{\pgfqpoint{5.616335in}{1.461618in}}{\pgfqpoint{5.619607in}{1.453718in}}{\pgfqpoint{5.625431in}{1.447894in}}%
\pgfpathcurveto{\pgfqpoint{5.631255in}{1.442070in}}{\pgfqpoint{5.639155in}{1.438798in}}{\pgfqpoint{5.647391in}{1.438798in}}%
\pgfpathclose%
\pgfusepath{stroke,fill}%
\end{pgfscope}%
\begin{pgfscope}%
\pgfpathrectangle{\pgfqpoint{3.793912in}{0.557870in}}{\pgfqpoint{2.446088in}{1.684734in}}%
\pgfusepath{clip}%
\pgfsetbuttcap%
\pgfsetroundjoin%
\definecolor{currentfill}{rgb}{0.298039,0.447059,0.690196}%
\pgfsetfillcolor{currentfill}%
\pgfsetlinewidth{1.003750pt}%
\definecolor{currentstroke}{rgb}{0.298039,0.447059,0.690196}%
\pgfsetstrokecolor{currentstroke}%
\pgfsetdash{}{0pt}%
\pgfpathmoveto{\pgfqpoint{5.647391in}{1.438798in}}%
\pgfpathcurveto{\pgfqpoint{5.655627in}{1.438798in}}{\pgfqpoint{5.663527in}{1.442070in}}{\pgfqpoint{5.669351in}{1.447894in}}%
\pgfpathcurveto{\pgfqpoint{5.675175in}{1.453718in}}{\pgfqpoint{5.678448in}{1.461618in}}{\pgfqpoint{5.678448in}{1.469854in}}%
\pgfpathcurveto{\pgfqpoint{5.678448in}{1.478091in}}{\pgfqpoint{5.675175in}{1.485991in}}{\pgfqpoint{5.669351in}{1.491815in}}%
\pgfpathcurveto{\pgfqpoint{5.663527in}{1.497639in}}{\pgfqpoint{5.655627in}{1.500911in}}{\pgfqpoint{5.647391in}{1.500911in}}%
\pgfpathcurveto{\pgfqpoint{5.639155in}{1.500911in}}{\pgfqpoint{5.631255in}{1.497639in}}{\pgfqpoint{5.625431in}{1.491815in}}%
\pgfpathcurveto{\pgfqpoint{5.619607in}{1.485991in}}{\pgfqpoint{5.616335in}{1.478091in}}{\pgfqpoint{5.616335in}{1.469854in}}%
\pgfpathcurveto{\pgfqpoint{5.616335in}{1.461618in}}{\pgfqpoint{5.619607in}{1.453718in}}{\pgfqpoint{5.625431in}{1.447894in}}%
\pgfpathcurveto{\pgfqpoint{5.631255in}{1.442070in}}{\pgfqpoint{5.639155in}{1.438798in}}{\pgfqpoint{5.647391in}{1.438798in}}%
\pgfpathclose%
\pgfusepath{stroke,fill}%
\end{pgfscope}%
\begin{pgfscope}%
\pgfpathrectangle{\pgfqpoint{3.793912in}{0.557870in}}{\pgfqpoint{2.446088in}{1.684734in}}%
\pgfusepath{clip}%
\pgfsetbuttcap%
\pgfsetroundjoin%
\definecolor{currentfill}{rgb}{0.298039,0.447059,0.690196}%
\pgfsetfillcolor{currentfill}%
\pgfsetlinewidth{1.003750pt}%
\definecolor{currentstroke}{rgb}{0.298039,0.447059,0.690196}%
\pgfsetstrokecolor{currentstroke}%
\pgfsetdash{}{0pt}%
\pgfpathmoveto{\pgfqpoint{4.065572in}{2.119498in}}%
\pgfpathcurveto{\pgfqpoint{4.073809in}{2.119498in}}{\pgfqpoint{4.081709in}{2.122771in}}{\pgfqpoint{4.087533in}{2.128595in}}%
\pgfpathcurveto{\pgfqpoint{4.093357in}{2.134419in}}{\pgfqpoint{4.096629in}{2.142319in}}{\pgfqpoint{4.096629in}{2.150555in}}%
\pgfpathcurveto{\pgfqpoint{4.096629in}{2.158791in}}{\pgfqpoint{4.093357in}{2.166691in}}{\pgfqpoint{4.087533in}{2.172515in}}%
\pgfpathcurveto{\pgfqpoint{4.081709in}{2.178339in}}{\pgfqpoint{4.073809in}{2.181611in}}{\pgfqpoint{4.065572in}{2.181611in}}%
\pgfpathcurveto{\pgfqpoint{4.057336in}{2.181611in}}{\pgfqpoint{4.049436in}{2.178339in}}{\pgfqpoint{4.043612in}{2.172515in}}%
\pgfpathcurveto{\pgfqpoint{4.037788in}{2.166691in}}{\pgfqpoint{4.034516in}{2.158791in}}{\pgfqpoint{4.034516in}{2.150555in}}%
\pgfpathcurveto{\pgfqpoint{4.034516in}{2.142319in}}{\pgfqpoint{4.037788in}{2.134419in}}{\pgfqpoint{4.043612in}{2.128595in}}%
\pgfpathcurveto{\pgfqpoint{4.049436in}{2.122771in}}{\pgfqpoint{4.057336in}{2.119498in}}{\pgfqpoint{4.065572in}{2.119498in}}%
\pgfpathclose%
\pgfusepath{stroke,fill}%
\end{pgfscope}%
\begin{pgfscope}%
\pgfpathrectangle{\pgfqpoint{3.793912in}{0.557870in}}{\pgfqpoint{2.446088in}{1.684734in}}%
\pgfusepath{clip}%
\pgfsetbuttcap%
\pgfsetroundjoin%
\definecolor{currentfill}{rgb}{0.298039,0.447059,0.690196}%
\pgfsetfillcolor{currentfill}%
\pgfsetlinewidth{1.003750pt}%
\definecolor{currentstroke}{rgb}{0.298039,0.447059,0.690196}%
\pgfsetstrokecolor{currentstroke}%
\pgfsetdash{}{0pt}%
\pgfpathmoveto{\pgfqpoint{3.905098in}{2.119498in}}%
\pgfpathcurveto{\pgfqpoint{3.913334in}{2.119498in}}{\pgfqpoint{3.921234in}{2.122771in}}{\pgfqpoint{3.927058in}{2.128595in}}%
\pgfpathcurveto{\pgfqpoint{3.932882in}{2.134419in}}{\pgfqpoint{3.936155in}{2.142319in}}{\pgfqpoint{3.936155in}{2.150555in}}%
\pgfpathcurveto{\pgfqpoint{3.936155in}{2.158791in}}{\pgfqpoint{3.932882in}{2.166691in}}{\pgfqpoint{3.927058in}{2.172515in}}%
\pgfpathcurveto{\pgfqpoint{3.921234in}{2.178339in}}{\pgfqpoint{3.913334in}{2.181611in}}{\pgfqpoint{3.905098in}{2.181611in}}%
\pgfpathcurveto{\pgfqpoint{3.896862in}{2.181611in}}{\pgfqpoint{3.888962in}{2.178339in}}{\pgfqpoint{3.883138in}{2.172515in}}%
\pgfpathcurveto{\pgfqpoint{3.877314in}{2.166691in}}{\pgfqpoint{3.874042in}{2.158791in}}{\pgfqpoint{3.874042in}{2.150555in}}%
\pgfpathcurveto{\pgfqpoint{3.874042in}{2.142319in}}{\pgfqpoint{3.877314in}{2.134419in}}{\pgfqpoint{3.883138in}{2.128595in}}%
\pgfpathcurveto{\pgfqpoint{3.888962in}{2.122771in}}{\pgfqpoint{3.896862in}{2.119498in}}{\pgfqpoint{3.905098in}{2.119498in}}%
\pgfpathclose%
\pgfusepath{stroke,fill}%
\end{pgfscope}%
\begin{pgfscope}%
\pgfpathrectangle{\pgfqpoint{3.793912in}{0.557870in}}{\pgfqpoint{2.446088in}{1.684734in}}%
\pgfusepath{clip}%
\pgfsetbuttcap%
\pgfsetroundjoin%
\definecolor{currentfill}{rgb}{0.298039,0.447059,0.690196}%
\pgfsetfillcolor{currentfill}%
\pgfsetlinewidth{1.003750pt}%
\definecolor{currentstroke}{rgb}{0.298039,0.447059,0.690196}%
\pgfsetstrokecolor{currentstroke}%
\pgfsetdash{}{0pt}%
\pgfpathmoveto{\pgfqpoint{5.441067in}{1.144859in}}%
\pgfpathcurveto{\pgfqpoint{5.449303in}{1.144859in}}{\pgfqpoint{5.457203in}{1.148131in}}{\pgfqpoint{5.463027in}{1.153955in}}%
\pgfpathcurveto{\pgfqpoint{5.468851in}{1.159779in}}{\pgfqpoint{5.472123in}{1.167679in}}{\pgfqpoint{5.472123in}{1.175915in}}%
\pgfpathcurveto{\pgfqpoint{5.472123in}{1.184152in}}{\pgfqpoint{5.468851in}{1.192052in}}{\pgfqpoint{5.463027in}{1.197876in}}%
\pgfpathcurveto{\pgfqpoint{5.457203in}{1.203700in}}{\pgfqpoint{5.449303in}{1.206972in}}{\pgfqpoint{5.441067in}{1.206972in}}%
\pgfpathcurveto{\pgfqpoint{5.432831in}{1.206972in}}{\pgfqpoint{5.424931in}{1.203700in}}{\pgfqpoint{5.419107in}{1.197876in}}%
\pgfpathcurveto{\pgfqpoint{5.413283in}{1.192052in}}{\pgfqpoint{5.410010in}{1.184152in}}{\pgfqpoint{5.410010in}{1.175915in}}%
\pgfpathcurveto{\pgfqpoint{5.410010in}{1.167679in}}{\pgfqpoint{5.413283in}{1.159779in}}{\pgfqpoint{5.419107in}{1.153955in}}%
\pgfpathcurveto{\pgfqpoint{5.424931in}{1.148131in}}{\pgfqpoint{5.432831in}{1.144859in}}{\pgfqpoint{5.441067in}{1.144859in}}%
\pgfpathclose%
\pgfusepath{stroke,fill}%
\end{pgfscope}%
\begin{pgfscope}%
\pgfpathrectangle{\pgfqpoint{3.793912in}{0.557870in}}{\pgfqpoint{2.446088in}{1.684734in}}%
\pgfusepath{clip}%
\pgfsetbuttcap%
\pgfsetroundjoin%
\definecolor{currentfill}{rgb}{0.298039,0.447059,0.690196}%
\pgfsetfillcolor{currentfill}%
\pgfsetlinewidth{1.003750pt}%
\definecolor{currentstroke}{rgb}{0.298039,0.447059,0.690196}%
\pgfsetstrokecolor{currentstroke}%
\pgfsetdash{}{0pt}%
\pgfpathmoveto{\pgfqpoint{3.905098in}{2.119498in}}%
\pgfpathcurveto{\pgfqpoint{3.913334in}{2.119498in}}{\pgfqpoint{3.921234in}{2.122771in}}{\pgfqpoint{3.927058in}{2.128595in}}%
\pgfpathcurveto{\pgfqpoint{3.932882in}{2.134419in}}{\pgfqpoint{3.936155in}{2.142319in}}{\pgfqpoint{3.936155in}{2.150555in}}%
\pgfpathcurveto{\pgfqpoint{3.936155in}{2.158791in}}{\pgfqpoint{3.932882in}{2.166691in}}{\pgfqpoint{3.927058in}{2.172515in}}%
\pgfpathcurveto{\pgfqpoint{3.921234in}{2.178339in}}{\pgfqpoint{3.913334in}{2.181611in}}{\pgfqpoint{3.905098in}{2.181611in}}%
\pgfpathcurveto{\pgfqpoint{3.896862in}{2.181611in}}{\pgfqpoint{3.888962in}{2.178339in}}{\pgfqpoint{3.883138in}{2.172515in}}%
\pgfpathcurveto{\pgfqpoint{3.877314in}{2.166691in}}{\pgfqpoint{3.874042in}{2.158791in}}{\pgfqpoint{3.874042in}{2.150555in}}%
\pgfpathcurveto{\pgfqpoint{3.874042in}{2.142319in}}{\pgfqpoint{3.877314in}{2.134419in}}{\pgfqpoint{3.883138in}{2.128595in}}%
\pgfpathcurveto{\pgfqpoint{3.888962in}{2.122771in}}{\pgfqpoint{3.896862in}{2.119498in}}{\pgfqpoint{3.905098in}{2.119498in}}%
\pgfpathclose%
\pgfusepath{stroke,fill}%
\end{pgfscope}%
\begin{pgfscope}%
\pgfpathrectangle{\pgfqpoint{3.793912in}{0.557870in}}{\pgfqpoint{2.446088in}{1.684734in}}%
\pgfusepath{clip}%
\pgfsetbuttcap%
\pgfsetroundjoin%
\definecolor{currentfill}{rgb}{0.298039,0.447059,0.690196}%
\pgfsetfillcolor{currentfill}%
\pgfsetlinewidth{1.003750pt}%
\definecolor{currentstroke}{rgb}{0.298039,0.447059,0.690196}%
\pgfsetstrokecolor{currentstroke}%
\pgfsetdash{}{0pt}%
\pgfpathmoveto{\pgfqpoint{5.395217in}{1.423327in}}%
\pgfpathcurveto{\pgfqpoint{5.403453in}{1.423327in}}{\pgfqpoint{5.411353in}{1.426600in}}{\pgfqpoint{5.417177in}{1.432424in}}%
\pgfpathcurveto{\pgfqpoint{5.423001in}{1.438248in}}{\pgfqpoint{5.426274in}{1.446148in}}{\pgfqpoint{5.426274in}{1.454384in}}%
\pgfpathcurveto{\pgfqpoint{5.426274in}{1.462620in}}{\pgfqpoint{5.423001in}{1.470520in}}{\pgfqpoint{5.417177in}{1.476344in}}%
\pgfpathcurveto{\pgfqpoint{5.411353in}{1.482168in}}{\pgfqpoint{5.403453in}{1.485440in}}{\pgfqpoint{5.395217in}{1.485440in}}%
\pgfpathcurveto{\pgfqpoint{5.386981in}{1.485440in}}{\pgfqpoint{5.379081in}{1.482168in}}{\pgfqpoint{5.373257in}{1.476344in}}%
\pgfpathcurveto{\pgfqpoint{5.367433in}{1.470520in}}{\pgfqpoint{5.364161in}{1.462620in}}{\pgfqpoint{5.364161in}{1.454384in}}%
\pgfpathcurveto{\pgfqpoint{5.364161in}{1.446148in}}{\pgfqpoint{5.367433in}{1.438248in}}{\pgfqpoint{5.373257in}{1.432424in}}%
\pgfpathcurveto{\pgfqpoint{5.379081in}{1.426600in}}{\pgfqpoint{5.386981in}{1.423327in}}{\pgfqpoint{5.395217in}{1.423327in}}%
\pgfpathclose%
\pgfusepath{stroke,fill}%
\end{pgfscope}%
\begin{pgfscope}%
\pgfpathrectangle{\pgfqpoint{3.793912in}{0.557870in}}{\pgfqpoint{2.446088in}{1.684734in}}%
\pgfusepath{clip}%
\pgfsetbuttcap%
\pgfsetroundjoin%
\definecolor{currentfill}{rgb}{0.298039,0.447059,0.690196}%
\pgfsetfillcolor{currentfill}%
\pgfsetlinewidth{1.003750pt}%
\definecolor{currentstroke}{rgb}{0.298039,0.447059,0.690196}%
\pgfsetstrokecolor{currentstroke}%
\pgfsetdash{}{0pt}%
\pgfpathmoveto{\pgfqpoint{3.905098in}{2.119498in}}%
\pgfpathcurveto{\pgfqpoint{3.913334in}{2.119498in}}{\pgfqpoint{3.921234in}{2.122771in}}{\pgfqpoint{3.927058in}{2.128595in}}%
\pgfpathcurveto{\pgfqpoint{3.932882in}{2.134419in}}{\pgfqpoint{3.936155in}{2.142319in}}{\pgfqpoint{3.936155in}{2.150555in}}%
\pgfpathcurveto{\pgfqpoint{3.936155in}{2.158791in}}{\pgfqpoint{3.932882in}{2.166691in}}{\pgfqpoint{3.927058in}{2.172515in}}%
\pgfpathcurveto{\pgfqpoint{3.921234in}{2.178339in}}{\pgfqpoint{3.913334in}{2.181611in}}{\pgfqpoint{3.905098in}{2.181611in}}%
\pgfpathcurveto{\pgfqpoint{3.896862in}{2.181611in}}{\pgfqpoint{3.888962in}{2.178339in}}{\pgfqpoint{3.883138in}{2.172515in}}%
\pgfpathcurveto{\pgfqpoint{3.877314in}{2.166691in}}{\pgfqpoint{3.874042in}{2.158791in}}{\pgfqpoint{3.874042in}{2.150555in}}%
\pgfpathcurveto{\pgfqpoint{3.874042in}{2.142319in}}{\pgfqpoint{3.877314in}{2.134419in}}{\pgfqpoint{3.883138in}{2.128595in}}%
\pgfpathcurveto{\pgfqpoint{3.888962in}{2.122771in}}{\pgfqpoint{3.896862in}{2.119498in}}{\pgfqpoint{3.905098in}{2.119498in}}%
\pgfpathclose%
\pgfusepath{stroke,fill}%
\end{pgfscope}%
\begin{pgfscope}%
\pgfpathrectangle{\pgfqpoint{3.793912in}{0.557870in}}{\pgfqpoint{2.446088in}{1.684734in}}%
\pgfusepath{clip}%
\pgfsetbuttcap%
\pgfsetroundjoin%
\definecolor{currentfill}{rgb}{0.298039,0.447059,0.690196}%
\pgfsetfillcolor{currentfill}%
\pgfsetlinewidth{1.003750pt}%
\definecolor{currentstroke}{rgb}{0.298039,0.447059,0.690196}%
\pgfsetstrokecolor{currentstroke}%
\pgfsetdash{}{0pt}%
\pgfpathmoveto{\pgfqpoint{3.905098in}{2.119498in}}%
\pgfpathcurveto{\pgfqpoint{3.913334in}{2.119498in}}{\pgfqpoint{3.921234in}{2.122771in}}{\pgfqpoint{3.927058in}{2.128595in}}%
\pgfpathcurveto{\pgfqpoint{3.932882in}{2.134419in}}{\pgfqpoint{3.936155in}{2.142319in}}{\pgfqpoint{3.936155in}{2.150555in}}%
\pgfpathcurveto{\pgfqpoint{3.936155in}{2.158791in}}{\pgfqpoint{3.932882in}{2.166691in}}{\pgfqpoint{3.927058in}{2.172515in}}%
\pgfpathcurveto{\pgfqpoint{3.921234in}{2.178339in}}{\pgfqpoint{3.913334in}{2.181611in}}{\pgfqpoint{3.905098in}{2.181611in}}%
\pgfpathcurveto{\pgfqpoint{3.896862in}{2.181611in}}{\pgfqpoint{3.888962in}{2.178339in}}{\pgfqpoint{3.883138in}{2.172515in}}%
\pgfpathcurveto{\pgfqpoint{3.877314in}{2.166691in}}{\pgfqpoint{3.874042in}{2.158791in}}{\pgfqpoint{3.874042in}{2.150555in}}%
\pgfpathcurveto{\pgfqpoint{3.874042in}{2.142319in}}{\pgfqpoint{3.877314in}{2.134419in}}{\pgfqpoint{3.883138in}{2.128595in}}%
\pgfpathcurveto{\pgfqpoint{3.888962in}{2.122771in}}{\pgfqpoint{3.896862in}{2.119498in}}{\pgfqpoint{3.905098in}{2.119498in}}%
\pgfpathclose%
\pgfusepath{stroke,fill}%
\end{pgfscope}%
\begin{pgfscope}%
\pgfpathrectangle{\pgfqpoint{3.793912in}{0.557870in}}{\pgfqpoint{2.446088in}{1.684734in}}%
\pgfusepath{clip}%
\pgfsetbuttcap%
\pgfsetroundjoin%
\definecolor{currentfill}{rgb}{0.298039,0.447059,0.690196}%
\pgfsetfillcolor{currentfill}%
\pgfsetlinewidth{1.003750pt}%
\definecolor{currentstroke}{rgb}{0.298039,0.447059,0.690196}%
\pgfsetstrokecolor{currentstroke}%
\pgfsetdash{}{0pt}%
\pgfpathmoveto{\pgfqpoint{3.905098in}{2.119498in}}%
\pgfpathcurveto{\pgfqpoint{3.913334in}{2.119498in}}{\pgfqpoint{3.921234in}{2.122771in}}{\pgfqpoint{3.927058in}{2.128595in}}%
\pgfpathcurveto{\pgfqpoint{3.932882in}{2.134419in}}{\pgfqpoint{3.936155in}{2.142319in}}{\pgfqpoint{3.936155in}{2.150555in}}%
\pgfpathcurveto{\pgfqpoint{3.936155in}{2.158791in}}{\pgfqpoint{3.932882in}{2.166691in}}{\pgfqpoint{3.927058in}{2.172515in}}%
\pgfpathcurveto{\pgfqpoint{3.921234in}{2.178339in}}{\pgfqpoint{3.913334in}{2.181611in}}{\pgfqpoint{3.905098in}{2.181611in}}%
\pgfpathcurveto{\pgfqpoint{3.896862in}{2.181611in}}{\pgfqpoint{3.888962in}{2.178339in}}{\pgfqpoint{3.883138in}{2.172515in}}%
\pgfpathcurveto{\pgfqpoint{3.877314in}{2.166691in}}{\pgfqpoint{3.874042in}{2.158791in}}{\pgfqpoint{3.874042in}{2.150555in}}%
\pgfpathcurveto{\pgfqpoint{3.874042in}{2.142319in}}{\pgfqpoint{3.877314in}{2.134419in}}{\pgfqpoint{3.883138in}{2.128595in}}%
\pgfpathcurveto{\pgfqpoint{3.888962in}{2.122771in}}{\pgfqpoint{3.896862in}{2.119498in}}{\pgfqpoint{3.905098in}{2.119498in}}%
\pgfpathclose%
\pgfusepath{stroke,fill}%
\end{pgfscope}%
\begin{pgfscope}%
\pgfpathrectangle{\pgfqpoint{3.793912in}{0.557870in}}{\pgfqpoint{2.446088in}{1.684734in}}%
\pgfusepath{clip}%
\pgfsetbuttcap%
\pgfsetroundjoin%
\definecolor{currentfill}{rgb}{0.298039,0.447059,0.690196}%
\pgfsetfillcolor{currentfill}%
\pgfsetlinewidth{1.003750pt}%
\definecolor{currentstroke}{rgb}{0.298039,0.447059,0.690196}%
\pgfsetstrokecolor{currentstroke}%
\pgfsetdash{}{0pt}%
\pgfpathmoveto{\pgfqpoint{3.905098in}{2.119498in}}%
\pgfpathcurveto{\pgfqpoint{3.913334in}{2.119498in}}{\pgfqpoint{3.921234in}{2.122771in}}{\pgfqpoint{3.927058in}{2.128595in}}%
\pgfpathcurveto{\pgfqpoint{3.932882in}{2.134419in}}{\pgfqpoint{3.936155in}{2.142319in}}{\pgfqpoint{3.936155in}{2.150555in}}%
\pgfpathcurveto{\pgfqpoint{3.936155in}{2.158791in}}{\pgfqpoint{3.932882in}{2.166691in}}{\pgfqpoint{3.927058in}{2.172515in}}%
\pgfpathcurveto{\pgfqpoint{3.921234in}{2.178339in}}{\pgfqpoint{3.913334in}{2.181611in}}{\pgfqpoint{3.905098in}{2.181611in}}%
\pgfpathcurveto{\pgfqpoint{3.896862in}{2.181611in}}{\pgfqpoint{3.888962in}{2.178339in}}{\pgfqpoint{3.883138in}{2.172515in}}%
\pgfpathcurveto{\pgfqpoint{3.877314in}{2.166691in}}{\pgfqpoint{3.874042in}{2.158791in}}{\pgfqpoint{3.874042in}{2.150555in}}%
\pgfpathcurveto{\pgfqpoint{3.874042in}{2.142319in}}{\pgfqpoint{3.877314in}{2.134419in}}{\pgfqpoint{3.883138in}{2.128595in}}%
\pgfpathcurveto{\pgfqpoint{3.888962in}{2.122771in}}{\pgfqpoint{3.896862in}{2.119498in}}{\pgfqpoint{3.905098in}{2.119498in}}%
\pgfpathclose%
\pgfusepath{stroke,fill}%
\end{pgfscope}%
\begin{pgfscope}%
\pgfpathrectangle{\pgfqpoint{3.793912in}{0.557870in}}{\pgfqpoint{2.446088in}{1.684734in}}%
\pgfusepath{clip}%
\pgfsetbuttcap%
\pgfsetroundjoin%
\definecolor{currentfill}{rgb}{0.298039,0.447059,0.690196}%
\pgfsetfillcolor{currentfill}%
\pgfsetlinewidth{1.003750pt}%
\definecolor{currentstroke}{rgb}{0.298039,0.447059,0.690196}%
\pgfsetstrokecolor{currentstroke}%
\pgfsetdash{}{0pt}%
\pgfpathmoveto{\pgfqpoint{3.905098in}{2.119498in}}%
\pgfpathcurveto{\pgfqpoint{3.913334in}{2.119498in}}{\pgfqpoint{3.921234in}{2.122771in}}{\pgfqpoint{3.927058in}{2.128595in}}%
\pgfpathcurveto{\pgfqpoint{3.932882in}{2.134419in}}{\pgfqpoint{3.936155in}{2.142319in}}{\pgfqpoint{3.936155in}{2.150555in}}%
\pgfpathcurveto{\pgfqpoint{3.936155in}{2.158791in}}{\pgfqpoint{3.932882in}{2.166691in}}{\pgfqpoint{3.927058in}{2.172515in}}%
\pgfpathcurveto{\pgfqpoint{3.921234in}{2.178339in}}{\pgfqpoint{3.913334in}{2.181611in}}{\pgfqpoint{3.905098in}{2.181611in}}%
\pgfpathcurveto{\pgfqpoint{3.896862in}{2.181611in}}{\pgfqpoint{3.888962in}{2.178339in}}{\pgfqpoint{3.883138in}{2.172515in}}%
\pgfpathcurveto{\pgfqpoint{3.877314in}{2.166691in}}{\pgfqpoint{3.874042in}{2.158791in}}{\pgfqpoint{3.874042in}{2.150555in}}%
\pgfpathcurveto{\pgfqpoint{3.874042in}{2.142319in}}{\pgfqpoint{3.877314in}{2.134419in}}{\pgfqpoint{3.883138in}{2.128595in}}%
\pgfpathcurveto{\pgfqpoint{3.888962in}{2.122771in}}{\pgfqpoint{3.896862in}{2.119498in}}{\pgfqpoint{3.905098in}{2.119498in}}%
\pgfpathclose%
\pgfusepath{stroke,fill}%
\end{pgfscope}%
\begin{pgfscope}%
\pgfpathrectangle{\pgfqpoint{3.793912in}{0.557870in}}{\pgfqpoint{2.446088in}{1.684734in}}%
\pgfusepath{clip}%
\pgfsetbuttcap%
\pgfsetroundjoin%
\definecolor{currentfill}{rgb}{0.298039,0.447059,0.690196}%
\pgfsetfillcolor{currentfill}%
\pgfsetlinewidth{1.003750pt}%
\definecolor{currentstroke}{rgb}{0.298039,0.447059,0.690196}%
\pgfsetstrokecolor{currentstroke}%
\pgfsetdash{}{0pt}%
\pgfpathmoveto{\pgfqpoint{3.905098in}{2.119498in}}%
\pgfpathcurveto{\pgfqpoint{3.913334in}{2.119498in}}{\pgfqpoint{3.921234in}{2.122771in}}{\pgfqpoint{3.927058in}{2.128595in}}%
\pgfpathcurveto{\pgfqpoint{3.932882in}{2.134419in}}{\pgfqpoint{3.936155in}{2.142319in}}{\pgfqpoint{3.936155in}{2.150555in}}%
\pgfpathcurveto{\pgfqpoint{3.936155in}{2.158791in}}{\pgfqpoint{3.932882in}{2.166691in}}{\pgfqpoint{3.927058in}{2.172515in}}%
\pgfpathcurveto{\pgfqpoint{3.921234in}{2.178339in}}{\pgfqpoint{3.913334in}{2.181611in}}{\pgfqpoint{3.905098in}{2.181611in}}%
\pgfpathcurveto{\pgfqpoint{3.896862in}{2.181611in}}{\pgfqpoint{3.888962in}{2.178339in}}{\pgfqpoint{3.883138in}{2.172515in}}%
\pgfpathcurveto{\pgfqpoint{3.877314in}{2.166691in}}{\pgfqpoint{3.874042in}{2.158791in}}{\pgfqpoint{3.874042in}{2.150555in}}%
\pgfpathcurveto{\pgfqpoint{3.874042in}{2.142319in}}{\pgfqpoint{3.877314in}{2.134419in}}{\pgfqpoint{3.883138in}{2.128595in}}%
\pgfpathcurveto{\pgfqpoint{3.888962in}{2.122771in}}{\pgfqpoint{3.896862in}{2.119498in}}{\pgfqpoint{3.905098in}{2.119498in}}%
\pgfpathclose%
\pgfusepath{stroke,fill}%
\end{pgfscope}%
\begin{pgfscope}%
\pgfpathrectangle{\pgfqpoint{3.793912in}{0.557870in}}{\pgfqpoint{2.446088in}{1.684734in}}%
\pgfusepath{clip}%
\pgfsetbuttcap%
\pgfsetroundjoin%
\definecolor{currentfill}{rgb}{0.298039,0.447059,0.690196}%
\pgfsetfillcolor{currentfill}%
\pgfsetlinewidth{1.003750pt}%
\definecolor{currentstroke}{rgb}{0.298039,0.447059,0.690196}%
\pgfsetstrokecolor{currentstroke}%
\pgfsetdash{}{0pt}%
\pgfpathmoveto{\pgfqpoint{5.234743in}{1.531621in}}%
\pgfpathcurveto{\pgfqpoint{5.242979in}{1.531621in}}{\pgfqpoint{5.250879in}{1.534893in}}{\pgfqpoint{5.256703in}{1.540717in}}%
\pgfpathcurveto{\pgfqpoint{5.262527in}{1.546541in}}{\pgfqpoint{5.265799in}{1.554441in}}{\pgfqpoint{5.265799in}{1.562677in}}%
\pgfpathcurveto{\pgfqpoint{5.265799in}{1.570913in}}{\pgfqpoint{5.262527in}{1.578813in}}{\pgfqpoint{5.256703in}{1.584637in}}%
\pgfpathcurveto{\pgfqpoint{5.250879in}{1.590461in}}{\pgfqpoint{5.242979in}{1.593734in}}{\pgfqpoint{5.234743in}{1.593734in}}%
\pgfpathcurveto{\pgfqpoint{5.226506in}{1.593734in}}{\pgfqpoint{5.218606in}{1.590461in}}{\pgfqpoint{5.212782in}{1.584637in}}%
\pgfpathcurveto{\pgfqpoint{5.206959in}{1.578813in}}{\pgfqpoint{5.203686in}{1.570913in}}{\pgfqpoint{5.203686in}{1.562677in}}%
\pgfpathcurveto{\pgfqpoint{5.203686in}{1.554441in}}{\pgfqpoint{5.206959in}{1.546541in}}{\pgfqpoint{5.212782in}{1.540717in}}%
\pgfpathcurveto{\pgfqpoint{5.218606in}{1.534893in}}{\pgfqpoint{5.226506in}{1.531621in}}{\pgfqpoint{5.234743in}{1.531621in}}%
\pgfpathclose%
\pgfusepath{stroke,fill}%
\end{pgfscope}%
\begin{pgfscope}%
\pgfpathrectangle{\pgfqpoint{3.793912in}{0.557870in}}{\pgfqpoint{2.446088in}{1.684734in}}%
\pgfusepath{clip}%
\pgfsetbuttcap%
\pgfsetroundjoin%
\definecolor{currentfill}{rgb}{0.298039,0.447059,0.690196}%
\pgfsetfillcolor{currentfill}%
\pgfsetlinewidth{1.003750pt}%
\definecolor{currentstroke}{rgb}{0.298039,0.447059,0.690196}%
\pgfsetstrokecolor{currentstroke}%
\pgfsetdash{}{0pt}%
\pgfpathmoveto{\pgfqpoint{3.905098in}{2.119498in}}%
\pgfpathcurveto{\pgfqpoint{3.913334in}{2.119498in}}{\pgfqpoint{3.921234in}{2.122771in}}{\pgfqpoint{3.927058in}{2.128595in}}%
\pgfpathcurveto{\pgfqpoint{3.932882in}{2.134419in}}{\pgfqpoint{3.936155in}{2.142319in}}{\pgfqpoint{3.936155in}{2.150555in}}%
\pgfpathcurveto{\pgfqpoint{3.936155in}{2.158791in}}{\pgfqpoint{3.932882in}{2.166691in}}{\pgfqpoint{3.927058in}{2.172515in}}%
\pgfpathcurveto{\pgfqpoint{3.921234in}{2.178339in}}{\pgfqpoint{3.913334in}{2.181611in}}{\pgfqpoint{3.905098in}{2.181611in}}%
\pgfpathcurveto{\pgfqpoint{3.896862in}{2.181611in}}{\pgfqpoint{3.888962in}{2.178339in}}{\pgfqpoint{3.883138in}{2.172515in}}%
\pgfpathcurveto{\pgfqpoint{3.877314in}{2.166691in}}{\pgfqpoint{3.874042in}{2.158791in}}{\pgfqpoint{3.874042in}{2.150555in}}%
\pgfpathcurveto{\pgfqpoint{3.874042in}{2.142319in}}{\pgfqpoint{3.877314in}{2.134419in}}{\pgfqpoint{3.883138in}{2.128595in}}%
\pgfpathcurveto{\pgfqpoint{3.888962in}{2.122771in}}{\pgfqpoint{3.896862in}{2.119498in}}{\pgfqpoint{3.905098in}{2.119498in}}%
\pgfpathclose%
\pgfusepath{stroke,fill}%
\end{pgfscope}%
\begin{pgfscope}%
\pgfpathrectangle{\pgfqpoint{3.793912in}{0.557870in}}{\pgfqpoint{2.446088in}{1.684734in}}%
\pgfusepath{clip}%
\pgfsetbuttcap%
\pgfsetroundjoin%
\definecolor{currentfill}{rgb}{0.298039,0.447059,0.690196}%
\pgfsetfillcolor{currentfill}%
\pgfsetlinewidth{1.003750pt}%
\definecolor{currentstroke}{rgb}{0.298039,0.447059,0.690196}%
\pgfsetstrokecolor{currentstroke}%
\pgfsetdash{}{0pt}%
\pgfpathmoveto{\pgfqpoint{3.905098in}{2.119498in}}%
\pgfpathcurveto{\pgfqpoint{3.913334in}{2.119498in}}{\pgfqpoint{3.921234in}{2.122771in}}{\pgfqpoint{3.927058in}{2.128595in}}%
\pgfpathcurveto{\pgfqpoint{3.932882in}{2.134419in}}{\pgfqpoint{3.936155in}{2.142319in}}{\pgfqpoint{3.936155in}{2.150555in}}%
\pgfpathcurveto{\pgfqpoint{3.936155in}{2.158791in}}{\pgfqpoint{3.932882in}{2.166691in}}{\pgfqpoint{3.927058in}{2.172515in}}%
\pgfpathcurveto{\pgfqpoint{3.921234in}{2.178339in}}{\pgfqpoint{3.913334in}{2.181611in}}{\pgfqpoint{3.905098in}{2.181611in}}%
\pgfpathcurveto{\pgfqpoint{3.896862in}{2.181611in}}{\pgfqpoint{3.888962in}{2.178339in}}{\pgfqpoint{3.883138in}{2.172515in}}%
\pgfpathcurveto{\pgfqpoint{3.877314in}{2.166691in}}{\pgfqpoint{3.874042in}{2.158791in}}{\pgfqpoint{3.874042in}{2.150555in}}%
\pgfpathcurveto{\pgfqpoint{3.874042in}{2.142319in}}{\pgfqpoint{3.877314in}{2.134419in}}{\pgfqpoint{3.883138in}{2.128595in}}%
\pgfpathcurveto{\pgfqpoint{3.888962in}{2.122771in}}{\pgfqpoint{3.896862in}{2.119498in}}{\pgfqpoint{3.905098in}{2.119498in}}%
\pgfpathclose%
\pgfusepath{stroke,fill}%
\end{pgfscope}%
\begin{pgfscope}%
\pgfpathrectangle{\pgfqpoint{3.793912in}{0.557870in}}{\pgfqpoint{2.446088in}{1.684734in}}%
\pgfusepath{clip}%
\pgfsetbuttcap%
\pgfsetroundjoin%
\definecolor{currentfill}{rgb}{0.298039,0.447059,0.690196}%
\pgfsetfillcolor{currentfill}%
\pgfsetlinewidth{1.003750pt}%
\definecolor{currentstroke}{rgb}{0.298039,0.447059,0.690196}%
\pgfsetstrokecolor{currentstroke}%
\pgfsetdash{}{0pt}%
\pgfpathmoveto{\pgfqpoint{3.905098in}{2.119498in}}%
\pgfpathcurveto{\pgfqpoint{3.913334in}{2.119498in}}{\pgfqpoint{3.921234in}{2.122771in}}{\pgfqpoint{3.927058in}{2.128595in}}%
\pgfpathcurveto{\pgfqpoint{3.932882in}{2.134419in}}{\pgfqpoint{3.936155in}{2.142319in}}{\pgfqpoint{3.936155in}{2.150555in}}%
\pgfpathcurveto{\pgfqpoint{3.936155in}{2.158791in}}{\pgfqpoint{3.932882in}{2.166691in}}{\pgfqpoint{3.927058in}{2.172515in}}%
\pgfpathcurveto{\pgfqpoint{3.921234in}{2.178339in}}{\pgfqpoint{3.913334in}{2.181611in}}{\pgfqpoint{3.905098in}{2.181611in}}%
\pgfpathcurveto{\pgfqpoint{3.896862in}{2.181611in}}{\pgfqpoint{3.888962in}{2.178339in}}{\pgfqpoint{3.883138in}{2.172515in}}%
\pgfpathcurveto{\pgfqpoint{3.877314in}{2.166691in}}{\pgfqpoint{3.874042in}{2.158791in}}{\pgfqpoint{3.874042in}{2.150555in}}%
\pgfpathcurveto{\pgfqpoint{3.874042in}{2.142319in}}{\pgfqpoint{3.877314in}{2.134419in}}{\pgfqpoint{3.883138in}{2.128595in}}%
\pgfpathcurveto{\pgfqpoint{3.888962in}{2.122771in}}{\pgfqpoint{3.896862in}{2.119498in}}{\pgfqpoint{3.905098in}{2.119498in}}%
\pgfpathclose%
\pgfusepath{stroke,fill}%
\end{pgfscope}%
\begin{pgfscope}%
\pgfpathrectangle{\pgfqpoint{3.793912in}{0.557870in}}{\pgfqpoint{2.446088in}{1.684734in}}%
\pgfusepath{clip}%
\pgfsetbuttcap%
\pgfsetroundjoin%
\definecolor{currentfill}{rgb}{0.298039,0.447059,0.690196}%
\pgfsetfillcolor{currentfill}%
\pgfsetlinewidth{1.003750pt}%
\definecolor{currentstroke}{rgb}{0.298039,0.447059,0.690196}%
\pgfsetstrokecolor{currentstroke}%
\pgfsetdash{}{0pt}%
\pgfpathmoveto{\pgfqpoint{3.905098in}{2.119498in}}%
\pgfpathcurveto{\pgfqpoint{3.913334in}{2.119498in}}{\pgfqpoint{3.921234in}{2.122771in}}{\pgfqpoint{3.927058in}{2.128595in}}%
\pgfpathcurveto{\pgfqpoint{3.932882in}{2.134419in}}{\pgfqpoint{3.936155in}{2.142319in}}{\pgfqpoint{3.936155in}{2.150555in}}%
\pgfpathcurveto{\pgfqpoint{3.936155in}{2.158791in}}{\pgfqpoint{3.932882in}{2.166691in}}{\pgfqpoint{3.927058in}{2.172515in}}%
\pgfpathcurveto{\pgfqpoint{3.921234in}{2.178339in}}{\pgfqpoint{3.913334in}{2.181611in}}{\pgfqpoint{3.905098in}{2.181611in}}%
\pgfpathcurveto{\pgfqpoint{3.896862in}{2.181611in}}{\pgfqpoint{3.888962in}{2.178339in}}{\pgfqpoint{3.883138in}{2.172515in}}%
\pgfpathcurveto{\pgfqpoint{3.877314in}{2.166691in}}{\pgfqpoint{3.874042in}{2.158791in}}{\pgfqpoint{3.874042in}{2.150555in}}%
\pgfpathcurveto{\pgfqpoint{3.874042in}{2.142319in}}{\pgfqpoint{3.877314in}{2.134419in}}{\pgfqpoint{3.883138in}{2.128595in}}%
\pgfpathcurveto{\pgfqpoint{3.888962in}{2.122771in}}{\pgfqpoint{3.896862in}{2.119498in}}{\pgfqpoint{3.905098in}{2.119498in}}%
\pgfpathclose%
\pgfusepath{stroke,fill}%
\end{pgfscope}%
\begin{pgfscope}%
\pgfpathrectangle{\pgfqpoint{3.793912in}{0.557870in}}{\pgfqpoint{2.446088in}{1.684734in}}%
\pgfusepath{clip}%
\pgfsetbuttcap%
\pgfsetroundjoin%
\definecolor{currentfill}{rgb}{0.298039,0.447059,0.690196}%
\pgfsetfillcolor{currentfill}%
\pgfsetlinewidth{1.003750pt}%
\definecolor{currentstroke}{rgb}{0.298039,0.447059,0.690196}%
\pgfsetstrokecolor{currentstroke}%
\pgfsetdash{}{0pt}%
\pgfpathmoveto{\pgfqpoint{5.670316in}{1.438798in}}%
\pgfpathcurveto{\pgfqpoint{5.678552in}{1.438798in}}{\pgfqpoint{5.686452in}{1.442070in}}{\pgfqpoint{5.692276in}{1.447894in}}%
\pgfpathcurveto{\pgfqpoint{5.698100in}{1.453718in}}{\pgfqpoint{5.701373in}{1.461618in}}{\pgfqpoint{5.701373in}{1.469854in}}%
\pgfpathcurveto{\pgfqpoint{5.701373in}{1.478091in}}{\pgfqpoint{5.698100in}{1.485991in}}{\pgfqpoint{5.692276in}{1.491815in}}%
\pgfpathcurveto{\pgfqpoint{5.686452in}{1.497639in}}{\pgfqpoint{5.678552in}{1.500911in}}{\pgfqpoint{5.670316in}{1.500911in}}%
\pgfpathcurveto{\pgfqpoint{5.662080in}{1.500911in}}{\pgfqpoint{5.654180in}{1.497639in}}{\pgfqpoint{5.648356in}{1.491815in}}%
\pgfpathcurveto{\pgfqpoint{5.642532in}{1.485991in}}{\pgfqpoint{5.639260in}{1.478091in}}{\pgfqpoint{5.639260in}{1.469854in}}%
\pgfpathcurveto{\pgfqpoint{5.639260in}{1.461618in}}{\pgfqpoint{5.642532in}{1.453718in}}{\pgfqpoint{5.648356in}{1.447894in}}%
\pgfpathcurveto{\pgfqpoint{5.654180in}{1.442070in}}{\pgfqpoint{5.662080in}{1.438798in}}{\pgfqpoint{5.670316in}{1.438798in}}%
\pgfpathclose%
\pgfusepath{stroke,fill}%
\end{pgfscope}%
\begin{pgfscope}%
\pgfpathrectangle{\pgfqpoint{3.793912in}{0.557870in}}{\pgfqpoint{2.446088in}{1.684734in}}%
\pgfusepath{clip}%
\pgfsetbuttcap%
\pgfsetroundjoin%
\definecolor{currentfill}{rgb}{0.298039,0.447059,0.690196}%
\pgfsetfillcolor{currentfill}%
\pgfsetlinewidth{1.003750pt}%
\definecolor{currentstroke}{rgb}{0.298039,0.447059,0.690196}%
\pgfsetstrokecolor{currentstroke}%
\pgfsetdash{}{0pt}%
\pgfpathmoveto{\pgfqpoint{5.624466in}{1.531621in}}%
\pgfpathcurveto{\pgfqpoint{5.632702in}{1.531621in}}{\pgfqpoint{5.640603in}{1.534893in}}{\pgfqpoint{5.646426in}{1.540717in}}%
\pgfpathcurveto{\pgfqpoint{5.652250in}{1.546541in}}{\pgfqpoint{5.655523in}{1.554441in}}{\pgfqpoint{5.655523in}{1.562677in}}%
\pgfpathcurveto{\pgfqpoint{5.655523in}{1.570913in}}{\pgfqpoint{5.652250in}{1.578813in}}{\pgfqpoint{5.646426in}{1.584637in}}%
\pgfpathcurveto{\pgfqpoint{5.640603in}{1.590461in}}{\pgfqpoint{5.632702in}{1.593734in}}{\pgfqpoint{5.624466in}{1.593734in}}%
\pgfpathcurveto{\pgfqpoint{5.616230in}{1.593734in}}{\pgfqpoint{5.608330in}{1.590461in}}{\pgfqpoint{5.602506in}{1.584637in}}%
\pgfpathcurveto{\pgfqpoint{5.596682in}{1.578813in}}{\pgfqpoint{5.593410in}{1.570913in}}{\pgfqpoint{5.593410in}{1.562677in}}%
\pgfpathcurveto{\pgfqpoint{5.593410in}{1.554441in}}{\pgfqpoint{5.596682in}{1.546541in}}{\pgfqpoint{5.602506in}{1.540717in}}%
\pgfpathcurveto{\pgfqpoint{5.608330in}{1.534893in}}{\pgfqpoint{5.616230in}{1.531621in}}{\pgfqpoint{5.624466in}{1.531621in}}%
\pgfpathclose%
\pgfusepath{stroke,fill}%
\end{pgfscope}%
\begin{pgfscope}%
\pgfpathrectangle{\pgfqpoint{3.793912in}{0.557870in}}{\pgfqpoint{2.446088in}{1.684734in}}%
\pgfusepath{clip}%
\pgfsetbuttcap%
\pgfsetroundjoin%
\definecolor{currentfill}{rgb}{0.298039,0.447059,0.690196}%
\pgfsetfillcolor{currentfill}%
\pgfsetlinewidth{1.003750pt}%
\definecolor{currentstroke}{rgb}{0.298039,0.447059,0.690196}%
\pgfsetstrokecolor{currentstroke}%
\pgfsetdash{}{0pt}%
\pgfpathmoveto{\pgfqpoint{5.647391in}{1.469739in}}%
\pgfpathcurveto{\pgfqpoint{5.655627in}{1.469739in}}{\pgfqpoint{5.663527in}{1.473011in}}{\pgfqpoint{5.669351in}{1.478835in}}%
\pgfpathcurveto{\pgfqpoint{5.675175in}{1.484659in}}{\pgfqpoint{5.678448in}{1.492559in}}{\pgfqpoint{5.678448in}{1.500795in}}%
\pgfpathcurveto{\pgfqpoint{5.678448in}{1.509032in}}{\pgfqpoint{5.675175in}{1.516932in}}{\pgfqpoint{5.669351in}{1.522756in}}%
\pgfpathcurveto{\pgfqpoint{5.663527in}{1.528579in}}{\pgfqpoint{5.655627in}{1.531852in}}{\pgfqpoint{5.647391in}{1.531852in}}%
\pgfpathcurveto{\pgfqpoint{5.639155in}{1.531852in}}{\pgfqpoint{5.631255in}{1.528579in}}{\pgfqpoint{5.625431in}{1.522756in}}%
\pgfpathcurveto{\pgfqpoint{5.619607in}{1.516932in}}{\pgfqpoint{5.616335in}{1.509032in}}{\pgfqpoint{5.616335in}{1.500795in}}%
\pgfpathcurveto{\pgfqpoint{5.616335in}{1.492559in}}{\pgfqpoint{5.619607in}{1.484659in}}{\pgfqpoint{5.625431in}{1.478835in}}%
\pgfpathcurveto{\pgfqpoint{5.631255in}{1.473011in}}{\pgfqpoint{5.639155in}{1.469739in}}{\pgfqpoint{5.647391in}{1.469739in}}%
\pgfpathclose%
\pgfusepath{stroke,fill}%
\end{pgfscope}%
\begin{pgfscope}%
\pgfpathrectangle{\pgfqpoint{3.793912in}{0.557870in}}{\pgfqpoint{2.446088in}{1.684734in}}%
\pgfusepath{clip}%
\pgfsetbuttcap%
\pgfsetroundjoin%
\definecolor{currentfill}{rgb}{0.298039,0.447059,0.690196}%
\pgfsetfillcolor{currentfill}%
\pgfsetlinewidth{1.003750pt}%
\definecolor{currentstroke}{rgb}{0.298039,0.447059,0.690196}%
\pgfsetstrokecolor{currentstroke}%
\pgfsetdash{}{0pt}%
\pgfpathmoveto{\pgfqpoint{4.042647in}{2.088557in}}%
\pgfpathcurveto{\pgfqpoint{4.050884in}{2.088557in}}{\pgfqpoint{4.058784in}{2.091830in}}{\pgfqpoint{4.064608in}{2.097654in}}%
\pgfpathcurveto{\pgfqpoint{4.070432in}{2.103478in}}{\pgfqpoint{4.073704in}{2.111378in}}{\pgfqpoint{4.073704in}{2.119614in}}%
\pgfpathcurveto{\pgfqpoint{4.073704in}{2.127850in}}{\pgfqpoint{4.070432in}{2.135750in}}{\pgfqpoint{4.064608in}{2.141574in}}%
\pgfpathcurveto{\pgfqpoint{4.058784in}{2.147398in}}{\pgfqpoint{4.050884in}{2.150670in}}{\pgfqpoint{4.042647in}{2.150670in}}%
\pgfpathcurveto{\pgfqpoint{4.034411in}{2.150670in}}{\pgfqpoint{4.026511in}{2.147398in}}{\pgfqpoint{4.020687in}{2.141574in}}%
\pgfpathcurveto{\pgfqpoint{4.014863in}{2.135750in}}{\pgfqpoint{4.011591in}{2.127850in}}{\pgfqpoint{4.011591in}{2.119614in}}%
\pgfpathcurveto{\pgfqpoint{4.011591in}{2.111378in}}{\pgfqpoint{4.014863in}{2.103478in}}{\pgfqpoint{4.020687in}{2.097654in}}%
\pgfpathcurveto{\pgfqpoint{4.026511in}{2.091830in}}{\pgfqpoint{4.034411in}{2.088557in}}{\pgfqpoint{4.042647in}{2.088557in}}%
\pgfpathclose%
\pgfusepath{stroke,fill}%
\end{pgfscope}%
\begin{pgfscope}%
\pgfpathrectangle{\pgfqpoint{3.793912in}{0.557870in}}{\pgfqpoint{2.446088in}{1.684734in}}%
\pgfusepath{clip}%
\pgfsetbuttcap%
\pgfsetroundjoin%
\definecolor{currentfill}{rgb}{0.298039,0.447059,0.690196}%
\pgfsetfillcolor{currentfill}%
\pgfsetlinewidth{1.003750pt}%
\definecolor{currentstroke}{rgb}{0.298039,0.447059,0.690196}%
\pgfsetstrokecolor{currentstroke}%
\pgfsetdash{}{0pt}%
\pgfpathmoveto{\pgfqpoint{4.982569in}{1.918382in}}%
\pgfpathcurveto{\pgfqpoint{4.990805in}{1.918382in}}{\pgfqpoint{4.998705in}{1.921655in}}{\pgfqpoint{5.004529in}{1.927479in}}%
\pgfpathcurveto{\pgfqpoint{5.010353in}{1.933302in}}{\pgfqpoint{5.013625in}{1.941203in}}{\pgfqpoint{5.013625in}{1.949439in}}%
\pgfpathcurveto{\pgfqpoint{5.013625in}{1.957675in}}{\pgfqpoint{5.010353in}{1.965575in}}{\pgfqpoint{5.004529in}{1.971399in}}%
\pgfpathcurveto{\pgfqpoint{4.998705in}{1.977223in}}{\pgfqpoint{4.990805in}{1.980495in}}{\pgfqpoint{4.982569in}{1.980495in}}%
\pgfpathcurveto{\pgfqpoint{4.974332in}{1.980495in}}{\pgfqpoint{4.966432in}{1.977223in}}{\pgfqpoint{4.960608in}{1.971399in}}%
\pgfpathcurveto{\pgfqpoint{4.954785in}{1.965575in}}{\pgfqpoint{4.951512in}{1.957675in}}{\pgfqpoint{4.951512in}{1.949439in}}%
\pgfpathcurveto{\pgfqpoint{4.951512in}{1.941203in}}{\pgfqpoint{4.954785in}{1.933302in}}{\pgfqpoint{4.960608in}{1.927479in}}%
\pgfpathcurveto{\pgfqpoint{4.966432in}{1.921655in}}{\pgfqpoint{4.974332in}{1.918382in}}{\pgfqpoint{4.982569in}{1.918382in}}%
\pgfpathclose%
\pgfusepath{stroke,fill}%
\end{pgfscope}%
\begin{pgfscope}%
\pgfpathrectangle{\pgfqpoint{3.793912in}{0.557870in}}{\pgfqpoint{2.446088in}{1.684734in}}%
\pgfusepath{clip}%
\pgfsetbuttcap%
\pgfsetroundjoin%
\definecolor{currentfill}{rgb}{0.298039,0.447059,0.690196}%
\pgfsetfillcolor{currentfill}%
\pgfsetlinewidth{1.003750pt}%
\definecolor{currentstroke}{rgb}{0.298039,0.447059,0.690196}%
\pgfsetstrokecolor{currentstroke}%
\pgfsetdash{}{0pt}%
\pgfpathmoveto{\pgfqpoint{3.905098in}{2.119498in}}%
\pgfpathcurveto{\pgfqpoint{3.913334in}{2.119498in}}{\pgfqpoint{3.921234in}{2.122771in}}{\pgfqpoint{3.927058in}{2.128595in}}%
\pgfpathcurveto{\pgfqpoint{3.932882in}{2.134419in}}{\pgfqpoint{3.936155in}{2.142319in}}{\pgfqpoint{3.936155in}{2.150555in}}%
\pgfpathcurveto{\pgfqpoint{3.936155in}{2.158791in}}{\pgfqpoint{3.932882in}{2.166691in}}{\pgfqpoint{3.927058in}{2.172515in}}%
\pgfpathcurveto{\pgfqpoint{3.921234in}{2.178339in}}{\pgfqpoint{3.913334in}{2.181611in}}{\pgfqpoint{3.905098in}{2.181611in}}%
\pgfpathcurveto{\pgfqpoint{3.896862in}{2.181611in}}{\pgfqpoint{3.888962in}{2.178339in}}{\pgfqpoint{3.883138in}{2.172515in}}%
\pgfpathcurveto{\pgfqpoint{3.877314in}{2.166691in}}{\pgfqpoint{3.874042in}{2.158791in}}{\pgfqpoint{3.874042in}{2.150555in}}%
\pgfpathcurveto{\pgfqpoint{3.874042in}{2.142319in}}{\pgfqpoint{3.877314in}{2.134419in}}{\pgfqpoint{3.883138in}{2.128595in}}%
\pgfpathcurveto{\pgfqpoint{3.888962in}{2.122771in}}{\pgfqpoint{3.896862in}{2.119498in}}{\pgfqpoint{3.905098in}{2.119498in}}%
\pgfpathclose%
\pgfusepath{stroke,fill}%
\end{pgfscope}%
\begin{pgfscope}%
\pgfpathrectangle{\pgfqpoint{3.793912in}{0.557870in}}{\pgfqpoint{2.446088in}{1.684734in}}%
\pgfusepath{clip}%
\pgfsetbuttcap%
\pgfsetroundjoin%
\definecolor{currentfill}{rgb}{0.298039,0.447059,0.690196}%
\pgfsetfillcolor{currentfill}%
\pgfsetlinewidth{1.003750pt}%
\definecolor{currentstroke}{rgb}{0.298039,0.447059,0.690196}%
\pgfsetstrokecolor{currentstroke}%
\pgfsetdash{}{0pt}%
\pgfpathmoveto{\pgfqpoint{5.463992in}{1.732737in}}%
\pgfpathcurveto{\pgfqpoint{5.472228in}{1.732737in}}{\pgfqpoint{5.480128in}{1.736009in}}{\pgfqpoint{5.485952in}{1.741833in}}%
\pgfpathcurveto{\pgfqpoint{5.491776in}{1.747657in}}{\pgfqpoint{5.495048in}{1.755557in}}{\pgfqpoint{5.495048in}{1.763793in}}%
\pgfpathcurveto{\pgfqpoint{5.495048in}{1.772029in}}{\pgfqpoint{5.491776in}{1.779930in}}{\pgfqpoint{5.485952in}{1.785753in}}%
\pgfpathcurveto{\pgfqpoint{5.480128in}{1.791577in}}{\pgfqpoint{5.472228in}{1.794850in}}{\pgfqpoint{5.463992in}{1.794850in}}%
\pgfpathcurveto{\pgfqpoint{5.455756in}{1.794850in}}{\pgfqpoint{5.447856in}{1.791577in}}{\pgfqpoint{5.442032in}{1.785753in}}%
\pgfpathcurveto{\pgfqpoint{5.436208in}{1.779930in}}{\pgfqpoint{5.432935in}{1.772029in}}{\pgfqpoint{5.432935in}{1.763793in}}%
\pgfpathcurveto{\pgfqpoint{5.432935in}{1.755557in}}{\pgfqpoint{5.436208in}{1.747657in}}{\pgfqpoint{5.442032in}{1.741833in}}%
\pgfpathcurveto{\pgfqpoint{5.447856in}{1.736009in}}{\pgfqpoint{5.455756in}{1.732737in}}{\pgfqpoint{5.463992in}{1.732737in}}%
\pgfpathclose%
\pgfusepath{stroke,fill}%
\end{pgfscope}%
\begin{pgfscope}%
\pgfpathrectangle{\pgfqpoint{3.793912in}{0.557870in}}{\pgfqpoint{2.446088in}{1.684734in}}%
\pgfusepath{clip}%
\pgfsetbuttcap%
\pgfsetroundjoin%
\definecolor{currentfill}{rgb}{0.298039,0.447059,0.690196}%
\pgfsetfillcolor{currentfill}%
\pgfsetlinewidth{1.003750pt}%
\definecolor{currentstroke}{rgb}{0.298039,0.447059,0.690196}%
\pgfsetstrokecolor{currentstroke}%
\pgfsetdash{}{0pt}%
\pgfpathmoveto{\pgfqpoint{3.905098in}{2.119498in}}%
\pgfpathcurveto{\pgfqpoint{3.913334in}{2.119498in}}{\pgfqpoint{3.921234in}{2.122771in}}{\pgfqpoint{3.927058in}{2.128595in}}%
\pgfpathcurveto{\pgfqpoint{3.932882in}{2.134419in}}{\pgfqpoint{3.936155in}{2.142319in}}{\pgfqpoint{3.936155in}{2.150555in}}%
\pgfpathcurveto{\pgfqpoint{3.936155in}{2.158791in}}{\pgfqpoint{3.932882in}{2.166691in}}{\pgfqpoint{3.927058in}{2.172515in}}%
\pgfpathcurveto{\pgfqpoint{3.921234in}{2.178339in}}{\pgfqpoint{3.913334in}{2.181611in}}{\pgfqpoint{3.905098in}{2.181611in}}%
\pgfpathcurveto{\pgfqpoint{3.896862in}{2.181611in}}{\pgfqpoint{3.888962in}{2.178339in}}{\pgfqpoint{3.883138in}{2.172515in}}%
\pgfpathcurveto{\pgfqpoint{3.877314in}{2.166691in}}{\pgfqpoint{3.874042in}{2.158791in}}{\pgfqpoint{3.874042in}{2.150555in}}%
\pgfpathcurveto{\pgfqpoint{3.874042in}{2.142319in}}{\pgfqpoint{3.877314in}{2.134419in}}{\pgfqpoint{3.883138in}{2.128595in}}%
\pgfpathcurveto{\pgfqpoint{3.888962in}{2.122771in}}{\pgfqpoint{3.896862in}{2.119498in}}{\pgfqpoint{3.905098in}{2.119498in}}%
\pgfpathclose%
\pgfusepath{stroke,fill}%
\end{pgfscope}%
\begin{pgfscope}%
\pgfpathrectangle{\pgfqpoint{3.793912in}{0.557870in}}{\pgfqpoint{2.446088in}{1.684734in}}%
\pgfusepath{clip}%
\pgfsetbuttcap%
\pgfsetroundjoin%
\definecolor{currentfill}{rgb}{0.298039,0.447059,0.690196}%
\pgfsetfillcolor{currentfill}%
\pgfsetlinewidth{1.003750pt}%
\definecolor{currentstroke}{rgb}{0.298039,0.447059,0.690196}%
\pgfsetstrokecolor{currentstroke}%
\pgfsetdash{}{0pt}%
\pgfpathmoveto{\pgfqpoint{5.326442in}{1.810089in}}%
\pgfpathcurveto{\pgfqpoint{5.334679in}{1.810089in}}{\pgfqpoint{5.342579in}{1.813361in}}{\pgfqpoint{5.348403in}{1.819185in}}%
\pgfpathcurveto{\pgfqpoint{5.354227in}{1.825009in}}{\pgfqpoint{5.357499in}{1.832909in}}{\pgfqpoint{5.357499in}{1.841146in}}%
\pgfpathcurveto{\pgfqpoint{5.357499in}{1.849382in}}{\pgfqpoint{5.354227in}{1.857282in}}{\pgfqpoint{5.348403in}{1.863106in}}%
\pgfpathcurveto{\pgfqpoint{5.342579in}{1.868930in}}{\pgfqpoint{5.334679in}{1.872202in}}{\pgfqpoint{5.326442in}{1.872202in}}%
\pgfpathcurveto{\pgfqpoint{5.318206in}{1.872202in}}{\pgfqpoint{5.310306in}{1.868930in}}{\pgfqpoint{5.304482in}{1.863106in}}%
\pgfpathcurveto{\pgfqpoint{5.298658in}{1.857282in}}{\pgfqpoint{5.295386in}{1.849382in}}{\pgfqpoint{5.295386in}{1.841146in}}%
\pgfpathcurveto{\pgfqpoint{5.295386in}{1.832909in}}{\pgfqpoint{5.298658in}{1.825009in}}{\pgfqpoint{5.304482in}{1.819185in}}%
\pgfpathcurveto{\pgfqpoint{5.310306in}{1.813361in}}{\pgfqpoint{5.318206in}{1.810089in}}{\pgfqpoint{5.326442in}{1.810089in}}%
\pgfpathclose%
\pgfusepath{stroke,fill}%
\end{pgfscope}%
\begin{pgfscope}%
\pgfpathrectangle{\pgfqpoint{3.793912in}{0.557870in}}{\pgfqpoint{2.446088in}{1.684734in}}%
\pgfusepath{clip}%
\pgfsetbuttcap%
\pgfsetroundjoin%
\definecolor{currentfill}{rgb}{0.298039,0.447059,0.690196}%
\pgfsetfillcolor{currentfill}%
\pgfsetlinewidth{1.003750pt}%
\definecolor{currentstroke}{rgb}{0.298039,0.447059,0.690196}%
\pgfsetstrokecolor{currentstroke}%
\pgfsetdash{}{0pt}%
\pgfpathmoveto{\pgfqpoint{3.905098in}{2.119498in}}%
\pgfpathcurveto{\pgfqpoint{3.913334in}{2.119498in}}{\pgfqpoint{3.921234in}{2.122771in}}{\pgfqpoint{3.927058in}{2.128595in}}%
\pgfpathcurveto{\pgfqpoint{3.932882in}{2.134419in}}{\pgfqpoint{3.936155in}{2.142319in}}{\pgfqpoint{3.936155in}{2.150555in}}%
\pgfpathcurveto{\pgfqpoint{3.936155in}{2.158791in}}{\pgfqpoint{3.932882in}{2.166691in}}{\pgfqpoint{3.927058in}{2.172515in}}%
\pgfpathcurveto{\pgfqpoint{3.921234in}{2.178339in}}{\pgfqpoint{3.913334in}{2.181611in}}{\pgfqpoint{3.905098in}{2.181611in}}%
\pgfpathcurveto{\pgfqpoint{3.896862in}{2.181611in}}{\pgfqpoint{3.888962in}{2.178339in}}{\pgfqpoint{3.883138in}{2.172515in}}%
\pgfpathcurveto{\pgfqpoint{3.877314in}{2.166691in}}{\pgfqpoint{3.874042in}{2.158791in}}{\pgfqpoint{3.874042in}{2.150555in}}%
\pgfpathcurveto{\pgfqpoint{3.874042in}{2.142319in}}{\pgfqpoint{3.877314in}{2.134419in}}{\pgfqpoint{3.883138in}{2.128595in}}%
\pgfpathcurveto{\pgfqpoint{3.888962in}{2.122771in}}{\pgfqpoint{3.896862in}{2.119498in}}{\pgfqpoint{3.905098in}{2.119498in}}%
\pgfpathclose%
\pgfusepath{stroke,fill}%
\end{pgfscope}%
\begin{pgfscope}%
\pgfpathrectangle{\pgfqpoint{3.793912in}{0.557870in}}{\pgfqpoint{2.446088in}{1.684734in}}%
\pgfusepath{clip}%
\pgfsetbuttcap%
\pgfsetroundjoin%
\definecolor{currentfill}{rgb}{0.298039,0.447059,0.690196}%
\pgfsetfillcolor{currentfill}%
\pgfsetlinewidth{1.003750pt}%
\definecolor{currentstroke}{rgb}{0.298039,0.447059,0.690196}%
\pgfsetstrokecolor{currentstroke}%
\pgfsetdash{}{0pt}%
\pgfpathmoveto{\pgfqpoint{3.905098in}{2.119498in}}%
\pgfpathcurveto{\pgfqpoint{3.913334in}{2.119498in}}{\pgfqpoint{3.921234in}{2.122771in}}{\pgfqpoint{3.927058in}{2.128595in}}%
\pgfpathcurveto{\pgfqpoint{3.932882in}{2.134419in}}{\pgfqpoint{3.936155in}{2.142319in}}{\pgfqpoint{3.936155in}{2.150555in}}%
\pgfpathcurveto{\pgfqpoint{3.936155in}{2.158791in}}{\pgfqpoint{3.932882in}{2.166691in}}{\pgfqpoint{3.927058in}{2.172515in}}%
\pgfpathcurveto{\pgfqpoint{3.921234in}{2.178339in}}{\pgfqpoint{3.913334in}{2.181611in}}{\pgfqpoint{3.905098in}{2.181611in}}%
\pgfpathcurveto{\pgfqpoint{3.896862in}{2.181611in}}{\pgfqpoint{3.888962in}{2.178339in}}{\pgfqpoint{3.883138in}{2.172515in}}%
\pgfpathcurveto{\pgfqpoint{3.877314in}{2.166691in}}{\pgfqpoint{3.874042in}{2.158791in}}{\pgfqpoint{3.874042in}{2.150555in}}%
\pgfpathcurveto{\pgfqpoint{3.874042in}{2.142319in}}{\pgfqpoint{3.877314in}{2.134419in}}{\pgfqpoint{3.883138in}{2.128595in}}%
\pgfpathcurveto{\pgfqpoint{3.888962in}{2.122771in}}{\pgfqpoint{3.896862in}{2.119498in}}{\pgfqpoint{3.905098in}{2.119498in}}%
\pgfpathclose%
\pgfusepath{stroke,fill}%
\end{pgfscope}%
\begin{pgfscope}%
\pgfpathrectangle{\pgfqpoint{3.793912in}{0.557870in}}{\pgfqpoint{2.446088in}{1.684734in}}%
\pgfusepath{clip}%
\pgfsetbuttcap%
\pgfsetroundjoin%
\definecolor{currentfill}{rgb}{0.298039,0.447059,0.690196}%
\pgfsetfillcolor{currentfill}%
\pgfsetlinewidth{1.003750pt}%
\definecolor{currentstroke}{rgb}{0.298039,0.447059,0.690196}%
\pgfsetstrokecolor{currentstroke}%
\pgfsetdash{}{0pt}%
\pgfpathmoveto{\pgfqpoint{3.905098in}{2.119498in}}%
\pgfpathcurveto{\pgfqpoint{3.913334in}{2.119498in}}{\pgfqpoint{3.921234in}{2.122771in}}{\pgfqpoint{3.927058in}{2.128595in}}%
\pgfpathcurveto{\pgfqpoint{3.932882in}{2.134419in}}{\pgfqpoint{3.936155in}{2.142319in}}{\pgfqpoint{3.936155in}{2.150555in}}%
\pgfpathcurveto{\pgfqpoint{3.936155in}{2.158791in}}{\pgfqpoint{3.932882in}{2.166691in}}{\pgfqpoint{3.927058in}{2.172515in}}%
\pgfpathcurveto{\pgfqpoint{3.921234in}{2.178339in}}{\pgfqpoint{3.913334in}{2.181611in}}{\pgfqpoint{3.905098in}{2.181611in}}%
\pgfpathcurveto{\pgfqpoint{3.896862in}{2.181611in}}{\pgfqpoint{3.888962in}{2.178339in}}{\pgfqpoint{3.883138in}{2.172515in}}%
\pgfpathcurveto{\pgfqpoint{3.877314in}{2.166691in}}{\pgfqpoint{3.874042in}{2.158791in}}{\pgfqpoint{3.874042in}{2.150555in}}%
\pgfpathcurveto{\pgfqpoint{3.874042in}{2.142319in}}{\pgfqpoint{3.877314in}{2.134419in}}{\pgfqpoint{3.883138in}{2.128595in}}%
\pgfpathcurveto{\pgfqpoint{3.888962in}{2.122771in}}{\pgfqpoint{3.896862in}{2.119498in}}{\pgfqpoint{3.905098in}{2.119498in}}%
\pgfpathclose%
\pgfusepath{stroke,fill}%
\end{pgfscope}%
\begin{pgfscope}%
\pgfpathrectangle{\pgfqpoint{3.793912in}{0.557870in}}{\pgfqpoint{2.446088in}{1.684734in}}%
\pgfusepath{clip}%
\pgfsetbuttcap%
\pgfsetroundjoin%
\definecolor{currentfill}{rgb}{0.298039,0.447059,0.690196}%
\pgfsetfillcolor{currentfill}%
\pgfsetlinewidth{1.003750pt}%
\definecolor{currentstroke}{rgb}{0.298039,0.447059,0.690196}%
\pgfsetstrokecolor{currentstroke}%
\pgfsetdash{}{0pt}%
\pgfpathmoveto{\pgfqpoint{3.905098in}{2.119498in}}%
\pgfpathcurveto{\pgfqpoint{3.913334in}{2.119498in}}{\pgfqpoint{3.921234in}{2.122771in}}{\pgfqpoint{3.927058in}{2.128595in}}%
\pgfpathcurveto{\pgfqpoint{3.932882in}{2.134419in}}{\pgfqpoint{3.936155in}{2.142319in}}{\pgfqpoint{3.936155in}{2.150555in}}%
\pgfpathcurveto{\pgfqpoint{3.936155in}{2.158791in}}{\pgfqpoint{3.932882in}{2.166691in}}{\pgfqpoint{3.927058in}{2.172515in}}%
\pgfpathcurveto{\pgfqpoint{3.921234in}{2.178339in}}{\pgfqpoint{3.913334in}{2.181611in}}{\pgfqpoint{3.905098in}{2.181611in}}%
\pgfpathcurveto{\pgfqpoint{3.896862in}{2.181611in}}{\pgfqpoint{3.888962in}{2.178339in}}{\pgfqpoint{3.883138in}{2.172515in}}%
\pgfpathcurveto{\pgfqpoint{3.877314in}{2.166691in}}{\pgfqpoint{3.874042in}{2.158791in}}{\pgfqpoint{3.874042in}{2.150555in}}%
\pgfpathcurveto{\pgfqpoint{3.874042in}{2.142319in}}{\pgfqpoint{3.877314in}{2.134419in}}{\pgfqpoint{3.883138in}{2.128595in}}%
\pgfpathcurveto{\pgfqpoint{3.888962in}{2.122771in}}{\pgfqpoint{3.896862in}{2.119498in}}{\pgfqpoint{3.905098in}{2.119498in}}%
\pgfpathclose%
\pgfusepath{stroke,fill}%
\end{pgfscope}%
\begin{pgfscope}%
\pgfpathrectangle{\pgfqpoint{3.793912in}{0.557870in}}{\pgfqpoint{2.446088in}{1.684734in}}%
\pgfusepath{clip}%
\pgfsetbuttcap%
\pgfsetroundjoin%
\definecolor{currentfill}{rgb}{0.298039,0.447059,0.690196}%
\pgfsetfillcolor{currentfill}%
\pgfsetlinewidth{1.003750pt}%
\definecolor{currentstroke}{rgb}{0.298039,0.447059,0.690196}%
\pgfsetstrokecolor{currentstroke}%
\pgfsetdash{}{0pt}%
\pgfpathmoveto{\pgfqpoint{4.661620in}{1.686325in}}%
\pgfpathcurveto{\pgfqpoint{4.669856in}{1.686325in}}{\pgfqpoint{4.677756in}{1.689598in}}{\pgfqpoint{4.683580in}{1.695422in}}%
\pgfpathcurveto{\pgfqpoint{4.689404in}{1.701245in}}{\pgfqpoint{4.692677in}{1.709146in}}{\pgfqpoint{4.692677in}{1.717382in}}%
\pgfpathcurveto{\pgfqpoint{4.692677in}{1.725618in}}{\pgfqpoint{4.689404in}{1.733518in}}{\pgfqpoint{4.683580in}{1.739342in}}%
\pgfpathcurveto{\pgfqpoint{4.677756in}{1.745166in}}{\pgfqpoint{4.669856in}{1.748438in}}{\pgfqpoint{4.661620in}{1.748438in}}%
\pgfpathcurveto{\pgfqpoint{4.653384in}{1.748438in}}{\pgfqpoint{4.645484in}{1.745166in}}{\pgfqpoint{4.639660in}{1.739342in}}%
\pgfpathcurveto{\pgfqpoint{4.633836in}{1.733518in}}{\pgfqpoint{4.630564in}{1.725618in}}{\pgfqpoint{4.630564in}{1.717382in}}%
\pgfpathcurveto{\pgfqpoint{4.630564in}{1.709146in}}{\pgfqpoint{4.633836in}{1.701245in}}{\pgfqpoint{4.639660in}{1.695422in}}%
\pgfpathcurveto{\pgfqpoint{4.645484in}{1.689598in}}{\pgfqpoint{4.653384in}{1.686325in}}{\pgfqpoint{4.661620in}{1.686325in}}%
\pgfpathclose%
\pgfusepath{stroke,fill}%
\end{pgfscope}%
\begin{pgfscope}%
\pgfpathrectangle{\pgfqpoint{3.793912in}{0.557870in}}{\pgfqpoint{2.446088in}{1.684734in}}%
\pgfusepath{clip}%
\pgfsetbuttcap%
\pgfsetroundjoin%
\definecolor{currentfill}{rgb}{0.298039,0.447059,0.690196}%
\pgfsetfillcolor{currentfill}%
\pgfsetlinewidth{1.003750pt}%
\definecolor{currentstroke}{rgb}{0.298039,0.447059,0.690196}%
\pgfsetstrokecolor{currentstroke}%
\pgfsetdash{}{0pt}%
\pgfpathmoveto{\pgfqpoint{3.905098in}{2.119498in}}%
\pgfpathcurveto{\pgfqpoint{3.913334in}{2.119498in}}{\pgfqpoint{3.921234in}{2.122771in}}{\pgfqpoint{3.927058in}{2.128595in}}%
\pgfpathcurveto{\pgfqpoint{3.932882in}{2.134419in}}{\pgfqpoint{3.936155in}{2.142319in}}{\pgfqpoint{3.936155in}{2.150555in}}%
\pgfpathcurveto{\pgfqpoint{3.936155in}{2.158791in}}{\pgfqpoint{3.932882in}{2.166691in}}{\pgfqpoint{3.927058in}{2.172515in}}%
\pgfpathcurveto{\pgfqpoint{3.921234in}{2.178339in}}{\pgfqpoint{3.913334in}{2.181611in}}{\pgfqpoint{3.905098in}{2.181611in}}%
\pgfpathcurveto{\pgfqpoint{3.896862in}{2.181611in}}{\pgfqpoint{3.888962in}{2.178339in}}{\pgfqpoint{3.883138in}{2.172515in}}%
\pgfpathcurveto{\pgfqpoint{3.877314in}{2.166691in}}{\pgfqpoint{3.874042in}{2.158791in}}{\pgfqpoint{3.874042in}{2.150555in}}%
\pgfpathcurveto{\pgfqpoint{3.874042in}{2.142319in}}{\pgfqpoint{3.877314in}{2.134419in}}{\pgfqpoint{3.883138in}{2.128595in}}%
\pgfpathcurveto{\pgfqpoint{3.888962in}{2.122771in}}{\pgfqpoint{3.896862in}{2.119498in}}{\pgfqpoint{3.905098in}{2.119498in}}%
\pgfpathclose%
\pgfusepath{stroke,fill}%
\end{pgfscope}%
\begin{pgfscope}%
\pgfpathrectangle{\pgfqpoint{3.793912in}{0.557870in}}{\pgfqpoint{2.446088in}{1.684734in}}%
\pgfusepath{clip}%
\pgfsetbuttcap%
\pgfsetroundjoin%
\definecolor{currentfill}{rgb}{0.298039,0.447059,0.690196}%
\pgfsetfillcolor{currentfill}%
\pgfsetlinewidth{1.003750pt}%
\definecolor{currentstroke}{rgb}{0.298039,0.447059,0.690196}%
\pgfsetstrokecolor{currentstroke}%
\pgfsetdash{}{0pt}%
\pgfpathmoveto{\pgfqpoint{4.799169in}{1.608973in}}%
\pgfpathcurveto{\pgfqpoint{4.807406in}{1.608973in}}{\pgfqpoint{4.815306in}{1.612245in}}{\pgfqpoint{4.821130in}{1.618069in}}%
\pgfpathcurveto{\pgfqpoint{4.826954in}{1.623893in}}{\pgfqpoint{4.830226in}{1.631793in}}{\pgfqpoint{4.830226in}{1.640029in}}%
\pgfpathcurveto{\pgfqpoint{4.830226in}{1.648266in}}{\pgfqpoint{4.826954in}{1.656166in}}{\pgfqpoint{4.821130in}{1.661990in}}%
\pgfpathcurveto{\pgfqpoint{4.815306in}{1.667814in}}{\pgfqpoint{4.807406in}{1.671086in}}{\pgfqpoint{4.799169in}{1.671086in}}%
\pgfpathcurveto{\pgfqpoint{4.790933in}{1.671086in}}{\pgfqpoint{4.783033in}{1.667814in}}{\pgfqpoint{4.777209in}{1.661990in}}%
\pgfpathcurveto{\pgfqpoint{4.771385in}{1.656166in}}{\pgfqpoint{4.768113in}{1.648266in}}{\pgfqpoint{4.768113in}{1.640029in}}%
\pgfpathcurveto{\pgfqpoint{4.768113in}{1.631793in}}{\pgfqpoint{4.771385in}{1.623893in}}{\pgfqpoint{4.777209in}{1.618069in}}%
\pgfpathcurveto{\pgfqpoint{4.783033in}{1.612245in}}{\pgfqpoint{4.790933in}{1.608973in}}{\pgfqpoint{4.799169in}{1.608973in}}%
\pgfpathclose%
\pgfusepath{stroke,fill}%
\end{pgfscope}%
\begin{pgfscope}%
\pgfpathrectangle{\pgfqpoint{3.793912in}{0.557870in}}{\pgfqpoint{2.446088in}{1.684734in}}%
\pgfusepath{clip}%
\pgfsetbuttcap%
\pgfsetroundjoin%
\definecolor{currentfill}{rgb}{0.298039,0.447059,0.690196}%
\pgfsetfillcolor{currentfill}%
\pgfsetlinewidth{1.003750pt}%
\definecolor{currentstroke}{rgb}{0.298039,0.447059,0.690196}%
\pgfsetstrokecolor{currentstroke}%
\pgfsetdash{}{0pt}%
\pgfpathmoveto{\pgfqpoint{3.905098in}{2.119498in}}%
\pgfpathcurveto{\pgfqpoint{3.913334in}{2.119498in}}{\pgfqpoint{3.921234in}{2.122771in}}{\pgfqpoint{3.927058in}{2.128595in}}%
\pgfpathcurveto{\pgfqpoint{3.932882in}{2.134419in}}{\pgfqpoint{3.936155in}{2.142319in}}{\pgfqpoint{3.936155in}{2.150555in}}%
\pgfpathcurveto{\pgfqpoint{3.936155in}{2.158791in}}{\pgfqpoint{3.932882in}{2.166691in}}{\pgfqpoint{3.927058in}{2.172515in}}%
\pgfpathcurveto{\pgfqpoint{3.921234in}{2.178339in}}{\pgfqpoint{3.913334in}{2.181611in}}{\pgfqpoint{3.905098in}{2.181611in}}%
\pgfpathcurveto{\pgfqpoint{3.896862in}{2.181611in}}{\pgfqpoint{3.888962in}{2.178339in}}{\pgfqpoint{3.883138in}{2.172515in}}%
\pgfpathcurveto{\pgfqpoint{3.877314in}{2.166691in}}{\pgfqpoint{3.874042in}{2.158791in}}{\pgfqpoint{3.874042in}{2.150555in}}%
\pgfpathcurveto{\pgfqpoint{3.874042in}{2.142319in}}{\pgfqpoint{3.877314in}{2.134419in}}{\pgfqpoint{3.883138in}{2.128595in}}%
\pgfpathcurveto{\pgfqpoint{3.888962in}{2.122771in}}{\pgfqpoint{3.896862in}{2.119498in}}{\pgfqpoint{3.905098in}{2.119498in}}%
\pgfpathclose%
\pgfusepath{stroke,fill}%
\end{pgfscope}%
\begin{pgfscope}%
\pgfpathrectangle{\pgfqpoint{3.793912in}{0.557870in}}{\pgfqpoint{2.446088in}{1.684734in}}%
\pgfusepath{clip}%
\pgfsetbuttcap%
\pgfsetroundjoin%
\definecolor{currentfill}{rgb}{0.298039,0.447059,0.690196}%
\pgfsetfillcolor{currentfill}%
\pgfsetlinewidth{1.003750pt}%
\definecolor{currentstroke}{rgb}{0.298039,0.447059,0.690196}%
\pgfsetstrokecolor{currentstroke}%
\pgfsetdash{}{0pt}%
\pgfpathmoveto{\pgfqpoint{3.905098in}{2.119498in}}%
\pgfpathcurveto{\pgfqpoint{3.913334in}{2.119498in}}{\pgfqpoint{3.921234in}{2.122771in}}{\pgfqpoint{3.927058in}{2.128595in}}%
\pgfpathcurveto{\pgfqpoint{3.932882in}{2.134419in}}{\pgfqpoint{3.936155in}{2.142319in}}{\pgfqpoint{3.936155in}{2.150555in}}%
\pgfpathcurveto{\pgfqpoint{3.936155in}{2.158791in}}{\pgfqpoint{3.932882in}{2.166691in}}{\pgfqpoint{3.927058in}{2.172515in}}%
\pgfpathcurveto{\pgfqpoint{3.921234in}{2.178339in}}{\pgfqpoint{3.913334in}{2.181611in}}{\pgfqpoint{3.905098in}{2.181611in}}%
\pgfpathcurveto{\pgfqpoint{3.896862in}{2.181611in}}{\pgfqpoint{3.888962in}{2.178339in}}{\pgfqpoint{3.883138in}{2.172515in}}%
\pgfpathcurveto{\pgfqpoint{3.877314in}{2.166691in}}{\pgfqpoint{3.874042in}{2.158791in}}{\pgfqpoint{3.874042in}{2.150555in}}%
\pgfpathcurveto{\pgfqpoint{3.874042in}{2.142319in}}{\pgfqpoint{3.877314in}{2.134419in}}{\pgfqpoint{3.883138in}{2.128595in}}%
\pgfpathcurveto{\pgfqpoint{3.888962in}{2.122771in}}{\pgfqpoint{3.896862in}{2.119498in}}{\pgfqpoint{3.905098in}{2.119498in}}%
\pgfpathclose%
\pgfusepath{stroke,fill}%
\end{pgfscope}%
\begin{pgfscope}%
\pgfpathrectangle{\pgfqpoint{3.793912in}{0.557870in}}{\pgfqpoint{2.446088in}{1.684734in}}%
\pgfusepath{clip}%
\pgfsetbuttcap%
\pgfsetroundjoin%
\definecolor{currentfill}{rgb}{0.298039,0.447059,0.690196}%
\pgfsetfillcolor{currentfill}%
\pgfsetlinewidth{1.003750pt}%
\definecolor{currentstroke}{rgb}{0.298039,0.447059,0.690196}%
\pgfsetstrokecolor{currentstroke}%
\pgfsetdash{}{0pt}%
\pgfpathmoveto{\pgfqpoint{3.905098in}{2.119498in}}%
\pgfpathcurveto{\pgfqpoint{3.913334in}{2.119498in}}{\pgfqpoint{3.921234in}{2.122771in}}{\pgfqpoint{3.927058in}{2.128595in}}%
\pgfpathcurveto{\pgfqpoint{3.932882in}{2.134419in}}{\pgfqpoint{3.936155in}{2.142319in}}{\pgfqpoint{3.936155in}{2.150555in}}%
\pgfpathcurveto{\pgfqpoint{3.936155in}{2.158791in}}{\pgfqpoint{3.932882in}{2.166691in}}{\pgfqpoint{3.927058in}{2.172515in}}%
\pgfpathcurveto{\pgfqpoint{3.921234in}{2.178339in}}{\pgfqpoint{3.913334in}{2.181611in}}{\pgfqpoint{3.905098in}{2.181611in}}%
\pgfpathcurveto{\pgfqpoint{3.896862in}{2.181611in}}{\pgfqpoint{3.888962in}{2.178339in}}{\pgfqpoint{3.883138in}{2.172515in}}%
\pgfpathcurveto{\pgfqpoint{3.877314in}{2.166691in}}{\pgfqpoint{3.874042in}{2.158791in}}{\pgfqpoint{3.874042in}{2.150555in}}%
\pgfpathcurveto{\pgfqpoint{3.874042in}{2.142319in}}{\pgfqpoint{3.877314in}{2.134419in}}{\pgfqpoint{3.883138in}{2.128595in}}%
\pgfpathcurveto{\pgfqpoint{3.888962in}{2.122771in}}{\pgfqpoint{3.896862in}{2.119498in}}{\pgfqpoint{3.905098in}{2.119498in}}%
\pgfpathclose%
\pgfusepath{stroke,fill}%
\end{pgfscope}%
\begin{pgfscope}%
\pgfpathrectangle{\pgfqpoint{3.793912in}{0.557870in}}{\pgfqpoint{2.446088in}{1.684734in}}%
\pgfusepath{clip}%
\pgfsetbuttcap%
\pgfsetroundjoin%
\definecolor{currentfill}{rgb}{0.298039,0.447059,0.690196}%
\pgfsetfillcolor{currentfill}%
\pgfsetlinewidth{1.003750pt}%
\definecolor{currentstroke}{rgb}{0.298039,0.447059,0.690196}%
\pgfsetstrokecolor{currentstroke}%
\pgfsetdash{}{0pt}%
\pgfpathmoveto{\pgfqpoint{3.950948in}{2.119498in}}%
\pgfpathcurveto{\pgfqpoint{3.959184in}{2.119498in}}{\pgfqpoint{3.967084in}{2.122771in}}{\pgfqpoint{3.972908in}{2.128595in}}%
\pgfpathcurveto{\pgfqpoint{3.978732in}{2.134419in}}{\pgfqpoint{3.982004in}{2.142319in}}{\pgfqpoint{3.982004in}{2.150555in}}%
\pgfpathcurveto{\pgfqpoint{3.982004in}{2.158791in}}{\pgfqpoint{3.978732in}{2.166691in}}{\pgfqpoint{3.972908in}{2.172515in}}%
\pgfpathcurveto{\pgfqpoint{3.967084in}{2.178339in}}{\pgfqpoint{3.959184in}{2.181611in}}{\pgfqpoint{3.950948in}{2.181611in}}%
\pgfpathcurveto{\pgfqpoint{3.942712in}{2.181611in}}{\pgfqpoint{3.934812in}{2.178339in}}{\pgfqpoint{3.928988in}{2.172515in}}%
\pgfpathcurveto{\pgfqpoint{3.923164in}{2.166691in}}{\pgfqpoint{3.919891in}{2.158791in}}{\pgfqpoint{3.919891in}{2.150555in}}%
\pgfpathcurveto{\pgfqpoint{3.919891in}{2.142319in}}{\pgfqpoint{3.923164in}{2.134419in}}{\pgfqpoint{3.928988in}{2.128595in}}%
\pgfpathcurveto{\pgfqpoint{3.934812in}{2.122771in}}{\pgfqpoint{3.942712in}{2.119498in}}{\pgfqpoint{3.950948in}{2.119498in}}%
\pgfpathclose%
\pgfusepath{stroke,fill}%
\end{pgfscope}%
\begin{pgfscope}%
\pgfpathrectangle{\pgfqpoint{3.793912in}{0.557870in}}{\pgfqpoint{2.446088in}{1.684734in}}%
\pgfusepath{clip}%
\pgfsetbuttcap%
\pgfsetroundjoin%
\definecolor{currentfill}{rgb}{0.298039,0.447059,0.690196}%
\pgfsetfillcolor{currentfill}%
\pgfsetlinewidth{1.003750pt}%
\definecolor{currentstroke}{rgb}{0.298039,0.447059,0.690196}%
\pgfsetstrokecolor{currentstroke}%
\pgfsetdash{}{0pt}%
\pgfpathmoveto{\pgfqpoint{3.905098in}{2.119498in}}%
\pgfpathcurveto{\pgfqpoint{3.913334in}{2.119498in}}{\pgfqpoint{3.921234in}{2.122771in}}{\pgfqpoint{3.927058in}{2.128595in}}%
\pgfpathcurveto{\pgfqpoint{3.932882in}{2.134419in}}{\pgfqpoint{3.936155in}{2.142319in}}{\pgfqpoint{3.936155in}{2.150555in}}%
\pgfpathcurveto{\pgfqpoint{3.936155in}{2.158791in}}{\pgfqpoint{3.932882in}{2.166691in}}{\pgfqpoint{3.927058in}{2.172515in}}%
\pgfpathcurveto{\pgfqpoint{3.921234in}{2.178339in}}{\pgfqpoint{3.913334in}{2.181611in}}{\pgfqpoint{3.905098in}{2.181611in}}%
\pgfpathcurveto{\pgfqpoint{3.896862in}{2.181611in}}{\pgfqpoint{3.888962in}{2.178339in}}{\pgfqpoint{3.883138in}{2.172515in}}%
\pgfpathcurveto{\pgfqpoint{3.877314in}{2.166691in}}{\pgfqpoint{3.874042in}{2.158791in}}{\pgfqpoint{3.874042in}{2.150555in}}%
\pgfpathcurveto{\pgfqpoint{3.874042in}{2.142319in}}{\pgfqpoint{3.877314in}{2.134419in}}{\pgfqpoint{3.883138in}{2.128595in}}%
\pgfpathcurveto{\pgfqpoint{3.888962in}{2.122771in}}{\pgfqpoint{3.896862in}{2.119498in}}{\pgfqpoint{3.905098in}{2.119498in}}%
\pgfpathclose%
\pgfusepath{stroke,fill}%
\end{pgfscope}%
\begin{pgfscope}%
\pgfpathrectangle{\pgfqpoint{3.793912in}{0.557870in}}{\pgfqpoint{2.446088in}{1.684734in}}%
\pgfusepath{clip}%
\pgfsetbuttcap%
\pgfsetroundjoin%
\definecolor{currentfill}{rgb}{0.298039,0.447059,0.690196}%
\pgfsetfillcolor{currentfill}%
\pgfsetlinewidth{1.003750pt}%
\definecolor{currentstroke}{rgb}{0.298039,0.447059,0.690196}%
\pgfsetstrokecolor{currentstroke}%
\pgfsetdash{}{0pt}%
\pgfpathmoveto{\pgfqpoint{4.730395in}{1.810089in}}%
\pgfpathcurveto{\pgfqpoint{4.738631in}{1.810089in}}{\pgfqpoint{4.746531in}{1.813361in}}{\pgfqpoint{4.752355in}{1.819185in}}%
\pgfpathcurveto{\pgfqpoint{4.758179in}{1.825009in}}{\pgfqpoint{4.761451in}{1.832909in}}{\pgfqpoint{4.761451in}{1.841146in}}%
\pgfpathcurveto{\pgfqpoint{4.761451in}{1.849382in}}{\pgfqpoint{4.758179in}{1.857282in}}{\pgfqpoint{4.752355in}{1.863106in}}%
\pgfpathcurveto{\pgfqpoint{4.746531in}{1.868930in}}{\pgfqpoint{4.738631in}{1.872202in}}{\pgfqpoint{4.730395in}{1.872202in}}%
\pgfpathcurveto{\pgfqpoint{4.722158in}{1.872202in}}{\pgfqpoint{4.714258in}{1.868930in}}{\pgfqpoint{4.708434in}{1.863106in}}%
\pgfpathcurveto{\pgfqpoint{4.702611in}{1.857282in}}{\pgfqpoint{4.699338in}{1.849382in}}{\pgfqpoint{4.699338in}{1.841146in}}%
\pgfpathcurveto{\pgfqpoint{4.699338in}{1.832909in}}{\pgfqpoint{4.702611in}{1.825009in}}{\pgfqpoint{4.708434in}{1.819185in}}%
\pgfpathcurveto{\pgfqpoint{4.714258in}{1.813361in}}{\pgfqpoint{4.722158in}{1.810089in}}{\pgfqpoint{4.730395in}{1.810089in}}%
\pgfpathclose%
\pgfusepath{stroke,fill}%
\end{pgfscope}%
\begin{pgfscope}%
\pgfpathrectangle{\pgfqpoint{3.793912in}{0.557870in}}{\pgfqpoint{2.446088in}{1.684734in}}%
\pgfusepath{clip}%
\pgfsetbuttcap%
\pgfsetroundjoin%
\definecolor{currentfill}{rgb}{0.298039,0.447059,0.690196}%
\pgfsetfillcolor{currentfill}%
\pgfsetlinewidth{1.003750pt}%
\definecolor{currentstroke}{rgb}{0.298039,0.447059,0.690196}%
\pgfsetstrokecolor{currentstroke}%
\pgfsetdash{}{0pt}%
\pgfpathmoveto{\pgfqpoint{3.905098in}{2.119498in}}%
\pgfpathcurveto{\pgfqpoint{3.913334in}{2.119498in}}{\pgfqpoint{3.921234in}{2.122771in}}{\pgfqpoint{3.927058in}{2.128595in}}%
\pgfpathcurveto{\pgfqpoint{3.932882in}{2.134419in}}{\pgfqpoint{3.936155in}{2.142319in}}{\pgfqpoint{3.936155in}{2.150555in}}%
\pgfpathcurveto{\pgfqpoint{3.936155in}{2.158791in}}{\pgfqpoint{3.932882in}{2.166691in}}{\pgfqpoint{3.927058in}{2.172515in}}%
\pgfpathcurveto{\pgfqpoint{3.921234in}{2.178339in}}{\pgfqpoint{3.913334in}{2.181611in}}{\pgfqpoint{3.905098in}{2.181611in}}%
\pgfpathcurveto{\pgfqpoint{3.896862in}{2.181611in}}{\pgfqpoint{3.888962in}{2.178339in}}{\pgfqpoint{3.883138in}{2.172515in}}%
\pgfpathcurveto{\pgfqpoint{3.877314in}{2.166691in}}{\pgfqpoint{3.874042in}{2.158791in}}{\pgfqpoint{3.874042in}{2.150555in}}%
\pgfpathcurveto{\pgfqpoint{3.874042in}{2.142319in}}{\pgfqpoint{3.877314in}{2.134419in}}{\pgfqpoint{3.883138in}{2.128595in}}%
\pgfpathcurveto{\pgfqpoint{3.888962in}{2.122771in}}{\pgfqpoint{3.896862in}{2.119498in}}{\pgfqpoint{3.905098in}{2.119498in}}%
\pgfpathclose%
\pgfusepath{stroke,fill}%
\end{pgfscope}%
\begin{pgfscope}%
\pgfpathrectangle{\pgfqpoint{3.793912in}{0.557870in}}{\pgfqpoint{2.446088in}{1.684734in}}%
\pgfusepath{clip}%
\pgfsetbuttcap%
\pgfsetroundjoin%
\definecolor{currentfill}{rgb}{0.298039,0.447059,0.690196}%
\pgfsetfillcolor{currentfill}%
\pgfsetlinewidth{1.003750pt}%
\definecolor{currentstroke}{rgb}{0.298039,0.447059,0.690196}%
\pgfsetstrokecolor{currentstroke}%
\pgfsetdash{}{0pt}%
\pgfpathmoveto{\pgfqpoint{3.950948in}{2.104028in}}%
\pgfpathcurveto{\pgfqpoint{3.959184in}{2.104028in}}{\pgfqpoint{3.967084in}{2.107300in}}{\pgfqpoint{3.972908in}{2.113124in}}%
\pgfpathcurveto{\pgfqpoint{3.978732in}{2.118948in}}{\pgfqpoint{3.982004in}{2.126848in}}{\pgfqpoint{3.982004in}{2.135084in}}%
\pgfpathcurveto{\pgfqpoint{3.982004in}{2.143321in}}{\pgfqpoint{3.978732in}{2.151221in}}{\pgfqpoint{3.972908in}{2.157045in}}%
\pgfpathcurveto{\pgfqpoint{3.967084in}{2.162869in}}{\pgfqpoint{3.959184in}{2.166141in}}{\pgfqpoint{3.950948in}{2.166141in}}%
\pgfpathcurveto{\pgfqpoint{3.942712in}{2.166141in}}{\pgfqpoint{3.934812in}{2.162869in}}{\pgfqpoint{3.928988in}{2.157045in}}%
\pgfpathcurveto{\pgfqpoint{3.923164in}{2.151221in}}{\pgfqpoint{3.919891in}{2.143321in}}{\pgfqpoint{3.919891in}{2.135084in}}%
\pgfpathcurveto{\pgfqpoint{3.919891in}{2.126848in}}{\pgfqpoint{3.923164in}{2.118948in}}{\pgfqpoint{3.928988in}{2.113124in}}%
\pgfpathcurveto{\pgfqpoint{3.934812in}{2.107300in}}{\pgfqpoint{3.942712in}{2.104028in}}{\pgfqpoint{3.950948in}{2.104028in}}%
\pgfpathclose%
\pgfusepath{stroke,fill}%
\end{pgfscope}%
\begin{pgfscope}%
\pgfpathrectangle{\pgfqpoint{3.793912in}{0.557870in}}{\pgfqpoint{2.446088in}{1.684734in}}%
\pgfusepath{clip}%
\pgfsetbuttcap%
\pgfsetroundjoin%
\definecolor{currentfill}{rgb}{0.298039,0.447059,0.690196}%
\pgfsetfillcolor{currentfill}%
\pgfsetlinewidth{1.003750pt}%
\definecolor{currentstroke}{rgb}{0.298039,0.447059,0.690196}%
\pgfsetstrokecolor{currentstroke}%
\pgfsetdash{}{0pt}%
\pgfpathmoveto{\pgfqpoint{3.905098in}{2.119498in}}%
\pgfpathcurveto{\pgfqpoint{3.913334in}{2.119498in}}{\pgfqpoint{3.921234in}{2.122771in}}{\pgfqpoint{3.927058in}{2.128595in}}%
\pgfpathcurveto{\pgfqpoint{3.932882in}{2.134419in}}{\pgfqpoint{3.936155in}{2.142319in}}{\pgfqpoint{3.936155in}{2.150555in}}%
\pgfpathcurveto{\pgfqpoint{3.936155in}{2.158791in}}{\pgfqpoint{3.932882in}{2.166691in}}{\pgfqpoint{3.927058in}{2.172515in}}%
\pgfpathcurveto{\pgfqpoint{3.921234in}{2.178339in}}{\pgfqpoint{3.913334in}{2.181611in}}{\pgfqpoint{3.905098in}{2.181611in}}%
\pgfpathcurveto{\pgfqpoint{3.896862in}{2.181611in}}{\pgfqpoint{3.888962in}{2.178339in}}{\pgfqpoint{3.883138in}{2.172515in}}%
\pgfpathcurveto{\pgfqpoint{3.877314in}{2.166691in}}{\pgfqpoint{3.874042in}{2.158791in}}{\pgfqpoint{3.874042in}{2.150555in}}%
\pgfpathcurveto{\pgfqpoint{3.874042in}{2.142319in}}{\pgfqpoint{3.877314in}{2.134419in}}{\pgfqpoint{3.883138in}{2.128595in}}%
\pgfpathcurveto{\pgfqpoint{3.888962in}{2.122771in}}{\pgfqpoint{3.896862in}{2.119498in}}{\pgfqpoint{3.905098in}{2.119498in}}%
\pgfpathclose%
\pgfusepath{stroke,fill}%
\end{pgfscope}%
\begin{pgfscope}%
\pgfpathrectangle{\pgfqpoint{3.793912in}{0.557870in}}{\pgfqpoint{2.446088in}{1.684734in}}%
\pgfusepath{clip}%
\pgfsetbuttcap%
\pgfsetroundjoin%
\definecolor{currentfill}{rgb}{0.298039,0.447059,0.690196}%
\pgfsetfillcolor{currentfill}%
\pgfsetlinewidth{1.003750pt}%
\definecolor{currentstroke}{rgb}{0.298039,0.447059,0.690196}%
\pgfsetstrokecolor{currentstroke}%
\pgfsetdash{}{0pt}%
\pgfpathmoveto{\pgfqpoint{5.509842in}{1.748207in}}%
\pgfpathcurveto{\pgfqpoint{5.518078in}{1.748207in}}{\pgfqpoint{5.525978in}{1.751479in}}{\pgfqpoint{5.531802in}{1.757303in}}%
\pgfpathcurveto{\pgfqpoint{5.537626in}{1.763127in}}{\pgfqpoint{5.540898in}{1.771027in}}{\pgfqpoint{5.540898in}{1.779264in}}%
\pgfpathcurveto{\pgfqpoint{5.540898in}{1.787500in}}{\pgfqpoint{5.537626in}{1.795400in}}{\pgfqpoint{5.531802in}{1.801224in}}%
\pgfpathcurveto{\pgfqpoint{5.525978in}{1.807048in}}{\pgfqpoint{5.518078in}{1.810320in}}{\pgfqpoint{5.509842in}{1.810320in}}%
\pgfpathcurveto{\pgfqpoint{5.501605in}{1.810320in}}{\pgfqpoint{5.493705in}{1.807048in}}{\pgfqpoint{5.487881in}{1.801224in}}%
\pgfpathcurveto{\pgfqpoint{5.482057in}{1.795400in}}{\pgfqpoint{5.478785in}{1.787500in}}{\pgfqpoint{5.478785in}{1.779264in}}%
\pgfpathcurveto{\pgfqpoint{5.478785in}{1.771027in}}{\pgfqpoint{5.482057in}{1.763127in}}{\pgfqpoint{5.487881in}{1.757303in}}%
\pgfpathcurveto{\pgfqpoint{5.493705in}{1.751479in}}{\pgfqpoint{5.501605in}{1.748207in}}{\pgfqpoint{5.509842in}{1.748207in}}%
\pgfpathclose%
\pgfusepath{stroke,fill}%
\end{pgfscope}%
\begin{pgfscope}%
\pgfpathrectangle{\pgfqpoint{3.793912in}{0.557870in}}{\pgfqpoint{2.446088in}{1.684734in}}%
\pgfusepath{clip}%
\pgfsetbuttcap%
\pgfsetroundjoin%
\definecolor{currentfill}{rgb}{0.298039,0.447059,0.690196}%
\pgfsetfillcolor{currentfill}%
\pgfsetlinewidth{1.003750pt}%
\definecolor{currentstroke}{rgb}{0.298039,0.447059,0.690196}%
\pgfsetstrokecolor{currentstroke}%
\pgfsetdash{}{0pt}%
\pgfpathmoveto{\pgfqpoint{3.905098in}{2.119498in}}%
\pgfpathcurveto{\pgfqpoint{3.913334in}{2.119498in}}{\pgfqpoint{3.921234in}{2.122771in}}{\pgfqpoint{3.927058in}{2.128595in}}%
\pgfpathcurveto{\pgfqpoint{3.932882in}{2.134419in}}{\pgfqpoint{3.936155in}{2.142319in}}{\pgfqpoint{3.936155in}{2.150555in}}%
\pgfpathcurveto{\pgfqpoint{3.936155in}{2.158791in}}{\pgfqpoint{3.932882in}{2.166691in}}{\pgfqpoint{3.927058in}{2.172515in}}%
\pgfpathcurveto{\pgfqpoint{3.921234in}{2.178339in}}{\pgfqpoint{3.913334in}{2.181611in}}{\pgfqpoint{3.905098in}{2.181611in}}%
\pgfpathcurveto{\pgfqpoint{3.896862in}{2.181611in}}{\pgfqpoint{3.888962in}{2.178339in}}{\pgfqpoint{3.883138in}{2.172515in}}%
\pgfpathcurveto{\pgfqpoint{3.877314in}{2.166691in}}{\pgfqpoint{3.874042in}{2.158791in}}{\pgfqpoint{3.874042in}{2.150555in}}%
\pgfpathcurveto{\pgfqpoint{3.874042in}{2.142319in}}{\pgfqpoint{3.877314in}{2.134419in}}{\pgfqpoint{3.883138in}{2.128595in}}%
\pgfpathcurveto{\pgfqpoint{3.888962in}{2.122771in}}{\pgfqpoint{3.896862in}{2.119498in}}{\pgfqpoint{3.905098in}{2.119498in}}%
\pgfpathclose%
\pgfusepath{stroke,fill}%
\end{pgfscope}%
\begin{pgfscope}%
\pgfpathrectangle{\pgfqpoint{3.793912in}{0.557870in}}{\pgfqpoint{2.446088in}{1.684734in}}%
\pgfusepath{clip}%
\pgfsetbuttcap%
\pgfsetroundjoin%
\definecolor{currentfill}{rgb}{0.298039,0.447059,0.690196}%
\pgfsetfillcolor{currentfill}%
\pgfsetlinewidth{1.003750pt}%
\definecolor{currentstroke}{rgb}{0.298039,0.447059,0.690196}%
\pgfsetstrokecolor{currentstroke}%
\pgfsetdash{}{0pt}%
\pgfpathmoveto{\pgfqpoint{5.395217in}{1.856500in}}%
\pgfpathcurveto{\pgfqpoint{5.403453in}{1.856500in}}{\pgfqpoint{5.411353in}{1.859773in}}{\pgfqpoint{5.417177in}{1.865597in}}%
\pgfpathcurveto{\pgfqpoint{5.423001in}{1.871421in}}{\pgfqpoint{5.426274in}{1.879321in}}{\pgfqpoint{5.426274in}{1.887557in}}%
\pgfpathcurveto{\pgfqpoint{5.426274in}{1.895793in}}{\pgfqpoint{5.423001in}{1.903693in}}{\pgfqpoint{5.417177in}{1.909517in}}%
\pgfpathcurveto{\pgfqpoint{5.411353in}{1.915341in}}{\pgfqpoint{5.403453in}{1.918613in}}{\pgfqpoint{5.395217in}{1.918613in}}%
\pgfpathcurveto{\pgfqpoint{5.386981in}{1.918613in}}{\pgfqpoint{5.379081in}{1.915341in}}{\pgfqpoint{5.373257in}{1.909517in}}%
\pgfpathcurveto{\pgfqpoint{5.367433in}{1.903693in}}{\pgfqpoint{5.364161in}{1.895793in}}{\pgfqpoint{5.364161in}{1.887557in}}%
\pgfpathcurveto{\pgfqpoint{5.364161in}{1.879321in}}{\pgfqpoint{5.367433in}{1.871421in}}{\pgfqpoint{5.373257in}{1.865597in}}%
\pgfpathcurveto{\pgfqpoint{5.379081in}{1.859773in}}{\pgfqpoint{5.386981in}{1.856500in}}{\pgfqpoint{5.395217in}{1.856500in}}%
\pgfpathclose%
\pgfusepath{stroke,fill}%
\end{pgfscope}%
\begin{pgfscope}%
\pgfpathrectangle{\pgfqpoint{3.793912in}{0.557870in}}{\pgfqpoint{2.446088in}{1.684734in}}%
\pgfusepath{clip}%
\pgfsetbuttcap%
\pgfsetroundjoin%
\definecolor{currentfill}{rgb}{0.298039,0.447059,0.690196}%
\pgfsetfillcolor{currentfill}%
\pgfsetlinewidth{1.003750pt}%
\definecolor{currentstroke}{rgb}{0.298039,0.447059,0.690196}%
\pgfsetstrokecolor{currentstroke}%
\pgfsetdash{}{0pt}%
\pgfpathmoveto{\pgfqpoint{3.905098in}{2.119498in}}%
\pgfpathcurveto{\pgfqpoint{3.913334in}{2.119498in}}{\pgfqpoint{3.921234in}{2.122771in}}{\pgfqpoint{3.927058in}{2.128595in}}%
\pgfpathcurveto{\pgfqpoint{3.932882in}{2.134419in}}{\pgfqpoint{3.936155in}{2.142319in}}{\pgfqpoint{3.936155in}{2.150555in}}%
\pgfpathcurveto{\pgfqpoint{3.936155in}{2.158791in}}{\pgfqpoint{3.932882in}{2.166691in}}{\pgfqpoint{3.927058in}{2.172515in}}%
\pgfpathcurveto{\pgfqpoint{3.921234in}{2.178339in}}{\pgfqpoint{3.913334in}{2.181611in}}{\pgfqpoint{3.905098in}{2.181611in}}%
\pgfpathcurveto{\pgfqpoint{3.896862in}{2.181611in}}{\pgfqpoint{3.888962in}{2.178339in}}{\pgfqpoint{3.883138in}{2.172515in}}%
\pgfpathcurveto{\pgfqpoint{3.877314in}{2.166691in}}{\pgfqpoint{3.874042in}{2.158791in}}{\pgfqpoint{3.874042in}{2.150555in}}%
\pgfpathcurveto{\pgfqpoint{3.874042in}{2.142319in}}{\pgfqpoint{3.877314in}{2.134419in}}{\pgfqpoint{3.883138in}{2.128595in}}%
\pgfpathcurveto{\pgfqpoint{3.888962in}{2.122771in}}{\pgfqpoint{3.896862in}{2.119498in}}{\pgfqpoint{3.905098in}{2.119498in}}%
\pgfpathclose%
\pgfusepath{stroke,fill}%
\end{pgfscope}%
\begin{pgfscope}%
\pgfpathrectangle{\pgfqpoint{3.793912in}{0.557870in}}{\pgfqpoint{2.446088in}{1.684734in}}%
\pgfusepath{clip}%
\pgfsetbuttcap%
\pgfsetroundjoin%
\definecolor{currentfill}{rgb}{0.298039,0.447059,0.690196}%
\pgfsetfillcolor{currentfill}%
\pgfsetlinewidth{1.003750pt}%
\definecolor{currentstroke}{rgb}{0.298039,0.447059,0.690196}%
\pgfsetstrokecolor{currentstroke}%
\pgfsetdash{}{0pt}%
\pgfpathmoveto{\pgfqpoint{3.905098in}{2.119498in}}%
\pgfpathcurveto{\pgfqpoint{3.913334in}{2.119498in}}{\pgfqpoint{3.921234in}{2.122771in}}{\pgfqpoint{3.927058in}{2.128595in}}%
\pgfpathcurveto{\pgfqpoint{3.932882in}{2.134419in}}{\pgfqpoint{3.936155in}{2.142319in}}{\pgfqpoint{3.936155in}{2.150555in}}%
\pgfpathcurveto{\pgfqpoint{3.936155in}{2.158791in}}{\pgfqpoint{3.932882in}{2.166691in}}{\pgfqpoint{3.927058in}{2.172515in}}%
\pgfpathcurveto{\pgfqpoint{3.921234in}{2.178339in}}{\pgfqpoint{3.913334in}{2.181611in}}{\pgfqpoint{3.905098in}{2.181611in}}%
\pgfpathcurveto{\pgfqpoint{3.896862in}{2.181611in}}{\pgfqpoint{3.888962in}{2.178339in}}{\pgfqpoint{3.883138in}{2.172515in}}%
\pgfpathcurveto{\pgfqpoint{3.877314in}{2.166691in}}{\pgfqpoint{3.874042in}{2.158791in}}{\pgfqpoint{3.874042in}{2.150555in}}%
\pgfpathcurveto{\pgfqpoint{3.874042in}{2.142319in}}{\pgfqpoint{3.877314in}{2.134419in}}{\pgfqpoint{3.883138in}{2.128595in}}%
\pgfpathcurveto{\pgfqpoint{3.888962in}{2.122771in}}{\pgfqpoint{3.896862in}{2.119498in}}{\pgfqpoint{3.905098in}{2.119498in}}%
\pgfpathclose%
\pgfusepath{stroke,fill}%
\end{pgfscope}%
\begin{pgfscope}%
\pgfpathrectangle{\pgfqpoint{3.793912in}{0.557870in}}{\pgfqpoint{2.446088in}{1.684734in}}%
\pgfusepath{clip}%
\pgfsetbuttcap%
\pgfsetroundjoin%
\definecolor{currentfill}{rgb}{0.298039,0.447059,0.690196}%
\pgfsetfillcolor{currentfill}%
\pgfsetlinewidth{1.003750pt}%
\definecolor{currentstroke}{rgb}{0.298039,0.447059,0.690196}%
\pgfsetstrokecolor{currentstroke}%
\pgfsetdash{}{0pt}%
\pgfpathmoveto{\pgfqpoint{5.234743in}{1.732737in}}%
\pgfpathcurveto{\pgfqpoint{5.242979in}{1.732737in}}{\pgfqpoint{5.250879in}{1.736009in}}{\pgfqpoint{5.256703in}{1.741833in}}%
\pgfpathcurveto{\pgfqpoint{5.262527in}{1.747657in}}{\pgfqpoint{5.265799in}{1.755557in}}{\pgfqpoint{5.265799in}{1.763793in}}%
\pgfpathcurveto{\pgfqpoint{5.265799in}{1.772029in}}{\pgfqpoint{5.262527in}{1.779930in}}{\pgfqpoint{5.256703in}{1.785753in}}%
\pgfpathcurveto{\pgfqpoint{5.250879in}{1.791577in}}{\pgfqpoint{5.242979in}{1.794850in}}{\pgfqpoint{5.234743in}{1.794850in}}%
\pgfpathcurveto{\pgfqpoint{5.226506in}{1.794850in}}{\pgfqpoint{5.218606in}{1.791577in}}{\pgfqpoint{5.212782in}{1.785753in}}%
\pgfpathcurveto{\pgfqpoint{5.206959in}{1.779930in}}{\pgfqpoint{5.203686in}{1.772029in}}{\pgfqpoint{5.203686in}{1.763793in}}%
\pgfpathcurveto{\pgfqpoint{5.203686in}{1.755557in}}{\pgfqpoint{5.206959in}{1.747657in}}{\pgfqpoint{5.212782in}{1.741833in}}%
\pgfpathcurveto{\pgfqpoint{5.218606in}{1.736009in}}{\pgfqpoint{5.226506in}{1.732737in}}{\pgfqpoint{5.234743in}{1.732737in}}%
\pgfpathclose%
\pgfusepath{stroke,fill}%
\end{pgfscope}%
\begin{pgfscope}%
\pgfpathrectangle{\pgfqpoint{3.793912in}{0.557870in}}{\pgfqpoint{2.446088in}{1.684734in}}%
\pgfusepath{clip}%
\pgfsetbuttcap%
\pgfsetroundjoin%
\definecolor{currentfill}{rgb}{0.298039,0.447059,0.690196}%
\pgfsetfillcolor{currentfill}%
\pgfsetlinewidth{1.003750pt}%
\definecolor{currentstroke}{rgb}{0.298039,0.447059,0.690196}%
\pgfsetstrokecolor{currentstroke}%
\pgfsetdash{}{0pt}%
\pgfpathmoveto{\pgfqpoint{5.624466in}{1.500680in}}%
\pgfpathcurveto{\pgfqpoint{5.632702in}{1.500680in}}{\pgfqpoint{5.640603in}{1.503952in}}{\pgfqpoint{5.646426in}{1.509776in}}%
\pgfpathcurveto{\pgfqpoint{5.652250in}{1.515600in}}{\pgfqpoint{5.655523in}{1.523500in}}{\pgfqpoint{5.655523in}{1.531736in}}%
\pgfpathcurveto{\pgfqpoint{5.655523in}{1.539972in}}{\pgfqpoint{5.652250in}{1.547873in}}{\pgfqpoint{5.646426in}{1.553696in}}%
\pgfpathcurveto{\pgfqpoint{5.640603in}{1.559520in}}{\pgfqpoint{5.632702in}{1.562793in}}{\pgfqpoint{5.624466in}{1.562793in}}%
\pgfpathcurveto{\pgfqpoint{5.616230in}{1.562793in}}{\pgfqpoint{5.608330in}{1.559520in}}{\pgfqpoint{5.602506in}{1.553696in}}%
\pgfpathcurveto{\pgfqpoint{5.596682in}{1.547873in}}{\pgfqpoint{5.593410in}{1.539972in}}{\pgfqpoint{5.593410in}{1.531736in}}%
\pgfpathcurveto{\pgfqpoint{5.593410in}{1.523500in}}{\pgfqpoint{5.596682in}{1.515600in}}{\pgfqpoint{5.602506in}{1.509776in}}%
\pgfpathcurveto{\pgfqpoint{5.608330in}{1.503952in}}{\pgfqpoint{5.616230in}{1.500680in}}{\pgfqpoint{5.624466in}{1.500680in}}%
\pgfpathclose%
\pgfusepath{stroke,fill}%
\end{pgfscope}%
\begin{pgfscope}%
\pgfpathrectangle{\pgfqpoint{3.793912in}{0.557870in}}{\pgfqpoint{2.446088in}{1.684734in}}%
\pgfusepath{clip}%
\pgfsetbuttcap%
\pgfsetroundjoin%
\definecolor{currentfill}{rgb}{0.298039,0.447059,0.690196}%
\pgfsetfillcolor{currentfill}%
\pgfsetlinewidth{1.003750pt}%
\definecolor{currentstroke}{rgb}{0.298039,0.447059,0.690196}%
\pgfsetstrokecolor{currentstroke}%
\pgfsetdash{}{0pt}%
\pgfpathmoveto{\pgfqpoint{5.647391in}{1.454268in}}%
\pgfpathcurveto{\pgfqpoint{5.655627in}{1.454268in}}{\pgfqpoint{5.663527in}{1.457541in}}{\pgfqpoint{5.669351in}{1.463365in}}%
\pgfpathcurveto{\pgfqpoint{5.675175in}{1.469188in}}{\pgfqpoint{5.678448in}{1.477089in}}{\pgfqpoint{5.678448in}{1.485325in}}%
\pgfpathcurveto{\pgfqpoint{5.678448in}{1.493561in}}{\pgfqpoint{5.675175in}{1.501461in}}{\pgfqpoint{5.669351in}{1.507285in}}%
\pgfpathcurveto{\pgfqpoint{5.663527in}{1.513109in}}{\pgfqpoint{5.655627in}{1.516381in}}{\pgfqpoint{5.647391in}{1.516381in}}%
\pgfpathcurveto{\pgfqpoint{5.639155in}{1.516381in}}{\pgfqpoint{5.631255in}{1.513109in}}{\pgfqpoint{5.625431in}{1.507285in}}%
\pgfpathcurveto{\pgfqpoint{5.619607in}{1.501461in}}{\pgfqpoint{5.616335in}{1.493561in}}{\pgfqpoint{5.616335in}{1.485325in}}%
\pgfpathcurveto{\pgfqpoint{5.616335in}{1.477089in}}{\pgfqpoint{5.619607in}{1.469188in}}{\pgfqpoint{5.625431in}{1.463365in}}%
\pgfpathcurveto{\pgfqpoint{5.631255in}{1.457541in}}{\pgfqpoint{5.639155in}{1.454268in}}{\pgfqpoint{5.647391in}{1.454268in}}%
\pgfpathclose%
\pgfusepath{stroke,fill}%
\end{pgfscope}%
\begin{pgfscope}%
\pgfpathrectangle{\pgfqpoint{3.793912in}{0.557870in}}{\pgfqpoint{2.446088in}{1.684734in}}%
\pgfusepath{clip}%
\pgfsetbuttcap%
\pgfsetroundjoin%
\definecolor{currentfill}{rgb}{0.298039,0.447059,0.690196}%
\pgfsetfillcolor{currentfill}%
\pgfsetlinewidth{1.003750pt}%
\definecolor{currentstroke}{rgb}{0.298039,0.447059,0.690196}%
\pgfsetstrokecolor{currentstroke}%
\pgfsetdash{}{0pt}%
\pgfpathmoveto{\pgfqpoint{5.693241in}{1.392386in}}%
\pgfpathcurveto{\pgfqpoint{5.701477in}{1.392386in}}{\pgfqpoint{5.709377in}{1.395659in}}{\pgfqpoint{5.715201in}{1.401483in}}%
\pgfpathcurveto{\pgfqpoint{5.721025in}{1.407307in}}{\pgfqpoint{5.724297in}{1.415207in}}{\pgfqpoint{5.724297in}{1.423443in}}%
\pgfpathcurveto{\pgfqpoint{5.724297in}{1.431679in}}{\pgfqpoint{5.721025in}{1.439579in}}{\pgfqpoint{5.715201in}{1.445403in}}%
\pgfpathcurveto{\pgfqpoint{5.709377in}{1.451227in}}{\pgfqpoint{5.701477in}{1.454499in}}{\pgfqpoint{5.693241in}{1.454499in}}%
\pgfpathcurveto{\pgfqpoint{5.685005in}{1.454499in}}{\pgfqpoint{5.677105in}{1.451227in}}{\pgfqpoint{5.671281in}{1.445403in}}%
\pgfpathcurveto{\pgfqpoint{5.665457in}{1.439579in}}{\pgfqpoint{5.662184in}{1.431679in}}{\pgfqpoint{5.662184in}{1.423443in}}%
\pgfpathcurveto{\pgfqpoint{5.662184in}{1.415207in}}{\pgfqpoint{5.665457in}{1.407307in}}{\pgfqpoint{5.671281in}{1.401483in}}%
\pgfpathcurveto{\pgfqpoint{5.677105in}{1.395659in}}{\pgfqpoint{5.685005in}{1.392386in}}{\pgfqpoint{5.693241in}{1.392386in}}%
\pgfpathclose%
\pgfusepath{stroke,fill}%
\end{pgfscope}%
\begin{pgfscope}%
\pgfpathrectangle{\pgfqpoint{3.793912in}{0.557870in}}{\pgfqpoint{2.446088in}{1.684734in}}%
\pgfusepath{clip}%
\pgfsetbuttcap%
\pgfsetroundjoin%
\definecolor{currentfill}{rgb}{0.298039,0.447059,0.690196}%
\pgfsetfillcolor{currentfill}%
\pgfsetlinewidth{1.003750pt}%
\definecolor{currentstroke}{rgb}{0.298039,0.447059,0.690196}%
\pgfsetstrokecolor{currentstroke}%
\pgfsetdash{}{0pt}%
\pgfpathmoveto{\pgfqpoint{6.037115in}{1.005625in}}%
\pgfpathcurveto{\pgfqpoint{6.045351in}{1.005625in}}{\pgfqpoint{6.053251in}{1.008897in}}{\pgfqpoint{6.059075in}{1.014721in}}%
\pgfpathcurveto{\pgfqpoint{6.064899in}{1.020545in}}{\pgfqpoint{6.068171in}{1.028445in}}{\pgfqpoint{6.068171in}{1.036681in}}%
\pgfpathcurveto{\pgfqpoint{6.068171in}{1.044918in}}{\pgfqpoint{6.064899in}{1.052818in}}{\pgfqpoint{6.059075in}{1.058642in}}%
\pgfpathcurveto{\pgfqpoint{6.053251in}{1.064465in}}{\pgfqpoint{6.045351in}{1.067738in}}{\pgfqpoint{6.037115in}{1.067738in}}%
\pgfpathcurveto{\pgfqpoint{6.028878in}{1.067738in}}{\pgfqpoint{6.020978in}{1.064465in}}{\pgfqpoint{6.015154in}{1.058642in}}%
\pgfpathcurveto{\pgfqpoint{6.009330in}{1.052818in}}{\pgfqpoint{6.006058in}{1.044918in}}{\pgfqpoint{6.006058in}{1.036681in}}%
\pgfpathcurveto{\pgfqpoint{6.006058in}{1.028445in}}{\pgfqpoint{6.009330in}{1.020545in}}{\pgfqpoint{6.015154in}{1.014721in}}%
\pgfpathcurveto{\pgfqpoint{6.020978in}{1.008897in}}{\pgfqpoint{6.028878in}{1.005625in}}{\pgfqpoint{6.037115in}{1.005625in}}%
\pgfpathclose%
\pgfusepath{stroke,fill}%
\end{pgfscope}%
\begin{pgfscope}%
\pgfpathrectangle{\pgfqpoint{3.793912in}{0.557870in}}{\pgfqpoint{2.446088in}{1.684734in}}%
\pgfusepath{clip}%
\pgfsetbuttcap%
\pgfsetroundjoin%
\definecolor{currentfill}{rgb}{0.298039,0.447059,0.690196}%
\pgfsetfillcolor{currentfill}%
\pgfsetlinewidth{1.003750pt}%
\definecolor{currentstroke}{rgb}{0.298039,0.447059,0.690196}%
\pgfsetstrokecolor{currentstroke}%
\pgfsetdash{}{0pt}%
\pgfpathmoveto{\pgfqpoint{5.876640in}{1.052036in}}%
\pgfpathcurveto{\pgfqpoint{5.884876in}{1.052036in}}{\pgfqpoint{5.892777in}{1.055308in}}{\pgfqpoint{5.898600in}{1.061132in}}%
\pgfpathcurveto{\pgfqpoint{5.904424in}{1.066956in}}{\pgfqpoint{5.907697in}{1.074856in}}{\pgfqpoint{5.907697in}{1.083093in}}%
\pgfpathcurveto{\pgfqpoint{5.907697in}{1.091329in}}{\pgfqpoint{5.904424in}{1.099229in}}{\pgfqpoint{5.898600in}{1.105053in}}%
\pgfpathcurveto{\pgfqpoint{5.892777in}{1.110877in}}{\pgfqpoint{5.884876in}{1.114149in}}{\pgfqpoint{5.876640in}{1.114149in}}%
\pgfpathcurveto{\pgfqpoint{5.868404in}{1.114149in}}{\pgfqpoint{5.860504in}{1.110877in}}{\pgfqpoint{5.854680in}{1.105053in}}%
\pgfpathcurveto{\pgfqpoint{5.848856in}{1.099229in}}{\pgfqpoint{5.845584in}{1.091329in}}{\pgfqpoint{5.845584in}{1.083093in}}%
\pgfpathcurveto{\pgfqpoint{5.845584in}{1.074856in}}{\pgfqpoint{5.848856in}{1.066956in}}{\pgfqpoint{5.854680in}{1.061132in}}%
\pgfpathcurveto{\pgfqpoint{5.860504in}{1.055308in}}{\pgfqpoint{5.868404in}{1.052036in}}{\pgfqpoint{5.876640in}{1.052036in}}%
\pgfpathclose%
\pgfusepath{stroke,fill}%
\end{pgfscope}%
\begin{pgfscope}%
\pgfpathrectangle{\pgfqpoint{3.793912in}{0.557870in}}{\pgfqpoint{2.446088in}{1.684734in}}%
\pgfusepath{clip}%
\pgfsetbuttcap%
\pgfsetroundjoin%
\definecolor{currentfill}{rgb}{0.298039,0.447059,0.690196}%
\pgfsetfillcolor{currentfill}%
\pgfsetlinewidth{1.003750pt}%
\definecolor{currentstroke}{rgb}{0.298039,0.447059,0.690196}%
\pgfsetstrokecolor{currentstroke}%
\pgfsetdash{}{0pt}%
\pgfpathmoveto{\pgfqpoint{3.905098in}{2.119498in}}%
\pgfpathcurveto{\pgfqpoint{3.913334in}{2.119498in}}{\pgfqpoint{3.921234in}{2.122771in}}{\pgfqpoint{3.927058in}{2.128595in}}%
\pgfpathcurveto{\pgfqpoint{3.932882in}{2.134419in}}{\pgfqpoint{3.936155in}{2.142319in}}{\pgfqpoint{3.936155in}{2.150555in}}%
\pgfpathcurveto{\pgfqpoint{3.936155in}{2.158791in}}{\pgfqpoint{3.932882in}{2.166691in}}{\pgfqpoint{3.927058in}{2.172515in}}%
\pgfpathcurveto{\pgfqpoint{3.921234in}{2.178339in}}{\pgfqpoint{3.913334in}{2.181611in}}{\pgfqpoint{3.905098in}{2.181611in}}%
\pgfpathcurveto{\pgfqpoint{3.896862in}{2.181611in}}{\pgfqpoint{3.888962in}{2.178339in}}{\pgfqpoint{3.883138in}{2.172515in}}%
\pgfpathcurveto{\pgfqpoint{3.877314in}{2.166691in}}{\pgfqpoint{3.874042in}{2.158791in}}{\pgfqpoint{3.874042in}{2.150555in}}%
\pgfpathcurveto{\pgfqpoint{3.874042in}{2.142319in}}{\pgfqpoint{3.877314in}{2.134419in}}{\pgfqpoint{3.883138in}{2.128595in}}%
\pgfpathcurveto{\pgfqpoint{3.888962in}{2.122771in}}{\pgfqpoint{3.896862in}{2.119498in}}{\pgfqpoint{3.905098in}{2.119498in}}%
\pgfpathclose%
\pgfusepath{stroke,fill}%
\end{pgfscope}%
\begin{pgfscope}%
\pgfpathrectangle{\pgfqpoint{3.793912in}{0.557870in}}{\pgfqpoint{2.446088in}{1.684734in}}%
\pgfusepath{clip}%
\pgfsetbuttcap%
\pgfsetroundjoin%
\definecolor{currentfill}{rgb}{0.298039,0.447059,0.690196}%
\pgfsetfillcolor{currentfill}%
\pgfsetlinewidth{1.003750pt}%
\definecolor{currentstroke}{rgb}{0.298039,0.447059,0.690196}%
\pgfsetstrokecolor{currentstroke}%
\pgfsetdash{}{0pt}%
\pgfpathmoveto{\pgfqpoint{5.784941in}{0.959213in}}%
\pgfpathcurveto{\pgfqpoint{5.793177in}{0.959213in}}{\pgfqpoint{5.801077in}{0.962486in}}{\pgfqpoint{5.806901in}{0.968310in}}%
\pgfpathcurveto{\pgfqpoint{5.812725in}{0.974134in}}{\pgfqpoint{5.815997in}{0.982034in}}{\pgfqpoint{5.815997in}{0.990270in}}%
\pgfpathcurveto{\pgfqpoint{5.815997in}{0.998506in}}{\pgfqpoint{5.812725in}{1.006406in}}{\pgfqpoint{5.806901in}{1.012230in}}%
\pgfpathcurveto{\pgfqpoint{5.801077in}{1.018054in}}{\pgfqpoint{5.793177in}{1.021326in}}{\pgfqpoint{5.784941in}{1.021326in}}%
\pgfpathcurveto{\pgfqpoint{5.776704in}{1.021326in}}{\pgfqpoint{5.768804in}{1.018054in}}{\pgfqpoint{5.762980in}{1.012230in}}%
\pgfpathcurveto{\pgfqpoint{5.757156in}{1.006406in}}{\pgfqpoint{5.753884in}{0.998506in}}{\pgfqpoint{5.753884in}{0.990270in}}%
\pgfpathcurveto{\pgfqpoint{5.753884in}{0.982034in}}{\pgfqpoint{5.757156in}{0.974134in}}{\pgfqpoint{5.762980in}{0.968310in}}%
\pgfpathcurveto{\pgfqpoint{5.768804in}{0.962486in}}{\pgfqpoint{5.776704in}{0.959213in}}{\pgfqpoint{5.784941in}{0.959213in}}%
\pgfpathclose%
\pgfusepath{stroke,fill}%
\end{pgfscope}%
\begin{pgfscope}%
\pgfpathrectangle{\pgfqpoint{3.793912in}{0.557870in}}{\pgfqpoint{2.446088in}{1.684734in}}%
\pgfusepath{clip}%
\pgfsetbuttcap%
\pgfsetroundjoin%
\definecolor{currentfill}{rgb}{0.298039,0.447059,0.690196}%
\pgfsetfillcolor{currentfill}%
\pgfsetlinewidth{1.003750pt}%
\definecolor{currentstroke}{rgb}{0.298039,0.447059,0.690196}%
\pgfsetstrokecolor{currentstroke}%
\pgfsetdash{}{0pt}%
\pgfpathmoveto{\pgfqpoint{3.905098in}{2.119498in}}%
\pgfpathcurveto{\pgfqpoint{3.913334in}{2.119498in}}{\pgfqpoint{3.921234in}{2.122771in}}{\pgfqpoint{3.927058in}{2.128595in}}%
\pgfpathcurveto{\pgfqpoint{3.932882in}{2.134419in}}{\pgfqpoint{3.936155in}{2.142319in}}{\pgfqpoint{3.936155in}{2.150555in}}%
\pgfpathcurveto{\pgfqpoint{3.936155in}{2.158791in}}{\pgfqpoint{3.932882in}{2.166691in}}{\pgfqpoint{3.927058in}{2.172515in}}%
\pgfpathcurveto{\pgfqpoint{3.921234in}{2.178339in}}{\pgfqpoint{3.913334in}{2.181611in}}{\pgfqpoint{3.905098in}{2.181611in}}%
\pgfpathcurveto{\pgfqpoint{3.896862in}{2.181611in}}{\pgfqpoint{3.888962in}{2.178339in}}{\pgfqpoint{3.883138in}{2.172515in}}%
\pgfpathcurveto{\pgfqpoint{3.877314in}{2.166691in}}{\pgfqpoint{3.874042in}{2.158791in}}{\pgfqpoint{3.874042in}{2.150555in}}%
\pgfpathcurveto{\pgfqpoint{3.874042in}{2.142319in}}{\pgfqpoint{3.877314in}{2.134419in}}{\pgfqpoint{3.883138in}{2.128595in}}%
\pgfpathcurveto{\pgfqpoint{3.888962in}{2.122771in}}{\pgfqpoint{3.896862in}{2.119498in}}{\pgfqpoint{3.905098in}{2.119498in}}%
\pgfpathclose%
\pgfusepath{stroke,fill}%
\end{pgfscope}%
\begin{pgfscope}%
\pgfpathrectangle{\pgfqpoint{3.793912in}{0.557870in}}{\pgfqpoint{2.446088in}{1.684734in}}%
\pgfusepath{clip}%
\pgfsetbuttcap%
\pgfsetroundjoin%
\definecolor{currentfill}{rgb}{0.298039,0.447059,0.690196}%
\pgfsetfillcolor{currentfill}%
\pgfsetlinewidth{1.003750pt}%
\definecolor{currentstroke}{rgb}{0.298039,0.447059,0.690196}%
\pgfsetstrokecolor{currentstroke}%
\pgfsetdash{}{0pt}%
\pgfpathmoveto{\pgfqpoint{5.762016in}{0.990154in}}%
\pgfpathcurveto{\pgfqpoint{5.770252in}{0.990154in}}{\pgfqpoint{5.778152in}{0.993427in}}{\pgfqpoint{5.783976in}{0.999251in}}%
\pgfpathcurveto{\pgfqpoint{5.789800in}{1.005074in}}{\pgfqpoint{5.793072in}{1.012974in}}{\pgfqpoint{5.793072in}{1.021211in}}%
\pgfpathcurveto{\pgfqpoint{5.793072in}{1.029447in}}{\pgfqpoint{5.789800in}{1.037347in}}{\pgfqpoint{5.783976in}{1.043171in}}%
\pgfpathcurveto{\pgfqpoint{5.778152in}{1.048995in}}{\pgfqpoint{5.770252in}{1.052267in}}{\pgfqpoint{5.762016in}{1.052267in}}%
\pgfpathcurveto{\pgfqpoint{5.753779in}{1.052267in}}{\pgfqpoint{5.745879in}{1.048995in}}{\pgfqpoint{5.740055in}{1.043171in}}%
\pgfpathcurveto{\pgfqpoint{5.734231in}{1.037347in}}{\pgfqpoint{5.730959in}{1.029447in}}{\pgfqpoint{5.730959in}{1.021211in}}%
\pgfpathcurveto{\pgfqpoint{5.730959in}{1.012974in}}{\pgfqpoint{5.734231in}{1.005074in}}{\pgfqpoint{5.740055in}{0.999251in}}%
\pgfpathcurveto{\pgfqpoint{5.745879in}{0.993427in}}{\pgfqpoint{5.753779in}{0.990154in}}{\pgfqpoint{5.762016in}{0.990154in}}%
\pgfpathclose%
\pgfusepath{stroke,fill}%
\end{pgfscope}%
\begin{pgfscope}%
\pgfpathrectangle{\pgfqpoint{3.793912in}{0.557870in}}{\pgfqpoint{2.446088in}{1.684734in}}%
\pgfusepath{clip}%
\pgfsetbuttcap%
\pgfsetroundjoin%
\definecolor{currentfill}{rgb}{0.298039,0.447059,0.690196}%
\pgfsetfillcolor{currentfill}%
\pgfsetlinewidth{1.003750pt}%
\definecolor{currentstroke}{rgb}{0.298039,0.447059,0.690196}%
\pgfsetstrokecolor{currentstroke}%
\pgfsetdash{}{0pt}%
\pgfpathmoveto{\pgfqpoint{3.905098in}{2.119498in}}%
\pgfpathcurveto{\pgfqpoint{3.913334in}{2.119498in}}{\pgfqpoint{3.921234in}{2.122771in}}{\pgfqpoint{3.927058in}{2.128595in}}%
\pgfpathcurveto{\pgfqpoint{3.932882in}{2.134419in}}{\pgfqpoint{3.936155in}{2.142319in}}{\pgfqpoint{3.936155in}{2.150555in}}%
\pgfpathcurveto{\pgfqpoint{3.936155in}{2.158791in}}{\pgfqpoint{3.932882in}{2.166691in}}{\pgfqpoint{3.927058in}{2.172515in}}%
\pgfpathcurveto{\pgfqpoint{3.921234in}{2.178339in}}{\pgfqpoint{3.913334in}{2.181611in}}{\pgfqpoint{3.905098in}{2.181611in}}%
\pgfpathcurveto{\pgfqpoint{3.896862in}{2.181611in}}{\pgfqpoint{3.888962in}{2.178339in}}{\pgfqpoint{3.883138in}{2.172515in}}%
\pgfpathcurveto{\pgfqpoint{3.877314in}{2.166691in}}{\pgfqpoint{3.874042in}{2.158791in}}{\pgfqpoint{3.874042in}{2.150555in}}%
\pgfpathcurveto{\pgfqpoint{3.874042in}{2.142319in}}{\pgfqpoint{3.877314in}{2.134419in}}{\pgfqpoint{3.883138in}{2.128595in}}%
\pgfpathcurveto{\pgfqpoint{3.888962in}{2.122771in}}{\pgfqpoint{3.896862in}{2.119498in}}{\pgfqpoint{3.905098in}{2.119498in}}%
\pgfpathclose%
\pgfusepath{stroke,fill}%
\end{pgfscope}%
\begin{pgfscope}%
\pgfpathrectangle{\pgfqpoint{3.793912in}{0.557870in}}{\pgfqpoint{2.446088in}{1.684734in}}%
\pgfusepath{clip}%
\pgfsetbuttcap%
\pgfsetroundjoin%
\definecolor{currentfill}{rgb}{0.298039,0.447059,0.690196}%
\pgfsetfillcolor{currentfill}%
\pgfsetlinewidth{1.003750pt}%
\definecolor{currentstroke}{rgb}{0.298039,0.447059,0.690196}%
\pgfsetstrokecolor{currentstroke}%
\pgfsetdash{}{0pt}%
\pgfpathmoveto{\pgfqpoint{3.905098in}{2.119498in}}%
\pgfpathcurveto{\pgfqpoint{3.913334in}{2.119498in}}{\pgfqpoint{3.921234in}{2.122771in}}{\pgfqpoint{3.927058in}{2.128595in}}%
\pgfpathcurveto{\pgfqpoint{3.932882in}{2.134419in}}{\pgfqpoint{3.936155in}{2.142319in}}{\pgfqpoint{3.936155in}{2.150555in}}%
\pgfpathcurveto{\pgfqpoint{3.936155in}{2.158791in}}{\pgfqpoint{3.932882in}{2.166691in}}{\pgfqpoint{3.927058in}{2.172515in}}%
\pgfpathcurveto{\pgfqpoint{3.921234in}{2.178339in}}{\pgfqpoint{3.913334in}{2.181611in}}{\pgfqpoint{3.905098in}{2.181611in}}%
\pgfpathcurveto{\pgfqpoint{3.896862in}{2.181611in}}{\pgfqpoint{3.888962in}{2.178339in}}{\pgfqpoint{3.883138in}{2.172515in}}%
\pgfpathcurveto{\pgfqpoint{3.877314in}{2.166691in}}{\pgfqpoint{3.874042in}{2.158791in}}{\pgfqpoint{3.874042in}{2.150555in}}%
\pgfpathcurveto{\pgfqpoint{3.874042in}{2.142319in}}{\pgfqpoint{3.877314in}{2.134419in}}{\pgfqpoint{3.883138in}{2.128595in}}%
\pgfpathcurveto{\pgfqpoint{3.888962in}{2.122771in}}{\pgfqpoint{3.896862in}{2.119498in}}{\pgfqpoint{3.905098in}{2.119498in}}%
\pgfpathclose%
\pgfusepath{stroke,fill}%
\end{pgfscope}%
\begin{pgfscope}%
\pgfpathrectangle{\pgfqpoint{3.793912in}{0.557870in}}{\pgfqpoint{2.446088in}{1.684734in}}%
\pgfusepath{clip}%
\pgfsetbuttcap%
\pgfsetroundjoin%
\definecolor{currentfill}{rgb}{0.298039,0.447059,0.690196}%
\pgfsetfillcolor{currentfill}%
\pgfsetlinewidth{1.003750pt}%
\definecolor{currentstroke}{rgb}{0.298039,0.447059,0.690196}%
\pgfsetstrokecolor{currentstroke}%
\pgfsetdash{}{0pt}%
\pgfpathmoveto{\pgfqpoint{4.409446in}{1.810089in}}%
\pgfpathcurveto{\pgfqpoint{4.417682in}{1.810089in}}{\pgfqpoint{4.425582in}{1.813361in}}{\pgfqpoint{4.431406in}{1.819185in}}%
\pgfpathcurveto{\pgfqpoint{4.437230in}{1.825009in}}{\pgfqpoint{4.440503in}{1.832909in}}{\pgfqpoint{4.440503in}{1.841146in}}%
\pgfpathcurveto{\pgfqpoint{4.440503in}{1.849382in}}{\pgfqpoint{4.437230in}{1.857282in}}{\pgfqpoint{4.431406in}{1.863106in}}%
\pgfpathcurveto{\pgfqpoint{4.425582in}{1.868930in}}{\pgfqpoint{4.417682in}{1.872202in}}{\pgfqpoint{4.409446in}{1.872202in}}%
\pgfpathcurveto{\pgfqpoint{4.401210in}{1.872202in}}{\pgfqpoint{4.393310in}{1.868930in}}{\pgfqpoint{4.387486in}{1.863106in}}%
\pgfpathcurveto{\pgfqpoint{4.381662in}{1.857282in}}{\pgfqpoint{4.378390in}{1.849382in}}{\pgfqpoint{4.378390in}{1.841146in}}%
\pgfpathcurveto{\pgfqpoint{4.378390in}{1.832909in}}{\pgfqpoint{4.381662in}{1.825009in}}{\pgfqpoint{4.387486in}{1.819185in}}%
\pgfpathcurveto{\pgfqpoint{4.393310in}{1.813361in}}{\pgfqpoint{4.401210in}{1.810089in}}{\pgfqpoint{4.409446in}{1.810089in}}%
\pgfpathclose%
\pgfusepath{stroke,fill}%
\end{pgfscope}%
\begin{pgfscope}%
\pgfpathrectangle{\pgfqpoint{3.793912in}{0.557870in}}{\pgfqpoint{2.446088in}{1.684734in}}%
\pgfusepath{clip}%
\pgfsetbuttcap%
\pgfsetroundjoin%
\definecolor{currentfill}{rgb}{0.298039,0.447059,0.690196}%
\pgfsetfillcolor{currentfill}%
\pgfsetlinewidth{1.003750pt}%
\definecolor{currentstroke}{rgb}{0.298039,0.447059,0.690196}%
\pgfsetstrokecolor{currentstroke}%
\pgfsetdash{}{0pt}%
\pgfpathmoveto{\pgfqpoint{3.905098in}{2.119498in}}%
\pgfpathcurveto{\pgfqpoint{3.913334in}{2.119498in}}{\pgfqpoint{3.921234in}{2.122771in}}{\pgfqpoint{3.927058in}{2.128595in}}%
\pgfpathcurveto{\pgfqpoint{3.932882in}{2.134419in}}{\pgfqpoint{3.936155in}{2.142319in}}{\pgfqpoint{3.936155in}{2.150555in}}%
\pgfpathcurveto{\pgfqpoint{3.936155in}{2.158791in}}{\pgfqpoint{3.932882in}{2.166691in}}{\pgfqpoint{3.927058in}{2.172515in}}%
\pgfpathcurveto{\pgfqpoint{3.921234in}{2.178339in}}{\pgfqpoint{3.913334in}{2.181611in}}{\pgfqpoint{3.905098in}{2.181611in}}%
\pgfpathcurveto{\pgfqpoint{3.896862in}{2.181611in}}{\pgfqpoint{3.888962in}{2.178339in}}{\pgfqpoint{3.883138in}{2.172515in}}%
\pgfpathcurveto{\pgfqpoint{3.877314in}{2.166691in}}{\pgfqpoint{3.874042in}{2.158791in}}{\pgfqpoint{3.874042in}{2.150555in}}%
\pgfpathcurveto{\pgfqpoint{3.874042in}{2.142319in}}{\pgfqpoint{3.877314in}{2.134419in}}{\pgfqpoint{3.883138in}{2.128595in}}%
\pgfpathcurveto{\pgfqpoint{3.888962in}{2.122771in}}{\pgfqpoint{3.896862in}{2.119498in}}{\pgfqpoint{3.905098in}{2.119498in}}%
\pgfpathclose%
\pgfusepath{stroke,fill}%
\end{pgfscope}%
\begin{pgfscope}%
\pgfpathrectangle{\pgfqpoint{3.793912in}{0.557870in}}{\pgfqpoint{2.446088in}{1.684734in}}%
\pgfusepath{clip}%
\pgfsetbuttcap%
\pgfsetroundjoin%
\definecolor{currentfill}{rgb}{0.298039,0.447059,0.690196}%
\pgfsetfillcolor{currentfill}%
\pgfsetlinewidth{1.003750pt}%
\definecolor{currentstroke}{rgb}{0.298039,0.447059,0.690196}%
\pgfsetstrokecolor{currentstroke}%
\pgfsetdash{}{0pt}%
\pgfpathmoveto{\pgfqpoint{5.349367in}{1.253152in}}%
\pgfpathcurveto{\pgfqpoint{5.357604in}{1.253152in}}{\pgfqpoint{5.365504in}{1.256425in}}{\pgfqpoint{5.371328in}{1.262248in}}%
\pgfpathcurveto{\pgfqpoint{5.377151in}{1.268072in}}{\pgfqpoint{5.380424in}{1.275972in}}{\pgfqpoint{5.380424in}{1.284209in}}%
\pgfpathcurveto{\pgfqpoint{5.380424in}{1.292445in}}{\pgfqpoint{5.377151in}{1.300345in}}{\pgfqpoint{5.371328in}{1.306169in}}%
\pgfpathcurveto{\pgfqpoint{5.365504in}{1.311993in}}{\pgfqpoint{5.357604in}{1.315265in}}{\pgfqpoint{5.349367in}{1.315265in}}%
\pgfpathcurveto{\pgfqpoint{5.341131in}{1.315265in}}{\pgfqpoint{5.333231in}{1.311993in}}{\pgfqpoint{5.327407in}{1.306169in}}%
\pgfpathcurveto{\pgfqpoint{5.321583in}{1.300345in}}{\pgfqpoint{5.318311in}{1.292445in}}{\pgfqpoint{5.318311in}{1.284209in}}%
\pgfpathcurveto{\pgfqpoint{5.318311in}{1.275972in}}{\pgfqpoint{5.321583in}{1.268072in}}{\pgfqpoint{5.327407in}{1.262248in}}%
\pgfpathcurveto{\pgfqpoint{5.333231in}{1.256425in}}{\pgfqpoint{5.341131in}{1.253152in}}{\pgfqpoint{5.349367in}{1.253152in}}%
\pgfpathclose%
\pgfusepath{stroke,fill}%
\end{pgfscope}%
\begin{pgfscope}%
\pgfpathrectangle{\pgfqpoint{3.793912in}{0.557870in}}{\pgfqpoint{2.446088in}{1.684734in}}%
\pgfusepath{clip}%
\pgfsetbuttcap%
\pgfsetroundjoin%
\definecolor{currentfill}{rgb}{0.298039,0.447059,0.690196}%
\pgfsetfillcolor{currentfill}%
\pgfsetlinewidth{1.003750pt}%
\definecolor{currentstroke}{rgb}{0.298039,0.447059,0.690196}%
\pgfsetstrokecolor{currentstroke}%
\pgfsetdash{}{0pt}%
\pgfpathmoveto{\pgfqpoint{5.326442in}{1.268623in}}%
\pgfpathcurveto{\pgfqpoint{5.334679in}{1.268623in}}{\pgfqpoint{5.342579in}{1.271895in}}{\pgfqpoint{5.348403in}{1.277719in}}%
\pgfpathcurveto{\pgfqpoint{5.354227in}{1.283543in}}{\pgfqpoint{5.357499in}{1.291443in}}{\pgfqpoint{5.357499in}{1.299679in}}%
\pgfpathcurveto{\pgfqpoint{5.357499in}{1.307915in}}{\pgfqpoint{5.354227in}{1.315816in}}{\pgfqpoint{5.348403in}{1.321639in}}%
\pgfpathcurveto{\pgfqpoint{5.342579in}{1.327463in}}{\pgfqpoint{5.334679in}{1.330736in}}{\pgfqpoint{5.326442in}{1.330736in}}%
\pgfpathcurveto{\pgfqpoint{5.318206in}{1.330736in}}{\pgfqpoint{5.310306in}{1.327463in}}{\pgfqpoint{5.304482in}{1.321639in}}%
\pgfpathcurveto{\pgfqpoint{5.298658in}{1.315816in}}{\pgfqpoint{5.295386in}{1.307915in}}{\pgfqpoint{5.295386in}{1.299679in}}%
\pgfpathcurveto{\pgfqpoint{5.295386in}{1.291443in}}{\pgfqpoint{5.298658in}{1.283543in}}{\pgfqpoint{5.304482in}{1.277719in}}%
\pgfpathcurveto{\pgfqpoint{5.310306in}{1.271895in}}{\pgfqpoint{5.318206in}{1.268623in}}{\pgfqpoint{5.326442in}{1.268623in}}%
\pgfpathclose%
\pgfusepath{stroke,fill}%
\end{pgfscope}%
\begin{pgfscope}%
\pgfpathrectangle{\pgfqpoint{3.793912in}{0.557870in}}{\pgfqpoint{2.446088in}{1.684734in}}%
\pgfusepath{clip}%
\pgfsetbuttcap%
\pgfsetroundjoin%
\definecolor{currentfill}{rgb}{0.298039,0.447059,0.690196}%
\pgfsetfillcolor{currentfill}%
\pgfsetlinewidth{1.003750pt}%
\definecolor{currentstroke}{rgb}{0.298039,0.447059,0.690196}%
\pgfsetstrokecolor{currentstroke}%
\pgfsetdash{}{0pt}%
\pgfpathmoveto{\pgfqpoint{3.905098in}{2.119498in}}%
\pgfpathcurveto{\pgfqpoint{3.913334in}{2.119498in}}{\pgfqpoint{3.921234in}{2.122771in}}{\pgfqpoint{3.927058in}{2.128595in}}%
\pgfpathcurveto{\pgfqpoint{3.932882in}{2.134419in}}{\pgfqpoint{3.936155in}{2.142319in}}{\pgfqpoint{3.936155in}{2.150555in}}%
\pgfpathcurveto{\pgfqpoint{3.936155in}{2.158791in}}{\pgfqpoint{3.932882in}{2.166691in}}{\pgfqpoint{3.927058in}{2.172515in}}%
\pgfpathcurveto{\pgfqpoint{3.921234in}{2.178339in}}{\pgfqpoint{3.913334in}{2.181611in}}{\pgfqpoint{3.905098in}{2.181611in}}%
\pgfpathcurveto{\pgfqpoint{3.896862in}{2.181611in}}{\pgfqpoint{3.888962in}{2.178339in}}{\pgfqpoint{3.883138in}{2.172515in}}%
\pgfpathcurveto{\pgfqpoint{3.877314in}{2.166691in}}{\pgfqpoint{3.874042in}{2.158791in}}{\pgfqpoint{3.874042in}{2.150555in}}%
\pgfpathcurveto{\pgfqpoint{3.874042in}{2.142319in}}{\pgfqpoint{3.877314in}{2.134419in}}{\pgfqpoint{3.883138in}{2.128595in}}%
\pgfpathcurveto{\pgfqpoint{3.888962in}{2.122771in}}{\pgfqpoint{3.896862in}{2.119498in}}{\pgfqpoint{3.905098in}{2.119498in}}%
\pgfpathclose%
\pgfusepath{stroke,fill}%
\end{pgfscope}%
\begin{pgfscope}%
\pgfpathrectangle{\pgfqpoint{3.793912in}{0.557870in}}{\pgfqpoint{2.446088in}{1.684734in}}%
\pgfusepath{clip}%
\pgfsetbuttcap%
\pgfsetroundjoin%
\definecolor{currentfill}{rgb}{0.298039,0.447059,0.690196}%
\pgfsetfillcolor{currentfill}%
\pgfsetlinewidth{1.003750pt}%
\definecolor{currentstroke}{rgb}{0.298039,0.447059,0.690196}%
\pgfsetstrokecolor{currentstroke}%
\pgfsetdash{}{0pt}%
\pgfpathmoveto{\pgfqpoint{3.905098in}{2.119498in}}%
\pgfpathcurveto{\pgfqpoint{3.913334in}{2.119498in}}{\pgfqpoint{3.921234in}{2.122771in}}{\pgfqpoint{3.927058in}{2.128595in}}%
\pgfpathcurveto{\pgfqpoint{3.932882in}{2.134419in}}{\pgfqpoint{3.936155in}{2.142319in}}{\pgfqpoint{3.936155in}{2.150555in}}%
\pgfpathcurveto{\pgfqpoint{3.936155in}{2.158791in}}{\pgfqpoint{3.932882in}{2.166691in}}{\pgfqpoint{3.927058in}{2.172515in}}%
\pgfpathcurveto{\pgfqpoint{3.921234in}{2.178339in}}{\pgfqpoint{3.913334in}{2.181611in}}{\pgfqpoint{3.905098in}{2.181611in}}%
\pgfpathcurveto{\pgfqpoint{3.896862in}{2.181611in}}{\pgfqpoint{3.888962in}{2.178339in}}{\pgfqpoint{3.883138in}{2.172515in}}%
\pgfpathcurveto{\pgfqpoint{3.877314in}{2.166691in}}{\pgfqpoint{3.874042in}{2.158791in}}{\pgfqpoint{3.874042in}{2.150555in}}%
\pgfpathcurveto{\pgfqpoint{3.874042in}{2.142319in}}{\pgfqpoint{3.877314in}{2.134419in}}{\pgfqpoint{3.883138in}{2.128595in}}%
\pgfpathcurveto{\pgfqpoint{3.888962in}{2.122771in}}{\pgfqpoint{3.896862in}{2.119498in}}{\pgfqpoint{3.905098in}{2.119498in}}%
\pgfpathclose%
\pgfusepath{stroke,fill}%
\end{pgfscope}%
\begin{pgfscope}%
\pgfpathrectangle{\pgfqpoint{3.793912in}{0.557870in}}{\pgfqpoint{2.446088in}{1.684734in}}%
\pgfusepath{clip}%
\pgfsetbuttcap%
\pgfsetroundjoin%
\definecolor{currentfill}{rgb}{0.298039,0.447059,0.690196}%
\pgfsetfillcolor{currentfill}%
\pgfsetlinewidth{1.003750pt}%
\definecolor{currentstroke}{rgb}{0.298039,0.447059,0.690196}%
\pgfsetstrokecolor{currentstroke}%
\pgfsetdash{}{0pt}%
\pgfpathmoveto{\pgfqpoint{5.853715in}{1.237682in}}%
\pgfpathcurveto{\pgfqpoint{5.861952in}{1.237682in}}{\pgfqpoint{5.869852in}{1.240954in}}{\pgfqpoint{5.875676in}{1.246778in}}%
\pgfpathcurveto{\pgfqpoint{5.881499in}{1.252602in}}{\pgfqpoint{5.884772in}{1.260502in}}{\pgfqpoint{5.884772in}{1.268738in}}%
\pgfpathcurveto{\pgfqpoint{5.884772in}{1.276975in}}{\pgfqpoint{5.881499in}{1.284875in}}{\pgfqpoint{5.875676in}{1.290699in}}%
\pgfpathcurveto{\pgfqpoint{5.869852in}{1.296522in}}{\pgfqpoint{5.861952in}{1.299795in}}{\pgfqpoint{5.853715in}{1.299795in}}%
\pgfpathcurveto{\pgfqpoint{5.845479in}{1.299795in}}{\pgfqpoint{5.837579in}{1.296522in}}{\pgfqpoint{5.831755in}{1.290699in}}%
\pgfpathcurveto{\pgfqpoint{5.825931in}{1.284875in}}{\pgfqpoint{5.822659in}{1.276975in}}{\pgfqpoint{5.822659in}{1.268738in}}%
\pgfpathcurveto{\pgfqpoint{5.822659in}{1.260502in}}{\pgfqpoint{5.825931in}{1.252602in}}{\pgfqpoint{5.831755in}{1.246778in}}%
\pgfpathcurveto{\pgfqpoint{5.837579in}{1.240954in}}{\pgfqpoint{5.845479in}{1.237682in}}{\pgfqpoint{5.853715in}{1.237682in}}%
\pgfpathclose%
\pgfusepath{stroke,fill}%
\end{pgfscope}%
\begin{pgfscope}%
\pgfpathrectangle{\pgfqpoint{3.793912in}{0.557870in}}{\pgfqpoint{2.446088in}{1.684734in}}%
\pgfusepath{clip}%
\pgfsetbuttcap%
\pgfsetroundjoin%
\definecolor{currentfill}{rgb}{0.298039,0.447059,0.690196}%
\pgfsetfillcolor{currentfill}%
\pgfsetlinewidth{1.003750pt}%
\definecolor{currentstroke}{rgb}{0.298039,0.447059,0.690196}%
\pgfsetstrokecolor{currentstroke}%
\pgfsetdash{}{0pt}%
\pgfpathmoveto{\pgfqpoint{5.211818in}{1.856500in}}%
\pgfpathcurveto{\pgfqpoint{5.220054in}{1.856500in}}{\pgfqpoint{5.227954in}{1.859773in}}{\pgfqpoint{5.233778in}{1.865597in}}%
\pgfpathcurveto{\pgfqpoint{5.239602in}{1.871421in}}{\pgfqpoint{5.242874in}{1.879321in}}{\pgfqpoint{5.242874in}{1.887557in}}%
\pgfpathcurveto{\pgfqpoint{5.242874in}{1.895793in}}{\pgfqpoint{5.239602in}{1.903693in}}{\pgfqpoint{5.233778in}{1.909517in}}%
\pgfpathcurveto{\pgfqpoint{5.227954in}{1.915341in}}{\pgfqpoint{5.220054in}{1.918613in}}{\pgfqpoint{5.211818in}{1.918613in}}%
\pgfpathcurveto{\pgfqpoint{5.203582in}{1.918613in}}{\pgfqpoint{5.195682in}{1.915341in}}{\pgfqpoint{5.189858in}{1.909517in}}%
\pgfpathcurveto{\pgfqpoint{5.184034in}{1.903693in}}{\pgfqpoint{5.180761in}{1.895793in}}{\pgfqpoint{5.180761in}{1.887557in}}%
\pgfpathcurveto{\pgfqpoint{5.180761in}{1.879321in}}{\pgfqpoint{5.184034in}{1.871421in}}{\pgfqpoint{5.189858in}{1.865597in}}%
\pgfpathcurveto{\pgfqpoint{5.195682in}{1.859773in}}{\pgfqpoint{5.203582in}{1.856500in}}{\pgfqpoint{5.211818in}{1.856500in}}%
\pgfpathclose%
\pgfusepath{stroke,fill}%
\end{pgfscope}%
\begin{pgfscope}%
\pgfpathrectangle{\pgfqpoint{3.793912in}{0.557870in}}{\pgfqpoint{2.446088in}{1.684734in}}%
\pgfusepath{clip}%
\pgfsetbuttcap%
\pgfsetroundjoin%
\definecolor{currentfill}{rgb}{0.298039,0.447059,0.690196}%
\pgfsetfillcolor{currentfill}%
\pgfsetlinewidth{1.003750pt}%
\definecolor{currentstroke}{rgb}{0.298039,0.447059,0.690196}%
\pgfsetstrokecolor{currentstroke}%
\pgfsetdash{}{0pt}%
\pgfpathmoveto{\pgfqpoint{5.670316in}{1.438798in}}%
\pgfpathcurveto{\pgfqpoint{5.678552in}{1.438798in}}{\pgfqpoint{5.686452in}{1.442070in}}{\pgfqpoint{5.692276in}{1.447894in}}%
\pgfpathcurveto{\pgfqpoint{5.698100in}{1.453718in}}{\pgfqpoint{5.701373in}{1.461618in}}{\pgfqpoint{5.701373in}{1.469854in}}%
\pgfpathcurveto{\pgfqpoint{5.701373in}{1.478091in}}{\pgfqpoint{5.698100in}{1.485991in}}{\pgfqpoint{5.692276in}{1.491815in}}%
\pgfpathcurveto{\pgfqpoint{5.686452in}{1.497639in}}{\pgfqpoint{5.678552in}{1.500911in}}{\pgfqpoint{5.670316in}{1.500911in}}%
\pgfpathcurveto{\pgfqpoint{5.662080in}{1.500911in}}{\pgfqpoint{5.654180in}{1.497639in}}{\pgfqpoint{5.648356in}{1.491815in}}%
\pgfpathcurveto{\pgfqpoint{5.642532in}{1.485991in}}{\pgfqpoint{5.639260in}{1.478091in}}{\pgfqpoint{5.639260in}{1.469854in}}%
\pgfpathcurveto{\pgfqpoint{5.639260in}{1.461618in}}{\pgfqpoint{5.642532in}{1.453718in}}{\pgfqpoint{5.648356in}{1.447894in}}%
\pgfpathcurveto{\pgfqpoint{5.654180in}{1.442070in}}{\pgfqpoint{5.662080in}{1.438798in}}{\pgfqpoint{5.670316in}{1.438798in}}%
\pgfpathclose%
\pgfusepath{stroke,fill}%
\end{pgfscope}%
\begin{pgfscope}%
\pgfpathrectangle{\pgfqpoint{3.793912in}{0.557870in}}{\pgfqpoint{2.446088in}{1.684734in}}%
\pgfusepath{clip}%
\pgfsetbuttcap%
\pgfsetroundjoin%
\definecolor{currentfill}{rgb}{0.298039,0.447059,0.690196}%
\pgfsetfillcolor{currentfill}%
\pgfsetlinewidth{1.003750pt}%
\definecolor{currentstroke}{rgb}{0.298039,0.447059,0.690196}%
\pgfsetstrokecolor{currentstroke}%
\pgfsetdash{}{0pt}%
\pgfpathmoveto{\pgfqpoint{5.647391in}{1.500680in}}%
\pgfpathcurveto{\pgfqpoint{5.655627in}{1.500680in}}{\pgfqpoint{5.663527in}{1.503952in}}{\pgfqpoint{5.669351in}{1.509776in}}%
\pgfpathcurveto{\pgfqpoint{5.675175in}{1.515600in}}{\pgfqpoint{5.678448in}{1.523500in}}{\pgfqpoint{5.678448in}{1.531736in}}%
\pgfpathcurveto{\pgfqpoint{5.678448in}{1.539972in}}{\pgfqpoint{5.675175in}{1.547873in}}{\pgfqpoint{5.669351in}{1.553696in}}%
\pgfpathcurveto{\pgfqpoint{5.663527in}{1.559520in}}{\pgfqpoint{5.655627in}{1.562793in}}{\pgfqpoint{5.647391in}{1.562793in}}%
\pgfpathcurveto{\pgfqpoint{5.639155in}{1.562793in}}{\pgfqpoint{5.631255in}{1.559520in}}{\pgfqpoint{5.625431in}{1.553696in}}%
\pgfpathcurveto{\pgfqpoint{5.619607in}{1.547873in}}{\pgfqpoint{5.616335in}{1.539972in}}{\pgfqpoint{5.616335in}{1.531736in}}%
\pgfpathcurveto{\pgfqpoint{5.616335in}{1.523500in}}{\pgfqpoint{5.619607in}{1.515600in}}{\pgfqpoint{5.625431in}{1.509776in}}%
\pgfpathcurveto{\pgfqpoint{5.631255in}{1.503952in}}{\pgfqpoint{5.639155in}{1.500680in}}{\pgfqpoint{5.647391in}{1.500680in}}%
\pgfpathclose%
\pgfusepath{stroke,fill}%
\end{pgfscope}%
\begin{pgfscope}%
\pgfpathrectangle{\pgfqpoint{3.793912in}{0.557870in}}{\pgfqpoint{2.446088in}{1.684734in}}%
\pgfusepath{clip}%
\pgfsetbuttcap%
\pgfsetroundjoin%
\definecolor{currentfill}{rgb}{0.298039,0.447059,0.690196}%
\pgfsetfillcolor{currentfill}%
\pgfsetlinewidth{1.003750pt}%
\definecolor{currentstroke}{rgb}{0.298039,0.447059,0.690196}%
\pgfsetstrokecolor{currentstroke}%
\pgfsetdash{}{0pt}%
\pgfpathmoveto{\pgfqpoint{4.890869in}{1.995735in}}%
\pgfpathcurveto{\pgfqpoint{4.899105in}{1.995735in}}{\pgfqpoint{4.907005in}{1.999007in}}{\pgfqpoint{4.912829in}{2.004831in}}%
\pgfpathcurveto{\pgfqpoint{4.918653in}{2.010655in}}{\pgfqpoint{4.921926in}{2.018555in}}{\pgfqpoint{4.921926in}{2.026791in}}%
\pgfpathcurveto{\pgfqpoint{4.921926in}{2.035027in}}{\pgfqpoint{4.918653in}{2.042927in}}{\pgfqpoint{4.912829in}{2.048751in}}%
\pgfpathcurveto{\pgfqpoint{4.907005in}{2.054575in}}{\pgfqpoint{4.899105in}{2.057848in}}{\pgfqpoint{4.890869in}{2.057848in}}%
\pgfpathcurveto{\pgfqpoint{4.882633in}{2.057848in}}{\pgfqpoint{4.874733in}{2.054575in}}{\pgfqpoint{4.868909in}{2.048751in}}%
\pgfpathcurveto{\pgfqpoint{4.863085in}{2.042927in}}{\pgfqpoint{4.859813in}{2.035027in}}{\pgfqpoint{4.859813in}{2.026791in}}%
\pgfpathcurveto{\pgfqpoint{4.859813in}{2.018555in}}{\pgfqpoint{4.863085in}{2.010655in}}{\pgfqpoint{4.868909in}{2.004831in}}%
\pgfpathcurveto{\pgfqpoint{4.874733in}{1.999007in}}{\pgfqpoint{4.882633in}{1.995735in}}{\pgfqpoint{4.890869in}{1.995735in}}%
\pgfpathclose%
\pgfusepath{stroke,fill}%
\end{pgfscope}%
\begin{pgfscope}%
\pgfsetrectcap%
\pgfsetmiterjoin%
\pgfsetlinewidth{1.254687pt}%
\definecolor{currentstroke}{rgb}{1.000000,1.000000,1.000000}%
\pgfsetstrokecolor{currentstroke}%
\pgfsetdash{}{0pt}%
\pgfpathmoveto{\pgfqpoint{3.793912in}{0.557870in}}%
\pgfpathlineto{\pgfqpoint{3.793912in}{2.242604in}}%
\pgfusepath{stroke}%
\end{pgfscope}%
\begin{pgfscope}%
\pgfsetrectcap%
\pgfsetmiterjoin%
\pgfsetlinewidth{1.254687pt}%
\definecolor{currentstroke}{rgb}{1.000000,1.000000,1.000000}%
\pgfsetstrokecolor{currentstroke}%
\pgfsetdash{}{0pt}%
\pgfpathmoveto{\pgfqpoint{6.240000in}{0.557870in}}%
\pgfpathlineto{\pgfqpoint{6.240000in}{2.242604in}}%
\pgfusepath{stroke}%
\end{pgfscope}%
\begin{pgfscope}%
\pgfsetrectcap%
\pgfsetmiterjoin%
\pgfsetlinewidth{1.254687pt}%
\definecolor{currentstroke}{rgb}{1.000000,1.000000,1.000000}%
\pgfsetstrokecolor{currentstroke}%
\pgfsetdash{}{0pt}%
\pgfpathmoveto{\pgfqpoint{3.793912in}{0.557870in}}%
\pgfpathlineto{\pgfqpoint{6.240000in}{0.557870in}}%
\pgfusepath{stroke}%
\end{pgfscope}%
\begin{pgfscope}%
\pgfsetrectcap%
\pgfsetmiterjoin%
\pgfsetlinewidth{1.254687pt}%
\definecolor{currentstroke}{rgb}{1.000000,1.000000,1.000000}%
\pgfsetstrokecolor{currentstroke}%
\pgfsetdash{}{0pt}%
\pgfpathmoveto{\pgfqpoint{3.793912in}{2.242604in}}%
\pgfpathlineto{\pgfqpoint{6.240000in}{2.242604in}}%
\pgfusepath{stroke}%
\end{pgfscope}%
\begin{pgfscope}%
\definecolor{textcolor}{rgb}{0.150000,0.150000,0.150000}%
\pgfsetstrokecolor{textcolor}%
\pgfsetfillcolor{textcolor}%
\pgftext[x=5.016956in,y=2.325938in,,base]{\color{textcolor}\sffamily\fontsize{11.000000}{13.200000}\selectfont (b)}%
\end{pgfscope}%
\end{pgfpicture}%
\makeatother%
\endgroup%

    \caption{(a) Distribution plot of \acrshort{dor} of all \acrshort{tsc} models evaluated at two cluster centers when applied to classify heart failure.
             (b) Scatter plot of the same models sensitivity, and specificity.}
    \label{fig:tsc_hf_dor_sens_spec_dist}
\end{figure}

Figure \ref{fig:tsc_hf_dor_sens_spec_dist}a shows that the \acrshort{dor} is close to zero for many of the two-cluster-center models, However, the best performing models are able to acheive a \acrshort{dor} above ten, these models are listed in table \ref{tab:tsc_hf_dor_sens_spec_dist}. From the scatterplot in figure \ref{fig:tsc_hf_dor_sens_spec_dist}b one can see that the distribution of sensitivity, and specificity are quite widespread. Sensitivity and specificity scores range from 0 to 1. Common to the top 18 models in terms of \acrshort{dor} is that they all use data from a single view, and \acrshort{2ch} is the only view that is represented among the five models with highest \acrshort{dor}. What else is worth noting is that almost all the models using normalization or z-normalization as preprocessing score below the models that use scaling, or no preprocessing at all. These observations can be confirmed from the table\ref{tab:tsc_hf_raw_results} in the appendix.  From table \ref{tab:tsc_hf_dor_sens_spec_dist} one can see that the two best-performing models in terms of \acrshort{dor} received the exact same score in all metrics. \textit{gls/2CH/regular/centroid/2}, and \textit{gls/2CH/scaled/centroid/2} differ only in the way of preprocessing, the former does not preprocess the curves before clustering, and the latter uses scaling. However, for these two cases preprocessing did not matter as they have the exact same cluster assignments as well.\bigskip

\begin{table*}
    \centering
    \ra{1.3}
    \begin{tabular}{lrrrr}
        \toprule
        Dataset-model             &  Accuracy &  Sensitivity &  Specificity & \acrshort{dor} \\
        \midrule
        \acrshort{gls}/2CH/regular/centroid/2 &      0.76 &         0.87 &         0.64 & 11.72 \\
        \acrshort{gls}/2CH/scaled/centroid/2  &      0.76 &         0.87 &         0.64 & 11.72 \\
        \acrshort{gls}/2CH/regular/average/2  &      0.75 &         0.85 &         0.65 & 10.38 \\
        \acrshort{gls}/2CH/scaled/average/2   &      0.75 &         0.85 &         0.65 & 10.38 \\
        \acrshort{gls}-rls/2CH/scaled/ward/2  &      0.74 &         0.82 &         0.67 &  9.14 \\
        \bottomrule
    \end{tabular}
    \caption{The accuracy, \acrshort{dor}, sensitivity and specicity scores of the five best performing two-cluster-center\acrshort{tsc} models in terms of \acrshort{dor}, at detecting heart failure.
             The \textbf{Dataset-model} column indicates \textit{Dataset used}$/$\textit{View used}$/$\textit{Type of preprocessing used}$/$\textit{Linkage criteria of model}$/$\textit{Number of cluster centers}.}
    \label{tab:tsc_hf_dor_sens_spec_dist}
\end{table*}

\begin{table*}[htb]
    \centering
    \ra{1.3}
    \begin{tabular}{lr}
        \toprule
        Dataset-model             &  \acrshort{ari} \\
        \midrule
        \acrshort{gls}/2CH/regular/centroid/2 & 0.25 \\
        \acrshort{gls}/2CH/scaled/centroid/2  & 0.25 \\
        \acrshort{gls}/2CH/scaled/centroid/3  & 0.24 \\
        \acrshort{gls}/2CH/regular/centroid/3 & 0.24 \\
        \acrshort{gls}/2CH/scaled/average/2   & 0.24 \\
        \bottomrule
    \end{tabular}
    \caption{The five highest \acrshort{ari} scores attained when applying\acrshort{tsc} for detecting heart failure.
             The \textbf{Dataset-model} column indicates \textit{Dataset used}$/$\textit{View used}$/$\textit{Linkage criteria of model}$/$\textit{Number of cluster centers}.}
    \label{tab:tsc_hf_ari}
\end{table*}

\begin{figure}[htb]
    \centering
    %% Creator: Matplotlib, PGF backend
%%
%% To include the figure in your LaTeX document, write
%%   \input{<filename>.pgf}
%%
%% Make sure the required packages are loaded in your preamble
%%   \usepackage{pgf}
%%
%% Figures using additional raster images can only be included by \input if
%% they are in the same directory as the main LaTeX file. For loading figures
%% from other directories you can use the `import` package
%%   \usepackage{import}
%% and then include the figures with
%%   \import{<path to file>}{<filename>.pgf}
%%
%% Matplotlib used the following preamble
%%
\begingroup%
\makeatletter%
\begin{pgfpicture}%
\pgfpathrectangle{\pgfpointorigin}{\pgfqpoint{6.336707in}{2.540000in}}%
\pgfusepath{use as bounding box, clip}%
\begin{pgfscope}%
\pgfsetbuttcap%
\pgfsetmiterjoin%
\definecolor{currentfill}{rgb}{1.000000,1.000000,1.000000}%
\pgfsetfillcolor{currentfill}%
\pgfsetlinewidth{0.000000pt}%
\definecolor{currentstroke}{rgb}{1.000000,1.000000,1.000000}%
\pgfsetstrokecolor{currentstroke}%
\pgfsetdash{}{0pt}%
\pgfpathmoveto{\pgfqpoint{0.000000in}{0.000000in}}%
\pgfpathlineto{\pgfqpoint{6.336707in}{0.000000in}}%
\pgfpathlineto{\pgfqpoint{6.336707in}{2.540000in}}%
\pgfpathlineto{\pgfqpoint{0.000000in}{2.540000in}}%
\pgfpathclose%
\pgfusepath{fill}%
\end{pgfscope}%
\begin{pgfscope}%
\pgfsetbuttcap%
\pgfsetmiterjoin%
\definecolor{currentfill}{rgb}{0.917647,0.917647,0.949020}%
\pgfsetfillcolor{currentfill}%
\pgfsetlinewidth{0.000000pt}%
\definecolor{currentstroke}{rgb}{0.000000,0.000000,0.000000}%
\pgfsetstrokecolor{currentstroke}%
\pgfsetstrokeopacity{0.000000}%
\pgfsetdash{}{0pt}%
\pgfpathmoveto{\pgfqpoint{0.693056in}{0.574768in}}%
\pgfpathlineto{\pgfqpoint{3.073153in}{0.574768in}}%
\pgfpathlineto{\pgfqpoint{3.073153in}{2.242604in}}%
\pgfpathlineto{\pgfqpoint{0.693056in}{2.242604in}}%
\pgfpathclose%
\pgfusepath{fill}%
\end{pgfscope}%
\begin{pgfscope}%
\pgfpathrectangle{\pgfqpoint{0.693056in}{0.574768in}}{\pgfqpoint{2.380097in}{1.667836in}}%
\pgfusepath{clip}%
\pgfsetroundcap%
\pgfsetroundjoin%
\pgfsetlinewidth{1.003750pt}%
\definecolor{currentstroke}{rgb}{1.000000,1.000000,1.000000}%
\pgfsetstrokecolor{currentstroke}%
\pgfsetdash{}{0pt}%
\pgfpathmoveto{\pgfqpoint{0.801242in}{0.574768in}}%
\pgfpathlineto{\pgfqpoint{0.801242in}{2.242604in}}%
\pgfusepath{stroke}%
\end{pgfscope}%
\begin{pgfscope}%
\definecolor{textcolor}{rgb}{0.150000,0.150000,0.150000}%
\pgfsetstrokecolor{textcolor}%
\pgfsetfillcolor{textcolor}%
\pgftext[x=0.801242in,y=0.442824in,,top]{\color{textcolor}\sffamily\fontsize{11.000000}{13.200000}\selectfont \(\displaystyle 0.0\)}%
\end{pgfscope}%
\begin{pgfscope}%
\pgfpathrectangle{\pgfqpoint{0.693056in}{0.574768in}}{\pgfqpoint{2.380097in}{1.667836in}}%
\pgfusepath{clip}%
\pgfsetroundcap%
\pgfsetroundjoin%
\pgfsetlinewidth{1.003750pt}%
\definecolor{currentstroke}{rgb}{1.000000,1.000000,1.000000}%
\pgfsetstrokecolor{currentstroke}%
\pgfsetdash{}{0pt}%
\pgfpathmoveto{\pgfqpoint{1.454644in}{0.574768in}}%
\pgfpathlineto{\pgfqpoint{1.454644in}{2.242604in}}%
\pgfusepath{stroke}%
\end{pgfscope}%
\begin{pgfscope}%
\definecolor{textcolor}{rgb}{0.150000,0.150000,0.150000}%
\pgfsetstrokecolor{textcolor}%
\pgfsetfillcolor{textcolor}%
\pgftext[x=1.454644in,y=0.442824in,,top]{\color{textcolor}\sffamily\fontsize{11.000000}{13.200000}\selectfont \(\displaystyle 0.5\)}%
\end{pgfscope}%
\begin{pgfscope}%
\pgfpathrectangle{\pgfqpoint{0.693056in}{0.574768in}}{\pgfqpoint{2.380097in}{1.667836in}}%
\pgfusepath{clip}%
\pgfsetroundcap%
\pgfsetroundjoin%
\pgfsetlinewidth{1.003750pt}%
\definecolor{currentstroke}{rgb}{1.000000,1.000000,1.000000}%
\pgfsetstrokecolor{currentstroke}%
\pgfsetdash{}{0pt}%
\pgfpathmoveto{\pgfqpoint{2.108046in}{0.574768in}}%
\pgfpathlineto{\pgfqpoint{2.108046in}{2.242604in}}%
\pgfusepath{stroke}%
\end{pgfscope}%
\begin{pgfscope}%
\definecolor{textcolor}{rgb}{0.150000,0.150000,0.150000}%
\pgfsetstrokecolor{textcolor}%
\pgfsetfillcolor{textcolor}%
\pgftext[x=2.108046in,y=0.442824in,,top]{\color{textcolor}\sffamily\fontsize{11.000000}{13.200000}\selectfont \(\displaystyle 1.0\)}%
\end{pgfscope}%
\begin{pgfscope}%
\pgfpathrectangle{\pgfqpoint{0.693056in}{0.574768in}}{\pgfqpoint{2.380097in}{1.667836in}}%
\pgfusepath{clip}%
\pgfsetroundcap%
\pgfsetroundjoin%
\pgfsetlinewidth{1.003750pt}%
\definecolor{currentstroke}{rgb}{1.000000,1.000000,1.000000}%
\pgfsetstrokecolor{currentstroke}%
\pgfsetdash{}{0pt}%
\pgfpathmoveto{\pgfqpoint{2.761448in}{0.574768in}}%
\pgfpathlineto{\pgfqpoint{2.761448in}{2.242604in}}%
\pgfusepath{stroke}%
\end{pgfscope}%
\begin{pgfscope}%
\definecolor{textcolor}{rgb}{0.150000,0.150000,0.150000}%
\pgfsetstrokecolor{textcolor}%
\pgfsetfillcolor{textcolor}%
\pgftext[x=2.761448in,y=0.442824in,,top]{\color{textcolor}\sffamily\fontsize{11.000000}{13.200000}\selectfont \(\displaystyle 1.5\)}%
\end{pgfscope}%
\begin{pgfscope}%
\definecolor{textcolor}{rgb}{0.150000,0.150000,0.150000}%
\pgfsetstrokecolor{textcolor}%
\pgfsetfillcolor{textcolor}%
\pgftext[x=1.883105in,y=0.252083in,,top]{\color{textcolor}\sffamily\fontsize{11.000000}{13.200000}\selectfont Time [s]}%
\end{pgfscope}%
\begin{pgfscope}%
\pgfpathrectangle{\pgfqpoint{0.693056in}{0.574768in}}{\pgfqpoint{2.380097in}{1.667836in}}%
\pgfusepath{clip}%
\pgfsetroundcap%
\pgfsetroundjoin%
\pgfsetlinewidth{1.003750pt}%
\definecolor{currentstroke}{rgb}{1.000000,1.000000,1.000000}%
\pgfsetstrokecolor{currentstroke}%
\pgfsetdash{}{0pt}%
\pgfpathmoveto{\pgfqpoint{0.693056in}{0.677249in}}%
\pgfpathlineto{\pgfqpoint{3.073153in}{0.677249in}}%
\pgfusepath{stroke}%
\end{pgfscope}%
\begin{pgfscope}%
\definecolor{textcolor}{rgb}{0.150000,0.150000,0.150000}%
\pgfsetstrokecolor{textcolor}%
\pgfsetfillcolor{textcolor}%
\pgftext[x=0.290741in,y=0.624442in,left,base]{\color{textcolor}\sffamily\fontsize{11.000000}{13.200000}\selectfont \(\displaystyle -15\)}%
\end{pgfscope}%
\begin{pgfscope}%
\pgfpathrectangle{\pgfqpoint{0.693056in}{0.574768in}}{\pgfqpoint{2.380097in}{1.667836in}}%
\pgfusepath{clip}%
\pgfsetroundcap%
\pgfsetroundjoin%
\pgfsetlinewidth{1.003750pt}%
\definecolor{currentstroke}{rgb}{1.000000,1.000000,1.000000}%
\pgfsetstrokecolor{currentstroke}%
\pgfsetdash{}{0pt}%
\pgfpathmoveto{\pgfqpoint{0.693056in}{1.106474in}}%
\pgfpathlineto{\pgfqpoint{3.073153in}{1.106474in}}%
\pgfusepath{stroke}%
\end{pgfscope}%
\begin{pgfscope}%
\definecolor{textcolor}{rgb}{0.150000,0.150000,0.150000}%
\pgfsetstrokecolor{textcolor}%
\pgfsetfillcolor{textcolor}%
\pgftext[x=0.290741in,y=1.053667in,left,base]{\color{textcolor}\sffamily\fontsize{11.000000}{13.200000}\selectfont \(\displaystyle -10\)}%
\end{pgfscope}%
\begin{pgfscope}%
\pgfpathrectangle{\pgfqpoint{0.693056in}{0.574768in}}{\pgfqpoint{2.380097in}{1.667836in}}%
\pgfusepath{clip}%
\pgfsetroundcap%
\pgfsetroundjoin%
\pgfsetlinewidth{1.003750pt}%
\definecolor{currentstroke}{rgb}{1.000000,1.000000,1.000000}%
\pgfsetstrokecolor{currentstroke}%
\pgfsetdash{}{0pt}%
\pgfpathmoveto{\pgfqpoint{0.693056in}{1.535699in}}%
\pgfpathlineto{\pgfqpoint{3.073153in}{1.535699in}}%
\pgfusepath{stroke}%
\end{pgfscope}%
\begin{pgfscope}%
\definecolor{textcolor}{rgb}{0.150000,0.150000,0.150000}%
\pgfsetstrokecolor{textcolor}%
\pgfsetfillcolor{textcolor}%
\pgftext[x=0.366782in,y=1.482892in,left,base]{\color{textcolor}\sffamily\fontsize{11.000000}{13.200000}\selectfont \(\displaystyle -5\)}%
\end{pgfscope}%
\begin{pgfscope}%
\pgfpathrectangle{\pgfqpoint{0.693056in}{0.574768in}}{\pgfqpoint{2.380097in}{1.667836in}}%
\pgfusepath{clip}%
\pgfsetroundcap%
\pgfsetroundjoin%
\pgfsetlinewidth{1.003750pt}%
\definecolor{currentstroke}{rgb}{1.000000,1.000000,1.000000}%
\pgfsetstrokecolor{currentstroke}%
\pgfsetdash{}{0pt}%
\pgfpathmoveto{\pgfqpoint{0.693056in}{1.964924in}}%
\pgfpathlineto{\pgfqpoint{3.073153in}{1.964924in}}%
\pgfusepath{stroke}%
\end{pgfscope}%
\begin{pgfscope}%
\definecolor{textcolor}{rgb}{0.150000,0.150000,0.150000}%
\pgfsetstrokecolor{textcolor}%
\pgfsetfillcolor{textcolor}%
\pgftext[x=0.485070in,y=1.912117in,left,base]{\color{textcolor}\sffamily\fontsize{11.000000}{13.200000}\selectfont \(\displaystyle 0\)}%
\end{pgfscope}%
\begin{pgfscope}%
\definecolor{textcolor}{rgb}{0.150000,0.150000,0.150000}%
\pgfsetstrokecolor{textcolor}%
\pgfsetfillcolor{textcolor}%
\pgftext[x=0.235185in,y=1.408686in,,bottom,rotate=90.000000]{\color{textcolor}\sffamily\fontsize{11.000000}{13.200000}\selectfont GLS}%
\end{pgfscope}%
\begin{pgfscope}%
\pgfpathrectangle{\pgfqpoint{0.693056in}{0.574768in}}{\pgfqpoint{2.380097in}{1.667836in}}%
\pgfusepath{clip}%
\pgfsetroundcap%
\pgfsetroundjoin%
\pgfsetlinewidth{1.505625pt}%
\definecolor{currentstroke}{rgb}{0.298039,0.447059,0.690196}%
\pgfsetstrokecolor{currentstroke}%
\pgfsetdash{}{0pt}%
\pgfpathmoveto{\pgfqpoint{0.801242in}{1.575632in}}%
\pgfpathlineto{\pgfqpoint{0.822665in}{1.579029in}}%
\pgfpathlineto{\pgfqpoint{0.844088in}{1.582737in}}%
\pgfpathlineto{\pgfqpoint{0.865511in}{1.586608in}}%
\pgfpathlineto{\pgfqpoint{0.886934in}{1.590076in}}%
\pgfpathlineto{\pgfqpoint{0.908357in}{1.592827in}}%
\pgfpathlineto{\pgfqpoint{0.929780in}{1.595583in}}%
\pgfpathlineto{\pgfqpoint{0.951203in}{1.601079in}}%
\pgfpathlineto{\pgfqpoint{0.972626in}{1.614740in}}%
\pgfpathlineto{\pgfqpoint{0.994049in}{1.643100in}}%
\pgfpathlineto{\pgfqpoint{1.015472in}{1.689370in}}%
\pgfpathlineto{\pgfqpoint{1.036895in}{1.749710in}}%
\pgfpathlineto{\pgfqpoint{1.058318in}{1.814052in}}%
\pgfpathlineto{\pgfqpoint{1.079741in}{1.872013in}}%
\pgfpathlineto{\pgfqpoint{1.101164in}{1.917587in}}%
\pgfpathlineto{\pgfqpoint{1.122587in}{1.948388in}}%
\pgfpathlineto{\pgfqpoint{1.144010in}{1.964924in}}%
\pgfpathlineto{\pgfqpoint{1.165433in}{1.970235in}}%
\pgfpathlineto{\pgfqpoint{1.186856in}{1.968312in}}%
\pgfpathlineto{\pgfqpoint{1.208280in}{1.960487in}}%
\pgfpathlineto{\pgfqpoint{1.229703in}{1.943079in}}%
\pgfpathlineto{\pgfqpoint{1.251126in}{1.910179in}}%
\pgfpathlineto{\pgfqpoint{1.272549in}{1.860635in}}%
\pgfpathlineto{\pgfqpoint{1.293972in}{1.801043in}}%
\pgfpathlineto{\pgfqpoint{1.315395in}{1.740516in}}%
\pgfpathlineto{\pgfqpoint{1.336818in}{1.684166in}}%
\pgfpathlineto{\pgfqpoint{1.358241in}{1.631101in}}%
\pgfpathlineto{\pgfqpoint{1.379664in}{1.579547in}}%
\pgfpathlineto{\pgfqpoint{1.401087in}{1.529570in}}%
\pgfpathlineto{\pgfqpoint{1.422510in}{1.480979in}}%
\pgfpathlineto{\pgfqpoint{1.443933in}{1.432548in}}%
\pgfpathlineto{\pgfqpoint{1.465356in}{1.384065in}}%
\pgfpathlineto{\pgfqpoint{1.486779in}{1.336156in}}%
\pgfpathlineto{\pgfqpoint{1.508202in}{1.290294in}}%
\pgfpathlineto{\pgfqpoint{1.529625in}{1.248057in}}%
\pgfpathlineto{\pgfqpoint{1.551048in}{1.210262in}}%
\pgfpathlineto{\pgfqpoint{1.572471in}{1.176787in}}%
\pgfpathlineto{\pgfqpoint{1.593894in}{1.147091in}}%
\pgfpathlineto{\pgfqpoint{1.615317in}{1.120775in}}%
\pgfpathlineto{\pgfqpoint{1.636740in}{1.097727in}}%
\pgfpathlineto{\pgfqpoint{1.658163in}{1.078717in}}%
\pgfpathlineto{\pgfqpoint{1.679586in}{1.066330in}}%
\pgfpathlineto{\pgfqpoint{1.701009in}{1.063619in}}%
\pgfpathlineto{\pgfqpoint{1.722432in}{1.071138in}}%
\pgfpathlineto{\pgfqpoint{1.743855in}{1.085032in}}%
\pgfpathlineto{\pgfqpoint{1.765278in}{1.097365in}}%
\pgfpathlineto{\pgfqpoint{1.786701in}{1.099285in}}%
\pgfpathlineto{\pgfqpoint{1.808124in}{1.086452in}}%
\pgfpathlineto{\pgfqpoint{1.829547in}{1.061460in}}%
\pgfpathlineto{\pgfqpoint{1.850970in}{1.031653in}}%
\pgfpathlineto{\pgfqpoint{1.872393in}{1.005769in}}%
\pgfpathlineto{\pgfqpoint{1.893816in}{0.992812in}}%
\pgfpathlineto{\pgfqpoint{1.915239in}{1.000673in}}%
\pgfpathlineto{\pgfqpoint{1.936662in}{1.031437in}}%
\pgfpathlineto{\pgfqpoint{1.958085in}{1.078781in}}%
\pgfpathlineto{\pgfqpoint{1.979508in}{1.130851in}}%
\pgfpathlineto{\pgfqpoint{2.000931in}{1.177416in}}%
\pgfpathlineto{\pgfqpoint{2.022354in}{1.214820in}}%
\pgfpathlineto{\pgfqpoint{2.043777in}{1.245437in}}%
\pgfpathlineto{\pgfqpoint{2.065200in}{1.273182in}}%
\pgfpathlineto{\pgfqpoint{2.086623in}{1.300003in}}%
\pgfpathlineto{\pgfqpoint{2.108046in}{1.325497in}}%
\pgfpathlineto{\pgfqpoint{2.129469in}{1.348215in}}%
\pgfpathlineto{\pgfqpoint{2.150892in}{1.367437in}}%
\pgfpathlineto{\pgfqpoint{2.172315in}{1.383860in}}%
\pgfpathlineto{\pgfqpoint{2.193738in}{1.398694in}}%
\pgfpathlineto{\pgfqpoint{2.215161in}{1.413311in}}%
\pgfpathlineto{\pgfqpoint{2.236584in}{1.429172in}}%
\pgfpathlineto{\pgfqpoint{2.258007in}{1.447164in}}%
\pgfpathlineto{\pgfqpoint{2.279430in}{1.467148in}}%
\pgfpathlineto{\pgfqpoint{2.300853in}{1.487700in}}%
\pgfpathlineto{\pgfqpoint{2.322276in}{1.506573in}}%
\pgfpathlineto{\pgfqpoint{2.343699in}{1.522142in}}%
\pgfpathlineto{\pgfqpoint{2.365122in}{1.534170in}}%
\pgfpathlineto{\pgfqpoint{2.386545in}{1.543680in}}%
\pgfpathlineto{\pgfqpoint{2.407968in}{1.551691in}}%
\pgfpathlineto{\pgfqpoint{2.429391in}{1.558334in}}%
\pgfpathlineto{\pgfqpoint{2.450814in}{1.563495in}}%
\pgfpathlineto{\pgfqpoint{2.472237in}{1.567205in}}%
\pgfpathlineto{\pgfqpoint{2.493660in}{1.569953in}}%
\pgfpathlineto{\pgfqpoint{2.515083in}{1.572566in}}%
\pgfpathlineto{\pgfqpoint{2.536507in}{1.575673in}}%
\pgfpathlineto{\pgfqpoint{2.557930in}{1.579284in}}%
\pgfpathlineto{\pgfqpoint{2.579353in}{1.582749in}}%
\pgfpathlineto{\pgfqpoint{2.600776in}{1.585092in}}%
\pgfpathlineto{\pgfqpoint{2.622199in}{1.585858in}}%
\pgfpathlineto{\pgfqpoint{2.643622in}{1.586780in}}%
\pgfpathlineto{\pgfqpoint{2.665045in}{1.593153in}}%
\pgfpathlineto{\pgfqpoint{2.686468in}{1.612863in}}%
\pgfpathlineto{\pgfqpoint{2.707891in}{1.651360in}}%
\pgfpathlineto{\pgfqpoint{2.729314in}{1.706307in}}%
\pgfpathlineto{\pgfqpoint{2.750737in}{1.768745in}}%
\pgfpathlineto{\pgfqpoint{2.772160in}{1.828986in}}%
\pgfpathlineto{\pgfqpoint{2.793583in}{1.880852in}}%
\pgfpathlineto{\pgfqpoint{2.815006in}{1.921088in}}%
\pgfpathlineto{\pgfqpoint{2.836429in}{1.948503in}}%
\pgfpathlineto{\pgfqpoint{2.857852in}{1.964924in}}%
\pgfpathlineto{\pgfqpoint{2.879275in}{1.973475in}}%
\pgfpathlineto{\pgfqpoint{2.900698in}{1.976401in}}%
\pgfpathlineto{\pgfqpoint{2.922121in}{1.973436in}}%
\pgfpathlineto{\pgfqpoint{2.943544in}{1.960879in}}%
\pgfpathlineto{\pgfqpoint{2.964967in}{1.933565in}}%
\pgfusepath{stroke}%
\end{pgfscope}%
\begin{pgfscope}%
\pgfpathrectangle{\pgfqpoint{0.693056in}{0.574768in}}{\pgfqpoint{2.380097in}{1.667836in}}%
\pgfusepath{clip}%
\pgfsetroundcap%
\pgfsetroundjoin%
\pgfsetlinewidth{1.505625pt}%
\definecolor{currentstroke}{rgb}{0.866667,0.517647,0.321569}%
\pgfsetstrokecolor{currentstroke}%
\pgfsetdash{}{0pt}%
\pgfpathmoveto{\pgfqpoint{0.801242in}{1.524355in}}%
\pgfpathlineto{\pgfqpoint{0.819911in}{1.520952in}}%
\pgfpathlineto{\pgfqpoint{0.838579in}{1.525155in}}%
\pgfpathlineto{\pgfqpoint{0.857248in}{1.546025in}}%
\pgfpathlineto{\pgfqpoint{0.875917in}{1.591506in}}%
\pgfpathlineto{\pgfqpoint{0.894585in}{1.662917in}}%
\pgfpathlineto{\pgfqpoint{0.913254in}{1.750410in}}%
\pgfpathlineto{\pgfqpoint{0.931923in}{1.837630in}}%
\pgfpathlineto{\pgfqpoint{0.950591in}{1.910596in}}%
\pgfpathlineto{\pgfqpoint{0.969260in}{1.960293in}}%
\pgfpathlineto{\pgfqpoint{0.987928in}{1.983570in}}%
\pgfpathlineto{\pgfqpoint{1.006597in}{1.983037in}}%
\pgfpathlineto{\pgfqpoint{1.025266in}{1.964924in}}%
\pgfpathlineto{\pgfqpoint{1.043934in}{1.938418in}}%
\pgfpathlineto{\pgfqpoint{1.062603in}{1.914914in}}%
\pgfpathlineto{\pgfqpoint{1.081272in}{1.902650in}}%
\pgfpathlineto{\pgfqpoint{1.099940in}{1.900808in}}%
\pgfpathlineto{\pgfqpoint{1.118609in}{1.901870in}}%
\pgfpathlineto{\pgfqpoint{1.137278in}{1.899427in}}%
\pgfpathlineto{\pgfqpoint{1.155946in}{1.890793in}}%
\pgfpathlineto{\pgfqpoint{1.174615in}{1.872649in}}%
\pgfpathlineto{\pgfqpoint{1.193283in}{1.839522in}}%
\pgfpathlineto{\pgfqpoint{1.211952in}{1.787986in}}%
\pgfpathlineto{\pgfqpoint{1.230621in}{1.720747in}}%
\pgfpathlineto{\pgfqpoint{1.249289in}{1.643690in}}%
\pgfpathlineto{\pgfqpoint{1.267958in}{1.563026in}}%
\pgfpathlineto{\pgfqpoint{1.286627in}{1.483047in}}%
\pgfpathlineto{\pgfqpoint{1.305295in}{1.404789in}}%
\pgfpathlineto{\pgfqpoint{1.323964in}{1.326959in}}%
\pgfpathlineto{\pgfqpoint{1.342632in}{1.248431in}}%
\pgfpathlineto{\pgfqpoint{1.361301in}{1.170127in}}%
\pgfpathlineto{\pgfqpoint{1.379970in}{1.095560in}}%
\pgfpathlineto{\pgfqpoint{1.398638in}{1.026508in}}%
\pgfpathlineto{\pgfqpoint{1.417307in}{0.962053in}}%
\pgfpathlineto{\pgfqpoint{1.435976in}{0.901460in}}%
\pgfpathlineto{\pgfqpoint{1.454644in}{0.845733in}}%
\pgfpathlineto{\pgfqpoint{1.473313in}{0.798609in}}%
\pgfpathlineto{\pgfqpoint{1.491981in}{0.763104in}}%
\pgfpathlineto{\pgfqpoint{1.510650in}{0.736636in}}%
\pgfpathlineto{\pgfqpoint{1.529319in}{0.712036in}}%
\pgfpathlineto{\pgfqpoint{1.547987in}{0.685853in}}%
\pgfpathlineto{\pgfqpoint{1.566656in}{0.662824in}}%
\pgfpathlineto{\pgfqpoint{1.585325in}{0.650579in}}%
\pgfpathlineto{\pgfqpoint{1.603993in}{0.652476in}}%
\pgfpathlineto{\pgfqpoint{1.622662in}{0.666805in}}%
\pgfpathlineto{\pgfqpoint{1.641330in}{0.689924in}}%
\pgfpathlineto{\pgfqpoint{1.659999in}{0.720084in}}%
\pgfpathlineto{\pgfqpoint{1.678668in}{0.758520in}}%
\pgfpathlineto{\pgfqpoint{1.697336in}{0.807890in}}%
\pgfpathlineto{\pgfqpoint{1.716005in}{0.869900in}}%
\pgfpathlineto{\pgfqpoint{1.734674in}{0.942850in}}%
\pgfpathlineto{\pgfqpoint{1.753342in}{1.019662in}}%
\pgfpathlineto{\pgfqpoint{1.772011in}{1.092746in}}%
\pgfpathlineto{\pgfqpoint{1.790680in}{1.162154in}}%
\pgfpathlineto{\pgfqpoint{1.809348in}{1.232041in}}%
\pgfpathlineto{\pgfqpoint{1.828017in}{1.303101in}}%
\pgfpathlineto{\pgfqpoint{1.846685in}{1.369873in}}%
\pgfpathlineto{\pgfqpoint{1.865354in}{1.424763in}}%
\pgfpathlineto{\pgfqpoint{1.884023in}{1.464746in}}%
\pgfpathlineto{\pgfqpoint{1.902691in}{1.492805in}}%
\pgfpathlineto{\pgfqpoint{1.921360in}{1.513682in}}%
\pgfpathlineto{\pgfqpoint{1.940029in}{1.530249in}}%
\pgfpathlineto{\pgfqpoint{1.958697in}{1.542668in}}%
\pgfpathlineto{\pgfqpoint{1.977366in}{1.549792in}}%
\pgfpathlineto{\pgfqpoint{1.996034in}{1.550906in}}%
\pgfpathlineto{\pgfqpoint{2.014703in}{1.547659in}}%
\pgfpathlineto{\pgfqpoint{2.033372in}{1.545708in}}%
\pgfpathlineto{\pgfqpoint{2.052040in}{1.555237in}}%
\pgfpathlineto{\pgfqpoint{2.070709in}{1.588020in}}%
\pgfpathlineto{\pgfqpoint{2.089378in}{1.650250in}}%
\pgfpathlineto{\pgfqpoint{2.108046in}{1.735819in}}%
\pgfpathlineto{\pgfqpoint{2.126715in}{1.827866in}}%
\pgfpathlineto{\pgfqpoint{2.145383in}{1.906452in}}%
\pgfpathlineto{\pgfqpoint{2.164052in}{1.957664in}}%
\pgfpathlineto{\pgfqpoint{2.182721in}{1.980498in}}%
\pgfpathlineto{\pgfqpoint{2.201389in}{1.981063in}}%
\pgfpathlineto{\pgfqpoint{2.220058in}{1.964924in}}%
\pgfpathlineto{\pgfqpoint{2.238727in}{1.938816in}}%
\pgfpathlineto{\pgfqpoint{2.257395in}{1.913446in}}%
\pgfpathlineto{\pgfqpoint{2.276064in}{1.898581in}}%
\pgfpathlineto{\pgfqpoint{2.294732in}{1.894322in}}%
\pgfpathlineto{\pgfqpoint{2.313401in}{1.891731in}}%
\pgfpathlineto{\pgfqpoint{2.332070in}{1.883194in}}%
\pgfpathlineto{\pgfqpoint{2.350738in}{1.867435in}}%
\pgfpathlineto{\pgfqpoint{2.369407in}{1.846780in}}%
\pgfpathlineto{\pgfqpoint{2.388076in}{1.821354in}}%
\pgfpathlineto{\pgfqpoint{2.406744in}{1.784658in}}%
\pgfpathlineto{\pgfqpoint{2.425413in}{1.729588in}}%
\pgfpathlineto{\pgfqpoint{2.444081in}{1.659730in}}%
\pgfusepath{stroke}%
\end{pgfscope}%
\begin{pgfscope}%
\pgfpathrectangle{\pgfqpoint{0.693056in}{0.574768in}}{\pgfqpoint{2.380097in}{1.667836in}}%
\pgfusepath{clip}%
\pgfsetroundcap%
\pgfsetroundjoin%
\pgfsetlinewidth{1.505625pt}%
\definecolor{currentstroke}{rgb}{0.333333,0.658824,0.407843}%
\pgfsetstrokecolor{currentstroke}%
\pgfsetdash{}{0pt}%
\pgfpathmoveto{\pgfqpoint{0.801242in}{1.453338in}}%
\pgfpathlineto{\pgfqpoint{0.822665in}{1.458406in}}%
\pgfpathlineto{\pgfqpoint{0.844088in}{1.461636in}}%
\pgfpathlineto{\pgfqpoint{0.865511in}{1.463025in}}%
\pgfpathlineto{\pgfqpoint{0.886934in}{1.466661in}}%
\pgfpathlineto{\pgfqpoint{0.908357in}{1.482688in}}%
\pgfpathlineto{\pgfqpoint{0.929780in}{1.525436in}}%
\pgfpathlineto{\pgfqpoint{0.951203in}{1.603877in}}%
\pgfpathlineto{\pgfqpoint{0.972626in}{1.711009in}}%
\pgfpathlineto{\pgfqpoint{0.994049in}{1.825132in}}%
\pgfpathlineto{\pgfqpoint{1.015472in}{1.919170in}}%
\pgfpathlineto{\pgfqpoint{1.036895in}{1.969473in}}%
\pgfpathlineto{\pgfqpoint{1.058318in}{1.964924in}}%
\pgfpathlineto{\pgfqpoint{1.079741in}{1.913373in}}%
\pgfpathlineto{\pgfqpoint{1.101164in}{1.840631in}}%
\pgfpathlineto{\pgfqpoint{1.122587in}{1.777808in}}%
\pgfpathlineto{\pgfqpoint{1.144010in}{1.743844in}}%
\pgfpathlineto{\pgfqpoint{1.165433in}{1.735438in}}%
\pgfpathlineto{\pgfqpoint{1.186856in}{1.730970in}}%
\pgfpathlineto{\pgfqpoint{1.208280in}{1.704622in}}%
\pgfpathlineto{\pgfqpoint{1.229703in}{1.643092in}}%
\pgfpathlineto{\pgfqpoint{1.251126in}{1.551965in}}%
\pgfpathlineto{\pgfqpoint{1.272549in}{1.446305in}}%
\pgfpathlineto{\pgfqpoint{1.293972in}{1.339671in}}%
\pgfpathlineto{\pgfqpoint{1.315395in}{1.238982in}}%
\pgfpathlineto{\pgfqpoint{1.336818in}{1.147615in}}%
\pgfpathlineto{\pgfqpoint{1.358241in}{1.066475in}}%
\pgfpathlineto{\pgfqpoint{1.379664in}{0.992217in}}%
\pgfpathlineto{\pgfqpoint{1.401087in}{0.920998in}}%
\pgfpathlineto{\pgfqpoint{1.422510in}{0.852794in}}%
\pgfpathlineto{\pgfqpoint{1.443933in}{0.791237in}}%
\pgfpathlineto{\pgfqpoint{1.465356in}{0.743975in}}%
\pgfpathlineto{\pgfqpoint{1.486779in}{0.719975in}}%
\pgfpathlineto{\pgfqpoint{1.508202in}{0.722812in}}%
\pgfpathlineto{\pgfqpoint{1.529625in}{0.746744in}}%
\pgfpathlineto{\pgfqpoint{1.551048in}{0.781609in}}%
\pgfpathlineto{\pgfqpoint{1.572471in}{0.820623in}}%
\pgfpathlineto{\pgfqpoint{1.593894in}{0.862559in}}%
\pgfpathlineto{\pgfqpoint{1.615317in}{0.908076in}}%
\pgfpathlineto{\pgfqpoint{1.636740in}{0.954971in}}%
\pgfpathlineto{\pgfqpoint{1.658163in}{0.998131in}}%
\pgfpathlineto{\pgfqpoint{1.679586in}{1.034331in}}%
\pgfpathlineto{\pgfqpoint{1.701009in}{1.066644in}}%
\pgfpathlineto{\pgfqpoint{1.722432in}{1.101205in}}%
\pgfpathlineto{\pgfqpoint{1.743855in}{1.140730in}}%
\pgfpathlineto{\pgfqpoint{1.765278in}{1.181876in}}%
\pgfpathlineto{\pgfqpoint{1.786701in}{1.220066in}}%
\pgfpathlineto{\pgfqpoint{1.808124in}{1.254835in}}%
\pgfpathlineto{\pgfqpoint{1.829547in}{1.287922in}}%
\pgfpathlineto{\pgfqpoint{1.850970in}{1.320239in}}%
\pgfpathlineto{\pgfqpoint{1.872393in}{1.350866in}}%
\pgfpathlineto{\pgfqpoint{1.893816in}{1.377642in}}%
\pgfpathlineto{\pgfqpoint{1.915239in}{1.398750in}}%
\pgfpathlineto{\pgfqpoint{1.936662in}{1.414007in}}%
\pgfpathlineto{\pgfqpoint{1.958085in}{1.425630in}}%
\pgfpathlineto{\pgfqpoint{1.979508in}{1.436568in}}%
\pgfpathlineto{\pgfqpoint{2.000931in}{1.446434in}}%
\pgfpathlineto{\pgfqpoint{2.022354in}{1.453858in}}%
\pgfpathlineto{\pgfqpoint{2.043777in}{1.458678in}}%
\pgfpathlineto{\pgfqpoint{2.065200in}{1.460540in}}%
\pgfpathlineto{\pgfqpoint{2.086623in}{1.458835in}}%
\pgfpathlineto{\pgfqpoint{2.108046in}{1.454520in}}%
\pgfpathlineto{\pgfqpoint{2.129469in}{1.449779in}}%
\pgfpathlineto{\pgfqpoint{2.150892in}{1.447394in}}%
\pgfpathlineto{\pgfqpoint{2.172315in}{1.448036in}}%
\pgfpathlineto{\pgfqpoint{2.193738in}{1.449718in}}%
\pgfpathlineto{\pgfqpoint{2.215161in}{1.450722in}}%
\pgfpathlineto{\pgfqpoint{2.236584in}{1.453346in}}%
\pgfpathlineto{\pgfqpoint{2.258007in}{1.466173in}}%
\pgfpathlineto{\pgfqpoint{2.279430in}{1.502082in}}%
\pgfpathlineto{\pgfqpoint{2.300853in}{1.570122in}}%
\pgfpathlineto{\pgfqpoint{2.322276in}{1.668101in}}%
\pgfpathlineto{\pgfqpoint{2.343699in}{1.780357in}}%
\pgfpathlineto{\pgfqpoint{2.365122in}{1.881826in}}%
\pgfpathlineto{\pgfqpoint{2.386545in}{1.948424in}}%
\pgfpathlineto{\pgfqpoint{2.407968in}{1.964924in}}%
\pgfpathlineto{\pgfqpoint{2.429391in}{1.929864in}}%
\pgfpathlineto{\pgfqpoint{2.450814in}{1.858447in}}%
\pgfpathlineto{\pgfqpoint{2.472237in}{1.779325in}}%
\pgfpathlineto{\pgfqpoint{2.493660in}{1.719750in}}%
\pgfpathlineto{\pgfqpoint{2.515083in}{1.687727in}}%
\pgfpathlineto{\pgfqpoint{2.536507in}{1.669153in}}%
\pgfpathlineto{\pgfqpoint{2.557930in}{1.641323in}}%
\pgfpathlineto{\pgfqpoint{2.579353in}{1.588501in}}%
\pgfpathlineto{\pgfqpoint{2.600776in}{1.511076in}}%
\pgfpathlineto{\pgfqpoint{2.622199in}{1.422289in}}%
\pgfpathlineto{\pgfqpoint{2.643622in}{1.331567in}}%
\pgfpathlineto{\pgfqpoint{2.665045in}{1.240259in}}%
\pgfusepath{stroke}%
\end{pgfscope}%
\begin{pgfscope}%
\pgfpathrectangle{\pgfqpoint{0.693056in}{0.574768in}}{\pgfqpoint{2.380097in}{1.667836in}}%
\pgfusepath{clip}%
\pgfsetroundcap%
\pgfsetroundjoin%
\pgfsetlinewidth{1.505625pt}%
\definecolor{currentstroke}{rgb}{0.768627,0.305882,0.321569}%
\pgfsetstrokecolor{currentstroke}%
\pgfsetdash{}{0pt}%
\pgfpathmoveto{\pgfqpoint{0.801242in}{1.730158in}}%
\pgfpathlineto{\pgfqpoint{0.822665in}{1.835092in}}%
\pgfpathlineto{\pgfqpoint{0.844088in}{1.917736in}}%
\pgfpathlineto{\pgfqpoint{0.865511in}{1.964243in}}%
\pgfpathlineto{\pgfqpoint{0.886934in}{1.978108in}}%
\pgfpathlineto{\pgfqpoint{0.908357in}{1.974136in}}%
\pgfpathlineto{\pgfqpoint{0.929780in}{1.966116in}}%
\pgfpathlineto{\pgfqpoint{0.951203in}{1.961731in}}%
\pgfpathlineto{\pgfqpoint{0.972626in}{1.964924in}}%
\pgfpathlineto{\pgfqpoint{0.994049in}{1.979213in}}%
\pgfpathlineto{\pgfqpoint{1.015472in}{2.007841in}}%
\pgfpathlineto{\pgfqpoint{1.036895in}{2.049795in}}%
\pgfpathlineto{\pgfqpoint{1.058318in}{2.097038in}}%
\pgfpathlineto{\pgfqpoint{1.079741in}{2.137938in}}%
\pgfpathlineto{\pgfqpoint{1.101164in}{2.162631in}}%
\pgfpathlineto{\pgfqpoint{1.122587in}{2.166793in}}%
\pgfpathlineto{\pgfqpoint{1.144010in}{2.153755in}}%
\pgfpathlineto{\pgfqpoint{1.165433in}{2.133075in}}%
\pgfpathlineto{\pgfqpoint{1.186856in}{2.114325in}}%
\pgfpathlineto{\pgfqpoint{1.208280in}{2.100527in}}%
\pgfpathlineto{\pgfqpoint{1.229703in}{2.087293in}}%
\pgfpathlineto{\pgfqpoint{1.251126in}{2.067974in}}%
\pgfpathlineto{\pgfqpoint{1.272549in}{2.039065in}}%
\pgfpathlineto{\pgfqpoint{1.293972in}{2.001690in}}%
\pgfpathlineto{\pgfqpoint{1.315395in}{1.958945in}}%
\pgfpathlineto{\pgfqpoint{1.336818in}{1.913689in}}%
\pgfpathlineto{\pgfqpoint{1.358241in}{1.867909in}}%
\pgfpathlineto{\pgfqpoint{1.379664in}{1.822600in}}%
\pgfpathlineto{\pgfqpoint{1.401087in}{1.777697in}}%
\pgfpathlineto{\pgfqpoint{1.422510in}{1.732977in}}%
\pgfpathlineto{\pgfqpoint{1.443933in}{1.690944in}}%
\pgfpathlineto{\pgfqpoint{1.465356in}{1.657573in}}%
\pgfpathlineto{\pgfqpoint{1.486779in}{1.638596in}}%
\pgfpathlineto{\pgfqpoint{1.508202in}{1.634359in}}%
\pgfpathlineto{\pgfqpoint{1.529625in}{1.637046in}}%
\pgfpathlineto{\pgfqpoint{1.551048in}{1.635394in}}%
\pgfpathlineto{\pgfqpoint{1.572471in}{1.624103in}}%
\pgfpathlineto{\pgfqpoint{1.593894in}{1.609944in}}%
\pgfpathlineto{\pgfqpoint{1.615317in}{1.610218in}}%
\pgfpathlineto{\pgfqpoint{1.636740in}{1.642780in}}%
\pgfpathlineto{\pgfqpoint{1.658163in}{1.712427in}}%
\pgfpathlineto{\pgfqpoint{1.679586in}{1.804550in}}%
\pgfpathlineto{\pgfqpoint{1.701009in}{1.893065in}}%
\pgfpathlineto{\pgfqpoint{1.722432in}{1.955780in}}%
\pgfpathlineto{\pgfqpoint{1.743855in}{1.985795in}}%
\pgfpathlineto{\pgfqpoint{1.765278in}{1.990903in}}%
\pgfpathlineto{\pgfqpoint{1.786701in}{1.983829in}}%
\pgfpathlineto{\pgfqpoint{1.808124in}{1.973982in}}%
\pgfpathlineto{\pgfqpoint{1.829547in}{1.966477in}}%
\pgfpathlineto{\pgfqpoint{1.850970in}{1.964924in}}%
\pgfpathlineto{\pgfqpoint{1.872393in}{1.973719in}}%
\pgfpathlineto{\pgfqpoint{1.893816in}{1.997036in}}%
\pgfpathlineto{\pgfqpoint{1.915239in}{2.034589in}}%
\pgfpathlineto{\pgfqpoint{1.936662in}{2.078838in}}%
\pgfpathlineto{\pgfqpoint{1.958085in}{2.117030in}}%
\pgfpathlineto{\pgfqpoint{1.979508in}{2.137366in}}%
\pgfpathlineto{\pgfqpoint{2.000931in}{2.135323in}}%
\pgfpathlineto{\pgfqpoint{2.022354in}{2.117352in}}%
\pgfusepath{stroke}%
\end{pgfscope}%
\begin{pgfscope}%
\pgfpathrectangle{\pgfqpoint{0.693056in}{0.574768in}}{\pgfqpoint{2.380097in}{1.667836in}}%
\pgfusepath{clip}%
\pgfsetroundcap%
\pgfsetroundjoin%
\pgfsetlinewidth{1.505625pt}%
\definecolor{currentstroke}{rgb}{0.505882,0.447059,0.701961}%
\pgfsetstrokecolor{currentstroke}%
\pgfsetdash{}{0pt}%
\pgfpathmoveto{\pgfqpoint{0.801242in}{1.530109in}}%
\pgfpathlineto{\pgfqpoint{0.822665in}{1.555994in}}%
\pgfpathlineto{\pgfqpoint{0.844088in}{1.590974in}}%
\pgfpathlineto{\pgfqpoint{0.865511in}{1.640607in}}%
\pgfpathlineto{\pgfqpoint{0.886934in}{1.703113in}}%
\pgfpathlineto{\pgfqpoint{0.908357in}{1.771381in}}%
\pgfpathlineto{\pgfqpoint{0.929780in}{1.836527in}}%
\pgfpathlineto{\pgfqpoint{0.951203in}{1.891797in}}%
\pgfpathlineto{\pgfqpoint{0.972626in}{1.933357in}}%
\pgfpathlineto{\pgfqpoint{0.994049in}{1.958299in}}%
\pgfpathlineto{\pgfqpoint{1.015472in}{1.964924in}}%
\pgfpathlineto{\pgfqpoint{1.036895in}{1.954418in}}%
\pgfpathlineto{\pgfqpoint{1.058318in}{1.932491in}}%
\pgfpathlineto{\pgfqpoint{1.079741in}{1.908496in}}%
\pgfpathlineto{\pgfqpoint{1.101164in}{1.888844in}}%
\pgfpathlineto{\pgfqpoint{1.122587in}{1.870115in}}%
\pgfpathlineto{\pgfqpoint{1.144010in}{1.843344in}}%
\pgfpathlineto{\pgfqpoint{1.165433in}{1.802160in}}%
\pgfpathlineto{\pgfqpoint{1.186856in}{1.747336in}}%
\pgfpathlineto{\pgfqpoint{1.208280in}{1.683716in}}%
\pgfpathlineto{\pgfqpoint{1.229703in}{1.613157in}}%
\pgfpathlineto{\pgfqpoint{1.251126in}{1.536243in}}%
\pgfpathlineto{\pgfqpoint{1.272549in}{1.457728in}}%
\pgfpathlineto{\pgfqpoint{1.293972in}{1.383260in}}%
\pgfpathlineto{\pgfqpoint{1.315395in}{1.312663in}}%
\pgfpathlineto{\pgfqpoint{1.336818in}{1.242311in}}%
\pgfpathlineto{\pgfqpoint{1.358241in}{1.174581in}}%
\pgfpathlineto{\pgfqpoint{1.379664in}{1.117780in}}%
\pgfpathlineto{\pgfqpoint{1.401087in}{1.076799in}}%
\pgfpathlineto{\pgfqpoint{1.422510in}{1.049766in}}%
\pgfpathlineto{\pgfqpoint{1.443933in}{1.033003in}}%
\pgfpathlineto{\pgfqpoint{1.465356in}{1.024054in}}%
\pgfpathlineto{\pgfqpoint{1.486779in}{1.020475in}}%
\pgfpathlineto{\pgfqpoint{1.508202in}{1.019878in}}%
\pgfpathlineto{\pgfqpoint{1.529625in}{1.021278in}}%
\pgfpathlineto{\pgfqpoint{1.551048in}{1.024618in}}%
\pgfpathlineto{\pgfqpoint{1.572471in}{1.030830in}}%
\pgfpathlineto{\pgfqpoint{1.593894in}{1.041735in}}%
\pgfpathlineto{\pgfqpoint{1.615317in}{1.059731in}}%
\pgfpathlineto{\pgfqpoint{1.636740in}{1.088680in}}%
\pgfpathlineto{\pgfqpoint{1.658163in}{1.131480in}}%
\pgfpathlineto{\pgfqpoint{1.679586in}{1.187601in}}%
\pgfpathlineto{\pgfqpoint{1.701009in}{1.251407in}}%
\pgfpathlineto{\pgfqpoint{1.722432in}{1.313793in}}%
\pgfpathlineto{\pgfqpoint{1.743855in}{1.367642in}}%
\pgfpathlineto{\pgfqpoint{1.765278in}{1.411309in}}%
\pgfpathlineto{\pgfqpoint{1.786701in}{1.443631in}}%
\pgfpathlineto{\pgfqpoint{1.808124in}{1.465503in}}%
\pgfpathlineto{\pgfqpoint{1.829547in}{1.483221in}}%
\pgfpathlineto{\pgfqpoint{1.850970in}{1.501415in}}%
\pgfpathlineto{\pgfqpoint{1.872393in}{1.520160in}}%
\pgfpathlineto{\pgfqpoint{1.893816in}{1.542096in}}%
\pgfpathlineto{\pgfqpoint{1.915239in}{1.574414in}}%
\pgfpathlineto{\pgfqpoint{1.936662in}{1.622234in}}%
\pgfpathlineto{\pgfqpoint{1.958085in}{1.685226in}}%
\pgfpathlineto{\pgfqpoint{1.979508in}{1.760744in}}%
\pgfpathlineto{\pgfqpoint{2.000931in}{1.842742in}}%
\pgfpathlineto{\pgfqpoint{2.022354in}{1.917875in}}%
\pgfpathlineto{\pgfqpoint{2.043777in}{1.970774in}}%
\pgfpathlineto{\pgfqpoint{2.065200in}{1.997116in}}%
\pgfpathlineto{\pgfqpoint{2.086623in}{2.001705in}}%
\pgfpathlineto{\pgfqpoint{2.108046in}{1.989511in}}%
\pgfpathlineto{\pgfqpoint{2.129469in}{1.964924in}}%
\pgfpathlineto{\pgfqpoint{2.150892in}{1.934468in}}%
\pgfpathlineto{\pgfqpoint{2.172315in}{1.904114in}}%
\pgfpathlineto{\pgfqpoint{2.193738in}{1.874346in}}%
\pgfpathlineto{\pgfqpoint{2.215161in}{1.840481in}}%
\pgfpathlineto{\pgfqpoint{2.236584in}{1.795764in}}%
\pgfpathlineto{\pgfqpoint{2.258007in}{1.734726in}}%
\pgfpathlineto{\pgfqpoint{2.279430in}{1.657352in}}%
\pgfpathlineto{\pgfqpoint{2.300853in}{1.569567in}}%
\pgfpathlineto{\pgfqpoint{2.322276in}{1.477492in}}%
\pgfpathlineto{\pgfqpoint{2.343699in}{1.384085in}}%
\pgfusepath{stroke}%
\end{pgfscope}%
\begin{pgfscope}%
\pgfsetrectcap%
\pgfsetmiterjoin%
\pgfsetlinewidth{1.254687pt}%
\definecolor{currentstroke}{rgb}{1.000000,1.000000,1.000000}%
\pgfsetstrokecolor{currentstroke}%
\pgfsetdash{}{0pt}%
\pgfpathmoveto{\pgfqpoint{0.693056in}{0.574768in}}%
\pgfpathlineto{\pgfqpoint{0.693056in}{2.242604in}}%
\pgfusepath{stroke}%
\end{pgfscope}%
\begin{pgfscope}%
\pgfsetrectcap%
\pgfsetmiterjoin%
\pgfsetlinewidth{1.254687pt}%
\definecolor{currentstroke}{rgb}{1.000000,1.000000,1.000000}%
\pgfsetstrokecolor{currentstroke}%
\pgfsetdash{}{0pt}%
\pgfpathmoveto{\pgfqpoint{3.073153in}{0.574768in}}%
\pgfpathlineto{\pgfqpoint{3.073153in}{2.242604in}}%
\pgfusepath{stroke}%
\end{pgfscope}%
\begin{pgfscope}%
\pgfsetrectcap%
\pgfsetmiterjoin%
\pgfsetlinewidth{1.254687pt}%
\definecolor{currentstroke}{rgb}{1.000000,1.000000,1.000000}%
\pgfsetstrokecolor{currentstroke}%
\pgfsetdash{}{0pt}%
\pgfpathmoveto{\pgfqpoint{0.693056in}{0.574768in}}%
\pgfpathlineto{\pgfqpoint{3.073153in}{0.574768in}}%
\pgfusepath{stroke}%
\end{pgfscope}%
\begin{pgfscope}%
\pgfsetrectcap%
\pgfsetmiterjoin%
\pgfsetlinewidth{1.254687pt}%
\definecolor{currentstroke}{rgb}{1.000000,1.000000,1.000000}%
\pgfsetstrokecolor{currentstroke}%
\pgfsetdash{}{0pt}%
\pgfpathmoveto{\pgfqpoint{0.693056in}{2.242604in}}%
\pgfpathlineto{\pgfqpoint{3.073153in}{2.242604in}}%
\pgfusepath{stroke}%
\end{pgfscope}%
\begin{pgfscope}%
\definecolor{textcolor}{rgb}{0.150000,0.150000,0.150000}%
\pgfsetstrokecolor{textcolor}%
\pgfsetfillcolor{textcolor}%
\pgftext[x=1.883105in,y=2.325938in,,base]{\color{textcolor}\sffamily\fontsize{11.000000}{13.200000}\selectfont (a) Cluster 1 members}%
\end{pgfscope}%
\begin{pgfscope}%
\pgfsetbuttcap%
\pgfsetmiterjoin%
\definecolor{currentfill}{rgb}{0.917647,0.917647,0.949020}%
\pgfsetfillcolor{currentfill}%
\pgfsetlinewidth{0.000000pt}%
\definecolor{currentstroke}{rgb}{0.000000,0.000000,0.000000}%
\pgfsetstrokecolor{currentstroke}%
\pgfsetstrokeopacity{0.000000}%
\pgfsetdash{}{0pt}%
\pgfpathmoveto{\pgfqpoint{3.846209in}{0.574768in}}%
\pgfpathlineto{\pgfqpoint{6.226306in}{0.574768in}}%
\pgfpathlineto{\pgfqpoint{6.226306in}{2.242604in}}%
\pgfpathlineto{\pgfqpoint{3.846209in}{2.242604in}}%
\pgfpathclose%
\pgfusepath{fill}%
\end{pgfscope}%
\begin{pgfscope}%
\pgfpathrectangle{\pgfqpoint{3.846209in}{0.574768in}}{\pgfqpoint{2.380097in}{1.667836in}}%
\pgfusepath{clip}%
\pgfsetroundcap%
\pgfsetroundjoin%
\pgfsetlinewidth{1.003750pt}%
\definecolor{currentstroke}{rgb}{1.000000,1.000000,1.000000}%
\pgfsetstrokecolor{currentstroke}%
\pgfsetdash{}{0pt}%
\pgfpathmoveto{\pgfqpoint{3.954395in}{0.574768in}}%
\pgfpathlineto{\pgfqpoint{3.954395in}{2.242604in}}%
\pgfusepath{stroke}%
\end{pgfscope}%
\begin{pgfscope}%
\definecolor{textcolor}{rgb}{0.150000,0.150000,0.150000}%
\pgfsetstrokecolor{textcolor}%
\pgfsetfillcolor{textcolor}%
\pgftext[x=3.954395in,y=0.442824in,,top]{\color{textcolor}\sffamily\fontsize{11.000000}{13.200000}\selectfont \(\displaystyle 0.0\)}%
\end{pgfscope}%
\begin{pgfscope}%
\pgfpathrectangle{\pgfqpoint{3.846209in}{0.574768in}}{\pgfqpoint{2.380097in}{1.667836in}}%
\pgfusepath{clip}%
\pgfsetroundcap%
\pgfsetroundjoin%
\pgfsetlinewidth{1.003750pt}%
\definecolor{currentstroke}{rgb}{1.000000,1.000000,1.000000}%
\pgfsetstrokecolor{currentstroke}%
\pgfsetdash{}{0pt}%
\pgfpathmoveto{\pgfqpoint{4.682778in}{0.574768in}}%
\pgfpathlineto{\pgfqpoint{4.682778in}{2.242604in}}%
\pgfusepath{stroke}%
\end{pgfscope}%
\begin{pgfscope}%
\definecolor{textcolor}{rgb}{0.150000,0.150000,0.150000}%
\pgfsetstrokecolor{textcolor}%
\pgfsetfillcolor{textcolor}%
\pgftext[x=4.682778in,y=0.442824in,,top]{\color{textcolor}\sffamily\fontsize{11.000000}{13.200000}\selectfont \(\displaystyle 0.5\)}%
\end{pgfscope}%
\begin{pgfscope}%
\pgfpathrectangle{\pgfqpoint{3.846209in}{0.574768in}}{\pgfqpoint{2.380097in}{1.667836in}}%
\pgfusepath{clip}%
\pgfsetroundcap%
\pgfsetroundjoin%
\pgfsetlinewidth{1.003750pt}%
\definecolor{currentstroke}{rgb}{1.000000,1.000000,1.000000}%
\pgfsetstrokecolor{currentstroke}%
\pgfsetdash{}{0pt}%
\pgfpathmoveto{\pgfqpoint{5.411160in}{0.574768in}}%
\pgfpathlineto{\pgfqpoint{5.411160in}{2.242604in}}%
\pgfusepath{stroke}%
\end{pgfscope}%
\begin{pgfscope}%
\definecolor{textcolor}{rgb}{0.150000,0.150000,0.150000}%
\pgfsetstrokecolor{textcolor}%
\pgfsetfillcolor{textcolor}%
\pgftext[x=5.411160in,y=0.442824in,,top]{\color{textcolor}\sffamily\fontsize{11.000000}{13.200000}\selectfont \(\displaystyle 1.0\)}%
\end{pgfscope}%
\begin{pgfscope}%
\pgfpathrectangle{\pgfqpoint{3.846209in}{0.574768in}}{\pgfqpoint{2.380097in}{1.667836in}}%
\pgfusepath{clip}%
\pgfsetroundcap%
\pgfsetroundjoin%
\pgfsetlinewidth{1.003750pt}%
\definecolor{currentstroke}{rgb}{1.000000,1.000000,1.000000}%
\pgfsetstrokecolor{currentstroke}%
\pgfsetdash{}{0pt}%
\pgfpathmoveto{\pgfqpoint{6.139543in}{0.574768in}}%
\pgfpathlineto{\pgfqpoint{6.139543in}{2.242604in}}%
\pgfusepath{stroke}%
\end{pgfscope}%
\begin{pgfscope}%
\definecolor{textcolor}{rgb}{0.150000,0.150000,0.150000}%
\pgfsetstrokecolor{textcolor}%
\pgfsetfillcolor{textcolor}%
\pgftext[x=6.139543in,y=0.442824in,,top]{\color{textcolor}\sffamily\fontsize{11.000000}{13.200000}\selectfont \(\displaystyle 1.5\)}%
\end{pgfscope}%
\begin{pgfscope}%
\definecolor{textcolor}{rgb}{0.150000,0.150000,0.150000}%
\pgfsetstrokecolor{textcolor}%
\pgfsetfillcolor{textcolor}%
\pgftext[x=5.036258in,y=0.252083in,,top]{\color{textcolor}\sffamily\fontsize{11.000000}{13.200000}\selectfont Time [s]}%
\end{pgfscope}%
\begin{pgfscope}%
\pgfpathrectangle{\pgfqpoint{3.846209in}{0.574768in}}{\pgfqpoint{2.380097in}{1.667836in}}%
\pgfusepath{clip}%
\pgfsetroundcap%
\pgfsetroundjoin%
\pgfsetlinewidth{1.003750pt}%
\definecolor{currentstroke}{rgb}{1.000000,1.000000,1.000000}%
\pgfsetstrokecolor{currentstroke}%
\pgfsetdash{}{0pt}%
\pgfpathmoveto{\pgfqpoint{3.846209in}{0.829935in}}%
\pgfpathlineto{\pgfqpoint{6.226306in}{0.829935in}}%
\pgfusepath{stroke}%
\end{pgfscope}%
\begin{pgfscope}%
\definecolor{textcolor}{rgb}{0.150000,0.150000,0.150000}%
\pgfsetstrokecolor{textcolor}%
\pgfsetfillcolor{textcolor}%
\pgftext[x=3.443894in,y=0.777128in,left,base]{\color{textcolor}\sffamily\fontsize{11.000000}{13.200000}\selectfont \(\displaystyle -20\)}%
\end{pgfscope}%
\begin{pgfscope}%
\pgfpathrectangle{\pgfqpoint{3.846209in}{0.574768in}}{\pgfqpoint{2.380097in}{1.667836in}}%
\pgfusepath{clip}%
\pgfsetroundcap%
\pgfsetroundjoin%
\pgfsetlinewidth{1.003750pt}%
\definecolor{currentstroke}{rgb}{1.000000,1.000000,1.000000}%
\pgfsetstrokecolor{currentstroke}%
\pgfsetdash{}{0pt}%
\pgfpathmoveto{\pgfqpoint{3.846209in}{1.490377in}}%
\pgfpathlineto{\pgfqpoint{6.226306in}{1.490377in}}%
\pgfusepath{stroke}%
\end{pgfscope}%
\begin{pgfscope}%
\definecolor{textcolor}{rgb}{0.150000,0.150000,0.150000}%
\pgfsetstrokecolor{textcolor}%
\pgfsetfillcolor{textcolor}%
\pgftext[x=3.443894in,y=1.437570in,left,base]{\color{textcolor}\sffamily\fontsize{11.000000}{13.200000}\selectfont \(\displaystyle -10\)}%
\end{pgfscope}%
\begin{pgfscope}%
\pgfpathrectangle{\pgfqpoint{3.846209in}{0.574768in}}{\pgfqpoint{2.380097in}{1.667836in}}%
\pgfusepath{clip}%
\pgfsetroundcap%
\pgfsetroundjoin%
\pgfsetlinewidth{1.003750pt}%
\definecolor{currentstroke}{rgb}{1.000000,1.000000,1.000000}%
\pgfsetstrokecolor{currentstroke}%
\pgfsetdash{}{0pt}%
\pgfpathmoveto{\pgfqpoint{3.846209in}{2.150819in}}%
\pgfpathlineto{\pgfqpoint{6.226306in}{2.150819in}}%
\pgfusepath{stroke}%
\end{pgfscope}%
\begin{pgfscope}%
\definecolor{textcolor}{rgb}{0.150000,0.150000,0.150000}%
\pgfsetstrokecolor{textcolor}%
\pgfsetfillcolor{textcolor}%
\pgftext[x=3.638223in,y=2.098013in,left,base]{\color{textcolor}\sffamily\fontsize{11.000000}{13.200000}\selectfont \(\displaystyle 0\)}%
\end{pgfscope}%
\begin{pgfscope}%
\definecolor{textcolor}{rgb}{0.150000,0.150000,0.150000}%
\pgfsetstrokecolor{textcolor}%
\pgfsetfillcolor{textcolor}%
\pgftext[x=3.388338in,y=1.408686in,,bottom,rotate=90.000000]{\color{textcolor}\sffamily\fontsize{11.000000}{13.200000}\selectfont GLS}%
\end{pgfscope}%
\begin{pgfscope}%
\pgfpathrectangle{\pgfqpoint{3.846209in}{0.574768in}}{\pgfqpoint{2.380097in}{1.667836in}}%
\pgfusepath{clip}%
\pgfsetroundcap%
\pgfsetroundjoin%
\pgfsetlinewidth{1.505625pt}%
\definecolor{currentstroke}{rgb}{0.298039,0.447059,0.690196}%
\pgfsetstrokecolor{currentstroke}%
\pgfsetdash{}{0pt}%
\pgfpathmoveto{\pgfqpoint{3.954395in}{1.535856in}}%
\pgfpathlineto{\pgfqpoint{3.976807in}{1.589270in}}%
\pgfpathlineto{\pgfqpoint{3.999219in}{1.652728in}}%
\pgfpathlineto{\pgfqpoint{4.021631in}{1.733193in}}%
\pgfpathlineto{\pgfqpoint{4.044042in}{1.831472in}}%
\pgfpathlineto{\pgfqpoint{4.066454in}{1.938659in}}%
\pgfpathlineto{\pgfqpoint{4.088866in}{2.036550in}}%
\pgfpathlineto{\pgfqpoint{4.111278in}{2.107013in}}%
\pgfpathlineto{\pgfqpoint{4.133689in}{2.143703in}}%
\pgfpathlineto{\pgfqpoint{4.156101in}{2.150819in}}%
\pgfpathlineto{\pgfqpoint{4.178513in}{2.136634in}}%
\pgfpathlineto{\pgfqpoint{4.200925in}{2.111671in}}%
\pgfpathlineto{\pgfqpoint{4.223337in}{2.085283in}}%
\pgfpathlineto{\pgfqpoint{4.245748in}{2.059323in}}%
\pgfpathlineto{\pgfqpoint{4.268160in}{2.027243in}}%
\pgfpathlineto{\pgfqpoint{4.290572in}{1.980430in}}%
\pgfpathlineto{\pgfqpoint{4.312984in}{1.916713in}}%
\pgfpathlineto{\pgfqpoint{4.335395in}{1.842457in}}%
\pgfpathlineto{\pgfqpoint{4.357807in}{1.765868in}}%
\pgfpathlineto{\pgfqpoint{4.380219in}{1.688121in}}%
\pgfpathlineto{\pgfqpoint{4.402631in}{1.606504in}}%
\pgfpathlineto{\pgfqpoint{4.425042in}{1.523089in}}%
\pgfpathlineto{\pgfqpoint{4.447454in}{1.443429in}}%
\pgfpathlineto{\pgfqpoint{4.469866in}{1.369986in}}%
\pgfpathlineto{\pgfqpoint{4.492278in}{1.302534in}}%
\pgfpathlineto{\pgfqpoint{4.514690in}{1.241341in}}%
\pgfpathlineto{\pgfqpoint{4.537101in}{1.187879in}}%
\pgfpathlineto{\pgfqpoint{4.559513in}{1.143717in}}%
\pgfpathlineto{\pgfqpoint{4.581925in}{1.110345in}}%
\pgfpathlineto{\pgfqpoint{4.604337in}{1.088979in}}%
\pgfpathlineto{\pgfqpoint{4.626748in}{1.078501in}}%
\pgfpathlineto{\pgfqpoint{4.649160in}{1.074747in}}%
\pgfpathlineto{\pgfqpoint{4.671572in}{1.073035in}}%
\pgfpathlineto{\pgfqpoint{4.693984in}{1.071678in}}%
\pgfpathlineto{\pgfqpoint{4.716395in}{1.072256in}}%
\pgfpathlineto{\pgfqpoint{4.738807in}{1.078172in}}%
\pgfpathlineto{\pgfqpoint{4.761219in}{1.093553in}}%
\pgfpathlineto{\pgfqpoint{4.783631in}{1.121635in}}%
\pgfpathlineto{\pgfqpoint{4.806043in}{1.163091in}}%
\pgfpathlineto{\pgfqpoint{4.828454in}{1.214682in}}%
\pgfpathlineto{\pgfqpoint{4.850866in}{1.271897in}}%
\pgfpathlineto{\pgfqpoint{4.873278in}{1.331435in}}%
\pgfpathlineto{\pgfqpoint{4.895690in}{1.391887in}}%
\pgfpathlineto{\pgfqpoint{4.918101in}{1.452489in}}%
\pgfpathlineto{\pgfqpoint{4.940513in}{1.511130in}}%
\pgfpathlineto{\pgfqpoint{4.962925in}{1.565741in}}%
\pgfpathlineto{\pgfqpoint{4.985337in}{1.618775in}}%
\pgfpathlineto{\pgfqpoint{5.007748in}{1.677371in}}%
\pgfpathlineto{\pgfqpoint{5.030160in}{1.747611in}}%
\pgfpathlineto{\pgfqpoint{5.052572in}{1.833171in}}%
\pgfpathlineto{\pgfqpoint{5.074984in}{1.932540in}}%
\pgfpathlineto{\pgfqpoint{5.097396in}{2.030385in}}%
\pgfpathlineto{\pgfqpoint{5.119807in}{2.103892in}}%
\pgfpathlineto{\pgfqpoint{5.142219in}{2.142189in}}%
\pgfpathlineto{\pgfqpoint{5.164631in}{2.150819in}}%
\pgfpathlineto{\pgfqpoint{5.187043in}{2.140219in}}%
\pgfpathlineto{\pgfqpoint{5.209454in}{2.119195in}}%
\pgfpathlineto{\pgfqpoint{5.231866in}{2.094095in}}%
\pgfpathlineto{\pgfqpoint{5.254278in}{2.066335in}}%
\pgfpathlineto{\pgfqpoint{5.276690in}{2.031294in}}%
\pgfpathlineto{\pgfqpoint{5.299102in}{1.982329in}}%
\pgfpathlineto{\pgfqpoint{5.321513in}{1.916569in}}%
\pgfpathlineto{\pgfqpoint{5.343925in}{1.836861in}}%
\pgfpathlineto{\pgfqpoint{5.366337in}{1.749408in}}%
\pgfusepath{stroke}%
\end{pgfscope}%
\begin{pgfscope}%
\pgfpathrectangle{\pgfqpoint{3.846209in}{0.574768in}}{\pgfqpoint{2.380097in}{1.667836in}}%
\pgfusepath{clip}%
\pgfsetroundcap%
\pgfsetroundjoin%
\pgfsetlinewidth{1.505625pt}%
\definecolor{currentstroke}{rgb}{0.866667,0.517647,0.321569}%
\pgfsetstrokecolor{currentstroke}%
\pgfsetdash{}{0pt}%
\pgfpathmoveto{\pgfqpoint{3.954395in}{1.711617in}}%
\pgfpathlineto{\pgfqpoint{3.975206in}{1.706849in}}%
\pgfpathlineto{\pgfqpoint{3.996017in}{1.702309in}}%
\pgfpathlineto{\pgfqpoint{4.016828in}{1.700577in}}%
\pgfpathlineto{\pgfqpoint{4.037639in}{1.706934in}}%
\pgfpathlineto{\pgfqpoint{4.058450in}{1.729909in}}%
\pgfpathlineto{\pgfqpoint{4.079261in}{1.777216in}}%
\pgfpathlineto{\pgfqpoint{4.100072in}{1.849048in}}%
\pgfpathlineto{\pgfqpoint{4.120883in}{1.935753in}}%
\pgfpathlineto{\pgfqpoint{4.141694in}{2.020412in}}%
\pgfpathlineto{\pgfqpoint{4.162505in}{2.088051in}}%
\pgfpathlineto{\pgfqpoint{4.183315in}{2.131414in}}%
\pgfpathlineto{\pgfqpoint{4.204126in}{2.150819in}}%
\pgfpathlineto{\pgfqpoint{4.224937in}{2.150743in}}%
\pgfpathlineto{\pgfqpoint{4.245748in}{2.138486in}}%
\pgfpathlineto{\pgfqpoint{4.266559in}{2.122508in}}%
\pgfpathlineto{\pgfqpoint{4.287370in}{2.104880in}}%
\pgfpathlineto{\pgfqpoint{4.308181in}{2.076107in}}%
\pgfpathlineto{\pgfqpoint{4.328992in}{2.018544in}}%
\pgfpathlineto{\pgfqpoint{4.349803in}{1.919811in}}%
\pgfpathlineto{\pgfqpoint{4.370614in}{1.786311in}}%
\pgfpathlineto{\pgfqpoint{4.391425in}{1.640875in}}%
\pgfpathlineto{\pgfqpoint{4.412236in}{1.505597in}}%
\pgfpathlineto{\pgfqpoint{4.433047in}{1.390076in}}%
\pgfpathlineto{\pgfqpoint{4.453858in}{1.293144in}}%
\pgfpathlineto{\pgfqpoint{4.474669in}{1.210011in}}%
\pgfpathlineto{\pgfqpoint{4.495479in}{1.135480in}}%
\pgfpathlineto{\pgfqpoint{4.516290in}{1.066060in}}%
\pgfpathlineto{\pgfqpoint{4.537101in}{0.997819in}}%
\pgfpathlineto{\pgfqpoint{4.557912in}{0.925752in}}%
\pgfpathlineto{\pgfqpoint{4.578723in}{0.857526in}}%
\pgfpathlineto{\pgfqpoint{4.599534in}{0.799882in}}%
\pgfpathlineto{\pgfqpoint{4.620345in}{0.755309in}}%
\pgfpathlineto{\pgfqpoint{4.641156in}{0.724955in}}%
\pgfpathlineto{\pgfqpoint{4.661967in}{0.712304in}}%
\pgfpathlineto{\pgfqpoint{4.682778in}{0.719478in}}%
\pgfpathlineto{\pgfqpoint{4.703589in}{0.740790in}}%
\pgfpathlineto{\pgfqpoint{4.724400in}{0.763272in}}%
\pgfpathlineto{\pgfqpoint{4.745211in}{0.776875in}}%
\pgfpathlineto{\pgfqpoint{4.766022in}{0.782190in}}%
\pgfpathlineto{\pgfqpoint{4.786832in}{0.788891in}}%
\pgfpathlineto{\pgfqpoint{4.807643in}{0.808460in}}%
\pgfpathlineto{\pgfqpoint{4.828454in}{0.847816in}}%
\pgfpathlineto{\pgfqpoint{4.849265in}{0.907401in}}%
\pgfpathlineto{\pgfqpoint{4.870076in}{0.984626in}}%
\pgfpathlineto{\pgfqpoint{4.890887in}{1.076085in}}%
\pgfpathlineto{\pgfqpoint{4.911698in}{1.173939in}}%
\pgfpathlineto{\pgfqpoint{4.932509in}{1.265915in}}%
\pgfpathlineto{\pgfqpoint{4.953320in}{1.342534in}}%
\pgfpathlineto{\pgfqpoint{4.974131in}{1.402773in}}%
\pgfpathlineto{\pgfqpoint{4.994942in}{1.452042in}}%
\pgfpathlineto{\pgfqpoint{5.015753in}{1.496585in}}%
\pgfpathlineto{\pgfqpoint{5.036564in}{1.539442in}}%
\pgfpathlineto{\pgfqpoint{5.057375in}{1.581888in}}%
\pgfpathlineto{\pgfqpoint{5.078185in}{1.623976in}}%
\pgfpathlineto{\pgfqpoint{5.098996in}{1.662318in}}%
\pgfpathlineto{\pgfqpoint{5.119807in}{1.692006in}}%
\pgfpathlineto{\pgfqpoint{5.140618in}{1.710631in}}%
\pgfpathlineto{\pgfqpoint{5.161429in}{1.719819in}}%
\pgfpathlineto{\pgfqpoint{5.182240in}{1.722874in}}%
\pgfpathlineto{\pgfqpoint{5.203051in}{1.722948in}}%
\pgfpathlineto{\pgfqpoint{5.223862in}{1.721784in}}%
\pgfpathlineto{\pgfqpoint{5.244673in}{1.720352in}}%
\pgfpathlineto{\pgfqpoint{5.265484in}{1.719435in}}%
\pgfpathlineto{\pgfqpoint{5.286295in}{1.719306in}}%
\pgfpathlineto{\pgfqpoint{5.307106in}{1.722354in}}%
\pgfpathlineto{\pgfqpoint{5.327917in}{1.735964in}}%
\pgfpathlineto{\pgfqpoint{5.348728in}{1.770544in}}%
\pgfpathlineto{\pgfqpoint{5.369538in}{1.832279in}}%
\pgfpathlineto{\pgfqpoint{5.390349in}{1.916244in}}%
\pgfpathlineto{\pgfqpoint{5.411160in}{2.005804in}}%
\pgfpathlineto{\pgfqpoint{5.431971in}{2.081959in}}%
\pgfpathlineto{\pgfqpoint{5.452782in}{2.131719in}}%
\pgfpathlineto{\pgfqpoint{5.473593in}{2.150819in}}%
\pgfpathlineto{\pgfqpoint{5.494404in}{2.142339in}}%
\pgfpathlineto{\pgfqpoint{5.515215in}{2.115189in}}%
\pgfpathlineto{\pgfqpoint{5.536026in}{2.080674in}}%
\pgfpathlineto{\pgfqpoint{5.556837in}{2.044968in}}%
\pgfpathlineto{\pgfqpoint{5.577648in}{2.003398in}}%
\pgfpathlineto{\pgfqpoint{5.598459in}{1.941992in}}%
\pgfpathlineto{\pgfqpoint{5.619270in}{1.847648in}}%
\pgfpathlineto{\pgfqpoint{5.640081in}{1.719123in}}%
\pgfpathlineto{\pgfqpoint{5.660892in}{1.569865in}}%
\pgfpathlineto{\pgfqpoint{5.681702in}{1.419716in}}%
\pgfpathlineto{\pgfqpoint{5.702513in}{1.281719in}}%
\pgfpathlineto{\pgfqpoint{5.723324in}{1.155395in}}%
\pgfusepath{stroke}%
\end{pgfscope}%
\begin{pgfscope}%
\pgfpathrectangle{\pgfqpoint{3.846209in}{0.574768in}}{\pgfqpoint{2.380097in}{1.667836in}}%
\pgfusepath{clip}%
\pgfsetroundcap%
\pgfsetroundjoin%
\pgfsetlinewidth{1.505625pt}%
\definecolor{currentstroke}{rgb}{0.333333,0.658824,0.407843}%
\pgfsetstrokecolor{currentstroke}%
\pgfsetdash{}{0pt}%
\pgfpathmoveto{\pgfqpoint{3.954395in}{1.774712in}}%
\pgfpathlineto{\pgfqpoint{3.975206in}{1.778172in}}%
\pgfpathlineto{\pgfqpoint{3.996017in}{1.781355in}}%
\pgfpathlineto{\pgfqpoint{4.016828in}{1.785594in}}%
\pgfpathlineto{\pgfqpoint{4.037639in}{1.796719in}}%
\pgfpathlineto{\pgfqpoint{4.058450in}{1.824570in}}%
\pgfpathlineto{\pgfqpoint{4.079261in}{1.876132in}}%
\pgfpathlineto{\pgfqpoint{4.100072in}{1.948488in}}%
\pgfpathlineto{\pgfqpoint{4.120883in}{2.026762in}}%
\pgfpathlineto{\pgfqpoint{4.141694in}{2.091753in}}%
\pgfpathlineto{\pgfqpoint{4.162505in}{2.132386in}}%
\pgfpathlineto{\pgfqpoint{4.183315in}{2.149488in}}%
\pgfpathlineto{\pgfqpoint{4.204126in}{2.150819in}}%
\pgfpathlineto{\pgfqpoint{4.224937in}{2.144713in}}%
\pgfpathlineto{\pgfqpoint{4.245748in}{2.136479in}}%
\pgfpathlineto{\pgfqpoint{4.266559in}{2.126636in}}%
\pgfpathlineto{\pgfqpoint{4.287370in}{2.110748in}}%
\pgfpathlineto{\pgfqpoint{4.308181in}{2.082788in}}%
\pgfpathlineto{\pgfqpoint{4.328992in}{2.038638in}}%
\pgfpathlineto{\pgfqpoint{4.349803in}{1.979562in}}%
\pgfpathlineto{\pgfqpoint{4.370614in}{1.912128in}}%
\pgfpathlineto{\pgfqpoint{4.391425in}{1.842447in}}%
\pgfpathlineto{\pgfqpoint{4.412236in}{1.773205in}}%
\pgfpathlineto{\pgfqpoint{4.433047in}{1.704111in}}%
\pgfpathlineto{\pgfqpoint{4.453858in}{1.633044in}}%
\pgfpathlineto{\pgfqpoint{4.474669in}{1.559184in}}%
\pgfpathlineto{\pgfqpoint{4.495479in}{1.485593in}}%
\pgfpathlineto{\pgfqpoint{4.516290in}{1.416544in}}%
\pgfpathlineto{\pgfqpoint{4.537101in}{1.352996in}}%
\pgfpathlineto{\pgfqpoint{4.557912in}{1.293786in}}%
\pgfpathlineto{\pgfqpoint{4.578723in}{1.237463in}}%
\pgfpathlineto{\pgfqpoint{4.599534in}{1.183667in}}%
\pgfpathlineto{\pgfqpoint{4.620345in}{1.133654in}}%
\pgfpathlineto{\pgfqpoint{4.641156in}{1.089852in}}%
\pgfpathlineto{\pgfqpoint{4.661967in}{1.053841in}}%
\pgfpathlineto{\pgfqpoint{4.682778in}{1.027141in}}%
\pgfpathlineto{\pgfqpoint{4.703589in}{1.011808in}}%
\pgfpathlineto{\pgfqpoint{4.724400in}{1.008422in}}%
\pgfpathlineto{\pgfqpoint{4.745211in}{1.014313in}}%
\pgfpathlineto{\pgfqpoint{4.766022in}{1.025758in}}%
\pgfpathlineto{\pgfqpoint{4.786832in}{1.042046in}}%
\pgfpathlineto{\pgfqpoint{4.807643in}{1.066887in}}%
\pgfpathlineto{\pgfqpoint{4.828454in}{1.105862in}}%
\pgfpathlineto{\pgfqpoint{4.849265in}{1.161884in}}%
\pgfpathlineto{\pgfqpoint{4.870076in}{1.233763in}}%
\pgfpathlineto{\pgfqpoint{4.890887in}{1.316811in}}%
\pgfpathlineto{\pgfqpoint{4.911698in}{1.404176in}}%
\pgfpathlineto{\pgfqpoint{4.932509in}{1.489632in}}%
\pgfpathlineto{\pgfqpoint{4.953320in}{1.569603in}}%
\pgfpathlineto{\pgfqpoint{4.974131in}{1.642323in}}%
\pgfpathlineto{\pgfqpoint{4.994942in}{1.705041in}}%
\pgfpathlineto{\pgfqpoint{5.015753in}{1.753247in}}%
\pgfpathlineto{\pgfqpoint{5.036564in}{1.783440in}}%
\pgfpathlineto{\pgfqpoint{5.057375in}{1.795987in}}%
\pgfpathlineto{\pgfqpoint{5.078185in}{1.795433in}}%
\pgfpathlineto{\pgfqpoint{5.098996in}{1.788213in}}%
\pgfpathlineto{\pgfqpoint{5.119807in}{1.779780in}}%
\pgfpathlineto{\pgfqpoint{5.140618in}{1.773294in}}%
\pgfpathlineto{\pgfqpoint{5.161429in}{1.769852in}}%
\pgfpathlineto{\pgfqpoint{5.182240in}{1.768887in}}%
\pgfpathlineto{\pgfqpoint{5.203051in}{1.769107in}}%
\pgfpathlineto{\pgfqpoint{5.223862in}{1.769051in}}%
\pgfpathlineto{\pgfqpoint{5.244673in}{1.767343in}}%
\pgfpathlineto{\pgfqpoint{5.265484in}{1.763368in}}%
\pgfpathlineto{\pgfqpoint{5.286295in}{1.757793in}}%
\pgfpathlineto{\pgfqpoint{5.307106in}{1.752668in}}%
\pgfpathlineto{\pgfqpoint{5.327917in}{1.751632in}}%
\pgfpathlineto{\pgfqpoint{5.348728in}{1.760033in}}%
\pgfpathlineto{\pgfqpoint{5.369538in}{1.784554in}}%
\pgfpathlineto{\pgfqpoint{5.390349in}{1.831250in}}%
\pgfpathlineto{\pgfqpoint{5.411160in}{1.901133in}}%
\pgfpathlineto{\pgfqpoint{5.431971in}{1.984908in}}%
\pgfpathlineto{\pgfqpoint{5.452782in}{2.063681in}}%
\pgfpathlineto{\pgfqpoint{5.473593in}{2.120021in}}%
\pgfpathlineto{\pgfqpoint{5.494404in}{2.147698in}}%
\pgfpathlineto{\pgfqpoint{5.515215in}{2.150819in}}%
\pgfpathlineto{\pgfqpoint{5.536026in}{2.138922in}}%
\pgfpathlineto{\pgfqpoint{5.556837in}{2.122422in}}%
\pgfpathlineto{\pgfqpoint{5.577648in}{2.107247in}}%
\pgfpathlineto{\pgfqpoint{5.598459in}{2.090789in}}%
\pgfpathlineto{\pgfqpoint{5.619270in}{2.065040in}}%
\pgfpathlineto{\pgfqpoint{5.640081in}{2.022793in}}%
\pgfpathlineto{\pgfqpoint{5.660892in}{1.962974in}}%
\pgfpathlineto{\pgfqpoint{5.681702in}{1.891368in}}%
\pgfpathlineto{\pgfqpoint{5.702513in}{1.814330in}}%
\pgfpathlineto{\pgfqpoint{5.723324in}{1.734654in}}%
\pgfpathlineto{\pgfqpoint{5.744135in}{1.652087in}}%
\pgfpathlineto{\pgfqpoint{5.764946in}{1.567110in}}%
\pgfusepath{stroke}%
\end{pgfscope}%
\begin{pgfscope}%
\pgfpathrectangle{\pgfqpoint{3.846209in}{0.574768in}}{\pgfqpoint{2.380097in}{1.667836in}}%
\pgfusepath{clip}%
\pgfsetroundcap%
\pgfsetroundjoin%
\pgfsetlinewidth{1.505625pt}%
\definecolor{currentstroke}{rgb}{0.768627,0.305882,0.321569}%
\pgfsetstrokecolor{currentstroke}%
\pgfsetdash{}{0pt}%
\pgfpathmoveto{\pgfqpoint{3.954395in}{1.751339in}}%
\pgfpathlineto{\pgfqpoint{3.978277in}{1.816065in}}%
\pgfpathlineto{\pgfqpoint{4.002158in}{1.892784in}}%
\pgfpathlineto{\pgfqpoint{4.026039in}{1.978703in}}%
\pgfpathlineto{\pgfqpoint{4.049921in}{2.057831in}}%
\pgfpathlineto{\pgfqpoint{4.073802in}{2.114953in}}%
\pgfpathlineto{\pgfqpoint{4.097684in}{2.144771in}}%
\pgfpathlineto{\pgfqpoint{4.121565in}{2.150819in}}%
\pgfpathlineto{\pgfqpoint{4.145446in}{2.140172in}}%
\pgfpathlineto{\pgfqpoint{4.169328in}{2.118281in}}%
\pgfpathlineto{\pgfqpoint{4.193209in}{2.083717in}}%
\pgfpathlineto{\pgfqpoint{4.217091in}{2.026657in}}%
\pgfpathlineto{\pgfqpoint{4.240972in}{1.937545in}}%
\pgfpathlineto{\pgfqpoint{4.264853in}{1.817646in}}%
\pgfpathlineto{\pgfqpoint{4.288735in}{1.680281in}}%
\pgfpathlineto{\pgfqpoint{4.312616in}{1.541374in}}%
\pgfpathlineto{\pgfqpoint{4.336498in}{1.412401in}}%
\pgfpathlineto{\pgfqpoint{4.360379in}{1.297313in}}%
\pgfpathlineto{\pgfqpoint{4.384260in}{1.194330in}}%
\pgfpathlineto{\pgfqpoint{4.408142in}{1.100160in}}%
\pgfpathlineto{\pgfqpoint{4.432023in}{1.013511in}}%
\pgfpathlineto{\pgfqpoint{4.455905in}{0.933009in}}%
\pgfpathlineto{\pgfqpoint{4.479786in}{0.856656in}}%
\pgfpathlineto{\pgfqpoint{4.503667in}{0.788150in}}%
\pgfpathlineto{\pgfqpoint{4.527549in}{0.732963in}}%
\pgfpathlineto{\pgfqpoint{4.551430in}{0.694093in}}%
\pgfpathlineto{\pgfqpoint{4.575312in}{0.670160in}}%
\pgfpathlineto{\pgfqpoint{4.599193in}{0.657316in}}%
\pgfpathlineto{\pgfqpoint{4.623074in}{0.651710in}}%
\pgfpathlineto{\pgfqpoint{4.646956in}{0.650579in}}%
\pgfpathlineto{\pgfqpoint{4.670837in}{0.653683in}}%
\pgfpathlineto{\pgfqpoint{4.694719in}{0.664831in}}%
\pgfpathlineto{\pgfqpoint{4.718600in}{0.692443in}}%
\pgfpathlineto{\pgfqpoint{4.742481in}{0.747091in}}%
\pgfpathlineto{\pgfqpoint{4.766363in}{0.835424in}}%
\pgfpathlineto{\pgfqpoint{4.790244in}{0.953978in}}%
\pgfpathlineto{\pgfqpoint{4.814125in}{1.089041in}}%
\pgfpathlineto{\pgfqpoint{4.838007in}{1.224657in}}%
\pgfpathlineto{\pgfqpoint{4.861888in}{1.349814in}}%
\pgfpathlineto{\pgfqpoint{4.885770in}{1.458323in}}%
\pgfpathlineto{\pgfqpoint{4.909651in}{1.545940in}}%
\pgfpathlineto{\pgfqpoint{4.933532in}{1.610481in}}%
\pgfpathlineto{\pgfqpoint{4.957414in}{1.652946in}}%
\pgfpathlineto{\pgfqpoint{4.981295in}{1.678777in}}%
\pgfpathlineto{\pgfqpoint{5.005177in}{1.694856in}}%
\pgfpathlineto{\pgfqpoint{5.029058in}{1.706145in}}%
\pgfpathlineto{\pgfqpoint{5.052939in}{1.715518in}}%
\pgfpathlineto{\pgfqpoint{5.076821in}{1.724440in}}%
\pgfpathlineto{\pgfqpoint{5.100702in}{1.733292in}}%
\pgfpathlineto{\pgfqpoint{5.124584in}{1.741661in}}%
\pgfpathlineto{\pgfqpoint{5.148465in}{1.748977in}}%
\pgfpathlineto{\pgfqpoint{5.172346in}{1.754874in}}%
\pgfpathlineto{\pgfqpoint{5.196228in}{1.761099in}}%
\pgfpathlineto{\pgfqpoint{5.220109in}{1.774796in}}%
\pgfpathlineto{\pgfqpoint{5.243991in}{1.808844in}}%
\pgfpathlineto{\pgfqpoint{5.267872in}{1.873283in}}%
\pgfpathlineto{\pgfqpoint{5.291753in}{1.962400in}}%
\pgfpathlineto{\pgfqpoint{5.315635in}{2.053643in}}%
\pgfpathlineto{\pgfqpoint{5.339516in}{2.120904in}}%
\pgfpathlineto{\pgfqpoint{5.363398in}{2.152153in}}%
\pgfpathlineto{\pgfqpoint{5.387279in}{2.150819in}}%
\pgfpathlineto{\pgfqpoint{5.411160in}{2.126426in}}%
\pgfpathlineto{\pgfqpoint{5.435042in}{2.087207in}}%
\pgfpathlineto{\pgfqpoint{5.458923in}{2.034793in}}%
\pgfpathlineto{\pgfqpoint{5.482805in}{1.960818in}}%
\pgfpathlineto{\pgfqpoint{5.506686in}{1.854052in}}%
\pgfpathlineto{\pgfqpoint{5.530567in}{1.713635in}}%
\pgfpathlineto{\pgfqpoint{5.554449in}{1.554361in}}%
\pgfpathlineto{\pgfqpoint{5.578330in}{1.396856in}}%
\pgfpathlineto{\pgfqpoint{5.602212in}{1.252173in}}%
\pgfpathlineto{\pgfqpoint{5.626093in}{1.118177in}}%
\pgfusepath{stroke}%
\end{pgfscope}%
\begin{pgfscope}%
\pgfpathrectangle{\pgfqpoint{3.846209in}{0.574768in}}{\pgfqpoint{2.380097in}{1.667836in}}%
\pgfusepath{clip}%
\pgfsetroundcap%
\pgfsetroundjoin%
\pgfsetlinewidth{1.505625pt}%
\definecolor{currentstroke}{rgb}{0.505882,0.447059,0.701961}%
\pgfsetstrokecolor{currentstroke}%
\pgfsetdash{}{0pt}%
\pgfpathmoveto{\pgfqpoint{3.954395in}{1.605250in}}%
\pgfpathlineto{\pgfqpoint{3.975818in}{1.613793in}}%
\pgfpathlineto{\pgfqpoint{3.997241in}{1.623610in}}%
\pgfpathlineto{\pgfqpoint{4.018664in}{1.633920in}}%
\pgfpathlineto{\pgfqpoint{4.040087in}{1.644862in}}%
\pgfpathlineto{\pgfqpoint{4.061510in}{1.662646in}}%
\pgfpathlineto{\pgfqpoint{4.082933in}{1.698971in}}%
\pgfpathlineto{\pgfqpoint{4.104356in}{1.762407in}}%
\pgfpathlineto{\pgfqpoint{4.125779in}{1.849930in}}%
\pgfpathlineto{\pgfqpoint{4.147202in}{1.946438in}}%
\pgfpathlineto{\pgfqpoint{4.168625in}{2.033542in}}%
\pgfpathlineto{\pgfqpoint{4.190048in}{2.099294in}}%
\pgfpathlineto{\pgfqpoint{4.211471in}{2.140108in}}%
\pgfpathlineto{\pgfqpoint{4.232894in}{2.156472in}}%
\pgfpathlineto{\pgfqpoint{4.254317in}{2.150819in}}%
\pgfpathlineto{\pgfqpoint{4.275741in}{2.128653in}}%
\pgfpathlineto{\pgfqpoint{4.297164in}{2.098171in}}%
\pgfpathlineto{\pgfqpoint{4.318587in}{2.066407in}}%
\pgfpathlineto{\pgfqpoint{4.340010in}{2.034502in}}%
\pgfpathlineto{\pgfqpoint{4.361433in}{1.997034in}}%
\pgfpathlineto{\pgfqpoint{4.382856in}{1.946853in}}%
\pgfpathlineto{\pgfqpoint{4.404279in}{1.881941in}}%
\pgfpathlineto{\pgfqpoint{4.425702in}{1.807475in}}%
\pgfpathlineto{\pgfqpoint{4.447125in}{1.730326in}}%
\pgfpathlineto{\pgfqpoint{4.468548in}{1.654024in}}%
\pgfpathlineto{\pgfqpoint{4.489971in}{1.579428in}}%
\pgfpathlineto{\pgfqpoint{4.511394in}{1.507295in}}%
\pgfpathlineto{\pgfqpoint{4.532817in}{1.438525in}}%
\pgfpathlineto{\pgfqpoint{4.554240in}{1.374056in}}%
\pgfpathlineto{\pgfqpoint{4.575663in}{1.314062in}}%
\pgfpathlineto{\pgfqpoint{4.597086in}{1.257300in}}%
\pgfpathlineto{\pgfqpoint{4.618509in}{1.202266in}}%
\pgfpathlineto{\pgfqpoint{4.639932in}{1.148336in}}%
\pgfpathlineto{\pgfqpoint{4.661355in}{1.095987in}}%
\pgfpathlineto{\pgfqpoint{4.682778in}{1.045292in}}%
\pgfpathlineto{\pgfqpoint{4.704201in}{0.995698in}}%
\pgfpathlineto{\pgfqpoint{4.725624in}{0.946898in}}%
\pgfpathlineto{\pgfqpoint{4.747047in}{0.899997in}}%
\pgfpathlineto{\pgfqpoint{4.768470in}{0.857560in}}%
\pgfpathlineto{\pgfqpoint{4.789893in}{0.822163in}}%
\pgfpathlineto{\pgfqpoint{4.811316in}{0.795563in}}%
\pgfpathlineto{\pgfqpoint{4.832739in}{0.778579in}}%
\pgfpathlineto{\pgfqpoint{4.854162in}{0.770722in}}%
\pgfpathlineto{\pgfqpoint{4.875585in}{0.770013in}}%
\pgfpathlineto{\pgfqpoint{4.897008in}{0.774121in}}%
\pgfpathlineto{\pgfqpoint{4.918431in}{0.781928in}}%
\pgfpathlineto{\pgfqpoint{4.939854in}{0.793914in}}%
\pgfpathlineto{\pgfqpoint{4.961277in}{0.811201in}}%
\pgfpathlineto{\pgfqpoint{4.982700in}{0.835646in}}%
\pgfpathlineto{\pgfqpoint{5.004123in}{0.870816in}}%
\pgfpathlineto{\pgfqpoint{5.025546in}{0.920364in}}%
\pgfpathlineto{\pgfqpoint{5.046969in}{0.984183in}}%
\pgfpathlineto{\pgfqpoint{5.068392in}{1.058038in}}%
\pgfpathlineto{\pgfqpoint{5.089815in}{1.135385in}}%
\pgfpathlineto{\pgfqpoint{5.111238in}{1.209819in}}%
\pgfpathlineto{\pgfqpoint{5.132661in}{1.277920in}}%
\pgfpathlineto{\pgfqpoint{5.154084in}{1.339550in}}%
\pgfpathlineto{\pgfqpoint{5.175507in}{1.395850in}}%
\pgfpathlineto{\pgfqpoint{5.196930in}{1.447027in}}%
\pgfpathlineto{\pgfqpoint{5.218353in}{1.490810in}}%
\pgfpathlineto{\pgfqpoint{5.239776in}{1.524273in}}%
\pgfpathlineto{\pgfqpoint{5.261199in}{1.546489in}}%
\pgfpathlineto{\pgfqpoint{5.282622in}{1.559374in}}%
\pgfpathlineto{\pgfqpoint{5.304045in}{1.566447in}}%
\pgfpathlineto{\pgfqpoint{5.325468in}{1.570665in}}%
\pgfpathlineto{\pgfqpoint{5.346891in}{1.573482in}}%
\pgfpathlineto{\pgfqpoint{5.368314in}{1.576054in}}%
\pgfpathlineto{\pgfqpoint{5.389737in}{1.579936in}}%
\pgfpathlineto{\pgfqpoint{5.411160in}{1.586527in}}%
\pgfpathlineto{\pgfqpoint{5.432583in}{1.595922in}}%
\pgfpathlineto{\pgfqpoint{5.454006in}{1.606835in}}%
\pgfpathlineto{\pgfqpoint{5.475429in}{1.617723in}}%
\pgfpathlineto{\pgfqpoint{5.496852in}{1.627634in}}%
\pgfpathlineto{\pgfqpoint{5.518275in}{1.636225in}}%
\pgfpathlineto{\pgfqpoint{5.539698in}{1.643399in}}%
\pgfpathlineto{\pgfqpoint{5.561121in}{1.649574in}}%
\pgfpathlineto{\pgfqpoint{5.582544in}{1.656472in}}%
\pgfpathlineto{\pgfqpoint{5.603968in}{1.668679in}}%
\pgfpathlineto{\pgfqpoint{5.625391in}{1.694582in}}%
\pgfpathlineto{\pgfqpoint{5.646814in}{1.743315in}}%
\pgfpathlineto{\pgfqpoint{5.668237in}{1.818415in}}%
\pgfpathlineto{\pgfqpoint{5.689660in}{1.912338in}}%
\pgfpathlineto{\pgfqpoint{5.711083in}{2.006801in}}%
\pgfpathlineto{\pgfqpoint{5.732506in}{2.083900in}}%
\pgfpathlineto{\pgfqpoint{5.753929in}{2.135907in}}%
\pgfpathlineto{\pgfqpoint{5.775352in}{2.162942in}}%
\pgfpathlineto{\pgfqpoint{5.796775in}{2.166793in}}%
\pgfpathlineto{\pgfqpoint{5.818198in}{2.150819in}}%
\pgfpathlineto{\pgfqpoint{5.839621in}{2.123548in}}%
\pgfpathlineto{\pgfqpoint{5.861044in}{2.095483in}}%
\pgfpathlineto{\pgfqpoint{5.882467in}{2.071377in}}%
\pgfpathlineto{\pgfqpoint{5.903890in}{2.046104in}}%
\pgfpathlineto{\pgfqpoint{5.925313in}{2.008234in}}%
\pgfpathlineto{\pgfqpoint{5.946736in}{1.949686in}}%
\pgfpathlineto{\pgfqpoint{5.968159in}{1.873010in}}%
\pgfpathlineto{\pgfqpoint{5.989582in}{1.789400in}}%
\pgfpathlineto{\pgfqpoint{6.011005in}{1.709159in}}%
\pgfpathlineto{\pgfqpoint{6.032428in}{1.634675in}}%
\pgfpathlineto{\pgfqpoint{6.053851in}{1.563251in}}%
\pgfpathlineto{\pgfqpoint{6.075274in}{1.492273in}}%
\pgfpathlineto{\pgfqpoint{6.096697in}{1.421634in}}%
\pgfpathlineto{\pgfqpoint{6.118120in}{1.351614in}}%
\pgfusepath{stroke}%
\end{pgfscope}%
\begin{pgfscope}%
\pgfsetrectcap%
\pgfsetmiterjoin%
\pgfsetlinewidth{1.254687pt}%
\definecolor{currentstroke}{rgb}{1.000000,1.000000,1.000000}%
\pgfsetstrokecolor{currentstroke}%
\pgfsetdash{}{0pt}%
\pgfpathmoveto{\pgfqpoint{3.846209in}{0.574768in}}%
\pgfpathlineto{\pgfqpoint{3.846209in}{2.242604in}}%
\pgfusepath{stroke}%
\end{pgfscope}%
\begin{pgfscope}%
\pgfsetrectcap%
\pgfsetmiterjoin%
\pgfsetlinewidth{1.254687pt}%
\definecolor{currentstroke}{rgb}{1.000000,1.000000,1.000000}%
\pgfsetstrokecolor{currentstroke}%
\pgfsetdash{}{0pt}%
\pgfpathmoveto{\pgfqpoint{6.226306in}{0.574768in}}%
\pgfpathlineto{\pgfqpoint{6.226306in}{2.242604in}}%
\pgfusepath{stroke}%
\end{pgfscope}%
\begin{pgfscope}%
\pgfsetrectcap%
\pgfsetmiterjoin%
\pgfsetlinewidth{1.254687pt}%
\definecolor{currentstroke}{rgb}{1.000000,1.000000,1.000000}%
\pgfsetstrokecolor{currentstroke}%
\pgfsetdash{}{0pt}%
\pgfpathmoveto{\pgfqpoint{3.846209in}{0.574768in}}%
\pgfpathlineto{\pgfqpoint{6.226306in}{0.574768in}}%
\pgfusepath{stroke}%
\end{pgfscope}%
\begin{pgfscope}%
\pgfsetrectcap%
\pgfsetmiterjoin%
\pgfsetlinewidth{1.254687pt}%
\definecolor{currentstroke}{rgb}{1.000000,1.000000,1.000000}%
\pgfsetstrokecolor{currentstroke}%
\pgfsetdash{}{0pt}%
\pgfpathmoveto{\pgfqpoint{3.846209in}{2.242604in}}%
\pgfpathlineto{\pgfqpoint{6.226306in}{2.242604in}}%
\pgfusepath{stroke}%
\end{pgfscope}%
\begin{pgfscope}%
\definecolor{textcolor}{rgb}{0.150000,0.150000,0.150000}%
\pgfsetstrokecolor{textcolor}%
\pgfsetfillcolor{textcolor}%
\pgftext[x=5.036258in,y=2.325938in,,base]{\color{textcolor}\sffamily\fontsize{11.000000}{13.200000}\selectfont (b) Cluster 2 members}%
\end{pgfscope}%
\end{pgfpicture}%
\makeatother%
\endgroup%

    \caption{Here the curves of five random cluster members assigned by the \textit{gls/2CH/regular/centroid/2} model.
             Each plot depicts the \acrshort{2ch} \acrshort{gls} curves for five random cluster members from the \textit{gls/2CH/regular/centroid/2} model. 
             (a) and (b) contain members from cluster 1 and 2 respectively. Only five curves are included to avoid making the plot too chaotic.}
    \label{fig:tsc_hf_best_meth_5_samples}
\end{figure}

\newpage

The majority of \acrshort{ari} scores are close to zero, but 17 models evaluated at different numbers of cluster centers are able to acheive an \acrshort{ari} score above $0.20$. As with \acrshort{dor}, the general trends for models with a high \acrshort{ari} score is that they use data from a single view, use scaling or no preprocessing at all. From table \ref{tab:tsc_hf_ari} one can see that the top five models only use the \acrshort{gls} curve from the \acrshort{2ch} view. In addition, one can also see that the two models with the highest \acrshort{ari} ($0.25$) are the clustering models evaluated at two cluster centers that perform best in terms of \acrshort{dor} as well. This means that there most likely are no models evaluated at a number of cluster centers higher than two that will perform better than \textit{gls/2CH/regular/centroid/2}, or \textit{gls/2CH/scaled/centroid/2}. Figure \ref{fig:tsc_hf_best_meth_5_samples} shows the \acrshort{2ch} \acrshort{gls} curves of five random cluster members from the \textit{gls/2CH/regular/centroid/2} model. Although one caANNot make any conclusive statements about what the general similarities between cluster members are, from the plots in figure \ref{fig:tsc_hf_best_meth_5_samples} it seems like the curves of cluster 2 are smooth, while the curves of cluster 1 are more irregular in shape, which makes sense as this clustering algorithm uses a shape-based distance measure. Since \textit{gls/2CH/regular/centroid/2} is one of two models to acheive the highest \acrshort{dor} ($11.72$), accuracy ($0.76$), and \acrshort{ari} ($0.25$) it is chosen as the best of the\acrshort{tsc} models at identifying heart failure among patients. \textit{gls/2CH/regular/centroid/2} is chosen over \textit{gls/2CH/scaled/centroid/2} because it does not require preprocessing.
\bigskip

\newpage

\subsection{Peak-value Clustering}

\begin{figure}[htb]
    \centering
    %% Creator: Matplotlib, PGF backend
%%
%% To include the figure in your LaTeX document, write
%%   \input{<filename>.pgf}
%%
%% Make sure the required packages are loaded in your preamble
%%   \usepackage{pgf}
%%
%% Figures using additional raster images can only be included by \input if
%% they are in the same directory as the main LaTeX file. For loading figures
%% from other directories you can use the `import` package
%%   \usepackage{import}
%% and then include the figures with
%%   \import{<path to file>}{<filename>.pgf}
%%
%% Matplotlib used the following preamble
%%
\begingroup%
\makeatletter%
\begin{pgfpicture}%
\pgfpathrectangle{\pgfpointorigin}{\pgfqpoint{6.360543in}{2.540000in}}%
\pgfusepath{use as bounding box, clip}%
\begin{pgfscope}%
\pgfsetbuttcap%
\pgfsetmiterjoin%
\definecolor{currentfill}{rgb}{1.000000,1.000000,1.000000}%
\pgfsetfillcolor{currentfill}%
\pgfsetlinewidth{0.000000pt}%
\definecolor{currentstroke}{rgb}{1.000000,1.000000,1.000000}%
\pgfsetstrokecolor{currentstroke}%
\pgfsetdash{}{0pt}%
\pgfpathmoveto{\pgfqpoint{0.000000in}{0.000000in}}%
\pgfpathlineto{\pgfqpoint{6.360543in}{0.000000in}}%
\pgfpathlineto{\pgfqpoint{6.360543in}{2.540000in}}%
\pgfpathlineto{\pgfqpoint{0.000000in}{2.540000in}}%
\pgfpathclose%
\pgfusepath{fill}%
\end{pgfscope}%
\begin{pgfscope}%
\pgfsetbuttcap%
\pgfsetmiterjoin%
\definecolor{currentfill}{rgb}{0.917647,0.917647,0.949020}%
\pgfsetfillcolor{currentfill}%
\pgfsetlinewidth{0.000000pt}%
\definecolor{currentstroke}{rgb}{0.000000,0.000000,0.000000}%
\pgfsetstrokecolor{currentstroke}%
\pgfsetstrokeopacity{0.000000}%
\pgfsetdash{}{0pt}%
\pgfpathmoveto{\pgfqpoint{0.498727in}{0.557870in}}%
\pgfpathlineto{\pgfqpoint{3.020856in}{0.557870in}}%
\pgfpathlineto{\pgfqpoint{3.020856in}{2.242604in}}%
\pgfpathlineto{\pgfqpoint{0.498727in}{2.242604in}}%
\pgfpathclose%
\pgfusepath{fill}%
\end{pgfscope}%
\begin{pgfscope}%
\pgfpathrectangle{\pgfqpoint{0.498727in}{0.557870in}}{\pgfqpoint{2.522130in}{1.684734in}}%
\pgfusepath{clip}%
\pgfsetroundcap%
\pgfsetroundjoin%
\pgfsetlinewidth{1.003750pt}%
\definecolor{currentstroke}{rgb}{1.000000,1.000000,1.000000}%
\pgfsetstrokecolor{currentstroke}%
\pgfsetdash{}{0pt}%
\pgfpathmoveto{\pgfqpoint{0.613369in}{0.557870in}}%
\pgfpathlineto{\pgfqpoint{0.613369in}{2.242604in}}%
\pgfusepath{stroke}%
\end{pgfscope}%
\begin{pgfscope}%
\definecolor{textcolor}{rgb}{0.150000,0.150000,0.150000}%
\pgfsetstrokecolor{textcolor}%
\pgfsetfillcolor{textcolor}%
\pgftext[x=0.613369in,y=0.425926in,,top]{\color{textcolor}\sffamily\fontsize{11.000000}{13.200000}\selectfont \(\displaystyle 0\)}%
\end{pgfscope}%
\begin{pgfscope}%
\pgfpathrectangle{\pgfqpoint{0.498727in}{0.557870in}}{\pgfqpoint{2.522130in}{1.684734in}}%
\pgfusepath{clip}%
\pgfsetroundcap%
\pgfsetroundjoin%
\pgfsetlinewidth{1.003750pt}%
\definecolor{currentstroke}{rgb}{1.000000,1.000000,1.000000}%
\pgfsetstrokecolor{currentstroke}%
\pgfsetdash{}{0pt}%
\pgfpathmoveto{\pgfqpoint{1.586809in}{0.557870in}}%
\pgfpathlineto{\pgfqpoint{1.586809in}{2.242604in}}%
\pgfusepath{stroke}%
\end{pgfscope}%
\begin{pgfscope}%
\definecolor{textcolor}{rgb}{0.150000,0.150000,0.150000}%
\pgfsetstrokecolor{textcolor}%
\pgfsetfillcolor{textcolor}%
\pgftext[x=1.586809in,y=0.425926in,,top]{\color{textcolor}\sffamily\fontsize{11.000000}{13.200000}\selectfont \(\displaystyle 5\)}%
\end{pgfscope}%
\begin{pgfscope}%
\pgfpathrectangle{\pgfqpoint{0.498727in}{0.557870in}}{\pgfqpoint{2.522130in}{1.684734in}}%
\pgfusepath{clip}%
\pgfsetroundcap%
\pgfsetroundjoin%
\pgfsetlinewidth{1.003750pt}%
\definecolor{currentstroke}{rgb}{1.000000,1.000000,1.000000}%
\pgfsetstrokecolor{currentstroke}%
\pgfsetdash{}{0pt}%
\pgfpathmoveto{\pgfqpoint{2.560248in}{0.557870in}}%
\pgfpathlineto{\pgfqpoint{2.560248in}{2.242604in}}%
\pgfusepath{stroke}%
\end{pgfscope}%
\begin{pgfscope}%
\definecolor{textcolor}{rgb}{0.150000,0.150000,0.150000}%
\pgfsetstrokecolor{textcolor}%
\pgfsetfillcolor{textcolor}%
\pgftext[x=2.560248in,y=0.425926in,,top]{\color{textcolor}\sffamily\fontsize{11.000000}{13.200000}\selectfont \(\displaystyle 10\)}%
\end{pgfscope}%
\begin{pgfscope}%
\definecolor{textcolor}{rgb}{0.150000,0.150000,0.150000}%
\pgfsetstrokecolor{textcolor}%
\pgfsetfillcolor{textcolor}%
\pgftext[x=1.759792in,y=0.235185in,,top]{\color{textcolor}\sffamily\fontsize{11.000000}{13.200000}\selectfont DOR}%
\end{pgfscope}%
\begin{pgfscope}%
\pgfpathrectangle{\pgfqpoint{0.498727in}{0.557870in}}{\pgfqpoint{2.522130in}{1.684734in}}%
\pgfusepath{clip}%
\pgfsetroundcap%
\pgfsetroundjoin%
\pgfsetlinewidth{1.003750pt}%
\definecolor{currentstroke}{rgb}{1.000000,1.000000,1.000000}%
\pgfsetstrokecolor{currentstroke}%
\pgfsetdash{}{0pt}%
\pgfpathmoveto{\pgfqpoint{0.498727in}{0.557870in}}%
\pgfpathlineto{\pgfqpoint{3.020856in}{0.557870in}}%
\pgfusepath{stroke}%
\end{pgfscope}%
\begin{pgfscope}%
\definecolor{textcolor}{rgb}{0.150000,0.150000,0.150000}%
\pgfsetstrokecolor{textcolor}%
\pgfsetfillcolor{textcolor}%
\pgftext[x=0.290741in,y=0.505064in,left,base]{\color{textcolor}\sffamily\fontsize{11.000000}{13.200000}\selectfont \(\displaystyle 0\)}%
\end{pgfscope}%
\begin{pgfscope}%
\pgfpathrectangle{\pgfqpoint{0.498727in}{0.557870in}}{\pgfqpoint{2.522130in}{1.684734in}}%
\pgfusepath{clip}%
\pgfsetroundcap%
\pgfsetroundjoin%
\pgfsetlinewidth{1.003750pt}%
\definecolor{currentstroke}{rgb}{1.000000,1.000000,1.000000}%
\pgfsetstrokecolor{currentstroke}%
\pgfsetdash{}{0pt}%
\pgfpathmoveto{\pgfqpoint{0.498727in}{1.016301in}}%
\pgfpathlineto{\pgfqpoint{3.020856in}{1.016301in}}%
\pgfusepath{stroke}%
\end{pgfscope}%
\begin{pgfscope}%
\definecolor{textcolor}{rgb}{0.150000,0.150000,0.150000}%
\pgfsetstrokecolor{textcolor}%
\pgfsetfillcolor{textcolor}%
\pgftext[x=0.290741in,y=0.963495in,left,base]{\color{textcolor}\sffamily\fontsize{11.000000}{13.200000}\selectfont \(\displaystyle 2\)}%
\end{pgfscope}%
\begin{pgfscope}%
\pgfpathrectangle{\pgfqpoint{0.498727in}{0.557870in}}{\pgfqpoint{2.522130in}{1.684734in}}%
\pgfusepath{clip}%
\pgfsetroundcap%
\pgfsetroundjoin%
\pgfsetlinewidth{1.003750pt}%
\definecolor{currentstroke}{rgb}{1.000000,1.000000,1.000000}%
\pgfsetstrokecolor{currentstroke}%
\pgfsetdash{}{0pt}%
\pgfpathmoveto{\pgfqpoint{0.498727in}{1.474732in}}%
\pgfpathlineto{\pgfqpoint{3.020856in}{1.474732in}}%
\pgfusepath{stroke}%
\end{pgfscope}%
\begin{pgfscope}%
\definecolor{textcolor}{rgb}{0.150000,0.150000,0.150000}%
\pgfsetstrokecolor{textcolor}%
\pgfsetfillcolor{textcolor}%
\pgftext[x=0.290741in,y=1.421926in,left,base]{\color{textcolor}\sffamily\fontsize{11.000000}{13.200000}\selectfont \(\displaystyle 4\)}%
\end{pgfscope}%
\begin{pgfscope}%
\pgfpathrectangle{\pgfqpoint{0.498727in}{0.557870in}}{\pgfqpoint{2.522130in}{1.684734in}}%
\pgfusepath{clip}%
\pgfsetroundcap%
\pgfsetroundjoin%
\pgfsetlinewidth{1.003750pt}%
\definecolor{currentstroke}{rgb}{1.000000,1.000000,1.000000}%
\pgfsetstrokecolor{currentstroke}%
\pgfsetdash{}{0pt}%
\pgfpathmoveto{\pgfqpoint{0.498727in}{1.933163in}}%
\pgfpathlineto{\pgfqpoint{3.020856in}{1.933163in}}%
\pgfusepath{stroke}%
\end{pgfscope}%
\begin{pgfscope}%
\definecolor{textcolor}{rgb}{0.150000,0.150000,0.150000}%
\pgfsetstrokecolor{textcolor}%
\pgfsetfillcolor{textcolor}%
\pgftext[x=0.290741in,y=1.880357in,left,base]{\color{textcolor}\sffamily\fontsize{11.000000}{13.200000}\selectfont \(\displaystyle 6\)}%
\end{pgfscope}%
\begin{pgfscope}%
\definecolor{textcolor}{rgb}{0.150000,0.150000,0.150000}%
\pgfsetstrokecolor{textcolor}%
\pgfsetfillcolor{textcolor}%
\pgftext[x=0.235185in,y=1.400237in,,bottom,rotate=90.000000]{\color{textcolor}\sffamily\fontsize{11.000000}{13.200000}\selectfont Occurance}%
\end{pgfscope}%
\begin{pgfscope}%
\pgfpathrectangle{\pgfqpoint{0.498727in}{0.557870in}}{\pgfqpoint{2.522130in}{1.684734in}}%
\pgfusepath{clip}%
\pgfsetbuttcap%
\pgfsetmiterjoin%
\definecolor{currentfill}{rgb}{0.298039,0.447059,0.690196}%
\pgfsetfillcolor{currentfill}%
\pgfsetfillopacity{0.400000}%
\pgfsetlinewidth{1.003750pt}%
\definecolor{currentstroke}{rgb}{1.000000,1.000000,1.000000}%
\pgfsetstrokecolor{currentstroke}%
\pgfsetstrokeopacity{0.400000}%
\pgfsetdash{}{0pt}%
\pgfpathmoveto{\pgfqpoint{0.613369in}{0.557870in}}%
\pgfpathlineto{\pgfqpoint{0.842654in}{0.557870in}}%
\pgfpathlineto{\pgfqpoint{0.842654in}{2.162379in}}%
\pgfpathlineto{\pgfqpoint{0.613369in}{2.162379in}}%
\pgfpathclose%
\pgfusepath{stroke,fill}%
\end{pgfscope}%
\begin{pgfscope}%
\pgfpathrectangle{\pgfqpoint{0.498727in}{0.557870in}}{\pgfqpoint{2.522130in}{1.684734in}}%
\pgfusepath{clip}%
\pgfsetbuttcap%
\pgfsetmiterjoin%
\definecolor{currentfill}{rgb}{0.298039,0.447059,0.690196}%
\pgfsetfillcolor{currentfill}%
\pgfsetfillopacity{0.400000}%
\pgfsetlinewidth{1.003750pt}%
\definecolor{currentstroke}{rgb}{1.000000,1.000000,1.000000}%
\pgfsetstrokecolor{currentstroke}%
\pgfsetstrokeopacity{0.400000}%
\pgfsetdash{}{0pt}%
\pgfpathmoveto{\pgfqpoint{0.842654in}{0.557870in}}%
\pgfpathlineto{\pgfqpoint{1.071938in}{0.557870in}}%
\pgfpathlineto{\pgfqpoint{1.071938in}{0.557870in}}%
\pgfpathlineto{\pgfqpoint{0.842654in}{0.557870in}}%
\pgfpathclose%
\pgfusepath{stroke,fill}%
\end{pgfscope}%
\begin{pgfscope}%
\pgfpathrectangle{\pgfqpoint{0.498727in}{0.557870in}}{\pgfqpoint{2.522130in}{1.684734in}}%
\pgfusepath{clip}%
\pgfsetbuttcap%
\pgfsetmiterjoin%
\definecolor{currentfill}{rgb}{0.298039,0.447059,0.690196}%
\pgfsetfillcolor{currentfill}%
\pgfsetfillopacity{0.400000}%
\pgfsetlinewidth{1.003750pt}%
\definecolor{currentstroke}{rgb}{1.000000,1.000000,1.000000}%
\pgfsetstrokecolor{currentstroke}%
\pgfsetstrokeopacity{0.400000}%
\pgfsetdash{}{0pt}%
\pgfpathmoveto{\pgfqpoint{1.071938in}{0.557870in}}%
\pgfpathlineto{\pgfqpoint{1.301223in}{0.557870in}}%
\pgfpathlineto{\pgfqpoint{1.301223in}{0.787086in}}%
\pgfpathlineto{\pgfqpoint{1.071938in}{0.787086in}}%
\pgfpathclose%
\pgfusepath{stroke,fill}%
\end{pgfscope}%
\begin{pgfscope}%
\pgfpathrectangle{\pgfqpoint{0.498727in}{0.557870in}}{\pgfqpoint{2.522130in}{1.684734in}}%
\pgfusepath{clip}%
\pgfsetbuttcap%
\pgfsetmiterjoin%
\definecolor{currentfill}{rgb}{0.298039,0.447059,0.690196}%
\pgfsetfillcolor{currentfill}%
\pgfsetfillopacity{0.400000}%
\pgfsetlinewidth{1.003750pt}%
\definecolor{currentstroke}{rgb}{1.000000,1.000000,1.000000}%
\pgfsetstrokecolor{currentstroke}%
\pgfsetstrokeopacity{0.400000}%
\pgfsetdash{}{0pt}%
\pgfpathmoveto{\pgfqpoint{1.301223in}{0.557870in}}%
\pgfpathlineto{\pgfqpoint{1.530507in}{0.557870in}}%
\pgfpathlineto{\pgfqpoint{1.530507in}{1.245517in}}%
\pgfpathlineto{\pgfqpoint{1.301223in}{1.245517in}}%
\pgfpathclose%
\pgfusepath{stroke,fill}%
\end{pgfscope}%
\begin{pgfscope}%
\pgfpathrectangle{\pgfqpoint{0.498727in}{0.557870in}}{\pgfqpoint{2.522130in}{1.684734in}}%
\pgfusepath{clip}%
\pgfsetbuttcap%
\pgfsetmiterjoin%
\definecolor{currentfill}{rgb}{0.298039,0.447059,0.690196}%
\pgfsetfillcolor{currentfill}%
\pgfsetfillopacity{0.400000}%
\pgfsetlinewidth{1.003750pt}%
\definecolor{currentstroke}{rgb}{1.000000,1.000000,1.000000}%
\pgfsetstrokecolor{currentstroke}%
\pgfsetstrokeopacity{0.400000}%
\pgfsetdash{}{0pt}%
\pgfpathmoveto{\pgfqpoint{1.530507in}{0.557870in}}%
\pgfpathlineto{\pgfqpoint{1.759792in}{0.557870in}}%
\pgfpathlineto{\pgfqpoint{1.759792in}{1.016301in}}%
\pgfpathlineto{\pgfqpoint{1.530507in}{1.016301in}}%
\pgfpathclose%
\pgfusepath{stroke,fill}%
\end{pgfscope}%
\begin{pgfscope}%
\pgfpathrectangle{\pgfqpoint{0.498727in}{0.557870in}}{\pgfqpoint{2.522130in}{1.684734in}}%
\pgfusepath{clip}%
\pgfsetbuttcap%
\pgfsetmiterjoin%
\definecolor{currentfill}{rgb}{0.298039,0.447059,0.690196}%
\pgfsetfillcolor{currentfill}%
\pgfsetfillopacity{0.400000}%
\pgfsetlinewidth{1.003750pt}%
\definecolor{currentstroke}{rgb}{1.000000,1.000000,1.000000}%
\pgfsetstrokecolor{currentstroke}%
\pgfsetstrokeopacity{0.400000}%
\pgfsetdash{}{0pt}%
\pgfpathmoveto{\pgfqpoint{1.759792in}{0.557870in}}%
\pgfpathlineto{\pgfqpoint{1.989076in}{0.557870in}}%
\pgfpathlineto{\pgfqpoint{1.989076in}{1.245517in}}%
\pgfpathlineto{\pgfqpoint{1.759792in}{1.245517in}}%
\pgfpathclose%
\pgfusepath{stroke,fill}%
\end{pgfscope}%
\begin{pgfscope}%
\pgfpathrectangle{\pgfqpoint{0.498727in}{0.557870in}}{\pgfqpoint{2.522130in}{1.684734in}}%
\pgfusepath{clip}%
\pgfsetbuttcap%
\pgfsetmiterjoin%
\definecolor{currentfill}{rgb}{0.298039,0.447059,0.690196}%
\pgfsetfillcolor{currentfill}%
\pgfsetfillopacity{0.400000}%
\pgfsetlinewidth{1.003750pt}%
\definecolor{currentstroke}{rgb}{1.000000,1.000000,1.000000}%
\pgfsetstrokecolor{currentstroke}%
\pgfsetstrokeopacity{0.400000}%
\pgfsetdash{}{0pt}%
\pgfpathmoveto{\pgfqpoint{1.989076in}{0.557870in}}%
\pgfpathlineto{\pgfqpoint{2.218361in}{0.557870in}}%
\pgfpathlineto{\pgfqpoint{2.218361in}{0.787086in}}%
\pgfpathlineto{\pgfqpoint{1.989076in}{0.787086in}}%
\pgfpathclose%
\pgfusepath{stroke,fill}%
\end{pgfscope}%
\begin{pgfscope}%
\pgfpathrectangle{\pgfqpoint{0.498727in}{0.557870in}}{\pgfqpoint{2.522130in}{1.684734in}}%
\pgfusepath{clip}%
\pgfsetbuttcap%
\pgfsetmiterjoin%
\definecolor{currentfill}{rgb}{0.298039,0.447059,0.690196}%
\pgfsetfillcolor{currentfill}%
\pgfsetfillopacity{0.400000}%
\pgfsetlinewidth{1.003750pt}%
\definecolor{currentstroke}{rgb}{1.000000,1.000000,1.000000}%
\pgfsetstrokecolor{currentstroke}%
\pgfsetstrokeopacity{0.400000}%
\pgfsetdash{}{0pt}%
\pgfpathmoveto{\pgfqpoint{2.218361in}{0.557870in}}%
\pgfpathlineto{\pgfqpoint{2.447645in}{0.557870in}}%
\pgfpathlineto{\pgfqpoint{2.447645in}{0.787086in}}%
\pgfpathlineto{\pgfqpoint{2.218361in}{0.787086in}}%
\pgfpathclose%
\pgfusepath{stroke,fill}%
\end{pgfscope}%
\begin{pgfscope}%
\pgfpathrectangle{\pgfqpoint{0.498727in}{0.557870in}}{\pgfqpoint{2.522130in}{1.684734in}}%
\pgfusepath{clip}%
\pgfsetbuttcap%
\pgfsetmiterjoin%
\definecolor{currentfill}{rgb}{0.298039,0.447059,0.690196}%
\pgfsetfillcolor{currentfill}%
\pgfsetfillopacity{0.400000}%
\pgfsetlinewidth{1.003750pt}%
\definecolor{currentstroke}{rgb}{1.000000,1.000000,1.000000}%
\pgfsetstrokecolor{currentstroke}%
\pgfsetstrokeopacity{0.400000}%
\pgfsetdash{}{0pt}%
\pgfpathmoveto{\pgfqpoint{2.447645in}{0.557870in}}%
\pgfpathlineto{\pgfqpoint{2.676930in}{0.557870in}}%
\pgfpathlineto{\pgfqpoint{2.676930in}{0.557870in}}%
\pgfpathlineto{\pgfqpoint{2.447645in}{0.557870in}}%
\pgfpathclose%
\pgfusepath{stroke,fill}%
\end{pgfscope}%
\begin{pgfscope}%
\pgfpathrectangle{\pgfqpoint{0.498727in}{0.557870in}}{\pgfqpoint{2.522130in}{1.684734in}}%
\pgfusepath{clip}%
\pgfsetbuttcap%
\pgfsetmiterjoin%
\definecolor{currentfill}{rgb}{0.298039,0.447059,0.690196}%
\pgfsetfillcolor{currentfill}%
\pgfsetfillopacity{0.400000}%
\pgfsetlinewidth{1.003750pt}%
\definecolor{currentstroke}{rgb}{1.000000,1.000000,1.000000}%
\pgfsetstrokecolor{currentstroke}%
\pgfsetstrokeopacity{0.400000}%
\pgfsetdash{}{0pt}%
\pgfpathmoveto{\pgfqpoint{2.676930in}{0.557870in}}%
\pgfpathlineto{\pgfqpoint{2.906214in}{0.557870in}}%
\pgfpathlineto{\pgfqpoint{2.906214in}{1.245517in}}%
\pgfpathlineto{\pgfqpoint{2.676930in}{1.245517in}}%
\pgfpathclose%
\pgfusepath{stroke,fill}%
\end{pgfscope}%
\begin{pgfscope}%
\pgfsetrectcap%
\pgfsetmiterjoin%
\pgfsetlinewidth{1.254687pt}%
\definecolor{currentstroke}{rgb}{1.000000,1.000000,1.000000}%
\pgfsetstrokecolor{currentstroke}%
\pgfsetdash{}{0pt}%
\pgfpathmoveto{\pgfqpoint{0.498727in}{0.557870in}}%
\pgfpathlineto{\pgfqpoint{0.498727in}{2.242604in}}%
\pgfusepath{stroke}%
\end{pgfscope}%
\begin{pgfscope}%
\pgfsetrectcap%
\pgfsetmiterjoin%
\pgfsetlinewidth{1.254687pt}%
\definecolor{currentstroke}{rgb}{1.000000,1.000000,1.000000}%
\pgfsetstrokecolor{currentstroke}%
\pgfsetdash{}{0pt}%
\pgfpathmoveto{\pgfqpoint{3.020856in}{0.557870in}}%
\pgfpathlineto{\pgfqpoint{3.020856in}{2.242604in}}%
\pgfusepath{stroke}%
\end{pgfscope}%
\begin{pgfscope}%
\pgfsetrectcap%
\pgfsetmiterjoin%
\pgfsetlinewidth{1.254687pt}%
\definecolor{currentstroke}{rgb}{1.000000,1.000000,1.000000}%
\pgfsetstrokecolor{currentstroke}%
\pgfsetdash{}{0pt}%
\pgfpathmoveto{\pgfqpoint{0.498727in}{0.557870in}}%
\pgfpathlineto{\pgfqpoint{3.020856in}{0.557870in}}%
\pgfusepath{stroke}%
\end{pgfscope}%
\begin{pgfscope}%
\pgfsetrectcap%
\pgfsetmiterjoin%
\pgfsetlinewidth{1.254687pt}%
\definecolor{currentstroke}{rgb}{1.000000,1.000000,1.000000}%
\pgfsetstrokecolor{currentstroke}%
\pgfsetdash{}{0pt}%
\pgfpathmoveto{\pgfqpoint{0.498727in}{2.242604in}}%
\pgfpathlineto{\pgfqpoint{3.020856in}{2.242604in}}%
\pgfusepath{stroke}%
\end{pgfscope}%
\begin{pgfscope}%
\definecolor{textcolor}{rgb}{0.150000,0.150000,0.150000}%
\pgfsetstrokecolor{textcolor}%
\pgfsetfillcolor{textcolor}%
\pgftext[x=1.759792in,y=2.325938in,,base]{\color{textcolor}\sffamily\fontsize{11.000000}{13.200000}\selectfont (a)}%
\end{pgfscope}%
\begin{pgfscope}%
\pgfsetbuttcap%
\pgfsetmiterjoin%
\definecolor{currentfill}{rgb}{0.917647,0.917647,0.949020}%
\pgfsetfillcolor{currentfill}%
\pgfsetlinewidth{0.000000pt}%
\definecolor{currentstroke}{rgb}{0.000000,0.000000,0.000000}%
\pgfsetstrokecolor{currentstroke}%
\pgfsetstrokeopacity{0.000000}%
\pgfsetdash{}{0pt}%
\pgfpathmoveto{\pgfqpoint{3.717870in}{0.557870in}}%
\pgfpathlineto{\pgfqpoint{6.240000in}{0.557870in}}%
\pgfpathlineto{\pgfqpoint{6.240000in}{2.242604in}}%
\pgfpathlineto{\pgfqpoint{3.717870in}{2.242604in}}%
\pgfpathclose%
\pgfusepath{fill}%
\end{pgfscope}%
\begin{pgfscope}%
\pgfpathrectangle{\pgfqpoint{3.717870in}{0.557870in}}{\pgfqpoint{2.522130in}{1.684734in}}%
\pgfusepath{clip}%
\pgfsetroundcap%
\pgfsetroundjoin%
\pgfsetlinewidth{1.003750pt}%
\definecolor{currentstroke}{rgb}{1.000000,1.000000,1.000000}%
\pgfsetstrokecolor{currentstroke}%
\pgfsetdash{}{0pt}%
\pgfpathmoveto{\pgfqpoint{3.832513in}{0.557870in}}%
\pgfpathlineto{\pgfqpoint{3.832513in}{2.242604in}}%
\pgfusepath{stroke}%
\end{pgfscope}%
\begin{pgfscope}%
\definecolor{textcolor}{rgb}{0.150000,0.150000,0.150000}%
\pgfsetstrokecolor{textcolor}%
\pgfsetfillcolor{textcolor}%
\pgftext[x=3.832513in,y=0.425926in,,top]{\color{textcolor}\sffamily\fontsize{11.000000}{13.200000}\selectfont \(\displaystyle 0.00\)}%
\end{pgfscope}%
\begin{pgfscope}%
\pgfpathrectangle{\pgfqpoint{3.717870in}{0.557870in}}{\pgfqpoint{2.522130in}{1.684734in}}%
\pgfusepath{clip}%
\pgfsetroundcap%
\pgfsetroundjoin%
\pgfsetlinewidth{1.003750pt}%
\definecolor{currentstroke}{rgb}{1.000000,1.000000,1.000000}%
\pgfsetstrokecolor{currentstroke}%
\pgfsetdash{}{0pt}%
\pgfpathmoveto{\pgfqpoint{4.405724in}{0.557870in}}%
\pgfpathlineto{\pgfqpoint{4.405724in}{2.242604in}}%
\pgfusepath{stroke}%
\end{pgfscope}%
\begin{pgfscope}%
\definecolor{textcolor}{rgb}{0.150000,0.150000,0.150000}%
\pgfsetstrokecolor{textcolor}%
\pgfsetfillcolor{textcolor}%
\pgftext[x=4.405724in,y=0.425926in,,top]{\color{textcolor}\sffamily\fontsize{11.000000}{13.200000}\selectfont \(\displaystyle 0.25\)}%
\end{pgfscope}%
\begin{pgfscope}%
\pgfpathrectangle{\pgfqpoint{3.717870in}{0.557870in}}{\pgfqpoint{2.522130in}{1.684734in}}%
\pgfusepath{clip}%
\pgfsetroundcap%
\pgfsetroundjoin%
\pgfsetlinewidth{1.003750pt}%
\definecolor{currentstroke}{rgb}{1.000000,1.000000,1.000000}%
\pgfsetstrokecolor{currentstroke}%
\pgfsetdash{}{0pt}%
\pgfpathmoveto{\pgfqpoint{4.978935in}{0.557870in}}%
\pgfpathlineto{\pgfqpoint{4.978935in}{2.242604in}}%
\pgfusepath{stroke}%
\end{pgfscope}%
\begin{pgfscope}%
\definecolor{textcolor}{rgb}{0.150000,0.150000,0.150000}%
\pgfsetstrokecolor{textcolor}%
\pgfsetfillcolor{textcolor}%
\pgftext[x=4.978935in,y=0.425926in,,top]{\color{textcolor}\sffamily\fontsize{11.000000}{13.200000}\selectfont \(\displaystyle 0.50\)}%
\end{pgfscope}%
\begin{pgfscope}%
\pgfpathrectangle{\pgfqpoint{3.717870in}{0.557870in}}{\pgfqpoint{2.522130in}{1.684734in}}%
\pgfusepath{clip}%
\pgfsetroundcap%
\pgfsetroundjoin%
\pgfsetlinewidth{1.003750pt}%
\definecolor{currentstroke}{rgb}{1.000000,1.000000,1.000000}%
\pgfsetstrokecolor{currentstroke}%
\pgfsetdash{}{0pt}%
\pgfpathmoveto{\pgfqpoint{5.552146in}{0.557870in}}%
\pgfpathlineto{\pgfqpoint{5.552146in}{2.242604in}}%
\pgfusepath{stroke}%
\end{pgfscope}%
\begin{pgfscope}%
\definecolor{textcolor}{rgb}{0.150000,0.150000,0.150000}%
\pgfsetstrokecolor{textcolor}%
\pgfsetfillcolor{textcolor}%
\pgftext[x=5.552146in,y=0.425926in,,top]{\color{textcolor}\sffamily\fontsize{11.000000}{13.200000}\selectfont \(\displaystyle 0.75\)}%
\end{pgfscope}%
\begin{pgfscope}%
\pgfpathrectangle{\pgfqpoint{3.717870in}{0.557870in}}{\pgfqpoint{2.522130in}{1.684734in}}%
\pgfusepath{clip}%
\pgfsetroundcap%
\pgfsetroundjoin%
\pgfsetlinewidth{1.003750pt}%
\definecolor{currentstroke}{rgb}{1.000000,1.000000,1.000000}%
\pgfsetstrokecolor{currentstroke}%
\pgfsetdash{}{0pt}%
\pgfpathmoveto{\pgfqpoint{6.125358in}{0.557870in}}%
\pgfpathlineto{\pgfqpoint{6.125358in}{2.242604in}}%
\pgfusepath{stroke}%
\end{pgfscope}%
\begin{pgfscope}%
\definecolor{textcolor}{rgb}{0.150000,0.150000,0.150000}%
\pgfsetstrokecolor{textcolor}%
\pgfsetfillcolor{textcolor}%
\pgftext[x=6.125358in,y=0.425926in,,top]{\color{textcolor}\sffamily\fontsize{11.000000}{13.200000}\selectfont \(\displaystyle 1.00\)}%
\end{pgfscope}%
\begin{pgfscope}%
\definecolor{textcolor}{rgb}{0.150000,0.150000,0.150000}%
\pgfsetstrokecolor{textcolor}%
\pgfsetfillcolor{textcolor}%
\pgftext[x=4.978935in,y=0.235185in,,top]{\color{textcolor}\sffamily\fontsize{11.000000}{13.200000}\selectfont Specificity}%
\end{pgfscope}%
\begin{pgfscope}%
\pgfpathrectangle{\pgfqpoint{3.717870in}{0.557870in}}{\pgfqpoint{2.522130in}{1.684734in}}%
\pgfusepath{clip}%
\pgfsetroundcap%
\pgfsetroundjoin%
\pgfsetlinewidth{1.003750pt}%
\definecolor{currentstroke}{rgb}{1.000000,1.000000,1.000000}%
\pgfsetstrokecolor{currentstroke}%
\pgfsetdash{}{0pt}%
\pgfpathmoveto{\pgfqpoint{3.717870in}{0.634449in}}%
\pgfpathlineto{\pgfqpoint{6.240000in}{0.634449in}}%
\pgfusepath{stroke}%
\end{pgfscope}%
\begin{pgfscope}%
\definecolor{textcolor}{rgb}{0.150000,0.150000,0.150000}%
\pgfsetstrokecolor{textcolor}%
\pgfsetfillcolor{textcolor}%
\pgftext[x=3.315555in,y=0.581642in,left,base]{\color{textcolor}\sffamily\fontsize{11.000000}{13.200000}\selectfont \(\displaystyle 0.00\)}%
\end{pgfscope}%
\begin{pgfscope}%
\pgfpathrectangle{\pgfqpoint{3.717870in}{0.557870in}}{\pgfqpoint{2.522130in}{1.684734in}}%
\pgfusepath{clip}%
\pgfsetroundcap%
\pgfsetroundjoin%
\pgfsetlinewidth{1.003750pt}%
\definecolor{currentstroke}{rgb}{1.000000,1.000000,1.000000}%
\pgfsetstrokecolor{currentstroke}%
\pgfsetdash{}{0pt}%
\pgfpathmoveto{\pgfqpoint{3.717870in}{1.017343in}}%
\pgfpathlineto{\pgfqpoint{6.240000in}{1.017343in}}%
\pgfusepath{stroke}%
\end{pgfscope}%
\begin{pgfscope}%
\definecolor{textcolor}{rgb}{0.150000,0.150000,0.150000}%
\pgfsetstrokecolor{textcolor}%
\pgfsetfillcolor{textcolor}%
\pgftext[x=3.315555in,y=0.964536in,left,base]{\color{textcolor}\sffamily\fontsize{11.000000}{13.200000}\selectfont \(\displaystyle 0.25\)}%
\end{pgfscope}%
\begin{pgfscope}%
\pgfpathrectangle{\pgfqpoint{3.717870in}{0.557870in}}{\pgfqpoint{2.522130in}{1.684734in}}%
\pgfusepath{clip}%
\pgfsetroundcap%
\pgfsetroundjoin%
\pgfsetlinewidth{1.003750pt}%
\definecolor{currentstroke}{rgb}{1.000000,1.000000,1.000000}%
\pgfsetstrokecolor{currentstroke}%
\pgfsetdash{}{0pt}%
\pgfpathmoveto{\pgfqpoint{3.717870in}{1.400237in}}%
\pgfpathlineto{\pgfqpoint{6.240000in}{1.400237in}}%
\pgfusepath{stroke}%
\end{pgfscope}%
\begin{pgfscope}%
\definecolor{textcolor}{rgb}{0.150000,0.150000,0.150000}%
\pgfsetstrokecolor{textcolor}%
\pgfsetfillcolor{textcolor}%
\pgftext[x=3.315555in,y=1.347431in,left,base]{\color{textcolor}\sffamily\fontsize{11.000000}{13.200000}\selectfont \(\displaystyle 0.50\)}%
\end{pgfscope}%
\begin{pgfscope}%
\pgfpathrectangle{\pgfqpoint{3.717870in}{0.557870in}}{\pgfqpoint{2.522130in}{1.684734in}}%
\pgfusepath{clip}%
\pgfsetroundcap%
\pgfsetroundjoin%
\pgfsetlinewidth{1.003750pt}%
\definecolor{currentstroke}{rgb}{1.000000,1.000000,1.000000}%
\pgfsetstrokecolor{currentstroke}%
\pgfsetdash{}{0pt}%
\pgfpathmoveto{\pgfqpoint{3.717870in}{1.783131in}}%
\pgfpathlineto{\pgfqpoint{6.240000in}{1.783131in}}%
\pgfusepath{stroke}%
\end{pgfscope}%
\begin{pgfscope}%
\definecolor{textcolor}{rgb}{0.150000,0.150000,0.150000}%
\pgfsetstrokecolor{textcolor}%
\pgfsetfillcolor{textcolor}%
\pgftext[x=3.315555in,y=1.730325in,left,base]{\color{textcolor}\sffamily\fontsize{11.000000}{13.200000}\selectfont \(\displaystyle 0.75\)}%
\end{pgfscope}%
\begin{pgfscope}%
\pgfpathrectangle{\pgfqpoint{3.717870in}{0.557870in}}{\pgfqpoint{2.522130in}{1.684734in}}%
\pgfusepath{clip}%
\pgfsetroundcap%
\pgfsetroundjoin%
\pgfsetlinewidth{1.003750pt}%
\definecolor{currentstroke}{rgb}{1.000000,1.000000,1.000000}%
\pgfsetstrokecolor{currentstroke}%
\pgfsetdash{}{0pt}%
\pgfpathmoveto{\pgfqpoint{3.717870in}{2.166025in}}%
\pgfpathlineto{\pgfqpoint{6.240000in}{2.166025in}}%
\pgfusepath{stroke}%
\end{pgfscope}%
\begin{pgfscope}%
\definecolor{textcolor}{rgb}{0.150000,0.150000,0.150000}%
\pgfsetstrokecolor{textcolor}%
\pgfsetfillcolor{textcolor}%
\pgftext[x=3.315555in,y=2.113219in,left,base]{\color{textcolor}\sffamily\fontsize{11.000000}{13.200000}\selectfont \(\displaystyle 1.00\)}%
\end{pgfscope}%
\begin{pgfscope}%
\definecolor{textcolor}{rgb}{0.150000,0.150000,0.150000}%
\pgfsetstrokecolor{textcolor}%
\pgfsetfillcolor{textcolor}%
\pgftext[x=3.260000in,y=1.400237in,,bottom,rotate=90.000000]{\color{textcolor}\sffamily\fontsize{11.000000}{13.200000}\selectfont Sensitivity}%
\end{pgfscope}%
\begin{pgfscope}%
\pgfpathrectangle{\pgfqpoint{3.717870in}{0.557870in}}{\pgfqpoint{2.522130in}{1.684734in}}%
\pgfusepath{clip}%
\pgfsetbuttcap%
\pgfsetroundjoin%
\definecolor{currentfill}{rgb}{0.298039,0.447059,0.690196}%
\pgfsetfillcolor{currentfill}%
\pgfsetlinewidth{1.003750pt}%
\definecolor{currentstroke}{rgb}{0.298039,0.447059,0.690196}%
\pgfsetstrokecolor{currentstroke}%
\pgfsetdash{}{0pt}%
\pgfpathmoveto{\pgfqpoint{6.102429in}{0.603393in}}%
\pgfpathcurveto{\pgfqpoint{6.110666in}{0.603393in}}{\pgfqpoint{6.118566in}{0.606665in}}{\pgfqpoint{6.124390in}{0.612489in}}%
\pgfpathcurveto{\pgfqpoint{6.130213in}{0.618313in}}{\pgfqpoint{6.133486in}{0.626213in}}{\pgfqpoint{6.133486in}{0.634449in}}%
\pgfpathcurveto{\pgfqpoint{6.133486in}{0.642685in}}{\pgfqpoint{6.130213in}{0.650585in}}{\pgfqpoint{6.124390in}{0.656409in}}%
\pgfpathcurveto{\pgfqpoint{6.118566in}{0.662233in}}{\pgfqpoint{6.110666in}{0.665506in}}{\pgfqpoint{6.102429in}{0.665506in}}%
\pgfpathcurveto{\pgfqpoint{6.094193in}{0.665506in}}{\pgfqpoint{6.086293in}{0.662233in}}{\pgfqpoint{6.080469in}{0.656409in}}%
\pgfpathcurveto{\pgfqpoint{6.074645in}{0.650585in}}{\pgfqpoint{6.071373in}{0.642685in}}{\pgfqpoint{6.071373in}{0.634449in}}%
\pgfpathcurveto{\pgfqpoint{6.071373in}{0.626213in}}{\pgfqpoint{6.074645in}{0.618313in}}{\pgfqpoint{6.080469in}{0.612489in}}%
\pgfpathcurveto{\pgfqpoint{6.086293in}{0.606665in}}{\pgfqpoint{6.094193in}{0.603393in}}{\pgfqpoint{6.102429in}{0.603393in}}%
\pgfpathclose%
\pgfusepath{stroke,fill}%
\end{pgfscope}%
\begin{pgfscope}%
\pgfpathrectangle{\pgfqpoint{3.717870in}{0.557870in}}{\pgfqpoint{2.522130in}{1.684734in}}%
\pgfusepath{clip}%
\pgfsetbuttcap%
\pgfsetroundjoin%
\definecolor{currentfill}{rgb}{0.298039,0.447059,0.690196}%
\pgfsetfillcolor{currentfill}%
\pgfsetlinewidth{1.003750pt}%
\definecolor{currentstroke}{rgb}{0.298039,0.447059,0.690196}%
\pgfsetstrokecolor{currentstroke}%
\pgfsetdash{}{0pt}%
\pgfpathmoveto{\pgfqpoint{5.735574in}{0.861342in}}%
\pgfpathcurveto{\pgfqpoint{5.743810in}{0.861342in}}{\pgfqpoint{5.751710in}{0.864615in}}{\pgfqpoint{5.757534in}{0.870439in}}%
\pgfpathcurveto{\pgfqpoint{5.763358in}{0.876262in}}{\pgfqpoint{5.766631in}{0.884162in}}{\pgfqpoint{5.766631in}{0.892399in}}%
\pgfpathcurveto{\pgfqpoint{5.766631in}{0.900635in}}{\pgfqpoint{5.763358in}{0.908535in}}{\pgfqpoint{5.757534in}{0.914359in}}%
\pgfpathcurveto{\pgfqpoint{5.751710in}{0.920183in}}{\pgfqpoint{5.743810in}{0.923455in}}{\pgfqpoint{5.735574in}{0.923455in}}%
\pgfpathcurveto{\pgfqpoint{5.727338in}{0.923455in}}{\pgfqpoint{5.719438in}{0.920183in}}{\pgfqpoint{5.713614in}{0.914359in}}%
\pgfpathcurveto{\pgfqpoint{5.707790in}{0.908535in}}{\pgfqpoint{5.704518in}{0.900635in}}{\pgfqpoint{5.704518in}{0.892399in}}%
\pgfpathcurveto{\pgfqpoint{5.704518in}{0.884162in}}{\pgfqpoint{5.707790in}{0.876262in}}{\pgfqpoint{5.713614in}{0.870439in}}%
\pgfpathcurveto{\pgfqpoint{5.719438in}{0.864615in}}{\pgfqpoint{5.727338in}{0.861342in}}{\pgfqpoint{5.735574in}{0.861342in}}%
\pgfpathclose%
\pgfusepath{stroke,fill}%
\end{pgfscope}%
\begin{pgfscope}%
\pgfpathrectangle{\pgfqpoint{3.717870in}{0.557870in}}{\pgfqpoint{2.522130in}{1.684734in}}%
\pgfusepath{clip}%
\pgfsetbuttcap%
\pgfsetroundjoin%
\definecolor{currentfill}{rgb}{0.298039,0.447059,0.690196}%
\pgfsetfillcolor{currentfill}%
\pgfsetlinewidth{1.003750pt}%
\definecolor{currentstroke}{rgb}{0.298039,0.447059,0.690196}%
\pgfsetstrokecolor{currentstroke}%
\pgfsetdash{}{0pt}%
\pgfpathmoveto{\pgfqpoint{6.102429in}{0.603393in}}%
\pgfpathcurveto{\pgfqpoint{6.110666in}{0.603393in}}{\pgfqpoint{6.118566in}{0.606665in}}{\pgfqpoint{6.124390in}{0.612489in}}%
\pgfpathcurveto{\pgfqpoint{6.130213in}{0.618313in}}{\pgfqpoint{6.133486in}{0.626213in}}{\pgfqpoint{6.133486in}{0.634449in}}%
\pgfpathcurveto{\pgfqpoint{6.133486in}{0.642685in}}{\pgfqpoint{6.130213in}{0.650585in}}{\pgfqpoint{6.124390in}{0.656409in}}%
\pgfpathcurveto{\pgfqpoint{6.118566in}{0.662233in}}{\pgfqpoint{6.110666in}{0.665506in}}{\pgfqpoint{6.102429in}{0.665506in}}%
\pgfpathcurveto{\pgfqpoint{6.094193in}{0.665506in}}{\pgfqpoint{6.086293in}{0.662233in}}{\pgfqpoint{6.080469in}{0.656409in}}%
\pgfpathcurveto{\pgfqpoint{6.074645in}{0.650585in}}{\pgfqpoint{6.071373in}{0.642685in}}{\pgfqpoint{6.071373in}{0.634449in}}%
\pgfpathcurveto{\pgfqpoint{6.071373in}{0.626213in}}{\pgfqpoint{6.074645in}{0.618313in}}{\pgfqpoint{6.080469in}{0.612489in}}%
\pgfpathcurveto{\pgfqpoint{6.086293in}{0.606665in}}{\pgfqpoint{6.094193in}{0.603393in}}{\pgfqpoint{6.102429in}{0.603393in}}%
\pgfpathclose%
\pgfusepath{stroke,fill}%
\end{pgfscope}%
\begin{pgfscope}%
\pgfpathrectangle{\pgfqpoint{3.717870in}{0.557870in}}{\pgfqpoint{2.522130in}{1.684734in}}%
\pgfusepath{clip}%
\pgfsetbuttcap%
\pgfsetroundjoin%
\definecolor{currentfill}{rgb}{0.298039,0.447059,0.690196}%
\pgfsetfillcolor{currentfill}%
\pgfsetlinewidth{1.003750pt}%
\definecolor{currentstroke}{rgb}{0.298039,0.447059,0.690196}%
\pgfsetstrokecolor{currentstroke}%
\pgfsetdash{}{0pt}%
\pgfpathmoveto{\pgfqpoint{5.529218in}{1.554582in}}%
\pgfpathcurveto{\pgfqpoint{5.537454in}{1.554582in}}{\pgfqpoint{5.545354in}{1.557854in}}{\pgfqpoint{5.551178in}{1.563678in}}%
\pgfpathcurveto{\pgfqpoint{5.557002in}{1.569502in}}{\pgfqpoint{5.560275in}{1.577402in}}{\pgfqpoint{5.560275in}{1.585639in}}%
\pgfpathcurveto{\pgfqpoint{5.560275in}{1.593875in}}{\pgfqpoint{5.557002in}{1.601775in}}{\pgfqpoint{5.551178in}{1.607599in}}%
\pgfpathcurveto{\pgfqpoint{5.545354in}{1.613423in}}{\pgfqpoint{5.537454in}{1.616695in}}{\pgfqpoint{5.529218in}{1.616695in}}%
\pgfpathcurveto{\pgfqpoint{5.520982in}{1.616695in}}{\pgfqpoint{5.513082in}{1.613423in}}{\pgfqpoint{5.507258in}{1.607599in}}%
\pgfpathcurveto{\pgfqpoint{5.501434in}{1.601775in}}{\pgfqpoint{5.498162in}{1.593875in}}{\pgfqpoint{5.498162in}{1.585639in}}%
\pgfpathcurveto{\pgfqpoint{5.498162in}{1.577402in}}{\pgfqpoint{5.501434in}{1.569502in}}{\pgfqpoint{5.507258in}{1.563678in}}%
\pgfpathcurveto{\pgfqpoint{5.513082in}{1.557854in}}{\pgfqpoint{5.520982in}{1.554582in}}{\pgfqpoint{5.529218in}{1.554582in}}%
\pgfpathclose%
\pgfusepath{stroke,fill}%
\end{pgfscope}%
\begin{pgfscope}%
\pgfpathrectangle{\pgfqpoint{3.717870in}{0.557870in}}{\pgfqpoint{2.522130in}{1.684734in}}%
\pgfusepath{clip}%
\pgfsetbuttcap%
\pgfsetroundjoin%
\definecolor{currentfill}{rgb}{0.298039,0.447059,0.690196}%
\pgfsetfillcolor{currentfill}%
\pgfsetlinewidth{1.003750pt}%
\definecolor{currentstroke}{rgb}{0.298039,0.447059,0.690196}%
\pgfsetstrokecolor{currentstroke}%
\pgfsetdash{}{0pt}%
\pgfpathmoveto{\pgfqpoint{5.495455in}{1.519180in}}%
\pgfpathcurveto{\pgfqpoint{5.503692in}{1.519180in}}{\pgfqpoint{5.511592in}{1.522453in}}{\pgfqpoint{5.517416in}{1.528277in}}%
\pgfpathcurveto{\pgfqpoint{5.523239in}{1.534101in}}{\pgfqpoint{5.526512in}{1.542001in}}{\pgfqpoint{5.526512in}{1.550237in}}%
\pgfpathcurveto{\pgfqpoint{5.526512in}{1.558473in}}{\pgfqpoint{5.523239in}{1.566373in}}{\pgfqpoint{5.517416in}{1.572197in}}%
\pgfpathcurveto{\pgfqpoint{5.511592in}{1.578021in}}{\pgfqpoint{5.503692in}{1.581293in}}{\pgfqpoint{5.495455in}{1.581293in}}%
\pgfpathcurveto{\pgfqpoint{5.487219in}{1.581293in}}{\pgfqpoint{5.479319in}{1.578021in}}{\pgfqpoint{5.473495in}{1.572197in}}%
\pgfpathcurveto{\pgfqpoint{5.467671in}{1.566373in}}{\pgfqpoint{5.464399in}{1.558473in}}{\pgfqpoint{5.464399in}{1.550237in}}%
\pgfpathcurveto{\pgfqpoint{5.464399in}{1.542001in}}{\pgfqpoint{5.467671in}{1.534101in}}{\pgfqpoint{5.473495in}{1.528277in}}%
\pgfpathcurveto{\pgfqpoint{5.479319in}{1.522453in}}{\pgfqpoint{5.487219in}{1.519180in}}{\pgfqpoint{5.495455in}{1.519180in}}%
\pgfpathclose%
\pgfusepath{stroke,fill}%
\end{pgfscope}%
\begin{pgfscope}%
\pgfpathrectangle{\pgfqpoint{3.717870in}{0.557870in}}{\pgfqpoint{2.522130in}{1.684734in}}%
\pgfusepath{clip}%
\pgfsetbuttcap%
\pgfsetroundjoin%
\definecolor{currentfill}{rgb}{0.298039,0.447059,0.690196}%
\pgfsetfillcolor{currentfill}%
\pgfsetlinewidth{1.003750pt}%
\definecolor{currentstroke}{rgb}{0.298039,0.447059,0.690196}%
\pgfsetstrokecolor{currentstroke}%
\pgfsetdash{}{0pt}%
\pgfpathmoveto{\pgfqpoint{5.646632in}{1.566549in}}%
\pgfpathcurveto{\pgfqpoint{5.654868in}{1.566549in}}{\pgfqpoint{5.662768in}{1.569821in}}{\pgfqpoint{5.668592in}{1.575645in}}%
\pgfpathcurveto{\pgfqpoint{5.674416in}{1.581469in}}{\pgfqpoint{5.677688in}{1.589369in}}{\pgfqpoint{5.677688in}{1.597605in}}%
\pgfpathcurveto{\pgfqpoint{5.677688in}{1.605842in}}{\pgfqpoint{5.674416in}{1.613742in}}{\pgfqpoint{5.668592in}{1.619566in}}%
\pgfpathcurveto{\pgfqpoint{5.662768in}{1.625389in}}{\pgfqpoint{5.654868in}{1.628662in}}{\pgfqpoint{5.646632in}{1.628662in}}%
\pgfpathcurveto{\pgfqpoint{5.638396in}{1.628662in}}{\pgfqpoint{5.630496in}{1.625389in}}{\pgfqpoint{5.624672in}{1.619566in}}%
\pgfpathcurveto{\pgfqpoint{5.618848in}{1.613742in}}{\pgfqpoint{5.615575in}{1.605842in}}{\pgfqpoint{5.615575in}{1.597605in}}%
\pgfpathcurveto{\pgfqpoint{5.615575in}{1.589369in}}{\pgfqpoint{5.618848in}{1.581469in}}{\pgfqpoint{5.624672in}{1.575645in}}%
\pgfpathcurveto{\pgfqpoint{5.630496in}{1.569821in}}{\pgfqpoint{5.638396in}{1.566549in}}{\pgfqpoint{5.646632in}{1.566549in}}%
\pgfpathclose%
\pgfusepath{stroke,fill}%
\end{pgfscope}%
\begin{pgfscope}%
\pgfpathrectangle{\pgfqpoint{3.717870in}{0.557870in}}{\pgfqpoint{2.522130in}{1.684734in}}%
\pgfusepath{clip}%
\pgfsetbuttcap%
\pgfsetroundjoin%
\definecolor{currentfill}{rgb}{0.298039,0.447059,0.690196}%
\pgfsetfillcolor{currentfill}%
\pgfsetlinewidth{1.003750pt}%
\definecolor{currentstroke}{rgb}{0.298039,0.447059,0.690196}%
\pgfsetstrokecolor{currentstroke}%
\pgfsetdash{}{0pt}%
\pgfpathmoveto{\pgfqpoint{5.621436in}{1.503391in}}%
\pgfpathcurveto{\pgfqpoint{5.629672in}{1.503391in}}{\pgfqpoint{5.637572in}{1.506663in}}{\pgfqpoint{5.643396in}{1.512487in}}%
\pgfpathcurveto{\pgfqpoint{5.649220in}{1.518311in}}{\pgfqpoint{5.652492in}{1.526211in}}{\pgfqpoint{5.652492in}{1.534448in}}%
\pgfpathcurveto{\pgfqpoint{5.652492in}{1.542684in}}{\pgfqpoint{5.649220in}{1.550584in}}{\pgfqpoint{5.643396in}{1.556408in}}%
\pgfpathcurveto{\pgfqpoint{5.637572in}{1.562232in}}{\pgfqpoint{5.629672in}{1.565504in}}{\pgfqpoint{5.621436in}{1.565504in}}%
\pgfpathcurveto{\pgfqpoint{5.613199in}{1.565504in}}{\pgfqpoint{5.605299in}{1.562232in}}{\pgfqpoint{5.599475in}{1.556408in}}%
\pgfpathcurveto{\pgfqpoint{5.593652in}{1.550584in}}{\pgfqpoint{5.590379in}{1.542684in}}{\pgfqpoint{5.590379in}{1.534448in}}%
\pgfpathcurveto{\pgfqpoint{5.590379in}{1.526211in}}{\pgfqpoint{5.593652in}{1.518311in}}{\pgfqpoint{5.599475in}{1.512487in}}%
\pgfpathcurveto{\pgfqpoint{5.605299in}{1.506663in}}{\pgfqpoint{5.613199in}{1.503391in}}{\pgfqpoint{5.621436in}{1.503391in}}%
\pgfpathclose%
\pgfusepath{stroke,fill}%
\end{pgfscope}%
\begin{pgfscope}%
\pgfpathrectangle{\pgfqpoint{3.717870in}{0.557870in}}{\pgfqpoint{2.522130in}{1.684734in}}%
\pgfusepath{clip}%
\pgfsetbuttcap%
\pgfsetroundjoin%
\definecolor{currentfill}{rgb}{0.298039,0.447059,0.690196}%
\pgfsetfillcolor{currentfill}%
\pgfsetlinewidth{1.003750pt}%
\definecolor{currentstroke}{rgb}{0.298039,0.447059,0.690196}%
\pgfsetstrokecolor{currentstroke}%
\pgfsetdash{}{0pt}%
\pgfpathmoveto{\pgfqpoint{6.100162in}{0.669267in}}%
\pgfpathcurveto{\pgfqpoint{6.108398in}{0.669267in}}{\pgfqpoint{6.116298in}{0.672539in}}{\pgfqpoint{6.122122in}{0.678363in}}%
\pgfpathcurveto{\pgfqpoint{6.127946in}{0.684187in}}{\pgfqpoint{6.131218in}{0.692087in}}{\pgfqpoint{6.131218in}{0.700323in}}%
\pgfpathcurveto{\pgfqpoint{6.131218in}{0.708560in}}{\pgfqpoint{6.127946in}{0.716460in}}{\pgfqpoint{6.122122in}{0.722284in}}%
\pgfpathcurveto{\pgfqpoint{6.116298in}{0.728108in}}{\pgfqpoint{6.108398in}{0.731380in}}{\pgfqpoint{6.100162in}{0.731380in}}%
\pgfpathcurveto{\pgfqpoint{6.091925in}{0.731380in}}{\pgfqpoint{6.084025in}{0.728108in}}{\pgfqpoint{6.078201in}{0.722284in}}%
\pgfpathcurveto{\pgfqpoint{6.072377in}{0.716460in}}{\pgfqpoint{6.069105in}{0.708560in}}{\pgfqpoint{6.069105in}{0.700323in}}%
\pgfpathcurveto{\pgfqpoint{6.069105in}{0.692087in}}{\pgfqpoint{6.072377in}{0.684187in}}{\pgfqpoint{6.078201in}{0.678363in}}%
\pgfpathcurveto{\pgfqpoint{6.084025in}{0.672539in}}{\pgfqpoint{6.091925in}{0.669267in}}{\pgfqpoint{6.100162in}{0.669267in}}%
\pgfpathclose%
\pgfusepath{stroke,fill}%
\end{pgfscope}%
\begin{pgfscope}%
\pgfpathrectangle{\pgfqpoint{3.717870in}{0.557870in}}{\pgfqpoint{2.522130in}{1.684734in}}%
\pgfusepath{clip}%
\pgfsetbuttcap%
\pgfsetroundjoin%
\definecolor{currentfill}{rgb}{0.298039,0.447059,0.690196}%
\pgfsetfillcolor{currentfill}%
\pgfsetlinewidth{1.003750pt}%
\definecolor{currentstroke}{rgb}{0.298039,0.447059,0.690196}%
\pgfsetstrokecolor{currentstroke}%
\pgfsetdash{}{0pt}%
\pgfpathmoveto{\pgfqpoint{3.908101in}{2.118500in}}%
\pgfpathcurveto{\pgfqpoint{3.916337in}{2.118500in}}{\pgfqpoint{3.924237in}{2.121773in}}{\pgfqpoint{3.930061in}{2.127597in}}%
\pgfpathcurveto{\pgfqpoint{3.935885in}{2.133420in}}{\pgfqpoint{3.939157in}{2.141321in}}{\pgfqpoint{3.939157in}{2.149557in}}%
\pgfpathcurveto{\pgfqpoint{3.939157in}{2.157793in}}{\pgfqpoint{3.935885in}{2.165693in}}{\pgfqpoint{3.930061in}{2.171517in}}%
\pgfpathcurveto{\pgfqpoint{3.924237in}{2.177341in}}{\pgfqpoint{3.916337in}{2.180613in}}{\pgfqpoint{3.908101in}{2.180613in}}%
\pgfpathcurveto{\pgfqpoint{3.899865in}{2.180613in}}{\pgfqpoint{3.891965in}{2.177341in}}{\pgfqpoint{3.886141in}{2.171517in}}%
\pgfpathcurveto{\pgfqpoint{3.880317in}{2.165693in}}{\pgfqpoint{3.877044in}{2.157793in}}{\pgfqpoint{3.877044in}{2.149557in}}%
\pgfpathcurveto{\pgfqpoint{3.877044in}{2.141321in}}{\pgfqpoint{3.880317in}{2.133420in}}{\pgfqpoint{3.886141in}{2.127597in}}%
\pgfpathcurveto{\pgfqpoint{3.891965in}{2.121773in}}{\pgfqpoint{3.899865in}{2.118500in}}{\pgfqpoint{3.908101in}{2.118500in}}%
\pgfpathclose%
\pgfusepath{stroke,fill}%
\end{pgfscope}%
\begin{pgfscope}%
\pgfpathrectangle{\pgfqpoint{3.717870in}{0.557870in}}{\pgfqpoint{2.522130in}{1.684734in}}%
\pgfusepath{clip}%
\pgfsetbuttcap%
\pgfsetroundjoin%
\definecolor{currentfill}{rgb}{0.298039,0.447059,0.690196}%
\pgfsetfillcolor{currentfill}%
\pgfsetlinewidth{1.003750pt}%
\definecolor{currentstroke}{rgb}{0.298039,0.447059,0.690196}%
\pgfsetstrokecolor{currentstroke}%
\pgfsetdash{}{0pt}%
\pgfpathmoveto{\pgfqpoint{5.621436in}{1.525632in}}%
\pgfpathcurveto{\pgfqpoint{5.629672in}{1.525632in}}{\pgfqpoint{5.637572in}{1.528904in}}{\pgfqpoint{5.643396in}{1.534728in}}%
\pgfpathcurveto{\pgfqpoint{5.649220in}{1.540552in}}{\pgfqpoint{5.652492in}{1.548452in}}{\pgfqpoint{5.652492in}{1.556689in}}%
\pgfpathcurveto{\pgfqpoint{5.652492in}{1.564925in}}{\pgfqpoint{5.649220in}{1.572825in}}{\pgfqpoint{5.643396in}{1.578649in}}%
\pgfpathcurveto{\pgfqpoint{5.637572in}{1.584473in}}{\pgfqpoint{5.629672in}{1.587745in}}{\pgfqpoint{5.621436in}{1.587745in}}%
\pgfpathcurveto{\pgfqpoint{5.613199in}{1.587745in}}{\pgfqpoint{5.605299in}{1.584473in}}{\pgfqpoint{5.599475in}{1.578649in}}%
\pgfpathcurveto{\pgfqpoint{5.593652in}{1.572825in}}{\pgfqpoint{5.590379in}{1.564925in}}{\pgfqpoint{5.590379in}{1.556689in}}%
\pgfpathcurveto{\pgfqpoint{5.590379in}{1.548452in}}{\pgfqpoint{5.593652in}{1.540552in}}{\pgfqpoint{5.599475in}{1.534728in}}%
\pgfpathcurveto{\pgfqpoint{5.605299in}{1.528904in}}{\pgfqpoint{5.613199in}{1.525632in}}{\pgfqpoint{5.621436in}{1.525632in}}%
\pgfpathclose%
\pgfusepath{stroke,fill}%
\end{pgfscope}%
\begin{pgfscope}%
\pgfpathrectangle{\pgfqpoint{3.717870in}{0.557870in}}{\pgfqpoint{2.522130in}{1.684734in}}%
\pgfusepath{clip}%
\pgfsetbuttcap%
\pgfsetroundjoin%
\definecolor{currentfill}{rgb}{0.298039,0.447059,0.690196}%
\pgfsetfillcolor{currentfill}%
\pgfsetlinewidth{1.003750pt}%
\definecolor{currentstroke}{rgb}{0.298039,0.447059,0.690196}%
\pgfsetstrokecolor{currentstroke}%
\pgfsetdash{}{0pt}%
\pgfpathmoveto{\pgfqpoint{5.483361in}{1.844775in}}%
\pgfpathcurveto{\pgfqpoint{5.491597in}{1.844775in}}{\pgfqpoint{5.499497in}{1.848048in}}{\pgfqpoint{5.505321in}{1.853872in}}%
\pgfpathcurveto{\pgfqpoint{5.511145in}{1.859696in}}{\pgfqpoint{5.514418in}{1.867596in}}{\pgfqpoint{5.514418in}{1.875832in}}%
\pgfpathcurveto{\pgfqpoint{5.514418in}{1.884068in}}{\pgfqpoint{5.511145in}{1.891968in}}{\pgfqpoint{5.505321in}{1.897792in}}%
\pgfpathcurveto{\pgfqpoint{5.499497in}{1.903616in}}{\pgfqpoint{5.491597in}{1.906888in}}{\pgfqpoint{5.483361in}{1.906888in}}%
\pgfpathcurveto{\pgfqpoint{5.475125in}{1.906888in}}{\pgfqpoint{5.467225in}{1.903616in}}{\pgfqpoint{5.461401in}{1.897792in}}%
\pgfpathcurveto{\pgfqpoint{5.455577in}{1.891968in}}{\pgfqpoint{5.452305in}{1.884068in}}{\pgfqpoint{5.452305in}{1.875832in}}%
\pgfpathcurveto{\pgfqpoint{5.452305in}{1.867596in}}{\pgfqpoint{5.455577in}{1.859696in}}{\pgfqpoint{5.461401in}{1.853872in}}%
\pgfpathcurveto{\pgfqpoint{5.467225in}{1.848048in}}{\pgfqpoint{5.475125in}{1.844775in}}{\pgfqpoint{5.483361in}{1.844775in}}%
\pgfpathclose%
\pgfusepath{stroke,fill}%
\end{pgfscope}%
\begin{pgfscope}%
\pgfpathrectangle{\pgfqpoint{3.717870in}{0.557870in}}{\pgfqpoint{2.522130in}{1.684734in}}%
\pgfusepath{clip}%
\pgfsetbuttcap%
\pgfsetroundjoin%
\definecolor{currentfill}{rgb}{0.298039,0.447059,0.690196}%
\pgfsetfillcolor{currentfill}%
\pgfsetlinewidth{1.003750pt}%
\definecolor{currentstroke}{rgb}{0.298039,0.447059,0.690196}%
\pgfsetstrokecolor{currentstroke}%
\pgfsetdash{}{0pt}%
\pgfpathmoveto{\pgfqpoint{5.322862in}{1.909263in}}%
\pgfpathcurveto{\pgfqpoint{5.331098in}{1.909263in}}{\pgfqpoint{5.338998in}{1.912535in}}{\pgfqpoint{5.344822in}{1.918359in}}%
\pgfpathcurveto{\pgfqpoint{5.350646in}{1.924183in}}{\pgfqpoint{5.353918in}{1.932083in}}{\pgfqpoint{5.353918in}{1.940319in}}%
\pgfpathcurveto{\pgfqpoint{5.353918in}{1.948556in}}{\pgfqpoint{5.350646in}{1.956456in}}{\pgfqpoint{5.344822in}{1.962280in}}%
\pgfpathcurveto{\pgfqpoint{5.338998in}{1.968104in}}{\pgfqpoint{5.331098in}{1.971376in}}{\pgfqpoint{5.322862in}{1.971376in}}%
\pgfpathcurveto{\pgfqpoint{5.314626in}{1.971376in}}{\pgfqpoint{5.306726in}{1.968104in}}{\pgfqpoint{5.300902in}{1.962280in}}%
\pgfpathcurveto{\pgfqpoint{5.295078in}{1.956456in}}{\pgfqpoint{5.291805in}{1.948556in}}{\pgfqpoint{5.291805in}{1.940319in}}%
\pgfpathcurveto{\pgfqpoint{5.291805in}{1.932083in}}{\pgfqpoint{5.295078in}{1.924183in}}{\pgfqpoint{5.300902in}{1.918359in}}%
\pgfpathcurveto{\pgfqpoint{5.306726in}{1.912535in}}{\pgfqpoint{5.314626in}{1.909263in}}{\pgfqpoint{5.322862in}{1.909263in}}%
\pgfpathclose%
\pgfusepath{stroke,fill}%
\end{pgfscope}%
\begin{pgfscope}%
\pgfpathrectangle{\pgfqpoint{3.717870in}{0.557870in}}{\pgfqpoint{2.522130in}{1.684734in}}%
\pgfusepath{clip}%
\pgfsetbuttcap%
\pgfsetroundjoin%
\definecolor{currentfill}{rgb}{0.298039,0.447059,0.690196}%
\pgfsetfillcolor{currentfill}%
\pgfsetlinewidth{1.003750pt}%
\definecolor{currentstroke}{rgb}{0.298039,0.447059,0.690196}%
\pgfsetstrokecolor{currentstroke}%
\pgfsetdash{}{0pt}%
\pgfpathmoveto{\pgfqpoint{5.277005in}{1.941507in}}%
\pgfpathcurveto{\pgfqpoint{5.285241in}{1.941507in}}{\pgfqpoint{5.293141in}{1.944779in}}{\pgfqpoint{5.298965in}{1.950603in}}%
\pgfpathcurveto{\pgfqpoint{5.304789in}{1.956427in}}{\pgfqpoint{5.308062in}{1.964327in}}{\pgfqpoint{5.308062in}{1.972563in}}%
\pgfpathcurveto{\pgfqpoint{5.308062in}{1.980799in}}{\pgfqpoint{5.304789in}{1.988699in}}{\pgfqpoint{5.298965in}{1.994523in}}%
\pgfpathcurveto{\pgfqpoint{5.293141in}{2.000347in}}{\pgfqpoint{5.285241in}{2.003620in}}{\pgfqpoint{5.277005in}{2.003620in}}%
\pgfpathcurveto{\pgfqpoint{5.268769in}{2.003620in}}{\pgfqpoint{5.260869in}{2.000347in}}{\pgfqpoint{5.255045in}{1.994523in}}%
\pgfpathcurveto{\pgfqpoint{5.249221in}{1.988699in}}{\pgfqpoint{5.245949in}{1.980799in}}{\pgfqpoint{5.245949in}{1.972563in}}%
\pgfpathcurveto{\pgfqpoint{5.245949in}{1.964327in}}{\pgfqpoint{5.249221in}{1.956427in}}{\pgfqpoint{5.255045in}{1.950603in}}%
\pgfpathcurveto{\pgfqpoint{5.260869in}{1.944779in}}{\pgfqpoint{5.268769in}{1.941507in}}{\pgfqpoint{5.277005in}{1.941507in}}%
\pgfpathclose%
\pgfusepath{stroke,fill}%
\end{pgfscope}%
\begin{pgfscope}%
\pgfpathrectangle{\pgfqpoint{3.717870in}{0.557870in}}{\pgfqpoint{2.522130in}{1.684734in}}%
\pgfusepath{clip}%
\pgfsetbuttcap%
\pgfsetroundjoin%
\definecolor{currentfill}{rgb}{0.298039,0.447059,0.690196}%
\pgfsetfillcolor{currentfill}%
\pgfsetlinewidth{1.003750pt}%
\definecolor{currentstroke}{rgb}{0.298039,0.447059,0.690196}%
\pgfsetstrokecolor{currentstroke}%
\pgfsetdash{}{0pt}%
\pgfpathmoveto{\pgfqpoint{3.832513in}{2.119179in}}%
\pgfpathcurveto{\pgfqpoint{3.840749in}{2.119179in}}{\pgfqpoint{3.848649in}{2.122452in}}{\pgfqpoint{3.854473in}{2.128276in}}%
\pgfpathcurveto{\pgfqpoint{3.860297in}{2.134100in}}{\pgfqpoint{3.863569in}{2.142000in}}{\pgfqpoint{3.863569in}{2.150236in}}%
\pgfpathcurveto{\pgfqpoint{3.863569in}{2.158472in}}{\pgfqpoint{3.860297in}{2.166372in}}{\pgfqpoint{3.854473in}{2.172196in}}%
\pgfpathcurveto{\pgfqpoint{3.848649in}{2.178020in}}{\pgfqpoint{3.840749in}{2.181292in}}{\pgfqpoint{3.832513in}{2.181292in}}%
\pgfpathcurveto{\pgfqpoint{3.824276in}{2.181292in}}{\pgfqpoint{3.816376in}{2.178020in}}{\pgfqpoint{3.810552in}{2.172196in}}%
\pgfpathcurveto{\pgfqpoint{3.804728in}{2.166372in}}{\pgfqpoint{3.801456in}{2.158472in}}{\pgfqpoint{3.801456in}{2.150236in}}%
\pgfpathcurveto{\pgfqpoint{3.801456in}{2.142000in}}{\pgfqpoint{3.804728in}{2.134100in}}{\pgfqpoint{3.810552in}{2.128276in}}%
\pgfpathcurveto{\pgfqpoint{3.816376in}{2.122452in}}{\pgfqpoint{3.824276in}{2.119179in}}{\pgfqpoint{3.832513in}{2.119179in}}%
\pgfpathclose%
\pgfusepath{stroke,fill}%
\end{pgfscope}%
\begin{pgfscope}%
\pgfpathrectangle{\pgfqpoint{3.717870in}{0.557870in}}{\pgfqpoint{2.522130in}{1.684734in}}%
\pgfusepath{clip}%
\pgfsetbuttcap%
\pgfsetroundjoin%
\definecolor{currentfill}{rgb}{0.298039,0.447059,0.690196}%
\pgfsetfillcolor{currentfill}%
\pgfsetlinewidth{1.003750pt}%
\definecolor{currentstroke}{rgb}{0.298039,0.447059,0.690196}%
\pgfsetstrokecolor{currentstroke}%
\pgfsetdash{}{0pt}%
\pgfpathmoveto{\pgfqpoint{5.218298in}{1.913917in}}%
\pgfpathcurveto{\pgfqpoint{5.226534in}{1.913917in}}{\pgfqpoint{5.234434in}{1.917189in}}{\pgfqpoint{5.240258in}{1.923013in}}%
\pgfpathcurveto{\pgfqpoint{5.246082in}{1.928837in}}{\pgfqpoint{5.249355in}{1.936737in}}{\pgfqpoint{5.249355in}{1.944973in}}%
\pgfpathcurveto{\pgfqpoint{5.249355in}{1.953209in}}{\pgfqpoint{5.246082in}{1.961109in}}{\pgfqpoint{5.240258in}{1.966933in}}%
\pgfpathcurveto{\pgfqpoint{5.234434in}{1.972757in}}{\pgfqpoint{5.226534in}{1.976030in}}{\pgfqpoint{5.218298in}{1.976030in}}%
\pgfpathcurveto{\pgfqpoint{5.210062in}{1.976030in}}{\pgfqpoint{5.202162in}{1.972757in}}{\pgfqpoint{5.196338in}{1.966933in}}%
\pgfpathcurveto{\pgfqpoint{5.190514in}{1.961109in}}{\pgfqpoint{5.187242in}{1.953209in}}{\pgfqpoint{5.187242in}{1.944973in}}%
\pgfpathcurveto{\pgfqpoint{5.187242in}{1.936737in}}{\pgfqpoint{5.190514in}{1.928837in}}{\pgfqpoint{5.196338in}{1.923013in}}%
\pgfpathcurveto{\pgfqpoint{5.202162in}{1.917189in}}{\pgfqpoint{5.210062in}{1.913917in}}{\pgfqpoint{5.218298in}{1.913917in}}%
\pgfpathclose%
\pgfusepath{stroke,fill}%
\end{pgfscope}%
\begin{pgfscope}%
\pgfpathrectangle{\pgfqpoint{3.717870in}{0.557870in}}{\pgfqpoint{2.522130in}{1.684734in}}%
\pgfusepath{clip}%
\pgfsetbuttcap%
\pgfsetroundjoin%
\definecolor{currentfill}{rgb}{0.298039,0.447059,0.690196}%
\pgfsetfillcolor{currentfill}%
\pgfsetlinewidth{1.003750pt}%
\definecolor{currentstroke}{rgb}{0.298039,0.447059,0.690196}%
\pgfsetstrokecolor{currentstroke}%
\pgfsetdash{}{0pt}%
\pgfpathmoveto{\pgfqpoint{3.832513in}{2.119179in}}%
\pgfpathcurveto{\pgfqpoint{3.840749in}{2.119179in}}{\pgfqpoint{3.848649in}{2.122452in}}{\pgfqpoint{3.854473in}{2.128276in}}%
\pgfpathcurveto{\pgfqpoint{3.860297in}{2.134100in}}{\pgfqpoint{3.863569in}{2.142000in}}{\pgfqpoint{3.863569in}{2.150236in}}%
\pgfpathcurveto{\pgfqpoint{3.863569in}{2.158472in}}{\pgfqpoint{3.860297in}{2.166372in}}{\pgfqpoint{3.854473in}{2.172196in}}%
\pgfpathcurveto{\pgfqpoint{3.848649in}{2.178020in}}{\pgfqpoint{3.840749in}{2.181292in}}{\pgfqpoint{3.832513in}{2.181292in}}%
\pgfpathcurveto{\pgfqpoint{3.824276in}{2.181292in}}{\pgfqpoint{3.816376in}{2.178020in}}{\pgfqpoint{3.810552in}{2.172196in}}%
\pgfpathcurveto{\pgfqpoint{3.804728in}{2.166372in}}{\pgfqpoint{3.801456in}{2.158472in}}{\pgfqpoint{3.801456in}{2.150236in}}%
\pgfpathcurveto{\pgfqpoint{3.801456in}{2.142000in}}{\pgfqpoint{3.804728in}{2.134100in}}{\pgfqpoint{3.810552in}{2.128276in}}%
\pgfpathcurveto{\pgfqpoint{3.816376in}{2.122452in}}{\pgfqpoint{3.824276in}{2.119179in}}{\pgfqpoint{3.832513in}{2.119179in}}%
\pgfpathclose%
\pgfusepath{stroke,fill}%
\end{pgfscope}%
\begin{pgfscope}%
\pgfpathrectangle{\pgfqpoint{3.717870in}{0.557870in}}{\pgfqpoint{2.522130in}{1.684734in}}%
\pgfusepath{clip}%
\pgfsetbuttcap%
\pgfsetroundjoin%
\definecolor{currentfill}{rgb}{0.298039,0.447059,0.690196}%
\pgfsetfillcolor{currentfill}%
\pgfsetlinewidth{1.003750pt}%
\definecolor{currentstroke}{rgb}{0.298039,0.447059,0.690196}%
\pgfsetstrokecolor{currentstroke}%
\pgfsetdash{}{0pt}%
\pgfpathmoveto{\pgfqpoint{5.697024in}{1.519180in}}%
\pgfpathcurveto{\pgfqpoint{5.705260in}{1.519180in}}{\pgfqpoint{5.713160in}{1.522453in}}{\pgfqpoint{5.718984in}{1.528277in}}%
\pgfpathcurveto{\pgfqpoint{5.724808in}{1.534101in}}{\pgfqpoint{5.728081in}{1.542001in}}{\pgfqpoint{5.728081in}{1.550237in}}%
\pgfpathcurveto{\pgfqpoint{5.728081in}{1.558473in}}{\pgfqpoint{5.724808in}{1.566373in}}{\pgfqpoint{5.718984in}{1.572197in}}%
\pgfpathcurveto{\pgfqpoint{5.713160in}{1.578021in}}{\pgfqpoint{5.705260in}{1.581293in}}{\pgfqpoint{5.697024in}{1.581293in}}%
\pgfpathcurveto{\pgfqpoint{5.688788in}{1.581293in}}{\pgfqpoint{5.680888in}{1.578021in}}{\pgfqpoint{5.675064in}{1.572197in}}%
\pgfpathcurveto{\pgfqpoint{5.669240in}{1.566373in}}{\pgfqpoint{5.665968in}{1.558473in}}{\pgfqpoint{5.665968in}{1.550237in}}%
\pgfpathcurveto{\pgfqpoint{5.665968in}{1.542001in}}{\pgfqpoint{5.669240in}{1.534101in}}{\pgfqpoint{5.675064in}{1.528277in}}%
\pgfpathcurveto{\pgfqpoint{5.680888in}{1.522453in}}{\pgfqpoint{5.688788in}{1.519180in}}{\pgfqpoint{5.697024in}{1.519180in}}%
\pgfpathclose%
\pgfusepath{stroke,fill}%
\end{pgfscope}%
\begin{pgfscope}%
\pgfpathrectangle{\pgfqpoint{3.717870in}{0.557870in}}{\pgfqpoint{2.522130in}{1.684734in}}%
\pgfusepath{clip}%
\pgfsetbuttcap%
\pgfsetroundjoin%
\definecolor{currentfill}{rgb}{0.298039,0.447059,0.690196}%
\pgfsetfillcolor{currentfill}%
\pgfsetlinewidth{1.003750pt}%
\definecolor{currentstroke}{rgb}{0.298039,0.447059,0.690196}%
\pgfsetstrokecolor{currentstroke}%
\pgfsetdash{}{0pt}%
\pgfpathmoveto{\pgfqpoint{3.832513in}{2.118500in}}%
\pgfpathcurveto{\pgfqpoint{3.840749in}{2.118500in}}{\pgfqpoint{3.848649in}{2.121773in}}{\pgfqpoint{3.854473in}{2.127597in}}%
\pgfpathcurveto{\pgfqpoint{3.860297in}{2.133420in}}{\pgfqpoint{3.863569in}{2.141321in}}{\pgfqpoint{3.863569in}{2.149557in}}%
\pgfpathcurveto{\pgfqpoint{3.863569in}{2.157793in}}{\pgfqpoint{3.860297in}{2.165693in}}{\pgfqpoint{3.854473in}{2.171517in}}%
\pgfpathcurveto{\pgfqpoint{3.848649in}{2.177341in}}{\pgfqpoint{3.840749in}{2.180613in}}{\pgfqpoint{3.832513in}{2.180613in}}%
\pgfpathcurveto{\pgfqpoint{3.824276in}{2.180613in}}{\pgfqpoint{3.816376in}{2.177341in}}{\pgfqpoint{3.810552in}{2.171517in}}%
\pgfpathcurveto{\pgfqpoint{3.804728in}{2.165693in}}{\pgfqpoint{3.801456in}{2.157793in}}{\pgfqpoint{3.801456in}{2.149557in}}%
\pgfpathcurveto{\pgfqpoint{3.801456in}{2.141321in}}{\pgfqpoint{3.804728in}{2.133420in}}{\pgfqpoint{3.810552in}{2.127597in}}%
\pgfpathcurveto{\pgfqpoint{3.816376in}{2.121773in}}{\pgfqpoint{3.824276in}{2.118500in}}{\pgfqpoint{3.832513in}{2.118500in}}%
\pgfpathclose%
\pgfusepath{stroke,fill}%
\end{pgfscope}%
\begin{pgfscope}%
\pgfpathrectangle{\pgfqpoint{3.717870in}{0.557870in}}{\pgfqpoint{2.522130in}{1.684734in}}%
\pgfusepath{clip}%
\pgfsetbuttcap%
\pgfsetroundjoin%
\definecolor{currentfill}{rgb}{0.298039,0.447059,0.690196}%
\pgfsetfillcolor{currentfill}%
\pgfsetlinewidth{1.003750pt}%
\definecolor{currentstroke}{rgb}{0.298039,0.447059,0.690196}%
\pgfsetstrokecolor{currentstroke}%
\pgfsetdash{}{0pt}%
\pgfpathmoveto{\pgfqpoint{5.394671in}{1.756192in}}%
\pgfpathcurveto{\pgfqpoint{5.402907in}{1.756192in}}{\pgfqpoint{5.410807in}{1.759464in}}{\pgfqpoint{5.416631in}{1.765288in}}%
\pgfpathcurveto{\pgfqpoint{5.422455in}{1.771112in}}{\pgfqpoint{5.425727in}{1.779012in}}{\pgfqpoint{5.425727in}{1.787248in}}%
\pgfpathcurveto{\pgfqpoint{5.425727in}{1.795485in}}{\pgfqpoint{5.422455in}{1.803385in}}{\pgfqpoint{5.416631in}{1.809209in}}%
\pgfpathcurveto{\pgfqpoint{5.410807in}{1.815033in}}{\pgfqpoint{5.402907in}{1.818305in}}{\pgfqpoint{5.394671in}{1.818305in}}%
\pgfpathcurveto{\pgfqpoint{5.386435in}{1.818305in}}{\pgfqpoint{5.378535in}{1.815033in}}{\pgfqpoint{5.372711in}{1.809209in}}%
\pgfpathcurveto{\pgfqpoint{5.366887in}{1.803385in}}{\pgfqpoint{5.363614in}{1.795485in}}{\pgfqpoint{5.363614in}{1.787248in}}%
\pgfpathcurveto{\pgfqpoint{5.363614in}{1.779012in}}{\pgfqpoint{5.366887in}{1.771112in}}{\pgfqpoint{5.372711in}{1.765288in}}%
\pgfpathcurveto{\pgfqpoint{5.378535in}{1.759464in}}{\pgfqpoint{5.386435in}{1.756192in}}{\pgfqpoint{5.394671in}{1.756192in}}%
\pgfpathclose%
\pgfusepath{stroke,fill}%
\end{pgfscope}%
\begin{pgfscope}%
\pgfpathrectangle{\pgfqpoint{3.717870in}{0.557870in}}{\pgfqpoint{2.522130in}{1.684734in}}%
\pgfusepath{clip}%
\pgfsetbuttcap%
\pgfsetroundjoin%
\definecolor{currentfill}{rgb}{0.298039,0.447059,0.690196}%
\pgfsetfillcolor{currentfill}%
\pgfsetlinewidth{1.003750pt}%
\definecolor{currentstroke}{rgb}{0.298039,0.447059,0.690196}%
\pgfsetstrokecolor{currentstroke}%
\pgfsetdash{}{0pt}%
\pgfpathmoveto{\pgfqpoint{3.832513in}{2.118500in}}%
\pgfpathcurveto{\pgfqpoint{3.840749in}{2.118500in}}{\pgfqpoint{3.848649in}{2.121773in}}{\pgfqpoint{3.854473in}{2.127597in}}%
\pgfpathcurveto{\pgfqpoint{3.860297in}{2.133420in}}{\pgfqpoint{3.863569in}{2.141321in}}{\pgfqpoint{3.863569in}{2.149557in}}%
\pgfpathcurveto{\pgfqpoint{3.863569in}{2.157793in}}{\pgfqpoint{3.860297in}{2.165693in}}{\pgfqpoint{3.854473in}{2.171517in}}%
\pgfpathcurveto{\pgfqpoint{3.848649in}{2.177341in}}{\pgfqpoint{3.840749in}{2.180613in}}{\pgfqpoint{3.832513in}{2.180613in}}%
\pgfpathcurveto{\pgfqpoint{3.824276in}{2.180613in}}{\pgfqpoint{3.816376in}{2.177341in}}{\pgfqpoint{3.810552in}{2.171517in}}%
\pgfpathcurveto{\pgfqpoint{3.804728in}{2.165693in}}{\pgfqpoint{3.801456in}{2.157793in}}{\pgfqpoint{3.801456in}{2.149557in}}%
\pgfpathcurveto{\pgfqpoint{3.801456in}{2.141321in}}{\pgfqpoint{3.804728in}{2.133420in}}{\pgfqpoint{3.810552in}{2.127597in}}%
\pgfpathcurveto{\pgfqpoint{3.816376in}{2.121773in}}{\pgfqpoint{3.824276in}{2.118500in}}{\pgfqpoint{3.832513in}{2.118500in}}%
\pgfpathclose%
\pgfusepath{stroke,fill}%
\end{pgfscope}%
\begin{pgfscope}%
\pgfpathrectangle{\pgfqpoint{3.717870in}{0.557870in}}{\pgfqpoint{2.522130in}{1.684734in}}%
\pgfusepath{clip}%
\pgfsetbuttcap%
\pgfsetroundjoin%
\definecolor{currentfill}{rgb}{0.298039,0.447059,0.690196}%
\pgfsetfillcolor{currentfill}%
\pgfsetlinewidth{1.003750pt}%
\definecolor{currentstroke}{rgb}{0.298039,0.447059,0.690196}%
\pgfsetstrokecolor{currentstroke}%
\pgfsetdash{}{0pt}%
\pgfpathmoveto{\pgfqpoint{5.218298in}{1.887940in}}%
\pgfpathcurveto{\pgfqpoint{5.226534in}{1.887940in}}{\pgfqpoint{5.234434in}{1.891213in}}{\pgfqpoint{5.240258in}{1.897037in}}%
\pgfpathcurveto{\pgfqpoint{5.246082in}{1.902861in}}{\pgfqpoint{5.249355in}{1.910761in}}{\pgfqpoint{5.249355in}{1.918997in}}%
\pgfpathcurveto{\pgfqpoint{5.249355in}{1.927233in}}{\pgfqpoint{5.246082in}{1.935133in}}{\pgfqpoint{5.240258in}{1.940957in}}%
\pgfpathcurveto{\pgfqpoint{5.234434in}{1.946781in}}{\pgfqpoint{5.226534in}{1.950053in}}{\pgfqpoint{5.218298in}{1.950053in}}%
\pgfpathcurveto{\pgfqpoint{5.210062in}{1.950053in}}{\pgfqpoint{5.202162in}{1.946781in}}{\pgfqpoint{5.196338in}{1.940957in}}%
\pgfpathcurveto{\pgfqpoint{5.190514in}{1.935133in}}{\pgfqpoint{5.187242in}{1.927233in}}{\pgfqpoint{5.187242in}{1.918997in}}%
\pgfpathcurveto{\pgfqpoint{5.187242in}{1.910761in}}{\pgfqpoint{5.190514in}{1.902861in}}{\pgfqpoint{5.196338in}{1.897037in}}%
\pgfpathcurveto{\pgfqpoint{5.202162in}{1.891213in}}{\pgfqpoint{5.210062in}{1.887940in}}{\pgfqpoint{5.218298in}{1.887940in}}%
\pgfpathclose%
\pgfusepath{stroke,fill}%
\end{pgfscope}%
\begin{pgfscope}%
\pgfsetrectcap%
\pgfsetmiterjoin%
\pgfsetlinewidth{1.254687pt}%
\definecolor{currentstroke}{rgb}{1.000000,1.000000,1.000000}%
\pgfsetstrokecolor{currentstroke}%
\pgfsetdash{}{0pt}%
\pgfpathmoveto{\pgfqpoint{3.717870in}{0.557870in}}%
\pgfpathlineto{\pgfqpoint{3.717870in}{2.242604in}}%
\pgfusepath{stroke}%
\end{pgfscope}%
\begin{pgfscope}%
\pgfsetrectcap%
\pgfsetmiterjoin%
\pgfsetlinewidth{1.254687pt}%
\definecolor{currentstroke}{rgb}{1.000000,1.000000,1.000000}%
\pgfsetstrokecolor{currentstroke}%
\pgfsetdash{}{0pt}%
\pgfpathmoveto{\pgfqpoint{6.240000in}{0.557870in}}%
\pgfpathlineto{\pgfqpoint{6.240000in}{2.242604in}}%
\pgfusepath{stroke}%
\end{pgfscope}%
\begin{pgfscope}%
\pgfsetrectcap%
\pgfsetmiterjoin%
\pgfsetlinewidth{1.254687pt}%
\definecolor{currentstroke}{rgb}{1.000000,1.000000,1.000000}%
\pgfsetstrokecolor{currentstroke}%
\pgfsetdash{}{0pt}%
\pgfpathmoveto{\pgfqpoint{3.717870in}{0.557870in}}%
\pgfpathlineto{\pgfqpoint{6.240000in}{0.557870in}}%
\pgfusepath{stroke}%
\end{pgfscope}%
\begin{pgfscope}%
\pgfsetrectcap%
\pgfsetmiterjoin%
\pgfsetlinewidth{1.254687pt}%
\definecolor{currentstroke}{rgb}{1.000000,1.000000,1.000000}%
\pgfsetstrokecolor{currentstroke}%
\pgfsetdash{}{0pt}%
\pgfpathmoveto{\pgfqpoint{3.717870in}{2.242604in}}%
\pgfpathlineto{\pgfqpoint{6.240000in}{2.242604in}}%
\pgfusepath{stroke}%
\end{pgfscope}%
\begin{pgfscope}%
\definecolor{textcolor}{rgb}{0.150000,0.150000,0.150000}%
\pgfsetstrokecolor{textcolor}%
\pgfsetfillcolor{textcolor}%
\pgftext[x=4.978935in,y=2.325938in,,base]{\color{textcolor}\sffamily\fontsize{11.000000}{13.200000}\selectfont (b)}%
\end{pgfscope}%
\end{pgfpicture}%
\makeatother%
\endgroup%

    \caption{(a) Distribution plot of \acrshort{dor} of all PVC models evaluated at two cluster centers when applied to classify heart failure.
             (b) Scatter plot of the same models sensitivity, and specificity.}
    \label{fig:pvc_hf_dor_sens_spec_dist}
\end{figure}

From figure \ref{fig:pvc_hf_dor_sens_spec_dist}a one can see that the majority of \acrshort{dor} scores are centered around zero, but there is a substantial number of models that acheive a \acrshort{dor} score above 10. The scatterplot in figure \ref{fig:pvc_hf_dor_sens_spec_dist}b shows that there is also a great spread in sensitivity, and specificity. A few models are spread along the edges of the plot acheiving a sensitivity or specificity score close to zero, but there are also models that aceive sensitivity and specificity scores above $0.7$. Common to the highest performing PVC models is that they all use the dataset that is a combination of peak systolic \acrshort{gls} values and \acrshort{ef} values. This can be confirmed from the complete table of results in the appendix \ref{tab:pvc_hf_raw_results}. From table \ref{tab:pvc_hf_dor_sens_spec_dist} one can see that \textit{gls-EF/ward/2} is the PVC model that acheives the highest \acrshort{dor} of $11.59$ when applied to classify heart failure. The \textit{gls-EF/complete/2} model acheives the second highest \acrshort{dor} of $10.85$, but its' specificity is nine points higher than \textit{gls-EF/ward/2}, while its sensitivity is only six points lower, and it also has the highest accuracy of all the PVC models applied to identify heart failure. \bigskip

\begin{table*}[htb]
    \centering
    \ra{1.3}
    \begin{tabular}{lrrrr}
        \toprule
        Dataset-model    &  Accuracy &  Sensitivity &  Specificity &   \acrshort{dor} \\
        \midrule
        gls-EF/ward/2     &      0.75 &         0.87 &         0.63 & 11.59 \\
        gls-EF/complete/2 &      0.76 &         0.81 &         0.72 & 10.85 \\
        gls-EF/average/2  &      0.75 &         0.85 &         0.65 & 10.58 \\
        rls-EF/complete/2 &      0.73 &         0.86 &         0.60 &  8.89 \\
        gls-rls-EF/ward/2 &      0.72 &         0.84 &         0.60 &  7.80 \\
        \bottomrule
    \end{tabular}
    \caption{The accuracy, \acrshort{dor}, sensitivity and specicity scores of the five best performing two-cluster-center PVC models in terms of \acrshort{dor}, at detecting heart failure.
             The \textbf{Dataset-model} column indicates \textit{Dataset used}$/$\textit{Linkage criteria of model}$/$\textit{Number of cluster centers}.}
    \label{tab:pvc_hf_dor_sens_spec_dist}
\end{table*}

\begin{table*}[htb]
    \centering
    \ra{1.3}
    \begin{tabular}{lr}
        \toprule
        Dataset-model    &  \acrshort{ari} \\
        \midrule
        gls-EF/complete/2 & 0.27 \\
        gls-EF/ward/2     & 0.24 \\
        gls-EF/average/2  & 0.24 \\
        rls-EF/complete/2 & 0.21 \\
        gls-EF/complete/3 & 0.21 \\
        \bottomrule
    \end{tabular}
    \caption{The five highest \acrshort{ari} scores attained when applying PVC for detecting heart failure.
             The \textbf{Dataset-model} column indicates \textit{Dataset used}$/$\textit{Linkage criteria of model}$/$\textit{Number of cluster centers}.}
    \label{tab:pvc_hf_ari}
\end{table*}

\newpage

Many of the \acrshort{ari} of PVC models for classifying heart failure are close to zero, but substantially more of the models score above zero in \acrshort{ari}. As with \acrshort{dor}, the models that acheive the highest \acrshort{ari} scores use datasets that are combinations of strain curves and \acrshort{ef} values. Table \ref{tab:pvc_hf_ari} shows that the three highest \acrshort{ari}s are attained by the same three models that acheived the highest \acrshort{dor}. This means that there are most likely no models evaluated at a higher number of cluster centers that will outperform \textit{ward/2}, or \textit{complete/2} at classifying heart failure. However, \textit{complete/2} acheives the highest \acrshort{ari}, although it only acheives the second highest \acrshort{dor}. \textit{complete/2} is chosen as the best performing PVC model when classifying heart failure, since it has the highest accuracy (76$\%$), highest \acrshort{ari} (0.27), and second highest \acrshort{dor} (10.85). In figure \ref{fig:scatter_gls_ef_hf_cluster_assignments} scatterplots patients are plotted with the dimensions: 4-chamber peak systolic \acrshort{gls}, 2-chamber peak systolic \acrshort{gls} and \acrshort{ef}. The colors of the points correspond to wheather the patient has heart failure or not, and which cluster the points belong to. The plots are actually a lower dimensional projection of the \acrshort{gls}-EF peak-value dataset. This particular projection was chosen as it was found to be the projection where heart failure patients were as separable as possible. From plots \ref{fig:scatter_gls_ef_hf_cluster_assignments}b-d one can see that the clusters are fairly separable, heart failure on the other hand is not as easy to separate in these dimensions as can be seen in plot \ref{fig:scatter_gls_ef_average2}. \textit{Ward/2} and \textit{complete/2} can in some sense be considered as binary classifiers where values under a certain threshold are categorized as heart failure.The \textit{ward/2} model has the highest threshold for what is considered heart failure, and \textit{complete/2} has the lowest, which explains their difference in sensitivity and specificity score. Since model \textit{complete/2} acheives the highest accuracy ($0.76$), highest \acrshort{ari} ($0.27$) and second highest \acrshort{dor} ($10.85$) it is chosen as the best PVC model to identify heart failure among patients. \bigskip

\begin{figure}[htb]
    \centering
    \begin{subfigure}[b]{0.49\textwidth}
        \centering
        \includegraphics[width=0.99\textwidth]{results/hf/scatter_gls_EF_hf.png}
        \caption{Heart failure.}
        \label{fig:scatter_gls_ef_hf}
    \end{subfigure}
    \begin{subfigure}[b]{0.49\textwidth}
        \centering
        \includegraphics[width=0.99\textwidth]{results/hf/scatter_gls_EF_ward2.png}
        \caption{\textit{Ward/2} cluster assignments.}
        \label{fig:scatter_gls_ef_ward2}
    \end{subfigure}\\
    \begin{subfigure}[b]{0.49\textwidth}
        \centering
        \includegraphics[width=0.99\textwidth]{results/hf/scatter_gls_EF_complete2.png}
        \caption{\textit{Complete/2} cluster assignments.}
        \label{fig:scatter_gls_ef_complete2}
    \end{subfigure}
    \begin{subfigure}[b]{0.49\textwidth}
        \centering
        \includegraphics[width=0.99\textwidth]{results/hf/scatter_gls_EF_average2.png}
        \caption{\textit{Average/2} cluster assignments.}
        \label{fig:scatter_gls_ef_average2}
    \end{subfigure}
    \caption{Scatterplot of peak \acrshort{gls} values in each view. Colors in the of the different dots are given by heart failure diagnosis, and cluster assignments of 
             ward/2, complete/2 and average/2 models. Numbers are not included on the axes because the point of the figure is to illustrate the separability 
             of clusters, and heart failure.}
             \label{fig:scatter_gls_ef_hf_cluster_assignments}
\end{figure}

\clearpage

\subsection{Deep Neural Network}

\begin{figure}[H]
    \centering
    % \includegraphics[width=\textwidth]{results/dl_hf_dor_sens_spec_dist.png}
    %% Creator: Matplotlib, PGF backend
%%
%% To include the figure in your LaTeX document, write
%%   \input{<filename>.pgf}
%%
%% Make sure the required packages are loaded in your preamble
%%   \usepackage{pgf}
%%
%% Figures using additional raster images can only be included by \input if
%% they are in the same directory as the main LaTeX file. For loading figures
%% from other directories you can use the `import` package
%%   \usepackage{import}
%% and then include the figures with
%%   \import{<path to file>}{<filename>.pgf}
%%
%% Matplotlib used the following preamble
%%
\begingroup%
\makeatletter%
\begin{pgfpicture}%
\pgfpathrectangle{\pgfpointorigin}{\pgfqpoint{6.246672in}{2.540000in}}%
\pgfusepath{use as bounding box, clip}%
\begin{pgfscope}%
\pgfsetbuttcap%
\pgfsetmiterjoin%
\definecolor{currentfill}{rgb}{1.000000,1.000000,1.000000}%
\pgfsetfillcolor{currentfill}%
\pgfsetlinewidth{0.000000pt}%
\definecolor{currentstroke}{rgb}{1.000000,1.000000,1.000000}%
\pgfsetstrokecolor{currentstroke}%
\pgfsetdash{}{0pt}%
\pgfpathmoveto{\pgfqpoint{0.000000in}{0.000000in}}%
\pgfpathlineto{\pgfqpoint{6.246672in}{0.000000in}}%
\pgfpathlineto{\pgfqpoint{6.246672in}{2.540000in}}%
\pgfpathlineto{\pgfqpoint{0.000000in}{2.540000in}}%
\pgfpathclose%
\pgfusepath{fill}%
\end{pgfscope}%
\begin{pgfscope}%
\pgfsetbuttcap%
\pgfsetmiterjoin%
\definecolor{currentfill}{rgb}{0.917647,0.917647,0.949020}%
\pgfsetfillcolor{currentfill}%
\pgfsetlinewidth{0.000000pt}%
\definecolor{currentstroke}{rgb}{0.000000,0.000000,0.000000}%
\pgfsetstrokecolor{currentstroke}%
\pgfsetstrokeopacity{0.000000}%
\pgfsetdash{}{0pt}%
\pgfpathmoveto{\pgfqpoint{0.574769in}{0.557870in}}%
\pgfpathlineto{\pgfqpoint{2.999734in}{0.557870in}}%
\pgfpathlineto{\pgfqpoint{2.999734in}{2.242604in}}%
\pgfpathlineto{\pgfqpoint{0.574769in}{2.242604in}}%
\pgfpathclose%
\pgfusepath{fill}%
\end{pgfscope}%
\begin{pgfscope}%
\pgfpathrectangle{\pgfqpoint{0.574769in}{0.557870in}}{\pgfqpoint{2.424965in}{1.684734in}}%
\pgfusepath{clip}%
\pgfsetroundcap%
\pgfsetroundjoin%
\pgfsetlinewidth{1.003750pt}%
\definecolor{currentstroke}{rgb}{1.000000,1.000000,1.000000}%
\pgfsetstrokecolor{currentstroke}%
\pgfsetdash{}{0pt}%
\pgfpathmoveto{\pgfqpoint{0.684994in}{0.557870in}}%
\pgfpathlineto{\pgfqpoint{0.684994in}{2.242604in}}%
\pgfusepath{stroke}%
\end{pgfscope}%
\begin{pgfscope}%
\definecolor{textcolor}{rgb}{0.150000,0.150000,0.150000}%
\pgfsetstrokecolor{textcolor}%
\pgfsetfillcolor{textcolor}%
\pgftext[x=0.684994in,y=0.425926in,,top]{\color{textcolor}\sffamily\fontsize{11.000000}{13.200000}\selectfont \(\displaystyle 0.0\)}%
\end{pgfscope}%
\begin{pgfscope}%
\pgfpathrectangle{\pgfqpoint{0.574769in}{0.557870in}}{\pgfqpoint{2.424965in}{1.684734in}}%
\pgfusepath{clip}%
\pgfsetroundcap%
\pgfsetroundjoin%
\pgfsetlinewidth{1.003750pt}%
\definecolor{currentstroke}{rgb}{1.000000,1.000000,1.000000}%
\pgfsetstrokecolor{currentstroke}%
\pgfsetdash{}{0pt}%
\pgfpathmoveto{\pgfqpoint{1.496956in}{0.557870in}}%
\pgfpathlineto{\pgfqpoint{1.496956in}{2.242604in}}%
\pgfusepath{stroke}%
\end{pgfscope}%
\begin{pgfscope}%
\definecolor{textcolor}{rgb}{0.150000,0.150000,0.150000}%
\pgfsetstrokecolor{textcolor}%
\pgfsetfillcolor{textcolor}%
\pgftext[x=1.496956in,y=0.425926in,,top]{\color{textcolor}\sffamily\fontsize{11.000000}{13.200000}\selectfont \(\displaystyle 0.5\)}%
\end{pgfscope}%
\begin{pgfscope}%
\pgfpathrectangle{\pgfqpoint{0.574769in}{0.557870in}}{\pgfqpoint{2.424965in}{1.684734in}}%
\pgfusepath{clip}%
\pgfsetroundcap%
\pgfsetroundjoin%
\pgfsetlinewidth{1.003750pt}%
\definecolor{currentstroke}{rgb}{1.000000,1.000000,1.000000}%
\pgfsetstrokecolor{currentstroke}%
\pgfsetdash{}{0pt}%
\pgfpathmoveto{\pgfqpoint{2.308918in}{0.557870in}}%
\pgfpathlineto{\pgfqpoint{2.308918in}{2.242604in}}%
\pgfusepath{stroke}%
\end{pgfscope}%
\begin{pgfscope}%
\definecolor{textcolor}{rgb}{0.150000,0.150000,0.150000}%
\pgfsetstrokecolor{textcolor}%
\pgfsetfillcolor{textcolor}%
\pgftext[x=2.308918in,y=0.425926in,,top]{\color{textcolor}\sffamily\fontsize{11.000000}{13.200000}\selectfont \(\displaystyle 1.0\)}%
\end{pgfscope}%
\begin{pgfscope}%
\definecolor{textcolor}{rgb}{0.150000,0.150000,0.150000}%
\pgfsetstrokecolor{textcolor}%
\pgfsetfillcolor{textcolor}%
\pgftext[x=1.787251in,y=0.235185in,,top]{\color{textcolor}\sffamily\fontsize{11.000000}{13.200000}\selectfont DOR}%
\end{pgfscope}%
\begin{pgfscope}%
\pgfpathrectangle{\pgfqpoint{0.574769in}{0.557870in}}{\pgfqpoint{2.424965in}{1.684734in}}%
\pgfusepath{clip}%
\pgfsetroundcap%
\pgfsetroundjoin%
\pgfsetlinewidth{1.003750pt}%
\definecolor{currentstroke}{rgb}{1.000000,1.000000,1.000000}%
\pgfsetstrokecolor{currentstroke}%
\pgfsetdash{}{0pt}%
\pgfpathmoveto{\pgfqpoint{0.574769in}{0.557870in}}%
\pgfpathlineto{\pgfqpoint{2.999734in}{0.557870in}}%
\pgfusepath{stroke}%
\end{pgfscope}%
\begin{pgfscope}%
\definecolor{textcolor}{rgb}{0.150000,0.150000,0.150000}%
\pgfsetstrokecolor{textcolor}%
\pgfsetfillcolor{textcolor}%
\pgftext[x=0.366783in,y=0.505064in,left,base]{\color{textcolor}\sffamily\fontsize{11.000000}{13.200000}\selectfont \(\displaystyle 0\)}%
\end{pgfscope}%
\begin{pgfscope}%
\pgfpathrectangle{\pgfqpoint{0.574769in}{0.557870in}}{\pgfqpoint{2.424965in}{1.684734in}}%
\pgfusepath{clip}%
\pgfsetroundcap%
\pgfsetroundjoin%
\pgfsetlinewidth{1.003750pt}%
\definecolor{currentstroke}{rgb}{1.000000,1.000000,1.000000}%
\pgfsetstrokecolor{currentstroke}%
\pgfsetdash{}{0pt}%
\pgfpathmoveto{\pgfqpoint{0.574769in}{1.174989in}}%
\pgfpathlineto{\pgfqpoint{2.999734in}{1.174989in}}%
\pgfusepath{stroke}%
\end{pgfscope}%
\begin{pgfscope}%
\definecolor{textcolor}{rgb}{0.150000,0.150000,0.150000}%
\pgfsetstrokecolor{textcolor}%
\pgfsetfillcolor{textcolor}%
\pgftext[x=0.366783in,y=1.122182in,left,base]{\color{textcolor}\sffamily\fontsize{11.000000}{13.200000}\selectfont \(\displaystyle 5\)}%
\end{pgfscope}%
\begin{pgfscope}%
\pgfpathrectangle{\pgfqpoint{0.574769in}{0.557870in}}{\pgfqpoint{2.424965in}{1.684734in}}%
\pgfusepath{clip}%
\pgfsetroundcap%
\pgfsetroundjoin%
\pgfsetlinewidth{1.003750pt}%
\definecolor{currentstroke}{rgb}{1.000000,1.000000,1.000000}%
\pgfsetstrokecolor{currentstroke}%
\pgfsetdash{}{0pt}%
\pgfpathmoveto{\pgfqpoint{0.574769in}{1.792108in}}%
\pgfpathlineto{\pgfqpoint{2.999734in}{1.792108in}}%
\pgfusepath{stroke}%
\end{pgfscope}%
\begin{pgfscope}%
\definecolor{textcolor}{rgb}{0.150000,0.150000,0.150000}%
\pgfsetstrokecolor{textcolor}%
\pgfsetfillcolor{textcolor}%
\pgftext[x=0.290741in,y=1.739301in,left,base]{\color{textcolor}\sffamily\fontsize{11.000000}{13.200000}\selectfont \(\displaystyle 10\)}%
\end{pgfscope}%
\begin{pgfscope}%
\definecolor{textcolor}{rgb}{0.150000,0.150000,0.150000}%
\pgfsetstrokecolor{textcolor}%
\pgfsetfillcolor{textcolor}%
\pgftext[x=0.235185in,y=1.400237in,,bottom,rotate=90.000000]{\color{textcolor}\sffamily\fontsize{11.000000}{13.200000}\selectfont Occurance}%
\end{pgfscope}%
\begin{pgfscope}%
\pgfpathrectangle{\pgfqpoint{0.574769in}{0.557870in}}{\pgfqpoint{2.424965in}{1.684734in}}%
\pgfusepath{clip}%
\pgfsetbuttcap%
\pgfsetmiterjoin%
\definecolor{currentfill}{rgb}{0.298039,0.447059,0.690196}%
\pgfsetfillcolor{currentfill}%
\pgfsetfillopacity{0.400000}%
\pgfsetlinewidth{1.003750pt}%
\definecolor{currentstroke}{rgb}{1.000000,1.000000,1.000000}%
\pgfsetstrokecolor{currentstroke}%
\pgfsetstrokeopacity{0.400000}%
\pgfsetdash{}{0pt}%
\pgfpathmoveto{\pgfqpoint{0.684994in}{0.557870in}}%
\pgfpathlineto{\pgfqpoint{0.905446in}{0.557870in}}%
\pgfpathlineto{\pgfqpoint{0.905446in}{2.162379in}}%
\pgfpathlineto{\pgfqpoint{0.684994in}{2.162379in}}%
\pgfpathclose%
\pgfusepath{stroke,fill}%
\end{pgfscope}%
\begin{pgfscope}%
\pgfpathrectangle{\pgfqpoint{0.574769in}{0.557870in}}{\pgfqpoint{2.424965in}{1.684734in}}%
\pgfusepath{clip}%
\pgfsetbuttcap%
\pgfsetmiterjoin%
\definecolor{currentfill}{rgb}{0.298039,0.447059,0.690196}%
\pgfsetfillcolor{currentfill}%
\pgfsetfillopacity{0.400000}%
\pgfsetlinewidth{1.003750pt}%
\definecolor{currentstroke}{rgb}{1.000000,1.000000,1.000000}%
\pgfsetstrokecolor{currentstroke}%
\pgfsetstrokeopacity{0.400000}%
\pgfsetdash{}{0pt}%
\pgfpathmoveto{\pgfqpoint{0.905446in}{0.557870in}}%
\pgfpathlineto{\pgfqpoint{1.125897in}{0.557870in}}%
\pgfpathlineto{\pgfqpoint{1.125897in}{1.174989in}}%
\pgfpathlineto{\pgfqpoint{0.905446in}{1.174989in}}%
\pgfpathclose%
\pgfusepath{stroke,fill}%
\end{pgfscope}%
\begin{pgfscope}%
\pgfpathrectangle{\pgfqpoint{0.574769in}{0.557870in}}{\pgfqpoint{2.424965in}{1.684734in}}%
\pgfusepath{clip}%
\pgfsetbuttcap%
\pgfsetmiterjoin%
\definecolor{currentfill}{rgb}{0.298039,0.447059,0.690196}%
\pgfsetfillcolor{currentfill}%
\pgfsetfillopacity{0.400000}%
\pgfsetlinewidth{1.003750pt}%
\definecolor{currentstroke}{rgb}{1.000000,1.000000,1.000000}%
\pgfsetstrokecolor{currentstroke}%
\pgfsetstrokeopacity{0.400000}%
\pgfsetdash{}{0pt}%
\pgfpathmoveto{\pgfqpoint{1.125897in}{0.557870in}}%
\pgfpathlineto{\pgfqpoint{1.346348in}{0.557870in}}%
\pgfpathlineto{\pgfqpoint{1.346348in}{1.298413in}}%
\pgfpathlineto{\pgfqpoint{1.125897in}{1.298413in}}%
\pgfpathclose%
\pgfusepath{stroke,fill}%
\end{pgfscope}%
\begin{pgfscope}%
\pgfpathrectangle{\pgfqpoint{0.574769in}{0.557870in}}{\pgfqpoint{2.424965in}{1.684734in}}%
\pgfusepath{clip}%
\pgfsetbuttcap%
\pgfsetmiterjoin%
\definecolor{currentfill}{rgb}{0.298039,0.447059,0.690196}%
\pgfsetfillcolor{currentfill}%
\pgfsetfillopacity{0.400000}%
\pgfsetlinewidth{1.003750pt}%
\definecolor{currentstroke}{rgb}{1.000000,1.000000,1.000000}%
\pgfsetstrokecolor{currentstroke}%
\pgfsetstrokeopacity{0.400000}%
\pgfsetdash{}{0pt}%
\pgfpathmoveto{\pgfqpoint{1.346348in}{0.557870in}}%
\pgfpathlineto{\pgfqpoint{1.566800in}{0.557870in}}%
\pgfpathlineto{\pgfqpoint{1.566800in}{0.928141in}}%
\pgfpathlineto{\pgfqpoint{1.346348in}{0.928141in}}%
\pgfpathclose%
\pgfusepath{stroke,fill}%
\end{pgfscope}%
\begin{pgfscope}%
\pgfpathrectangle{\pgfqpoint{0.574769in}{0.557870in}}{\pgfqpoint{2.424965in}{1.684734in}}%
\pgfusepath{clip}%
\pgfsetbuttcap%
\pgfsetmiterjoin%
\definecolor{currentfill}{rgb}{0.298039,0.447059,0.690196}%
\pgfsetfillcolor{currentfill}%
\pgfsetfillopacity{0.400000}%
\pgfsetlinewidth{1.003750pt}%
\definecolor{currentstroke}{rgb}{1.000000,1.000000,1.000000}%
\pgfsetstrokecolor{currentstroke}%
\pgfsetstrokeopacity{0.400000}%
\pgfsetdash{}{0pt}%
\pgfpathmoveto{\pgfqpoint{1.566800in}{0.557870in}}%
\pgfpathlineto{\pgfqpoint{1.787251in}{0.557870in}}%
\pgfpathlineto{\pgfqpoint{1.787251in}{0.557870in}}%
\pgfpathlineto{\pgfqpoint{1.566800in}{0.557870in}}%
\pgfpathclose%
\pgfusepath{stroke,fill}%
\end{pgfscope}%
\begin{pgfscope}%
\pgfpathrectangle{\pgfqpoint{0.574769in}{0.557870in}}{\pgfqpoint{2.424965in}{1.684734in}}%
\pgfusepath{clip}%
\pgfsetbuttcap%
\pgfsetmiterjoin%
\definecolor{currentfill}{rgb}{0.298039,0.447059,0.690196}%
\pgfsetfillcolor{currentfill}%
\pgfsetfillopacity{0.400000}%
\pgfsetlinewidth{1.003750pt}%
\definecolor{currentstroke}{rgb}{1.000000,1.000000,1.000000}%
\pgfsetstrokecolor{currentstroke}%
\pgfsetstrokeopacity{0.400000}%
\pgfsetdash{}{0pt}%
\pgfpathmoveto{\pgfqpoint{1.787251in}{0.557870in}}%
\pgfpathlineto{\pgfqpoint{2.007703in}{0.557870in}}%
\pgfpathlineto{\pgfqpoint{2.007703in}{0.681294in}}%
\pgfpathlineto{\pgfqpoint{1.787251in}{0.681294in}}%
\pgfpathclose%
\pgfusepath{stroke,fill}%
\end{pgfscope}%
\begin{pgfscope}%
\pgfpathrectangle{\pgfqpoint{0.574769in}{0.557870in}}{\pgfqpoint{2.424965in}{1.684734in}}%
\pgfusepath{clip}%
\pgfsetbuttcap%
\pgfsetmiterjoin%
\definecolor{currentfill}{rgb}{0.298039,0.447059,0.690196}%
\pgfsetfillcolor{currentfill}%
\pgfsetfillopacity{0.400000}%
\pgfsetlinewidth{1.003750pt}%
\definecolor{currentstroke}{rgb}{1.000000,1.000000,1.000000}%
\pgfsetstrokecolor{currentstroke}%
\pgfsetstrokeopacity{0.400000}%
\pgfsetdash{}{0pt}%
\pgfpathmoveto{\pgfqpoint{2.007703in}{0.557870in}}%
\pgfpathlineto{\pgfqpoint{2.228154in}{0.557870in}}%
\pgfpathlineto{\pgfqpoint{2.228154in}{0.804718in}}%
\pgfpathlineto{\pgfqpoint{2.007703in}{0.804718in}}%
\pgfpathclose%
\pgfusepath{stroke,fill}%
\end{pgfscope}%
\begin{pgfscope}%
\pgfpathrectangle{\pgfqpoint{0.574769in}{0.557870in}}{\pgfqpoint{2.424965in}{1.684734in}}%
\pgfusepath{clip}%
\pgfsetbuttcap%
\pgfsetmiterjoin%
\definecolor{currentfill}{rgb}{0.298039,0.447059,0.690196}%
\pgfsetfillcolor{currentfill}%
\pgfsetfillopacity{0.400000}%
\pgfsetlinewidth{1.003750pt}%
\definecolor{currentstroke}{rgb}{1.000000,1.000000,1.000000}%
\pgfsetstrokecolor{currentstroke}%
\pgfsetstrokeopacity{0.400000}%
\pgfsetdash{}{0pt}%
\pgfpathmoveto{\pgfqpoint{2.228154in}{0.557870in}}%
\pgfpathlineto{\pgfqpoint{2.448605in}{0.557870in}}%
\pgfpathlineto{\pgfqpoint{2.448605in}{0.804718in}}%
\pgfpathlineto{\pgfqpoint{2.228154in}{0.804718in}}%
\pgfpathclose%
\pgfusepath{stroke,fill}%
\end{pgfscope}%
\begin{pgfscope}%
\pgfpathrectangle{\pgfqpoint{0.574769in}{0.557870in}}{\pgfqpoint{2.424965in}{1.684734in}}%
\pgfusepath{clip}%
\pgfsetbuttcap%
\pgfsetmiterjoin%
\definecolor{currentfill}{rgb}{0.298039,0.447059,0.690196}%
\pgfsetfillcolor{currentfill}%
\pgfsetfillopacity{0.400000}%
\pgfsetlinewidth{1.003750pt}%
\definecolor{currentstroke}{rgb}{1.000000,1.000000,1.000000}%
\pgfsetstrokecolor{currentstroke}%
\pgfsetstrokeopacity{0.400000}%
\pgfsetdash{}{0pt}%
\pgfpathmoveto{\pgfqpoint{2.448605in}{0.557870in}}%
\pgfpathlineto{\pgfqpoint{2.669057in}{0.557870in}}%
\pgfpathlineto{\pgfqpoint{2.669057in}{0.804718in}}%
\pgfpathlineto{\pgfqpoint{2.448605in}{0.804718in}}%
\pgfpathclose%
\pgfusepath{stroke,fill}%
\end{pgfscope}%
\begin{pgfscope}%
\pgfpathrectangle{\pgfqpoint{0.574769in}{0.557870in}}{\pgfqpoint{2.424965in}{1.684734in}}%
\pgfusepath{clip}%
\pgfsetbuttcap%
\pgfsetmiterjoin%
\definecolor{currentfill}{rgb}{0.298039,0.447059,0.690196}%
\pgfsetfillcolor{currentfill}%
\pgfsetfillopacity{0.400000}%
\pgfsetlinewidth{1.003750pt}%
\definecolor{currentstroke}{rgb}{1.000000,1.000000,1.000000}%
\pgfsetstrokecolor{currentstroke}%
\pgfsetstrokeopacity{0.400000}%
\pgfsetdash{}{0pt}%
\pgfpathmoveto{\pgfqpoint{2.669057in}{0.557870in}}%
\pgfpathlineto{\pgfqpoint{2.889508in}{0.557870in}}%
\pgfpathlineto{\pgfqpoint{2.889508in}{0.804718in}}%
\pgfpathlineto{\pgfqpoint{2.669057in}{0.804718in}}%
\pgfpathclose%
\pgfusepath{stroke,fill}%
\end{pgfscope}%
\begin{pgfscope}%
\pgfsetrectcap%
\pgfsetmiterjoin%
\pgfsetlinewidth{1.254687pt}%
\definecolor{currentstroke}{rgb}{1.000000,1.000000,1.000000}%
\pgfsetstrokecolor{currentstroke}%
\pgfsetdash{}{0pt}%
\pgfpathmoveto{\pgfqpoint{0.574769in}{0.557870in}}%
\pgfpathlineto{\pgfqpoint{0.574769in}{2.242604in}}%
\pgfusepath{stroke}%
\end{pgfscope}%
\begin{pgfscope}%
\pgfsetrectcap%
\pgfsetmiterjoin%
\pgfsetlinewidth{1.254687pt}%
\definecolor{currentstroke}{rgb}{1.000000,1.000000,1.000000}%
\pgfsetstrokecolor{currentstroke}%
\pgfsetdash{}{0pt}%
\pgfpathmoveto{\pgfqpoint{2.999734in}{0.557870in}}%
\pgfpathlineto{\pgfqpoint{2.999734in}{2.242604in}}%
\pgfusepath{stroke}%
\end{pgfscope}%
\begin{pgfscope}%
\pgfsetrectcap%
\pgfsetmiterjoin%
\pgfsetlinewidth{1.254687pt}%
\definecolor{currentstroke}{rgb}{1.000000,1.000000,1.000000}%
\pgfsetstrokecolor{currentstroke}%
\pgfsetdash{}{0pt}%
\pgfpathmoveto{\pgfqpoint{0.574769in}{0.557870in}}%
\pgfpathlineto{\pgfqpoint{2.999734in}{0.557870in}}%
\pgfusepath{stroke}%
\end{pgfscope}%
\begin{pgfscope}%
\pgfsetrectcap%
\pgfsetmiterjoin%
\pgfsetlinewidth{1.254687pt}%
\definecolor{currentstroke}{rgb}{1.000000,1.000000,1.000000}%
\pgfsetstrokecolor{currentstroke}%
\pgfsetdash{}{0pt}%
\pgfpathmoveto{\pgfqpoint{0.574769in}{2.242604in}}%
\pgfpathlineto{\pgfqpoint{2.999734in}{2.242604in}}%
\pgfusepath{stroke}%
\end{pgfscope}%
\begin{pgfscope}%
\definecolor{textcolor}{rgb}{0.150000,0.150000,0.150000}%
\pgfsetstrokecolor{textcolor}%
\pgfsetfillcolor{textcolor}%
\pgftext[x=1.787251in,y=2.325938in,,base]{\color{textcolor}\sffamily\fontsize{11.000000}{13.200000}\selectfont (a)}%
\end{pgfscope}%
\begin{pgfscope}%
\pgfsetbuttcap%
\pgfsetmiterjoin%
\definecolor{currentfill}{rgb}{0.917647,0.917647,0.949020}%
\pgfsetfillcolor{currentfill}%
\pgfsetlinewidth{0.000000pt}%
\definecolor{currentstroke}{rgb}{0.000000,0.000000,0.000000}%
\pgfsetstrokecolor{currentstroke}%
\pgfsetstrokeopacity{0.000000}%
\pgfsetdash{}{0pt}%
\pgfpathmoveto{\pgfqpoint{3.696748in}{0.557870in}}%
\pgfpathlineto{\pgfqpoint{6.121713in}{0.557870in}}%
\pgfpathlineto{\pgfqpoint{6.121713in}{2.242604in}}%
\pgfpathlineto{\pgfqpoint{3.696748in}{2.242604in}}%
\pgfpathclose%
\pgfusepath{fill}%
\end{pgfscope}%
\begin{pgfscope}%
\pgfpathrectangle{\pgfqpoint{3.696748in}{0.557870in}}{\pgfqpoint{2.424965in}{1.684734in}}%
\pgfusepath{clip}%
\pgfsetroundcap%
\pgfsetroundjoin%
\pgfsetlinewidth{1.003750pt}%
\definecolor{currentstroke}{rgb}{1.000000,1.000000,1.000000}%
\pgfsetstrokecolor{currentstroke}%
\pgfsetdash{}{0pt}%
\pgfpathmoveto{\pgfqpoint{3.806973in}{0.557870in}}%
\pgfpathlineto{\pgfqpoint{3.806973in}{2.242604in}}%
\pgfusepath{stroke}%
\end{pgfscope}%
\begin{pgfscope}%
\definecolor{textcolor}{rgb}{0.150000,0.150000,0.150000}%
\pgfsetstrokecolor{textcolor}%
\pgfsetfillcolor{textcolor}%
\pgftext[x=3.806973in,y=0.425926in,,top]{\color{textcolor}\sffamily\fontsize{11.000000}{13.200000}\selectfont \(\displaystyle 0.00\)}%
\end{pgfscope}%
\begin{pgfscope}%
\pgfpathrectangle{\pgfqpoint{3.696748in}{0.557870in}}{\pgfqpoint{2.424965in}{1.684734in}}%
\pgfusepath{clip}%
\pgfsetroundcap%
\pgfsetroundjoin%
\pgfsetlinewidth{1.003750pt}%
\definecolor{currentstroke}{rgb}{1.000000,1.000000,1.000000}%
\pgfsetstrokecolor{currentstroke}%
\pgfsetdash{}{0pt}%
\pgfpathmoveto{\pgfqpoint{4.358102in}{0.557870in}}%
\pgfpathlineto{\pgfqpoint{4.358102in}{2.242604in}}%
\pgfusepath{stroke}%
\end{pgfscope}%
\begin{pgfscope}%
\definecolor{textcolor}{rgb}{0.150000,0.150000,0.150000}%
\pgfsetstrokecolor{textcolor}%
\pgfsetfillcolor{textcolor}%
\pgftext[x=4.358102in,y=0.425926in,,top]{\color{textcolor}\sffamily\fontsize{11.000000}{13.200000}\selectfont \(\displaystyle 0.25\)}%
\end{pgfscope}%
\begin{pgfscope}%
\pgfpathrectangle{\pgfqpoint{3.696748in}{0.557870in}}{\pgfqpoint{2.424965in}{1.684734in}}%
\pgfusepath{clip}%
\pgfsetroundcap%
\pgfsetroundjoin%
\pgfsetlinewidth{1.003750pt}%
\definecolor{currentstroke}{rgb}{1.000000,1.000000,1.000000}%
\pgfsetstrokecolor{currentstroke}%
\pgfsetdash{}{0pt}%
\pgfpathmoveto{\pgfqpoint{4.909230in}{0.557870in}}%
\pgfpathlineto{\pgfqpoint{4.909230in}{2.242604in}}%
\pgfusepath{stroke}%
\end{pgfscope}%
\begin{pgfscope}%
\definecolor{textcolor}{rgb}{0.150000,0.150000,0.150000}%
\pgfsetstrokecolor{textcolor}%
\pgfsetfillcolor{textcolor}%
\pgftext[x=4.909230in,y=0.425926in,,top]{\color{textcolor}\sffamily\fontsize{11.000000}{13.200000}\selectfont \(\displaystyle 0.50\)}%
\end{pgfscope}%
\begin{pgfscope}%
\pgfpathrectangle{\pgfqpoint{3.696748in}{0.557870in}}{\pgfqpoint{2.424965in}{1.684734in}}%
\pgfusepath{clip}%
\pgfsetroundcap%
\pgfsetroundjoin%
\pgfsetlinewidth{1.003750pt}%
\definecolor{currentstroke}{rgb}{1.000000,1.000000,1.000000}%
\pgfsetstrokecolor{currentstroke}%
\pgfsetdash{}{0pt}%
\pgfpathmoveto{\pgfqpoint{5.460359in}{0.557870in}}%
\pgfpathlineto{\pgfqpoint{5.460359in}{2.242604in}}%
\pgfusepath{stroke}%
\end{pgfscope}%
\begin{pgfscope}%
\definecolor{textcolor}{rgb}{0.150000,0.150000,0.150000}%
\pgfsetstrokecolor{textcolor}%
\pgfsetfillcolor{textcolor}%
\pgftext[x=5.460359in,y=0.425926in,,top]{\color{textcolor}\sffamily\fontsize{11.000000}{13.200000}\selectfont \(\displaystyle 0.75\)}%
\end{pgfscope}%
\begin{pgfscope}%
\pgfpathrectangle{\pgfqpoint{3.696748in}{0.557870in}}{\pgfqpoint{2.424965in}{1.684734in}}%
\pgfusepath{clip}%
\pgfsetroundcap%
\pgfsetroundjoin%
\pgfsetlinewidth{1.003750pt}%
\definecolor{currentstroke}{rgb}{1.000000,1.000000,1.000000}%
\pgfsetstrokecolor{currentstroke}%
\pgfsetdash{}{0pt}%
\pgfpathmoveto{\pgfqpoint{6.011487in}{0.557870in}}%
\pgfpathlineto{\pgfqpoint{6.011487in}{2.242604in}}%
\pgfusepath{stroke}%
\end{pgfscope}%
\begin{pgfscope}%
\definecolor{textcolor}{rgb}{0.150000,0.150000,0.150000}%
\pgfsetstrokecolor{textcolor}%
\pgfsetfillcolor{textcolor}%
\pgftext[x=6.011487in,y=0.425926in,,top]{\color{textcolor}\sffamily\fontsize{11.000000}{13.200000}\selectfont \(\displaystyle 1.00\)}%
\end{pgfscope}%
\begin{pgfscope}%
\definecolor{textcolor}{rgb}{0.150000,0.150000,0.150000}%
\pgfsetstrokecolor{textcolor}%
\pgfsetfillcolor{textcolor}%
\pgftext[x=4.909230in,y=0.235185in,,top]{\color{textcolor}\sffamily\fontsize{11.000000}{13.200000}\selectfont Specificity}%
\end{pgfscope}%
\begin{pgfscope}%
\pgfpathrectangle{\pgfqpoint{3.696748in}{0.557870in}}{\pgfqpoint{2.424965in}{1.684734in}}%
\pgfusepath{clip}%
\pgfsetroundcap%
\pgfsetroundjoin%
\pgfsetlinewidth{1.003750pt}%
\definecolor{currentstroke}{rgb}{1.000000,1.000000,1.000000}%
\pgfsetstrokecolor{currentstroke}%
\pgfsetdash{}{0pt}%
\pgfpathmoveto{\pgfqpoint{3.696748in}{0.634449in}}%
\pgfpathlineto{\pgfqpoint{6.121713in}{0.634449in}}%
\pgfusepath{stroke}%
\end{pgfscope}%
\begin{pgfscope}%
\definecolor{textcolor}{rgb}{0.150000,0.150000,0.150000}%
\pgfsetstrokecolor{textcolor}%
\pgfsetfillcolor{textcolor}%
\pgftext[x=3.294433in,y=0.581642in,left,base]{\color{textcolor}\sffamily\fontsize{11.000000}{13.200000}\selectfont \(\displaystyle 0.00\)}%
\end{pgfscope}%
\begin{pgfscope}%
\pgfpathrectangle{\pgfqpoint{3.696748in}{0.557870in}}{\pgfqpoint{2.424965in}{1.684734in}}%
\pgfusepath{clip}%
\pgfsetroundcap%
\pgfsetroundjoin%
\pgfsetlinewidth{1.003750pt}%
\definecolor{currentstroke}{rgb}{1.000000,1.000000,1.000000}%
\pgfsetstrokecolor{currentstroke}%
\pgfsetdash{}{0pt}%
\pgfpathmoveto{\pgfqpoint{3.696748in}{1.017343in}}%
\pgfpathlineto{\pgfqpoint{6.121713in}{1.017343in}}%
\pgfusepath{stroke}%
\end{pgfscope}%
\begin{pgfscope}%
\definecolor{textcolor}{rgb}{0.150000,0.150000,0.150000}%
\pgfsetstrokecolor{textcolor}%
\pgfsetfillcolor{textcolor}%
\pgftext[x=3.294433in,y=0.964536in,left,base]{\color{textcolor}\sffamily\fontsize{11.000000}{13.200000}\selectfont \(\displaystyle 0.25\)}%
\end{pgfscope}%
\begin{pgfscope}%
\pgfpathrectangle{\pgfqpoint{3.696748in}{0.557870in}}{\pgfqpoint{2.424965in}{1.684734in}}%
\pgfusepath{clip}%
\pgfsetroundcap%
\pgfsetroundjoin%
\pgfsetlinewidth{1.003750pt}%
\definecolor{currentstroke}{rgb}{1.000000,1.000000,1.000000}%
\pgfsetstrokecolor{currentstroke}%
\pgfsetdash{}{0pt}%
\pgfpathmoveto{\pgfqpoint{3.696748in}{1.400237in}}%
\pgfpathlineto{\pgfqpoint{6.121713in}{1.400237in}}%
\pgfusepath{stroke}%
\end{pgfscope}%
\begin{pgfscope}%
\definecolor{textcolor}{rgb}{0.150000,0.150000,0.150000}%
\pgfsetstrokecolor{textcolor}%
\pgfsetfillcolor{textcolor}%
\pgftext[x=3.294433in,y=1.347431in,left,base]{\color{textcolor}\sffamily\fontsize{11.000000}{13.200000}\selectfont \(\displaystyle 0.50\)}%
\end{pgfscope}%
\begin{pgfscope}%
\pgfpathrectangle{\pgfqpoint{3.696748in}{0.557870in}}{\pgfqpoint{2.424965in}{1.684734in}}%
\pgfusepath{clip}%
\pgfsetroundcap%
\pgfsetroundjoin%
\pgfsetlinewidth{1.003750pt}%
\definecolor{currentstroke}{rgb}{1.000000,1.000000,1.000000}%
\pgfsetstrokecolor{currentstroke}%
\pgfsetdash{}{0pt}%
\pgfpathmoveto{\pgfqpoint{3.696748in}{1.783131in}}%
\pgfpathlineto{\pgfqpoint{6.121713in}{1.783131in}}%
\pgfusepath{stroke}%
\end{pgfscope}%
\begin{pgfscope}%
\definecolor{textcolor}{rgb}{0.150000,0.150000,0.150000}%
\pgfsetstrokecolor{textcolor}%
\pgfsetfillcolor{textcolor}%
\pgftext[x=3.294433in,y=1.730325in,left,base]{\color{textcolor}\sffamily\fontsize{11.000000}{13.200000}\selectfont \(\displaystyle 0.75\)}%
\end{pgfscope}%
\begin{pgfscope}%
\pgfpathrectangle{\pgfqpoint{3.696748in}{0.557870in}}{\pgfqpoint{2.424965in}{1.684734in}}%
\pgfusepath{clip}%
\pgfsetroundcap%
\pgfsetroundjoin%
\pgfsetlinewidth{1.003750pt}%
\definecolor{currentstroke}{rgb}{1.000000,1.000000,1.000000}%
\pgfsetstrokecolor{currentstroke}%
\pgfsetdash{}{0pt}%
\pgfpathmoveto{\pgfqpoint{3.696748in}{2.166025in}}%
\pgfpathlineto{\pgfqpoint{6.121713in}{2.166025in}}%
\pgfusepath{stroke}%
\end{pgfscope}%
\begin{pgfscope}%
\definecolor{textcolor}{rgb}{0.150000,0.150000,0.150000}%
\pgfsetstrokecolor{textcolor}%
\pgfsetfillcolor{textcolor}%
\pgftext[x=3.294433in,y=2.113219in,left,base]{\color{textcolor}\sffamily\fontsize{11.000000}{13.200000}\selectfont \(\displaystyle 1.00\)}%
\end{pgfscope}%
\begin{pgfscope}%
\definecolor{textcolor}{rgb}{0.150000,0.150000,0.150000}%
\pgfsetstrokecolor{textcolor}%
\pgfsetfillcolor{textcolor}%
\pgftext[x=3.238877in,y=1.400237in,,bottom,rotate=90.000000]{\color{textcolor}\sffamily\fontsize{11.000000}{13.200000}\selectfont Sensitivity}%
\end{pgfscope}%
\begin{pgfscope}%
\pgfpathrectangle{\pgfqpoint{3.696748in}{0.557870in}}{\pgfqpoint{2.424965in}{1.684734in}}%
\pgfusepath{clip}%
\pgfsetbuttcap%
\pgfsetroundjoin%
\definecolor{currentfill}{rgb}{0.298039,0.447059,0.690196}%
\pgfsetfillcolor{currentfill}%
\pgfsetlinewidth{1.003750pt}%
\definecolor{currentstroke}{rgb}{0.298039,0.447059,0.690196}%
\pgfsetstrokecolor{currentstroke}%
\pgfsetdash{}{0pt}%
\pgfpathmoveto{\pgfqpoint{5.151727in}{1.315034in}}%
\pgfpathcurveto{\pgfqpoint{5.159963in}{1.315034in}}{\pgfqpoint{5.167863in}{1.318306in}}{\pgfqpoint{5.173687in}{1.324130in}}%
\pgfpathcurveto{\pgfqpoint{5.179511in}{1.329954in}}{\pgfqpoint{5.182783in}{1.337854in}}{\pgfqpoint{5.182783in}{1.346091in}}%
\pgfpathcurveto{\pgfqpoint{5.182783in}{1.354327in}}{\pgfqpoint{5.179511in}{1.362227in}}{\pgfqpoint{5.173687in}{1.368051in}}%
\pgfpathcurveto{\pgfqpoint{5.167863in}{1.373875in}}{\pgfqpoint{5.159963in}{1.377147in}}{\pgfqpoint{5.151727in}{1.377147in}}%
\pgfpathcurveto{\pgfqpoint{5.143490in}{1.377147in}}{\pgfqpoint{5.135590in}{1.373875in}}{\pgfqpoint{5.129766in}{1.368051in}}%
\pgfpathcurveto{\pgfqpoint{5.123943in}{1.362227in}}{\pgfqpoint{5.120670in}{1.354327in}}{\pgfqpoint{5.120670in}{1.346091in}}%
\pgfpathcurveto{\pgfqpoint{5.120670in}{1.337854in}}{\pgfqpoint{5.123943in}{1.329954in}}{\pgfqpoint{5.129766in}{1.324130in}}%
\pgfpathcurveto{\pgfqpoint{5.135590in}{1.318306in}}{\pgfqpoint{5.143490in}{1.315034in}}{\pgfqpoint{5.151727in}{1.315034in}}%
\pgfpathclose%
\pgfusepath{stroke,fill}%
\end{pgfscope}%
\begin{pgfscope}%
\pgfpathrectangle{\pgfqpoint{3.696748in}{0.557870in}}{\pgfqpoint{2.424965in}{1.684734in}}%
\pgfusepath{clip}%
\pgfsetbuttcap%
\pgfsetroundjoin%
\definecolor{currentfill}{rgb}{0.298039,0.447059,0.690196}%
\pgfsetfillcolor{currentfill}%
\pgfsetlinewidth{1.003750pt}%
\definecolor{currentstroke}{rgb}{0.298039,0.447059,0.690196}%
\pgfsetstrokecolor{currentstroke}%
\pgfsetdash{}{0pt}%
\pgfpathmoveto{\pgfqpoint{5.085591in}{1.345975in}}%
\pgfpathcurveto{\pgfqpoint{5.093828in}{1.345975in}}{\pgfqpoint{5.101728in}{1.349247in}}{\pgfqpoint{5.107552in}{1.355071in}}%
\pgfpathcurveto{\pgfqpoint{5.113376in}{1.360895in}}{\pgfqpoint{5.116648in}{1.368795in}}{\pgfqpoint{5.116648in}{1.377032in}}%
\pgfpathcurveto{\pgfqpoint{5.116648in}{1.385268in}}{\pgfqpoint{5.113376in}{1.393168in}}{\pgfqpoint{5.107552in}{1.398992in}}%
\pgfpathcurveto{\pgfqpoint{5.101728in}{1.404816in}}{\pgfqpoint{5.093828in}{1.408088in}}{\pgfqpoint{5.085591in}{1.408088in}}%
\pgfpathcurveto{\pgfqpoint{5.077355in}{1.408088in}}{\pgfqpoint{5.069455in}{1.404816in}}{\pgfqpoint{5.063631in}{1.398992in}}%
\pgfpathcurveto{\pgfqpoint{5.057807in}{1.393168in}}{\pgfqpoint{5.054535in}{1.385268in}}{\pgfqpoint{5.054535in}{1.377032in}}%
\pgfpathcurveto{\pgfqpoint{5.054535in}{1.368795in}}{\pgfqpoint{5.057807in}{1.360895in}}{\pgfqpoint{5.063631in}{1.355071in}}%
\pgfpathcurveto{\pgfqpoint{5.069455in}{1.349247in}}{\pgfqpoint{5.077355in}{1.345975in}}{\pgfqpoint{5.085591in}{1.345975in}}%
\pgfpathclose%
\pgfusepath{stroke,fill}%
\end{pgfscope}%
\begin{pgfscope}%
\pgfpathrectangle{\pgfqpoint{3.696748in}{0.557870in}}{\pgfqpoint{2.424965in}{1.684734in}}%
\pgfusepath{clip}%
\pgfsetbuttcap%
\pgfsetroundjoin%
\definecolor{currentfill}{rgb}{0.298039,0.447059,0.690196}%
\pgfsetfillcolor{currentfill}%
\pgfsetlinewidth{1.003750pt}%
\definecolor{currentstroke}{rgb}{0.298039,0.447059,0.690196}%
\pgfsetstrokecolor{currentstroke}%
\pgfsetdash{}{0pt}%
\pgfpathmoveto{\pgfqpoint{5.306043in}{1.154760in}}%
\pgfpathcurveto{\pgfqpoint{5.314279in}{1.154760in}}{\pgfqpoint{5.322179in}{1.158032in}}{\pgfqpoint{5.328003in}{1.163856in}}%
\pgfpathcurveto{\pgfqpoint{5.333827in}{1.169680in}}{\pgfqpoint{5.337099in}{1.177580in}}{\pgfqpoint{5.337099in}{1.185817in}}%
\pgfpathcurveto{\pgfqpoint{5.337099in}{1.194053in}}{\pgfqpoint{5.333827in}{1.201953in}}{\pgfqpoint{5.328003in}{1.207777in}}%
\pgfpathcurveto{\pgfqpoint{5.322179in}{1.213601in}}{\pgfqpoint{5.314279in}{1.216873in}}{\pgfqpoint{5.306043in}{1.216873in}}%
\pgfpathcurveto{\pgfqpoint{5.297806in}{1.216873in}}{\pgfqpoint{5.289906in}{1.213601in}}{\pgfqpoint{5.284082in}{1.207777in}}%
\pgfpathcurveto{\pgfqpoint{5.278259in}{1.201953in}}{\pgfqpoint{5.274986in}{1.194053in}}{\pgfqpoint{5.274986in}{1.185817in}}%
\pgfpathcurveto{\pgfqpoint{5.274986in}{1.177580in}}{\pgfqpoint{5.278259in}{1.169680in}}{\pgfqpoint{5.284082in}{1.163856in}}%
\pgfpathcurveto{\pgfqpoint{5.289906in}{1.158032in}}{\pgfqpoint{5.297806in}{1.154760in}}{\pgfqpoint{5.306043in}{1.154760in}}%
\pgfpathclose%
\pgfusepath{stroke,fill}%
\end{pgfscope}%
\begin{pgfscope}%
\pgfpathrectangle{\pgfqpoint{3.696748in}{0.557870in}}{\pgfqpoint{2.424965in}{1.684734in}}%
\pgfusepath{clip}%
\pgfsetbuttcap%
\pgfsetroundjoin%
\definecolor{currentfill}{rgb}{0.298039,0.447059,0.690196}%
\pgfsetfillcolor{currentfill}%
\pgfsetlinewidth{1.003750pt}%
\definecolor{currentstroke}{rgb}{0.298039,0.447059,0.690196}%
\pgfsetstrokecolor{currentstroke}%
\pgfsetdash{}{0pt}%
\pgfpathmoveto{\pgfqpoint{4.688779in}{1.572349in}}%
\pgfpathcurveto{\pgfqpoint{4.697015in}{1.572349in}}{\pgfqpoint{4.704915in}{1.575621in}}{\pgfqpoint{4.710739in}{1.581445in}}%
\pgfpathcurveto{\pgfqpoint{4.716563in}{1.587269in}}{\pgfqpoint{4.719835in}{1.595169in}}{\pgfqpoint{4.719835in}{1.603406in}}%
\pgfpathcurveto{\pgfqpoint{4.719835in}{1.611642in}}{\pgfqpoint{4.716563in}{1.619542in}}{\pgfqpoint{4.710739in}{1.625366in}}%
\pgfpathcurveto{\pgfqpoint{4.704915in}{1.631190in}}{\pgfqpoint{4.697015in}{1.634462in}}{\pgfqpoint{4.688779in}{1.634462in}}%
\pgfpathcurveto{\pgfqpoint{4.680543in}{1.634462in}}{\pgfqpoint{4.672643in}{1.631190in}}{\pgfqpoint{4.666819in}{1.625366in}}%
\pgfpathcurveto{\pgfqpoint{4.660995in}{1.619542in}}{\pgfqpoint{4.657722in}{1.611642in}}{\pgfqpoint{4.657722in}{1.603406in}}%
\pgfpathcurveto{\pgfqpoint{4.657722in}{1.595169in}}{\pgfqpoint{4.660995in}{1.587269in}}{\pgfqpoint{4.666819in}{1.581445in}}%
\pgfpathcurveto{\pgfqpoint{4.672643in}{1.575621in}}{\pgfqpoint{4.680543in}{1.572349in}}{\pgfqpoint{4.688779in}{1.572349in}}%
\pgfpathclose%
\pgfusepath{stroke,fill}%
\end{pgfscope}%
\begin{pgfscope}%
\pgfpathrectangle{\pgfqpoint{3.696748in}{0.557870in}}{\pgfqpoint{2.424965in}{1.684734in}}%
\pgfusepath{clip}%
\pgfsetbuttcap%
\pgfsetroundjoin%
\definecolor{currentfill}{rgb}{0.298039,0.447059,0.690196}%
\pgfsetfillcolor{currentfill}%
\pgfsetlinewidth{1.003750pt}%
\definecolor{currentstroke}{rgb}{0.298039,0.447059,0.690196}%
\pgfsetstrokecolor{currentstroke}%
\pgfsetdash{}{0pt}%
\pgfpathmoveto{\pgfqpoint{4.693324in}{1.531621in}}%
\pgfpathcurveto{\pgfqpoint{4.701561in}{1.531621in}}{\pgfqpoint{4.709461in}{1.534893in}}{\pgfqpoint{4.715285in}{1.540717in}}%
\pgfpathcurveto{\pgfqpoint{4.721108in}{1.546541in}}{\pgfqpoint{4.724381in}{1.554441in}}{\pgfqpoint{4.724381in}{1.562677in}}%
\pgfpathcurveto{\pgfqpoint{4.724381in}{1.570913in}}{\pgfqpoint{4.721108in}{1.578813in}}{\pgfqpoint{4.715285in}{1.584637in}}%
\pgfpathcurveto{\pgfqpoint{4.709461in}{1.590461in}}{\pgfqpoint{4.701561in}{1.593734in}}{\pgfqpoint{4.693324in}{1.593734in}}%
\pgfpathcurveto{\pgfqpoint{4.685088in}{1.593734in}}{\pgfqpoint{4.677188in}{1.590461in}}{\pgfqpoint{4.671364in}{1.584637in}}%
\pgfpathcurveto{\pgfqpoint{4.665540in}{1.578813in}}{\pgfqpoint{4.662268in}{1.570913in}}{\pgfqpoint{4.662268in}{1.562677in}}%
\pgfpathcurveto{\pgfqpoint{4.662268in}{1.554441in}}{\pgfqpoint{4.665540in}{1.546541in}}{\pgfqpoint{4.671364in}{1.540717in}}%
\pgfpathcurveto{\pgfqpoint{4.677188in}{1.534893in}}{\pgfqpoint{4.685088in}{1.531621in}}{\pgfqpoint{4.693324in}{1.531621in}}%
\pgfpathclose%
\pgfusepath{stroke,fill}%
\end{pgfscope}%
\begin{pgfscope}%
\pgfpathrectangle{\pgfqpoint{3.696748in}{0.557870in}}{\pgfqpoint{2.424965in}{1.684734in}}%
\pgfusepath{clip}%
\pgfsetbuttcap%
\pgfsetroundjoin%
\definecolor{currentfill}{rgb}{0.298039,0.447059,0.690196}%
\pgfsetfillcolor{currentfill}%
\pgfsetlinewidth{1.003750pt}%
\definecolor{currentstroke}{rgb}{0.298039,0.447059,0.690196}%
\pgfsetstrokecolor{currentstroke}%
\pgfsetdash{}{0pt}%
\pgfpathmoveto{\pgfqpoint{4.953321in}{1.345975in}}%
\pgfpathcurveto{\pgfqpoint{4.961557in}{1.345975in}}{\pgfqpoint{4.969457in}{1.349247in}}{\pgfqpoint{4.975281in}{1.355071in}}%
\pgfpathcurveto{\pgfqpoint{4.981105in}{1.360895in}}{\pgfqpoint{4.984377in}{1.368795in}}{\pgfqpoint{4.984377in}{1.377032in}}%
\pgfpathcurveto{\pgfqpoint{4.984377in}{1.385268in}}{\pgfqpoint{4.981105in}{1.393168in}}{\pgfqpoint{4.975281in}{1.398992in}}%
\pgfpathcurveto{\pgfqpoint{4.969457in}{1.404816in}}{\pgfqpoint{4.961557in}{1.408088in}}{\pgfqpoint{4.953321in}{1.408088in}}%
\pgfpathcurveto{\pgfqpoint{4.945084in}{1.408088in}}{\pgfqpoint{4.937184in}{1.404816in}}{\pgfqpoint{4.931360in}{1.398992in}}%
\pgfpathcurveto{\pgfqpoint{4.925536in}{1.393168in}}{\pgfqpoint{4.922264in}{1.385268in}}{\pgfqpoint{4.922264in}{1.377032in}}%
\pgfpathcurveto{\pgfqpoint{4.922264in}{1.368795in}}{\pgfqpoint{4.925536in}{1.360895in}}{\pgfqpoint{4.931360in}{1.355071in}}%
\pgfpathcurveto{\pgfqpoint{4.937184in}{1.349247in}}{\pgfqpoint{4.945084in}{1.345975in}}{\pgfqpoint{4.953321in}{1.345975in}}%
\pgfpathclose%
\pgfusepath{stroke,fill}%
\end{pgfscope}%
\begin{pgfscope}%
\pgfpathrectangle{\pgfqpoint{3.696748in}{0.557870in}}{\pgfqpoint{2.424965in}{1.684734in}}%
\pgfusepath{clip}%
\pgfsetbuttcap%
\pgfsetroundjoin%
\definecolor{currentfill}{rgb}{0.298039,0.447059,0.690196}%
\pgfsetfillcolor{currentfill}%
\pgfsetlinewidth{1.003750pt}%
\definecolor{currentstroke}{rgb}{0.298039,0.447059,0.690196}%
\pgfsetstrokecolor{currentstroke}%
\pgfsetdash{}{0pt}%
\pgfpathmoveto{\pgfqpoint{4.909230in}{1.345975in}}%
\pgfpathcurveto{\pgfqpoint{4.917467in}{1.345975in}}{\pgfqpoint{4.925367in}{1.349247in}}{\pgfqpoint{4.931191in}{1.355071in}}%
\pgfpathcurveto{\pgfqpoint{4.937014in}{1.360895in}}{\pgfqpoint{4.940287in}{1.368795in}}{\pgfqpoint{4.940287in}{1.377032in}}%
\pgfpathcurveto{\pgfqpoint{4.940287in}{1.385268in}}{\pgfqpoint{4.937014in}{1.393168in}}{\pgfqpoint{4.931191in}{1.398992in}}%
\pgfpathcurveto{\pgfqpoint{4.925367in}{1.404816in}}{\pgfqpoint{4.917467in}{1.408088in}}{\pgfqpoint{4.909230in}{1.408088in}}%
\pgfpathcurveto{\pgfqpoint{4.900994in}{1.408088in}}{\pgfqpoint{4.893094in}{1.404816in}}{\pgfqpoint{4.887270in}{1.398992in}}%
\pgfpathcurveto{\pgfqpoint{4.881446in}{1.393168in}}{\pgfqpoint{4.878174in}{1.385268in}}{\pgfqpoint{4.878174in}{1.377032in}}%
\pgfpathcurveto{\pgfqpoint{4.878174in}{1.368795in}}{\pgfqpoint{4.881446in}{1.360895in}}{\pgfqpoint{4.887270in}{1.355071in}}%
\pgfpathcurveto{\pgfqpoint{4.893094in}{1.349247in}}{\pgfqpoint{4.900994in}{1.345975in}}{\pgfqpoint{4.909230in}{1.345975in}}%
\pgfpathclose%
\pgfusepath{stroke,fill}%
\end{pgfscope}%
\begin{pgfscope}%
\pgfpathrectangle{\pgfqpoint{3.696748in}{0.557870in}}{\pgfqpoint{2.424965in}{1.684734in}}%
\pgfusepath{clip}%
\pgfsetbuttcap%
\pgfsetroundjoin%
\definecolor{currentfill}{rgb}{0.298039,0.447059,0.690196}%
\pgfsetfillcolor{currentfill}%
\pgfsetlinewidth{1.003750pt}%
\definecolor{currentstroke}{rgb}{0.298039,0.447059,0.690196}%
\pgfsetstrokecolor{currentstroke}%
\pgfsetdash{}{0pt}%
\pgfpathmoveto{\pgfqpoint{4.693324in}{1.476391in}}%
\pgfpathcurveto{\pgfqpoint{4.701561in}{1.476391in}}{\pgfqpoint{4.709461in}{1.479663in}}{\pgfqpoint{4.715285in}{1.485487in}}%
\pgfpathcurveto{\pgfqpoint{4.721108in}{1.491311in}}{\pgfqpoint{4.724381in}{1.499211in}}{\pgfqpoint{4.724381in}{1.507448in}}%
\pgfpathcurveto{\pgfqpoint{4.724381in}{1.515684in}}{\pgfqpoint{4.721108in}{1.523584in}}{\pgfqpoint{4.715285in}{1.529408in}}%
\pgfpathcurveto{\pgfqpoint{4.709461in}{1.535232in}}{\pgfqpoint{4.701561in}{1.538504in}}{\pgfqpoint{4.693324in}{1.538504in}}%
\pgfpathcurveto{\pgfqpoint{4.685088in}{1.538504in}}{\pgfqpoint{4.677188in}{1.535232in}}{\pgfqpoint{4.671364in}{1.529408in}}%
\pgfpathcurveto{\pgfqpoint{4.665540in}{1.523584in}}{\pgfqpoint{4.662268in}{1.515684in}}{\pgfqpoint{4.662268in}{1.507448in}}%
\pgfpathcurveto{\pgfqpoint{4.662268in}{1.499211in}}{\pgfqpoint{4.665540in}{1.491311in}}{\pgfqpoint{4.671364in}{1.485487in}}%
\pgfpathcurveto{\pgfqpoint{4.677188in}{1.479663in}}{\pgfqpoint{4.685088in}{1.476391in}}{\pgfqpoint{4.693324in}{1.476391in}}%
\pgfpathclose%
\pgfusepath{stroke,fill}%
\end{pgfscope}%
\begin{pgfscope}%
\pgfpathrectangle{\pgfqpoint{3.696748in}{0.557870in}}{\pgfqpoint{2.424965in}{1.684734in}}%
\pgfusepath{clip}%
\pgfsetbuttcap%
\pgfsetroundjoin%
\definecolor{currentfill}{rgb}{0.298039,0.447059,0.690196}%
\pgfsetfillcolor{currentfill}%
\pgfsetlinewidth{1.003750pt}%
\definecolor{currentstroke}{rgb}{0.298039,0.447059,0.690196}%
\pgfsetstrokecolor{currentstroke}%
\pgfsetdash{}{0pt}%
\pgfpathmoveto{\pgfqpoint{4.556508in}{1.537654in}}%
\pgfpathcurveto{\pgfqpoint{4.564744in}{1.537654in}}{\pgfqpoint{4.572644in}{1.540926in}}{\pgfqpoint{4.578468in}{1.546750in}}%
\pgfpathcurveto{\pgfqpoint{4.584292in}{1.552574in}}{\pgfqpoint{4.587565in}{1.560474in}}{\pgfqpoint{4.587565in}{1.568711in}}%
\pgfpathcurveto{\pgfqpoint{4.587565in}{1.576947in}}{\pgfqpoint{4.584292in}{1.584847in}}{\pgfqpoint{4.578468in}{1.590671in}}%
\pgfpathcurveto{\pgfqpoint{4.572644in}{1.596495in}}{\pgfqpoint{4.564744in}{1.599767in}}{\pgfqpoint{4.556508in}{1.599767in}}%
\pgfpathcurveto{\pgfqpoint{4.548272in}{1.599767in}}{\pgfqpoint{4.540372in}{1.596495in}}{\pgfqpoint{4.534548in}{1.590671in}}%
\pgfpathcurveto{\pgfqpoint{4.528724in}{1.584847in}}{\pgfqpoint{4.525452in}{1.576947in}}{\pgfqpoint{4.525452in}{1.568711in}}%
\pgfpathcurveto{\pgfqpoint{4.525452in}{1.560474in}}{\pgfqpoint{4.528724in}{1.552574in}}{\pgfqpoint{4.534548in}{1.546750in}}%
\pgfpathcurveto{\pgfqpoint{4.540372in}{1.540926in}}{\pgfqpoint{4.548272in}{1.537654in}}{\pgfqpoint{4.556508in}{1.537654in}}%
\pgfpathclose%
\pgfusepath{stroke,fill}%
\end{pgfscope}%
\begin{pgfscope}%
\pgfpathrectangle{\pgfqpoint{3.696748in}{0.557870in}}{\pgfqpoint{2.424965in}{1.684734in}}%
\pgfusepath{clip}%
\pgfsetbuttcap%
\pgfsetroundjoin%
\definecolor{currentfill}{rgb}{0.298039,0.447059,0.690196}%
\pgfsetfillcolor{currentfill}%
\pgfsetlinewidth{1.003750pt}%
\definecolor{currentstroke}{rgb}{0.298039,0.447059,0.690196}%
\pgfsetstrokecolor{currentstroke}%
\pgfsetdash{}{0pt}%
\pgfpathmoveto{\pgfqpoint{4.490373in}{1.399812in}}%
\pgfpathcurveto{\pgfqpoint{4.498609in}{1.399812in}}{\pgfqpoint{4.506509in}{1.403085in}}{\pgfqpoint{4.512333in}{1.408908in}}%
\pgfpathcurveto{\pgfqpoint{4.518157in}{1.414732in}}{\pgfqpoint{4.521429in}{1.422632in}}{\pgfqpoint{4.521429in}{1.430869in}}%
\pgfpathcurveto{\pgfqpoint{4.521429in}{1.439105in}}{\pgfqpoint{4.518157in}{1.447005in}}{\pgfqpoint{4.512333in}{1.452829in}}%
\pgfpathcurveto{\pgfqpoint{4.506509in}{1.458653in}}{\pgfqpoint{4.498609in}{1.461925in}}{\pgfqpoint{4.490373in}{1.461925in}}%
\pgfpathcurveto{\pgfqpoint{4.482136in}{1.461925in}}{\pgfqpoint{4.474236in}{1.458653in}}{\pgfqpoint{4.468412in}{1.452829in}}%
\pgfpathcurveto{\pgfqpoint{4.462588in}{1.447005in}}{\pgfqpoint{4.459316in}{1.439105in}}{\pgfqpoint{4.459316in}{1.430869in}}%
\pgfpathcurveto{\pgfqpoint{4.459316in}{1.422632in}}{\pgfqpoint{4.462588in}{1.414732in}}{\pgfqpoint{4.468412in}{1.408908in}}%
\pgfpathcurveto{\pgfqpoint{4.474236in}{1.403085in}}{\pgfqpoint{4.482136in}{1.399812in}}{\pgfqpoint{4.490373in}{1.399812in}}%
\pgfpathclose%
\pgfusepath{stroke,fill}%
\end{pgfscope}%
\begin{pgfscope}%
\pgfpathrectangle{\pgfqpoint{3.696748in}{0.557870in}}{\pgfqpoint{2.424965in}{1.684734in}}%
\pgfusepath{clip}%
\pgfsetbuttcap%
\pgfsetroundjoin%
\definecolor{currentfill}{rgb}{0.298039,0.447059,0.690196}%
\pgfsetfillcolor{currentfill}%
\pgfsetlinewidth{1.003750pt}%
\definecolor{currentstroke}{rgb}{0.298039,0.447059,0.690196}%
\pgfsetstrokecolor{currentstroke}%
\pgfsetdash{}{0pt}%
\pgfpathmoveto{\pgfqpoint{4.843095in}{1.119127in}}%
\pgfpathcurveto{\pgfqpoint{4.851331in}{1.119127in}}{\pgfqpoint{4.859231in}{1.122400in}}{\pgfqpoint{4.865055in}{1.128224in}}%
\pgfpathcurveto{\pgfqpoint{4.870879in}{1.134048in}}{\pgfqpoint{4.874151in}{1.141948in}}{\pgfqpoint{4.874151in}{1.150184in}}%
\pgfpathcurveto{\pgfqpoint{4.874151in}{1.158420in}}{\pgfqpoint{4.870879in}{1.166320in}}{\pgfqpoint{4.865055in}{1.172144in}}%
\pgfpathcurveto{\pgfqpoint{4.859231in}{1.177968in}}{\pgfqpoint{4.851331in}{1.181240in}}{\pgfqpoint{4.843095in}{1.181240in}}%
\pgfpathcurveto{\pgfqpoint{4.834859in}{1.181240in}}{\pgfqpoint{4.826958in}{1.177968in}}{\pgfqpoint{4.821135in}{1.172144in}}%
\pgfpathcurveto{\pgfqpoint{4.815311in}{1.166320in}}{\pgfqpoint{4.812038in}{1.158420in}}{\pgfqpoint{4.812038in}{1.150184in}}%
\pgfpathcurveto{\pgfqpoint{4.812038in}{1.141948in}}{\pgfqpoint{4.815311in}{1.134048in}}{\pgfqpoint{4.821135in}{1.128224in}}%
\pgfpathcurveto{\pgfqpoint{4.826958in}{1.122400in}}{\pgfqpoint{4.834859in}{1.119127in}}{\pgfqpoint{4.843095in}{1.119127in}}%
\pgfpathclose%
\pgfusepath{stroke,fill}%
\end{pgfscope}%
\begin{pgfscope}%
\pgfpathrectangle{\pgfqpoint{3.696748in}{0.557870in}}{\pgfqpoint{2.424965in}{1.684734in}}%
\pgfusepath{clip}%
\pgfsetbuttcap%
\pgfsetroundjoin%
\definecolor{currentfill}{rgb}{0.298039,0.447059,0.690196}%
\pgfsetfillcolor{currentfill}%
\pgfsetlinewidth{1.003750pt}%
\definecolor{currentstroke}{rgb}{0.298039,0.447059,0.690196}%
\pgfsetstrokecolor{currentstroke}%
\pgfsetdash{}{0pt}%
\pgfpathmoveto{\pgfqpoint{4.420601in}{1.423327in}}%
\pgfpathcurveto{\pgfqpoint{4.428837in}{1.423327in}}{\pgfqpoint{4.436737in}{1.426600in}}{\pgfqpoint{4.442561in}{1.432424in}}%
\pgfpathcurveto{\pgfqpoint{4.448385in}{1.438248in}}{\pgfqpoint{4.451657in}{1.446148in}}{\pgfqpoint{4.451657in}{1.454384in}}%
\pgfpathcurveto{\pgfqpoint{4.451657in}{1.462620in}}{\pgfqpoint{4.448385in}{1.470520in}}{\pgfqpoint{4.442561in}{1.476344in}}%
\pgfpathcurveto{\pgfqpoint{4.436737in}{1.482168in}}{\pgfqpoint{4.428837in}{1.485440in}}{\pgfqpoint{4.420601in}{1.485440in}}%
\pgfpathcurveto{\pgfqpoint{4.412365in}{1.485440in}}{\pgfqpoint{4.404465in}{1.482168in}}{\pgfqpoint{4.398641in}{1.476344in}}%
\pgfpathcurveto{\pgfqpoint{4.392817in}{1.470520in}}{\pgfqpoint{4.389544in}{1.462620in}}{\pgfqpoint{4.389544in}{1.454384in}}%
\pgfpathcurveto{\pgfqpoint{4.389544in}{1.446148in}}{\pgfqpoint{4.392817in}{1.438248in}}{\pgfqpoint{4.398641in}{1.432424in}}%
\pgfpathcurveto{\pgfqpoint{4.404465in}{1.426600in}}{\pgfqpoint{4.412365in}{1.423327in}}{\pgfqpoint{4.420601in}{1.423327in}}%
\pgfpathclose%
\pgfusepath{stroke,fill}%
\end{pgfscope}%
\begin{pgfscope}%
\pgfpathrectangle{\pgfqpoint{3.696748in}{0.557870in}}{\pgfqpoint{2.424965in}{1.684734in}}%
\pgfusepath{clip}%
\pgfsetbuttcap%
\pgfsetroundjoin%
\definecolor{currentfill}{rgb}{0.298039,0.447059,0.690196}%
\pgfsetfillcolor{currentfill}%
\pgfsetlinewidth{1.003750pt}%
\definecolor{currentstroke}{rgb}{0.298039,0.447059,0.690196}%
\pgfsetstrokecolor{currentstroke}%
\pgfsetdash{}{0pt}%
\pgfpathmoveto{\pgfqpoint{4.829686in}{1.067507in}}%
\pgfpathcurveto{\pgfqpoint{4.837922in}{1.067507in}}{\pgfqpoint{4.845822in}{1.070779in}}{\pgfqpoint{4.851646in}{1.076603in}}%
\pgfpathcurveto{\pgfqpoint{4.857470in}{1.082427in}}{\pgfqpoint{4.860742in}{1.090327in}}{\pgfqpoint{4.860742in}{1.098563in}}%
\pgfpathcurveto{\pgfqpoint{4.860742in}{1.106799in}}{\pgfqpoint{4.857470in}{1.114699in}}{\pgfqpoint{4.851646in}{1.120523in}}%
\pgfpathcurveto{\pgfqpoint{4.845822in}{1.126347in}}{\pgfqpoint{4.837922in}{1.129620in}}{\pgfqpoint{4.829686in}{1.129620in}}%
\pgfpathcurveto{\pgfqpoint{4.821450in}{1.129620in}}{\pgfqpoint{4.813550in}{1.126347in}}{\pgfqpoint{4.807726in}{1.120523in}}%
\pgfpathcurveto{\pgfqpoint{4.801902in}{1.114699in}}{\pgfqpoint{4.798629in}{1.106799in}}{\pgfqpoint{4.798629in}{1.098563in}}%
\pgfpathcurveto{\pgfqpoint{4.798629in}{1.090327in}}{\pgfqpoint{4.801902in}{1.082427in}}{\pgfqpoint{4.807726in}{1.076603in}}%
\pgfpathcurveto{\pgfqpoint{4.813550in}{1.070779in}}{\pgfqpoint{4.821450in}{1.067507in}}{\pgfqpoint{4.829686in}{1.067507in}}%
\pgfpathclose%
\pgfusepath{stroke,fill}%
\end{pgfscope}%
\begin{pgfscope}%
\pgfpathrectangle{\pgfqpoint{3.696748in}{0.557870in}}{\pgfqpoint{2.424965in}{1.684734in}}%
\pgfusepath{clip}%
\pgfsetbuttcap%
\pgfsetroundjoin%
\definecolor{currentfill}{rgb}{0.298039,0.447059,0.690196}%
\pgfsetfillcolor{currentfill}%
\pgfsetlinewidth{1.003750pt}%
\definecolor{currentstroke}{rgb}{0.298039,0.447059,0.690196}%
\pgfsetstrokecolor{currentstroke}%
\pgfsetdash{}{0pt}%
\pgfpathmoveto{\pgfqpoint{4.625143in}{1.166012in}}%
\pgfpathcurveto{\pgfqpoint{4.633380in}{1.166012in}}{\pgfqpoint{4.641280in}{1.169285in}}{\pgfqpoint{4.647104in}{1.175109in}}%
\pgfpathcurveto{\pgfqpoint{4.652928in}{1.180933in}}{\pgfqpoint{4.656200in}{1.188833in}}{\pgfqpoint{4.656200in}{1.197069in}}%
\pgfpathcurveto{\pgfqpoint{4.656200in}{1.205305in}}{\pgfqpoint{4.652928in}{1.213205in}}{\pgfqpoint{4.647104in}{1.219029in}}%
\pgfpathcurveto{\pgfqpoint{4.641280in}{1.224853in}}{\pgfqpoint{4.633380in}{1.228125in}}{\pgfqpoint{4.625143in}{1.228125in}}%
\pgfpathcurveto{\pgfqpoint{4.616907in}{1.228125in}}{\pgfqpoint{4.609007in}{1.224853in}}{\pgfqpoint{4.603183in}{1.219029in}}%
\pgfpathcurveto{\pgfqpoint{4.597359in}{1.213205in}}{\pgfqpoint{4.594087in}{1.205305in}}{\pgfqpoint{4.594087in}{1.197069in}}%
\pgfpathcurveto{\pgfqpoint{4.594087in}{1.188833in}}{\pgfqpoint{4.597359in}{1.180933in}}{\pgfqpoint{4.603183in}{1.175109in}}%
\pgfpathcurveto{\pgfqpoint{4.609007in}{1.169285in}}{\pgfqpoint{4.616907in}{1.166012in}}{\pgfqpoint{4.625143in}{1.166012in}}%
\pgfpathclose%
\pgfusepath{stroke,fill}%
\end{pgfscope}%
\begin{pgfscope}%
\pgfpathrectangle{\pgfqpoint{3.696748in}{0.557870in}}{\pgfqpoint{2.424965in}{1.684734in}}%
\pgfusepath{clip}%
\pgfsetbuttcap%
\pgfsetroundjoin%
\definecolor{currentfill}{rgb}{0.298039,0.447059,0.690196}%
\pgfsetfillcolor{currentfill}%
\pgfsetlinewidth{1.003750pt}%
\definecolor{currentstroke}{rgb}{0.298039,0.447059,0.690196}%
\pgfsetstrokecolor{currentstroke}%
\pgfsetdash{}{0pt}%
\pgfpathmoveto{\pgfqpoint{4.336057in}{1.369181in}}%
\pgfpathcurveto{\pgfqpoint{4.344293in}{1.369181in}}{\pgfqpoint{4.352193in}{1.372453in}}{\pgfqpoint{4.358017in}{1.378277in}}%
\pgfpathcurveto{\pgfqpoint{4.363841in}{1.384101in}}{\pgfqpoint{4.367113in}{1.392001in}}{\pgfqpoint{4.367113in}{1.400237in}}%
\pgfpathcurveto{\pgfqpoint{4.367113in}{1.408474in}}{\pgfqpoint{4.363841in}{1.416374in}}{\pgfqpoint{4.358017in}{1.422197in}}%
\pgfpathcurveto{\pgfqpoint{4.352193in}{1.428021in}}{\pgfqpoint{4.344293in}{1.431294in}}{\pgfqpoint{4.336057in}{1.431294in}}%
\pgfpathcurveto{\pgfqpoint{4.327820in}{1.431294in}}{\pgfqpoint{4.319920in}{1.428021in}}{\pgfqpoint{4.314096in}{1.422197in}}%
\pgfpathcurveto{\pgfqpoint{4.308272in}{1.416374in}}{\pgfqpoint{4.305000in}{1.408474in}}{\pgfqpoint{4.305000in}{1.400237in}}%
\pgfpathcurveto{\pgfqpoint{4.305000in}{1.392001in}}{\pgfqpoint{4.308272in}{1.384101in}}{\pgfqpoint{4.314096in}{1.378277in}}%
\pgfpathcurveto{\pgfqpoint{4.319920in}{1.372453in}}{\pgfqpoint{4.327820in}{1.369181in}}{\pgfqpoint{4.336057in}{1.369181in}}%
\pgfpathclose%
\pgfusepath{stroke,fill}%
\end{pgfscope}%
\begin{pgfscope}%
\pgfpathrectangle{\pgfqpoint{3.696748in}{0.557870in}}{\pgfqpoint{2.424965in}{1.684734in}}%
\pgfusepath{clip}%
\pgfsetbuttcap%
\pgfsetroundjoin%
\definecolor{currentfill}{rgb}{0.298039,0.447059,0.690196}%
\pgfsetfillcolor{currentfill}%
\pgfsetlinewidth{1.003750pt}%
\definecolor{currentstroke}{rgb}{0.298039,0.447059,0.690196}%
\pgfsetstrokecolor{currentstroke}%
\pgfsetdash{}{0pt}%
\pgfpathmoveto{\pgfqpoint{4.579690in}{1.154760in}}%
\pgfpathcurveto{\pgfqpoint{4.587926in}{1.154760in}}{\pgfqpoint{4.595826in}{1.158032in}}{\pgfqpoint{4.601650in}{1.163856in}}%
\pgfpathcurveto{\pgfqpoint{4.607474in}{1.169680in}}{\pgfqpoint{4.610746in}{1.177580in}}{\pgfqpoint{4.610746in}{1.185817in}}%
\pgfpathcurveto{\pgfqpoint{4.610746in}{1.194053in}}{\pgfqpoint{4.607474in}{1.201953in}}{\pgfqpoint{4.601650in}{1.207777in}}%
\pgfpathcurveto{\pgfqpoint{4.595826in}{1.213601in}}{\pgfqpoint{4.587926in}{1.216873in}}{\pgfqpoint{4.579690in}{1.216873in}}%
\pgfpathcurveto{\pgfqpoint{4.571453in}{1.216873in}}{\pgfqpoint{4.563553in}{1.213601in}}{\pgfqpoint{4.557729in}{1.207777in}}%
\pgfpathcurveto{\pgfqpoint{4.551905in}{1.201953in}}{\pgfqpoint{4.548633in}{1.194053in}}{\pgfqpoint{4.548633in}{1.185817in}}%
\pgfpathcurveto{\pgfqpoint{4.548633in}{1.177580in}}{\pgfqpoint{4.551905in}{1.169680in}}{\pgfqpoint{4.557729in}{1.163856in}}%
\pgfpathcurveto{\pgfqpoint{4.563553in}{1.158032in}}{\pgfqpoint{4.571453in}{1.154760in}}{\pgfqpoint{4.579690in}{1.154760in}}%
\pgfpathclose%
\pgfusepath{stroke,fill}%
\end{pgfscope}%
\begin{pgfscope}%
\pgfpathrectangle{\pgfqpoint{3.696748in}{0.557870in}}{\pgfqpoint{2.424965in}{1.684734in}}%
\pgfusepath{clip}%
\pgfsetbuttcap%
\pgfsetroundjoin%
\definecolor{currentfill}{rgb}{0.298039,0.447059,0.690196}%
\pgfsetfillcolor{currentfill}%
\pgfsetlinewidth{1.003750pt}%
\definecolor{currentstroke}{rgb}{0.298039,0.447059,0.690196}%
\pgfsetstrokecolor{currentstroke}%
\pgfsetdash{}{0pt}%
\pgfpathmoveto{\pgfqpoint{4.170605in}{1.500680in}}%
\pgfpathcurveto{\pgfqpoint{4.178841in}{1.500680in}}{\pgfqpoint{4.186741in}{1.503952in}}{\pgfqpoint{4.192565in}{1.509776in}}%
\pgfpathcurveto{\pgfqpoint{4.198389in}{1.515600in}}{\pgfqpoint{4.201661in}{1.523500in}}{\pgfqpoint{4.201661in}{1.531736in}}%
\pgfpathcurveto{\pgfqpoint{4.201661in}{1.539972in}}{\pgfqpoint{4.198389in}{1.547873in}}{\pgfqpoint{4.192565in}{1.553696in}}%
\pgfpathcurveto{\pgfqpoint{4.186741in}{1.559520in}}{\pgfqpoint{4.178841in}{1.562793in}}{\pgfqpoint{4.170605in}{1.562793in}}%
\pgfpathcurveto{\pgfqpoint{4.162368in}{1.562793in}}{\pgfqpoint{4.154468in}{1.559520in}}{\pgfqpoint{4.148644in}{1.553696in}}%
\pgfpathcurveto{\pgfqpoint{4.142820in}{1.547873in}}{\pgfqpoint{4.139548in}{1.539972in}}{\pgfqpoint{4.139548in}{1.531736in}}%
\pgfpathcurveto{\pgfqpoint{4.139548in}{1.523500in}}{\pgfqpoint{4.142820in}{1.515600in}}{\pgfqpoint{4.148644in}{1.509776in}}%
\pgfpathcurveto{\pgfqpoint{4.154468in}{1.503952in}}{\pgfqpoint{4.162368in}{1.500680in}}{\pgfqpoint{4.170605in}{1.500680in}}%
\pgfpathclose%
\pgfusepath{stroke,fill}%
\end{pgfscope}%
\begin{pgfscope}%
\pgfpathrectangle{\pgfqpoint{3.696748in}{0.557870in}}{\pgfqpoint{2.424965in}{1.684734in}}%
\pgfusepath{clip}%
\pgfsetbuttcap%
\pgfsetroundjoin%
\definecolor{currentfill}{rgb}{0.298039,0.447059,0.690196}%
\pgfsetfillcolor{currentfill}%
\pgfsetlinewidth{1.003750pt}%
\definecolor{currentstroke}{rgb}{0.298039,0.447059,0.690196}%
\pgfsetstrokecolor{currentstroke}%
\pgfsetdash{}{0pt}%
\pgfpathmoveto{\pgfqpoint{4.997411in}{0.897331in}}%
\pgfpathcurveto{\pgfqpoint{5.005647in}{0.897331in}}{\pgfqpoint{5.013547in}{0.900604in}}{\pgfqpoint{5.019371in}{0.906428in}}%
\pgfpathcurveto{\pgfqpoint{5.025195in}{0.912252in}}{\pgfqpoint{5.028467in}{0.920152in}}{\pgfqpoint{5.028467in}{0.928388in}}%
\pgfpathcurveto{\pgfqpoint{5.028467in}{0.936624in}}{\pgfqpoint{5.025195in}{0.944524in}}{\pgfqpoint{5.019371in}{0.950348in}}%
\pgfpathcurveto{\pgfqpoint{5.013547in}{0.956172in}}{\pgfqpoint{5.005647in}{0.959444in}}{\pgfqpoint{4.997411in}{0.959444in}}%
\pgfpathcurveto{\pgfqpoint{4.989175in}{0.959444in}}{\pgfqpoint{4.981274in}{0.956172in}}{\pgfqpoint{4.975451in}{0.950348in}}%
\pgfpathcurveto{\pgfqpoint{4.969627in}{0.944524in}}{\pgfqpoint{4.966354in}{0.936624in}}{\pgfqpoint{4.966354in}{0.928388in}}%
\pgfpathcurveto{\pgfqpoint{4.966354in}{0.920152in}}{\pgfqpoint{4.969627in}{0.912252in}}{\pgfqpoint{4.975451in}{0.906428in}}%
\pgfpathcurveto{\pgfqpoint{4.981274in}{0.900604in}}{\pgfqpoint{4.989175in}{0.897331in}}{\pgfqpoint{4.997411in}{0.897331in}}%
\pgfpathclose%
\pgfusepath{stroke,fill}%
\end{pgfscope}%
\begin{pgfscope}%
\pgfpathrectangle{\pgfqpoint{3.696748in}{0.557870in}}{\pgfqpoint{2.424965in}{1.684734in}}%
\pgfusepath{clip}%
\pgfsetbuttcap%
\pgfsetroundjoin%
\definecolor{currentfill}{rgb}{0.298039,0.447059,0.690196}%
\pgfsetfillcolor{currentfill}%
\pgfsetlinewidth{1.003750pt}%
\definecolor{currentstroke}{rgb}{0.298039,0.447059,0.690196}%
\pgfsetstrokecolor{currentstroke}%
\pgfsetdash{}{0pt}%
\pgfpathmoveto{\pgfqpoint{4.034243in}{1.593502in}}%
\pgfpathcurveto{\pgfqpoint{4.042479in}{1.593502in}}{\pgfqpoint{4.050379in}{1.596775in}}{\pgfqpoint{4.056203in}{1.602599in}}%
\pgfpathcurveto{\pgfqpoint{4.062027in}{1.608423in}}{\pgfqpoint{4.065299in}{1.616323in}}{\pgfqpoint{4.065299in}{1.624559in}}%
\pgfpathcurveto{\pgfqpoint{4.065299in}{1.632795in}}{\pgfqpoint{4.062027in}{1.640695in}}{\pgfqpoint{4.056203in}{1.646519in}}%
\pgfpathcurveto{\pgfqpoint{4.050379in}{1.652343in}}{\pgfqpoint{4.042479in}{1.655615in}}{\pgfqpoint{4.034243in}{1.655615in}}%
\pgfpathcurveto{\pgfqpoint{4.026007in}{1.655615in}}{\pgfqpoint{4.018107in}{1.652343in}}{\pgfqpoint{4.012283in}{1.646519in}}%
\pgfpathcurveto{\pgfqpoint{4.006459in}{1.640695in}}{\pgfqpoint{4.003186in}{1.632795in}}{\pgfqpoint{4.003186in}{1.624559in}}%
\pgfpathcurveto{\pgfqpoint{4.003186in}{1.616323in}}{\pgfqpoint{4.006459in}{1.608423in}}{\pgfqpoint{4.012283in}{1.602599in}}%
\pgfpathcurveto{\pgfqpoint{4.018107in}{1.596775in}}{\pgfqpoint{4.026007in}{1.593502in}}{\pgfqpoint{4.034243in}{1.593502in}}%
\pgfpathclose%
\pgfusepath{stroke,fill}%
\end{pgfscope}%
\begin{pgfscope}%
\pgfpathrectangle{\pgfqpoint{3.696748in}{0.557870in}}{\pgfqpoint{2.424965in}{1.684734in}}%
\pgfusepath{clip}%
\pgfsetbuttcap%
\pgfsetroundjoin%
\definecolor{currentfill}{rgb}{0.298039,0.447059,0.690196}%
\pgfsetfillcolor{currentfill}%
\pgfsetlinewidth{1.003750pt}%
\definecolor{currentstroke}{rgb}{0.298039,0.447059,0.690196}%
\pgfsetstrokecolor{currentstroke}%
\pgfsetdash{}{0pt}%
\pgfpathmoveto{\pgfqpoint{4.291966in}{1.237682in}}%
\pgfpathcurveto{\pgfqpoint{4.300203in}{1.237682in}}{\pgfqpoint{4.308103in}{1.240954in}}{\pgfqpoint{4.313927in}{1.246778in}}%
\pgfpathcurveto{\pgfqpoint{4.319751in}{1.252602in}}{\pgfqpoint{4.323023in}{1.260502in}}{\pgfqpoint{4.323023in}{1.268738in}}%
\pgfpathcurveto{\pgfqpoint{4.323023in}{1.276975in}}{\pgfqpoint{4.319751in}{1.284875in}}{\pgfqpoint{4.313927in}{1.290699in}}%
\pgfpathcurveto{\pgfqpoint{4.308103in}{1.296522in}}{\pgfqpoint{4.300203in}{1.299795in}}{\pgfqpoint{4.291966in}{1.299795in}}%
\pgfpathcurveto{\pgfqpoint{4.283730in}{1.299795in}}{\pgfqpoint{4.275830in}{1.296522in}}{\pgfqpoint{4.270006in}{1.290699in}}%
\pgfpathcurveto{\pgfqpoint{4.264182in}{1.284875in}}{\pgfqpoint{4.260910in}{1.276975in}}{\pgfqpoint{4.260910in}{1.268738in}}%
\pgfpathcurveto{\pgfqpoint{4.260910in}{1.260502in}}{\pgfqpoint{4.264182in}{1.252602in}}{\pgfqpoint{4.270006in}{1.246778in}}%
\pgfpathcurveto{\pgfqpoint{4.275830in}{1.240954in}}{\pgfqpoint{4.283730in}{1.237682in}}{\pgfqpoint{4.291966in}{1.237682in}}%
\pgfpathclose%
\pgfusepath{stroke,fill}%
\end{pgfscope}%
\begin{pgfscope}%
\pgfpathrectangle{\pgfqpoint{3.696748in}{0.557870in}}{\pgfqpoint{2.424965in}{1.684734in}}%
\pgfusepath{clip}%
\pgfsetbuttcap%
\pgfsetroundjoin%
\definecolor{currentfill}{rgb}{0.298039,0.447059,0.690196}%
\pgfsetfillcolor{currentfill}%
\pgfsetlinewidth{1.003750pt}%
\definecolor{currentstroke}{rgb}{0.298039,0.447059,0.690196}%
\pgfsetstrokecolor{currentstroke}%
\pgfsetdash{}{0pt}%
\pgfpathmoveto{\pgfqpoint{4.556963in}{0.994101in}}%
\pgfpathcurveto{\pgfqpoint{4.565199in}{0.994101in}}{\pgfqpoint{4.573099in}{0.997373in}}{\pgfqpoint{4.578923in}{1.003197in}}%
\pgfpathcurveto{\pgfqpoint{4.584747in}{1.009021in}}{\pgfqpoint{4.588019in}{1.016921in}}{\pgfqpoint{4.588019in}{1.025157in}}%
\pgfpathcurveto{\pgfqpoint{4.588019in}{1.033394in}}{\pgfqpoint{4.584747in}{1.041294in}}{\pgfqpoint{4.578923in}{1.047118in}}%
\pgfpathcurveto{\pgfqpoint{4.573099in}{1.052942in}}{\pgfqpoint{4.565199in}{1.056214in}}{\pgfqpoint{4.556963in}{1.056214in}}%
\pgfpathcurveto{\pgfqpoint{4.548726in}{1.056214in}}{\pgfqpoint{4.540826in}{1.052942in}}{\pgfqpoint{4.535002in}{1.047118in}}%
\pgfpathcurveto{\pgfqpoint{4.529178in}{1.041294in}}{\pgfqpoint{4.525906in}{1.033394in}}{\pgfqpoint{4.525906in}{1.025157in}}%
\pgfpathcurveto{\pgfqpoint{4.525906in}{1.016921in}}{\pgfqpoint{4.529178in}{1.009021in}}{\pgfqpoint{4.535002in}{1.003197in}}%
\pgfpathcurveto{\pgfqpoint{4.540826in}{0.997373in}}{\pgfqpoint{4.548726in}{0.994101in}}{\pgfqpoint{4.556963in}{0.994101in}}%
\pgfpathclose%
\pgfusepath{stroke,fill}%
\end{pgfscope}%
\begin{pgfscope}%
\pgfpathrectangle{\pgfqpoint{3.696748in}{0.557870in}}{\pgfqpoint{2.424965in}{1.684734in}}%
\pgfusepath{clip}%
\pgfsetbuttcap%
\pgfsetroundjoin%
\definecolor{currentfill}{rgb}{0.298039,0.447059,0.690196}%
\pgfsetfillcolor{currentfill}%
\pgfsetlinewidth{1.003750pt}%
\definecolor{currentstroke}{rgb}{0.298039,0.447059,0.690196}%
\pgfsetstrokecolor{currentstroke}%
\pgfsetdash{}{0pt}%
\pgfpathmoveto{\pgfqpoint{4.291966in}{1.134756in}}%
\pgfpathcurveto{\pgfqpoint{4.300203in}{1.134756in}}{\pgfqpoint{4.308103in}{1.138028in}}{\pgfqpoint{4.313927in}{1.143852in}}%
\pgfpathcurveto{\pgfqpoint{4.319751in}{1.149676in}}{\pgfqpoint{4.323023in}{1.157576in}}{\pgfqpoint{4.323023in}{1.165812in}}%
\pgfpathcurveto{\pgfqpoint{4.323023in}{1.174049in}}{\pgfqpoint{4.319751in}{1.181949in}}{\pgfqpoint{4.313927in}{1.187773in}}%
\pgfpathcurveto{\pgfqpoint{4.308103in}{1.193596in}}{\pgfqpoint{4.300203in}{1.196869in}}{\pgfqpoint{4.291966in}{1.196869in}}%
\pgfpathcurveto{\pgfqpoint{4.283730in}{1.196869in}}{\pgfqpoint{4.275830in}{1.193596in}}{\pgfqpoint{4.270006in}{1.187773in}}%
\pgfpathcurveto{\pgfqpoint{4.264182in}{1.181949in}}{\pgfqpoint{4.260910in}{1.174049in}}{\pgfqpoint{4.260910in}{1.165812in}}%
\pgfpathcurveto{\pgfqpoint{4.260910in}{1.157576in}}{\pgfqpoint{4.264182in}{1.149676in}}{\pgfqpoint{4.270006in}{1.143852in}}%
\pgfpathcurveto{\pgfqpoint{4.275830in}{1.138028in}}{\pgfqpoint{4.283730in}{1.134756in}}{\pgfqpoint{4.291966in}{1.134756in}}%
\pgfpathclose%
\pgfusepath{stroke,fill}%
\end{pgfscope}%
\begin{pgfscope}%
\pgfpathrectangle{\pgfqpoint{3.696748in}{0.557870in}}{\pgfqpoint{2.424965in}{1.684734in}}%
\pgfusepath{clip}%
\pgfsetbuttcap%
\pgfsetroundjoin%
\definecolor{currentfill}{rgb}{0.298039,0.447059,0.690196}%
\pgfsetfillcolor{currentfill}%
\pgfsetlinewidth{1.003750pt}%
\definecolor{currentstroke}{rgb}{0.298039,0.447059,0.690196}%
\pgfsetstrokecolor{currentstroke}%
\pgfsetdash{}{0pt}%
\pgfpathmoveto{\pgfqpoint{4.443328in}{0.990154in}}%
\pgfpathcurveto{\pgfqpoint{4.451564in}{0.990154in}}{\pgfqpoint{4.459464in}{0.993427in}}{\pgfqpoint{4.465288in}{0.999251in}}%
\pgfpathcurveto{\pgfqpoint{4.471112in}{1.005074in}}{\pgfqpoint{4.474384in}{1.012974in}}{\pgfqpoint{4.474384in}{1.021211in}}%
\pgfpathcurveto{\pgfqpoint{4.474384in}{1.029447in}}{\pgfqpoint{4.471112in}{1.037347in}}{\pgfqpoint{4.465288in}{1.043171in}}%
\pgfpathcurveto{\pgfqpoint{4.459464in}{1.048995in}}{\pgfqpoint{4.451564in}{1.052267in}}{\pgfqpoint{4.443328in}{1.052267in}}%
\pgfpathcurveto{\pgfqpoint{4.435092in}{1.052267in}}{\pgfqpoint{4.427192in}{1.048995in}}{\pgfqpoint{4.421368in}{1.043171in}}%
\pgfpathcurveto{\pgfqpoint{4.415544in}{1.037347in}}{\pgfqpoint{4.412271in}{1.029447in}}{\pgfqpoint{4.412271in}{1.021211in}}%
\pgfpathcurveto{\pgfqpoint{4.412271in}{1.012974in}}{\pgfqpoint{4.415544in}{1.005074in}}{\pgfqpoint{4.421368in}{0.999251in}}%
\pgfpathcurveto{\pgfqpoint{4.427192in}{0.993427in}}{\pgfqpoint{4.435092in}{0.990154in}}{\pgfqpoint{4.443328in}{0.990154in}}%
\pgfpathclose%
\pgfusepath{stroke,fill}%
\end{pgfscope}%
\begin{pgfscope}%
\pgfpathrectangle{\pgfqpoint{3.696748in}{0.557870in}}{\pgfqpoint{2.424965in}{1.684734in}}%
\pgfusepath{clip}%
\pgfsetbuttcap%
\pgfsetroundjoin%
\definecolor{currentfill}{rgb}{0.298039,0.447059,0.690196}%
\pgfsetfillcolor{currentfill}%
\pgfsetlinewidth{1.003750pt}%
\definecolor{currentstroke}{rgb}{0.298039,0.447059,0.690196}%
\pgfsetstrokecolor{currentstroke}%
\pgfsetdash{}{0pt}%
\pgfpathmoveto{\pgfqpoint{4.011516in}{1.392386in}}%
\pgfpathcurveto{\pgfqpoint{4.019752in}{1.392386in}}{\pgfqpoint{4.027652in}{1.395659in}}{\pgfqpoint{4.033476in}{1.401483in}}%
\pgfpathcurveto{\pgfqpoint{4.039300in}{1.407307in}}{\pgfqpoint{4.042572in}{1.415207in}}{\pgfqpoint{4.042572in}{1.423443in}}%
\pgfpathcurveto{\pgfqpoint{4.042572in}{1.431679in}}{\pgfqpoint{4.039300in}{1.439579in}}{\pgfqpoint{4.033476in}{1.445403in}}%
\pgfpathcurveto{\pgfqpoint{4.027652in}{1.451227in}}{\pgfqpoint{4.019752in}{1.454499in}}{\pgfqpoint{4.011516in}{1.454499in}}%
\pgfpathcurveto{\pgfqpoint{4.003280in}{1.454499in}}{\pgfqpoint{3.995380in}{1.451227in}}{\pgfqpoint{3.989556in}{1.445403in}}%
\pgfpathcurveto{\pgfqpoint{3.983732in}{1.439579in}}{\pgfqpoint{3.980459in}{1.431679in}}{\pgfqpoint{3.980459in}{1.423443in}}%
\pgfpathcurveto{\pgfqpoint{3.980459in}{1.415207in}}{\pgfqpoint{3.983732in}{1.407307in}}{\pgfqpoint{3.989556in}{1.401483in}}%
\pgfpathcurveto{\pgfqpoint{3.995380in}{1.395659in}}{\pgfqpoint{4.003280in}{1.392386in}}{\pgfqpoint{4.011516in}{1.392386in}}%
\pgfpathclose%
\pgfusepath{stroke,fill}%
\end{pgfscope}%
\begin{pgfscope}%
\pgfpathrectangle{\pgfqpoint{3.696748in}{0.557870in}}{\pgfqpoint{2.424965in}{1.684734in}}%
\pgfusepath{clip}%
\pgfsetbuttcap%
\pgfsetroundjoin%
\definecolor{currentfill}{rgb}{0.298039,0.447059,0.690196}%
\pgfsetfillcolor{currentfill}%
\pgfsetlinewidth{1.003750pt}%
\definecolor{currentstroke}{rgb}{0.298039,0.447059,0.690196}%
\pgfsetstrokecolor{currentstroke}%
\pgfsetdash{}{0pt}%
\pgfpathmoveto{\pgfqpoint{3.988789in}{1.400437in}}%
\pgfpathcurveto{\pgfqpoint{3.997025in}{1.400437in}}{\pgfqpoint{4.004925in}{1.403710in}}{\pgfqpoint{4.010749in}{1.409534in}}%
\pgfpathcurveto{\pgfqpoint{4.016573in}{1.415358in}}{\pgfqpoint{4.019845in}{1.423258in}}{\pgfqpoint{4.019845in}{1.431494in}}%
\pgfpathcurveto{\pgfqpoint{4.019845in}{1.439730in}}{\pgfqpoint{4.016573in}{1.447630in}}{\pgfqpoint{4.010749in}{1.453454in}}%
\pgfpathcurveto{\pgfqpoint{4.004925in}{1.459278in}}{\pgfqpoint{3.997025in}{1.462550in}}{\pgfqpoint{3.988789in}{1.462550in}}%
\pgfpathcurveto{\pgfqpoint{3.980553in}{1.462550in}}{\pgfqpoint{3.972653in}{1.459278in}}{\pgfqpoint{3.966829in}{1.453454in}}%
\pgfpathcurveto{\pgfqpoint{3.961005in}{1.447630in}}{\pgfqpoint{3.957732in}{1.439730in}}{\pgfqpoint{3.957732in}{1.431494in}}%
\pgfpathcurveto{\pgfqpoint{3.957732in}{1.423258in}}{\pgfqpoint{3.961005in}{1.415358in}}{\pgfqpoint{3.966829in}{1.409534in}}%
\pgfpathcurveto{\pgfqpoint{3.972653in}{1.403710in}}{\pgfqpoint{3.980553in}{1.400437in}}{\pgfqpoint{3.988789in}{1.400437in}}%
\pgfpathclose%
\pgfusepath{stroke,fill}%
\end{pgfscope}%
\begin{pgfscope}%
\pgfpathrectangle{\pgfqpoint{3.696748in}{0.557870in}}{\pgfqpoint{2.424965in}{1.684734in}}%
\pgfusepath{clip}%
\pgfsetbuttcap%
\pgfsetroundjoin%
\definecolor{currentfill}{rgb}{0.298039,0.447059,0.690196}%
\pgfsetfillcolor{currentfill}%
\pgfsetlinewidth{1.003750pt}%
\definecolor{currentstroke}{rgb}{0.298039,0.447059,0.690196}%
\pgfsetstrokecolor{currentstroke}%
\pgfsetdash{}{0pt}%
\pgfpathmoveto{\pgfqpoint{4.137650in}{1.144859in}}%
\pgfpathcurveto{\pgfqpoint{4.145887in}{1.144859in}}{\pgfqpoint{4.153787in}{1.148131in}}{\pgfqpoint{4.159611in}{1.153955in}}%
\pgfpathcurveto{\pgfqpoint{4.165435in}{1.159779in}}{\pgfqpoint{4.168707in}{1.167679in}}{\pgfqpoint{4.168707in}{1.175915in}}%
\pgfpathcurveto{\pgfqpoint{4.168707in}{1.184152in}}{\pgfqpoint{4.165435in}{1.192052in}}{\pgfqpoint{4.159611in}{1.197876in}}%
\pgfpathcurveto{\pgfqpoint{4.153787in}{1.203700in}}{\pgfqpoint{4.145887in}{1.206972in}}{\pgfqpoint{4.137650in}{1.206972in}}%
\pgfpathcurveto{\pgfqpoint{4.129414in}{1.206972in}}{\pgfqpoint{4.121514in}{1.203700in}}{\pgfqpoint{4.115690in}{1.197876in}}%
\pgfpathcurveto{\pgfqpoint{4.109866in}{1.192052in}}{\pgfqpoint{4.106594in}{1.184152in}}{\pgfqpoint{4.106594in}{1.175915in}}%
\pgfpathcurveto{\pgfqpoint{4.106594in}{1.167679in}}{\pgfqpoint{4.109866in}{1.159779in}}{\pgfqpoint{4.115690in}{1.153955in}}%
\pgfpathcurveto{\pgfqpoint{4.121514in}{1.148131in}}{\pgfqpoint{4.129414in}{1.144859in}}{\pgfqpoint{4.137650in}{1.144859in}}%
\pgfpathclose%
\pgfusepath{stroke,fill}%
\end{pgfscope}%
\begin{pgfscope}%
\pgfpathrectangle{\pgfqpoint{3.696748in}{0.557870in}}{\pgfqpoint{2.424965in}{1.684734in}}%
\pgfusepath{clip}%
\pgfsetbuttcap%
\pgfsetroundjoin%
\definecolor{currentfill}{rgb}{0.298039,0.447059,0.690196}%
\pgfsetfillcolor{currentfill}%
\pgfsetlinewidth{1.003750pt}%
\definecolor{currentstroke}{rgb}{0.298039,0.447059,0.690196}%
\pgfsetstrokecolor{currentstroke}%
\pgfsetdash{}{0pt}%
\pgfpathmoveto{\pgfqpoint{4.336057in}{0.943743in}}%
\pgfpathcurveto{\pgfqpoint{4.344293in}{0.943743in}}{\pgfqpoint{4.352193in}{0.947015in}}{\pgfqpoint{4.358017in}{0.952839in}}%
\pgfpathcurveto{\pgfqpoint{4.363841in}{0.958663in}}{\pgfqpoint{4.367113in}{0.966563in}}{\pgfqpoint{4.367113in}{0.974799in}}%
\pgfpathcurveto{\pgfqpoint{4.367113in}{0.983036in}}{\pgfqpoint{4.363841in}{0.990936in}}{\pgfqpoint{4.358017in}{0.996760in}}%
\pgfpathcurveto{\pgfqpoint{4.352193in}{1.002584in}}{\pgfqpoint{4.344293in}{1.005856in}}{\pgfqpoint{4.336057in}{1.005856in}}%
\pgfpathcurveto{\pgfqpoint{4.327820in}{1.005856in}}{\pgfqpoint{4.319920in}{1.002584in}}{\pgfqpoint{4.314096in}{0.996760in}}%
\pgfpathcurveto{\pgfqpoint{4.308272in}{0.990936in}}{\pgfqpoint{4.305000in}{0.983036in}}{\pgfqpoint{4.305000in}{0.974799in}}%
\pgfpathcurveto{\pgfqpoint{4.305000in}{0.966563in}}{\pgfqpoint{4.308272in}{0.958663in}}{\pgfqpoint{4.314096in}{0.952839in}}%
\pgfpathcurveto{\pgfqpoint{4.319920in}{0.947015in}}{\pgfqpoint{4.327820in}{0.943743in}}{\pgfqpoint{4.336057in}{0.943743in}}%
\pgfpathclose%
\pgfusepath{stroke,fill}%
\end{pgfscope}%
\begin{pgfscope}%
\pgfpathrectangle{\pgfqpoint{3.696748in}{0.557870in}}{\pgfqpoint{2.424965in}{1.684734in}}%
\pgfusepath{clip}%
\pgfsetbuttcap%
\pgfsetroundjoin%
\definecolor{currentfill}{rgb}{0.298039,0.447059,0.690196}%
\pgfsetfillcolor{currentfill}%
\pgfsetlinewidth{1.003750pt}%
\definecolor{currentstroke}{rgb}{0.298039,0.447059,0.690196}%
\pgfsetstrokecolor{currentstroke}%
\pgfsetdash{}{0pt}%
\pgfpathmoveto{\pgfqpoint{4.093560in}{1.160329in}}%
\pgfpathcurveto{\pgfqpoint{4.101796in}{1.160329in}}{\pgfqpoint{4.109697in}{1.163602in}}{\pgfqpoint{4.115520in}{1.169426in}}%
\pgfpathcurveto{\pgfqpoint{4.121344in}{1.175250in}}{\pgfqpoint{4.124617in}{1.183150in}}{\pgfqpoint{4.124617in}{1.191386in}}%
\pgfpathcurveto{\pgfqpoint{4.124617in}{1.199622in}}{\pgfqpoint{4.121344in}{1.207522in}}{\pgfqpoint{4.115520in}{1.213346in}}%
\pgfpathcurveto{\pgfqpoint{4.109697in}{1.219170in}}{\pgfqpoint{4.101796in}{1.222442in}}{\pgfqpoint{4.093560in}{1.222442in}}%
\pgfpathcurveto{\pgfqpoint{4.085324in}{1.222442in}}{\pgfqpoint{4.077424in}{1.219170in}}{\pgfqpoint{4.071600in}{1.213346in}}%
\pgfpathcurveto{\pgfqpoint{4.065776in}{1.207522in}}{\pgfqpoint{4.062504in}{1.199622in}}{\pgfqpoint{4.062504in}{1.191386in}}%
\pgfpathcurveto{\pgfqpoint{4.062504in}{1.183150in}}{\pgfqpoint{4.065776in}{1.175250in}}{\pgfqpoint{4.071600in}{1.169426in}}%
\pgfpathcurveto{\pgfqpoint{4.077424in}{1.163602in}}{\pgfqpoint{4.085324in}{1.160329in}}{\pgfqpoint{4.093560in}{1.160329in}}%
\pgfpathclose%
\pgfusepath{stroke,fill}%
\end{pgfscope}%
\begin{pgfscope}%
\pgfpathrectangle{\pgfqpoint{3.696748in}{0.557870in}}{\pgfqpoint{2.424965in}{1.684734in}}%
\pgfusepath{clip}%
\pgfsetbuttcap%
\pgfsetroundjoin%
\definecolor{currentfill}{rgb}{0.298039,0.447059,0.690196}%
\pgfsetfillcolor{currentfill}%
\pgfsetlinewidth{1.003750pt}%
\definecolor{currentstroke}{rgb}{0.298039,0.447059,0.690196}%
\pgfsetstrokecolor{currentstroke}%
\pgfsetdash{}{0pt}%
\pgfpathmoveto{\pgfqpoint{4.159696in}{1.025357in}}%
\pgfpathcurveto{\pgfqpoint{4.167932in}{1.025357in}}{\pgfqpoint{4.175832in}{1.028630in}}{\pgfqpoint{4.181656in}{1.034454in}}%
\pgfpathcurveto{\pgfqpoint{4.187480in}{1.040278in}}{\pgfqpoint{4.190752in}{1.048178in}}{\pgfqpoint{4.190752in}{1.056414in}}%
\pgfpathcurveto{\pgfqpoint{4.190752in}{1.064650in}}{\pgfqpoint{4.187480in}{1.072550in}}{\pgfqpoint{4.181656in}{1.078374in}}%
\pgfpathcurveto{\pgfqpoint{4.175832in}{1.084198in}}{\pgfqpoint{4.167932in}{1.087470in}}{\pgfqpoint{4.159696in}{1.087470in}}%
\pgfpathcurveto{\pgfqpoint{4.151459in}{1.087470in}}{\pgfqpoint{4.143559in}{1.084198in}}{\pgfqpoint{4.137735in}{1.078374in}}%
\pgfpathcurveto{\pgfqpoint{4.131911in}{1.072550in}}{\pgfqpoint{4.128639in}{1.064650in}}{\pgfqpoint{4.128639in}{1.056414in}}%
\pgfpathcurveto{\pgfqpoint{4.128639in}{1.048178in}}{\pgfqpoint{4.131911in}{1.040278in}}{\pgfqpoint{4.137735in}{1.034454in}}%
\pgfpathcurveto{\pgfqpoint{4.143559in}{1.028630in}}{\pgfqpoint{4.151459in}{1.025357in}}{\pgfqpoint{4.159696in}{1.025357in}}%
\pgfpathclose%
\pgfusepath{stroke,fill}%
\end{pgfscope}%
\begin{pgfscope}%
\pgfpathrectangle{\pgfqpoint{3.696748in}{0.557870in}}{\pgfqpoint{2.424965in}{1.684734in}}%
\pgfusepath{clip}%
\pgfsetbuttcap%
\pgfsetroundjoin%
\definecolor{currentfill}{rgb}{0.298039,0.447059,0.690196}%
\pgfsetfillcolor{currentfill}%
\pgfsetlinewidth{1.003750pt}%
\definecolor{currentstroke}{rgb}{0.298039,0.447059,0.690196}%
\pgfsetstrokecolor{currentstroke}%
\pgfsetdash{}{0pt}%
\pgfpathmoveto{\pgfqpoint{4.466055in}{0.806561in}}%
\pgfpathcurveto{\pgfqpoint{4.474291in}{0.806561in}}{\pgfqpoint{4.482191in}{0.809833in}}{\pgfqpoint{4.488015in}{0.815657in}}%
\pgfpathcurveto{\pgfqpoint{4.493839in}{0.821481in}}{\pgfqpoint{4.497111in}{0.829381in}}{\pgfqpoint{4.497111in}{0.837617in}}%
\pgfpathcurveto{\pgfqpoint{4.497111in}{0.845854in}}{\pgfqpoint{4.493839in}{0.853754in}}{\pgfqpoint{4.488015in}{0.859578in}}%
\pgfpathcurveto{\pgfqpoint{4.482191in}{0.865402in}}{\pgfqpoint{4.474291in}{0.868674in}}{\pgfqpoint{4.466055in}{0.868674in}}%
\pgfpathcurveto{\pgfqpoint{4.457819in}{0.868674in}}{\pgfqpoint{4.449918in}{0.865402in}}{\pgfqpoint{4.444095in}{0.859578in}}%
\pgfpathcurveto{\pgfqpoint{4.438271in}{0.853754in}}{\pgfqpoint{4.434998in}{0.845854in}}{\pgfqpoint{4.434998in}{0.837617in}}%
\pgfpathcurveto{\pgfqpoint{4.434998in}{0.829381in}}{\pgfqpoint{4.438271in}{0.821481in}}{\pgfqpoint{4.444095in}{0.815657in}}%
\pgfpathcurveto{\pgfqpoint{4.449918in}{0.809833in}}{\pgfqpoint{4.457819in}{0.806561in}}{\pgfqpoint{4.466055in}{0.806561in}}%
\pgfpathclose%
\pgfusepath{stroke,fill}%
\end{pgfscope}%
\begin{pgfscope}%
\pgfpathrectangle{\pgfqpoint{3.696748in}{0.557870in}}{\pgfqpoint{2.424965in}{1.684734in}}%
\pgfusepath{clip}%
\pgfsetbuttcap%
\pgfsetroundjoin%
\definecolor{currentfill}{rgb}{0.298039,0.447059,0.690196}%
\pgfsetfillcolor{currentfill}%
\pgfsetlinewidth{1.003750pt}%
\definecolor{currentstroke}{rgb}{0.298039,0.447059,0.690196}%
\pgfsetstrokecolor{currentstroke}%
\pgfsetdash{}{0pt}%
\pgfpathmoveto{\pgfqpoint{4.466055in}{0.806561in}}%
\pgfpathcurveto{\pgfqpoint{4.474291in}{0.806561in}}{\pgfqpoint{4.482191in}{0.809833in}}{\pgfqpoint{4.488015in}{0.815657in}}%
\pgfpathcurveto{\pgfqpoint{4.493839in}{0.821481in}}{\pgfqpoint{4.497111in}{0.829381in}}{\pgfqpoint{4.497111in}{0.837617in}}%
\pgfpathcurveto{\pgfqpoint{4.497111in}{0.845854in}}{\pgfqpoint{4.493839in}{0.853754in}}{\pgfqpoint{4.488015in}{0.859578in}}%
\pgfpathcurveto{\pgfqpoint{4.482191in}{0.865402in}}{\pgfqpoint{4.474291in}{0.868674in}}{\pgfqpoint{4.466055in}{0.868674in}}%
\pgfpathcurveto{\pgfqpoint{4.457819in}{0.868674in}}{\pgfqpoint{4.449918in}{0.865402in}}{\pgfqpoint{4.444095in}{0.859578in}}%
\pgfpathcurveto{\pgfqpoint{4.438271in}{0.853754in}}{\pgfqpoint{4.434998in}{0.845854in}}{\pgfqpoint{4.434998in}{0.837617in}}%
\pgfpathcurveto{\pgfqpoint{4.434998in}{0.829381in}}{\pgfqpoint{4.438271in}{0.821481in}}{\pgfqpoint{4.444095in}{0.815657in}}%
\pgfpathcurveto{\pgfqpoint{4.449918in}{0.809833in}}{\pgfqpoint{4.457819in}{0.806561in}}{\pgfqpoint{4.466055in}{0.806561in}}%
\pgfpathclose%
\pgfusepath{stroke,fill}%
\end{pgfscope}%
\begin{pgfscope}%
\pgfpathrectangle{\pgfqpoint{3.696748in}{0.557870in}}{\pgfqpoint{2.424965in}{1.684734in}}%
\pgfusepath{clip}%
\pgfsetbuttcap%
\pgfsetroundjoin%
\definecolor{currentfill}{rgb}{0.298039,0.447059,0.690196}%
\pgfsetfillcolor{currentfill}%
\pgfsetlinewidth{1.003750pt}%
\definecolor{currentstroke}{rgb}{0.298039,0.447059,0.690196}%
\pgfsetstrokecolor{currentstroke}%
\pgfsetdash{}{0pt}%
\pgfpathmoveto{\pgfqpoint{3.943335in}{1.322296in}}%
\pgfpathcurveto{\pgfqpoint{3.951571in}{1.322296in}}{\pgfqpoint{3.959471in}{1.325568in}}{\pgfqpoint{3.965295in}{1.331392in}}%
\pgfpathcurveto{\pgfqpoint{3.971119in}{1.337216in}}{\pgfqpoint{3.974392in}{1.345116in}}{\pgfqpoint{3.974392in}{1.353352in}}%
\pgfpathcurveto{\pgfqpoint{3.974392in}{1.361589in}}{\pgfqpoint{3.971119in}{1.369489in}}{\pgfqpoint{3.965295in}{1.375312in}}%
\pgfpathcurveto{\pgfqpoint{3.959471in}{1.381136in}}{\pgfqpoint{3.951571in}{1.384409in}}{\pgfqpoint{3.943335in}{1.384409in}}%
\pgfpathcurveto{\pgfqpoint{3.935099in}{1.384409in}}{\pgfqpoint{3.927199in}{1.381136in}}{\pgfqpoint{3.921375in}{1.375312in}}%
\pgfpathcurveto{\pgfqpoint{3.915551in}{1.369489in}}{\pgfqpoint{3.912279in}{1.361589in}}{\pgfqpoint{3.912279in}{1.353352in}}%
\pgfpathcurveto{\pgfqpoint{3.912279in}{1.345116in}}{\pgfqpoint{3.915551in}{1.337216in}}{\pgfqpoint{3.921375in}{1.331392in}}%
\pgfpathcurveto{\pgfqpoint{3.927199in}{1.325568in}}{\pgfqpoint{3.935099in}{1.322296in}}{\pgfqpoint{3.943335in}{1.322296in}}%
\pgfpathclose%
\pgfusepath{stroke,fill}%
\end{pgfscope}%
\begin{pgfscope}%
\pgfpathrectangle{\pgfqpoint{3.696748in}{0.557870in}}{\pgfqpoint{2.424965in}{1.684734in}}%
\pgfusepath{clip}%
\pgfsetbuttcap%
\pgfsetroundjoin%
\definecolor{currentfill}{rgb}{0.298039,0.447059,0.690196}%
\pgfsetfillcolor{currentfill}%
\pgfsetlinewidth{1.003750pt}%
\definecolor{currentstroke}{rgb}{0.298039,0.447059,0.690196}%
\pgfsetstrokecolor{currentstroke}%
\pgfsetdash{}{0pt}%
\pgfpathmoveto{\pgfqpoint{4.102424in}{1.021095in}}%
\pgfpathcurveto{\pgfqpoint{4.110660in}{1.021095in}}{\pgfqpoint{4.118560in}{1.024368in}}{\pgfqpoint{4.124384in}{1.030191in}}%
\pgfpathcurveto{\pgfqpoint{4.130208in}{1.036015in}}{\pgfqpoint{4.133480in}{1.043915in}}{\pgfqpoint{4.133480in}{1.052152in}}%
\pgfpathcurveto{\pgfqpoint{4.133480in}{1.060388in}}{\pgfqpoint{4.130208in}{1.068288in}}{\pgfqpoint{4.124384in}{1.074112in}}%
\pgfpathcurveto{\pgfqpoint{4.118560in}{1.079936in}}{\pgfqpoint{4.110660in}{1.083208in}}{\pgfqpoint{4.102424in}{1.083208in}}%
\pgfpathcurveto{\pgfqpoint{4.094187in}{1.083208in}}{\pgfqpoint{4.086287in}{1.079936in}}{\pgfqpoint{4.080463in}{1.074112in}}%
\pgfpathcurveto{\pgfqpoint{4.074639in}{1.068288in}}{\pgfqpoint{4.071367in}{1.060388in}}{\pgfqpoint{4.071367in}{1.052152in}}%
\pgfpathcurveto{\pgfqpoint{4.071367in}{1.043915in}}{\pgfqpoint{4.074639in}{1.036015in}}{\pgfqpoint{4.080463in}{1.030191in}}%
\pgfpathcurveto{\pgfqpoint{4.086287in}{1.024368in}}{\pgfqpoint{4.094187in}{1.021095in}}{\pgfqpoint{4.102424in}{1.021095in}}%
\pgfpathclose%
\pgfusepath{stroke,fill}%
\end{pgfscope}%
\begin{pgfscope}%
\pgfpathrectangle{\pgfqpoint{3.696748in}{0.557870in}}{\pgfqpoint{2.424965in}{1.684734in}}%
\pgfusepath{clip}%
\pgfsetbuttcap%
\pgfsetroundjoin%
\definecolor{currentfill}{rgb}{0.298039,0.447059,0.690196}%
\pgfsetfillcolor{currentfill}%
\pgfsetlinewidth{1.003750pt}%
\definecolor{currentstroke}{rgb}{0.298039,0.447059,0.690196}%
\pgfsetstrokecolor{currentstroke}%
\pgfsetdash{}{0pt}%
\pgfpathmoveto{\pgfqpoint{4.011516in}{1.098448in}}%
\pgfpathcurveto{\pgfqpoint{4.019752in}{1.098448in}}{\pgfqpoint{4.027652in}{1.101720in}}{\pgfqpoint{4.033476in}{1.107544in}}%
\pgfpathcurveto{\pgfqpoint{4.039300in}{1.113368in}}{\pgfqpoint{4.042572in}{1.121268in}}{\pgfqpoint{4.042572in}{1.129504in}}%
\pgfpathcurveto{\pgfqpoint{4.042572in}{1.137740in}}{\pgfqpoint{4.039300in}{1.145640in}}{\pgfqpoint{4.033476in}{1.151464in}}%
\pgfpathcurveto{\pgfqpoint{4.027652in}{1.157288in}}{\pgfqpoint{4.019752in}{1.160561in}}{\pgfqpoint{4.011516in}{1.160561in}}%
\pgfpathcurveto{\pgfqpoint{4.003280in}{1.160561in}}{\pgfqpoint{3.995380in}{1.157288in}}{\pgfqpoint{3.989556in}{1.151464in}}%
\pgfpathcurveto{\pgfqpoint{3.983732in}{1.145640in}}{\pgfqpoint{3.980459in}{1.137740in}}{\pgfqpoint{3.980459in}{1.129504in}}%
\pgfpathcurveto{\pgfqpoint{3.980459in}{1.121268in}}{\pgfqpoint{3.983732in}{1.113368in}}{\pgfqpoint{3.989556in}{1.107544in}}%
\pgfpathcurveto{\pgfqpoint{3.995380in}{1.101720in}}{\pgfqpoint{4.003280in}{1.098448in}}{\pgfqpoint{4.011516in}{1.098448in}}%
\pgfpathclose%
\pgfusepath{stroke,fill}%
\end{pgfscope}%
\begin{pgfscope}%
\pgfpathrectangle{\pgfqpoint{3.696748in}{0.557870in}}{\pgfqpoint{2.424965in}{1.684734in}}%
\pgfusepath{clip}%
\pgfsetbuttcap%
\pgfsetroundjoin%
\definecolor{currentfill}{rgb}{0.298039,0.447059,0.690196}%
\pgfsetfillcolor{currentfill}%
\pgfsetlinewidth{1.003750pt}%
\definecolor{currentstroke}{rgb}{0.298039,0.447059,0.690196}%
\pgfsetstrokecolor{currentstroke}%
\pgfsetdash{}{0pt}%
\pgfpathmoveto{\pgfqpoint{3.873109in}{1.244154in}}%
\pgfpathcurveto{\pgfqpoint{3.881345in}{1.244154in}}{\pgfqpoint{3.889245in}{1.247426in}}{\pgfqpoint{3.895069in}{1.253250in}}%
\pgfpathcurveto{\pgfqpoint{3.900893in}{1.259074in}}{\pgfqpoint{3.904165in}{1.266974in}}{\pgfqpoint{3.904165in}{1.275211in}}%
\pgfpathcurveto{\pgfqpoint{3.904165in}{1.283447in}}{\pgfqpoint{3.900893in}{1.291347in}}{\pgfqpoint{3.895069in}{1.297171in}}%
\pgfpathcurveto{\pgfqpoint{3.889245in}{1.302995in}}{\pgfqpoint{3.881345in}{1.306267in}}{\pgfqpoint{3.873109in}{1.306267in}}%
\pgfpathcurveto{\pgfqpoint{3.864873in}{1.306267in}}{\pgfqpoint{3.856972in}{1.302995in}}{\pgfqpoint{3.851149in}{1.297171in}}%
\pgfpathcurveto{\pgfqpoint{3.845325in}{1.291347in}}{\pgfqpoint{3.842052in}{1.283447in}}{\pgfqpoint{3.842052in}{1.275211in}}%
\pgfpathcurveto{\pgfqpoint{3.842052in}{1.266974in}}{\pgfqpoint{3.845325in}{1.259074in}}{\pgfqpoint{3.851149in}{1.253250in}}%
\pgfpathcurveto{\pgfqpoint{3.856972in}{1.247426in}}{\pgfqpoint{3.864873in}{1.244154in}}{\pgfqpoint{3.873109in}{1.244154in}}%
\pgfpathclose%
\pgfusepath{stroke,fill}%
\end{pgfscope}%
\begin{pgfscope}%
\pgfpathrectangle{\pgfqpoint{3.696748in}{0.557870in}}{\pgfqpoint{2.424965in}{1.684734in}}%
\pgfusepath{clip}%
\pgfsetbuttcap%
\pgfsetroundjoin%
\definecolor{currentfill}{rgb}{0.298039,0.447059,0.690196}%
\pgfsetfillcolor{currentfill}%
\pgfsetlinewidth{1.003750pt}%
\definecolor{currentstroke}{rgb}{0.298039,0.447059,0.690196}%
\pgfsetstrokecolor{currentstroke}%
\pgfsetdash{}{0pt}%
\pgfpathmoveto{\pgfqpoint{3.806973in}{1.246655in}}%
\pgfpathcurveto{\pgfqpoint{3.815210in}{1.246655in}}{\pgfqpoint{3.823110in}{1.249927in}}{\pgfqpoint{3.828934in}{1.255751in}}%
\pgfpathcurveto{\pgfqpoint{3.834758in}{1.261575in}}{\pgfqpoint{3.838030in}{1.269475in}}{\pgfqpoint{3.838030in}{1.277711in}}%
\pgfpathcurveto{\pgfqpoint{3.838030in}{1.285947in}}{\pgfqpoint{3.834758in}{1.293847in}}{\pgfqpoint{3.828934in}{1.299671in}}%
\pgfpathcurveto{\pgfqpoint{3.823110in}{1.305495in}}{\pgfqpoint{3.815210in}{1.308768in}}{\pgfqpoint{3.806973in}{1.308768in}}%
\pgfpathcurveto{\pgfqpoint{3.798737in}{1.308768in}}{\pgfqpoint{3.790837in}{1.305495in}}{\pgfqpoint{3.785013in}{1.299671in}}%
\pgfpathcurveto{\pgfqpoint{3.779189in}{1.293847in}}{\pgfqpoint{3.775917in}{1.285947in}}{\pgfqpoint{3.775917in}{1.277711in}}%
\pgfpathcurveto{\pgfqpoint{3.775917in}{1.269475in}}{\pgfqpoint{3.779189in}{1.261575in}}{\pgfqpoint{3.785013in}{1.255751in}}%
\pgfpathcurveto{\pgfqpoint{3.790837in}{1.249927in}}{\pgfqpoint{3.798737in}{1.246655in}}{\pgfqpoint{3.806973in}{1.246655in}}%
\pgfpathclose%
\pgfusepath{stroke,fill}%
\end{pgfscope}%
\begin{pgfscope}%
\pgfsetrectcap%
\pgfsetmiterjoin%
\pgfsetlinewidth{1.254687pt}%
\definecolor{currentstroke}{rgb}{1.000000,1.000000,1.000000}%
\pgfsetstrokecolor{currentstroke}%
\pgfsetdash{}{0pt}%
\pgfpathmoveto{\pgfqpoint{3.696748in}{0.557870in}}%
\pgfpathlineto{\pgfqpoint{3.696748in}{2.242604in}}%
\pgfusepath{stroke}%
\end{pgfscope}%
\begin{pgfscope}%
\pgfsetrectcap%
\pgfsetmiterjoin%
\pgfsetlinewidth{1.254687pt}%
\definecolor{currentstroke}{rgb}{1.000000,1.000000,1.000000}%
\pgfsetstrokecolor{currentstroke}%
\pgfsetdash{}{0pt}%
\pgfpathmoveto{\pgfqpoint{6.121713in}{0.557870in}}%
\pgfpathlineto{\pgfqpoint{6.121713in}{2.242604in}}%
\pgfusepath{stroke}%
\end{pgfscope}%
\begin{pgfscope}%
\pgfsetrectcap%
\pgfsetmiterjoin%
\pgfsetlinewidth{1.254687pt}%
\definecolor{currentstroke}{rgb}{1.000000,1.000000,1.000000}%
\pgfsetstrokecolor{currentstroke}%
\pgfsetdash{}{0pt}%
\pgfpathmoveto{\pgfqpoint{3.696748in}{0.557870in}}%
\pgfpathlineto{\pgfqpoint{6.121713in}{0.557870in}}%
\pgfusepath{stroke}%
\end{pgfscope}%
\begin{pgfscope}%
\pgfsetrectcap%
\pgfsetmiterjoin%
\pgfsetlinewidth{1.254687pt}%
\definecolor{currentstroke}{rgb}{1.000000,1.000000,1.000000}%
\pgfsetstrokecolor{currentstroke}%
\pgfsetdash{}{0pt}%
\pgfpathmoveto{\pgfqpoint{3.696748in}{2.242604in}}%
\pgfpathlineto{\pgfqpoint{6.121713in}{2.242604in}}%
\pgfusepath{stroke}%
\end{pgfscope}%
\begin{pgfscope}%
\definecolor{textcolor}{rgb}{0.150000,0.150000,0.150000}%
\pgfsetstrokecolor{textcolor}%
\pgfsetfillcolor{textcolor}%
\pgftext[x=4.909230in,y=2.325938in,,base]{\color{textcolor}\sffamily\fontsize{11.000000}{13.200000}\selectfont (b)}%
\end{pgfscope}%
\end{pgfpicture}%
\makeatother%
\endgroup%

    \caption{(a) Distribution plot of \acrshort{dor} of all \acrshort{ann} models evaluated at two cluster centers when trained to predict heart failure.
             (b) Scatter plot of the same models sensitivity, and specificity.}
    \label{fig:dl_hf_dor_sens_spec_dist}
\end{figure}

From the distribution plot in figure \ref{fig:dl_hf_dor_sens_spec_dist}a one can see that the most frequent \acrshort{dor} by \acrshort{ann} models when training them to predict heart failure is zero. The highest \acrshort{dor} of $1.36$ is attained by using only the \acrshort{gls} curve from the \acrshort{4ch} view as input, as can be seen from table \ref{tab:dl_hf_dor_sens_spec_dist}. In the scatterplot in figure \ref{fig:dl_hf_dor_sens_spec_dist}b one can see that sensitivity scores vary between $0.15$ and $0.65$, and the specificity scores vary between $0$ and $0.68$. The majority of the \acrshort{ann} variations acheive a sensitivity, specificity and accuracy below $0.50$. The accuracy of the model variations are also fairly low, $0.54$ being the highest accuracy acheived. Since the heart failure dataset is fairly evenly distribution (recall figure \ref{fig:hf_ind_dist}) an accuracy of $0.54$ is not much better than what could be acheived by randomly guessing the label. The 11 highest \acrshort{dor}s attained by \acrshort{ann} models trained to classify heart failure are acheived using only curves from single views as input, and only \acrshort{gls}, or \acrshort{rls} curves. \textit{Gls/4CH/upsampled} will be considered the best model variation of the \acrshort{ann}s at predicting heart failure since it acheives the highest accuracy and \acrshort{dor} . \bigskip

\begin{table*}
    \centering
    \ra{1.3}
    \begin{tabular}{lrrrr}
        \toprule
        Dataset-model         &  Accuracy &  Sensitivity &  Specificity &  \acrshort{dor} \\
        \midrule
        gls/4CH/upsampled     &      0.54 &         0.46 &         0.61 & 1.36 \\
        rls/APLAX/regular     &      0.53 &         0.48 &         0.58 & 1.30 \\
        rls/4CH/regular       &      0.52 &         0.36 &         0.68 & 1.20 \\
        gls/APLAX/downsampled &      0.52 &         0.63 &         0.40 & 1.15 \\
        gls/2CH/downsampled   &      0.51 &         0.61 &         0.40 & 1.03 \\
        \bottomrule
    \end{tabular}
    \caption{The accuracy, \acrshort{dor}, sensitivity and specicity scores of the five best performing variations of the \acrshort{ann} in terms of \acrshort{dor}, at detecting heart failure.
             The \textbf{Dataset-model} column indicates \textit{Dataset used}$/$\textit{View used}$/$\textit{Whether curve has been upsampled, downsampled or is regular}.}
    \label{tab:dl_hf_dor_sens_spec_dist}
\end{table*}
\newpage

\subsection{Peak-value Supervised Classifiers}

\begin{figure}[htb]
    \centering
    % \includegraphics[width=\textwidth]{results/pvmlc_hf_dor_sens_spec_dist.png}
    %% Creator: Matplotlib, PGF backend
%%
%% To include the figure in your LaTeX document, write
%%   \input{<filename>.pgf}
%%
%% Make sure the required packages are loaded in your preamble
%%   \usepackage{pgf}
%%
%% Figures using additional raster images can only be included by \input if
%% they are in the same directory as the main LaTeX file. For loading figures
%% from other directories you can use the `import` package
%%   \usepackage{import}
%% and then include the figures with
%%   \import{<path to file>}{<filename>.pgf}
%%
%% Matplotlib used the following preamble
%%
\begingroup%
\makeatletter%
\begin{pgfpicture}%
\pgfpathrectangle{\pgfpointorigin}{\pgfqpoint{6.480559in}{2.540000in}}%
\pgfusepath{use as bounding box, clip}%
\begin{pgfscope}%
\pgfsetbuttcap%
\pgfsetmiterjoin%
\definecolor{currentfill}{rgb}{1.000000,1.000000,1.000000}%
\pgfsetfillcolor{currentfill}%
\pgfsetlinewidth{0.000000pt}%
\definecolor{currentstroke}{rgb}{1.000000,1.000000,1.000000}%
\pgfsetstrokecolor{currentstroke}%
\pgfsetdash{}{0pt}%
\pgfpathmoveto{\pgfqpoint{0.000000in}{0.000000in}}%
\pgfpathlineto{\pgfqpoint{6.480559in}{0.000000in}}%
\pgfpathlineto{\pgfqpoint{6.480559in}{2.540000in}}%
\pgfpathlineto{\pgfqpoint{0.000000in}{2.540000in}}%
\pgfpathclose%
\pgfusepath{fill}%
\end{pgfscope}%
\begin{pgfscope}%
\pgfsetbuttcap%
\pgfsetmiterjoin%
\definecolor{currentfill}{rgb}{0.917647,0.917647,0.949020}%
\pgfsetfillcolor{currentfill}%
\pgfsetlinewidth{0.000000pt}%
\definecolor{currentstroke}{rgb}{0.000000,0.000000,0.000000}%
\pgfsetstrokecolor{currentstroke}%
\pgfsetstrokeopacity{0.000000}%
\pgfsetdash{}{0pt}%
\pgfpathmoveto{\pgfqpoint{0.693056in}{0.557870in}}%
\pgfpathlineto{\pgfqpoint{3.177165in}{0.557870in}}%
\pgfpathlineto{\pgfqpoint{3.177165in}{2.242604in}}%
\pgfpathlineto{\pgfqpoint{0.693056in}{2.242604in}}%
\pgfpathclose%
\pgfusepath{fill}%
\end{pgfscope}%
\begin{pgfscope}%
\pgfpathrectangle{\pgfqpoint{0.693056in}{0.557870in}}{\pgfqpoint{2.484109in}{1.684734in}}%
\pgfusepath{clip}%
\pgfsetroundcap%
\pgfsetroundjoin%
\pgfsetlinewidth{1.003750pt}%
\definecolor{currentstroke}{rgb}{1.000000,1.000000,1.000000}%
\pgfsetstrokecolor{currentstroke}%
\pgfsetdash{}{0pt}%
\pgfpathmoveto{\pgfqpoint{0.824323in}{0.557870in}}%
\pgfpathlineto{\pgfqpoint{0.824323in}{2.242604in}}%
\pgfusepath{stroke}%
\end{pgfscope}%
\begin{pgfscope}%
\definecolor{textcolor}{rgb}{0.150000,0.150000,0.150000}%
\pgfsetstrokecolor{textcolor}%
\pgfsetfillcolor{textcolor}%
\pgftext[x=0.824323in,y=0.425926in,,top]{\color{textcolor}\sffamily\fontsize{11.000000}{13.200000}\selectfont \(\displaystyle 2\)}%
\end{pgfscope}%
\begin{pgfscope}%
\pgfpathrectangle{\pgfqpoint{0.693056in}{0.557870in}}{\pgfqpoint{2.484109in}{1.684734in}}%
\pgfusepath{clip}%
\pgfsetroundcap%
\pgfsetroundjoin%
\pgfsetlinewidth{1.003750pt}%
\definecolor{currentstroke}{rgb}{1.000000,1.000000,1.000000}%
\pgfsetstrokecolor{currentstroke}%
\pgfsetdash{}{0pt}%
\pgfpathmoveto{\pgfqpoint{1.429968in}{0.557870in}}%
\pgfpathlineto{\pgfqpoint{1.429968in}{2.242604in}}%
\pgfusepath{stroke}%
\end{pgfscope}%
\begin{pgfscope}%
\definecolor{textcolor}{rgb}{0.150000,0.150000,0.150000}%
\pgfsetstrokecolor{textcolor}%
\pgfsetfillcolor{textcolor}%
\pgftext[x=1.429968in,y=0.425926in,,top]{\color{textcolor}\sffamily\fontsize{11.000000}{13.200000}\selectfont \(\displaystyle 4\)}%
\end{pgfscope}%
\begin{pgfscope}%
\pgfpathrectangle{\pgfqpoint{0.693056in}{0.557870in}}{\pgfqpoint{2.484109in}{1.684734in}}%
\pgfusepath{clip}%
\pgfsetroundcap%
\pgfsetroundjoin%
\pgfsetlinewidth{1.003750pt}%
\definecolor{currentstroke}{rgb}{1.000000,1.000000,1.000000}%
\pgfsetstrokecolor{currentstroke}%
\pgfsetdash{}{0pt}%
\pgfpathmoveto{\pgfqpoint{2.035614in}{0.557870in}}%
\pgfpathlineto{\pgfqpoint{2.035614in}{2.242604in}}%
\pgfusepath{stroke}%
\end{pgfscope}%
\begin{pgfscope}%
\definecolor{textcolor}{rgb}{0.150000,0.150000,0.150000}%
\pgfsetstrokecolor{textcolor}%
\pgfsetfillcolor{textcolor}%
\pgftext[x=2.035614in,y=0.425926in,,top]{\color{textcolor}\sffamily\fontsize{11.000000}{13.200000}\selectfont \(\displaystyle 6\)}%
\end{pgfscope}%
\begin{pgfscope}%
\pgfpathrectangle{\pgfqpoint{0.693056in}{0.557870in}}{\pgfqpoint{2.484109in}{1.684734in}}%
\pgfusepath{clip}%
\pgfsetroundcap%
\pgfsetroundjoin%
\pgfsetlinewidth{1.003750pt}%
\definecolor{currentstroke}{rgb}{1.000000,1.000000,1.000000}%
\pgfsetstrokecolor{currentstroke}%
\pgfsetdash{}{0pt}%
\pgfpathmoveto{\pgfqpoint{2.641260in}{0.557870in}}%
\pgfpathlineto{\pgfqpoint{2.641260in}{2.242604in}}%
\pgfusepath{stroke}%
\end{pgfscope}%
\begin{pgfscope}%
\definecolor{textcolor}{rgb}{0.150000,0.150000,0.150000}%
\pgfsetstrokecolor{textcolor}%
\pgfsetfillcolor{textcolor}%
\pgftext[x=2.641260in,y=0.425926in,,top]{\color{textcolor}\sffamily\fontsize{11.000000}{13.200000}\selectfont \(\displaystyle 8\)}%
\end{pgfscope}%
\begin{pgfscope}%
\definecolor{textcolor}{rgb}{0.150000,0.150000,0.150000}%
\pgfsetstrokecolor{textcolor}%
\pgfsetfillcolor{textcolor}%
\pgftext[x=1.935110in,y=0.235185in,,top]{\color{textcolor}\sffamily\fontsize{11.000000}{13.200000}\selectfont DOR}%
\end{pgfscope}%
\begin{pgfscope}%
\pgfpathrectangle{\pgfqpoint{0.693056in}{0.557870in}}{\pgfqpoint{2.484109in}{1.684734in}}%
\pgfusepath{clip}%
\pgfsetroundcap%
\pgfsetroundjoin%
\pgfsetlinewidth{1.003750pt}%
\definecolor{currentstroke}{rgb}{1.000000,1.000000,1.000000}%
\pgfsetstrokecolor{currentstroke}%
\pgfsetdash{}{0pt}%
\pgfpathmoveto{\pgfqpoint{0.693056in}{0.557870in}}%
\pgfpathlineto{\pgfqpoint{3.177165in}{0.557870in}}%
\pgfusepath{stroke}%
\end{pgfscope}%
\begin{pgfscope}%
\definecolor{textcolor}{rgb}{0.150000,0.150000,0.150000}%
\pgfsetstrokecolor{textcolor}%
\pgfsetfillcolor{textcolor}%
\pgftext[x=0.366783in,y=0.505064in,left,base]{\color{textcolor}\sffamily\fontsize{11.000000}{13.200000}\selectfont \(\displaystyle 0.0\)}%
\end{pgfscope}%
\begin{pgfscope}%
\pgfpathrectangle{\pgfqpoint{0.693056in}{0.557870in}}{\pgfqpoint{2.484109in}{1.684734in}}%
\pgfusepath{clip}%
\pgfsetroundcap%
\pgfsetroundjoin%
\pgfsetlinewidth{1.003750pt}%
\definecolor{currentstroke}{rgb}{1.000000,1.000000,1.000000}%
\pgfsetstrokecolor{currentstroke}%
\pgfsetdash{}{0pt}%
\pgfpathmoveto{\pgfqpoint{0.693056in}{0.958997in}}%
\pgfpathlineto{\pgfqpoint{3.177165in}{0.958997in}}%
\pgfusepath{stroke}%
\end{pgfscope}%
\begin{pgfscope}%
\definecolor{textcolor}{rgb}{0.150000,0.150000,0.150000}%
\pgfsetstrokecolor{textcolor}%
\pgfsetfillcolor{textcolor}%
\pgftext[x=0.366783in,y=0.906191in,left,base]{\color{textcolor}\sffamily\fontsize{11.000000}{13.200000}\selectfont \(\displaystyle 2.5\)}%
\end{pgfscope}%
\begin{pgfscope}%
\pgfpathrectangle{\pgfqpoint{0.693056in}{0.557870in}}{\pgfqpoint{2.484109in}{1.684734in}}%
\pgfusepath{clip}%
\pgfsetroundcap%
\pgfsetroundjoin%
\pgfsetlinewidth{1.003750pt}%
\definecolor{currentstroke}{rgb}{1.000000,1.000000,1.000000}%
\pgfsetstrokecolor{currentstroke}%
\pgfsetdash{}{0pt}%
\pgfpathmoveto{\pgfqpoint{0.693056in}{1.360125in}}%
\pgfpathlineto{\pgfqpoint{3.177165in}{1.360125in}}%
\pgfusepath{stroke}%
\end{pgfscope}%
\begin{pgfscope}%
\definecolor{textcolor}{rgb}{0.150000,0.150000,0.150000}%
\pgfsetstrokecolor{textcolor}%
\pgfsetfillcolor{textcolor}%
\pgftext[x=0.366783in,y=1.307318in,left,base]{\color{textcolor}\sffamily\fontsize{11.000000}{13.200000}\selectfont \(\displaystyle 5.0\)}%
\end{pgfscope}%
\begin{pgfscope}%
\pgfpathrectangle{\pgfqpoint{0.693056in}{0.557870in}}{\pgfqpoint{2.484109in}{1.684734in}}%
\pgfusepath{clip}%
\pgfsetroundcap%
\pgfsetroundjoin%
\pgfsetlinewidth{1.003750pt}%
\definecolor{currentstroke}{rgb}{1.000000,1.000000,1.000000}%
\pgfsetstrokecolor{currentstroke}%
\pgfsetdash{}{0pt}%
\pgfpathmoveto{\pgfqpoint{0.693056in}{1.761252in}}%
\pgfpathlineto{\pgfqpoint{3.177165in}{1.761252in}}%
\pgfusepath{stroke}%
\end{pgfscope}%
\begin{pgfscope}%
\definecolor{textcolor}{rgb}{0.150000,0.150000,0.150000}%
\pgfsetstrokecolor{textcolor}%
\pgfsetfillcolor{textcolor}%
\pgftext[x=0.366783in,y=1.708445in,left,base]{\color{textcolor}\sffamily\fontsize{11.000000}{13.200000}\selectfont \(\displaystyle 7.5\)}%
\end{pgfscope}%
\begin{pgfscope}%
\pgfpathrectangle{\pgfqpoint{0.693056in}{0.557870in}}{\pgfqpoint{2.484109in}{1.684734in}}%
\pgfusepath{clip}%
\pgfsetroundcap%
\pgfsetroundjoin%
\pgfsetlinewidth{1.003750pt}%
\definecolor{currentstroke}{rgb}{1.000000,1.000000,1.000000}%
\pgfsetstrokecolor{currentstroke}%
\pgfsetdash{}{0pt}%
\pgfpathmoveto{\pgfqpoint{0.693056in}{2.162379in}}%
\pgfpathlineto{\pgfqpoint{3.177165in}{2.162379in}}%
\pgfusepath{stroke}%
\end{pgfscope}%
\begin{pgfscope}%
\definecolor{textcolor}{rgb}{0.150000,0.150000,0.150000}%
\pgfsetstrokecolor{textcolor}%
\pgfsetfillcolor{textcolor}%
\pgftext[x=0.290741in,y=2.109572in,left,base]{\color{textcolor}\sffamily\fontsize{11.000000}{13.200000}\selectfont \(\displaystyle 10.0\)}%
\end{pgfscope}%
\begin{pgfscope}%
\definecolor{textcolor}{rgb}{0.150000,0.150000,0.150000}%
\pgfsetstrokecolor{textcolor}%
\pgfsetfillcolor{textcolor}%
\pgftext[x=0.235185in,y=1.400237in,,bottom,rotate=90.000000]{\color{textcolor}\sffamily\fontsize{11.000000}{13.200000}\selectfont Occurance}%
\end{pgfscope}%
\begin{pgfscope}%
\pgfpathrectangle{\pgfqpoint{0.693056in}{0.557870in}}{\pgfqpoint{2.484109in}{1.684734in}}%
\pgfusepath{clip}%
\pgfsetbuttcap%
\pgfsetmiterjoin%
\definecolor{currentfill}{rgb}{0.298039,0.447059,0.690196}%
\pgfsetfillcolor{currentfill}%
\pgfsetfillopacity{0.400000}%
\pgfsetlinewidth{1.003750pt}%
\definecolor{currentstroke}{rgb}{1.000000,1.000000,1.000000}%
\pgfsetstrokecolor{currentstroke}%
\pgfsetstrokeopacity{0.400000}%
\pgfsetdash{}{0pt}%
\pgfpathmoveto{\pgfqpoint{0.805970in}{0.557870in}}%
\pgfpathlineto{\pgfqpoint{1.031798in}{0.557870in}}%
\pgfpathlineto{\pgfqpoint{1.031798in}{1.360125in}}%
\pgfpathlineto{\pgfqpoint{0.805970in}{1.360125in}}%
\pgfpathclose%
\pgfusepath{stroke,fill}%
\end{pgfscope}%
\begin{pgfscope}%
\pgfpathrectangle{\pgfqpoint{0.693056in}{0.557870in}}{\pgfqpoint{2.484109in}{1.684734in}}%
\pgfusepath{clip}%
\pgfsetbuttcap%
\pgfsetmiterjoin%
\definecolor{currentfill}{rgb}{0.298039,0.447059,0.690196}%
\pgfsetfillcolor{currentfill}%
\pgfsetfillopacity{0.400000}%
\pgfsetlinewidth{1.003750pt}%
\definecolor{currentstroke}{rgb}{1.000000,1.000000,1.000000}%
\pgfsetstrokecolor{currentstroke}%
\pgfsetstrokeopacity{0.400000}%
\pgfsetdash{}{0pt}%
\pgfpathmoveto{\pgfqpoint{1.031798in}{0.557870in}}%
\pgfpathlineto{\pgfqpoint{1.257626in}{0.557870in}}%
\pgfpathlineto{\pgfqpoint{1.257626in}{1.520575in}}%
\pgfpathlineto{\pgfqpoint{1.031798in}{1.520575in}}%
\pgfpathclose%
\pgfusepath{stroke,fill}%
\end{pgfscope}%
\begin{pgfscope}%
\pgfpathrectangle{\pgfqpoint{0.693056in}{0.557870in}}{\pgfqpoint{2.484109in}{1.684734in}}%
\pgfusepath{clip}%
\pgfsetbuttcap%
\pgfsetmiterjoin%
\definecolor{currentfill}{rgb}{0.298039,0.447059,0.690196}%
\pgfsetfillcolor{currentfill}%
\pgfsetfillopacity{0.400000}%
\pgfsetlinewidth{1.003750pt}%
\definecolor{currentstroke}{rgb}{1.000000,1.000000,1.000000}%
\pgfsetstrokecolor{currentstroke}%
\pgfsetstrokeopacity{0.400000}%
\pgfsetdash{}{0pt}%
\pgfpathmoveto{\pgfqpoint{1.257626in}{0.557870in}}%
\pgfpathlineto{\pgfqpoint{1.483454in}{0.557870in}}%
\pgfpathlineto{\pgfqpoint{1.483454in}{0.718321in}}%
\pgfpathlineto{\pgfqpoint{1.257626in}{0.718321in}}%
\pgfpathclose%
\pgfusepath{stroke,fill}%
\end{pgfscope}%
\begin{pgfscope}%
\pgfpathrectangle{\pgfqpoint{0.693056in}{0.557870in}}{\pgfqpoint{2.484109in}{1.684734in}}%
\pgfusepath{clip}%
\pgfsetbuttcap%
\pgfsetmiterjoin%
\definecolor{currentfill}{rgb}{0.298039,0.447059,0.690196}%
\pgfsetfillcolor{currentfill}%
\pgfsetfillopacity{0.400000}%
\pgfsetlinewidth{1.003750pt}%
\definecolor{currentstroke}{rgb}{1.000000,1.000000,1.000000}%
\pgfsetstrokecolor{currentstroke}%
\pgfsetstrokeopacity{0.400000}%
\pgfsetdash{}{0pt}%
\pgfpathmoveto{\pgfqpoint{1.483454in}{0.557870in}}%
\pgfpathlineto{\pgfqpoint{1.709282in}{0.557870in}}%
\pgfpathlineto{\pgfqpoint{1.709282in}{2.001928in}}%
\pgfpathlineto{\pgfqpoint{1.483454in}{2.001928in}}%
\pgfpathclose%
\pgfusepath{stroke,fill}%
\end{pgfscope}%
\begin{pgfscope}%
\pgfpathrectangle{\pgfqpoint{0.693056in}{0.557870in}}{\pgfqpoint{2.484109in}{1.684734in}}%
\pgfusepath{clip}%
\pgfsetbuttcap%
\pgfsetmiterjoin%
\definecolor{currentfill}{rgb}{0.298039,0.447059,0.690196}%
\pgfsetfillcolor{currentfill}%
\pgfsetfillopacity{0.400000}%
\pgfsetlinewidth{1.003750pt}%
\definecolor{currentstroke}{rgb}{1.000000,1.000000,1.000000}%
\pgfsetstrokecolor{currentstroke}%
\pgfsetstrokeopacity{0.400000}%
\pgfsetdash{}{0pt}%
\pgfpathmoveto{\pgfqpoint{1.709282in}{0.557870in}}%
\pgfpathlineto{\pgfqpoint{1.935110in}{0.557870in}}%
\pgfpathlineto{\pgfqpoint{1.935110in}{1.841477in}}%
\pgfpathlineto{\pgfqpoint{1.709282in}{1.841477in}}%
\pgfpathclose%
\pgfusepath{stroke,fill}%
\end{pgfscope}%
\begin{pgfscope}%
\pgfpathrectangle{\pgfqpoint{0.693056in}{0.557870in}}{\pgfqpoint{2.484109in}{1.684734in}}%
\pgfusepath{clip}%
\pgfsetbuttcap%
\pgfsetmiterjoin%
\definecolor{currentfill}{rgb}{0.298039,0.447059,0.690196}%
\pgfsetfillcolor{currentfill}%
\pgfsetfillopacity{0.400000}%
\pgfsetlinewidth{1.003750pt}%
\definecolor{currentstroke}{rgb}{1.000000,1.000000,1.000000}%
\pgfsetstrokecolor{currentstroke}%
\pgfsetstrokeopacity{0.400000}%
\pgfsetdash{}{0pt}%
\pgfpathmoveto{\pgfqpoint{1.935110in}{0.557870in}}%
\pgfpathlineto{\pgfqpoint{2.160938in}{0.557870in}}%
\pgfpathlineto{\pgfqpoint{2.160938in}{1.841477in}}%
\pgfpathlineto{\pgfqpoint{1.935110in}{1.841477in}}%
\pgfpathclose%
\pgfusepath{stroke,fill}%
\end{pgfscope}%
\begin{pgfscope}%
\pgfpathrectangle{\pgfqpoint{0.693056in}{0.557870in}}{\pgfqpoint{2.484109in}{1.684734in}}%
\pgfusepath{clip}%
\pgfsetbuttcap%
\pgfsetmiterjoin%
\definecolor{currentfill}{rgb}{0.298039,0.447059,0.690196}%
\pgfsetfillcolor{currentfill}%
\pgfsetfillopacity{0.400000}%
\pgfsetlinewidth{1.003750pt}%
\definecolor{currentstroke}{rgb}{1.000000,1.000000,1.000000}%
\pgfsetstrokecolor{currentstroke}%
\pgfsetstrokeopacity{0.400000}%
\pgfsetdash{}{0pt}%
\pgfpathmoveto{\pgfqpoint{2.160938in}{0.557870in}}%
\pgfpathlineto{\pgfqpoint{2.386766in}{0.557870in}}%
\pgfpathlineto{\pgfqpoint{2.386766in}{2.162379in}}%
\pgfpathlineto{\pgfqpoint{2.160938in}{2.162379in}}%
\pgfpathclose%
\pgfusepath{stroke,fill}%
\end{pgfscope}%
\begin{pgfscope}%
\pgfpathrectangle{\pgfqpoint{0.693056in}{0.557870in}}{\pgfqpoint{2.484109in}{1.684734in}}%
\pgfusepath{clip}%
\pgfsetbuttcap%
\pgfsetmiterjoin%
\definecolor{currentfill}{rgb}{0.298039,0.447059,0.690196}%
\pgfsetfillcolor{currentfill}%
\pgfsetfillopacity{0.400000}%
\pgfsetlinewidth{1.003750pt}%
\definecolor{currentstroke}{rgb}{1.000000,1.000000,1.000000}%
\pgfsetstrokecolor{currentstroke}%
\pgfsetstrokeopacity{0.400000}%
\pgfsetdash{}{0pt}%
\pgfpathmoveto{\pgfqpoint{2.386766in}{0.557870in}}%
\pgfpathlineto{\pgfqpoint{2.612594in}{0.557870in}}%
\pgfpathlineto{\pgfqpoint{2.612594in}{1.841477in}}%
\pgfpathlineto{\pgfqpoint{2.386766in}{1.841477in}}%
\pgfpathclose%
\pgfusepath{stroke,fill}%
\end{pgfscope}%
\begin{pgfscope}%
\pgfpathrectangle{\pgfqpoint{0.693056in}{0.557870in}}{\pgfqpoint{2.484109in}{1.684734in}}%
\pgfusepath{clip}%
\pgfsetbuttcap%
\pgfsetmiterjoin%
\definecolor{currentfill}{rgb}{0.298039,0.447059,0.690196}%
\pgfsetfillcolor{currentfill}%
\pgfsetfillopacity{0.400000}%
\pgfsetlinewidth{1.003750pt}%
\definecolor{currentstroke}{rgb}{1.000000,1.000000,1.000000}%
\pgfsetstrokecolor{currentstroke}%
\pgfsetstrokeopacity{0.400000}%
\pgfsetdash{}{0pt}%
\pgfpathmoveto{\pgfqpoint{2.612594in}{0.557870in}}%
\pgfpathlineto{\pgfqpoint{2.838422in}{0.557870in}}%
\pgfpathlineto{\pgfqpoint{2.838422in}{0.718321in}}%
\pgfpathlineto{\pgfqpoint{2.612594in}{0.718321in}}%
\pgfpathclose%
\pgfusepath{stroke,fill}%
\end{pgfscope}%
\begin{pgfscope}%
\pgfpathrectangle{\pgfqpoint{0.693056in}{0.557870in}}{\pgfqpoint{2.484109in}{1.684734in}}%
\pgfusepath{clip}%
\pgfsetbuttcap%
\pgfsetmiterjoin%
\definecolor{currentfill}{rgb}{0.298039,0.447059,0.690196}%
\pgfsetfillcolor{currentfill}%
\pgfsetfillopacity{0.400000}%
\pgfsetlinewidth{1.003750pt}%
\definecolor{currentstroke}{rgb}{1.000000,1.000000,1.000000}%
\pgfsetstrokecolor{currentstroke}%
\pgfsetstrokeopacity{0.400000}%
\pgfsetdash{}{0pt}%
\pgfpathmoveto{\pgfqpoint{2.838422in}{0.557870in}}%
\pgfpathlineto{\pgfqpoint{3.064250in}{0.557870in}}%
\pgfpathlineto{\pgfqpoint{3.064250in}{1.520575in}}%
\pgfpathlineto{\pgfqpoint{2.838422in}{1.520575in}}%
\pgfpathclose%
\pgfusepath{stroke,fill}%
\end{pgfscope}%
\begin{pgfscope}%
\pgfsetrectcap%
\pgfsetmiterjoin%
\pgfsetlinewidth{1.254687pt}%
\definecolor{currentstroke}{rgb}{1.000000,1.000000,1.000000}%
\pgfsetstrokecolor{currentstroke}%
\pgfsetdash{}{0pt}%
\pgfpathmoveto{\pgfqpoint{0.693056in}{0.557870in}}%
\pgfpathlineto{\pgfqpoint{0.693056in}{2.242604in}}%
\pgfusepath{stroke}%
\end{pgfscope}%
\begin{pgfscope}%
\pgfsetrectcap%
\pgfsetmiterjoin%
\pgfsetlinewidth{1.254687pt}%
\definecolor{currentstroke}{rgb}{1.000000,1.000000,1.000000}%
\pgfsetstrokecolor{currentstroke}%
\pgfsetdash{}{0pt}%
\pgfpathmoveto{\pgfqpoint{3.177165in}{0.557870in}}%
\pgfpathlineto{\pgfqpoint{3.177165in}{2.242604in}}%
\pgfusepath{stroke}%
\end{pgfscope}%
\begin{pgfscope}%
\pgfsetrectcap%
\pgfsetmiterjoin%
\pgfsetlinewidth{1.254687pt}%
\definecolor{currentstroke}{rgb}{1.000000,1.000000,1.000000}%
\pgfsetstrokecolor{currentstroke}%
\pgfsetdash{}{0pt}%
\pgfpathmoveto{\pgfqpoint{0.693056in}{0.557870in}}%
\pgfpathlineto{\pgfqpoint{3.177165in}{0.557870in}}%
\pgfusepath{stroke}%
\end{pgfscope}%
\begin{pgfscope}%
\pgfsetrectcap%
\pgfsetmiterjoin%
\pgfsetlinewidth{1.254687pt}%
\definecolor{currentstroke}{rgb}{1.000000,1.000000,1.000000}%
\pgfsetstrokecolor{currentstroke}%
\pgfsetdash{}{0pt}%
\pgfpathmoveto{\pgfqpoint{0.693056in}{2.242604in}}%
\pgfpathlineto{\pgfqpoint{3.177165in}{2.242604in}}%
\pgfusepath{stroke}%
\end{pgfscope}%
\begin{pgfscope}%
\definecolor{textcolor}{rgb}{0.150000,0.150000,0.150000}%
\pgfsetstrokecolor{textcolor}%
\pgfsetfillcolor{textcolor}%
\pgftext[x=1.935110in,y=2.325938in,,base]{\color{textcolor}\sffamily\fontsize{11.000000}{13.200000}\selectfont (a)}%
\end{pgfscope}%
\begin{pgfscope}%
\pgfsetbuttcap%
\pgfsetmiterjoin%
\definecolor{currentfill}{rgb}{0.917647,0.917647,0.949020}%
\pgfsetfillcolor{currentfill}%
\pgfsetlinewidth{0.000000pt}%
\definecolor{currentstroke}{rgb}{0.000000,0.000000,0.000000}%
\pgfsetstrokecolor{currentstroke}%
\pgfsetstrokeopacity{0.000000}%
\pgfsetdash{}{0pt}%
\pgfpathmoveto{\pgfqpoint{3.874179in}{0.557870in}}%
\pgfpathlineto{\pgfqpoint{6.358287in}{0.557870in}}%
\pgfpathlineto{\pgfqpoint{6.358287in}{2.242604in}}%
\pgfpathlineto{\pgfqpoint{3.874179in}{2.242604in}}%
\pgfpathclose%
\pgfusepath{fill}%
\end{pgfscope}%
\begin{pgfscope}%
\pgfpathrectangle{\pgfqpoint{3.874179in}{0.557870in}}{\pgfqpoint{2.484109in}{1.684734in}}%
\pgfusepath{clip}%
\pgfsetroundcap%
\pgfsetroundjoin%
\pgfsetlinewidth{1.003750pt}%
\definecolor{currentstroke}{rgb}{1.000000,1.000000,1.000000}%
\pgfsetstrokecolor{currentstroke}%
\pgfsetdash{}{0pt}%
\pgfpathmoveto{\pgfqpoint{3.987093in}{0.557870in}}%
\pgfpathlineto{\pgfqpoint{3.987093in}{2.242604in}}%
\pgfusepath{stroke}%
\end{pgfscope}%
\begin{pgfscope}%
\definecolor{textcolor}{rgb}{0.150000,0.150000,0.150000}%
\pgfsetstrokecolor{textcolor}%
\pgfsetfillcolor{textcolor}%
\pgftext[x=3.987093in,y=0.425926in,,top]{\color{textcolor}\sffamily\fontsize{11.000000}{13.200000}\selectfont \(\displaystyle 0.00\)}%
\end{pgfscope}%
\begin{pgfscope}%
\pgfpathrectangle{\pgfqpoint{3.874179in}{0.557870in}}{\pgfqpoint{2.484109in}{1.684734in}}%
\pgfusepath{clip}%
\pgfsetroundcap%
\pgfsetroundjoin%
\pgfsetlinewidth{1.003750pt}%
\definecolor{currentstroke}{rgb}{1.000000,1.000000,1.000000}%
\pgfsetstrokecolor{currentstroke}%
\pgfsetdash{}{0pt}%
\pgfpathmoveto{\pgfqpoint{4.551663in}{0.557870in}}%
\pgfpathlineto{\pgfqpoint{4.551663in}{2.242604in}}%
\pgfusepath{stroke}%
\end{pgfscope}%
\begin{pgfscope}%
\definecolor{textcolor}{rgb}{0.150000,0.150000,0.150000}%
\pgfsetstrokecolor{textcolor}%
\pgfsetfillcolor{textcolor}%
\pgftext[x=4.551663in,y=0.425926in,,top]{\color{textcolor}\sffamily\fontsize{11.000000}{13.200000}\selectfont \(\displaystyle 0.25\)}%
\end{pgfscope}%
\begin{pgfscope}%
\pgfpathrectangle{\pgfqpoint{3.874179in}{0.557870in}}{\pgfqpoint{2.484109in}{1.684734in}}%
\pgfusepath{clip}%
\pgfsetroundcap%
\pgfsetroundjoin%
\pgfsetlinewidth{1.003750pt}%
\definecolor{currentstroke}{rgb}{1.000000,1.000000,1.000000}%
\pgfsetstrokecolor{currentstroke}%
\pgfsetdash{}{0pt}%
\pgfpathmoveto{\pgfqpoint{5.116233in}{0.557870in}}%
\pgfpathlineto{\pgfqpoint{5.116233in}{2.242604in}}%
\pgfusepath{stroke}%
\end{pgfscope}%
\begin{pgfscope}%
\definecolor{textcolor}{rgb}{0.150000,0.150000,0.150000}%
\pgfsetstrokecolor{textcolor}%
\pgfsetfillcolor{textcolor}%
\pgftext[x=5.116233in,y=0.425926in,,top]{\color{textcolor}\sffamily\fontsize{11.000000}{13.200000}\selectfont \(\displaystyle 0.50\)}%
\end{pgfscope}%
\begin{pgfscope}%
\pgfpathrectangle{\pgfqpoint{3.874179in}{0.557870in}}{\pgfqpoint{2.484109in}{1.684734in}}%
\pgfusepath{clip}%
\pgfsetroundcap%
\pgfsetroundjoin%
\pgfsetlinewidth{1.003750pt}%
\definecolor{currentstroke}{rgb}{1.000000,1.000000,1.000000}%
\pgfsetstrokecolor{currentstroke}%
\pgfsetdash{}{0pt}%
\pgfpathmoveto{\pgfqpoint{5.680803in}{0.557870in}}%
\pgfpathlineto{\pgfqpoint{5.680803in}{2.242604in}}%
\pgfusepath{stroke}%
\end{pgfscope}%
\begin{pgfscope}%
\definecolor{textcolor}{rgb}{0.150000,0.150000,0.150000}%
\pgfsetstrokecolor{textcolor}%
\pgfsetfillcolor{textcolor}%
\pgftext[x=5.680803in,y=0.425926in,,top]{\color{textcolor}\sffamily\fontsize{11.000000}{13.200000}\selectfont \(\displaystyle 0.75\)}%
\end{pgfscope}%
\begin{pgfscope}%
\pgfpathrectangle{\pgfqpoint{3.874179in}{0.557870in}}{\pgfqpoint{2.484109in}{1.684734in}}%
\pgfusepath{clip}%
\pgfsetroundcap%
\pgfsetroundjoin%
\pgfsetlinewidth{1.003750pt}%
\definecolor{currentstroke}{rgb}{1.000000,1.000000,1.000000}%
\pgfsetstrokecolor{currentstroke}%
\pgfsetdash{}{0pt}%
\pgfpathmoveto{\pgfqpoint{6.245373in}{0.557870in}}%
\pgfpathlineto{\pgfqpoint{6.245373in}{2.242604in}}%
\pgfusepath{stroke}%
\end{pgfscope}%
\begin{pgfscope}%
\definecolor{textcolor}{rgb}{0.150000,0.150000,0.150000}%
\pgfsetstrokecolor{textcolor}%
\pgfsetfillcolor{textcolor}%
\pgftext[x=6.245373in,y=0.425926in,,top]{\color{textcolor}\sffamily\fontsize{11.000000}{13.200000}\selectfont \(\displaystyle 1.00\)}%
\end{pgfscope}%
\begin{pgfscope}%
\definecolor{textcolor}{rgb}{0.150000,0.150000,0.150000}%
\pgfsetstrokecolor{textcolor}%
\pgfsetfillcolor{textcolor}%
\pgftext[x=5.116233in,y=0.235185in,,top]{\color{textcolor}\sffamily\fontsize{11.000000}{13.200000}\selectfont Specificity}%
\end{pgfscope}%
\begin{pgfscope}%
\pgfpathrectangle{\pgfqpoint{3.874179in}{0.557870in}}{\pgfqpoint{2.484109in}{1.684734in}}%
\pgfusepath{clip}%
\pgfsetroundcap%
\pgfsetroundjoin%
\pgfsetlinewidth{1.003750pt}%
\definecolor{currentstroke}{rgb}{1.000000,1.000000,1.000000}%
\pgfsetstrokecolor{currentstroke}%
\pgfsetdash{}{0pt}%
\pgfpathmoveto{\pgfqpoint{3.874179in}{0.634449in}}%
\pgfpathlineto{\pgfqpoint{6.358287in}{0.634449in}}%
\pgfusepath{stroke}%
\end{pgfscope}%
\begin{pgfscope}%
\definecolor{textcolor}{rgb}{0.150000,0.150000,0.150000}%
\pgfsetstrokecolor{textcolor}%
\pgfsetfillcolor{textcolor}%
\pgftext[x=3.471863in,y=0.581642in,left,base]{\color{textcolor}\sffamily\fontsize{11.000000}{13.200000}\selectfont \(\displaystyle 0.00\)}%
\end{pgfscope}%
\begin{pgfscope}%
\pgfpathrectangle{\pgfqpoint{3.874179in}{0.557870in}}{\pgfqpoint{2.484109in}{1.684734in}}%
\pgfusepath{clip}%
\pgfsetroundcap%
\pgfsetroundjoin%
\pgfsetlinewidth{1.003750pt}%
\definecolor{currentstroke}{rgb}{1.000000,1.000000,1.000000}%
\pgfsetstrokecolor{currentstroke}%
\pgfsetdash{}{0pt}%
\pgfpathmoveto{\pgfqpoint{3.874179in}{1.017343in}}%
\pgfpathlineto{\pgfqpoint{6.358287in}{1.017343in}}%
\pgfusepath{stroke}%
\end{pgfscope}%
\begin{pgfscope}%
\definecolor{textcolor}{rgb}{0.150000,0.150000,0.150000}%
\pgfsetstrokecolor{textcolor}%
\pgfsetfillcolor{textcolor}%
\pgftext[x=3.471863in,y=0.964536in,left,base]{\color{textcolor}\sffamily\fontsize{11.000000}{13.200000}\selectfont \(\displaystyle 0.25\)}%
\end{pgfscope}%
\begin{pgfscope}%
\pgfpathrectangle{\pgfqpoint{3.874179in}{0.557870in}}{\pgfqpoint{2.484109in}{1.684734in}}%
\pgfusepath{clip}%
\pgfsetroundcap%
\pgfsetroundjoin%
\pgfsetlinewidth{1.003750pt}%
\definecolor{currentstroke}{rgb}{1.000000,1.000000,1.000000}%
\pgfsetstrokecolor{currentstroke}%
\pgfsetdash{}{0pt}%
\pgfpathmoveto{\pgfqpoint{3.874179in}{1.400237in}}%
\pgfpathlineto{\pgfqpoint{6.358287in}{1.400237in}}%
\pgfusepath{stroke}%
\end{pgfscope}%
\begin{pgfscope}%
\definecolor{textcolor}{rgb}{0.150000,0.150000,0.150000}%
\pgfsetstrokecolor{textcolor}%
\pgfsetfillcolor{textcolor}%
\pgftext[x=3.471863in,y=1.347431in,left,base]{\color{textcolor}\sffamily\fontsize{11.000000}{13.200000}\selectfont \(\displaystyle 0.50\)}%
\end{pgfscope}%
\begin{pgfscope}%
\pgfpathrectangle{\pgfqpoint{3.874179in}{0.557870in}}{\pgfqpoint{2.484109in}{1.684734in}}%
\pgfusepath{clip}%
\pgfsetroundcap%
\pgfsetroundjoin%
\pgfsetlinewidth{1.003750pt}%
\definecolor{currentstroke}{rgb}{1.000000,1.000000,1.000000}%
\pgfsetstrokecolor{currentstroke}%
\pgfsetdash{}{0pt}%
\pgfpathmoveto{\pgfqpoint{3.874179in}{1.783131in}}%
\pgfpathlineto{\pgfqpoint{6.358287in}{1.783131in}}%
\pgfusepath{stroke}%
\end{pgfscope}%
\begin{pgfscope}%
\definecolor{textcolor}{rgb}{0.150000,0.150000,0.150000}%
\pgfsetstrokecolor{textcolor}%
\pgfsetfillcolor{textcolor}%
\pgftext[x=3.471863in,y=1.730325in,left,base]{\color{textcolor}\sffamily\fontsize{11.000000}{13.200000}\selectfont \(\displaystyle 0.75\)}%
\end{pgfscope}%
\begin{pgfscope}%
\pgfpathrectangle{\pgfqpoint{3.874179in}{0.557870in}}{\pgfqpoint{2.484109in}{1.684734in}}%
\pgfusepath{clip}%
\pgfsetroundcap%
\pgfsetroundjoin%
\pgfsetlinewidth{1.003750pt}%
\definecolor{currentstroke}{rgb}{1.000000,1.000000,1.000000}%
\pgfsetstrokecolor{currentstroke}%
\pgfsetdash{}{0pt}%
\pgfpathmoveto{\pgfqpoint{3.874179in}{2.166025in}}%
\pgfpathlineto{\pgfqpoint{6.358287in}{2.166025in}}%
\pgfusepath{stroke}%
\end{pgfscope}%
\begin{pgfscope}%
\definecolor{textcolor}{rgb}{0.150000,0.150000,0.150000}%
\pgfsetstrokecolor{textcolor}%
\pgfsetfillcolor{textcolor}%
\pgftext[x=3.471863in,y=2.113219in,left,base]{\color{textcolor}\sffamily\fontsize{11.000000}{13.200000}\selectfont \(\displaystyle 1.00\)}%
\end{pgfscope}%
\begin{pgfscope}%
\definecolor{textcolor}{rgb}{0.150000,0.150000,0.150000}%
\pgfsetstrokecolor{textcolor}%
\pgfsetfillcolor{textcolor}%
\pgftext[x=3.416308in,y=1.400237in,,bottom,rotate=90.000000]{\color{textcolor}\sffamily\fontsize{11.000000}{13.200000}\selectfont Sensitivity}%
\end{pgfscope}%
\begin{pgfscope}%
\pgfpathrectangle{\pgfqpoint{3.874179in}{0.557870in}}{\pgfqpoint{2.484109in}{1.684734in}}%
\pgfusepath{clip}%
\pgfsetbuttcap%
\pgfsetroundjoin%
\definecolor{currentfill}{rgb}{0.298039,0.447059,0.690196}%
\pgfsetfillcolor{currentfill}%
\pgfsetlinewidth{1.003750pt}%
\definecolor{currentstroke}{rgb}{0.298039,0.447059,0.690196}%
\pgfsetstrokecolor{currentstroke}%
\pgfsetdash{}{0pt}%
\pgfpathmoveto{\pgfqpoint{5.629478in}{1.248267in}}%
\pgfpathcurveto{\pgfqpoint{5.637715in}{1.248267in}}{\pgfqpoint{5.645615in}{1.251539in}}{\pgfqpoint{5.651439in}{1.257363in}}%
\pgfpathcurveto{\pgfqpoint{5.657263in}{1.263187in}}{\pgfqpoint{5.660535in}{1.271087in}}{\pgfqpoint{5.660535in}{1.279323in}}%
\pgfpathcurveto{\pgfqpoint{5.660535in}{1.287560in}}{\pgfqpoint{5.657263in}{1.295460in}}{\pgfqpoint{5.651439in}{1.301284in}}%
\pgfpathcurveto{\pgfqpoint{5.645615in}{1.307107in}}{\pgfqpoint{5.637715in}{1.310380in}}{\pgfqpoint{5.629478in}{1.310380in}}%
\pgfpathcurveto{\pgfqpoint{5.621242in}{1.310380in}}{\pgfqpoint{5.613342in}{1.307107in}}{\pgfqpoint{5.607518in}{1.301284in}}%
\pgfpathcurveto{\pgfqpoint{5.601694in}{1.295460in}}{\pgfqpoint{5.598422in}{1.287560in}}{\pgfqpoint{5.598422in}{1.279323in}}%
\pgfpathcurveto{\pgfqpoint{5.598422in}{1.271087in}}{\pgfqpoint{5.601694in}{1.263187in}}{\pgfqpoint{5.607518in}{1.257363in}}%
\pgfpathcurveto{\pgfqpoint{5.613342in}{1.251539in}}{\pgfqpoint{5.621242in}{1.248267in}}{\pgfqpoint{5.629478in}{1.248267in}}%
\pgfpathclose%
\pgfusepath{stroke,fill}%
\end{pgfscope}%
\begin{pgfscope}%
\pgfpathrectangle{\pgfqpoint{3.874179in}{0.557870in}}{\pgfqpoint{2.484109in}{1.684734in}}%
\pgfusepath{clip}%
\pgfsetbuttcap%
\pgfsetroundjoin%
\definecolor{currentfill}{rgb}{0.298039,0.447059,0.690196}%
\pgfsetfillcolor{currentfill}%
\pgfsetlinewidth{1.003750pt}%
\definecolor{currentstroke}{rgb}{0.298039,0.447059,0.690196}%
\pgfsetstrokecolor{currentstroke}%
\pgfsetdash{}{0pt}%
\pgfpathmoveto{\pgfqpoint{5.216601in}{1.575038in}}%
\pgfpathcurveto{\pgfqpoint{5.224837in}{1.575038in}}{\pgfqpoint{5.232737in}{1.578310in}}{\pgfqpoint{5.238561in}{1.584134in}}%
\pgfpathcurveto{\pgfqpoint{5.244385in}{1.589958in}}{\pgfqpoint{5.247657in}{1.597858in}}{\pgfqpoint{5.247657in}{1.606094in}}%
\pgfpathcurveto{\pgfqpoint{5.247657in}{1.614331in}}{\pgfqpoint{5.244385in}{1.622231in}}{\pgfqpoint{5.238561in}{1.628055in}}%
\pgfpathcurveto{\pgfqpoint{5.232737in}{1.633878in}}{\pgfqpoint{5.224837in}{1.637151in}}{\pgfqpoint{5.216601in}{1.637151in}}%
\pgfpathcurveto{\pgfqpoint{5.208365in}{1.637151in}}{\pgfqpoint{5.200465in}{1.633878in}}{\pgfqpoint{5.194641in}{1.628055in}}%
\pgfpathcurveto{\pgfqpoint{5.188817in}{1.622231in}}{\pgfqpoint{5.185544in}{1.614331in}}{\pgfqpoint{5.185544in}{1.606094in}}%
\pgfpathcurveto{\pgfqpoint{5.185544in}{1.597858in}}{\pgfqpoint{5.188817in}{1.589958in}}{\pgfqpoint{5.194641in}{1.584134in}}%
\pgfpathcurveto{\pgfqpoint{5.200465in}{1.578310in}}{\pgfqpoint{5.208365in}{1.575038in}}{\pgfqpoint{5.216601in}{1.575038in}}%
\pgfpathclose%
\pgfusepath{stroke,fill}%
\end{pgfscope}%
\begin{pgfscope}%
\pgfpathrectangle{\pgfqpoint{3.874179in}{0.557870in}}{\pgfqpoint{2.484109in}{1.684734in}}%
\pgfusepath{clip}%
\pgfsetbuttcap%
\pgfsetroundjoin%
\definecolor{currentfill}{rgb}{0.298039,0.447059,0.690196}%
\pgfsetfillcolor{currentfill}%
\pgfsetlinewidth{1.003750pt}%
\definecolor{currentstroke}{rgb}{0.298039,0.447059,0.690196}%
\pgfsetstrokecolor{currentstroke}%
\pgfsetdash{}{0pt}%
\pgfpathmoveto{\pgfqpoint{5.675100in}{1.280511in}}%
\pgfpathcurveto{\pgfqpoint{5.683337in}{1.280511in}}{\pgfqpoint{5.691237in}{1.283783in}}{\pgfqpoint{5.697061in}{1.289607in}}%
\pgfpathcurveto{\pgfqpoint{5.702884in}{1.295431in}}{\pgfqpoint{5.706157in}{1.303331in}}{\pgfqpoint{5.706157in}{1.311567in}}%
\pgfpathcurveto{\pgfqpoint{5.706157in}{1.319803in}}{\pgfqpoint{5.702884in}{1.327703in}}{\pgfqpoint{5.697061in}{1.333527in}}%
\pgfpathcurveto{\pgfqpoint{5.691237in}{1.339351in}}{\pgfqpoint{5.683337in}{1.342624in}}{\pgfqpoint{5.675100in}{1.342624in}}%
\pgfpathcurveto{\pgfqpoint{5.666864in}{1.342624in}}{\pgfqpoint{5.658964in}{1.339351in}}{\pgfqpoint{5.653140in}{1.333527in}}%
\pgfpathcurveto{\pgfqpoint{5.647316in}{1.327703in}}{\pgfqpoint{5.644044in}{1.319803in}}{\pgfqpoint{5.644044in}{1.311567in}}%
\pgfpathcurveto{\pgfqpoint{5.644044in}{1.303331in}}{\pgfqpoint{5.647316in}{1.295431in}}{\pgfqpoint{5.653140in}{1.289607in}}%
\pgfpathcurveto{\pgfqpoint{5.658964in}{1.283783in}}{\pgfqpoint{5.666864in}{1.280511in}}{\pgfqpoint{5.675100in}{1.280511in}}%
\pgfpathclose%
\pgfusepath{stroke,fill}%
\end{pgfscope}%
\begin{pgfscope}%
\pgfpathrectangle{\pgfqpoint{3.874179in}{0.557870in}}{\pgfqpoint{2.484109in}{1.684734in}}%
\pgfusepath{clip}%
\pgfsetbuttcap%
\pgfsetroundjoin%
\definecolor{currentfill}{rgb}{0.298039,0.447059,0.690196}%
\pgfsetfillcolor{currentfill}%
\pgfsetlinewidth{1.003750pt}%
\definecolor{currentstroke}{rgb}{0.298039,0.447059,0.690196}%
\pgfsetstrokecolor{currentstroke}%
\pgfsetdash{}{0pt}%
\pgfpathmoveto{\pgfqpoint{6.154130in}{0.748489in}}%
\pgfpathcurveto{\pgfqpoint{6.162366in}{0.748489in}}{\pgfqpoint{6.170266in}{0.751762in}}{\pgfqpoint{6.176090in}{0.757586in}}%
\pgfpathcurveto{\pgfqpoint{6.181914in}{0.763409in}}{\pgfqpoint{6.185186in}{0.771310in}}{\pgfqpoint{6.185186in}{0.779546in}}%
\pgfpathcurveto{\pgfqpoint{6.185186in}{0.787782in}}{\pgfqpoint{6.181914in}{0.795682in}}{\pgfqpoint{6.176090in}{0.801506in}}%
\pgfpathcurveto{\pgfqpoint{6.170266in}{0.807330in}}{\pgfqpoint{6.162366in}{0.810602in}}{\pgfqpoint{6.154130in}{0.810602in}}%
\pgfpathcurveto{\pgfqpoint{6.145893in}{0.810602in}}{\pgfqpoint{6.137993in}{0.807330in}}{\pgfqpoint{6.132169in}{0.801506in}}%
\pgfpathcurveto{\pgfqpoint{6.126345in}{0.795682in}}{\pgfqpoint{6.123073in}{0.787782in}}{\pgfqpoint{6.123073in}{0.779546in}}%
\pgfpathcurveto{\pgfqpoint{6.123073in}{0.771310in}}{\pgfqpoint{6.126345in}{0.763409in}}{\pgfqpoint{6.132169in}{0.757586in}}%
\pgfpathcurveto{\pgfqpoint{6.137993in}{0.751762in}}{\pgfqpoint{6.145893in}{0.748489in}}{\pgfqpoint{6.154130in}{0.748489in}}%
\pgfpathclose%
\pgfusepath{stroke,fill}%
\end{pgfscope}%
\begin{pgfscope}%
\pgfpathrectangle{\pgfqpoint{3.874179in}{0.557870in}}{\pgfqpoint{2.484109in}{1.684734in}}%
\pgfusepath{clip}%
\pgfsetbuttcap%
\pgfsetroundjoin%
\definecolor{currentfill}{rgb}{0.298039,0.447059,0.690196}%
\pgfsetfillcolor{currentfill}%
\pgfsetlinewidth{1.003750pt}%
\definecolor{currentstroke}{rgb}{0.298039,0.447059,0.690196}%
\pgfsetstrokecolor{currentstroke}%
\pgfsetdash{}{0pt}%
\pgfpathmoveto{\pgfqpoint{5.467521in}{1.492695in}}%
\pgfpathcurveto{\pgfqpoint{5.475757in}{1.492695in}}{\pgfqpoint{5.483657in}{1.495967in}}{\pgfqpoint{5.489481in}{1.501791in}}%
\pgfpathcurveto{\pgfqpoint{5.495305in}{1.507615in}}{\pgfqpoint{5.498577in}{1.515515in}}{\pgfqpoint{5.498577in}{1.523751in}}%
\pgfpathcurveto{\pgfqpoint{5.498577in}{1.531988in}}{\pgfqpoint{5.495305in}{1.539888in}}{\pgfqpoint{5.489481in}{1.545712in}}%
\pgfpathcurveto{\pgfqpoint{5.483657in}{1.551536in}}{\pgfqpoint{5.475757in}{1.554808in}}{\pgfqpoint{5.467521in}{1.554808in}}%
\pgfpathcurveto{\pgfqpoint{5.459285in}{1.554808in}}{\pgfqpoint{5.451385in}{1.551536in}}{\pgfqpoint{5.445561in}{1.545712in}}%
\pgfpathcurveto{\pgfqpoint{5.439737in}{1.539888in}}{\pgfqpoint{5.436464in}{1.531988in}}{\pgfqpoint{5.436464in}{1.523751in}}%
\pgfpathcurveto{\pgfqpoint{5.436464in}{1.515515in}}{\pgfqpoint{5.439737in}{1.507615in}}{\pgfqpoint{5.445561in}{1.501791in}}%
\pgfpathcurveto{\pgfqpoint{5.451385in}{1.495967in}}{\pgfqpoint{5.459285in}{1.492695in}}{\pgfqpoint{5.467521in}{1.492695in}}%
\pgfpathclose%
\pgfusepath{stroke,fill}%
\end{pgfscope}%
\begin{pgfscope}%
\pgfpathrectangle{\pgfqpoint{3.874179in}{0.557870in}}{\pgfqpoint{2.484109in}{1.684734in}}%
\pgfusepath{clip}%
\pgfsetbuttcap%
\pgfsetroundjoin%
\definecolor{currentfill}{rgb}{0.298039,0.447059,0.690196}%
\pgfsetfillcolor{currentfill}%
\pgfsetlinewidth{1.003750pt}%
\definecolor{currentstroke}{rgb}{0.298039,0.447059,0.690196}%
\pgfsetstrokecolor{currentstroke}%
\pgfsetdash{}{0pt}%
\pgfpathmoveto{\pgfqpoint{5.720722in}{1.296632in}}%
\pgfpathcurveto{\pgfqpoint{5.728958in}{1.296632in}}{\pgfqpoint{5.736858in}{1.299905in}}{\pgfqpoint{5.742682in}{1.305729in}}%
\pgfpathcurveto{\pgfqpoint{5.748506in}{1.311553in}}{\pgfqpoint{5.751779in}{1.319453in}}{\pgfqpoint{5.751779in}{1.327689in}}%
\pgfpathcurveto{\pgfqpoint{5.751779in}{1.335925in}}{\pgfqpoint{5.748506in}{1.343825in}}{\pgfqpoint{5.742682in}{1.349649in}}%
\pgfpathcurveto{\pgfqpoint{5.736858in}{1.355473in}}{\pgfqpoint{5.728958in}{1.358745in}}{\pgfqpoint{5.720722in}{1.358745in}}%
\pgfpathcurveto{\pgfqpoint{5.712486in}{1.358745in}}{\pgfqpoint{5.704586in}{1.355473in}}{\pgfqpoint{5.698762in}{1.349649in}}%
\pgfpathcurveto{\pgfqpoint{5.692938in}{1.343825in}}{\pgfqpoint{5.689666in}{1.335925in}}{\pgfqpoint{5.689666in}{1.327689in}}%
\pgfpathcurveto{\pgfqpoint{5.689666in}{1.319453in}}{\pgfqpoint{5.692938in}{1.311553in}}{\pgfqpoint{5.698762in}{1.305729in}}%
\pgfpathcurveto{\pgfqpoint{5.704586in}{1.299905in}}{\pgfqpoint{5.712486in}{1.296632in}}{\pgfqpoint{5.720722in}{1.296632in}}%
\pgfpathclose%
\pgfusepath{stroke,fill}%
\end{pgfscope}%
\begin{pgfscope}%
\pgfpathrectangle{\pgfqpoint{3.874179in}{0.557870in}}{\pgfqpoint{2.484109in}{1.684734in}}%
\pgfusepath{clip}%
\pgfsetbuttcap%
\pgfsetroundjoin%
\definecolor{currentfill}{rgb}{0.298039,0.447059,0.690196}%
\pgfsetfillcolor{currentfill}%
\pgfsetlinewidth{1.003750pt}%
\definecolor{currentstroke}{rgb}{0.298039,0.447059,0.690196}%
\pgfsetstrokecolor{currentstroke}%
\pgfsetdash{}{0pt}%
\pgfpathmoveto{\pgfqpoint{5.424180in}{1.602948in}}%
\pgfpathcurveto{\pgfqpoint{5.432417in}{1.602948in}}{\pgfqpoint{5.440317in}{1.606220in}}{\pgfqpoint{5.446141in}{1.612044in}}%
\pgfpathcurveto{\pgfqpoint{5.451964in}{1.617868in}}{\pgfqpoint{5.455237in}{1.625768in}}{\pgfqpoint{5.455237in}{1.634004in}}%
\pgfpathcurveto{\pgfqpoint{5.455237in}{1.642240in}}{\pgfqpoint{5.451964in}{1.650140in}}{\pgfqpoint{5.446141in}{1.655964in}}%
\pgfpathcurveto{\pgfqpoint{5.440317in}{1.661788in}}{\pgfqpoint{5.432417in}{1.665061in}}{\pgfqpoint{5.424180in}{1.665061in}}%
\pgfpathcurveto{\pgfqpoint{5.415944in}{1.665061in}}{\pgfqpoint{5.408044in}{1.661788in}}{\pgfqpoint{5.402220in}{1.655964in}}%
\pgfpathcurveto{\pgfqpoint{5.396396in}{1.650140in}}{\pgfqpoint{5.393124in}{1.642240in}}{\pgfqpoint{5.393124in}{1.634004in}}%
\pgfpathcurveto{\pgfqpoint{5.393124in}{1.625768in}}{\pgfqpoint{5.396396in}{1.617868in}}{\pgfqpoint{5.402220in}{1.612044in}}%
\pgfpathcurveto{\pgfqpoint{5.408044in}{1.606220in}}{\pgfqpoint{5.415944in}{1.602948in}}{\pgfqpoint{5.424180in}{1.602948in}}%
\pgfpathclose%
\pgfusepath{stroke,fill}%
\end{pgfscope}%
\begin{pgfscope}%
\pgfpathrectangle{\pgfqpoint{3.874179in}{0.557870in}}{\pgfqpoint{2.484109in}{1.684734in}}%
\pgfusepath{clip}%
\pgfsetbuttcap%
\pgfsetroundjoin%
\definecolor{currentfill}{rgb}{0.298039,0.447059,0.690196}%
\pgfsetfillcolor{currentfill}%
\pgfsetlinewidth{1.003750pt}%
\definecolor{currentstroke}{rgb}{0.298039,0.447059,0.690196}%
\pgfsetstrokecolor{currentstroke}%
\pgfsetdash{}{0pt}%
\pgfpathmoveto{\pgfqpoint{5.342061in}{1.657381in}}%
\pgfpathcurveto{\pgfqpoint{5.350297in}{1.657381in}}{\pgfqpoint{5.358197in}{1.660653in}}{\pgfqpoint{5.364021in}{1.666477in}}%
\pgfpathcurveto{\pgfqpoint{5.369845in}{1.672301in}}{\pgfqpoint{5.373117in}{1.680201in}}{\pgfqpoint{5.373117in}{1.688437in}}%
\pgfpathcurveto{\pgfqpoint{5.373117in}{1.696673in}}{\pgfqpoint{5.369845in}{1.704573in}}{\pgfqpoint{5.364021in}{1.710397in}}%
\pgfpathcurveto{\pgfqpoint{5.358197in}{1.716221in}}{\pgfqpoint{5.350297in}{1.719494in}}{\pgfqpoint{5.342061in}{1.719494in}}%
\pgfpathcurveto{\pgfqpoint{5.333825in}{1.719494in}}{\pgfqpoint{5.325925in}{1.716221in}}{\pgfqpoint{5.320101in}{1.710397in}}%
\pgfpathcurveto{\pgfqpoint{5.314277in}{1.704573in}}{\pgfqpoint{5.311004in}{1.696673in}}{\pgfqpoint{5.311004in}{1.688437in}}%
\pgfpathcurveto{\pgfqpoint{5.311004in}{1.680201in}}{\pgfqpoint{5.314277in}{1.672301in}}{\pgfqpoint{5.320101in}{1.666477in}}%
\pgfpathcurveto{\pgfqpoint{5.325925in}{1.660653in}}{\pgfqpoint{5.333825in}{1.657381in}}{\pgfqpoint{5.342061in}{1.657381in}}%
\pgfpathclose%
\pgfusepath{stroke,fill}%
\end{pgfscope}%
\begin{pgfscope}%
\pgfpathrectangle{\pgfqpoint{3.874179in}{0.557870in}}{\pgfqpoint{2.484109in}{1.684734in}}%
\pgfusepath{clip}%
\pgfsetbuttcap%
\pgfsetroundjoin%
\definecolor{currentfill}{rgb}{0.298039,0.447059,0.690196}%
\pgfsetfillcolor{currentfill}%
\pgfsetlinewidth{1.003750pt}%
\definecolor{currentstroke}{rgb}{0.298039,0.447059,0.690196}%
\pgfsetstrokecolor{currentstroke}%
\pgfsetdash{}{0pt}%
\pgfpathmoveto{\pgfqpoint{5.517705in}{1.542101in}}%
\pgfpathcurveto{\pgfqpoint{5.525941in}{1.542101in}}{\pgfqpoint{5.533841in}{1.545373in}}{\pgfqpoint{5.539665in}{1.551197in}}%
\pgfpathcurveto{\pgfqpoint{5.545489in}{1.557021in}}{\pgfqpoint{5.548761in}{1.564921in}}{\pgfqpoint{5.548761in}{1.573157in}}%
\pgfpathcurveto{\pgfqpoint{5.548761in}{1.581393in}}{\pgfqpoint{5.545489in}{1.589293in}}{\pgfqpoint{5.539665in}{1.595117in}}%
\pgfpathcurveto{\pgfqpoint{5.533841in}{1.600941in}}{\pgfqpoint{5.525941in}{1.604214in}}{\pgfqpoint{5.517705in}{1.604214in}}%
\pgfpathcurveto{\pgfqpoint{5.509469in}{1.604214in}}{\pgfqpoint{5.501569in}{1.600941in}}{\pgfqpoint{5.495745in}{1.595117in}}%
\pgfpathcurveto{\pgfqpoint{5.489921in}{1.589293in}}{\pgfqpoint{5.486648in}{1.581393in}}{\pgfqpoint{5.486648in}{1.573157in}}%
\pgfpathcurveto{\pgfqpoint{5.486648in}{1.564921in}}{\pgfqpoint{5.489921in}{1.557021in}}{\pgfqpoint{5.495745in}{1.551197in}}%
\pgfpathcurveto{\pgfqpoint{5.501569in}{1.545373in}}{\pgfqpoint{5.509469in}{1.542101in}}{\pgfqpoint{5.517705in}{1.542101in}}%
\pgfpathclose%
\pgfusepath{stroke,fill}%
\end{pgfscope}%
\begin{pgfscope}%
\pgfpathrectangle{\pgfqpoint{3.874179in}{0.557870in}}{\pgfqpoint{2.484109in}{1.684734in}}%
\pgfusepath{clip}%
\pgfsetbuttcap%
\pgfsetroundjoin%
\definecolor{currentfill}{rgb}{0.298039,0.447059,0.690196}%
\pgfsetfillcolor{currentfill}%
\pgfsetlinewidth{1.003750pt}%
\definecolor{currentstroke}{rgb}{0.298039,0.447059,0.690196}%
\pgfsetstrokecolor{currentstroke}%
\pgfsetdash{}{0pt}%
\pgfpathmoveto{\pgfqpoint{5.291877in}{1.692864in}}%
\pgfpathcurveto{\pgfqpoint{5.300113in}{1.692864in}}{\pgfqpoint{5.308013in}{1.696137in}}{\pgfqpoint{5.313837in}{1.701961in}}%
\pgfpathcurveto{\pgfqpoint{5.319661in}{1.707785in}}{\pgfqpoint{5.322933in}{1.715685in}}{\pgfqpoint{5.322933in}{1.723921in}}%
\pgfpathcurveto{\pgfqpoint{5.322933in}{1.732157in}}{\pgfqpoint{5.319661in}{1.740057in}}{\pgfqpoint{5.313837in}{1.745881in}}%
\pgfpathcurveto{\pgfqpoint{5.308013in}{1.751705in}}{\pgfqpoint{5.300113in}{1.754977in}}{\pgfqpoint{5.291877in}{1.754977in}}%
\pgfpathcurveto{\pgfqpoint{5.283641in}{1.754977in}}{\pgfqpoint{5.275741in}{1.751705in}}{\pgfqpoint{5.269917in}{1.745881in}}%
\pgfpathcurveto{\pgfqpoint{5.264093in}{1.740057in}}{\pgfqpoint{5.260820in}{1.732157in}}{\pgfqpoint{5.260820in}{1.723921in}}%
\pgfpathcurveto{\pgfqpoint{5.260820in}{1.715685in}}{\pgfqpoint{5.264093in}{1.707785in}}{\pgfqpoint{5.269917in}{1.701961in}}%
\pgfpathcurveto{\pgfqpoint{5.275741in}{1.696137in}}{\pgfqpoint{5.283641in}{1.692864in}}{\pgfqpoint{5.291877in}{1.692864in}}%
\pgfpathclose%
\pgfusepath{stroke,fill}%
\end{pgfscope}%
\begin{pgfscope}%
\pgfpathrectangle{\pgfqpoint{3.874179in}{0.557870in}}{\pgfqpoint{2.484109in}{1.684734in}}%
\pgfusepath{clip}%
\pgfsetbuttcap%
\pgfsetroundjoin%
\definecolor{currentfill}{rgb}{0.298039,0.447059,0.690196}%
\pgfsetfillcolor{currentfill}%
\pgfsetlinewidth{1.003750pt}%
\definecolor{currentstroke}{rgb}{0.298039,0.447059,0.690196}%
\pgfsetstrokecolor{currentstroke}%
\pgfsetdash{}{0pt}%
\pgfpathmoveto{\pgfqpoint{5.291877in}{1.692864in}}%
\pgfpathcurveto{\pgfqpoint{5.300113in}{1.692864in}}{\pgfqpoint{5.308013in}{1.696137in}}{\pgfqpoint{5.313837in}{1.701961in}}%
\pgfpathcurveto{\pgfqpoint{5.319661in}{1.707785in}}{\pgfqpoint{5.322933in}{1.715685in}}{\pgfqpoint{5.322933in}{1.723921in}}%
\pgfpathcurveto{\pgfqpoint{5.322933in}{1.732157in}}{\pgfqpoint{5.319661in}{1.740057in}}{\pgfqpoint{5.313837in}{1.745881in}}%
\pgfpathcurveto{\pgfqpoint{5.308013in}{1.751705in}}{\pgfqpoint{5.300113in}{1.754977in}}{\pgfqpoint{5.291877in}{1.754977in}}%
\pgfpathcurveto{\pgfqpoint{5.283641in}{1.754977in}}{\pgfqpoint{5.275741in}{1.751705in}}{\pgfqpoint{5.269917in}{1.745881in}}%
\pgfpathcurveto{\pgfqpoint{5.264093in}{1.740057in}}{\pgfqpoint{5.260820in}{1.732157in}}{\pgfqpoint{5.260820in}{1.723921in}}%
\pgfpathcurveto{\pgfqpoint{5.260820in}{1.715685in}}{\pgfqpoint{5.264093in}{1.707785in}}{\pgfqpoint{5.269917in}{1.701961in}}%
\pgfpathcurveto{\pgfqpoint{5.275741in}{1.696137in}}{\pgfqpoint{5.283641in}{1.692864in}}{\pgfqpoint{5.291877in}{1.692864in}}%
\pgfpathclose%
\pgfusepath{stroke,fill}%
\end{pgfscope}%
\begin{pgfscope}%
\pgfpathrectangle{\pgfqpoint{3.874179in}{0.557870in}}{\pgfqpoint{2.484109in}{1.684734in}}%
\pgfusepath{clip}%
\pgfsetbuttcap%
\pgfsetroundjoin%
\definecolor{currentfill}{rgb}{0.298039,0.447059,0.690196}%
\pgfsetfillcolor{currentfill}%
\pgfsetlinewidth{1.003750pt}%
\definecolor{currentstroke}{rgb}{0.298039,0.447059,0.690196}%
\pgfsetstrokecolor{currentstroke}%
\pgfsetdash{}{0pt}%
\pgfpathmoveto{\pgfqpoint{5.355747in}{1.715801in}}%
\pgfpathcurveto{\pgfqpoint{5.363984in}{1.715801in}}{\pgfqpoint{5.371884in}{1.719073in}}{\pgfqpoint{5.377708in}{1.724897in}}%
\pgfpathcurveto{\pgfqpoint{5.383532in}{1.730721in}}{\pgfqpoint{5.386804in}{1.738621in}}{\pgfqpoint{5.386804in}{1.746857in}}%
\pgfpathcurveto{\pgfqpoint{5.386804in}{1.755093in}}{\pgfqpoint{5.383532in}{1.762993in}}{\pgfqpoint{5.377708in}{1.768817in}}%
\pgfpathcurveto{\pgfqpoint{5.371884in}{1.774641in}}{\pgfqpoint{5.363984in}{1.777914in}}{\pgfqpoint{5.355747in}{1.777914in}}%
\pgfpathcurveto{\pgfqpoint{5.347511in}{1.777914in}}{\pgfqpoint{5.339611in}{1.774641in}}{\pgfqpoint{5.333787in}{1.768817in}}%
\pgfpathcurveto{\pgfqpoint{5.327963in}{1.762993in}}{\pgfqpoint{5.324691in}{1.755093in}}{\pgfqpoint{5.324691in}{1.746857in}}%
\pgfpathcurveto{\pgfqpoint{5.324691in}{1.738621in}}{\pgfqpoint{5.327963in}{1.730721in}}{\pgfqpoint{5.333787in}{1.724897in}}%
\pgfpathcurveto{\pgfqpoint{5.339611in}{1.719073in}}{\pgfqpoint{5.347511in}{1.715801in}}{\pgfqpoint{5.355747in}{1.715801in}}%
\pgfpathclose%
\pgfusepath{stroke,fill}%
\end{pgfscope}%
\begin{pgfscope}%
\pgfpathrectangle{\pgfqpoint{3.874179in}{0.557870in}}{\pgfqpoint{2.484109in}{1.684734in}}%
\pgfusepath{clip}%
\pgfsetbuttcap%
\pgfsetroundjoin%
\definecolor{currentfill}{rgb}{0.298039,0.447059,0.690196}%
\pgfsetfillcolor{currentfill}%
\pgfsetlinewidth{1.003750pt}%
\definecolor{currentstroke}{rgb}{0.298039,0.447059,0.690196}%
\pgfsetstrokecolor{currentstroke}%
\pgfsetdash{}{0pt}%
\pgfpathmoveto{\pgfqpoint{5.668257in}{1.509163in}}%
\pgfpathcurveto{\pgfqpoint{5.676493in}{1.509163in}}{\pgfqpoint{5.684393in}{1.512436in}}{\pgfqpoint{5.690217in}{1.518260in}}%
\pgfpathcurveto{\pgfqpoint{5.696041in}{1.524084in}}{\pgfqpoint{5.699314in}{1.531984in}}{\pgfqpoint{5.699314in}{1.540220in}}%
\pgfpathcurveto{\pgfqpoint{5.699314in}{1.548456in}}{\pgfqpoint{5.696041in}{1.556356in}}{\pgfqpoint{5.690217in}{1.562180in}}%
\pgfpathcurveto{\pgfqpoint{5.684393in}{1.568004in}}{\pgfqpoint{5.676493in}{1.571276in}}{\pgfqpoint{5.668257in}{1.571276in}}%
\pgfpathcurveto{\pgfqpoint{5.660021in}{1.571276in}}{\pgfqpoint{5.652121in}{1.568004in}}{\pgfqpoint{5.646297in}{1.562180in}}%
\pgfpathcurveto{\pgfqpoint{5.640473in}{1.556356in}}{\pgfqpoint{5.637201in}{1.548456in}}{\pgfqpoint{5.637201in}{1.540220in}}%
\pgfpathcurveto{\pgfqpoint{5.637201in}{1.531984in}}{\pgfqpoint{5.640473in}{1.524084in}}{\pgfqpoint{5.646297in}{1.518260in}}%
\pgfpathcurveto{\pgfqpoint{5.652121in}{1.512436in}}{\pgfqpoint{5.660021in}{1.509163in}}{\pgfqpoint{5.668257in}{1.509163in}}%
\pgfpathclose%
\pgfusepath{stroke,fill}%
\end{pgfscope}%
\begin{pgfscope}%
\pgfpathrectangle{\pgfqpoint{3.874179in}{0.557870in}}{\pgfqpoint{2.484109in}{1.684734in}}%
\pgfusepath{clip}%
\pgfsetbuttcap%
\pgfsetroundjoin%
\definecolor{currentfill}{rgb}{0.298039,0.447059,0.690196}%
\pgfsetfillcolor{currentfill}%
\pgfsetlinewidth{1.003750pt}%
\definecolor{currentstroke}{rgb}{0.298039,0.447059,0.690196}%
\pgfsetstrokecolor{currentstroke}%
\pgfsetdash{}{0pt}%
\pgfpathmoveto{\pgfqpoint{5.467521in}{1.661285in}}%
\pgfpathcurveto{\pgfqpoint{5.475757in}{1.661285in}}{\pgfqpoint{5.483657in}{1.664558in}}{\pgfqpoint{5.489481in}{1.670382in}}%
\pgfpathcurveto{\pgfqpoint{5.495305in}{1.676206in}}{\pgfqpoint{5.498577in}{1.684106in}}{\pgfqpoint{5.498577in}{1.692342in}}%
\pgfpathcurveto{\pgfqpoint{5.498577in}{1.700578in}}{\pgfqpoint{5.495305in}{1.708478in}}{\pgfqpoint{5.489481in}{1.714302in}}%
\pgfpathcurveto{\pgfqpoint{5.483657in}{1.720126in}}{\pgfqpoint{5.475757in}{1.723398in}}{\pgfqpoint{5.467521in}{1.723398in}}%
\pgfpathcurveto{\pgfqpoint{5.459285in}{1.723398in}}{\pgfqpoint{5.451385in}{1.720126in}}{\pgfqpoint{5.445561in}{1.714302in}}%
\pgfpathcurveto{\pgfqpoint{5.439737in}{1.708478in}}{\pgfqpoint{5.436464in}{1.700578in}}{\pgfqpoint{5.436464in}{1.692342in}}%
\pgfpathcurveto{\pgfqpoint{5.436464in}{1.684106in}}{\pgfqpoint{5.439737in}{1.676206in}}{\pgfqpoint{5.445561in}{1.670382in}}%
\pgfpathcurveto{\pgfqpoint{5.451385in}{1.664558in}}{\pgfqpoint{5.459285in}{1.661285in}}{\pgfqpoint{5.467521in}{1.661285in}}%
\pgfpathclose%
\pgfusepath{stroke,fill}%
\end{pgfscope}%
\begin{pgfscope}%
\pgfpathrectangle{\pgfqpoint{3.874179in}{0.557870in}}{\pgfqpoint{2.484109in}{1.684734in}}%
\pgfusepath{clip}%
\pgfsetbuttcap%
\pgfsetroundjoin%
\definecolor{currentfill}{rgb}{0.298039,0.447059,0.690196}%
\pgfsetfillcolor{currentfill}%
\pgfsetlinewidth{1.003750pt}%
\definecolor{currentstroke}{rgb}{0.298039,0.447059,0.690196}%
\pgfsetstrokecolor{currentstroke}%
\pgfsetdash{}{0pt}%
\pgfpathmoveto{\pgfqpoint{5.515424in}{1.635191in}}%
\pgfpathcurveto{\pgfqpoint{5.523660in}{1.635191in}}{\pgfqpoint{5.531560in}{1.638464in}}{\pgfqpoint{5.537384in}{1.644288in}}%
\pgfpathcurveto{\pgfqpoint{5.543208in}{1.650112in}}{\pgfqpoint{5.546480in}{1.658012in}}{\pgfqpoint{5.546480in}{1.666248in}}%
\pgfpathcurveto{\pgfqpoint{5.546480in}{1.674484in}}{\pgfqpoint{5.543208in}{1.682384in}}{\pgfqpoint{5.537384in}{1.688208in}}%
\pgfpathcurveto{\pgfqpoint{5.531560in}{1.694032in}}{\pgfqpoint{5.523660in}{1.697304in}}{\pgfqpoint{5.515424in}{1.697304in}}%
\pgfpathcurveto{\pgfqpoint{5.507188in}{1.697304in}}{\pgfqpoint{5.499288in}{1.694032in}}{\pgfqpoint{5.493464in}{1.688208in}}%
\pgfpathcurveto{\pgfqpoint{5.487640in}{1.682384in}}{\pgfqpoint{5.484367in}{1.674484in}}{\pgfqpoint{5.484367in}{1.666248in}}%
\pgfpathcurveto{\pgfqpoint{5.484367in}{1.658012in}}{\pgfqpoint{5.487640in}{1.650112in}}{\pgfqpoint{5.493464in}{1.644288in}}%
\pgfpathcurveto{\pgfqpoint{5.499288in}{1.638464in}}{\pgfqpoint{5.507188in}{1.635191in}}{\pgfqpoint{5.515424in}{1.635191in}}%
\pgfpathclose%
\pgfusepath{stroke,fill}%
\end{pgfscope}%
\begin{pgfscope}%
\pgfpathrectangle{\pgfqpoint{3.874179in}{0.557870in}}{\pgfqpoint{2.484109in}{1.684734in}}%
\pgfusepath{clip}%
\pgfsetbuttcap%
\pgfsetroundjoin%
\definecolor{currentfill}{rgb}{0.298039,0.447059,0.690196}%
\pgfsetfillcolor{currentfill}%
\pgfsetlinewidth{1.003750pt}%
\definecolor{currentstroke}{rgb}{0.298039,0.447059,0.690196}%
\pgfsetstrokecolor{currentstroke}%
\pgfsetdash{}{0pt}%
\pgfpathmoveto{\pgfqpoint{5.467521in}{1.677075in}}%
\pgfpathcurveto{\pgfqpoint{5.475757in}{1.677075in}}{\pgfqpoint{5.483657in}{1.680347in}}{\pgfqpoint{5.489481in}{1.686171in}}%
\pgfpathcurveto{\pgfqpoint{5.495305in}{1.691995in}}{\pgfqpoint{5.498577in}{1.699895in}}{\pgfqpoint{5.498577in}{1.708131in}}%
\pgfpathcurveto{\pgfqpoint{5.498577in}{1.716368in}}{\pgfqpoint{5.495305in}{1.724268in}}{\pgfqpoint{5.489481in}{1.730092in}}%
\pgfpathcurveto{\pgfqpoint{5.483657in}{1.735916in}}{\pgfqpoint{5.475757in}{1.739188in}}{\pgfqpoint{5.467521in}{1.739188in}}%
\pgfpathcurveto{\pgfqpoint{5.459285in}{1.739188in}}{\pgfqpoint{5.451385in}{1.735916in}}{\pgfqpoint{5.445561in}{1.730092in}}%
\pgfpathcurveto{\pgfqpoint{5.439737in}{1.724268in}}{\pgfqpoint{5.436464in}{1.716368in}}{\pgfqpoint{5.436464in}{1.708131in}}%
\pgfpathcurveto{\pgfqpoint{5.436464in}{1.699895in}}{\pgfqpoint{5.439737in}{1.691995in}}{\pgfqpoint{5.445561in}{1.686171in}}%
\pgfpathcurveto{\pgfqpoint{5.451385in}{1.680347in}}{\pgfqpoint{5.459285in}{1.677075in}}{\pgfqpoint{5.467521in}{1.677075in}}%
\pgfpathclose%
\pgfusepath{stroke,fill}%
\end{pgfscope}%
\begin{pgfscope}%
\pgfpathrectangle{\pgfqpoint{3.874179in}{0.557870in}}{\pgfqpoint{2.484109in}{1.684734in}}%
\pgfusepath{clip}%
\pgfsetbuttcap%
\pgfsetroundjoin%
\definecolor{currentfill}{rgb}{0.298039,0.447059,0.690196}%
\pgfsetfillcolor{currentfill}%
\pgfsetlinewidth{1.003750pt}%
\definecolor{currentstroke}{rgb}{0.298039,0.447059,0.690196}%
\pgfsetstrokecolor{currentstroke}%
\pgfsetdash{}{0pt}%
\pgfpathmoveto{\pgfqpoint{5.442429in}{1.692864in}}%
\pgfpathcurveto{\pgfqpoint{5.450665in}{1.692864in}}{\pgfqpoint{5.458565in}{1.696137in}}{\pgfqpoint{5.464389in}{1.701961in}}%
\pgfpathcurveto{\pgfqpoint{5.470213in}{1.707785in}}{\pgfqpoint{5.473485in}{1.715685in}}{\pgfqpoint{5.473485in}{1.723921in}}%
\pgfpathcurveto{\pgfqpoint{5.473485in}{1.732157in}}{\pgfqpoint{5.470213in}{1.740057in}}{\pgfqpoint{5.464389in}{1.745881in}}%
\pgfpathcurveto{\pgfqpoint{5.458565in}{1.751705in}}{\pgfqpoint{5.450665in}{1.754977in}}{\pgfqpoint{5.442429in}{1.754977in}}%
\pgfpathcurveto{\pgfqpoint{5.434193in}{1.754977in}}{\pgfqpoint{5.426293in}{1.751705in}}{\pgfqpoint{5.420469in}{1.745881in}}%
\pgfpathcurveto{\pgfqpoint{5.414645in}{1.740057in}}{\pgfqpoint{5.411372in}{1.732157in}}{\pgfqpoint{5.411372in}{1.723921in}}%
\pgfpathcurveto{\pgfqpoint{5.411372in}{1.715685in}}{\pgfqpoint{5.414645in}{1.707785in}}{\pgfqpoint{5.420469in}{1.701961in}}%
\pgfpathcurveto{\pgfqpoint{5.426293in}{1.696137in}}{\pgfqpoint{5.434193in}{1.692864in}}{\pgfqpoint{5.442429in}{1.692864in}}%
\pgfpathclose%
\pgfusepath{stroke,fill}%
\end{pgfscope}%
\begin{pgfscope}%
\pgfpathrectangle{\pgfqpoint{3.874179in}{0.557870in}}{\pgfqpoint{2.484109in}{1.684734in}}%
\pgfusepath{clip}%
\pgfsetbuttcap%
\pgfsetroundjoin%
\definecolor{currentfill}{rgb}{0.298039,0.447059,0.690196}%
\pgfsetfillcolor{currentfill}%
\pgfsetlinewidth{1.003750pt}%
\definecolor{currentstroke}{rgb}{0.298039,0.447059,0.690196}%
\pgfsetstrokecolor{currentstroke}%
\pgfsetdash{}{0pt}%
\pgfpathmoveto{\pgfqpoint{5.592981in}{1.613917in}}%
\pgfpathcurveto{\pgfqpoint{5.601217in}{1.613917in}}{\pgfqpoint{5.609117in}{1.617189in}}{\pgfqpoint{5.614941in}{1.623013in}}%
\pgfpathcurveto{\pgfqpoint{5.620765in}{1.628837in}}{\pgfqpoint{5.624038in}{1.636737in}}{\pgfqpoint{5.624038in}{1.644974in}}%
\pgfpathcurveto{\pgfqpoint{5.624038in}{1.653210in}}{\pgfqpoint{5.620765in}{1.661110in}}{\pgfqpoint{5.614941in}{1.666934in}}%
\pgfpathcurveto{\pgfqpoint{5.609117in}{1.672758in}}{\pgfqpoint{5.601217in}{1.676030in}}{\pgfqpoint{5.592981in}{1.676030in}}%
\pgfpathcurveto{\pgfqpoint{5.584745in}{1.676030in}}{\pgfqpoint{5.576845in}{1.672758in}}{\pgfqpoint{5.571021in}{1.666934in}}%
\pgfpathcurveto{\pgfqpoint{5.565197in}{1.661110in}}{\pgfqpoint{5.561925in}{1.653210in}}{\pgfqpoint{5.561925in}{1.644974in}}%
\pgfpathcurveto{\pgfqpoint{5.561925in}{1.636737in}}{\pgfqpoint{5.565197in}{1.628837in}}{\pgfqpoint{5.571021in}{1.623013in}}%
\pgfpathcurveto{\pgfqpoint{5.576845in}{1.617189in}}{\pgfqpoint{5.584745in}{1.613917in}}{\pgfqpoint{5.592981in}{1.613917in}}%
\pgfpathclose%
\pgfusepath{stroke,fill}%
\end{pgfscope}%
\begin{pgfscope}%
\pgfpathrectangle{\pgfqpoint{3.874179in}{0.557870in}}{\pgfqpoint{2.484109in}{1.684734in}}%
\pgfusepath{clip}%
\pgfsetbuttcap%
\pgfsetroundjoin%
\definecolor{currentfill}{rgb}{0.298039,0.447059,0.690196}%
\pgfsetfillcolor{currentfill}%
\pgfsetlinewidth{1.003750pt}%
\definecolor{currentstroke}{rgb}{0.298039,0.447059,0.690196}%
\pgfsetstrokecolor{currentstroke}%
\pgfsetdash{}{0pt}%
\pgfpathmoveto{\pgfqpoint{5.652289in}{1.570704in}}%
\pgfpathcurveto{\pgfqpoint{5.660526in}{1.570704in}}{\pgfqpoint{5.668426in}{1.573976in}}{\pgfqpoint{5.674250in}{1.579800in}}%
\pgfpathcurveto{\pgfqpoint{5.680074in}{1.585624in}}{\pgfqpoint{5.683346in}{1.593524in}}{\pgfqpoint{5.683346in}{1.601760in}}%
\pgfpathcurveto{\pgfqpoint{5.683346in}{1.609997in}}{\pgfqpoint{5.680074in}{1.617897in}}{\pgfqpoint{5.674250in}{1.623721in}}%
\pgfpathcurveto{\pgfqpoint{5.668426in}{1.629545in}}{\pgfqpoint{5.660526in}{1.632817in}}{\pgfqpoint{5.652289in}{1.632817in}}%
\pgfpathcurveto{\pgfqpoint{5.644053in}{1.632817in}}{\pgfqpoint{5.636153in}{1.629545in}}{\pgfqpoint{5.630329in}{1.623721in}}%
\pgfpathcurveto{\pgfqpoint{5.624505in}{1.617897in}}{\pgfqpoint{5.621233in}{1.609997in}}{\pgfqpoint{5.621233in}{1.601760in}}%
\pgfpathcurveto{\pgfqpoint{5.621233in}{1.593524in}}{\pgfqpoint{5.624505in}{1.585624in}}{\pgfqpoint{5.630329in}{1.579800in}}%
\pgfpathcurveto{\pgfqpoint{5.636153in}{1.573976in}}{\pgfqpoint{5.644053in}{1.570704in}}{\pgfqpoint{5.652289in}{1.570704in}}%
\pgfpathclose%
\pgfusepath{stroke,fill}%
\end{pgfscope}%
\begin{pgfscope}%
\pgfpathrectangle{\pgfqpoint{3.874179in}{0.557870in}}{\pgfqpoint{2.484109in}{1.684734in}}%
\pgfusepath{clip}%
\pgfsetbuttcap%
\pgfsetroundjoin%
\definecolor{currentfill}{rgb}{0.298039,0.447059,0.690196}%
\pgfsetfillcolor{currentfill}%
\pgfsetlinewidth{1.003750pt}%
\definecolor{currentstroke}{rgb}{0.298039,0.447059,0.690196}%
\pgfsetstrokecolor{currentstroke}%
\pgfsetdash{}{0pt}%
\pgfpathmoveto{\pgfqpoint{5.517705in}{1.673849in}}%
\pgfpathcurveto{\pgfqpoint{5.525941in}{1.673849in}}{\pgfqpoint{5.533841in}{1.677121in}}{\pgfqpoint{5.539665in}{1.682945in}}%
\pgfpathcurveto{\pgfqpoint{5.545489in}{1.688769in}}{\pgfqpoint{5.548761in}{1.696669in}}{\pgfqpoint{5.548761in}{1.704906in}}%
\pgfpathcurveto{\pgfqpoint{5.548761in}{1.713142in}}{\pgfqpoint{5.545489in}{1.721042in}}{\pgfqpoint{5.539665in}{1.726866in}}%
\pgfpathcurveto{\pgfqpoint{5.533841in}{1.732690in}}{\pgfqpoint{5.525941in}{1.735962in}}{\pgfqpoint{5.517705in}{1.735962in}}%
\pgfpathcurveto{\pgfqpoint{5.509469in}{1.735962in}}{\pgfqpoint{5.501569in}{1.732690in}}{\pgfqpoint{5.495745in}{1.726866in}}%
\pgfpathcurveto{\pgfqpoint{5.489921in}{1.721042in}}{\pgfqpoint{5.486648in}{1.713142in}}{\pgfqpoint{5.486648in}{1.704906in}}%
\pgfpathcurveto{\pgfqpoint{5.486648in}{1.696669in}}{\pgfqpoint{5.489921in}{1.688769in}}{\pgfqpoint{5.495745in}{1.682945in}}%
\pgfpathcurveto{\pgfqpoint{5.501569in}{1.677121in}}{\pgfqpoint{5.509469in}{1.673849in}}{\pgfqpoint{5.517705in}{1.673849in}}%
\pgfpathclose%
\pgfusepath{stroke,fill}%
\end{pgfscope}%
\begin{pgfscope}%
\pgfpathrectangle{\pgfqpoint{3.874179in}{0.557870in}}{\pgfqpoint{2.484109in}{1.684734in}}%
\pgfusepath{clip}%
\pgfsetbuttcap%
\pgfsetroundjoin%
\definecolor{currentfill}{rgb}{0.298039,0.447059,0.690196}%
\pgfsetfillcolor{currentfill}%
\pgfsetlinewidth{1.003750pt}%
\definecolor{currentstroke}{rgb}{0.298039,0.447059,0.690196}%
\pgfsetstrokecolor{currentstroke}%
\pgfsetdash{}{0pt}%
\pgfpathmoveto{\pgfqpoint{5.467521in}{1.706786in}}%
\pgfpathcurveto{\pgfqpoint{5.475757in}{1.706786in}}{\pgfqpoint{5.483657in}{1.710059in}}{\pgfqpoint{5.489481in}{1.715882in}}%
\pgfpathcurveto{\pgfqpoint{5.495305in}{1.721706in}}{\pgfqpoint{5.498577in}{1.729606in}}{\pgfqpoint{5.498577in}{1.737843in}}%
\pgfpathcurveto{\pgfqpoint{5.498577in}{1.746079in}}{\pgfqpoint{5.495305in}{1.753979in}}{\pgfqpoint{5.489481in}{1.759803in}}%
\pgfpathcurveto{\pgfqpoint{5.483657in}{1.765627in}}{\pgfqpoint{5.475757in}{1.768899in}}{\pgfqpoint{5.467521in}{1.768899in}}%
\pgfpathcurveto{\pgfqpoint{5.459285in}{1.768899in}}{\pgfqpoint{5.451385in}{1.765627in}}{\pgfqpoint{5.445561in}{1.759803in}}%
\pgfpathcurveto{\pgfqpoint{5.439737in}{1.753979in}}{\pgfqpoint{5.436464in}{1.746079in}}{\pgfqpoint{5.436464in}{1.737843in}}%
\pgfpathcurveto{\pgfqpoint{5.436464in}{1.729606in}}{\pgfqpoint{5.439737in}{1.721706in}}{\pgfqpoint{5.445561in}{1.715882in}}%
\pgfpathcurveto{\pgfqpoint{5.451385in}{1.710059in}}{\pgfqpoint{5.459285in}{1.706786in}}{\pgfqpoint{5.467521in}{1.706786in}}%
\pgfpathclose%
\pgfusepath{stroke,fill}%
\end{pgfscope}%
\begin{pgfscope}%
\pgfpathrectangle{\pgfqpoint{3.874179in}{0.557870in}}{\pgfqpoint{2.484109in}{1.684734in}}%
\pgfusepath{clip}%
\pgfsetbuttcap%
\pgfsetroundjoin%
\definecolor{currentfill}{rgb}{0.298039,0.447059,0.690196}%
\pgfsetfillcolor{currentfill}%
\pgfsetlinewidth{1.003750pt}%
\definecolor{currentstroke}{rgb}{0.298039,0.447059,0.690196}%
\pgfsetstrokecolor{currentstroke}%
\pgfsetdash{}{0pt}%
\pgfpathmoveto{\pgfqpoint{5.561046in}{1.651313in}}%
\pgfpathcurveto{\pgfqpoint{5.569282in}{1.651313in}}{\pgfqpoint{5.577182in}{1.654586in}}{\pgfqpoint{5.583006in}{1.660409in}}%
\pgfpathcurveto{\pgfqpoint{5.588830in}{1.666233in}}{\pgfqpoint{5.592102in}{1.674133in}}{\pgfqpoint{5.592102in}{1.682370in}}%
\pgfpathcurveto{\pgfqpoint{5.592102in}{1.690606in}}{\pgfqpoint{5.588830in}{1.698506in}}{\pgfqpoint{5.583006in}{1.704330in}}%
\pgfpathcurveto{\pgfqpoint{5.577182in}{1.710154in}}{\pgfqpoint{5.569282in}{1.713426in}}{\pgfqpoint{5.561046in}{1.713426in}}%
\pgfpathcurveto{\pgfqpoint{5.552809in}{1.713426in}}{\pgfqpoint{5.544909in}{1.710154in}}{\pgfqpoint{5.539085in}{1.704330in}}%
\pgfpathcurveto{\pgfqpoint{5.533262in}{1.698506in}}{\pgfqpoint{5.529989in}{1.690606in}}{\pgfqpoint{5.529989in}{1.682370in}}%
\pgfpathcurveto{\pgfqpoint{5.529989in}{1.674133in}}{\pgfqpoint{5.533262in}{1.666233in}}{\pgfqpoint{5.539085in}{1.660409in}}%
\pgfpathcurveto{\pgfqpoint{5.544909in}{1.654586in}}{\pgfqpoint{5.552809in}{1.651313in}}{\pgfqpoint{5.561046in}{1.651313in}}%
\pgfpathclose%
\pgfusepath{stroke,fill}%
\end{pgfscope}%
\begin{pgfscope}%
\pgfpathrectangle{\pgfqpoint{3.874179in}{0.557870in}}{\pgfqpoint{2.484109in}{1.684734in}}%
\pgfusepath{clip}%
\pgfsetbuttcap%
\pgfsetroundjoin%
\definecolor{currentfill}{rgb}{0.298039,0.447059,0.690196}%
\pgfsetfillcolor{currentfill}%
\pgfsetlinewidth{1.003750pt}%
\definecolor{currentstroke}{rgb}{0.298039,0.447059,0.690196}%
\pgfsetstrokecolor{currentstroke}%
\pgfsetdash{}{0pt}%
\pgfpathmoveto{\pgfqpoint{5.367153in}{1.772660in}}%
\pgfpathcurveto{\pgfqpoint{5.375389in}{1.772660in}}{\pgfqpoint{5.383289in}{1.775933in}}{\pgfqpoint{5.389113in}{1.781757in}}%
\pgfpathcurveto{\pgfqpoint{5.394937in}{1.787581in}}{\pgfqpoint{5.398209in}{1.795481in}}{\pgfqpoint{5.398209in}{1.803717in}}%
\pgfpathcurveto{\pgfqpoint{5.398209in}{1.811953in}}{\pgfqpoint{5.394937in}{1.819853in}}{\pgfqpoint{5.389113in}{1.825677in}}%
\pgfpathcurveto{\pgfqpoint{5.383289in}{1.831501in}}{\pgfqpoint{5.375389in}{1.834773in}}{\pgfqpoint{5.367153in}{1.834773in}}%
\pgfpathcurveto{\pgfqpoint{5.358917in}{1.834773in}}{\pgfqpoint{5.351017in}{1.831501in}}{\pgfqpoint{5.345193in}{1.825677in}}%
\pgfpathcurveto{\pgfqpoint{5.339369in}{1.819853in}}{\pgfqpoint{5.336096in}{1.811953in}}{\pgfqpoint{5.336096in}{1.803717in}}%
\pgfpathcurveto{\pgfqpoint{5.336096in}{1.795481in}}{\pgfqpoint{5.339369in}{1.787581in}}{\pgfqpoint{5.345193in}{1.781757in}}%
\pgfpathcurveto{\pgfqpoint{5.351017in}{1.775933in}}{\pgfqpoint{5.358917in}{1.772660in}}{\pgfqpoint{5.367153in}{1.772660in}}%
\pgfpathclose%
\pgfusepath{stroke,fill}%
\end{pgfscope}%
\begin{pgfscope}%
\pgfpathrectangle{\pgfqpoint{3.874179in}{0.557870in}}{\pgfqpoint{2.484109in}{1.684734in}}%
\pgfusepath{clip}%
\pgfsetbuttcap%
\pgfsetroundjoin%
\definecolor{currentfill}{rgb}{0.298039,0.447059,0.690196}%
\pgfsetfillcolor{currentfill}%
\pgfsetlinewidth{1.003750pt}%
\definecolor{currentstroke}{rgb}{0.298039,0.447059,0.690196}%
\pgfsetstrokecolor{currentstroke}%
\pgfsetdash{}{0pt}%
\pgfpathmoveto{\pgfqpoint{5.517705in}{1.690318in}}%
\pgfpathcurveto{\pgfqpoint{5.525941in}{1.690318in}}{\pgfqpoint{5.533841in}{1.693590in}}{\pgfqpoint{5.539665in}{1.699414in}}%
\pgfpathcurveto{\pgfqpoint{5.545489in}{1.705238in}}{\pgfqpoint{5.548761in}{1.713138in}}{\pgfqpoint{5.548761in}{1.721374in}}%
\pgfpathcurveto{\pgfqpoint{5.548761in}{1.729610in}}{\pgfqpoint{5.545489in}{1.737511in}}{\pgfqpoint{5.539665in}{1.743334in}}%
\pgfpathcurveto{\pgfqpoint{5.533841in}{1.749158in}}{\pgfqpoint{5.525941in}{1.752431in}}{\pgfqpoint{5.517705in}{1.752431in}}%
\pgfpathcurveto{\pgfqpoint{5.509469in}{1.752431in}}{\pgfqpoint{5.501569in}{1.749158in}}{\pgfqpoint{5.495745in}{1.743334in}}%
\pgfpathcurveto{\pgfqpoint{5.489921in}{1.737511in}}{\pgfqpoint{5.486648in}{1.729610in}}{\pgfqpoint{5.486648in}{1.721374in}}%
\pgfpathcurveto{\pgfqpoint{5.486648in}{1.713138in}}{\pgfqpoint{5.489921in}{1.705238in}}{\pgfqpoint{5.495745in}{1.699414in}}%
\pgfpathcurveto{\pgfqpoint{5.501569in}{1.693590in}}{\pgfqpoint{5.509469in}{1.690318in}}{\pgfqpoint{5.517705in}{1.690318in}}%
\pgfpathclose%
\pgfusepath{stroke,fill}%
\end{pgfscope}%
\begin{pgfscope}%
\pgfpathrectangle{\pgfqpoint{3.874179in}{0.557870in}}{\pgfqpoint{2.484109in}{1.684734in}}%
\pgfusepath{clip}%
\pgfsetbuttcap%
\pgfsetroundjoin%
\definecolor{currentfill}{rgb}{0.298039,0.447059,0.690196}%
\pgfsetfillcolor{currentfill}%
\pgfsetlinewidth{1.003750pt}%
\definecolor{currentstroke}{rgb}{0.298039,0.447059,0.690196}%
\pgfsetstrokecolor{currentstroke}%
\pgfsetdash{}{0pt}%
\pgfpathmoveto{\pgfqpoint{5.392245in}{1.771812in}}%
\pgfpathcurveto{\pgfqpoint{5.400481in}{1.771812in}}{\pgfqpoint{5.408381in}{1.775084in}}{\pgfqpoint{5.414205in}{1.780908in}}%
\pgfpathcurveto{\pgfqpoint{5.420029in}{1.786732in}}{\pgfqpoint{5.423301in}{1.794632in}}{\pgfqpoint{5.423301in}{1.802868in}}%
\pgfpathcurveto{\pgfqpoint{5.423301in}{1.811104in}}{\pgfqpoint{5.420029in}{1.819004in}}{\pgfqpoint{5.414205in}{1.824828in}}%
\pgfpathcurveto{\pgfqpoint{5.408381in}{1.830652in}}{\pgfqpoint{5.400481in}{1.833925in}}{\pgfqpoint{5.392245in}{1.833925in}}%
\pgfpathcurveto{\pgfqpoint{5.384009in}{1.833925in}}{\pgfqpoint{5.376109in}{1.830652in}}{\pgfqpoint{5.370285in}{1.824828in}}%
\pgfpathcurveto{\pgfqpoint{5.364461in}{1.819004in}}{\pgfqpoint{5.361188in}{1.811104in}}{\pgfqpoint{5.361188in}{1.802868in}}%
\pgfpathcurveto{\pgfqpoint{5.361188in}{1.794632in}}{\pgfqpoint{5.364461in}{1.786732in}}{\pgfqpoint{5.370285in}{1.780908in}}%
\pgfpathcurveto{\pgfqpoint{5.376109in}{1.775084in}}{\pgfqpoint{5.384009in}{1.771812in}}{\pgfqpoint{5.392245in}{1.771812in}}%
\pgfpathclose%
\pgfusepath{stroke,fill}%
\end{pgfscope}%
\begin{pgfscope}%
\pgfpathrectangle{\pgfqpoint{3.874179in}{0.557870in}}{\pgfqpoint{2.484109in}{1.684734in}}%
\pgfusepath{clip}%
\pgfsetbuttcap%
\pgfsetroundjoin%
\definecolor{currentfill}{rgb}{0.298039,0.447059,0.690196}%
\pgfsetfillcolor{currentfill}%
\pgfsetlinewidth{1.003750pt}%
\definecolor{currentstroke}{rgb}{0.298039,0.447059,0.690196}%
\pgfsetstrokecolor{currentstroke}%
\pgfsetdash{}{0pt}%
\pgfpathmoveto{\pgfqpoint{5.492613in}{1.724443in}}%
\pgfpathcurveto{\pgfqpoint{5.500849in}{1.724443in}}{\pgfqpoint{5.508749in}{1.727716in}}{\pgfqpoint{5.514573in}{1.733539in}}%
\pgfpathcurveto{\pgfqpoint{5.520397in}{1.739363in}}{\pgfqpoint{5.523669in}{1.747263in}}{\pgfqpoint{5.523669in}{1.755500in}}%
\pgfpathcurveto{\pgfqpoint{5.523669in}{1.763736in}}{\pgfqpoint{5.520397in}{1.771636in}}{\pgfqpoint{5.514573in}{1.777460in}}%
\pgfpathcurveto{\pgfqpoint{5.508749in}{1.783284in}}{\pgfqpoint{5.500849in}{1.786556in}}{\pgfqpoint{5.492613in}{1.786556in}}%
\pgfpathcurveto{\pgfqpoint{5.484377in}{1.786556in}}{\pgfqpoint{5.476477in}{1.783284in}}{\pgfqpoint{5.470653in}{1.777460in}}%
\pgfpathcurveto{\pgfqpoint{5.464829in}{1.771636in}}{\pgfqpoint{5.461556in}{1.763736in}}{\pgfqpoint{5.461556in}{1.755500in}}%
\pgfpathcurveto{\pgfqpoint{5.461556in}{1.747263in}}{\pgfqpoint{5.464829in}{1.739363in}}{\pgfqpoint{5.470653in}{1.733539in}}%
\pgfpathcurveto{\pgfqpoint{5.476477in}{1.727716in}}{\pgfqpoint{5.484377in}{1.724443in}}{\pgfqpoint{5.492613in}{1.724443in}}%
\pgfpathclose%
\pgfusepath{stroke,fill}%
\end{pgfscope}%
\begin{pgfscope}%
\pgfpathrectangle{\pgfqpoint{3.874179in}{0.557870in}}{\pgfqpoint{2.484109in}{1.684734in}}%
\pgfusepath{clip}%
\pgfsetbuttcap%
\pgfsetroundjoin%
\definecolor{currentfill}{rgb}{0.298039,0.447059,0.690196}%
\pgfsetfillcolor{currentfill}%
\pgfsetlinewidth{1.003750pt}%
\definecolor{currentstroke}{rgb}{0.298039,0.447059,0.690196}%
\pgfsetstrokecolor{currentstroke}%
\pgfsetdash{}{0pt}%
\pgfpathmoveto{\pgfqpoint{5.567889in}{1.677075in}}%
\pgfpathcurveto{\pgfqpoint{5.576125in}{1.677075in}}{\pgfqpoint{5.584025in}{1.680347in}}{\pgfqpoint{5.589849in}{1.686171in}}%
\pgfpathcurveto{\pgfqpoint{5.595673in}{1.691995in}}{\pgfqpoint{5.598946in}{1.699895in}}{\pgfqpoint{5.598946in}{1.708131in}}%
\pgfpathcurveto{\pgfqpoint{5.598946in}{1.716368in}}{\pgfqpoint{5.595673in}{1.724268in}}{\pgfqpoint{5.589849in}{1.730092in}}%
\pgfpathcurveto{\pgfqpoint{5.584025in}{1.735916in}}{\pgfqpoint{5.576125in}{1.739188in}}{\pgfqpoint{5.567889in}{1.739188in}}%
\pgfpathcurveto{\pgfqpoint{5.559653in}{1.739188in}}{\pgfqpoint{5.551753in}{1.735916in}}{\pgfqpoint{5.545929in}{1.730092in}}%
\pgfpathcurveto{\pgfqpoint{5.540105in}{1.724268in}}{\pgfqpoint{5.536833in}{1.716368in}}{\pgfqpoint{5.536833in}{1.708131in}}%
\pgfpathcurveto{\pgfqpoint{5.536833in}{1.699895in}}{\pgfqpoint{5.540105in}{1.691995in}}{\pgfqpoint{5.545929in}{1.686171in}}%
\pgfpathcurveto{\pgfqpoint{5.551753in}{1.680347in}}{\pgfqpoint{5.559653in}{1.677075in}}{\pgfqpoint{5.567889in}{1.677075in}}%
\pgfpathclose%
\pgfusepath{stroke,fill}%
\end{pgfscope}%
\begin{pgfscope}%
\pgfpathrectangle{\pgfqpoint{3.874179in}{0.557870in}}{\pgfqpoint{2.484109in}{1.684734in}}%
\pgfusepath{clip}%
\pgfsetbuttcap%
\pgfsetroundjoin%
\definecolor{currentfill}{rgb}{0.298039,0.447059,0.690196}%
\pgfsetfillcolor{currentfill}%
\pgfsetlinewidth{1.003750pt}%
\definecolor{currentstroke}{rgb}{0.298039,0.447059,0.690196}%
\pgfsetstrokecolor{currentstroke}%
\pgfsetdash{}{0pt}%
\pgfpathmoveto{\pgfqpoint{5.467521in}{1.739723in}}%
\pgfpathcurveto{\pgfqpoint{5.475757in}{1.739723in}}{\pgfqpoint{5.483657in}{1.742996in}}{\pgfqpoint{5.489481in}{1.748820in}}%
\pgfpathcurveto{\pgfqpoint{5.495305in}{1.754644in}}{\pgfqpoint{5.498577in}{1.762544in}}{\pgfqpoint{5.498577in}{1.770780in}}%
\pgfpathcurveto{\pgfqpoint{5.498577in}{1.779016in}}{\pgfqpoint{5.495305in}{1.786916in}}{\pgfqpoint{5.489481in}{1.792740in}}%
\pgfpathcurveto{\pgfqpoint{5.483657in}{1.798564in}}{\pgfqpoint{5.475757in}{1.801836in}}{\pgfqpoint{5.467521in}{1.801836in}}%
\pgfpathcurveto{\pgfqpoint{5.459285in}{1.801836in}}{\pgfqpoint{5.451385in}{1.798564in}}{\pgfqpoint{5.445561in}{1.792740in}}%
\pgfpathcurveto{\pgfqpoint{5.439737in}{1.786916in}}{\pgfqpoint{5.436464in}{1.779016in}}{\pgfqpoint{5.436464in}{1.770780in}}%
\pgfpathcurveto{\pgfqpoint{5.436464in}{1.762544in}}{\pgfqpoint{5.439737in}{1.754644in}}{\pgfqpoint{5.445561in}{1.748820in}}%
\pgfpathcurveto{\pgfqpoint{5.451385in}{1.742996in}}{\pgfqpoint{5.459285in}{1.739723in}}{\pgfqpoint{5.467521in}{1.739723in}}%
\pgfpathclose%
\pgfusepath{stroke,fill}%
\end{pgfscope}%
\begin{pgfscope}%
\pgfpathrectangle{\pgfqpoint{3.874179in}{0.557870in}}{\pgfqpoint{2.484109in}{1.684734in}}%
\pgfusepath{clip}%
\pgfsetbuttcap%
\pgfsetroundjoin%
\definecolor{currentfill}{rgb}{0.298039,0.447059,0.690196}%
\pgfsetfillcolor{currentfill}%
\pgfsetlinewidth{1.003750pt}%
\definecolor{currentstroke}{rgb}{0.298039,0.447059,0.690196}%
\pgfsetstrokecolor{currentstroke}%
\pgfsetdash{}{0pt}%
\pgfpathmoveto{\pgfqpoint{5.417337in}{1.771812in}}%
\pgfpathcurveto{\pgfqpoint{5.425573in}{1.771812in}}{\pgfqpoint{5.433473in}{1.775084in}}{\pgfqpoint{5.439297in}{1.780908in}}%
\pgfpathcurveto{\pgfqpoint{5.445121in}{1.786732in}}{\pgfqpoint{5.448393in}{1.794632in}}{\pgfqpoint{5.448393in}{1.802868in}}%
\pgfpathcurveto{\pgfqpoint{5.448393in}{1.811104in}}{\pgfqpoint{5.445121in}{1.819004in}}{\pgfqpoint{5.439297in}{1.824828in}}%
\pgfpathcurveto{\pgfqpoint{5.433473in}{1.830652in}}{\pgfqpoint{5.425573in}{1.833925in}}{\pgfqpoint{5.417337in}{1.833925in}}%
\pgfpathcurveto{\pgfqpoint{5.409101in}{1.833925in}}{\pgfqpoint{5.401201in}{1.830652in}}{\pgfqpoint{5.395377in}{1.824828in}}%
\pgfpathcurveto{\pgfqpoint{5.389553in}{1.819004in}}{\pgfqpoint{5.386280in}{1.811104in}}{\pgfqpoint{5.386280in}{1.802868in}}%
\pgfpathcurveto{\pgfqpoint{5.386280in}{1.794632in}}{\pgfqpoint{5.389553in}{1.786732in}}{\pgfqpoint{5.395377in}{1.780908in}}%
\pgfpathcurveto{\pgfqpoint{5.401201in}{1.775084in}}{\pgfqpoint{5.409101in}{1.771812in}}{\pgfqpoint{5.417337in}{1.771812in}}%
\pgfpathclose%
\pgfusepath{stroke,fill}%
\end{pgfscope}%
\begin{pgfscope}%
\pgfpathrectangle{\pgfqpoint{3.874179in}{0.557870in}}{\pgfqpoint{2.484109in}{1.684734in}}%
\pgfusepath{clip}%
\pgfsetbuttcap%
\pgfsetroundjoin%
\definecolor{currentfill}{rgb}{0.298039,0.447059,0.690196}%
\pgfsetfillcolor{currentfill}%
\pgfsetlinewidth{1.003750pt}%
\definecolor{currentstroke}{rgb}{0.298039,0.447059,0.690196}%
\pgfsetstrokecolor{currentstroke}%
\pgfsetdash{}{0pt}%
\pgfpathmoveto{\pgfqpoint{5.542797in}{1.706786in}}%
\pgfpathcurveto{\pgfqpoint{5.551033in}{1.706786in}}{\pgfqpoint{5.558933in}{1.710059in}}{\pgfqpoint{5.564757in}{1.715882in}}%
\pgfpathcurveto{\pgfqpoint{5.570581in}{1.721706in}}{\pgfqpoint{5.573853in}{1.729606in}}{\pgfqpoint{5.573853in}{1.737843in}}%
\pgfpathcurveto{\pgfqpoint{5.573853in}{1.746079in}}{\pgfqpoint{5.570581in}{1.753979in}}{\pgfqpoint{5.564757in}{1.759803in}}%
\pgfpathcurveto{\pgfqpoint{5.558933in}{1.765627in}}{\pgfqpoint{5.551033in}{1.768899in}}{\pgfqpoint{5.542797in}{1.768899in}}%
\pgfpathcurveto{\pgfqpoint{5.534561in}{1.768899in}}{\pgfqpoint{5.526661in}{1.765627in}}{\pgfqpoint{5.520837in}{1.759803in}}%
\pgfpathcurveto{\pgfqpoint{5.515013in}{1.753979in}}{\pgfqpoint{5.511740in}{1.746079in}}{\pgfqpoint{5.511740in}{1.737843in}}%
\pgfpathcurveto{\pgfqpoint{5.511740in}{1.729606in}}{\pgfqpoint{5.515013in}{1.721706in}}{\pgfqpoint{5.520837in}{1.715882in}}%
\pgfpathcurveto{\pgfqpoint{5.526661in}{1.710059in}}{\pgfqpoint{5.534561in}{1.706786in}}{\pgfqpoint{5.542797in}{1.706786in}}%
\pgfpathclose%
\pgfusepath{stroke,fill}%
\end{pgfscope}%
\begin{pgfscope}%
\pgfpathrectangle{\pgfqpoint{3.874179in}{0.557870in}}{\pgfqpoint{2.484109in}{1.684734in}}%
\pgfusepath{clip}%
\pgfsetbuttcap%
\pgfsetroundjoin%
\definecolor{currentfill}{rgb}{0.298039,0.447059,0.690196}%
\pgfsetfillcolor{currentfill}%
\pgfsetlinewidth{1.003750pt}%
\definecolor{currentstroke}{rgb}{0.298039,0.447059,0.690196}%
\pgfsetstrokecolor{currentstroke}%
\pgfsetdash{}{0pt}%
\pgfpathmoveto{\pgfqpoint{5.538235in}{1.715801in}}%
\pgfpathcurveto{\pgfqpoint{5.546471in}{1.715801in}}{\pgfqpoint{5.554371in}{1.719073in}}{\pgfqpoint{5.560195in}{1.724897in}}%
\pgfpathcurveto{\pgfqpoint{5.566019in}{1.730721in}}{\pgfqpoint{5.569291in}{1.738621in}}{\pgfqpoint{5.569291in}{1.746857in}}%
\pgfpathcurveto{\pgfqpoint{5.569291in}{1.755093in}}{\pgfqpoint{5.566019in}{1.762993in}}{\pgfqpoint{5.560195in}{1.768817in}}%
\pgfpathcurveto{\pgfqpoint{5.554371in}{1.774641in}}{\pgfqpoint{5.546471in}{1.777914in}}{\pgfqpoint{5.538235in}{1.777914in}}%
\pgfpathcurveto{\pgfqpoint{5.529999in}{1.777914in}}{\pgfqpoint{5.522098in}{1.774641in}}{\pgfqpoint{5.516275in}{1.768817in}}%
\pgfpathcurveto{\pgfqpoint{5.510451in}{1.762993in}}{\pgfqpoint{5.507178in}{1.755093in}}{\pgfqpoint{5.507178in}{1.746857in}}%
\pgfpathcurveto{\pgfqpoint{5.507178in}{1.738621in}}{\pgfqpoint{5.510451in}{1.730721in}}{\pgfqpoint{5.516275in}{1.724897in}}%
\pgfpathcurveto{\pgfqpoint{5.522098in}{1.719073in}}{\pgfqpoint{5.529999in}{1.715801in}}{\pgfqpoint{5.538235in}{1.715801in}}%
\pgfpathclose%
\pgfusepath{stroke,fill}%
\end{pgfscope}%
\begin{pgfscope}%
\pgfpathrectangle{\pgfqpoint{3.874179in}{0.557870in}}{\pgfqpoint{2.484109in}{1.684734in}}%
\pgfusepath{clip}%
\pgfsetbuttcap%
\pgfsetroundjoin%
\definecolor{currentfill}{rgb}{0.298039,0.447059,0.690196}%
\pgfsetfillcolor{currentfill}%
\pgfsetlinewidth{1.003750pt}%
\definecolor{currentstroke}{rgb}{0.298039,0.447059,0.690196}%
\pgfsetstrokecolor{currentstroke}%
\pgfsetdash{}{0pt}%
\pgfpathmoveto{\pgfqpoint{5.515424in}{1.731922in}}%
\pgfpathcurveto{\pgfqpoint{5.523660in}{1.731922in}}{\pgfqpoint{5.531560in}{1.735195in}}{\pgfqpoint{5.537384in}{1.741019in}}%
\pgfpathcurveto{\pgfqpoint{5.543208in}{1.746843in}}{\pgfqpoint{5.546480in}{1.754743in}}{\pgfqpoint{5.546480in}{1.762979in}}%
\pgfpathcurveto{\pgfqpoint{5.546480in}{1.771215in}}{\pgfqpoint{5.543208in}{1.779115in}}{\pgfqpoint{5.537384in}{1.784939in}}%
\pgfpathcurveto{\pgfqpoint{5.531560in}{1.790763in}}{\pgfqpoint{5.523660in}{1.794035in}}{\pgfqpoint{5.515424in}{1.794035in}}%
\pgfpathcurveto{\pgfqpoint{5.507188in}{1.794035in}}{\pgfqpoint{5.499288in}{1.790763in}}{\pgfqpoint{5.493464in}{1.784939in}}%
\pgfpathcurveto{\pgfqpoint{5.487640in}{1.779115in}}{\pgfqpoint{5.484367in}{1.771215in}}{\pgfqpoint{5.484367in}{1.762979in}}%
\pgfpathcurveto{\pgfqpoint{5.484367in}{1.754743in}}{\pgfqpoint{5.487640in}{1.746843in}}{\pgfqpoint{5.493464in}{1.741019in}}%
\pgfpathcurveto{\pgfqpoint{5.499288in}{1.735195in}}{\pgfqpoint{5.507188in}{1.731922in}}{\pgfqpoint{5.515424in}{1.731922in}}%
\pgfpathclose%
\pgfusepath{stroke,fill}%
\end{pgfscope}%
\begin{pgfscope}%
\pgfpathrectangle{\pgfqpoint{3.874179in}{0.557870in}}{\pgfqpoint{2.484109in}{1.684734in}}%
\pgfusepath{clip}%
\pgfsetbuttcap%
\pgfsetroundjoin%
\definecolor{currentfill}{rgb}{0.298039,0.447059,0.690196}%
\pgfsetfillcolor{currentfill}%
\pgfsetlinewidth{1.003750pt}%
\definecolor{currentstroke}{rgb}{0.298039,0.447059,0.690196}%
\pgfsetstrokecolor{currentstroke}%
\pgfsetdash{}{0pt}%
\pgfpathmoveto{\pgfqpoint{5.417337in}{1.787601in}}%
\pgfpathcurveto{\pgfqpoint{5.425573in}{1.787601in}}{\pgfqpoint{5.433473in}{1.790873in}}{\pgfqpoint{5.439297in}{1.796697in}}%
\pgfpathcurveto{\pgfqpoint{5.445121in}{1.802521in}}{\pgfqpoint{5.448393in}{1.810421in}}{\pgfqpoint{5.448393in}{1.818658in}}%
\pgfpathcurveto{\pgfqpoint{5.448393in}{1.826894in}}{\pgfqpoint{5.445121in}{1.834794in}}{\pgfqpoint{5.439297in}{1.840618in}}%
\pgfpathcurveto{\pgfqpoint{5.433473in}{1.846442in}}{\pgfqpoint{5.425573in}{1.849714in}}{\pgfqpoint{5.417337in}{1.849714in}}%
\pgfpathcurveto{\pgfqpoint{5.409101in}{1.849714in}}{\pgfqpoint{5.401201in}{1.846442in}}{\pgfqpoint{5.395377in}{1.840618in}}%
\pgfpathcurveto{\pgfqpoint{5.389553in}{1.834794in}}{\pgfqpoint{5.386280in}{1.826894in}}{\pgfqpoint{5.386280in}{1.818658in}}%
\pgfpathcurveto{\pgfqpoint{5.386280in}{1.810421in}}{\pgfqpoint{5.389553in}{1.802521in}}{\pgfqpoint{5.395377in}{1.796697in}}%
\pgfpathcurveto{\pgfqpoint{5.401201in}{1.790873in}}{\pgfqpoint{5.409101in}{1.787601in}}{\pgfqpoint{5.417337in}{1.787601in}}%
\pgfpathclose%
\pgfusepath{stroke,fill}%
\end{pgfscope}%
\begin{pgfscope}%
\pgfpathrectangle{\pgfqpoint{3.874179in}{0.557870in}}{\pgfqpoint{2.484109in}{1.684734in}}%
\pgfusepath{clip}%
\pgfsetbuttcap%
\pgfsetroundjoin%
\definecolor{currentfill}{rgb}{0.298039,0.447059,0.690196}%
\pgfsetfillcolor{currentfill}%
\pgfsetlinewidth{1.003750pt}%
\definecolor{currentstroke}{rgb}{0.298039,0.447059,0.690196}%
\pgfsetstrokecolor{currentstroke}%
\pgfsetdash{}{0pt}%
\pgfpathmoveto{\pgfqpoint{5.606668in}{1.683557in}}%
\pgfpathcurveto{\pgfqpoint{5.614904in}{1.683557in}}{\pgfqpoint{5.622804in}{1.686829in}}{\pgfqpoint{5.628628in}{1.692653in}}%
\pgfpathcurveto{\pgfqpoint{5.634452in}{1.698477in}}{\pgfqpoint{5.637724in}{1.706377in}}{\pgfqpoint{5.637724in}{1.714613in}}%
\pgfpathcurveto{\pgfqpoint{5.637724in}{1.722850in}}{\pgfqpoint{5.634452in}{1.730750in}}{\pgfqpoint{5.628628in}{1.736574in}}%
\pgfpathcurveto{\pgfqpoint{5.622804in}{1.742398in}}{\pgfqpoint{5.614904in}{1.745670in}}{\pgfqpoint{5.606668in}{1.745670in}}%
\pgfpathcurveto{\pgfqpoint{5.598431in}{1.745670in}}{\pgfqpoint{5.590531in}{1.742398in}}{\pgfqpoint{5.584707in}{1.736574in}}%
\pgfpathcurveto{\pgfqpoint{5.578883in}{1.730750in}}{\pgfqpoint{5.575611in}{1.722850in}}{\pgfqpoint{5.575611in}{1.714613in}}%
\pgfpathcurveto{\pgfqpoint{5.575611in}{1.706377in}}{\pgfqpoint{5.578883in}{1.698477in}}{\pgfqpoint{5.584707in}{1.692653in}}%
\pgfpathcurveto{\pgfqpoint{5.590531in}{1.686829in}}{\pgfqpoint{5.598431in}{1.683557in}}{\pgfqpoint{5.606668in}{1.683557in}}%
\pgfpathclose%
\pgfusepath{stroke,fill}%
\end{pgfscope}%
\begin{pgfscope}%
\pgfpathrectangle{\pgfqpoint{3.874179in}{0.557870in}}{\pgfqpoint{2.484109in}{1.684734in}}%
\pgfusepath{clip}%
\pgfsetbuttcap%
\pgfsetroundjoin%
\definecolor{currentfill}{rgb}{0.298039,0.447059,0.690196}%
\pgfsetfillcolor{currentfill}%
\pgfsetlinewidth{1.003750pt}%
\definecolor{currentstroke}{rgb}{0.298039,0.447059,0.690196}%
\pgfsetstrokecolor{currentstroke}%
\pgfsetdash{}{0pt}%
\pgfpathmoveto{\pgfqpoint{5.492613in}{1.756022in}}%
\pgfpathcurveto{\pgfqpoint{5.500849in}{1.756022in}}{\pgfqpoint{5.508749in}{1.759294in}}{\pgfqpoint{5.514573in}{1.765118in}}%
\pgfpathcurveto{\pgfqpoint{5.520397in}{1.770942in}}{\pgfqpoint{5.523669in}{1.778842in}}{\pgfqpoint{5.523669in}{1.787079in}}%
\pgfpathcurveto{\pgfqpoint{5.523669in}{1.795315in}}{\pgfqpoint{5.520397in}{1.803215in}}{\pgfqpoint{5.514573in}{1.809039in}}%
\pgfpathcurveto{\pgfqpoint{5.508749in}{1.814863in}}{\pgfqpoint{5.500849in}{1.818135in}}{\pgfqpoint{5.492613in}{1.818135in}}%
\pgfpathcurveto{\pgfqpoint{5.484377in}{1.818135in}}{\pgfqpoint{5.476477in}{1.814863in}}{\pgfqpoint{5.470653in}{1.809039in}}%
\pgfpathcurveto{\pgfqpoint{5.464829in}{1.803215in}}{\pgfqpoint{5.461556in}{1.795315in}}{\pgfqpoint{5.461556in}{1.787079in}}%
\pgfpathcurveto{\pgfqpoint{5.461556in}{1.778842in}}{\pgfqpoint{5.464829in}{1.770942in}}{\pgfqpoint{5.470653in}{1.765118in}}%
\pgfpathcurveto{\pgfqpoint{5.476477in}{1.759294in}}{\pgfqpoint{5.484377in}{1.756022in}}{\pgfqpoint{5.492613in}{1.756022in}}%
\pgfpathclose%
\pgfusepath{stroke,fill}%
\end{pgfscope}%
\begin{pgfscope}%
\pgfpathrectangle{\pgfqpoint{3.874179in}{0.557870in}}{\pgfqpoint{2.484109in}{1.684734in}}%
\pgfusepath{clip}%
\pgfsetbuttcap%
\pgfsetroundjoin%
\definecolor{currentfill}{rgb}{0.298039,0.447059,0.690196}%
\pgfsetfillcolor{currentfill}%
\pgfsetlinewidth{1.003750pt}%
\definecolor{currentstroke}{rgb}{0.298039,0.447059,0.690196}%
\pgfsetstrokecolor{currentstroke}%
\pgfsetdash{}{0pt}%
\pgfpathmoveto{\pgfqpoint{5.538235in}{1.731922in}}%
\pgfpathcurveto{\pgfqpoint{5.546471in}{1.731922in}}{\pgfqpoint{5.554371in}{1.735195in}}{\pgfqpoint{5.560195in}{1.741019in}}%
\pgfpathcurveto{\pgfqpoint{5.566019in}{1.746843in}}{\pgfqpoint{5.569291in}{1.754743in}}{\pgfqpoint{5.569291in}{1.762979in}}%
\pgfpathcurveto{\pgfqpoint{5.569291in}{1.771215in}}{\pgfqpoint{5.566019in}{1.779115in}}{\pgfqpoint{5.560195in}{1.784939in}}%
\pgfpathcurveto{\pgfqpoint{5.554371in}{1.790763in}}{\pgfqpoint{5.546471in}{1.794035in}}{\pgfqpoint{5.538235in}{1.794035in}}%
\pgfpathcurveto{\pgfqpoint{5.529999in}{1.794035in}}{\pgfqpoint{5.522098in}{1.790763in}}{\pgfqpoint{5.516275in}{1.784939in}}%
\pgfpathcurveto{\pgfqpoint{5.510451in}{1.779115in}}{\pgfqpoint{5.507178in}{1.771215in}}{\pgfqpoint{5.507178in}{1.762979in}}%
\pgfpathcurveto{\pgfqpoint{5.507178in}{1.754743in}}{\pgfqpoint{5.510451in}{1.746843in}}{\pgfqpoint{5.516275in}{1.741019in}}%
\pgfpathcurveto{\pgfqpoint{5.522098in}{1.735195in}}{\pgfqpoint{5.529999in}{1.731922in}}{\pgfqpoint{5.538235in}{1.731922in}}%
\pgfpathclose%
\pgfusepath{stroke,fill}%
\end{pgfscope}%
\begin{pgfscope}%
\pgfpathrectangle{\pgfqpoint{3.874179in}{0.557870in}}{\pgfqpoint{2.484109in}{1.684734in}}%
\pgfusepath{clip}%
\pgfsetbuttcap%
\pgfsetroundjoin%
\definecolor{currentfill}{rgb}{0.298039,0.447059,0.690196}%
\pgfsetfillcolor{currentfill}%
\pgfsetlinewidth{1.003750pt}%
\definecolor{currentstroke}{rgb}{0.298039,0.447059,0.690196}%
\pgfsetstrokecolor{currentstroke}%
\pgfsetdash{}{0pt}%
\pgfpathmoveto{\pgfqpoint{5.517705in}{1.756192in}}%
\pgfpathcurveto{\pgfqpoint{5.525941in}{1.756192in}}{\pgfqpoint{5.533841in}{1.759464in}}{\pgfqpoint{5.539665in}{1.765288in}}%
\pgfpathcurveto{\pgfqpoint{5.545489in}{1.771112in}}{\pgfqpoint{5.548761in}{1.779012in}}{\pgfqpoint{5.548761in}{1.787248in}}%
\pgfpathcurveto{\pgfqpoint{5.548761in}{1.795485in}}{\pgfqpoint{5.545489in}{1.803385in}}{\pgfqpoint{5.539665in}{1.809209in}}%
\pgfpathcurveto{\pgfqpoint{5.533841in}{1.815033in}}{\pgfqpoint{5.525941in}{1.818305in}}{\pgfqpoint{5.517705in}{1.818305in}}%
\pgfpathcurveto{\pgfqpoint{5.509469in}{1.818305in}}{\pgfqpoint{5.501569in}{1.815033in}}{\pgfqpoint{5.495745in}{1.809209in}}%
\pgfpathcurveto{\pgfqpoint{5.489921in}{1.803385in}}{\pgfqpoint{5.486648in}{1.795485in}}{\pgfqpoint{5.486648in}{1.787248in}}%
\pgfpathcurveto{\pgfqpoint{5.486648in}{1.779012in}}{\pgfqpoint{5.489921in}{1.771112in}}{\pgfqpoint{5.495745in}{1.765288in}}%
\pgfpathcurveto{\pgfqpoint{5.501569in}{1.759464in}}{\pgfqpoint{5.509469in}{1.756192in}}{\pgfqpoint{5.517705in}{1.756192in}}%
\pgfpathclose%
\pgfusepath{stroke,fill}%
\end{pgfscope}%
\begin{pgfscope}%
\pgfpathrectangle{\pgfqpoint{3.874179in}{0.557870in}}{\pgfqpoint{2.484109in}{1.684734in}}%
\pgfusepath{clip}%
\pgfsetbuttcap%
\pgfsetroundjoin%
\definecolor{currentfill}{rgb}{0.298039,0.447059,0.690196}%
\pgfsetfillcolor{currentfill}%
\pgfsetlinewidth{1.003750pt}%
\definecolor{currentstroke}{rgb}{0.298039,0.447059,0.690196}%
\pgfsetstrokecolor{currentstroke}%
\pgfsetdash{}{0pt}%
\pgfpathmoveto{\pgfqpoint{5.492613in}{1.771812in}}%
\pgfpathcurveto{\pgfqpoint{5.500849in}{1.771812in}}{\pgfqpoint{5.508749in}{1.775084in}}{\pgfqpoint{5.514573in}{1.780908in}}%
\pgfpathcurveto{\pgfqpoint{5.520397in}{1.786732in}}{\pgfqpoint{5.523669in}{1.794632in}}{\pgfqpoint{5.523669in}{1.802868in}}%
\pgfpathcurveto{\pgfqpoint{5.523669in}{1.811104in}}{\pgfqpoint{5.520397in}{1.819004in}}{\pgfqpoint{5.514573in}{1.824828in}}%
\pgfpathcurveto{\pgfqpoint{5.508749in}{1.830652in}}{\pgfqpoint{5.500849in}{1.833925in}}{\pgfqpoint{5.492613in}{1.833925in}}%
\pgfpathcurveto{\pgfqpoint{5.484377in}{1.833925in}}{\pgfqpoint{5.476477in}{1.830652in}}{\pgfqpoint{5.470653in}{1.824828in}}%
\pgfpathcurveto{\pgfqpoint{5.464829in}{1.819004in}}{\pgfqpoint{5.461556in}{1.811104in}}{\pgfqpoint{5.461556in}{1.802868in}}%
\pgfpathcurveto{\pgfqpoint{5.461556in}{1.794632in}}{\pgfqpoint{5.464829in}{1.786732in}}{\pgfqpoint{5.470653in}{1.780908in}}%
\pgfpathcurveto{\pgfqpoint{5.476477in}{1.775084in}}{\pgfqpoint{5.484377in}{1.771812in}}{\pgfqpoint{5.492613in}{1.771812in}}%
\pgfpathclose%
\pgfusepath{stroke,fill}%
\end{pgfscope}%
\begin{pgfscope}%
\pgfpathrectangle{\pgfqpoint{3.874179in}{0.557870in}}{\pgfqpoint{2.484109in}{1.684734in}}%
\pgfusepath{clip}%
\pgfsetbuttcap%
\pgfsetroundjoin%
\definecolor{currentfill}{rgb}{0.298039,0.447059,0.690196}%
\pgfsetfillcolor{currentfill}%
\pgfsetlinewidth{1.003750pt}%
\definecolor{currentstroke}{rgb}{0.298039,0.447059,0.690196}%
\pgfsetstrokecolor{currentstroke}%
\pgfsetdash{}{0pt}%
\pgfpathmoveto{\pgfqpoint{5.561046in}{1.731922in}}%
\pgfpathcurveto{\pgfqpoint{5.569282in}{1.731922in}}{\pgfqpoint{5.577182in}{1.735195in}}{\pgfqpoint{5.583006in}{1.741019in}}%
\pgfpathcurveto{\pgfqpoint{5.588830in}{1.746843in}}{\pgfqpoint{5.592102in}{1.754743in}}{\pgfqpoint{5.592102in}{1.762979in}}%
\pgfpathcurveto{\pgfqpoint{5.592102in}{1.771215in}}{\pgfqpoint{5.588830in}{1.779115in}}{\pgfqpoint{5.583006in}{1.784939in}}%
\pgfpathcurveto{\pgfqpoint{5.577182in}{1.790763in}}{\pgfqpoint{5.569282in}{1.794035in}}{\pgfqpoint{5.561046in}{1.794035in}}%
\pgfpathcurveto{\pgfqpoint{5.552809in}{1.794035in}}{\pgfqpoint{5.544909in}{1.790763in}}{\pgfqpoint{5.539085in}{1.784939in}}%
\pgfpathcurveto{\pgfqpoint{5.533262in}{1.779115in}}{\pgfqpoint{5.529989in}{1.771215in}}{\pgfqpoint{5.529989in}{1.762979in}}%
\pgfpathcurveto{\pgfqpoint{5.529989in}{1.754743in}}{\pgfqpoint{5.533262in}{1.746843in}}{\pgfqpoint{5.539085in}{1.741019in}}%
\pgfpathcurveto{\pgfqpoint{5.544909in}{1.735195in}}{\pgfqpoint{5.552809in}{1.731922in}}{\pgfqpoint{5.561046in}{1.731922in}}%
\pgfpathclose%
\pgfusepath{stroke,fill}%
\end{pgfscope}%
\begin{pgfscope}%
\pgfpathrectangle{\pgfqpoint{3.874179in}{0.557870in}}{\pgfqpoint{2.484109in}{1.684734in}}%
\pgfusepath{clip}%
\pgfsetbuttcap%
\pgfsetroundjoin%
\definecolor{currentfill}{rgb}{0.298039,0.447059,0.690196}%
\pgfsetfillcolor{currentfill}%
\pgfsetlinewidth{1.003750pt}%
\definecolor{currentstroke}{rgb}{0.298039,0.447059,0.690196}%
\pgfsetstrokecolor{currentstroke}%
\pgfsetdash{}{0pt}%
\pgfpathmoveto{\pgfqpoint{5.467521in}{1.789129in}}%
\pgfpathcurveto{\pgfqpoint{5.475757in}{1.789129in}}{\pgfqpoint{5.483657in}{1.792401in}}{\pgfqpoint{5.489481in}{1.798225in}}%
\pgfpathcurveto{\pgfqpoint{5.495305in}{1.804049in}}{\pgfqpoint{5.498577in}{1.811949in}}{\pgfqpoint{5.498577in}{1.820186in}}%
\pgfpathcurveto{\pgfqpoint{5.498577in}{1.828422in}}{\pgfqpoint{5.495305in}{1.836322in}}{\pgfqpoint{5.489481in}{1.842146in}}%
\pgfpathcurveto{\pgfqpoint{5.483657in}{1.847970in}}{\pgfqpoint{5.475757in}{1.851242in}}{\pgfqpoint{5.467521in}{1.851242in}}%
\pgfpathcurveto{\pgfqpoint{5.459285in}{1.851242in}}{\pgfqpoint{5.451385in}{1.847970in}}{\pgfqpoint{5.445561in}{1.842146in}}%
\pgfpathcurveto{\pgfqpoint{5.439737in}{1.836322in}}{\pgfqpoint{5.436464in}{1.828422in}}{\pgfqpoint{5.436464in}{1.820186in}}%
\pgfpathcurveto{\pgfqpoint{5.436464in}{1.811949in}}{\pgfqpoint{5.439737in}{1.804049in}}{\pgfqpoint{5.445561in}{1.798225in}}%
\pgfpathcurveto{\pgfqpoint{5.451385in}{1.792401in}}{\pgfqpoint{5.459285in}{1.789129in}}{\pgfqpoint{5.467521in}{1.789129in}}%
\pgfpathclose%
\pgfusepath{stroke,fill}%
\end{pgfscope}%
\begin{pgfscope}%
\pgfpathrectangle{\pgfqpoint{3.874179in}{0.557870in}}{\pgfqpoint{2.484109in}{1.684734in}}%
\pgfusepath{clip}%
\pgfsetbuttcap%
\pgfsetroundjoin%
\definecolor{currentfill}{rgb}{0.298039,0.447059,0.690196}%
\pgfsetfillcolor{currentfill}%
\pgfsetlinewidth{1.003750pt}%
\definecolor{currentstroke}{rgb}{0.298039,0.447059,0.690196}%
\pgfsetstrokecolor{currentstroke}%
\pgfsetdash{}{0pt}%
\pgfpathmoveto{\pgfqpoint{5.542797in}{1.756192in}}%
\pgfpathcurveto{\pgfqpoint{5.551033in}{1.756192in}}{\pgfqpoint{5.558933in}{1.759464in}}{\pgfqpoint{5.564757in}{1.765288in}}%
\pgfpathcurveto{\pgfqpoint{5.570581in}{1.771112in}}{\pgfqpoint{5.573853in}{1.779012in}}{\pgfqpoint{5.573853in}{1.787248in}}%
\pgfpathcurveto{\pgfqpoint{5.573853in}{1.795485in}}{\pgfqpoint{5.570581in}{1.803385in}}{\pgfqpoint{5.564757in}{1.809209in}}%
\pgfpathcurveto{\pgfqpoint{5.558933in}{1.815033in}}{\pgfqpoint{5.551033in}{1.818305in}}{\pgfqpoint{5.542797in}{1.818305in}}%
\pgfpathcurveto{\pgfqpoint{5.534561in}{1.818305in}}{\pgfqpoint{5.526661in}{1.815033in}}{\pgfqpoint{5.520837in}{1.809209in}}%
\pgfpathcurveto{\pgfqpoint{5.515013in}{1.803385in}}{\pgfqpoint{5.511740in}{1.795485in}}{\pgfqpoint{5.511740in}{1.787248in}}%
\pgfpathcurveto{\pgfqpoint{5.511740in}{1.779012in}}{\pgfqpoint{5.515013in}{1.771112in}}{\pgfqpoint{5.520837in}{1.765288in}}%
\pgfpathcurveto{\pgfqpoint{5.526661in}{1.759464in}}{\pgfqpoint{5.534561in}{1.756192in}}{\pgfqpoint{5.542797in}{1.756192in}}%
\pgfpathclose%
\pgfusepath{stroke,fill}%
\end{pgfscope}%
\begin{pgfscope}%
\pgfpathrectangle{\pgfqpoint{3.874179in}{0.557870in}}{\pgfqpoint{2.484109in}{1.684734in}}%
\pgfusepath{clip}%
\pgfsetbuttcap%
\pgfsetroundjoin%
\definecolor{currentfill}{rgb}{0.298039,0.447059,0.690196}%
\pgfsetfillcolor{currentfill}%
\pgfsetlinewidth{1.003750pt}%
\definecolor{currentstroke}{rgb}{0.298039,0.447059,0.690196}%
\pgfsetstrokecolor{currentstroke}%
\pgfsetdash{}{0pt}%
\pgfpathmoveto{\pgfqpoint{5.367153in}{1.850759in}}%
\pgfpathcurveto{\pgfqpoint{5.375389in}{1.850759in}}{\pgfqpoint{5.383289in}{1.854031in}}{\pgfqpoint{5.389113in}{1.859855in}}%
\pgfpathcurveto{\pgfqpoint{5.394937in}{1.865679in}}{\pgfqpoint{5.398209in}{1.873579in}}{\pgfqpoint{5.398209in}{1.881815in}}%
\pgfpathcurveto{\pgfqpoint{5.398209in}{1.890052in}}{\pgfqpoint{5.394937in}{1.897952in}}{\pgfqpoint{5.389113in}{1.903776in}}%
\pgfpathcurveto{\pgfqpoint{5.383289in}{1.909600in}}{\pgfqpoint{5.375389in}{1.912872in}}{\pgfqpoint{5.367153in}{1.912872in}}%
\pgfpathcurveto{\pgfqpoint{5.358917in}{1.912872in}}{\pgfqpoint{5.351017in}{1.909600in}}{\pgfqpoint{5.345193in}{1.903776in}}%
\pgfpathcurveto{\pgfqpoint{5.339369in}{1.897952in}}{\pgfqpoint{5.336096in}{1.890052in}}{\pgfqpoint{5.336096in}{1.881815in}}%
\pgfpathcurveto{\pgfqpoint{5.336096in}{1.873579in}}{\pgfqpoint{5.339369in}{1.865679in}}{\pgfqpoint{5.345193in}{1.859855in}}%
\pgfpathcurveto{\pgfqpoint{5.351017in}{1.854031in}}{\pgfqpoint{5.358917in}{1.850759in}}{\pgfqpoint{5.367153in}{1.850759in}}%
\pgfpathclose%
\pgfusepath{stroke,fill}%
\end{pgfscope}%
\begin{pgfscope}%
\pgfpathrectangle{\pgfqpoint{3.874179in}{0.557870in}}{\pgfqpoint{2.484109in}{1.684734in}}%
\pgfusepath{clip}%
\pgfsetbuttcap%
\pgfsetroundjoin%
\definecolor{currentfill}{rgb}{0.298039,0.447059,0.690196}%
\pgfsetfillcolor{currentfill}%
\pgfsetlinewidth{1.003750pt}%
\definecolor{currentstroke}{rgb}{0.298039,0.447059,0.690196}%
\pgfsetstrokecolor{currentstroke}%
\pgfsetdash{}{0pt}%
\pgfpathmoveto{\pgfqpoint{5.629478in}{1.715801in}}%
\pgfpathcurveto{\pgfqpoint{5.637715in}{1.715801in}}{\pgfqpoint{5.645615in}{1.719073in}}{\pgfqpoint{5.651439in}{1.724897in}}%
\pgfpathcurveto{\pgfqpoint{5.657263in}{1.730721in}}{\pgfqpoint{5.660535in}{1.738621in}}{\pgfqpoint{5.660535in}{1.746857in}}%
\pgfpathcurveto{\pgfqpoint{5.660535in}{1.755093in}}{\pgfqpoint{5.657263in}{1.762993in}}{\pgfqpoint{5.651439in}{1.768817in}}%
\pgfpathcurveto{\pgfqpoint{5.645615in}{1.774641in}}{\pgfqpoint{5.637715in}{1.777914in}}{\pgfqpoint{5.629478in}{1.777914in}}%
\pgfpathcurveto{\pgfqpoint{5.621242in}{1.777914in}}{\pgfqpoint{5.613342in}{1.774641in}}{\pgfqpoint{5.607518in}{1.768817in}}%
\pgfpathcurveto{\pgfqpoint{5.601694in}{1.762993in}}{\pgfqpoint{5.598422in}{1.755093in}}{\pgfqpoint{5.598422in}{1.746857in}}%
\pgfpathcurveto{\pgfqpoint{5.598422in}{1.738621in}}{\pgfqpoint{5.601694in}{1.730721in}}{\pgfqpoint{5.607518in}{1.724897in}}%
\pgfpathcurveto{\pgfqpoint{5.613342in}{1.719073in}}{\pgfqpoint{5.621242in}{1.715801in}}{\pgfqpoint{5.629478in}{1.715801in}}%
\pgfpathclose%
\pgfusepath{stroke,fill}%
\end{pgfscope}%
\begin{pgfscope}%
\pgfpathrectangle{\pgfqpoint{3.874179in}{0.557870in}}{\pgfqpoint{2.484109in}{1.684734in}}%
\pgfusepath{clip}%
\pgfsetbuttcap%
\pgfsetroundjoin%
\definecolor{currentfill}{rgb}{0.298039,0.447059,0.690196}%
\pgfsetfillcolor{currentfill}%
\pgfsetlinewidth{1.003750pt}%
\definecolor{currentstroke}{rgb}{0.298039,0.447059,0.690196}%
\pgfsetstrokecolor{currentstroke}%
\pgfsetdash{}{0pt}%
\pgfpathmoveto{\pgfqpoint{5.592981in}{1.739723in}}%
\pgfpathcurveto{\pgfqpoint{5.601217in}{1.739723in}}{\pgfqpoint{5.609117in}{1.742996in}}{\pgfqpoint{5.614941in}{1.748820in}}%
\pgfpathcurveto{\pgfqpoint{5.620765in}{1.754644in}}{\pgfqpoint{5.624038in}{1.762544in}}{\pgfqpoint{5.624038in}{1.770780in}}%
\pgfpathcurveto{\pgfqpoint{5.624038in}{1.779016in}}{\pgfqpoint{5.620765in}{1.786916in}}{\pgfqpoint{5.614941in}{1.792740in}}%
\pgfpathcurveto{\pgfqpoint{5.609117in}{1.798564in}}{\pgfqpoint{5.601217in}{1.801836in}}{\pgfqpoint{5.592981in}{1.801836in}}%
\pgfpathcurveto{\pgfqpoint{5.584745in}{1.801836in}}{\pgfqpoint{5.576845in}{1.798564in}}{\pgfqpoint{5.571021in}{1.792740in}}%
\pgfpathcurveto{\pgfqpoint{5.565197in}{1.786916in}}{\pgfqpoint{5.561925in}{1.779016in}}{\pgfqpoint{5.561925in}{1.770780in}}%
\pgfpathcurveto{\pgfqpoint{5.561925in}{1.762544in}}{\pgfqpoint{5.565197in}{1.754644in}}{\pgfqpoint{5.571021in}{1.748820in}}%
\pgfpathcurveto{\pgfqpoint{5.576845in}{1.742996in}}{\pgfqpoint{5.584745in}{1.739723in}}{\pgfqpoint{5.592981in}{1.739723in}}%
\pgfpathclose%
\pgfusepath{stroke,fill}%
\end{pgfscope}%
\begin{pgfscope}%
\pgfpathrectangle{\pgfqpoint{3.874179in}{0.557870in}}{\pgfqpoint{2.484109in}{1.684734in}}%
\pgfusepath{clip}%
\pgfsetbuttcap%
\pgfsetroundjoin%
\definecolor{currentfill}{rgb}{0.298039,0.447059,0.690196}%
\pgfsetfillcolor{currentfill}%
\pgfsetlinewidth{1.003750pt}%
\definecolor{currentstroke}{rgb}{0.298039,0.447059,0.690196}%
\pgfsetstrokecolor{currentstroke}%
\pgfsetdash{}{0pt}%
\pgfpathmoveto{\pgfqpoint{5.675100in}{1.683557in}}%
\pgfpathcurveto{\pgfqpoint{5.683337in}{1.683557in}}{\pgfqpoint{5.691237in}{1.686829in}}{\pgfqpoint{5.697061in}{1.692653in}}%
\pgfpathcurveto{\pgfqpoint{5.702884in}{1.698477in}}{\pgfqpoint{5.706157in}{1.706377in}}{\pgfqpoint{5.706157in}{1.714613in}}%
\pgfpathcurveto{\pgfqpoint{5.706157in}{1.722850in}}{\pgfqpoint{5.702884in}{1.730750in}}{\pgfqpoint{5.697061in}{1.736574in}}%
\pgfpathcurveto{\pgfqpoint{5.691237in}{1.742398in}}{\pgfqpoint{5.683337in}{1.745670in}}{\pgfqpoint{5.675100in}{1.745670in}}%
\pgfpathcurveto{\pgfqpoint{5.666864in}{1.745670in}}{\pgfqpoint{5.658964in}{1.742398in}}{\pgfqpoint{5.653140in}{1.736574in}}%
\pgfpathcurveto{\pgfqpoint{5.647316in}{1.730750in}}{\pgfqpoint{5.644044in}{1.722850in}}{\pgfqpoint{5.644044in}{1.714613in}}%
\pgfpathcurveto{\pgfqpoint{5.644044in}{1.706377in}}{\pgfqpoint{5.647316in}{1.698477in}}{\pgfqpoint{5.653140in}{1.692653in}}%
\pgfpathcurveto{\pgfqpoint{5.658964in}{1.686829in}}{\pgfqpoint{5.666864in}{1.683557in}}{\pgfqpoint{5.675100in}{1.683557in}}%
\pgfpathclose%
\pgfusepath{stroke,fill}%
\end{pgfscope}%
\begin{pgfscope}%
\pgfpathrectangle{\pgfqpoint{3.874179in}{0.557870in}}{\pgfqpoint{2.484109in}{1.684734in}}%
\pgfusepath{clip}%
\pgfsetbuttcap%
\pgfsetroundjoin%
\definecolor{currentfill}{rgb}{0.298039,0.447059,0.690196}%
\pgfsetfillcolor{currentfill}%
\pgfsetlinewidth{1.003750pt}%
\definecolor{currentstroke}{rgb}{0.298039,0.447059,0.690196}%
\pgfsetstrokecolor{currentstroke}%
\pgfsetdash{}{0pt}%
\pgfpathmoveto{\pgfqpoint{5.567889in}{1.756192in}}%
\pgfpathcurveto{\pgfqpoint{5.576125in}{1.756192in}}{\pgfqpoint{5.584025in}{1.759464in}}{\pgfqpoint{5.589849in}{1.765288in}}%
\pgfpathcurveto{\pgfqpoint{5.595673in}{1.771112in}}{\pgfqpoint{5.598946in}{1.779012in}}{\pgfqpoint{5.598946in}{1.787248in}}%
\pgfpathcurveto{\pgfqpoint{5.598946in}{1.795485in}}{\pgfqpoint{5.595673in}{1.803385in}}{\pgfqpoint{5.589849in}{1.809209in}}%
\pgfpathcurveto{\pgfqpoint{5.584025in}{1.815033in}}{\pgfqpoint{5.576125in}{1.818305in}}{\pgfqpoint{5.567889in}{1.818305in}}%
\pgfpathcurveto{\pgfqpoint{5.559653in}{1.818305in}}{\pgfqpoint{5.551753in}{1.815033in}}{\pgfqpoint{5.545929in}{1.809209in}}%
\pgfpathcurveto{\pgfqpoint{5.540105in}{1.803385in}}{\pgfqpoint{5.536833in}{1.795485in}}{\pgfqpoint{5.536833in}{1.787248in}}%
\pgfpathcurveto{\pgfqpoint{5.536833in}{1.779012in}}{\pgfqpoint{5.540105in}{1.771112in}}{\pgfqpoint{5.545929in}{1.765288in}}%
\pgfpathcurveto{\pgfqpoint{5.551753in}{1.759464in}}{\pgfqpoint{5.559653in}{1.756192in}}{\pgfqpoint{5.567889in}{1.756192in}}%
\pgfpathclose%
\pgfusepath{stroke,fill}%
\end{pgfscope}%
\begin{pgfscope}%
\pgfpathrectangle{\pgfqpoint{3.874179in}{0.557870in}}{\pgfqpoint{2.484109in}{1.684734in}}%
\pgfusepath{clip}%
\pgfsetbuttcap%
\pgfsetroundjoin%
\definecolor{currentfill}{rgb}{0.298039,0.447059,0.690196}%
\pgfsetfillcolor{currentfill}%
\pgfsetlinewidth{1.003750pt}%
\definecolor{currentstroke}{rgb}{0.298039,0.447059,0.690196}%
\pgfsetstrokecolor{currentstroke}%
\pgfsetdash{}{0pt}%
\pgfpathmoveto{\pgfqpoint{5.542797in}{1.772660in}}%
\pgfpathcurveto{\pgfqpoint{5.551033in}{1.772660in}}{\pgfqpoint{5.558933in}{1.775933in}}{\pgfqpoint{5.564757in}{1.781757in}}%
\pgfpathcurveto{\pgfqpoint{5.570581in}{1.787581in}}{\pgfqpoint{5.573853in}{1.795481in}}{\pgfqpoint{5.573853in}{1.803717in}}%
\pgfpathcurveto{\pgfqpoint{5.573853in}{1.811953in}}{\pgfqpoint{5.570581in}{1.819853in}}{\pgfqpoint{5.564757in}{1.825677in}}%
\pgfpathcurveto{\pgfqpoint{5.558933in}{1.831501in}}{\pgfqpoint{5.551033in}{1.834773in}}{\pgfqpoint{5.542797in}{1.834773in}}%
\pgfpathcurveto{\pgfqpoint{5.534561in}{1.834773in}}{\pgfqpoint{5.526661in}{1.831501in}}{\pgfqpoint{5.520837in}{1.825677in}}%
\pgfpathcurveto{\pgfqpoint{5.515013in}{1.819853in}}{\pgfqpoint{5.511740in}{1.811953in}}{\pgfqpoint{5.511740in}{1.803717in}}%
\pgfpathcurveto{\pgfqpoint{5.511740in}{1.795481in}}{\pgfqpoint{5.515013in}{1.787581in}}{\pgfqpoint{5.520837in}{1.781757in}}%
\pgfpathcurveto{\pgfqpoint{5.526661in}{1.775933in}}{\pgfqpoint{5.534561in}{1.772660in}}{\pgfqpoint{5.542797in}{1.772660in}}%
\pgfpathclose%
\pgfusepath{stroke,fill}%
\end{pgfscope}%
\begin{pgfscope}%
\pgfpathrectangle{\pgfqpoint{3.874179in}{0.557870in}}{\pgfqpoint{2.484109in}{1.684734in}}%
\pgfusepath{clip}%
\pgfsetbuttcap%
\pgfsetroundjoin%
\definecolor{currentfill}{rgb}{0.298039,0.447059,0.690196}%
\pgfsetfillcolor{currentfill}%
\pgfsetlinewidth{1.003750pt}%
\definecolor{currentstroke}{rgb}{0.298039,0.447059,0.690196}%
\pgfsetstrokecolor{currentstroke}%
\pgfsetdash{}{0pt}%
\pgfpathmoveto{\pgfqpoint{5.517705in}{1.789129in}}%
\pgfpathcurveto{\pgfqpoint{5.525941in}{1.789129in}}{\pgfqpoint{5.533841in}{1.792401in}}{\pgfqpoint{5.539665in}{1.798225in}}%
\pgfpathcurveto{\pgfqpoint{5.545489in}{1.804049in}}{\pgfqpoint{5.548761in}{1.811949in}}{\pgfqpoint{5.548761in}{1.820186in}}%
\pgfpathcurveto{\pgfqpoint{5.548761in}{1.828422in}}{\pgfqpoint{5.545489in}{1.836322in}}{\pgfqpoint{5.539665in}{1.842146in}}%
\pgfpathcurveto{\pgfqpoint{5.533841in}{1.847970in}}{\pgfqpoint{5.525941in}{1.851242in}}{\pgfqpoint{5.517705in}{1.851242in}}%
\pgfpathcurveto{\pgfqpoint{5.509469in}{1.851242in}}{\pgfqpoint{5.501569in}{1.847970in}}{\pgfqpoint{5.495745in}{1.842146in}}%
\pgfpathcurveto{\pgfqpoint{5.489921in}{1.836322in}}{\pgfqpoint{5.486648in}{1.828422in}}{\pgfqpoint{5.486648in}{1.820186in}}%
\pgfpathcurveto{\pgfqpoint{5.486648in}{1.811949in}}{\pgfqpoint{5.489921in}{1.804049in}}{\pgfqpoint{5.495745in}{1.798225in}}%
\pgfpathcurveto{\pgfqpoint{5.501569in}{1.792401in}}{\pgfqpoint{5.509469in}{1.789129in}}{\pgfqpoint{5.517705in}{1.789129in}}%
\pgfpathclose%
\pgfusepath{stroke,fill}%
\end{pgfscope}%
\begin{pgfscope}%
\pgfpathrectangle{\pgfqpoint{3.874179in}{0.557870in}}{\pgfqpoint{2.484109in}{1.684734in}}%
\pgfusepath{clip}%
\pgfsetbuttcap%
\pgfsetroundjoin%
\definecolor{currentfill}{rgb}{0.298039,0.447059,0.690196}%
\pgfsetfillcolor{currentfill}%
\pgfsetlinewidth{1.003750pt}%
\definecolor{currentstroke}{rgb}{0.298039,0.447059,0.690196}%
\pgfsetstrokecolor{currentstroke}%
\pgfsetdash{}{0pt}%
\pgfpathmoveto{\pgfqpoint{5.643165in}{1.723255in}}%
\pgfpathcurveto{\pgfqpoint{5.651401in}{1.723255in}}{\pgfqpoint{5.659301in}{1.726527in}}{\pgfqpoint{5.665125in}{1.732351in}}%
\pgfpathcurveto{\pgfqpoint{5.670949in}{1.738175in}}{\pgfqpoint{5.674222in}{1.746075in}}{\pgfqpoint{5.674222in}{1.754311in}}%
\pgfpathcurveto{\pgfqpoint{5.674222in}{1.762548in}}{\pgfqpoint{5.670949in}{1.770448in}}{\pgfqpoint{5.665125in}{1.776272in}}%
\pgfpathcurveto{\pgfqpoint{5.659301in}{1.782095in}}{\pgfqpoint{5.651401in}{1.785368in}}{\pgfqpoint{5.643165in}{1.785368in}}%
\pgfpathcurveto{\pgfqpoint{5.634929in}{1.785368in}}{\pgfqpoint{5.627029in}{1.782095in}}{\pgfqpoint{5.621205in}{1.776272in}}%
\pgfpathcurveto{\pgfqpoint{5.615381in}{1.770448in}}{\pgfqpoint{5.612109in}{1.762548in}}{\pgfqpoint{5.612109in}{1.754311in}}%
\pgfpathcurveto{\pgfqpoint{5.612109in}{1.746075in}}{\pgfqpoint{5.615381in}{1.738175in}}{\pgfqpoint{5.621205in}{1.732351in}}%
\pgfpathcurveto{\pgfqpoint{5.627029in}{1.726527in}}{\pgfqpoint{5.634929in}{1.723255in}}{\pgfqpoint{5.643165in}{1.723255in}}%
\pgfpathclose%
\pgfusepath{stroke,fill}%
\end{pgfscope}%
\begin{pgfscope}%
\pgfpathrectangle{\pgfqpoint{3.874179in}{0.557870in}}{\pgfqpoint{2.484109in}{1.684734in}}%
\pgfusepath{clip}%
\pgfsetbuttcap%
\pgfsetroundjoin%
\definecolor{currentfill}{rgb}{0.298039,0.447059,0.690196}%
\pgfsetfillcolor{currentfill}%
\pgfsetlinewidth{1.003750pt}%
\definecolor{currentstroke}{rgb}{0.298039,0.447059,0.690196}%
\pgfsetstrokecolor{currentstroke}%
\pgfsetdash{}{0pt}%
\pgfpathmoveto{\pgfqpoint{5.592981in}{1.756022in}}%
\pgfpathcurveto{\pgfqpoint{5.601217in}{1.756022in}}{\pgfqpoint{5.609117in}{1.759294in}}{\pgfqpoint{5.614941in}{1.765118in}}%
\pgfpathcurveto{\pgfqpoint{5.620765in}{1.770942in}}{\pgfqpoint{5.624038in}{1.778842in}}{\pgfqpoint{5.624038in}{1.787079in}}%
\pgfpathcurveto{\pgfqpoint{5.624038in}{1.795315in}}{\pgfqpoint{5.620765in}{1.803215in}}{\pgfqpoint{5.614941in}{1.809039in}}%
\pgfpathcurveto{\pgfqpoint{5.609117in}{1.814863in}}{\pgfqpoint{5.601217in}{1.818135in}}{\pgfqpoint{5.592981in}{1.818135in}}%
\pgfpathcurveto{\pgfqpoint{5.584745in}{1.818135in}}{\pgfqpoint{5.576845in}{1.814863in}}{\pgfqpoint{5.571021in}{1.809039in}}%
\pgfpathcurveto{\pgfqpoint{5.565197in}{1.803215in}}{\pgfqpoint{5.561925in}{1.795315in}}{\pgfqpoint{5.561925in}{1.787079in}}%
\pgfpathcurveto{\pgfqpoint{5.561925in}{1.778842in}}{\pgfqpoint{5.565197in}{1.770942in}}{\pgfqpoint{5.571021in}{1.765118in}}%
\pgfpathcurveto{\pgfqpoint{5.576845in}{1.759294in}}{\pgfqpoint{5.584745in}{1.756022in}}{\pgfqpoint{5.592981in}{1.756022in}}%
\pgfpathclose%
\pgfusepath{stroke,fill}%
\end{pgfscope}%
\begin{pgfscope}%
\pgfpathrectangle{\pgfqpoint{3.874179in}{0.557870in}}{\pgfqpoint{2.484109in}{1.684734in}}%
\pgfusepath{clip}%
\pgfsetbuttcap%
\pgfsetroundjoin%
\definecolor{currentfill}{rgb}{0.298039,0.447059,0.690196}%
\pgfsetfillcolor{currentfill}%
\pgfsetlinewidth{1.003750pt}%
\definecolor{currentstroke}{rgb}{0.298039,0.447059,0.690196}%
\pgfsetstrokecolor{currentstroke}%
\pgfsetdash{}{0pt}%
\pgfpathmoveto{\pgfqpoint{5.618073in}{1.740233in}}%
\pgfpathcurveto{\pgfqpoint{5.626309in}{1.740233in}}{\pgfqpoint{5.634209in}{1.743505in}}{\pgfqpoint{5.640033in}{1.749329in}}%
\pgfpathcurveto{\pgfqpoint{5.645857in}{1.755153in}}{\pgfqpoint{5.649130in}{1.763053in}}{\pgfqpoint{5.649130in}{1.771289in}}%
\pgfpathcurveto{\pgfqpoint{5.649130in}{1.779525in}}{\pgfqpoint{5.645857in}{1.787426in}}{\pgfqpoint{5.640033in}{1.793249in}}%
\pgfpathcurveto{\pgfqpoint{5.634209in}{1.799073in}}{\pgfqpoint{5.626309in}{1.802346in}}{\pgfqpoint{5.618073in}{1.802346in}}%
\pgfpathcurveto{\pgfqpoint{5.609837in}{1.802346in}}{\pgfqpoint{5.601937in}{1.799073in}}{\pgfqpoint{5.596113in}{1.793249in}}%
\pgfpathcurveto{\pgfqpoint{5.590289in}{1.787426in}}{\pgfqpoint{5.587017in}{1.779525in}}{\pgfqpoint{5.587017in}{1.771289in}}%
\pgfpathcurveto{\pgfqpoint{5.587017in}{1.763053in}}{\pgfqpoint{5.590289in}{1.755153in}}{\pgfqpoint{5.596113in}{1.749329in}}%
\pgfpathcurveto{\pgfqpoint{5.601937in}{1.743505in}}{\pgfqpoint{5.609837in}{1.740233in}}{\pgfqpoint{5.618073in}{1.740233in}}%
\pgfpathclose%
\pgfusepath{stroke,fill}%
\end{pgfscope}%
\begin{pgfscope}%
\pgfpathrectangle{\pgfqpoint{3.874179in}{0.557870in}}{\pgfqpoint{2.484109in}{1.684734in}}%
\pgfusepath{clip}%
\pgfsetbuttcap%
\pgfsetroundjoin%
\definecolor{currentfill}{rgb}{0.298039,0.447059,0.690196}%
\pgfsetfillcolor{currentfill}%
\pgfsetlinewidth{1.003750pt}%
\definecolor{currentstroke}{rgb}{0.298039,0.447059,0.690196}%
\pgfsetstrokecolor{currentstroke}%
\pgfsetdash{}{0pt}%
\pgfpathmoveto{\pgfqpoint{5.567889in}{1.772660in}}%
\pgfpathcurveto{\pgfqpoint{5.576125in}{1.772660in}}{\pgfqpoint{5.584025in}{1.775933in}}{\pgfqpoint{5.589849in}{1.781757in}}%
\pgfpathcurveto{\pgfqpoint{5.595673in}{1.787581in}}{\pgfqpoint{5.598946in}{1.795481in}}{\pgfqpoint{5.598946in}{1.803717in}}%
\pgfpathcurveto{\pgfqpoint{5.598946in}{1.811953in}}{\pgfqpoint{5.595673in}{1.819853in}}{\pgfqpoint{5.589849in}{1.825677in}}%
\pgfpathcurveto{\pgfqpoint{5.584025in}{1.831501in}}{\pgfqpoint{5.576125in}{1.834773in}}{\pgfqpoint{5.567889in}{1.834773in}}%
\pgfpathcurveto{\pgfqpoint{5.559653in}{1.834773in}}{\pgfqpoint{5.551753in}{1.831501in}}{\pgfqpoint{5.545929in}{1.825677in}}%
\pgfpathcurveto{\pgfqpoint{5.540105in}{1.819853in}}{\pgfqpoint{5.536833in}{1.811953in}}{\pgfqpoint{5.536833in}{1.803717in}}%
\pgfpathcurveto{\pgfqpoint{5.536833in}{1.795481in}}{\pgfqpoint{5.540105in}{1.787581in}}{\pgfqpoint{5.545929in}{1.781757in}}%
\pgfpathcurveto{\pgfqpoint{5.551753in}{1.775933in}}{\pgfqpoint{5.559653in}{1.772660in}}{\pgfqpoint{5.567889in}{1.772660in}}%
\pgfpathclose%
\pgfusepath{stroke,fill}%
\end{pgfscope}%
\begin{pgfscope}%
\pgfpathrectangle{\pgfqpoint{3.874179in}{0.557870in}}{\pgfqpoint{2.484109in}{1.684734in}}%
\pgfusepath{clip}%
\pgfsetbuttcap%
\pgfsetroundjoin%
\definecolor{currentfill}{rgb}{0.298039,0.447059,0.690196}%
\pgfsetfillcolor{currentfill}%
\pgfsetlinewidth{1.003750pt}%
\definecolor{currentstroke}{rgb}{0.298039,0.447059,0.690196}%
\pgfsetstrokecolor{currentstroke}%
\pgfsetdash{}{0pt}%
\pgfpathmoveto{\pgfqpoint{5.492613in}{1.819180in}}%
\pgfpathcurveto{\pgfqpoint{5.500849in}{1.819180in}}{\pgfqpoint{5.508749in}{1.822452in}}{\pgfqpoint{5.514573in}{1.828276in}}%
\pgfpathcurveto{\pgfqpoint{5.520397in}{1.834100in}}{\pgfqpoint{5.523669in}{1.842000in}}{\pgfqpoint{5.523669in}{1.850236in}}%
\pgfpathcurveto{\pgfqpoint{5.523669in}{1.858473in}}{\pgfqpoint{5.520397in}{1.866373in}}{\pgfqpoint{5.514573in}{1.872197in}}%
\pgfpathcurveto{\pgfqpoint{5.508749in}{1.878021in}}{\pgfqpoint{5.500849in}{1.881293in}}{\pgfqpoint{5.492613in}{1.881293in}}%
\pgfpathcurveto{\pgfqpoint{5.484377in}{1.881293in}}{\pgfqpoint{5.476477in}{1.878021in}}{\pgfqpoint{5.470653in}{1.872197in}}%
\pgfpathcurveto{\pgfqpoint{5.464829in}{1.866373in}}{\pgfqpoint{5.461556in}{1.858473in}}{\pgfqpoint{5.461556in}{1.850236in}}%
\pgfpathcurveto{\pgfqpoint{5.461556in}{1.842000in}}{\pgfqpoint{5.464829in}{1.834100in}}{\pgfqpoint{5.470653in}{1.828276in}}%
\pgfpathcurveto{\pgfqpoint{5.476477in}{1.822452in}}{\pgfqpoint{5.484377in}{1.819180in}}{\pgfqpoint{5.492613in}{1.819180in}}%
\pgfpathclose%
\pgfusepath{stroke,fill}%
\end{pgfscope}%
\begin{pgfscope}%
\pgfpathrectangle{\pgfqpoint{3.874179in}{0.557870in}}{\pgfqpoint{2.484109in}{1.684734in}}%
\pgfusepath{clip}%
\pgfsetbuttcap%
\pgfsetroundjoin%
\definecolor{currentfill}{rgb}{0.298039,0.447059,0.690196}%
\pgfsetfillcolor{currentfill}%
\pgfsetlinewidth{1.003750pt}%
\definecolor{currentstroke}{rgb}{0.298039,0.447059,0.690196}%
\pgfsetstrokecolor{currentstroke}%
\pgfsetdash{}{0pt}%
\pgfpathmoveto{\pgfqpoint{5.538235in}{1.796410in}}%
\pgfpathcurveto{\pgfqpoint{5.546471in}{1.796410in}}{\pgfqpoint{5.554371in}{1.799682in}}{\pgfqpoint{5.560195in}{1.805506in}}%
\pgfpathcurveto{\pgfqpoint{5.566019in}{1.811330in}}{\pgfqpoint{5.569291in}{1.819230in}}{\pgfqpoint{5.569291in}{1.827466in}}%
\pgfpathcurveto{\pgfqpoint{5.569291in}{1.835703in}}{\pgfqpoint{5.566019in}{1.843603in}}{\pgfqpoint{5.560195in}{1.849427in}}%
\pgfpathcurveto{\pgfqpoint{5.554371in}{1.855251in}}{\pgfqpoint{5.546471in}{1.858523in}}{\pgfqpoint{5.538235in}{1.858523in}}%
\pgfpathcurveto{\pgfqpoint{5.529999in}{1.858523in}}{\pgfqpoint{5.522098in}{1.855251in}}{\pgfqpoint{5.516275in}{1.849427in}}%
\pgfpathcurveto{\pgfqpoint{5.510451in}{1.843603in}}{\pgfqpoint{5.507178in}{1.835703in}}{\pgfqpoint{5.507178in}{1.827466in}}%
\pgfpathcurveto{\pgfqpoint{5.507178in}{1.819230in}}{\pgfqpoint{5.510451in}{1.811330in}}{\pgfqpoint{5.516275in}{1.805506in}}%
\pgfpathcurveto{\pgfqpoint{5.522098in}{1.799682in}}{\pgfqpoint{5.529999in}{1.796410in}}{\pgfqpoint{5.538235in}{1.796410in}}%
\pgfpathclose%
\pgfusepath{stroke,fill}%
\end{pgfscope}%
\begin{pgfscope}%
\pgfpathrectangle{\pgfqpoint{3.874179in}{0.557870in}}{\pgfqpoint{2.484109in}{1.684734in}}%
\pgfusepath{clip}%
\pgfsetbuttcap%
\pgfsetroundjoin%
\definecolor{currentfill}{rgb}{0.298039,0.447059,0.690196}%
\pgfsetfillcolor{currentfill}%
\pgfsetlinewidth{1.003750pt}%
\definecolor{currentstroke}{rgb}{0.298039,0.447059,0.690196}%
\pgfsetstrokecolor{currentstroke}%
\pgfsetdash{}{0pt}%
\pgfpathmoveto{\pgfqpoint{5.652289in}{1.731922in}}%
\pgfpathcurveto{\pgfqpoint{5.660526in}{1.731922in}}{\pgfqpoint{5.668426in}{1.735195in}}{\pgfqpoint{5.674250in}{1.741019in}}%
\pgfpathcurveto{\pgfqpoint{5.680074in}{1.746843in}}{\pgfqpoint{5.683346in}{1.754743in}}{\pgfqpoint{5.683346in}{1.762979in}}%
\pgfpathcurveto{\pgfqpoint{5.683346in}{1.771215in}}{\pgfqpoint{5.680074in}{1.779115in}}{\pgfqpoint{5.674250in}{1.784939in}}%
\pgfpathcurveto{\pgfqpoint{5.668426in}{1.790763in}}{\pgfqpoint{5.660526in}{1.794035in}}{\pgfqpoint{5.652289in}{1.794035in}}%
\pgfpathcurveto{\pgfqpoint{5.644053in}{1.794035in}}{\pgfqpoint{5.636153in}{1.790763in}}{\pgfqpoint{5.630329in}{1.784939in}}%
\pgfpathcurveto{\pgfqpoint{5.624505in}{1.779115in}}{\pgfqpoint{5.621233in}{1.771215in}}{\pgfqpoint{5.621233in}{1.762979in}}%
\pgfpathcurveto{\pgfqpoint{5.621233in}{1.754743in}}{\pgfqpoint{5.624505in}{1.746843in}}{\pgfqpoint{5.630329in}{1.741019in}}%
\pgfpathcurveto{\pgfqpoint{5.636153in}{1.735195in}}{\pgfqpoint{5.644053in}{1.731922in}}{\pgfqpoint{5.652289in}{1.731922in}}%
\pgfpathclose%
\pgfusepath{stroke,fill}%
\end{pgfscope}%
\begin{pgfscope}%
\pgfpathrectangle{\pgfqpoint{3.874179in}{0.557870in}}{\pgfqpoint{2.484109in}{1.684734in}}%
\pgfusepath{clip}%
\pgfsetbuttcap%
\pgfsetroundjoin%
\definecolor{currentfill}{rgb}{0.298039,0.447059,0.690196}%
\pgfsetfillcolor{currentfill}%
\pgfsetlinewidth{1.003750pt}%
\definecolor{currentstroke}{rgb}{0.298039,0.447059,0.690196}%
\pgfsetstrokecolor{currentstroke}%
\pgfsetdash{}{0pt}%
\pgfpathmoveto{\pgfqpoint{5.542797in}{1.803390in}}%
\pgfpathcurveto{\pgfqpoint{5.551033in}{1.803390in}}{\pgfqpoint{5.558933in}{1.806663in}}{\pgfqpoint{5.564757in}{1.812487in}}%
\pgfpathcurveto{\pgfqpoint{5.570581in}{1.818311in}}{\pgfqpoint{5.573853in}{1.826211in}}{\pgfqpoint{5.573853in}{1.834447in}}%
\pgfpathcurveto{\pgfqpoint{5.573853in}{1.842683in}}{\pgfqpoint{5.570581in}{1.850583in}}{\pgfqpoint{5.564757in}{1.856407in}}%
\pgfpathcurveto{\pgfqpoint{5.558933in}{1.862231in}}{\pgfqpoint{5.551033in}{1.865503in}}{\pgfqpoint{5.542797in}{1.865503in}}%
\pgfpathcurveto{\pgfqpoint{5.534561in}{1.865503in}}{\pgfqpoint{5.526661in}{1.862231in}}{\pgfqpoint{5.520837in}{1.856407in}}%
\pgfpathcurveto{\pgfqpoint{5.515013in}{1.850583in}}{\pgfqpoint{5.511740in}{1.842683in}}{\pgfqpoint{5.511740in}{1.834447in}}%
\pgfpathcurveto{\pgfqpoint{5.511740in}{1.826211in}}{\pgfqpoint{5.515013in}{1.818311in}}{\pgfqpoint{5.520837in}{1.812487in}}%
\pgfpathcurveto{\pgfqpoint{5.526661in}{1.806663in}}{\pgfqpoint{5.534561in}{1.803390in}}{\pgfqpoint{5.542797in}{1.803390in}}%
\pgfpathclose%
\pgfusepath{stroke,fill}%
\end{pgfscope}%
\begin{pgfscope}%
\pgfpathrectangle{\pgfqpoint{3.874179in}{0.557870in}}{\pgfqpoint{2.484109in}{1.684734in}}%
\pgfusepath{clip}%
\pgfsetbuttcap%
\pgfsetroundjoin%
\definecolor{currentfill}{rgb}{0.298039,0.447059,0.690196}%
\pgfsetfillcolor{currentfill}%
\pgfsetlinewidth{1.003750pt}%
\definecolor{currentstroke}{rgb}{0.298039,0.447059,0.690196}%
\pgfsetstrokecolor{currentstroke}%
\pgfsetdash{}{0pt}%
\pgfpathmoveto{\pgfqpoint{5.675100in}{1.748044in}}%
\pgfpathcurveto{\pgfqpoint{5.683337in}{1.748044in}}{\pgfqpoint{5.691237in}{1.751317in}}{\pgfqpoint{5.697061in}{1.757141in}}%
\pgfpathcurveto{\pgfqpoint{5.702884in}{1.762964in}}{\pgfqpoint{5.706157in}{1.770865in}}{\pgfqpoint{5.706157in}{1.779101in}}%
\pgfpathcurveto{\pgfqpoint{5.706157in}{1.787337in}}{\pgfqpoint{5.702884in}{1.795237in}}{\pgfqpoint{5.697061in}{1.801061in}}%
\pgfpathcurveto{\pgfqpoint{5.691237in}{1.806885in}}{\pgfqpoint{5.683337in}{1.810157in}}{\pgfqpoint{5.675100in}{1.810157in}}%
\pgfpathcurveto{\pgfqpoint{5.666864in}{1.810157in}}{\pgfqpoint{5.658964in}{1.806885in}}{\pgfqpoint{5.653140in}{1.801061in}}%
\pgfpathcurveto{\pgfqpoint{5.647316in}{1.795237in}}{\pgfqpoint{5.644044in}{1.787337in}}{\pgfqpoint{5.644044in}{1.779101in}}%
\pgfpathcurveto{\pgfqpoint{5.644044in}{1.770865in}}{\pgfqpoint{5.647316in}{1.762964in}}{\pgfqpoint{5.653140in}{1.757141in}}%
\pgfpathcurveto{\pgfqpoint{5.658964in}{1.751317in}}{\pgfqpoint{5.666864in}{1.748044in}}{\pgfqpoint{5.675100in}{1.748044in}}%
\pgfpathclose%
\pgfusepath{stroke,fill}%
\end{pgfscope}%
\begin{pgfscope}%
\pgfpathrectangle{\pgfqpoint{3.874179in}{0.557870in}}{\pgfqpoint{2.484109in}{1.684734in}}%
\pgfusepath{clip}%
\pgfsetbuttcap%
\pgfsetroundjoin%
\definecolor{currentfill}{rgb}{0.298039,0.447059,0.690196}%
\pgfsetfillcolor{currentfill}%
\pgfsetlinewidth{1.003750pt}%
\definecolor{currentstroke}{rgb}{0.298039,0.447059,0.690196}%
\pgfsetstrokecolor{currentstroke}%
\pgfsetdash{}{0pt}%
\pgfpathmoveto{\pgfqpoint{5.652289in}{1.764166in}}%
\pgfpathcurveto{\pgfqpoint{5.660526in}{1.764166in}}{\pgfqpoint{5.668426in}{1.767438in}}{\pgfqpoint{5.674250in}{1.773262in}}%
\pgfpathcurveto{\pgfqpoint{5.680074in}{1.779086in}}{\pgfqpoint{5.683346in}{1.786986in}}{\pgfqpoint{5.683346in}{1.795223in}}%
\pgfpathcurveto{\pgfqpoint{5.683346in}{1.803459in}}{\pgfqpoint{5.680074in}{1.811359in}}{\pgfqpoint{5.674250in}{1.817183in}}%
\pgfpathcurveto{\pgfqpoint{5.668426in}{1.823007in}}{\pgfqpoint{5.660526in}{1.826279in}}{\pgfqpoint{5.652289in}{1.826279in}}%
\pgfpathcurveto{\pgfqpoint{5.644053in}{1.826279in}}{\pgfqpoint{5.636153in}{1.823007in}}{\pgfqpoint{5.630329in}{1.817183in}}%
\pgfpathcurveto{\pgfqpoint{5.624505in}{1.811359in}}{\pgfqpoint{5.621233in}{1.803459in}}{\pgfqpoint{5.621233in}{1.795223in}}%
\pgfpathcurveto{\pgfqpoint{5.621233in}{1.786986in}}{\pgfqpoint{5.624505in}{1.779086in}}{\pgfqpoint{5.630329in}{1.773262in}}%
\pgfpathcurveto{\pgfqpoint{5.636153in}{1.767438in}}{\pgfqpoint{5.644053in}{1.764166in}}{\pgfqpoint{5.652289in}{1.764166in}}%
\pgfpathclose%
\pgfusepath{stroke,fill}%
\end{pgfscope}%
\begin{pgfscope}%
\pgfpathrectangle{\pgfqpoint{3.874179in}{0.557870in}}{\pgfqpoint{2.484109in}{1.684734in}}%
\pgfusepath{clip}%
\pgfsetbuttcap%
\pgfsetroundjoin%
\definecolor{currentfill}{rgb}{0.298039,0.447059,0.690196}%
\pgfsetfillcolor{currentfill}%
\pgfsetlinewidth{1.003750pt}%
\definecolor{currentstroke}{rgb}{0.298039,0.447059,0.690196}%
\pgfsetstrokecolor{currentstroke}%
\pgfsetdash{}{0pt}%
\pgfpathmoveto{\pgfqpoint{5.629478in}{1.780288in}}%
\pgfpathcurveto{\pgfqpoint{5.637715in}{1.780288in}}{\pgfqpoint{5.645615in}{1.783560in}}{\pgfqpoint{5.651439in}{1.789384in}}%
\pgfpathcurveto{\pgfqpoint{5.657263in}{1.795208in}}{\pgfqpoint{5.660535in}{1.803108in}}{\pgfqpoint{5.660535in}{1.811345in}}%
\pgfpathcurveto{\pgfqpoint{5.660535in}{1.819581in}}{\pgfqpoint{5.657263in}{1.827481in}}{\pgfqpoint{5.651439in}{1.833305in}}%
\pgfpathcurveto{\pgfqpoint{5.645615in}{1.839129in}}{\pgfqpoint{5.637715in}{1.842401in}}{\pgfqpoint{5.629478in}{1.842401in}}%
\pgfpathcurveto{\pgfqpoint{5.621242in}{1.842401in}}{\pgfqpoint{5.613342in}{1.839129in}}{\pgfqpoint{5.607518in}{1.833305in}}%
\pgfpathcurveto{\pgfqpoint{5.601694in}{1.827481in}}{\pgfqpoint{5.598422in}{1.819581in}}{\pgfqpoint{5.598422in}{1.811345in}}%
\pgfpathcurveto{\pgfqpoint{5.598422in}{1.803108in}}{\pgfqpoint{5.601694in}{1.795208in}}{\pgfqpoint{5.607518in}{1.789384in}}%
\pgfpathcurveto{\pgfqpoint{5.613342in}{1.783560in}}{\pgfqpoint{5.621242in}{1.780288in}}{\pgfqpoint{5.629478in}{1.780288in}}%
\pgfpathclose%
\pgfusepath{stroke,fill}%
\end{pgfscope}%
\begin{pgfscope}%
\pgfpathrectangle{\pgfqpoint{3.874179in}{0.557870in}}{\pgfqpoint{2.484109in}{1.684734in}}%
\pgfusepath{clip}%
\pgfsetbuttcap%
\pgfsetroundjoin%
\definecolor{currentfill}{rgb}{0.298039,0.447059,0.690196}%
\pgfsetfillcolor{currentfill}%
\pgfsetlinewidth{1.003750pt}%
\definecolor{currentstroke}{rgb}{0.298039,0.447059,0.690196}%
\pgfsetstrokecolor{currentstroke}%
\pgfsetdash{}{0pt}%
\pgfpathmoveto{\pgfqpoint{5.668257in}{1.756022in}}%
\pgfpathcurveto{\pgfqpoint{5.676493in}{1.756022in}}{\pgfqpoint{5.684393in}{1.759294in}}{\pgfqpoint{5.690217in}{1.765118in}}%
\pgfpathcurveto{\pgfqpoint{5.696041in}{1.770942in}}{\pgfqpoint{5.699314in}{1.778842in}}{\pgfqpoint{5.699314in}{1.787079in}}%
\pgfpathcurveto{\pgfqpoint{5.699314in}{1.795315in}}{\pgfqpoint{5.696041in}{1.803215in}}{\pgfqpoint{5.690217in}{1.809039in}}%
\pgfpathcurveto{\pgfqpoint{5.684393in}{1.814863in}}{\pgfqpoint{5.676493in}{1.818135in}}{\pgfqpoint{5.668257in}{1.818135in}}%
\pgfpathcurveto{\pgfqpoint{5.660021in}{1.818135in}}{\pgfqpoint{5.652121in}{1.814863in}}{\pgfqpoint{5.646297in}{1.809039in}}%
\pgfpathcurveto{\pgfqpoint{5.640473in}{1.803215in}}{\pgfqpoint{5.637201in}{1.795315in}}{\pgfqpoint{5.637201in}{1.787079in}}%
\pgfpathcurveto{\pgfqpoint{5.637201in}{1.778842in}}{\pgfqpoint{5.640473in}{1.770942in}}{\pgfqpoint{5.646297in}{1.765118in}}%
\pgfpathcurveto{\pgfqpoint{5.652121in}{1.759294in}}{\pgfqpoint{5.660021in}{1.756022in}}{\pgfqpoint{5.668257in}{1.756022in}}%
\pgfpathclose%
\pgfusepath{stroke,fill}%
\end{pgfscope}%
\begin{pgfscope}%
\pgfpathrectangle{\pgfqpoint{3.874179in}{0.557870in}}{\pgfqpoint{2.484109in}{1.684734in}}%
\pgfusepath{clip}%
\pgfsetbuttcap%
\pgfsetroundjoin%
\definecolor{currentfill}{rgb}{0.298039,0.447059,0.690196}%
\pgfsetfillcolor{currentfill}%
\pgfsetlinewidth{1.003750pt}%
\definecolor{currentstroke}{rgb}{0.298039,0.447059,0.690196}%
\pgfsetstrokecolor{currentstroke}%
\pgfsetdash{}{0pt}%
\pgfpathmoveto{\pgfqpoint{5.668257in}{1.771812in}}%
\pgfpathcurveto{\pgfqpoint{5.676493in}{1.771812in}}{\pgfqpoint{5.684393in}{1.775084in}}{\pgfqpoint{5.690217in}{1.780908in}}%
\pgfpathcurveto{\pgfqpoint{5.696041in}{1.786732in}}{\pgfqpoint{5.699314in}{1.794632in}}{\pgfqpoint{5.699314in}{1.802868in}}%
\pgfpathcurveto{\pgfqpoint{5.699314in}{1.811104in}}{\pgfqpoint{5.696041in}{1.819004in}}{\pgfqpoint{5.690217in}{1.824828in}}%
\pgfpathcurveto{\pgfqpoint{5.684393in}{1.830652in}}{\pgfqpoint{5.676493in}{1.833925in}}{\pgfqpoint{5.668257in}{1.833925in}}%
\pgfpathcurveto{\pgfqpoint{5.660021in}{1.833925in}}{\pgfqpoint{5.652121in}{1.830652in}}{\pgfqpoint{5.646297in}{1.824828in}}%
\pgfpathcurveto{\pgfqpoint{5.640473in}{1.819004in}}{\pgfqpoint{5.637201in}{1.811104in}}{\pgfqpoint{5.637201in}{1.802868in}}%
\pgfpathcurveto{\pgfqpoint{5.637201in}{1.794632in}}{\pgfqpoint{5.640473in}{1.786732in}}{\pgfqpoint{5.646297in}{1.780908in}}%
\pgfpathcurveto{\pgfqpoint{5.652121in}{1.775084in}}{\pgfqpoint{5.660021in}{1.771812in}}{\pgfqpoint{5.668257in}{1.771812in}}%
\pgfpathclose%
\pgfusepath{stroke,fill}%
\end{pgfscope}%
\begin{pgfscope}%
\pgfpathrectangle{\pgfqpoint{3.874179in}{0.557870in}}{\pgfqpoint{2.484109in}{1.684734in}}%
\pgfusepath{clip}%
\pgfsetbuttcap%
\pgfsetroundjoin%
\definecolor{currentfill}{rgb}{0.298039,0.447059,0.690196}%
\pgfsetfillcolor{currentfill}%
\pgfsetlinewidth{1.003750pt}%
\definecolor{currentstroke}{rgb}{0.298039,0.447059,0.690196}%
\pgfsetstrokecolor{currentstroke}%
\pgfsetdash{}{0pt}%
\pgfpathmoveto{\pgfqpoint{5.629478in}{1.796410in}}%
\pgfpathcurveto{\pgfqpoint{5.637715in}{1.796410in}}{\pgfqpoint{5.645615in}{1.799682in}}{\pgfqpoint{5.651439in}{1.805506in}}%
\pgfpathcurveto{\pgfqpoint{5.657263in}{1.811330in}}{\pgfqpoint{5.660535in}{1.819230in}}{\pgfqpoint{5.660535in}{1.827466in}}%
\pgfpathcurveto{\pgfqpoint{5.660535in}{1.835703in}}{\pgfqpoint{5.657263in}{1.843603in}}{\pgfqpoint{5.651439in}{1.849427in}}%
\pgfpathcurveto{\pgfqpoint{5.645615in}{1.855251in}}{\pgfqpoint{5.637715in}{1.858523in}}{\pgfqpoint{5.629478in}{1.858523in}}%
\pgfpathcurveto{\pgfqpoint{5.621242in}{1.858523in}}{\pgfqpoint{5.613342in}{1.855251in}}{\pgfqpoint{5.607518in}{1.849427in}}%
\pgfpathcurveto{\pgfqpoint{5.601694in}{1.843603in}}{\pgfqpoint{5.598422in}{1.835703in}}{\pgfqpoint{5.598422in}{1.827466in}}%
\pgfpathcurveto{\pgfqpoint{5.598422in}{1.819230in}}{\pgfqpoint{5.601694in}{1.811330in}}{\pgfqpoint{5.607518in}{1.805506in}}%
\pgfpathcurveto{\pgfqpoint{5.613342in}{1.799682in}}{\pgfqpoint{5.621242in}{1.796410in}}{\pgfqpoint{5.629478in}{1.796410in}}%
\pgfpathclose%
\pgfusepath{stroke,fill}%
\end{pgfscope}%
\begin{pgfscope}%
\pgfsetrectcap%
\pgfsetmiterjoin%
\pgfsetlinewidth{1.254687pt}%
\definecolor{currentstroke}{rgb}{1.000000,1.000000,1.000000}%
\pgfsetstrokecolor{currentstroke}%
\pgfsetdash{}{0pt}%
\pgfpathmoveto{\pgfqpoint{3.874179in}{0.557870in}}%
\pgfpathlineto{\pgfqpoint{3.874179in}{2.242604in}}%
\pgfusepath{stroke}%
\end{pgfscope}%
\begin{pgfscope}%
\pgfsetrectcap%
\pgfsetmiterjoin%
\pgfsetlinewidth{1.254687pt}%
\definecolor{currentstroke}{rgb}{1.000000,1.000000,1.000000}%
\pgfsetstrokecolor{currentstroke}%
\pgfsetdash{}{0pt}%
\pgfpathmoveto{\pgfqpoint{6.358287in}{0.557870in}}%
\pgfpathlineto{\pgfqpoint{6.358287in}{2.242604in}}%
\pgfusepath{stroke}%
\end{pgfscope}%
\begin{pgfscope}%
\pgfsetrectcap%
\pgfsetmiterjoin%
\pgfsetlinewidth{1.254687pt}%
\definecolor{currentstroke}{rgb}{1.000000,1.000000,1.000000}%
\pgfsetstrokecolor{currentstroke}%
\pgfsetdash{}{0pt}%
\pgfpathmoveto{\pgfqpoint{3.874179in}{0.557870in}}%
\pgfpathlineto{\pgfqpoint{6.358287in}{0.557870in}}%
\pgfusepath{stroke}%
\end{pgfscope}%
\begin{pgfscope}%
\pgfsetrectcap%
\pgfsetmiterjoin%
\pgfsetlinewidth{1.254687pt}%
\definecolor{currentstroke}{rgb}{1.000000,1.000000,1.000000}%
\pgfsetstrokecolor{currentstroke}%
\pgfsetdash{}{0pt}%
\pgfpathmoveto{\pgfqpoint{3.874179in}{2.242604in}}%
\pgfpathlineto{\pgfqpoint{6.358287in}{2.242604in}}%
\pgfusepath{stroke}%
\end{pgfscope}%
\begin{pgfscope}%
\definecolor{textcolor}{rgb}{0.150000,0.150000,0.150000}%
\pgfsetstrokecolor{textcolor}%
\pgfsetfillcolor{textcolor}%
\pgftext[x=5.116233in,y=2.325938in,,base]{\color{textcolor}\sffamily\fontsize{11.000000}{13.200000}\selectfont (b)}%
\end{pgfscope}%
\end{pgfpicture}%
\makeatother%
\endgroup%

    \caption{(a) Distribution plot of \acrshort{dor} of all PVSC models evaluated at two cluster centers when trained to predict heart failure.
             (b) Scatter plot of the same models sensitivity, and specificity.}
    \label{fig:pvmlc_hf_dor_sens_spec_dis}
\end{figure}

From the distribution plot depicted in figure \ref{fig:pvmlc_hf_dor_sens_spec_dis}a one can see that the PVSC models overall acheive relatively high \acrshort{dor}s, 
with a range of approximately two to nine.
The scatterplot in figure \ref{fig:pvmlc_hf_dor_sens_spec_dis}b shows that the models are quite concentrated in terms of sensitivity and specificity scores. 
The majority of the models achieve sensitivity, and specificity scores in the ranges $0.6$ to $0.75$, with some outliers acheiving specificity below $0.5$ and sensitivity above $0.75$.
What is even more concentrated are the accuracy scores of the models. 
As can be seen in table \ref{tab:pvmlc_hf_dor_sens_spec_dis}, the accuracy of top five PVSC models are all $0.75$
As with PVC all the best performing PVSC models use a combination of \acrshort{ef} and peak systolic strain values, 
and no specific \acrshort{ml} model seems to outperform the others on all the datasets in term of \acrshort{dor}.
The table also shows that the highest \acrshort{dor} of $9.4$ is acheived by model \textit{gls-EF/Gaissian-Process}. 
Although the \acrshort{dor}, sensitivity and specificity scores are very similar for the five best performing models \textit{gls-EF/Gaussian-Process} is chosen as the PVSC model that performs best at predicting heart failure as it acheives the highest \acrshort{dor}.


\begin{table*}
    \centering
    \ra{1.3}
    \begin{tabular}{lrrrr}
        \toprule
        Dataset-model           &  Accuracy &  Sensitivity &  Specificity &  \acrshort{dor} \\
        \midrule
        gls-EF/Gaussian-Process &      0.75 &         0.78 &         0.73 & 9.40 \\
        rls-EF/MLP              &      0.75 &         0.76 &         0.74 & 9.37 \\
        rls-EF/Linear-SVM       &      0.75 &         0.75 &         0.74 & 8.86 \\
        gls-EF/Ada-Boost        &      0.75 &         0.77 &         0.73 & 8.85 \\
        gls-EF/Naive-Bayes      &      0.75 &         0.76 &         0.74 & 8.79 \\
        \bottomrule
    \end{tabular}
    \caption{The accuracy, \acrshort{dor}, sensitivity and specicity scores of the five best performing PVSC in terms of \acrshort{dor}, at detecting heart failure. The \textbf{Dataset-model} column indicates \textit{Dataset used}$/$\textit{The specific \acrshort{ml} model used}.}
    \label{tab:pvmlc_hf_dor_sens_spec_dis}
\end{table*}

\clearpage
\subsection{Comparisons}

\begin{table*}
    \centering
    \ra{1.3}
    \begin{tabular}{lcccc}
        \toprule
        Dataset-model                           &  Accuracy &  Sensitivity &  Specificity &  \acrshort{dor} \\
        \midrule
        \textbf{TSC}-gls/2CH/regular/centroid/2 &      0.76 &         0.87 &         0.64 & 11.72 \\
        \textbf{PVC}-gls-EF/complete/2          &      0.76 &         0.81 &         0.72 & 10.85 \\
        \textbf{ANN}-gls/4CH/upsampled          &      0.54 &         0.46 &         0.61 & 1.36 \\
        \textbf{PVSC}-gls-EF/Gaussian-Process   &      0.75 &         0.78 &         0.73 & 9.40 \\
        \midrule
        Dataset-model                           &  TP &  TN &  FP &  FN \\
        \midrule
        \textbf{TSC}-gls/2CH/regular/centroid/2 &  86 &  62 &  35 &  13 \\
        \textbf{PVC}-gls-EF/complete/2          &  77 &  72 &  28 &  18 \\
        \textbf{ANN}-gls/4CH/upsampled          &  46 &  61 &  39 &  53 \\
        \textbf{PVSC}-gls-EF/Gaussian-Process   &  74 &  72 &  27 &  21 \\
        \bottomrule
    \end{tabular}
    \caption{A table comparing the best contenders within each model group for predicting heart failure among patients. The top table compare the models by their accuracy, sensitivity, specificity and \acrshort{dor}, and the bottom table shows the number of TPs, TNs, FPs and FNs that the different models attain.}
    \label{tab:hf_compare}
\end{table*}

With exeption of the \acrshort{ann}, the models performance of the different models are very close in terms of \acrshort{dor} and accuracy. From table \ref{tab:hf_compare} one can see that the\acrshort{tsc} model \textit{gls/2CH/regular/centroid/2} achieves the highest sensitivity of all the models applied to predict heart failure, but it achieves the second lowest specificity of the four model groups. This can be confirmed by the fact that it attains 86 TPs, and 35 FPs. The PVSC model \textit{gls-EF/Gaussian-Process} attains the most balanced score in terms of sensitivity and specificity, and the highest specificity score of all the model groups. However, the PVC model \textit{gls-EF/complete/2} attains a higher accuracy, sensitivity and \acrshort{dor} than the PVSC model. One can also see that the PVC model attains more TP, the same number of TN, fewer FP and fewer FN than the PVSC model. It should also be noted that the PVC model and the PVSC model are using the same dataset which is a combination of peak systolic \acrshort{gls} values, and \acrshort{ef}. To conclude this particular case study, the PVC model is picked as the best model at predicting heart failure among patients as it achieves the highest accuracy of the model groups, highest number of TN, and one of the most balanced combinations of sensitivity, and specificity. Recall the scores of the simple threshold classifier using \acrshort{ef}, and a lower threshold of $45\%$ mentioned in section \ref{sec:target}: Accuracy of 0.77, sensitivity of 0.86, specificity of 0.69 and \acrshort{dor} of 13.48. The \acrshort{ef} threshold classifier perfoms best in terms of overall accuracy and \acrshort{dor}, but is outperformed by the best \acrshort{tsc} model in terms of sensitivity, and the best \acrshort{pvc} and \acrshort{pvsc} models in terms of specificity. Since the \acrshort{ef} threshold classifier attains the highest accuracy and \acrshort{dor}, a sensitivity that is only $1\%$ below the best sensitivity score, and specificity that is only $3\%$ lower than the highest specificity score, it is arguably better than all the models. This speaks volumes about the underperformance of the models, when applied to predict heart failure, especially the \acrshort{pvc}, and \acrshort{pvsc} models that use \acrshort{ef} as an input parameter. 

\newpage


\section{Case Study: Patient Diagnosis}

\subsection{Time-series Clustering}

\begin{figure}[H]
    \centering
    % \includegraphics[width=\textwidth]{results/tsc_ind_dor_sens_spec_dist.png}
    %% Creator: Matplotlib, PGF backend
%%
%% To include the figure in your LaTeX document, write
%%   \input{<filename>.pgf}
%%
%% Make sure the required packages are loaded in your preamble
%%   \usepackage{pgf}
%%
%% Figures using additional raster images can only be included by \input if
%% they are in the same directory as the main LaTeX file. For loading figures
%% from other directories you can use the `import` package
%%   \usepackage{import}
%% and then include the figures with
%%   \import{<path to file>}{<filename>.pgf}
%%
%% Matplotlib used the following preamble
%%
\begingroup%
\makeatletter%
\begin{pgfpicture}%
\pgfpathrectangle{\pgfpointorigin}{\pgfqpoint{6.364000in}{2.540000in}}%
\pgfusepath{use as bounding box, clip}%
\begin{pgfscope}%
\pgfsetbuttcap%
\pgfsetmiterjoin%
\definecolor{currentfill}{rgb}{1.000000,1.000000,1.000000}%
\pgfsetfillcolor{currentfill}%
\pgfsetlinewidth{0.000000pt}%
\definecolor{currentstroke}{rgb}{1.000000,1.000000,1.000000}%
\pgfsetstrokecolor{currentstroke}%
\pgfsetdash{}{0pt}%
\pgfpathmoveto{\pgfqpoint{0.000000in}{0.000000in}}%
\pgfpathlineto{\pgfqpoint{6.364000in}{0.000000in}}%
\pgfpathlineto{\pgfqpoint{6.364000in}{2.540000in}}%
\pgfpathlineto{\pgfqpoint{0.000000in}{2.540000in}}%
\pgfpathclose%
\pgfusepath{fill}%
\end{pgfscope}%
\begin{pgfscope}%
\pgfsetbuttcap%
\pgfsetmiterjoin%
\definecolor{currentfill}{rgb}{0.917647,0.917647,0.949020}%
\pgfsetfillcolor{currentfill}%
\pgfsetlinewidth{0.000000pt}%
\definecolor{currentstroke}{rgb}{0.000000,0.000000,0.000000}%
\pgfsetstrokecolor{currentstroke}%
\pgfsetstrokeopacity{0.000000}%
\pgfsetdash{}{0pt}%
\pgfpathmoveto{\pgfqpoint{0.650810in}{0.557870in}}%
\pgfpathlineto{\pgfqpoint{3.096898in}{0.557870in}}%
\pgfpathlineto{\pgfqpoint{3.096898in}{2.242604in}}%
\pgfpathlineto{\pgfqpoint{0.650810in}{2.242604in}}%
\pgfpathclose%
\pgfusepath{fill}%
\end{pgfscope}%
\begin{pgfscope}%
\pgfpathrectangle{\pgfqpoint{0.650810in}{0.557870in}}{\pgfqpoint{2.446088in}{1.684734in}}%
\pgfusepath{clip}%
\pgfsetroundcap%
\pgfsetroundjoin%
\pgfsetlinewidth{1.003750pt}%
\definecolor{currentstroke}{rgb}{1.000000,1.000000,1.000000}%
\pgfsetstrokecolor{currentstroke}%
\pgfsetdash{}{0pt}%
\pgfpathmoveto{\pgfqpoint{0.761996in}{0.557870in}}%
\pgfpathlineto{\pgfqpoint{0.761996in}{2.242604in}}%
\pgfusepath{stroke}%
\end{pgfscope}%
\begin{pgfscope}%
\definecolor{textcolor}{rgb}{0.150000,0.150000,0.150000}%
\pgfsetstrokecolor{textcolor}%
\pgfsetfillcolor{textcolor}%
\pgftext[x=0.761996in,y=0.425926in,,top]{\color{textcolor}\sffamily\fontsize{11.000000}{13.200000}\selectfont \(\displaystyle 0\)}%
\end{pgfscope}%
\begin{pgfscope}%
\pgfpathrectangle{\pgfqpoint{0.650810in}{0.557870in}}{\pgfqpoint{2.446088in}{1.684734in}}%
\pgfusepath{clip}%
\pgfsetroundcap%
\pgfsetroundjoin%
\pgfsetlinewidth{1.003750pt}%
\definecolor{currentstroke}{rgb}{1.000000,1.000000,1.000000}%
\pgfsetstrokecolor{currentstroke}%
\pgfsetdash{}{0pt}%
\pgfpathmoveto{\pgfqpoint{1.426412in}{0.557870in}}%
\pgfpathlineto{\pgfqpoint{1.426412in}{2.242604in}}%
\pgfusepath{stroke}%
\end{pgfscope}%
\begin{pgfscope}%
\definecolor{textcolor}{rgb}{0.150000,0.150000,0.150000}%
\pgfsetstrokecolor{textcolor}%
\pgfsetfillcolor{textcolor}%
\pgftext[x=1.426412in,y=0.425926in,,top]{\color{textcolor}\sffamily\fontsize{11.000000}{13.200000}\selectfont \(\displaystyle 10\)}%
\end{pgfscope}%
\begin{pgfscope}%
\pgfpathrectangle{\pgfqpoint{0.650810in}{0.557870in}}{\pgfqpoint{2.446088in}{1.684734in}}%
\pgfusepath{clip}%
\pgfsetroundcap%
\pgfsetroundjoin%
\pgfsetlinewidth{1.003750pt}%
\definecolor{currentstroke}{rgb}{1.000000,1.000000,1.000000}%
\pgfsetstrokecolor{currentstroke}%
\pgfsetdash{}{0pt}%
\pgfpathmoveto{\pgfqpoint{2.090828in}{0.557870in}}%
\pgfpathlineto{\pgfqpoint{2.090828in}{2.242604in}}%
\pgfusepath{stroke}%
\end{pgfscope}%
\begin{pgfscope}%
\definecolor{textcolor}{rgb}{0.150000,0.150000,0.150000}%
\pgfsetstrokecolor{textcolor}%
\pgfsetfillcolor{textcolor}%
\pgftext[x=2.090828in,y=0.425926in,,top]{\color{textcolor}\sffamily\fontsize{11.000000}{13.200000}\selectfont \(\displaystyle 20\)}%
\end{pgfscope}%
\begin{pgfscope}%
\pgfpathrectangle{\pgfqpoint{0.650810in}{0.557870in}}{\pgfqpoint{2.446088in}{1.684734in}}%
\pgfusepath{clip}%
\pgfsetroundcap%
\pgfsetroundjoin%
\pgfsetlinewidth{1.003750pt}%
\definecolor{currentstroke}{rgb}{1.000000,1.000000,1.000000}%
\pgfsetstrokecolor{currentstroke}%
\pgfsetdash{}{0pt}%
\pgfpathmoveto{\pgfqpoint{2.755243in}{0.557870in}}%
\pgfpathlineto{\pgfqpoint{2.755243in}{2.242604in}}%
\pgfusepath{stroke}%
\end{pgfscope}%
\begin{pgfscope}%
\definecolor{textcolor}{rgb}{0.150000,0.150000,0.150000}%
\pgfsetstrokecolor{textcolor}%
\pgfsetfillcolor{textcolor}%
\pgftext[x=2.755243in,y=0.425926in,,top]{\color{textcolor}\sffamily\fontsize{11.000000}{13.200000}\selectfont \(\displaystyle 30\)}%
\end{pgfscope}%
\begin{pgfscope}%
\definecolor{textcolor}{rgb}{0.150000,0.150000,0.150000}%
\pgfsetstrokecolor{textcolor}%
\pgfsetfillcolor{textcolor}%
\pgftext[x=1.873854in,y=0.235185in,,top]{\color{textcolor}\sffamily\fontsize{11.000000}{13.200000}\selectfont DOR}%
\end{pgfscope}%
\begin{pgfscope}%
\pgfpathrectangle{\pgfqpoint{0.650810in}{0.557870in}}{\pgfqpoint{2.446088in}{1.684734in}}%
\pgfusepath{clip}%
\pgfsetroundcap%
\pgfsetroundjoin%
\pgfsetlinewidth{1.003750pt}%
\definecolor{currentstroke}{rgb}{1.000000,1.000000,1.000000}%
\pgfsetstrokecolor{currentstroke}%
\pgfsetdash{}{0pt}%
\pgfpathmoveto{\pgfqpoint{0.650810in}{0.557870in}}%
\pgfpathlineto{\pgfqpoint{3.096898in}{0.557870in}}%
\pgfusepath{stroke}%
\end{pgfscope}%
\begin{pgfscope}%
\definecolor{textcolor}{rgb}{0.150000,0.150000,0.150000}%
\pgfsetstrokecolor{textcolor}%
\pgfsetfillcolor{textcolor}%
\pgftext[x=0.442824in,y=0.505064in,left,base]{\color{textcolor}\sffamily\fontsize{11.000000}{13.200000}\selectfont \(\displaystyle 0\)}%
\end{pgfscope}%
\begin{pgfscope}%
\pgfpathrectangle{\pgfqpoint{0.650810in}{0.557870in}}{\pgfqpoint{2.446088in}{1.684734in}}%
\pgfusepath{clip}%
\pgfsetroundcap%
\pgfsetroundjoin%
\pgfsetlinewidth{1.003750pt}%
\definecolor{currentstroke}{rgb}{1.000000,1.000000,1.000000}%
\pgfsetstrokecolor{currentstroke}%
\pgfsetdash{}{0pt}%
\pgfpathmoveto{\pgfqpoint{0.650810in}{0.934515in}}%
\pgfpathlineto{\pgfqpoint{3.096898in}{0.934515in}}%
\pgfusepath{stroke}%
\end{pgfscope}%
\begin{pgfscope}%
\definecolor{textcolor}{rgb}{0.150000,0.150000,0.150000}%
\pgfsetstrokecolor{textcolor}%
\pgfsetfillcolor{textcolor}%
\pgftext[x=0.366783in,y=0.881709in,left,base]{\color{textcolor}\sffamily\fontsize{11.000000}{13.200000}\selectfont \(\displaystyle 50\)}%
\end{pgfscope}%
\begin{pgfscope}%
\pgfpathrectangle{\pgfqpoint{0.650810in}{0.557870in}}{\pgfqpoint{2.446088in}{1.684734in}}%
\pgfusepath{clip}%
\pgfsetroundcap%
\pgfsetroundjoin%
\pgfsetlinewidth{1.003750pt}%
\definecolor{currentstroke}{rgb}{1.000000,1.000000,1.000000}%
\pgfsetstrokecolor{currentstroke}%
\pgfsetdash{}{0pt}%
\pgfpathmoveto{\pgfqpoint{0.650810in}{1.311161in}}%
\pgfpathlineto{\pgfqpoint{3.096898in}{1.311161in}}%
\pgfusepath{stroke}%
\end{pgfscope}%
\begin{pgfscope}%
\definecolor{textcolor}{rgb}{0.150000,0.150000,0.150000}%
\pgfsetstrokecolor{textcolor}%
\pgfsetfillcolor{textcolor}%
\pgftext[x=0.290741in,y=1.258354in,left,base]{\color{textcolor}\sffamily\fontsize{11.000000}{13.200000}\selectfont \(\displaystyle 100\)}%
\end{pgfscope}%
\begin{pgfscope}%
\pgfpathrectangle{\pgfqpoint{0.650810in}{0.557870in}}{\pgfqpoint{2.446088in}{1.684734in}}%
\pgfusepath{clip}%
\pgfsetroundcap%
\pgfsetroundjoin%
\pgfsetlinewidth{1.003750pt}%
\definecolor{currentstroke}{rgb}{1.000000,1.000000,1.000000}%
\pgfsetstrokecolor{currentstroke}%
\pgfsetdash{}{0pt}%
\pgfpathmoveto{\pgfqpoint{0.650810in}{1.687806in}}%
\pgfpathlineto{\pgfqpoint{3.096898in}{1.687806in}}%
\pgfusepath{stroke}%
\end{pgfscope}%
\begin{pgfscope}%
\definecolor{textcolor}{rgb}{0.150000,0.150000,0.150000}%
\pgfsetstrokecolor{textcolor}%
\pgfsetfillcolor{textcolor}%
\pgftext[x=0.290741in,y=1.634999in,left,base]{\color{textcolor}\sffamily\fontsize{11.000000}{13.200000}\selectfont \(\displaystyle 150\)}%
\end{pgfscope}%
\begin{pgfscope}%
\pgfpathrectangle{\pgfqpoint{0.650810in}{0.557870in}}{\pgfqpoint{2.446088in}{1.684734in}}%
\pgfusepath{clip}%
\pgfsetroundcap%
\pgfsetroundjoin%
\pgfsetlinewidth{1.003750pt}%
\definecolor{currentstroke}{rgb}{1.000000,1.000000,1.000000}%
\pgfsetstrokecolor{currentstroke}%
\pgfsetdash{}{0pt}%
\pgfpathmoveto{\pgfqpoint{0.650810in}{2.064451in}}%
\pgfpathlineto{\pgfqpoint{3.096898in}{2.064451in}}%
\pgfusepath{stroke}%
\end{pgfscope}%
\begin{pgfscope}%
\definecolor{textcolor}{rgb}{0.150000,0.150000,0.150000}%
\pgfsetstrokecolor{textcolor}%
\pgfsetfillcolor{textcolor}%
\pgftext[x=0.290741in,y=2.011644in,left,base]{\color{textcolor}\sffamily\fontsize{11.000000}{13.200000}\selectfont \(\displaystyle 200\)}%
\end{pgfscope}%
\begin{pgfscope}%
\definecolor{textcolor}{rgb}{0.150000,0.150000,0.150000}%
\pgfsetstrokecolor{textcolor}%
\pgfsetfillcolor{textcolor}%
\pgftext[x=0.235185in,y=1.400237in,,bottom,rotate=90.000000]{\color{textcolor}\sffamily\fontsize{11.000000}{13.200000}\selectfont Occurance}%
\end{pgfscope}%
\begin{pgfscope}%
\pgfpathrectangle{\pgfqpoint{0.650810in}{0.557870in}}{\pgfqpoint{2.446088in}{1.684734in}}%
\pgfusepath{clip}%
\pgfsetbuttcap%
\pgfsetmiterjoin%
\definecolor{currentfill}{rgb}{0.298039,0.447059,0.690196}%
\pgfsetfillcolor{currentfill}%
\pgfsetfillopacity{0.400000}%
\pgfsetlinewidth{1.003750pt}%
\definecolor{currentstroke}{rgb}{1.000000,1.000000,1.000000}%
\pgfsetstrokecolor{currentstroke}%
\pgfsetstrokeopacity{0.400000}%
\pgfsetdash{}{0pt}%
\pgfpathmoveto{\pgfqpoint{0.761996in}{0.557870in}}%
\pgfpathlineto{\pgfqpoint{0.984368in}{0.557870in}}%
\pgfpathlineto{\pgfqpoint{0.984368in}{2.162379in}}%
\pgfpathlineto{\pgfqpoint{0.761996in}{2.162379in}}%
\pgfpathclose%
\pgfusepath{stroke,fill}%
\end{pgfscope}%
\begin{pgfscope}%
\pgfpathrectangle{\pgfqpoint{0.650810in}{0.557870in}}{\pgfqpoint{2.446088in}{1.684734in}}%
\pgfusepath{clip}%
\pgfsetbuttcap%
\pgfsetmiterjoin%
\definecolor{currentfill}{rgb}{0.298039,0.447059,0.690196}%
\pgfsetfillcolor{currentfill}%
\pgfsetfillopacity{0.400000}%
\pgfsetlinewidth{1.003750pt}%
\definecolor{currentstroke}{rgb}{1.000000,1.000000,1.000000}%
\pgfsetstrokecolor{currentstroke}%
\pgfsetstrokeopacity{0.400000}%
\pgfsetdash{}{0pt}%
\pgfpathmoveto{\pgfqpoint{0.984368in}{0.557870in}}%
\pgfpathlineto{\pgfqpoint{1.206739in}{0.557870in}}%
\pgfpathlineto{\pgfqpoint{1.206739in}{0.610601in}}%
\pgfpathlineto{\pgfqpoint{0.984368in}{0.610601in}}%
\pgfpathclose%
\pgfusepath{stroke,fill}%
\end{pgfscope}%
\begin{pgfscope}%
\pgfpathrectangle{\pgfqpoint{0.650810in}{0.557870in}}{\pgfqpoint{2.446088in}{1.684734in}}%
\pgfusepath{clip}%
\pgfsetbuttcap%
\pgfsetmiterjoin%
\definecolor{currentfill}{rgb}{0.298039,0.447059,0.690196}%
\pgfsetfillcolor{currentfill}%
\pgfsetfillopacity{0.400000}%
\pgfsetlinewidth{1.003750pt}%
\definecolor{currentstroke}{rgb}{1.000000,1.000000,1.000000}%
\pgfsetstrokecolor{currentstroke}%
\pgfsetstrokeopacity{0.400000}%
\pgfsetdash{}{0pt}%
\pgfpathmoveto{\pgfqpoint{1.206739in}{0.557870in}}%
\pgfpathlineto{\pgfqpoint{1.429111in}{0.557870in}}%
\pgfpathlineto{\pgfqpoint{1.429111in}{0.588002in}}%
\pgfpathlineto{\pgfqpoint{1.206739in}{0.588002in}}%
\pgfpathclose%
\pgfusepath{stroke,fill}%
\end{pgfscope}%
\begin{pgfscope}%
\pgfpathrectangle{\pgfqpoint{0.650810in}{0.557870in}}{\pgfqpoint{2.446088in}{1.684734in}}%
\pgfusepath{clip}%
\pgfsetbuttcap%
\pgfsetmiterjoin%
\definecolor{currentfill}{rgb}{0.298039,0.447059,0.690196}%
\pgfsetfillcolor{currentfill}%
\pgfsetfillopacity{0.400000}%
\pgfsetlinewidth{1.003750pt}%
\definecolor{currentstroke}{rgb}{1.000000,1.000000,1.000000}%
\pgfsetstrokecolor{currentstroke}%
\pgfsetstrokeopacity{0.400000}%
\pgfsetdash{}{0pt}%
\pgfpathmoveto{\pgfqpoint{1.429111in}{0.557870in}}%
\pgfpathlineto{\pgfqpoint{1.651483in}{0.557870in}}%
\pgfpathlineto{\pgfqpoint{1.651483in}{0.866719in}}%
\pgfpathlineto{\pgfqpoint{1.429111in}{0.866719in}}%
\pgfpathclose%
\pgfusepath{stroke,fill}%
\end{pgfscope}%
\begin{pgfscope}%
\pgfpathrectangle{\pgfqpoint{0.650810in}{0.557870in}}{\pgfqpoint{2.446088in}{1.684734in}}%
\pgfusepath{clip}%
\pgfsetbuttcap%
\pgfsetmiterjoin%
\definecolor{currentfill}{rgb}{0.298039,0.447059,0.690196}%
\pgfsetfillcolor{currentfill}%
\pgfsetfillopacity{0.400000}%
\pgfsetlinewidth{1.003750pt}%
\definecolor{currentstroke}{rgb}{1.000000,1.000000,1.000000}%
\pgfsetstrokecolor{currentstroke}%
\pgfsetstrokeopacity{0.400000}%
\pgfsetdash{}{0pt}%
\pgfpathmoveto{\pgfqpoint{1.651483in}{0.557870in}}%
\pgfpathlineto{\pgfqpoint{1.873854in}{0.557870in}}%
\pgfpathlineto{\pgfqpoint{1.873854in}{0.685930in}}%
\pgfpathlineto{\pgfqpoint{1.651483in}{0.685930in}}%
\pgfpathclose%
\pgfusepath{stroke,fill}%
\end{pgfscope}%
\begin{pgfscope}%
\pgfpathrectangle{\pgfqpoint{0.650810in}{0.557870in}}{\pgfqpoint{2.446088in}{1.684734in}}%
\pgfusepath{clip}%
\pgfsetbuttcap%
\pgfsetmiterjoin%
\definecolor{currentfill}{rgb}{0.298039,0.447059,0.690196}%
\pgfsetfillcolor{currentfill}%
\pgfsetfillopacity{0.400000}%
\pgfsetlinewidth{1.003750pt}%
\definecolor{currentstroke}{rgb}{1.000000,1.000000,1.000000}%
\pgfsetstrokecolor{currentstroke}%
\pgfsetstrokeopacity{0.400000}%
\pgfsetdash{}{0pt}%
\pgfpathmoveto{\pgfqpoint{1.873854in}{0.557870in}}%
\pgfpathlineto{\pgfqpoint{2.096226in}{0.557870in}}%
\pgfpathlineto{\pgfqpoint{2.096226in}{0.580469in}}%
\pgfpathlineto{\pgfqpoint{1.873854in}{0.580469in}}%
\pgfpathclose%
\pgfusepath{stroke,fill}%
\end{pgfscope}%
\begin{pgfscope}%
\pgfpathrectangle{\pgfqpoint{0.650810in}{0.557870in}}{\pgfqpoint{2.446088in}{1.684734in}}%
\pgfusepath{clip}%
\pgfsetbuttcap%
\pgfsetmiterjoin%
\definecolor{currentfill}{rgb}{0.298039,0.447059,0.690196}%
\pgfsetfillcolor{currentfill}%
\pgfsetfillopacity{0.400000}%
\pgfsetlinewidth{1.003750pt}%
\definecolor{currentstroke}{rgb}{1.000000,1.000000,1.000000}%
\pgfsetstrokecolor{currentstroke}%
\pgfsetstrokeopacity{0.400000}%
\pgfsetdash{}{0pt}%
\pgfpathmoveto{\pgfqpoint{2.096226in}{0.557870in}}%
\pgfpathlineto{\pgfqpoint{2.318598in}{0.557870in}}%
\pgfpathlineto{\pgfqpoint{2.318598in}{0.618134in}}%
\pgfpathlineto{\pgfqpoint{2.096226in}{0.618134in}}%
\pgfpathclose%
\pgfusepath{stroke,fill}%
\end{pgfscope}%
\begin{pgfscope}%
\pgfpathrectangle{\pgfqpoint{0.650810in}{0.557870in}}{\pgfqpoint{2.446088in}{1.684734in}}%
\pgfusepath{clip}%
\pgfsetbuttcap%
\pgfsetmiterjoin%
\definecolor{currentfill}{rgb}{0.298039,0.447059,0.690196}%
\pgfsetfillcolor{currentfill}%
\pgfsetfillopacity{0.400000}%
\pgfsetlinewidth{1.003750pt}%
\definecolor{currentstroke}{rgb}{1.000000,1.000000,1.000000}%
\pgfsetstrokecolor{currentstroke}%
\pgfsetstrokeopacity{0.400000}%
\pgfsetdash{}{0pt}%
\pgfpathmoveto{\pgfqpoint{2.318598in}{0.557870in}}%
\pgfpathlineto{\pgfqpoint{2.540969in}{0.557870in}}%
\pgfpathlineto{\pgfqpoint{2.540969in}{0.588002in}}%
\pgfpathlineto{\pgfqpoint{2.318598in}{0.588002in}}%
\pgfpathclose%
\pgfusepath{stroke,fill}%
\end{pgfscope}%
\begin{pgfscope}%
\pgfpathrectangle{\pgfqpoint{0.650810in}{0.557870in}}{\pgfqpoint{2.446088in}{1.684734in}}%
\pgfusepath{clip}%
\pgfsetbuttcap%
\pgfsetmiterjoin%
\definecolor{currentfill}{rgb}{0.298039,0.447059,0.690196}%
\pgfsetfillcolor{currentfill}%
\pgfsetfillopacity{0.400000}%
\pgfsetlinewidth{1.003750pt}%
\definecolor{currentstroke}{rgb}{1.000000,1.000000,1.000000}%
\pgfsetstrokecolor{currentstroke}%
\pgfsetstrokeopacity{0.400000}%
\pgfsetdash{}{0pt}%
\pgfpathmoveto{\pgfqpoint{2.540969in}{0.557870in}}%
\pgfpathlineto{\pgfqpoint{2.763341in}{0.557870in}}%
\pgfpathlineto{\pgfqpoint{2.763341in}{0.572936in}}%
\pgfpathlineto{\pgfqpoint{2.540969in}{0.572936in}}%
\pgfpathclose%
\pgfusepath{stroke,fill}%
\end{pgfscope}%
\begin{pgfscope}%
\pgfpathrectangle{\pgfqpoint{0.650810in}{0.557870in}}{\pgfqpoint{2.446088in}{1.684734in}}%
\pgfusepath{clip}%
\pgfsetbuttcap%
\pgfsetmiterjoin%
\definecolor{currentfill}{rgb}{0.298039,0.447059,0.690196}%
\pgfsetfillcolor{currentfill}%
\pgfsetfillopacity{0.400000}%
\pgfsetlinewidth{1.003750pt}%
\definecolor{currentstroke}{rgb}{1.000000,1.000000,1.000000}%
\pgfsetstrokecolor{currentstroke}%
\pgfsetstrokeopacity{0.400000}%
\pgfsetdash{}{0pt}%
\pgfpathmoveto{\pgfqpoint{2.763341in}{0.557870in}}%
\pgfpathlineto{\pgfqpoint{2.985712in}{0.557870in}}%
\pgfpathlineto{\pgfqpoint{2.985712in}{0.588002in}}%
\pgfpathlineto{\pgfqpoint{2.763341in}{0.588002in}}%
\pgfpathclose%
\pgfusepath{stroke,fill}%
\end{pgfscope}%
\begin{pgfscope}%
\pgfsetrectcap%
\pgfsetmiterjoin%
\pgfsetlinewidth{1.254687pt}%
\definecolor{currentstroke}{rgb}{1.000000,1.000000,1.000000}%
\pgfsetstrokecolor{currentstroke}%
\pgfsetdash{}{0pt}%
\pgfpathmoveto{\pgfqpoint{0.650810in}{0.557870in}}%
\pgfpathlineto{\pgfqpoint{0.650810in}{2.242604in}}%
\pgfusepath{stroke}%
\end{pgfscope}%
\begin{pgfscope}%
\pgfsetrectcap%
\pgfsetmiterjoin%
\pgfsetlinewidth{1.254687pt}%
\definecolor{currentstroke}{rgb}{1.000000,1.000000,1.000000}%
\pgfsetstrokecolor{currentstroke}%
\pgfsetdash{}{0pt}%
\pgfpathmoveto{\pgfqpoint{3.096898in}{0.557870in}}%
\pgfpathlineto{\pgfqpoint{3.096898in}{2.242604in}}%
\pgfusepath{stroke}%
\end{pgfscope}%
\begin{pgfscope}%
\pgfsetrectcap%
\pgfsetmiterjoin%
\pgfsetlinewidth{1.254687pt}%
\definecolor{currentstroke}{rgb}{1.000000,1.000000,1.000000}%
\pgfsetstrokecolor{currentstroke}%
\pgfsetdash{}{0pt}%
\pgfpathmoveto{\pgfqpoint{0.650810in}{0.557870in}}%
\pgfpathlineto{\pgfqpoint{3.096898in}{0.557870in}}%
\pgfusepath{stroke}%
\end{pgfscope}%
\begin{pgfscope}%
\pgfsetrectcap%
\pgfsetmiterjoin%
\pgfsetlinewidth{1.254687pt}%
\definecolor{currentstroke}{rgb}{1.000000,1.000000,1.000000}%
\pgfsetstrokecolor{currentstroke}%
\pgfsetdash{}{0pt}%
\pgfpathmoveto{\pgfqpoint{0.650810in}{2.242604in}}%
\pgfpathlineto{\pgfqpoint{3.096898in}{2.242604in}}%
\pgfusepath{stroke}%
\end{pgfscope}%
\begin{pgfscope}%
\definecolor{textcolor}{rgb}{0.150000,0.150000,0.150000}%
\pgfsetstrokecolor{textcolor}%
\pgfsetfillcolor{textcolor}%
\pgftext[x=1.873854in,y=2.325938in,,base]{\color{textcolor}\sffamily\fontsize{11.000000}{13.200000}\selectfont (a)}%
\end{pgfscope}%
\begin{pgfscope}%
\pgfsetbuttcap%
\pgfsetmiterjoin%
\definecolor{currentfill}{rgb}{0.917647,0.917647,0.949020}%
\pgfsetfillcolor{currentfill}%
\pgfsetlinewidth{0.000000pt}%
\definecolor{currentstroke}{rgb}{0.000000,0.000000,0.000000}%
\pgfsetstrokecolor{currentstroke}%
\pgfsetstrokeopacity{0.000000}%
\pgfsetdash{}{0pt}%
\pgfpathmoveto{\pgfqpoint{3.793912in}{0.557870in}}%
\pgfpathlineto{\pgfqpoint{6.240000in}{0.557870in}}%
\pgfpathlineto{\pgfqpoint{6.240000in}{2.242604in}}%
\pgfpathlineto{\pgfqpoint{3.793912in}{2.242604in}}%
\pgfpathclose%
\pgfusepath{fill}%
\end{pgfscope}%
\begin{pgfscope}%
\pgfpathrectangle{\pgfqpoint{3.793912in}{0.557870in}}{\pgfqpoint{2.446088in}{1.684734in}}%
\pgfusepath{clip}%
\pgfsetroundcap%
\pgfsetroundjoin%
\pgfsetlinewidth{1.003750pt}%
\definecolor{currentstroke}{rgb}{1.000000,1.000000,1.000000}%
\pgfsetstrokecolor{currentstroke}%
\pgfsetdash{}{0pt}%
\pgfpathmoveto{\pgfqpoint{3.905098in}{0.557870in}}%
\pgfpathlineto{\pgfqpoint{3.905098in}{2.242604in}}%
\pgfusepath{stroke}%
\end{pgfscope}%
\begin{pgfscope}%
\definecolor{textcolor}{rgb}{0.150000,0.150000,0.150000}%
\pgfsetstrokecolor{textcolor}%
\pgfsetfillcolor{textcolor}%
\pgftext[x=3.905098in,y=0.425926in,,top]{\color{textcolor}\sffamily\fontsize{11.000000}{13.200000}\selectfont \(\displaystyle 0.00\)}%
\end{pgfscope}%
\begin{pgfscope}%
\pgfpathrectangle{\pgfqpoint{3.793912in}{0.557870in}}{\pgfqpoint{2.446088in}{1.684734in}}%
\pgfusepath{clip}%
\pgfsetroundcap%
\pgfsetroundjoin%
\pgfsetlinewidth{1.003750pt}%
\definecolor{currentstroke}{rgb}{1.000000,1.000000,1.000000}%
\pgfsetstrokecolor{currentstroke}%
\pgfsetdash{}{0pt}%
\pgfpathmoveto{\pgfqpoint{4.461027in}{0.557870in}}%
\pgfpathlineto{\pgfqpoint{4.461027in}{2.242604in}}%
\pgfusepath{stroke}%
\end{pgfscope}%
\begin{pgfscope}%
\definecolor{textcolor}{rgb}{0.150000,0.150000,0.150000}%
\pgfsetstrokecolor{textcolor}%
\pgfsetfillcolor{textcolor}%
\pgftext[x=4.461027in,y=0.425926in,,top]{\color{textcolor}\sffamily\fontsize{11.000000}{13.200000}\selectfont \(\displaystyle 0.25\)}%
\end{pgfscope}%
\begin{pgfscope}%
\pgfpathrectangle{\pgfqpoint{3.793912in}{0.557870in}}{\pgfqpoint{2.446088in}{1.684734in}}%
\pgfusepath{clip}%
\pgfsetroundcap%
\pgfsetroundjoin%
\pgfsetlinewidth{1.003750pt}%
\definecolor{currentstroke}{rgb}{1.000000,1.000000,1.000000}%
\pgfsetstrokecolor{currentstroke}%
\pgfsetdash{}{0pt}%
\pgfpathmoveto{\pgfqpoint{5.016956in}{0.557870in}}%
\pgfpathlineto{\pgfqpoint{5.016956in}{2.242604in}}%
\pgfusepath{stroke}%
\end{pgfscope}%
\begin{pgfscope}%
\definecolor{textcolor}{rgb}{0.150000,0.150000,0.150000}%
\pgfsetstrokecolor{textcolor}%
\pgfsetfillcolor{textcolor}%
\pgftext[x=5.016956in,y=0.425926in,,top]{\color{textcolor}\sffamily\fontsize{11.000000}{13.200000}\selectfont \(\displaystyle 0.50\)}%
\end{pgfscope}%
\begin{pgfscope}%
\pgfpathrectangle{\pgfqpoint{3.793912in}{0.557870in}}{\pgfqpoint{2.446088in}{1.684734in}}%
\pgfusepath{clip}%
\pgfsetroundcap%
\pgfsetroundjoin%
\pgfsetlinewidth{1.003750pt}%
\definecolor{currentstroke}{rgb}{1.000000,1.000000,1.000000}%
\pgfsetstrokecolor{currentstroke}%
\pgfsetdash{}{0pt}%
\pgfpathmoveto{\pgfqpoint{5.572885in}{0.557870in}}%
\pgfpathlineto{\pgfqpoint{5.572885in}{2.242604in}}%
\pgfusepath{stroke}%
\end{pgfscope}%
\begin{pgfscope}%
\definecolor{textcolor}{rgb}{0.150000,0.150000,0.150000}%
\pgfsetstrokecolor{textcolor}%
\pgfsetfillcolor{textcolor}%
\pgftext[x=5.572885in,y=0.425926in,,top]{\color{textcolor}\sffamily\fontsize{11.000000}{13.200000}\selectfont \(\displaystyle 0.75\)}%
\end{pgfscope}%
\begin{pgfscope}%
\pgfpathrectangle{\pgfqpoint{3.793912in}{0.557870in}}{\pgfqpoint{2.446088in}{1.684734in}}%
\pgfusepath{clip}%
\pgfsetroundcap%
\pgfsetroundjoin%
\pgfsetlinewidth{1.003750pt}%
\definecolor{currentstroke}{rgb}{1.000000,1.000000,1.000000}%
\pgfsetstrokecolor{currentstroke}%
\pgfsetdash{}{0pt}%
\pgfpathmoveto{\pgfqpoint{6.128814in}{0.557870in}}%
\pgfpathlineto{\pgfqpoint{6.128814in}{2.242604in}}%
\pgfusepath{stroke}%
\end{pgfscope}%
\begin{pgfscope}%
\definecolor{textcolor}{rgb}{0.150000,0.150000,0.150000}%
\pgfsetstrokecolor{textcolor}%
\pgfsetfillcolor{textcolor}%
\pgftext[x=6.128814in,y=0.425926in,,top]{\color{textcolor}\sffamily\fontsize{11.000000}{13.200000}\selectfont \(\displaystyle 1.00\)}%
\end{pgfscope}%
\begin{pgfscope}%
\definecolor{textcolor}{rgb}{0.150000,0.150000,0.150000}%
\pgfsetstrokecolor{textcolor}%
\pgfsetfillcolor{textcolor}%
\pgftext[x=5.016956in,y=0.235185in,,top]{\color{textcolor}\sffamily\fontsize{11.000000}{13.200000}\selectfont Specificity}%
\end{pgfscope}%
\begin{pgfscope}%
\pgfpathrectangle{\pgfqpoint{3.793912in}{0.557870in}}{\pgfqpoint{2.446088in}{1.684734in}}%
\pgfusepath{clip}%
\pgfsetroundcap%
\pgfsetroundjoin%
\pgfsetlinewidth{1.003750pt}%
\definecolor{currentstroke}{rgb}{1.000000,1.000000,1.000000}%
\pgfsetstrokecolor{currentstroke}%
\pgfsetdash{}{0pt}%
\pgfpathmoveto{\pgfqpoint{3.793912in}{0.634449in}}%
\pgfpathlineto{\pgfqpoint{6.240000in}{0.634449in}}%
\pgfusepath{stroke}%
\end{pgfscope}%
\begin{pgfscope}%
\definecolor{textcolor}{rgb}{0.150000,0.150000,0.150000}%
\pgfsetstrokecolor{textcolor}%
\pgfsetfillcolor{textcolor}%
\pgftext[x=3.391597in,y=0.581642in,left,base]{\color{textcolor}\sffamily\fontsize{11.000000}{13.200000}\selectfont \(\displaystyle 0.00\)}%
\end{pgfscope}%
\begin{pgfscope}%
\pgfpathrectangle{\pgfqpoint{3.793912in}{0.557870in}}{\pgfqpoint{2.446088in}{1.684734in}}%
\pgfusepath{clip}%
\pgfsetroundcap%
\pgfsetroundjoin%
\pgfsetlinewidth{1.003750pt}%
\definecolor{currentstroke}{rgb}{1.000000,1.000000,1.000000}%
\pgfsetstrokecolor{currentstroke}%
\pgfsetdash{}{0pt}%
\pgfpathmoveto{\pgfqpoint{3.793912in}{1.017343in}}%
\pgfpathlineto{\pgfqpoint{6.240000in}{1.017343in}}%
\pgfusepath{stroke}%
\end{pgfscope}%
\begin{pgfscope}%
\definecolor{textcolor}{rgb}{0.150000,0.150000,0.150000}%
\pgfsetstrokecolor{textcolor}%
\pgfsetfillcolor{textcolor}%
\pgftext[x=3.391597in,y=0.964536in,left,base]{\color{textcolor}\sffamily\fontsize{11.000000}{13.200000}\selectfont \(\displaystyle 0.25\)}%
\end{pgfscope}%
\begin{pgfscope}%
\pgfpathrectangle{\pgfqpoint{3.793912in}{0.557870in}}{\pgfqpoint{2.446088in}{1.684734in}}%
\pgfusepath{clip}%
\pgfsetroundcap%
\pgfsetroundjoin%
\pgfsetlinewidth{1.003750pt}%
\definecolor{currentstroke}{rgb}{1.000000,1.000000,1.000000}%
\pgfsetstrokecolor{currentstroke}%
\pgfsetdash{}{0pt}%
\pgfpathmoveto{\pgfqpoint{3.793912in}{1.400237in}}%
\pgfpathlineto{\pgfqpoint{6.240000in}{1.400237in}}%
\pgfusepath{stroke}%
\end{pgfscope}%
\begin{pgfscope}%
\definecolor{textcolor}{rgb}{0.150000,0.150000,0.150000}%
\pgfsetstrokecolor{textcolor}%
\pgfsetfillcolor{textcolor}%
\pgftext[x=3.391597in,y=1.347431in,left,base]{\color{textcolor}\sffamily\fontsize{11.000000}{13.200000}\selectfont \(\displaystyle 0.50\)}%
\end{pgfscope}%
\begin{pgfscope}%
\pgfpathrectangle{\pgfqpoint{3.793912in}{0.557870in}}{\pgfqpoint{2.446088in}{1.684734in}}%
\pgfusepath{clip}%
\pgfsetroundcap%
\pgfsetroundjoin%
\pgfsetlinewidth{1.003750pt}%
\definecolor{currentstroke}{rgb}{1.000000,1.000000,1.000000}%
\pgfsetstrokecolor{currentstroke}%
\pgfsetdash{}{0pt}%
\pgfpathmoveto{\pgfqpoint{3.793912in}{1.783131in}}%
\pgfpathlineto{\pgfqpoint{6.240000in}{1.783131in}}%
\pgfusepath{stroke}%
\end{pgfscope}%
\begin{pgfscope}%
\definecolor{textcolor}{rgb}{0.150000,0.150000,0.150000}%
\pgfsetstrokecolor{textcolor}%
\pgfsetfillcolor{textcolor}%
\pgftext[x=3.391597in,y=1.730325in,left,base]{\color{textcolor}\sffamily\fontsize{11.000000}{13.200000}\selectfont \(\displaystyle 0.75\)}%
\end{pgfscope}%
\begin{pgfscope}%
\pgfpathrectangle{\pgfqpoint{3.793912in}{0.557870in}}{\pgfqpoint{2.446088in}{1.684734in}}%
\pgfusepath{clip}%
\pgfsetroundcap%
\pgfsetroundjoin%
\pgfsetlinewidth{1.003750pt}%
\definecolor{currentstroke}{rgb}{1.000000,1.000000,1.000000}%
\pgfsetstrokecolor{currentstroke}%
\pgfsetdash{}{0pt}%
\pgfpathmoveto{\pgfqpoint{3.793912in}{2.166025in}}%
\pgfpathlineto{\pgfqpoint{6.240000in}{2.166025in}}%
\pgfusepath{stroke}%
\end{pgfscope}%
\begin{pgfscope}%
\definecolor{textcolor}{rgb}{0.150000,0.150000,0.150000}%
\pgfsetstrokecolor{textcolor}%
\pgfsetfillcolor{textcolor}%
\pgftext[x=3.391597in,y=2.113219in,left,base]{\color{textcolor}\sffamily\fontsize{11.000000}{13.200000}\selectfont \(\displaystyle 1.00\)}%
\end{pgfscope}%
\begin{pgfscope}%
\definecolor{textcolor}{rgb}{0.150000,0.150000,0.150000}%
\pgfsetstrokecolor{textcolor}%
\pgfsetfillcolor{textcolor}%
\pgftext[x=3.336042in,y=1.400237in,,bottom,rotate=90.000000]{\color{textcolor}\sffamily\fontsize{11.000000}{13.200000}\selectfont Sensitivity}%
\end{pgfscope}%
\begin{pgfscope}%
\pgfpathrectangle{\pgfqpoint{3.793912in}{0.557870in}}{\pgfqpoint{2.446088in}{1.684734in}}%
\pgfusepath{clip}%
\pgfsetbuttcap%
\pgfsetroundjoin%
\definecolor{currentfill}{rgb}{0.298039,0.447059,0.690196}%
\pgfsetfillcolor{currentfill}%
\pgfsetlinewidth{1.003750pt}%
\definecolor{currentstroke}{rgb}{0.298039,0.447059,0.690196}%
\pgfsetstrokecolor{currentstroke}%
\pgfsetdash{}{0pt}%
\pgfpathmoveto{\pgfqpoint{3.905098in}{2.125798in}}%
\pgfpathcurveto{\pgfqpoint{3.913334in}{2.125798in}}{\pgfqpoint{3.921234in}{2.129070in}}{\pgfqpoint{3.927058in}{2.134894in}}%
\pgfpathcurveto{\pgfqpoint{3.932882in}{2.140718in}}{\pgfqpoint{3.936155in}{2.148618in}}{\pgfqpoint{3.936155in}{2.156854in}}%
\pgfpathcurveto{\pgfqpoint{3.936155in}{2.165091in}}{\pgfqpoint{3.932882in}{2.172991in}}{\pgfqpoint{3.927058in}{2.178814in}}%
\pgfpathcurveto{\pgfqpoint{3.921234in}{2.184638in}}{\pgfqpoint{3.913334in}{2.187911in}}{\pgfqpoint{3.905098in}{2.187911in}}%
\pgfpathcurveto{\pgfqpoint{3.896862in}{2.187911in}}{\pgfqpoint{3.888962in}{2.184638in}}{\pgfqpoint{3.883138in}{2.178814in}}%
\pgfpathcurveto{\pgfqpoint{3.877314in}{2.172991in}}{\pgfqpoint{3.874042in}{2.165091in}}{\pgfqpoint{3.874042in}{2.156854in}}%
\pgfpathcurveto{\pgfqpoint{3.874042in}{2.148618in}}{\pgfqpoint{3.877314in}{2.140718in}}{\pgfqpoint{3.883138in}{2.134894in}}%
\pgfpathcurveto{\pgfqpoint{3.888962in}{2.129070in}}{\pgfqpoint{3.896862in}{2.125798in}}{\pgfqpoint{3.905098in}{2.125798in}}%
\pgfpathclose%
\pgfusepath{stroke,fill}%
\end{pgfscope}%
\begin{pgfscope}%
\pgfpathrectangle{\pgfqpoint{3.793912in}{0.557870in}}{\pgfqpoint{2.446088in}{1.684734in}}%
\pgfusepath{clip}%
\pgfsetbuttcap%
\pgfsetroundjoin%
\definecolor{currentfill}{rgb}{0.298039,0.447059,0.690196}%
\pgfsetfillcolor{currentfill}%
\pgfsetlinewidth{1.003750pt}%
\definecolor{currentstroke}{rgb}{0.298039,0.447059,0.690196}%
\pgfsetstrokecolor{currentstroke}%
\pgfsetdash{}{0pt}%
\pgfpathmoveto{\pgfqpoint{5.975454in}{1.364595in}}%
\pgfpathcurveto{\pgfqpoint{5.983691in}{1.364595in}}{\pgfqpoint{5.991591in}{1.367867in}}{\pgfqpoint{5.997415in}{1.373691in}}%
\pgfpathcurveto{\pgfqpoint{6.003239in}{1.379515in}}{\pgfqpoint{6.006511in}{1.387415in}}{\pgfqpoint{6.006511in}{1.395652in}}%
\pgfpathcurveto{\pgfqpoint{6.006511in}{1.403888in}}{\pgfqpoint{6.003239in}{1.411788in}}{\pgfqpoint{5.997415in}{1.417612in}}%
\pgfpathcurveto{\pgfqpoint{5.991591in}{1.423436in}}{\pgfqpoint{5.983691in}{1.426708in}}{\pgfqpoint{5.975454in}{1.426708in}}%
\pgfpathcurveto{\pgfqpoint{5.967218in}{1.426708in}}{\pgfqpoint{5.959318in}{1.423436in}}{\pgfqpoint{5.953494in}{1.417612in}}%
\pgfpathcurveto{\pgfqpoint{5.947670in}{1.411788in}}{\pgfqpoint{5.944398in}{1.403888in}}{\pgfqpoint{5.944398in}{1.395652in}}%
\pgfpathcurveto{\pgfqpoint{5.944398in}{1.387415in}}{\pgfqpoint{5.947670in}{1.379515in}}{\pgfqpoint{5.953494in}{1.373691in}}%
\pgfpathcurveto{\pgfqpoint{5.959318in}{1.367867in}}{\pgfqpoint{5.967218in}{1.364595in}}{\pgfqpoint{5.975454in}{1.364595in}}%
\pgfpathclose%
\pgfusepath{stroke,fill}%
\end{pgfscope}%
\begin{pgfscope}%
\pgfpathrectangle{\pgfqpoint{3.793912in}{0.557870in}}{\pgfqpoint{2.446088in}{1.684734in}}%
\pgfusepath{clip}%
\pgfsetbuttcap%
\pgfsetroundjoin%
\definecolor{currentfill}{rgb}{0.298039,0.447059,0.690196}%
\pgfsetfillcolor{currentfill}%
\pgfsetlinewidth{1.003750pt}%
\definecolor{currentstroke}{rgb}{0.298039,0.447059,0.690196}%
\pgfsetstrokecolor{currentstroke}%
\pgfsetdash{}{0pt}%
\pgfpathmoveto{\pgfqpoint{5.975454in}{1.364595in}}%
\pgfpathcurveto{\pgfqpoint{5.983691in}{1.364595in}}{\pgfqpoint{5.991591in}{1.367867in}}{\pgfqpoint{5.997415in}{1.373691in}}%
\pgfpathcurveto{\pgfqpoint{6.003239in}{1.379515in}}{\pgfqpoint{6.006511in}{1.387415in}}{\pgfqpoint{6.006511in}{1.395652in}}%
\pgfpathcurveto{\pgfqpoint{6.006511in}{1.403888in}}{\pgfqpoint{6.003239in}{1.411788in}}{\pgfqpoint{5.997415in}{1.417612in}}%
\pgfpathcurveto{\pgfqpoint{5.991591in}{1.423436in}}{\pgfqpoint{5.983691in}{1.426708in}}{\pgfqpoint{5.975454in}{1.426708in}}%
\pgfpathcurveto{\pgfqpoint{5.967218in}{1.426708in}}{\pgfqpoint{5.959318in}{1.423436in}}{\pgfqpoint{5.953494in}{1.417612in}}%
\pgfpathcurveto{\pgfqpoint{5.947670in}{1.411788in}}{\pgfqpoint{5.944398in}{1.403888in}}{\pgfqpoint{5.944398in}{1.395652in}}%
\pgfpathcurveto{\pgfqpoint{5.944398in}{1.387415in}}{\pgfqpoint{5.947670in}{1.379515in}}{\pgfqpoint{5.953494in}{1.373691in}}%
\pgfpathcurveto{\pgfqpoint{5.959318in}{1.367867in}}{\pgfqpoint{5.967218in}{1.364595in}}{\pgfqpoint{5.975454in}{1.364595in}}%
\pgfpathclose%
\pgfusepath{stroke,fill}%
\end{pgfscope}%
\begin{pgfscope}%
\pgfpathrectangle{\pgfqpoint{3.793912in}{0.557870in}}{\pgfqpoint{2.446088in}{1.684734in}}%
\pgfusepath{clip}%
\pgfsetbuttcap%
\pgfsetroundjoin%
\definecolor{currentfill}{rgb}{0.298039,0.447059,0.690196}%
\pgfsetfillcolor{currentfill}%
\pgfsetlinewidth{1.003750pt}%
\definecolor{currentstroke}{rgb}{0.298039,0.447059,0.690196}%
\pgfsetstrokecolor{currentstroke}%
\pgfsetdash{}{0pt}%
\pgfpathmoveto{\pgfqpoint{5.975454in}{1.437964in}}%
\pgfpathcurveto{\pgfqpoint{5.983691in}{1.437964in}}{\pgfqpoint{5.991591in}{1.441236in}}{\pgfqpoint{5.997415in}{1.447060in}}%
\pgfpathcurveto{\pgfqpoint{6.003239in}{1.452884in}}{\pgfqpoint{6.006511in}{1.460784in}}{\pgfqpoint{6.006511in}{1.469021in}}%
\pgfpathcurveto{\pgfqpoint{6.006511in}{1.477257in}}{\pgfqpoint{6.003239in}{1.485157in}}{\pgfqpoint{5.997415in}{1.490981in}}%
\pgfpathcurveto{\pgfqpoint{5.991591in}{1.496805in}}{\pgfqpoint{5.983691in}{1.500077in}}{\pgfqpoint{5.975454in}{1.500077in}}%
\pgfpathcurveto{\pgfqpoint{5.967218in}{1.500077in}}{\pgfqpoint{5.959318in}{1.496805in}}{\pgfqpoint{5.953494in}{1.490981in}}%
\pgfpathcurveto{\pgfqpoint{5.947670in}{1.485157in}}{\pgfqpoint{5.944398in}{1.477257in}}{\pgfqpoint{5.944398in}{1.469021in}}%
\pgfpathcurveto{\pgfqpoint{5.944398in}{1.460784in}}{\pgfqpoint{5.947670in}{1.452884in}}{\pgfqpoint{5.953494in}{1.447060in}}%
\pgfpathcurveto{\pgfqpoint{5.959318in}{1.441236in}}{\pgfqpoint{5.967218in}{1.437964in}}{\pgfqpoint{5.975454in}{1.437964in}}%
\pgfpathclose%
\pgfusepath{stroke,fill}%
\end{pgfscope}%
\begin{pgfscope}%
\pgfpathrectangle{\pgfqpoint{3.793912in}{0.557870in}}{\pgfqpoint{2.446088in}{1.684734in}}%
\pgfusepath{clip}%
\pgfsetbuttcap%
\pgfsetroundjoin%
\definecolor{currentfill}{rgb}{0.298039,0.447059,0.690196}%
\pgfsetfillcolor{currentfill}%
\pgfsetlinewidth{1.003750pt}%
\definecolor{currentstroke}{rgb}{0.298039,0.447059,0.690196}%
\pgfsetstrokecolor{currentstroke}%
\pgfsetdash{}{0pt}%
\pgfpathmoveto{\pgfqpoint{5.975454in}{1.538846in}}%
\pgfpathcurveto{\pgfqpoint{5.983691in}{1.538846in}}{\pgfqpoint{5.991591in}{1.542119in}}{\pgfqpoint{5.997415in}{1.547943in}}%
\pgfpathcurveto{\pgfqpoint{6.003239in}{1.553767in}}{\pgfqpoint{6.006511in}{1.561667in}}{\pgfqpoint{6.006511in}{1.569903in}}%
\pgfpathcurveto{\pgfqpoint{6.006511in}{1.578139in}}{\pgfqpoint{6.003239in}{1.586039in}}{\pgfqpoint{5.997415in}{1.591863in}}%
\pgfpathcurveto{\pgfqpoint{5.991591in}{1.597687in}}{\pgfqpoint{5.983691in}{1.600959in}}{\pgfqpoint{5.975454in}{1.600959in}}%
\pgfpathcurveto{\pgfqpoint{5.967218in}{1.600959in}}{\pgfqpoint{5.959318in}{1.597687in}}{\pgfqpoint{5.953494in}{1.591863in}}%
\pgfpathcurveto{\pgfqpoint{5.947670in}{1.586039in}}{\pgfqpoint{5.944398in}{1.578139in}}{\pgfqpoint{5.944398in}{1.569903in}}%
\pgfpathcurveto{\pgfqpoint{5.944398in}{1.561667in}}{\pgfqpoint{5.947670in}{1.553767in}}{\pgfqpoint{5.953494in}{1.547943in}}%
\pgfpathcurveto{\pgfqpoint{5.959318in}{1.542119in}}{\pgfqpoint{5.967218in}{1.538846in}}{\pgfqpoint{5.975454in}{1.538846in}}%
\pgfpathclose%
\pgfusepath{stroke,fill}%
\end{pgfscope}%
\begin{pgfscope}%
\pgfpathrectangle{\pgfqpoint{3.793912in}{0.557870in}}{\pgfqpoint{2.446088in}{1.684734in}}%
\pgfusepath{clip}%
\pgfsetbuttcap%
\pgfsetroundjoin%
\definecolor{currentfill}{rgb}{0.298039,0.447059,0.690196}%
\pgfsetfillcolor{currentfill}%
\pgfsetlinewidth{1.003750pt}%
\definecolor{currentstroke}{rgb}{0.298039,0.447059,0.690196}%
\pgfsetstrokecolor{currentstroke}%
\pgfsetdash{}{0pt}%
\pgfpathmoveto{\pgfqpoint{4.058458in}{1.924033in}}%
\pgfpathcurveto{\pgfqpoint{4.066694in}{1.924033in}}{\pgfqpoint{4.074594in}{1.927306in}}{\pgfqpoint{4.080418in}{1.933129in}}%
\pgfpathcurveto{\pgfqpoint{4.086242in}{1.938953in}}{\pgfqpoint{4.089514in}{1.946853in}}{\pgfqpoint{4.089514in}{1.955090in}}%
\pgfpathcurveto{\pgfqpoint{4.089514in}{1.963326in}}{\pgfqpoint{4.086242in}{1.971226in}}{\pgfqpoint{4.080418in}{1.977050in}}%
\pgfpathcurveto{\pgfqpoint{4.074594in}{1.982874in}}{\pgfqpoint{4.066694in}{1.986146in}}{\pgfqpoint{4.058458in}{1.986146in}}%
\pgfpathcurveto{\pgfqpoint{4.050221in}{1.986146in}}{\pgfqpoint{4.042321in}{1.982874in}}{\pgfqpoint{4.036498in}{1.977050in}}%
\pgfpathcurveto{\pgfqpoint{4.030674in}{1.971226in}}{\pgfqpoint{4.027401in}{1.963326in}}{\pgfqpoint{4.027401in}{1.955090in}}%
\pgfpathcurveto{\pgfqpoint{4.027401in}{1.946853in}}{\pgfqpoint{4.030674in}{1.938953in}}{\pgfqpoint{4.036498in}{1.933129in}}%
\pgfpathcurveto{\pgfqpoint{4.042321in}{1.927306in}}{\pgfqpoint{4.050221in}{1.924033in}}{\pgfqpoint{4.058458in}{1.924033in}}%
\pgfpathclose%
\pgfusepath{stroke,fill}%
\end{pgfscope}%
\begin{pgfscope}%
\pgfpathrectangle{\pgfqpoint{3.793912in}{0.557870in}}{\pgfqpoint{2.446088in}{1.684734in}}%
\pgfusepath{clip}%
\pgfsetbuttcap%
\pgfsetroundjoin%
\definecolor{currentfill}{rgb}{0.298039,0.447059,0.690196}%
\pgfsetfillcolor{currentfill}%
\pgfsetlinewidth{1.003750pt}%
\definecolor{currentstroke}{rgb}{0.298039,0.447059,0.690196}%
\pgfsetstrokecolor{currentstroke}%
\pgfsetdash{}{0pt}%
\pgfpathmoveto{\pgfqpoint{5.745415in}{1.850664in}}%
\pgfpathcurveto{\pgfqpoint{5.753651in}{1.850664in}}{\pgfqpoint{5.761551in}{1.853937in}}{\pgfqpoint{5.767375in}{1.859761in}}%
\pgfpathcurveto{\pgfqpoint{5.773199in}{1.865584in}}{\pgfqpoint{5.776471in}{1.873484in}}{\pgfqpoint{5.776471in}{1.881721in}}%
\pgfpathcurveto{\pgfqpoint{5.776471in}{1.889957in}}{\pgfqpoint{5.773199in}{1.897857in}}{\pgfqpoint{5.767375in}{1.903681in}}%
\pgfpathcurveto{\pgfqpoint{5.761551in}{1.909505in}}{\pgfqpoint{5.753651in}{1.912777in}}{\pgfqpoint{5.745415in}{1.912777in}}%
\pgfpathcurveto{\pgfqpoint{5.737179in}{1.912777in}}{\pgfqpoint{5.729279in}{1.909505in}}{\pgfqpoint{5.723455in}{1.903681in}}%
\pgfpathcurveto{\pgfqpoint{5.717631in}{1.897857in}}{\pgfqpoint{5.714358in}{1.889957in}}{\pgfqpoint{5.714358in}{1.881721in}}%
\pgfpathcurveto{\pgfqpoint{5.714358in}{1.873484in}}{\pgfqpoint{5.717631in}{1.865584in}}{\pgfqpoint{5.723455in}{1.859761in}}%
\pgfpathcurveto{\pgfqpoint{5.729279in}{1.853937in}}{\pgfqpoint{5.737179in}{1.850664in}}{\pgfqpoint{5.745415in}{1.850664in}}%
\pgfpathclose%
\pgfusepath{stroke,fill}%
\end{pgfscope}%
\begin{pgfscope}%
\pgfpathrectangle{\pgfqpoint{3.793912in}{0.557870in}}{\pgfqpoint{2.446088in}{1.684734in}}%
\pgfusepath{clip}%
\pgfsetbuttcap%
\pgfsetroundjoin%
\definecolor{currentfill}{rgb}{0.298039,0.447059,0.690196}%
\pgfsetfillcolor{currentfill}%
\pgfsetlinewidth{1.003750pt}%
\definecolor{currentstroke}{rgb}{0.298039,0.447059,0.690196}%
\pgfsetstrokecolor{currentstroke}%
\pgfsetdash{}{0pt}%
\pgfpathmoveto{\pgfqpoint{3.905098in}{2.125798in}}%
\pgfpathcurveto{\pgfqpoint{3.913334in}{2.125798in}}{\pgfqpoint{3.921234in}{2.129070in}}{\pgfqpoint{3.927058in}{2.134894in}}%
\pgfpathcurveto{\pgfqpoint{3.932882in}{2.140718in}}{\pgfqpoint{3.936155in}{2.148618in}}{\pgfqpoint{3.936155in}{2.156854in}}%
\pgfpathcurveto{\pgfqpoint{3.936155in}{2.165091in}}{\pgfqpoint{3.932882in}{2.172991in}}{\pgfqpoint{3.927058in}{2.178814in}}%
\pgfpathcurveto{\pgfqpoint{3.921234in}{2.184638in}}{\pgfqpoint{3.913334in}{2.187911in}}{\pgfqpoint{3.905098in}{2.187911in}}%
\pgfpathcurveto{\pgfqpoint{3.896862in}{2.187911in}}{\pgfqpoint{3.888962in}{2.184638in}}{\pgfqpoint{3.883138in}{2.178814in}}%
\pgfpathcurveto{\pgfqpoint{3.877314in}{2.172991in}}{\pgfqpoint{3.874042in}{2.165091in}}{\pgfqpoint{3.874042in}{2.156854in}}%
\pgfpathcurveto{\pgfqpoint{3.874042in}{2.148618in}}{\pgfqpoint{3.877314in}{2.140718in}}{\pgfqpoint{3.883138in}{2.134894in}}%
\pgfpathcurveto{\pgfqpoint{3.888962in}{2.129070in}}{\pgfqpoint{3.896862in}{2.125798in}}{\pgfqpoint{3.905098in}{2.125798in}}%
\pgfpathclose%
\pgfusepath{stroke,fill}%
\end{pgfscope}%
\begin{pgfscope}%
\pgfpathrectangle{\pgfqpoint{3.793912in}{0.557870in}}{\pgfqpoint{2.446088in}{1.684734in}}%
\pgfusepath{clip}%
\pgfsetbuttcap%
\pgfsetroundjoin%
\definecolor{currentfill}{rgb}{0.298039,0.447059,0.690196}%
\pgfsetfillcolor{currentfill}%
\pgfsetlinewidth{1.003750pt}%
\definecolor{currentstroke}{rgb}{0.298039,0.447059,0.690196}%
\pgfsetstrokecolor{currentstroke}%
\pgfsetdash{}{0pt}%
\pgfpathmoveto{\pgfqpoint{4.441857in}{1.924033in}}%
\pgfpathcurveto{\pgfqpoint{4.450093in}{1.924033in}}{\pgfqpoint{4.457993in}{1.927306in}}{\pgfqpoint{4.463817in}{1.933129in}}%
\pgfpathcurveto{\pgfqpoint{4.469641in}{1.938953in}}{\pgfqpoint{4.472914in}{1.946853in}}{\pgfqpoint{4.472914in}{1.955090in}}%
\pgfpathcurveto{\pgfqpoint{4.472914in}{1.963326in}}{\pgfqpoint{4.469641in}{1.971226in}}{\pgfqpoint{4.463817in}{1.977050in}}%
\pgfpathcurveto{\pgfqpoint{4.457993in}{1.982874in}}{\pgfqpoint{4.450093in}{1.986146in}}{\pgfqpoint{4.441857in}{1.986146in}}%
\pgfpathcurveto{\pgfqpoint{4.433621in}{1.986146in}}{\pgfqpoint{4.425721in}{1.982874in}}{\pgfqpoint{4.419897in}{1.977050in}}%
\pgfpathcurveto{\pgfqpoint{4.414073in}{1.971226in}}{\pgfqpoint{4.410801in}{1.963326in}}{\pgfqpoint{4.410801in}{1.955090in}}%
\pgfpathcurveto{\pgfqpoint{4.410801in}{1.946853in}}{\pgfqpoint{4.414073in}{1.938953in}}{\pgfqpoint{4.419897in}{1.933129in}}%
\pgfpathcurveto{\pgfqpoint{4.425721in}{1.927306in}}{\pgfqpoint{4.433621in}{1.924033in}}{\pgfqpoint{4.441857in}{1.924033in}}%
\pgfpathclose%
\pgfusepath{stroke,fill}%
\end{pgfscope}%
\begin{pgfscope}%
\pgfpathrectangle{\pgfqpoint{3.793912in}{0.557870in}}{\pgfqpoint{2.446088in}{1.684734in}}%
\pgfusepath{clip}%
\pgfsetbuttcap%
\pgfsetroundjoin%
\definecolor{currentfill}{rgb}{0.298039,0.447059,0.690196}%
\pgfsetfillcolor{currentfill}%
\pgfsetlinewidth{1.003750pt}%
\definecolor{currentstroke}{rgb}{0.298039,0.447059,0.690196}%
\pgfsetstrokecolor{currentstroke}%
\pgfsetdash{}{0pt}%
\pgfpathmoveto{\pgfqpoint{3.905098in}{2.125798in}}%
\pgfpathcurveto{\pgfqpoint{3.913334in}{2.125798in}}{\pgfqpoint{3.921234in}{2.129070in}}{\pgfqpoint{3.927058in}{2.134894in}}%
\pgfpathcurveto{\pgfqpoint{3.932882in}{2.140718in}}{\pgfqpoint{3.936155in}{2.148618in}}{\pgfqpoint{3.936155in}{2.156854in}}%
\pgfpathcurveto{\pgfqpoint{3.936155in}{2.165091in}}{\pgfqpoint{3.932882in}{2.172991in}}{\pgfqpoint{3.927058in}{2.178814in}}%
\pgfpathcurveto{\pgfqpoint{3.921234in}{2.184638in}}{\pgfqpoint{3.913334in}{2.187911in}}{\pgfqpoint{3.905098in}{2.187911in}}%
\pgfpathcurveto{\pgfqpoint{3.896862in}{2.187911in}}{\pgfqpoint{3.888962in}{2.184638in}}{\pgfqpoint{3.883138in}{2.178814in}}%
\pgfpathcurveto{\pgfqpoint{3.877314in}{2.172991in}}{\pgfqpoint{3.874042in}{2.165091in}}{\pgfqpoint{3.874042in}{2.156854in}}%
\pgfpathcurveto{\pgfqpoint{3.874042in}{2.148618in}}{\pgfqpoint{3.877314in}{2.140718in}}{\pgfqpoint{3.883138in}{2.134894in}}%
\pgfpathcurveto{\pgfqpoint{3.888962in}{2.129070in}}{\pgfqpoint{3.896862in}{2.125798in}}{\pgfqpoint{3.905098in}{2.125798in}}%
\pgfpathclose%
\pgfusepath{stroke,fill}%
\end{pgfscope}%
\begin{pgfscope}%
\pgfpathrectangle{\pgfqpoint{3.793912in}{0.557870in}}{\pgfqpoint{2.446088in}{1.684734in}}%
\pgfusepath{clip}%
\pgfsetbuttcap%
\pgfsetroundjoin%
\definecolor{currentfill}{rgb}{0.298039,0.447059,0.690196}%
\pgfsetfillcolor{currentfill}%
\pgfsetlinewidth{1.003750pt}%
\definecolor{currentstroke}{rgb}{0.298039,0.447059,0.690196}%
\pgfsetstrokecolor{currentstroke}%
\pgfsetdash{}{0pt}%
\pgfpathmoveto{\pgfqpoint{4.978616in}{1.648900in}}%
\pgfpathcurveto{\pgfqpoint{4.986852in}{1.648900in}}{\pgfqpoint{4.994753in}{1.652172in}}{\pgfqpoint{5.000576in}{1.657996in}}%
\pgfpathcurveto{\pgfqpoint{5.006400in}{1.663820in}}{\pgfqpoint{5.009673in}{1.671720in}}{\pgfqpoint{5.009673in}{1.679956in}}%
\pgfpathcurveto{\pgfqpoint{5.009673in}{1.688193in}}{\pgfqpoint{5.006400in}{1.696093in}}{\pgfqpoint{5.000576in}{1.701917in}}%
\pgfpathcurveto{\pgfqpoint{4.994753in}{1.707740in}}{\pgfqpoint{4.986852in}{1.711013in}}{\pgfqpoint{4.978616in}{1.711013in}}%
\pgfpathcurveto{\pgfqpoint{4.970380in}{1.711013in}}{\pgfqpoint{4.962480in}{1.707740in}}{\pgfqpoint{4.956656in}{1.701917in}}%
\pgfpathcurveto{\pgfqpoint{4.950832in}{1.696093in}}{\pgfqpoint{4.947560in}{1.688193in}}{\pgfqpoint{4.947560in}{1.679956in}}%
\pgfpathcurveto{\pgfqpoint{4.947560in}{1.671720in}}{\pgfqpoint{4.950832in}{1.663820in}}{\pgfqpoint{4.956656in}{1.657996in}}%
\pgfpathcurveto{\pgfqpoint{4.962480in}{1.652172in}}{\pgfqpoint{4.970380in}{1.648900in}}{\pgfqpoint{4.978616in}{1.648900in}}%
\pgfpathclose%
\pgfusepath{stroke,fill}%
\end{pgfscope}%
\begin{pgfscope}%
\pgfpathrectangle{\pgfqpoint{3.793912in}{0.557870in}}{\pgfqpoint{2.446088in}{1.684734in}}%
\pgfusepath{clip}%
\pgfsetbuttcap%
\pgfsetroundjoin%
\definecolor{currentfill}{rgb}{0.298039,0.447059,0.690196}%
\pgfsetfillcolor{currentfill}%
\pgfsetlinewidth{1.003750pt}%
\definecolor{currentstroke}{rgb}{0.298039,0.447059,0.690196}%
\pgfsetstrokecolor{currentstroke}%
\pgfsetdash{}{0pt}%
\pgfpathmoveto{\pgfqpoint{3.905098in}{2.125798in}}%
\pgfpathcurveto{\pgfqpoint{3.913334in}{2.125798in}}{\pgfqpoint{3.921234in}{2.129070in}}{\pgfqpoint{3.927058in}{2.134894in}}%
\pgfpathcurveto{\pgfqpoint{3.932882in}{2.140718in}}{\pgfqpoint{3.936155in}{2.148618in}}{\pgfqpoint{3.936155in}{2.156854in}}%
\pgfpathcurveto{\pgfqpoint{3.936155in}{2.165091in}}{\pgfqpoint{3.932882in}{2.172991in}}{\pgfqpoint{3.927058in}{2.178814in}}%
\pgfpathcurveto{\pgfqpoint{3.921234in}{2.184638in}}{\pgfqpoint{3.913334in}{2.187911in}}{\pgfqpoint{3.905098in}{2.187911in}}%
\pgfpathcurveto{\pgfqpoint{3.896862in}{2.187911in}}{\pgfqpoint{3.888962in}{2.184638in}}{\pgfqpoint{3.883138in}{2.178814in}}%
\pgfpathcurveto{\pgfqpoint{3.877314in}{2.172991in}}{\pgfqpoint{3.874042in}{2.165091in}}{\pgfqpoint{3.874042in}{2.156854in}}%
\pgfpathcurveto{\pgfqpoint{3.874042in}{2.148618in}}{\pgfqpoint{3.877314in}{2.140718in}}{\pgfqpoint{3.883138in}{2.134894in}}%
\pgfpathcurveto{\pgfqpoint{3.888962in}{2.129070in}}{\pgfqpoint{3.896862in}{2.125798in}}{\pgfqpoint{3.905098in}{2.125798in}}%
\pgfpathclose%
\pgfusepath{stroke,fill}%
\end{pgfscope}%
\begin{pgfscope}%
\pgfpathrectangle{\pgfqpoint{3.793912in}{0.557870in}}{\pgfqpoint{2.446088in}{1.684734in}}%
\pgfusepath{clip}%
\pgfsetbuttcap%
\pgfsetroundjoin%
\definecolor{currentfill}{rgb}{0.298039,0.447059,0.690196}%
\pgfsetfillcolor{currentfill}%
\pgfsetlinewidth{1.003750pt}%
\definecolor{currentstroke}{rgb}{0.298039,0.447059,0.690196}%
\pgfsetstrokecolor{currentstroke}%
\pgfsetdash{}{0pt}%
\pgfpathmoveto{\pgfqpoint{3.905098in}{2.125798in}}%
\pgfpathcurveto{\pgfqpoint{3.913334in}{2.125798in}}{\pgfqpoint{3.921234in}{2.129070in}}{\pgfqpoint{3.927058in}{2.134894in}}%
\pgfpathcurveto{\pgfqpoint{3.932882in}{2.140718in}}{\pgfqpoint{3.936155in}{2.148618in}}{\pgfqpoint{3.936155in}{2.156854in}}%
\pgfpathcurveto{\pgfqpoint{3.936155in}{2.165091in}}{\pgfqpoint{3.932882in}{2.172991in}}{\pgfqpoint{3.927058in}{2.178814in}}%
\pgfpathcurveto{\pgfqpoint{3.921234in}{2.184638in}}{\pgfqpoint{3.913334in}{2.187911in}}{\pgfqpoint{3.905098in}{2.187911in}}%
\pgfpathcurveto{\pgfqpoint{3.896862in}{2.187911in}}{\pgfqpoint{3.888962in}{2.184638in}}{\pgfqpoint{3.883138in}{2.178814in}}%
\pgfpathcurveto{\pgfqpoint{3.877314in}{2.172991in}}{\pgfqpoint{3.874042in}{2.165091in}}{\pgfqpoint{3.874042in}{2.156854in}}%
\pgfpathcurveto{\pgfqpoint{3.874042in}{2.148618in}}{\pgfqpoint{3.877314in}{2.140718in}}{\pgfqpoint{3.883138in}{2.134894in}}%
\pgfpathcurveto{\pgfqpoint{3.888962in}{2.129070in}}{\pgfqpoint{3.896862in}{2.125798in}}{\pgfqpoint{3.905098in}{2.125798in}}%
\pgfpathclose%
\pgfusepath{stroke,fill}%
\end{pgfscope}%
\begin{pgfscope}%
\pgfpathrectangle{\pgfqpoint{3.793912in}{0.557870in}}{\pgfqpoint{2.446088in}{1.684734in}}%
\pgfusepath{clip}%
\pgfsetbuttcap%
\pgfsetroundjoin%
\definecolor{currentfill}{rgb}{0.298039,0.447059,0.690196}%
\pgfsetfillcolor{currentfill}%
\pgfsetlinewidth{1.003750pt}%
\definecolor{currentstroke}{rgb}{0.298039,0.447059,0.690196}%
\pgfsetstrokecolor{currentstroke}%
\pgfsetdash{}{0pt}%
\pgfpathmoveto{\pgfqpoint{3.905098in}{2.125798in}}%
\pgfpathcurveto{\pgfqpoint{3.913334in}{2.125798in}}{\pgfqpoint{3.921234in}{2.129070in}}{\pgfqpoint{3.927058in}{2.134894in}}%
\pgfpathcurveto{\pgfqpoint{3.932882in}{2.140718in}}{\pgfqpoint{3.936155in}{2.148618in}}{\pgfqpoint{3.936155in}{2.156854in}}%
\pgfpathcurveto{\pgfqpoint{3.936155in}{2.165091in}}{\pgfqpoint{3.932882in}{2.172991in}}{\pgfqpoint{3.927058in}{2.178814in}}%
\pgfpathcurveto{\pgfqpoint{3.921234in}{2.184638in}}{\pgfqpoint{3.913334in}{2.187911in}}{\pgfqpoint{3.905098in}{2.187911in}}%
\pgfpathcurveto{\pgfqpoint{3.896862in}{2.187911in}}{\pgfqpoint{3.888962in}{2.184638in}}{\pgfqpoint{3.883138in}{2.178814in}}%
\pgfpathcurveto{\pgfqpoint{3.877314in}{2.172991in}}{\pgfqpoint{3.874042in}{2.165091in}}{\pgfqpoint{3.874042in}{2.156854in}}%
\pgfpathcurveto{\pgfqpoint{3.874042in}{2.148618in}}{\pgfqpoint{3.877314in}{2.140718in}}{\pgfqpoint{3.883138in}{2.134894in}}%
\pgfpathcurveto{\pgfqpoint{3.888962in}{2.129070in}}{\pgfqpoint{3.896862in}{2.125798in}}{\pgfqpoint{3.905098in}{2.125798in}}%
\pgfpathclose%
\pgfusepath{stroke,fill}%
\end{pgfscope}%
\begin{pgfscope}%
\pgfpathrectangle{\pgfqpoint{3.793912in}{0.557870in}}{\pgfqpoint{2.446088in}{1.684734in}}%
\pgfusepath{clip}%
\pgfsetbuttcap%
\pgfsetroundjoin%
\definecolor{currentfill}{rgb}{0.298039,0.447059,0.690196}%
\pgfsetfillcolor{currentfill}%
\pgfsetlinewidth{1.003750pt}%
\definecolor{currentstroke}{rgb}{0.298039,0.447059,0.690196}%
\pgfsetstrokecolor{currentstroke}%
\pgfsetdash{}{0pt}%
\pgfpathmoveto{\pgfqpoint{3.905098in}{2.125798in}}%
\pgfpathcurveto{\pgfqpoint{3.913334in}{2.125798in}}{\pgfqpoint{3.921234in}{2.129070in}}{\pgfqpoint{3.927058in}{2.134894in}}%
\pgfpathcurveto{\pgfqpoint{3.932882in}{2.140718in}}{\pgfqpoint{3.936155in}{2.148618in}}{\pgfqpoint{3.936155in}{2.156854in}}%
\pgfpathcurveto{\pgfqpoint{3.936155in}{2.165091in}}{\pgfqpoint{3.932882in}{2.172991in}}{\pgfqpoint{3.927058in}{2.178814in}}%
\pgfpathcurveto{\pgfqpoint{3.921234in}{2.184638in}}{\pgfqpoint{3.913334in}{2.187911in}}{\pgfqpoint{3.905098in}{2.187911in}}%
\pgfpathcurveto{\pgfqpoint{3.896862in}{2.187911in}}{\pgfqpoint{3.888962in}{2.184638in}}{\pgfqpoint{3.883138in}{2.178814in}}%
\pgfpathcurveto{\pgfqpoint{3.877314in}{2.172991in}}{\pgfqpoint{3.874042in}{2.165091in}}{\pgfqpoint{3.874042in}{2.156854in}}%
\pgfpathcurveto{\pgfqpoint{3.874042in}{2.148618in}}{\pgfqpoint{3.877314in}{2.140718in}}{\pgfqpoint{3.883138in}{2.134894in}}%
\pgfpathcurveto{\pgfqpoint{3.888962in}{2.129070in}}{\pgfqpoint{3.896862in}{2.125798in}}{\pgfqpoint{3.905098in}{2.125798in}}%
\pgfpathclose%
\pgfusepath{stroke,fill}%
\end{pgfscope}%
\begin{pgfscope}%
\pgfpathrectangle{\pgfqpoint{3.793912in}{0.557870in}}{\pgfqpoint{2.446088in}{1.684734in}}%
\pgfusepath{clip}%
\pgfsetbuttcap%
\pgfsetroundjoin%
\definecolor{currentfill}{rgb}{0.298039,0.447059,0.690196}%
\pgfsetfillcolor{currentfill}%
\pgfsetlinewidth{1.003750pt}%
\definecolor{currentstroke}{rgb}{0.298039,0.447059,0.690196}%
\pgfsetstrokecolor{currentstroke}%
\pgfsetdash{}{0pt}%
\pgfpathmoveto{\pgfqpoint{5.131976in}{1.639729in}}%
\pgfpathcurveto{\pgfqpoint{5.140212in}{1.639729in}}{\pgfqpoint{5.148112in}{1.643001in}}{\pgfqpoint{5.153936in}{1.648825in}}%
\pgfpathcurveto{\pgfqpoint{5.159760in}{1.654649in}}{\pgfqpoint{5.163032in}{1.662549in}}{\pgfqpoint{5.163032in}{1.670785in}}%
\pgfpathcurveto{\pgfqpoint{5.163032in}{1.679021in}}{\pgfqpoint{5.159760in}{1.686921in}}{\pgfqpoint{5.153936in}{1.692745in}}%
\pgfpathcurveto{\pgfqpoint{5.148112in}{1.698569in}}{\pgfqpoint{5.140212in}{1.701842in}}{\pgfqpoint{5.131976in}{1.701842in}}%
\pgfpathcurveto{\pgfqpoint{5.123740in}{1.701842in}}{\pgfqpoint{5.115840in}{1.698569in}}{\pgfqpoint{5.110016in}{1.692745in}}%
\pgfpathcurveto{\pgfqpoint{5.104192in}{1.686921in}}{\pgfqpoint{5.100919in}{1.679021in}}{\pgfqpoint{5.100919in}{1.670785in}}%
\pgfpathcurveto{\pgfqpoint{5.100919in}{1.662549in}}{\pgfqpoint{5.104192in}{1.654649in}}{\pgfqpoint{5.110016in}{1.648825in}}%
\pgfpathcurveto{\pgfqpoint{5.115840in}{1.643001in}}{\pgfqpoint{5.123740in}{1.639729in}}{\pgfqpoint{5.131976in}{1.639729in}}%
\pgfpathclose%
\pgfusepath{stroke,fill}%
\end{pgfscope}%
\begin{pgfscope}%
\pgfpathrectangle{\pgfqpoint{3.793912in}{0.557870in}}{\pgfqpoint{2.446088in}{1.684734in}}%
\pgfusepath{clip}%
\pgfsetbuttcap%
\pgfsetroundjoin%
\definecolor{currentfill}{rgb}{0.298039,0.447059,0.690196}%
\pgfsetfillcolor{currentfill}%
\pgfsetlinewidth{1.003750pt}%
\definecolor{currentstroke}{rgb}{0.298039,0.447059,0.690196}%
\pgfsetstrokecolor{currentstroke}%
\pgfsetdash{}{0pt}%
\pgfpathmoveto{\pgfqpoint{3.905098in}{2.125798in}}%
\pgfpathcurveto{\pgfqpoint{3.913334in}{2.125798in}}{\pgfqpoint{3.921234in}{2.129070in}}{\pgfqpoint{3.927058in}{2.134894in}}%
\pgfpathcurveto{\pgfqpoint{3.932882in}{2.140718in}}{\pgfqpoint{3.936155in}{2.148618in}}{\pgfqpoint{3.936155in}{2.156854in}}%
\pgfpathcurveto{\pgfqpoint{3.936155in}{2.165091in}}{\pgfqpoint{3.932882in}{2.172991in}}{\pgfqpoint{3.927058in}{2.178814in}}%
\pgfpathcurveto{\pgfqpoint{3.921234in}{2.184638in}}{\pgfqpoint{3.913334in}{2.187911in}}{\pgfqpoint{3.905098in}{2.187911in}}%
\pgfpathcurveto{\pgfqpoint{3.896862in}{2.187911in}}{\pgfqpoint{3.888962in}{2.184638in}}{\pgfqpoint{3.883138in}{2.178814in}}%
\pgfpathcurveto{\pgfqpoint{3.877314in}{2.172991in}}{\pgfqpoint{3.874042in}{2.165091in}}{\pgfqpoint{3.874042in}{2.156854in}}%
\pgfpathcurveto{\pgfqpoint{3.874042in}{2.148618in}}{\pgfqpoint{3.877314in}{2.140718in}}{\pgfqpoint{3.883138in}{2.134894in}}%
\pgfpathcurveto{\pgfqpoint{3.888962in}{2.129070in}}{\pgfqpoint{3.896862in}{2.125798in}}{\pgfqpoint{3.905098in}{2.125798in}}%
\pgfpathclose%
\pgfusepath{stroke,fill}%
\end{pgfscope}%
\begin{pgfscope}%
\pgfpathrectangle{\pgfqpoint{3.793912in}{0.557870in}}{\pgfqpoint{2.446088in}{1.684734in}}%
\pgfusepath{clip}%
\pgfsetbuttcap%
\pgfsetroundjoin%
\definecolor{currentfill}{rgb}{0.298039,0.447059,0.690196}%
\pgfsetfillcolor{currentfill}%
\pgfsetlinewidth{1.003750pt}%
\definecolor{currentstroke}{rgb}{0.298039,0.447059,0.690196}%
\pgfsetstrokecolor{currentstroke}%
\pgfsetdash{}{0pt}%
\pgfpathmoveto{\pgfqpoint{5.131976in}{1.401280in}}%
\pgfpathcurveto{\pgfqpoint{5.140212in}{1.401280in}}{\pgfqpoint{5.148112in}{1.404552in}}{\pgfqpoint{5.153936in}{1.410376in}}%
\pgfpathcurveto{\pgfqpoint{5.159760in}{1.416200in}}{\pgfqpoint{5.163032in}{1.424100in}}{\pgfqpoint{5.163032in}{1.432336in}}%
\pgfpathcurveto{\pgfqpoint{5.163032in}{1.440572in}}{\pgfqpoint{5.159760in}{1.448472in}}{\pgfqpoint{5.153936in}{1.454296in}}%
\pgfpathcurveto{\pgfqpoint{5.148112in}{1.460120in}}{\pgfqpoint{5.140212in}{1.463393in}}{\pgfqpoint{5.131976in}{1.463393in}}%
\pgfpathcurveto{\pgfqpoint{5.123740in}{1.463393in}}{\pgfqpoint{5.115840in}{1.460120in}}{\pgfqpoint{5.110016in}{1.454296in}}%
\pgfpathcurveto{\pgfqpoint{5.104192in}{1.448472in}}{\pgfqpoint{5.100919in}{1.440572in}}{\pgfqpoint{5.100919in}{1.432336in}}%
\pgfpathcurveto{\pgfqpoint{5.100919in}{1.424100in}}{\pgfqpoint{5.104192in}{1.416200in}}{\pgfqpoint{5.110016in}{1.410376in}}%
\pgfpathcurveto{\pgfqpoint{5.115840in}{1.404552in}}{\pgfqpoint{5.123740in}{1.401280in}}{\pgfqpoint{5.131976in}{1.401280in}}%
\pgfpathclose%
\pgfusepath{stroke,fill}%
\end{pgfscope}%
\begin{pgfscope}%
\pgfpathrectangle{\pgfqpoint{3.793912in}{0.557870in}}{\pgfqpoint{2.446088in}{1.684734in}}%
\pgfusepath{clip}%
\pgfsetbuttcap%
\pgfsetroundjoin%
\definecolor{currentfill}{rgb}{0.298039,0.447059,0.690196}%
\pgfsetfillcolor{currentfill}%
\pgfsetlinewidth{1.003750pt}%
\definecolor{currentstroke}{rgb}{0.298039,0.447059,0.690196}%
\pgfsetstrokecolor{currentstroke}%
\pgfsetdash{}{0pt}%
\pgfpathmoveto{\pgfqpoint{3.905098in}{2.125798in}}%
\pgfpathcurveto{\pgfqpoint{3.913334in}{2.125798in}}{\pgfqpoint{3.921234in}{2.129070in}}{\pgfqpoint{3.927058in}{2.134894in}}%
\pgfpathcurveto{\pgfqpoint{3.932882in}{2.140718in}}{\pgfqpoint{3.936155in}{2.148618in}}{\pgfqpoint{3.936155in}{2.156854in}}%
\pgfpathcurveto{\pgfqpoint{3.936155in}{2.165091in}}{\pgfqpoint{3.932882in}{2.172991in}}{\pgfqpoint{3.927058in}{2.178814in}}%
\pgfpathcurveto{\pgfqpoint{3.921234in}{2.184638in}}{\pgfqpoint{3.913334in}{2.187911in}}{\pgfqpoint{3.905098in}{2.187911in}}%
\pgfpathcurveto{\pgfqpoint{3.896862in}{2.187911in}}{\pgfqpoint{3.888962in}{2.184638in}}{\pgfqpoint{3.883138in}{2.178814in}}%
\pgfpathcurveto{\pgfqpoint{3.877314in}{2.172991in}}{\pgfqpoint{3.874042in}{2.165091in}}{\pgfqpoint{3.874042in}{2.156854in}}%
\pgfpathcurveto{\pgfqpoint{3.874042in}{2.148618in}}{\pgfqpoint{3.877314in}{2.140718in}}{\pgfqpoint{3.883138in}{2.134894in}}%
\pgfpathcurveto{\pgfqpoint{3.888962in}{2.129070in}}{\pgfqpoint{3.896862in}{2.125798in}}{\pgfqpoint{3.905098in}{2.125798in}}%
\pgfpathclose%
\pgfusepath{stroke,fill}%
\end{pgfscope}%
\begin{pgfscope}%
\pgfpathrectangle{\pgfqpoint{3.793912in}{0.557870in}}{\pgfqpoint{2.446088in}{1.684734in}}%
\pgfusepath{clip}%
\pgfsetbuttcap%
\pgfsetroundjoin%
\definecolor{currentfill}{rgb}{0.298039,0.447059,0.690196}%
\pgfsetfillcolor{currentfill}%
\pgfsetlinewidth{1.003750pt}%
\definecolor{currentstroke}{rgb}{0.298039,0.447059,0.690196}%
\pgfsetstrokecolor{currentstroke}%
\pgfsetdash{}{0pt}%
\pgfpathmoveto{\pgfqpoint{3.905098in}{2.125798in}}%
\pgfpathcurveto{\pgfqpoint{3.913334in}{2.125798in}}{\pgfqpoint{3.921234in}{2.129070in}}{\pgfqpoint{3.927058in}{2.134894in}}%
\pgfpathcurveto{\pgfqpoint{3.932882in}{2.140718in}}{\pgfqpoint{3.936155in}{2.148618in}}{\pgfqpoint{3.936155in}{2.156854in}}%
\pgfpathcurveto{\pgfqpoint{3.936155in}{2.165091in}}{\pgfqpoint{3.932882in}{2.172991in}}{\pgfqpoint{3.927058in}{2.178814in}}%
\pgfpathcurveto{\pgfqpoint{3.921234in}{2.184638in}}{\pgfqpoint{3.913334in}{2.187911in}}{\pgfqpoint{3.905098in}{2.187911in}}%
\pgfpathcurveto{\pgfqpoint{3.896862in}{2.187911in}}{\pgfqpoint{3.888962in}{2.184638in}}{\pgfqpoint{3.883138in}{2.178814in}}%
\pgfpathcurveto{\pgfqpoint{3.877314in}{2.172991in}}{\pgfqpoint{3.874042in}{2.165091in}}{\pgfqpoint{3.874042in}{2.156854in}}%
\pgfpathcurveto{\pgfqpoint{3.874042in}{2.148618in}}{\pgfqpoint{3.877314in}{2.140718in}}{\pgfqpoint{3.883138in}{2.134894in}}%
\pgfpathcurveto{\pgfqpoint{3.888962in}{2.129070in}}{\pgfqpoint{3.896862in}{2.125798in}}{\pgfqpoint{3.905098in}{2.125798in}}%
\pgfpathclose%
\pgfusepath{stroke,fill}%
\end{pgfscope}%
\begin{pgfscope}%
\pgfpathrectangle{\pgfqpoint{3.793912in}{0.557870in}}{\pgfqpoint{2.446088in}{1.684734in}}%
\pgfusepath{clip}%
\pgfsetbuttcap%
\pgfsetroundjoin%
\definecolor{currentfill}{rgb}{0.298039,0.447059,0.690196}%
\pgfsetfillcolor{currentfill}%
\pgfsetlinewidth{1.003750pt}%
\definecolor{currentstroke}{rgb}{0.298039,0.447059,0.690196}%
\pgfsetstrokecolor{currentstroke}%
\pgfsetdash{}{0pt}%
\pgfpathmoveto{\pgfqpoint{3.905098in}{2.125798in}}%
\pgfpathcurveto{\pgfqpoint{3.913334in}{2.125798in}}{\pgfqpoint{3.921234in}{2.129070in}}{\pgfqpoint{3.927058in}{2.134894in}}%
\pgfpathcurveto{\pgfqpoint{3.932882in}{2.140718in}}{\pgfqpoint{3.936155in}{2.148618in}}{\pgfqpoint{3.936155in}{2.156854in}}%
\pgfpathcurveto{\pgfqpoint{3.936155in}{2.165091in}}{\pgfqpoint{3.932882in}{2.172991in}}{\pgfqpoint{3.927058in}{2.178814in}}%
\pgfpathcurveto{\pgfqpoint{3.921234in}{2.184638in}}{\pgfqpoint{3.913334in}{2.187911in}}{\pgfqpoint{3.905098in}{2.187911in}}%
\pgfpathcurveto{\pgfqpoint{3.896862in}{2.187911in}}{\pgfqpoint{3.888962in}{2.184638in}}{\pgfqpoint{3.883138in}{2.178814in}}%
\pgfpathcurveto{\pgfqpoint{3.877314in}{2.172991in}}{\pgfqpoint{3.874042in}{2.165091in}}{\pgfqpoint{3.874042in}{2.156854in}}%
\pgfpathcurveto{\pgfqpoint{3.874042in}{2.148618in}}{\pgfqpoint{3.877314in}{2.140718in}}{\pgfqpoint{3.883138in}{2.134894in}}%
\pgfpathcurveto{\pgfqpoint{3.888962in}{2.129070in}}{\pgfqpoint{3.896862in}{2.125798in}}{\pgfqpoint{3.905098in}{2.125798in}}%
\pgfpathclose%
\pgfusepath{stroke,fill}%
\end{pgfscope}%
\begin{pgfscope}%
\pgfpathrectangle{\pgfqpoint{3.793912in}{0.557870in}}{\pgfqpoint{2.446088in}{1.684734in}}%
\pgfusepath{clip}%
\pgfsetbuttcap%
\pgfsetroundjoin%
\definecolor{currentfill}{rgb}{0.298039,0.447059,0.690196}%
\pgfsetfillcolor{currentfill}%
\pgfsetlinewidth{1.003750pt}%
\definecolor{currentstroke}{rgb}{0.298039,0.447059,0.690196}%
\pgfsetstrokecolor{currentstroke}%
\pgfsetdash{}{0pt}%
\pgfpathmoveto{\pgfqpoint{3.905098in}{2.125798in}}%
\pgfpathcurveto{\pgfqpoint{3.913334in}{2.125798in}}{\pgfqpoint{3.921234in}{2.129070in}}{\pgfqpoint{3.927058in}{2.134894in}}%
\pgfpathcurveto{\pgfqpoint{3.932882in}{2.140718in}}{\pgfqpoint{3.936155in}{2.148618in}}{\pgfqpoint{3.936155in}{2.156854in}}%
\pgfpathcurveto{\pgfqpoint{3.936155in}{2.165091in}}{\pgfqpoint{3.932882in}{2.172991in}}{\pgfqpoint{3.927058in}{2.178814in}}%
\pgfpathcurveto{\pgfqpoint{3.921234in}{2.184638in}}{\pgfqpoint{3.913334in}{2.187911in}}{\pgfqpoint{3.905098in}{2.187911in}}%
\pgfpathcurveto{\pgfqpoint{3.896862in}{2.187911in}}{\pgfqpoint{3.888962in}{2.184638in}}{\pgfqpoint{3.883138in}{2.178814in}}%
\pgfpathcurveto{\pgfqpoint{3.877314in}{2.172991in}}{\pgfqpoint{3.874042in}{2.165091in}}{\pgfqpoint{3.874042in}{2.156854in}}%
\pgfpathcurveto{\pgfqpoint{3.874042in}{2.148618in}}{\pgfqpoint{3.877314in}{2.140718in}}{\pgfqpoint{3.883138in}{2.134894in}}%
\pgfpathcurveto{\pgfqpoint{3.888962in}{2.129070in}}{\pgfqpoint{3.896862in}{2.125798in}}{\pgfqpoint{3.905098in}{2.125798in}}%
\pgfpathclose%
\pgfusepath{stroke,fill}%
\end{pgfscope}%
\begin{pgfscope}%
\pgfpathrectangle{\pgfqpoint{3.793912in}{0.557870in}}{\pgfqpoint{2.446088in}{1.684734in}}%
\pgfusepath{clip}%
\pgfsetbuttcap%
\pgfsetroundjoin%
\definecolor{currentfill}{rgb}{0.298039,0.447059,0.690196}%
\pgfsetfillcolor{currentfill}%
\pgfsetlinewidth{1.003750pt}%
\definecolor{currentstroke}{rgb}{0.298039,0.447059,0.690196}%
\pgfsetstrokecolor{currentstroke}%
\pgfsetdash{}{0pt}%
\pgfpathmoveto{\pgfqpoint{5.975454in}{1.428793in}}%
\pgfpathcurveto{\pgfqpoint{5.983691in}{1.428793in}}{\pgfqpoint{5.991591in}{1.432065in}}{\pgfqpoint{5.997415in}{1.437889in}}%
\pgfpathcurveto{\pgfqpoint{6.003239in}{1.443713in}}{\pgfqpoint{6.006511in}{1.451613in}}{\pgfqpoint{6.006511in}{1.459849in}}%
\pgfpathcurveto{\pgfqpoint{6.006511in}{1.468086in}}{\pgfqpoint{6.003239in}{1.475986in}}{\pgfqpoint{5.997415in}{1.481810in}}%
\pgfpathcurveto{\pgfqpoint{5.991591in}{1.487634in}}{\pgfqpoint{5.983691in}{1.490906in}}{\pgfqpoint{5.975454in}{1.490906in}}%
\pgfpathcurveto{\pgfqpoint{5.967218in}{1.490906in}}{\pgfqpoint{5.959318in}{1.487634in}}{\pgfqpoint{5.953494in}{1.481810in}}%
\pgfpathcurveto{\pgfqpoint{5.947670in}{1.475986in}}{\pgfqpoint{5.944398in}{1.468086in}}{\pgfqpoint{5.944398in}{1.459849in}}%
\pgfpathcurveto{\pgfqpoint{5.944398in}{1.451613in}}{\pgfqpoint{5.947670in}{1.443713in}}{\pgfqpoint{5.953494in}{1.437889in}}%
\pgfpathcurveto{\pgfqpoint{5.959318in}{1.432065in}}{\pgfqpoint{5.967218in}{1.428793in}}{\pgfqpoint{5.975454in}{1.428793in}}%
\pgfpathclose%
\pgfusepath{stroke,fill}%
\end{pgfscope}%
\begin{pgfscope}%
\pgfpathrectangle{\pgfqpoint{3.793912in}{0.557870in}}{\pgfqpoint{2.446088in}{1.684734in}}%
\pgfusepath{clip}%
\pgfsetbuttcap%
\pgfsetroundjoin%
\definecolor{currentfill}{rgb}{0.298039,0.447059,0.690196}%
\pgfsetfillcolor{currentfill}%
\pgfsetlinewidth{1.003750pt}%
\definecolor{currentstroke}{rgb}{0.298039,0.447059,0.690196}%
\pgfsetstrokecolor{currentstroke}%
\pgfsetdash{}{0pt}%
\pgfpathmoveto{\pgfqpoint{5.975454in}{1.529675in}}%
\pgfpathcurveto{\pgfqpoint{5.983691in}{1.529675in}}{\pgfqpoint{5.991591in}{1.532948in}}{\pgfqpoint{5.997415in}{1.538771in}}%
\pgfpathcurveto{\pgfqpoint{6.003239in}{1.544595in}}{\pgfqpoint{6.006511in}{1.552495in}}{\pgfqpoint{6.006511in}{1.560732in}}%
\pgfpathcurveto{\pgfqpoint{6.006511in}{1.568968in}}{\pgfqpoint{6.003239in}{1.576868in}}{\pgfqpoint{5.997415in}{1.582692in}}%
\pgfpathcurveto{\pgfqpoint{5.991591in}{1.588516in}}{\pgfqpoint{5.983691in}{1.591788in}}{\pgfqpoint{5.975454in}{1.591788in}}%
\pgfpathcurveto{\pgfqpoint{5.967218in}{1.591788in}}{\pgfqpoint{5.959318in}{1.588516in}}{\pgfqpoint{5.953494in}{1.582692in}}%
\pgfpathcurveto{\pgfqpoint{5.947670in}{1.576868in}}{\pgfqpoint{5.944398in}{1.568968in}}{\pgfqpoint{5.944398in}{1.560732in}}%
\pgfpathcurveto{\pgfqpoint{5.944398in}{1.552495in}}{\pgfqpoint{5.947670in}{1.544595in}}{\pgfqpoint{5.953494in}{1.538771in}}%
\pgfpathcurveto{\pgfqpoint{5.959318in}{1.532948in}}{\pgfqpoint{5.967218in}{1.529675in}}{\pgfqpoint{5.975454in}{1.529675in}}%
\pgfpathclose%
\pgfusepath{stroke,fill}%
\end{pgfscope}%
\begin{pgfscope}%
\pgfpathrectangle{\pgfqpoint{3.793912in}{0.557870in}}{\pgfqpoint{2.446088in}{1.684734in}}%
\pgfusepath{clip}%
\pgfsetbuttcap%
\pgfsetroundjoin%
\definecolor{currentfill}{rgb}{0.298039,0.447059,0.690196}%
\pgfsetfillcolor{currentfill}%
\pgfsetlinewidth{1.003750pt}%
\definecolor{currentstroke}{rgb}{0.298039,0.447059,0.690196}%
\pgfsetstrokecolor{currentstroke}%
\pgfsetdash{}{0pt}%
\pgfpathmoveto{\pgfqpoint{5.975454in}{1.437964in}}%
\pgfpathcurveto{\pgfqpoint{5.983691in}{1.437964in}}{\pgfqpoint{5.991591in}{1.441236in}}{\pgfqpoint{5.997415in}{1.447060in}}%
\pgfpathcurveto{\pgfqpoint{6.003239in}{1.452884in}}{\pgfqpoint{6.006511in}{1.460784in}}{\pgfqpoint{6.006511in}{1.469021in}}%
\pgfpathcurveto{\pgfqpoint{6.006511in}{1.477257in}}{\pgfqpoint{6.003239in}{1.485157in}}{\pgfqpoint{5.997415in}{1.490981in}}%
\pgfpathcurveto{\pgfqpoint{5.991591in}{1.496805in}}{\pgfqpoint{5.983691in}{1.500077in}}{\pgfqpoint{5.975454in}{1.500077in}}%
\pgfpathcurveto{\pgfqpoint{5.967218in}{1.500077in}}{\pgfqpoint{5.959318in}{1.496805in}}{\pgfqpoint{5.953494in}{1.490981in}}%
\pgfpathcurveto{\pgfqpoint{5.947670in}{1.485157in}}{\pgfqpoint{5.944398in}{1.477257in}}{\pgfqpoint{5.944398in}{1.469021in}}%
\pgfpathcurveto{\pgfqpoint{5.944398in}{1.460784in}}{\pgfqpoint{5.947670in}{1.452884in}}{\pgfqpoint{5.953494in}{1.447060in}}%
\pgfpathcurveto{\pgfqpoint{5.959318in}{1.441236in}}{\pgfqpoint{5.967218in}{1.437964in}}{\pgfqpoint{5.975454in}{1.437964in}}%
\pgfpathclose%
\pgfusepath{stroke,fill}%
\end{pgfscope}%
\begin{pgfscope}%
\pgfpathrectangle{\pgfqpoint{3.793912in}{0.557870in}}{\pgfqpoint{2.446088in}{1.684734in}}%
\pgfusepath{clip}%
\pgfsetbuttcap%
\pgfsetroundjoin%
\definecolor{currentfill}{rgb}{0.298039,0.447059,0.690196}%
\pgfsetfillcolor{currentfill}%
\pgfsetlinewidth{1.003750pt}%
\definecolor{currentstroke}{rgb}{0.298039,0.447059,0.690196}%
\pgfsetstrokecolor{currentstroke}%
\pgfsetdash{}{0pt}%
\pgfpathmoveto{\pgfqpoint{5.975454in}{1.364595in}}%
\pgfpathcurveto{\pgfqpoint{5.983691in}{1.364595in}}{\pgfqpoint{5.991591in}{1.367867in}}{\pgfqpoint{5.997415in}{1.373691in}}%
\pgfpathcurveto{\pgfqpoint{6.003239in}{1.379515in}}{\pgfqpoint{6.006511in}{1.387415in}}{\pgfqpoint{6.006511in}{1.395652in}}%
\pgfpathcurveto{\pgfqpoint{6.006511in}{1.403888in}}{\pgfqpoint{6.003239in}{1.411788in}}{\pgfqpoint{5.997415in}{1.417612in}}%
\pgfpathcurveto{\pgfqpoint{5.991591in}{1.423436in}}{\pgfqpoint{5.983691in}{1.426708in}}{\pgfqpoint{5.975454in}{1.426708in}}%
\pgfpathcurveto{\pgfqpoint{5.967218in}{1.426708in}}{\pgfqpoint{5.959318in}{1.423436in}}{\pgfqpoint{5.953494in}{1.417612in}}%
\pgfpathcurveto{\pgfqpoint{5.947670in}{1.411788in}}{\pgfqpoint{5.944398in}{1.403888in}}{\pgfqpoint{5.944398in}{1.395652in}}%
\pgfpathcurveto{\pgfqpoint{5.944398in}{1.387415in}}{\pgfqpoint{5.947670in}{1.379515in}}{\pgfqpoint{5.953494in}{1.373691in}}%
\pgfpathcurveto{\pgfqpoint{5.959318in}{1.367867in}}{\pgfqpoint{5.967218in}{1.364595in}}{\pgfqpoint{5.975454in}{1.364595in}}%
\pgfpathclose%
\pgfusepath{stroke,fill}%
\end{pgfscope}%
\begin{pgfscope}%
\pgfpathrectangle{\pgfqpoint{3.793912in}{0.557870in}}{\pgfqpoint{2.446088in}{1.684734in}}%
\pgfusepath{clip}%
\pgfsetbuttcap%
\pgfsetroundjoin%
\definecolor{currentfill}{rgb}{0.298039,0.447059,0.690196}%
\pgfsetfillcolor{currentfill}%
\pgfsetlinewidth{1.003750pt}%
\definecolor{currentstroke}{rgb}{0.298039,0.447059,0.690196}%
\pgfsetstrokecolor{currentstroke}%
\pgfsetdash{}{0pt}%
\pgfpathmoveto{\pgfqpoint{5.822095in}{1.456306in}}%
\pgfpathcurveto{\pgfqpoint{5.830331in}{1.456306in}}{\pgfqpoint{5.838231in}{1.459579in}}{\pgfqpoint{5.844055in}{1.465403in}}%
\pgfpathcurveto{\pgfqpoint{5.849879in}{1.471226in}}{\pgfqpoint{5.853151in}{1.479127in}}{\pgfqpoint{5.853151in}{1.487363in}}%
\pgfpathcurveto{\pgfqpoint{5.853151in}{1.495599in}}{\pgfqpoint{5.849879in}{1.503499in}}{\pgfqpoint{5.844055in}{1.509323in}}%
\pgfpathcurveto{\pgfqpoint{5.838231in}{1.515147in}}{\pgfqpoint{5.830331in}{1.518419in}}{\pgfqpoint{5.822095in}{1.518419in}}%
\pgfpathcurveto{\pgfqpoint{5.813858in}{1.518419in}}{\pgfqpoint{5.805958in}{1.515147in}}{\pgfqpoint{5.800134in}{1.509323in}}%
\pgfpathcurveto{\pgfqpoint{5.794311in}{1.503499in}}{\pgfqpoint{5.791038in}{1.495599in}}{\pgfqpoint{5.791038in}{1.487363in}}%
\pgfpathcurveto{\pgfqpoint{5.791038in}{1.479127in}}{\pgfqpoint{5.794311in}{1.471226in}}{\pgfqpoint{5.800134in}{1.465403in}}%
\pgfpathcurveto{\pgfqpoint{5.805958in}{1.459579in}}{\pgfqpoint{5.813858in}{1.456306in}}{\pgfqpoint{5.822095in}{1.456306in}}%
\pgfpathclose%
\pgfusepath{stroke,fill}%
\end{pgfscope}%
\begin{pgfscope}%
\pgfpathrectangle{\pgfqpoint{3.793912in}{0.557870in}}{\pgfqpoint{2.446088in}{1.684734in}}%
\pgfusepath{clip}%
\pgfsetbuttcap%
\pgfsetroundjoin%
\definecolor{currentfill}{rgb}{0.298039,0.447059,0.690196}%
\pgfsetfillcolor{currentfill}%
\pgfsetlinewidth{1.003750pt}%
\definecolor{currentstroke}{rgb}{0.298039,0.447059,0.690196}%
\pgfsetstrokecolor{currentstroke}%
\pgfsetdash{}{0pt}%
\pgfpathmoveto{\pgfqpoint{4.058458in}{2.006573in}}%
\pgfpathcurveto{\pgfqpoint{4.066694in}{2.006573in}}{\pgfqpoint{4.074594in}{2.009846in}}{\pgfqpoint{4.080418in}{2.015669in}}%
\pgfpathcurveto{\pgfqpoint{4.086242in}{2.021493in}}{\pgfqpoint{4.089514in}{2.029393in}}{\pgfqpoint{4.089514in}{2.037630in}}%
\pgfpathcurveto{\pgfqpoint{4.089514in}{2.045866in}}{\pgfqpoint{4.086242in}{2.053766in}}{\pgfqpoint{4.080418in}{2.059590in}}%
\pgfpathcurveto{\pgfqpoint{4.074594in}{2.065414in}}{\pgfqpoint{4.066694in}{2.068686in}}{\pgfqpoint{4.058458in}{2.068686in}}%
\pgfpathcurveto{\pgfqpoint{4.050221in}{2.068686in}}{\pgfqpoint{4.042321in}{2.065414in}}{\pgfqpoint{4.036498in}{2.059590in}}%
\pgfpathcurveto{\pgfqpoint{4.030674in}{2.053766in}}{\pgfqpoint{4.027401in}{2.045866in}}{\pgfqpoint{4.027401in}{2.037630in}}%
\pgfpathcurveto{\pgfqpoint{4.027401in}{2.029393in}}{\pgfqpoint{4.030674in}{2.021493in}}{\pgfqpoint{4.036498in}{2.015669in}}%
\pgfpathcurveto{\pgfqpoint{4.042321in}{2.009846in}}{\pgfqpoint{4.050221in}{2.006573in}}{\pgfqpoint{4.058458in}{2.006573in}}%
\pgfpathclose%
\pgfusepath{stroke,fill}%
\end{pgfscope}%
\begin{pgfscope}%
\pgfpathrectangle{\pgfqpoint{3.793912in}{0.557870in}}{\pgfqpoint{2.446088in}{1.684734in}}%
\pgfusepath{clip}%
\pgfsetbuttcap%
\pgfsetroundjoin%
\definecolor{currentfill}{rgb}{0.298039,0.447059,0.690196}%
\pgfsetfillcolor{currentfill}%
\pgfsetlinewidth{1.003750pt}%
\definecolor{currentstroke}{rgb}{0.298039,0.447059,0.690196}%
\pgfsetstrokecolor{currentstroke}%
\pgfsetdash{}{0pt}%
\pgfpathmoveto{\pgfqpoint{5.898775in}{1.557189in}}%
\pgfpathcurveto{\pgfqpoint{5.907011in}{1.557189in}}{\pgfqpoint{5.914911in}{1.560461in}}{\pgfqpoint{5.920735in}{1.566285in}}%
\pgfpathcurveto{\pgfqpoint{5.926559in}{1.572109in}}{\pgfqpoint{5.929831in}{1.580009in}}{\pgfqpoint{5.929831in}{1.588245in}}%
\pgfpathcurveto{\pgfqpoint{5.929831in}{1.596481in}}{\pgfqpoint{5.926559in}{1.604381in}}{\pgfqpoint{5.920735in}{1.610205in}}%
\pgfpathcurveto{\pgfqpoint{5.914911in}{1.616029in}}{\pgfqpoint{5.907011in}{1.619302in}}{\pgfqpoint{5.898775in}{1.619302in}}%
\pgfpathcurveto{\pgfqpoint{5.890538in}{1.619302in}}{\pgfqpoint{5.882638in}{1.616029in}}{\pgfqpoint{5.876814in}{1.610205in}}%
\pgfpathcurveto{\pgfqpoint{5.870990in}{1.604381in}}{\pgfqpoint{5.867718in}{1.596481in}}{\pgfqpoint{5.867718in}{1.588245in}}%
\pgfpathcurveto{\pgfqpoint{5.867718in}{1.580009in}}{\pgfqpoint{5.870990in}{1.572109in}}{\pgfqpoint{5.876814in}{1.566285in}}%
\pgfpathcurveto{\pgfqpoint{5.882638in}{1.560461in}}{\pgfqpoint{5.890538in}{1.557189in}}{\pgfqpoint{5.898775in}{1.557189in}}%
\pgfpathclose%
\pgfusepath{stroke,fill}%
\end{pgfscope}%
\begin{pgfscope}%
\pgfpathrectangle{\pgfqpoint{3.793912in}{0.557870in}}{\pgfqpoint{2.446088in}{1.684734in}}%
\pgfusepath{clip}%
\pgfsetbuttcap%
\pgfsetroundjoin%
\definecolor{currentfill}{rgb}{0.298039,0.447059,0.690196}%
\pgfsetfillcolor{currentfill}%
\pgfsetlinewidth{1.003750pt}%
\definecolor{currentstroke}{rgb}{0.298039,0.447059,0.690196}%
\pgfsetstrokecolor{currentstroke}%
\pgfsetdash{}{0pt}%
\pgfpathmoveto{\pgfqpoint{4.058458in}{1.621386in}}%
\pgfpathcurveto{\pgfqpoint{4.066694in}{1.621386in}}{\pgfqpoint{4.074594in}{1.624659in}}{\pgfqpoint{4.080418in}{1.630483in}}%
\pgfpathcurveto{\pgfqpoint{4.086242in}{1.636307in}}{\pgfqpoint{4.089514in}{1.644207in}}{\pgfqpoint{4.089514in}{1.652443in}}%
\pgfpathcurveto{\pgfqpoint{4.089514in}{1.660679in}}{\pgfqpoint{4.086242in}{1.668579in}}{\pgfqpoint{4.080418in}{1.674403in}}%
\pgfpathcurveto{\pgfqpoint{4.074594in}{1.680227in}}{\pgfqpoint{4.066694in}{1.683499in}}{\pgfqpoint{4.058458in}{1.683499in}}%
\pgfpathcurveto{\pgfqpoint{4.050221in}{1.683499in}}{\pgfqpoint{4.042321in}{1.680227in}}{\pgfqpoint{4.036498in}{1.674403in}}%
\pgfpathcurveto{\pgfqpoint{4.030674in}{1.668579in}}{\pgfqpoint{4.027401in}{1.660679in}}{\pgfqpoint{4.027401in}{1.652443in}}%
\pgfpathcurveto{\pgfqpoint{4.027401in}{1.644207in}}{\pgfqpoint{4.030674in}{1.636307in}}{\pgfqpoint{4.036498in}{1.630483in}}%
\pgfpathcurveto{\pgfqpoint{4.042321in}{1.624659in}}{\pgfqpoint{4.050221in}{1.621386in}}{\pgfqpoint{4.058458in}{1.621386in}}%
\pgfpathclose%
\pgfusepath{stroke,fill}%
\end{pgfscope}%
\begin{pgfscope}%
\pgfpathrectangle{\pgfqpoint{3.793912in}{0.557870in}}{\pgfqpoint{2.446088in}{1.684734in}}%
\pgfusepath{clip}%
\pgfsetbuttcap%
\pgfsetroundjoin%
\definecolor{currentfill}{rgb}{0.298039,0.447059,0.690196}%
\pgfsetfillcolor{currentfill}%
\pgfsetlinewidth{1.003750pt}%
\definecolor{currentstroke}{rgb}{0.298039,0.447059,0.690196}%
\pgfsetstrokecolor{currentstroke}%
\pgfsetdash{}{0pt}%
\pgfpathmoveto{\pgfqpoint{5.898775in}{1.584702in}}%
\pgfpathcurveto{\pgfqpoint{5.907011in}{1.584702in}}{\pgfqpoint{5.914911in}{1.587974in}}{\pgfqpoint{5.920735in}{1.593798in}}%
\pgfpathcurveto{\pgfqpoint{5.926559in}{1.599622in}}{\pgfqpoint{5.929831in}{1.607522in}}{\pgfqpoint{5.929831in}{1.615758in}}%
\pgfpathcurveto{\pgfqpoint{5.929831in}{1.623995in}}{\pgfqpoint{5.926559in}{1.631895in}}{\pgfqpoint{5.920735in}{1.637719in}}%
\pgfpathcurveto{\pgfqpoint{5.914911in}{1.643543in}}{\pgfqpoint{5.907011in}{1.646815in}}{\pgfqpoint{5.898775in}{1.646815in}}%
\pgfpathcurveto{\pgfqpoint{5.890538in}{1.646815in}}{\pgfqpoint{5.882638in}{1.643543in}}{\pgfqpoint{5.876814in}{1.637719in}}%
\pgfpathcurveto{\pgfqpoint{5.870990in}{1.631895in}}{\pgfqpoint{5.867718in}{1.623995in}}{\pgfqpoint{5.867718in}{1.615758in}}%
\pgfpathcurveto{\pgfqpoint{5.867718in}{1.607522in}}{\pgfqpoint{5.870990in}{1.599622in}}{\pgfqpoint{5.876814in}{1.593798in}}%
\pgfpathcurveto{\pgfqpoint{5.882638in}{1.587974in}}{\pgfqpoint{5.890538in}{1.584702in}}{\pgfqpoint{5.898775in}{1.584702in}}%
\pgfpathclose%
\pgfusepath{stroke,fill}%
\end{pgfscope}%
\begin{pgfscope}%
\pgfpathrectangle{\pgfqpoint{3.793912in}{0.557870in}}{\pgfqpoint{2.446088in}{1.684734in}}%
\pgfusepath{clip}%
\pgfsetbuttcap%
\pgfsetroundjoin%
\definecolor{currentfill}{rgb}{0.298039,0.447059,0.690196}%
\pgfsetfillcolor{currentfill}%
\pgfsetlinewidth{1.003750pt}%
\definecolor{currentstroke}{rgb}{0.298039,0.447059,0.690196}%
\pgfsetstrokecolor{currentstroke}%
\pgfsetdash{}{0pt}%
\pgfpathmoveto{\pgfqpoint{4.058458in}{1.878178in}}%
\pgfpathcurveto{\pgfqpoint{4.066694in}{1.878178in}}{\pgfqpoint{4.074594in}{1.881450in}}{\pgfqpoint{4.080418in}{1.887274in}}%
\pgfpathcurveto{\pgfqpoint{4.086242in}{1.893098in}}{\pgfqpoint{4.089514in}{1.900998in}}{\pgfqpoint{4.089514in}{1.909234in}}%
\pgfpathcurveto{\pgfqpoint{4.089514in}{1.917470in}}{\pgfqpoint{4.086242in}{1.925370in}}{\pgfqpoint{4.080418in}{1.931194in}}%
\pgfpathcurveto{\pgfqpoint{4.074594in}{1.937018in}}{\pgfqpoint{4.066694in}{1.940291in}}{\pgfqpoint{4.058458in}{1.940291in}}%
\pgfpathcurveto{\pgfqpoint{4.050221in}{1.940291in}}{\pgfqpoint{4.042321in}{1.937018in}}{\pgfqpoint{4.036498in}{1.931194in}}%
\pgfpathcurveto{\pgfqpoint{4.030674in}{1.925370in}}{\pgfqpoint{4.027401in}{1.917470in}}{\pgfqpoint{4.027401in}{1.909234in}}%
\pgfpathcurveto{\pgfqpoint{4.027401in}{1.900998in}}{\pgfqpoint{4.030674in}{1.893098in}}{\pgfqpoint{4.036498in}{1.887274in}}%
\pgfpathcurveto{\pgfqpoint{4.042321in}{1.881450in}}{\pgfqpoint{4.050221in}{1.878178in}}{\pgfqpoint{4.058458in}{1.878178in}}%
\pgfpathclose%
\pgfusepath{stroke,fill}%
\end{pgfscope}%
\begin{pgfscope}%
\pgfpathrectangle{\pgfqpoint{3.793912in}{0.557870in}}{\pgfqpoint{2.446088in}{1.684734in}}%
\pgfusepath{clip}%
\pgfsetbuttcap%
\pgfsetroundjoin%
\definecolor{currentfill}{rgb}{0.298039,0.447059,0.690196}%
\pgfsetfillcolor{currentfill}%
\pgfsetlinewidth{1.003750pt}%
\definecolor{currentstroke}{rgb}{0.298039,0.447059,0.690196}%
\pgfsetstrokecolor{currentstroke}%
\pgfsetdash{}{0pt}%
\pgfpathmoveto{\pgfqpoint{4.058458in}{1.878178in}}%
\pgfpathcurveto{\pgfqpoint{4.066694in}{1.878178in}}{\pgfqpoint{4.074594in}{1.881450in}}{\pgfqpoint{4.080418in}{1.887274in}}%
\pgfpathcurveto{\pgfqpoint{4.086242in}{1.893098in}}{\pgfqpoint{4.089514in}{1.900998in}}{\pgfqpoint{4.089514in}{1.909234in}}%
\pgfpathcurveto{\pgfqpoint{4.089514in}{1.917470in}}{\pgfqpoint{4.086242in}{1.925370in}}{\pgfqpoint{4.080418in}{1.931194in}}%
\pgfpathcurveto{\pgfqpoint{4.074594in}{1.937018in}}{\pgfqpoint{4.066694in}{1.940291in}}{\pgfqpoint{4.058458in}{1.940291in}}%
\pgfpathcurveto{\pgfqpoint{4.050221in}{1.940291in}}{\pgfqpoint{4.042321in}{1.937018in}}{\pgfqpoint{4.036498in}{1.931194in}}%
\pgfpathcurveto{\pgfqpoint{4.030674in}{1.925370in}}{\pgfqpoint{4.027401in}{1.917470in}}{\pgfqpoint{4.027401in}{1.909234in}}%
\pgfpathcurveto{\pgfqpoint{4.027401in}{1.900998in}}{\pgfqpoint{4.030674in}{1.893098in}}{\pgfqpoint{4.036498in}{1.887274in}}%
\pgfpathcurveto{\pgfqpoint{4.042321in}{1.881450in}}{\pgfqpoint{4.050221in}{1.878178in}}{\pgfqpoint{4.058458in}{1.878178in}}%
\pgfpathclose%
\pgfusepath{stroke,fill}%
\end{pgfscope}%
\begin{pgfscope}%
\pgfpathrectangle{\pgfqpoint{3.793912in}{0.557870in}}{\pgfqpoint{2.446088in}{1.684734in}}%
\pgfusepath{clip}%
\pgfsetbuttcap%
\pgfsetroundjoin%
\definecolor{currentfill}{rgb}{0.298039,0.447059,0.690196}%
\pgfsetfillcolor{currentfill}%
\pgfsetlinewidth{1.003750pt}%
\definecolor{currentstroke}{rgb}{0.298039,0.447059,0.690196}%
\pgfsetstrokecolor{currentstroke}%
\pgfsetdash{}{0pt}%
\pgfpathmoveto{\pgfqpoint{3.905098in}{2.125798in}}%
\pgfpathcurveto{\pgfqpoint{3.913334in}{2.125798in}}{\pgfqpoint{3.921234in}{2.129070in}}{\pgfqpoint{3.927058in}{2.134894in}}%
\pgfpathcurveto{\pgfqpoint{3.932882in}{2.140718in}}{\pgfqpoint{3.936155in}{2.148618in}}{\pgfqpoint{3.936155in}{2.156854in}}%
\pgfpathcurveto{\pgfqpoint{3.936155in}{2.165091in}}{\pgfqpoint{3.932882in}{2.172991in}}{\pgfqpoint{3.927058in}{2.178814in}}%
\pgfpathcurveto{\pgfqpoint{3.921234in}{2.184638in}}{\pgfqpoint{3.913334in}{2.187911in}}{\pgfqpoint{3.905098in}{2.187911in}}%
\pgfpathcurveto{\pgfqpoint{3.896862in}{2.187911in}}{\pgfqpoint{3.888962in}{2.184638in}}{\pgfqpoint{3.883138in}{2.178814in}}%
\pgfpathcurveto{\pgfqpoint{3.877314in}{2.172991in}}{\pgfqpoint{3.874042in}{2.165091in}}{\pgfqpoint{3.874042in}{2.156854in}}%
\pgfpathcurveto{\pgfqpoint{3.874042in}{2.148618in}}{\pgfqpoint{3.877314in}{2.140718in}}{\pgfqpoint{3.883138in}{2.134894in}}%
\pgfpathcurveto{\pgfqpoint{3.888962in}{2.129070in}}{\pgfqpoint{3.896862in}{2.125798in}}{\pgfqpoint{3.905098in}{2.125798in}}%
\pgfpathclose%
\pgfusepath{stroke,fill}%
\end{pgfscope}%
\begin{pgfscope}%
\pgfpathrectangle{\pgfqpoint{3.793912in}{0.557870in}}{\pgfqpoint{2.446088in}{1.684734in}}%
\pgfusepath{clip}%
\pgfsetbuttcap%
\pgfsetroundjoin%
\definecolor{currentfill}{rgb}{0.298039,0.447059,0.690196}%
\pgfsetfillcolor{currentfill}%
\pgfsetlinewidth{1.003750pt}%
\definecolor{currentstroke}{rgb}{0.298039,0.447059,0.690196}%
\pgfsetstrokecolor{currentstroke}%
\pgfsetdash{}{0pt}%
\pgfpathmoveto{\pgfqpoint{3.905098in}{2.125798in}}%
\pgfpathcurveto{\pgfqpoint{3.913334in}{2.125798in}}{\pgfqpoint{3.921234in}{2.129070in}}{\pgfqpoint{3.927058in}{2.134894in}}%
\pgfpathcurveto{\pgfqpoint{3.932882in}{2.140718in}}{\pgfqpoint{3.936155in}{2.148618in}}{\pgfqpoint{3.936155in}{2.156854in}}%
\pgfpathcurveto{\pgfqpoint{3.936155in}{2.165091in}}{\pgfqpoint{3.932882in}{2.172991in}}{\pgfqpoint{3.927058in}{2.178814in}}%
\pgfpathcurveto{\pgfqpoint{3.921234in}{2.184638in}}{\pgfqpoint{3.913334in}{2.187911in}}{\pgfqpoint{3.905098in}{2.187911in}}%
\pgfpathcurveto{\pgfqpoint{3.896862in}{2.187911in}}{\pgfqpoint{3.888962in}{2.184638in}}{\pgfqpoint{3.883138in}{2.178814in}}%
\pgfpathcurveto{\pgfqpoint{3.877314in}{2.172991in}}{\pgfqpoint{3.874042in}{2.165091in}}{\pgfqpoint{3.874042in}{2.156854in}}%
\pgfpathcurveto{\pgfqpoint{3.874042in}{2.148618in}}{\pgfqpoint{3.877314in}{2.140718in}}{\pgfqpoint{3.883138in}{2.134894in}}%
\pgfpathcurveto{\pgfqpoint{3.888962in}{2.129070in}}{\pgfqpoint{3.896862in}{2.125798in}}{\pgfqpoint{3.905098in}{2.125798in}}%
\pgfpathclose%
\pgfusepath{stroke,fill}%
\end{pgfscope}%
\begin{pgfscope}%
\pgfpathrectangle{\pgfqpoint{3.793912in}{0.557870in}}{\pgfqpoint{2.446088in}{1.684734in}}%
\pgfusepath{clip}%
\pgfsetbuttcap%
\pgfsetroundjoin%
\definecolor{currentfill}{rgb}{0.298039,0.447059,0.690196}%
\pgfsetfillcolor{currentfill}%
\pgfsetlinewidth{1.003750pt}%
\definecolor{currentstroke}{rgb}{0.298039,0.447059,0.690196}%
\pgfsetstrokecolor{currentstroke}%
\pgfsetdash{}{0pt}%
\pgfpathmoveto{\pgfqpoint{3.905098in}{2.125798in}}%
\pgfpathcurveto{\pgfqpoint{3.913334in}{2.125798in}}{\pgfqpoint{3.921234in}{2.129070in}}{\pgfqpoint{3.927058in}{2.134894in}}%
\pgfpathcurveto{\pgfqpoint{3.932882in}{2.140718in}}{\pgfqpoint{3.936155in}{2.148618in}}{\pgfqpoint{3.936155in}{2.156854in}}%
\pgfpathcurveto{\pgfqpoint{3.936155in}{2.165091in}}{\pgfqpoint{3.932882in}{2.172991in}}{\pgfqpoint{3.927058in}{2.178814in}}%
\pgfpathcurveto{\pgfqpoint{3.921234in}{2.184638in}}{\pgfqpoint{3.913334in}{2.187911in}}{\pgfqpoint{3.905098in}{2.187911in}}%
\pgfpathcurveto{\pgfqpoint{3.896862in}{2.187911in}}{\pgfqpoint{3.888962in}{2.184638in}}{\pgfqpoint{3.883138in}{2.178814in}}%
\pgfpathcurveto{\pgfqpoint{3.877314in}{2.172991in}}{\pgfqpoint{3.874042in}{2.165091in}}{\pgfqpoint{3.874042in}{2.156854in}}%
\pgfpathcurveto{\pgfqpoint{3.874042in}{2.148618in}}{\pgfqpoint{3.877314in}{2.140718in}}{\pgfqpoint{3.883138in}{2.134894in}}%
\pgfpathcurveto{\pgfqpoint{3.888962in}{2.129070in}}{\pgfqpoint{3.896862in}{2.125798in}}{\pgfqpoint{3.905098in}{2.125798in}}%
\pgfpathclose%
\pgfusepath{stroke,fill}%
\end{pgfscope}%
\begin{pgfscope}%
\pgfpathrectangle{\pgfqpoint{3.793912in}{0.557870in}}{\pgfqpoint{2.446088in}{1.684734in}}%
\pgfusepath{clip}%
\pgfsetbuttcap%
\pgfsetroundjoin%
\definecolor{currentfill}{rgb}{0.298039,0.447059,0.690196}%
\pgfsetfillcolor{currentfill}%
\pgfsetlinewidth{1.003750pt}%
\definecolor{currentstroke}{rgb}{0.298039,0.447059,0.690196}%
\pgfsetstrokecolor{currentstroke}%
\pgfsetdash{}{0pt}%
\pgfpathmoveto{\pgfqpoint{4.901936in}{1.621386in}}%
\pgfpathcurveto{\pgfqpoint{4.910173in}{1.621386in}}{\pgfqpoint{4.918073in}{1.624659in}}{\pgfqpoint{4.923897in}{1.630483in}}%
\pgfpathcurveto{\pgfqpoint{4.929721in}{1.636307in}}{\pgfqpoint{4.932993in}{1.644207in}}{\pgfqpoint{4.932993in}{1.652443in}}%
\pgfpathcurveto{\pgfqpoint{4.932993in}{1.660679in}}{\pgfqpoint{4.929721in}{1.668579in}}{\pgfqpoint{4.923897in}{1.674403in}}%
\pgfpathcurveto{\pgfqpoint{4.918073in}{1.680227in}}{\pgfqpoint{4.910173in}{1.683499in}}{\pgfqpoint{4.901936in}{1.683499in}}%
\pgfpathcurveto{\pgfqpoint{4.893700in}{1.683499in}}{\pgfqpoint{4.885800in}{1.680227in}}{\pgfqpoint{4.879976in}{1.674403in}}%
\pgfpathcurveto{\pgfqpoint{4.874152in}{1.668579in}}{\pgfqpoint{4.870880in}{1.660679in}}{\pgfqpoint{4.870880in}{1.652443in}}%
\pgfpathcurveto{\pgfqpoint{4.870880in}{1.644207in}}{\pgfqpoint{4.874152in}{1.636307in}}{\pgfqpoint{4.879976in}{1.630483in}}%
\pgfpathcurveto{\pgfqpoint{4.885800in}{1.624659in}}{\pgfqpoint{4.893700in}{1.621386in}}{\pgfqpoint{4.901936in}{1.621386in}}%
\pgfpathclose%
\pgfusepath{stroke,fill}%
\end{pgfscope}%
\begin{pgfscope}%
\pgfpathrectangle{\pgfqpoint{3.793912in}{0.557870in}}{\pgfqpoint{2.446088in}{1.684734in}}%
\pgfusepath{clip}%
\pgfsetbuttcap%
\pgfsetroundjoin%
\definecolor{currentfill}{rgb}{0.298039,0.447059,0.690196}%
\pgfsetfillcolor{currentfill}%
\pgfsetlinewidth{1.003750pt}%
\definecolor{currentstroke}{rgb}{0.298039,0.447059,0.690196}%
\pgfsetstrokecolor{currentstroke}%
\pgfsetdash{}{0pt}%
\pgfpathmoveto{\pgfqpoint{3.905098in}{2.125798in}}%
\pgfpathcurveto{\pgfqpoint{3.913334in}{2.125798in}}{\pgfqpoint{3.921234in}{2.129070in}}{\pgfqpoint{3.927058in}{2.134894in}}%
\pgfpathcurveto{\pgfqpoint{3.932882in}{2.140718in}}{\pgfqpoint{3.936155in}{2.148618in}}{\pgfqpoint{3.936155in}{2.156854in}}%
\pgfpathcurveto{\pgfqpoint{3.936155in}{2.165091in}}{\pgfqpoint{3.932882in}{2.172991in}}{\pgfqpoint{3.927058in}{2.178814in}}%
\pgfpathcurveto{\pgfqpoint{3.921234in}{2.184638in}}{\pgfqpoint{3.913334in}{2.187911in}}{\pgfqpoint{3.905098in}{2.187911in}}%
\pgfpathcurveto{\pgfqpoint{3.896862in}{2.187911in}}{\pgfqpoint{3.888962in}{2.184638in}}{\pgfqpoint{3.883138in}{2.178814in}}%
\pgfpathcurveto{\pgfqpoint{3.877314in}{2.172991in}}{\pgfqpoint{3.874042in}{2.165091in}}{\pgfqpoint{3.874042in}{2.156854in}}%
\pgfpathcurveto{\pgfqpoint{3.874042in}{2.148618in}}{\pgfqpoint{3.877314in}{2.140718in}}{\pgfqpoint{3.883138in}{2.134894in}}%
\pgfpathcurveto{\pgfqpoint{3.888962in}{2.129070in}}{\pgfqpoint{3.896862in}{2.125798in}}{\pgfqpoint{3.905098in}{2.125798in}}%
\pgfpathclose%
\pgfusepath{stroke,fill}%
\end{pgfscope}%
\begin{pgfscope}%
\pgfpathrectangle{\pgfqpoint{3.793912in}{0.557870in}}{\pgfqpoint{2.446088in}{1.684734in}}%
\pgfusepath{clip}%
\pgfsetbuttcap%
\pgfsetroundjoin%
\definecolor{currentfill}{rgb}{0.298039,0.447059,0.690196}%
\pgfsetfillcolor{currentfill}%
\pgfsetlinewidth{1.003750pt}%
\definecolor{currentstroke}{rgb}{0.298039,0.447059,0.690196}%
\pgfsetstrokecolor{currentstroke}%
\pgfsetdash{}{0pt}%
\pgfpathmoveto{\pgfqpoint{3.905098in}{2.125798in}}%
\pgfpathcurveto{\pgfqpoint{3.913334in}{2.125798in}}{\pgfqpoint{3.921234in}{2.129070in}}{\pgfqpoint{3.927058in}{2.134894in}}%
\pgfpathcurveto{\pgfqpoint{3.932882in}{2.140718in}}{\pgfqpoint{3.936155in}{2.148618in}}{\pgfqpoint{3.936155in}{2.156854in}}%
\pgfpathcurveto{\pgfqpoint{3.936155in}{2.165091in}}{\pgfqpoint{3.932882in}{2.172991in}}{\pgfqpoint{3.927058in}{2.178814in}}%
\pgfpathcurveto{\pgfqpoint{3.921234in}{2.184638in}}{\pgfqpoint{3.913334in}{2.187911in}}{\pgfqpoint{3.905098in}{2.187911in}}%
\pgfpathcurveto{\pgfqpoint{3.896862in}{2.187911in}}{\pgfqpoint{3.888962in}{2.184638in}}{\pgfqpoint{3.883138in}{2.178814in}}%
\pgfpathcurveto{\pgfqpoint{3.877314in}{2.172991in}}{\pgfqpoint{3.874042in}{2.165091in}}{\pgfqpoint{3.874042in}{2.156854in}}%
\pgfpathcurveto{\pgfqpoint{3.874042in}{2.148618in}}{\pgfqpoint{3.877314in}{2.140718in}}{\pgfqpoint{3.883138in}{2.134894in}}%
\pgfpathcurveto{\pgfqpoint{3.888962in}{2.129070in}}{\pgfqpoint{3.896862in}{2.125798in}}{\pgfqpoint{3.905098in}{2.125798in}}%
\pgfpathclose%
\pgfusepath{stroke,fill}%
\end{pgfscope}%
\begin{pgfscope}%
\pgfpathrectangle{\pgfqpoint{3.793912in}{0.557870in}}{\pgfqpoint{2.446088in}{1.684734in}}%
\pgfusepath{clip}%
\pgfsetbuttcap%
\pgfsetroundjoin%
\definecolor{currentfill}{rgb}{0.298039,0.447059,0.690196}%
\pgfsetfillcolor{currentfill}%
\pgfsetlinewidth{1.003750pt}%
\definecolor{currentstroke}{rgb}{0.298039,0.447059,0.690196}%
\pgfsetstrokecolor{currentstroke}%
\pgfsetdash{}{0pt}%
\pgfpathmoveto{\pgfqpoint{3.905098in}{2.125798in}}%
\pgfpathcurveto{\pgfqpoint{3.913334in}{2.125798in}}{\pgfqpoint{3.921234in}{2.129070in}}{\pgfqpoint{3.927058in}{2.134894in}}%
\pgfpathcurveto{\pgfqpoint{3.932882in}{2.140718in}}{\pgfqpoint{3.936155in}{2.148618in}}{\pgfqpoint{3.936155in}{2.156854in}}%
\pgfpathcurveto{\pgfqpoint{3.936155in}{2.165091in}}{\pgfqpoint{3.932882in}{2.172991in}}{\pgfqpoint{3.927058in}{2.178814in}}%
\pgfpathcurveto{\pgfqpoint{3.921234in}{2.184638in}}{\pgfqpoint{3.913334in}{2.187911in}}{\pgfqpoint{3.905098in}{2.187911in}}%
\pgfpathcurveto{\pgfqpoint{3.896862in}{2.187911in}}{\pgfqpoint{3.888962in}{2.184638in}}{\pgfqpoint{3.883138in}{2.178814in}}%
\pgfpathcurveto{\pgfqpoint{3.877314in}{2.172991in}}{\pgfqpoint{3.874042in}{2.165091in}}{\pgfqpoint{3.874042in}{2.156854in}}%
\pgfpathcurveto{\pgfqpoint{3.874042in}{2.148618in}}{\pgfqpoint{3.877314in}{2.140718in}}{\pgfqpoint{3.883138in}{2.134894in}}%
\pgfpathcurveto{\pgfqpoint{3.888962in}{2.129070in}}{\pgfqpoint{3.896862in}{2.125798in}}{\pgfqpoint{3.905098in}{2.125798in}}%
\pgfpathclose%
\pgfusepath{stroke,fill}%
\end{pgfscope}%
\begin{pgfscope}%
\pgfpathrectangle{\pgfqpoint{3.793912in}{0.557870in}}{\pgfqpoint{2.446088in}{1.684734in}}%
\pgfusepath{clip}%
\pgfsetbuttcap%
\pgfsetroundjoin%
\definecolor{currentfill}{rgb}{0.298039,0.447059,0.690196}%
\pgfsetfillcolor{currentfill}%
\pgfsetlinewidth{1.003750pt}%
\definecolor{currentstroke}{rgb}{0.298039,0.447059,0.690196}%
\pgfsetstrokecolor{currentstroke}%
\pgfsetdash{}{0pt}%
\pgfpathmoveto{\pgfqpoint{3.905098in}{2.125798in}}%
\pgfpathcurveto{\pgfqpoint{3.913334in}{2.125798in}}{\pgfqpoint{3.921234in}{2.129070in}}{\pgfqpoint{3.927058in}{2.134894in}}%
\pgfpathcurveto{\pgfqpoint{3.932882in}{2.140718in}}{\pgfqpoint{3.936155in}{2.148618in}}{\pgfqpoint{3.936155in}{2.156854in}}%
\pgfpathcurveto{\pgfqpoint{3.936155in}{2.165091in}}{\pgfqpoint{3.932882in}{2.172991in}}{\pgfqpoint{3.927058in}{2.178814in}}%
\pgfpathcurveto{\pgfqpoint{3.921234in}{2.184638in}}{\pgfqpoint{3.913334in}{2.187911in}}{\pgfqpoint{3.905098in}{2.187911in}}%
\pgfpathcurveto{\pgfqpoint{3.896862in}{2.187911in}}{\pgfqpoint{3.888962in}{2.184638in}}{\pgfqpoint{3.883138in}{2.178814in}}%
\pgfpathcurveto{\pgfqpoint{3.877314in}{2.172991in}}{\pgfqpoint{3.874042in}{2.165091in}}{\pgfqpoint{3.874042in}{2.156854in}}%
\pgfpathcurveto{\pgfqpoint{3.874042in}{2.148618in}}{\pgfqpoint{3.877314in}{2.140718in}}{\pgfqpoint{3.883138in}{2.134894in}}%
\pgfpathcurveto{\pgfqpoint{3.888962in}{2.129070in}}{\pgfqpoint{3.896862in}{2.125798in}}{\pgfqpoint{3.905098in}{2.125798in}}%
\pgfpathclose%
\pgfusepath{stroke,fill}%
\end{pgfscope}%
\begin{pgfscope}%
\pgfpathrectangle{\pgfqpoint{3.793912in}{0.557870in}}{\pgfqpoint{2.446088in}{1.684734in}}%
\pgfusepath{clip}%
\pgfsetbuttcap%
\pgfsetroundjoin%
\definecolor{currentfill}{rgb}{0.298039,0.447059,0.690196}%
\pgfsetfillcolor{currentfill}%
\pgfsetlinewidth{1.003750pt}%
\definecolor{currentstroke}{rgb}{0.298039,0.447059,0.690196}%
\pgfsetstrokecolor{currentstroke}%
\pgfsetdash{}{0pt}%
\pgfpathmoveto{\pgfqpoint{4.978616in}{1.502162in}}%
\pgfpathcurveto{\pgfqpoint{4.986852in}{1.502162in}}{\pgfqpoint{4.994753in}{1.505434in}}{\pgfqpoint{5.000576in}{1.511258in}}%
\pgfpathcurveto{\pgfqpoint{5.006400in}{1.517082in}}{\pgfqpoint{5.009673in}{1.524982in}}{\pgfqpoint{5.009673in}{1.533218in}}%
\pgfpathcurveto{\pgfqpoint{5.009673in}{1.541455in}}{\pgfqpoint{5.006400in}{1.549355in}}{\pgfqpoint{5.000576in}{1.555179in}}%
\pgfpathcurveto{\pgfqpoint{4.994753in}{1.561003in}}{\pgfqpoint{4.986852in}{1.564275in}}{\pgfqpoint{4.978616in}{1.564275in}}%
\pgfpathcurveto{\pgfqpoint{4.970380in}{1.564275in}}{\pgfqpoint{4.962480in}{1.561003in}}{\pgfqpoint{4.956656in}{1.555179in}}%
\pgfpathcurveto{\pgfqpoint{4.950832in}{1.549355in}}{\pgfqpoint{4.947560in}{1.541455in}}{\pgfqpoint{4.947560in}{1.533218in}}%
\pgfpathcurveto{\pgfqpoint{4.947560in}{1.524982in}}{\pgfqpoint{4.950832in}{1.517082in}}{\pgfqpoint{4.956656in}{1.511258in}}%
\pgfpathcurveto{\pgfqpoint{4.962480in}{1.505434in}}{\pgfqpoint{4.970380in}{1.502162in}}{\pgfqpoint{4.978616in}{1.502162in}}%
\pgfpathclose%
\pgfusepath{stroke,fill}%
\end{pgfscope}%
\begin{pgfscope}%
\pgfpathrectangle{\pgfqpoint{3.793912in}{0.557870in}}{\pgfqpoint{2.446088in}{1.684734in}}%
\pgfusepath{clip}%
\pgfsetbuttcap%
\pgfsetroundjoin%
\definecolor{currentfill}{rgb}{0.298039,0.447059,0.690196}%
\pgfsetfillcolor{currentfill}%
\pgfsetlinewidth{1.003750pt}%
\definecolor{currentstroke}{rgb}{0.298039,0.447059,0.690196}%
\pgfsetstrokecolor{currentstroke}%
\pgfsetdash{}{0pt}%
\pgfpathmoveto{\pgfqpoint{3.905098in}{2.125798in}}%
\pgfpathcurveto{\pgfqpoint{3.913334in}{2.125798in}}{\pgfqpoint{3.921234in}{2.129070in}}{\pgfqpoint{3.927058in}{2.134894in}}%
\pgfpathcurveto{\pgfqpoint{3.932882in}{2.140718in}}{\pgfqpoint{3.936155in}{2.148618in}}{\pgfqpoint{3.936155in}{2.156854in}}%
\pgfpathcurveto{\pgfqpoint{3.936155in}{2.165091in}}{\pgfqpoint{3.932882in}{2.172991in}}{\pgfqpoint{3.927058in}{2.178814in}}%
\pgfpathcurveto{\pgfqpoint{3.921234in}{2.184638in}}{\pgfqpoint{3.913334in}{2.187911in}}{\pgfqpoint{3.905098in}{2.187911in}}%
\pgfpathcurveto{\pgfqpoint{3.896862in}{2.187911in}}{\pgfqpoint{3.888962in}{2.184638in}}{\pgfqpoint{3.883138in}{2.178814in}}%
\pgfpathcurveto{\pgfqpoint{3.877314in}{2.172991in}}{\pgfqpoint{3.874042in}{2.165091in}}{\pgfqpoint{3.874042in}{2.156854in}}%
\pgfpathcurveto{\pgfqpoint{3.874042in}{2.148618in}}{\pgfqpoint{3.877314in}{2.140718in}}{\pgfqpoint{3.883138in}{2.134894in}}%
\pgfpathcurveto{\pgfqpoint{3.888962in}{2.129070in}}{\pgfqpoint{3.896862in}{2.125798in}}{\pgfqpoint{3.905098in}{2.125798in}}%
\pgfpathclose%
\pgfusepath{stroke,fill}%
\end{pgfscope}%
\begin{pgfscope}%
\pgfpathrectangle{\pgfqpoint{3.793912in}{0.557870in}}{\pgfqpoint{2.446088in}{1.684734in}}%
\pgfusepath{clip}%
\pgfsetbuttcap%
\pgfsetroundjoin%
\definecolor{currentfill}{rgb}{0.298039,0.447059,0.690196}%
\pgfsetfillcolor{currentfill}%
\pgfsetlinewidth{1.003750pt}%
\definecolor{currentstroke}{rgb}{0.298039,0.447059,0.690196}%
\pgfsetstrokecolor{currentstroke}%
\pgfsetdash{}{0pt}%
\pgfpathmoveto{\pgfqpoint{4.901936in}{1.676413in}}%
\pgfpathcurveto{\pgfqpoint{4.910173in}{1.676413in}}{\pgfqpoint{4.918073in}{1.679685in}}{\pgfqpoint{4.923897in}{1.685509in}}%
\pgfpathcurveto{\pgfqpoint{4.929721in}{1.691333in}}{\pgfqpoint{4.932993in}{1.699233in}}{\pgfqpoint{4.932993in}{1.707470in}}%
\pgfpathcurveto{\pgfqpoint{4.932993in}{1.715706in}}{\pgfqpoint{4.929721in}{1.723606in}}{\pgfqpoint{4.923897in}{1.729430in}}%
\pgfpathcurveto{\pgfqpoint{4.918073in}{1.735254in}}{\pgfqpoint{4.910173in}{1.738526in}}{\pgfqpoint{4.901936in}{1.738526in}}%
\pgfpathcurveto{\pgfqpoint{4.893700in}{1.738526in}}{\pgfqpoint{4.885800in}{1.735254in}}{\pgfqpoint{4.879976in}{1.729430in}}%
\pgfpathcurveto{\pgfqpoint{4.874152in}{1.723606in}}{\pgfqpoint{4.870880in}{1.715706in}}{\pgfqpoint{4.870880in}{1.707470in}}%
\pgfpathcurveto{\pgfqpoint{4.870880in}{1.699233in}}{\pgfqpoint{4.874152in}{1.691333in}}{\pgfqpoint{4.879976in}{1.685509in}}%
\pgfpathcurveto{\pgfqpoint{4.885800in}{1.679685in}}{\pgfqpoint{4.893700in}{1.676413in}}{\pgfqpoint{4.901936in}{1.676413in}}%
\pgfpathclose%
\pgfusepath{stroke,fill}%
\end{pgfscope}%
\begin{pgfscope}%
\pgfpathrectangle{\pgfqpoint{3.793912in}{0.557870in}}{\pgfqpoint{2.446088in}{1.684734in}}%
\pgfusepath{clip}%
\pgfsetbuttcap%
\pgfsetroundjoin%
\definecolor{currentfill}{rgb}{0.298039,0.447059,0.690196}%
\pgfsetfillcolor{currentfill}%
\pgfsetlinewidth{1.003750pt}%
\definecolor{currentstroke}{rgb}{0.298039,0.447059,0.690196}%
\pgfsetstrokecolor{currentstroke}%
\pgfsetdash{}{0pt}%
\pgfpathmoveto{\pgfqpoint{3.905098in}{2.125798in}}%
\pgfpathcurveto{\pgfqpoint{3.913334in}{2.125798in}}{\pgfqpoint{3.921234in}{2.129070in}}{\pgfqpoint{3.927058in}{2.134894in}}%
\pgfpathcurveto{\pgfqpoint{3.932882in}{2.140718in}}{\pgfqpoint{3.936155in}{2.148618in}}{\pgfqpoint{3.936155in}{2.156854in}}%
\pgfpathcurveto{\pgfqpoint{3.936155in}{2.165091in}}{\pgfqpoint{3.932882in}{2.172991in}}{\pgfqpoint{3.927058in}{2.178814in}}%
\pgfpathcurveto{\pgfqpoint{3.921234in}{2.184638in}}{\pgfqpoint{3.913334in}{2.187911in}}{\pgfqpoint{3.905098in}{2.187911in}}%
\pgfpathcurveto{\pgfqpoint{3.896862in}{2.187911in}}{\pgfqpoint{3.888962in}{2.184638in}}{\pgfqpoint{3.883138in}{2.178814in}}%
\pgfpathcurveto{\pgfqpoint{3.877314in}{2.172991in}}{\pgfqpoint{3.874042in}{2.165091in}}{\pgfqpoint{3.874042in}{2.156854in}}%
\pgfpathcurveto{\pgfqpoint{3.874042in}{2.148618in}}{\pgfqpoint{3.877314in}{2.140718in}}{\pgfqpoint{3.883138in}{2.134894in}}%
\pgfpathcurveto{\pgfqpoint{3.888962in}{2.129070in}}{\pgfqpoint{3.896862in}{2.125798in}}{\pgfqpoint{3.905098in}{2.125798in}}%
\pgfpathclose%
\pgfusepath{stroke,fill}%
\end{pgfscope}%
\begin{pgfscope}%
\pgfpathrectangle{\pgfqpoint{3.793912in}{0.557870in}}{\pgfqpoint{2.446088in}{1.684734in}}%
\pgfusepath{clip}%
\pgfsetbuttcap%
\pgfsetroundjoin%
\definecolor{currentfill}{rgb}{0.298039,0.447059,0.690196}%
\pgfsetfillcolor{currentfill}%
\pgfsetlinewidth{1.003750pt}%
\definecolor{currentstroke}{rgb}{0.298039,0.447059,0.690196}%
\pgfsetstrokecolor{currentstroke}%
\pgfsetdash{}{0pt}%
\pgfpathmoveto{\pgfqpoint{3.905098in}{2.125798in}}%
\pgfpathcurveto{\pgfqpoint{3.913334in}{2.125798in}}{\pgfqpoint{3.921234in}{2.129070in}}{\pgfqpoint{3.927058in}{2.134894in}}%
\pgfpathcurveto{\pgfqpoint{3.932882in}{2.140718in}}{\pgfqpoint{3.936155in}{2.148618in}}{\pgfqpoint{3.936155in}{2.156854in}}%
\pgfpathcurveto{\pgfqpoint{3.936155in}{2.165091in}}{\pgfqpoint{3.932882in}{2.172991in}}{\pgfqpoint{3.927058in}{2.178814in}}%
\pgfpathcurveto{\pgfqpoint{3.921234in}{2.184638in}}{\pgfqpoint{3.913334in}{2.187911in}}{\pgfqpoint{3.905098in}{2.187911in}}%
\pgfpathcurveto{\pgfqpoint{3.896862in}{2.187911in}}{\pgfqpoint{3.888962in}{2.184638in}}{\pgfqpoint{3.883138in}{2.178814in}}%
\pgfpathcurveto{\pgfqpoint{3.877314in}{2.172991in}}{\pgfqpoint{3.874042in}{2.165091in}}{\pgfqpoint{3.874042in}{2.156854in}}%
\pgfpathcurveto{\pgfqpoint{3.874042in}{2.148618in}}{\pgfqpoint{3.877314in}{2.140718in}}{\pgfqpoint{3.883138in}{2.134894in}}%
\pgfpathcurveto{\pgfqpoint{3.888962in}{2.129070in}}{\pgfqpoint{3.896862in}{2.125798in}}{\pgfqpoint{3.905098in}{2.125798in}}%
\pgfpathclose%
\pgfusepath{stroke,fill}%
\end{pgfscope}%
\begin{pgfscope}%
\pgfpathrectangle{\pgfqpoint{3.793912in}{0.557870in}}{\pgfqpoint{2.446088in}{1.684734in}}%
\pgfusepath{clip}%
\pgfsetbuttcap%
\pgfsetroundjoin%
\definecolor{currentfill}{rgb}{0.298039,0.447059,0.690196}%
\pgfsetfillcolor{currentfill}%
\pgfsetlinewidth{1.003750pt}%
\definecolor{currentstroke}{rgb}{0.298039,0.447059,0.690196}%
\pgfsetstrokecolor{currentstroke}%
\pgfsetdash{}{0pt}%
\pgfpathmoveto{\pgfqpoint{3.905098in}{2.125798in}}%
\pgfpathcurveto{\pgfqpoint{3.913334in}{2.125798in}}{\pgfqpoint{3.921234in}{2.129070in}}{\pgfqpoint{3.927058in}{2.134894in}}%
\pgfpathcurveto{\pgfqpoint{3.932882in}{2.140718in}}{\pgfqpoint{3.936155in}{2.148618in}}{\pgfqpoint{3.936155in}{2.156854in}}%
\pgfpathcurveto{\pgfqpoint{3.936155in}{2.165091in}}{\pgfqpoint{3.932882in}{2.172991in}}{\pgfqpoint{3.927058in}{2.178814in}}%
\pgfpathcurveto{\pgfqpoint{3.921234in}{2.184638in}}{\pgfqpoint{3.913334in}{2.187911in}}{\pgfqpoint{3.905098in}{2.187911in}}%
\pgfpathcurveto{\pgfqpoint{3.896862in}{2.187911in}}{\pgfqpoint{3.888962in}{2.184638in}}{\pgfqpoint{3.883138in}{2.178814in}}%
\pgfpathcurveto{\pgfqpoint{3.877314in}{2.172991in}}{\pgfqpoint{3.874042in}{2.165091in}}{\pgfqpoint{3.874042in}{2.156854in}}%
\pgfpathcurveto{\pgfqpoint{3.874042in}{2.148618in}}{\pgfqpoint{3.877314in}{2.140718in}}{\pgfqpoint{3.883138in}{2.134894in}}%
\pgfpathcurveto{\pgfqpoint{3.888962in}{2.129070in}}{\pgfqpoint{3.896862in}{2.125798in}}{\pgfqpoint{3.905098in}{2.125798in}}%
\pgfpathclose%
\pgfusepath{stroke,fill}%
\end{pgfscope}%
\begin{pgfscope}%
\pgfpathrectangle{\pgfqpoint{3.793912in}{0.557870in}}{\pgfqpoint{2.446088in}{1.684734in}}%
\pgfusepath{clip}%
\pgfsetbuttcap%
\pgfsetroundjoin%
\definecolor{currentfill}{rgb}{0.298039,0.447059,0.690196}%
\pgfsetfillcolor{currentfill}%
\pgfsetlinewidth{1.003750pt}%
\definecolor{currentstroke}{rgb}{0.298039,0.447059,0.690196}%
\pgfsetstrokecolor{currentstroke}%
\pgfsetdash{}{0pt}%
\pgfpathmoveto{\pgfqpoint{5.898775in}{1.557189in}}%
\pgfpathcurveto{\pgfqpoint{5.907011in}{1.557189in}}{\pgfqpoint{5.914911in}{1.560461in}}{\pgfqpoint{5.920735in}{1.566285in}}%
\pgfpathcurveto{\pgfqpoint{5.926559in}{1.572109in}}{\pgfqpoint{5.929831in}{1.580009in}}{\pgfqpoint{5.929831in}{1.588245in}}%
\pgfpathcurveto{\pgfqpoint{5.929831in}{1.596481in}}{\pgfqpoint{5.926559in}{1.604381in}}{\pgfqpoint{5.920735in}{1.610205in}}%
\pgfpathcurveto{\pgfqpoint{5.914911in}{1.616029in}}{\pgfqpoint{5.907011in}{1.619302in}}{\pgfqpoint{5.898775in}{1.619302in}}%
\pgfpathcurveto{\pgfqpoint{5.890538in}{1.619302in}}{\pgfqpoint{5.882638in}{1.616029in}}{\pgfqpoint{5.876814in}{1.610205in}}%
\pgfpathcurveto{\pgfqpoint{5.870990in}{1.604381in}}{\pgfqpoint{5.867718in}{1.596481in}}{\pgfqpoint{5.867718in}{1.588245in}}%
\pgfpathcurveto{\pgfqpoint{5.867718in}{1.580009in}}{\pgfqpoint{5.870990in}{1.572109in}}{\pgfqpoint{5.876814in}{1.566285in}}%
\pgfpathcurveto{\pgfqpoint{5.882638in}{1.560461in}}{\pgfqpoint{5.890538in}{1.557189in}}{\pgfqpoint{5.898775in}{1.557189in}}%
\pgfpathclose%
\pgfusepath{stroke,fill}%
\end{pgfscope}%
\begin{pgfscope}%
\pgfpathrectangle{\pgfqpoint{3.793912in}{0.557870in}}{\pgfqpoint{2.446088in}{1.684734in}}%
\pgfusepath{clip}%
\pgfsetbuttcap%
\pgfsetroundjoin%
\definecolor{currentfill}{rgb}{0.298039,0.447059,0.690196}%
\pgfsetfillcolor{currentfill}%
\pgfsetlinewidth{1.003750pt}%
\definecolor{currentstroke}{rgb}{0.298039,0.447059,0.690196}%
\pgfsetstrokecolor{currentstroke}%
\pgfsetdash{}{0pt}%
\pgfpathmoveto{\pgfqpoint{4.058458in}{1.621386in}}%
\pgfpathcurveto{\pgfqpoint{4.066694in}{1.621386in}}{\pgfqpoint{4.074594in}{1.624659in}}{\pgfqpoint{4.080418in}{1.630483in}}%
\pgfpathcurveto{\pgfqpoint{4.086242in}{1.636307in}}{\pgfqpoint{4.089514in}{1.644207in}}{\pgfqpoint{4.089514in}{1.652443in}}%
\pgfpathcurveto{\pgfqpoint{4.089514in}{1.660679in}}{\pgfqpoint{4.086242in}{1.668579in}}{\pgfqpoint{4.080418in}{1.674403in}}%
\pgfpathcurveto{\pgfqpoint{4.074594in}{1.680227in}}{\pgfqpoint{4.066694in}{1.683499in}}{\pgfqpoint{4.058458in}{1.683499in}}%
\pgfpathcurveto{\pgfqpoint{4.050221in}{1.683499in}}{\pgfqpoint{4.042321in}{1.680227in}}{\pgfqpoint{4.036498in}{1.674403in}}%
\pgfpathcurveto{\pgfqpoint{4.030674in}{1.668579in}}{\pgfqpoint{4.027401in}{1.660679in}}{\pgfqpoint{4.027401in}{1.652443in}}%
\pgfpathcurveto{\pgfqpoint{4.027401in}{1.644207in}}{\pgfqpoint{4.030674in}{1.636307in}}{\pgfqpoint{4.036498in}{1.630483in}}%
\pgfpathcurveto{\pgfqpoint{4.042321in}{1.624659in}}{\pgfqpoint{4.050221in}{1.621386in}}{\pgfqpoint{4.058458in}{1.621386in}}%
\pgfpathclose%
\pgfusepath{stroke,fill}%
\end{pgfscope}%
\begin{pgfscope}%
\pgfpathrectangle{\pgfqpoint{3.793912in}{0.557870in}}{\pgfqpoint{2.446088in}{1.684734in}}%
\pgfusepath{clip}%
\pgfsetbuttcap%
\pgfsetroundjoin%
\definecolor{currentfill}{rgb}{0.298039,0.447059,0.690196}%
\pgfsetfillcolor{currentfill}%
\pgfsetlinewidth{1.003750pt}%
\definecolor{currentstroke}{rgb}{0.298039,0.447059,0.690196}%
\pgfsetstrokecolor{currentstroke}%
\pgfsetdash{}{0pt}%
\pgfpathmoveto{\pgfqpoint{5.898775in}{1.584702in}}%
\pgfpathcurveto{\pgfqpoint{5.907011in}{1.584702in}}{\pgfqpoint{5.914911in}{1.587974in}}{\pgfqpoint{5.920735in}{1.593798in}}%
\pgfpathcurveto{\pgfqpoint{5.926559in}{1.599622in}}{\pgfqpoint{5.929831in}{1.607522in}}{\pgfqpoint{5.929831in}{1.615758in}}%
\pgfpathcurveto{\pgfqpoint{5.929831in}{1.623995in}}{\pgfqpoint{5.926559in}{1.631895in}}{\pgfqpoint{5.920735in}{1.637719in}}%
\pgfpathcurveto{\pgfqpoint{5.914911in}{1.643543in}}{\pgfqpoint{5.907011in}{1.646815in}}{\pgfqpoint{5.898775in}{1.646815in}}%
\pgfpathcurveto{\pgfqpoint{5.890538in}{1.646815in}}{\pgfqpoint{5.882638in}{1.643543in}}{\pgfqpoint{5.876814in}{1.637719in}}%
\pgfpathcurveto{\pgfqpoint{5.870990in}{1.631895in}}{\pgfqpoint{5.867718in}{1.623995in}}{\pgfqpoint{5.867718in}{1.615758in}}%
\pgfpathcurveto{\pgfqpoint{5.867718in}{1.607522in}}{\pgfqpoint{5.870990in}{1.599622in}}{\pgfqpoint{5.876814in}{1.593798in}}%
\pgfpathcurveto{\pgfqpoint{5.882638in}{1.587974in}}{\pgfqpoint{5.890538in}{1.584702in}}{\pgfqpoint{5.898775in}{1.584702in}}%
\pgfpathclose%
\pgfusepath{stroke,fill}%
\end{pgfscope}%
\begin{pgfscope}%
\pgfpathrectangle{\pgfqpoint{3.793912in}{0.557870in}}{\pgfqpoint{2.446088in}{1.684734in}}%
\pgfusepath{clip}%
\pgfsetbuttcap%
\pgfsetroundjoin%
\definecolor{currentfill}{rgb}{0.298039,0.447059,0.690196}%
\pgfsetfillcolor{currentfill}%
\pgfsetlinewidth{1.003750pt}%
\definecolor{currentstroke}{rgb}{0.298039,0.447059,0.690196}%
\pgfsetstrokecolor{currentstroke}%
\pgfsetdash{}{0pt}%
\pgfpathmoveto{\pgfqpoint{4.058458in}{1.878178in}}%
\pgfpathcurveto{\pgfqpoint{4.066694in}{1.878178in}}{\pgfqpoint{4.074594in}{1.881450in}}{\pgfqpoint{4.080418in}{1.887274in}}%
\pgfpathcurveto{\pgfqpoint{4.086242in}{1.893098in}}{\pgfqpoint{4.089514in}{1.900998in}}{\pgfqpoint{4.089514in}{1.909234in}}%
\pgfpathcurveto{\pgfqpoint{4.089514in}{1.917470in}}{\pgfqpoint{4.086242in}{1.925370in}}{\pgfqpoint{4.080418in}{1.931194in}}%
\pgfpathcurveto{\pgfqpoint{4.074594in}{1.937018in}}{\pgfqpoint{4.066694in}{1.940291in}}{\pgfqpoint{4.058458in}{1.940291in}}%
\pgfpathcurveto{\pgfqpoint{4.050221in}{1.940291in}}{\pgfqpoint{4.042321in}{1.937018in}}{\pgfqpoint{4.036498in}{1.931194in}}%
\pgfpathcurveto{\pgfqpoint{4.030674in}{1.925370in}}{\pgfqpoint{4.027401in}{1.917470in}}{\pgfqpoint{4.027401in}{1.909234in}}%
\pgfpathcurveto{\pgfqpoint{4.027401in}{1.900998in}}{\pgfqpoint{4.030674in}{1.893098in}}{\pgfqpoint{4.036498in}{1.887274in}}%
\pgfpathcurveto{\pgfqpoint{4.042321in}{1.881450in}}{\pgfqpoint{4.050221in}{1.878178in}}{\pgfqpoint{4.058458in}{1.878178in}}%
\pgfpathclose%
\pgfusepath{stroke,fill}%
\end{pgfscope}%
\begin{pgfscope}%
\pgfpathrectangle{\pgfqpoint{3.793912in}{0.557870in}}{\pgfqpoint{2.446088in}{1.684734in}}%
\pgfusepath{clip}%
\pgfsetbuttcap%
\pgfsetroundjoin%
\definecolor{currentfill}{rgb}{0.298039,0.447059,0.690196}%
\pgfsetfillcolor{currentfill}%
\pgfsetlinewidth{1.003750pt}%
\definecolor{currentstroke}{rgb}{0.298039,0.447059,0.690196}%
\pgfsetstrokecolor{currentstroke}%
\pgfsetdash{}{0pt}%
\pgfpathmoveto{\pgfqpoint{4.058458in}{1.878178in}}%
\pgfpathcurveto{\pgfqpoint{4.066694in}{1.878178in}}{\pgfqpoint{4.074594in}{1.881450in}}{\pgfqpoint{4.080418in}{1.887274in}}%
\pgfpathcurveto{\pgfqpoint{4.086242in}{1.893098in}}{\pgfqpoint{4.089514in}{1.900998in}}{\pgfqpoint{4.089514in}{1.909234in}}%
\pgfpathcurveto{\pgfqpoint{4.089514in}{1.917470in}}{\pgfqpoint{4.086242in}{1.925370in}}{\pgfqpoint{4.080418in}{1.931194in}}%
\pgfpathcurveto{\pgfqpoint{4.074594in}{1.937018in}}{\pgfqpoint{4.066694in}{1.940291in}}{\pgfqpoint{4.058458in}{1.940291in}}%
\pgfpathcurveto{\pgfqpoint{4.050221in}{1.940291in}}{\pgfqpoint{4.042321in}{1.937018in}}{\pgfqpoint{4.036498in}{1.931194in}}%
\pgfpathcurveto{\pgfqpoint{4.030674in}{1.925370in}}{\pgfqpoint{4.027401in}{1.917470in}}{\pgfqpoint{4.027401in}{1.909234in}}%
\pgfpathcurveto{\pgfqpoint{4.027401in}{1.900998in}}{\pgfqpoint{4.030674in}{1.893098in}}{\pgfqpoint{4.036498in}{1.887274in}}%
\pgfpathcurveto{\pgfqpoint{4.042321in}{1.881450in}}{\pgfqpoint{4.050221in}{1.878178in}}{\pgfqpoint{4.058458in}{1.878178in}}%
\pgfpathclose%
\pgfusepath{stroke,fill}%
\end{pgfscope}%
\begin{pgfscope}%
\pgfpathrectangle{\pgfqpoint{3.793912in}{0.557870in}}{\pgfqpoint{2.446088in}{1.684734in}}%
\pgfusepath{clip}%
\pgfsetbuttcap%
\pgfsetroundjoin%
\definecolor{currentfill}{rgb}{0.298039,0.447059,0.690196}%
\pgfsetfillcolor{currentfill}%
\pgfsetlinewidth{1.003750pt}%
\definecolor{currentstroke}{rgb}{0.298039,0.447059,0.690196}%
\pgfsetstrokecolor{currentstroke}%
\pgfsetdash{}{0pt}%
\pgfpathmoveto{\pgfqpoint{5.975454in}{1.538846in}}%
\pgfpathcurveto{\pgfqpoint{5.983691in}{1.538846in}}{\pgfqpoint{5.991591in}{1.542119in}}{\pgfqpoint{5.997415in}{1.547943in}}%
\pgfpathcurveto{\pgfqpoint{6.003239in}{1.553767in}}{\pgfqpoint{6.006511in}{1.561667in}}{\pgfqpoint{6.006511in}{1.569903in}}%
\pgfpathcurveto{\pgfqpoint{6.006511in}{1.578139in}}{\pgfqpoint{6.003239in}{1.586039in}}{\pgfqpoint{5.997415in}{1.591863in}}%
\pgfpathcurveto{\pgfqpoint{5.991591in}{1.597687in}}{\pgfqpoint{5.983691in}{1.600959in}}{\pgfqpoint{5.975454in}{1.600959in}}%
\pgfpathcurveto{\pgfqpoint{5.967218in}{1.600959in}}{\pgfqpoint{5.959318in}{1.597687in}}{\pgfqpoint{5.953494in}{1.591863in}}%
\pgfpathcurveto{\pgfqpoint{5.947670in}{1.586039in}}{\pgfqpoint{5.944398in}{1.578139in}}{\pgfqpoint{5.944398in}{1.569903in}}%
\pgfpathcurveto{\pgfqpoint{5.944398in}{1.561667in}}{\pgfqpoint{5.947670in}{1.553767in}}{\pgfqpoint{5.953494in}{1.547943in}}%
\pgfpathcurveto{\pgfqpoint{5.959318in}{1.542119in}}{\pgfqpoint{5.967218in}{1.538846in}}{\pgfqpoint{5.975454in}{1.538846in}}%
\pgfpathclose%
\pgfusepath{stroke,fill}%
\end{pgfscope}%
\begin{pgfscope}%
\pgfpathrectangle{\pgfqpoint{3.793912in}{0.557870in}}{\pgfqpoint{2.446088in}{1.684734in}}%
\pgfusepath{clip}%
\pgfsetbuttcap%
\pgfsetroundjoin%
\definecolor{currentfill}{rgb}{0.298039,0.447059,0.690196}%
\pgfsetfillcolor{currentfill}%
\pgfsetlinewidth{1.003750pt}%
\definecolor{currentstroke}{rgb}{0.298039,0.447059,0.690196}%
\pgfsetstrokecolor{currentstroke}%
\pgfsetdash{}{0pt}%
\pgfpathmoveto{\pgfqpoint{5.975454in}{1.667242in}}%
\pgfpathcurveto{\pgfqpoint{5.983691in}{1.667242in}}{\pgfqpoint{5.991591in}{1.670514in}}{\pgfqpoint{5.997415in}{1.676338in}}%
\pgfpathcurveto{\pgfqpoint{6.003239in}{1.682162in}}{\pgfqpoint{6.006511in}{1.690062in}}{\pgfqpoint{6.006511in}{1.698298in}}%
\pgfpathcurveto{\pgfqpoint{6.006511in}{1.706535in}}{\pgfqpoint{6.003239in}{1.714435in}}{\pgfqpoint{5.997415in}{1.720259in}}%
\pgfpathcurveto{\pgfqpoint{5.991591in}{1.726083in}}{\pgfqpoint{5.983691in}{1.729355in}}{\pgfqpoint{5.975454in}{1.729355in}}%
\pgfpathcurveto{\pgfqpoint{5.967218in}{1.729355in}}{\pgfqpoint{5.959318in}{1.726083in}}{\pgfqpoint{5.953494in}{1.720259in}}%
\pgfpathcurveto{\pgfqpoint{5.947670in}{1.714435in}}{\pgfqpoint{5.944398in}{1.706535in}}{\pgfqpoint{5.944398in}{1.698298in}}%
\pgfpathcurveto{\pgfqpoint{5.944398in}{1.690062in}}{\pgfqpoint{5.947670in}{1.682162in}}{\pgfqpoint{5.953494in}{1.676338in}}%
\pgfpathcurveto{\pgfqpoint{5.959318in}{1.670514in}}{\pgfqpoint{5.967218in}{1.667242in}}{\pgfqpoint{5.975454in}{1.667242in}}%
\pgfpathclose%
\pgfusepath{stroke,fill}%
\end{pgfscope}%
\begin{pgfscope}%
\pgfpathrectangle{\pgfqpoint{3.793912in}{0.557870in}}{\pgfqpoint{2.446088in}{1.684734in}}%
\pgfusepath{clip}%
\pgfsetbuttcap%
\pgfsetroundjoin%
\definecolor{currentfill}{rgb}{0.298039,0.447059,0.690196}%
\pgfsetfillcolor{currentfill}%
\pgfsetlinewidth{1.003750pt}%
\definecolor{currentstroke}{rgb}{0.298039,0.447059,0.690196}%
\pgfsetstrokecolor{currentstroke}%
\pgfsetdash{}{0pt}%
\pgfpathmoveto{\pgfqpoint{5.975454in}{1.630558in}}%
\pgfpathcurveto{\pgfqpoint{5.983691in}{1.630558in}}{\pgfqpoint{5.991591in}{1.633830in}}{\pgfqpoint{5.997415in}{1.639654in}}%
\pgfpathcurveto{\pgfqpoint{6.003239in}{1.645478in}}{\pgfqpoint{6.006511in}{1.653378in}}{\pgfqpoint{6.006511in}{1.661614in}}%
\pgfpathcurveto{\pgfqpoint{6.006511in}{1.669850in}}{\pgfqpoint{6.003239in}{1.677750in}}{\pgfqpoint{5.997415in}{1.683574in}}%
\pgfpathcurveto{\pgfqpoint{5.991591in}{1.689398in}}{\pgfqpoint{5.983691in}{1.692671in}}{\pgfqpoint{5.975454in}{1.692671in}}%
\pgfpathcurveto{\pgfqpoint{5.967218in}{1.692671in}}{\pgfqpoint{5.959318in}{1.689398in}}{\pgfqpoint{5.953494in}{1.683574in}}%
\pgfpathcurveto{\pgfqpoint{5.947670in}{1.677750in}}{\pgfqpoint{5.944398in}{1.669850in}}{\pgfqpoint{5.944398in}{1.661614in}}%
\pgfpathcurveto{\pgfqpoint{5.944398in}{1.653378in}}{\pgfqpoint{5.947670in}{1.645478in}}{\pgfqpoint{5.953494in}{1.639654in}}%
\pgfpathcurveto{\pgfqpoint{5.959318in}{1.633830in}}{\pgfqpoint{5.967218in}{1.630558in}}{\pgfqpoint{5.975454in}{1.630558in}}%
\pgfpathclose%
\pgfusepath{stroke,fill}%
\end{pgfscope}%
\begin{pgfscope}%
\pgfpathrectangle{\pgfqpoint{3.793912in}{0.557870in}}{\pgfqpoint{2.446088in}{1.684734in}}%
\pgfusepath{clip}%
\pgfsetbuttcap%
\pgfsetroundjoin%
\definecolor{currentfill}{rgb}{0.298039,0.447059,0.690196}%
\pgfsetfillcolor{currentfill}%
\pgfsetlinewidth{1.003750pt}%
\definecolor{currentstroke}{rgb}{0.298039,0.447059,0.690196}%
\pgfsetstrokecolor{currentstroke}%
\pgfsetdash{}{0pt}%
\pgfpathmoveto{\pgfqpoint{5.975454in}{1.694755in}}%
\pgfpathcurveto{\pgfqpoint{5.983691in}{1.694755in}}{\pgfqpoint{5.991591in}{1.698028in}}{\pgfqpoint{5.997415in}{1.703852in}}%
\pgfpathcurveto{\pgfqpoint{6.003239in}{1.709675in}}{\pgfqpoint{6.006511in}{1.717576in}}{\pgfqpoint{6.006511in}{1.725812in}}%
\pgfpathcurveto{\pgfqpoint{6.006511in}{1.734048in}}{\pgfqpoint{6.003239in}{1.741948in}}{\pgfqpoint{5.997415in}{1.747772in}}%
\pgfpathcurveto{\pgfqpoint{5.991591in}{1.753596in}}{\pgfqpoint{5.983691in}{1.756868in}}{\pgfqpoint{5.975454in}{1.756868in}}%
\pgfpathcurveto{\pgfqpoint{5.967218in}{1.756868in}}{\pgfqpoint{5.959318in}{1.753596in}}{\pgfqpoint{5.953494in}{1.747772in}}%
\pgfpathcurveto{\pgfqpoint{5.947670in}{1.741948in}}{\pgfqpoint{5.944398in}{1.734048in}}{\pgfqpoint{5.944398in}{1.725812in}}%
\pgfpathcurveto{\pgfqpoint{5.944398in}{1.717576in}}{\pgfqpoint{5.947670in}{1.709675in}}{\pgfqpoint{5.953494in}{1.703852in}}%
\pgfpathcurveto{\pgfqpoint{5.959318in}{1.698028in}}{\pgfqpoint{5.967218in}{1.694755in}}{\pgfqpoint{5.975454in}{1.694755in}}%
\pgfpathclose%
\pgfusepath{stroke,fill}%
\end{pgfscope}%
\begin{pgfscope}%
\pgfpathrectangle{\pgfqpoint{3.793912in}{0.557870in}}{\pgfqpoint{2.446088in}{1.684734in}}%
\pgfusepath{clip}%
\pgfsetbuttcap%
\pgfsetroundjoin%
\definecolor{currentfill}{rgb}{0.298039,0.447059,0.690196}%
\pgfsetfillcolor{currentfill}%
\pgfsetlinewidth{1.003750pt}%
\definecolor{currentstroke}{rgb}{0.298039,0.447059,0.690196}%
\pgfsetstrokecolor{currentstroke}%
\pgfsetdash{}{0pt}%
\pgfpathmoveto{\pgfqpoint{5.975454in}{1.318740in}}%
\pgfpathcurveto{\pgfqpoint{5.983691in}{1.318740in}}{\pgfqpoint{5.991591in}{1.322012in}}{\pgfqpoint{5.997415in}{1.327836in}}%
\pgfpathcurveto{\pgfqpoint{6.003239in}{1.333660in}}{\pgfqpoint{6.006511in}{1.341560in}}{\pgfqpoint{6.006511in}{1.349796in}}%
\pgfpathcurveto{\pgfqpoint{6.006511in}{1.358032in}}{\pgfqpoint{6.003239in}{1.365932in}}{\pgfqpoint{5.997415in}{1.371756in}}%
\pgfpathcurveto{\pgfqpoint{5.991591in}{1.377580in}}{\pgfqpoint{5.983691in}{1.380853in}}{\pgfqpoint{5.975454in}{1.380853in}}%
\pgfpathcurveto{\pgfqpoint{5.967218in}{1.380853in}}{\pgfqpoint{5.959318in}{1.377580in}}{\pgfqpoint{5.953494in}{1.371756in}}%
\pgfpathcurveto{\pgfqpoint{5.947670in}{1.365932in}}{\pgfqpoint{5.944398in}{1.358032in}}{\pgfqpoint{5.944398in}{1.349796in}}%
\pgfpathcurveto{\pgfqpoint{5.944398in}{1.341560in}}{\pgfqpoint{5.947670in}{1.333660in}}{\pgfqpoint{5.953494in}{1.327836in}}%
\pgfpathcurveto{\pgfqpoint{5.959318in}{1.322012in}}{\pgfqpoint{5.967218in}{1.318740in}}{\pgfqpoint{5.975454in}{1.318740in}}%
\pgfpathclose%
\pgfusepath{stroke,fill}%
\end{pgfscope}%
\begin{pgfscope}%
\pgfpathrectangle{\pgfqpoint{3.793912in}{0.557870in}}{\pgfqpoint{2.446088in}{1.684734in}}%
\pgfusepath{clip}%
\pgfsetbuttcap%
\pgfsetroundjoin%
\definecolor{currentfill}{rgb}{0.298039,0.447059,0.690196}%
\pgfsetfillcolor{currentfill}%
\pgfsetlinewidth{1.003750pt}%
\definecolor{currentstroke}{rgb}{0.298039,0.447059,0.690196}%
\pgfsetstrokecolor{currentstroke}%
\pgfsetdash{}{0pt}%
\pgfpathmoveto{\pgfqpoint{5.975454in}{1.318740in}}%
\pgfpathcurveto{\pgfqpoint{5.983691in}{1.318740in}}{\pgfqpoint{5.991591in}{1.322012in}}{\pgfqpoint{5.997415in}{1.327836in}}%
\pgfpathcurveto{\pgfqpoint{6.003239in}{1.333660in}}{\pgfqpoint{6.006511in}{1.341560in}}{\pgfqpoint{6.006511in}{1.349796in}}%
\pgfpathcurveto{\pgfqpoint{6.006511in}{1.358032in}}{\pgfqpoint{6.003239in}{1.365932in}}{\pgfqpoint{5.997415in}{1.371756in}}%
\pgfpathcurveto{\pgfqpoint{5.991591in}{1.377580in}}{\pgfqpoint{5.983691in}{1.380853in}}{\pgfqpoint{5.975454in}{1.380853in}}%
\pgfpathcurveto{\pgfqpoint{5.967218in}{1.380853in}}{\pgfqpoint{5.959318in}{1.377580in}}{\pgfqpoint{5.953494in}{1.371756in}}%
\pgfpathcurveto{\pgfqpoint{5.947670in}{1.365932in}}{\pgfqpoint{5.944398in}{1.358032in}}{\pgfqpoint{5.944398in}{1.349796in}}%
\pgfpathcurveto{\pgfqpoint{5.944398in}{1.341560in}}{\pgfqpoint{5.947670in}{1.333660in}}{\pgfqpoint{5.953494in}{1.327836in}}%
\pgfpathcurveto{\pgfqpoint{5.959318in}{1.322012in}}{\pgfqpoint{5.967218in}{1.318740in}}{\pgfqpoint{5.975454in}{1.318740in}}%
\pgfpathclose%
\pgfusepath{stroke,fill}%
\end{pgfscope}%
\begin{pgfscope}%
\pgfpathrectangle{\pgfqpoint{3.793912in}{0.557870in}}{\pgfqpoint{2.446088in}{1.684734in}}%
\pgfusepath{clip}%
\pgfsetbuttcap%
\pgfsetroundjoin%
\definecolor{currentfill}{rgb}{0.298039,0.447059,0.690196}%
\pgfsetfillcolor{currentfill}%
\pgfsetlinewidth{1.003750pt}%
\definecolor{currentstroke}{rgb}{0.298039,0.447059,0.690196}%
\pgfsetstrokecolor{currentstroke}%
\pgfsetdash{}{0pt}%
\pgfpathmoveto{\pgfqpoint{3.905098in}{2.125798in}}%
\pgfpathcurveto{\pgfqpoint{3.913334in}{2.125798in}}{\pgfqpoint{3.921234in}{2.129070in}}{\pgfqpoint{3.927058in}{2.134894in}}%
\pgfpathcurveto{\pgfqpoint{3.932882in}{2.140718in}}{\pgfqpoint{3.936155in}{2.148618in}}{\pgfqpoint{3.936155in}{2.156854in}}%
\pgfpathcurveto{\pgfqpoint{3.936155in}{2.165091in}}{\pgfqpoint{3.932882in}{2.172991in}}{\pgfqpoint{3.927058in}{2.178814in}}%
\pgfpathcurveto{\pgfqpoint{3.921234in}{2.184638in}}{\pgfqpoint{3.913334in}{2.187911in}}{\pgfqpoint{3.905098in}{2.187911in}}%
\pgfpathcurveto{\pgfqpoint{3.896862in}{2.187911in}}{\pgfqpoint{3.888962in}{2.184638in}}{\pgfqpoint{3.883138in}{2.178814in}}%
\pgfpathcurveto{\pgfqpoint{3.877314in}{2.172991in}}{\pgfqpoint{3.874042in}{2.165091in}}{\pgfqpoint{3.874042in}{2.156854in}}%
\pgfpathcurveto{\pgfqpoint{3.874042in}{2.148618in}}{\pgfqpoint{3.877314in}{2.140718in}}{\pgfqpoint{3.883138in}{2.134894in}}%
\pgfpathcurveto{\pgfqpoint{3.888962in}{2.129070in}}{\pgfqpoint{3.896862in}{2.125798in}}{\pgfqpoint{3.905098in}{2.125798in}}%
\pgfpathclose%
\pgfusepath{stroke,fill}%
\end{pgfscope}%
\begin{pgfscope}%
\pgfpathrectangle{\pgfqpoint{3.793912in}{0.557870in}}{\pgfqpoint{2.446088in}{1.684734in}}%
\pgfusepath{clip}%
\pgfsetbuttcap%
\pgfsetroundjoin%
\definecolor{currentfill}{rgb}{0.298039,0.447059,0.690196}%
\pgfsetfillcolor{currentfill}%
\pgfsetlinewidth{1.003750pt}%
\definecolor{currentstroke}{rgb}{0.298039,0.447059,0.690196}%
\pgfsetstrokecolor{currentstroke}%
\pgfsetdash{}{0pt}%
\pgfpathmoveto{\pgfqpoint{3.905098in}{2.125798in}}%
\pgfpathcurveto{\pgfqpoint{3.913334in}{2.125798in}}{\pgfqpoint{3.921234in}{2.129070in}}{\pgfqpoint{3.927058in}{2.134894in}}%
\pgfpathcurveto{\pgfqpoint{3.932882in}{2.140718in}}{\pgfqpoint{3.936155in}{2.148618in}}{\pgfqpoint{3.936155in}{2.156854in}}%
\pgfpathcurveto{\pgfqpoint{3.936155in}{2.165091in}}{\pgfqpoint{3.932882in}{2.172991in}}{\pgfqpoint{3.927058in}{2.178814in}}%
\pgfpathcurveto{\pgfqpoint{3.921234in}{2.184638in}}{\pgfqpoint{3.913334in}{2.187911in}}{\pgfqpoint{3.905098in}{2.187911in}}%
\pgfpathcurveto{\pgfqpoint{3.896862in}{2.187911in}}{\pgfqpoint{3.888962in}{2.184638in}}{\pgfqpoint{3.883138in}{2.178814in}}%
\pgfpathcurveto{\pgfqpoint{3.877314in}{2.172991in}}{\pgfqpoint{3.874042in}{2.165091in}}{\pgfqpoint{3.874042in}{2.156854in}}%
\pgfpathcurveto{\pgfqpoint{3.874042in}{2.148618in}}{\pgfqpoint{3.877314in}{2.140718in}}{\pgfqpoint{3.883138in}{2.134894in}}%
\pgfpathcurveto{\pgfqpoint{3.888962in}{2.129070in}}{\pgfqpoint{3.896862in}{2.125798in}}{\pgfqpoint{3.905098in}{2.125798in}}%
\pgfpathclose%
\pgfusepath{stroke,fill}%
\end{pgfscope}%
\begin{pgfscope}%
\pgfpathrectangle{\pgfqpoint{3.793912in}{0.557870in}}{\pgfqpoint{2.446088in}{1.684734in}}%
\pgfusepath{clip}%
\pgfsetbuttcap%
\pgfsetroundjoin%
\definecolor{currentfill}{rgb}{0.298039,0.447059,0.690196}%
\pgfsetfillcolor{currentfill}%
\pgfsetlinewidth{1.003750pt}%
\definecolor{currentstroke}{rgb}{0.298039,0.447059,0.690196}%
\pgfsetstrokecolor{currentstroke}%
\pgfsetdash{}{0pt}%
\pgfpathmoveto{\pgfqpoint{3.905098in}{2.125798in}}%
\pgfpathcurveto{\pgfqpoint{3.913334in}{2.125798in}}{\pgfqpoint{3.921234in}{2.129070in}}{\pgfqpoint{3.927058in}{2.134894in}}%
\pgfpathcurveto{\pgfqpoint{3.932882in}{2.140718in}}{\pgfqpoint{3.936155in}{2.148618in}}{\pgfqpoint{3.936155in}{2.156854in}}%
\pgfpathcurveto{\pgfqpoint{3.936155in}{2.165091in}}{\pgfqpoint{3.932882in}{2.172991in}}{\pgfqpoint{3.927058in}{2.178814in}}%
\pgfpathcurveto{\pgfqpoint{3.921234in}{2.184638in}}{\pgfqpoint{3.913334in}{2.187911in}}{\pgfqpoint{3.905098in}{2.187911in}}%
\pgfpathcurveto{\pgfqpoint{3.896862in}{2.187911in}}{\pgfqpoint{3.888962in}{2.184638in}}{\pgfqpoint{3.883138in}{2.178814in}}%
\pgfpathcurveto{\pgfqpoint{3.877314in}{2.172991in}}{\pgfqpoint{3.874042in}{2.165091in}}{\pgfqpoint{3.874042in}{2.156854in}}%
\pgfpathcurveto{\pgfqpoint{3.874042in}{2.148618in}}{\pgfqpoint{3.877314in}{2.140718in}}{\pgfqpoint{3.883138in}{2.134894in}}%
\pgfpathcurveto{\pgfqpoint{3.888962in}{2.129070in}}{\pgfqpoint{3.896862in}{2.125798in}}{\pgfqpoint{3.905098in}{2.125798in}}%
\pgfpathclose%
\pgfusepath{stroke,fill}%
\end{pgfscope}%
\begin{pgfscope}%
\pgfpathrectangle{\pgfqpoint{3.793912in}{0.557870in}}{\pgfqpoint{2.446088in}{1.684734in}}%
\pgfusepath{clip}%
\pgfsetbuttcap%
\pgfsetroundjoin%
\definecolor{currentfill}{rgb}{0.298039,0.447059,0.690196}%
\pgfsetfillcolor{currentfill}%
\pgfsetlinewidth{1.003750pt}%
\definecolor{currentstroke}{rgb}{0.298039,0.447059,0.690196}%
\pgfsetstrokecolor{currentstroke}%
\pgfsetdash{}{0pt}%
\pgfpathmoveto{\pgfqpoint{4.518537in}{1.492991in}}%
\pgfpathcurveto{\pgfqpoint{4.526773in}{1.492991in}}{\pgfqpoint{4.534673in}{1.496263in}}{\pgfqpoint{4.540497in}{1.502087in}}%
\pgfpathcurveto{\pgfqpoint{4.546321in}{1.507911in}}{\pgfqpoint{4.549593in}{1.515811in}}{\pgfqpoint{4.549593in}{1.524047in}}%
\pgfpathcurveto{\pgfqpoint{4.549593in}{1.532284in}}{\pgfqpoint{4.546321in}{1.540184in}}{\pgfqpoint{4.540497in}{1.546008in}}%
\pgfpathcurveto{\pgfqpoint{4.534673in}{1.551831in}}{\pgfqpoint{4.526773in}{1.555104in}}{\pgfqpoint{4.518537in}{1.555104in}}%
\pgfpathcurveto{\pgfqpoint{4.510301in}{1.555104in}}{\pgfqpoint{4.502401in}{1.551831in}}{\pgfqpoint{4.496577in}{1.546008in}}%
\pgfpathcurveto{\pgfqpoint{4.490753in}{1.540184in}}{\pgfqpoint{4.487480in}{1.532284in}}{\pgfqpoint{4.487480in}{1.524047in}}%
\pgfpathcurveto{\pgfqpoint{4.487480in}{1.515811in}}{\pgfqpoint{4.490753in}{1.507911in}}{\pgfqpoint{4.496577in}{1.502087in}}%
\pgfpathcurveto{\pgfqpoint{4.502401in}{1.496263in}}{\pgfqpoint{4.510301in}{1.492991in}}{\pgfqpoint{4.518537in}{1.492991in}}%
\pgfpathclose%
\pgfusepath{stroke,fill}%
\end{pgfscope}%
\begin{pgfscope}%
\pgfpathrectangle{\pgfqpoint{3.793912in}{0.557870in}}{\pgfqpoint{2.446088in}{1.684734in}}%
\pgfusepath{clip}%
\pgfsetbuttcap%
\pgfsetroundjoin%
\definecolor{currentfill}{rgb}{0.298039,0.447059,0.690196}%
\pgfsetfillcolor{currentfill}%
\pgfsetlinewidth{1.003750pt}%
\definecolor{currentstroke}{rgb}{0.298039,0.447059,0.690196}%
\pgfsetstrokecolor{currentstroke}%
\pgfsetdash{}{0pt}%
\pgfpathmoveto{\pgfqpoint{3.905098in}{2.125798in}}%
\pgfpathcurveto{\pgfqpoint{3.913334in}{2.125798in}}{\pgfqpoint{3.921234in}{2.129070in}}{\pgfqpoint{3.927058in}{2.134894in}}%
\pgfpathcurveto{\pgfqpoint{3.932882in}{2.140718in}}{\pgfqpoint{3.936155in}{2.148618in}}{\pgfqpoint{3.936155in}{2.156854in}}%
\pgfpathcurveto{\pgfqpoint{3.936155in}{2.165091in}}{\pgfqpoint{3.932882in}{2.172991in}}{\pgfqpoint{3.927058in}{2.178814in}}%
\pgfpathcurveto{\pgfqpoint{3.921234in}{2.184638in}}{\pgfqpoint{3.913334in}{2.187911in}}{\pgfqpoint{3.905098in}{2.187911in}}%
\pgfpathcurveto{\pgfqpoint{3.896862in}{2.187911in}}{\pgfqpoint{3.888962in}{2.184638in}}{\pgfqpoint{3.883138in}{2.178814in}}%
\pgfpathcurveto{\pgfqpoint{3.877314in}{2.172991in}}{\pgfqpoint{3.874042in}{2.165091in}}{\pgfqpoint{3.874042in}{2.156854in}}%
\pgfpathcurveto{\pgfqpoint{3.874042in}{2.148618in}}{\pgfqpoint{3.877314in}{2.140718in}}{\pgfqpoint{3.883138in}{2.134894in}}%
\pgfpathcurveto{\pgfqpoint{3.888962in}{2.129070in}}{\pgfqpoint{3.896862in}{2.125798in}}{\pgfqpoint{3.905098in}{2.125798in}}%
\pgfpathclose%
\pgfusepath{stroke,fill}%
\end{pgfscope}%
\begin{pgfscope}%
\pgfpathrectangle{\pgfqpoint{3.793912in}{0.557870in}}{\pgfqpoint{2.446088in}{1.684734in}}%
\pgfusepath{clip}%
\pgfsetbuttcap%
\pgfsetroundjoin%
\definecolor{currentfill}{rgb}{0.298039,0.447059,0.690196}%
\pgfsetfillcolor{currentfill}%
\pgfsetlinewidth{1.003750pt}%
\definecolor{currentstroke}{rgb}{0.298039,0.447059,0.690196}%
\pgfsetstrokecolor{currentstroke}%
\pgfsetdash{}{0pt}%
\pgfpathmoveto{\pgfqpoint{3.905098in}{2.125798in}}%
\pgfpathcurveto{\pgfqpoint{3.913334in}{2.125798in}}{\pgfqpoint{3.921234in}{2.129070in}}{\pgfqpoint{3.927058in}{2.134894in}}%
\pgfpathcurveto{\pgfqpoint{3.932882in}{2.140718in}}{\pgfqpoint{3.936155in}{2.148618in}}{\pgfqpoint{3.936155in}{2.156854in}}%
\pgfpathcurveto{\pgfqpoint{3.936155in}{2.165091in}}{\pgfqpoint{3.932882in}{2.172991in}}{\pgfqpoint{3.927058in}{2.178814in}}%
\pgfpathcurveto{\pgfqpoint{3.921234in}{2.184638in}}{\pgfqpoint{3.913334in}{2.187911in}}{\pgfqpoint{3.905098in}{2.187911in}}%
\pgfpathcurveto{\pgfqpoint{3.896862in}{2.187911in}}{\pgfqpoint{3.888962in}{2.184638in}}{\pgfqpoint{3.883138in}{2.178814in}}%
\pgfpathcurveto{\pgfqpoint{3.877314in}{2.172991in}}{\pgfqpoint{3.874042in}{2.165091in}}{\pgfqpoint{3.874042in}{2.156854in}}%
\pgfpathcurveto{\pgfqpoint{3.874042in}{2.148618in}}{\pgfqpoint{3.877314in}{2.140718in}}{\pgfqpoint{3.883138in}{2.134894in}}%
\pgfpathcurveto{\pgfqpoint{3.888962in}{2.129070in}}{\pgfqpoint{3.896862in}{2.125798in}}{\pgfqpoint{3.905098in}{2.125798in}}%
\pgfpathclose%
\pgfusepath{stroke,fill}%
\end{pgfscope}%
\begin{pgfscope}%
\pgfpathrectangle{\pgfqpoint{3.793912in}{0.557870in}}{\pgfqpoint{2.446088in}{1.684734in}}%
\pgfusepath{clip}%
\pgfsetbuttcap%
\pgfsetroundjoin%
\definecolor{currentfill}{rgb}{0.298039,0.447059,0.690196}%
\pgfsetfillcolor{currentfill}%
\pgfsetlinewidth{1.003750pt}%
\definecolor{currentstroke}{rgb}{0.298039,0.447059,0.690196}%
\pgfsetstrokecolor{currentstroke}%
\pgfsetdash{}{0pt}%
\pgfpathmoveto{\pgfqpoint{3.905098in}{2.125798in}}%
\pgfpathcurveto{\pgfqpoint{3.913334in}{2.125798in}}{\pgfqpoint{3.921234in}{2.129070in}}{\pgfqpoint{3.927058in}{2.134894in}}%
\pgfpathcurveto{\pgfqpoint{3.932882in}{2.140718in}}{\pgfqpoint{3.936155in}{2.148618in}}{\pgfqpoint{3.936155in}{2.156854in}}%
\pgfpathcurveto{\pgfqpoint{3.936155in}{2.165091in}}{\pgfqpoint{3.932882in}{2.172991in}}{\pgfqpoint{3.927058in}{2.178814in}}%
\pgfpathcurveto{\pgfqpoint{3.921234in}{2.184638in}}{\pgfqpoint{3.913334in}{2.187911in}}{\pgfqpoint{3.905098in}{2.187911in}}%
\pgfpathcurveto{\pgfqpoint{3.896862in}{2.187911in}}{\pgfqpoint{3.888962in}{2.184638in}}{\pgfqpoint{3.883138in}{2.178814in}}%
\pgfpathcurveto{\pgfqpoint{3.877314in}{2.172991in}}{\pgfqpoint{3.874042in}{2.165091in}}{\pgfqpoint{3.874042in}{2.156854in}}%
\pgfpathcurveto{\pgfqpoint{3.874042in}{2.148618in}}{\pgfqpoint{3.877314in}{2.140718in}}{\pgfqpoint{3.883138in}{2.134894in}}%
\pgfpathcurveto{\pgfqpoint{3.888962in}{2.129070in}}{\pgfqpoint{3.896862in}{2.125798in}}{\pgfqpoint{3.905098in}{2.125798in}}%
\pgfpathclose%
\pgfusepath{stroke,fill}%
\end{pgfscope}%
\begin{pgfscope}%
\pgfpathrectangle{\pgfqpoint{3.793912in}{0.557870in}}{\pgfqpoint{2.446088in}{1.684734in}}%
\pgfusepath{clip}%
\pgfsetbuttcap%
\pgfsetroundjoin%
\definecolor{currentfill}{rgb}{0.298039,0.447059,0.690196}%
\pgfsetfillcolor{currentfill}%
\pgfsetlinewidth{1.003750pt}%
\definecolor{currentstroke}{rgb}{0.298039,0.447059,0.690196}%
\pgfsetstrokecolor{currentstroke}%
\pgfsetdash{}{0pt}%
\pgfpathmoveto{\pgfqpoint{3.905098in}{2.125798in}}%
\pgfpathcurveto{\pgfqpoint{3.913334in}{2.125798in}}{\pgfqpoint{3.921234in}{2.129070in}}{\pgfqpoint{3.927058in}{2.134894in}}%
\pgfpathcurveto{\pgfqpoint{3.932882in}{2.140718in}}{\pgfqpoint{3.936155in}{2.148618in}}{\pgfqpoint{3.936155in}{2.156854in}}%
\pgfpathcurveto{\pgfqpoint{3.936155in}{2.165091in}}{\pgfqpoint{3.932882in}{2.172991in}}{\pgfqpoint{3.927058in}{2.178814in}}%
\pgfpathcurveto{\pgfqpoint{3.921234in}{2.184638in}}{\pgfqpoint{3.913334in}{2.187911in}}{\pgfqpoint{3.905098in}{2.187911in}}%
\pgfpathcurveto{\pgfqpoint{3.896862in}{2.187911in}}{\pgfqpoint{3.888962in}{2.184638in}}{\pgfqpoint{3.883138in}{2.178814in}}%
\pgfpathcurveto{\pgfqpoint{3.877314in}{2.172991in}}{\pgfqpoint{3.874042in}{2.165091in}}{\pgfqpoint{3.874042in}{2.156854in}}%
\pgfpathcurveto{\pgfqpoint{3.874042in}{2.148618in}}{\pgfqpoint{3.877314in}{2.140718in}}{\pgfqpoint{3.883138in}{2.134894in}}%
\pgfpathcurveto{\pgfqpoint{3.888962in}{2.129070in}}{\pgfqpoint{3.896862in}{2.125798in}}{\pgfqpoint{3.905098in}{2.125798in}}%
\pgfpathclose%
\pgfusepath{stroke,fill}%
\end{pgfscope}%
\begin{pgfscope}%
\pgfpathrectangle{\pgfqpoint{3.793912in}{0.557870in}}{\pgfqpoint{2.446088in}{1.684734in}}%
\pgfusepath{clip}%
\pgfsetbuttcap%
\pgfsetroundjoin%
\definecolor{currentfill}{rgb}{0.298039,0.447059,0.690196}%
\pgfsetfillcolor{currentfill}%
\pgfsetlinewidth{1.003750pt}%
\definecolor{currentstroke}{rgb}{0.298039,0.447059,0.690196}%
\pgfsetstrokecolor{currentstroke}%
\pgfsetdash{}{0pt}%
\pgfpathmoveto{\pgfqpoint{3.905098in}{2.125798in}}%
\pgfpathcurveto{\pgfqpoint{3.913334in}{2.125798in}}{\pgfqpoint{3.921234in}{2.129070in}}{\pgfqpoint{3.927058in}{2.134894in}}%
\pgfpathcurveto{\pgfqpoint{3.932882in}{2.140718in}}{\pgfqpoint{3.936155in}{2.148618in}}{\pgfqpoint{3.936155in}{2.156854in}}%
\pgfpathcurveto{\pgfqpoint{3.936155in}{2.165091in}}{\pgfqpoint{3.932882in}{2.172991in}}{\pgfqpoint{3.927058in}{2.178814in}}%
\pgfpathcurveto{\pgfqpoint{3.921234in}{2.184638in}}{\pgfqpoint{3.913334in}{2.187911in}}{\pgfqpoint{3.905098in}{2.187911in}}%
\pgfpathcurveto{\pgfqpoint{3.896862in}{2.187911in}}{\pgfqpoint{3.888962in}{2.184638in}}{\pgfqpoint{3.883138in}{2.178814in}}%
\pgfpathcurveto{\pgfqpoint{3.877314in}{2.172991in}}{\pgfqpoint{3.874042in}{2.165091in}}{\pgfqpoint{3.874042in}{2.156854in}}%
\pgfpathcurveto{\pgfqpoint{3.874042in}{2.148618in}}{\pgfqpoint{3.877314in}{2.140718in}}{\pgfqpoint{3.883138in}{2.134894in}}%
\pgfpathcurveto{\pgfqpoint{3.888962in}{2.129070in}}{\pgfqpoint{3.896862in}{2.125798in}}{\pgfqpoint{3.905098in}{2.125798in}}%
\pgfpathclose%
\pgfusepath{stroke,fill}%
\end{pgfscope}%
\begin{pgfscope}%
\pgfpathrectangle{\pgfqpoint{3.793912in}{0.557870in}}{\pgfqpoint{2.446088in}{1.684734in}}%
\pgfusepath{clip}%
\pgfsetbuttcap%
\pgfsetroundjoin%
\definecolor{currentfill}{rgb}{0.298039,0.447059,0.690196}%
\pgfsetfillcolor{currentfill}%
\pgfsetlinewidth{1.003750pt}%
\definecolor{currentstroke}{rgb}{0.298039,0.447059,0.690196}%
\pgfsetstrokecolor{currentstroke}%
\pgfsetdash{}{0pt}%
\pgfpathmoveto{\pgfqpoint{3.905098in}{2.125798in}}%
\pgfpathcurveto{\pgfqpoint{3.913334in}{2.125798in}}{\pgfqpoint{3.921234in}{2.129070in}}{\pgfqpoint{3.927058in}{2.134894in}}%
\pgfpathcurveto{\pgfqpoint{3.932882in}{2.140718in}}{\pgfqpoint{3.936155in}{2.148618in}}{\pgfqpoint{3.936155in}{2.156854in}}%
\pgfpathcurveto{\pgfqpoint{3.936155in}{2.165091in}}{\pgfqpoint{3.932882in}{2.172991in}}{\pgfqpoint{3.927058in}{2.178814in}}%
\pgfpathcurveto{\pgfqpoint{3.921234in}{2.184638in}}{\pgfqpoint{3.913334in}{2.187911in}}{\pgfqpoint{3.905098in}{2.187911in}}%
\pgfpathcurveto{\pgfqpoint{3.896862in}{2.187911in}}{\pgfqpoint{3.888962in}{2.184638in}}{\pgfqpoint{3.883138in}{2.178814in}}%
\pgfpathcurveto{\pgfqpoint{3.877314in}{2.172991in}}{\pgfqpoint{3.874042in}{2.165091in}}{\pgfqpoint{3.874042in}{2.156854in}}%
\pgfpathcurveto{\pgfqpoint{3.874042in}{2.148618in}}{\pgfqpoint{3.877314in}{2.140718in}}{\pgfqpoint{3.883138in}{2.134894in}}%
\pgfpathcurveto{\pgfqpoint{3.888962in}{2.129070in}}{\pgfqpoint{3.896862in}{2.125798in}}{\pgfqpoint{3.905098in}{2.125798in}}%
\pgfpathclose%
\pgfusepath{stroke,fill}%
\end{pgfscope}%
\begin{pgfscope}%
\pgfpathrectangle{\pgfqpoint{3.793912in}{0.557870in}}{\pgfqpoint{2.446088in}{1.684734in}}%
\pgfusepath{clip}%
\pgfsetbuttcap%
\pgfsetroundjoin%
\definecolor{currentfill}{rgb}{0.298039,0.447059,0.690196}%
\pgfsetfillcolor{currentfill}%
\pgfsetlinewidth{1.003750pt}%
\definecolor{currentstroke}{rgb}{0.298039,0.447059,0.690196}%
\pgfsetstrokecolor{currentstroke}%
\pgfsetdash{}{0pt}%
\pgfpathmoveto{\pgfqpoint{4.595217in}{1.566360in}}%
\pgfpathcurveto{\pgfqpoint{4.603453in}{1.566360in}}{\pgfqpoint{4.611353in}{1.569632in}}{\pgfqpoint{4.617177in}{1.575456in}}%
\pgfpathcurveto{\pgfqpoint{4.623001in}{1.581280in}}{\pgfqpoint{4.626273in}{1.589180in}}{\pgfqpoint{4.626273in}{1.597416in}}%
\pgfpathcurveto{\pgfqpoint{4.626273in}{1.605652in}}{\pgfqpoint{4.623001in}{1.613553in}}{\pgfqpoint{4.617177in}{1.619376in}}%
\pgfpathcurveto{\pgfqpoint{4.611353in}{1.625200in}}{\pgfqpoint{4.603453in}{1.628473in}}{\pgfqpoint{4.595217in}{1.628473in}}%
\pgfpathcurveto{\pgfqpoint{4.586981in}{1.628473in}}{\pgfqpoint{4.579081in}{1.625200in}}{\pgfqpoint{4.573257in}{1.619376in}}%
\pgfpathcurveto{\pgfqpoint{4.567433in}{1.613553in}}{\pgfqpoint{4.564160in}{1.605652in}}{\pgfqpoint{4.564160in}{1.597416in}}%
\pgfpathcurveto{\pgfqpoint{4.564160in}{1.589180in}}{\pgfqpoint{4.567433in}{1.581280in}}{\pgfqpoint{4.573257in}{1.575456in}}%
\pgfpathcurveto{\pgfqpoint{4.579081in}{1.569632in}}{\pgfqpoint{4.586981in}{1.566360in}}{\pgfqpoint{4.595217in}{1.566360in}}%
\pgfpathclose%
\pgfusepath{stroke,fill}%
\end{pgfscope}%
\begin{pgfscope}%
\pgfpathrectangle{\pgfqpoint{3.793912in}{0.557870in}}{\pgfqpoint{2.446088in}{1.684734in}}%
\pgfusepath{clip}%
\pgfsetbuttcap%
\pgfsetroundjoin%
\definecolor{currentfill}{rgb}{0.298039,0.447059,0.690196}%
\pgfsetfillcolor{currentfill}%
\pgfsetlinewidth{1.003750pt}%
\definecolor{currentstroke}{rgb}{0.298039,0.447059,0.690196}%
\pgfsetstrokecolor{currentstroke}%
\pgfsetdash{}{0pt}%
\pgfpathmoveto{\pgfqpoint{3.905098in}{2.125798in}}%
\pgfpathcurveto{\pgfqpoint{3.913334in}{2.125798in}}{\pgfqpoint{3.921234in}{2.129070in}}{\pgfqpoint{3.927058in}{2.134894in}}%
\pgfpathcurveto{\pgfqpoint{3.932882in}{2.140718in}}{\pgfqpoint{3.936155in}{2.148618in}}{\pgfqpoint{3.936155in}{2.156854in}}%
\pgfpathcurveto{\pgfqpoint{3.936155in}{2.165091in}}{\pgfqpoint{3.932882in}{2.172991in}}{\pgfqpoint{3.927058in}{2.178814in}}%
\pgfpathcurveto{\pgfqpoint{3.921234in}{2.184638in}}{\pgfqpoint{3.913334in}{2.187911in}}{\pgfqpoint{3.905098in}{2.187911in}}%
\pgfpathcurveto{\pgfqpoint{3.896862in}{2.187911in}}{\pgfqpoint{3.888962in}{2.184638in}}{\pgfqpoint{3.883138in}{2.178814in}}%
\pgfpathcurveto{\pgfqpoint{3.877314in}{2.172991in}}{\pgfqpoint{3.874042in}{2.165091in}}{\pgfqpoint{3.874042in}{2.156854in}}%
\pgfpathcurveto{\pgfqpoint{3.874042in}{2.148618in}}{\pgfqpoint{3.877314in}{2.140718in}}{\pgfqpoint{3.883138in}{2.134894in}}%
\pgfpathcurveto{\pgfqpoint{3.888962in}{2.129070in}}{\pgfqpoint{3.896862in}{2.125798in}}{\pgfqpoint{3.905098in}{2.125798in}}%
\pgfpathclose%
\pgfusepath{stroke,fill}%
\end{pgfscope}%
\begin{pgfscope}%
\pgfpathrectangle{\pgfqpoint{3.793912in}{0.557870in}}{\pgfqpoint{2.446088in}{1.684734in}}%
\pgfusepath{clip}%
\pgfsetbuttcap%
\pgfsetroundjoin%
\definecolor{currentfill}{rgb}{0.298039,0.447059,0.690196}%
\pgfsetfillcolor{currentfill}%
\pgfsetlinewidth{1.003750pt}%
\definecolor{currentstroke}{rgb}{0.298039,0.447059,0.690196}%
\pgfsetstrokecolor{currentstroke}%
\pgfsetdash{}{0pt}%
\pgfpathmoveto{\pgfqpoint{3.905098in}{2.125798in}}%
\pgfpathcurveto{\pgfqpoint{3.913334in}{2.125798in}}{\pgfqpoint{3.921234in}{2.129070in}}{\pgfqpoint{3.927058in}{2.134894in}}%
\pgfpathcurveto{\pgfqpoint{3.932882in}{2.140718in}}{\pgfqpoint{3.936155in}{2.148618in}}{\pgfqpoint{3.936155in}{2.156854in}}%
\pgfpathcurveto{\pgfqpoint{3.936155in}{2.165091in}}{\pgfqpoint{3.932882in}{2.172991in}}{\pgfqpoint{3.927058in}{2.178814in}}%
\pgfpathcurveto{\pgfqpoint{3.921234in}{2.184638in}}{\pgfqpoint{3.913334in}{2.187911in}}{\pgfqpoint{3.905098in}{2.187911in}}%
\pgfpathcurveto{\pgfqpoint{3.896862in}{2.187911in}}{\pgfqpoint{3.888962in}{2.184638in}}{\pgfqpoint{3.883138in}{2.178814in}}%
\pgfpathcurveto{\pgfqpoint{3.877314in}{2.172991in}}{\pgfqpoint{3.874042in}{2.165091in}}{\pgfqpoint{3.874042in}{2.156854in}}%
\pgfpathcurveto{\pgfqpoint{3.874042in}{2.148618in}}{\pgfqpoint{3.877314in}{2.140718in}}{\pgfqpoint{3.883138in}{2.134894in}}%
\pgfpathcurveto{\pgfqpoint{3.888962in}{2.129070in}}{\pgfqpoint{3.896862in}{2.125798in}}{\pgfqpoint{3.905098in}{2.125798in}}%
\pgfpathclose%
\pgfusepath{stroke,fill}%
\end{pgfscope}%
\begin{pgfscope}%
\pgfpathrectangle{\pgfqpoint{3.793912in}{0.557870in}}{\pgfqpoint{2.446088in}{1.684734in}}%
\pgfusepath{clip}%
\pgfsetbuttcap%
\pgfsetroundjoin%
\definecolor{currentfill}{rgb}{0.298039,0.447059,0.690196}%
\pgfsetfillcolor{currentfill}%
\pgfsetlinewidth{1.003750pt}%
\definecolor{currentstroke}{rgb}{0.298039,0.447059,0.690196}%
\pgfsetstrokecolor{currentstroke}%
\pgfsetdash{}{0pt}%
\pgfpathmoveto{\pgfqpoint{3.905098in}{2.125798in}}%
\pgfpathcurveto{\pgfqpoint{3.913334in}{2.125798in}}{\pgfqpoint{3.921234in}{2.129070in}}{\pgfqpoint{3.927058in}{2.134894in}}%
\pgfpathcurveto{\pgfqpoint{3.932882in}{2.140718in}}{\pgfqpoint{3.936155in}{2.148618in}}{\pgfqpoint{3.936155in}{2.156854in}}%
\pgfpathcurveto{\pgfqpoint{3.936155in}{2.165091in}}{\pgfqpoint{3.932882in}{2.172991in}}{\pgfqpoint{3.927058in}{2.178814in}}%
\pgfpathcurveto{\pgfqpoint{3.921234in}{2.184638in}}{\pgfqpoint{3.913334in}{2.187911in}}{\pgfqpoint{3.905098in}{2.187911in}}%
\pgfpathcurveto{\pgfqpoint{3.896862in}{2.187911in}}{\pgfqpoint{3.888962in}{2.184638in}}{\pgfqpoint{3.883138in}{2.178814in}}%
\pgfpathcurveto{\pgfqpoint{3.877314in}{2.172991in}}{\pgfqpoint{3.874042in}{2.165091in}}{\pgfqpoint{3.874042in}{2.156854in}}%
\pgfpathcurveto{\pgfqpoint{3.874042in}{2.148618in}}{\pgfqpoint{3.877314in}{2.140718in}}{\pgfqpoint{3.883138in}{2.134894in}}%
\pgfpathcurveto{\pgfqpoint{3.888962in}{2.129070in}}{\pgfqpoint{3.896862in}{2.125798in}}{\pgfqpoint{3.905098in}{2.125798in}}%
\pgfpathclose%
\pgfusepath{stroke,fill}%
\end{pgfscope}%
\begin{pgfscope}%
\pgfpathrectangle{\pgfqpoint{3.793912in}{0.557870in}}{\pgfqpoint{2.446088in}{1.684734in}}%
\pgfusepath{clip}%
\pgfsetbuttcap%
\pgfsetroundjoin%
\definecolor{currentfill}{rgb}{0.298039,0.447059,0.690196}%
\pgfsetfillcolor{currentfill}%
\pgfsetlinewidth{1.003750pt}%
\definecolor{currentstroke}{rgb}{0.298039,0.447059,0.690196}%
\pgfsetstrokecolor{currentstroke}%
\pgfsetdash{}{0pt}%
\pgfpathmoveto{\pgfqpoint{5.975454in}{1.538846in}}%
\pgfpathcurveto{\pgfqpoint{5.983691in}{1.538846in}}{\pgfqpoint{5.991591in}{1.542119in}}{\pgfqpoint{5.997415in}{1.547943in}}%
\pgfpathcurveto{\pgfqpoint{6.003239in}{1.553767in}}{\pgfqpoint{6.006511in}{1.561667in}}{\pgfqpoint{6.006511in}{1.569903in}}%
\pgfpathcurveto{\pgfqpoint{6.006511in}{1.578139in}}{\pgfqpoint{6.003239in}{1.586039in}}{\pgfqpoint{5.997415in}{1.591863in}}%
\pgfpathcurveto{\pgfqpoint{5.991591in}{1.597687in}}{\pgfqpoint{5.983691in}{1.600959in}}{\pgfqpoint{5.975454in}{1.600959in}}%
\pgfpathcurveto{\pgfqpoint{5.967218in}{1.600959in}}{\pgfqpoint{5.959318in}{1.597687in}}{\pgfqpoint{5.953494in}{1.591863in}}%
\pgfpathcurveto{\pgfqpoint{5.947670in}{1.586039in}}{\pgfqpoint{5.944398in}{1.578139in}}{\pgfqpoint{5.944398in}{1.569903in}}%
\pgfpathcurveto{\pgfqpoint{5.944398in}{1.561667in}}{\pgfqpoint{5.947670in}{1.553767in}}{\pgfqpoint{5.953494in}{1.547943in}}%
\pgfpathcurveto{\pgfqpoint{5.959318in}{1.542119in}}{\pgfqpoint{5.967218in}{1.538846in}}{\pgfqpoint{5.975454in}{1.538846in}}%
\pgfpathclose%
\pgfusepath{stroke,fill}%
\end{pgfscope}%
\begin{pgfscope}%
\pgfpathrectangle{\pgfqpoint{3.793912in}{0.557870in}}{\pgfqpoint{2.446088in}{1.684734in}}%
\pgfusepath{clip}%
\pgfsetbuttcap%
\pgfsetroundjoin%
\definecolor{currentfill}{rgb}{0.298039,0.447059,0.690196}%
\pgfsetfillcolor{currentfill}%
\pgfsetlinewidth{1.003750pt}%
\definecolor{currentstroke}{rgb}{0.298039,0.447059,0.690196}%
\pgfsetstrokecolor{currentstroke}%
\pgfsetdash{}{0pt}%
\pgfpathmoveto{\pgfqpoint{5.975454in}{1.667242in}}%
\pgfpathcurveto{\pgfqpoint{5.983691in}{1.667242in}}{\pgfqpoint{5.991591in}{1.670514in}}{\pgfqpoint{5.997415in}{1.676338in}}%
\pgfpathcurveto{\pgfqpoint{6.003239in}{1.682162in}}{\pgfqpoint{6.006511in}{1.690062in}}{\pgfqpoint{6.006511in}{1.698298in}}%
\pgfpathcurveto{\pgfqpoint{6.006511in}{1.706535in}}{\pgfqpoint{6.003239in}{1.714435in}}{\pgfqpoint{5.997415in}{1.720259in}}%
\pgfpathcurveto{\pgfqpoint{5.991591in}{1.726083in}}{\pgfqpoint{5.983691in}{1.729355in}}{\pgfqpoint{5.975454in}{1.729355in}}%
\pgfpathcurveto{\pgfqpoint{5.967218in}{1.729355in}}{\pgfqpoint{5.959318in}{1.726083in}}{\pgfqpoint{5.953494in}{1.720259in}}%
\pgfpathcurveto{\pgfqpoint{5.947670in}{1.714435in}}{\pgfqpoint{5.944398in}{1.706535in}}{\pgfqpoint{5.944398in}{1.698298in}}%
\pgfpathcurveto{\pgfqpoint{5.944398in}{1.690062in}}{\pgfqpoint{5.947670in}{1.682162in}}{\pgfqpoint{5.953494in}{1.676338in}}%
\pgfpathcurveto{\pgfqpoint{5.959318in}{1.670514in}}{\pgfqpoint{5.967218in}{1.667242in}}{\pgfqpoint{5.975454in}{1.667242in}}%
\pgfpathclose%
\pgfusepath{stroke,fill}%
\end{pgfscope}%
\begin{pgfscope}%
\pgfpathrectangle{\pgfqpoint{3.793912in}{0.557870in}}{\pgfqpoint{2.446088in}{1.684734in}}%
\pgfusepath{clip}%
\pgfsetbuttcap%
\pgfsetroundjoin%
\definecolor{currentfill}{rgb}{0.298039,0.447059,0.690196}%
\pgfsetfillcolor{currentfill}%
\pgfsetlinewidth{1.003750pt}%
\definecolor{currentstroke}{rgb}{0.298039,0.447059,0.690196}%
\pgfsetstrokecolor{currentstroke}%
\pgfsetdash{}{0pt}%
\pgfpathmoveto{\pgfqpoint{5.975454in}{1.630558in}}%
\pgfpathcurveto{\pgfqpoint{5.983691in}{1.630558in}}{\pgfqpoint{5.991591in}{1.633830in}}{\pgfqpoint{5.997415in}{1.639654in}}%
\pgfpathcurveto{\pgfqpoint{6.003239in}{1.645478in}}{\pgfqpoint{6.006511in}{1.653378in}}{\pgfqpoint{6.006511in}{1.661614in}}%
\pgfpathcurveto{\pgfqpoint{6.006511in}{1.669850in}}{\pgfqpoint{6.003239in}{1.677750in}}{\pgfqpoint{5.997415in}{1.683574in}}%
\pgfpathcurveto{\pgfqpoint{5.991591in}{1.689398in}}{\pgfqpoint{5.983691in}{1.692671in}}{\pgfqpoint{5.975454in}{1.692671in}}%
\pgfpathcurveto{\pgfqpoint{5.967218in}{1.692671in}}{\pgfqpoint{5.959318in}{1.689398in}}{\pgfqpoint{5.953494in}{1.683574in}}%
\pgfpathcurveto{\pgfqpoint{5.947670in}{1.677750in}}{\pgfqpoint{5.944398in}{1.669850in}}{\pgfqpoint{5.944398in}{1.661614in}}%
\pgfpathcurveto{\pgfqpoint{5.944398in}{1.653378in}}{\pgfqpoint{5.947670in}{1.645478in}}{\pgfqpoint{5.953494in}{1.639654in}}%
\pgfpathcurveto{\pgfqpoint{5.959318in}{1.633830in}}{\pgfqpoint{5.967218in}{1.630558in}}{\pgfqpoint{5.975454in}{1.630558in}}%
\pgfpathclose%
\pgfusepath{stroke,fill}%
\end{pgfscope}%
\begin{pgfscope}%
\pgfpathrectangle{\pgfqpoint{3.793912in}{0.557870in}}{\pgfqpoint{2.446088in}{1.684734in}}%
\pgfusepath{clip}%
\pgfsetbuttcap%
\pgfsetroundjoin%
\definecolor{currentfill}{rgb}{0.298039,0.447059,0.690196}%
\pgfsetfillcolor{currentfill}%
\pgfsetlinewidth{1.003750pt}%
\definecolor{currentstroke}{rgb}{0.298039,0.447059,0.690196}%
\pgfsetstrokecolor{currentstroke}%
\pgfsetdash{}{0pt}%
\pgfpathmoveto{\pgfqpoint{5.975454in}{1.694755in}}%
\pgfpathcurveto{\pgfqpoint{5.983691in}{1.694755in}}{\pgfqpoint{5.991591in}{1.698028in}}{\pgfqpoint{5.997415in}{1.703852in}}%
\pgfpathcurveto{\pgfqpoint{6.003239in}{1.709675in}}{\pgfqpoint{6.006511in}{1.717576in}}{\pgfqpoint{6.006511in}{1.725812in}}%
\pgfpathcurveto{\pgfqpoint{6.006511in}{1.734048in}}{\pgfqpoint{6.003239in}{1.741948in}}{\pgfqpoint{5.997415in}{1.747772in}}%
\pgfpathcurveto{\pgfqpoint{5.991591in}{1.753596in}}{\pgfqpoint{5.983691in}{1.756868in}}{\pgfqpoint{5.975454in}{1.756868in}}%
\pgfpathcurveto{\pgfqpoint{5.967218in}{1.756868in}}{\pgfqpoint{5.959318in}{1.753596in}}{\pgfqpoint{5.953494in}{1.747772in}}%
\pgfpathcurveto{\pgfqpoint{5.947670in}{1.741948in}}{\pgfqpoint{5.944398in}{1.734048in}}{\pgfqpoint{5.944398in}{1.725812in}}%
\pgfpathcurveto{\pgfqpoint{5.944398in}{1.717576in}}{\pgfqpoint{5.947670in}{1.709675in}}{\pgfqpoint{5.953494in}{1.703852in}}%
\pgfpathcurveto{\pgfqpoint{5.959318in}{1.698028in}}{\pgfqpoint{5.967218in}{1.694755in}}{\pgfqpoint{5.975454in}{1.694755in}}%
\pgfpathclose%
\pgfusepath{stroke,fill}%
\end{pgfscope}%
\begin{pgfscope}%
\pgfpathrectangle{\pgfqpoint{3.793912in}{0.557870in}}{\pgfqpoint{2.446088in}{1.684734in}}%
\pgfusepath{clip}%
\pgfsetbuttcap%
\pgfsetroundjoin%
\definecolor{currentfill}{rgb}{0.298039,0.447059,0.690196}%
\pgfsetfillcolor{currentfill}%
\pgfsetlinewidth{1.003750pt}%
\definecolor{currentstroke}{rgb}{0.298039,0.447059,0.690196}%
\pgfsetstrokecolor{currentstroke}%
\pgfsetdash{}{0pt}%
\pgfpathmoveto{\pgfqpoint{5.975454in}{1.318740in}}%
\pgfpathcurveto{\pgfqpoint{5.983691in}{1.318740in}}{\pgfqpoint{5.991591in}{1.322012in}}{\pgfqpoint{5.997415in}{1.327836in}}%
\pgfpathcurveto{\pgfqpoint{6.003239in}{1.333660in}}{\pgfqpoint{6.006511in}{1.341560in}}{\pgfqpoint{6.006511in}{1.349796in}}%
\pgfpathcurveto{\pgfqpoint{6.006511in}{1.358032in}}{\pgfqpoint{6.003239in}{1.365932in}}{\pgfqpoint{5.997415in}{1.371756in}}%
\pgfpathcurveto{\pgfqpoint{5.991591in}{1.377580in}}{\pgfqpoint{5.983691in}{1.380853in}}{\pgfqpoint{5.975454in}{1.380853in}}%
\pgfpathcurveto{\pgfqpoint{5.967218in}{1.380853in}}{\pgfqpoint{5.959318in}{1.377580in}}{\pgfqpoint{5.953494in}{1.371756in}}%
\pgfpathcurveto{\pgfqpoint{5.947670in}{1.365932in}}{\pgfqpoint{5.944398in}{1.358032in}}{\pgfqpoint{5.944398in}{1.349796in}}%
\pgfpathcurveto{\pgfqpoint{5.944398in}{1.341560in}}{\pgfqpoint{5.947670in}{1.333660in}}{\pgfqpoint{5.953494in}{1.327836in}}%
\pgfpathcurveto{\pgfqpoint{5.959318in}{1.322012in}}{\pgfqpoint{5.967218in}{1.318740in}}{\pgfqpoint{5.975454in}{1.318740in}}%
\pgfpathclose%
\pgfusepath{stroke,fill}%
\end{pgfscope}%
\begin{pgfscope}%
\pgfpathrectangle{\pgfqpoint{3.793912in}{0.557870in}}{\pgfqpoint{2.446088in}{1.684734in}}%
\pgfusepath{clip}%
\pgfsetbuttcap%
\pgfsetroundjoin%
\definecolor{currentfill}{rgb}{0.298039,0.447059,0.690196}%
\pgfsetfillcolor{currentfill}%
\pgfsetlinewidth{1.003750pt}%
\definecolor{currentstroke}{rgb}{0.298039,0.447059,0.690196}%
\pgfsetstrokecolor{currentstroke}%
\pgfsetdash{}{0pt}%
\pgfpathmoveto{\pgfqpoint{5.975454in}{1.318740in}}%
\pgfpathcurveto{\pgfqpoint{5.983691in}{1.318740in}}{\pgfqpoint{5.991591in}{1.322012in}}{\pgfqpoint{5.997415in}{1.327836in}}%
\pgfpathcurveto{\pgfqpoint{6.003239in}{1.333660in}}{\pgfqpoint{6.006511in}{1.341560in}}{\pgfqpoint{6.006511in}{1.349796in}}%
\pgfpathcurveto{\pgfqpoint{6.006511in}{1.358032in}}{\pgfqpoint{6.003239in}{1.365932in}}{\pgfqpoint{5.997415in}{1.371756in}}%
\pgfpathcurveto{\pgfqpoint{5.991591in}{1.377580in}}{\pgfqpoint{5.983691in}{1.380853in}}{\pgfqpoint{5.975454in}{1.380853in}}%
\pgfpathcurveto{\pgfqpoint{5.967218in}{1.380853in}}{\pgfqpoint{5.959318in}{1.377580in}}{\pgfqpoint{5.953494in}{1.371756in}}%
\pgfpathcurveto{\pgfqpoint{5.947670in}{1.365932in}}{\pgfqpoint{5.944398in}{1.358032in}}{\pgfqpoint{5.944398in}{1.349796in}}%
\pgfpathcurveto{\pgfqpoint{5.944398in}{1.341560in}}{\pgfqpoint{5.947670in}{1.333660in}}{\pgfqpoint{5.953494in}{1.327836in}}%
\pgfpathcurveto{\pgfqpoint{5.959318in}{1.322012in}}{\pgfqpoint{5.967218in}{1.318740in}}{\pgfqpoint{5.975454in}{1.318740in}}%
\pgfpathclose%
\pgfusepath{stroke,fill}%
\end{pgfscope}%
\begin{pgfscope}%
\pgfpathrectangle{\pgfqpoint{3.793912in}{0.557870in}}{\pgfqpoint{2.446088in}{1.684734in}}%
\pgfusepath{clip}%
\pgfsetbuttcap%
\pgfsetroundjoin%
\definecolor{currentfill}{rgb}{0.298039,0.447059,0.690196}%
\pgfsetfillcolor{currentfill}%
\pgfsetlinewidth{1.003750pt}%
\definecolor{currentstroke}{rgb}{0.298039,0.447059,0.690196}%
\pgfsetstrokecolor{currentstroke}%
\pgfsetdash{}{0pt}%
\pgfpathmoveto{\pgfqpoint{3.905098in}{2.125798in}}%
\pgfpathcurveto{\pgfqpoint{3.913334in}{2.125798in}}{\pgfqpoint{3.921234in}{2.129070in}}{\pgfqpoint{3.927058in}{2.134894in}}%
\pgfpathcurveto{\pgfqpoint{3.932882in}{2.140718in}}{\pgfqpoint{3.936155in}{2.148618in}}{\pgfqpoint{3.936155in}{2.156854in}}%
\pgfpathcurveto{\pgfqpoint{3.936155in}{2.165091in}}{\pgfqpoint{3.932882in}{2.172991in}}{\pgfqpoint{3.927058in}{2.178814in}}%
\pgfpathcurveto{\pgfqpoint{3.921234in}{2.184638in}}{\pgfqpoint{3.913334in}{2.187911in}}{\pgfqpoint{3.905098in}{2.187911in}}%
\pgfpathcurveto{\pgfqpoint{3.896862in}{2.187911in}}{\pgfqpoint{3.888962in}{2.184638in}}{\pgfqpoint{3.883138in}{2.178814in}}%
\pgfpathcurveto{\pgfqpoint{3.877314in}{2.172991in}}{\pgfqpoint{3.874042in}{2.165091in}}{\pgfqpoint{3.874042in}{2.156854in}}%
\pgfpathcurveto{\pgfqpoint{3.874042in}{2.148618in}}{\pgfqpoint{3.877314in}{2.140718in}}{\pgfqpoint{3.883138in}{2.134894in}}%
\pgfpathcurveto{\pgfqpoint{3.888962in}{2.129070in}}{\pgfqpoint{3.896862in}{2.125798in}}{\pgfqpoint{3.905098in}{2.125798in}}%
\pgfpathclose%
\pgfusepath{stroke,fill}%
\end{pgfscope}%
\begin{pgfscope}%
\pgfpathrectangle{\pgfqpoint{3.793912in}{0.557870in}}{\pgfqpoint{2.446088in}{1.684734in}}%
\pgfusepath{clip}%
\pgfsetbuttcap%
\pgfsetroundjoin%
\definecolor{currentfill}{rgb}{0.298039,0.447059,0.690196}%
\pgfsetfillcolor{currentfill}%
\pgfsetlinewidth{1.003750pt}%
\definecolor{currentstroke}{rgb}{0.298039,0.447059,0.690196}%
\pgfsetstrokecolor{currentstroke}%
\pgfsetdash{}{0pt}%
\pgfpathmoveto{\pgfqpoint{4.058458in}{1.584702in}}%
\pgfpathcurveto{\pgfqpoint{4.066694in}{1.584702in}}{\pgfqpoint{4.074594in}{1.587974in}}{\pgfqpoint{4.080418in}{1.593798in}}%
\pgfpathcurveto{\pgfqpoint{4.086242in}{1.599622in}}{\pgfqpoint{4.089514in}{1.607522in}}{\pgfqpoint{4.089514in}{1.615758in}}%
\pgfpathcurveto{\pgfqpoint{4.089514in}{1.623995in}}{\pgfqpoint{4.086242in}{1.631895in}}{\pgfqpoint{4.080418in}{1.637719in}}%
\pgfpathcurveto{\pgfqpoint{4.074594in}{1.643543in}}{\pgfqpoint{4.066694in}{1.646815in}}{\pgfqpoint{4.058458in}{1.646815in}}%
\pgfpathcurveto{\pgfqpoint{4.050221in}{1.646815in}}{\pgfqpoint{4.042321in}{1.643543in}}{\pgfqpoint{4.036498in}{1.637719in}}%
\pgfpathcurveto{\pgfqpoint{4.030674in}{1.631895in}}{\pgfqpoint{4.027401in}{1.623995in}}{\pgfqpoint{4.027401in}{1.615758in}}%
\pgfpathcurveto{\pgfqpoint{4.027401in}{1.607522in}}{\pgfqpoint{4.030674in}{1.599622in}}{\pgfqpoint{4.036498in}{1.593798in}}%
\pgfpathcurveto{\pgfqpoint{4.042321in}{1.587974in}}{\pgfqpoint{4.050221in}{1.584702in}}{\pgfqpoint{4.058458in}{1.584702in}}%
\pgfpathclose%
\pgfusepath{stroke,fill}%
\end{pgfscope}%
\begin{pgfscope}%
\pgfpathrectangle{\pgfqpoint{3.793912in}{0.557870in}}{\pgfqpoint{2.446088in}{1.684734in}}%
\pgfusepath{clip}%
\pgfsetbuttcap%
\pgfsetroundjoin%
\definecolor{currentfill}{rgb}{0.298039,0.447059,0.690196}%
\pgfsetfillcolor{currentfill}%
\pgfsetlinewidth{1.003750pt}%
\definecolor{currentstroke}{rgb}{0.298039,0.447059,0.690196}%
\pgfsetstrokecolor{currentstroke}%
\pgfsetdash{}{0pt}%
\pgfpathmoveto{\pgfqpoint{5.975454in}{1.392109in}}%
\pgfpathcurveto{\pgfqpoint{5.983691in}{1.392109in}}{\pgfqpoint{5.991591in}{1.395381in}}{\pgfqpoint{5.997415in}{1.401205in}}%
\pgfpathcurveto{\pgfqpoint{6.003239in}{1.407029in}}{\pgfqpoint{6.006511in}{1.414929in}}{\pgfqpoint{6.006511in}{1.423165in}}%
\pgfpathcurveto{\pgfqpoint{6.006511in}{1.431401in}}{\pgfqpoint{6.003239in}{1.439301in}}{\pgfqpoint{5.997415in}{1.445125in}}%
\pgfpathcurveto{\pgfqpoint{5.991591in}{1.450949in}}{\pgfqpoint{5.983691in}{1.454222in}}{\pgfqpoint{5.975454in}{1.454222in}}%
\pgfpathcurveto{\pgfqpoint{5.967218in}{1.454222in}}{\pgfqpoint{5.959318in}{1.450949in}}{\pgfqpoint{5.953494in}{1.445125in}}%
\pgfpathcurveto{\pgfqpoint{5.947670in}{1.439301in}}{\pgfqpoint{5.944398in}{1.431401in}}{\pgfqpoint{5.944398in}{1.423165in}}%
\pgfpathcurveto{\pgfqpoint{5.944398in}{1.414929in}}{\pgfqpoint{5.947670in}{1.407029in}}{\pgfqpoint{5.953494in}{1.401205in}}%
\pgfpathcurveto{\pgfqpoint{5.959318in}{1.395381in}}{\pgfqpoint{5.967218in}{1.392109in}}{\pgfqpoint{5.975454in}{1.392109in}}%
\pgfpathclose%
\pgfusepath{stroke,fill}%
\end{pgfscope}%
\begin{pgfscope}%
\pgfpathrectangle{\pgfqpoint{3.793912in}{0.557870in}}{\pgfqpoint{2.446088in}{1.684734in}}%
\pgfusepath{clip}%
\pgfsetbuttcap%
\pgfsetroundjoin%
\definecolor{currentfill}{rgb}{0.298039,0.447059,0.690196}%
\pgfsetfillcolor{currentfill}%
\pgfsetlinewidth{1.003750pt}%
\definecolor{currentstroke}{rgb}{0.298039,0.447059,0.690196}%
\pgfsetstrokecolor{currentstroke}%
\pgfsetdash{}{0pt}%
\pgfpathmoveto{\pgfqpoint{5.975454in}{1.419622in}}%
\pgfpathcurveto{\pgfqpoint{5.983691in}{1.419622in}}{\pgfqpoint{5.991591in}{1.422894in}}{\pgfqpoint{5.997415in}{1.428718in}}%
\pgfpathcurveto{\pgfqpoint{6.003239in}{1.434542in}}{\pgfqpoint{6.006511in}{1.442442in}}{\pgfqpoint{6.006511in}{1.450678in}}%
\pgfpathcurveto{\pgfqpoint{6.006511in}{1.458915in}}{\pgfqpoint{6.003239in}{1.466815in}}{\pgfqpoint{5.997415in}{1.472639in}}%
\pgfpathcurveto{\pgfqpoint{5.991591in}{1.478463in}}{\pgfqpoint{5.983691in}{1.481735in}}{\pgfqpoint{5.975454in}{1.481735in}}%
\pgfpathcurveto{\pgfqpoint{5.967218in}{1.481735in}}{\pgfqpoint{5.959318in}{1.478463in}}{\pgfqpoint{5.953494in}{1.472639in}}%
\pgfpathcurveto{\pgfqpoint{5.947670in}{1.466815in}}{\pgfqpoint{5.944398in}{1.458915in}}{\pgfqpoint{5.944398in}{1.450678in}}%
\pgfpathcurveto{\pgfqpoint{5.944398in}{1.442442in}}{\pgfqpoint{5.947670in}{1.434542in}}{\pgfqpoint{5.953494in}{1.428718in}}%
\pgfpathcurveto{\pgfqpoint{5.959318in}{1.422894in}}{\pgfqpoint{5.967218in}{1.419622in}}{\pgfqpoint{5.975454in}{1.419622in}}%
\pgfpathclose%
\pgfusepath{stroke,fill}%
\end{pgfscope}%
\begin{pgfscope}%
\pgfpathrectangle{\pgfqpoint{3.793912in}{0.557870in}}{\pgfqpoint{2.446088in}{1.684734in}}%
\pgfusepath{clip}%
\pgfsetbuttcap%
\pgfsetroundjoin%
\definecolor{currentfill}{rgb}{0.298039,0.447059,0.690196}%
\pgfsetfillcolor{currentfill}%
\pgfsetlinewidth{1.003750pt}%
\definecolor{currentstroke}{rgb}{0.298039,0.447059,0.690196}%
\pgfsetstrokecolor{currentstroke}%
\pgfsetdash{}{0pt}%
\pgfpathmoveto{\pgfqpoint{5.975454in}{1.428793in}}%
\pgfpathcurveto{\pgfqpoint{5.983691in}{1.428793in}}{\pgfqpoint{5.991591in}{1.432065in}}{\pgfqpoint{5.997415in}{1.437889in}}%
\pgfpathcurveto{\pgfqpoint{6.003239in}{1.443713in}}{\pgfqpoint{6.006511in}{1.451613in}}{\pgfqpoint{6.006511in}{1.459849in}}%
\pgfpathcurveto{\pgfqpoint{6.006511in}{1.468086in}}{\pgfqpoint{6.003239in}{1.475986in}}{\pgfqpoint{5.997415in}{1.481810in}}%
\pgfpathcurveto{\pgfqpoint{5.991591in}{1.487634in}}{\pgfqpoint{5.983691in}{1.490906in}}{\pgfqpoint{5.975454in}{1.490906in}}%
\pgfpathcurveto{\pgfqpoint{5.967218in}{1.490906in}}{\pgfqpoint{5.959318in}{1.487634in}}{\pgfqpoint{5.953494in}{1.481810in}}%
\pgfpathcurveto{\pgfqpoint{5.947670in}{1.475986in}}{\pgfqpoint{5.944398in}{1.468086in}}{\pgfqpoint{5.944398in}{1.459849in}}%
\pgfpathcurveto{\pgfqpoint{5.944398in}{1.451613in}}{\pgfqpoint{5.947670in}{1.443713in}}{\pgfqpoint{5.953494in}{1.437889in}}%
\pgfpathcurveto{\pgfqpoint{5.959318in}{1.432065in}}{\pgfqpoint{5.967218in}{1.428793in}}{\pgfqpoint{5.975454in}{1.428793in}}%
\pgfpathclose%
\pgfusepath{stroke,fill}%
\end{pgfscope}%
\begin{pgfscope}%
\pgfpathrectangle{\pgfqpoint{3.793912in}{0.557870in}}{\pgfqpoint{2.446088in}{1.684734in}}%
\pgfusepath{clip}%
\pgfsetbuttcap%
\pgfsetroundjoin%
\definecolor{currentfill}{rgb}{0.298039,0.447059,0.690196}%
\pgfsetfillcolor{currentfill}%
\pgfsetlinewidth{1.003750pt}%
\definecolor{currentstroke}{rgb}{0.298039,0.447059,0.690196}%
\pgfsetstrokecolor{currentstroke}%
\pgfsetdash{}{0pt}%
\pgfpathmoveto{\pgfqpoint{5.975454in}{1.419622in}}%
\pgfpathcurveto{\pgfqpoint{5.983691in}{1.419622in}}{\pgfqpoint{5.991591in}{1.422894in}}{\pgfqpoint{5.997415in}{1.428718in}}%
\pgfpathcurveto{\pgfqpoint{6.003239in}{1.434542in}}{\pgfqpoint{6.006511in}{1.442442in}}{\pgfqpoint{6.006511in}{1.450678in}}%
\pgfpathcurveto{\pgfqpoint{6.006511in}{1.458915in}}{\pgfqpoint{6.003239in}{1.466815in}}{\pgfqpoint{5.997415in}{1.472639in}}%
\pgfpathcurveto{\pgfqpoint{5.991591in}{1.478463in}}{\pgfqpoint{5.983691in}{1.481735in}}{\pgfqpoint{5.975454in}{1.481735in}}%
\pgfpathcurveto{\pgfqpoint{5.967218in}{1.481735in}}{\pgfqpoint{5.959318in}{1.478463in}}{\pgfqpoint{5.953494in}{1.472639in}}%
\pgfpathcurveto{\pgfqpoint{5.947670in}{1.466815in}}{\pgfqpoint{5.944398in}{1.458915in}}{\pgfqpoint{5.944398in}{1.450678in}}%
\pgfpathcurveto{\pgfqpoint{5.944398in}{1.442442in}}{\pgfqpoint{5.947670in}{1.434542in}}{\pgfqpoint{5.953494in}{1.428718in}}%
\pgfpathcurveto{\pgfqpoint{5.959318in}{1.422894in}}{\pgfqpoint{5.967218in}{1.419622in}}{\pgfqpoint{5.975454in}{1.419622in}}%
\pgfpathclose%
\pgfusepath{stroke,fill}%
\end{pgfscope}%
\begin{pgfscope}%
\pgfpathrectangle{\pgfqpoint{3.793912in}{0.557870in}}{\pgfqpoint{2.446088in}{1.684734in}}%
\pgfusepath{clip}%
\pgfsetbuttcap%
\pgfsetroundjoin%
\definecolor{currentfill}{rgb}{0.298039,0.447059,0.690196}%
\pgfsetfillcolor{currentfill}%
\pgfsetlinewidth{1.003750pt}%
\definecolor{currentstroke}{rgb}{0.298039,0.447059,0.690196}%
\pgfsetstrokecolor{currentstroke}%
\pgfsetdash{}{0pt}%
\pgfpathmoveto{\pgfqpoint{4.058458in}{1.722269in}}%
\pgfpathcurveto{\pgfqpoint{4.066694in}{1.722269in}}{\pgfqpoint{4.074594in}{1.725541in}}{\pgfqpoint{4.080418in}{1.731365in}}%
\pgfpathcurveto{\pgfqpoint{4.086242in}{1.737189in}}{\pgfqpoint{4.089514in}{1.745089in}}{\pgfqpoint{4.089514in}{1.753325in}}%
\pgfpathcurveto{\pgfqpoint{4.089514in}{1.761561in}}{\pgfqpoint{4.086242in}{1.769461in}}{\pgfqpoint{4.080418in}{1.775285in}}%
\pgfpathcurveto{\pgfqpoint{4.074594in}{1.781109in}}{\pgfqpoint{4.066694in}{1.784382in}}{\pgfqpoint{4.058458in}{1.784382in}}%
\pgfpathcurveto{\pgfqpoint{4.050221in}{1.784382in}}{\pgfqpoint{4.042321in}{1.781109in}}{\pgfqpoint{4.036498in}{1.775285in}}%
\pgfpathcurveto{\pgfqpoint{4.030674in}{1.769461in}}{\pgfqpoint{4.027401in}{1.761561in}}{\pgfqpoint{4.027401in}{1.753325in}}%
\pgfpathcurveto{\pgfqpoint{4.027401in}{1.745089in}}{\pgfqpoint{4.030674in}{1.737189in}}{\pgfqpoint{4.036498in}{1.731365in}}%
\pgfpathcurveto{\pgfqpoint{4.042321in}{1.725541in}}{\pgfqpoint{4.050221in}{1.722269in}}{\pgfqpoint{4.058458in}{1.722269in}}%
\pgfpathclose%
\pgfusepath{stroke,fill}%
\end{pgfscope}%
\begin{pgfscope}%
\pgfpathrectangle{\pgfqpoint{3.793912in}{0.557870in}}{\pgfqpoint{2.446088in}{1.684734in}}%
\pgfusepath{clip}%
\pgfsetbuttcap%
\pgfsetroundjoin%
\definecolor{currentfill}{rgb}{0.298039,0.447059,0.690196}%
\pgfsetfillcolor{currentfill}%
\pgfsetlinewidth{1.003750pt}%
\definecolor{currentstroke}{rgb}{0.298039,0.447059,0.690196}%
\pgfsetstrokecolor{currentstroke}%
\pgfsetdash{}{0pt}%
\pgfpathmoveto{\pgfqpoint{3.905098in}{2.125798in}}%
\pgfpathcurveto{\pgfqpoint{3.913334in}{2.125798in}}{\pgfqpoint{3.921234in}{2.129070in}}{\pgfqpoint{3.927058in}{2.134894in}}%
\pgfpathcurveto{\pgfqpoint{3.932882in}{2.140718in}}{\pgfqpoint{3.936155in}{2.148618in}}{\pgfqpoint{3.936155in}{2.156854in}}%
\pgfpathcurveto{\pgfqpoint{3.936155in}{2.165091in}}{\pgfqpoint{3.932882in}{2.172991in}}{\pgfqpoint{3.927058in}{2.178814in}}%
\pgfpathcurveto{\pgfqpoint{3.921234in}{2.184638in}}{\pgfqpoint{3.913334in}{2.187911in}}{\pgfqpoint{3.905098in}{2.187911in}}%
\pgfpathcurveto{\pgfqpoint{3.896862in}{2.187911in}}{\pgfqpoint{3.888962in}{2.184638in}}{\pgfqpoint{3.883138in}{2.178814in}}%
\pgfpathcurveto{\pgfqpoint{3.877314in}{2.172991in}}{\pgfqpoint{3.874042in}{2.165091in}}{\pgfqpoint{3.874042in}{2.156854in}}%
\pgfpathcurveto{\pgfqpoint{3.874042in}{2.148618in}}{\pgfqpoint{3.877314in}{2.140718in}}{\pgfqpoint{3.883138in}{2.134894in}}%
\pgfpathcurveto{\pgfqpoint{3.888962in}{2.129070in}}{\pgfqpoint{3.896862in}{2.125798in}}{\pgfqpoint{3.905098in}{2.125798in}}%
\pgfpathclose%
\pgfusepath{stroke,fill}%
\end{pgfscope}%
\begin{pgfscope}%
\pgfpathrectangle{\pgfqpoint{3.793912in}{0.557870in}}{\pgfqpoint{2.446088in}{1.684734in}}%
\pgfusepath{clip}%
\pgfsetbuttcap%
\pgfsetroundjoin%
\definecolor{currentfill}{rgb}{0.298039,0.447059,0.690196}%
\pgfsetfillcolor{currentfill}%
\pgfsetlinewidth{1.003750pt}%
\definecolor{currentstroke}{rgb}{0.298039,0.447059,0.690196}%
\pgfsetstrokecolor{currentstroke}%
\pgfsetdash{}{0pt}%
\pgfpathmoveto{\pgfqpoint{4.288497in}{1.850664in}}%
\pgfpathcurveto{\pgfqpoint{4.296734in}{1.850664in}}{\pgfqpoint{4.304634in}{1.853937in}}{\pgfqpoint{4.310458in}{1.859761in}}%
\pgfpathcurveto{\pgfqpoint{4.316282in}{1.865584in}}{\pgfqpoint{4.319554in}{1.873484in}}{\pgfqpoint{4.319554in}{1.881721in}}%
\pgfpathcurveto{\pgfqpoint{4.319554in}{1.889957in}}{\pgfqpoint{4.316282in}{1.897857in}}{\pgfqpoint{4.310458in}{1.903681in}}%
\pgfpathcurveto{\pgfqpoint{4.304634in}{1.909505in}}{\pgfqpoint{4.296734in}{1.912777in}}{\pgfqpoint{4.288497in}{1.912777in}}%
\pgfpathcurveto{\pgfqpoint{4.280261in}{1.912777in}}{\pgfqpoint{4.272361in}{1.909505in}}{\pgfqpoint{4.266537in}{1.903681in}}%
\pgfpathcurveto{\pgfqpoint{4.260713in}{1.897857in}}{\pgfqpoint{4.257441in}{1.889957in}}{\pgfqpoint{4.257441in}{1.881721in}}%
\pgfpathcurveto{\pgfqpoint{4.257441in}{1.873484in}}{\pgfqpoint{4.260713in}{1.865584in}}{\pgfqpoint{4.266537in}{1.859761in}}%
\pgfpathcurveto{\pgfqpoint{4.272361in}{1.853937in}}{\pgfqpoint{4.280261in}{1.850664in}}{\pgfqpoint{4.288497in}{1.850664in}}%
\pgfpathclose%
\pgfusepath{stroke,fill}%
\end{pgfscope}%
\begin{pgfscope}%
\pgfpathrectangle{\pgfqpoint{3.793912in}{0.557870in}}{\pgfqpoint{2.446088in}{1.684734in}}%
\pgfusepath{clip}%
\pgfsetbuttcap%
\pgfsetroundjoin%
\definecolor{currentfill}{rgb}{0.298039,0.447059,0.690196}%
\pgfsetfillcolor{currentfill}%
\pgfsetlinewidth{1.003750pt}%
\definecolor{currentstroke}{rgb}{0.298039,0.447059,0.690196}%
\pgfsetstrokecolor{currentstroke}%
\pgfsetdash{}{0pt}%
\pgfpathmoveto{\pgfqpoint{3.905098in}{2.125798in}}%
\pgfpathcurveto{\pgfqpoint{3.913334in}{2.125798in}}{\pgfqpoint{3.921234in}{2.129070in}}{\pgfqpoint{3.927058in}{2.134894in}}%
\pgfpathcurveto{\pgfqpoint{3.932882in}{2.140718in}}{\pgfqpoint{3.936155in}{2.148618in}}{\pgfqpoint{3.936155in}{2.156854in}}%
\pgfpathcurveto{\pgfqpoint{3.936155in}{2.165091in}}{\pgfqpoint{3.932882in}{2.172991in}}{\pgfqpoint{3.927058in}{2.178814in}}%
\pgfpathcurveto{\pgfqpoint{3.921234in}{2.184638in}}{\pgfqpoint{3.913334in}{2.187911in}}{\pgfqpoint{3.905098in}{2.187911in}}%
\pgfpathcurveto{\pgfqpoint{3.896862in}{2.187911in}}{\pgfqpoint{3.888962in}{2.184638in}}{\pgfqpoint{3.883138in}{2.178814in}}%
\pgfpathcurveto{\pgfqpoint{3.877314in}{2.172991in}}{\pgfqpoint{3.874042in}{2.165091in}}{\pgfqpoint{3.874042in}{2.156854in}}%
\pgfpathcurveto{\pgfqpoint{3.874042in}{2.148618in}}{\pgfqpoint{3.877314in}{2.140718in}}{\pgfqpoint{3.883138in}{2.134894in}}%
\pgfpathcurveto{\pgfqpoint{3.888962in}{2.129070in}}{\pgfqpoint{3.896862in}{2.125798in}}{\pgfqpoint{3.905098in}{2.125798in}}%
\pgfpathclose%
\pgfusepath{stroke,fill}%
\end{pgfscope}%
\begin{pgfscope}%
\pgfpathrectangle{\pgfqpoint{3.793912in}{0.557870in}}{\pgfqpoint{2.446088in}{1.684734in}}%
\pgfusepath{clip}%
\pgfsetbuttcap%
\pgfsetroundjoin%
\definecolor{currentfill}{rgb}{0.298039,0.447059,0.690196}%
\pgfsetfillcolor{currentfill}%
\pgfsetlinewidth{1.003750pt}%
\definecolor{currentstroke}{rgb}{0.298039,0.447059,0.690196}%
\pgfsetstrokecolor{currentstroke}%
\pgfsetdash{}{0pt}%
\pgfpathmoveto{\pgfqpoint{4.978616in}{1.832322in}}%
\pgfpathcurveto{\pgfqpoint{4.986852in}{1.832322in}}{\pgfqpoint{4.994753in}{1.835594in}}{\pgfqpoint{5.000576in}{1.841418in}}%
\pgfpathcurveto{\pgfqpoint{5.006400in}{1.847242in}}{\pgfqpoint{5.009673in}{1.855142in}}{\pgfqpoint{5.009673in}{1.863379in}}%
\pgfpathcurveto{\pgfqpoint{5.009673in}{1.871615in}}{\pgfqpoint{5.006400in}{1.879515in}}{\pgfqpoint{5.000576in}{1.885339in}}%
\pgfpathcurveto{\pgfqpoint{4.994753in}{1.891163in}}{\pgfqpoint{4.986852in}{1.894435in}}{\pgfqpoint{4.978616in}{1.894435in}}%
\pgfpathcurveto{\pgfqpoint{4.970380in}{1.894435in}}{\pgfqpoint{4.962480in}{1.891163in}}{\pgfqpoint{4.956656in}{1.885339in}}%
\pgfpathcurveto{\pgfqpoint{4.950832in}{1.879515in}}{\pgfqpoint{4.947560in}{1.871615in}}{\pgfqpoint{4.947560in}{1.863379in}}%
\pgfpathcurveto{\pgfqpoint{4.947560in}{1.855142in}}{\pgfqpoint{4.950832in}{1.847242in}}{\pgfqpoint{4.956656in}{1.841418in}}%
\pgfpathcurveto{\pgfqpoint{4.962480in}{1.835594in}}{\pgfqpoint{4.970380in}{1.832322in}}{\pgfqpoint{4.978616in}{1.832322in}}%
\pgfpathclose%
\pgfusepath{stroke,fill}%
\end{pgfscope}%
\begin{pgfscope}%
\pgfpathrectangle{\pgfqpoint{3.793912in}{0.557870in}}{\pgfqpoint{2.446088in}{1.684734in}}%
\pgfusepath{clip}%
\pgfsetbuttcap%
\pgfsetroundjoin%
\definecolor{currentfill}{rgb}{0.298039,0.447059,0.690196}%
\pgfsetfillcolor{currentfill}%
\pgfsetlinewidth{1.003750pt}%
\definecolor{currentstroke}{rgb}{0.298039,0.447059,0.690196}%
\pgfsetstrokecolor{currentstroke}%
\pgfsetdash{}{0pt}%
\pgfpathmoveto{\pgfqpoint{3.905098in}{2.125798in}}%
\pgfpathcurveto{\pgfqpoint{3.913334in}{2.125798in}}{\pgfqpoint{3.921234in}{2.129070in}}{\pgfqpoint{3.927058in}{2.134894in}}%
\pgfpathcurveto{\pgfqpoint{3.932882in}{2.140718in}}{\pgfqpoint{3.936155in}{2.148618in}}{\pgfqpoint{3.936155in}{2.156854in}}%
\pgfpathcurveto{\pgfqpoint{3.936155in}{2.165091in}}{\pgfqpoint{3.932882in}{2.172991in}}{\pgfqpoint{3.927058in}{2.178814in}}%
\pgfpathcurveto{\pgfqpoint{3.921234in}{2.184638in}}{\pgfqpoint{3.913334in}{2.187911in}}{\pgfqpoint{3.905098in}{2.187911in}}%
\pgfpathcurveto{\pgfqpoint{3.896862in}{2.187911in}}{\pgfqpoint{3.888962in}{2.184638in}}{\pgfqpoint{3.883138in}{2.178814in}}%
\pgfpathcurveto{\pgfqpoint{3.877314in}{2.172991in}}{\pgfqpoint{3.874042in}{2.165091in}}{\pgfqpoint{3.874042in}{2.156854in}}%
\pgfpathcurveto{\pgfqpoint{3.874042in}{2.148618in}}{\pgfqpoint{3.877314in}{2.140718in}}{\pgfqpoint{3.883138in}{2.134894in}}%
\pgfpathcurveto{\pgfqpoint{3.888962in}{2.129070in}}{\pgfqpoint{3.896862in}{2.125798in}}{\pgfqpoint{3.905098in}{2.125798in}}%
\pgfpathclose%
\pgfusepath{stroke,fill}%
\end{pgfscope}%
\begin{pgfscope}%
\pgfpathrectangle{\pgfqpoint{3.793912in}{0.557870in}}{\pgfqpoint{2.446088in}{1.684734in}}%
\pgfusepath{clip}%
\pgfsetbuttcap%
\pgfsetroundjoin%
\definecolor{currentfill}{rgb}{0.298039,0.447059,0.690196}%
\pgfsetfillcolor{currentfill}%
\pgfsetlinewidth{1.003750pt}%
\definecolor{currentstroke}{rgb}{0.298039,0.447059,0.690196}%
\pgfsetstrokecolor{currentstroke}%
\pgfsetdash{}{0pt}%
\pgfpathmoveto{\pgfqpoint{3.905098in}{2.125798in}}%
\pgfpathcurveto{\pgfqpoint{3.913334in}{2.125798in}}{\pgfqpoint{3.921234in}{2.129070in}}{\pgfqpoint{3.927058in}{2.134894in}}%
\pgfpathcurveto{\pgfqpoint{3.932882in}{2.140718in}}{\pgfqpoint{3.936155in}{2.148618in}}{\pgfqpoint{3.936155in}{2.156854in}}%
\pgfpathcurveto{\pgfqpoint{3.936155in}{2.165091in}}{\pgfqpoint{3.932882in}{2.172991in}}{\pgfqpoint{3.927058in}{2.178814in}}%
\pgfpathcurveto{\pgfqpoint{3.921234in}{2.184638in}}{\pgfqpoint{3.913334in}{2.187911in}}{\pgfqpoint{3.905098in}{2.187911in}}%
\pgfpathcurveto{\pgfqpoint{3.896862in}{2.187911in}}{\pgfqpoint{3.888962in}{2.184638in}}{\pgfqpoint{3.883138in}{2.178814in}}%
\pgfpathcurveto{\pgfqpoint{3.877314in}{2.172991in}}{\pgfqpoint{3.874042in}{2.165091in}}{\pgfqpoint{3.874042in}{2.156854in}}%
\pgfpathcurveto{\pgfqpoint{3.874042in}{2.148618in}}{\pgfqpoint{3.877314in}{2.140718in}}{\pgfqpoint{3.883138in}{2.134894in}}%
\pgfpathcurveto{\pgfqpoint{3.888962in}{2.129070in}}{\pgfqpoint{3.896862in}{2.125798in}}{\pgfqpoint{3.905098in}{2.125798in}}%
\pgfpathclose%
\pgfusepath{stroke,fill}%
\end{pgfscope}%
\begin{pgfscope}%
\pgfpathrectangle{\pgfqpoint{3.793912in}{0.557870in}}{\pgfqpoint{2.446088in}{1.684734in}}%
\pgfusepath{clip}%
\pgfsetbuttcap%
\pgfsetroundjoin%
\definecolor{currentfill}{rgb}{0.298039,0.447059,0.690196}%
\pgfsetfillcolor{currentfill}%
\pgfsetlinewidth{1.003750pt}%
\definecolor{currentstroke}{rgb}{0.298039,0.447059,0.690196}%
\pgfsetstrokecolor{currentstroke}%
\pgfsetdash{}{0pt}%
\pgfpathmoveto{\pgfqpoint{3.905098in}{2.125798in}}%
\pgfpathcurveto{\pgfqpoint{3.913334in}{2.125798in}}{\pgfqpoint{3.921234in}{2.129070in}}{\pgfqpoint{3.927058in}{2.134894in}}%
\pgfpathcurveto{\pgfqpoint{3.932882in}{2.140718in}}{\pgfqpoint{3.936155in}{2.148618in}}{\pgfqpoint{3.936155in}{2.156854in}}%
\pgfpathcurveto{\pgfqpoint{3.936155in}{2.165091in}}{\pgfqpoint{3.932882in}{2.172991in}}{\pgfqpoint{3.927058in}{2.178814in}}%
\pgfpathcurveto{\pgfqpoint{3.921234in}{2.184638in}}{\pgfqpoint{3.913334in}{2.187911in}}{\pgfqpoint{3.905098in}{2.187911in}}%
\pgfpathcurveto{\pgfqpoint{3.896862in}{2.187911in}}{\pgfqpoint{3.888962in}{2.184638in}}{\pgfqpoint{3.883138in}{2.178814in}}%
\pgfpathcurveto{\pgfqpoint{3.877314in}{2.172991in}}{\pgfqpoint{3.874042in}{2.165091in}}{\pgfqpoint{3.874042in}{2.156854in}}%
\pgfpathcurveto{\pgfqpoint{3.874042in}{2.148618in}}{\pgfqpoint{3.877314in}{2.140718in}}{\pgfqpoint{3.883138in}{2.134894in}}%
\pgfpathcurveto{\pgfqpoint{3.888962in}{2.129070in}}{\pgfqpoint{3.896862in}{2.125798in}}{\pgfqpoint{3.905098in}{2.125798in}}%
\pgfpathclose%
\pgfusepath{stroke,fill}%
\end{pgfscope}%
\begin{pgfscope}%
\pgfpathrectangle{\pgfqpoint{3.793912in}{0.557870in}}{\pgfqpoint{2.446088in}{1.684734in}}%
\pgfusepath{clip}%
\pgfsetbuttcap%
\pgfsetroundjoin%
\definecolor{currentfill}{rgb}{0.298039,0.447059,0.690196}%
\pgfsetfillcolor{currentfill}%
\pgfsetlinewidth{1.003750pt}%
\definecolor{currentstroke}{rgb}{0.298039,0.447059,0.690196}%
\pgfsetstrokecolor{currentstroke}%
\pgfsetdash{}{0pt}%
\pgfpathmoveto{\pgfqpoint{3.905098in}{2.125798in}}%
\pgfpathcurveto{\pgfqpoint{3.913334in}{2.125798in}}{\pgfqpoint{3.921234in}{2.129070in}}{\pgfqpoint{3.927058in}{2.134894in}}%
\pgfpathcurveto{\pgfqpoint{3.932882in}{2.140718in}}{\pgfqpoint{3.936155in}{2.148618in}}{\pgfqpoint{3.936155in}{2.156854in}}%
\pgfpathcurveto{\pgfqpoint{3.936155in}{2.165091in}}{\pgfqpoint{3.932882in}{2.172991in}}{\pgfqpoint{3.927058in}{2.178814in}}%
\pgfpathcurveto{\pgfqpoint{3.921234in}{2.184638in}}{\pgfqpoint{3.913334in}{2.187911in}}{\pgfqpoint{3.905098in}{2.187911in}}%
\pgfpathcurveto{\pgfqpoint{3.896862in}{2.187911in}}{\pgfqpoint{3.888962in}{2.184638in}}{\pgfqpoint{3.883138in}{2.178814in}}%
\pgfpathcurveto{\pgfqpoint{3.877314in}{2.172991in}}{\pgfqpoint{3.874042in}{2.165091in}}{\pgfqpoint{3.874042in}{2.156854in}}%
\pgfpathcurveto{\pgfqpoint{3.874042in}{2.148618in}}{\pgfqpoint{3.877314in}{2.140718in}}{\pgfqpoint{3.883138in}{2.134894in}}%
\pgfpathcurveto{\pgfqpoint{3.888962in}{2.129070in}}{\pgfqpoint{3.896862in}{2.125798in}}{\pgfqpoint{3.905098in}{2.125798in}}%
\pgfpathclose%
\pgfusepath{stroke,fill}%
\end{pgfscope}%
\begin{pgfscope}%
\pgfpathrectangle{\pgfqpoint{3.793912in}{0.557870in}}{\pgfqpoint{2.446088in}{1.684734in}}%
\pgfusepath{clip}%
\pgfsetbuttcap%
\pgfsetroundjoin%
\definecolor{currentfill}{rgb}{0.298039,0.447059,0.690196}%
\pgfsetfillcolor{currentfill}%
\pgfsetlinewidth{1.003750pt}%
\definecolor{currentstroke}{rgb}{0.298039,0.447059,0.690196}%
\pgfsetstrokecolor{currentstroke}%
\pgfsetdash{}{0pt}%
\pgfpathmoveto{\pgfqpoint{4.978616in}{1.603044in}}%
\pgfpathcurveto{\pgfqpoint{4.986852in}{1.603044in}}{\pgfqpoint{4.994753in}{1.606316in}}{\pgfqpoint{5.000576in}{1.612140in}}%
\pgfpathcurveto{\pgfqpoint{5.006400in}{1.617964in}}{\pgfqpoint{5.009673in}{1.625864in}}{\pgfqpoint{5.009673in}{1.634101in}}%
\pgfpathcurveto{\pgfqpoint{5.009673in}{1.642337in}}{\pgfqpoint{5.006400in}{1.650237in}}{\pgfqpoint{5.000576in}{1.656061in}}%
\pgfpathcurveto{\pgfqpoint{4.994753in}{1.661885in}}{\pgfqpoint{4.986852in}{1.665157in}}{\pgfqpoint{4.978616in}{1.665157in}}%
\pgfpathcurveto{\pgfqpoint{4.970380in}{1.665157in}}{\pgfqpoint{4.962480in}{1.661885in}}{\pgfqpoint{4.956656in}{1.656061in}}%
\pgfpathcurveto{\pgfqpoint{4.950832in}{1.650237in}}{\pgfqpoint{4.947560in}{1.642337in}}{\pgfqpoint{4.947560in}{1.634101in}}%
\pgfpathcurveto{\pgfqpoint{4.947560in}{1.625864in}}{\pgfqpoint{4.950832in}{1.617964in}}{\pgfqpoint{4.956656in}{1.612140in}}%
\pgfpathcurveto{\pgfqpoint{4.962480in}{1.606316in}}{\pgfqpoint{4.970380in}{1.603044in}}{\pgfqpoint{4.978616in}{1.603044in}}%
\pgfpathclose%
\pgfusepath{stroke,fill}%
\end{pgfscope}%
\begin{pgfscope}%
\pgfpathrectangle{\pgfqpoint{3.793912in}{0.557870in}}{\pgfqpoint{2.446088in}{1.684734in}}%
\pgfusepath{clip}%
\pgfsetbuttcap%
\pgfsetroundjoin%
\definecolor{currentfill}{rgb}{0.298039,0.447059,0.690196}%
\pgfsetfillcolor{currentfill}%
\pgfsetlinewidth{1.003750pt}%
\definecolor{currentstroke}{rgb}{0.298039,0.447059,0.690196}%
\pgfsetstrokecolor{currentstroke}%
\pgfsetdash{}{0pt}%
\pgfpathmoveto{\pgfqpoint{3.905098in}{2.125798in}}%
\pgfpathcurveto{\pgfqpoint{3.913334in}{2.125798in}}{\pgfqpoint{3.921234in}{2.129070in}}{\pgfqpoint{3.927058in}{2.134894in}}%
\pgfpathcurveto{\pgfqpoint{3.932882in}{2.140718in}}{\pgfqpoint{3.936155in}{2.148618in}}{\pgfqpoint{3.936155in}{2.156854in}}%
\pgfpathcurveto{\pgfqpoint{3.936155in}{2.165091in}}{\pgfqpoint{3.932882in}{2.172991in}}{\pgfqpoint{3.927058in}{2.178814in}}%
\pgfpathcurveto{\pgfqpoint{3.921234in}{2.184638in}}{\pgfqpoint{3.913334in}{2.187911in}}{\pgfqpoint{3.905098in}{2.187911in}}%
\pgfpathcurveto{\pgfqpoint{3.896862in}{2.187911in}}{\pgfqpoint{3.888962in}{2.184638in}}{\pgfqpoint{3.883138in}{2.178814in}}%
\pgfpathcurveto{\pgfqpoint{3.877314in}{2.172991in}}{\pgfqpoint{3.874042in}{2.165091in}}{\pgfqpoint{3.874042in}{2.156854in}}%
\pgfpathcurveto{\pgfqpoint{3.874042in}{2.148618in}}{\pgfqpoint{3.877314in}{2.140718in}}{\pgfqpoint{3.883138in}{2.134894in}}%
\pgfpathcurveto{\pgfqpoint{3.888962in}{2.129070in}}{\pgfqpoint{3.896862in}{2.125798in}}{\pgfqpoint{3.905098in}{2.125798in}}%
\pgfpathclose%
\pgfusepath{stroke,fill}%
\end{pgfscope}%
\begin{pgfscope}%
\pgfpathrectangle{\pgfqpoint{3.793912in}{0.557870in}}{\pgfqpoint{2.446088in}{1.684734in}}%
\pgfusepath{clip}%
\pgfsetbuttcap%
\pgfsetroundjoin%
\definecolor{currentfill}{rgb}{0.298039,0.447059,0.690196}%
\pgfsetfillcolor{currentfill}%
\pgfsetlinewidth{1.003750pt}%
\definecolor{currentstroke}{rgb}{0.298039,0.447059,0.690196}%
\pgfsetstrokecolor{currentstroke}%
\pgfsetdash{}{0pt}%
\pgfpathmoveto{\pgfqpoint{4.978616in}{1.639729in}}%
\pgfpathcurveto{\pgfqpoint{4.986852in}{1.639729in}}{\pgfqpoint{4.994753in}{1.643001in}}{\pgfqpoint{5.000576in}{1.648825in}}%
\pgfpathcurveto{\pgfqpoint{5.006400in}{1.654649in}}{\pgfqpoint{5.009673in}{1.662549in}}{\pgfqpoint{5.009673in}{1.670785in}}%
\pgfpathcurveto{\pgfqpoint{5.009673in}{1.679021in}}{\pgfqpoint{5.006400in}{1.686921in}}{\pgfqpoint{5.000576in}{1.692745in}}%
\pgfpathcurveto{\pgfqpoint{4.994753in}{1.698569in}}{\pgfqpoint{4.986852in}{1.701842in}}{\pgfqpoint{4.978616in}{1.701842in}}%
\pgfpathcurveto{\pgfqpoint{4.970380in}{1.701842in}}{\pgfqpoint{4.962480in}{1.698569in}}{\pgfqpoint{4.956656in}{1.692745in}}%
\pgfpathcurveto{\pgfqpoint{4.950832in}{1.686921in}}{\pgfqpoint{4.947560in}{1.679021in}}{\pgfqpoint{4.947560in}{1.670785in}}%
\pgfpathcurveto{\pgfqpoint{4.947560in}{1.662549in}}{\pgfqpoint{4.950832in}{1.654649in}}{\pgfqpoint{4.956656in}{1.648825in}}%
\pgfpathcurveto{\pgfqpoint{4.962480in}{1.643001in}}{\pgfqpoint{4.970380in}{1.639729in}}{\pgfqpoint{4.978616in}{1.639729in}}%
\pgfpathclose%
\pgfusepath{stroke,fill}%
\end{pgfscope}%
\begin{pgfscope}%
\pgfpathrectangle{\pgfqpoint{3.793912in}{0.557870in}}{\pgfqpoint{2.446088in}{1.684734in}}%
\pgfusepath{clip}%
\pgfsetbuttcap%
\pgfsetroundjoin%
\definecolor{currentfill}{rgb}{0.298039,0.447059,0.690196}%
\pgfsetfillcolor{currentfill}%
\pgfsetlinewidth{1.003750pt}%
\definecolor{currentstroke}{rgb}{0.298039,0.447059,0.690196}%
\pgfsetstrokecolor{currentstroke}%
\pgfsetdash{}{0pt}%
\pgfpathmoveto{\pgfqpoint{3.905098in}{2.125798in}}%
\pgfpathcurveto{\pgfqpoint{3.913334in}{2.125798in}}{\pgfqpoint{3.921234in}{2.129070in}}{\pgfqpoint{3.927058in}{2.134894in}}%
\pgfpathcurveto{\pgfqpoint{3.932882in}{2.140718in}}{\pgfqpoint{3.936155in}{2.148618in}}{\pgfqpoint{3.936155in}{2.156854in}}%
\pgfpathcurveto{\pgfqpoint{3.936155in}{2.165091in}}{\pgfqpoint{3.932882in}{2.172991in}}{\pgfqpoint{3.927058in}{2.178814in}}%
\pgfpathcurveto{\pgfqpoint{3.921234in}{2.184638in}}{\pgfqpoint{3.913334in}{2.187911in}}{\pgfqpoint{3.905098in}{2.187911in}}%
\pgfpathcurveto{\pgfqpoint{3.896862in}{2.187911in}}{\pgfqpoint{3.888962in}{2.184638in}}{\pgfqpoint{3.883138in}{2.178814in}}%
\pgfpathcurveto{\pgfqpoint{3.877314in}{2.172991in}}{\pgfqpoint{3.874042in}{2.165091in}}{\pgfqpoint{3.874042in}{2.156854in}}%
\pgfpathcurveto{\pgfqpoint{3.874042in}{2.148618in}}{\pgfqpoint{3.877314in}{2.140718in}}{\pgfqpoint{3.883138in}{2.134894in}}%
\pgfpathcurveto{\pgfqpoint{3.888962in}{2.129070in}}{\pgfqpoint{3.896862in}{2.125798in}}{\pgfqpoint{3.905098in}{2.125798in}}%
\pgfpathclose%
\pgfusepath{stroke,fill}%
\end{pgfscope}%
\begin{pgfscope}%
\pgfpathrectangle{\pgfqpoint{3.793912in}{0.557870in}}{\pgfqpoint{2.446088in}{1.684734in}}%
\pgfusepath{clip}%
\pgfsetbuttcap%
\pgfsetroundjoin%
\definecolor{currentfill}{rgb}{0.298039,0.447059,0.690196}%
\pgfsetfillcolor{currentfill}%
\pgfsetlinewidth{1.003750pt}%
\definecolor{currentstroke}{rgb}{0.298039,0.447059,0.690196}%
\pgfsetstrokecolor{currentstroke}%
\pgfsetdash{}{0pt}%
\pgfpathmoveto{\pgfqpoint{3.905098in}{2.125798in}}%
\pgfpathcurveto{\pgfqpoint{3.913334in}{2.125798in}}{\pgfqpoint{3.921234in}{2.129070in}}{\pgfqpoint{3.927058in}{2.134894in}}%
\pgfpathcurveto{\pgfqpoint{3.932882in}{2.140718in}}{\pgfqpoint{3.936155in}{2.148618in}}{\pgfqpoint{3.936155in}{2.156854in}}%
\pgfpathcurveto{\pgfqpoint{3.936155in}{2.165091in}}{\pgfqpoint{3.932882in}{2.172991in}}{\pgfqpoint{3.927058in}{2.178814in}}%
\pgfpathcurveto{\pgfqpoint{3.921234in}{2.184638in}}{\pgfqpoint{3.913334in}{2.187911in}}{\pgfqpoint{3.905098in}{2.187911in}}%
\pgfpathcurveto{\pgfqpoint{3.896862in}{2.187911in}}{\pgfqpoint{3.888962in}{2.184638in}}{\pgfqpoint{3.883138in}{2.178814in}}%
\pgfpathcurveto{\pgfqpoint{3.877314in}{2.172991in}}{\pgfqpoint{3.874042in}{2.165091in}}{\pgfqpoint{3.874042in}{2.156854in}}%
\pgfpathcurveto{\pgfqpoint{3.874042in}{2.148618in}}{\pgfqpoint{3.877314in}{2.140718in}}{\pgfqpoint{3.883138in}{2.134894in}}%
\pgfpathcurveto{\pgfqpoint{3.888962in}{2.129070in}}{\pgfqpoint{3.896862in}{2.125798in}}{\pgfqpoint{3.905098in}{2.125798in}}%
\pgfpathclose%
\pgfusepath{stroke,fill}%
\end{pgfscope}%
\begin{pgfscope}%
\pgfpathrectangle{\pgfqpoint{3.793912in}{0.557870in}}{\pgfqpoint{2.446088in}{1.684734in}}%
\pgfusepath{clip}%
\pgfsetbuttcap%
\pgfsetroundjoin%
\definecolor{currentfill}{rgb}{0.298039,0.447059,0.690196}%
\pgfsetfillcolor{currentfill}%
\pgfsetlinewidth{1.003750pt}%
\definecolor{currentstroke}{rgb}{0.298039,0.447059,0.690196}%
\pgfsetstrokecolor{currentstroke}%
\pgfsetdash{}{0pt}%
\pgfpathmoveto{\pgfqpoint{5.822095in}{1.795638in}}%
\pgfpathcurveto{\pgfqpoint{5.830331in}{1.795638in}}{\pgfqpoint{5.838231in}{1.798910in}}{\pgfqpoint{5.844055in}{1.804734in}}%
\pgfpathcurveto{\pgfqpoint{5.849879in}{1.810558in}}{\pgfqpoint{5.853151in}{1.818458in}}{\pgfqpoint{5.853151in}{1.826694in}}%
\pgfpathcurveto{\pgfqpoint{5.853151in}{1.834930in}}{\pgfqpoint{5.849879in}{1.842830in}}{\pgfqpoint{5.844055in}{1.848654in}}%
\pgfpathcurveto{\pgfqpoint{5.838231in}{1.854478in}}{\pgfqpoint{5.830331in}{1.857751in}}{\pgfqpoint{5.822095in}{1.857751in}}%
\pgfpathcurveto{\pgfqpoint{5.813858in}{1.857751in}}{\pgfqpoint{5.805958in}{1.854478in}}{\pgfqpoint{5.800134in}{1.848654in}}%
\pgfpathcurveto{\pgfqpoint{5.794311in}{1.842830in}}{\pgfqpoint{5.791038in}{1.834930in}}{\pgfqpoint{5.791038in}{1.826694in}}%
\pgfpathcurveto{\pgfqpoint{5.791038in}{1.818458in}}{\pgfqpoint{5.794311in}{1.810558in}}{\pgfqpoint{5.800134in}{1.804734in}}%
\pgfpathcurveto{\pgfqpoint{5.805958in}{1.798910in}}{\pgfqpoint{5.813858in}{1.795638in}}{\pgfqpoint{5.822095in}{1.795638in}}%
\pgfpathclose%
\pgfusepath{stroke,fill}%
\end{pgfscope}%
\begin{pgfscope}%
\pgfpathrectangle{\pgfqpoint{3.793912in}{0.557870in}}{\pgfqpoint{2.446088in}{1.684734in}}%
\pgfusepath{clip}%
\pgfsetbuttcap%
\pgfsetroundjoin%
\definecolor{currentfill}{rgb}{0.298039,0.447059,0.690196}%
\pgfsetfillcolor{currentfill}%
\pgfsetlinewidth{1.003750pt}%
\definecolor{currentstroke}{rgb}{0.298039,0.447059,0.690196}%
\pgfsetstrokecolor{currentstroke}%
\pgfsetdash{}{0pt}%
\pgfpathmoveto{\pgfqpoint{5.975454in}{1.263713in}}%
\pgfpathcurveto{\pgfqpoint{5.983691in}{1.263713in}}{\pgfqpoint{5.991591in}{1.266985in}}{\pgfqpoint{5.997415in}{1.272809in}}%
\pgfpathcurveto{\pgfqpoint{6.003239in}{1.278633in}}{\pgfqpoint{6.006511in}{1.286533in}}{\pgfqpoint{6.006511in}{1.294769in}}%
\pgfpathcurveto{\pgfqpoint{6.006511in}{1.303006in}}{\pgfqpoint{6.003239in}{1.310906in}}{\pgfqpoint{5.997415in}{1.316730in}}%
\pgfpathcurveto{\pgfqpoint{5.991591in}{1.322554in}}{\pgfqpoint{5.983691in}{1.325826in}}{\pgfqpoint{5.975454in}{1.325826in}}%
\pgfpathcurveto{\pgfqpoint{5.967218in}{1.325826in}}{\pgfqpoint{5.959318in}{1.322554in}}{\pgfqpoint{5.953494in}{1.316730in}}%
\pgfpathcurveto{\pgfqpoint{5.947670in}{1.310906in}}{\pgfqpoint{5.944398in}{1.303006in}}{\pgfqpoint{5.944398in}{1.294769in}}%
\pgfpathcurveto{\pgfqpoint{5.944398in}{1.286533in}}{\pgfqpoint{5.947670in}{1.278633in}}{\pgfqpoint{5.953494in}{1.272809in}}%
\pgfpathcurveto{\pgfqpoint{5.959318in}{1.266985in}}{\pgfqpoint{5.967218in}{1.263713in}}{\pgfqpoint{5.975454in}{1.263713in}}%
\pgfpathclose%
\pgfusepath{stroke,fill}%
\end{pgfscope}%
\begin{pgfscope}%
\pgfpathrectangle{\pgfqpoint{3.793912in}{0.557870in}}{\pgfqpoint{2.446088in}{1.684734in}}%
\pgfusepath{clip}%
\pgfsetbuttcap%
\pgfsetroundjoin%
\definecolor{currentfill}{rgb}{0.298039,0.447059,0.690196}%
\pgfsetfillcolor{currentfill}%
\pgfsetlinewidth{1.003750pt}%
\definecolor{currentstroke}{rgb}{0.298039,0.447059,0.690196}%
\pgfsetstrokecolor{currentstroke}%
\pgfsetdash{}{0pt}%
\pgfpathmoveto{\pgfqpoint{5.975454in}{1.282055in}}%
\pgfpathcurveto{\pgfqpoint{5.983691in}{1.282055in}}{\pgfqpoint{5.991591in}{1.285327in}}{\pgfqpoint{5.997415in}{1.291151in}}%
\pgfpathcurveto{\pgfqpoint{6.003239in}{1.296975in}}{\pgfqpoint{6.006511in}{1.304875in}}{\pgfqpoint{6.006511in}{1.313112in}}%
\pgfpathcurveto{\pgfqpoint{6.006511in}{1.321348in}}{\pgfqpoint{6.003239in}{1.329248in}}{\pgfqpoint{5.997415in}{1.335072in}}%
\pgfpathcurveto{\pgfqpoint{5.991591in}{1.340896in}}{\pgfqpoint{5.983691in}{1.344168in}}{\pgfqpoint{5.975454in}{1.344168in}}%
\pgfpathcurveto{\pgfqpoint{5.967218in}{1.344168in}}{\pgfqpoint{5.959318in}{1.340896in}}{\pgfqpoint{5.953494in}{1.335072in}}%
\pgfpathcurveto{\pgfqpoint{5.947670in}{1.329248in}}{\pgfqpoint{5.944398in}{1.321348in}}{\pgfqpoint{5.944398in}{1.313112in}}%
\pgfpathcurveto{\pgfqpoint{5.944398in}{1.304875in}}{\pgfqpoint{5.947670in}{1.296975in}}{\pgfqpoint{5.953494in}{1.291151in}}%
\pgfpathcurveto{\pgfqpoint{5.959318in}{1.285327in}}{\pgfqpoint{5.967218in}{1.282055in}}{\pgfqpoint{5.975454in}{1.282055in}}%
\pgfpathclose%
\pgfusepath{stroke,fill}%
\end{pgfscope}%
\begin{pgfscope}%
\pgfpathrectangle{\pgfqpoint{3.793912in}{0.557870in}}{\pgfqpoint{2.446088in}{1.684734in}}%
\pgfusepath{clip}%
\pgfsetbuttcap%
\pgfsetroundjoin%
\definecolor{currentfill}{rgb}{0.298039,0.447059,0.690196}%
\pgfsetfillcolor{currentfill}%
\pgfsetlinewidth{1.003750pt}%
\definecolor{currentstroke}{rgb}{0.298039,0.447059,0.690196}%
\pgfsetstrokecolor{currentstroke}%
\pgfsetdash{}{0pt}%
\pgfpathmoveto{\pgfqpoint{3.981778in}{2.052429in}}%
\pgfpathcurveto{\pgfqpoint{3.990014in}{2.052429in}}{\pgfqpoint{3.997914in}{2.055701in}}{\pgfqpoint{4.003738in}{2.061525in}}%
\pgfpathcurveto{\pgfqpoint{4.009562in}{2.067349in}}{\pgfqpoint{4.012834in}{2.075249in}}{\pgfqpoint{4.012834in}{2.083485in}}%
\pgfpathcurveto{\pgfqpoint{4.012834in}{2.091722in}}{\pgfqpoint{4.009562in}{2.099622in}}{\pgfqpoint{4.003738in}{2.105446in}}%
\pgfpathcurveto{\pgfqpoint{3.997914in}{2.111270in}}{\pgfqpoint{3.990014in}{2.114542in}}{\pgfqpoint{3.981778in}{2.114542in}}%
\pgfpathcurveto{\pgfqpoint{3.973542in}{2.114542in}}{\pgfqpoint{3.965642in}{2.111270in}}{\pgfqpoint{3.959818in}{2.105446in}}%
\pgfpathcurveto{\pgfqpoint{3.953994in}{2.099622in}}{\pgfqpoint{3.950721in}{2.091722in}}{\pgfqpoint{3.950721in}{2.083485in}}%
\pgfpathcurveto{\pgfqpoint{3.950721in}{2.075249in}}{\pgfqpoint{3.953994in}{2.067349in}}{\pgfqpoint{3.959818in}{2.061525in}}%
\pgfpathcurveto{\pgfqpoint{3.965642in}{2.055701in}}{\pgfqpoint{3.973542in}{2.052429in}}{\pgfqpoint{3.981778in}{2.052429in}}%
\pgfpathclose%
\pgfusepath{stroke,fill}%
\end{pgfscope}%
\begin{pgfscope}%
\pgfpathrectangle{\pgfqpoint{3.793912in}{0.557870in}}{\pgfqpoint{2.446088in}{1.684734in}}%
\pgfusepath{clip}%
\pgfsetbuttcap%
\pgfsetroundjoin%
\definecolor{currentfill}{rgb}{0.298039,0.447059,0.690196}%
\pgfsetfillcolor{currentfill}%
\pgfsetlinewidth{1.003750pt}%
\definecolor{currentstroke}{rgb}{0.298039,0.447059,0.690196}%
\pgfsetstrokecolor{currentstroke}%
\pgfsetdash{}{0pt}%
\pgfpathmoveto{\pgfqpoint{3.905098in}{2.125798in}}%
\pgfpathcurveto{\pgfqpoint{3.913334in}{2.125798in}}{\pgfqpoint{3.921234in}{2.129070in}}{\pgfqpoint{3.927058in}{2.134894in}}%
\pgfpathcurveto{\pgfqpoint{3.932882in}{2.140718in}}{\pgfqpoint{3.936155in}{2.148618in}}{\pgfqpoint{3.936155in}{2.156854in}}%
\pgfpathcurveto{\pgfqpoint{3.936155in}{2.165091in}}{\pgfqpoint{3.932882in}{2.172991in}}{\pgfqpoint{3.927058in}{2.178814in}}%
\pgfpathcurveto{\pgfqpoint{3.921234in}{2.184638in}}{\pgfqpoint{3.913334in}{2.187911in}}{\pgfqpoint{3.905098in}{2.187911in}}%
\pgfpathcurveto{\pgfqpoint{3.896862in}{2.187911in}}{\pgfqpoint{3.888962in}{2.184638in}}{\pgfqpoint{3.883138in}{2.178814in}}%
\pgfpathcurveto{\pgfqpoint{3.877314in}{2.172991in}}{\pgfqpoint{3.874042in}{2.165091in}}{\pgfqpoint{3.874042in}{2.156854in}}%
\pgfpathcurveto{\pgfqpoint{3.874042in}{2.148618in}}{\pgfqpoint{3.877314in}{2.140718in}}{\pgfqpoint{3.883138in}{2.134894in}}%
\pgfpathcurveto{\pgfqpoint{3.888962in}{2.129070in}}{\pgfqpoint{3.896862in}{2.125798in}}{\pgfqpoint{3.905098in}{2.125798in}}%
\pgfpathclose%
\pgfusepath{stroke,fill}%
\end{pgfscope}%
\begin{pgfscope}%
\pgfpathrectangle{\pgfqpoint{3.793912in}{0.557870in}}{\pgfqpoint{2.446088in}{1.684734in}}%
\pgfusepath{clip}%
\pgfsetbuttcap%
\pgfsetroundjoin%
\definecolor{currentfill}{rgb}{0.298039,0.447059,0.690196}%
\pgfsetfillcolor{currentfill}%
\pgfsetlinewidth{1.003750pt}%
\definecolor{currentstroke}{rgb}{0.298039,0.447059,0.690196}%
\pgfsetstrokecolor{currentstroke}%
\pgfsetdash{}{0pt}%
\pgfpathmoveto{\pgfqpoint{3.905098in}{2.125798in}}%
\pgfpathcurveto{\pgfqpoint{3.913334in}{2.125798in}}{\pgfqpoint{3.921234in}{2.129070in}}{\pgfqpoint{3.927058in}{2.134894in}}%
\pgfpathcurveto{\pgfqpoint{3.932882in}{2.140718in}}{\pgfqpoint{3.936155in}{2.148618in}}{\pgfqpoint{3.936155in}{2.156854in}}%
\pgfpathcurveto{\pgfqpoint{3.936155in}{2.165091in}}{\pgfqpoint{3.932882in}{2.172991in}}{\pgfqpoint{3.927058in}{2.178814in}}%
\pgfpathcurveto{\pgfqpoint{3.921234in}{2.184638in}}{\pgfqpoint{3.913334in}{2.187911in}}{\pgfqpoint{3.905098in}{2.187911in}}%
\pgfpathcurveto{\pgfqpoint{3.896862in}{2.187911in}}{\pgfqpoint{3.888962in}{2.184638in}}{\pgfqpoint{3.883138in}{2.178814in}}%
\pgfpathcurveto{\pgfqpoint{3.877314in}{2.172991in}}{\pgfqpoint{3.874042in}{2.165091in}}{\pgfqpoint{3.874042in}{2.156854in}}%
\pgfpathcurveto{\pgfqpoint{3.874042in}{2.148618in}}{\pgfqpoint{3.877314in}{2.140718in}}{\pgfqpoint{3.883138in}{2.134894in}}%
\pgfpathcurveto{\pgfqpoint{3.888962in}{2.129070in}}{\pgfqpoint{3.896862in}{2.125798in}}{\pgfqpoint{3.905098in}{2.125798in}}%
\pgfpathclose%
\pgfusepath{stroke,fill}%
\end{pgfscope}%
\begin{pgfscope}%
\pgfpathrectangle{\pgfqpoint{3.793912in}{0.557870in}}{\pgfqpoint{2.446088in}{1.684734in}}%
\pgfusepath{clip}%
\pgfsetbuttcap%
\pgfsetroundjoin%
\definecolor{currentfill}{rgb}{0.298039,0.447059,0.690196}%
\pgfsetfillcolor{currentfill}%
\pgfsetlinewidth{1.003750pt}%
\definecolor{currentstroke}{rgb}{0.298039,0.447059,0.690196}%
\pgfsetstrokecolor{currentstroke}%
\pgfsetdash{}{0pt}%
\pgfpathmoveto{\pgfqpoint{4.288497in}{1.703926in}}%
\pgfpathcurveto{\pgfqpoint{4.296734in}{1.703926in}}{\pgfqpoint{4.304634in}{1.707199in}}{\pgfqpoint{4.310458in}{1.713023in}}%
\pgfpathcurveto{\pgfqpoint{4.316282in}{1.718847in}}{\pgfqpoint{4.319554in}{1.726747in}}{\pgfqpoint{4.319554in}{1.734983in}}%
\pgfpathcurveto{\pgfqpoint{4.319554in}{1.743219in}}{\pgfqpoint{4.316282in}{1.751119in}}{\pgfqpoint{4.310458in}{1.756943in}}%
\pgfpathcurveto{\pgfqpoint{4.304634in}{1.762767in}}{\pgfqpoint{4.296734in}{1.766039in}}{\pgfqpoint{4.288497in}{1.766039in}}%
\pgfpathcurveto{\pgfqpoint{4.280261in}{1.766039in}}{\pgfqpoint{4.272361in}{1.762767in}}{\pgfqpoint{4.266537in}{1.756943in}}%
\pgfpathcurveto{\pgfqpoint{4.260713in}{1.751119in}}{\pgfqpoint{4.257441in}{1.743219in}}{\pgfqpoint{4.257441in}{1.734983in}}%
\pgfpathcurveto{\pgfqpoint{4.257441in}{1.726747in}}{\pgfqpoint{4.260713in}{1.718847in}}{\pgfqpoint{4.266537in}{1.713023in}}%
\pgfpathcurveto{\pgfqpoint{4.272361in}{1.707199in}}{\pgfqpoint{4.280261in}{1.703926in}}{\pgfqpoint{4.288497in}{1.703926in}}%
\pgfpathclose%
\pgfusepath{stroke,fill}%
\end{pgfscope}%
\begin{pgfscope}%
\pgfpathrectangle{\pgfqpoint{3.793912in}{0.557870in}}{\pgfqpoint{2.446088in}{1.684734in}}%
\pgfusepath{clip}%
\pgfsetbuttcap%
\pgfsetroundjoin%
\definecolor{currentfill}{rgb}{0.298039,0.447059,0.690196}%
\pgfsetfillcolor{currentfill}%
\pgfsetlinewidth{1.003750pt}%
\definecolor{currentstroke}{rgb}{0.298039,0.447059,0.690196}%
\pgfsetstrokecolor{currentstroke}%
\pgfsetdash{}{0pt}%
\pgfpathmoveto{\pgfqpoint{3.905098in}{2.125798in}}%
\pgfpathcurveto{\pgfqpoint{3.913334in}{2.125798in}}{\pgfqpoint{3.921234in}{2.129070in}}{\pgfqpoint{3.927058in}{2.134894in}}%
\pgfpathcurveto{\pgfqpoint{3.932882in}{2.140718in}}{\pgfqpoint{3.936155in}{2.148618in}}{\pgfqpoint{3.936155in}{2.156854in}}%
\pgfpathcurveto{\pgfqpoint{3.936155in}{2.165091in}}{\pgfqpoint{3.932882in}{2.172991in}}{\pgfqpoint{3.927058in}{2.178814in}}%
\pgfpathcurveto{\pgfqpoint{3.921234in}{2.184638in}}{\pgfqpoint{3.913334in}{2.187911in}}{\pgfqpoint{3.905098in}{2.187911in}}%
\pgfpathcurveto{\pgfqpoint{3.896862in}{2.187911in}}{\pgfqpoint{3.888962in}{2.184638in}}{\pgfqpoint{3.883138in}{2.178814in}}%
\pgfpathcurveto{\pgfqpoint{3.877314in}{2.172991in}}{\pgfqpoint{3.874042in}{2.165091in}}{\pgfqpoint{3.874042in}{2.156854in}}%
\pgfpathcurveto{\pgfqpoint{3.874042in}{2.148618in}}{\pgfqpoint{3.877314in}{2.140718in}}{\pgfqpoint{3.883138in}{2.134894in}}%
\pgfpathcurveto{\pgfqpoint{3.888962in}{2.129070in}}{\pgfqpoint{3.896862in}{2.125798in}}{\pgfqpoint{3.905098in}{2.125798in}}%
\pgfpathclose%
\pgfusepath{stroke,fill}%
\end{pgfscope}%
\begin{pgfscope}%
\pgfpathrectangle{\pgfqpoint{3.793912in}{0.557870in}}{\pgfqpoint{2.446088in}{1.684734in}}%
\pgfusepath{clip}%
\pgfsetbuttcap%
\pgfsetroundjoin%
\definecolor{currentfill}{rgb}{0.298039,0.447059,0.690196}%
\pgfsetfillcolor{currentfill}%
\pgfsetlinewidth{1.003750pt}%
\definecolor{currentstroke}{rgb}{0.298039,0.447059,0.690196}%
\pgfsetstrokecolor{currentstroke}%
\pgfsetdash{}{0pt}%
\pgfpathmoveto{\pgfqpoint{5.055296in}{1.364595in}}%
\pgfpathcurveto{\pgfqpoint{5.063532in}{1.364595in}}{\pgfqpoint{5.071432in}{1.367867in}}{\pgfqpoint{5.077256in}{1.373691in}}%
\pgfpathcurveto{\pgfqpoint{5.083080in}{1.379515in}}{\pgfqpoint{5.086353in}{1.387415in}}{\pgfqpoint{5.086353in}{1.395652in}}%
\pgfpathcurveto{\pgfqpoint{5.086353in}{1.403888in}}{\pgfqpoint{5.083080in}{1.411788in}}{\pgfqpoint{5.077256in}{1.417612in}}%
\pgfpathcurveto{\pgfqpoint{5.071432in}{1.423436in}}{\pgfqpoint{5.063532in}{1.426708in}}{\pgfqpoint{5.055296in}{1.426708in}}%
\pgfpathcurveto{\pgfqpoint{5.047060in}{1.426708in}}{\pgfqpoint{5.039160in}{1.423436in}}{\pgfqpoint{5.033336in}{1.417612in}}%
\pgfpathcurveto{\pgfqpoint{5.027512in}{1.411788in}}{\pgfqpoint{5.024240in}{1.403888in}}{\pgfqpoint{5.024240in}{1.395652in}}%
\pgfpathcurveto{\pgfqpoint{5.024240in}{1.387415in}}{\pgfqpoint{5.027512in}{1.379515in}}{\pgfqpoint{5.033336in}{1.373691in}}%
\pgfpathcurveto{\pgfqpoint{5.039160in}{1.367867in}}{\pgfqpoint{5.047060in}{1.364595in}}{\pgfqpoint{5.055296in}{1.364595in}}%
\pgfpathclose%
\pgfusepath{stroke,fill}%
\end{pgfscope}%
\begin{pgfscope}%
\pgfpathrectangle{\pgfqpoint{3.793912in}{0.557870in}}{\pgfqpoint{2.446088in}{1.684734in}}%
\pgfusepath{clip}%
\pgfsetbuttcap%
\pgfsetroundjoin%
\definecolor{currentfill}{rgb}{0.298039,0.447059,0.690196}%
\pgfsetfillcolor{currentfill}%
\pgfsetlinewidth{1.003750pt}%
\definecolor{currentstroke}{rgb}{0.298039,0.447059,0.690196}%
\pgfsetstrokecolor{currentstroke}%
\pgfsetdash{}{0pt}%
\pgfpathmoveto{\pgfqpoint{3.905098in}{2.125798in}}%
\pgfpathcurveto{\pgfqpoint{3.913334in}{2.125798in}}{\pgfqpoint{3.921234in}{2.129070in}}{\pgfqpoint{3.927058in}{2.134894in}}%
\pgfpathcurveto{\pgfqpoint{3.932882in}{2.140718in}}{\pgfqpoint{3.936155in}{2.148618in}}{\pgfqpoint{3.936155in}{2.156854in}}%
\pgfpathcurveto{\pgfqpoint{3.936155in}{2.165091in}}{\pgfqpoint{3.932882in}{2.172991in}}{\pgfqpoint{3.927058in}{2.178814in}}%
\pgfpathcurveto{\pgfqpoint{3.921234in}{2.184638in}}{\pgfqpoint{3.913334in}{2.187911in}}{\pgfqpoint{3.905098in}{2.187911in}}%
\pgfpathcurveto{\pgfqpoint{3.896862in}{2.187911in}}{\pgfqpoint{3.888962in}{2.184638in}}{\pgfqpoint{3.883138in}{2.178814in}}%
\pgfpathcurveto{\pgfqpoint{3.877314in}{2.172991in}}{\pgfqpoint{3.874042in}{2.165091in}}{\pgfqpoint{3.874042in}{2.156854in}}%
\pgfpathcurveto{\pgfqpoint{3.874042in}{2.148618in}}{\pgfqpoint{3.877314in}{2.140718in}}{\pgfqpoint{3.883138in}{2.134894in}}%
\pgfpathcurveto{\pgfqpoint{3.888962in}{2.129070in}}{\pgfqpoint{3.896862in}{2.125798in}}{\pgfqpoint{3.905098in}{2.125798in}}%
\pgfpathclose%
\pgfusepath{stroke,fill}%
\end{pgfscope}%
\begin{pgfscope}%
\pgfpathrectangle{\pgfqpoint{3.793912in}{0.557870in}}{\pgfqpoint{2.446088in}{1.684734in}}%
\pgfusepath{clip}%
\pgfsetbuttcap%
\pgfsetroundjoin%
\definecolor{currentfill}{rgb}{0.298039,0.447059,0.690196}%
\pgfsetfillcolor{currentfill}%
\pgfsetlinewidth{1.003750pt}%
\definecolor{currentstroke}{rgb}{0.298039,0.447059,0.690196}%
\pgfsetstrokecolor{currentstroke}%
\pgfsetdash{}{0pt}%
\pgfpathmoveto{\pgfqpoint{3.905098in}{2.125798in}}%
\pgfpathcurveto{\pgfqpoint{3.913334in}{2.125798in}}{\pgfqpoint{3.921234in}{2.129070in}}{\pgfqpoint{3.927058in}{2.134894in}}%
\pgfpathcurveto{\pgfqpoint{3.932882in}{2.140718in}}{\pgfqpoint{3.936155in}{2.148618in}}{\pgfqpoint{3.936155in}{2.156854in}}%
\pgfpathcurveto{\pgfqpoint{3.936155in}{2.165091in}}{\pgfqpoint{3.932882in}{2.172991in}}{\pgfqpoint{3.927058in}{2.178814in}}%
\pgfpathcurveto{\pgfqpoint{3.921234in}{2.184638in}}{\pgfqpoint{3.913334in}{2.187911in}}{\pgfqpoint{3.905098in}{2.187911in}}%
\pgfpathcurveto{\pgfqpoint{3.896862in}{2.187911in}}{\pgfqpoint{3.888962in}{2.184638in}}{\pgfqpoint{3.883138in}{2.178814in}}%
\pgfpathcurveto{\pgfqpoint{3.877314in}{2.172991in}}{\pgfqpoint{3.874042in}{2.165091in}}{\pgfqpoint{3.874042in}{2.156854in}}%
\pgfpathcurveto{\pgfqpoint{3.874042in}{2.148618in}}{\pgfqpoint{3.877314in}{2.140718in}}{\pgfqpoint{3.883138in}{2.134894in}}%
\pgfpathcurveto{\pgfqpoint{3.888962in}{2.129070in}}{\pgfqpoint{3.896862in}{2.125798in}}{\pgfqpoint{3.905098in}{2.125798in}}%
\pgfpathclose%
\pgfusepath{stroke,fill}%
\end{pgfscope}%
\begin{pgfscope}%
\pgfpathrectangle{\pgfqpoint{3.793912in}{0.557870in}}{\pgfqpoint{2.446088in}{1.684734in}}%
\pgfusepath{clip}%
\pgfsetbuttcap%
\pgfsetroundjoin%
\definecolor{currentfill}{rgb}{0.298039,0.447059,0.690196}%
\pgfsetfillcolor{currentfill}%
\pgfsetlinewidth{1.003750pt}%
\definecolor{currentstroke}{rgb}{0.298039,0.447059,0.690196}%
\pgfsetstrokecolor{currentstroke}%
\pgfsetdash{}{0pt}%
\pgfpathmoveto{\pgfqpoint{4.211818in}{2.125798in}}%
\pgfpathcurveto{\pgfqpoint{4.220054in}{2.125798in}}{\pgfqpoint{4.227954in}{2.129070in}}{\pgfqpoint{4.233778in}{2.134894in}}%
\pgfpathcurveto{\pgfqpoint{4.239602in}{2.140718in}}{\pgfqpoint{4.242874in}{2.148618in}}{\pgfqpoint{4.242874in}{2.156854in}}%
\pgfpathcurveto{\pgfqpoint{4.242874in}{2.165091in}}{\pgfqpoint{4.239602in}{2.172991in}}{\pgfqpoint{4.233778in}{2.178814in}}%
\pgfpathcurveto{\pgfqpoint{4.227954in}{2.184638in}}{\pgfqpoint{4.220054in}{2.187911in}}{\pgfqpoint{4.211818in}{2.187911in}}%
\pgfpathcurveto{\pgfqpoint{4.203581in}{2.187911in}}{\pgfqpoint{4.195681in}{2.184638in}}{\pgfqpoint{4.189857in}{2.178814in}}%
\pgfpathcurveto{\pgfqpoint{4.184033in}{2.172991in}}{\pgfqpoint{4.180761in}{2.165091in}}{\pgfqpoint{4.180761in}{2.156854in}}%
\pgfpathcurveto{\pgfqpoint{4.180761in}{2.148618in}}{\pgfqpoint{4.184033in}{2.140718in}}{\pgfqpoint{4.189857in}{2.134894in}}%
\pgfpathcurveto{\pgfqpoint{4.195681in}{2.129070in}}{\pgfqpoint{4.203581in}{2.125798in}}{\pgfqpoint{4.211818in}{2.125798in}}%
\pgfpathclose%
\pgfusepath{stroke,fill}%
\end{pgfscope}%
\begin{pgfscope}%
\pgfpathrectangle{\pgfqpoint{3.793912in}{0.557870in}}{\pgfqpoint{2.446088in}{1.684734in}}%
\pgfusepath{clip}%
\pgfsetbuttcap%
\pgfsetroundjoin%
\definecolor{currentfill}{rgb}{0.298039,0.447059,0.690196}%
\pgfsetfillcolor{currentfill}%
\pgfsetlinewidth{1.003750pt}%
\definecolor{currentstroke}{rgb}{0.298039,0.447059,0.690196}%
\pgfsetstrokecolor{currentstroke}%
\pgfsetdash{}{0pt}%
\pgfpathmoveto{\pgfqpoint{3.905098in}{2.125798in}}%
\pgfpathcurveto{\pgfqpoint{3.913334in}{2.125798in}}{\pgfqpoint{3.921234in}{2.129070in}}{\pgfqpoint{3.927058in}{2.134894in}}%
\pgfpathcurveto{\pgfqpoint{3.932882in}{2.140718in}}{\pgfqpoint{3.936155in}{2.148618in}}{\pgfqpoint{3.936155in}{2.156854in}}%
\pgfpathcurveto{\pgfqpoint{3.936155in}{2.165091in}}{\pgfqpoint{3.932882in}{2.172991in}}{\pgfqpoint{3.927058in}{2.178814in}}%
\pgfpathcurveto{\pgfqpoint{3.921234in}{2.184638in}}{\pgfqpoint{3.913334in}{2.187911in}}{\pgfqpoint{3.905098in}{2.187911in}}%
\pgfpathcurveto{\pgfqpoint{3.896862in}{2.187911in}}{\pgfqpoint{3.888962in}{2.184638in}}{\pgfqpoint{3.883138in}{2.178814in}}%
\pgfpathcurveto{\pgfqpoint{3.877314in}{2.172991in}}{\pgfqpoint{3.874042in}{2.165091in}}{\pgfqpoint{3.874042in}{2.156854in}}%
\pgfpathcurveto{\pgfqpoint{3.874042in}{2.148618in}}{\pgfqpoint{3.877314in}{2.140718in}}{\pgfqpoint{3.883138in}{2.134894in}}%
\pgfpathcurveto{\pgfqpoint{3.888962in}{2.129070in}}{\pgfqpoint{3.896862in}{2.125798in}}{\pgfqpoint{3.905098in}{2.125798in}}%
\pgfpathclose%
\pgfusepath{stroke,fill}%
\end{pgfscope}%
\begin{pgfscope}%
\pgfpathrectangle{\pgfqpoint{3.793912in}{0.557870in}}{\pgfqpoint{2.446088in}{1.684734in}}%
\pgfusepath{clip}%
\pgfsetbuttcap%
\pgfsetroundjoin%
\definecolor{currentfill}{rgb}{0.298039,0.447059,0.690196}%
\pgfsetfillcolor{currentfill}%
\pgfsetlinewidth{1.003750pt}%
\definecolor{currentstroke}{rgb}{0.298039,0.447059,0.690196}%
\pgfsetstrokecolor{currentstroke}%
\pgfsetdash{}{0pt}%
\pgfpathmoveto{\pgfqpoint{4.901936in}{1.447135in}}%
\pgfpathcurveto{\pgfqpoint{4.910173in}{1.447135in}}{\pgfqpoint{4.918073in}{1.450408in}}{\pgfqpoint{4.923897in}{1.456231in}}%
\pgfpathcurveto{\pgfqpoint{4.929721in}{1.462055in}}{\pgfqpoint{4.932993in}{1.469955in}}{\pgfqpoint{4.932993in}{1.478192in}}%
\pgfpathcurveto{\pgfqpoint{4.932993in}{1.486428in}}{\pgfqpoint{4.929721in}{1.494328in}}{\pgfqpoint{4.923897in}{1.500152in}}%
\pgfpathcurveto{\pgfqpoint{4.918073in}{1.505976in}}{\pgfqpoint{4.910173in}{1.509248in}}{\pgfqpoint{4.901936in}{1.509248in}}%
\pgfpathcurveto{\pgfqpoint{4.893700in}{1.509248in}}{\pgfqpoint{4.885800in}{1.505976in}}{\pgfqpoint{4.879976in}{1.500152in}}%
\pgfpathcurveto{\pgfqpoint{4.874152in}{1.494328in}}{\pgfqpoint{4.870880in}{1.486428in}}{\pgfqpoint{4.870880in}{1.478192in}}%
\pgfpathcurveto{\pgfqpoint{4.870880in}{1.469955in}}{\pgfqpoint{4.874152in}{1.462055in}}{\pgfqpoint{4.879976in}{1.456231in}}%
\pgfpathcurveto{\pgfqpoint{4.885800in}{1.450408in}}{\pgfqpoint{4.893700in}{1.447135in}}{\pgfqpoint{4.901936in}{1.447135in}}%
\pgfpathclose%
\pgfusepath{stroke,fill}%
\end{pgfscope}%
\begin{pgfscope}%
\pgfpathrectangle{\pgfqpoint{3.793912in}{0.557870in}}{\pgfqpoint{2.446088in}{1.684734in}}%
\pgfusepath{clip}%
\pgfsetbuttcap%
\pgfsetroundjoin%
\definecolor{currentfill}{rgb}{0.298039,0.447059,0.690196}%
\pgfsetfillcolor{currentfill}%
\pgfsetlinewidth{1.003750pt}%
\definecolor{currentstroke}{rgb}{0.298039,0.447059,0.690196}%
\pgfsetstrokecolor{currentstroke}%
\pgfsetdash{}{0pt}%
\pgfpathmoveto{\pgfqpoint{3.905098in}{2.116627in}}%
\pgfpathcurveto{\pgfqpoint{3.913334in}{2.116627in}}{\pgfqpoint{3.921234in}{2.119899in}}{\pgfqpoint{3.927058in}{2.125723in}}%
\pgfpathcurveto{\pgfqpoint{3.932882in}{2.131547in}}{\pgfqpoint{3.936155in}{2.139447in}}{\pgfqpoint{3.936155in}{2.147683in}}%
\pgfpathcurveto{\pgfqpoint{3.936155in}{2.155919in}}{\pgfqpoint{3.932882in}{2.163819in}}{\pgfqpoint{3.927058in}{2.169643in}}%
\pgfpathcurveto{\pgfqpoint{3.921234in}{2.175467in}}{\pgfqpoint{3.913334in}{2.178740in}}{\pgfqpoint{3.905098in}{2.178740in}}%
\pgfpathcurveto{\pgfqpoint{3.896862in}{2.178740in}}{\pgfqpoint{3.888962in}{2.175467in}}{\pgfqpoint{3.883138in}{2.169643in}}%
\pgfpathcurveto{\pgfqpoint{3.877314in}{2.163819in}}{\pgfqpoint{3.874042in}{2.155919in}}{\pgfqpoint{3.874042in}{2.147683in}}%
\pgfpathcurveto{\pgfqpoint{3.874042in}{2.139447in}}{\pgfqpoint{3.877314in}{2.131547in}}{\pgfqpoint{3.883138in}{2.125723in}}%
\pgfpathcurveto{\pgfqpoint{3.888962in}{2.119899in}}{\pgfqpoint{3.896862in}{2.116627in}}{\pgfqpoint{3.905098in}{2.116627in}}%
\pgfpathclose%
\pgfusepath{stroke,fill}%
\end{pgfscope}%
\begin{pgfscope}%
\pgfpathrectangle{\pgfqpoint{3.793912in}{0.557870in}}{\pgfqpoint{2.446088in}{1.684734in}}%
\pgfusepath{clip}%
\pgfsetbuttcap%
\pgfsetroundjoin%
\definecolor{currentfill}{rgb}{0.298039,0.447059,0.690196}%
\pgfsetfillcolor{currentfill}%
\pgfsetlinewidth{1.003750pt}%
\definecolor{currentstroke}{rgb}{0.298039,0.447059,0.690196}%
\pgfsetstrokecolor{currentstroke}%
\pgfsetdash{}{0pt}%
\pgfpathmoveto{\pgfqpoint{4.978616in}{1.474649in}}%
\pgfpathcurveto{\pgfqpoint{4.986852in}{1.474649in}}{\pgfqpoint{4.994753in}{1.477921in}}{\pgfqpoint{5.000576in}{1.483745in}}%
\pgfpathcurveto{\pgfqpoint{5.006400in}{1.489569in}}{\pgfqpoint{5.009673in}{1.497469in}}{\pgfqpoint{5.009673in}{1.505705in}}%
\pgfpathcurveto{\pgfqpoint{5.009673in}{1.513941in}}{\pgfqpoint{5.006400in}{1.521841in}}{\pgfqpoint{5.000576in}{1.527665in}}%
\pgfpathcurveto{\pgfqpoint{4.994753in}{1.533489in}}{\pgfqpoint{4.986852in}{1.536762in}}{\pgfqpoint{4.978616in}{1.536762in}}%
\pgfpathcurveto{\pgfqpoint{4.970380in}{1.536762in}}{\pgfqpoint{4.962480in}{1.533489in}}{\pgfqpoint{4.956656in}{1.527665in}}%
\pgfpathcurveto{\pgfqpoint{4.950832in}{1.521841in}}{\pgfqpoint{4.947560in}{1.513941in}}{\pgfqpoint{4.947560in}{1.505705in}}%
\pgfpathcurveto{\pgfqpoint{4.947560in}{1.497469in}}{\pgfqpoint{4.950832in}{1.489569in}}{\pgfqpoint{4.956656in}{1.483745in}}%
\pgfpathcurveto{\pgfqpoint{4.962480in}{1.477921in}}{\pgfqpoint{4.970380in}{1.474649in}}{\pgfqpoint{4.978616in}{1.474649in}}%
\pgfpathclose%
\pgfusepath{stroke,fill}%
\end{pgfscope}%
\begin{pgfscope}%
\pgfpathrectangle{\pgfqpoint{3.793912in}{0.557870in}}{\pgfqpoint{2.446088in}{1.684734in}}%
\pgfusepath{clip}%
\pgfsetbuttcap%
\pgfsetroundjoin%
\definecolor{currentfill}{rgb}{0.298039,0.447059,0.690196}%
\pgfsetfillcolor{currentfill}%
\pgfsetlinewidth{1.003750pt}%
\definecolor{currentstroke}{rgb}{0.298039,0.447059,0.690196}%
\pgfsetstrokecolor{currentstroke}%
\pgfsetdash{}{0pt}%
\pgfpathmoveto{\pgfqpoint{3.905098in}{2.125798in}}%
\pgfpathcurveto{\pgfqpoint{3.913334in}{2.125798in}}{\pgfqpoint{3.921234in}{2.129070in}}{\pgfqpoint{3.927058in}{2.134894in}}%
\pgfpathcurveto{\pgfqpoint{3.932882in}{2.140718in}}{\pgfqpoint{3.936155in}{2.148618in}}{\pgfqpoint{3.936155in}{2.156854in}}%
\pgfpathcurveto{\pgfqpoint{3.936155in}{2.165091in}}{\pgfqpoint{3.932882in}{2.172991in}}{\pgfqpoint{3.927058in}{2.178814in}}%
\pgfpathcurveto{\pgfqpoint{3.921234in}{2.184638in}}{\pgfqpoint{3.913334in}{2.187911in}}{\pgfqpoint{3.905098in}{2.187911in}}%
\pgfpathcurveto{\pgfqpoint{3.896862in}{2.187911in}}{\pgfqpoint{3.888962in}{2.184638in}}{\pgfqpoint{3.883138in}{2.178814in}}%
\pgfpathcurveto{\pgfqpoint{3.877314in}{2.172991in}}{\pgfqpoint{3.874042in}{2.165091in}}{\pgfqpoint{3.874042in}{2.156854in}}%
\pgfpathcurveto{\pgfqpoint{3.874042in}{2.148618in}}{\pgfqpoint{3.877314in}{2.140718in}}{\pgfqpoint{3.883138in}{2.134894in}}%
\pgfpathcurveto{\pgfqpoint{3.888962in}{2.129070in}}{\pgfqpoint{3.896862in}{2.125798in}}{\pgfqpoint{3.905098in}{2.125798in}}%
\pgfpathclose%
\pgfusepath{stroke,fill}%
\end{pgfscope}%
\begin{pgfscope}%
\pgfpathrectangle{\pgfqpoint{3.793912in}{0.557870in}}{\pgfqpoint{2.446088in}{1.684734in}}%
\pgfusepath{clip}%
\pgfsetbuttcap%
\pgfsetroundjoin%
\definecolor{currentfill}{rgb}{0.298039,0.447059,0.690196}%
\pgfsetfillcolor{currentfill}%
\pgfsetlinewidth{1.003750pt}%
\definecolor{currentstroke}{rgb}{0.298039,0.447059,0.690196}%
\pgfsetstrokecolor{currentstroke}%
\pgfsetdash{}{0pt}%
\pgfpathmoveto{\pgfqpoint{3.905098in}{2.125798in}}%
\pgfpathcurveto{\pgfqpoint{3.913334in}{2.125798in}}{\pgfqpoint{3.921234in}{2.129070in}}{\pgfqpoint{3.927058in}{2.134894in}}%
\pgfpathcurveto{\pgfqpoint{3.932882in}{2.140718in}}{\pgfqpoint{3.936155in}{2.148618in}}{\pgfqpoint{3.936155in}{2.156854in}}%
\pgfpathcurveto{\pgfqpoint{3.936155in}{2.165091in}}{\pgfqpoint{3.932882in}{2.172991in}}{\pgfqpoint{3.927058in}{2.178814in}}%
\pgfpathcurveto{\pgfqpoint{3.921234in}{2.184638in}}{\pgfqpoint{3.913334in}{2.187911in}}{\pgfqpoint{3.905098in}{2.187911in}}%
\pgfpathcurveto{\pgfqpoint{3.896862in}{2.187911in}}{\pgfqpoint{3.888962in}{2.184638in}}{\pgfqpoint{3.883138in}{2.178814in}}%
\pgfpathcurveto{\pgfqpoint{3.877314in}{2.172991in}}{\pgfqpoint{3.874042in}{2.165091in}}{\pgfqpoint{3.874042in}{2.156854in}}%
\pgfpathcurveto{\pgfqpoint{3.874042in}{2.148618in}}{\pgfqpoint{3.877314in}{2.140718in}}{\pgfqpoint{3.883138in}{2.134894in}}%
\pgfpathcurveto{\pgfqpoint{3.888962in}{2.129070in}}{\pgfqpoint{3.896862in}{2.125798in}}{\pgfqpoint{3.905098in}{2.125798in}}%
\pgfpathclose%
\pgfusepath{stroke,fill}%
\end{pgfscope}%
\begin{pgfscope}%
\pgfpathrectangle{\pgfqpoint{3.793912in}{0.557870in}}{\pgfqpoint{2.446088in}{1.684734in}}%
\pgfusepath{clip}%
\pgfsetbuttcap%
\pgfsetroundjoin%
\definecolor{currentfill}{rgb}{0.298039,0.447059,0.690196}%
\pgfsetfillcolor{currentfill}%
\pgfsetlinewidth{1.003750pt}%
\definecolor{currentstroke}{rgb}{0.298039,0.447059,0.690196}%
\pgfsetstrokecolor{currentstroke}%
\pgfsetdash{}{0pt}%
\pgfpathmoveto{\pgfqpoint{3.905098in}{2.116627in}}%
\pgfpathcurveto{\pgfqpoint{3.913334in}{2.116627in}}{\pgfqpoint{3.921234in}{2.119899in}}{\pgfqpoint{3.927058in}{2.125723in}}%
\pgfpathcurveto{\pgfqpoint{3.932882in}{2.131547in}}{\pgfqpoint{3.936155in}{2.139447in}}{\pgfqpoint{3.936155in}{2.147683in}}%
\pgfpathcurveto{\pgfqpoint{3.936155in}{2.155919in}}{\pgfqpoint{3.932882in}{2.163819in}}{\pgfqpoint{3.927058in}{2.169643in}}%
\pgfpathcurveto{\pgfqpoint{3.921234in}{2.175467in}}{\pgfqpoint{3.913334in}{2.178740in}}{\pgfqpoint{3.905098in}{2.178740in}}%
\pgfpathcurveto{\pgfqpoint{3.896862in}{2.178740in}}{\pgfqpoint{3.888962in}{2.175467in}}{\pgfqpoint{3.883138in}{2.169643in}}%
\pgfpathcurveto{\pgfqpoint{3.877314in}{2.163819in}}{\pgfqpoint{3.874042in}{2.155919in}}{\pgfqpoint{3.874042in}{2.147683in}}%
\pgfpathcurveto{\pgfqpoint{3.874042in}{2.139447in}}{\pgfqpoint{3.877314in}{2.131547in}}{\pgfqpoint{3.883138in}{2.125723in}}%
\pgfpathcurveto{\pgfqpoint{3.888962in}{2.119899in}}{\pgfqpoint{3.896862in}{2.116627in}}{\pgfqpoint{3.905098in}{2.116627in}}%
\pgfpathclose%
\pgfusepath{stroke,fill}%
\end{pgfscope}%
\begin{pgfscope}%
\pgfpathrectangle{\pgfqpoint{3.793912in}{0.557870in}}{\pgfqpoint{2.446088in}{1.684734in}}%
\pgfusepath{clip}%
\pgfsetbuttcap%
\pgfsetroundjoin%
\definecolor{currentfill}{rgb}{0.298039,0.447059,0.690196}%
\pgfsetfillcolor{currentfill}%
\pgfsetlinewidth{1.003750pt}%
\definecolor{currentstroke}{rgb}{0.298039,0.447059,0.690196}%
\pgfsetstrokecolor{currentstroke}%
\pgfsetdash{}{0pt}%
\pgfpathmoveto{\pgfqpoint{3.905098in}{2.125798in}}%
\pgfpathcurveto{\pgfqpoint{3.913334in}{2.125798in}}{\pgfqpoint{3.921234in}{2.129070in}}{\pgfqpoint{3.927058in}{2.134894in}}%
\pgfpathcurveto{\pgfqpoint{3.932882in}{2.140718in}}{\pgfqpoint{3.936155in}{2.148618in}}{\pgfqpoint{3.936155in}{2.156854in}}%
\pgfpathcurveto{\pgfqpoint{3.936155in}{2.165091in}}{\pgfqpoint{3.932882in}{2.172991in}}{\pgfqpoint{3.927058in}{2.178814in}}%
\pgfpathcurveto{\pgfqpoint{3.921234in}{2.184638in}}{\pgfqpoint{3.913334in}{2.187911in}}{\pgfqpoint{3.905098in}{2.187911in}}%
\pgfpathcurveto{\pgfqpoint{3.896862in}{2.187911in}}{\pgfqpoint{3.888962in}{2.184638in}}{\pgfqpoint{3.883138in}{2.178814in}}%
\pgfpathcurveto{\pgfqpoint{3.877314in}{2.172991in}}{\pgfqpoint{3.874042in}{2.165091in}}{\pgfqpoint{3.874042in}{2.156854in}}%
\pgfpathcurveto{\pgfqpoint{3.874042in}{2.148618in}}{\pgfqpoint{3.877314in}{2.140718in}}{\pgfqpoint{3.883138in}{2.134894in}}%
\pgfpathcurveto{\pgfqpoint{3.888962in}{2.129070in}}{\pgfqpoint{3.896862in}{2.125798in}}{\pgfqpoint{3.905098in}{2.125798in}}%
\pgfpathclose%
\pgfusepath{stroke,fill}%
\end{pgfscope}%
\begin{pgfscope}%
\pgfpathrectangle{\pgfqpoint{3.793912in}{0.557870in}}{\pgfqpoint{2.446088in}{1.684734in}}%
\pgfusepath{clip}%
\pgfsetbuttcap%
\pgfsetroundjoin%
\definecolor{currentfill}{rgb}{0.298039,0.447059,0.690196}%
\pgfsetfillcolor{currentfill}%
\pgfsetlinewidth{1.003750pt}%
\definecolor{currentstroke}{rgb}{0.298039,0.447059,0.690196}%
\pgfsetstrokecolor{currentstroke}%
\pgfsetdash{}{0pt}%
\pgfpathmoveto{\pgfqpoint{5.975454in}{1.327911in}}%
\pgfpathcurveto{\pgfqpoint{5.983691in}{1.327911in}}{\pgfqpoint{5.991591in}{1.331183in}}{\pgfqpoint{5.997415in}{1.337007in}}%
\pgfpathcurveto{\pgfqpoint{6.003239in}{1.342831in}}{\pgfqpoint{6.006511in}{1.350731in}}{\pgfqpoint{6.006511in}{1.358967in}}%
\pgfpathcurveto{\pgfqpoint{6.006511in}{1.367203in}}{\pgfqpoint{6.003239in}{1.375104in}}{\pgfqpoint{5.997415in}{1.380927in}}%
\pgfpathcurveto{\pgfqpoint{5.991591in}{1.386751in}}{\pgfqpoint{5.983691in}{1.390024in}}{\pgfqpoint{5.975454in}{1.390024in}}%
\pgfpathcurveto{\pgfqpoint{5.967218in}{1.390024in}}{\pgfqpoint{5.959318in}{1.386751in}}{\pgfqpoint{5.953494in}{1.380927in}}%
\pgfpathcurveto{\pgfqpoint{5.947670in}{1.375104in}}{\pgfqpoint{5.944398in}{1.367203in}}{\pgfqpoint{5.944398in}{1.358967in}}%
\pgfpathcurveto{\pgfqpoint{5.944398in}{1.350731in}}{\pgfqpoint{5.947670in}{1.342831in}}{\pgfqpoint{5.953494in}{1.337007in}}%
\pgfpathcurveto{\pgfqpoint{5.959318in}{1.331183in}}{\pgfqpoint{5.967218in}{1.327911in}}{\pgfqpoint{5.975454in}{1.327911in}}%
\pgfpathclose%
\pgfusepath{stroke,fill}%
\end{pgfscope}%
\begin{pgfscope}%
\pgfpathrectangle{\pgfqpoint{3.793912in}{0.557870in}}{\pgfqpoint{2.446088in}{1.684734in}}%
\pgfusepath{clip}%
\pgfsetbuttcap%
\pgfsetroundjoin%
\definecolor{currentfill}{rgb}{0.298039,0.447059,0.690196}%
\pgfsetfillcolor{currentfill}%
\pgfsetlinewidth{1.003750pt}%
\definecolor{currentstroke}{rgb}{0.298039,0.447059,0.690196}%
\pgfsetstrokecolor{currentstroke}%
\pgfsetdash{}{0pt}%
\pgfpathmoveto{\pgfqpoint{5.975454in}{1.263713in}}%
\pgfpathcurveto{\pgfqpoint{5.983691in}{1.263713in}}{\pgfqpoint{5.991591in}{1.266985in}}{\pgfqpoint{5.997415in}{1.272809in}}%
\pgfpathcurveto{\pgfqpoint{6.003239in}{1.278633in}}{\pgfqpoint{6.006511in}{1.286533in}}{\pgfqpoint{6.006511in}{1.294769in}}%
\pgfpathcurveto{\pgfqpoint{6.006511in}{1.303006in}}{\pgfqpoint{6.003239in}{1.310906in}}{\pgfqpoint{5.997415in}{1.316730in}}%
\pgfpathcurveto{\pgfqpoint{5.991591in}{1.322554in}}{\pgfqpoint{5.983691in}{1.325826in}}{\pgfqpoint{5.975454in}{1.325826in}}%
\pgfpathcurveto{\pgfqpoint{5.967218in}{1.325826in}}{\pgfqpoint{5.959318in}{1.322554in}}{\pgfqpoint{5.953494in}{1.316730in}}%
\pgfpathcurveto{\pgfqpoint{5.947670in}{1.310906in}}{\pgfqpoint{5.944398in}{1.303006in}}{\pgfqpoint{5.944398in}{1.294769in}}%
\pgfpathcurveto{\pgfqpoint{5.944398in}{1.286533in}}{\pgfqpoint{5.947670in}{1.278633in}}{\pgfqpoint{5.953494in}{1.272809in}}%
\pgfpathcurveto{\pgfqpoint{5.959318in}{1.266985in}}{\pgfqpoint{5.967218in}{1.263713in}}{\pgfqpoint{5.975454in}{1.263713in}}%
\pgfpathclose%
\pgfusepath{stroke,fill}%
\end{pgfscope}%
\begin{pgfscope}%
\pgfpathrectangle{\pgfqpoint{3.793912in}{0.557870in}}{\pgfqpoint{2.446088in}{1.684734in}}%
\pgfusepath{clip}%
\pgfsetbuttcap%
\pgfsetroundjoin%
\definecolor{currentfill}{rgb}{0.298039,0.447059,0.690196}%
\pgfsetfillcolor{currentfill}%
\pgfsetlinewidth{1.003750pt}%
\definecolor{currentstroke}{rgb}{0.298039,0.447059,0.690196}%
\pgfsetstrokecolor{currentstroke}%
\pgfsetdash{}{0pt}%
\pgfpathmoveto{\pgfqpoint{5.975454in}{1.318740in}}%
\pgfpathcurveto{\pgfqpoint{5.983691in}{1.318740in}}{\pgfqpoint{5.991591in}{1.322012in}}{\pgfqpoint{5.997415in}{1.327836in}}%
\pgfpathcurveto{\pgfqpoint{6.003239in}{1.333660in}}{\pgfqpoint{6.006511in}{1.341560in}}{\pgfqpoint{6.006511in}{1.349796in}}%
\pgfpathcurveto{\pgfqpoint{6.006511in}{1.358032in}}{\pgfqpoint{6.003239in}{1.365932in}}{\pgfqpoint{5.997415in}{1.371756in}}%
\pgfpathcurveto{\pgfqpoint{5.991591in}{1.377580in}}{\pgfqpoint{5.983691in}{1.380853in}}{\pgfqpoint{5.975454in}{1.380853in}}%
\pgfpathcurveto{\pgfqpoint{5.967218in}{1.380853in}}{\pgfqpoint{5.959318in}{1.377580in}}{\pgfqpoint{5.953494in}{1.371756in}}%
\pgfpathcurveto{\pgfqpoint{5.947670in}{1.365932in}}{\pgfqpoint{5.944398in}{1.358032in}}{\pgfqpoint{5.944398in}{1.349796in}}%
\pgfpathcurveto{\pgfqpoint{5.944398in}{1.341560in}}{\pgfqpoint{5.947670in}{1.333660in}}{\pgfqpoint{5.953494in}{1.327836in}}%
\pgfpathcurveto{\pgfqpoint{5.959318in}{1.322012in}}{\pgfqpoint{5.967218in}{1.318740in}}{\pgfqpoint{5.975454in}{1.318740in}}%
\pgfpathclose%
\pgfusepath{stroke,fill}%
\end{pgfscope}%
\begin{pgfscope}%
\pgfpathrectangle{\pgfqpoint{3.793912in}{0.557870in}}{\pgfqpoint{2.446088in}{1.684734in}}%
\pgfusepath{clip}%
\pgfsetbuttcap%
\pgfsetroundjoin%
\definecolor{currentfill}{rgb}{0.298039,0.447059,0.690196}%
\pgfsetfillcolor{currentfill}%
\pgfsetlinewidth{1.003750pt}%
\definecolor{currentstroke}{rgb}{0.298039,0.447059,0.690196}%
\pgfsetstrokecolor{currentstroke}%
\pgfsetdash{}{0pt}%
\pgfpathmoveto{\pgfqpoint{3.905098in}{2.107456in}}%
\pgfpathcurveto{\pgfqpoint{3.913334in}{2.107456in}}{\pgfqpoint{3.921234in}{2.110728in}}{\pgfqpoint{3.927058in}{2.116552in}}%
\pgfpathcurveto{\pgfqpoint{3.932882in}{2.122376in}}{\pgfqpoint{3.936155in}{2.130276in}}{\pgfqpoint{3.936155in}{2.138512in}}%
\pgfpathcurveto{\pgfqpoint{3.936155in}{2.146748in}}{\pgfqpoint{3.932882in}{2.154648in}}{\pgfqpoint{3.927058in}{2.160472in}}%
\pgfpathcurveto{\pgfqpoint{3.921234in}{2.166296in}}{\pgfqpoint{3.913334in}{2.169569in}}{\pgfqpoint{3.905098in}{2.169569in}}%
\pgfpathcurveto{\pgfqpoint{3.896862in}{2.169569in}}{\pgfqpoint{3.888962in}{2.166296in}}{\pgfqpoint{3.883138in}{2.160472in}}%
\pgfpathcurveto{\pgfqpoint{3.877314in}{2.154648in}}{\pgfqpoint{3.874042in}{2.146748in}}{\pgfqpoint{3.874042in}{2.138512in}}%
\pgfpathcurveto{\pgfqpoint{3.874042in}{2.130276in}}{\pgfqpoint{3.877314in}{2.122376in}}{\pgfqpoint{3.883138in}{2.116552in}}%
\pgfpathcurveto{\pgfqpoint{3.888962in}{2.110728in}}{\pgfqpoint{3.896862in}{2.107456in}}{\pgfqpoint{3.905098in}{2.107456in}}%
\pgfpathclose%
\pgfusepath{stroke,fill}%
\end{pgfscope}%
\begin{pgfscope}%
\pgfpathrectangle{\pgfqpoint{3.793912in}{0.557870in}}{\pgfqpoint{2.446088in}{1.684734in}}%
\pgfusepath{clip}%
\pgfsetbuttcap%
\pgfsetroundjoin%
\definecolor{currentfill}{rgb}{0.298039,0.447059,0.690196}%
\pgfsetfillcolor{currentfill}%
\pgfsetlinewidth{1.003750pt}%
\definecolor{currentstroke}{rgb}{0.298039,0.447059,0.690196}%
\pgfsetstrokecolor{currentstroke}%
\pgfsetdash{}{0pt}%
\pgfpathmoveto{\pgfqpoint{5.975454in}{1.309568in}}%
\pgfpathcurveto{\pgfqpoint{5.983691in}{1.309568in}}{\pgfqpoint{5.991591in}{1.312841in}}{\pgfqpoint{5.997415in}{1.318665in}}%
\pgfpathcurveto{\pgfqpoint{6.003239in}{1.324489in}}{\pgfqpoint{6.006511in}{1.332389in}}{\pgfqpoint{6.006511in}{1.340625in}}%
\pgfpathcurveto{\pgfqpoint{6.006511in}{1.348861in}}{\pgfqpoint{6.003239in}{1.356761in}}{\pgfqpoint{5.997415in}{1.362585in}}%
\pgfpathcurveto{\pgfqpoint{5.991591in}{1.368409in}}{\pgfqpoint{5.983691in}{1.371681in}}{\pgfqpoint{5.975454in}{1.371681in}}%
\pgfpathcurveto{\pgfqpoint{5.967218in}{1.371681in}}{\pgfqpoint{5.959318in}{1.368409in}}{\pgfqpoint{5.953494in}{1.362585in}}%
\pgfpathcurveto{\pgfqpoint{5.947670in}{1.356761in}}{\pgfqpoint{5.944398in}{1.348861in}}{\pgfqpoint{5.944398in}{1.340625in}}%
\pgfpathcurveto{\pgfqpoint{5.944398in}{1.332389in}}{\pgfqpoint{5.947670in}{1.324489in}}{\pgfqpoint{5.953494in}{1.318665in}}%
\pgfpathcurveto{\pgfqpoint{5.959318in}{1.312841in}}{\pgfqpoint{5.967218in}{1.309568in}}{\pgfqpoint{5.975454in}{1.309568in}}%
\pgfpathclose%
\pgfusepath{stroke,fill}%
\end{pgfscope}%
\begin{pgfscope}%
\pgfpathrectangle{\pgfqpoint{3.793912in}{0.557870in}}{\pgfqpoint{2.446088in}{1.684734in}}%
\pgfusepath{clip}%
\pgfsetbuttcap%
\pgfsetroundjoin%
\definecolor{currentfill}{rgb}{0.298039,0.447059,0.690196}%
\pgfsetfillcolor{currentfill}%
\pgfsetlinewidth{1.003750pt}%
\definecolor{currentstroke}{rgb}{0.298039,0.447059,0.690196}%
\pgfsetstrokecolor{currentstroke}%
\pgfsetdash{}{0pt}%
\pgfpathmoveto{\pgfqpoint{3.905098in}{2.125798in}}%
\pgfpathcurveto{\pgfqpoint{3.913334in}{2.125798in}}{\pgfqpoint{3.921234in}{2.129070in}}{\pgfqpoint{3.927058in}{2.134894in}}%
\pgfpathcurveto{\pgfqpoint{3.932882in}{2.140718in}}{\pgfqpoint{3.936155in}{2.148618in}}{\pgfqpoint{3.936155in}{2.156854in}}%
\pgfpathcurveto{\pgfqpoint{3.936155in}{2.165091in}}{\pgfqpoint{3.932882in}{2.172991in}}{\pgfqpoint{3.927058in}{2.178814in}}%
\pgfpathcurveto{\pgfqpoint{3.921234in}{2.184638in}}{\pgfqpoint{3.913334in}{2.187911in}}{\pgfqpoint{3.905098in}{2.187911in}}%
\pgfpathcurveto{\pgfqpoint{3.896862in}{2.187911in}}{\pgfqpoint{3.888962in}{2.184638in}}{\pgfqpoint{3.883138in}{2.178814in}}%
\pgfpathcurveto{\pgfqpoint{3.877314in}{2.172991in}}{\pgfqpoint{3.874042in}{2.165091in}}{\pgfqpoint{3.874042in}{2.156854in}}%
\pgfpathcurveto{\pgfqpoint{3.874042in}{2.148618in}}{\pgfqpoint{3.877314in}{2.140718in}}{\pgfqpoint{3.883138in}{2.134894in}}%
\pgfpathcurveto{\pgfqpoint{3.888962in}{2.129070in}}{\pgfqpoint{3.896862in}{2.125798in}}{\pgfqpoint{3.905098in}{2.125798in}}%
\pgfpathclose%
\pgfusepath{stroke,fill}%
\end{pgfscope}%
\begin{pgfscope}%
\pgfpathrectangle{\pgfqpoint{3.793912in}{0.557870in}}{\pgfqpoint{2.446088in}{1.684734in}}%
\pgfusepath{clip}%
\pgfsetbuttcap%
\pgfsetroundjoin%
\definecolor{currentfill}{rgb}{0.298039,0.447059,0.690196}%
\pgfsetfillcolor{currentfill}%
\pgfsetlinewidth{1.003750pt}%
\definecolor{currentstroke}{rgb}{0.298039,0.447059,0.690196}%
\pgfsetstrokecolor{currentstroke}%
\pgfsetdash{}{0pt}%
\pgfpathmoveto{\pgfqpoint{5.898775in}{1.419622in}}%
\pgfpathcurveto{\pgfqpoint{5.907011in}{1.419622in}}{\pgfqpoint{5.914911in}{1.422894in}}{\pgfqpoint{5.920735in}{1.428718in}}%
\pgfpathcurveto{\pgfqpoint{5.926559in}{1.434542in}}{\pgfqpoint{5.929831in}{1.442442in}}{\pgfqpoint{5.929831in}{1.450678in}}%
\pgfpathcurveto{\pgfqpoint{5.929831in}{1.458915in}}{\pgfqpoint{5.926559in}{1.466815in}}{\pgfqpoint{5.920735in}{1.472639in}}%
\pgfpathcurveto{\pgfqpoint{5.914911in}{1.478463in}}{\pgfqpoint{5.907011in}{1.481735in}}{\pgfqpoint{5.898775in}{1.481735in}}%
\pgfpathcurveto{\pgfqpoint{5.890538in}{1.481735in}}{\pgfqpoint{5.882638in}{1.478463in}}{\pgfqpoint{5.876814in}{1.472639in}}%
\pgfpathcurveto{\pgfqpoint{5.870990in}{1.466815in}}{\pgfqpoint{5.867718in}{1.458915in}}{\pgfqpoint{5.867718in}{1.450678in}}%
\pgfpathcurveto{\pgfqpoint{5.867718in}{1.442442in}}{\pgfqpoint{5.870990in}{1.434542in}}{\pgfqpoint{5.876814in}{1.428718in}}%
\pgfpathcurveto{\pgfqpoint{5.882638in}{1.422894in}}{\pgfqpoint{5.890538in}{1.419622in}}{\pgfqpoint{5.898775in}{1.419622in}}%
\pgfpathclose%
\pgfusepath{stroke,fill}%
\end{pgfscope}%
\begin{pgfscope}%
\pgfpathrectangle{\pgfqpoint{3.793912in}{0.557870in}}{\pgfqpoint{2.446088in}{1.684734in}}%
\pgfusepath{clip}%
\pgfsetbuttcap%
\pgfsetroundjoin%
\definecolor{currentfill}{rgb}{0.298039,0.447059,0.690196}%
\pgfsetfillcolor{currentfill}%
\pgfsetlinewidth{1.003750pt}%
\definecolor{currentstroke}{rgb}{0.298039,0.447059,0.690196}%
\pgfsetstrokecolor{currentstroke}%
\pgfsetdash{}{0pt}%
\pgfpathmoveto{\pgfqpoint{3.981778in}{2.125798in}}%
\pgfpathcurveto{\pgfqpoint{3.990014in}{2.125798in}}{\pgfqpoint{3.997914in}{2.129070in}}{\pgfqpoint{4.003738in}{2.134894in}}%
\pgfpathcurveto{\pgfqpoint{4.009562in}{2.140718in}}{\pgfqpoint{4.012834in}{2.148618in}}{\pgfqpoint{4.012834in}{2.156854in}}%
\pgfpathcurveto{\pgfqpoint{4.012834in}{2.165091in}}{\pgfqpoint{4.009562in}{2.172991in}}{\pgfqpoint{4.003738in}{2.178814in}}%
\pgfpathcurveto{\pgfqpoint{3.997914in}{2.184638in}}{\pgfqpoint{3.990014in}{2.187911in}}{\pgfqpoint{3.981778in}{2.187911in}}%
\pgfpathcurveto{\pgfqpoint{3.973542in}{2.187911in}}{\pgfqpoint{3.965642in}{2.184638in}}{\pgfqpoint{3.959818in}{2.178814in}}%
\pgfpathcurveto{\pgfqpoint{3.953994in}{2.172991in}}{\pgfqpoint{3.950721in}{2.165091in}}{\pgfqpoint{3.950721in}{2.156854in}}%
\pgfpathcurveto{\pgfqpoint{3.950721in}{2.148618in}}{\pgfqpoint{3.953994in}{2.140718in}}{\pgfqpoint{3.959818in}{2.134894in}}%
\pgfpathcurveto{\pgfqpoint{3.965642in}{2.129070in}}{\pgfqpoint{3.973542in}{2.125798in}}{\pgfqpoint{3.981778in}{2.125798in}}%
\pgfpathclose%
\pgfusepath{stroke,fill}%
\end{pgfscope}%
\begin{pgfscope}%
\pgfpathrectangle{\pgfqpoint{3.793912in}{0.557870in}}{\pgfqpoint{2.446088in}{1.684734in}}%
\pgfusepath{clip}%
\pgfsetbuttcap%
\pgfsetroundjoin%
\definecolor{currentfill}{rgb}{0.298039,0.447059,0.690196}%
\pgfsetfillcolor{currentfill}%
\pgfsetlinewidth{1.003750pt}%
\definecolor{currentstroke}{rgb}{0.298039,0.447059,0.690196}%
\pgfsetstrokecolor{currentstroke}%
\pgfsetdash{}{0pt}%
\pgfpathmoveto{\pgfqpoint{5.975454in}{1.263713in}}%
\pgfpathcurveto{\pgfqpoint{5.983691in}{1.263713in}}{\pgfqpoint{5.991591in}{1.266985in}}{\pgfqpoint{5.997415in}{1.272809in}}%
\pgfpathcurveto{\pgfqpoint{6.003239in}{1.278633in}}{\pgfqpoint{6.006511in}{1.286533in}}{\pgfqpoint{6.006511in}{1.294769in}}%
\pgfpathcurveto{\pgfqpoint{6.006511in}{1.303006in}}{\pgfqpoint{6.003239in}{1.310906in}}{\pgfqpoint{5.997415in}{1.316730in}}%
\pgfpathcurveto{\pgfqpoint{5.991591in}{1.322554in}}{\pgfqpoint{5.983691in}{1.325826in}}{\pgfqpoint{5.975454in}{1.325826in}}%
\pgfpathcurveto{\pgfqpoint{5.967218in}{1.325826in}}{\pgfqpoint{5.959318in}{1.322554in}}{\pgfqpoint{5.953494in}{1.316730in}}%
\pgfpathcurveto{\pgfqpoint{5.947670in}{1.310906in}}{\pgfqpoint{5.944398in}{1.303006in}}{\pgfqpoint{5.944398in}{1.294769in}}%
\pgfpathcurveto{\pgfqpoint{5.944398in}{1.286533in}}{\pgfqpoint{5.947670in}{1.278633in}}{\pgfqpoint{5.953494in}{1.272809in}}%
\pgfpathcurveto{\pgfqpoint{5.959318in}{1.266985in}}{\pgfqpoint{5.967218in}{1.263713in}}{\pgfqpoint{5.975454in}{1.263713in}}%
\pgfpathclose%
\pgfusepath{stroke,fill}%
\end{pgfscope}%
\begin{pgfscope}%
\pgfpathrectangle{\pgfqpoint{3.793912in}{0.557870in}}{\pgfqpoint{2.446088in}{1.684734in}}%
\pgfusepath{clip}%
\pgfsetbuttcap%
\pgfsetroundjoin%
\definecolor{currentfill}{rgb}{0.298039,0.447059,0.690196}%
\pgfsetfillcolor{currentfill}%
\pgfsetlinewidth{1.003750pt}%
\definecolor{currentstroke}{rgb}{0.298039,0.447059,0.690196}%
\pgfsetstrokecolor{currentstroke}%
\pgfsetdash{}{0pt}%
\pgfpathmoveto{\pgfqpoint{3.905098in}{2.125798in}}%
\pgfpathcurveto{\pgfqpoint{3.913334in}{2.125798in}}{\pgfqpoint{3.921234in}{2.129070in}}{\pgfqpoint{3.927058in}{2.134894in}}%
\pgfpathcurveto{\pgfqpoint{3.932882in}{2.140718in}}{\pgfqpoint{3.936155in}{2.148618in}}{\pgfqpoint{3.936155in}{2.156854in}}%
\pgfpathcurveto{\pgfqpoint{3.936155in}{2.165091in}}{\pgfqpoint{3.932882in}{2.172991in}}{\pgfqpoint{3.927058in}{2.178814in}}%
\pgfpathcurveto{\pgfqpoint{3.921234in}{2.184638in}}{\pgfqpoint{3.913334in}{2.187911in}}{\pgfqpoint{3.905098in}{2.187911in}}%
\pgfpathcurveto{\pgfqpoint{3.896862in}{2.187911in}}{\pgfqpoint{3.888962in}{2.184638in}}{\pgfqpoint{3.883138in}{2.178814in}}%
\pgfpathcurveto{\pgfqpoint{3.877314in}{2.172991in}}{\pgfqpoint{3.874042in}{2.165091in}}{\pgfqpoint{3.874042in}{2.156854in}}%
\pgfpathcurveto{\pgfqpoint{3.874042in}{2.148618in}}{\pgfqpoint{3.877314in}{2.140718in}}{\pgfqpoint{3.883138in}{2.134894in}}%
\pgfpathcurveto{\pgfqpoint{3.888962in}{2.129070in}}{\pgfqpoint{3.896862in}{2.125798in}}{\pgfqpoint{3.905098in}{2.125798in}}%
\pgfpathclose%
\pgfusepath{stroke,fill}%
\end{pgfscope}%
\begin{pgfscope}%
\pgfpathrectangle{\pgfqpoint{3.793912in}{0.557870in}}{\pgfqpoint{2.446088in}{1.684734in}}%
\pgfusepath{clip}%
\pgfsetbuttcap%
\pgfsetroundjoin%
\definecolor{currentfill}{rgb}{0.298039,0.447059,0.690196}%
\pgfsetfillcolor{currentfill}%
\pgfsetlinewidth{1.003750pt}%
\definecolor{currentstroke}{rgb}{0.298039,0.447059,0.690196}%
\pgfsetstrokecolor{currentstroke}%
\pgfsetdash{}{0pt}%
\pgfpathmoveto{\pgfqpoint{4.365177in}{2.116627in}}%
\pgfpathcurveto{\pgfqpoint{4.373414in}{2.116627in}}{\pgfqpoint{4.381314in}{2.119899in}}{\pgfqpoint{4.387137in}{2.125723in}}%
\pgfpathcurveto{\pgfqpoint{4.392961in}{2.131547in}}{\pgfqpoint{4.396234in}{2.139447in}}{\pgfqpoint{4.396234in}{2.147683in}}%
\pgfpathcurveto{\pgfqpoint{4.396234in}{2.155919in}}{\pgfqpoint{4.392961in}{2.163819in}}{\pgfqpoint{4.387137in}{2.169643in}}%
\pgfpathcurveto{\pgfqpoint{4.381314in}{2.175467in}}{\pgfqpoint{4.373414in}{2.178740in}}{\pgfqpoint{4.365177in}{2.178740in}}%
\pgfpathcurveto{\pgfqpoint{4.356941in}{2.178740in}}{\pgfqpoint{4.349041in}{2.175467in}}{\pgfqpoint{4.343217in}{2.169643in}}%
\pgfpathcurveto{\pgfqpoint{4.337393in}{2.163819in}}{\pgfqpoint{4.334121in}{2.155919in}}{\pgfqpoint{4.334121in}{2.147683in}}%
\pgfpathcurveto{\pgfqpoint{4.334121in}{2.139447in}}{\pgfqpoint{4.337393in}{2.131547in}}{\pgfqpoint{4.343217in}{2.125723in}}%
\pgfpathcurveto{\pgfqpoint{4.349041in}{2.119899in}}{\pgfqpoint{4.356941in}{2.116627in}}{\pgfqpoint{4.365177in}{2.116627in}}%
\pgfpathclose%
\pgfusepath{stroke,fill}%
\end{pgfscope}%
\begin{pgfscope}%
\pgfpathrectangle{\pgfqpoint{3.793912in}{0.557870in}}{\pgfqpoint{2.446088in}{1.684734in}}%
\pgfusepath{clip}%
\pgfsetbuttcap%
\pgfsetroundjoin%
\definecolor{currentfill}{rgb}{0.298039,0.447059,0.690196}%
\pgfsetfillcolor{currentfill}%
\pgfsetlinewidth{1.003750pt}%
\definecolor{currentstroke}{rgb}{0.298039,0.447059,0.690196}%
\pgfsetstrokecolor{currentstroke}%
\pgfsetdash{}{0pt}%
\pgfpathmoveto{\pgfqpoint{3.905098in}{2.125798in}}%
\pgfpathcurveto{\pgfqpoint{3.913334in}{2.125798in}}{\pgfqpoint{3.921234in}{2.129070in}}{\pgfqpoint{3.927058in}{2.134894in}}%
\pgfpathcurveto{\pgfqpoint{3.932882in}{2.140718in}}{\pgfqpoint{3.936155in}{2.148618in}}{\pgfqpoint{3.936155in}{2.156854in}}%
\pgfpathcurveto{\pgfqpoint{3.936155in}{2.165091in}}{\pgfqpoint{3.932882in}{2.172991in}}{\pgfqpoint{3.927058in}{2.178814in}}%
\pgfpathcurveto{\pgfqpoint{3.921234in}{2.184638in}}{\pgfqpoint{3.913334in}{2.187911in}}{\pgfqpoint{3.905098in}{2.187911in}}%
\pgfpathcurveto{\pgfqpoint{3.896862in}{2.187911in}}{\pgfqpoint{3.888962in}{2.184638in}}{\pgfqpoint{3.883138in}{2.178814in}}%
\pgfpathcurveto{\pgfqpoint{3.877314in}{2.172991in}}{\pgfqpoint{3.874042in}{2.165091in}}{\pgfqpoint{3.874042in}{2.156854in}}%
\pgfpathcurveto{\pgfqpoint{3.874042in}{2.148618in}}{\pgfqpoint{3.877314in}{2.140718in}}{\pgfqpoint{3.883138in}{2.134894in}}%
\pgfpathcurveto{\pgfqpoint{3.888962in}{2.129070in}}{\pgfqpoint{3.896862in}{2.125798in}}{\pgfqpoint{3.905098in}{2.125798in}}%
\pgfpathclose%
\pgfusepath{stroke,fill}%
\end{pgfscope}%
\begin{pgfscope}%
\pgfpathrectangle{\pgfqpoint{3.793912in}{0.557870in}}{\pgfqpoint{2.446088in}{1.684734in}}%
\pgfusepath{clip}%
\pgfsetbuttcap%
\pgfsetroundjoin%
\definecolor{currentfill}{rgb}{0.298039,0.447059,0.690196}%
\pgfsetfillcolor{currentfill}%
\pgfsetlinewidth{1.003750pt}%
\definecolor{currentstroke}{rgb}{0.298039,0.447059,0.690196}%
\pgfsetstrokecolor{currentstroke}%
\pgfsetdash{}{0pt}%
\pgfpathmoveto{\pgfqpoint{4.901936in}{1.795638in}}%
\pgfpathcurveto{\pgfqpoint{4.910173in}{1.795638in}}{\pgfqpoint{4.918073in}{1.798910in}}{\pgfqpoint{4.923897in}{1.804734in}}%
\pgfpathcurveto{\pgfqpoint{4.929721in}{1.810558in}}{\pgfqpoint{4.932993in}{1.818458in}}{\pgfqpoint{4.932993in}{1.826694in}}%
\pgfpathcurveto{\pgfqpoint{4.932993in}{1.834930in}}{\pgfqpoint{4.929721in}{1.842830in}}{\pgfqpoint{4.923897in}{1.848654in}}%
\pgfpathcurveto{\pgfqpoint{4.918073in}{1.854478in}}{\pgfqpoint{4.910173in}{1.857751in}}{\pgfqpoint{4.901936in}{1.857751in}}%
\pgfpathcurveto{\pgfqpoint{4.893700in}{1.857751in}}{\pgfqpoint{4.885800in}{1.854478in}}{\pgfqpoint{4.879976in}{1.848654in}}%
\pgfpathcurveto{\pgfqpoint{4.874152in}{1.842830in}}{\pgfqpoint{4.870880in}{1.834930in}}{\pgfqpoint{4.870880in}{1.826694in}}%
\pgfpathcurveto{\pgfqpoint{4.870880in}{1.818458in}}{\pgfqpoint{4.874152in}{1.810558in}}{\pgfqpoint{4.879976in}{1.804734in}}%
\pgfpathcurveto{\pgfqpoint{4.885800in}{1.798910in}}{\pgfqpoint{4.893700in}{1.795638in}}{\pgfqpoint{4.901936in}{1.795638in}}%
\pgfpathclose%
\pgfusepath{stroke,fill}%
\end{pgfscope}%
\begin{pgfscope}%
\pgfpathrectangle{\pgfqpoint{3.793912in}{0.557870in}}{\pgfqpoint{2.446088in}{1.684734in}}%
\pgfusepath{clip}%
\pgfsetbuttcap%
\pgfsetroundjoin%
\definecolor{currentfill}{rgb}{0.298039,0.447059,0.690196}%
\pgfsetfillcolor{currentfill}%
\pgfsetlinewidth{1.003750pt}%
\definecolor{currentstroke}{rgb}{0.298039,0.447059,0.690196}%
\pgfsetstrokecolor{currentstroke}%
\pgfsetdash{}{0pt}%
\pgfpathmoveto{\pgfqpoint{3.905098in}{2.125798in}}%
\pgfpathcurveto{\pgfqpoint{3.913334in}{2.125798in}}{\pgfqpoint{3.921234in}{2.129070in}}{\pgfqpoint{3.927058in}{2.134894in}}%
\pgfpathcurveto{\pgfqpoint{3.932882in}{2.140718in}}{\pgfqpoint{3.936155in}{2.148618in}}{\pgfqpoint{3.936155in}{2.156854in}}%
\pgfpathcurveto{\pgfqpoint{3.936155in}{2.165091in}}{\pgfqpoint{3.932882in}{2.172991in}}{\pgfqpoint{3.927058in}{2.178814in}}%
\pgfpathcurveto{\pgfqpoint{3.921234in}{2.184638in}}{\pgfqpoint{3.913334in}{2.187911in}}{\pgfqpoint{3.905098in}{2.187911in}}%
\pgfpathcurveto{\pgfqpoint{3.896862in}{2.187911in}}{\pgfqpoint{3.888962in}{2.184638in}}{\pgfqpoint{3.883138in}{2.178814in}}%
\pgfpathcurveto{\pgfqpoint{3.877314in}{2.172991in}}{\pgfqpoint{3.874042in}{2.165091in}}{\pgfqpoint{3.874042in}{2.156854in}}%
\pgfpathcurveto{\pgfqpoint{3.874042in}{2.148618in}}{\pgfqpoint{3.877314in}{2.140718in}}{\pgfqpoint{3.883138in}{2.134894in}}%
\pgfpathcurveto{\pgfqpoint{3.888962in}{2.129070in}}{\pgfqpoint{3.896862in}{2.125798in}}{\pgfqpoint{3.905098in}{2.125798in}}%
\pgfpathclose%
\pgfusepath{stroke,fill}%
\end{pgfscope}%
\begin{pgfscope}%
\pgfpathrectangle{\pgfqpoint{3.793912in}{0.557870in}}{\pgfqpoint{2.446088in}{1.684734in}}%
\pgfusepath{clip}%
\pgfsetbuttcap%
\pgfsetroundjoin%
\definecolor{currentfill}{rgb}{0.298039,0.447059,0.690196}%
\pgfsetfillcolor{currentfill}%
\pgfsetlinewidth{1.003750pt}%
\definecolor{currentstroke}{rgb}{0.298039,0.447059,0.690196}%
\pgfsetstrokecolor{currentstroke}%
\pgfsetdash{}{0pt}%
\pgfpathmoveto{\pgfqpoint{4.901936in}{1.859835in}}%
\pgfpathcurveto{\pgfqpoint{4.910173in}{1.859835in}}{\pgfqpoint{4.918073in}{1.863108in}}{\pgfqpoint{4.923897in}{1.868932in}}%
\pgfpathcurveto{\pgfqpoint{4.929721in}{1.874756in}}{\pgfqpoint{4.932993in}{1.882656in}}{\pgfqpoint{4.932993in}{1.890892in}}%
\pgfpathcurveto{\pgfqpoint{4.932993in}{1.899128in}}{\pgfqpoint{4.929721in}{1.907028in}}{\pgfqpoint{4.923897in}{1.912852in}}%
\pgfpathcurveto{\pgfqpoint{4.918073in}{1.918676in}}{\pgfqpoint{4.910173in}{1.921948in}}{\pgfqpoint{4.901936in}{1.921948in}}%
\pgfpathcurveto{\pgfqpoint{4.893700in}{1.921948in}}{\pgfqpoint{4.885800in}{1.918676in}}{\pgfqpoint{4.879976in}{1.912852in}}%
\pgfpathcurveto{\pgfqpoint{4.874152in}{1.907028in}}{\pgfqpoint{4.870880in}{1.899128in}}{\pgfqpoint{4.870880in}{1.890892in}}%
\pgfpathcurveto{\pgfqpoint{4.870880in}{1.882656in}}{\pgfqpoint{4.874152in}{1.874756in}}{\pgfqpoint{4.879976in}{1.868932in}}%
\pgfpathcurveto{\pgfqpoint{4.885800in}{1.863108in}}{\pgfqpoint{4.893700in}{1.859835in}}{\pgfqpoint{4.901936in}{1.859835in}}%
\pgfpathclose%
\pgfusepath{stroke,fill}%
\end{pgfscope}%
\begin{pgfscope}%
\pgfpathrectangle{\pgfqpoint{3.793912in}{0.557870in}}{\pgfqpoint{2.446088in}{1.684734in}}%
\pgfusepath{clip}%
\pgfsetbuttcap%
\pgfsetroundjoin%
\definecolor{currentfill}{rgb}{0.298039,0.447059,0.690196}%
\pgfsetfillcolor{currentfill}%
\pgfsetlinewidth{1.003750pt}%
\definecolor{currentstroke}{rgb}{0.298039,0.447059,0.690196}%
\pgfsetstrokecolor{currentstroke}%
\pgfsetdash{}{0pt}%
\pgfpathmoveto{\pgfqpoint{3.905098in}{2.125798in}}%
\pgfpathcurveto{\pgfqpoint{3.913334in}{2.125798in}}{\pgfqpoint{3.921234in}{2.129070in}}{\pgfqpoint{3.927058in}{2.134894in}}%
\pgfpathcurveto{\pgfqpoint{3.932882in}{2.140718in}}{\pgfqpoint{3.936155in}{2.148618in}}{\pgfqpoint{3.936155in}{2.156854in}}%
\pgfpathcurveto{\pgfqpoint{3.936155in}{2.165091in}}{\pgfqpoint{3.932882in}{2.172991in}}{\pgfqpoint{3.927058in}{2.178814in}}%
\pgfpathcurveto{\pgfqpoint{3.921234in}{2.184638in}}{\pgfqpoint{3.913334in}{2.187911in}}{\pgfqpoint{3.905098in}{2.187911in}}%
\pgfpathcurveto{\pgfqpoint{3.896862in}{2.187911in}}{\pgfqpoint{3.888962in}{2.184638in}}{\pgfqpoint{3.883138in}{2.178814in}}%
\pgfpathcurveto{\pgfqpoint{3.877314in}{2.172991in}}{\pgfqpoint{3.874042in}{2.165091in}}{\pgfqpoint{3.874042in}{2.156854in}}%
\pgfpathcurveto{\pgfqpoint{3.874042in}{2.148618in}}{\pgfqpoint{3.877314in}{2.140718in}}{\pgfqpoint{3.883138in}{2.134894in}}%
\pgfpathcurveto{\pgfqpoint{3.888962in}{2.129070in}}{\pgfqpoint{3.896862in}{2.125798in}}{\pgfqpoint{3.905098in}{2.125798in}}%
\pgfpathclose%
\pgfusepath{stroke,fill}%
\end{pgfscope}%
\begin{pgfscope}%
\pgfpathrectangle{\pgfqpoint{3.793912in}{0.557870in}}{\pgfqpoint{2.446088in}{1.684734in}}%
\pgfusepath{clip}%
\pgfsetbuttcap%
\pgfsetroundjoin%
\definecolor{currentfill}{rgb}{0.298039,0.447059,0.690196}%
\pgfsetfillcolor{currentfill}%
\pgfsetlinewidth{1.003750pt}%
\definecolor{currentstroke}{rgb}{0.298039,0.447059,0.690196}%
\pgfsetstrokecolor{currentstroke}%
\pgfsetdash{}{0pt}%
\pgfpathmoveto{\pgfqpoint{3.905098in}{2.125798in}}%
\pgfpathcurveto{\pgfqpoint{3.913334in}{2.125798in}}{\pgfqpoint{3.921234in}{2.129070in}}{\pgfqpoint{3.927058in}{2.134894in}}%
\pgfpathcurveto{\pgfqpoint{3.932882in}{2.140718in}}{\pgfqpoint{3.936155in}{2.148618in}}{\pgfqpoint{3.936155in}{2.156854in}}%
\pgfpathcurveto{\pgfqpoint{3.936155in}{2.165091in}}{\pgfqpoint{3.932882in}{2.172991in}}{\pgfqpoint{3.927058in}{2.178814in}}%
\pgfpathcurveto{\pgfqpoint{3.921234in}{2.184638in}}{\pgfqpoint{3.913334in}{2.187911in}}{\pgfqpoint{3.905098in}{2.187911in}}%
\pgfpathcurveto{\pgfqpoint{3.896862in}{2.187911in}}{\pgfqpoint{3.888962in}{2.184638in}}{\pgfqpoint{3.883138in}{2.178814in}}%
\pgfpathcurveto{\pgfqpoint{3.877314in}{2.172991in}}{\pgfqpoint{3.874042in}{2.165091in}}{\pgfqpoint{3.874042in}{2.156854in}}%
\pgfpathcurveto{\pgfqpoint{3.874042in}{2.148618in}}{\pgfqpoint{3.877314in}{2.140718in}}{\pgfqpoint{3.883138in}{2.134894in}}%
\pgfpathcurveto{\pgfqpoint{3.888962in}{2.129070in}}{\pgfqpoint{3.896862in}{2.125798in}}{\pgfqpoint{3.905098in}{2.125798in}}%
\pgfpathclose%
\pgfusepath{stroke,fill}%
\end{pgfscope}%
\begin{pgfscope}%
\pgfpathrectangle{\pgfqpoint{3.793912in}{0.557870in}}{\pgfqpoint{2.446088in}{1.684734in}}%
\pgfusepath{clip}%
\pgfsetbuttcap%
\pgfsetroundjoin%
\definecolor{currentfill}{rgb}{0.298039,0.447059,0.690196}%
\pgfsetfillcolor{currentfill}%
\pgfsetlinewidth{1.003750pt}%
\definecolor{currentstroke}{rgb}{0.298039,0.447059,0.690196}%
\pgfsetstrokecolor{currentstroke}%
\pgfsetdash{}{0pt}%
\pgfpathmoveto{\pgfqpoint{3.905098in}{2.125798in}}%
\pgfpathcurveto{\pgfqpoint{3.913334in}{2.125798in}}{\pgfqpoint{3.921234in}{2.129070in}}{\pgfqpoint{3.927058in}{2.134894in}}%
\pgfpathcurveto{\pgfqpoint{3.932882in}{2.140718in}}{\pgfqpoint{3.936155in}{2.148618in}}{\pgfqpoint{3.936155in}{2.156854in}}%
\pgfpathcurveto{\pgfqpoint{3.936155in}{2.165091in}}{\pgfqpoint{3.932882in}{2.172991in}}{\pgfqpoint{3.927058in}{2.178814in}}%
\pgfpathcurveto{\pgfqpoint{3.921234in}{2.184638in}}{\pgfqpoint{3.913334in}{2.187911in}}{\pgfqpoint{3.905098in}{2.187911in}}%
\pgfpathcurveto{\pgfqpoint{3.896862in}{2.187911in}}{\pgfqpoint{3.888962in}{2.184638in}}{\pgfqpoint{3.883138in}{2.178814in}}%
\pgfpathcurveto{\pgfqpoint{3.877314in}{2.172991in}}{\pgfqpoint{3.874042in}{2.165091in}}{\pgfqpoint{3.874042in}{2.156854in}}%
\pgfpathcurveto{\pgfqpoint{3.874042in}{2.148618in}}{\pgfqpoint{3.877314in}{2.140718in}}{\pgfqpoint{3.883138in}{2.134894in}}%
\pgfpathcurveto{\pgfqpoint{3.888962in}{2.129070in}}{\pgfqpoint{3.896862in}{2.125798in}}{\pgfqpoint{3.905098in}{2.125798in}}%
\pgfpathclose%
\pgfusepath{stroke,fill}%
\end{pgfscope}%
\begin{pgfscope}%
\pgfpathrectangle{\pgfqpoint{3.793912in}{0.557870in}}{\pgfqpoint{2.446088in}{1.684734in}}%
\pgfusepath{clip}%
\pgfsetbuttcap%
\pgfsetroundjoin%
\definecolor{currentfill}{rgb}{0.298039,0.447059,0.690196}%
\pgfsetfillcolor{currentfill}%
\pgfsetlinewidth{1.003750pt}%
\definecolor{currentstroke}{rgb}{0.298039,0.447059,0.690196}%
\pgfsetstrokecolor{currentstroke}%
\pgfsetdash{}{0pt}%
\pgfpathmoveto{\pgfqpoint{3.905098in}{2.125798in}}%
\pgfpathcurveto{\pgfqpoint{3.913334in}{2.125798in}}{\pgfqpoint{3.921234in}{2.129070in}}{\pgfqpoint{3.927058in}{2.134894in}}%
\pgfpathcurveto{\pgfqpoint{3.932882in}{2.140718in}}{\pgfqpoint{3.936155in}{2.148618in}}{\pgfqpoint{3.936155in}{2.156854in}}%
\pgfpathcurveto{\pgfqpoint{3.936155in}{2.165091in}}{\pgfqpoint{3.932882in}{2.172991in}}{\pgfqpoint{3.927058in}{2.178814in}}%
\pgfpathcurveto{\pgfqpoint{3.921234in}{2.184638in}}{\pgfqpoint{3.913334in}{2.187911in}}{\pgfqpoint{3.905098in}{2.187911in}}%
\pgfpathcurveto{\pgfqpoint{3.896862in}{2.187911in}}{\pgfqpoint{3.888962in}{2.184638in}}{\pgfqpoint{3.883138in}{2.178814in}}%
\pgfpathcurveto{\pgfqpoint{3.877314in}{2.172991in}}{\pgfqpoint{3.874042in}{2.165091in}}{\pgfqpoint{3.874042in}{2.156854in}}%
\pgfpathcurveto{\pgfqpoint{3.874042in}{2.148618in}}{\pgfqpoint{3.877314in}{2.140718in}}{\pgfqpoint{3.883138in}{2.134894in}}%
\pgfpathcurveto{\pgfqpoint{3.888962in}{2.129070in}}{\pgfqpoint{3.896862in}{2.125798in}}{\pgfqpoint{3.905098in}{2.125798in}}%
\pgfpathclose%
\pgfusepath{stroke,fill}%
\end{pgfscope}%
\begin{pgfscope}%
\pgfpathrectangle{\pgfqpoint{3.793912in}{0.557870in}}{\pgfqpoint{2.446088in}{1.684734in}}%
\pgfusepath{clip}%
\pgfsetbuttcap%
\pgfsetroundjoin%
\definecolor{currentfill}{rgb}{0.298039,0.447059,0.690196}%
\pgfsetfillcolor{currentfill}%
\pgfsetlinewidth{1.003750pt}%
\definecolor{currentstroke}{rgb}{0.298039,0.447059,0.690196}%
\pgfsetstrokecolor{currentstroke}%
\pgfsetdash{}{0pt}%
\pgfpathmoveto{\pgfqpoint{3.905098in}{2.125798in}}%
\pgfpathcurveto{\pgfqpoint{3.913334in}{2.125798in}}{\pgfqpoint{3.921234in}{2.129070in}}{\pgfqpoint{3.927058in}{2.134894in}}%
\pgfpathcurveto{\pgfqpoint{3.932882in}{2.140718in}}{\pgfqpoint{3.936155in}{2.148618in}}{\pgfqpoint{3.936155in}{2.156854in}}%
\pgfpathcurveto{\pgfqpoint{3.936155in}{2.165091in}}{\pgfqpoint{3.932882in}{2.172991in}}{\pgfqpoint{3.927058in}{2.178814in}}%
\pgfpathcurveto{\pgfqpoint{3.921234in}{2.184638in}}{\pgfqpoint{3.913334in}{2.187911in}}{\pgfqpoint{3.905098in}{2.187911in}}%
\pgfpathcurveto{\pgfqpoint{3.896862in}{2.187911in}}{\pgfqpoint{3.888962in}{2.184638in}}{\pgfqpoint{3.883138in}{2.178814in}}%
\pgfpathcurveto{\pgfqpoint{3.877314in}{2.172991in}}{\pgfqpoint{3.874042in}{2.165091in}}{\pgfqpoint{3.874042in}{2.156854in}}%
\pgfpathcurveto{\pgfqpoint{3.874042in}{2.148618in}}{\pgfqpoint{3.877314in}{2.140718in}}{\pgfqpoint{3.883138in}{2.134894in}}%
\pgfpathcurveto{\pgfqpoint{3.888962in}{2.129070in}}{\pgfqpoint{3.896862in}{2.125798in}}{\pgfqpoint{3.905098in}{2.125798in}}%
\pgfpathclose%
\pgfusepath{stroke,fill}%
\end{pgfscope}%
\begin{pgfscope}%
\pgfpathrectangle{\pgfqpoint{3.793912in}{0.557870in}}{\pgfqpoint{2.446088in}{1.684734in}}%
\pgfusepath{clip}%
\pgfsetbuttcap%
\pgfsetroundjoin%
\definecolor{currentfill}{rgb}{0.298039,0.447059,0.690196}%
\pgfsetfillcolor{currentfill}%
\pgfsetlinewidth{1.003750pt}%
\definecolor{currentstroke}{rgb}{0.298039,0.447059,0.690196}%
\pgfsetstrokecolor{currentstroke}%
\pgfsetdash{}{0pt}%
\pgfpathmoveto{\pgfqpoint{3.905098in}{2.125798in}}%
\pgfpathcurveto{\pgfqpoint{3.913334in}{2.125798in}}{\pgfqpoint{3.921234in}{2.129070in}}{\pgfqpoint{3.927058in}{2.134894in}}%
\pgfpathcurveto{\pgfqpoint{3.932882in}{2.140718in}}{\pgfqpoint{3.936155in}{2.148618in}}{\pgfqpoint{3.936155in}{2.156854in}}%
\pgfpathcurveto{\pgfqpoint{3.936155in}{2.165091in}}{\pgfqpoint{3.932882in}{2.172991in}}{\pgfqpoint{3.927058in}{2.178814in}}%
\pgfpathcurveto{\pgfqpoint{3.921234in}{2.184638in}}{\pgfqpoint{3.913334in}{2.187911in}}{\pgfqpoint{3.905098in}{2.187911in}}%
\pgfpathcurveto{\pgfqpoint{3.896862in}{2.187911in}}{\pgfqpoint{3.888962in}{2.184638in}}{\pgfqpoint{3.883138in}{2.178814in}}%
\pgfpathcurveto{\pgfqpoint{3.877314in}{2.172991in}}{\pgfqpoint{3.874042in}{2.165091in}}{\pgfqpoint{3.874042in}{2.156854in}}%
\pgfpathcurveto{\pgfqpoint{3.874042in}{2.148618in}}{\pgfqpoint{3.877314in}{2.140718in}}{\pgfqpoint{3.883138in}{2.134894in}}%
\pgfpathcurveto{\pgfqpoint{3.888962in}{2.129070in}}{\pgfqpoint{3.896862in}{2.125798in}}{\pgfqpoint{3.905098in}{2.125798in}}%
\pgfpathclose%
\pgfusepath{stroke,fill}%
\end{pgfscope}%
\begin{pgfscope}%
\pgfpathrectangle{\pgfqpoint{3.793912in}{0.557870in}}{\pgfqpoint{2.446088in}{1.684734in}}%
\pgfusepath{clip}%
\pgfsetbuttcap%
\pgfsetroundjoin%
\definecolor{currentfill}{rgb}{0.298039,0.447059,0.690196}%
\pgfsetfillcolor{currentfill}%
\pgfsetlinewidth{1.003750pt}%
\definecolor{currentstroke}{rgb}{0.298039,0.447059,0.690196}%
\pgfsetstrokecolor{currentstroke}%
\pgfsetdash{}{0pt}%
\pgfpathmoveto{\pgfqpoint{4.978616in}{1.392109in}}%
\pgfpathcurveto{\pgfqpoint{4.986852in}{1.392109in}}{\pgfqpoint{4.994753in}{1.395381in}}{\pgfqpoint{5.000576in}{1.401205in}}%
\pgfpathcurveto{\pgfqpoint{5.006400in}{1.407029in}}{\pgfqpoint{5.009673in}{1.414929in}}{\pgfqpoint{5.009673in}{1.423165in}}%
\pgfpathcurveto{\pgfqpoint{5.009673in}{1.431401in}}{\pgfqpoint{5.006400in}{1.439301in}}{\pgfqpoint{5.000576in}{1.445125in}}%
\pgfpathcurveto{\pgfqpoint{4.994753in}{1.450949in}}{\pgfqpoint{4.986852in}{1.454222in}}{\pgfqpoint{4.978616in}{1.454222in}}%
\pgfpathcurveto{\pgfqpoint{4.970380in}{1.454222in}}{\pgfqpoint{4.962480in}{1.450949in}}{\pgfqpoint{4.956656in}{1.445125in}}%
\pgfpathcurveto{\pgfqpoint{4.950832in}{1.439301in}}{\pgfqpoint{4.947560in}{1.431401in}}{\pgfqpoint{4.947560in}{1.423165in}}%
\pgfpathcurveto{\pgfqpoint{4.947560in}{1.414929in}}{\pgfqpoint{4.950832in}{1.407029in}}{\pgfqpoint{4.956656in}{1.401205in}}%
\pgfpathcurveto{\pgfqpoint{4.962480in}{1.395381in}}{\pgfqpoint{4.970380in}{1.392109in}}{\pgfqpoint{4.978616in}{1.392109in}}%
\pgfpathclose%
\pgfusepath{stroke,fill}%
\end{pgfscope}%
\begin{pgfscope}%
\pgfpathrectangle{\pgfqpoint{3.793912in}{0.557870in}}{\pgfqpoint{2.446088in}{1.684734in}}%
\pgfusepath{clip}%
\pgfsetbuttcap%
\pgfsetroundjoin%
\definecolor{currentfill}{rgb}{0.298039,0.447059,0.690196}%
\pgfsetfillcolor{currentfill}%
\pgfsetlinewidth{1.003750pt}%
\definecolor{currentstroke}{rgb}{0.298039,0.447059,0.690196}%
\pgfsetstrokecolor{currentstroke}%
\pgfsetdash{}{0pt}%
\pgfpathmoveto{\pgfqpoint{3.905098in}{2.125798in}}%
\pgfpathcurveto{\pgfqpoint{3.913334in}{2.125798in}}{\pgfqpoint{3.921234in}{2.129070in}}{\pgfqpoint{3.927058in}{2.134894in}}%
\pgfpathcurveto{\pgfqpoint{3.932882in}{2.140718in}}{\pgfqpoint{3.936155in}{2.148618in}}{\pgfqpoint{3.936155in}{2.156854in}}%
\pgfpathcurveto{\pgfqpoint{3.936155in}{2.165091in}}{\pgfqpoint{3.932882in}{2.172991in}}{\pgfqpoint{3.927058in}{2.178814in}}%
\pgfpathcurveto{\pgfqpoint{3.921234in}{2.184638in}}{\pgfqpoint{3.913334in}{2.187911in}}{\pgfqpoint{3.905098in}{2.187911in}}%
\pgfpathcurveto{\pgfqpoint{3.896862in}{2.187911in}}{\pgfqpoint{3.888962in}{2.184638in}}{\pgfqpoint{3.883138in}{2.178814in}}%
\pgfpathcurveto{\pgfqpoint{3.877314in}{2.172991in}}{\pgfqpoint{3.874042in}{2.165091in}}{\pgfqpoint{3.874042in}{2.156854in}}%
\pgfpathcurveto{\pgfqpoint{3.874042in}{2.148618in}}{\pgfqpoint{3.877314in}{2.140718in}}{\pgfqpoint{3.883138in}{2.134894in}}%
\pgfpathcurveto{\pgfqpoint{3.888962in}{2.129070in}}{\pgfqpoint{3.896862in}{2.125798in}}{\pgfqpoint{3.905098in}{2.125798in}}%
\pgfpathclose%
\pgfusepath{stroke,fill}%
\end{pgfscope}%
\begin{pgfscope}%
\pgfpathrectangle{\pgfqpoint{3.793912in}{0.557870in}}{\pgfqpoint{2.446088in}{1.684734in}}%
\pgfusepath{clip}%
\pgfsetbuttcap%
\pgfsetroundjoin%
\definecolor{currentfill}{rgb}{0.298039,0.447059,0.690196}%
\pgfsetfillcolor{currentfill}%
\pgfsetlinewidth{1.003750pt}%
\definecolor{currentstroke}{rgb}{0.298039,0.447059,0.690196}%
\pgfsetstrokecolor{currentstroke}%
\pgfsetdash{}{0pt}%
\pgfpathmoveto{\pgfqpoint{3.905098in}{2.125798in}}%
\pgfpathcurveto{\pgfqpoint{3.913334in}{2.125798in}}{\pgfqpoint{3.921234in}{2.129070in}}{\pgfqpoint{3.927058in}{2.134894in}}%
\pgfpathcurveto{\pgfqpoint{3.932882in}{2.140718in}}{\pgfqpoint{3.936155in}{2.148618in}}{\pgfqpoint{3.936155in}{2.156854in}}%
\pgfpathcurveto{\pgfqpoint{3.936155in}{2.165091in}}{\pgfqpoint{3.932882in}{2.172991in}}{\pgfqpoint{3.927058in}{2.178814in}}%
\pgfpathcurveto{\pgfqpoint{3.921234in}{2.184638in}}{\pgfqpoint{3.913334in}{2.187911in}}{\pgfqpoint{3.905098in}{2.187911in}}%
\pgfpathcurveto{\pgfqpoint{3.896862in}{2.187911in}}{\pgfqpoint{3.888962in}{2.184638in}}{\pgfqpoint{3.883138in}{2.178814in}}%
\pgfpathcurveto{\pgfqpoint{3.877314in}{2.172991in}}{\pgfqpoint{3.874042in}{2.165091in}}{\pgfqpoint{3.874042in}{2.156854in}}%
\pgfpathcurveto{\pgfqpoint{3.874042in}{2.148618in}}{\pgfqpoint{3.877314in}{2.140718in}}{\pgfqpoint{3.883138in}{2.134894in}}%
\pgfpathcurveto{\pgfqpoint{3.888962in}{2.129070in}}{\pgfqpoint{3.896862in}{2.125798in}}{\pgfqpoint{3.905098in}{2.125798in}}%
\pgfpathclose%
\pgfusepath{stroke,fill}%
\end{pgfscope}%
\begin{pgfscope}%
\pgfpathrectangle{\pgfqpoint{3.793912in}{0.557870in}}{\pgfqpoint{2.446088in}{1.684734in}}%
\pgfusepath{clip}%
\pgfsetbuttcap%
\pgfsetroundjoin%
\definecolor{currentfill}{rgb}{0.298039,0.447059,0.690196}%
\pgfsetfillcolor{currentfill}%
\pgfsetlinewidth{1.003750pt}%
\definecolor{currentstroke}{rgb}{0.298039,0.447059,0.690196}%
\pgfsetstrokecolor{currentstroke}%
\pgfsetdash{}{0pt}%
\pgfpathmoveto{\pgfqpoint{3.905098in}{2.125798in}}%
\pgfpathcurveto{\pgfqpoint{3.913334in}{2.125798in}}{\pgfqpoint{3.921234in}{2.129070in}}{\pgfqpoint{3.927058in}{2.134894in}}%
\pgfpathcurveto{\pgfqpoint{3.932882in}{2.140718in}}{\pgfqpoint{3.936155in}{2.148618in}}{\pgfqpoint{3.936155in}{2.156854in}}%
\pgfpathcurveto{\pgfqpoint{3.936155in}{2.165091in}}{\pgfqpoint{3.932882in}{2.172991in}}{\pgfqpoint{3.927058in}{2.178814in}}%
\pgfpathcurveto{\pgfqpoint{3.921234in}{2.184638in}}{\pgfqpoint{3.913334in}{2.187911in}}{\pgfqpoint{3.905098in}{2.187911in}}%
\pgfpathcurveto{\pgfqpoint{3.896862in}{2.187911in}}{\pgfqpoint{3.888962in}{2.184638in}}{\pgfqpoint{3.883138in}{2.178814in}}%
\pgfpathcurveto{\pgfqpoint{3.877314in}{2.172991in}}{\pgfqpoint{3.874042in}{2.165091in}}{\pgfqpoint{3.874042in}{2.156854in}}%
\pgfpathcurveto{\pgfqpoint{3.874042in}{2.148618in}}{\pgfqpoint{3.877314in}{2.140718in}}{\pgfqpoint{3.883138in}{2.134894in}}%
\pgfpathcurveto{\pgfqpoint{3.888962in}{2.129070in}}{\pgfqpoint{3.896862in}{2.125798in}}{\pgfqpoint{3.905098in}{2.125798in}}%
\pgfpathclose%
\pgfusepath{stroke,fill}%
\end{pgfscope}%
\begin{pgfscope}%
\pgfpathrectangle{\pgfqpoint{3.793912in}{0.557870in}}{\pgfqpoint{2.446088in}{1.684734in}}%
\pgfusepath{clip}%
\pgfsetbuttcap%
\pgfsetroundjoin%
\definecolor{currentfill}{rgb}{0.298039,0.447059,0.690196}%
\pgfsetfillcolor{currentfill}%
\pgfsetlinewidth{1.003750pt}%
\definecolor{currentstroke}{rgb}{0.298039,0.447059,0.690196}%
\pgfsetstrokecolor{currentstroke}%
\pgfsetdash{}{0pt}%
\pgfpathmoveto{\pgfqpoint{3.905098in}{2.125798in}}%
\pgfpathcurveto{\pgfqpoint{3.913334in}{2.125798in}}{\pgfqpoint{3.921234in}{2.129070in}}{\pgfqpoint{3.927058in}{2.134894in}}%
\pgfpathcurveto{\pgfqpoint{3.932882in}{2.140718in}}{\pgfqpoint{3.936155in}{2.148618in}}{\pgfqpoint{3.936155in}{2.156854in}}%
\pgfpathcurveto{\pgfqpoint{3.936155in}{2.165091in}}{\pgfqpoint{3.932882in}{2.172991in}}{\pgfqpoint{3.927058in}{2.178814in}}%
\pgfpathcurveto{\pgfqpoint{3.921234in}{2.184638in}}{\pgfqpoint{3.913334in}{2.187911in}}{\pgfqpoint{3.905098in}{2.187911in}}%
\pgfpathcurveto{\pgfqpoint{3.896862in}{2.187911in}}{\pgfqpoint{3.888962in}{2.184638in}}{\pgfqpoint{3.883138in}{2.178814in}}%
\pgfpathcurveto{\pgfqpoint{3.877314in}{2.172991in}}{\pgfqpoint{3.874042in}{2.165091in}}{\pgfqpoint{3.874042in}{2.156854in}}%
\pgfpathcurveto{\pgfqpoint{3.874042in}{2.148618in}}{\pgfqpoint{3.877314in}{2.140718in}}{\pgfqpoint{3.883138in}{2.134894in}}%
\pgfpathcurveto{\pgfqpoint{3.888962in}{2.129070in}}{\pgfqpoint{3.896862in}{2.125798in}}{\pgfqpoint{3.905098in}{2.125798in}}%
\pgfpathclose%
\pgfusepath{stroke,fill}%
\end{pgfscope}%
\begin{pgfscope}%
\pgfpathrectangle{\pgfqpoint{3.793912in}{0.557870in}}{\pgfqpoint{2.446088in}{1.684734in}}%
\pgfusepath{clip}%
\pgfsetbuttcap%
\pgfsetroundjoin%
\definecolor{currentfill}{rgb}{0.298039,0.447059,0.690196}%
\pgfsetfillcolor{currentfill}%
\pgfsetlinewidth{1.003750pt}%
\definecolor{currentstroke}{rgb}{0.298039,0.447059,0.690196}%
\pgfsetstrokecolor{currentstroke}%
\pgfsetdash{}{0pt}%
\pgfpathmoveto{\pgfqpoint{5.898775in}{1.346253in}}%
\pgfpathcurveto{\pgfqpoint{5.907011in}{1.346253in}}{\pgfqpoint{5.914911in}{1.349525in}}{\pgfqpoint{5.920735in}{1.355349in}}%
\pgfpathcurveto{\pgfqpoint{5.926559in}{1.361173in}}{\pgfqpoint{5.929831in}{1.369073in}}{\pgfqpoint{5.929831in}{1.377309in}}%
\pgfpathcurveto{\pgfqpoint{5.929831in}{1.385546in}}{\pgfqpoint{5.926559in}{1.393446in}}{\pgfqpoint{5.920735in}{1.399270in}}%
\pgfpathcurveto{\pgfqpoint{5.914911in}{1.405094in}}{\pgfqpoint{5.907011in}{1.408366in}}{\pgfqpoint{5.898775in}{1.408366in}}%
\pgfpathcurveto{\pgfqpoint{5.890538in}{1.408366in}}{\pgfqpoint{5.882638in}{1.405094in}}{\pgfqpoint{5.876814in}{1.399270in}}%
\pgfpathcurveto{\pgfqpoint{5.870990in}{1.393446in}}{\pgfqpoint{5.867718in}{1.385546in}}{\pgfqpoint{5.867718in}{1.377309in}}%
\pgfpathcurveto{\pgfqpoint{5.867718in}{1.369073in}}{\pgfqpoint{5.870990in}{1.361173in}}{\pgfqpoint{5.876814in}{1.355349in}}%
\pgfpathcurveto{\pgfqpoint{5.882638in}{1.349525in}}{\pgfqpoint{5.890538in}{1.346253in}}{\pgfqpoint{5.898775in}{1.346253in}}%
\pgfpathclose%
\pgfusepath{stroke,fill}%
\end{pgfscope}%
\begin{pgfscope}%
\pgfpathrectangle{\pgfqpoint{3.793912in}{0.557870in}}{\pgfqpoint{2.446088in}{1.684734in}}%
\pgfusepath{clip}%
\pgfsetbuttcap%
\pgfsetroundjoin%
\definecolor{currentfill}{rgb}{0.298039,0.447059,0.690196}%
\pgfsetfillcolor{currentfill}%
\pgfsetlinewidth{1.003750pt}%
\definecolor{currentstroke}{rgb}{0.298039,0.447059,0.690196}%
\pgfsetstrokecolor{currentstroke}%
\pgfsetdash{}{0pt}%
\pgfpathmoveto{\pgfqpoint{5.975454in}{1.291226in}}%
\pgfpathcurveto{\pgfqpoint{5.983691in}{1.291226in}}{\pgfqpoint{5.991591in}{1.294499in}}{\pgfqpoint{5.997415in}{1.300322in}}%
\pgfpathcurveto{\pgfqpoint{6.003239in}{1.306146in}}{\pgfqpoint{6.006511in}{1.314046in}}{\pgfqpoint{6.006511in}{1.322283in}}%
\pgfpathcurveto{\pgfqpoint{6.006511in}{1.330519in}}{\pgfqpoint{6.003239in}{1.338419in}}{\pgfqpoint{5.997415in}{1.344243in}}%
\pgfpathcurveto{\pgfqpoint{5.991591in}{1.350067in}}{\pgfqpoint{5.983691in}{1.353339in}}{\pgfqpoint{5.975454in}{1.353339in}}%
\pgfpathcurveto{\pgfqpoint{5.967218in}{1.353339in}}{\pgfqpoint{5.959318in}{1.350067in}}{\pgfqpoint{5.953494in}{1.344243in}}%
\pgfpathcurveto{\pgfqpoint{5.947670in}{1.338419in}}{\pgfqpoint{5.944398in}{1.330519in}}{\pgfqpoint{5.944398in}{1.322283in}}%
\pgfpathcurveto{\pgfqpoint{5.944398in}{1.314046in}}{\pgfqpoint{5.947670in}{1.306146in}}{\pgfqpoint{5.953494in}{1.300322in}}%
\pgfpathcurveto{\pgfqpoint{5.959318in}{1.294499in}}{\pgfqpoint{5.967218in}{1.291226in}}{\pgfqpoint{5.975454in}{1.291226in}}%
\pgfpathclose%
\pgfusepath{stroke,fill}%
\end{pgfscope}%
\begin{pgfscope}%
\pgfpathrectangle{\pgfqpoint{3.793912in}{0.557870in}}{\pgfqpoint{2.446088in}{1.684734in}}%
\pgfusepath{clip}%
\pgfsetbuttcap%
\pgfsetroundjoin%
\definecolor{currentfill}{rgb}{0.298039,0.447059,0.690196}%
\pgfsetfillcolor{currentfill}%
\pgfsetlinewidth{1.003750pt}%
\definecolor{currentstroke}{rgb}{0.298039,0.447059,0.690196}%
\pgfsetstrokecolor{currentstroke}%
\pgfsetdash{}{0pt}%
\pgfpathmoveto{\pgfqpoint{4.058458in}{1.492991in}}%
\pgfpathcurveto{\pgfqpoint{4.066694in}{1.492991in}}{\pgfqpoint{4.074594in}{1.496263in}}{\pgfqpoint{4.080418in}{1.502087in}}%
\pgfpathcurveto{\pgfqpoint{4.086242in}{1.507911in}}{\pgfqpoint{4.089514in}{1.515811in}}{\pgfqpoint{4.089514in}{1.524047in}}%
\pgfpathcurveto{\pgfqpoint{4.089514in}{1.532284in}}{\pgfqpoint{4.086242in}{1.540184in}}{\pgfqpoint{4.080418in}{1.546008in}}%
\pgfpathcurveto{\pgfqpoint{4.074594in}{1.551831in}}{\pgfqpoint{4.066694in}{1.555104in}}{\pgfqpoint{4.058458in}{1.555104in}}%
\pgfpathcurveto{\pgfqpoint{4.050221in}{1.555104in}}{\pgfqpoint{4.042321in}{1.551831in}}{\pgfqpoint{4.036498in}{1.546008in}}%
\pgfpathcurveto{\pgfqpoint{4.030674in}{1.540184in}}{\pgfqpoint{4.027401in}{1.532284in}}{\pgfqpoint{4.027401in}{1.524047in}}%
\pgfpathcurveto{\pgfqpoint{4.027401in}{1.515811in}}{\pgfqpoint{4.030674in}{1.507911in}}{\pgfqpoint{4.036498in}{1.502087in}}%
\pgfpathcurveto{\pgfqpoint{4.042321in}{1.496263in}}{\pgfqpoint{4.050221in}{1.492991in}}{\pgfqpoint{4.058458in}{1.492991in}}%
\pgfpathclose%
\pgfusepath{stroke,fill}%
\end{pgfscope}%
\begin{pgfscope}%
\pgfpathrectangle{\pgfqpoint{3.793912in}{0.557870in}}{\pgfqpoint{2.446088in}{1.684734in}}%
\pgfusepath{clip}%
\pgfsetbuttcap%
\pgfsetroundjoin%
\definecolor{currentfill}{rgb}{0.298039,0.447059,0.690196}%
\pgfsetfillcolor{currentfill}%
\pgfsetlinewidth{1.003750pt}%
\definecolor{currentstroke}{rgb}{0.298039,0.447059,0.690196}%
\pgfsetstrokecolor{currentstroke}%
\pgfsetdash{}{0pt}%
\pgfpathmoveto{\pgfqpoint{3.981778in}{2.125798in}}%
\pgfpathcurveto{\pgfqpoint{3.990014in}{2.125798in}}{\pgfqpoint{3.997914in}{2.129070in}}{\pgfqpoint{4.003738in}{2.134894in}}%
\pgfpathcurveto{\pgfqpoint{4.009562in}{2.140718in}}{\pgfqpoint{4.012834in}{2.148618in}}{\pgfqpoint{4.012834in}{2.156854in}}%
\pgfpathcurveto{\pgfqpoint{4.012834in}{2.165091in}}{\pgfqpoint{4.009562in}{2.172991in}}{\pgfqpoint{4.003738in}{2.178814in}}%
\pgfpathcurveto{\pgfqpoint{3.997914in}{2.184638in}}{\pgfqpoint{3.990014in}{2.187911in}}{\pgfqpoint{3.981778in}{2.187911in}}%
\pgfpathcurveto{\pgfqpoint{3.973542in}{2.187911in}}{\pgfqpoint{3.965642in}{2.184638in}}{\pgfqpoint{3.959818in}{2.178814in}}%
\pgfpathcurveto{\pgfqpoint{3.953994in}{2.172991in}}{\pgfqpoint{3.950721in}{2.165091in}}{\pgfqpoint{3.950721in}{2.156854in}}%
\pgfpathcurveto{\pgfqpoint{3.950721in}{2.148618in}}{\pgfqpoint{3.953994in}{2.140718in}}{\pgfqpoint{3.959818in}{2.134894in}}%
\pgfpathcurveto{\pgfqpoint{3.965642in}{2.129070in}}{\pgfqpoint{3.973542in}{2.125798in}}{\pgfqpoint{3.981778in}{2.125798in}}%
\pgfpathclose%
\pgfusepath{stroke,fill}%
\end{pgfscope}%
\begin{pgfscope}%
\pgfpathrectangle{\pgfqpoint{3.793912in}{0.557870in}}{\pgfqpoint{2.446088in}{1.684734in}}%
\pgfusepath{clip}%
\pgfsetbuttcap%
\pgfsetroundjoin%
\definecolor{currentfill}{rgb}{0.298039,0.447059,0.690196}%
\pgfsetfillcolor{currentfill}%
\pgfsetlinewidth{1.003750pt}%
\definecolor{currentstroke}{rgb}{0.298039,0.447059,0.690196}%
\pgfsetstrokecolor{currentstroke}%
\pgfsetdash{}{0pt}%
\pgfpathmoveto{\pgfqpoint{3.905098in}{2.125798in}}%
\pgfpathcurveto{\pgfqpoint{3.913334in}{2.125798in}}{\pgfqpoint{3.921234in}{2.129070in}}{\pgfqpoint{3.927058in}{2.134894in}}%
\pgfpathcurveto{\pgfqpoint{3.932882in}{2.140718in}}{\pgfqpoint{3.936155in}{2.148618in}}{\pgfqpoint{3.936155in}{2.156854in}}%
\pgfpathcurveto{\pgfqpoint{3.936155in}{2.165091in}}{\pgfqpoint{3.932882in}{2.172991in}}{\pgfqpoint{3.927058in}{2.178814in}}%
\pgfpathcurveto{\pgfqpoint{3.921234in}{2.184638in}}{\pgfqpoint{3.913334in}{2.187911in}}{\pgfqpoint{3.905098in}{2.187911in}}%
\pgfpathcurveto{\pgfqpoint{3.896862in}{2.187911in}}{\pgfqpoint{3.888962in}{2.184638in}}{\pgfqpoint{3.883138in}{2.178814in}}%
\pgfpathcurveto{\pgfqpoint{3.877314in}{2.172991in}}{\pgfqpoint{3.874042in}{2.165091in}}{\pgfqpoint{3.874042in}{2.156854in}}%
\pgfpathcurveto{\pgfqpoint{3.874042in}{2.148618in}}{\pgfqpoint{3.877314in}{2.140718in}}{\pgfqpoint{3.883138in}{2.134894in}}%
\pgfpathcurveto{\pgfqpoint{3.888962in}{2.129070in}}{\pgfqpoint{3.896862in}{2.125798in}}{\pgfqpoint{3.905098in}{2.125798in}}%
\pgfpathclose%
\pgfusepath{stroke,fill}%
\end{pgfscope}%
\begin{pgfscope}%
\pgfpathrectangle{\pgfqpoint{3.793912in}{0.557870in}}{\pgfqpoint{2.446088in}{1.684734in}}%
\pgfusepath{clip}%
\pgfsetbuttcap%
\pgfsetroundjoin%
\definecolor{currentfill}{rgb}{0.298039,0.447059,0.690196}%
\pgfsetfillcolor{currentfill}%
\pgfsetlinewidth{1.003750pt}%
\definecolor{currentstroke}{rgb}{0.298039,0.447059,0.690196}%
\pgfsetstrokecolor{currentstroke}%
\pgfsetdash{}{0pt}%
\pgfpathmoveto{\pgfqpoint{5.975454in}{1.447135in}}%
\pgfpathcurveto{\pgfqpoint{5.983691in}{1.447135in}}{\pgfqpoint{5.991591in}{1.450408in}}{\pgfqpoint{5.997415in}{1.456231in}}%
\pgfpathcurveto{\pgfqpoint{6.003239in}{1.462055in}}{\pgfqpoint{6.006511in}{1.469955in}}{\pgfqpoint{6.006511in}{1.478192in}}%
\pgfpathcurveto{\pgfqpoint{6.006511in}{1.486428in}}{\pgfqpoint{6.003239in}{1.494328in}}{\pgfqpoint{5.997415in}{1.500152in}}%
\pgfpathcurveto{\pgfqpoint{5.991591in}{1.505976in}}{\pgfqpoint{5.983691in}{1.509248in}}{\pgfqpoint{5.975454in}{1.509248in}}%
\pgfpathcurveto{\pgfqpoint{5.967218in}{1.509248in}}{\pgfqpoint{5.959318in}{1.505976in}}{\pgfqpoint{5.953494in}{1.500152in}}%
\pgfpathcurveto{\pgfqpoint{5.947670in}{1.494328in}}{\pgfqpoint{5.944398in}{1.486428in}}{\pgfqpoint{5.944398in}{1.478192in}}%
\pgfpathcurveto{\pgfqpoint{5.944398in}{1.469955in}}{\pgfqpoint{5.947670in}{1.462055in}}{\pgfqpoint{5.953494in}{1.456231in}}%
\pgfpathcurveto{\pgfqpoint{5.959318in}{1.450408in}}{\pgfqpoint{5.967218in}{1.447135in}}{\pgfqpoint{5.975454in}{1.447135in}}%
\pgfpathclose%
\pgfusepath{stroke,fill}%
\end{pgfscope}%
\begin{pgfscope}%
\pgfpathrectangle{\pgfqpoint{3.793912in}{0.557870in}}{\pgfqpoint{2.446088in}{1.684734in}}%
\pgfusepath{clip}%
\pgfsetbuttcap%
\pgfsetroundjoin%
\definecolor{currentfill}{rgb}{0.298039,0.447059,0.690196}%
\pgfsetfillcolor{currentfill}%
\pgfsetlinewidth{1.003750pt}%
\definecolor{currentstroke}{rgb}{0.298039,0.447059,0.690196}%
\pgfsetstrokecolor{currentstroke}%
\pgfsetdash{}{0pt}%
\pgfpathmoveto{\pgfqpoint{3.905098in}{2.125798in}}%
\pgfpathcurveto{\pgfqpoint{3.913334in}{2.125798in}}{\pgfqpoint{3.921234in}{2.129070in}}{\pgfqpoint{3.927058in}{2.134894in}}%
\pgfpathcurveto{\pgfqpoint{3.932882in}{2.140718in}}{\pgfqpoint{3.936155in}{2.148618in}}{\pgfqpoint{3.936155in}{2.156854in}}%
\pgfpathcurveto{\pgfqpoint{3.936155in}{2.165091in}}{\pgfqpoint{3.932882in}{2.172991in}}{\pgfqpoint{3.927058in}{2.178814in}}%
\pgfpathcurveto{\pgfqpoint{3.921234in}{2.184638in}}{\pgfqpoint{3.913334in}{2.187911in}}{\pgfqpoint{3.905098in}{2.187911in}}%
\pgfpathcurveto{\pgfqpoint{3.896862in}{2.187911in}}{\pgfqpoint{3.888962in}{2.184638in}}{\pgfqpoint{3.883138in}{2.178814in}}%
\pgfpathcurveto{\pgfqpoint{3.877314in}{2.172991in}}{\pgfqpoint{3.874042in}{2.165091in}}{\pgfqpoint{3.874042in}{2.156854in}}%
\pgfpathcurveto{\pgfqpoint{3.874042in}{2.148618in}}{\pgfqpoint{3.877314in}{2.140718in}}{\pgfqpoint{3.883138in}{2.134894in}}%
\pgfpathcurveto{\pgfqpoint{3.888962in}{2.129070in}}{\pgfqpoint{3.896862in}{2.125798in}}{\pgfqpoint{3.905098in}{2.125798in}}%
\pgfpathclose%
\pgfusepath{stroke,fill}%
\end{pgfscope}%
\begin{pgfscope}%
\pgfpathrectangle{\pgfqpoint{3.793912in}{0.557870in}}{\pgfqpoint{2.446088in}{1.684734in}}%
\pgfusepath{clip}%
\pgfsetbuttcap%
\pgfsetroundjoin%
\definecolor{currentfill}{rgb}{0.298039,0.447059,0.690196}%
\pgfsetfillcolor{currentfill}%
\pgfsetlinewidth{1.003750pt}%
\definecolor{currentstroke}{rgb}{0.298039,0.447059,0.690196}%
\pgfsetstrokecolor{currentstroke}%
\pgfsetdash{}{0pt}%
\pgfpathmoveto{\pgfqpoint{4.058458in}{1.575531in}}%
\pgfpathcurveto{\pgfqpoint{4.066694in}{1.575531in}}{\pgfqpoint{4.074594in}{1.578803in}}{\pgfqpoint{4.080418in}{1.584627in}}%
\pgfpathcurveto{\pgfqpoint{4.086242in}{1.590451in}}{\pgfqpoint{4.089514in}{1.598351in}}{\pgfqpoint{4.089514in}{1.606587in}}%
\pgfpathcurveto{\pgfqpoint{4.089514in}{1.614824in}}{\pgfqpoint{4.086242in}{1.622724in}}{\pgfqpoint{4.080418in}{1.628548in}}%
\pgfpathcurveto{\pgfqpoint{4.074594in}{1.634372in}}{\pgfqpoint{4.066694in}{1.637644in}}{\pgfqpoint{4.058458in}{1.637644in}}%
\pgfpathcurveto{\pgfqpoint{4.050221in}{1.637644in}}{\pgfqpoint{4.042321in}{1.634372in}}{\pgfqpoint{4.036498in}{1.628548in}}%
\pgfpathcurveto{\pgfqpoint{4.030674in}{1.622724in}}{\pgfqpoint{4.027401in}{1.614824in}}{\pgfqpoint{4.027401in}{1.606587in}}%
\pgfpathcurveto{\pgfqpoint{4.027401in}{1.598351in}}{\pgfqpoint{4.030674in}{1.590451in}}{\pgfqpoint{4.036498in}{1.584627in}}%
\pgfpathcurveto{\pgfqpoint{4.042321in}{1.578803in}}{\pgfqpoint{4.050221in}{1.575531in}}{\pgfqpoint{4.058458in}{1.575531in}}%
\pgfpathclose%
\pgfusepath{stroke,fill}%
\end{pgfscope}%
\begin{pgfscope}%
\pgfpathrectangle{\pgfqpoint{3.793912in}{0.557870in}}{\pgfqpoint{2.446088in}{1.684734in}}%
\pgfusepath{clip}%
\pgfsetbuttcap%
\pgfsetroundjoin%
\definecolor{currentfill}{rgb}{0.298039,0.447059,0.690196}%
\pgfsetfillcolor{currentfill}%
\pgfsetlinewidth{1.003750pt}%
\definecolor{currentstroke}{rgb}{0.298039,0.447059,0.690196}%
\pgfsetstrokecolor{currentstroke}%
\pgfsetdash{}{0pt}%
\pgfpathmoveto{\pgfqpoint{3.905098in}{2.125798in}}%
\pgfpathcurveto{\pgfqpoint{3.913334in}{2.125798in}}{\pgfqpoint{3.921234in}{2.129070in}}{\pgfqpoint{3.927058in}{2.134894in}}%
\pgfpathcurveto{\pgfqpoint{3.932882in}{2.140718in}}{\pgfqpoint{3.936155in}{2.148618in}}{\pgfqpoint{3.936155in}{2.156854in}}%
\pgfpathcurveto{\pgfqpoint{3.936155in}{2.165091in}}{\pgfqpoint{3.932882in}{2.172991in}}{\pgfqpoint{3.927058in}{2.178814in}}%
\pgfpathcurveto{\pgfqpoint{3.921234in}{2.184638in}}{\pgfqpoint{3.913334in}{2.187911in}}{\pgfqpoint{3.905098in}{2.187911in}}%
\pgfpathcurveto{\pgfqpoint{3.896862in}{2.187911in}}{\pgfqpoint{3.888962in}{2.184638in}}{\pgfqpoint{3.883138in}{2.178814in}}%
\pgfpathcurveto{\pgfqpoint{3.877314in}{2.172991in}}{\pgfqpoint{3.874042in}{2.165091in}}{\pgfqpoint{3.874042in}{2.156854in}}%
\pgfpathcurveto{\pgfqpoint{3.874042in}{2.148618in}}{\pgfqpoint{3.877314in}{2.140718in}}{\pgfqpoint{3.883138in}{2.134894in}}%
\pgfpathcurveto{\pgfqpoint{3.888962in}{2.129070in}}{\pgfqpoint{3.896862in}{2.125798in}}{\pgfqpoint{3.905098in}{2.125798in}}%
\pgfpathclose%
\pgfusepath{stroke,fill}%
\end{pgfscope}%
\begin{pgfscope}%
\pgfpathrectangle{\pgfqpoint{3.793912in}{0.557870in}}{\pgfqpoint{2.446088in}{1.684734in}}%
\pgfusepath{clip}%
\pgfsetbuttcap%
\pgfsetroundjoin%
\definecolor{currentfill}{rgb}{0.298039,0.447059,0.690196}%
\pgfsetfillcolor{currentfill}%
\pgfsetlinewidth{1.003750pt}%
\definecolor{currentstroke}{rgb}{0.298039,0.447059,0.690196}%
\pgfsetstrokecolor{currentstroke}%
\pgfsetdash{}{0pt}%
\pgfpathmoveto{\pgfqpoint{3.905098in}{2.125798in}}%
\pgfpathcurveto{\pgfqpoint{3.913334in}{2.125798in}}{\pgfqpoint{3.921234in}{2.129070in}}{\pgfqpoint{3.927058in}{2.134894in}}%
\pgfpathcurveto{\pgfqpoint{3.932882in}{2.140718in}}{\pgfqpoint{3.936155in}{2.148618in}}{\pgfqpoint{3.936155in}{2.156854in}}%
\pgfpathcurveto{\pgfqpoint{3.936155in}{2.165091in}}{\pgfqpoint{3.932882in}{2.172991in}}{\pgfqpoint{3.927058in}{2.178814in}}%
\pgfpathcurveto{\pgfqpoint{3.921234in}{2.184638in}}{\pgfqpoint{3.913334in}{2.187911in}}{\pgfqpoint{3.905098in}{2.187911in}}%
\pgfpathcurveto{\pgfqpoint{3.896862in}{2.187911in}}{\pgfqpoint{3.888962in}{2.184638in}}{\pgfqpoint{3.883138in}{2.178814in}}%
\pgfpathcurveto{\pgfqpoint{3.877314in}{2.172991in}}{\pgfqpoint{3.874042in}{2.165091in}}{\pgfqpoint{3.874042in}{2.156854in}}%
\pgfpathcurveto{\pgfqpoint{3.874042in}{2.148618in}}{\pgfqpoint{3.877314in}{2.140718in}}{\pgfqpoint{3.883138in}{2.134894in}}%
\pgfpathcurveto{\pgfqpoint{3.888962in}{2.129070in}}{\pgfqpoint{3.896862in}{2.125798in}}{\pgfqpoint{3.905098in}{2.125798in}}%
\pgfpathclose%
\pgfusepath{stroke,fill}%
\end{pgfscope}%
\begin{pgfscope}%
\pgfpathrectangle{\pgfqpoint{3.793912in}{0.557870in}}{\pgfqpoint{2.446088in}{1.684734in}}%
\pgfusepath{clip}%
\pgfsetbuttcap%
\pgfsetroundjoin%
\definecolor{currentfill}{rgb}{0.298039,0.447059,0.690196}%
\pgfsetfillcolor{currentfill}%
\pgfsetlinewidth{1.003750pt}%
\definecolor{currentstroke}{rgb}{0.298039,0.447059,0.690196}%
\pgfsetstrokecolor{currentstroke}%
\pgfsetdash{}{0pt}%
\pgfpathmoveto{\pgfqpoint{3.905098in}{2.125798in}}%
\pgfpathcurveto{\pgfqpoint{3.913334in}{2.125798in}}{\pgfqpoint{3.921234in}{2.129070in}}{\pgfqpoint{3.927058in}{2.134894in}}%
\pgfpathcurveto{\pgfqpoint{3.932882in}{2.140718in}}{\pgfqpoint{3.936155in}{2.148618in}}{\pgfqpoint{3.936155in}{2.156854in}}%
\pgfpathcurveto{\pgfqpoint{3.936155in}{2.165091in}}{\pgfqpoint{3.932882in}{2.172991in}}{\pgfqpoint{3.927058in}{2.178814in}}%
\pgfpathcurveto{\pgfqpoint{3.921234in}{2.184638in}}{\pgfqpoint{3.913334in}{2.187911in}}{\pgfqpoint{3.905098in}{2.187911in}}%
\pgfpathcurveto{\pgfqpoint{3.896862in}{2.187911in}}{\pgfqpoint{3.888962in}{2.184638in}}{\pgfqpoint{3.883138in}{2.178814in}}%
\pgfpathcurveto{\pgfqpoint{3.877314in}{2.172991in}}{\pgfqpoint{3.874042in}{2.165091in}}{\pgfqpoint{3.874042in}{2.156854in}}%
\pgfpathcurveto{\pgfqpoint{3.874042in}{2.148618in}}{\pgfqpoint{3.877314in}{2.140718in}}{\pgfqpoint{3.883138in}{2.134894in}}%
\pgfpathcurveto{\pgfqpoint{3.888962in}{2.129070in}}{\pgfqpoint{3.896862in}{2.125798in}}{\pgfqpoint{3.905098in}{2.125798in}}%
\pgfpathclose%
\pgfusepath{stroke,fill}%
\end{pgfscope}%
\begin{pgfscope}%
\pgfpathrectangle{\pgfqpoint{3.793912in}{0.557870in}}{\pgfqpoint{2.446088in}{1.684734in}}%
\pgfusepath{clip}%
\pgfsetbuttcap%
\pgfsetroundjoin%
\definecolor{currentfill}{rgb}{0.298039,0.447059,0.690196}%
\pgfsetfillcolor{currentfill}%
\pgfsetlinewidth{1.003750pt}%
\definecolor{currentstroke}{rgb}{0.298039,0.447059,0.690196}%
\pgfsetstrokecolor{currentstroke}%
\pgfsetdash{}{0pt}%
\pgfpathmoveto{\pgfqpoint{4.978616in}{1.804809in}}%
\pgfpathcurveto{\pgfqpoint{4.986852in}{1.804809in}}{\pgfqpoint{4.994753in}{1.808081in}}{\pgfqpoint{5.000576in}{1.813905in}}%
\pgfpathcurveto{\pgfqpoint{5.006400in}{1.819729in}}{\pgfqpoint{5.009673in}{1.827629in}}{\pgfqpoint{5.009673in}{1.835865in}}%
\pgfpathcurveto{\pgfqpoint{5.009673in}{1.844101in}}{\pgfqpoint{5.006400in}{1.852002in}}{\pgfqpoint{5.000576in}{1.857825in}}%
\pgfpathcurveto{\pgfqpoint{4.994753in}{1.863649in}}{\pgfqpoint{4.986852in}{1.866922in}}{\pgfqpoint{4.978616in}{1.866922in}}%
\pgfpathcurveto{\pgfqpoint{4.970380in}{1.866922in}}{\pgfqpoint{4.962480in}{1.863649in}}{\pgfqpoint{4.956656in}{1.857825in}}%
\pgfpathcurveto{\pgfqpoint{4.950832in}{1.852002in}}{\pgfqpoint{4.947560in}{1.844101in}}{\pgfqpoint{4.947560in}{1.835865in}}%
\pgfpathcurveto{\pgfqpoint{4.947560in}{1.827629in}}{\pgfqpoint{4.950832in}{1.819729in}}{\pgfqpoint{4.956656in}{1.813905in}}%
\pgfpathcurveto{\pgfqpoint{4.962480in}{1.808081in}}{\pgfqpoint{4.970380in}{1.804809in}}{\pgfqpoint{4.978616in}{1.804809in}}%
\pgfpathclose%
\pgfusepath{stroke,fill}%
\end{pgfscope}%
\begin{pgfscope}%
\pgfpathrectangle{\pgfqpoint{3.793912in}{0.557870in}}{\pgfqpoint{2.446088in}{1.684734in}}%
\pgfusepath{clip}%
\pgfsetbuttcap%
\pgfsetroundjoin%
\definecolor{currentfill}{rgb}{0.298039,0.447059,0.690196}%
\pgfsetfillcolor{currentfill}%
\pgfsetlinewidth{1.003750pt}%
\definecolor{currentstroke}{rgb}{0.298039,0.447059,0.690196}%
\pgfsetstrokecolor{currentstroke}%
\pgfsetdash{}{0pt}%
\pgfpathmoveto{\pgfqpoint{3.905098in}{2.125798in}}%
\pgfpathcurveto{\pgfqpoint{3.913334in}{2.125798in}}{\pgfqpoint{3.921234in}{2.129070in}}{\pgfqpoint{3.927058in}{2.134894in}}%
\pgfpathcurveto{\pgfqpoint{3.932882in}{2.140718in}}{\pgfqpoint{3.936155in}{2.148618in}}{\pgfqpoint{3.936155in}{2.156854in}}%
\pgfpathcurveto{\pgfqpoint{3.936155in}{2.165091in}}{\pgfqpoint{3.932882in}{2.172991in}}{\pgfqpoint{3.927058in}{2.178814in}}%
\pgfpathcurveto{\pgfqpoint{3.921234in}{2.184638in}}{\pgfqpoint{3.913334in}{2.187911in}}{\pgfqpoint{3.905098in}{2.187911in}}%
\pgfpathcurveto{\pgfqpoint{3.896862in}{2.187911in}}{\pgfqpoint{3.888962in}{2.184638in}}{\pgfqpoint{3.883138in}{2.178814in}}%
\pgfpathcurveto{\pgfqpoint{3.877314in}{2.172991in}}{\pgfqpoint{3.874042in}{2.165091in}}{\pgfqpoint{3.874042in}{2.156854in}}%
\pgfpathcurveto{\pgfqpoint{3.874042in}{2.148618in}}{\pgfqpoint{3.877314in}{2.140718in}}{\pgfqpoint{3.883138in}{2.134894in}}%
\pgfpathcurveto{\pgfqpoint{3.888962in}{2.129070in}}{\pgfqpoint{3.896862in}{2.125798in}}{\pgfqpoint{3.905098in}{2.125798in}}%
\pgfpathclose%
\pgfusepath{stroke,fill}%
\end{pgfscope}%
\begin{pgfscope}%
\pgfpathrectangle{\pgfqpoint{3.793912in}{0.557870in}}{\pgfqpoint{2.446088in}{1.684734in}}%
\pgfusepath{clip}%
\pgfsetbuttcap%
\pgfsetroundjoin%
\definecolor{currentfill}{rgb}{0.298039,0.447059,0.690196}%
\pgfsetfillcolor{currentfill}%
\pgfsetlinewidth{1.003750pt}%
\definecolor{currentstroke}{rgb}{0.298039,0.447059,0.690196}%
\pgfsetstrokecolor{currentstroke}%
\pgfsetdash{}{0pt}%
\pgfpathmoveto{\pgfqpoint{5.131976in}{1.758953in}}%
\pgfpathcurveto{\pgfqpoint{5.140212in}{1.758953in}}{\pgfqpoint{5.148112in}{1.762225in}}{\pgfqpoint{5.153936in}{1.768049in}}%
\pgfpathcurveto{\pgfqpoint{5.159760in}{1.773873in}}{\pgfqpoint{5.163032in}{1.781773in}}{\pgfqpoint{5.163032in}{1.790010in}}%
\pgfpathcurveto{\pgfqpoint{5.163032in}{1.798246in}}{\pgfqpoint{5.159760in}{1.806146in}}{\pgfqpoint{5.153936in}{1.811970in}}%
\pgfpathcurveto{\pgfqpoint{5.148112in}{1.817794in}}{\pgfqpoint{5.140212in}{1.821066in}}{\pgfqpoint{5.131976in}{1.821066in}}%
\pgfpathcurveto{\pgfqpoint{5.123740in}{1.821066in}}{\pgfqpoint{5.115840in}{1.817794in}}{\pgfqpoint{5.110016in}{1.811970in}}%
\pgfpathcurveto{\pgfqpoint{5.104192in}{1.806146in}}{\pgfqpoint{5.100919in}{1.798246in}}{\pgfqpoint{5.100919in}{1.790010in}}%
\pgfpathcurveto{\pgfqpoint{5.100919in}{1.781773in}}{\pgfqpoint{5.104192in}{1.773873in}}{\pgfqpoint{5.110016in}{1.768049in}}%
\pgfpathcurveto{\pgfqpoint{5.115840in}{1.762225in}}{\pgfqpoint{5.123740in}{1.758953in}}{\pgfqpoint{5.131976in}{1.758953in}}%
\pgfpathclose%
\pgfusepath{stroke,fill}%
\end{pgfscope}%
\begin{pgfscope}%
\pgfpathrectangle{\pgfqpoint{3.793912in}{0.557870in}}{\pgfqpoint{2.446088in}{1.684734in}}%
\pgfusepath{clip}%
\pgfsetbuttcap%
\pgfsetroundjoin%
\definecolor{currentfill}{rgb}{0.298039,0.447059,0.690196}%
\pgfsetfillcolor{currentfill}%
\pgfsetlinewidth{1.003750pt}%
\definecolor{currentstroke}{rgb}{0.298039,0.447059,0.690196}%
\pgfsetstrokecolor{currentstroke}%
\pgfsetdash{}{0pt}%
\pgfpathmoveto{\pgfqpoint{3.905098in}{2.125798in}}%
\pgfpathcurveto{\pgfqpoint{3.913334in}{2.125798in}}{\pgfqpoint{3.921234in}{2.129070in}}{\pgfqpoint{3.927058in}{2.134894in}}%
\pgfpathcurveto{\pgfqpoint{3.932882in}{2.140718in}}{\pgfqpoint{3.936155in}{2.148618in}}{\pgfqpoint{3.936155in}{2.156854in}}%
\pgfpathcurveto{\pgfqpoint{3.936155in}{2.165091in}}{\pgfqpoint{3.932882in}{2.172991in}}{\pgfqpoint{3.927058in}{2.178814in}}%
\pgfpathcurveto{\pgfqpoint{3.921234in}{2.184638in}}{\pgfqpoint{3.913334in}{2.187911in}}{\pgfqpoint{3.905098in}{2.187911in}}%
\pgfpathcurveto{\pgfqpoint{3.896862in}{2.187911in}}{\pgfqpoint{3.888962in}{2.184638in}}{\pgfqpoint{3.883138in}{2.178814in}}%
\pgfpathcurveto{\pgfqpoint{3.877314in}{2.172991in}}{\pgfqpoint{3.874042in}{2.165091in}}{\pgfqpoint{3.874042in}{2.156854in}}%
\pgfpathcurveto{\pgfqpoint{3.874042in}{2.148618in}}{\pgfqpoint{3.877314in}{2.140718in}}{\pgfqpoint{3.883138in}{2.134894in}}%
\pgfpathcurveto{\pgfqpoint{3.888962in}{2.129070in}}{\pgfqpoint{3.896862in}{2.125798in}}{\pgfqpoint{3.905098in}{2.125798in}}%
\pgfpathclose%
\pgfusepath{stroke,fill}%
\end{pgfscope}%
\begin{pgfscope}%
\pgfpathrectangle{\pgfqpoint{3.793912in}{0.557870in}}{\pgfqpoint{2.446088in}{1.684734in}}%
\pgfusepath{clip}%
\pgfsetbuttcap%
\pgfsetroundjoin%
\definecolor{currentfill}{rgb}{0.298039,0.447059,0.690196}%
\pgfsetfillcolor{currentfill}%
\pgfsetlinewidth{1.003750pt}%
\definecolor{currentstroke}{rgb}{0.298039,0.447059,0.690196}%
\pgfsetstrokecolor{currentstroke}%
\pgfsetdash{}{0pt}%
\pgfpathmoveto{\pgfqpoint{3.905098in}{1.777295in}}%
\pgfpathcurveto{\pgfqpoint{3.913334in}{1.777295in}}{\pgfqpoint{3.921234in}{1.780568in}}{\pgfqpoint{3.927058in}{1.786392in}}%
\pgfpathcurveto{\pgfqpoint{3.932882in}{1.792216in}}{\pgfqpoint{3.936155in}{1.800116in}}{\pgfqpoint{3.936155in}{1.808352in}}%
\pgfpathcurveto{\pgfqpoint{3.936155in}{1.816588in}}{\pgfqpoint{3.932882in}{1.824488in}}{\pgfqpoint{3.927058in}{1.830312in}}%
\pgfpathcurveto{\pgfqpoint{3.921234in}{1.836136in}}{\pgfqpoint{3.913334in}{1.839408in}}{\pgfqpoint{3.905098in}{1.839408in}}%
\pgfpathcurveto{\pgfqpoint{3.896862in}{1.839408in}}{\pgfqpoint{3.888962in}{1.836136in}}{\pgfqpoint{3.883138in}{1.830312in}}%
\pgfpathcurveto{\pgfqpoint{3.877314in}{1.824488in}}{\pgfqpoint{3.874042in}{1.816588in}}{\pgfqpoint{3.874042in}{1.808352in}}%
\pgfpathcurveto{\pgfqpoint{3.874042in}{1.800116in}}{\pgfqpoint{3.877314in}{1.792216in}}{\pgfqpoint{3.883138in}{1.786392in}}%
\pgfpathcurveto{\pgfqpoint{3.888962in}{1.780568in}}{\pgfqpoint{3.896862in}{1.777295in}}{\pgfqpoint{3.905098in}{1.777295in}}%
\pgfpathclose%
\pgfusepath{stroke,fill}%
\end{pgfscope}%
\begin{pgfscope}%
\pgfpathrectangle{\pgfqpoint{3.793912in}{0.557870in}}{\pgfqpoint{2.446088in}{1.684734in}}%
\pgfusepath{clip}%
\pgfsetbuttcap%
\pgfsetroundjoin%
\definecolor{currentfill}{rgb}{0.298039,0.447059,0.690196}%
\pgfsetfillcolor{currentfill}%
\pgfsetlinewidth{1.003750pt}%
\definecolor{currentstroke}{rgb}{0.298039,0.447059,0.690196}%
\pgfsetstrokecolor{currentstroke}%
\pgfsetdash{}{0pt}%
\pgfpathmoveto{\pgfqpoint{3.905098in}{2.125798in}}%
\pgfpathcurveto{\pgfqpoint{3.913334in}{2.125798in}}{\pgfqpoint{3.921234in}{2.129070in}}{\pgfqpoint{3.927058in}{2.134894in}}%
\pgfpathcurveto{\pgfqpoint{3.932882in}{2.140718in}}{\pgfqpoint{3.936155in}{2.148618in}}{\pgfqpoint{3.936155in}{2.156854in}}%
\pgfpathcurveto{\pgfqpoint{3.936155in}{2.165091in}}{\pgfqpoint{3.932882in}{2.172991in}}{\pgfqpoint{3.927058in}{2.178814in}}%
\pgfpathcurveto{\pgfqpoint{3.921234in}{2.184638in}}{\pgfqpoint{3.913334in}{2.187911in}}{\pgfqpoint{3.905098in}{2.187911in}}%
\pgfpathcurveto{\pgfqpoint{3.896862in}{2.187911in}}{\pgfqpoint{3.888962in}{2.184638in}}{\pgfqpoint{3.883138in}{2.178814in}}%
\pgfpathcurveto{\pgfqpoint{3.877314in}{2.172991in}}{\pgfqpoint{3.874042in}{2.165091in}}{\pgfqpoint{3.874042in}{2.156854in}}%
\pgfpathcurveto{\pgfqpoint{3.874042in}{2.148618in}}{\pgfqpoint{3.877314in}{2.140718in}}{\pgfqpoint{3.883138in}{2.134894in}}%
\pgfpathcurveto{\pgfqpoint{3.888962in}{2.129070in}}{\pgfqpoint{3.896862in}{2.125798in}}{\pgfqpoint{3.905098in}{2.125798in}}%
\pgfpathclose%
\pgfusepath{stroke,fill}%
\end{pgfscope}%
\begin{pgfscope}%
\pgfpathrectangle{\pgfqpoint{3.793912in}{0.557870in}}{\pgfqpoint{2.446088in}{1.684734in}}%
\pgfusepath{clip}%
\pgfsetbuttcap%
\pgfsetroundjoin%
\definecolor{currentfill}{rgb}{0.298039,0.447059,0.690196}%
\pgfsetfillcolor{currentfill}%
\pgfsetlinewidth{1.003750pt}%
\definecolor{currentstroke}{rgb}{0.298039,0.447059,0.690196}%
\pgfsetstrokecolor{currentstroke}%
\pgfsetdash{}{0pt}%
\pgfpathmoveto{\pgfqpoint{3.905098in}{2.107456in}}%
\pgfpathcurveto{\pgfqpoint{3.913334in}{2.107456in}}{\pgfqpoint{3.921234in}{2.110728in}}{\pgfqpoint{3.927058in}{2.116552in}}%
\pgfpathcurveto{\pgfqpoint{3.932882in}{2.122376in}}{\pgfqpoint{3.936155in}{2.130276in}}{\pgfqpoint{3.936155in}{2.138512in}}%
\pgfpathcurveto{\pgfqpoint{3.936155in}{2.146748in}}{\pgfqpoint{3.932882in}{2.154648in}}{\pgfqpoint{3.927058in}{2.160472in}}%
\pgfpathcurveto{\pgfqpoint{3.921234in}{2.166296in}}{\pgfqpoint{3.913334in}{2.169569in}}{\pgfqpoint{3.905098in}{2.169569in}}%
\pgfpathcurveto{\pgfqpoint{3.896862in}{2.169569in}}{\pgfqpoint{3.888962in}{2.166296in}}{\pgfqpoint{3.883138in}{2.160472in}}%
\pgfpathcurveto{\pgfqpoint{3.877314in}{2.154648in}}{\pgfqpoint{3.874042in}{2.146748in}}{\pgfqpoint{3.874042in}{2.138512in}}%
\pgfpathcurveto{\pgfqpoint{3.874042in}{2.130276in}}{\pgfqpoint{3.877314in}{2.122376in}}{\pgfqpoint{3.883138in}{2.116552in}}%
\pgfpathcurveto{\pgfqpoint{3.888962in}{2.110728in}}{\pgfqpoint{3.896862in}{2.107456in}}{\pgfqpoint{3.905098in}{2.107456in}}%
\pgfpathclose%
\pgfusepath{stroke,fill}%
\end{pgfscope}%
\begin{pgfscope}%
\pgfpathrectangle{\pgfqpoint{3.793912in}{0.557870in}}{\pgfqpoint{2.446088in}{1.684734in}}%
\pgfusepath{clip}%
\pgfsetbuttcap%
\pgfsetroundjoin%
\definecolor{currentfill}{rgb}{0.298039,0.447059,0.690196}%
\pgfsetfillcolor{currentfill}%
\pgfsetlinewidth{1.003750pt}%
\definecolor{currentstroke}{rgb}{0.298039,0.447059,0.690196}%
\pgfsetstrokecolor{currentstroke}%
\pgfsetdash{}{0pt}%
\pgfpathmoveto{\pgfqpoint{3.905098in}{2.125798in}}%
\pgfpathcurveto{\pgfqpoint{3.913334in}{2.125798in}}{\pgfqpoint{3.921234in}{2.129070in}}{\pgfqpoint{3.927058in}{2.134894in}}%
\pgfpathcurveto{\pgfqpoint{3.932882in}{2.140718in}}{\pgfqpoint{3.936155in}{2.148618in}}{\pgfqpoint{3.936155in}{2.156854in}}%
\pgfpathcurveto{\pgfqpoint{3.936155in}{2.165091in}}{\pgfqpoint{3.932882in}{2.172991in}}{\pgfqpoint{3.927058in}{2.178814in}}%
\pgfpathcurveto{\pgfqpoint{3.921234in}{2.184638in}}{\pgfqpoint{3.913334in}{2.187911in}}{\pgfqpoint{3.905098in}{2.187911in}}%
\pgfpathcurveto{\pgfqpoint{3.896862in}{2.187911in}}{\pgfqpoint{3.888962in}{2.184638in}}{\pgfqpoint{3.883138in}{2.178814in}}%
\pgfpathcurveto{\pgfqpoint{3.877314in}{2.172991in}}{\pgfqpoint{3.874042in}{2.165091in}}{\pgfqpoint{3.874042in}{2.156854in}}%
\pgfpathcurveto{\pgfqpoint{3.874042in}{2.148618in}}{\pgfqpoint{3.877314in}{2.140718in}}{\pgfqpoint{3.883138in}{2.134894in}}%
\pgfpathcurveto{\pgfqpoint{3.888962in}{2.129070in}}{\pgfqpoint{3.896862in}{2.125798in}}{\pgfqpoint{3.905098in}{2.125798in}}%
\pgfpathclose%
\pgfusepath{stroke,fill}%
\end{pgfscope}%
\begin{pgfscope}%
\pgfpathrectangle{\pgfqpoint{3.793912in}{0.557870in}}{\pgfqpoint{2.446088in}{1.684734in}}%
\pgfusepath{clip}%
\pgfsetbuttcap%
\pgfsetroundjoin%
\definecolor{currentfill}{rgb}{0.298039,0.447059,0.690196}%
\pgfsetfillcolor{currentfill}%
\pgfsetlinewidth{1.003750pt}%
\definecolor{currentstroke}{rgb}{0.298039,0.447059,0.690196}%
\pgfsetstrokecolor{currentstroke}%
\pgfsetdash{}{0pt}%
\pgfpathmoveto{\pgfqpoint{5.131976in}{1.447135in}}%
\pgfpathcurveto{\pgfqpoint{5.140212in}{1.447135in}}{\pgfqpoint{5.148112in}{1.450408in}}{\pgfqpoint{5.153936in}{1.456231in}}%
\pgfpathcurveto{\pgfqpoint{5.159760in}{1.462055in}}{\pgfqpoint{5.163032in}{1.469955in}}{\pgfqpoint{5.163032in}{1.478192in}}%
\pgfpathcurveto{\pgfqpoint{5.163032in}{1.486428in}}{\pgfqpoint{5.159760in}{1.494328in}}{\pgfqpoint{5.153936in}{1.500152in}}%
\pgfpathcurveto{\pgfqpoint{5.148112in}{1.505976in}}{\pgfqpoint{5.140212in}{1.509248in}}{\pgfqpoint{5.131976in}{1.509248in}}%
\pgfpathcurveto{\pgfqpoint{5.123740in}{1.509248in}}{\pgfqpoint{5.115840in}{1.505976in}}{\pgfqpoint{5.110016in}{1.500152in}}%
\pgfpathcurveto{\pgfqpoint{5.104192in}{1.494328in}}{\pgfqpoint{5.100919in}{1.486428in}}{\pgfqpoint{5.100919in}{1.478192in}}%
\pgfpathcurveto{\pgfqpoint{5.100919in}{1.469955in}}{\pgfqpoint{5.104192in}{1.462055in}}{\pgfqpoint{5.110016in}{1.456231in}}%
\pgfpathcurveto{\pgfqpoint{5.115840in}{1.450408in}}{\pgfqpoint{5.123740in}{1.447135in}}{\pgfqpoint{5.131976in}{1.447135in}}%
\pgfpathclose%
\pgfusepath{stroke,fill}%
\end{pgfscope}%
\begin{pgfscope}%
\pgfpathrectangle{\pgfqpoint{3.793912in}{0.557870in}}{\pgfqpoint{2.446088in}{1.684734in}}%
\pgfusepath{clip}%
\pgfsetbuttcap%
\pgfsetroundjoin%
\definecolor{currentfill}{rgb}{0.298039,0.447059,0.690196}%
\pgfsetfillcolor{currentfill}%
\pgfsetlinewidth{1.003750pt}%
\definecolor{currentstroke}{rgb}{0.298039,0.447059,0.690196}%
\pgfsetstrokecolor{currentstroke}%
\pgfsetdash{}{0pt}%
\pgfpathmoveto{\pgfqpoint{3.905098in}{2.125798in}}%
\pgfpathcurveto{\pgfqpoint{3.913334in}{2.125798in}}{\pgfqpoint{3.921234in}{2.129070in}}{\pgfqpoint{3.927058in}{2.134894in}}%
\pgfpathcurveto{\pgfqpoint{3.932882in}{2.140718in}}{\pgfqpoint{3.936155in}{2.148618in}}{\pgfqpoint{3.936155in}{2.156854in}}%
\pgfpathcurveto{\pgfqpoint{3.936155in}{2.165091in}}{\pgfqpoint{3.932882in}{2.172991in}}{\pgfqpoint{3.927058in}{2.178814in}}%
\pgfpathcurveto{\pgfqpoint{3.921234in}{2.184638in}}{\pgfqpoint{3.913334in}{2.187911in}}{\pgfqpoint{3.905098in}{2.187911in}}%
\pgfpathcurveto{\pgfqpoint{3.896862in}{2.187911in}}{\pgfqpoint{3.888962in}{2.184638in}}{\pgfqpoint{3.883138in}{2.178814in}}%
\pgfpathcurveto{\pgfqpoint{3.877314in}{2.172991in}}{\pgfqpoint{3.874042in}{2.165091in}}{\pgfqpoint{3.874042in}{2.156854in}}%
\pgfpathcurveto{\pgfqpoint{3.874042in}{2.148618in}}{\pgfqpoint{3.877314in}{2.140718in}}{\pgfqpoint{3.883138in}{2.134894in}}%
\pgfpathcurveto{\pgfqpoint{3.888962in}{2.129070in}}{\pgfqpoint{3.896862in}{2.125798in}}{\pgfqpoint{3.905098in}{2.125798in}}%
\pgfpathclose%
\pgfusepath{stroke,fill}%
\end{pgfscope}%
\begin{pgfscope}%
\pgfpathrectangle{\pgfqpoint{3.793912in}{0.557870in}}{\pgfqpoint{2.446088in}{1.684734in}}%
\pgfusepath{clip}%
\pgfsetbuttcap%
\pgfsetroundjoin%
\definecolor{currentfill}{rgb}{0.298039,0.447059,0.690196}%
\pgfsetfillcolor{currentfill}%
\pgfsetlinewidth{1.003750pt}%
\definecolor{currentstroke}{rgb}{0.298039,0.447059,0.690196}%
\pgfsetstrokecolor{currentstroke}%
\pgfsetdash{}{0pt}%
\pgfpathmoveto{\pgfqpoint{4.058458in}{2.125798in}}%
\pgfpathcurveto{\pgfqpoint{4.066694in}{2.125798in}}{\pgfqpoint{4.074594in}{2.129070in}}{\pgfqpoint{4.080418in}{2.134894in}}%
\pgfpathcurveto{\pgfqpoint{4.086242in}{2.140718in}}{\pgfqpoint{4.089514in}{2.148618in}}{\pgfqpoint{4.089514in}{2.156854in}}%
\pgfpathcurveto{\pgfqpoint{4.089514in}{2.165091in}}{\pgfqpoint{4.086242in}{2.172991in}}{\pgfqpoint{4.080418in}{2.178814in}}%
\pgfpathcurveto{\pgfqpoint{4.074594in}{2.184638in}}{\pgfqpoint{4.066694in}{2.187911in}}{\pgfqpoint{4.058458in}{2.187911in}}%
\pgfpathcurveto{\pgfqpoint{4.050221in}{2.187911in}}{\pgfqpoint{4.042321in}{2.184638in}}{\pgfqpoint{4.036498in}{2.178814in}}%
\pgfpathcurveto{\pgfqpoint{4.030674in}{2.172991in}}{\pgfqpoint{4.027401in}{2.165091in}}{\pgfqpoint{4.027401in}{2.156854in}}%
\pgfpathcurveto{\pgfqpoint{4.027401in}{2.148618in}}{\pgfqpoint{4.030674in}{2.140718in}}{\pgfqpoint{4.036498in}{2.134894in}}%
\pgfpathcurveto{\pgfqpoint{4.042321in}{2.129070in}}{\pgfqpoint{4.050221in}{2.125798in}}{\pgfqpoint{4.058458in}{2.125798in}}%
\pgfpathclose%
\pgfusepath{stroke,fill}%
\end{pgfscope}%
\begin{pgfscope}%
\pgfpathrectangle{\pgfqpoint{3.793912in}{0.557870in}}{\pgfqpoint{2.446088in}{1.684734in}}%
\pgfusepath{clip}%
\pgfsetbuttcap%
\pgfsetroundjoin%
\definecolor{currentfill}{rgb}{0.298039,0.447059,0.690196}%
\pgfsetfillcolor{currentfill}%
\pgfsetlinewidth{1.003750pt}%
\definecolor{currentstroke}{rgb}{0.298039,0.447059,0.690196}%
\pgfsetstrokecolor{currentstroke}%
\pgfsetdash{}{0pt}%
\pgfpathmoveto{\pgfqpoint{3.905098in}{2.125798in}}%
\pgfpathcurveto{\pgfqpoint{3.913334in}{2.125798in}}{\pgfqpoint{3.921234in}{2.129070in}}{\pgfqpoint{3.927058in}{2.134894in}}%
\pgfpathcurveto{\pgfqpoint{3.932882in}{2.140718in}}{\pgfqpoint{3.936155in}{2.148618in}}{\pgfqpoint{3.936155in}{2.156854in}}%
\pgfpathcurveto{\pgfqpoint{3.936155in}{2.165091in}}{\pgfqpoint{3.932882in}{2.172991in}}{\pgfqpoint{3.927058in}{2.178814in}}%
\pgfpathcurveto{\pgfqpoint{3.921234in}{2.184638in}}{\pgfqpoint{3.913334in}{2.187911in}}{\pgfqpoint{3.905098in}{2.187911in}}%
\pgfpathcurveto{\pgfqpoint{3.896862in}{2.187911in}}{\pgfqpoint{3.888962in}{2.184638in}}{\pgfqpoint{3.883138in}{2.178814in}}%
\pgfpathcurveto{\pgfqpoint{3.877314in}{2.172991in}}{\pgfqpoint{3.874042in}{2.165091in}}{\pgfqpoint{3.874042in}{2.156854in}}%
\pgfpathcurveto{\pgfqpoint{3.874042in}{2.148618in}}{\pgfqpoint{3.877314in}{2.140718in}}{\pgfqpoint{3.883138in}{2.134894in}}%
\pgfpathcurveto{\pgfqpoint{3.888962in}{2.129070in}}{\pgfqpoint{3.896862in}{2.125798in}}{\pgfqpoint{3.905098in}{2.125798in}}%
\pgfpathclose%
\pgfusepath{stroke,fill}%
\end{pgfscope}%
\begin{pgfscope}%
\pgfpathrectangle{\pgfqpoint{3.793912in}{0.557870in}}{\pgfqpoint{2.446088in}{1.684734in}}%
\pgfusepath{clip}%
\pgfsetbuttcap%
\pgfsetroundjoin%
\definecolor{currentfill}{rgb}{0.298039,0.447059,0.690196}%
\pgfsetfillcolor{currentfill}%
\pgfsetlinewidth{1.003750pt}%
\definecolor{currentstroke}{rgb}{0.298039,0.447059,0.690196}%
\pgfsetstrokecolor{currentstroke}%
\pgfsetdash{}{0pt}%
\pgfpathmoveto{\pgfqpoint{5.975454in}{1.373766in}}%
\pgfpathcurveto{\pgfqpoint{5.983691in}{1.373766in}}{\pgfqpoint{5.991591in}{1.377039in}}{\pgfqpoint{5.997415in}{1.382863in}}%
\pgfpathcurveto{\pgfqpoint{6.003239in}{1.388686in}}{\pgfqpoint{6.006511in}{1.396586in}}{\pgfqpoint{6.006511in}{1.404823in}}%
\pgfpathcurveto{\pgfqpoint{6.006511in}{1.413059in}}{\pgfqpoint{6.003239in}{1.420959in}}{\pgfqpoint{5.997415in}{1.426783in}}%
\pgfpathcurveto{\pgfqpoint{5.991591in}{1.432607in}}{\pgfqpoint{5.983691in}{1.435879in}}{\pgfqpoint{5.975454in}{1.435879in}}%
\pgfpathcurveto{\pgfqpoint{5.967218in}{1.435879in}}{\pgfqpoint{5.959318in}{1.432607in}}{\pgfqpoint{5.953494in}{1.426783in}}%
\pgfpathcurveto{\pgfqpoint{5.947670in}{1.420959in}}{\pgfqpoint{5.944398in}{1.413059in}}{\pgfqpoint{5.944398in}{1.404823in}}%
\pgfpathcurveto{\pgfqpoint{5.944398in}{1.396586in}}{\pgfqpoint{5.947670in}{1.388686in}}{\pgfqpoint{5.953494in}{1.382863in}}%
\pgfpathcurveto{\pgfqpoint{5.959318in}{1.377039in}}{\pgfqpoint{5.967218in}{1.373766in}}{\pgfqpoint{5.975454in}{1.373766in}}%
\pgfpathclose%
\pgfusepath{stroke,fill}%
\end{pgfscope}%
\begin{pgfscope}%
\pgfpathrectangle{\pgfqpoint{3.793912in}{0.557870in}}{\pgfqpoint{2.446088in}{1.684734in}}%
\pgfusepath{clip}%
\pgfsetbuttcap%
\pgfsetroundjoin%
\definecolor{currentfill}{rgb}{0.298039,0.447059,0.690196}%
\pgfsetfillcolor{currentfill}%
\pgfsetlinewidth{1.003750pt}%
\definecolor{currentstroke}{rgb}{0.298039,0.447059,0.690196}%
\pgfsetstrokecolor{currentstroke}%
\pgfsetdash{}{0pt}%
\pgfpathmoveto{\pgfqpoint{3.905098in}{2.125798in}}%
\pgfpathcurveto{\pgfqpoint{3.913334in}{2.125798in}}{\pgfqpoint{3.921234in}{2.129070in}}{\pgfqpoint{3.927058in}{2.134894in}}%
\pgfpathcurveto{\pgfqpoint{3.932882in}{2.140718in}}{\pgfqpoint{3.936155in}{2.148618in}}{\pgfqpoint{3.936155in}{2.156854in}}%
\pgfpathcurveto{\pgfqpoint{3.936155in}{2.165091in}}{\pgfqpoint{3.932882in}{2.172991in}}{\pgfqpoint{3.927058in}{2.178814in}}%
\pgfpathcurveto{\pgfqpoint{3.921234in}{2.184638in}}{\pgfqpoint{3.913334in}{2.187911in}}{\pgfqpoint{3.905098in}{2.187911in}}%
\pgfpathcurveto{\pgfqpoint{3.896862in}{2.187911in}}{\pgfqpoint{3.888962in}{2.184638in}}{\pgfqpoint{3.883138in}{2.178814in}}%
\pgfpathcurveto{\pgfqpoint{3.877314in}{2.172991in}}{\pgfqpoint{3.874042in}{2.165091in}}{\pgfqpoint{3.874042in}{2.156854in}}%
\pgfpathcurveto{\pgfqpoint{3.874042in}{2.148618in}}{\pgfqpoint{3.877314in}{2.140718in}}{\pgfqpoint{3.883138in}{2.134894in}}%
\pgfpathcurveto{\pgfqpoint{3.888962in}{2.129070in}}{\pgfqpoint{3.896862in}{2.125798in}}{\pgfqpoint{3.905098in}{2.125798in}}%
\pgfpathclose%
\pgfusepath{stroke,fill}%
\end{pgfscope}%
\begin{pgfscope}%
\pgfpathrectangle{\pgfqpoint{3.793912in}{0.557870in}}{\pgfqpoint{2.446088in}{1.684734in}}%
\pgfusepath{clip}%
\pgfsetbuttcap%
\pgfsetroundjoin%
\definecolor{currentfill}{rgb}{0.298039,0.447059,0.690196}%
\pgfsetfillcolor{currentfill}%
\pgfsetlinewidth{1.003750pt}%
\definecolor{currentstroke}{rgb}{0.298039,0.447059,0.690196}%
\pgfsetstrokecolor{currentstroke}%
\pgfsetdash{}{0pt}%
\pgfpathmoveto{\pgfqpoint{5.975454in}{1.575531in}}%
\pgfpathcurveto{\pgfqpoint{5.983691in}{1.575531in}}{\pgfqpoint{5.991591in}{1.578803in}}{\pgfqpoint{5.997415in}{1.584627in}}%
\pgfpathcurveto{\pgfqpoint{6.003239in}{1.590451in}}{\pgfqpoint{6.006511in}{1.598351in}}{\pgfqpoint{6.006511in}{1.606587in}}%
\pgfpathcurveto{\pgfqpoint{6.006511in}{1.614824in}}{\pgfqpoint{6.003239in}{1.622724in}}{\pgfqpoint{5.997415in}{1.628548in}}%
\pgfpathcurveto{\pgfqpoint{5.991591in}{1.634372in}}{\pgfqpoint{5.983691in}{1.637644in}}{\pgfqpoint{5.975454in}{1.637644in}}%
\pgfpathcurveto{\pgfqpoint{5.967218in}{1.637644in}}{\pgfqpoint{5.959318in}{1.634372in}}{\pgfqpoint{5.953494in}{1.628548in}}%
\pgfpathcurveto{\pgfqpoint{5.947670in}{1.622724in}}{\pgfqpoint{5.944398in}{1.614824in}}{\pgfqpoint{5.944398in}{1.606587in}}%
\pgfpathcurveto{\pgfqpoint{5.944398in}{1.598351in}}{\pgfqpoint{5.947670in}{1.590451in}}{\pgfqpoint{5.953494in}{1.584627in}}%
\pgfpathcurveto{\pgfqpoint{5.959318in}{1.578803in}}{\pgfqpoint{5.967218in}{1.575531in}}{\pgfqpoint{5.975454in}{1.575531in}}%
\pgfpathclose%
\pgfusepath{stroke,fill}%
\end{pgfscope}%
\begin{pgfscope}%
\pgfpathrectangle{\pgfqpoint{3.793912in}{0.557870in}}{\pgfqpoint{2.446088in}{1.684734in}}%
\pgfusepath{clip}%
\pgfsetbuttcap%
\pgfsetroundjoin%
\definecolor{currentfill}{rgb}{0.298039,0.447059,0.690196}%
\pgfsetfillcolor{currentfill}%
\pgfsetlinewidth{1.003750pt}%
\definecolor{currentstroke}{rgb}{0.298039,0.447059,0.690196}%
\pgfsetstrokecolor{currentstroke}%
\pgfsetdash{}{0pt}%
\pgfpathmoveto{\pgfqpoint{3.905098in}{2.125798in}}%
\pgfpathcurveto{\pgfqpoint{3.913334in}{2.125798in}}{\pgfqpoint{3.921234in}{2.129070in}}{\pgfqpoint{3.927058in}{2.134894in}}%
\pgfpathcurveto{\pgfqpoint{3.932882in}{2.140718in}}{\pgfqpoint{3.936155in}{2.148618in}}{\pgfqpoint{3.936155in}{2.156854in}}%
\pgfpathcurveto{\pgfqpoint{3.936155in}{2.165091in}}{\pgfqpoint{3.932882in}{2.172991in}}{\pgfqpoint{3.927058in}{2.178814in}}%
\pgfpathcurveto{\pgfqpoint{3.921234in}{2.184638in}}{\pgfqpoint{3.913334in}{2.187911in}}{\pgfqpoint{3.905098in}{2.187911in}}%
\pgfpathcurveto{\pgfqpoint{3.896862in}{2.187911in}}{\pgfqpoint{3.888962in}{2.184638in}}{\pgfqpoint{3.883138in}{2.178814in}}%
\pgfpathcurveto{\pgfqpoint{3.877314in}{2.172991in}}{\pgfqpoint{3.874042in}{2.165091in}}{\pgfqpoint{3.874042in}{2.156854in}}%
\pgfpathcurveto{\pgfqpoint{3.874042in}{2.148618in}}{\pgfqpoint{3.877314in}{2.140718in}}{\pgfqpoint{3.883138in}{2.134894in}}%
\pgfpathcurveto{\pgfqpoint{3.888962in}{2.129070in}}{\pgfqpoint{3.896862in}{2.125798in}}{\pgfqpoint{3.905098in}{2.125798in}}%
\pgfpathclose%
\pgfusepath{stroke,fill}%
\end{pgfscope}%
\begin{pgfscope}%
\pgfpathrectangle{\pgfqpoint{3.793912in}{0.557870in}}{\pgfqpoint{2.446088in}{1.684734in}}%
\pgfusepath{clip}%
\pgfsetbuttcap%
\pgfsetroundjoin%
\definecolor{currentfill}{rgb}{0.298039,0.447059,0.690196}%
\pgfsetfillcolor{currentfill}%
\pgfsetlinewidth{1.003750pt}%
\definecolor{currentstroke}{rgb}{0.298039,0.447059,0.690196}%
\pgfsetstrokecolor{currentstroke}%
\pgfsetdash{}{0pt}%
\pgfpathmoveto{\pgfqpoint{3.905098in}{2.125798in}}%
\pgfpathcurveto{\pgfqpoint{3.913334in}{2.125798in}}{\pgfqpoint{3.921234in}{2.129070in}}{\pgfqpoint{3.927058in}{2.134894in}}%
\pgfpathcurveto{\pgfqpoint{3.932882in}{2.140718in}}{\pgfqpoint{3.936155in}{2.148618in}}{\pgfqpoint{3.936155in}{2.156854in}}%
\pgfpathcurveto{\pgfqpoint{3.936155in}{2.165091in}}{\pgfqpoint{3.932882in}{2.172991in}}{\pgfqpoint{3.927058in}{2.178814in}}%
\pgfpathcurveto{\pgfqpoint{3.921234in}{2.184638in}}{\pgfqpoint{3.913334in}{2.187911in}}{\pgfqpoint{3.905098in}{2.187911in}}%
\pgfpathcurveto{\pgfqpoint{3.896862in}{2.187911in}}{\pgfqpoint{3.888962in}{2.184638in}}{\pgfqpoint{3.883138in}{2.178814in}}%
\pgfpathcurveto{\pgfqpoint{3.877314in}{2.172991in}}{\pgfqpoint{3.874042in}{2.165091in}}{\pgfqpoint{3.874042in}{2.156854in}}%
\pgfpathcurveto{\pgfqpoint{3.874042in}{2.148618in}}{\pgfqpoint{3.877314in}{2.140718in}}{\pgfqpoint{3.883138in}{2.134894in}}%
\pgfpathcurveto{\pgfqpoint{3.888962in}{2.129070in}}{\pgfqpoint{3.896862in}{2.125798in}}{\pgfqpoint{3.905098in}{2.125798in}}%
\pgfpathclose%
\pgfusepath{stroke,fill}%
\end{pgfscope}%
\begin{pgfscope}%
\pgfpathrectangle{\pgfqpoint{3.793912in}{0.557870in}}{\pgfqpoint{2.446088in}{1.684734in}}%
\pgfusepath{clip}%
\pgfsetbuttcap%
\pgfsetroundjoin%
\definecolor{currentfill}{rgb}{0.298039,0.447059,0.690196}%
\pgfsetfillcolor{currentfill}%
\pgfsetlinewidth{1.003750pt}%
\definecolor{currentstroke}{rgb}{0.298039,0.447059,0.690196}%
\pgfsetstrokecolor{currentstroke}%
\pgfsetdash{}{0pt}%
\pgfpathmoveto{\pgfqpoint{3.905098in}{2.125798in}}%
\pgfpathcurveto{\pgfqpoint{3.913334in}{2.125798in}}{\pgfqpoint{3.921234in}{2.129070in}}{\pgfqpoint{3.927058in}{2.134894in}}%
\pgfpathcurveto{\pgfqpoint{3.932882in}{2.140718in}}{\pgfqpoint{3.936155in}{2.148618in}}{\pgfqpoint{3.936155in}{2.156854in}}%
\pgfpathcurveto{\pgfqpoint{3.936155in}{2.165091in}}{\pgfqpoint{3.932882in}{2.172991in}}{\pgfqpoint{3.927058in}{2.178814in}}%
\pgfpathcurveto{\pgfqpoint{3.921234in}{2.184638in}}{\pgfqpoint{3.913334in}{2.187911in}}{\pgfqpoint{3.905098in}{2.187911in}}%
\pgfpathcurveto{\pgfqpoint{3.896862in}{2.187911in}}{\pgfqpoint{3.888962in}{2.184638in}}{\pgfqpoint{3.883138in}{2.178814in}}%
\pgfpathcurveto{\pgfqpoint{3.877314in}{2.172991in}}{\pgfqpoint{3.874042in}{2.165091in}}{\pgfqpoint{3.874042in}{2.156854in}}%
\pgfpathcurveto{\pgfqpoint{3.874042in}{2.148618in}}{\pgfqpoint{3.877314in}{2.140718in}}{\pgfqpoint{3.883138in}{2.134894in}}%
\pgfpathcurveto{\pgfqpoint{3.888962in}{2.129070in}}{\pgfqpoint{3.896862in}{2.125798in}}{\pgfqpoint{3.905098in}{2.125798in}}%
\pgfpathclose%
\pgfusepath{stroke,fill}%
\end{pgfscope}%
\begin{pgfscope}%
\pgfpathrectangle{\pgfqpoint{3.793912in}{0.557870in}}{\pgfqpoint{2.446088in}{1.684734in}}%
\pgfusepath{clip}%
\pgfsetbuttcap%
\pgfsetroundjoin%
\definecolor{currentfill}{rgb}{0.298039,0.447059,0.690196}%
\pgfsetfillcolor{currentfill}%
\pgfsetlinewidth{1.003750pt}%
\definecolor{currentstroke}{rgb}{0.298039,0.447059,0.690196}%
\pgfsetstrokecolor{currentstroke}%
\pgfsetdash{}{0pt}%
\pgfpathmoveto{\pgfqpoint{3.905098in}{2.125798in}}%
\pgfpathcurveto{\pgfqpoint{3.913334in}{2.125798in}}{\pgfqpoint{3.921234in}{2.129070in}}{\pgfqpoint{3.927058in}{2.134894in}}%
\pgfpathcurveto{\pgfqpoint{3.932882in}{2.140718in}}{\pgfqpoint{3.936155in}{2.148618in}}{\pgfqpoint{3.936155in}{2.156854in}}%
\pgfpathcurveto{\pgfqpoint{3.936155in}{2.165091in}}{\pgfqpoint{3.932882in}{2.172991in}}{\pgfqpoint{3.927058in}{2.178814in}}%
\pgfpathcurveto{\pgfqpoint{3.921234in}{2.184638in}}{\pgfqpoint{3.913334in}{2.187911in}}{\pgfqpoint{3.905098in}{2.187911in}}%
\pgfpathcurveto{\pgfqpoint{3.896862in}{2.187911in}}{\pgfqpoint{3.888962in}{2.184638in}}{\pgfqpoint{3.883138in}{2.178814in}}%
\pgfpathcurveto{\pgfqpoint{3.877314in}{2.172991in}}{\pgfqpoint{3.874042in}{2.165091in}}{\pgfqpoint{3.874042in}{2.156854in}}%
\pgfpathcurveto{\pgfqpoint{3.874042in}{2.148618in}}{\pgfqpoint{3.877314in}{2.140718in}}{\pgfqpoint{3.883138in}{2.134894in}}%
\pgfpathcurveto{\pgfqpoint{3.888962in}{2.129070in}}{\pgfqpoint{3.896862in}{2.125798in}}{\pgfqpoint{3.905098in}{2.125798in}}%
\pgfpathclose%
\pgfusepath{stroke,fill}%
\end{pgfscope}%
\begin{pgfscope}%
\pgfpathrectangle{\pgfqpoint{3.793912in}{0.557870in}}{\pgfqpoint{2.446088in}{1.684734in}}%
\pgfusepath{clip}%
\pgfsetbuttcap%
\pgfsetroundjoin%
\definecolor{currentfill}{rgb}{0.298039,0.447059,0.690196}%
\pgfsetfillcolor{currentfill}%
\pgfsetlinewidth{1.003750pt}%
\definecolor{currentstroke}{rgb}{0.298039,0.447059,0.690196}%
\pgfsetstrokecolor{currentstroke}%
\pgfsetdash{}{0pt}%
\pgfpathmoveto{\pgfqpoint{5.975454in}{1.272884in}}%
\pgfpathcurveto{\pgfqpoint{5.983691in}{1.272884in}}{\pgfqpoint{5.991591in}{1.276156in}}{\pgfqpoint{5.997415in}{1.281980in}}%
\pgfpathcurveto{\pgfqpoint{6.003239in}{1.287804in}}{\pgfqpoint{6.006511in}{1.295704in}}{\pgfqpoint{6.006511in}{1.303941in}}%
\pgfpathcurveto{\pgfqpoint{6.006511in}{1.312177in}}{\pgfqpoint{6.003239in}{1.320077in}}{\pgfqpoint{5.997415in}{1.325901in}}%
\pgfpathcurveto{\pgfqpoint{5.991591in}{1.331725in}}{\pgfqpoint{5.983691in}{1.334997in}}{\pgfqpoint{5.975454in}{1.334997in}}%
\pgfpathcurveto{\pgfqpoint{5.967218in}{1.334997in}}{\pgfqpoint{5.959318in}{1.331725in}}{\pgfqpoint{5.953494in}{1.325901in}}%
\pgfpathcurveto{\pgfqpoint{5.947670in}{1.320077in}}{\pgfqpoint{5.944398in}{1.312177in}}{\pgfqpoint{5.944398in}{1.303941in}}%
\pgfpathcurveto{\pgfqpoint{5.944398in}{1.295704in}}{\pgfqpoint{5.947670in}{1.287804in}}{\pgfqpoint{5.953494in}{1.281980in}}%
\pgfpathcurveto{\pgfqpoint{5.959318in}{1.276156in}}{\pgfqpoint{5.967218in}{1.272884in}}{\pgfqpoint{5.975454in}{1.272884in}}%
\pgfpathclose%
\pgfusepath{stroke,fill}%
\end{pgfscope}%
\begin{pgfscope}%
\pgfpathrectangle{\pgfqpoint{3.793912in}{0.557870in}}{\pgfqpoint{2.446088in}{1.684734in}}%
\pgfusepath{clip}%
\pgfsetbuttcap%
\pgfsetroundjoin%
\definecolor{currentfill}{rgb}{0.298039,0.447059,0.690196}%
\pgfsetfillcolor{currentfill}%
\pgfsetlinewidth{1.003750pt}%
\definecolor{currentstroke}{rgb}{0.298039,0.447059,0.690196}%
\pgfsetstrokecolor{currentstroke}%
\pgfsetdash{}{0pt}%
\pgfpathmoveto{\pgfqpoint{3.905098in}{2.125798in}}%
\pgfpathcurveto{\pgfqpoint{3.913334in}{2.125798in}}{\pgfqpoint{3.921234in}{2.129070in}}{\pgfqpoint{3.927058in}{2.134894in}}%
\pgfpathcurveto{\pgfqpoint{3.932882in}{2.140718in}}{\pgfqpoint{3.936155in}{2.148618in}}{\pgfqpoint{3.936155in}{2.156854in}}%
\pgfpathcurveto{\pgfqpoint{3.936155in}{2.165091in}}{\pgfqpoint{3.932882in}{2.172991in}}{\pgfqpoint{3.927058in}{2.178814in}}%
\pgfpathcurveto{\pgfqpoint{3.921234in}{2.184638in}}{\pgfqpoint{3.913334in}{2.187911in}}{\pgfqpoint{3.905098in}{2.187911in}}%
\pgfpathcurveto{\pgfqpoint{3.896862in}{2.187911in}}{\pgfqpoint{3.888962in}{2.184638in}}{\pgfqpoint{3.883138in}{2.178814in}}%
\pgfpathcurveto{\pgfqpoint{3.877314in}{2.172991in}}{\pgfqpoint{3.874042in}{2.165091in}}{\pgfqpoint{3.874042in}{2.156854in}}%
\pgfpathcurveto{\pgfqpoint{3.874042in}{2.148618in}}{\pgfqpoint{3.877314in}{2.140718in}}{\pgfqpoint{3.883138in}{2.134894in}}%
\pgfpathcurveto{\pgfqpoint{3.888962in}{2.129070in}}{\pgfqpoint{3.896862in}{2.125798in}}{\pgfqpoint{3.905098in}{2.125798in}}%
\pgfpathclose%
\pgfusepath{stroke,fill}%
\end{pgfscope}%
\begin{pgfscope}%
\pgfpathrectangle{\pgfqpoint{3.793912in}{0.557870in}}{\pgfqpoint{2.446088in}{1.684734in}}%
\pgfusepath{clip}%
\pgfsetbuttcap%
\pgfsetroundjoin%
\definecolor{currentfill}{rgb}{0.298039,0.447059,0.690196}%
\pgfsetfillcolor{currentfill}%
\pgfsetlinewidth{1.003750pt}%
\definecolor{currentstroke}{rgb}{0.298039,0.447059,0.690196}%
\pgfsetstrokecolor{currentstroke}%
\pgfsetdash{}{0pt}%
\pgfpathmoveto{\pgfqpoint{5.975454in}{1.318740in}}%
\pgfpathcurveto{\pgfqpoint{5.983691in}{1.318740in}}{\pgfqpoint{5.991591in}{1.322012in}}{\pgfqpoint{5.997415in}{1.327836in}}%
\pgfpathcurveto{\pgfqpoint{6.003239in}{1.333660in}}{\pgfqpoint{6.006511in}{1.341560in}}{\pgfqpoint{6.006511in}{1.349796in}}%
\pgfpathcurveto{\pgfqpoint{6.006511in}{1.358032in}}{\pgfqpoint{6.003239in}{1.365932in}}{\pgfqpoint{5.997415in}{1.371756in}}%
\pgfpathcurveto{\pgfqpoint{5.991591in}{1.377580in}}{\pgfqpoint{5.983691in}{1.380853in}}{\pgfqpoint{5.975454in}{1.380853in}}%
\pgfpathcurveto{\pgfqpoint{5.967218in}{1.380853in}}{\pgfqpoint{5.959318in}{1.377580in}}{\pgfqpoint{5.953494in}{1.371756in}}%
\pgfpathcurveto{\pgfqpoint{5.947670in}{1.365932in}}{\pgfqpoint{5.944398in}{1.358032in}}{\pgfqpoint{5.944398in}{1.349796in}}%
\pgfpathcurveto{\pgfqpoint{5.944398in}{1.341560in}}{\pgfqpoint{5.947670in}{1.333660in}}{\pgfqpoint{5.953494in}{1.327836in}}%
\pgfpathcurveto{\pgfqpoint{5.959318in}{1.322012in}}{\pgfqpoint{5.967218in}{1.318740in}}{\pgfqpoint{5.975454in}{1.318740in}}%
\pgfpathclose%
\pgfusepath{stroke,fill}%
\end{pgfscope}%
\begin{pgfscope}%
\pgfpathrectangle{\pgfqpoint{3.793912in}{0.557870in}}{\pgfqpoint{2.446088in}{1.684734in}}%
\pgfusepath{clip}%
\pgfsetbuttcap%
\pgfsetroundjoin%
\definecolor{currentfill}{rgb}{0.298039,0.447059,0.690196}%
\pgfsetfillcolor{currentfill}%
\pgfsetlinewidth{1.003750pt}%
\definecolor{currentstroke}{rgb}{0.298039,0.447059,0.690196}%
\pgfsetstrokecolor{currentstroke}%
\pgfsetdash{}{0pt}%
\pgfpathmoveto{\pgfqpoint{3.905098in}{2.125798in}}%
\pgfpathcurveto{\pgfqpoint{3.913334in}{2.125798in}}{\pgfqpoint{3.921234in}{2.129070in}}{\pgfqpoint{3.927058in}{2.134894in}}%
\pgfpathcurveto{\pgfqpoint{3.932882in}{2.140718in}}{\pgfqpoint{3.936155in}{2.148618in}}{\pgfqpoint{3.936155in}{2.156854in}}%
\pgfpathcurveto{\pgfqpoint{3.936155in}{2.165091in}}{\pgfqpoint{3.932882in}{2.172991in}}{\pgfqpoint{3.927058in}{2.178814in}}%
\pgfpathcurveto{\pgfqpoint{3.921234in}{2.184638in}}{\pgfqpoint{3.913334in}{2.187911in}}{\pgfqpoint{3.905098in}{2.187911in}}%
\pgfpathcurveto{\pgfqpoint{3.896862in}{2.187911in}}{\pgfqpoint{3.888962in}{2.184638in}}{\pgfqpoint{3.883138in}{2.178814in}}%
\pgfpathcurveto{\pgfqpoint{3.877314in}{2.172991in}}{\pgfqpoint{3.874042in}{2.165091in}}{\pgfqpoint{3.874042in}{2.156854in}}%
\pgfpathcurveto{\pgfqpoint{3.874042in}{2.148618in}}{\pgfqpoint{3.877314in}{2.140718in}}{\pgfqpoint{3.883138in}{2.134894in}}%
\pgfpathcurveto{\pgfqpoint{3.888962in}{2.129070in}}{\pgfqpoint{3.896862in}{2.125798in}}{\pgfqpoint{3.905098in}{2.125798in}}%
\pgfpathclose%
\pgfusepath{stroke,fill}%
\end{pgfscope}%
\begin{pgfscope}%
\pgfpathrectangle{\pgfqpoint{3.793912in}{0.557870in}}{\pgfqpoint{2.446088in}{1.684734in}}%
\pgfusepath{clip}%
\pgfsetbuttcap%
\pgfsetroundjoin%
\definecolor{currentfill}{rgb}{0.298039,0.447059,0.690196}%
\pgfsetfillcolor{currentfill}%
\pgfsetlinewidth{1.003750pt}%
\definecolor{currentstroke}{rgb}{0.298039,0.447059,0.690196}%
\pgfsetstrokecolor{currentstroke}%
\pgfsetdash{}{0pt}%
\pgfpathmoveto{\pgfqpoint{4.058458in}{1.593873in}}%
\pgfpathcurveto{\pgfqpoint{4.066694in}{1.593873in}}{\pgfqpoint{4.074594in}{1.597145in}}{\pgfqpoint{4.080418in}{1.602969in}}%
\pgfpathcurveto{\pgfqpoint{4.086242in}{1.608793in}}{\pgfqpoint{4.089514in}{1.616693in}}{\pgfqpoint{4.089514in}{1.624930in}}%
\pgfpathcurveto{\pgfqpoint{4.089514in}{1.633166in}}{\pgfqpoint{4.086242in}{1.641066in}}{\pgfqpoint{4.080418in}{1.646890in}}%
\pgfpathcurveto{\pgfqpoint{4.074594in}{1.652714in}}{\pgfqpoint{4.066694in}{1.655986in}}{\pgfqpoint{4.058458in}{1.655986in}}%
\pgfpathcurveto{\pgfqpoint{4.050221in}{1.655986in}}{\pgfqpoint{4.042321in}{1.652714in}}{\pgfqpoint{4.036498in}{1.646890in}}%
\pgfpathcurveto{\pgfqpoint{4.030674in}{1.641066in}}{\pgfqpoint{4.027401in}{1.633166in}}{\pgfqpoint{4.027401in}{1.624930in}}%
\pgfpathcurveto{\pgfqpoint{4.027401in}{1.616693in}}{\pgfqpoint{4.030674in}{1.608793in}}{\pgfqpoint{4.036498in}{1.602969in}}%
\pgfpathcurveto{\pgfqpoint{4.042321in}{1.597145in}}{\pgfqpoint{4.050221in}{1.593873in}}{\pgfqpoint{4.058458in}{1.593873in}}%
\pgfpathclose%
\pgfusepath{stroke,fill}%
\end{pgfscope}%
\begin{pgfscope}%
\pgfpathrectangle{\pgfqpoint{3.793912in}{0.557870in}}{\pgfqpoint{2.446088in}{1.684734in}}%
\pgfusepath{clip}%
\pgfsetbuttcap%
\pgfsetroundjoin%
\definecolor{currentfill}{rgb}{0.298039,0.447059,0.690196}%
\pgfsetfillcolor{currentfill}%
\pgfsetlinewidth{1.003750pt}%
\definecolor{currentstroke}{rgb}{0.298039,0.447059,0.690196}%
\pgfsetstrokecolor{currentstroke}%
\pgfsetdash{}{0pt}%
\pgfpathmoveto{\pgfqpoint{3.905098in}{2.125798in}}%
\pgfpathcurveto{\pgfqpoint{3.913334in}{2.125798in}}{\pgfqpoint{3.921234in}{2.129070in}}{\pgfqpoint{3.927058in}{2.134894in}}%
\pgfpathcurveto{\pgfqpoint{3.932882in}{2.140718in}}{\pgfqpoint{3.936155in}{2.148618in}}{\pgfqpoint{3.936155in}{2.156854in}}%
\pgfpathcurveto{\pgfqpoint{3.936155in}{2.165091in}}{\pgfqpoint{3.932882in}{2.172991in}}{\pgfqpoint{3.927058in}{2.178814in}}%
\pgfpathcurveto{\pgfqpoint{3.921234in}{2.184638in}}{\pgfqpoint{3.913334in}{2.187911in}}{\pgfqpoint{3.905098in}{2.187911in}}%
\pgfpathcurveto{\pgfqpoint{3.896862in}{2.187911in}}{\pgfqpoint{3.888962in}{2.184638in}}{\pgfqpoint{3.883138in}{2.178814in}}%
\pgfpathcurveto{\pgfqpoint{3.877314in}{2.172991in}}{\pgfqpoint{3.874042in}{2.165091in}}{\pgfqpoint{3.874042in}{2.156854in}}%
\pgfpathcurveto{\pgfqpoint{3.874042in}{2.148618in}}{\pgfqpoint{3.877314in}{2.140718in}}{\pgfqpoint{3.883138in}{2.134894in}}%
\pgfpathcurveto{\pgfqpoint{3.888962in}{2.129070in}}{\pgfqpoint{3.896862in}{2.125798in}}{\pgfqpoint{3.905098in}{2.125798in}}%
\pgfpathclose%
\pgfusepath{stroke,fill}%
\end{pgfscope}%
\begin{pgfscope}%
\pgfpathrectangle{\pgfqpoint{3.793912in}{0.557870in}}{\pgfqpoint{2.446088in}{1.684734in}}%
\pgfusepath{clip}%
\pgfsetbuttcap%
\pgfsetroundjoin%
\definecolor{currentfill}{rgb}{0.298039,0.447059,0.690196}%
\pgfsetfillcolor{currentfill}%
\pgfsetlinewidth{1.003750pt}%
\definecolor{currentstroke}{rgb}{0.298039,0.447059,0.690196}%
\pgfsetstrokecolor{currentstroke}%
\pgfsetdash{}{0pt}%
\pgfpathmoveto{\pgfqpoint{4.671897in}{1.648900in}}%
\pgfpathcurveto{\pgfqpoint{4.680133in}{1.648900in}}{\pgfqpoint{4.688033in}{1.652172in}}{\pgfqpoint{4.693857in}{1.657996in}}%
\pgfpathcurveto{\pgfqpoint{4.699681in}{1.663820in}}{\pgfqpoint{4.702953in}{1.671720in}}{\pgfqpoint{4.702953in}{1.679956in}}%
\pgfpathcurveto{\pgfqpoint{4.702953in}{1.688193in}}{\pgfqpoint{4.699681in}{1.696093in}}{\pgfqpoint{4.693857in}{1.701917in}}%
\pgfpathcurveto{\pgfqpoint{4.688033in}{1.707740in}}{\pgfqpoint{4.680133in}{1.711013in}}{\pgfqpoint{4.671897in}{1.711013in}}%
\pgfpathcurveto{\pgfqpoint{4.663660in}{1.711013in}}{\pgfqpoint{4.655760in}{1.707740in}}{\pgfqpoint{4.649936in}{1.701917in}}%
\pgfpathcurveto{\pgfqpoint{4.644113in}{1.696093in}}{\pgfqpoint{4.640840in}{1.688193in}}{\pgfqpoint{4.640840in}{1.679956in}}%
\pgfpathcurveto{\pgfqpoint{4.640840in}{1.671720in}}{\pgfqpoint{4.644113in}{1.663820in}}{\pgfqpoint{4.649936in}{1.657996in}}%
\pgfpathcurveto{\pgfqpoint{4.655760in}{1.652172in}}{\pgfqpoint{4.663660in}{1.648900in}}{\pgfqpoint{4.671897in}{1.648900in}}%
\pgfpathclose%
\pgfusepath{stroke,fill}%
\end{pgfscope}%
\begin{pgfscope}%
\pgfpathrectangle{\pgfqpoint{3.793912in}{0.557870in}}{\pgfqpoint{2.446088in}{1.684734in}}%
\pgfusepath{clip}%
\pgfsetbuttcap%
\pgfsetroundjoin%
\definecolor{currentfill}{rgb}{0.298039,0.447059,0.690196}%
\pgfsetfillcolor{currentfill}%
\pgfsetlinewidth{1.003750pt}%
\definecolor{currentstroke}{rgb}{0.298039,0.447059,0.690196}%
\pgfsetstrokecolor{currentstroke}%
\pgfsetdash{}{0pt}%
\pgfpathmoveto{\pgfqpoint{3.905098in}{2.116627in}}%
\pgfpathcurveto{\pgfqpoint{3.913334in}{2.116627in}}{\pgfqpoint{3.921234in}{2.119899in}}{\pgfqpoint{3.927058in}{2.125723in}}%
\pgfpathcurveto{\pgfqpoint{3.932882in}{2.131547in}}{\pgfqpoint{3.936155in}{2.139447in}}{\pgfqpoint{3.936155in}{2.147683in}}%
\pgfpathcurveto{\pgfqpoint{3.936155in}{2.155919in}}{\pgfqpoint{3.932882in}{2.163819in}}{\pgfqpoint{3.927058in}{2.169643in}}%
\pgfpathcurveto{\pgfqpoint{3.921234in}{2.175467in}}{\pgfqpoint{3.913334in}{2.178740in}}{\pgfqpoint{3.905098in}{2.178740in}}%
\pgfpathcurveto{\pgfqpoint{3.896862in}{2.178740in}}{\pgfqpoint{3.888962in}{2.175467in}}{\pgfqpoint{3.883138in}{2.169643in}}%
\pgfpathcurveto{\pgfqpoint{3.877314in}{2.163819in}}{\pgfqpoint{3.874042in}{2.155919in}}{\pgfqpoint{3.874042in}{2.147683in}}%
\pgfpathcurveto{\pgfqpoint{3.874042in}{2.139447in}}{\pgfqpoint{3.877314in}{2.131547in}}{\pgfqpoint{3.883138in}{2.125723in}}%
\pgfpathcurveto{\pgfqpoint{3.888962in}{2.119899in}}{\pgfqpoint{3.896862in}{2.116627in}}{\pgfqpoint{3.905098in}{2.116627in}}%
\pgfpathclose%
\pgfusepath{stroke,fill}%
\end{pgfscope}%
\begin{pgfscope}%
\pgfpathrectangle{\pgfqpoint{3.793912in}{0.557870in}}{\pgfqpoint{2.446088in}{1.684734in}}%
\pgfusepath{clip}%
\pgfsetbuttcap%
\pgfsetroundjoin%
\definecolor{currentfill}{rgb}{0.298039,0.447059,0.690196}%
\pgfsetfillcolor{currentfill}%
\pgfsetlinewidth{1.003750pt}%
\definecolor{currentstroke}{rgb}{0.298039,0.447059,0.690196}%
\pgfsetstrokecolor{currentstroke}%
\pgfsetdash{}{0pt}%
\pgfpathmoveto{\pgfqpoint{4.288497in}{1.813980in}}%
\pgfpathcurveto{\pgfqpoint{4.296734in}{1.813980in}}{\pgfqpoint{4.304634in}{1.817252in}}{\pgfqpoint{4.310458in}{1.823076in}}%
\pgfpathcurveto{\pgfqpoint{4.316282in}{1.828900in}}{\pgfqpoint{4.319554in}{1.836800in}}{\pgfqpoint{4.319554in}{1.845036in}}%
\pgfpathcurveto{\pgfqpoint{4.319554in}{1.853273in}}{\pgfqpoint{4.316282in}{1.861173in}}{\pgfqpoint{4.310458in}{1.866997in}}%
\pgfpathcurveto{\pgfqpoint{4.304634in}{1.872821in}}{\pgfqpoint{4.296734in}{1.876093in}}{\pgfqpoint{4.288497in}{1.876093in}}%
\pgfpathcurveto{\pgfqpoint{4.280261in}{1.876093in}}{\pgfqpoint{4.272361in}{1.872821in}}{\pgfqpoint{4.266537in}{1.866997in}}%
\pgfpathcurveto{\pgfqpoint{4.260713in}{1.861173in}}{\pgfqpoint{4.257441in}{1.853273in}}{\pgfqpoint{4.257441in}{1.845036in}}%
\pgfpathcurveto{\pgfqpoint{4.257441in}{1.836800in}}{\pgfqpoint{4.260713in}{1.828900in}}{\pgfqpoint{4.266537in}{1.823076in}}%
\pgfpathcurveto{\pgfqpoint{4.272361in}{1.817252in}}{\pgfqpoint{4.280261in}{1.813980in}}{\pgfqpoint{4.288497in}{1.813980in}}%
\pgfpathclose%
\pgfusepath{stroke,fill}%
\end{pgfscope}%
\begin{pgfscope}%
\pgfpathrectangle{\pgfqpoint{3.793912in}{0.557870in}}{\pgfqpoint{2.446088in}{1.684734in}}%
\pgfusepath{clip}%
\pgfsetbuttcap%
\pgfsetroundjoin%
\definecolor{currentfill}{rgb}{0.298039,0.447059,0.690196}%
\pgfsetfillcolor{currentfill}%
\pgfsetlinewidth{1.003750pt}%
\definecolor{currentstroke}{rgb}{0.298039,0.447059,0.690196}%
\pgfsetstrokecolor{currentstroke}%
\pgfsetdash{}{0pt}%
\pgfpathmoveto{\pgfqpoint{3.905098in}{2.125798in}}%
\pgfpathcurveto{\pgfqpoint{3.913334in}{2.125798in}}{\pgfqpoint{3.921234in}{2.129070in}}{\pgfqpoint{3.927058in}{2.134894in}}%
\pgfpathcurveto{\pgfqpoint{3.932882in}{2.140718in}}{\pgfqpoint{3.936155in}{2.148618in}}{\pgfqpoint{3.936155in}{2.156854in}}%
\pgfpathcurveto{\pgfqpoint{3.936155in}{2.165091in}}{\pgfqpoint{3.932882in}{2.172991in}}{\pgfqpoint{3.927058in}{2.178814in}}%
\pgfpathcurveto{\pgfqpoint{3.921234in}{2.184638in}}{\pgfqpoint{3.913334in}{2.187911in}}{\pgfqpoint{3.905098in}{2.187911in}}%
\pgfpathcurveto{\pgfqpoint{3.896862in}{2.187911in}}{\pgfqpoint{3.888962in}{2.184638in}}{\pgfqpoint{3.883138in}{2.178814in}}%
\pgfpathcurveto{\pgfqpoint{3.877314in}{2.172991in}}{\pgfqpoint{3.874042in}{2.165091in}}{\pgfqpoint{3.874042in}{2.156854in}}%
\pgfpathcurveto{\pgfqpoint{3.874042in}{2.148618in}}{\pgfqpoint{3.877314in}{2.140718in}}{\pgfqpoint{3.883138in}{2.134894in}}%
\pgfpathcurveto{\pgfqpoint{3.888962in}{2.129070in}}{\pgfqpoint{3.896862in}{2.125798in}}{\pgfqpoint{3.905098in}{2.125798in}}%
\pgfpathclose%
\pgfusepath{stroke,fill}%
\end{pgfscope}%
\begin{pgfscope}%
\pgfpathrectangle{\pgfqpoint{3.793912in}{0.557870in}}{\pgfqpoint{2.446088in}{1.684734in}}%
\pgfusepath{clip}%
\pgfsetbuttcap%
\pgfsetroundjoin%
\definecolor{currentfill}{rgb}{0.298039,0.447059,0.690196}%
\pgfsetfillcolor{currentfill}%
\pgfsetlinewidth{1.003750pt}%
\definecolor{currentstroke}{rgb}{0.298039,0.447059,0.690196}%
\pgfsetstrokecolor{currentstroke}%
\pgfsetdash{}{0pt}%
\pgfpathmoveto{\pgfqpoint{3.905098in}{2.089113in}}%
\pgfpathcurveto{\pgfqpoint{3.913334in}{2.089113in}}{\pgfqpoint{3.921234in}{2.092386in}}{\pgfqpoint{3.927058in}{2.098210in}}%
\pgfpathcurveto{\pgfqpoint{3.932882in}{2.104033in}}{\pgfqpoint{3.936155in}{2.111933in}}{\pgfqpoint{3.936155in}{2.120170in}}%
\pgfpathcurveto{\pgfqpoint{3.936155in}{2.128406in}}{\pgfqpoint{3.932882in}{2.136306in}}{\pgfqpoint{3.927058in}{2.142130in}}%
\pgfpathcurveto{\pgfqpoint{3.921234in}{2.147954in}}{\pgfqpoint{3.913334in}{2.151226in}}{\pgfqpoint{3.905098in}{2.151226in}}%
\pgfpathcurveto{\pgfqpoint{3.896862in}{2.151226in}}{\pgfqpoint{3.888962in}{2.147954in}}{\pgfqpoint{3.883138in}{2.142130in}}%
\pgfpathcurveto{\pgfqpoint{3.877314in}{2.136306in}}{\pgfqpoint{3.874042in}{2.128406in}}{\pgfqpoint{3.874042in}{2.120170in}}%
\pgfpathcurveto{\pgfqpoint{3.874042in}{2.111933in}}{\pgfqpoint{3.877314in}{2.104033in}}{\pgfqpoint{3.883138in}{2.098210in}}%
\pgfpathcurveto{\pgfqpoint{3.888962in}{2.092386in}}{\pgfqpoint{3.896862in}{2.089113in}}{\pgfqpoint{3.905098in}{2.089113in}}%
\pgfpathclose%
\pgfusepath{stroke,fill}%
\end{pgfscope}%
\begin{pgfscope}%
\pgfpathrectangle{\pgfqpoint{3.793912in}{0.557870in}}{\pgfqpoint{2.446088in}{1.684734in}}%
\pgfusepath{clip}%
\pgfsetbuttcap%
\pgfsetroundjoin%
\definecolor{currentfill}{rgb}{0.298039,0.447059,0.690196}%
\pgfsetfillcolor{currentfill}%
\pgfsetlinewidth{1.003750pt}%
\definecolor{currentstroke}{rgb}{0.298039,0.447059,0.690196}%
\pgfsetstrokecolor{currentstroke}%
\pgfsetdash{}{0pt}%
\pgfpathmoveto{\pgfqpoint{3.905098in}{2.125798in}}%
\pgfpathcurveto{\pgfqpoint{3.913334in}{2.125798in}}{\pgfqpoint{3.921234in}{2.129070in}}{\pgfqpoint{3.927058in}{2.134894in}}%
\pgfpathcurveto{\pgfqpoint{3.932882in}{2.140718in}}{\pgfqpoint{3.936155in}{2.148618in}}{\pgfqpoint{3.936155in}{2.156854in}}%
\pgfpathcurveto{\pgfqpoint{3.936155in}{2.165091in}}{\pgfqpoint{3.932882in}{2.172991in}}{\pgfqpoint{3.927058in}{2.178814in}}%
\pgfpathcurveto{\pgfqpoint{3.921234in}{2.184638in}}{\pgfqpoint{3.913334in}{2.187911in}}{\pgfqpoint{3.905098in}{2.187911in}}%
\pgfpathcurveto{\pgfqpoint{3.896862in}{2.187911in}}{\pgfqpoint{3.888962in}{2.184638in}}{\pgfqpoint{3.883138in}{2.178814in}}%
\pgfpathcurveto{\pgfqpoint{3.877314in}{2.172991in}}{\pgfqpoint{3.874042in}{2.165091in}}{\pgfqpoint{3.874042in}{2.156854in}}%
\pgfpathcurveto{\pgfqpoint{3.874042in}{2.148618in}}{\pgfqpoint{3.877314in}{2.140718in}}{\pgfqpoint{3.883138in}{2.134894in}}%
\pgfpathcurveto{\pgfqpoint{3.888962in}{2.129070in}}{\pgfqpoint{3.896862in}{2.125798in}}{\pgfqpoint{3.905098in}{2.125798in}}%
\pgfpathclose%
\pgfusepath{stroke,fill}%
\end{pgfscope}%
\begin{pgfscope}%
\pgfpathrectangle{\pgfqpoint{3.793912in}{0.557870in}}{\pgfqpoint{2.446088in}{1.684734in}}%
\pgfusepath{clip}%
\pgfsetbuttcap%
\pgfsetroundjoin%
\definecolor{currentfill}{rgb}{0.298039,0.447059,0.690196}%
\pgfsetfillcolor{currentfill}%
\pgfsetlinewidth{1.003750pt}%
\definecolor{currentstroke}{rgb}{0.298039,0.447059,0.690196}%
\pgfsetstrokecolor{currentstroke}%
\pgfsetdash{}{0pt}%
\pgfpathmoveto{\pgfqpoint{3.905098in}{2.125798in}}%
\pgfpathcurveto{\pgfqpoint{3.913334in}{2.125798in}}{\pgfqpoint{3.921234in}{2.129070in}}{\pgfqpoint{3.927058in}{2.134894in}}%
\pgfpathcurveto{\pgfqpoint{3.932882in}{2.140718in}}{\pgfqpoint{3.936155in}{2.148618in}}{\pgfqpoint{3.936155in}{2.156854in}}%
\pgfpathcurveto{\pgfqpoint{3.936155in}{2.165091in}}{\pgfqpoint{3.932882in}{2.172991in}}{\pgfqpoint{3.927058in}{2.178814in}}%
\pgfpathcurveto{\pgfqpoint{3.921234in}{2.184638in}}{\pgfqpoint{3.913334in}{2.187911in}}{\pgfqpoint{3.905098in}{2.187911in}}%
\pgfpathcurveto{\pgfqpoint{3.896862in}{2.187911in}}{\pgfqpoint{3.888962in}{2.184638in}}{\pgfqpoint{3.883138in}{2.178814in}}%
\pgfpathcurveto{\pgfqpoint{3.877314in}{2.172991in}}{\pgfqpoint{3.874042in}{2.165091in}}{\pgfqpoint{3.874042in}{2.156854in}}%
\pgfpathcurveto{\pgfqpoint{3.874042in}{2.148618in}}{\pgfqpoint{3.877314in}{2.140718in}}{\pgfqpoint{3.883138in}{2.134894in}}%
\pgfpathcurveto{\pgfqpoint{3.888962in}{2.129070in}}{\pgfqpoint{3.896862in}{2.125798in}}{\pgfqpoint{3.905098in}{2.125798in}}%
\pgfpathclose%
\pgfusepath{stroke,fill}%
\end{pgfscope}%
\begin{pgfscope}%
\pgfpathrectangle{\pgfqpoint{3.793912in}{0.557870in}}{\pgfqpoint{2.446088in}{1.684734in}}%
\pgfusepath{clip}%
\pgfsetbuttcap%
\pgfsetroundjoin%
\definecolor{currentfill}{rgb}{0.298039,0.447059,0.690196}%
\pgfsetfillcolor{currentfill}%
\pgfsetlinewidth{1.003750pt}%
\definecolor{currentstroke}{rgb}{0.298039,0.447059,0.690196}%
\pgfsetstrokecolor{currentstroke}%
\pgfsetdash{}{0pt}%
\pgfpathmoveto{\pgfqpoint{4.748577in}{1.492991in}}%
\pgfpathcurveto{\pgfqpoint{4.756813in}{1.492991in}}{\pgfqpoint{4.764713in}{1.496263in}}{\pgfqpoint{4.770537in}{1.502087in}}%
\pgfpathcurveto{\pgfqpoint{4.776361in}{1.507911in}}{\pgfqpoint{4.779633in}{1.515811in}}{\pgfqpoint{4.779633in}{1.524047in}}%
\pgfpathcurveto{\pgfqpoint{4.779633in}{1.532284in}}{\pgfqpoint{4.776361in}{1.540184in}}{\pgfqpoint{4.770537in}{1.546008in}}%
\pgfpathcurveto{\pgfqpoint{4.764713in}{1.551831in}}{\pgfqpoint{4.756813in}{1.555104in}}{\pgfqpoint{4.748577in}{1.555104in}}%
\pgfpathcurveto{\pgfqpoint{4.740340in}{1.555104in}}{\pgfqpoint{4.732440in}{1.551831in}}{\pgfqpoint{4.726616in}{1.546008in}}%
\pgfpathcurveto{\pgfqpoint{4.720792in}{1.540184in}}{\pgfqpoint{4.717520in}{1.532284in}}{\pgfqpoint{4.717520in}{1.524047in}}%
\pgfpathcurveto{\pgfqpoint{4.717520in}{1.515811in}}{\pgfqpoint{4.720792in}{1.507911in}}{\pgfqpoint{4.726616in}{1.502087in}}%
\pgfpathcurveto{\pgfqpoint{4.732440in}{1.496263in}}{\pgfqpoint{4.740340in}{1.492991in}}{\pgfqpoint{4.748577in}{1.492991in}}%
\pgfpathclose%
\pgfusepath{stroke,fill}%
\end{pgfscope}%
\begin{pgfscope}%
\pgfpathrectangle{\pgfqpoint{3.793912in}{0.557870in}}{\pgfqpoint{2.446088in}{1.684734in}}%
\pgfusepath{clip}%
\pgfsetbuttcap%
\pgfsetroundjoin%
\definecolor{currentfill}{rgb}{0.298039,0.447059,0.690196}%
\pgfsetfillcolor{currentfill}%
\pgfsetlinewidth{1.003750pt}%
\definecolor{currentstroke}{rgb}{0.298039,0.447059,0.690196}%
\pgfsetstrokecolor{currentstroke}%
\pgfsetdash{}{0pt}%
\pgfpathmoveto{\pgfqpoint{3.905098in}{2.116627in}}%
\pgfpathcurveto{\pgfqpoint{3.913334in}{2.116627in}}{\pgfqpoint{3.921234in}{2.119899in}}{\pgfqpoint{3.927058in}{2.125723in}}%
\pgfpathcurveto{\pgfqpoint{3.932882in}{2.131547in}}{\pgfqpoint{3.936155in}{2.139447in}}{\pgfqpoint{3.936155in}{2.147683in}}%
\pgfpathcurveto{\pgfqpoint{3.936155in}{2.155919in}}{\pgfqpoint{3.932882in}{2.163819in}}{\pgfqpoint{3.927058in}{2.169643in}}%
\pgfpathcurveto{\pgfqpoint{3.921234in}{2.175467in}}{\pgfqpoint{3.913334in}{2.178740in}}{\pgfqpoint{3.905098in}{2.178740in}}%
\pgfpathcurveto{\pgfqpoint{3.896862in}{2.178740in}}{\pgfqpoint{3.888962in}{2.175467in}}{\pgfqpoint{3.883138in}{2.169643in}}%
\pgfpathcurveto{\pgfqpoint{3.877314in}{2.163819in}}{\pgfqpoint{3.874042in}{2.155919in}}{\pgfqpoint{3.874042in}{2.147683in}}%
\pgfpathcurveto{\pgfqpoint{3.874042in}{2.139447in}}{\pgfqpoint{3.877314in}{2.131547in}}{\pgfqpoint{3.883138in}{2.125723in}}%
\pgfpathcurveto{\pgfqpoint{3.888962in}{2.119899in}}{\pgfqpoint{3.896862in}{2.116627in}}{\pgfqpoint{3.905098in}{2.116627in}}%
\pgfpathclose%
\pgfusepath{stroke,fill}%
\end{pgfscope}%
\begin{pgfscope}%
\pgfpathrectangle{\pgfqpoint{3.793912in}{0.557870in}}{\pgfqpoint{2.446088in}{1.684734in}}%
\pgfusepath{clip}%
\pgfsetbuttcap%
\pgfsetroundjoin%
\definecolor{currentfill}{rgb}{0.298039,0.447059,0.690196}%
\pgfsetfillcolor{currentfill}%
\pgfsetlinewidth{1.003750pt}%
\definecolor{currentstroke}{rgb}{0.298039,0.447059,0.690196}%
\pgfsetstrokecolor{currentstroke}%
\pgfsetdash{}{0pt}%
\pgfpathmoveto{\pgfqpoint{4.978616in}{1.740611in}}%
\pgfpathcurveto{\pgfqpoint{4.986852in}{1.740611in}}{\pgfqpoint{4.994753in}{1.743883in}}{\pgfqpoint{5.000576in}{1.749707in}}%
\pgfpathcurveto{\pgfqpoint{5.006400in}{1.755531in}}{\pgfqpoint{5.009673in}{1.763431in}}{\pgfqpoint{5.009673in}{1.771667in}}%
\pgfpathcurveto{\pgfqpoint{5.009673in}{1.779904in}}{\pgfqpoint{5.006400in}{1.787804in}}{\pgfqpoint{5.000576in}{1.793628in}}%
\pgfpathcurveto{\pgfqpoint{4.994753in}{1.799452in}}{\pgfqpoint{4.986852in}{1.802724in}}{\pgfqpoint{4.978616in}{1.802724in}}%
\pgfpathcurveto{\pgfqpoint{4.970380in}{1.802724in}}{\pgfqpoint{4.962480in}{1.799452in}}{\pgfqpoint{4.956656in}{1.793628in}}%
\pgfpathcurveto{\pgfqpoint{4.950832in}{1.787804in}}{\pgfqpoint{4.947560in}{1.779904in}}{\pgfqpoint{4.947560in}{1.771667in}}%
\pgfpathcurveto{\pgfqpoint{4.947560in}{1.763431in}}{\pgfqpoint{4.950832in}{1.755531in}}{\pgfqpoint{4.956656in}{1.749707in}}%
\pgfpathcurveto{\pgfqpoint{4.962480in}{1.743883in}}{\pgfqpoint{4.970380in}{1.740611in}}{\pgfqpoint{4.978616in}{1.740611in}}%
\pgfpathclose%
\pgfusepath{stroke,fill}%
\end{pgfscope}%
\begin{pgfscope}%
\pgfpathrectangle{\pgfqpoint{3.793912in}{0.557870in}}{\pgfqpoint{2.446088in}{1.684734in}}%
\pgfusepath{clip}%
\pgfsetbuttcap%
\pgfsetroundjoin%
\definecolor{currentfill}{rgb}{0.298039,0.447059,0.690196}%
\pgfsetfillcolor{currentfill}%
\pgfsetlinewidth{1.003750pt}%
\definecolor{currentstroke}{rgb}{0.298039,0.447059,0.690196}%
\pgfsetstrokecolor{currentstroke}%
\pgfsetdash{}{0pt}%
\pgfpathmoveto{\pgfqpoint{3.905098in}{2.125798in}}%
\pgfpathcurveto{\pgfqpoint{3.913334in}{2.125798in}}{\pgfqpoint{3.921234in}{2.129070in}}{\pgfqpoint{3.927058in}{2.134894in}}%
\pgfpathcurveto{\pgfqpoint{3.932882in}{2.140718in}}{\pgfqpoint{3.936155in}{2.148618in}}{\pgfqpoint{3.936155in}{2.156854in}}%
\pgfpathcurveto{\pgfqpoint{3.936155in}{2.165091in}}{\pgfqpoint{3.932882in}{2.172991in}}{\pgfqpoint{3.927058in}{2.178814in}}%
\pgfpathcurveto{\pgfqpoint{3.921234in}{2.184638in}}{\pgfqpoint{3.913334in}{2.187911in}}{\pgfqpoint{3.905098in}{2.187911in}}%
\pgfpathcurveto{\pgfqpoint{3.896862in}{2.187911in}}{\pgfqpoint{3.888962in}{2.184638in}}{\pgfqpoint{3.883138in}{2.178814in}}%
\pgfpathcurveto{\pgfqpoint{3.877314in}{2.172991in}}{\pgfqpoint{3.874042in}{2.165091in}}{\pgfqpoint{3.874042in}{2.156854in}}%
\pgfpathcurveto{\pgfqpoint{3.874042in}{2.148618in}}{\pgfqpoint{3.877314in}{2.140718in}}{\pgfqpoint{3.883138in}{2.134894in}}%
\pgfpathcurveto{\pgfqpoint{3.888962in}{2.129070in}}{\pgfqpoint{3.896862in}{2.125798in}}{\pgfqpoint{3.905098in}{2.125798in}}%
\pgfpathclose%
\pgfusepath{stroke,fill}%
\end{pgfscope}%
\begin{pgfscope}%
\pgfpathrectangle{\pgfqpoint{3.793912in}{0.557870in}}{\pgfqpoint{2.446088in}{1.684734in}}%
\pgfusepath{clip}%
\pgfsetbuttcap%
\pgfsetroundjoin%
\definecolor{currentfill}{rgb}{0.298039,0.447059,0.690196}%
\pgfsetfillcolor{currentfill}%
\pgfsetlinewidth{1.003750pt}%
\definecolor{currentstroke}{rgb}{0.298039,0.447059,0.690196}%
\pgfsetstrokecolor{currentstroke}%
\pgfsetdash{}{0pt}%
\pgfpathmoveto{\pgfqpoint{3.905098in}{2.125798in}}%
\pgfpathcurveto{\pgfqpoint{3.913334in}{2.125798in}}{\pgfqpoint{3.921234in}{2.129070in}}{\pgfqpoint{3.927058in}{2.134894in}}%
\pgfpathcurveto{\pgfqpoint{3.932882in}{2.140718in}}{\pgfqpoint{3.936155in}{2.148618in}}{\pgfqpoint{3.936155in}{2.156854in}}%
\pgfpathcurveto{\pgfqpoint{3.936155in}{2.165091in}}{\pgfqpoint{3.932882in}{2.172991in}}{\pgfqpoint{3.927058in}{2.178814in}}%
\pgfpathcurveto{\pgfqpoint{3.921234in}{2.184638in}}{\pgfqpoint{3.913334in}{2.187911in}}{\pgfqpoint{3.905098in}{2.187911in}}%
\pgfpathcurveto{\pgfqpoint{3.896862in}{2.187911in}}{\pgfqpoint{3.888962in}{2.184638in}}{\pgfqpoint{3.883138in}{2.178814in}}%
\pgfpathcurveto{\pgfqpoint{3.877314in}{2.172991in}}{\pgfqpoint{3.874042in}{2.165091in}}{\pgfqpoint{3.874042in}{2.156854in}}%
\pgfpathcurveto{\pgfqpoint{3.874042in}{2.148618in}}{\pgfqpoint{3.877314in}{2.140718in}}{\pgfqpoint{3.883138in}{2.134894in}}%
\pgfpathcurveto{\pgfqpoint{3.888962in}{2.129070in}}{\pgfqpoint{3.896862in}{2.125798in}}{\pgfqpoint{3.905098in}{2.125798in}}%
\pgfpathclose%
\pgfusepath{stroke,fill}%
\end{pgfscope}%
\begin{pgfscope}%
\pgfpathrectangle{\pgfqpoint{3.793912in}{0.557870in}}{\pgfqpoint{2.446088in}{1.684734in}}%
\pgfusepath{clip}%
\pgfsetbuttcap%
\pgfsetroundjoin%
\definecolor{currentfill}{rgb}{0.298039,0.447059,0.690196}%
\pgfsetfillcolor{currentfill}%
\pgfsetlinewidth{1.003750pt}%
\definecolor{currentstroke}{rgb}{0.298039,0.447059,0.690196}%
\pgfsetstrokecolor{currentstroke}%
\pgfsetdash{}{0pt}%
\pgfpathmoveto{\pgfqpoint{3.905098in}{2.125798in}}%
\pgfpathcurveto{\pgfqpoint{3.913334in}{2.125798in}}{\pgfqpoint{3.921234in}{2.129070in}}{\pgfqpoint{3.927058in}{2.134894in}}%
\pgfpathcurveto{\pgfqpoint{3.932882in}{2.140718in}}{\pgfqpoint{3.936155in}{2.148618in}}{\pgfqpoint{3.936155in}{2.156854in}}%
\pgfpathcurveto{\pgfqpoint{3.936155in}{2.165091in}}{\pgfqpoint{3.932882in}{2.172991in}}{\pgfqpoint{3.927058in}{2.178814in}}%
\pgfpathcurveto{\pgfqpoint{3.921234in}{2.184638in}}{\pgfqpoint{3.913334in}{2.187911in}}{\pgfqpoint{3.905098in}{2.187911in}}%
\pgfpathcurveto{\pgfqpoint{3.896862in}{2.187911in}}{\pgfqpoint{3.888962in}{2.184638in}}{\pgfqpoint{3.883138in}{2.178814in}}%
\pgfpathcurveto{\pgfqpoint{3.877314in}{2.172991in}}{\pgfqpoint{3.874042in}{2.165091in}}{\pgfqpoint{3.874042in}{2.156854in}}%
\pgfpathcurveto{\pgfqpoint{3.874042in}{2.148618in}}{\pgfqpoint{3.877314in}{2.140718in}}{\pgfqpoint{3.883138in}{2.134894in}}%
\pgfpathcurveto{\pgfqpoint{3.888962in}{2.129070in}}{\pgfqpoint{3.896862in}{2.125798in}}{\pgfqpoint{3.905098in}{2.125798in}}%
\pgfpathclose%
\pgfusepath{stroke,fill}%
\end{pgfscope}%
\begin{pgfscope}%
\pgfpathrectangle{\pgfqpoint{3.793912in}{0.557870in}}{\pgfqpoint{2.446088in}{1.684734in}}%
\pgfusepath{clip}%
\pgfsetbuttcap%
\pgfsetroundjoin%
\definecolor{currentfill}{rgb}{0.298039,0.447059,0.690196}%
\pgfsetfillcolor{currentfill}%
\pgfsetlinewidth{1.003750pt}%
\definecolor{currentstroke}{rgb}{0.298039,0.447059,0.690196}%
\pgfsetstrokecolor{currentstroke}%
\pgfsetdash{}{0pt}%
\pgfpathmoveto{\pgfqpoint{3.905098in}{2.125798in}}%
\pgfpathcurveto{\pgfqpoint{3.913334in}{2.125798in}}{\pgfqpoint{3.921234in}{2.129070in}}{\pgfqpoint{3.927058in}{2.134894in}}%
\pgfpathcurveto{\pgfqpoint{3.932882in}{2.140718in}}{\pgfqpoint{3.936155in}{2.148618in}}{\pgfqpoint{3.936155in}{2.156854in}}%
\pgfpathcurveto{\pgfqpoint{3.936155in}{2.165091in}}{\pgfqpoint{3.932882in}{2.172991in}}{\pgfqpoint{3.927058in}{2.178814in}}%
\pgfpathcurveto{\pgfqpoint{3.921234in}{2.184638in}}{\pgfqpoint{3.913334in}{2.187911in}}{\pgfqpoint{3.905098in}{2.187911in}}%
\pgfpathcurveto{\pgfqpoint{3.896862in}{2.187911in}}{\pgfqpoint{3.888962in}{2.184638in}}{\pgfqpoint{3.883138in}{2.178814in}}%
\pgfpathcurveto{\pgfqpoint{3.877314in}{2.172991in}}{\pgfqpoint{3.874042in}{2.165091in}}{\pgfqpoint{3.874042in}{2.156854in}}%
\pgfpathcurveto{\pgfqpoint{3.874042in}{2.148618in}}{\pgfqpoint{3.877314in}{2.140718in}}{\pgfqpoint{3.883138in}{2.134894in}}%
\pgfpathcurveto{\pgfqpoint{3.888962in}{2.129070in}}{\pgfqpoint{3.896862in}{2.125798in}}{\pgfqpoint{3.905098in}{2.125798in}}%
\pgfpathclose%
\pgfusepath{stroke,fill}%
\end{pgfscope}%
\begin{pgfscope}%
\pgfpathrectangle{\pgfqpoint{3.793912in}{0.557870in}}{\pgfqpoint{2.446088in}{1.684734in}}%
\pgfusepath{clip}%
\pgfsetbuttcap%
\pgfsetroundjoin%
\definecolor{currentfill}{rgb}{0.298039,0.447059,0.690196}%
\pgfsetfillcolor{currentfill}%
\pgfsetlinewidth{1.003750pt}%
\definecolor{currentstroke}{rgb}{0.298039,0.447059,0.690196}%
\pgfsetstrokecolor{currentstroke}%
\pgfsetdash{}{0pt}%
\pgfpathmoveto{\pgfqpoint{5.975454in}{1.327911in}}%
\pgfpathcurveto{\pgfqpoint{5.983691in}{1.327911in}}{\pgfqpoint{5.991591in}{1.331183in}}{\pgfqpoint{5.997415in}{1.337007in}}%
\pgfpathcurveto{\pgfqpoint{6.003239in}{1.342831in}}{\pgfqpoint{6.006511in}{1.350731in}}{\pgfqpoint{6.006511in}{1.358967in}}%
\pgfpathcurveto{\pgfqpoint{6.006511in}{1.367203in}}{\pgfqpoint{6.003239in}{1.375104in}}{\pgfqpoint{5.997415in}{1.380927in}}%
\pgfpathcurveto{\pgfqpoint{5.991591in}{1.386751in}}{\pgfqpoint{5.983691in}{1.390024in}}{\pgfqpoint{5.975454in}{1.390024in}}%
\pgfpathcurveto{\pgfqpoint{5.967218in}{1.390024in}}{\pgfqpoint{5.959318in}{1.386751in}}{\pgfqpoint{5.953494in}{1.380927in}}%
\pgfpathcurveto{\pgfqpoint{5.947670in}{1.375104in}}{\pgfqpoint{5.944398in}{1.367203in}}{\pgfqpoint{5.944398in}{1.358967in}}%
\pgfpathcurveto{\pgfqpoint{5.944398in}{1.350731in}}{\pgfqpoint{5.947670in}{1.342831in}}{\pgfqpoint{5.953494in}{1.337007in}}%
\pgfpathcurveto{\pgfqpoint{5.959318in}{1.331183in}}{\pgfqpoint{5.967218in}{1.327911in}}{\pgfqpoint{5.975454in}{1.327911in}}%
\pgfpathclose%
\pgfusepath{stroke,fill}%
\end{pgfscope}%
\begin{pgfscope}%
\pgfpathrectangle{\pgfqpoint{3.793912in}{0.557870in}}{\pgfqpoint{2.446088in}{1.684734in}}%
\pgfusepath{clip}%
\pgfsetbuttcap%
\pgfsetroundjoin%
\definecolor{currentfill}{rgb}{0.298039,0.447059,0.690196}%
\pgfsetfillcolor{currentfill}%
\pgfsetlinewidth{1.003750pt}%
\definecolor{currentstroke}{rgb}{0.298039,0.447059,0.690196}%
\pgfsetstrokecolor{currentstroke}%
\pgfsetdash{}{0pt}%
\pgfpathmoveto{\pgfqpoint{5.898775in}{1.667242in}}%
\pgfpathcurveto{\pgfqpoint{5.907011in}{1.667242in}}{\pgfqpoint{5.914911in}{1.670514in}}{\pgfqpoint{5.920735in}{1.676338in}}%
\pgfpathcurveto{\pgfqpoint{5.926559in}{1.682162in}}{\pgfqpoint{5.929831in}{1.690062in}}{\pgfqpoint{5.929831in}{1.698298in}}%
\pgfpathcurveto{\pgfqpoint{5.929831in}{1.706535in}}{\pgfqpoint{5.926559in}{1.714435in}}{\pgfqpoint{5.920735in}{1.720259in}}%
\pgfpathcurveto{\pgfqpoint{5.914911in}{1.726083in}}{\pgfqpoint{5.907011in}{1.729355in}}{\pgfqpoint{5.898775in}{1.729355in}}%
\pgfpathcurveto{\pgfqpoint{5.890538in}{1.729355in}}{\pgfqpoint{5.882638in}{1.726083in}}{\pgfqpoint{5.876814in}{1.720259in}}%
\pgfpathcurveto{\pgfqpoint{5.870990in}{1.714435in}}{\pgfqpoint{5.867718in}{1.706535in}}{\pgfqpoint{5.867718in}{1.698298in}}%
\pgfpathcurveto{\pgfqpoint{5.867718in}{1.690062in}}{\pgfqpoint{5.870990in}{1.682162in}}{\pgfqpoint{5.876814in}{1.676338in}}%
\pgfpathcurveto{\pgfqpoint{5.882638in}{1.670514in}}{\pgfqpoint{5.890538in}{1.667242in}}{\pgfqpoint{5.898775in}{1.667242in}}%
\pgfpathclose%
\pgfusepath{stroke,fill}%
\end{pgfscope}%
\begin{pgfscope}%
\pgfpathrectangle{\pgfqpoint{3.793912in}{0.557870in}}{\pgfqpoint{2.446088in}{1.684734in}}%
\pgfusepath{clip}%
\pgfsetbuttcap%
\pgfsetroundjoin%
\definecolor{currentfill}{rgb}{0.298039,0.447059,0.690196}%
\pgfsetfillcolor{currentfill}%
\pgfsetlinewidth{1.003750pt}%
\definecolor{currentstroke}{rgb}{0.298039,0.447059,0.690196}%
\pgfsetstrokecolor{currentstroke}%
\pgfsetdash{}{0pt}%
\pgfpathmoveto{\pgfqpoint{5.975454in}{1.327911in}}%
\pgfpathcurveto{\pgfqpoint{5.983691in}{1.327911in}}{\pgfqpoint{5.991591in}{1.331183in}}{\pgfqpoint{5.997415in}{1.337007in}}%
\pgfpathcurveto{\pgfqpoint{6.003239in}{1.342831in}}{\pgfqpoint{6.006511in}{1.350731in}}{\pgfqpoint{6.006511in}{1.358967in}}%
\pgfpathcurveto{\pgfqpoint{6.006511in}{1.367203in}}{\pgfqpoint{6.003239in}{1.375104in}}{\pgfqpoint{5.997415in}{1.380927in}}%
\pgfpathcurveto{\pgfqpoint{5.991591in}{1.386751in}}{\pgfqpoint{5.983691in}{1.390024in}}{\pgfqpoint{5.975454in}{1.390024in}}%
\pgfpathcurveto{\pgfqpoint{5.967218in}{1.390024in}}{\pgfqpoint{5.959318in}{1.386751in}}{\pgfqpoint{5.953494in}{1.380927in}}%
\pgfpathcurveto{\pgfqpoint{5.947670in}{1.375104in}}{\pgfqpoint{5.944398in}{1.367203in}}{\pgfqpoint{5.944398in}{1.358967in}}%
\pgfpathcurveto{\pgfqpoint{5.944398in}{1.350731in}}{\pgfqpoint{5.947670in}{1.342831in}}{\pgfqpoint{5.953494in}{1.337007in}}%
\pgfpathcurveto{\pgfqpoint{5.959318in}{1.331183in}}{\pgfqpoint{5.967218in}{1.327911in}}{\pgfqpoint{5.975454in}{1.327911in}}%
\pgfpathclose%
\pgfusepath{stroke,fill}%
\end{pgfscope}%
\begin{pgfscope}%
\pgfpathrectangle{\pgfqpoint{3.793912in}{0.557870in}}{\pgfqpoint{2.446088in}{1.684734in}}%
\pgfusepath{clip}%
\pgfsetbuttcap%
\pgfsetroundjoin%
\definecolor{currentfill}{rgb}{0.298039,0.447059,0.690196}%
\pgfsetfillcolor{currentfill}%
\pgfsetlinewidth{1.003750pt}%
\definecolor{currentstroke}{rgb}{0.298039,0.447059,0.690196}%
\pgfsetstrokecolor{currentstroke}%
\pgfsetdash{}{0pt}%
\pgfpathmoveto{\pgfqpoint{3.905098in}{2.125798in}}%
\pgfpathcurveto{\pgfqpoint{3.913334in}{2.125798in}}{\pgfqpoint{3.921234in}{2.129070in}}{\pgfqpoint{3.927058in}{2.134894in}}%
\pgfpathcurveto{\pgfqpoint{3.932882in}{2.140718in}}{\pgfqpoint{3.936155in}{2.148618in}}{\pgfqpoint{3.936155in}{2.156854in}}%
\pgfpathcurveto{\pgfqpoint{3.936155in}{2.165091in}}{\pgfqpoint{3.932882in}{2.172991in}}{\pgfqpoint{3.927058in}{2.178814in}}%
\pgfpathcurveto{\pgfqpoint{3.921234in}{2.184638in}}{\pgfqpoint{3.913334in}{2.187911in}}{\pgfqpoint{3.905098in}{2.187911in}}%
\pgfpathcurveto{\pgfqpoint{3.896862in}{2.187911in}}{\pgfqpoint{3.888962in}{2.184638in}}{\pgfqpoint{3.883138in}{2.178814in}}%
\pgfpathcurveto{\pgfqpoint{3.877314in}{2.172991in}}{\pgfqpoint{3.874042in}{2.165091in}}{\pgfqpoint{3.874042in}{2.156854in}}%
\pgfpathcurveto{\pgfqpoint{3.874042in}{2.148618in}}{\pgfqpoint{3.877314in}{2.140718in}}{\pgfqpoint{3.883138in}{2.134894in}}%
\pgfpathcurveto{\pgfqpoint{3.888962in}{2.129070in}}{\pgfqpoint{3.896862in}{2.125798in}}{\pgfqpoint{3.905098in}{2.125798in}}%
\pgfpathclose%
\pgfusepath{stroke,fill}%
\end{pgfscope}%
\begin{pgfscope}%
\pgfpathrectangle{\pgfqpoint{3.793912in}{0.557870in}}{\pgfqpoint{2.446088in}{1.684734in}}%
\pgfusepath{clip}%
\pgfsetbuttcap%
\pgfsetroundjoin%
\definecolor{currentfill}{rgb}{0.298039,0.447059,0.690196}%
\pgfsetfillcolor{currentfill}%
\pgfsetlinewidth{1.003750pt}%
\definecolor{currentstroke}{rgb}{0.298039,0.447059,0.690196}%
\pgfsetstrokecolor{currentstroke}%
\pgfsetdash{}{0pt}%
\pgfpathmoveto{\pgfqpoint{5.592055in}{1.841493in}}%
\pgfpathcurveto{\pgfqpoint{5.600291in}{1.841493in}}{\pgfqpoint{5.608191in}{1.844765in}}{\pgfqpoint{5.614015in}{1.850589in}}%
\pgfpathcurveto{\pgfqpoint{5.619839in}{1.856413in}}{\pgfqpoint{5.623112in}{1.864313in}}{\pgfqpoint{5.623112in}{1.872550in}}%
\pgfpathcurveto{\pgfqpoint{5.623112in}{1.880786in}}{\pgfqpoint{5.619839in}{1.888686in}}{\pgfqpoint{5.614015in}{1.894510in}}%
\pgfpathcurveto{\pgfqpoint{5.608191in}{1.900334in}}{\pgfqpoint{5.600291in}{1.903606in}}{\pgfqpoint{5.592055in}{1.903606in}}%
\pgfpathcurveto{\pgfqpoint{5.583819in}{1.903606in}}{\pgfqpoint{5.575919in}{1.900334in}}{\pgfqpoint{5.570095in}{1.894510in}}%
\pgfpathcurveto{\pgfqpoint{5.564271in}{1.888686in}}{\pgfqpoint{5.560999in}{1.880786in}}{\pgfqpoint{5.560999in}{1.872550in}}%
\pgfpathcurveto{\pgfqpoint{5.560999in}{1.864313in}}{\pgfqpoint{5.564271in}{1.856413in}}{\pgfqpoint{5.570095in}{1.850589in}}%
\pgfpathcurveto{\pgfqpoint{5.575919in}{1.844765in}}{\pgfqpoint{5.583819in}{1.841493in}}{\pgfqpoint{5.592055in}{1.841493in}}%
\pgfpathclose%
\pgfusepath{stroke,fill}%
\end{pgfscope}%
\begin{pgfscope}%
\pgfpathrectangle{\pgfqpoint{3.793912in}{0.557870in}}{\pgfqpoint{2.446088in}{1.684734in}}%
\pgfusepath{clip}%
\pgfsetbuttcap%
\pgfsetroundjoin%
\definecolor{currentfill}{rgb}{0.298039,0.447059,0.690196}%
\pgfsetfillcolor{currentfill}%
\pgfsetlinewidth{1.003750pt}%
\definecolor{currentstroke}{rgb}{0.298039,0.447059,0.690196}%
\pgfsetstrokecolor{currentstroke}%
\pgfsetdash{}{0pt}%
\pgfpathmoveto{\pgfqpoint{3.905098in}{2.125798in}}%
\pgfpathcurveto{\pgfqpoint{3.913334in}{2.125798in}}{\pgfqpoint{3.921234in}{2.129070in}}{\pgfqpoint{3.927058in}{2.134894in}}%
\pgfpathcurveto{\pgfqpoint{3.932882in}{2.140718in}}{\pgfqpoint{3.936155in}{2.148618in}}{\pgfqpoint{3.936155in}{2.156854in}}%
\pgfpathcurveto{\pgfqpoint{3.936155in}{2.165091in}}{\pgfqpoint{3.932882in}{2.172991in}}{\pgfqpoint{3.927058in}{2.178814in}}%
\pgfpathcurveto{\pgfqpoint{3.921234in}{2.184638in}}{\pgfqpoint{3.913334in}{2.187911in}}{\pgfqpoint{3.905098in}{2.187911in}}%
\pgfpathcurveto{\pgfqpoint{3.896862in}{2.187911in}}{\pgfqpoint{3.888962in}{2.184638in}}{\pgfqpoint{3.883138in}{2.178814in}}%
\pgfpathcurveto{\pgfqpoint{3.877314in}{2.172991in}}{\pgfqpoint{3.874042in}{2.165091in}}{\pgfqpoint{3.874042in}{2.156854in}}%
\pgfpathcurveto{\pgfqpoint{3.874042in}{2.148618in}}{\pgfqpoint{3.877314in}{2.140718in}}{\pgfqpoint{3.883138in}{2.134894in}}%
\pgfpathcurveto{\pgfqpoint{3.888962in}{2.129070in}}{\pgfqpoint{3.896862in}{2.125798in}}{\pgfqpoint{3.905098in}{2.125798in}}%
\pgfpathclose%
\pgfusepath{stroke,fill}%
\end{pgfscope}%
\begin{pgfscope}%
\pgfpathrectangle{\pgfqpoint{3.793912in}{0.557870in}}{\pgfqpoint{2.446088in}{1.684734in}}%
\pgfusepath{clip}%
\pgfsetbuttcap%
\pgfsetroundjoin%
\definecolor{currentfill}{rgb}{0.298039,0.447059,0.690196}%
\pgfsetfillcolor{currentfill}%
\pgfsetlinewidth{1.003750pt}%
\definecolor{currentstroke}{rgb}{0.298039,0.447059,0.690196}%
\pgfsetstrokecolor{currentstroke}%
\pgfsetdash{}{0pt}%
\pgfpathmoveto{\pgfqpoint{5.975454in}{1.337082in}}%
\pgfpathcurveto{\pgfqpoint{5.983691in}{1.337082in}}{\pgfqpoint{5.991591in}{1.340354in}}{\pgfqpoint{5.997415in}{1.346178in}}%
\pgfpathcurveto{\pgfqpoint{6.003239in}{1.352002in}}{\pgfqpoint{6.006511in}{1.359902in}}{\pgfqpoint{6.006511in}{1.368138in}}%
\pgfpathcurveto{\pgfqpoint{6.006511in}{1.376375in}}{\pgfqpoint{6.003239in}{1.384275in}}{\pgfqpoint{5.997415in}{1.390099in}}%
\pgfpathcurveto{\pgfqpoint{5.991591in}{1.395923in}}{\pgfqpoint{5.983691in}{1.399195in}}{\pgfqpoint{5.975454in}{1.399195in}}%
\pgfpathcurveto{\pgfqpoint{5.967218in}{1.399195in}}{\pgfqpoint{5.959318in}{1.395923in}}{\pgfqpoint{5.953494in}{1.390099in}}%
\pgfpathcurveto{\pgfqpoint{5.947670in}{1.384275in}}{\pgfqpoint{5.944398in}{1.376375in}}{\pgfqpoint{5.944398in}{1.368138in}}%
\pgfpathcurveto{\pgfqpoint{5.944398in}{1.359902in}}{\pgfqpoint{5.947670in}{1.352002in}}{\pgfqpoint{5.953494in}{1.346178in}}%
\pgfpathcurveto{\pgfqpoint{5.959318in}{1.340354in}}{\pgfqpoint{5.967218in}{1.337082in}}{\pgfqpoint{5.975454in}{1.337082in}}%
\pgfpathclose%
\pgfusepath{stroke,fill}%
\end{pgfscope}%
\begin{pgfscope}%
\pgfpathrectangle{\pgfqpoint{3.793912in}{0.557870in}}{\pgfqpoint{2.446088in}{1.684734in}}%
\pgfusepath{clip}%
\pgfsetbuttcap%
\pgfsetroundjoin%
\definecolor{currentfill}{rgb}{0.298039,0.447059,0.690196}%
\pgfsetfillcolor{currentfill}%
\pgfsetlinewidth{1.003750pt}%
\definecolor{currentstroke}{rgb}{0.298039,0.447059,0.690196}%
\pgfsetstrokecolor{currentstroke}%
\pgfsetdash{}{0pt}%
\pgfpathmoveto{\pgfqpoint{5.975454in}{1.318740in}}%
\pgfpathcurveto{\pgfqpoint{5.983691in}{1.318740in}}{\pgfqpoint{5.991591in}{1.322012in}}{\pgfqpoint{5.997415in}{1.327836in}}%
\pgfpathcurveto{\pgfqpoint{6.003239in}{1.333660in}}{\pgfqpoint{6.006511in}{1.341560in}}{\pgfqpoint{6.006511in}{1.349796in}}%
\pgfpathcurveto{\pgfqpoint{6.006511in}{1.358032in}}{\pgfqpoint{6.003239in}{1.365932in}}{\pgfqpoint{5.997415in}{1.371756in}}%
\pgfpathcurveto{\pgfqpoint{5.991591in}{1.377580in}}{\pgfqpoint{5.983691in}{1.380853in}}{\pgfqpoint{5.975454in}{1.380853in}}%
\pgfpathcurveto{\pgfqpoint{5.967218in}{1.380853in}}{\pgfqpoint{5.959318in}{1.377580in}}{\pgfqpoint{5.953494in}{1.371756in}}%
\pgfpathcurveto{\pgfqpoint{5.947670in}{1.365932in}}{\pgfqpoint{5.944398in}{1.358032in}}{\pgfqpoint{5.944398in}{1.349796in}}%
\pgfpathcurveto{\pgfqpoint{5.944398in}{1.341560in}}{\pgfqpoint{5.947670in}{1.333660in}}{\pgfqpoint{5.953494in}{1.327836in}}%
\pgfpathcurveto{\pgfqpoint{5.959318in}{1.322012in}}{\pgfqpoint{5.967218in}{1.318740in}}{\pgfqpoint{5.975454in}{1.318740in}}%
\pgfpathclose%
\pgfusepath{stroke,fill}%
\end{pgfscope}%
\begin{pgfscope}%
\pgfpathrectangle{\pgfqpoint{3.793912in}{0.557870in}}{\pgfqpoint{2.446088in}{1.684734in}}%
\pgfusepath{clip}%
\pgfsetbuttcap%
\pgfsetroundjoin%
\definecolor{currentfill}{rgb}{0.298039,0.447059,0.690196}%
\pgfsetfillcolor{currentfill}%
\pgfsetlinewidth{1.003750pt}%
\definecolor{currentstroke}{rgb}{0.298039,0.447059,0.690196}%
\pgfsetstrokecolor{currentstroke}%
\pgfsetdash{}{0pt}%
\pgfpathmoveto{\pgfqpoint{5.975454in}{1.355424in}}%
\pgfpathcurveto{\pgfqpoint{5.983691in}{1.355424in}}{\pgfqpoint{5.991591in}{1.358696in}}{\pgfqpoint{5.997415in}{1.364520in}}%
\pgfpathcurveto{\pgfqpoint{6.003239in}{1.370344in}}{\pgfqpoint{6.006511in}{1.378244in}}{\pgfqpoint{6.006511in}{1.386481in}}%
\pgfpathcurveto{\pgfqpoint{6.006511in}{1.394717in}}{\pgfqpoint{6.003239in}{1.402617in}}{\pgfqpoint{5.997415in}{1.408441in}}%
\pgfpathcurveto{\pgfqpoint{5.991591in}{1.414265in}}{\pgfqpoint{5.983691in}{1.417537in}}{\pgfqpoint{5.975454in}{1.417537in}}%
\pgfpathcurveto{\pgfqpoint{5.967218in}{1.417537in}}{\pgfqpoint{5.959318in}{1.414265in}}{\pgfqpoint{5.953494in}{1.408441in}}%
\pgfpathcurveto{\pgfqpoint{5.947670in}{1.402617in}}{\pgfqpoint{5.944398in}{1.394717in}}{\pgfqpoint{5.944398in}{1.386481in}}%
\pgfpathcurveto{\pgfqpoint{5.944398in}{1.378244in}}{\pgfqpoint{5.947670in}{1.370344in}}{\pgfqpoint{5.953494in}{1.364520in}}%
\pgfpathcurveto{\pgfqpoint{5.959318in}{1.358696in}}{\pgfqpoint{5.967218in}{1.355424in}}{\pgfqpoint{5.975454in}{1.355424in}}%
\pgfpathclose%
\pgfusepath{stroke,fill}%
\end{pgfscope}%
\begin{pgfscope}%
\pgfpathrectangle{\pgfqpoint{3.793912in}{0.557870in}}{\pgfqpoint{2.446088in}{1.684734in}}%
\pgfusepath{clip}%
\pgfsetbuttcap%
\pgfsetroundjoin%
\definecolor{currentfill}{rgb}{0.298039,0.447059,0.690196}%
\pgfsetfillcolor{currentfill}%
\pgfsetlinewidth{1.003750pt}%
\definecolor{currentstroke}{rgb}{0.298039,0.447059,0.690196}%
\pgfsetstrokecolor{currentstroke}%
\pgfsetdash{}{0pt}%
\pgfpathmoveto{\pgfqpoint{5.975454in}{1.373766in}}%
\pgfpathcurveto{\pgfqpoint{5.983691in}{1.373766in}}{\pgfqpoint{5.991591in}{1.377039in}}{\pgfqpoint{5.997415in}{1.382863in}}%
\pgfpathcurveto{\pgfqpoint{6.003239in}{1.388686in}}{\pgfqpoint{6.006511in}{1.396586in}}{\pgfqpoint{6.006511in}{1.404823in}}%
\pgfpathcurveto{\pgfqpoint{6.006511in}{1.413059in}}{\pgfqpoint{6.003239in}{1.420959in}}{\pgfqpoint{5.997415in}{1.426783in}}%
\pgfpathcurveto{\pgfqpoint{5.991591in}{1.432607in}}{\pgfqpoint{5.983691in}{1.435879in}}{\pgfqpoint{5.975454in}{1.435879in}}%
\pgfpathcurveto{\pgfqpoint{5.967218in}{1.435879in}}{\pgfqpoint{5.959318in}{1.432607in}}{\pgfqpoint{5.953494in}{1.426783in}}%
\pgfpathcurveto{\pgfqpoint{5.947670in}{1.420959in}}{\pgfqpoint{5.944398in}{1.413059in}}{\pgfqpoint{5.944398in}{1.404823in}}%
\pgfpathcurveto{\pgfqpoint{5.944398in}{1.396586in}}{\pgfqpoint{5.947670in}{1.388686in}}{\pgfqpoint{5.953494in}{1.382863in}}%
\pgfpathcurveto{\pgfqpoint{5.959318in}{1.377039in}}{\pgfqpoint{5.967218in}{1.373766in}}{\pgfqpoint{5.975454in}{1.373766in}}%
\pgfpathclose%
\pgfusepath{stroke,fill}%
\end{pgfscope}%
\begin{pgfscope}%
\pgfpathrectangle{\pgfqpoint{3.793912in}{0.557870in}}{\pgfqpoint{2.446088in}{1.684734in}}%
\pgfusepath{clip}%
\pgfsetbuttcap%
\pgfsetroundjoin%
\definecolor{currentfill}{rgb}{0.298039,0.447059,0.690196}%
\pgfsetfillcolor{currentfill}%
\pgfsetlinewidth{1.003750pt}%
\definecolor{currentstroke}{rgb}{0.298039,0.447059,0.690196}%
\pgfsetstrokecolor{currentstroke}%
\pgfsetdash{}{0pt}%
\pgfpathmoveto{\pgfqpoint{3.905098in}{2.125798in}}%
\pgfpathcurveto{\pgfqpoint{3.913334in}{2.125798in}}{\pgfqpoint{3.921234in}{2.129070in}}{\pgfqpoint{3.927058in}{2.134894in}}%
\pgfpathcurveto{\pgfqpoint{3.932882in}{2.140718in}}{\pgfqpoint{3.936155in}{2.148618in}}{\pgfqpoint{3.936155in}{2.156854in}}%
\pgfpathcurveto{\pgfqpoint{3.936155in}{2.165091in}}{\pgfqpoint{3.932882in}{2.172991in}}{\pgfqpoint{3.927058in}{2.178814in}}%
\pgfpathcurveto{\pgfqpoint{3.921234in}{2.184638in}}{\pgfqpoint{3.913334in}{2.187911in}}{\pgfqpoint{3.905098in}{2.187911in}}%
\pgfpathcurveto{\pgfqpoint{3.896862in}{2.187911in}}{\pgfqpoint{3.888962in}{2.184638in}}{\pgfqpoint{3.883138in}{2.178814in}}%
\pgfpathcurveto{\pgfqpoint{3.877314in}{2.172991in}}{\pgfqpoint{3.874042in}{2.165091in}}{\pgfqpoint{3.874042in}{2.156854in}}%
\pgfpathcurveto{\pgfqpoint{3.874042in}{2.148618in}}{\pgfqpoint{3.877314in}{2.140718in}}{\pgfqpoint{3.883138in}{2.134894in}}%
\pgfpathcurveto{\pgfqpoint{3.888962in}{2.129070in}}{\pgfqpoint{3.896862in}{2.125798in}}{\pgfqpoint{3.905098in}{2.125798in}}%
\pgfpathclose%
\pgfusepath{stroke,fill}%
\end{pgfscope}%
\begin{pgfscope}%
\pgfpathrectangle{\pgfqpoint{3.793912in}{0.557870in}}{\pgfqpoint{2.446088in}{1.684734in}}%
\pgfusepath{clip}%
\pgfsetbuttcap%
\pgfsetroundjoin%
\definecolor{currentfill}{rgb}{0.298039,0.447059,0.690196}%
\pgfsetfillcolor{currentfill}%
\pgfsetlinewidth{1.003750pt}%
\definecolor{currentstroke}{rgb}{0.298039,0.447059,0.690196}%
\pgfsetstrokecolor{currentstroke}%
\pgfsetdash{}{0pt}%
\pgfpathmoveto{\pgfqpoint{4.058458in}{1.575531in}}%
\pgfpathcurveto{\pgfqpoint{4.066694in}{1.575531in}}{\pgfqpoint{4.074594in}{1.578803in}}{\pgfqpoint{4.080418in}{1.584627in}}%
\pgfpathcurveto{\pgfqpoint{4.086242in}{1.590451in}}{\pgfqpoint{4.089514in}{1.598351in}}{\pgfqpoint{4.089514in}{1.606587in}}%
\pgfpathcurveto{\pgfqpoint{4.089514in}{1.614824in}}{\pgfqpoint{4.086242in}{1.622724in}}{\pgfqpoint{4.080418in}{1.628548in}}%
\pgfpathcurveto{\pgfqpoint{4.074594in}{1.634372in}}{\pgfqpoint{4.066694in}{1.637644in}}{\pgfqpoint{4.058458in}{1.637644in}}%
\pgfpathcurveto{\pgfqpoint{4.050221in}{1.637644in}}{\pgfqpoint{4.042321in}{1.634372in}}{\pgfqpoint{4.036498in}{1.628548in}}%
\pgfpathcurveto{\pgfqpoint{4.030674in}{1.622724in}}{\pgfqpoint{4.027401in}{1.614824in}}{\pgfqpoint{4.027401in}{1.606587in}}%
\pgfpathcurveto{\pgfqpoint{4.027401in}{1.598351in}}{\pgfqpoint{4.030674in}{1.590451in}}{\pgfqpoint{4.036498in}{1.584627in}}%
\pgfpathcurveto{\pgfqpoint{4.042321in}{1.578803in}}{\pgfqpoint{4.050221in}{1.575531in}}{\pgfqpoint{4.058458in}{1.575531in}}%
\pgfpathclose%
\pgfusepath{stroke,fill}%
\end{pgfscope}%
\begin{pgfscope}%
\pgfpathrectangle{\pgfqpoint{3.793912in}{0.557870in}}{\pgfqpoint{2.446088in}{1.684734in}}%
\pgfusepath{clip}%
\pgfsetbuttcap%
\pgfsetroundjoin%
\definecolor{currentfill}{rgb}{0.298039,0.447059,0.690196}%
\pgfsetfillcolor{currentfill}%
\pgfsetlinewidth{1.003750pt}%
\definecolor{currentstroke}{rgb}{0.298039,0.447059,0.690196}%
\pgfsetstrokecolor{currentstroke}%
\pgfsetdash{}{0pt}%
\pgfpathmoveto{\pgfqpoint{3.905098in}{2.125798in}}%
\pgfpathcurveto{\pgfqpoint{3.913334in}{2.125798in}}{\pgfqpoint{3.921234in}{2.129070in}}{\pgfqpoint{3.927058in}{2.134894in}}%
\pgfpathcurveto{\pgfqpoint{3.932882in}{2.140718in}}{\pgfqpoint{3.936155in}{2.148618in}}{\pgfqpoint{3.936155in}{2.156854in}}%
\pgfpathcurveto{\pgfqpoint{3.936155in}{2.165091in}}{\pgfqpoint{3.932882in}{2.172991in}}{\pgfqpoint{3.927058in}{2.178814in}}%
\pgfpathcurveto{\pgfqpoint{3.921234in}{2.184638in}}{\pgfqpoint{3.913334in}{2.187911in}}{\pgfqpoint{3.905098in}{2.187911in}}%
\pgfpathcurveto{\pgfqpoint{3.896862in}{2.187911in}}{\pgfqpoint{3.888962in}{2.184638in}}{\pgfqpoint{3.883138in}{2.178814in}}%
\pgfpathcurveto{\pgfqpoint{3.877314in}{2.172991in}}{\pgfqpoint{3.874042in}{2.165091in}}{\pgfqpoint{3.874042in}{2.156854in}}%
\pgfpathcurveto{\pgfqpoint{3.874042in}{2.148618in}}{\pgfqpoint{3.877314in}{2.140718in}}{\pgfqpoint{3.883138in}{2.134894in}}%
\pgfpathcurveto{\pgfqpoint{3.888962in}{2.129070in}}{\pgfqpoint{3.896862in}{2.125798in}}{\pgfqpoint{3.905098in}{2.125798in}}%
\pgfpathclose%
\pgfusepath{stroke,fill}%
\end{pgfscope}%
\begin{pgfscope}%
\pgfpathrectangle{\pgfqpoint{3.793912in}{0.557870in}}{\pgfqpoint{2.446088in}{1.684734in}}%
\pgfusepath{clip}%
\pgfsetbuttcap%
\pgfsetroundjoin%
\definecolor{currentfill}{rgb}{0.298039,0.447059,0.690196}%
\pgfsetfillcolor{currentfill}%
\pgfsetlinewidth{1.003750pt}%
\definecolor{currentstroke}{rgb}{0.298039,0.447059,0.690196}%
\pgfsetstrokecolor{currentstroke}%
\pgfsetdash{}{0pt}%
\pgfpathmoveto{\pgfqpoint{4.058458in}{1.905691in}}%
\pgfpathcurveto{\pgfqpoint{4.066694in}{1.905691in}}{\pgfqpoint{4.074594in}{1.908963in}}{\pgfqpoint{4.080418in}{1.914787in}}%
\pgfpathcurveto{\pgfqpoint{4.086242in}{1.920611in}}{\pgfqpoint{4.089514in}{1.928511in}}{\pgfqpoint{4.089514in}{1.936747in}}%
\pgfpathcurveto{\pgfqpoint{4.089514in}{1.944984in}}{\pgfqpoint{4.086242in}{1.952884in}}{\pgfqpoint{4.080418in}{1.958708in}}%
\pgfpathcurveto{\pgfqpoint{4.074594in}{1.964532in}}{\pgfqpoint{4.066694in}{1.967804in}}{\pgfqpoint{4.058458in}{1.967804in}}%
\pgfpathcurveto{\pgfqpoint{4.050221in}{1.967804in}}{\pgfqpoint{4.042321in}{1.964532in}}{\pgfqpoint{4.036498in}{1.958708in}}%
\pgfpathcurveto{\pgfqpoint{4.030674in}{1.952884in}}{\pgfqpoint{4.027401in}{1.944984in}}{\pgfqpoint{4.027401in}{1.936747in}}%
\pgfpathcurveto{\pgfqpoint{4.027401in}{1.928511in}}{\pgfqpoint{4.030674in}{1.920611in}}{\pgfqpoint{4.036498in}{1.914787in}}%
\pgfpathcurveto{\pgfqpoint{4.042321in}{1.908963in}}{\pgfqpoint{4.050221in}{1.905691in}}{\pgfqpoint{4.058458in}{1.905691in}}%
\pgfpathclose%
\pgfusepath{stroke,fill}%
\end{pgfscope}%
\begin{pgfscope}%
\pgfpathrectangle{\pgfqpoint{3.793912in}{0.557870in}}{\pgfqpoint{2.446088in}{1.684734in}}%
\pgfusepath{clip}%
\pgfsetbuttcap%
\pgfsetroundjoin%
\definecolor{currentfill}{rgb}{0.298039,0.447059,0.690196}%
\pgfsetfillcolor{currentfill}%
\pgfsetlinewidth{1.003750pt}%
\definecolor{currentstroke}{rgb}{0.298039,0.447059,0.690196}%
\pgfsetstrokecolor{currentstroke}%
\pgfsetdash{}{0pt}%
\pgfpathmoveto{\pgfqpoint{3.905098in}{2.116627in}}%
\pgfpathcurveto{\pgfqpoint{3.913334in}{2.116627in}}{\pgfqpoint{3.921234in}{2.119899in}}{\pgfqpoint{3.927058in}{2.125723in}}%
\pgfpathcurveto{\pgfqpoint{3.932882in}{2.131547in}}{\pgfqpoint{3.936155in}{2.139447in}}{\pgfqpoint{3.936155in}{2.147683in}}%
\pgfpathcurveto{\pgfqpoint{3.936155in}{2.155919in}}{\pgfqpoint{3.932882in}{2.163819in}}{\pgfqpoint{3.927058in}{2.169643in}}%
\pgfpathcurveto{\pgfqpoint{3.921234in}{2.175467in}}{\pgfqpoint{3.913334in}{2.178740in}}{\pgfqpoint{3.905098in}{2.178740in}}%
\pgfpathcurveto{\pgfqpoint{3.896862in}{2.178740in}}{\pgfqpoint{3.888962in}{2.175467in}}{\pgfqpoint{3.883138in}{2.169643in}}%
\pgfpathcurveto{\pgfqpoint{3.877314in}{2.163819in}}{\pgfqpoint{3.874042in}{2.155919in}}{\pgfqpoint{3.874042in}{2.147683in}}%
\pgfpathcurveto{\pgfqpoint{3.874042in}{2.139447in}}{\pgfqpoint{3.877314in}{2.131547in}}{\pgfqpoint{3.883138in}{2.125723in}}%
\pgfpathcurveto{\pgfqpoint{3.888962in}{2.119899in}}{\pgfqpoint{3.896862in}{2.116627in}}{\pgfqpoint{3.905098in}{2.116627in}}%
\pgfpathclose%
\pgfusepath{stroke,fill}%
\end{pgfscope}%
\begin{pgfscope}%
\pgfpathrectangle{\pgfqpoint{3.793912in}{0.557870in}}{\pgfqpoint{2.446088in}{1.684734in}}%
\pgfusepath{clip}%
\pgfsetbuttcap%
\pgfsetroundjoin%
\definecolor{currentfill}{rgb}{0.298039,0.447059,0.690196}%
\pgfsetfillcolor{currentfill}%
\pgfsetlinewidth{1.003750pt}%
\definecolor{currentstroke}{rgb}{0.298039,0.447059,0.690196}%
\pgfsetstrokecolor{currentstroke}%
\pgfsetdash{}{0pt}%
\pgfpathmoveto{\pgfqpoint{3.905098in}{1.777295in}}%
\pgfpathcurveto{\pgfqpoint{3.913334in}{1.777295in}}{\pgfqpoint{3.921234in}{1.780568in}}{\pgfqpoint{3.927058in}{1.786392in}}%
\pgfpathcurveto{\pgfqpoint{3.932882in}{1.792216in}}{\pgfqpoint{3.936155in}{1.800116in}}{\pgfqpoint{3.936155in}{1.808352in}}%
\pgfpathcurveto{\pgfqpoint{3.936155in}{1.816588in}}{\pgfqpoint{3.932882in}{1.824488in}}{\pgfqpoint{3.927058in}{1.830312in}}%
\pgfpathcurveto{\pgfqpoint{3.921234in}{1.836136in}}{\pgfqpoint{3.913334in}{1.839408in}}{\pgfqpoint{3.905098in}{1.839408in}}%
\pgfpathcurveto{\pgfqpoint{3.896862in}{1.839408in}}{\pgfqpoint{3.888962in}{1.836136in}}{\pgfqpoint{3.883138in}{1.830312in}}%
\pgfpathcurveto{\pgfqpoint{3.877314in}{1.824488in}}{\pgfqpoint{3.874042in}{1.816588in}}{\pgfqpoint{3.874042in}{1.808352in}}%
\pgfpathcurveto{\pgfqpoint{3.874042in}{1.800116in}}{\pgfqpoint{3.877314in}{1.792216in}}{\pgfqpoint{3.883138in}{1.786392in}}%
\pgfpathcurveto{\pgfqpoint{3.888962in}{1.780568in}}{\pgfqpoint{3.896862in}{1.777295in}}{\pgfqpoint{3.905098in}{1.777295in}}%
\pgfpathclose%
\pgfusepath{stroke,fill}%
\end{pgfscope}%
\begin{pgfscope}%
\pgfpathrectangle{\pgfqpoint{3.793912in}{0.557870in}}{\pgfqpoint{2.446088in}{1.684734in}}%
\pgfusepath{clip}%
\pgfsetbuttcap%
\pgfsetroundjoin%
\definecolor{currentfill}{rgb}{0.298039,0.447059,0.690196}%
\pgfsetfillcolor{currentfill}%
\pgfsetlinewidth{1.003750pt}%
\definecolor{currentstroke}{rgb}{0.298039,0.447059,0.690196}%
\pgfsetstrokecolor{currentstroke}%
\pgfsetdash{}{0pt}%
\pgfpathmoveto{\pgfqpoint{3.905098in}{2.125798in}}%
\pgfpathcurveto{\pgfqpoint{3.913334in}{2.125798in}}{\pgfqpoint{3.921234in}{2.129070in}}{\pgfqpoint{3.927058in}{2.134894in}}%
\pgfpathcurveto{\pgfqpoint{3.932882in}{2.140718in}}{\pgfqpoint{3.936155in}{2.148618in}}{\pgfqpoint{3.936155in}{2.156854in}}%
\pgfpathcurveto{\pgfqpoint{3.936155in}{2.165091in}}{\pgfqpoint{3.932882in}{2.172991in}}{\pgfqpoint{3.927058in}{2.178814in}}%
\pgfpathcurveto{\pgfqpoint{3.921234in}{2.184638in}}{\pgfqpoint{3.913334in}{2.187911in}}{\pgfqpoint{3.905098in}{2.187911in}}%
\pgfpathcurveto{\pgfqpoint{3.896862in}{2.187911in}}{\pgfqpoint{3.888962in}{2.184638in}}{\pgfqpoint{3.883138in}{2.178814in}}%
\pgfpathcurveto{\pgfqpoint{3.877314in}{2.172991in}}{\pgfqpoint{3.874042in}{2.165091in}}{\pgfqpoint{3.874042in}{2.156854in}}%
\pgfpathcurveto{\pgfqpoint{3.874042in}{2.148618in}}{\pgfqpoint{3.877314in}{2.140718in}}{\pgfqpoint{3.883138in}{2.134894in}}%
\pgfpathcurveto{\pgfqpoint{3.888962in}{2.129070in}}{\pgfqpoint{3.896862in}{2.125798in}}{\pgfqpoint{3.905098in}{2.125798in}}%
\pgfpathclose%
\pgfusepath{stroke,fill}%
\end{pgfscope}%
\begin{pgfscope}%
\pgfpathrectangle{\pgfqpoint{3.793912in}{0.557870in}}{\pgfqpoint{2.446088in}{1.684734in}}%
\pgfusepath{clip}%
\pgfsetbuttcap%
\pgfsetroundjoin%
\definecolor{currentfill}{rgb}{0.298039,0.447059,0.690196}%
\pgfsetfillcolor{currentfill}%
\pgfsetlinewidth{1.003750pt}%
\definecolor{currentstroke}{rgb}{0.298039,0.447059,0.690196}%
\pgfsetstrokecolor{currentstroke}%
\pgfsetdash{}{0pt}%
\pgfpathmoveto{\pgfqpoint{3.905098in}{2.125798in}}%
\pgfpathcurveto{\pgfqpoint{3.913334in}{2.125798in}}{\pgfqpoint{3.921234in}{2.129070in}}{\pgfqpoint{3.927058in}{2.134894in}}%
\pgfpathcurveto{\pgfqpoint{3.932882in}{2.140718in}}{\pgfqpoint{3.936155in}{2.148618in}}{\pgfqpoint{3.936155in}{2.156854in}}%
\pgfpathcurveto{\pgfqpoint{3.936155in}{2.165091in}}{\pgfqpoint{3.932882in}{2.172991in}}{\pgfqpoint{3.927058in}{2.178814in}}%
\pgfpathcurveto{\pgfqpoint{3.921234in}{2.184638in}}{\pgfqpoint{3.913334in}{2.187911in}}{\pgfqpoint{3.905098in}{2.187911in}}%
\pgfpathcurveto{\pgfqpoint{3.896862in}{2.187911in}}{\pgfqpoint{3.888962in}{2.184638in}}{\pgfqpoint{3.883138in}{2.178814in}}%
\pgfpathcurveto{\pgfqpoint{3.877314in}{2.172991in}}{\pgfqpoint{3.874042in}{2.165091in}}{\pgfqpoint{3.874042in}{2.156854in}}%
\pgfpathcurveto{\pgfqpoint{3.874042in}{2.148618in}}{\pgfqpoint{3.877314in}{2.140718in}}{\pgfqpoint{3.883138in}{2.134894in}}%
\pgfpathcurveto{\pgfqpoint{3.888962in}{2.129070in}}{\pgfqpoint{3.896862in}{2.125798in}}{\pgfqpoint{3.905098in}{2.125798in}}%
\pgfpathclose%
\pgfusepath{stroke,fill}%
\end{pgfscope}%
\begin{pgfscope}%
\pgfpathrectangle{\pgfqpoint{3.793912in}{0.557870in}}{\pgfqpoint{2.446088in}{1.684734in}}%
\pgfusepath{clip}%
\pgfsetbuttcap%
\pgfsetroundjoin%
\definecolor{currentfill}{rgb}{0.298039,0.447059,0.690196}%
\pgfsetfillcolor{currentfill}%
\pgfsetlinewidth{1.003750pt}%
\definecolor{currentstroke}{rgb}{0.298039,0.447059,0.690196}%
\pgfsetstrokecolor{currentstroke}%
\pgfsetdash{}{0pt}%
\pgfpathmoveto{\pgfqpoint{3.905098in}{2.125798in}}%
\pgfpathcurveto{\pgfqpoint{3.913334in}{2.125798in}}{\pgfqpoint{3.921234in}{2.129070in}}{\pgfqpoint{3.927058in}{2.134894in}}%
\pgfpathcurveto{\pgfqpoint{3.932882in}{2.140718in}}{\pgfqpoint{3.936155in}{2.148618in}}{\pgfqpoint{3.936155in}{2.156854in}}%
\pgfpathcurveto{\pgfqpoint{3.936155in}{2.165091in}}{\pgfqpoint{3.932882in}{2.172991in}}{\pgfqpoint{3.927058in}{2.178814in}}%
\pgfpathcurveto{\pgfqpoint{3.921234in}{2.184638in}}{\pgfqpoint{3.913334in}{2.187911in}}{\pgfqpoint{3.905098in}{2.187911in}}%
\pgfpathcurveto{\pgfqpoint{3.896862in}{2.187911in}}{\pgfqpoint{3.888962in}{2.184638in}}{\pgfqpoint{3.883138in}{2.178814in}}%
\pgfpathcurveto{\pgfqpoint{3.877314in}{2.172991in}}{\pgfqpoint{3.874042in}{2.165091in}}{\pgfqpoint{3.874042in}{2.156854in}}%
\pgfpathcurveto{\pgfqpoint{3.874042in}{2.148618in}}{\pgfqpoint{3.877314in}{2.140718in}}{\pgfqpoint{3.883138in}{2.134894in}}%
\pgfpathcurveto{\pgfqpoint{3.888962in}{2.129070in}}{\pgfqpoint{3.896862in}{2.125798in}}{\pgfqpoint{3.905098in}{2.125798in}}%
\pgfpathclose%
\pgfusepath{stroke,fill}%
\end{pgfscope}%
\begin{pgfscope}%
\pgfpathrectangle{\pgfqpoint{3.793912in}{0.557870in}}{\pgfqpoint{2.446088in}{1.684734in}}%
\pgfusepath{clip}%
\pgfsetbuttcap%
\pgfsetroundjoin%
\definecolor{currentfill}{rgb}{0.298039,0.447059,0.690196}%
\pgfsetfillcolor{currentfill}%
\pgfsetlinewidth{1.003750pt}%
\definecolor{currentstroke}{rgb}{0.298039,0.447059,0.690196}%
\pgfsetstrokecolor{currentstroke}%
\pgfsetdash{}{0pt}%
\pgfpathmoveto{\pgfqpoint{3.905098in}{2.125798in}}%
\pgfpathcurveto{\pgfqpoint{3.913334in}{2.125798in}}{\pgfqpoint{3.921234in}{2.129070in}}{\pgfqpoint{3.927058in}{2.134894in}}%
\pgfpathcurveto{\pgfqpoint{3.932882in}{2.140718in}}{\pgfqpoint{3.936155in}{2.148618in}}{\pgfqpoint{3.936155in}{2.156854in}}%
\pgfpathcurveto{\pgfqpoint{3.936155in}{2.165091in}}{\pgfqpoint{3.932882in}{2.172991in}}{\pgfqpoint{3.927058in}{2.178814in}}%
\pgfpathcurveto{\pgfqpoint{3.921234in}{2.184638in}}{\pgfqpoint{3.913334in}{2.187911in}}{\pgfqpoint{3.905098in}{2.187911in}}%
\pgfpathcurveto{\pgfqpoint{3.896862in}{2.187911in}}{\pgfqpoint{3.888962in}{2.184638in}}{\pgfqpoint{3.883138in}{2.178814in}}%
\pgfpathcurveto{\pgfqpoint{3.877314in}{2.172991in}}{\pgfqpoint{3.874042in}{2.165091in}}{\pgfqpoint{3.874042in}{2.156854in}}%
\pgfpathcurveto{\pgfqpoint{3.874042in}{2.148618in}}{\pgfqpoint{3.877314in}{2.140718in}}{\pgfqpoint{3.883138in}{2.134894in}}%
\pgfpathcurveto{\pgfqpoint{3.888962in}{2.129070in}}{\pgfqpoint{3.896862in}{2.125798in}}{\pgfqpoint{3.905098in}{2.125798in}}%
\pgfpathclose%
\pgfusepath{stroke,fill}%
\end{pgfscope}%
\begin{pgfscope}%
\pgfpathrectangle{\pgfqpoint{3.793912in}{0.557870in}}{\pgfqpoint{2.446088in}{1.684734in}}%
\pgfusepath{clip}%
\pgfsetbuttcap%
\pgfsetroundjoin%
\definecolor{currentfill}{rgb}{0.298039,0.447059,0.690196}%
\pgfsetfillcolor{currentfill}%
\pgfsetlinewidth{1.003750pt}%
\definecolor{currentstroke}{rgb}{0.298039,0.447059,0.690196}%
\pgfsetstrokecolor{currentstroke}%
\pgfsetdash{}{0pt}%
\pgfpathmoveto{\pgfqpoint{4.671897in}{1.428793in}}%
\pgfpathcurveto{\pgfqpoint{4.680133in}{1.428793in}}{\pgfqpoint{4.688033in}{1.432065in}}{\pgfqpoint{4.693857in}{1.437889in}}%
\pgfpathcurveto{\pgfqpoint{4.699681in}{1.443713in}}{\pgfqpoint{4.702953in}{1.451613in}}{\pgfqpoint{4.702953in}{1.459849in}}%
\pgfpathcurveto{\pgfqpoint{4.702953in}{1.468086in}}{\pgfqpoint{4.699681in}{1.475986in}}{\pgfqpoint{4.693857in}{1.481810in}}%
\pgfpathcurveto{\pgfqpoint{4.688033in}{1.487634in}}{\pgfqpoint{4.680133in}{1.490906in}}{\pgfqpoint{4.671897in}{1.490906in}}%
\pgfpathcurveto{\pgfqpoint{4.663660in}{1.490906in}}{\pgfqpoint{4.655760in}{1.487634in}}{\pgfqpoint{4.649936in}{1.481810in}}%
\pgfpathcurveto{\pgfqpoint{4.644113in}{1.475986in}}{\pgfqpoint{4.640840in}{1.468086in}}{\pgfqpoint{4.640840in}{1.459849in}}%
\pgfpathcurveto{\pgfqpoint{4.640840in}{1.451613in}}{\pgfqpoint{4.644113in}{1.443713in}}{\pgfqpoint{4.649936in}{1.437889in}}%
\pgfpathcurveto{\pgfqpoint{4.655760in}{1.432065in}}{\pgfqpoint{4.663660in}{1.428793in}}{\pgfqpoint{4.671897in}{1.428793in}}%
\pgfpathclose%
\pgfusepath{stroke,fill}%
\end{pgfscope}%
\begin{pgfscope}%
\pgfpathrectangle{\pgfqpoint{3.793912in}{0.557870in}}{\pgfqpoint{2.446088in}{1.684734in}}%
\pgfusepath{clip}%
\pgfsetbuttcap%
\pgfsetroundjoin%
\definecolor{currentfill}{rgb}{0.298039,0.447059,0.690196}%
\pgfsetfillcolor{currentfill}%
\pgfsetlinewidth{1.003750pt}%
\definecolor{currentstroke}{rgb}{0.298039,0.447059,0.690196}%
\pgfsetstrokecolor{currentstroke}%
\pgfsetdash{}{0pt}%
\pgfpathmoveto{\pgfqpoint{3.905098in}{2.116627in}}%
\pgfpathcurveto{\pgfqpoint{3.913334in}{2.116627in}}{\pgfqpoint{3.921234in}{2.119899in}}{\pgfqpoint{3.927058in}{2.125723in}}%
\pgfpathcurveto{\pgfqpoint{3.932882in}{2.131547in}}{\pgfqpoint{3.936155in}{2.139447in}}{\pgfqpoint{3.936155in}{2.147683in}}%
\pgfpathcurveto{\pgfqpoint{3.936155in}{2.155919in}}{\pgfqpoint{3.932882in}{2.163819in}}{\pgfqpoint{3.927058in}{2.169643in}}%
\pgfpathcurveto{\pgfqpoint{3.921234in}{2.175467in}}{\pgfqpoint{3.913334in}{2.178740in}}{\pgfqpoint{3.905098in}{2.178740in}}%
\pgfpathcurveto{\pgfqpoint{3.896862in}{2.178740in}}{\pgfqpoint{3.888962in}{2.175467in}}{\pgfqpoint{3.883138in}{2.169643in}}%
\pgfpathcurveto{\pgfqpoint{3.877314in}{2.163819in}}{\pgfqpoint{3.874042in}{2.155919in}}{\pgfqpoint{3.874042in}{2.147683in}}%
\pgfpathcurveto{\pgfqpoint{3.874042in}{2.139447in}}{\pgfqpoint{3.877314in}{2.131547in}}{\pgfqpoint{3.883138in}{2.125723in}}%
\pgfpathcurveto{\pgfqpoint{3.888962in}{2.119899in}}{\pgfqpoint{3.896862in}{2.116627in}}{\pgfqpoint{3.905098in}{2.116627in}}%
\pgfpathclose%
\pgfusepath{stroke,fill}%
\end{pgfscope}%
\begin{pgfscope}%
\pgfpathrectangle{\pgfqpoint{3.793912in}{0.557870in}}{\pgfqpoint{2.446088in}{1.684734in}}%
\pgfusepath{clip}%
\pgfsetbuttcap%
\pgfsetroundjoin%
\definecolor{currentfill}{rgb}{0.298039,0.447059,0.690196}%
\pgfsetfillcolor{currentfill}%
\pgfsetlinewidth{1.003750pt}%
\definecolor{currentstroke}{rgb}{0.298039,0.447059,0.690196}%
\pgfsetstrokecolor{currentstroke}%
\pgfsetdash{}{0pt}%
\pgfpathmoveto{\pgfqpoint{5.055296in}{1.456306in}}%
\pgfpathcurveto{\pgfqpoint{5.063532in}{1.456306in}}{\pgfqpoint{5.071432in}{1.459579in}}{\pgfqpoint{5.077256in}{1.465403in}}%
\pgfpathcurveto{\pgfqpoint{5.083080in}{1.471226in}}{\pgfqpoint{5.086353in}{1.479127in}}{\pgfqpoint{5.086353in}{1.487363in}}%
\pgfpathcurveto{\pgfqpoint{5.086353in}{1.495599in}}{\pgfqpoint{5.083080in}{1.503499in}}{\pgfqpoint{5.077256in}{1.509323in}}%
\pgfpathcurveto{\pgfqpoint{5.071432in}{1.515147in}}{\pgfqpoint{5.063532in}{1.518419in}}{\pgfqpoint{5.055296in}{1.518419in}}%
\pgfpathcurveto{\pgfqpoint{5.047060in}{1.518419in}}{\pgfqpoint{5.039160in}{1.515147in}}{\pgfqpoint{5.033336in}{1.509323in}}%
\pgfpathcurveto{\pgfqpoint{5.027512in}{1.503499in}}{\pgfqpoint{5.024240in}{1.495599in}}{\pgfqpoint{5.024240in}{1.487363in}}%
\pgfpathcurveto{\pgfqpoint{5.024240in}{1.479127in}}{\pgfqpoint{5.027512in}{1.471226in}}{\pgfqpoint{5.033336in}{1.465403in}}%
\pgfpathcurveto{\pgfqpoint{5.039160in}{1.459579in}}{\pgfqpoint{5.047060in}{1.456306in}}{\pgfqpoint{5.055296in}{1.456306in}}%
\pgfpathclose%
\pgfusepath{stroke,fill}%
\end{pgfscope}%
\begin{pgfscope}%
\pgfpathrectangle{\pgfqpoint{3.793912in}{0.557870in}}{\pgfqpoint{2.446088in}{1.684734in}}%
\pgfusepath{clip}%
\pgfsetbuttcap%
\pgfsetroundjoin%
\definecolor{currentfill}{rgb}{0.298039,0.447059,0.690196}%
\pgfsetfillcolor{currentfill}%
\pgfsetlinewidth{1.003750pt}%
\definecolor{currentstroke}{rgb}{0.298039,0.447059,0.690196}%
\pgfsetstrokecolor{currentstroke}%
\pgfsetdash{}{0pt}%
\pgfpathmoveto{\pgfqpoint{3.905098in}{2.125798in}}%
\pgfpathcurveto{\pgfqpoint{3.913334in}{2.125798in}}{\pgfqpoint{3.921234in}{2.129070in}}{\pgfqpoint{3.927058in}{2.134894in}}%
\pgfpathcurveto{\pgfqpoint{3.932882in}{2.140718in}}{\pgfqpoint{3.936155in}{2.148618in}}{\pgfqpoint{3.936155in}{2.156854in}}%
\pgfpathcurveto{\pgfqpoint{3.936155in}{2.165091in}}{\pgfqpoint{3.932882in}{2.172991in}}{\pgfqpoint{3.927058in}{2.178814in}}%
\pgfpathcurveto{\pgfqpoint{3.921234in}{2.184638in}}{\pgfqpoint{3.913334in}{2.187911in}}{\pgfqpoint{3.905098in}{2.187911in}}%
\pgfpathcurveto{\pgfqpoint{3.896862in}{2.187911in}}{\pgfqpoint{3.888962in}{2.184638in}}{\pgfqpoint{3.883138in}{2.178814in}}%
\pgfpathcurveto{\pgfqpoint{3.877314in}{2.172991in}}{\pgfqpoint{3.874042in}{2.165091in}}{\pgfqpoint{3.874042in}{2.156854in}}%
\pgfpathcurveto{\pgfqpoint{3.874042in}{2.148618in}}{\pgfqpoint{3.877314in}{2.140718in}}{\pgfqpoint{3.883138in}{2.134894in}}%
\pgfpathcurveto{\pgfqpoint{3.888962in}{2.129070in}}{\pgfqpoint{3.896862in}{2.125798in}}{\pgfqpoint{3.905098in}{2.125798in}}%
\pgfpathclose%
\pgfusepath{stroke,fill}%
\end{pgfscope}%
\begin{pgfscope}%
\pgfpathrectangle{\pgfqpoint{3.793912in}{0.557870in}}{\pgfqpoint{2.446088in}{1.684734in}}%
\pgfusepath{clip}%
\pgfsetbuttcap%
\pgfsetroundjoin%
\definecolor{currentfill}{rgb}{0.298039,0.447059,0.690196}%
\pgfsetfillcolor{currentfill}%
\pgfsetlinewidth{1.003750pt}%
\definecolor{currentstroke}{rgb}{0.298039,0.447059,0.690196}%
\pgfsetstrokecolor{currentstroke}%
\pgfsetdash{}{0pt}%
\pgfpathmoveto{\pgfqpoint{3.905098in}{2.116627in}}%
\pgfpathcurveto{\pgfqpoint{3.913334in}{2.116627in}}{\pgfqpoint{3.921234in}{2.119899in}}{\pgfqpoint{3.927058in}{2.125723in}}%
\pgfpathcurveto{\pgfqpoint{3.932882in}{2.131547in}}{\pgfqpoint{3.936155in}{2.139447in}}{\pgfqpoint{3.936155in}{2.147683in}}%
\pgfpathcurveto{\pgfqpoint{3.936155in}{2.155919in}}{\pgfqpoint{3.932882in}{2.163819in}}{\pgfqpoint{3.927058in}{2.169643in}}%
\pgfpathcurveto{\pgfqpoint{3.921234in}{2.175467in}}{\pgfqpoint{3.913334in}{2.178740in}}{\pgfqpoint{3.905098in}{2.178740in}}%
\pgfpathcurveto{\pgfqpoint{3.896862in}{2.178740in}}{\pgfqpoint{3.888962in}{2.175467in}}{\pgfqpoint{3.883138in}{2.169643in}}%
\pgfpathcurveto{\pgfqpoint{3.877314in}{2.163819in}}{\pgfqpoint{3.874042in}{2.155919in}}{\pgfqpoint{3.874042in}{2.147683in}}%
\pgfpathcurveto{\pgfqpoint{3.874042in}{2.139447in}}{\pgfqpoint{3.877314in}{2.131547in}}{\pgfqpoint{3.883138in}{2.125723in}}%
\pgfpathcurveto{\pgfqpoint{3.888962in}{2.119899in}}{\pgfqpoint{3.896862in}{2.116627in}}{\pgfqpoint{3.905098in}{2.116627in}}%
\pgfpathclose%
\pgfusepath{stroke,fill}%
\end{pgfscope}%
\begin{pgfscope}%
\pgfpathrectangle{\pgfqpoint{3.793912in}{0.557870in}}{\pgfqpoint{2.446088in}{1.684734in}}%
\pgfusepath{clip}%
\pgfsetbuttcap%
\pgfsetroundjoin%
\definecolor{currentfill}{rgb}{0.298039,0.447059,0.690196}%
\pgfsetfillcolor{currentfill}%
\pgfsetlinewidth{1.003750pt}%
\definecolor{currentstroke}{rgb}{0.298039,0.447059,0.690196}%
\pgfsetstrokecolor{currentstroke}%
\pgfsetdash{}{0pt}%
\pgfpathmoveto{\pgfqpoint{3.905098in}{2.125798in}}%
\pgfpathcurveto{\pgfqpoint{3.913334in}{2.125798in}}{\pgfqpoint{3.921234in}{2.129070in}}{\pgfqpoint{3.927058in}{2.134894in}}%
\pgfpathcurveto{\pgfqpoint{3.932882in}{2.140718in}}{\pgfqpoint{3.936155in}{2.148618in}}{\pgfqpoint{3.936155in}{2.156854in}}%
\pgfpathcurveto{\pgfqpoint{3.936155in}{2.165091in}}{\pgfqpoint{3.932882in}{2.172991in}}{\pgfqpoint{3.927058in}{2.178814in}}%
\pgfpathcurveto{\pgfqpoint{3.921234in}{2.184638in}}{\pgfqpoint{3.913334in}{2.187911in}}{\pgfqpoint{3.905098in}{2.187911in}}%
\pgfpathcurveto{\pgfqpoint{3.896862in}{2.187911in}}{\pgfqpoint{3.888962in}{2.184638in}}{\pgfqpoint{3.883138in}{2.178814in}}%
\pgfpathcurveto{\pgfqpoint{3.877314in}{2.172991in}}{\pgfqpoint{3.874042in}{2.165091in}}{\pgfqpoint{3.874042in}{2.156854in}}%
\pgfpathcurveto{\pgfqpoint{3.874042in}{2.148618in}}{\pgfqpoint{3.877314in}{2.140718in}}{\pgfqpoint{3.883138in}{2.134894in}}%
\pgfpathcurveto{\pgfqpoint{3.888962in}{2.129070in}}{\pgfqpoint{3.896862in}{2.125798in}}{\pgfqpoint{3.905098in}{2.125798in}}%
\pgfpathclose%
\pgfusepath{stroke,fill}%
\end{pgfscope}%
\begin{pgfscope}%
\pgfpathrectangle{\pgfqpoint{3.793912in}{0.557870in}}{\pgfqpoint{2.446088in}{1.684734in}}%
\pgfusepath{clip}%
\pgfsetbuttcap%
\pgfsetroundjoin%
\definecolor{currentfill}{rgb}{0.298039,0.447059,0.690196}%
\pgfsetfillcolor{currentfill}%
\pgfsetlinewidth{1.003750pt}%
\definecolor{currentstroke}{rgb}{0.298039,0.447059,0.690196}%
\pgfsetstrokecolor{currentstroke}%
\pgfsetdash{}{0pt}%
\pgfpathmoveto{\pgfqpoint{5.975454in}{1.263713in}}%
\pgfpathcurveto{\pgfqpoint{5.983691in}{1.263713in}}{\pgfqpoint{5.991591in}{1.266985in}}{\pgfqpoint{5.997415in}{1.272809in}}%
\pgfpathcurveto{\pgfqpoint{6.003239in}{1.278633in}}{\pgfqpoint{6.006511in}{1.286533in}}{\pgfqpoint{6.006511in}{1.294769in}}%
\pgfpathcurveto{\pgfqpoint{6.006511in}{1.303006in}}{\pgfqpoint{6.003239in}{1.310906in}}{\pgfqpoint{5.997415in}{1.316730in}}%
\pgfpathcurveto{\pgfqpoint{5.991591in}{1.322554in}}{\pgfqpoint{5.983691in}{1.325826in}}{\pgfqpoint{5.975454in}{1.325826in}}%
\pgfpathcurveto{\pgfqpoint{5.967218in}{1.325826in}}{\pgfqpoint{5.959318in}{1.322554in}}{\pgfqpoint{5.953494in}{1.316730in}}%
\pgfpathcurveto{\pgfqpoint{5.947670in}{1.310906in}}{\pgfqpoint{5.944398in}{1.303006in}}{\pgfqpoint{5.944398in}{1.294769in}}%
\pgfpathcurveto{\pgfqpoint{5.944398in}{1.286533in}}{\pgfqpoint{5.947670in}{1.278633in}}{\pgfqpoint{5.953494in}{1.272809in}}%
\pgfpathcurveto{\pgfqpoint{5.959318in}{1.266985in}}{\pgfqpoint{5.967218in}{1.263713in}}{\pgfqpoint{5.975454in}{1.263713in}}%
\pgfpathclose%
\pgfusepath{stroke,fill}%
\end{pgfscope}%
\begin{pgfscope}%
\pgfpathrectangle{\pgfqpoint{3.793912in}{0.557870in}}{\pgfqpoint{2.446088in}{1.684734in}}%
\pgfusepath{clip}%
\pgfsetbuttcap%
\pgfsetroundjoin%
\definecolor{currentfill}{rgb}{0.298039,0.447059,0.690196}%
\pgfsetfillcolor{currentfill}%
\pgfsetlinewidth{1.003750pt}%
\definecolor{currentstroke}{rgb}{0.298039,0.447059,0.690196}%
\pgfsetstrokecolor{currentstroke}%
\pgfsetdash{}{0pt}%
\pgfpathmoveto{\pgfqpoint{5.975454in}{1.346253in}}%
\pgfpathcurveto{\pgfqpoint{5.983691in}{1.346253in}}{\pgfqpoint{5.991591in}{1.349525in}}{\pgfqpoint{5.997415in}{1.355349in}}%
\pgfpathcurveto{\pgfqpoint{6.003239in}{1.361173in}}{\pgfqpoint{6.006511in}{1.369073in}}{\pgfqpoint{6.006511in}{1.377309in}}%
\pgfpathcurveto{\pgfqpoint{6.006511in}{1.385546in}}{\pgfqpoint{6.003239in}{1.393446in}}{\pgfqpoint{5.997415in}{1.399270in}}%
\pgfpathcurveto{\pgfqpoint{5.991591in}{1.405094in}}{\pgfqpoint{5.983691in}{1.408366in}}{\pgfqpoint{5.975454in}{1.408366in}}%
\pgfpathcurveto{\pgfqpoint{5.967218in}{1.408366in}}{\pgfqpoint{5.959318in}{1.405094in}}{\pgfqpoint{5.953494in}{1.399270in}}%
\pgfpathcurveto{\pgfqpoint{5.947670in}{1.393446in}}{\pgfqpoint{5.944398in}{1.385546in}}{\pgfqpoint{5.944398in}{1.377309in}}%
\pgfpathcurveto{\pgfqpoint{5.944398in}{1.369073in}}{\pgfqpoint{5.947670in}{1.361173in}}{\pgfqpoint{5.953494in}{1.355349in}}%
\pgfpathcurveto{\pgfqpoint{5.959318in}{1.349525in}}{\pgfqpoint{5.967218in}{1.346253in}}{\pgfqpoint{5.975454in}{1.346253in}}%
\pgfpathclose%
\pgfusepath{stroke,fill}%
\end{pgfscope}%
\begin{pgfscope}%
\pgfpathrectangle{\pgfqpoint{3.793912in}{0.557870in}}{\pgfqpoint{2.446088in}{1.684734in}}%
\pgfusepath{clip}%
\pgfsetbuttcap%
\pgfsetroundjoin%
\definecolor{currentfill}{rgb}{0.298039,0.447059,0.690196}%
\pgfsetfillcolor{currentfill}%
\pgfsetlinewidth{1.003750pt}%
\definecolor{currentstroke}{rgb}{0.298039,0.447059,0.690196}%
\pgfsetstrokecolor{currentstroke}%
\pgfsetdash{}{0pt}%
\pgfpathmoveto{\pgfqpoint{5.975454in}{1.282055in}}%
\pgfpathcurveto{\pgfqpoint{5.983691in}{1.282055in}}{\pgfqpoint{5.991591in}{1.285327in}}{\pgfqpoint{5.997415in}{1.291151in}}%
\pgfpathcurveto{\pgfqpoint{6.003239in}{1.296975in}}{\pgfqpoint{6.006511in}{1.304875in}}{\pgfqpoint{6.006511in}{1.313112in}}%
\pgfpathcurveto{\pgfqpoint{6.006511in}{1.321348in}}{\pgfqpoint{6.003239in}{1.329248in}}{\pgfqpoint{5.997415in}{1.335072in}}%
\pgfpathcurveto{\pgfqpoint{5.991591in}{1.340896in}}{\pgfqpoint{5.983691in}{1.344168in}}{\pgfqpoint{5.975454in}{1.344168in}}%
\pgfpathcurveto{\pgfqpoint{5.967218in}{1.344168in}}{\pgfqpoint{5.959318in}{1.340896in}}{\pgfqpoint{5.953494in}{1.335072in}}%
\pgfpathcurveto{\pgfqpoint{5.947670in}{1.329248in}}{\pgfqpoint{5.944398in}{1.321348in}}{\pgfqpoint{5.944398in}{1.313112in}}%
\pgfpathcurveto{\pgfqpoint{5.944398in}{1.304875in}}{\pgfqpoint{5.947670in}{1.296975in}}{\pgfqpoint{5.953494in}{1.291151in}}%
\pgfpathcurveto{\pgfqpoint{5.959318in}{1.285327in}}{\pgfqpoint{5.967218in}{1.282055in}}{\pgfqpoint{5.975454in}{1.282055in}}%
\pgfpathclose%
\pgfusepath{stroke,fill}%
\end{pgfscope}%
\begin{pgfscope}%
\pgfpathrectangle{\pgfqpoint{3.793912in}{0.557870in}}{\pgfqpoint{2.446088in}{1.684734in}}%
\pgfusepath{clip}%
\pgfsetbuttcap%
\pgfsetroundjoin%
\definecolor{currentfill}{rgb}{0.298039,0.447059,0.690196}%
\pgfsetfillcolor{currentfill}%
\pgfsetlinewidth{1.003750pt}%
\definecolor{currentstroke}{rgb}{0.298039,0.447059,0.690196}%
\pgfsetstrokecolor{currentstroke}%
\pgfsetdash{}{0pt}%
\pgfpathmoveto{\pgfqpoint{5.975454in}{1.364595in}}%
\pgfpathcurveto{\pgfqpoint{5.983691in}{1.364595in}}{\pgfqpoint{5.991591in}{1.367867in}}{\pgfqpoint{5.997415in}{1.373691in}}%
\pgfpathcurveto{\pgfqpoint{6.003239in}{1.379515in}}{\pgfqpoint{6.006511in}{1.387415in}}{\pgfqpoint{6.006511in}{1.395652in}}%
\pgfpathcurveto{\pgfqpoint{6.006511in}{1.403888in}}{\pgfqpoint{6.003239in}{1.411788in}}{\pgfqpoint{5.997415in}{1.417612in}}%
\pgfpathcurveto{\pgfqpoint{5.991591in}{1.423436in}}{\pgfqpoint{5.983691in}{1.426708in}}{\pgfqpoint{5.975454in}{1.426708in}}%
\pgfpathcurveto{\pgfqpoint{5.967218in}{1.426708in}}{\pgfqpoint{5.959318in}{1.423436in}}{\pgfqpoint{5.953494in}{1.417612in}}%
\pgfpathcurveto{\pgfqpoint{5.947670in}{1.411788in}}{\pgfqpoint{5.944398in}{1.403888in}}{\pgfqpoint{5.944398in}{1.395652in}}%
\pgfpathcurveto{\pgfqpoint{5.944398in}{1.387415in}}{\pgfqpoint{5.947670in}{1.379515in}}{\pgfqpoint{5.953494in}{1.373691in}}%
\pgfpathcurveto{\pgfqpoint{5.959318in}{1.367867in}}{\pgfqpoint{5.967218in}{1.364595in}}{\pgfqpoint{5.975454in}{1.364595in}}%
\pgfpathclose%
\pgfusepath{stroke,fill}%
\end{pgfscope}%
\begin{pgfscope}%
\pgfpathrectangle{\pgfqpoint{3.793912in}{0.557870in}}{\pgfqpoint{2.446088in}{1.684734in}}%
\pgfusepath{clip}%
\pgfsetbuttcap%
\pgfsetroundjoin%
\definecolor{currentfill}{rgb}{0.298039,0.447059,0.690196}%
\pgfsetfillcolor{currentfill}%
\pgfsetlinewidth{1.003750pt}%
\definecolor{currentstroke}{rgb}{0.298039,0.447059,0.690196}%
\pgfsetstrokecolor{currentstroke}%
\pgfsetdash{}{0pt}%
\pgfpathmoveto{\pgfqpoint{3.905098in}{2.125798in}}%
\pgfpathcurveto{\pgfqpoint{3.913334in}{2.125798in}}{\pgfqpoint{3.921234in}{2.129070in}}{\pgfqpoint{3.927058in}{2.134894in}}%
\pgfpathcurveto{\pgfqpoint{3.932882in}{2.140718in}}{\pgfqpoint{3.936155in}{2.148618in}}{\pgfqpoint{3.936155in}{2.156854in}}%
\pgfpathcurveto{\pgfqpoint{3.936155in}{2.165091in}}{\pgfqpoint{3.932882in}{2.172991in}}{\pgfqpoint{3.927058in}{2.178814in}}%
\pgfpathcurveto{\pgfqpoint{3.921234in}{2.184638in}}{\pgfqpoint{3.913334in}{2.187911in}}{\pgfqpoint{3.905098in}{2.187911in}}%
\pgfpathcurveto{\pgfqpoint{3.896862in}{2.187911in}}{\pgfqpoint{3.888962in}{2.184638in}}{\pgfqpoint{3.883138in}{2.178814in}}%
\pgfpathcurveto{\pgfqpoint{3.877314in}{2.172991in}}{\pgfqpoint{3.874042in}{2.165091in}}{\pgfqpoint{3.874042in}{2.156854in}}%
\pgfpathcurveto{\pgfqpoint{3.874042in}{2.148618in}}{\pgfqpoint{3.877314in}{2.140718in}}{\pgfqpoint{3.883138in}{2.134894in}}%
\pgfpathcurveto{\pgfqpoint{3.888962in}{2.129070in}}{\pgfqpoint{3.896862in}{2.125798in}}{\pgfqpoint{3.905098in}{2.125798in}}%
\pgfpathclose%
\pgfusepath{stroke,fill}%
\end{pgfscope}%
\begin{pgfscope}%
\pgfpathrectangle{\pgfqpoint{3.793912in}{0.557870in}}{\pgfqpoint{2.446088in}{1.684734in}}%
\pgfusepath{clip}%
\pgfsetbuttcap%
\pgfsetroundjoin%
\definecolor{currentfill}{rgb}{0.298039,0.447059,0.690196}%
\pgfsetfillcolor{currentfill}%
\pgfsetlinewidth{1.003750pt}%
\definecolor{currentstroke}{rgb}{0.298039,0.447059,0.690196}%
\pgfsetstrokecolor{currentstroke}%
\pgfsetdash{}{0pt}%
\pgfpathmoveto{\pgfqpoint{5.975454in}{1.337082in}}%
\pgfpathcurveto{\pgfqpoint{5.983691in}{1.337082in}}{\pgfqpoint{5.991591in}{1.340354in}}{\pgfqpoint{5.997415in}{1.346178in}}%
\pgfpathcurveto{\pgfqpoint{6.003239in}{1.352002in}}{\pgfqpoint{6.006511in}{1.359902in}}{\pgfqpoint{6.006511in}{1.368138in}}%
\pgfpathcurveto{\pgfqpoint{6.006511in}{1.376375in}}{\pgfqpoint{6.003239in}{1.384275in}}{\pgfqpoint{5.997415in}{1.390099in}}%
\pgfpathcurveto{\pgfqpoint{5.991591in}{1.395923in}}{\pgfqpoint{5.983691in}{1.399195in}}{\pgfqpoint{5.975454in}{1.399195in}}%
\pgfpathcurveto{\pgfqpoint{5.967218in}{1.399195in}}{\pgfqpoint{5.959318in}{1.395923in}}{\pgfqpoint{5.953494in}{1.390099in}}%
\pgfpathcurveto{\pgfqpoint{5.947670in}{1.384275in}}{\pgfqpoint{5.944398in}{1.376375in}}{\pgfqpoint{5.944398in}{1.368138in}}%
\pgfpathcurveto{\pgfqpoint{5.944398in}{1.359902in}}{\pgfqpoint{5.947670in}{1.352002in}}{\pgfqpoint{5.953494in}{1.346178in}}%
\pgfpathcurveto{\pgfqpoint{5.959318in}{1.340354in}}{\pgfqpoint{5.967218in}{1.337082in}}{\pgfqpoint{5.975454in}{1.337082in}}%
\pgfpathclose%
\pgfusepath{stroke,fill}%
\end{pgfscope}%
\begin{pgfscope}%
\pgfpathrectangle{\pgfqpoint{3.793912in}{0.557870in}}{\pgfqpoint{2.446088in}{1.684734in}}%
\pgfusepath{clip}%
\pgfsetbuttcap%
\pgfsetroundjoin%
\definecolor{currentfill}{rgb}{0.298039,0.447059,0.690196}%
\pgfsetfillcolor{currentfill}%
\pgfsetlinewidth{1.003750pt}%
\definecolor{currentstroke}{rgb}{0.298039,0.447059,0.690196}%
\pgfsetstrokecolor{currentstroke}%
\pgfsetdash{}{0pt}%
\pgfpathmoveto{\pgfqpoint{3.905098in}{2.125798in}}%
\pgfpathcurveto{\pgfqpoint{3.913334in}{2.125798in}}{\pgfqpoint{3.921234in}{2.129070in}}{\pgfqpoint{3.927058in}{2.134894in}}%
\pgfpathcurveto{\pgfqpoint{3.932882in}{2.140718in}}{\pgfqpoint{3.936155in}{2.148618in}}{\pgfqpoint{3.936155in}{2.156854in}}%
\pgfpathcurveto{\pgfqpoint{3.936155in}{2.165091in}}{\pgfqpoint{3.932882in}{2.172991in}}{\pgfqpoint{3.927058in}{2.178814in}}%
\pgfpathcurveto{\pgfqpoint{3.921234in}{2.184638in}}{\pgfqpoint{3.913334in}{2.187911in}}{\pgfqpoint{3.905098in}{2.187911in}}%
\pgfpathcurveto{\pgfqpoint{3.896862in}{2.187911in}}{\pgfqpoint{3.888962in}{2.184638in}}{\pgfqpoint{3.883138in}{2.178814in}}%
\pgfpathcurveto{\pgfqpoint{3.877314in}{2.172991in}}{\pgfqpoint{3.874042in}{2.165091in}}{\pgfqpoint{3.874042in}{2.156854in}}%
\pgfpathcurveto{\pgfqpoint{3.874042in}{2.148618in}}{\pgfqpoint{3.877314in}{2.140718in}}{\pgfqpoint{3.883138in}{2.134894in}}%
\pgfpathcurveto{\pgfqpoint{3.888962in}{2.129070in}}{\pgfqpoint{3.896862in}{2.125798in}}{\pgfqpoint{3.905098in}{2.125798in}}%
\pgfpathclose%
\pgfusepath{stroke,fill}%
\end{pgfscope}%
\begin{pgfscope}%
\pgfpathrectangle{\pgfqpoint{3.793912in}{0.557870in}}{\pgfqpoint{2.446088in}{1.684734in}}%
\pgfusepath{clip}%
\pgfsetbuttcap%
\pgfsetroundjoin%
\definecolor{currentfill}{rgb}{0.298039,0.447059,0.690196}%
\pgfsetfillcolor{currentfill}%
\pgfsetlinewidth{1.003750pt}%
\definecolor{currentstroke}{rgb}{0.298039,0.447059,0.690196}%
\pgfsetstrokecolor{currentstroke}%
\pgfsetdash{}{0pt}%
\pgfpathmoveto{\pgfqpoint{5.438695in}{1.933204in}}%
\pgfpathcurveto{\pgfqpoint{5.446932in}{1.933204in}}{\pgfqpoint{5.454832in}{1.936477in}}{\pgfqpoint{5.460656in}{1.942301in}}%
\pgfpathcurveto{\pgfqpoint{5.466480in}{1.948124in}}{\pgfqpoint{5.469752in}{1.956025in}}{\pgfqpoint{5.469752in}{1.964261in}}%
\pgfpathcurveto{\pgfqpoint{5.469752in}{1.972497in}}{\pgfqpoint{5.466480in}{1.980397in}}{\pgfqpoint{5.460656in}{1.986221in}}%
\pgfpathcurveto{\pgfqpoint{5.454832in}{1.992045in}}{\pgfqpoint{5.446932in}{1.995317in}}{\pgfqpoint{5.438695in}{1.995317in}}%
\pgfpathcurveto{\pgfqpoint{5.430459in}{1.995317in}}{\pgfqpoint{5.422559in}{1.992045in}}{\pgfqpoint{5.416735in}{1.986221in}}%
\pgfpathcurveto{\pgfqpoint{5.410911in}{1.980397in}}{\pgfqpoint{5.407639in}{1.972497in}}{\pgfqpoint{5.407639in}{1.964261in}}%
\pgfpathcurveto{\pgfqpoint{5.407639in}{1.956025in}}{\pgfqpoint{5.410911in}{1.948124in}}{\pgfqpoint{5.416735in}{1.942301in}}%
\pgfpathcurveto{\pgfqpoint{5.422559in}{1.936477in}}{\pgfqpoint{5.430459in}{1.933204in}}{\pgfqpoint{5.438695in}{1.933204in}}%
\pgfpathclose%
\pgfusepath{stroke,fill}%
\end{pgfscope}%
\begin{pgfscope}%
\pgfpathrectangle{\pgfqpoint{3.793912in}{0.557870in}}{\pgfqpoint{2.446088in}{1.684734in}}%
\pgfusepath{clip}%
\pgfsetbuttcap%
\pgfsetroundjoin%
\definecolor{currentfill}{rgb}{0.298039,0.447059,0.690196}%
\pgfsetfillcolor{currentfill}%
\pgfsetlinewidth{1.003750pt}%
\definecolor{currentstroke}{rgb}{0.298039,0.447059,0.690196}%
\pgfsetstrokecolor{currentstroke}%
\pgfsetdash{}{0pt}%
\pgfpathmoveto{\pgfqpoint{5.975454in}{1.272884in}}%
\pgfpathcurveto{\pgfqpoint{5.983691in}{1.272884in}}{\pgfqpoint{5.991591in}{1.276156in}}{\pgfqpoint{5.997415in}{1.281980in}}%
\pgfpathcurveto{\pgfqpoint{6.003239in}{1.287804in}}{\pgfqpoint{6.006511in}{1.295704in}}{\pgfqpoint{6.006511in}{1.303941in}}%
\pgfpathcurveto{\pgfqpoint{6.006511in}{1.312177in}}{\pgfqpoint{6.003239in}{1.320077in}}{\pgfqpoint{5.997415in}{1.325901in}}%
\pgfpathcurveto{\pgfqpoint{5.991591in}{1.331725in}}{\pgfqpoint{5.983691in}{1.334997in}}{\pgfqpoint{5.975454in}{1.334997in}}%
\pgfpathcurveto{\pgfqpoint{5.967218in}{1.334997in}}{\pgfqpoint{5.959318in}{1.331725in}}{\pgfqpoint{5.953494in}{1.325901in}}%
\pgfpathcurveto{\pgfqpoint{5.947670in}{1.320077in}}{\pgfqpoint{5.944398in}{1.312177in}}{\pgfqpoint{5.944398in}{1.303941in}}%
\pgfpathcurveto{\pgfqpoint{5.944398in}{1.295704in}}{\pgfqpoint{5.947670in}{1.287804in}}{\pgfqpoint{5.953494in}{1.281980in}}%
\pgfpathcurveto{\pgfqpoint{5.959318in}{1.276156in}}{\pgfqpoint{5.967218in}{1.272884in}}{\pgfqpoint{5.975454in}{1.272884in}}%
\pgfpathclose%
\pgfusepath{stroke,fill}%
\end{pgfscope}%
\begin{pgfscope}%
\pgfpathrectangle{\pgfqpoint{3.793912in}{0.557870in}}{\pgfqpoint{2.446088in}{1.684734in}}%
\pgfusepath{clip}%
\pgfsetbuttcap%
\pgfsetroundjoin%
\definecolor{currentfill}{rgb}{0.298039,0.447059,0.690196}%
\pgfsetfillcolor{currentfill}%
\pgfsetlinewidth{1.003750pt}%
\definecolor{currentstroke}{rgb}{0.298039,0.447059,0.690196}%
\pgfsetstrokecolor{currentstroke}%
\pgfsetdash{}{0pt}%
\pgfpathmoveto{\pgfqpoint{5.975454in}{1.272884in}}%
\pgfpathcurveto{\pgfqpoint{5.983691in}{1.272884in}}{\pgfqpoint{5.991591in}{1.276156in}}{\pgfqpoint{5.997415in}{1.281980in}}%
\pgfpathcurveto{\pgfqpoint{6.003239in}{1.287804in}}{\pgfqpoint{6.006511in}{1.295704in}}{\pgfqpoint{6.006511in}{1.303941in}}%
\pgfpathcurveto{\pgfqpoint{6.006511in}{1.312177in}}{\pgfqpoint{6.003239in}{1.320077in}}{\pgfqpoint{5.997415in}{1.325901in}}%
\pgfpathcurveto{\pgfqpoint{5.991591in}{1.331725in}}{\pgfqpoint{5.983691in}{1.334997in}}{\pgfqpoint{5.975454in}{1.334997in}}%
\pgfpathcurveto{\pgfqpoint{5.967218in}{1.334997in}}{\pgfqpoint{5.959318in}{1.331725in}}{\pgfqpoint{5.953494in}{1.325901in}}%
\pgfpathcurveto{\pgfqpoint{5.947670in}{1.320077in}}{\pgfqpoint{5.944398in}{1.312177in}}{\pgfqpoint{5.944398in}{1.303941in}}%
\pgfpathcurveto{\pgfqpoint{5.944398in}{1.295704in}}{\pgfqpoint{5.947670in}{1.287804in}}{\pgfqpoint{5.953494in}{1.281980in}}%
\pgfpathcurveto{\pgfqpoint{5.959318in}{1.276156in}}{\pgfqpoint{5.967218in}{1.272884in}}{\pgfqpoint{5.975454in}{1.272884in}}%
\pgfpathclose%
\pgfusepath{stroke,fill}%
\end{pgfscope}%
\begin{pgfscope}%
\pgfpathrectangle{\pgfqpoint{3.793912in}{0.557870in}}{\pgfqpoint{2.446088in}{1.684734in}}%
\pgfusepath{clip}%
\pgfsetbuttcap%
\pgfsetroundjoin%
\definecolor{currentfill}{rgb}{0.298039,0.447059,0.690196}%
\pgfsetfillcolor{currentfill}%
\pgfsetlinewidth{1.003750pt}%
\definecolor{currentstroke}{rgb}{0.298039,0.447059,0.690196}%
\pgfsetstrokecolor{currentstroke}%
\pgfsetdash{}{0pt}%
\pgfpathmoveto{\pgfqpoint{4.288497in}{2.107456in}}%
\pgfpathcurveto{\pgfqpoint{4.296734in}{2.107456in}}{\pgfqpoint{4.304634in}{2.110728in}}{\pgfqpoint{4.310458in}{2.116552in}}%
\pgfpathcurveto{\pgfqpoint{4.316282in}{2.122376in}}{\pgfqpoint{4.319554in}{2.130276in}}{\pgfqpoint{4.319554in}{2.138512in}}%
\pgfpathcurveto{\pgfqpoint{4.319554in}{2.146748in}}{\pgfqpoint{4.316282in}{2.154648in}}{\pgfqpoint{4.310458in}{2.160472in}}%
\pgfpathcurveto{\pgfqpoint{4.304634in}{2.166296in}}{\pgfqpoint{4.296734in}{2.169569in}}{\pgfqpoint{4.288497in}{2.169569in}}%
\pgfpathcurveto{\pgfqpoint{4.280261in}{2.169569in}}{\pgfqpoint{4.272361in}{2.166296in}}{\pgfqpoint{4.266537in}{2.160472in}}%
\pgfpathcurveto{\pgfqpoint{4.260713in}{2.154648in}}{\pgfqpoint{4.257441in}{2.146748in}}{\pgfqpoint{4.257441in}{2.138512in}}%
\pgfpathcurveto{\pgfqpoint{4.257441in}{2.130276in}}{\pgfqpoint{4.260713in}{2.122376in}}{\pgfqpoint{4.266537in}{2.116552in}}%
\pgfpathcurveto{\pgfqpoint{4.272361in}{2.110728in}}{\pgfqpoint{4.280261in}{2.107456in}}{\pgfqpoint{4.288497in}{2.107456in}}%
\pgfpathclose%
\pgfusepath{stroke,fill}%
\end{pgfscope}%
\begin{pgfscope}%
\pgfpathrectangle{\pgfqpoint{3.793912in}{0.557870in}}{\pgfqpoint{2.446088in}{1.684734in}}%
\pgfusepath{clip}%
\pgfsetbuttcap%
\pgfsetroundjoin%
\definecolor{currentfill}{rgb}{0.298039,0.447059,0.690196}%
\pgfsetfillcolor{currentfill}%
\pgfsetlinewidth{1.003750pt}%
\definecolor{currentstroke}{rgb}{0.298039,0.447059,0.690196}%
\pgfsetstrokecolor{currentstroke}%
\pgfsetdash{}{0pt}%
\pgfpathmoveto{\pgfqpoint{3.905098in}{2.125798in}}%
\pgfpathcurveto{\pgfqpoint{3.913334in}{2.125798in}}{\pgfqpoint{3.921234in}{2.129070in}}{\pgfqpoint{3.927058in}{2.134894in}}%
\pgfpathcurveto{\pgfqpoint{3.932882in}{2.140718in}}{\pgfqpoint{3.936155in}{2.148618in}}{\pgfqpoint{3.936155in}{2.156854in}}%
\pgfpathcurveto{\pgfqpoint{3.936155in}{2.165091in}}{\pgfqpoint{3.932882in}{2.172991in}}{\pgfqpoint{3.927058in}{2.178814in}}%
\pgfpathcurveto{\pgfqpoint{3.921234in}{2.184638in}}{\pgfqpoint{3.913334in}{2.187911in}}{\pgfqpoint{3.905098in}{2.187911in}}%
\pgfpathcurveto{\pgfqpoint{3.896862in}{2.187911in}}{\pgfqpoint{3.888962in}{2.184638in}}{\pgfqpoint{3.883138in}{2.178814in}}%
\pgfpathcurveto{\pgfqpoint{3.877314in}{2.172991in}}{\pgfqpoint{3.874042in}{2.165091in}}{\pgfqpoint{3.874042in}{2.156854in}}%
\pgfpathcurveto{\pgfqpoint{3.874042in}{2.148618in}}{\pgfqpoint{3.877314in}{2.140718in}}{\pgfqpoint{3.883138in}{2.134894in}}%
\pgfpathcurveto{\pgfqpoint{3.888962in}{2.129070in}}{\pgfqpoint{3.896862in}{2.125798in}}{\pgfqpoint{3.905098in}{2.125798in}}%
\pgfpathclose%
\pgfusepath{stroke,fill}%
\end{pgfscope}%
\begin{pgfscope}%
\pgfpathrectangle{\pgfqpoint{3.793912in}{0.557870in}}{\pgfqpoint{2.446088in}{1.684734in}}%
\pgfusepath{clip}%
\pgfsetbuttcap%
\pgfsetroundjoin%
\definecolor{currentfill}{rgb}{0.298039,0.447059,0.690196}%
\pgfsetfillcolor{currentfill}%
\pgfsetlinewidth{1.003750pt}%
\definecolor{currentstroke}{rgb}{0.298039,0.447059,0.690196}%
\pgfsetstrokecolor{currentstroke}%
\pgfsetdash{}{0pt}%
\pgfpathmoveto{\pgfqpoint{4.978616in}{1.667242in}}%
\pgfpathcurveto{\pgfqpoint{4.986852in}{1.667242in}}{\pgfqpoint{4.994753in}{1.670514in}}{\pgfqpoint{5.000576in}{1.676338in}}%
\pgfpathcurveto{\pgfqpoint{5.006400in}{1.682162in}}{\pgfqpoint{5.009673in}{1.690062in}}{\pgfqpoint{5.009673in}{1.698298in}}%
\pgfpathcurveto{\pgfqpoint{5.009673in}{1.706535in}}{\pgfqpoint{5.006400in}{1.714435in}}{\pgfqpoint{5.000576in}{1.720259in}}%
\pgfpathcurveto{\pgfqpoint{4.994753in}{1.726083in}}{\pgfqpoint{4.986852in}{1.729355in}}{\pgfqpoint{4.978616in}{1.729355in}}%
\pgfpathcurveto{\pgfqpoint{4.970380in}{1.729355in}}{\pgfqpoint{4.962480in}{1.726083in}}{\pgfqpoint{4.956656in}{1.720259in}}%
\pgfpathcurveto{\pgfqpoint{4.950832in}{1.714435in}}{\pgfqpoint{4.947560in}{1.706535in}}{\pgfqpoint{4.947560in}{1.698298in}}%
\pgfpathcurveto{\pgfqpoint{4.947560in}{1.690062in}}{\pgfqpoint{4.950832in}{1.682162in}}{\pgfqpoint{4.956656in}{1.676338in}}%
\pgfpathcurveto{\pgfqpoint{4.962480in}{1.670514in}}{\pgfqpoint{4.970380in}{1.667242in}}{\pgfqpoint{4.978616in}{1.667242in}}%
\pgfpathclose%
\pgfusepath{stroke,fill}%
\end{pgfscope}%
\begin{pgfscope}%
\pgfpathrectangle{\pgfqpoint{3.793912in}{0.557870in}}{\pgfqpoint{2.446088in}{1.684734in}}%
\pgfusepath{clip}%
\pgfsetbuttcap%
\pgfsetroundjoin%
\definecolor{currentfill}{rgb}{0.298039,0.447059,0.690196}%
\pgfsetfillcolor{currentfill}%
\pgfsetlinewidth{1.003750pt}%
\definecolor{currentstroke}{rgb}{0.298039,0.447059,0.690196}%
\pgfsetstrokecolor{currentstroke}%
\pgfsetdash{}{0pt}%
\pgfpathmoveto{\pgfqpoint{3.905098in}{2.125798in}}%
\pgfpathcurveto{\pgfqpoint{3.913334in}{2.125798in}}{\pgfqpoint{3.921234in}{2.129070in}}{\pgfqpoint{3.927058in}{2.134894in}}%
\pgfpathcurveto{\pgfqpoint{3.932882in}{2.140718in}}{\pgfqpoint{3.936155in}{2.148618in}}{\pgfqpoint{3.936155in}{2.156854in}}%
\pgfpathcurveto{\pgfqpoint{3.936155in}{2.165091in}}{\pgfqpoint{3.932882in}{2.172991in}}{\pgfqpoint{3.927058in}{2.178814in}}%
\pgfpathcurveto{\pgfqpoint{3.921234in}{2.184638in}}{\pgfqpoint{3.913334in}{2.187911in}}{\pgfqpoint{3.905098in}{2.187911in}}%
\pgfpathcurveto{\pgfqpoint{3.896862in}{2.187911in}}{\pgfqpoint{3.888962in}{2.184638in}}{\pgfqpoint{3.883138in}{2.178814in}}%
\pgfpathcurveto{\pgfqpoint{3.877314in}{2.172991in}}{\pgfqpoint{3.874042in}{2.165091in}}{\pgfqpoint{3.874042in}{2.156854in}}%
\pgfpathcurveto{\pgfqpoint{3.874042in}{2.148618in}}{\pgfqpoint{3.877314in}{2.140718in}}{\pgfqpoint{3.883138in}{2.134894in}}%
\pgfpathcurveto{\pgfqpoint{3.888962in}{2.129070in}}{\pgfqpoint{3.896862in}{2.125798in}}{\pgfqpoint{3.905098in}{2.125798in}}%
\pgfpathclose%
\pgfusepath{stroke,fill}%
\end{pgfscope}%
\begin{pgfscope}%
\pgfpathrectangle{\pgfqpoint{3.793912in}{0.557870in}}{\pgfqpoint{2.446088in}{1.684734in}}%
\pgfusepath{clip}%
\pgfsetbuttcap%
\pgfsetroundjoin%
\definecolor{currentfill}{rgb}{0.298039,0.447059,0.690196}%
\pgfsetfillcolor{currentfill}%
\pgfsetlinewidth{1.003750pt}%
\definecolor{currentstroke}{rgb}{0.298039,0.447059,0.690196}%
\pgfsetstrokecolor{currentstroke}%
\pgfsetdash{}{0pt}%
\pgfpathmoveto{\pgfqpoint{4.901936in}{1.474649in}}%
\pgfpathcurveto{\pgfqpoint{4.910173in}{1.474649in}}{\pgfqpoint{4.918073in}{1.477921in}}{\pgfqpoint{4.923897in}{1.483745in}}%
\pgfpathcurveto{\pgfqpoint{4.929721in}{1.489569in}}{\pgfqpoint{4.932993in}{1.497469in}}{\pgfqpoint{4.932993in}{1.505705in}}%
\pgfpathcurveto{\pgfqpoint{4.932993in}{1.513941in}}{\pgfqpoint{4.929721in}{1.521841in}}{\pgfqpoint{4.923897in}{1.527665in}}%
\pgfpathcurveto{\pgfqpoint{4.918073in}{1.533489in}}{\pgfqpoint{4.910173in}{1.536762in}}{\pgfqpoint{4.901936in}{1.536762in}}%
\pgfpathcurveto{\pgfqpoint{4.893700in}{1.536762in}}{\pgfqpoint{4.885800in}{1.533489in}}{\pgfqpoint{4.879976in}{1.527665in}}%
\pgfpathcurveto{\pgfqpoint{4.874152in}{1.521841in}}{\pgfqpoint{4.870880in}{1.513941in}}{\pgfqpoint{4.870880in}{1.505705in}}%
\pgfpathcurveto{\pgfqpoint{4.870880in}{1.497469in}}{\pgfqpoint{4.874152in}{1.489569in}}{\pgfqpoint{4.879976in}{1.483745in}}%
\pgfpathcurveto{\pgfqpoint{4.885800in}{1.477921in}}{\pgfqpoint{4.893700in}{1.474649in}}{\pgfqpoint{4.901936in}{1.474649in}}%
\pgfpathclose%
\pgfusepath{stroke,fill}%
\end{pgfscope}%
\begin{pgfscope}%
\pgfpathrectangle{\pgfqpoint{3.793912in}{0.557870in}}{\pgfqpoint{2.446088in}{1.684734in}}%
\pgfusepath{clip}%
\pgfsetbuttcap%
\pgfsetroundjoin%
\definecolor{currentfill}{rgb}{0.298039,0.447059,0.690196}%
\pgfsetfillcolor{currentfill}%
\pgfsetlinewidth{1.003750pt}%
\definecolor{currentstroke}{rgb}{0.298039,0.447059,0.690196}%
\pgfsetstrokecolor{currentstroke}%
\pgfsetdash{}{0pt}%
\pgfpathmoveto{\pgfqpoint{3.905098in}{2.125798in}}%
\pgfpathcurveto{\pgfqpoint{3.913334in}{2.125798in}}{\pgfqpoint{3.921234in}{2.129070in}}{\pgfqpoint{3.927058in}{2.134894in}}%
\pgfpathcurveto{\pgfqpoint{3.932882in}{2.140718in}}{\pgfqpoint{3.936155in}{2.148618in}}{\pgfqpoint{3.936155in}{2.156854in}}%
\pgfpathcurveto{\pgfqpoint{3.936155in}{2.165091in}}{\pgfqpoint{3.932882in}{2.172991in}}{\pgfqpoint{3.927058in}{2.178814in}}%
\pgfpathcurveto{\pgfqpoint{3.921234in}{2.184638in}}{\pgfqpoint{3.913334in}{2.187911in}}{\pgfqpoint{3.905098in}{2.187911in}}%
\pgfpathcurveto{\pgfqpoint{3.896862in}{2.187911in}}{\pgfqpoint{3.888962in}{2.184638in}}{\pgfqpoint{3.883138in}{2.178814in}}%
\pgfpathcurveto{\pgfqpoint{3.877314in}{2.172991in}}{\pgfqpoint{3.874042in}{2.165091in}}{\pgfqpoint{3.874042in}{2.156854in}}%
\pgfpathcurveto{\pgfqpoint{3.874042in}{2.148618in}}{\pgfqpoint{3.877314in}{2.140718in}}{\pgfqpoint{3.883138in}{2.134894in}}%
\pgfpathcurveto{\pgfqpoint{3.888962in}{2.129070in}}{\pgfqpoint{3.896862in}{2.125798in}}{\pgfqpoint{3.905098in}{2.125798in}}%
\pgfpathclose%
\pgfusepath{stroke,fill}%
\end{pgfscope}%
\begin{pgfscope}%
\pgfpathrectangle{\pgfqpoint{3.793912in}{0.557870in}}{\pgfqpoint{2.446088in}{1.684734in}}%
\pgfusepath{clip}%
\pgfsetbuttcap%
\pgfsetroundjoin%
\definecolor{currentfill}{rgb}{0.298039,0.447059,0.690196}%
\pgfsetfillcolor{currentfill}%
\pgfsetlinewidth{1.003750pt}%
\definecolor{currentstroke}{rgb}{0.298039,0.447059,0.690196}%
\pgfsetstrokecolor{currentstroke}%
\pgfsetdash{}{0pt}%
\pgfpathmoveto{\pgfqpoint{3.905098in}{2.125798in}}%
\pgfpathcurveto{\pgfqpoint{3.913334in}{2.125798in}}{\pgfqpoint{3.921234in}{2.129070in}}{\pgfqpoint{3.927058in}{2.134894in}}%
\pgfpathcurveto{\pgfqpoint{3.932882in}{2.140718in}}{\pgfqpoint{3.936155in}{2.148618in}}{\pgfqpoint{3.936155in}{2.156854in}}%
\pgfpathcurveto{\pgfqpoint{3.936155in}{2.165091in}}{\pgfqpoint{3.932882in}{2.172991in}}{\pgfqpoint{3.927058in}{2.178814in}}%
\pgfpathcurveto{\pgfqpoint{3.921234in}{2.184638in}}{\pgfqpoint{3.913334in}{2.187911in}}{\pgfqpoint{3.905098in}{2.187911in}}%
\pgfpathcurveto{\pgfqpoint{3.896862in}{2.187911in}}{\pgfqpoint{3.888962in}{2.184638in}}{\pgfqpoint{3.883138in}{2.178814in}}%
\pgfpathcurveto{\pgfqpoint{3.877314in}{2.172991in}}{\pgfqpoint{3.874042in}{2.165091in}}{\pgfqpoint{3.874042in}{2.156854in}}%
\pgfpathcurveto{\pgfqpoint{3.874042in}{2.148618in}}{\pgfqpoint{3.877314in}{2.140718in}}{\pgfqpoint{3.883138in}{2.134894in}}%
\pgfpathcurveto{\pgfqpoint{3.888962in}{2.129070in}}{\pgfqpoint{3.896862in}{2.125798in}}{\pgfqpoint{3.905098in}{2.125798in}}%
\pgfpathclose%
\pgfusepath{stroke,fill}%
\end{pgfscope}%
\begin{pgfscope}%
\pgfpathrectangle{\pgfqpoint{3.793912in}{0.557870in}}{\pgfqpoint{2.446088in}{1.684734in}}%
\pgfusepath{clip}%
\pgfsetbuttcap%
\pgfsetroundjoin%
\definecolor{currentfill}{rgb}{0.298039,0.447059,0.690196}%
\pgfsetfillcolor{currentfill}%
\pgfsetlinewidth{1.003750pt}%
\definecolor{currentstroke}{rgb}{0.298039,0.447059,0.690196}%
\pgfsetstrokecolor{currentstroke}%
\pgfsetdash{}{0pt}%
\pgfpathmoveto{\pgfqpoint{3.905098in}{2.125798in}}%
\pgfpathcurveto{\pgfqpoint{3.913334in}{2.125798in}}{\pgfqpoint{3.921234in}{2.129070in}}{\pgfqpoint{3.927058in}{2.134894in}}%
\pgfpathcurveto{\pgfqpoint{3.932882in}{2.140718in}}{\pgfqpoint{3.936155in}{2.148618in}}{\pgfqpoint{3.936155in}{2.156854in}}%
\pgfpathcurveto{\pgfqpoint{3.936155in}{2.165091in}}{\pgfqpoint{3.932882in}{2.172991in}}{\pgfqpoint{3.927058in}{2.178814in}}%
\pgfpathcurveto{\pgfqpoint{3.921234in}{2.184638in}}{\pgfqpoint{3.913334in}{2.187911in}}{\pgfqpoint{3.905098in}{2.187911in}}%
\pgfpathcurveto{\pgfqpoint{3.896862in}{2.187911in}}{\pgfqpoint{3.888962in}{2.184638in}}{\pgfqpoint{3.883138in}{2.178814in}}%
\pgfpathcurveto{\pgfqpoint{3.877314in}{2.172991in}}{\pgfqpoint{3.874042in}{2.165091in}}{\pgfqpoint{3.874042in}{2.156854in}}%
\pgfpathcurveto{\pgfqpoint{3.874042in}{2.148618in}}{\pgfqpoint{3.877314in}{2.140718in}}{\pgfqpoint{3.883138in}{2.134894in}}%
\pgfpathcurveto{\pgfqpoint{3.888962in}{2.129070in}}{\pgfqpoint{3.896862in}{2.125798in}}{\pgfqpoint{3.905098in}{2.125798in}}%
\pgfpathclose%
\pgfusepath{stroke,fill}%
\end{pgfscope}%
\begin{pgfscope}%
\pgfpathrectangle{\pgfqpoint{3.793912in}{0.557870in}}{\pgfqpoint{2.446088in}{1.684734in}}%
\pgfusepath{clip}%
\pgfsetbuttcap%
\pgfsetroundjoin%
\definecolor{currentfill}{rgb}{0.298039,0.447059,0.690196}%
\pgfsetfillcolor{currentfill}%
\pgfsetlinewidth{1.003750pt}%
\definecolor{currentstroke}{rgb}{0.298039,0.447059,0.690196}%
\pgfsetstrokecolor{currentstroke}%
\pgfsetdash{}{0pt}%
\pgfpathmoveto{\pgfqpoint{3.905098in}{2.125798in}}%
\pgfpathcurveto{\pgfqpoint{3.913334in}{2.125798in}}{\pgfqpoint{3.921234in}{2.129070in}}{\pgfqpoint{3.927058in}{2.134894in}}%
\pgfpathcurveto{\pgfqpoint{3.932882in}{2.140718in}}{\pgfqpoint{3.936155in}{2.148618in}}{\pgfqpoint{3.936155in}{2.156854in}}%
\pgfpathcurveto{\pgfqpoint{3.936155in}{2.165091in}}{\pgfqpoint{3.932882in}{2.172991in}}{\pgfqpoint{3.927058in}{2.178814in}}%
\pgfpathcurveto{\pgfqpoint{3.921234in}{2.184638in}}{\pgfqpoint{3.913334in}{2.187911in}}{\pgfqpoint{3.905098in}{2.187911in}}%
\pgfpathcurveto{\pgfqpoint{3.896862in}{2.187911in}}{\pgfqpoint{3.888962in}{2.184638in}}{\pgfqpoint{3.883138in}{2.178814in}}%
\pgfpathcurveto{\pgfqpoint{3.877314in}{2.172991in}}{\pgfqpoint{3.874042in}{2.165091in}}{\pgfqpoint{3.874042in}{2.156854in}}%
\pgfpathcurveto{\pgfqpoint{3.874042in}{2.148618in}}{\pgfqpoint{3.877314in}{2.140718in}}{\pgfqpoint{3.883138in}{2.134894in}}%
\pgfpathcurveto{\pgfqpoint{3.888962in}{2.129070in}}{\pgfqpoint{3.896862in}{2.125798in}}{\pgfqpoint{3.905098in}{2.125798in}}%
\pgfpathclose%
\pgfusepath{stroke,fill}%
\end{pgfscope}%
\begin{pgfscope}%
\pgfpathrectangle{\pgfqpoint{3.793912in}{0.557870in}}{\pgfqpoint{2.446088in}{1.684734in}}%
\pgfusepath{clip}%
\pgfsetbuttcap%
\pgfsetroundjoin%
\definecolor{currentfill}{rgb}{0.298039,0.447059,0.690196}%
\pgfsetfillcolor{currentfill}%
\pgfsetlinewidth{1.003750pt}%
\definecolor{currentstroke}{rgb}{0.298039,0.447059,0.690196}%
\pgfsetstrokecolor{currentstroke}%
\pgfsetdash{}{0pt}%
\pgfpathmoveto{\pgfqpoint{3.905098in}{2.125798in}}%
\pgfpathcurveto{\pgfqpoint{3.913334in}{2.125798in}}{\pgfqpoint{3.921234in}{2.129070in}}{\pgfqpoint{3.927058in}{2.134894in}}%
\pgfpathcurveto{\pgfqpoint{3.932882in}{2.140718in}}{\pgfqpoint{3.936155in}{2.148618in}}{\pgfqpoint{3.936155in}{2.156854in}}%
\pgfpathcurveto{\pgfqpoint{3.936155in}{2.165091in}}{\pgfqpoint{3.932882in}{2.172991in}}{\pgfqpoint{3.927058in}{2.178814in}}%
\pgfpathcurveto{\pgfqpoint{3.921234in}{2.184638in}}{\pgfqpoint{3.913334in}{2.187911in}}{\pgfqpoint{3.905098in}{2.187911in}}%
\pgfpathcurveto{\pgfqpoint{3.896862in}{2.187911in}}{\pgfqpoint{3.888962in}{2.184638in}}{\pgfqpoint{3.883138in}{2.178814in}}%
\pgfpathcurveto{\pgfqpoint{3.877314in}{2.172991in}}{\pgfqpoint{3.874042in}{2.165091in}}{\pgfqpoint{3.874042in}{2.156854in}}%
\pgfpathcurveto{\pgfqpoint{3.874042in}{2.148618in}}{\pgfqpoint{3.877314in}{2.140718in}}{\pgfqpoint{3.883138in}{2.134894in}}%
\pgfpathcurveto{\pgfqpoint{3.888962in}{2.129070in}}{\pgfqpoint{3.896862in}{2.125798in}}{\pgfqpoint{3.905098in}{2.125798in}}%
\pgfpathclose%
\pgfusepath{stroke,fill}%
\end{pgfscope}%
\begin{pgfscope}%
\pgfpathrectangle{\pgfqpoint{3.793912in}{0.557870in}}{\pgfqpoint{2.446088in}{1.684734in}}%
\pgfusepath{clip}%
\pgfsetbuttcap%
\pgfsetroundjoin%
\definecolor{currentfill}{rgb}{0.298039,0.447059,0.690196}%
\pgfsetfillcolor{currentfill}%
\pgfsetlinewidth{1.003750pt}%
\definecolor{currentstroke}{rgb}{0.298039,0.447059,0.690196}%
\pgfsetstrokecolor{currentstroke}%
\pgfsetdash{}{0pt}%
\pgfpathmoveto{\pgfqpoint{3.905098in}{2.125798in}}%
\pgfpathcurveto{\pgfqpoint{3.913334in}{2.125798in}}{\pgfqpoint{3.921234in}{2.129070in}}{\pgfqpoint{3.927058in}{2.134894in}}%
\pgfpathcurveto{\pgfqpoint{3.932882in}{2.140718in}}{\pgfqpoint{3.936155in}{2.148618in}}{\pgfqpoint{3.936155in}{2.156854in}}%
\pgfpathcurveto{\pgfqpoint{3.936155in}{2.165091in}}{\pgfqpoint{3.932882in}{2.172991in}}{\pgfqpoint{3.927058in}{2.178814in}}%
\pgfpathcurveto{\pgfqpoint{3.921234in}{2.184638in}}{\pgfqpoint{3.913334in}{2.187911in}}{\pgfqpoint{3.905098in}{2.187911in}}%
\pgfpathcurveto{\pgfqpoint{3.896862in}{2.187911in}}{\pgfqpoint{3.888962in}{2.184638in}}{\pgfqpoint{3.883138in}{2.178814in}}%
\pgfpathcurveto{\pgfqpoint{3.877314in}{2.172991in}}{\pgfqpoint{3.874042in}{2.165091in}}{\pgfqpoint{3.874042in}{2.156854in}}%
\pgfpathcurveto{\pgfqpoint{3.874042in}{2.148618in}}{\pgfqpoint{3.877314in}{2.140718in}}{\pgfqpoint{3.883138in}{2.134894in}}%
\pgfpathcurveto{\pgfqpoint{3.888962in}{2.129070in}}{\pgfqpoint{3.896862in}{2.125798in}}{\pgfqpoint{3.905098in}{2.125798in}}%
\pgfpathclose%
\pgfusepath{stroke,fill}%
\end{pgfscope}%
\begin{pgfscope}%
\pgfpathrectangle{\pgfqpoint{3.793912in}{0.557870in}}{\pgfqpoint{2.446088in}{1.684734in}}%
\pgfusepath{clip}%
\pgfsetbuttcap%
\pgfsetroundjoin%
\definecolor{currentfill}{rgb}{0.298039,0.447059,0.690196}%
\pgfsetfillcolor{currentfill}%
\pgfsetlinewidth{1.003750pt}%
\definecolor{currentstroke}{rgb}{0.298039,0.447059,0.690196}%
\pgfsetstrokecolor{currentstroke}%
\pgfsetdash{}{0pt}%
\pgfpathmoveto{\pgfqpoint{5.055296in}{1.364595in}}%
\pgfpathcurveto{\pgfqpoint{5.063532in}{1.364595in}}{\pgfqpoint{5.071432in}{1.367867in}}{\pgfqpoint{5.077256in}{1.373691in}}%
\pgfpathcurveto{\pgfqpoint{5.083080in}{1.379515in}}{\pgfqpoint{5.086353in}{1.387415in}}{\pgfqpoint{5.086353in}{1.395652in}}%
\pgfpathcurveto{\pgfqpoint{5.086353in}{1.403888in}}{\pgfqpoint{5.083080in}{1.411788in}}{\pgfqpoint{5.077256in}{1.417612in}}%
\pgfpathcurveto{\pgfqpoint{5.071432in}{1.423436in}}{\pgfqpoint{5.063532in}{1.426708in}}{\pgfqpoint{5.055296in}{1.426708in}}%
\pgfpathcurveto{\pgfqpoint{5.047060in}{1.426708in}}{\pgfqpoint{5.039160in}{1.423436in}}{\pgfqpoint{5.033336in}{1.417612in}}%
\pgfpathcurveto{\pgfqpoint{5.027512in}{1.411788in}}{\pgfqpoint{5.024240in}{1.403888in}}{\pgfqpoint{5.024240in}{1.395652in}}%
\pgfpathcurveto{\pgfqpoint{5.024240in}{1.387415in}}{\pgfqpoint{5.027512in}{1.379515in}}{\pgfqpoint{5.033336in}{1.373691in}}%
\pgfpathcurveto{\pgfqpoint{5.039160in}{1.367867in}}{\pgfqpoint{5.047060in}{1.364595in}}{\pgfqpoint{5.055296in}{1.364595in}}%
\pgfpathclose%
\pgfusepath{stroke,fill}%
\end{pgfscope}%
\begin{pgfscope}%
\pgfpathrectangle{\pgfqpoint{3.793912in}{0.557870in}}{\pgfqpoint{2.446088in}{1.684734in}}%
\pgfusepath{clip}%
\pgfsetbuttcap%
\pgfsetroundjoin%
\definecolor{currentfill}{rgb}{0.298039,0.447059,0.690196}%
\pgfsetfillcolor{currentfill}%
\pgfsetlinewidth{1.003750pt}%
\definecolor{currentstroke}{rgb}{0.298039,0.447059,0.690196}%
\pgfsetstrokecolor{currentstroke}%
\pgfsetdash{}{0pt}%
\pgfpathmoveto{\pgfqpoint{3.905098in}{2.125798in}}%
\pgfpathcurveto{\pgfqpoint{3.913334in}{2.125798in}}{\pgfqpoint{3.921234in}{2.129070in}}{\pgfqpoint{3.927058in}{2.134894in}}%
\pgfpathcurveto{\pgfqpoint{3.932882in}{2.140718in}}{\pgfqpoint{3.936155in}{2.148618in}}{\pgfqpoint{3.936155in}{2.156854in}}%
\pgfpathcurveto{\pgfqpoint{3.936155in}{2.165091in}}{\pgfqpoint{3.932882in}{2.172991in}}{\pgfqpoint{3.927058in}{2.178814in}}%
\pgfpathcurveto{\pgfqpoint{3.921234in}{2.184638in}}{\pgfqpoint{3.913334in}{2.187911in}}{\pgfqpoint{3.905098in}{2.187911in}}%
\pgfpathcurveto{\pgfqpoint{3.896862in}{2.187911in}}{\pgfqpoint{3.888962in}{2.184638in}}{\pgfqpoint{3.883138in}{2.178814in}}%
\pgfpathcurveto{\pgfqpoint{3.877314in}{2.172991in}}{\pgfqpoint{3.874042in}{2.165091in}}{\pgfqpoint{3.874042in}{2.156854in}}%
\pgfpathcurveto{\pgfqpoint{3.874042in}{2.148618in}}{\pgfqpoint{3.877314in}{2.140718in}}{\pgfqpoint{3.883138in}{2.134894in}}%
\pgfpathcurveto{\pgfqpoint{3.888962in}{2.129070in}}{\pgfqpoint{3.896862in}{2.125798in}}{\pgfqpoint{3.905098in}{2.125798in}}%
\pgfpathclose%
\pgfusepath{stroke,fill}%
\end{pgfscope}%
\begin{pgfscope}%
\pgfpathrectangle{\pgfqpoint{3.793912in}{0.557870in}}{\pgfqpoint{2.446088in}{1.684734in}}%
\pgfusepath{clip}%
\pgfsetbuttcap%
\pgfsetroundjoin%
\definecolor{currentfill}{rgb}{0.298039,0.447059,0.690196}%
\pgfsetfillcolor{currentfill}%
\pgfsetlinewidth{1.003750pt}%
\definecolor{currentstroke}{rgb}{0.298039,0.447059,0.690196}%
\pgfsetstrokecolor{currentstroke}%
\pgfsetdash{}{0pt}%
\pgfpathmoveto{\pgfqpoint{3.905098in}{2.125798in}}%
\pgfpathcurveto{\pgfqpoint{3.913334in}{2.125798in}}{\pgfqpoint{3.921234in}{2.129070in}}{\pgfqpoint{3.927058in}{2.134894in}}%
\pgfpathcurveto{\pgfqpoint{3.932882in}{2.140718in}}{\pgfqpoint{3.936155in}{2.148618in}}{\pgfqpoint{3.936155in}{2.156854in}}%
\pgfpathcurveto{\pgfqpoint{3.936155in}{2.165091in}}{\pgfqpoint{3.932882in}{2.172991in}}{\pgfqpoint{3.927058in}{2.178814in}}%
\pgfpathcurveto{\pgfqpoint{3.921234in}{2.184638in}}{\pgfqpoint{3.913334in}{2.187911in}}{\pgfqpoint{3.905098in}{2.187911in}}%
\pgfpathcurveto{\pgfqpoint{3.896862in}{2.187911in}}{\pgfqpoint{3.888962in}{2.184638in}}{\pgfqpoint{3.883138in}{2.178814in}}%
\pgfpathcurveto{\pgfqpoint{3.877314in}{2.172991in}}{\pgfqpoint{3.874042in}{2.165091in}}{\pgfqpoint{3.874042in}{2.156854in}}%
\pgfpathcurveto{\pgfqpoint{3.874042in}{2.148618in}}{\pgfqpoint{3.877314in}{2.140718in}}{\pgfqpoint{3.883138in}{2.134894in}}%
\pgfpathcurveto{\pgfqpoint{3.888962in}{2.129070in}}{\pgfqpoint{3.896862in}{2.125798in}}{\pgfqpoint{3.905098in}{2.125798in}}%
\pgfpathclose%
\pgfusepath{stroke,fill}%
\end{pgfscope}%
\begin{pgfscope}%
\pgfpathrectangle{\pgfqpoint{3.793912in}{0.557870in}}{\pgfqpoint{2.446088in}{1.684734in}}%
\pgfusepath{clip}%
\pgfsetbuttcap%
\pgfsetroundjoin%
\definecolor{currentfill}{rgb}{0.298039,0.447059,0.690196}%
\pgfsetfillcolor{currentfill}%
\pgfsetlinewidth{1.003750pt}%
\definecolor{currentstroke}{rgb}{0.298039,0.447059,0.690196}%
\pgfsetstrokecolor{currentstroke}%
\pgfsetdash{}{0pt}%
\pgfpathmoveto{\pgfqpoint{3.905098in}{2.125798in}}%
\pgfpathcurveto{\pgfqpoint{3.913334in}{2.125798in}}{\pgfqpoint{3.921234in}{2.129070in}}{\pgfqpoint{3.927058in}{2.134894in}}%
\pgfpathcurveto{\pgfqpoint{3.932882in}{2.140718in}}{\pgfqpoint{3.936155in}{2.148618in}}{\pgfqpoint{3.936155in}{2.156854in}}%
\pgfpathcurveto{\pgfqpoint{3.936155in}{2.165091in}}{\pgfqpoint{3.932882in}{2.172991in}}{\pgfqpoint{3.927058in}{2.178814in}}%
\pgfpathcurveto{\pgfqpoint{3.921234in}{2.184638in}}{\pgfqpoint{3.913334in}{2.187911in}}{\pgfqpoint{3.905098in}{2.187911in}}%
\pgfpathcurveto{\pgfqpoint{3.896862in}{2.187911in}}{\pgfqpoint{3.888962in}{2.184638in}}{\pgfqpoint{3.883138in}{2.178814in}}%
\pgfpathcurveto{\pgfqpoint{3.877314in}{2.172991in}}{\pgfqpoint{3.874042in}{2.165091in}}{\pgfqpoint{3.874042in}{2.156854in}}%
\pgfpathcurveto{\pgfqpoint{3.874042in}{2.148618in}}{\pgfqpoint{3.877314in}{2.140718in}}{\pgfqpoint{3.883138in}{2.134894in}}%
\pgfpathcurveto{\pgfqpoint{3.888962in}{2.129070in}}{\pgfqpoint{3.896862in}{2.125798in}}{\pgfqpoint{3.905098in}{2.125798in}}%
\pgfpathclose%
\pgfusepath{stroke,fill}%
\end{pgfscope}%
\begin{pgfscope}%
\pgfpathrectangle{\pgfqpoint{3.793912in}{0.557870in}}{\pgfqpoint{2.446088in}{1.684734in}}%
\pgfusepath{clip}%
\pgfsetbuttcap%
\pgfsetroundjoin%
\definecolor{currentfill}{rgb}{0.298039,0.447059,0.690196}%
\pgfsetfillcolor{currentfill}%
\pgfsetlinewidth{1.003750pt}%
\definecolor{currentstroke}{rgb}{0.298039,0.447059,0.690196}%
\pgfsetstrokecolor{currentstroke}%
\pgfsetdash{}{0pt}%
\pgfpathmoveto{\pgfqpoint{3.905098in}{2.125798in}}%
\pgfpathcurveto{\pgfqpoint{3.913334in}{2.125798in}}{\pgfqpoint{3.921234in}{2.129070in}}{\pgfqpoint{3.927058in}{2.134894in}}%
\pgfpathcurveto{\pgfqpoint{3.932882in}{2.140718in}}{\pgfqpoint{3.936155in}{2.148618in}}{\pgfqpoint{3.936155in}{2.156854in}}%
\pgfpathcurveto{\pgfqpoint{3.936155in}{2.165091in}}{\pgfqpoint{3.932882in}{2.172991in}}{\pgfqpoint{3.927058in}{2.178814in}}%
\pgfpathcurveto{\pgfqpoint{3.921234in}{2.184638in}}{\pgfqpoint{3.913334in}{2.187911in}}{\pgfqpoint{3.905098in}{2.187911in}}%
\pgfpathcurveto{\pgfqpoint{3.896862in}{2.187911in}}{\pgfqpoint{3.888962in}{2.184638in}}{\pgfqpoint{3.883138in}{2.178814in}}%
\pgfpathcurveto{\pgfqpoint{3.877314in}{2.172991in}}{\pgfqpoint{3.874042in}{2.165091in}}{\pgfqpoint{3.874042in}{2.156854in}}%
\pgfpathcurveto{\pgfqpoint{3.874042in}{2.148618in}}{\pgfqpoint{3.877314in}{2.140718in}}{\pgfqpoint{3.883138in}{2.134894in}}%
\pgfpathcurveto{\pgfqpoint{3.888962in}{2.129070in}}{\pgfqpoint{3.896862in}{2.125798in}}{\pgfqpoint{3.905098in}{2.125798in}}%
\pgfpathclose%
\pgfusepath{stroke,fill}%
\end{pgfscope}%
\begin{pgfscope}%
\pgfpathrectangle{\pgfqpoint{3.793912in}{0.557870in}}{\pgfqpoint{2.446088in}{1.684734in}}%
\pgfusepath{clip}%
\pgfsetbuttcap%
\pgfsetroundjoin%
\definecolor{currentfill}{rgb}{0.298039,0.447059,0.690196}%
\pgfsetfillcolor{currentfill}%
\pgfsetlinewidth{1.003750pt}%
\definecolor{currentstroke}{rgb}{0.298039,0.447059,0.690196}%
\pgfsetstrokecolor{currentstroke}%
\pgfsetdash{}{0pt}%
\pgfpathmoveto{\pgfqpoint{5.975454in}{1.263713in}}%
\pgfpathcurveto{\pgfqpoint{5.983691in}{1.263713in}}{\pgfqpoint{5.991591in}{1.266985in}}{\pgfqpoint{5.997415in}{1.272809in}}%
\pgfpathcurveto{\pgfqpoint{6.003239in}{1.278633in}}{\pgfqpoint{6.006511in}{1.286533in}}{\pgfqpoint{6.006511in}{1.294769in}}%
\pgfpathcurveto{\pgfqpoint{6.006511in}{1.303006in}}{\pgfqpoint{6.003239in}{1.310906in}}{\pgfqpoint{5.997415in}{1.316730in}}%
\pgfpathcurveto{\pgfqpoint{5.991591in}{1.322554in}}{\pgfqpoint{5.983691in}{1.325826in}}{\pgfqpoint{5.975454in}{1.325826in}}%
\pgfpathcurveto{\pgfqpoint{5.967218in}{1.325826in}}{\pgfqpoint{5.959318in}{1.322554in}}{\pgfqpoint{5.953494in}{1.316730in}}%
\pgfpathcurveto{\pgfqpoint{5.947670in}{1.310906in}}{\pgfqpoint{5.944398in}{1.303006in}}{\pgfqpoint{5.944398in}{1.294769in}}%
\pgfpathcurveto{\pgfqpoint{5.944398in}{1.286533in}}{\pgfqpoint{5.947670in}{1.278633in}}{\pgfqpoint{5.953494in}{1.272809in}}%
\pgfpathcurveto{\pgfqpoint{5.959318in}{1.266985in}}{\pgfqpoint{5.967218in}{1.263713in}}{\pgfqpoint{5.975454in}{1.263713in}}%
\pgfpathclose%
\pgfusepath{stroke,fill}%
\end{pgfscope}%
\begin{pgfscope}%
\pgfpathrectangle{\pgfqpoint{3.793912in}{0.557870in}}{\pgfqpoint{2.446088in}{1.684734in}}%
\pgfusepath{clip}%
\pgfsetbuttcap%
\pgfsetroundjoin%
\definecolor{currentfill}{rgb}{0.298039,0.447059,0.690196}%
\pgfsetfillcolor{currentfill}%
\pgfsetlinewidth{1.003750pt}%
\definecolor{currentstroke}{rgb}{0.298039,0.447059,0.690196}%
\pgfsetstrokecolor{currentstroke}%
\pgfsetdash{}{0pt}%
\pgfpathmoveto{\pgfqpoint{5.975454in}{1.337082in}}%
\pgfpathcurveto{\pgfqpoint{5.983691in}{1.337082in}}{\pgfqpoint{5.991591in}{1.340354in}}{\pgfqpoint{5.997415in}{1.346178in}}%
\pgfpathcurveto{\pgfqpoint{6.003239in}{1.352002in}}{\pgfqpoint{6.006511in}{1.359902in}}{\pgfqpoint{6.006511in}{1.368138in}}%
\pgfpathcurveto{\pgfqpoint{6.006511in}{1.376375in}}{\pgfqpoint{6.003239in}{1.384275in}}{\pgfqpoint{5.997415in}{1.390099in}}%
\pgfpathcurveto{\pgfqpoint{5.991591in}{1.395923in}}{\pgfqpoint{5.983691in}{1.399195in}}{\pgfqpoint{5.975454in}{1.399195in}}%
\pgfpathcurveto{\pgfqpoint{5.967218in}{1.399195in}}{\pgfqpoint{5.959318in}{1.395923in}}{\pgfqpoint{5.953494in}{1.390099in}}%
\pgfpathcurveto{\pgfqpoint{5.947670in}{1.384275in}}{\pgfqpoint{5.944398in}{1.376375in}}{\pgfqpoint{5.944398in}{1.368138in}}%
\pgfpathcurveto{\pgfqpoint{5.944398in}{1.359902in}}{\pgfqpoint{5.947670in}{1.352002in}}{\pgfqpoint{5.953494in}{1.346178in}}%
\pgfpathcurveto{\pgfqpoint{5.959318in}{1.340354in}}{\pgfqpoint{5.967218in}{1.337082in}}{\pgfqpoint{5.975454in}{1.337082in}}%
\pgfpathclose%
\pgfusepath{stroke,fill}%
\end{pgfscope}%
\begin{pgfscope}%
\pgfpathrectangle{\pgfqpoint{3.793912in}{0.557870in}}{\pgfqpoint{2.446088in}{1.684734in}}%
\pgfusepath{clip}%
\pgfsetbuttcap%
\pgfsetroundjoin%
\definecolor{currentfill}{rgb}{0.298039,0.447059,0.690196}%
\pgfsetfillcolor{currentfill}%
\pgfsetlinewidth{1.003750pt}%
\definecolor{currentstroke}{rgb}{0.298039,0.447059,0.690196}%
\pgfsetstrokecolor{currentstroke}%
\pgfsetdash{}{0pt}%
\pgfpathmoveto{\pgfqpoint{5.975454in}{1.291226in}}%
\pgfpathcurveto{\pgfqpoint{5.983691in}{1.291226in}}{\pgfqpoint{5.991591in}{1.294499in}}{\pgfqpoint{5.997415in}{1.300322in}}%
\pgfpathcurveto{\pgfqpoint{6.003239in}{1.306146in}}{\pgfqpoint{6.006511in}{1.314046in}}{\pgfqpoint{6.006511in}{1.322283in}}%
\pgfpathcurveto{\pgfqpoint{6.006511in}{1.330519in}}{\pgfqpoint{6.003239in}{1.338419in}}{\pgfqpoint{5.997415in}{1.344243in}}%
\pgfpathcurveto{\pgfqpoint{5.991591in}{1.350067in}}{\pgfqpoint{5.983691in}{1.353339in}}{\pgfqpoint{5.975454in}{1.353339in}}%
\pgfpathcurveto{\pgfqpoint{5.967218in}{1.353339in}}{\pgfqpoint{5.959318in}{1.350067in}}{\pgfqpoint{5.953494in}{1.344243in}}%
\pgfpathcurveto{\pgfqpoint{5.947670in}{1.338419in}}{\pgfqpoint{5.944398in}{1.330519in}}{\pgfqpoint{5.944398in}{1.322283in}}%
\pgfpathcurveto{\pgfqpoint{5.944398in}{1.314046in}}{\pgfqpoint{5.947670in}{1.306146in}}{\pgfqpoint{5.953494in}{1.300322in}}%
\pgfpathcurveto{\pgfqpoint{5.959318in}{1.294499in}}{\pgfqpoint{5.967218in}{1.291226in}}{\pgfqpoint{5.975454in}{1.291226in}}%
\pgfpathclose%
\pgfusepath{stroke,fill}%
\end{pgfscope}%
\begin{pgfscope}%
\pgfpathrectangle{\pgfqpoint{3.793912in}{0.557870in}}{\pgfqpoint{2.446088in}{1.684734in}}%
\pgfusepath{clip}%
\pgfsetbuttcap%
\pgfsetroundjoin%
\definecolor{currentfill}{rgb}{0.298039,0.447059,0.690196}%
\pgfsetfillcolor{currentfill}%
\pgfsetlinewidth{1.003750pt}%
\definecolor{currentstroke}{rgb}{0.298039,0.447059,0.690196}%
\pgfsetstrokecolor{currentstroke}%
\pgfsetdash{}{0pt}%
\pgfpathmoveto{\pgfqpoint{4.365177in}{2.107456in}}%
\pgfpathcurveto{\pgfqpoint{4.373414in}{2.107456in}}{\pgfqpoint{4.381314in}{2.110728in}}{\pgfqpoint{4.387137in}{2.116552in}}%
\pgfpathcurveto{\pgfqpoint{4.392961in}{2.122376in}}{\pgfqpoint{4.396234in}{2.130276in}}{\pgfqpoint{4.396234in}{2.138512in}}%
\pgfpathcurveto{\pgfqpoint{4.396234in}{2.146748in}}{\pgfqpoint{4.392961in}{2.154648in}}{\pgfqpoint{4.387137in}{2.160472in}}%
\pgfpathcurveto{\pgfqpoint{4.381314in}{2.166296in}}{\pgfqpoint{4.373414in}{2.169569in}}{\pgfqpoint{4.365177in}{2.169569in}}%
\pgfpathcurveto{\pgfqpoint{4.356941in}{2.169569in}}{\pgfqpoint{4.349041in}{2.166296in}}{\pgfqpoint{4.343217in}{2.160472in}}%
\pgfpathcurveto{\pgfqpoint{4.337393in}{2.154648in}}{\pgfqpoint{4.334121in}{2.146748in}}{\pgfqpoint{4.334121in}{2.138512in}}%
\pgfpathcurveto{\pgfqpoint{4.334121in}{2.130276in}}{\pgfqpoint{4.337393in}{2.122376in}}{\pgfqpoint{4.343217in}{2.116552in}}%
\pgfpathcurveto{\pgfqpoint{4.349041in}{2.110728in}}{\pgfqpoint{4.356941in}{2.107456in}}{\pgfqpoint{4.365177in}{2.107456in}}%
\pgfpathclose%
\pgfusepath{stroke,fill}%
\end{pgfscope}%
\begin{pgfscope}%
\pgfpathrectangle{\pgfqpoint{3.793912in}{0.557870in}}{\pgfqpoint{2.446088in}{1.684734in}}%
\pgfusepath{clip}%
\pgfsetbuttcap%
\pgfsetroundjoin%
\definecolor{currentfill}{rgb}{0.298039,0.447059,0.690196}%
\pgfsetfillcolor{currentfill}%
\pgfsetlinewidth{1.003750pt}%
\definecolor{currentstroke}{rgb}{0.298039,0.447059,0.690196}%
\pgfsetstrokecolor{currentstroke}%
\pgfsetdash{}{0pt}%
\pgfpathmoveto{\pgfqpoint{5.745415in}{1.795638in}}%
\pgfpathcurveto{\pgfqpoint{5.753651in}{1.795638in}}{\pgfqpoint{5.761551in}{1.798910in}}{\pgfqpoint{5.767375in}{1.804734in}}%
\pgfpathcurveto{\pgfqpoint{5.773199in}{1.810558in}}{\pgfqpoint{5.776471in}{1.818458in}}{\pgfqpoint{5.776471in}{1.826694in}}%
\pgfpathcurveto{\pgfqpoint{5.776471in}{1.834930in}}{\pgfqpoint{5.773199in}{1.842830in}}{\pgfqpoint{5.767375in}{1.848654in}}%
\pgfpathcurveto{\pgfqpoint{5.761551in}{1.854478in}}{\pgfqpoint{5.753651in}{1.857751in}}{\pgfqpoint{5.745415in}{1.857751in}}%
\pgfpathcurveto{\pgfqpoint{5.737179in}{1.857751in}}{\pgfqpoint{5.729279in}{1.854478in}}{\pgfqpoint{5.723455in}{1.848654in}}%
\pgfpathcurveto{\pgfqpoint{5.717631in}{1.842830in}}{\pgfqpoint{5.714358in}{1.834930in}}{\pgfqpoint{5.714358in}{1.826694in}}%
\pgfpathcurveto{\pgfqpoint{5.714358in}{1.818458in}}{\pgfqpoint{5.717631in}{1.810558in}}{\pgfqpoint{5.723455in}{1.804734in}}%
\pgfpathcurveto{\pgfqpoint{5.729279in}{1.798910in}}{\pgfqpoint{5.737179in}{1.795638in}}{\pgfqpoint{5.745415in}{1.795638in}}%
\pgfpathclose%
\pgfusepath{stroke,fill}%
\end{pgfscope}%
\begin{pgfscope}%
\pgfpathrectangle{\pgfqpoint{3.793912in}{0.557870in}}{\pgfqpoint{2.446088in}{1.684734in}}%
\pgfusepath{clip}%
\pgfsetbuttcap%
\pgfsetroundjoin%
\definecolor{currentfill}{rgb}{0.298039,0.447059,0.690196}%
\pgfsetfillcolor{currentfill}%
\pgfsetlinewidth{1.003750pt}%
\definecolor{currentstroke}{rgb}{0.298039,0.447059,0.690196}%
\pgfsetstrokecolor{currentstroke}%
\pgfsetdash{}{0pt}%
\pgfpathmoveto{\pgfqpoint{3.905098in}{2.125798in}}%
\pgfpathcurveto{\pgfqpoint{3.913334in}{2.125798in}}{\pgfqpoint{3.921234in}{2.129070in}}{\pgfqpoint{3.927058in}{2.134894in}}%
\pgfpathcurveto{\pgfqpoint{3.932882in}{2.140718in}}{\pgfqpoint{3.936155in}{2.148618in}}{\pgfqpoint{3.936155in}{2.156854in}}%
\pgfpathcurveto{\pgfqpoint{3.936155in}{2.165091in}}{\pgfqpoint{3.932882in}{2.172991in}}{\pgfqpoint{3.927058in}{2.178814in}}%
\pgfpathcurveto{\pgfqpoint{3.921234in}{2.184638in}}{\pgfqpoint{3.913334in}{2.187911in}}{\pgfqpoint{3.905098in}{2.187911in}}%
\pgfpathcurveto{\pgfqpoint{3.896862in}{2.187911in}}{\pgfqpoint{3.888962in}{2.184638in}}{\pgfqpoint{3.883138in}{2.178814in}}%
\pgfpathcurveto{\pgfqpoint{3.877314in}{2.172991in}}{\pgfqpoint{3.874042in}{2.165091in}}{\pgfqpoint{3.874042in}{2.156854in}}%
\pgfpathcurveto{\pgfqpoint{3.874042in}{2.148618in}}{\pgfqpoint{3.877314in}{2.140718in}}{\pgfqpoint{3.883138in}{2.134894in}}%
\pgfpathcurveto{\pgfqpoint{3.888962in}{2.129070in}}{\pgfqpoint{3.896862in}{2.125798in}}{\pgfqpoint{3.905098in}{2.125798in}}%
\pgfpathclose%
\pgfusepath{stroke,fill}%
\end{pgfscope}%
\begin{pgfscope}%
\pgfpathrectangle{\pgfqpoint{3.793912in}{0.557870in}}{\pgfqpoint{2.446088in}{1.684734in}}%
\pgfusepath{clip}%
\pgfsetbuttcap%
\pgfsetroundjoin%
\definecolor{currentfill}{rgb}{0.298039,0.447059,0.690196}%
\pgfsetfillcolor{currentfill}%
\pgfsetlinewidth{1.003750pt}%
\definecolor{currentstroke}{rgb}{0.298039,0.447059,0.690196}%
\pgfsetstrokecolor{currentstroke}%
\pgfsetdash{}{0pt}%
\pgfpathmoveto{\pgfqpoint{5.975454in}{1.520504in}}%
\pgfpathcurveto{\pgfqpoint{5.983691in}{1.520504in}}{\pgfqpoint{5.991591in}{1.523776in}}{\pgfqpoint{5.997415in}{1.529600in}}%
\pgfpathcurveto{\pgfqpoint{6.003239in}{1.535424in}}{\pgfqpoint{6.006511in}{1.543324in}}{\pgfqpoint{6.006511in}{1.551561in}}%
\pgfpathcurveto{\pgfqpoint{6.006511in}{1.559797in}}{\pgfqpoint{6.003239in}{1.567697in}}{\pgfqpoint{5.997415in}{1.573521in}}%
\pgfpathcurveto{\pgfqpoint{5.991591in}{1.579345in}}{\pgfqpoint{5.983691in}{1.582617in}}{\pgfqpoint{5.975454in}{1.582617in}}%
\pgfpathcurveto{\pgfqpoint{5.967218in}{1.582617in}}{\pgfqpoint{5.959318in}{1.579345in}}{\pgfqpoint{5.953494in}{1.573521in}}%
\pgfpathcurveto{\pgfqpoint{5.947670in}{1.567697in}}{\pgfqpoint{5.944398in}{1.559797in}}{\pgfqpoint{5.944398in}{1.551561in}}%
\pgfpathcurveto{\pgfqpoint{5.944398in}{1.543324in}}{\pgfqpoint{5.947670in}{1.535424in}}{\pgfqpoint{5.953494in}{1.529600in}}%
\pgfpathcurveto{\pgfqpoint{5.959318in}{1.523776in}}{\pgfqpoint{5.967218in}{1.520504in}}{\pgfqpoint{5.975454in}{1.520504in}}%
\pgfpathclose%
\pgfusepath{stroke,fill}%
\end{pgfscope}%
\begin{pgfscope}%
\pgfpathrectangle{\pgfqpoint{3.793912in}{0.557870in}}{\pgfqpoint{2.446088in}{1.684734in}}%
\pgfusepath{clip}%
\pgfsetbuttcap%
\pgfsetroundjoin%
\definecolor{currentfill}{rgb}{0.298039,0.447059,0.690196}%
\pgfsetfillcolor{currentfill}%
\pgfsetlinewidth{1.003750pt}%
\definecolor{currentstroke}{rgb}{0.298039,0.447059,0.690196}%
\pgfsetstrokecolor{currentstroke}%
\pgfsetdash{}{0pt}%
\pgfpathmoveto{\pgfqpoint{3.905098in}{2.125798in}}%
\pgfpathcurveto{\pgfqpoint{3.913334in}{2.125798in}}{\pgfqpoint{3.921234in}{2.129070in}}{\pgfqpoint{3.927058in}{2.134894in}}%
\pgfpathcurveto{\pgfqpoint{3.932882in}{2.140718in}}{\pgfqpoint{3.936155in}{2.148618in}}{\pgfqpoint{3.936155in}{2.156854in}}%
\pgfpathcurveto{\pgfqpoint{3.936155in}{2.165091in}}{\pgfqpoint{3.932882in}{2.172991in}}{\pgfqpoint{3.927058in}{2.178814in}}%
\pgfpathcurveto{\pgfqpoint{3.921234in}{2.184638in}}{\pgfqpoint{3.913334in}{2.187911in}}{\pgfqpoint{3.905098in}{2.187911in}}%
\pgfpathcurveto{\pgfqpoint{3.896862in}{2.187911in}}{\pgfqpoint{3.888962in}{2.184638in}}{\pgfqpoint{3.883138in}{2.178814in}}%
\pgfpathcurveto{\pgfqpoint{3.877314in}{2.172991in}}{\pgfqpoint{3.874042in}{2.165091in}}{\pgfqpoint{3.874042in}{2.156854in}}%
\pgfpathcurveto{\pgfqpoint{3.874042in}{2.148618in}}{\pgfqpoint{3.877314in}{2.140718in}}{\pgfqpoint{3.883138in}{2.134894in}}%
\pgfpathcurveto{\pgfqpoint{3.888962in}{2.129070in}}{\pgfqpoint{3.896862in}{2.125798in}}{\pgfqpoint{3.905098in}{2.125798in}}%
\pgfpathclose%
\pgfusepath{stroke,fill}%
\end{pgfscope}%
\begin{pgfscope}%
\pgfpathrectangle{\pgfqpoint{3.793912in}{0.557870in}}{\pgfqpoint{2.446088in}{1.684734in}}%
\pgfusepath{clip}%
\pgfsetbuttcap%
\pgfsetroundjoin%
\definecolor{currentfill}{rgb}{0.298039,0.447059,0.690196}%
\pgfsetfillcolor{currentfill}%
\pgfsetlinewidth{1.003750pt}%
\definecolor{currentstroke}{rgb}{0.298039,0.447059,0.690196}%
\pgfsetstrokecolor{currentstroke}%
\pgfsetdash{}{0pt}%
\pgfpathmoveto{\pgfqpoint{5.975454in}{1.621386in}}%
\pgfpathcurveto{\pgfqpoint{5.983691in}{1.621386in}}{\pgfqpoint{5.991591in}{1.624659in}}{\pgfqpoint{5.997415in}{1.630483in}}%
\pgfpathcurveto{\pgfqpoint{6.003239in}{1.636307in}}{\pgfqpoint{6.006511in}{1.644207in}}{\pgfqpoint{6.006511in}{1.652443in}}%
\pgfpathcurveto{\pgfqpoint{6.006511in}{1.660679in}}{\pgfqpoint{6.003239in}{1.668579in}}{\pgfqpoint{5.997415in}{1.674403in}}%
\pgfpathcurveto{\pgfqpoint{5.991591in}{1.680227in}}{\pgfqpoint{5.983691in}{1.683499in}}{\pgfqpoint{5.975454in}{1.683499in}}%
\pgfpathcurveto{\pgfqpoint{5.967218in}{1.683499in}}{\pgfqpoint{5.959318in}{1.680227in}}{\pgfqpoint{5.953494in}{1.674403in}}%
\pgfpathcurveto{\pgfqpoint{5.947670in}{1.668579in}}{\pgfqpoint{5.944398in}{1.660679in}}{\pgfqpoint{5.944398in}{1.652443in}}%
\pgfpathcurveto{\pgfqpoint{5.944398in}{1.644207in}}{\pgfqpoint{5.947670in}{1.636307in}}{\pgfqpoint{5.953494in}{1.630483in}}%
\pgfpathcurveto{\pgfqpoint{5.959318in}{1.624659in}}{\pgfqpoint{5.967218in}{1.621386in}}{\pgfqpoint{5.975454in}{1.621386in}}%
\pgfpathclose%
\pgfusepath{stroke,fill}%
\end{pgfscope}%
\begin{pgfscope}%
\pgfpathrectangle{\pgfqpoint{3.793912in}{0.557870in}}{\pgfqpoint{2.446088in}{1.684734in}}%
\pgfusepath{clip}%
\pgfsetbuttcap%
\pgfsetroundjoin%
\definecolor{currentfill}{rgb}{0.298039,0.447059,0.690196}%
\pgfsetfillcolor{currentfill}%
\pgfsetlinewidth{1.003750pt}%
\definecolor{currentstroke}{rgb}{0.298039,0.447059,0.690196}%
\pgfsetstrokecolor{currentstroke}%
\pgfsetdash{}{0pt}%
\pgfpathmoveto{\pgfqpoint{3.905098in}{2.125798in}}%
\pgfpathcurveto{\pgfqpoint{3.913334in}{2.125798in}}{\pgfqpoint{3.921234in}{2.129070in}}{\pgfqpoint{3.927058in}{2.134894in}}%
\pgfpathcurveto{\pgfqpoint{3.932882in}{2.140718in}}{\pgfqpoint{3.936155in}{2.148618in}}{\pgfqpoint{3.936155in}{2.156854in}}%
\pgfpathcurveto{\pgfqpoint{3.936155in}{2.165091in}}{\pgfqpoint{3.932882in}{2.172991in}}{\pgfqpoint{3.927058in}{2.178814in}}%
\pgfpathcurveto{\pgfqpoint{3.921234in}{2.184638in}}{\pgfqpoint{3.913334in}{2.187911in}}{\pgfqpoint{3.905098in}{2.187911in}}%
\pgfpathcurveto{\pgfqpoint{3.896862in}{2.187911in}}{\pgfqpoint{3.888962in}{2.184638in}}{\pgfqpoint{3.883138in}{2.178814in}}%
\pgfpathcurveto{\pgfqpoint{3.877314in}{2.172991in}}{\pgfqpoint{3.874042in}{2.165091in}}{\pgfqpoint{3.874042in}{2.156854in}}%
\pgfpathcurveto{\pgfqpoint{3.874042in}{2.148618in}}{\pgfqpoint{3.877314in}{2.140718in}}{\pgfqpoint{3.883138in}{2.134894in}}%
\pgfpathcurveto{\pgfqpoint{3.888962in}{2.129070in}}{\pgfqpoint{3.896862in}{2.125798in}}{\pgfqpoint{3.905098in}{2.125798in}}%
\pgfpathclose%
\pgfusepath{stroke,fill}%
\end{pgfscope}%
\begin{pgfscope}%
\pgfpathrectangle{\pgfqpoint{3.793912in}{0.557870in}}{\pgfqpoint{2.446088in}{1.684734in}}%
\pgfusepath{clip}%
\pgfsetbuttcap%
\pgfsetroundjoin%
\definecolor{currentfill}{rgb}{0.298039,0.447059,0.690196}%
\pgfsetfillcolor{currentfill}%
\pgfsetlinewidth{1.003750pt}%
\definecolor{currentstroke}{rgb}{0.298039,0.447059,0.690196}%
\pgfsetstrokecolor{currentstroke}%
\pgfsetdash{}{0pt}%
\pgfpathmoveto{\pgfqpoint{3.905098in}{2.125798in}}%
\pgfpathcurveto{\pgfqpoint{3.913334in}{2.125798in}}{\pgfqpoint{3.921234in}{2.129070in}}{\pgfqpoint{3.927058in}{2.134894in}}%
\pgfpathcurveto{\pgfqpoint{3.932882in}{2.140718in}}{\pgfqpoint{3.936155in}{2.148618in}}{\pgfqpoint{3.936155in}{2.156854in}}%
\pgfpathcurveto{\pgfqpoint{3.936155in}{2.165091in}}{\pgfqpoint{3.932882in}{2.172991in}}{\pgfqpoint{3.927058in}{2.178814in}}%
\pgfpathcurveto{\pgfqpoint{3.921234in}{2.184638in}}{\pgfqpoint{3.913334in}{2.187911in}}{\pgfqpoint{3.905098in}{2.187911in}}%
\pgfpathcurveto{\pgfqpoint{3.896862in}{2.187911in}}{\pgfqpoint{3.888962in}{2.184638in}}{\pgfqpoint{3.883138in}{2.178814in}}%
\pgfpathcurveto{\pgfqpoint{3.877314in}{2.172991in}}{\pgfqpoint{3.874042in}{2.165091in}}{\pgfqpoint{3.874042in}{2.156854in}}%
\pgfpathcurveto{\pgfqpoint{3.874042in}{2.148618in}}{\pgfqpoint{3.877314in}{2.140718in}}{\pgfqpoint{3.883138in}{2.134894in}}%
\pgfpathcurveto{\pgfqpoint{3.888962in}{2.129070in}}{\pgfqpoint{3.896862in}{2.125798in}}{\pgfqpoint{3.905098in}{2.125798in}}%
\pgfpathclose%
\pgfusepath{stroke,fill}%
\end{pgfscope}%
\begin{pgfscope}%
\pgfpathrectangle{\pgfqpoint{3.793912in}{0.557870in}}{\pgfqpoint{2.446088in}{1.684734in}}%
\pgfusepath{clip}%
\pgfsetbuttcap%
\pgfsetroundjoin%
\definecolor{currentfill}{rgb}{0.298039,0.447059,0.690196}%
\pgfsetfillcolor{currentfill}%
\pgfsetlinewidth{1.003750pt}%
\definecolor{currentstroke}{rgb}{0.298039,0.447059,0.690196}%
\pgfsetstrokecolor{currentstroke}%
\pgfsetdash{}{0pt}%
\pgfpathmoveto{\pgfqpoint{3.905098in}{2.125798in}}%
\pgfpathcurveto{\pgfqpoint{3.913334in}{2.125798in}}{\pgfqpoint{3.921234in}{2.129070in}}{\pgfqpoint{3.927058in}{2.134894in}}%
\pgfpathcurveto{\pgfqpoint{3.932882in}{2.140718in}}{\pgfqpoint{3.936155in}{2.148618in}}{\pgfqpoint{3.936155in}{2.156854in}}%
\pgfpathcurveto{\pgfqpoint{3.936155in}{2.165091in}}{\pgfqpoint{3.932882in}{2.172991in}}{\pgfqpoint{3.927058in}{2.178814in}}%
\pgfpathcurveto{\pgfqpoint{3.921234in}{2.184638in}}{\pgfqpoint{3.913334in}{2.187911in}}{\pgfqpoint{3.905098in}{2.187911in}}%
\pgfpathcurveto{\pgfqpoint{3.896862in}{2.187911in}}{\pgfqpoint{3.888962in}{2.184638in}}{\pgfqpoint{3.883138in}{2.178814in}}%
\pgfpathcurveto{\pgfqpoint{3.877314in}{2.172991in}}{\pgfqpoint{3.874042in}{2.165091in}}{\pgfqpoint{3.874042in}{2.156854in}}%
\pgfpathcurveto{\pgfqpoint{3.874042in}{2.148618in}}{\pgfqpoint{3.877314in}{2.140718in}}{\pgfqpoint{3.883138in}{2.134894in}}%
\pgfpathcurveto{\pgfqpoint{3.888962in}{2.129070in}}{\pgfqpoint{3.896862in}{2.125798in}}{\pgfqpoint{3.905098in}{2.125798in}}%
\pgfpathclose%
\pgfusepath{stroke,fill}%
\end{pgfscope}%
\begin{pgfscope}%
\pgfpathrectangle{\pgfqpoint{3.793912in}{0.557870in}}{\pgfqpoint{2.446088in}{1.684734in}}%
\pgfusepath{clip}%
\pgfsetbuttcap%
\pgfsetroundjoin%
\definecolor{currentfill}{rgb}{0.298039,0.447059,0.690196}%
\pgfsetfillcolor{currentfill}%
\pgfsetlinewidth{1.003750pt}%
\definecolor{currentstroke}{rgb}{0.298039,0.447059,0.690196}%
\pgfsetstrokecolor{currentstroke}%
\pgfsetdash{}{0pt}%
\pgfpathmoveto{\pgfqpoint{3.905098in}{2.125798in}}%
\pgfpathcurveto{\pgfqpoint{3.913334in}{2.125798in}}{\pgfqpoint{3.921234in}{2.129070in}}{\pgfqpoint{3.927058in}{2.134894in}}%
\pgfpathcurveto{\pgfqpoint{3.932882in}{2.140718in}}{\pgfqpoint{3.936155in}{2.148618in}}{\pgfqpoint{3.936155in}{2.156854in}}%
\pgfpathcurveto{\pgfqpoint{3.936155in}{2.165091in}}{\pgfqpoint{3.932882in}{2.172991in}}{\pgfqpoint{3.927058in}{2.178814in}}%
\pgfpathcurveto{\pgfqpoint{3.921234in}{2.184638in}}{\pgfqpoint{3.913334in}{2.187911in}}{\pgfqpoint{3.905098in}{2.187911in}}%
\pgfpathcurveto{\pgfqpoint{3.896862in}{2.187911in}}{\pgfqpoint{3.888962in}{2.184638in}}{\pgfqpoint{3.883138in}{2.178814in}}%
\pgfpathcurveto{\pgfqpoint{3.877314in}{2.172991in}}{\pgfqpoint{3.874042in}{2.165091in}}{\pgfqpoint{3.874042in}{2.156854in}}%
\pgfpathcurveto{\pgfqpoint{3.874042in}{2.148618in}}{\pgfqpoint{3.877314in}{2.140718in}}{\pgfqpoint{3.883138in}{2.134894in}}%
\pgfpathcurveto{\pgfqpoint{3.888962in}{2.129070in}}{\pgfqpoint{3.896862in}{2.125798in}}{\pgfqpoint{3.905098in}{2.125798in}}%
\pgfpathclose%
\pgfusepath{stroke,fill}%
\end{pgfscope}%
\begin{pgfscope}%
\pgfpathrectangle{\pgfqpoint{3.793912in}{0.557870in}}{\pgfqpoint{2.446088in}{1.684734in}}%
\pgfusepath{clip}%
\pgfsetbuttcap%
\pgfsetroundjoin%
\definecolor{currentfill}{rgb}{0.298039,0.447059,0.690196}%
\pgfsetfillcolor{currentfill}%
\pgfsetlinewidth{1.003750pt}%
\definecolor{currentstroke}{rgb}{0.298039,0.447059,0.690196}%
\pgfsetstrokecolor{currentstroke}%
\pgfsetdash{}{0pt}%
\pgfpathmoveto{\pgfqpoint{5.055296in}{1.703926in}}%
\pgfpathcurveto{\pgfqpoint{5.063532in}{1.703926in}}{\pgfqpoint{5.071432in}{1.707199in}}{\pgfqpoint{5.077256in}{1.713023in}}%
\pgfpathcurveto{\pgfqpoint{5.083080in}{1.718847in}}{\pgfqpoint{5.086353in}{1.726747in}}{\pgfqpoint{5.086353in}{1.734983in}}%
\pgfpathcurveto{\pgfqpoint{5.086353in}{1.743219in}}{\pgfqpoint{5.083080in}{1.751119in}}{\pgfqpoint{5.077256in}{1.756943in}}%
\pgfpathcurveto{\pgfqpoint{5.071432in}{1.762767in}}{\pgfqpoint{5.063532in}{1.766039in}}{\pgfqpoint{5.055296in}{1.766039in}}%
\pgfpathcurveto{\pgfqpoint{5.047060in}{1.766039in}}{\pgfqpoint{5.039160in}{1.762767in}}{\pgfqpoint{5.033336in}{1.756943in}}%
\pgfpathcurveto{\pgfqpoint{5.027512in}{1.751119in}}{\pgfqpoint{5.024240in}{1.743219in}}{\pgfqpoint{5.024240in}{1.734983in}}%
\pgfpathcurveto{\pgfqpoint{5.024240in}{1.726747in}}{\pgfqpoint{5.027512in}{1.718847in}}{\pgfqpoint{5.033336in}{1.713023in}}%
\pgfpathcurveto{\pgfqpoint{5.039160in}{1.707199in}}{\pgfqpoint{5.047060in}{1.703926in}}{\pgfqpoint{5.055296in}{1.703926in}}%
\pgfpathclose%
\pgfusepath{stroke,fill}%
\end{pgfscope}%
\begin{pgfscope}%
\pgfpathrectangle{\pgfqpoint{3.793912in}{0.557870in}}{\pgfqpoint{2.446088in}{1.684734in}}%
\pgfusepath{clip}%
\pgfsetbuttcap%
\pgfsetroundjoin%
\definecolor{currentfill}{rgb}{0.298039,0.447059,0.690196}%
\pgfsetfillcolor{currentfill}%
\pgfsetlinewidth{1.003750pt}%
\definecolor{currentstroke}{rgb}{0.298039,0.447059,0.690196}%
\pgfsetstrokecolor{currentstroke}%
\pgfsetdash{}{0pt}%
\pgfpathmoveto{\pgfqpoint{3.905098in}{2.125798in}}%
\pgfpathcurveto{\pgfqpoint{3.913334in}{2.125798in}}{\pgfqpoint{3.921234in}{2.129070in}}{\pgfqpoint{3.927058in}{2.134894in}}%
\pgfpathcurveto{\pgfqpoint{3.932882in}{2.140718in}}{\pgfqpoint{3.936155in}{2.148618in}}{\pgfqpoint{3.936155in}{2.156854in}}%
\pgfpathcurveto{\pgfqpoint{3.936155in}{2.165091in}}{\pgfqpoint{3.932882in}{2.172991in}}{\pgfqpoint{3.927058in}{2.178814in}}%
\pgfpathcurveto{\pgfqpoint{3.921234in}{2.184638in}}{\pgfqpoint{3.913334in}{2.187911in}}{\pgfqpoint{3.905098in}{2.187911in}}%
\pgfpathcurveto{\pgfqpoint{3.896862in}{2.187911in}}{\pgfqpoint{3.888962in}{2.184638in}}{\pgfqpoint{3.883138in}{2.178814in}}%
\pgfpathcurveto{\pgfqpoint{3.877314in}{2.172991in}}{\pgfqpoint{3.874042in}{2.165091in}}{\pgfqpoint{3.874042in}{2.156854in}}%
\pgfpathcurveto{\pgfqpoint{3.874042in}{2.148618in}}{\pgfqpoint{3.877314in}{2.140718in}}{\pgfqpoint{3.883138in}{2.134894in}}%
\pgfpathcurveto{\pgfqpoint{3.888962in}{2.129070in}}{\pgfqpoint{3.896862in}{2.125798in}}{\pgfqpoint{3.905098in}{2.125798in}}%
\pgfpathclose%
\pgfusepath{stroke,fill}%
\end{pgfscope}%
\begin{pgfscope}%
\pgfpathrectangle{\pgfqpoint{3.793912in}{0.557870in}}{\pgfqpoint{2.446088in}{1.684734in}}%
\pgfusepath{clip}%
\pgfsetbuttcap%
\pgfsetroundjoin%
\definecolor{currentfill}{rgb}{0.298039,0.447059,0.690196}%
\pgfsetfillcolor{currentfill}%
\pgfsetlinewidth{1.003750pt}%
\definecolor{currentstroke}{rgb}{0.298039,0.447059,0.690196}%
\pgfsetstrokecolor{currentstroke}%
\pgfsetdash{}{0pt}%
\pgfpathmoveto{\pgfqpoint{5.208656in}{1.621386in}}%
\pgfpathcurveto{\pgfqpoint{5.216892in}{1.621386in}}{\pgfqpoint{5.224792in}{1.624659in}}{\pgfqpoint{5.230616in}{1.630483in}}%
\pgfpathcurveto{\pgfqpoint{5.236440in}{1.636307in}}{\pgfqpoint{5.239712in}{1.644207in}}{\pgfqpoint{5.239712in}{1.652443in}}%
\pgfpathcurveto{\pgfqpoint{5.239712in}{1.660679in}}{\pgfqpoint{5.236440in}{1.668579in}}{\pgfqpoint{5.230616in}{1.674403in}}%
\pgfpathcurveto{\pgfqpoint{5.224792in}{1.680227in}}{\pgfqpoint{5.216892in}{1.683499in}}{\pgfqpoint{5.208656in}{1.683499in}}%
\pgfpathcurveto{\pgfqpoint{5.200420in}{1.683499in}}{\pgfqpoint{5.192519in}{1.680227in}}{\pgfqpoint{5.186696in}{1.674403in}}%
\pgfpathcurveto{\pgfqpoint{5.180872in}{1.668579in}}{\pgfqpoint{5.177599in}{1.660679in}}{\pgfqpoint{5.177599in}{1.652443in}}%
\pgfpathcurveto{\pgfqpoint{5.177599in}{1.644207in}}{\pgfqpoint{5.180872in}{1.636307in}}{\pgfqpoint{5.186696in}{1.630483in}}%
\pgfpathcurveto{\pgfqpoint{5.192519in}{1.624659in}}{\pgfqpoint{5.200420in}{1.621386in}}{\pgfqpoint{5.208656in}{1.621386in}}%
\pgfpathclose%
\pgfusepath{stroke,fill}%
\end{pgfscope}%
\begin{pgfscope}%
\pgfpathrectangle{\pgfqpoint{3.793912in}{0.557870in}}{\pgfqpoint{2.446088in}{1.684734in}}%
\pgfusepath{clip}%
\pgfsetbuttcap%
\pgfsetroundjoin%
\definecolor{currentfill}{rgb}{0.298039,0.447059,0.690196}%
\pgfsetfillcolor{currentfill}%
\pgfsetlinewidth{1.003750pt}%
\definecolor{currentstroke}{rgb}{0.298039,0.447059,0.690196}%
\pgfsetstrokecolor{currentstroke}%
\pgfsetdash{}{0pt}%
\pgfpathmoveto{\pgfqpoint{3.905098in}{2.125798in}}%
\pgfpathcurveto{\pgfqpoint{3.913334in}{2.125798in}}{\pgfqpoint{3.921234in}{2.129070in}}{\pgfqpoint{3.927058in}{2.134894in}}%
\pgfpathcurveto{\pgfqpoint{3.932882in}{2.140718in}}{\pgfqpoint{3.936155in}{2.148618in}}{\pgfqpoint{3.936155in}{2.156854in}}%
\pgfpathcurveto{\pgfqpoint{3.936155in}{2.165091in}}{\pgfqpoint{3.932882in}{2.172991in}}{\pgfqpoint{3.927058in}{2.178814in}}%
\pgfpathcurveto{\pgfqpoint{3.921234in}{2.184638in}}{\pgfqpoint{3.913334in}{2.187911in}}{\pgfqpoint{3.905098in}{2.187911in}}%
\pgfpathcurveto{\pgfqpoint{3.896862in}{2.187911in}}{\pgfqpoint{3.888962in}{2.184638in}}{\pgfqpoint{3.883138in}{2.178814in}}%
\pgfpathcurveto{\pgfqpoint{3.877314in}{2.172991in}}{\pgfqpoint{3.874042in}{2.165091in}}{\pgfqpoint{3.874042in}{2.156854in}}%
\pgfpathcurveto{\pgfqpoint{3.874042in}{2.148618in}}{\pgfqpoint{3.877314in}{2.140718in}}{\pgfqpoint{3.883138in}{2.134894in}}%
\pgfpathcurveto{\pgfqpoint{3.888962in}{2.129070in}}{\pgfqpoint{3.896862in}{2.125798in}}{\pgfqpoint{3.905098in}{2.125798in}}%
\pgfpathclose%
\pgfusepath{stroke,fill}%
\end{pgfscope}%
\begin{pgfscope}%
\pgfpathrectangle{\pgfqpoint{3.793912in}{0.557870in}}{\pgfqpoint{2.446088in}{1.684734in}}%
\pgfusepath{clip}%
\pgfsetbuttcap%
\pgfsetroundjoin%
\definecolor{currentfill}{rgb}{0.298039,0.447059,0.690196}%
\pgfsetfillcolor{currentfill}%
\pgfsetlinewidth{1.003750pt}%
\definecolor{currentstroke}{rgb}{0.298039,0.447059,0.690196}%
\pgfsetstrokecolor{currentstroke}%
\pgfsetdash{}{0pt}%
\pgfpathmoveto{\pgfqpoint{3.905098in}{2.125798in}}%
\pgfpathcurveto{\pgfqpoint{3.913334in}{2.125798in}}{\pgfqpoint{3.921234in}{2.129070in}}{\pgfqpoint{3.927058in}{2.134894in}}%
\pgfpathcurveto{\pgfqpoint{3.932882in}{2.140718in}}{\pgfqpoint{3.936155in}{2.148618in}}{\pgfqpoint{3.936155in}{2.156854in}}%
\pgfpathcurveto{\pgfqpoint{3.936155in}{2.165091in}}{\pgfqpoint{3.932882in}{2.172991in}}{\pgfqpoint{3.927058in}{2.178814in}}%
\pgfpathcurveto{\pgfqpoint{3.921234in}{2.184638in}}{\pgfqpoint{3.913334in}{2.187911in}}{\pgfqpoint{3.905098in}{2.187911in}}%
\pgfpathcurveto{\pgfqpoint{3.896862in}{2.187911in}}{\pgfqpoint{3.888962in}{2.184638in}}{\pgfqpoint{3.883138in}{2.178814in}}%
\pgfpathcurveto{\pgfqpoint{3.877314in}{2.172991in}}{\pgfqpoint{3.874042in}{2.165091in}}{\pgfqpoint{3.874042in}{2.156854in}}%
\pgfpathcurveto{\pgfqpoint{3.874042in}{2.148618in}}{\pgfqpoint{3.877314in}{2.140718in}}{\pgfqpoint{3.883138in}{2.134894in}}%
\pgfpathcurveto{\pgfqpoint{3.888962in}{2.129070in}}{\pgfqpoint{3.896862in}{2.125798in}}{\pgfqpoint{3.905098in}{2.125798in}}%
\pgfpathclose%
\pgfusepath{stroke,fill}%
\end{pgfscope}%
\begin{pgfscope}%
\pgfpathrectangle{\pgfqpoint{3.793912in}{0.557870in}}{\pgfqpoint{2.446088in}{1.684734in}}%
\pgfusepath{clip}%
\pgfsetbuttcap%
\pgfsetroundjoin%
\definecolor{currentfill}{rgb}{0.298039,0.447059,0.690196}%
\pgfsetfillcolor{currentfill}%
\pgfsetlinewidth{1.003750pt}%
\definecolor{currentstroke}{rgb}{0.298039,0.447059,0.690196}%
\pgfsetstrokecolor{currentstroke}%
\pgfsetdash{}{0pt}%
\pgfpathmoveto{\pgfqpoint{3.905098in}{2.125798in}}%
\pgfpathcurveto{\pgfqpoint{3.913334in}{2.125798in}}{\pgfqpoint{3.921234in}{2.129070in}}{\pgfqpoint{3.927058in}{2.134894in}}%
\pgfpathcurveto{\pgfqpoint{3.932882in}{2.140718in}}{\pgfqpoint{3.936155in}{2.148618in}}{\pgfqpoint{3.936155in}{2.156854in}}%
\pgfpathcurveto{\pgfqpoint{3.936155in}{2.165091in}}{\pgfqpoint{3.932882in}{2.172991in}}{\pgfqpoint{3.927058in}{2.178814in}}%
\pgfpathcurveto{\pgfqpoint{3.921234in}{2.184638in}}{\pgfqpoint{3.913334in}{2.187911in}}{\pgfqpoint{3.905098in}{2.187911in}}%
\pgfpathcurveto{\pgfqpoint{3.896862in}{2.187911in}}{\pgfqpoint{3.888962in}{2.184638in}}{\pgfqpoint{3.883138in}{2.178814in}}%
\pgfpathcurveto{\pgfqpoint{3.877314in}{2.172991in}}{\pgfqpoint{3.874042in}{2.165091in}}{\pgfqpoint{3.874042in}{2.156854in}}%
\pgfpathcurveto{\pgfqpoint{3.874042in}{2.148618in}}{\pgfqpoint{3.877314in}{2.140718in}}{\pgfqpoint{3.883138in}{2.134894in}}%
\pgfpathcurveto{\pgfqpoint{3.888962in}{2.129070in}}{\pgfqpoint{3.896862in}{2.125798in}}{\pgfqpoint{3.905098in}{2.125798in}}%
\pgfpathclose%
\pgfusepath{stroke,fill}%
\end{pgfscope}%
\begin{pgfscope}%
\pgfpathrectangle{\pgfqpoint{3.793912in}{0.557870in}}{\pgfqpoint{2.446088in}{1.684734in}}%
\pgfusepath{clip}%
\pgfsetbuttcap%
\pgfsetroundjoin%
\definecolor{currentfill}{rgb}{0.298039,0.447059,0.690196}%
\pgfsetfillcolor{currentfill}%
\pgfsetlinewidth{1.003750pt}%
\definecolor{currentstroke}{rgb}{0.298039,0.447059,0.690196}%
\pgfsetstrokecolor{currentstroke}%
\pgfsetdash{}{0pt}%
\pgfpathmoveto{\pgfqpoint{4.058458in}{2.125798in}}%
\pgfpathcurveto{\pgfqpoint{4.066694in}{2.125798in}}{\pgfqpoint{4.074594in}{2.129070in}}{\pgfqpoint{4.080418in}{2.134894in}}%
\pgfpathcurveto{\pgfqpoint{4.086242in}{2.140718in}}{\pgfqpoint{4.089514in}{2.148618in}}{\pgfqpoint{4.089514in}{2.156854in}}%
\pgfpathcurveto{\pgfqpoint{4.089514in}{2.165091in}}{\pgfqpoint{4.086242in}{2.172991in}}{\pgfqpoint{4.080418in}{2.178814in}}%
\pgfpathcurveto{\pgfqpoint{4.074594in}{2.184638in}}{\pgfqpoint{4.066694in}{2.187911in}}{\pgfqpoint{4.058458in}{2.187911in}}%
\pgfpathcurveto{\pgfqpoint{4.050221in}{2.187911in}}{\pgfqpoint{4.042321in}{2.184638in}}{\pgfqpoint{4.036498in}{2.178814in}}%
\pgfpathcurveto{\pgfqpoint{4.030674in}{2.172991in}}{\pgfqpoint{4.027401in}{2.165091in}}{\pgfqpoint{4.027401in}{2.156854in}}%
\pgfpathcurveto{\pgfqpoint{4.027401in}{2.148618in}}{\pgfqpoint{4.030674in}{2.140718in}}{\pgfqpoint{4.036498in}{2.134894in}}%
\pgfpathcurveto{\pgfqpoint{4.042321in}{2.129070in}}{\pgfqpoint{4.050221in}{2.125798in}}{\pgfqpoint{4.058458in}{2.125798in}}%
\pgfpathclose%
\pgfusepath{stroke,fill}%
\end{pgfscope}%
\begin{pgfscope}%
\pgfpathrectangle{\pgfqpoint{3.793912in}{0.557870in}}{\pgfqpoint{2.446088in}{1.684734in}}%
\pgfusepath{clip}%
\pgfsetbuttcap%
\pgfsetroundjoin%
\definecolor{currentfill}{rgb}{0.298039,0.447059,0.690196}%
\pgfsetfillcolor{currentfill}%
\pgfsetlinewidth{1.003750pt}%
\definecolor{currentstroke}{rgb}{0.298039,0.447059,0.690196}%
\pgfsetstrokecolor{currentstroke}%
\pgfsetdash{}{0pt}%
\pgfpathmoveto{\pgfqpoint{3.905098in}{2.125798in}}%
\pgfpathcurveto{\pgfqpoint{3.913334in}{2.125798in}}{\pgfqpoint{3.921234in}{2.129070in}}{\pgfqpoint{3.927058in}{2.134894in}}%
\pgfpathcurveto{\pgfqpoint{3.932882in}{2.140718in}}{\pgfqpoint{3.936155in}{2.148618in}}{\pgfqpoint{3.936155in}{2.156854in}}%
\pgfpathcurveto{\pgfqpoint{3.936155in}{2.165091in}}{\pgfqpoint{3.932882in}{2.172991in}}{\pgfqpoint{3.927058in}{2.178814in}}%
\pgfpathcurveto{\pgfqpoint{3.921234in}{2.184638in}}{\pgfqpoint{3.913334in}{2.187911in}}{\pgfqpoint{3.905098in}{2.187911in}}%
\pgfpathcurveto{\pgfqpoint{3.896862in}{2.187911in}}{\pgfqpoint{3.888962in}{2.184638in}}{\pgfqpoint{3.883138in}{2.178814in}}%
\pgfpathcurveto{\pgfqpoint{3.877314in}{2.172991in}}{\pgfqpoint{3.874042in}{2.165091in}}{\pgfqpoint{3.874042in}{2.156854in}}%
\pgfpathcurveto{\pgfqpoint{3.874042in}{2.148618in}}{\pgfqpoint{3.877314in}{2.140718in}}{\pgfqpoint{3.883138in}{2.134894in}}%
\pgfpathcurveto{\pgfqpoint{3.888962in}{2.129070in}}{\pgfqpoint{3.896862in}{2.125798in}}{\pgfqpoint{3.905098in}{2.125798in}}%
\pgfpathclose%
\pgfusepath{stroke,fill}%
\end{pgfscope}%
\begin{pgfscope}%
\pgfpathrectangle{\pgfqpoint{3.793912in}{0.557870in}}{\pgfqpoint{2.446088in}{1.684734in}}%
\pgfusepath{clip}%
\pgfsetbuttcap%
\pgfsetroundjoin%
\definecolor{currentfill}{rgb}{0.298039,0.447059,0.690196}%
\pgfsetfillcolor{currentfill}%
\pgfsetlinewidth{1.003750pt}%
\definecolor{currentstroke}{rgb}{0.298039,0.447059,0.690196}%
\pgfsetstrokecolor{currentstroke}%
\pgfsetdash{}{0pt}%
\pgfpathmoveto{\pgfqpoint{4.671897in}{1.703926in}}%
\pgfpathcurveto{\pgfqpoint{4.680133in}{1.703926in}}{\pgfqpoint{4.688033in}{1.707199in}}{\pgfqpoint{4.693857in}{1.713023in}}%
\pgfpathcurveto{\pgfqpoint{4.699681in}{1.718847in}}{\pgfqpoint{4.702953in}{1.726747in}}{\pgfqpoint{4.702953in}{1.734983in}}%
\pgfpathcurveto{\pgfqpoint{4.702953in}{1.743219in}}{\pgfqpoint{4.699681in}{1.751119in}}{\pgfqpoint{4.693857in}{1.756943in}}%
\pgfpathcurveto{\pgfqpoint{4.688033in}{1.762767in}}{\pgfqpoint{4.680133in}{1.766039in}}{\pgfqpoint{4.671897in}{1.766039in}}%
\pgfpathcurveto{\pgfqpoint{4.663660in}{1.766039in}}{\pgfqpoint{4.655760in}{1.762767in}}{\pgfqpoint{4.649936in}{1.756943in}}%
\pgfpathcurveto{\pgfqpoint{4.644113in}{1.751119in}}{\pgfqpoint{4.640840in}{1.743219in}}{\pgfqpoint{4.640840in}{1.734983in}}%
\pgfpathcurveto{\pgfqpoint{4.640840in}{1.726747in}}{\pgfqpoint{4.644113in}{1.718847in}}{\pgfqpoint{4.649936in}{1.713023in}}%
\pgfpathcurveto{\pgfqpoint{4.655760in}{1.707199in}}{\pgfqpoint{4.663660in}{1.703926in}}{\pgfqpoint{4.671897in}{1.703926in}}%
\pgfpathclose%
\pgfusepath{stroke,fill}%
\end{pgfscope}%
\begin{pgfscope}%
\pgfpathrectangle{\pgfqpoint{3.793912in}{0.557870in}}{\pgfqpoint{2.446088in}{1.684734in}}%
\pgfusepath{clip}%
\pgfsetbuttcap%
\pgfsetroundjoin%
\definecolor{currentfill}{rgb}{0.298039,0.447059,0.690196}%
\pgfsetfillcolor{currentfill}%
\pgfsetlinewidth{1.003750pt}%
\definecolor{currentstroke}{rgb}{0.298039,0.447059,0.690196}%
\pgfsetstrokecolor{currentstroke}%
\pgfsetdash{}{0pt}%
\pgfpathmoveto{\pgfqpoint{3.905098in}{2.125798in}}%
\pgfpathcurveto{\pgfqpoint{3.913334in}{2.125798in}}{\pgfqpoint{3.921234in}{2.129070in}}{\pgfqpoint{3.927058in}{2.134894in}}%
\pgfpathcurveto{\pgfqpoint{3.932882in}{2.140718in}}{\pgfqpoint{3.936155in}{2.148618in}}{\pgfqpoint{3.936155in}{2.156854in}}%
\pgfpathcurveto{\pgfqpoint{3.936155in}{2.165091in}}{\pgfqpoint{3.932882in}{2.172991in}}{\pgfqpoint{3.927058in}{2.178814in}}%
\pgfpathcurveto{\pgfqpoint{3.921234in}{2.184638in}}{\pgfqpoint{3.913334in}{2.187911in}}{\pgfqpoint{3.905098in}{2.187911in}}%
\pgfpathcurveto{\pgfqpoint{3.896862in}{2.187911in}}{\pgfqpoint{3.888962in}{2.184638in}}{\pgfqpoint{3.883138in}{2.178814in}}%
\pgfpathcurveto{\pgfqpoint{3.877314in}{2.172991in}}{\pgfqpoint{3.874042in}{2.165091in}}{\pgfqpoint{3.874042in}{2.156854in}}%
\pgfpathcurveto{\pgfqpoint{3.874042in}{2.148618in}}{\pgfqpoint{3.877314in}{2.140718in}}{\pgfqpoint{3.883138in}{2.134894in}}%
\pgfpathcurveto{\pgfqpoint{3.888962in}{2.129070in}}{\pgfqpoint{3.896862in}{2.125798in}}{\pgfqpoint{3.905098in}{2.125798in}}%
\pgfpathclose%
\pgfusepath{stroke,fill}%
\end{pgfscope}%
\begin{pgfscope}%
\pgfpathrectangle{\pgfqpoint{3.793912in}{0.557870in}}{\pgfqpoint{2.446088in}{1.684734in}}%
\pgfusepath{clip}%
\pgfsetbuttcap%
\pgfsetroundjoin%
\definecolor{currentfill}{rgb}{0.298039,0.447059,0.690196}%
\pgfsetfillcolor{currentfill}%
\pgfsetlinewidth{1.003750pt}%
\definecolor{currentstroke}{rgb}{0.298039,0.447059,0.690196}%
\pgfsetstrokecolor{currentstroke}%
\pgfsetdash{}{0pt}%
\pgfpathmoveto{\pgfqpoint{4.058458in}{2.116627in}}%
\pgfpathcurveto{\pgfqpoint{4.066694in}{2.116627in}}{\pgfqpoint{4.074594in}{2.119899in}}{\pgfqpoint{4.080418in}{2.125723in}}%
\pgfpathcurveto{\pgfqpoint{4.086242in}{2.131547in}}{\pgfqpoint{4.089514in}{2.139447in}}{\pgfqpoint{4.089514in}{2.147683in}}%
\pgfpathcurveto{\pgfqpoint{4.089514in}{2.155919in}}{\pgfqpoint{4.086242in}{2.163819in}}{\pgfqpoint{4.080418in}{2.169643in}}%
\pgfpathcurveto{\pgfqpoint{4.074594in}{2.175467in}}{\pgfqpoint{4.066694in}{2.178740in}}{\pgfqpoint{4.058458in}{2.178740in}}%
\pgfpathcurveto{\pgfqpoint{4.050221in}{2.178740in}}{\pgfqpoint{4.042321in}{2.175467in}}{\pgfqpoint{4.036498in}{2.169643in}}%
\pgfpathcurveto{\pgfqpoint{4.030674in}{2.163819in}}{\pgfqpoint{4.027401in}{2.155919in}}{\pgfqpoint{4.027401in}{2.147683in}}%
\pgfpathcurveto{\pgfqpoint{4.027401in}{2.139447in}}{\pgfqpoint{4.030674in}{2.131547in}}{\pgfqpoint{4.036498in}{2.125723in}}%
\pgfpathcurveto{\pgfqpoint{4.042321in}{2.119899in}}{\pgfqpoint{4.050221in}{2.116627in}}{\pgfqpoint{4.058458in}{2.116627in}}%
\pgfpathclose%
\pgfusepath{stroke,fill}%
\end{pgfscope}%
\begin{pgfscope}%
\pgfpathrectangle{\pgfqpoint{3.793912in}{0.557870in}}{\pgfqpoint{2.446088in}{1.684734in}}%
\pgfusepath{clip}%
\pgfsetbuttcap%
\pgfsetroundjoin%
\definecolor{currentfill}{rgb}{0.298039,0.447059,0.690196}%
\pgfsetfillcolor{currentfill}%
\pgfsetlinewidth{1.003750pt}%
\definecolor{currentstroke}{rgb}{0.298039,0.447059,0.690196}%
\pgfsetstrokecolor{currentstroke}%
\pgfsetdash{}{0pt}%
\pgfpathmoveto{\pgfqpoint{3.905098in}{2.125798in}}%
\pgfpathcurveto{\pgfqpoint{3.913334in}{2.125798in}}{\pgfqpoint{3.921234in}{2.129070in}}{\pgfqpoint{3.927058in}{2.134894in}}%
\pgfpathcurveto{\pgfqpoint{3.932882in}{2.140718in}}{\pgfqpoint{3.936155in}{2.148618in}}{\pgfqpoint{3.936155in}{2.156854in}}%
\pgfpathcurveto{\pgfqpoint{3.936155in}{2.165091in}}{\pgfqpoint{3.932882in}{2.172991in}}{\pgfqpoint{3.927058in}{2.178814in}}%
\pgfpathcurveto{\pgfqpoint{3.921234in}{2.184638in}}{\pgfqpoint{3.913334in}{2.187911in}}{\pgfqpoint{3.905098in}{2.187911in}}%
\pgfpathcurveto{\pgfqpoint{3.896862in}{2.187911in}}{\pgfqpoint{3.888962in}{2.184638in}}{\pgfqpoint{3.883138in}{2.178814in}}%
\pgfpathcurveto{\pgfqpoint{3.877314in}{2.172991in}}{\pgfqpoint{3.874042in}{2.165091in}}{\pgfqpoint{3.874042in}{2.156854in}}%
\pgfpathcurveto{\pgfqpoint{3.874042in}{2.148618in}}{\pgfqpoint{3.877314in}{2.140718in}}{\pgfqpoint{3.883138in}{2.134894in}}%
\pgfpathcurveto{\pgfqpoint{3.888962in}{2.129070in}}{\pgfqpoint{3.896862in}{2.125798in}}{\pgfqpoint{3.905098in}{2.125798in}}%
\pgfpathclose%
\pgfusepath{stroke,fill}%
\end{pgfscope}%
\begin{pgfscope}%
\pgfpathrectangle{\pgfqpoint{3.793912in}{0.557870in}}{\pgfqpoint{2.446088in}{1.684734in}}%
\pgfusepath{clip}%
\pgfsetbuttcap%
\pgfsetroundjoin%
\definecolor{currentfill}{rgb}{0.298039,0.447059,0.690196}%
\pgfsetfillcolor{currentfill}%
\pgfsetlinewidth{1.003750pt}%
\definecolor{currentstroke}{rgb}{0.298039,0.447059,0.690196}%
\pgfsetstrokecolor{currentstroke}%
\pgfsetdash{}{0pt}%
\pgfpathmoveto{\pgfqpoint{5.975454in}{1.511333in}}%
\pgfpathcurveto{\pgfqpoint{5.983691in}{1.511333in}}{\pgfqpoint{5.991591in}{1.514605in}}{\pgfqpoint{5.997415in}{1.520429in}}%
\pgfpathcurveto{\pgfqpoint{6.003239in}{1.526253in}}{\pgfqpoint{6.006511in}{1.534153in}}{\pgfqpoint{6.006511in}{1.542390in}}%
\pgfpathcurveto{\pgfqpoint{6.006511in}{1.550626in}}{\pgfqpoint{6.003239in}{1.558526in}}{\pgfqpoint{5.997415in}{1.564350in}}%
\pgfpathcurveto{\pgfqpoint{5.991591in}{1.570174in}}{\pgfqpoint{5.983691in}{1.573446in}}{\pgfqpoint{5.975454in}{1.573446in}}%
\pgfpathcurveto{\pgfqpoint{5.967218in}{1.573446in}}{\pgfqpoint{5.959318in}{1.570174in}}{\pgfqpoint{5.953494in}{1.564350in}}%
\pgfpathcurveto{\pgfqpoint{5.947670in}{1.558526in}}{\pgfqpoint{5.944398in}{1.550626in}}{\pgfqpoint{5.944398in}{1.542390in}}%
\pgfpathcurveto{\pgfqpoint{5.944398in}{1.534153in}}{\pgfqpoint{5.947670in}{1.526253in}}{\pgfqpoint{5.953494in}{1.520429in}}%
\pgfpathcurveto{\pgfqpoint{5.959318in}{1.514605in}}{\pgfqpoint{5.967218in}{1.511333in}}{\pgfqpoint{5.975454in}{1.511333in}}%
\pgfpathclose%
\pgfusepath{stroke,fill}%
\end{pgfscope}%
\begin{pgfscope}%
\pgfpathrectangle{\pgfqpoint{3.793912in}{0.557870in}}{\pgfqpoint{2.446088in}{1.684734in}}%
\pgfusepath{clip}%
\pgfsetbuttcap%
\pgfsetroundjoin%
\definecolor{currentfill}{rgb}{0.298039,0.447059,0.690196}%
\pgfsetfillcolor{currentfill}%
\pgfsetlinewidth{1.003750pt}%
\definecolor{currentstroke}{rgb}{0.298039,0.447059,0.690196}%
\pgfsetstrokecolor{currentstroke}%
\pgfsetdash{}{0pt}%
\pgfpathmoveto{\pgfqpoint{3.905098in}{2.125798in}}%
\pgfpathcurveto{\pgfqpoint{3.913334in}{2.125798in}}{\pgfqpoint{3.921234in}{2.129070in}}{\pgfqpoint{3.927058in}{2.134894in}}%
\pgfpathcurveto{\pgfqpoint{3.932882in}{2.140718in}}{\pgfqpoint{3.936155in}{2.148618in}}{\pgfqpoint{3.936155in}{2.156854in}}%
\pgfpathcurveto{\pgfqpoint{3.936155in}{2.165091in}}{\pgfqpoint{3.932882in}{2.172991in}}{\pgfqpoint{3.927058in}{2.178814in}}%
\pgfpathcurveto{\pgfqpoint{3.921234in}{2.184638in}}{\pgfqpoint{3.913334in}{2.187911in}}{\pgfqpoint{3.905098in}{2.187911in}}%
\pgfpathcurveto{\pgfqpoint{3.896862in}{2.187911in}}{\pgfqpoint{3.888962in}{2.184638in}}{\pgfqpoint{3.883138in}{2.178814in}}%
\pgfpathcurveto{\pgfqpoint{3.877314in}{2.172991in}}{\pgfqpoint{3.874042in}{2.165091in}}{\pgfqpoint{3.874042in}{2.156854in}}%
\pgfpathcurveto{\pgfqpoint{3.874042in}{2.148618in}}{\pgfqpoint{3.877314in}{2.140718in}}{\pgfqpoint{3.883138in}{2.134894in}}%
\pgfpathcurveto{\pgfqpoint{3.888962in}{2.129070in}}{\pgfqpoint{3.896862in}{2.125798in}}{\pgfqpoint{3.905098in}{2.125798in}}%
\pgfpathclose%
\pgfusepath{stroke,fill}%
\end{pgfscope}%
\begin{pgfscope}%
\pgfpathrectangle{\pgfqpoint{3.793912in}{0.557870in}}{\pgfqpoint{2.446088in}{1.684734in}}%
\pgfusepath{clip}%
\pgfsetbuttcap%
\pgfsetroundjoin%
\definecolor{currentfill}{rgb}{0.298039,0.447059,0.690196}%
\pgfsetfillcolor{currentfill}%
\pgfsetlinewidth{1.003750pt}%
\definecolor{currentstroke}{rgb}{0.298039,0.447059,0.690196}%
\pgfsetstrokecolor{currentstroke}%
\pgfsetdash{}{0pt}%
\pgfpathmoveto{\pgfqpoint{5.975454in}{1.621386in}}%
\pgfpathcurveto{\pgfqpoint{5.983691in}{1.621386in}}{\pgfqpoint{5.991591in}{1.624659in}}{\pgfqpoint{5.997415in}{1.630483in}}%
\pgfpathcurveto{\pgfqpoint{6.003239in}{1.636307in}}{\pgfqpoint{6.006511in}{1.644207in}}{\pgfqpoint{6.006511in}{1.652443in}}%
\pgfpathcurveto{\pgfqpoint{6.006511in}{1.660679in}}{\pgfqpoint{6.003239in}{1.668579in}}{\pgfqpoint{5.997415in}{1.674403in}}%
\pgfpathcurveto{\pgfqpoint{5.991591in}{1.680227in}}{\pgfqpoint{5.983691in}{1.683499in}}{\pgfqpoint{5.975454in}{1.683499in}}%
\pgfpathcurveto{\pgfqpoint{5.967218in}{1.683499in}}{\pgfqpoint{5.959318in}{1.680227in}}{\pgfqpoint{5.953494in}{1.674403in}}%
\pgfpathcurveto{\pgfqpoint{5.947670in}{1.668579in}}{\pgfqpoint{5.944398in}{1.660679in}}{\pgfqpoint{5.944398in}{1.652443in}}%
\pgfpathcurveto{\pgfqpoint{5.944398in}{1.644207in}}{\pgfqpoint{5.947670in}{1.636307in}}{\pgfqpoint{5.953494in}{1.630483in}}%
\pgfpathcurveto{\pgfqpoint{5.959318in}{1.624659in}}{\pgfqpoint{5.967218in}{1.621386in}}{\pgfqpoint{5.975454in}{1.621386in}}%
\pgfpathclose%
\pgfusepath{stroke,fill}%
\end{pgfscope}%
\begin{pgfscope}%
\pgfpathrectangle{\pgfqpoint{3.793912in}{0.557870in}}{\pgfqpoint{2.446088in}{1.684734in}}%
\pgfusepath{clip}%
\pgfsetbuttcap%
\pgfsetroundjoin%
\definecolor{currentfill}{rgb}{0.298039,0.447059,0.690196}%
\pgfsetfillcolor{currentfill}%
\pgfsetlinewidth{1.003750pt}%
\definecolor{currentstroke}{rgb}{0.298039,0.447059,0.690196}%
\pgfsetstrokecolor{currentstroke}%
\pgfsetdash{}{0pt}%
\pgfpathmoveto{\pgfqpoint{3.905098in}{2.125798in}}%
\pgfpathcurveto{\pgfqpoint{3.913334in}{2.125798in}}{\pgfqpoint{3.921234in}{2.129070in}}{\pgfqpoint{3.927058in}{2.134894in}}%
\pgfpathcurveto{\pgfqpoint{3.932882in}{2.140718in}}{\pgfqpoint{3.936155in}{2.148618in}}{\pgfqpoint{3.936155in}{2.156854in}}%
\pgfpathcurveto{\pgfqpoint{3.936155in}{2.165091in}}{\pgfqpoint{3.932882in}{2.172991in}}{\pgfqpoint{3.927058in}{2.178814in}}%
\pgfpathcurveto{\pgfqpoint{3.921234in}{2.184638in}}{\pgfqpoint{3.913334in}{2.187911in}}{\pgfqpoint{3.905098in}{2.187911in}}%
\pgfpathcurveto{\pgfqpoint{3.896862in}{2.187911in}}{\pgfqpoint{3.888962in}{2.184638in}}{\pgfqpoint{3.883138in}{2.178814in}}%
\pgfpathcurveto{\pgfqpoint{3.877314in}{2.172991in}}{\pgfqpoint{3.874042in}{2.165091in}}{\pgfqpoint{3.874042in}{2.156854in}}%
\pgfpathcurveto{\pgfqpoint{3.874042in}{2.148618in}}{\pgfqpoint{3.877314in}{2.140718in}}{\pgfqpoint{3.883138in}{2.134894in}}%
\pgfpathcurveto{\pgfqpoint{3.888962in}{2.129070in}}{\pgfqpoint{3.896862in}{2.125798in}}{\pgfqpoint{3.905098in}{2.125798in}}%
\pgfpathclose%
\pgfusepath{stroke,fill}%
\end{pgfscope}%
\begin{pgfscope}%
\pgfpathrectangle{\pgfqpoint{3.793912in}{0.557870in}}{\pgfqpoint{2.446088in}{1.684734in}}%
\pgfusepath{clip}%
\pgfsetbuttcap%
\pgfsetroundjoin%
\definecolor{currentfill}{rgb}{0.298039,0.447059,0.690196}%
\pgfsetfillcolor{currentfill}%
\pgfsetlinewidth{1.003750pt}%
\definecolor{currentstroke}{rgb}{0.298039,0.447059,0.690196}%
\pgfsetstrokecolor{currentstroke}%
\pgfsetdash{}{0pt}%
\pgfpathmoveto{\pgfqpoint{3.905098in}{2.125798in}}%
\pgfpathcurveto{\pgfqpoint{3.913334in}{2.125798in}}{\pgfqpoint{3.921234in}{2.129070in}}{\pgfqpoint{3.927058in}{2.134894in}}%
\pgfpathcurveto{\pgfqpoint{3.932882in}{2.140718in}}{\pgfqpoint{3.936155in}{2.148618in}}{\pgfqpoint{3.936155in}{2.156854in}}%
\pgfpathcurveto{\pgfqpoint{3.936155in}{2.165091in}}{\pgfqpoint{3.932882in}{2.172991in}}{\pgfqpoint{3.927058in}{2.178814in}}%
\pgfpathcurveto{\pgfqpoint{3.921234in}{2.184638in}}{\pgfqpoint{3.913334in}{2.187911in}}{\pgfqpoint{3.905098in}{2.187911in}}%
\pgfpathcurveto{\pgfqpoint{3.896862in}{2.187911in}}{\pgfqpoint{3.888962in}{2.184638in}}{\pgfqpoint{3.883138in}{2.178814in}}%
\pgfpathcurveto{\pgfqpoint{3.877314in}{2.172991in}}{\pgfqpoint{3.874042in}{2.165091in}}{\pgfqpoint{3.874042in}{2.156854in}}%
\pgfpathcurveto{\pgfqpoint{3.874042in}{2.148618in}}{\pgfqpoint{3.877314in}{2.140718in}}{\pgfqpoint{3.883138in}{2.134894in}}%
\pgfpathcurveto{\pgfqpoint{3.888962in}{2.129070in}}{\pgfqpoint{3.896862in}{2.125798in}}{\pgfqpoint{3.905098in}{2.125798in}}%
\pgfpathclose%
\pgfusepath{stroke,fill}%
\end{pgfscope}%
\begin{pgfscope}%
\pgfpathrectangle{\pgfqpoint{3.793912in}{0.557870in}}{\pgfqpoint{2.446088in}{1.684734in}}%
\pgfusepath{clip}%
\pgfsetbuttcap%
\pgfsetroundjoin%
\definecolor{currentfill}{rgb}{0.298039,0.447059,0.690196}%
\pgfsetfillcolor{currentfill}%
\pgfsetlinewidth{1.003750pt}%
\definecolor{currentstroke}{rgb}{0.298039,0.447059,0.690196}%
\pgfsetstrokecolor{currentstroke}%
\pgfsetdash{}{0pt}%
\pgfpathmoveto{\pgfqpoint{3.905098in}{2.125798in}}%
\pgfpathcurveto{\pgfqpoint{3.913334in}{2.125798in}}{\pgfqpoint{3.921234in}{2.129070in}}{\pgfqpoint{3.927058in}{2.134894in}}%
\pgfpathcurveto{\pgfqpoint{3.932882in}{2.140718in}}{\pgfqpoint{3.936155in}{2.148618in}}{\pgfqpoint{3.936155in}{2.156854in}}%
\pgfpathcurveto{\pgfqpoint{3.936155in}{2.165091in}}{\pgfqpoint{3.932882in}{2.172991in}}{\pgfqpoint{3.927058in}{2.178814in}}%
\pgfpathcurveto{\pgfqpoint{3.921234in}{2.184638in}}{\pgfqpoint{3.913334in}{2.187911in}}{\pgfqpoint{3.905098in}{2.187911in}}%
\pgfpathcurveto{\pgfqpoint{3.896862in}{2.187911in}}{\pgfqpoint{3.888962in}{2.184638in}}{\pgfqpoint{3.883138in}{2.178814in}}%
\pgfpathcurveto{\pgfqpoint{3.877314in}{2.172991in}}{\pgfqpoint{3.874042in}{2.165091in}}{\pgfqpoint{3.874042in}{2.156854in}}%
\pgfpathcurveto{\pgfqpoint{3.874042in}{2.148618in}}{\pgfqpoint{3.877314in}{2.140718in}}{\pgfqpoint{3.883138in}{2.134894in}}%
\pgfpathcurveto{\pgfqpoint{3.888962in}{2.129070in}}{\pgfqpoint{3.896862in}{2.125798in}}{\pgfqpoint{3.905098in}{2.125798in}}%
\pgfpathclose%
\pgfusepath{stroke,fill}%
\end{pgfscope}%
\begin{pgfscope}%
\pgfpathrectangle{\pgfqpoint{3.793912in}{0.557870in}}{\pgfqpoint{2.446088in}{1.684734in}}%
\pgfusepath{clip}%
\pgfsetbuttcap%
\pgfsetroundjoin%
\definecolor{currentfill}{rgb}{0.298039,0.447059,0.690196}%
\pgfsetfillcolor{currentfill}%
\pgfsetlinewidth{1.003750pt}%
\definecolor{currentstroke}{rgb}{0.298039,0.447059,0.690196}%
\pgfsetstrokecolor{currentstroke}%
\pgfsetdash{}{0pt}%
\pgfpathmoveto{\pgfqpoint{5.745415in}{1.584702in}}%
\pgfpathcurveto{\pgfqpoint{5.753651in}{1.584702in}}{\pgfqpoint{5.761551in}{1.587974in}}{\pgfqpoint{5.767375in}{1.593798in}}%
\pgfpathcurveto{\pgfqpoint{5.773199in}{1.599622in}}{\pgfqpoint{5.776471in}{1.607522in}}{\pgfqpoint{5.776471in}{1.615758in}}%
\pgfpathcurveto{\pgfqpoint{5.776471in}{1.623995in}}{\pgfqpoint{5.773199in}{1.631895in}}{\pgfqpoint{5.767375in}{1.637719in}}%
\pgfpathcurveto{\pgfqpoint{5.761551in}{1.643543in}}{\pgfqpoint{5.753651in}{1.646815in}}{\pgfqpoint{5.745415in}{1.646815in}}%
\pgfpathcurveto{\pgfqpoint{5.737179in}{1.646815in}}{\pgfqpoint{5.729279in}{1.643543in}}{\pgfqpoint{5.723455in}{1.637719in}}%
\pgfpathcurveto{\pgfqpoint{5.717631in}{1.631895in}}{\pgfqpoint{5.714358in}{1.623995in}}{\pgfqpoint{5.714358in}{1.615758in}}%
\pgfpathcurveto{\pgfqpoint{5.714358in}{1.607522in}}{\pgfqpoint{5.717631in}{1.599622in}}{\pgfqpoint{5.723455in}{1.593798in}}%
\pgfpathcurveto{\pgfqpoint{5.729279in}{1.587974in}}{\pgfqpoint{5.737179in}{1.584702in}}{\pgfqpoint{5.745415in}{1.584702in}}%
\pgfpathclose%
\pgfusepath{stroke,fill}%
\end{pgfscope}%
\begin{pgfscope}%
\pgfpathrectangle{\pgfqpoint{3.793912in}{0.557870in}}{\pgfqpoint{2.446088in}{1.684734in}}%
\pgfusepath{clip}%
\pgfsetbuttcap%
\pgfsetroundjoin%
\definecolor{currentfill}{rgb}{0.298039,0.447059,0.690196}%
\pgfsetfillcolor{currentfill}%
\pgfsetlinewidth{1.003750pt}%
\definecolor{currentstroke}{rgb}{0.298039,0.447059,0.690196}%
\pgfsetstrokecolor{currentstroke}%
\pgfsetdash{}{0pt}%
\pgfpathmoveto{\pgfqpoint{5.975454in}{1.318740in}}%
\pgfpathcurveto{\pgfqpoint{5.983691in}{1.318740in}}{\pgfqpoint{5.991591in}{1.322012in}}{\pgfqpoint{5.997415in}{1.327836in}}%
\pgfpathcurveto{\pgfqpoint{6.003239in}{1.333660in}}{\pgfqpoint{6.006511in}{1.341560in}}{\pgfqpoint{6.006511in}{1.349796in}}%
\pgfpathcurveto{\pgfqpoint{6.006511in}{1.358032in}}{\pgfqpoint{6.003239in}{1.365932in}}{\pgfqpoint{5.997415in}{1.371756in}}%
\pgfpathcurveto{\pgfqpoint{5.991591in}{1.377580in}}{\pgfqpoint{5.983691in}{1.380853in}}{\pgfqpoint{5.975454in}{1.380853in}}%
\pgfpathcurveto{\pgfqpoint{5.967218in}{1.380853in}}{\pgfqpoint{5.959318in}{1.377580in}}{\pgfqpoint{5.953494in}{1.371756in}}%
\pgfpathcurveto{\pgfqpoint{5.947670in}{1.365932in}}{\pgfqpoint{5.944398in}{1.358032in}}{\pgfqpoint{5.944398in}{1.349796in}}%
\pgfpathcurveto{\pgfqpoint{5.944398in}{1.341560in}}{\pgfqpoint{5.947670in}{1.333660in}}{\pgfqpoint{5.953494in}{1.327836in}}%
\pgfpathcurveto{\pgfqpoint{5.959318in}{1.322012in}}{\pgfqpoint{5.967218in}{1.318740in}}{\pgfqpoint{5.975454in}{1.318740in}}%
\pgfpathclose%
\pgfusepath{stroke,fill}%
\end{pgfscope}%
\begin{pgfscope}%
\pgfpathrectangle{\pgfqpoint{3.793912in}{0.557870in}}{\pgfqpoint{2.446088in}{1.684734in}}%
\pgfusepath{clip}%
\pgfsetbuttcap%
\pgfsetroundjoin%
\definecolor{currentfill}{rgb}{0.298039,0.447059,0.690196}%
\pgfsetfillcolor{currentfill}%
\pgfsetlinewidth{1.003750pt}%
\definecolor{currentstroke}{rgb}{0.298039,0.447059,0.690196}%
\pgfsetstrokecolor{currentstroke}%
\pgfsetdash{}{0pt}%
\pgfpathmoveto{\pgfqpoint{5.975454in}{1.282055in}}%
\pgfpathcurveto{\pgfqpoint{5.983691in}{1.282055in}}{\pgfqpoint{5.991591in}{1.285327in}}{\pgfqpoint{5.997415in}{1.291151in}}%
\pgfpathcurveto{\pgfqpoint{6.003239in}{1.296975in}}{\pgfqpoint{6.006511in}{1.304875in}}{\pgfqpoint{6.006511in}{1.313112in}}%
\pgfpathcurveto{\pgfqpoint{6.006511in}{1.321348in}}{\pgfqpoint{6.003239in}{1.329248in}}{\pgfqpoint{5.997415in}{1.335072in}}%
\pgfpathcurveto{\pgfqpoint{5.991591in}{1.340896in}}{\pgfqpoint{5.983691in}{1.344168in}}{\pgfqpoint{5.975454in}{1.344168in}}%
\pgfpathcurveto{\pgfqpoint{5.967218in}{1.344168in}}{\pgfqpoint{5.959318in}{1.340896in}}{\pgfqpoint{5.953494in}{1.335072in}}%
\pgfpathcurveto{\pgfqpoint{5.947670in}{1.329248in}}{\pgfqpoint{5.944398in}{1.321348in}}{\pgfqpoint{5.944398in}{1.313112in}}%
\pgfpathcurveto{\pgfqpoint{5.944398in}{1.304875in}}{\pgfqpoint{5.947670in}{1.296975in}}{\pgfqpoint{5.953494in}{1.291151in}}%
\pgfpathcurveto{\pgfqpoint{5.959318in}{1.285327in}}{\pgfqpoint{5.967218in}{1.282055in}}{\pgfqpoint{5.975454in}{1.282055in}}%
\pgfpathclose%
\pgfusepath{stroke,fill}%
\end{pgfscope}%
\begin{pgfscope}%
\pgfpathrectangle{\pgfqpoint{3.793912in}{0.557870in}}{\pgfqpoint{2.446088in}{1.684734in}}%
\pgfusepath{clip}%
\pgfsetbuttcap%
\pgfsetroundjoin%
\definecolor{currentfill}{rgb}{0.298039,0.447059,0.690196}%
\pgfsetfillcolor{currentfill}%
\pgfsetlinewidth{1.003750pt}%
\definecolor{currentstroke}{rgb}{0.298039,0.447059,0.690196}%
\pgfsetstrokecolor{currentstroke}%
\pgfsetdash{}{0pt}%
\pgfpathmoveto{\pgfqpoint{4.058458in}{1.511333in}}%
\pgfpathcurveto{\pgfqpoint{4.066694in}{1.511333in}}{\pgfqpoint{4.074594in}{1.514605in}}{\pgfqpoint{4.080418in}{1.520429in}}%
\pgfpathcurveto{\pgfqpoint{4.086242in}{1.526253in}}{\pgfqpoint{4.089514in}{1.534153in}}{\pgfqpoint{4.089514in}{1.542390in}}%
\pgfpathcurveto{\pgfqpoint{4.089514in}{1.550626in}}{\pgfqpoint{4.086242in}{1.558526in}}{\pgfqpoint{4.080418in}{1.564350in}}%
\pgfpathcurveto{\pgfqpoint{4.074594in}{1.570174in}}{\pgfqpoint{4.066694in}{1.573446in}}{\pgfqpoint{4.058458in}{1.573446in}}%
\pgfpathcurveto{\pgfqpoint{4.050221in}{1.573446in}}{\pgfqpoint{4.042321in}{1.570174in}}{\pgfqpoint{4.036498in}{1.564350in}}%
\pgfpathcurveto{\pgfqpoint{4.030674in}{1.558526in}}{\pgfqpoint{4.027401in}{1.550626in}}{\pgfqpoint{4.027401in}{1.542390in}}%
\pgfpathcurveto{\pgfqpoint{4.027401in}{1.534153in}}{\pgfqpoint{4.030674in}{1.526253in}}{\pgfqpoint{4.036498in}{1.520429in}}%
\pgfpathcurveto{\pgfqpoint{4.042321in}{1.514605in}}{\pgfqpoint{4.050221in}{1.511333in}}{\pgfqpoint{4.058458in}{1.511333in}}%
\pgfpathclose%
\pgfusepath{stroke,fill}%
\end{pgfscope}%
\begin{pgfscope}%
\pgfpathrectangle{\pgfqpoint{3.793912in}{0.557870in}}{\pgfqpoint{2.446088in}{1.684734in}}%
\pgfusepath{clip}%
\pgfsetbuttcap%
\pgfsetroundjoin%
\definecolor{currentfill}{rgb}{0.298039,0.447059,0.690196}%
\pgfsetfillcolor{currentfill}%
\pgfsetlinewidth{1.003750pt}%
\definecolor{currentstroke}{rgb}{0.298039,0.447059,0.690196}%
\pgfsetstrokecolor{currentstroke}%
\pgfsetdash{}{0pt}%
\pgfpathmoveto{\pgfqpoint{3.981778in}{1.869007in}}%
\pgfpathcurveto{\pgfqpoint{3.990014in}{1.869007in}}{\pgfqpoint{3.997914in}{1.872279in}}{\pgfqpoint{4.003738in}{1.878103in}}%
\pgfpathcurveto{\pgfqpoint{4.009562in}{1.883927in}}{\pgfqpoint{4.012834in}{1.891827in}}{\pgfqpoint{4.012834in}{1.900063in}}%
\pgfpathcurveto{\pgfqpoint{4.012834in}{1.908299in}}{\pgfqpoint{4.009562in}{1.916199in}}{\pgfqpoint{4.003738in}{1.922023in}}%
\pgfpathcurveto{\pgfqpoint{3.997914in}{1.927847in}}{\pgfqpoint{3.990014in}{1.931120in}}{\pgfqpoint{3.981778in}{1.931120in}}%
\pgfpathcurveto{\pgfqpoint{3.973542in}{1.931120in}}{\pgfqpoint{3.965642in}{1.927847in}}{\pgfqpoint{3.959818in}{1.922023in}}%
\pgfpathcurveto{\pgfqpoint{3.953994in}{1.916199in}}{\pgfqpoint{3.950721in}{1.908299in}}{\pgfqpoint{3.950721in}{1.900063in}}%
\pgfpathcurveto{\pgfqpoint{3.950721in}{1.891827in}}{\pgfqpoint{3.953994in}{1.883927in}}{\pgfqpoint{3.959818in}{1.878103in}}%
\pgfpathcurveto{\pgfqpoint{3.965642in}{1.872279in}}{\pgfqpoint{3.973542in}{1.869007in}}{\pgfqpoint{3.981778in}{1.869007in}}%
\pgfpathclose%
\pgfusepath{stroke,fill}%
\end{pgfscope}%
\begin{pgfscope}%
\pgfpathrectangle{\pgfqpoint{3.793912in}{0.557870in}}{\pgfqpoint{2.446088in}{1.684734in}}%
\pgfusepath{clip}%
\pgfsetbuttcap%
\pgfsetroundjoin%
\definecolor{currentfill}{rgb}{0.298039,0.447059,0.690196}%
\pgfsetfillcolor{currentfill}%
\pgfsetlinewidth{1.003750pt}%
\definecolor{currentstroke}{rgb}{0.298039,0.447059,0.690196}%
\pgfsetstrokecolor{currentstroke}%
\pgfsetdash{}{0pt}%
\pgfpathmoveto{\pgfqpoint{3.981778in}{1.777295in}}%
\pgfpathcurveto{\pgfqpoint{3.990014in}{1.777295in}}{\pgfqpoint{3.997914in}{1.780568in}}{\pgfqpoint{4.003738in}{1.786392in}}%
\pgfpathcurveto{\pgfqpoint{4.009562in}{1.792216in}}{\pgfqpoint{4.012834in}{1.800116in}}{\pgfqpoint{4.012834in}{1.808352in}}%
\pgfpathcurveto{\pgfqpoint{4.012834in}{1.816588in}}{\pgfqpoint{4.009562in}{1.824488in}}{\pgfqpoint{4.003738in}{1.830312in}}%
\pgfpathcurveto{\pgfqpoint{3.997914in}{1.836136in}}{\pgfqpoint{3.990014in}{1.839408in}}{\pgfqpoint{3.981778in}{1.839408in}}%
\pgfpathcurveto{\pgfqpoint{3.973542in}{1.839408in}}{\pgfqpoint{3.965642in}{1.836136in}}{\pgfqpoint{3.959818in}{1.830312in}}%
\pgfpathcurveto{\pgfqpoint{3.953994in}{1.824488in}}{\pgfqpoint{3.950721in}{1.816588in}}{\pgfqpoint{3.950721in}{1.808352in}}%
\pgfpathcurveto{\pgfqpoint{3.950721in}{1.800116in}}{\pgfqpoint{3.953994in}{1.792216in}}{\pgfqpoint{3.959818in}{1.786392in}}%
\pgfpathcurveto{\pgfqpoint{3.965642in}{1.780568in}}{\pgfqpoint{3.973542in}{1.777295in}}{\pgfqpoint{3.981778in}{1.777295in}}%
\pgfpathclose%
\pgfusepath{stroke,fill}%
\end{pgfscope}%
\begin{pgfscope}%
\pgfpathrectangle{\pgfqpoint{3.793912in}{0.557870in}}{\pgfqpoint{2.446088in}{1.684734in}}%
\pgfusepath{clip}%
\pgfsetbuttcap%
\pgfsetroundjoin%
\definecolor{currentfill}{rgb}{0.298039,0.447059,0.690196}%
\pgfsetfillcolor{currentfill}%
\pgfsetlinewidth{1.003750pt}%
\definecolor{currentstroke}{rgb}{0.298039,0.447059,0.690196}%
\pgfsetstrokecolor{currentstroke}%
\pgfsetdash{}{0pt}%
\pgfpathmoveto{\pgfqpoint{3.905098in}{2.125798in}}%
\pgfpathcurveto{\pgfqpoint{3.913334in}{2.125798in}}{\pgfqpoint{3.921234in}{2.129070in}}{\pgfqpoint{3.927058in}{2.134894in}}%
\pgfpathcurveto{\pgfqpoint{3.932882in}{2.140718in}}{\pgfqpoint{3.936155in}{2.148618in}}{\pgfqpoint{3.936155in}{2.156854in}}%
\pgfpathcurveto{\pgfqpoint{3.936155in}{2.165091in}}{\pgfqpoint{3.932882in}{2.172991in}}{\pgfqpoint{3.927058in}{2.178814in}}%
\pgfpathcurveto{\pgfqpoint{3.921234in}{2.184638in}}{\pgfqpoint{3.913334in}{2.187911in}}{\pgfqpoint{3.905098in}{2.187911in}}%
\pgfpathcurveto{\pgfqpoint{3.896862in}{2.187911in}}{\pgfqpoint{3.888962in}{2.184638in}}{\pgfqpoint{3.883138in}{2.178814in}}%
\pgfpathcurveto{\pgfqpoint{3.877314in}{2.172991in}}{\pgfqpoint{3.874042in}{2.165091in}}{\pgfqpoint{3.874042in}{2.156854in}}%
\pgfpathcurveto{\pgfqpoint{3.874042in}{2.148618in}}{\pgfqpoint{3.877314in}{2.140718in}}{\pgfqpoint{3.883138in}{2.134894in}}%
\pgfpathcurveto{\pgfqpoint{3.888962in}{2.129070in}}{\pgfqpoint{3.896862in}{2.125798in}}{\pgfqpoint{3.905098in}{2.125798in}}%
\pgfpathclose%
\pgfusepath{stroke,fill}%
\end{pgfscope}%
\begin{pgfscope}%
\pgfpathrectangle{\pgfqpoint{3.793912in}{0.557870in}}{\pgfqpoint{2.446088in}{1.684734in}}%
\pgfusepath{clip}%
\pgfsetbuttcap%
\pgfsetroundjoin%
\definecolor{currentfill}{rgb}{0.298039,0.447059,0.690196}%
\pgfsetfillcolor{currentfill}%
\pgfsetlinewidth{1.003750pt}%
\definecolor{currentstroke}{rgb}{0.298039,0.447059,0.690196}%
\pgfsetstrokecolor{currentstroke}%
\pgfsetdash{}{0pt}%
\pgfpathmoveto{\pgfqpoint{4.365177in}{1.841493in}}%
\pgfpathcurveto{\pgfqpoint{4.373414in}{1.841493in}}{\pgfqpoint{4.381314in}{1.844765in}}{\pgfqpoint{4.387137in}{1.850589in}}%
\pgfpathcurveto{\pgfqpoint{4.392961in}{1.856413in}}{\pgfqpoint{4.396234in}{1.864313in}}{\pgfqpoint{4.396234in}{1.872550in}}%
\pgfpathcurveto{\pgfqpoint{4.396234in}{1.880786in}}{\pgfqpoint{4.392961in}{1.888686in}}{\pgfqpoint{4.387137in}{1.894510in}}%
\pgfpathcurveto{\pgfqpoint{4.381314in}{1.900334in}}{\pgfqpoint{4.373414in}{1.903606in}}{\pgfqpoint{4.365177in}{1.903606in}}%
\pgfpathcurveto{\pgfqpoint{4.356941in}{1.903606in}}{\pgfqpoint{4.349041in}{1.900334in}}{\pgfqpoint{4.343217in}{1.894510in}}%
\pgfpathcurveto{\pgfqpoint{4.337393in}{1.888686in}}{\pgfqpoint{4.334121in}{1.880786in}}{\pgfqpoint{4.334121in}{1.872550in}}%
\pgfpathcurveto{\pgfqpoint{4.334121in}{1.864313in}}{\pgfqpoint{4.337393in}{1.856413in}}{\pgfqpoint{4.343217in}{1.850589in}}%
\pgfpathcurveto{\pgfqpoint{4.349041in}{1.844765in}}{\pgfqpoint{4.356941in}{1.841493in}}{\pgfqpoint{4.365177in}{1.841493in}}%
\pgfpathclose%
\pgfusepath{stroke,fill}%
\end{pgfscope}%
\begin{pgfscope}%
\pgfpathrectangle{\pgfqpoint{3.793912in}{0.557870in}}{\pgfqpoint{2.446088in}{1.684734in}}%
\pgfusepath{clip}%
\pgfsetbuttcap%
\pgfsetroundjoin%
\definecolor{currentfill}{rgb}{0.298039,0.447059,0.690196}%
\pgfsetfillcolor{currentfill}%
\pgfsetlinewidth{1.003750pt}%
\definecolor{currentstroke}{rgb}{0.298039,0.447059,0.690196}%
\pgfsetstrokecolor{currentstroke}%
\pgfsetdash{}{0pt}%
\pgfpathmoveto{\pgfqpoint{3.905098in}{2.125798in}}%
\pgfpathcurveto{\pgfqpoint{3.913334in}{2.125798in}}{\pgfqpoint{3.921234in}{2.129070in}}{\pgfqpoint{3.927058in}{2.134894in}}%
\pgfpathcurveto{\pgfqpoint{3.932882in}{2.140718in}}{\pgfqpoint{3.936155in}{2.148618in}}{\pgfqpoint{3.936155in}{2.156854in}}%
\pgfpathcurveto{\pgfqpoint{3.936155in}{2.165091in}}{\pgfqpoint{3.932882in}{2.172991in}}{\pgfqpoint{3.927058in}{2.178814in}}%
\pgfpathcurveto{\pgfqpoint{3.921234in}{2.184638in}}{\pgfqpoint{3.913334in}{2.187911in}}{\pgfqpoint{3.905098in}{2.187911in}}%
\pgfpathcurveto{\pgfqpoint{3.896862in}{2.187911in}}{\pgfqpoint{3.888962in}{2.184638in}}{\pgfqpoint{3.883138in}{2.178814in}}%
\pgfpathcurveto{\pgfqpoint{3.877314in}{2.172991in}}{\pgfqpoint{3.874042in}{2.165091in}}{\pgfqpoint{3.874042in}{2.156854in}}%
\pgfpathcurveto{\pgfqpoint{3.874042in}{2.148618in}}{\pgfqpoint{3.877314in}{2.140718in}}{\pgfqpoint{3.883138in}{2.134894in}}%
\pgfpathcurveto{\pgfqpoint{3.888962in}{2.129070in}}{\pgfqpoint{3.896862in}{2.125798in}}{\pgfqpoint{3.905098in}{2.125798in}}%
\pgfpathclose%
\pgfusepath{stroke,fill}%
\end{pgfscope}%
\begin{pgfscope}%
\pgfpathrectangle{\pgfqpoint{3.793912in}{0.557870in}}{\pgfqpoint{2.446088in}{1.684734in}}%
\pgfusepath{clip}%
\pgfsetbuttcap%
\pgfsetroundjoin%
\definecolor{currentfill}{rgb}{0.298039,0.447059,0.690196}%
\pgfsetfillcolor{currentfill}%
\pgfsetlinewidth{1.003750pt}%
\definecolor{currentstroke}{rgb}{0.298039,0.447059,0.690196}%
\pgfsetstrokecolor{currentstroke}%
\pgfsetdash{}{0pt}%
\pgfpathmoveto{\pgfqpoint{4.365177in}{1.813980in}}%
\pgfpathcurveto{\pgfqpoint{4.373414in}{1.813980in}}{\pgfqpoint{4.381314in}{1.817252in}}{\pgfqpoint{4.387137in}{1.823076in}}%
\pgfpathcurveto{\pgfqpoint{4.392961in}{1.828900in}}{\pgfqpoint{4.396234in}{1.836800in}}{\pgfqpoint{4.396234in}{1.845036in}}%
\pgfpathcurveto{\pgfqpoint{4.396234in}{1.853273in}}{\pgfqpoint{4.392961in}{1.861173in}}{\pgfqpoint{4.387137in}{1.866997in}}%
\pgfpathcurveto{\pgfqpoint{4.381314in}{1.872821in}}{\pgfqpoint{4.373414in}{1.876093in}}{\pgfqpoint{4.365177in}{1.876093in}}%
\pgfpathcurveto{\pgfqpoint{4.356941in}{1.876093in}}{\pgfqpoint{4.349041in}{1.872821in}}{\pgfqpoint{4.343217in}{1.866997in}}%
\pgfpathcurveto{\pgfqpoint{4.337393in}{1.861173in}}{\pgfqpoint{4.334121in}{1.853273in}}{\pgfqpoint{4.334121in}{1.845036in}}%
\pgfpathcurveto{\pgfqpoint{4.334121in}{1.836800in}}{\pgfqpoint{4.337393in}{1.828900in}}{\pgfqpoint{4.343217in}{1.823076in}}%
\pgfpathcurveto{\pgfqpoint{4.349041in}{1.817252in}}{\pgfqpoint{4.356941in}{1.813980in}}{\pgfqpoint{4.365177in}{1.813980in}}%
\pgfpathclose%
\pgfusepath{stroke,fill}%
\end{pgfscope}%
\begin{pgfscope}%
\pgfpathrectangle{\pgfqpoint{3.793912in}{0.557870in}}{\pgfqpoint{2.446088in}{1.684734in}}%
\pgfusepath{clip}%
\pgfsetbuttcap%
\pgfsetroundjoin%
\definecolor{currentfill}{rgb}{0.298039,0.447059,0.690196}%
\pgfsetfillcolor{currentfill}%
\pgfsetlinewidth{1.003750pt}%
\definecolor{currentstroke}{rgb}{0.298039,0.447059,0.690196}%
\pgfsetstrokecolor{currentstroke}%
\pgfsetdash{}{0pt}%
\pgfpathmoveto{\pgfqpoint{3.905098in}{2.125798in}}%
\pgfpathcurveto{\pgfqpoint{3.913334in}{2.125798in}}{\pgfqpoint{3.921234in}{2.129070in}}{\pgfqpoint{3.927058in}{2.134894in}}%
\pgfpathcurveto{\pgfqpoint{3.932882in}{2.140718in}}{\pgfqpoint{3.936155in}{2.148618in}}{\pgfqpoint{3.936155in}{2.156854in}}%
\pgfpathcurveto{\pgfqpoint{3.936155in}{2.165091in}}{\pgfqpoint{3.932882in}{2.172991in}}{\pgfqpoint{3.927058in}{2.178814in}}%
\pgfpathcurveto{\pgfqpoint{3.921234in}{2.184638in}}{\pgfqpoint{3.913334in}{2.187911in}}{\pgfqpoint{3.905098in}{2.187911in}}%
\pgfpathcurveto{\pgfqpoint{3.896862in}{2.187911in}}{\pgfqpoint{3.888962in}{2.184638in}}{\pgfqpoint{3.883138in}{2.178814in}}%
\pgfpathcurveto{\pgfqpoint{3.877314in}{2.172991in}}{\pgfqpoint{3.874042in}{2.165091in}}{\pgfqpoint{3.874042in}{2.156854in}}%
\pgfpathcurveto{\pgfqpoint{3.874042in}{2.148618in}}{\pgfqpoint{3.877314in}{2.140718in}}{\pgfqpoint{3.883138in}{2.134894in}}%
\pgfpathcurveto{\pgfqpoint{3.888962in}{2.129070in}}{\pgfqpoint{3.896862in}{2.125798in}}{\pgfqpoint{3.905098in}{2.125798in}}%
\pgfpathclose%
\pgfusepath{stroke,fill}%
\end{pgfscope}%
\begin{pgfscope}%
\pgfpathrectangle{\pgfqpoint{3.793912in}{0.557870in}}{\pgfqpoint{2.446088in}{1.684734in}}%
\pgfusepath{clip}%
\pgfsetbuttcap%
\pgfsetroundjoin%
\definecolor{currentfill}{rgb}{0.298039,0.447059,0.690196}%
\pgfsetfillcolor{currentfill}%
\pgfsetlinewidth{1.003750pt}%
\definecolor{currentstroke}{rgb}{0.298039,0.447059,0.690196}%
\pgfsetstrokecolor{currentstroke}%
\pgfsetdash{}{0pt}%
\pgfpathmoveto{\pgfqpoint{3.905098in}{2.125798in}}%
\pgfpathcurveto{\pgfqpoint{3.913334in}{2.125798in}}{\pgfqpoint{3.921234in}{2.129070in}}{\pgfqpoint{3.927058in}{2.134894in}}%
\pgfpathcurveto{\pgfqpoint{3.932882in}{2.140718in}}{\pgfqpoint{3.936155in}{2.148618in}}{\pgfqpoint{3.936155in}{2.156854in}}%
\pgfpathcurveto{\pgfqpoint{3.936155in}{2.165091in}}{\pgfqpoint{3.932882in}{2.172991in}}{\pgfqpoint{3.927058in}{2.178814in}}%
\pgfpathcurveto{\pgfqpoint{3.921234in}{2.184638in}}{\pgfqpoint{3.913334in}{2.187911in}}{\pgfqpoint{3.905098in}{2.187911in}}%
\pgfpathcurveto{\pgfqpoint{3.896862in}{2.187911in}}{\pgfqpoint{3.888962in}{2.184638in}}{\pgfqpoint{3.883138in}{2.178814in}}%
\pgfpathcurveto{\pgfqpoint{3.877314in}{2.172991in}}{\pgfqpoint{3.874042in}{2.165091in}}{\pgfqpoint{3.874042in}{2.156854in}}%
\pgfpathcurveto{\pgfqpoint{3.874042in}{2.148618in}}{\pgfqpoint{3.877314in}{2.140718in}}{\pgfqpoint{3.883138in}{2.134894in}}%
\pgfpathcurveto{\pgfqpoint{3.888962in}{2.129070in}}{\pgfqpoint{3.896862in}{2.125798in}}{\pgfqpoint{3.905098in}{2.125798in}}%
\pgfpathclose%
\pgfusepath{stroke,fill}%
\end{pgfscope}%
\begin{pgfscope}%
\pgfpathrectangle{\pgfqpoint{3.793912in}{0.557870in}}{\pgfqpoint{2.446088in}{1.684734in}}%
\pgfusepath{clip}%
\pgfsetbuttcap%
\pgfsetroundjoin%
\definecolor{currentfill}{rgb}{0.298039,0.447059,0.690196}%
\pgfsetfillcolor{currentfill}%
\pgfsetlinewidth{1.003750pt}%
\definecolor{currentstroke}{rgb}{0.298039,0.447059,0.690196}%
\pgfsetstrokecolor{currentstroke}%
\pgfsetdash{}{0pt}%
\pgfpathmoveto{\pgfqpoint{4.825256in}{1.850664in}}%
\pgfpathcurveto{\pgfqpoint{4.833493in}{1.850664in}}{\pgfqpoint{4.841393in}{1.853937in}}{\pgfqpoint{4.847217in}{1.859761in}}%
\pgfpathcurveto{\pgfqpoint{4.853041in}{1.865584in}}{\pgfqpoint{4.856313in}{1.873484in}}{\pgfqpoint{4.856313in}{1.881721in}}%
\pgfpathcurveto{\pgfqpoint{4.856313in}{1.889957in}}{\pgfqpoint{4.853041in}{1.897857in}}{\pgfqpoint{4.847217in}{1.903681in}}%
\pgfpathcurveto{\pgfqpoint{4.841393in}{1.909505in}}{\pgfqpoint{4.833493in}{1.912777in}}{\pgfqpoint{4.825256in}{1.912777in}}%
\pgfpathcurveto{\pgfqpoint{4.817020in}{1.912777in}}{\pgfqpoint{4.809120in}{1.909505in}}{\pgfqpoint{4.803296in}{1.903681in}}%
\pgfpathcurveto{\pgfqpoint{4.797472in}{1.897857in}}{\pgfqpoint{4.794200in}{1.889957in}}{\pgfqpoint{4.794200in}{1.881721in}}%
\pgfpathcurveto{\pgfqpoint{4.794200in}{1.873484in}}{\pgfqpoint{4.797472in}{1.865584in}}{\pgfqpoint{4.803296in}{1.859761in}}%
\pgfpathcurveto{\pgfqpoint{4.809120in}{1.853937in}}{\pgfqpoint{4.817020in}{1.850664in}}{\pgfqpoint{4.825256in}{1.850664in}}%
\pgfpathclose%
\pgfusepath{stroke,fill}%
\end{pgfscope}%
\begin{pgfscope}%
\pgfpathrectangle{\pgfqpoint{3.793912in}{0.557870in}}{\pgfqpoint{2.446088in}{1.684734in}}%
\pgfusepath{clip}%
\pgfsetbuttcap%
\pgfsetroundjoin%
\definecolor{currentfill}{rgb}{0.298039,0.447059,0.690196}%
\pgfsetfillcolor{currentfill}%
\pgfsetlinewidth{1.003750pt}%
\definecolor{currentstroke}{rgb}{0.298039,0.447059,0.690196}%
\pgfsetstrokecolor{currentstroke}%
\pgfsetdash{}{0pt}%
\pgfpathmoveto{\pgfqpoint{3.905098in}{2.125798in}}%
\pgfpathcurveto{\pgfqpoint{3.913334in}{2.125798in}}{\pgfqpoint{3.921234in}{2.129070in}}{\pgfqpoint{3.927058in}{2.134894in}}%
\pgfpathcurveto{\pgfqpoint{3.932882in}{2.140718in}}{\pgfqpoint{3.936155in}{2.148618in}}{\pgfqpoint{3.936155in}{2.156854in}}%
\pgfpathcurveto{\pgfqpoint{3.936155in}{2.165091in}}{\pgfqpoint{3.932882in}{2.172991in}}{\pgfqpoint{3.927058in}{2.178814in}}%
\pgfpathcurveto{\pgfqpoint{3.921234in}{2.184638in}}{\pgfqpoint{3.913334in}{2.187911in}}{\pgfqpoint{3.905098in}{2.187911in}}%
\pgfpathcurveto{\pgfqpoint{3.896862in}{2.187911in}}{\pgfqpoint{3.888962in}{2.184638in}}{\pgfqpoint{3.883138in}{2.178814in}}%
\pgfpathcurveto{\pgfqpoint{3.877314in}{2.172991in}}{\pgfqpoint{3.874042in}{2.165091in}}{\pgfqpoint{3.874042in}{2.156854in}}%
\pgfpathcurveto{\pgfqpoint{3.874042in}{2.148618in}}{\pgfqpoint{3.877314in}{2.140718in}}{\pgfqpoint{3.883138in}{2.134894in}}%
\pgfpathcurveto{\pgfqpoint{3.888962in}{2.129070in}}{\pgfqpoint{3.896862in}{2.125798in}}{\pgfqpoint{3.905098in}{2.125798in}}%
\pgfpathclose%
\pgfusepath{stroke,fill}%
\end{pgfscope}%
\begin{pgfscope}%
\pgfpathrectangle{\pgfqpoint{3.793912in}{0.557870in}}{\pgfqpoint{2.446088in}{1.684734in}}%
\pgfusepath{clip}%
\pgfsetbuttcap%
\pgfsetroundjoin%
\definecolor{currentfill}{rgb}{0.298039,0.447059,0.690196}%
\pgfsetfillcolor{currentfill}%
\pgfsetlinewidth{1.003750pt}%
\definecolor{currentstroke}{rgb}{0.298039,0.447059,0.690196}%
\pgfsetstrokecolor{currentstroke}%
\pgfsetdash{}{0pt}%
\pgfpathmoveto{\pgfqpoint{4.595217in}{1.520504in}}%
\pgfpathcurveto{\pgfqpoint{4.603453in}{1.520504in}}{\pgfqpoint{4.611353in}{1.523776in}}{\pgfqpoint{4.617177in}{1.529600in}}%
\pgfpathcurveto{\pgfqpoint{4.623001in}{1.535424in}}{\pgfqpoint{4.626273in}{1.543324in}}{\pgfqpoint{4.626273in}{1.551561in}}%
\pgfpathcurveto{\pgfqpoint{4.626273in}{1.559797in}}{\pgfqpoint{4.623001in}{1.567697in}}{\pgfqpoint{4.617177in}{1.573521in}}%
\pgfpathcurveto{\pgfqpoint{4.611353in}{1.579345in}}{\pgfqpoint{4.603453in}{1.582617in}}{\pgfqpoint{4.595217in}{1.582617in}}%
\pgfpathcurveto{\pgfqpoint{4.586981in}{1.582617in}}{\pgfqpoint{4.579081in}{1.579345in}}{\pgfqpoint{4.573257in}{1.573521in}}%
\pgfpathcurveto{\pgfqpoint{4.567433in}{1.567697in}}{\pgfqpoint{4.564160in}{1.559797in}}{\pgfqpoint{4.564160in}{1.551561in}}%
\pgfpathcurveto{\pgfqpoint{4.564160in}{1.543324in}}{\pgfqpoint{4.567433in}{1.535424in}}{\pgfqpoint{4.573257in}{1.529600in}}%
\pgfpathcurveto{\pgfqpoint{4.579081in}{1.523776in}}{\pgfqpoint{4.586981in}{1.520504in}}{\pgfqpoint{4.595217in}{1.520504in}}%
\pgfpathclose%
\pgfusepath{stroke,fill}%
\end{pgfscope}%
\begin{pgfscope}%
\pgfpathrectangle{\pgfqpoint{3.793912in}{0.557870in}}{\pgfqpoint{2.446088in}{1.684734in}}%
\pgfusepath{clip}%
\pgfsetbuttcap%
\pgfsetroundjoin%
\definecolor{currentfill}{rgb}{0.298039,0.447059,0.690196}%
\pgfsetfillcolor{currentfill}%
\pgfsetlinewidth{1.003750pt}%
\definecolor{currentstroke}{rgb}{0.298039,0.447059,0.690196}%
\pgfsetstrokecolor{currentstroke}%
\pgfsetdash{}{0pt}%
\pgfpathmoveto{\pgfqpoint{3.905098in}{2.116627in}}%
\pgfpathcurveto{\pgfqpoint{3.913334in}{2.116627in}}{\pgfqpoint{3.921234in}{2.119899in}}{\pgfqpoint{3.927058in}{2.125723in}}%
\pgfpathcurveto{\pgfqpoint{3.932882in}{2.131547in}}{\pgfqpoint{3.936155in}{2.139447in}}{\pgfqpoint{3.936155in}{2.147683in}}%
\pgfpathcurveto{\pgfqpoint{3.936155in}{2.155919in}}{\pgfqpoint{3.932882in}{2.163819in}}{\pgfqpoint{3.927058in}{2.169643in}}%
\pgfpathcurveto{\pgfqpoint{3.921234in}{2.175467in}}{\pgfqpoint{3.913334in}{2.178740in}}{\pgfqpoint{3.905098in}{2.178740in}}%
\pgfpathcurveto{\pgfqpoint{3.896862in}{2.178740in}}{\pgfqpoint{3.888962in}{2.175467in}}{\pgfqpoint{3.883138in}{2.169643in}}%
\pgfpathcurveto{\pgfqpoint{3.877314in}{2.163819in}}{\pgfqpoint{3.874042in}{2.155919in}}{\pgfqpoint{3.874042in}{2.147683in}}%
\pgfpathcurveto{\pgfqpoint{3.874042in}{2.139447in}}{\pgfqpoint{3.877314in}{2.131547in}}{\pgfqpoint{3.883138in}{2.125723in}}%
\pgfpathcurveto{\pgfqpoint{3.888962in}{2.119899in}}{\pgfqpoint{3.896862in}{2.116627in}}{\pgfqpoint{3.905098in}{2.116627in}}%
\pgfpathclose%
\pgfusepath{stroke,fill}%
\end{pgfscope}%
\begin{pgfscope}%
\pgfpathrectangle{\pgfqpoint{3.793912in}{0.557870in}}{\pgfqpoint{2.446088in}{1.684734in}}%
\pgfusepath{clip}%
\pgfsetbuttcap%
\pgfsetroundjoin%
\definecolor{currentfill}{rgb}{0.298039,0.447059,0.690196}%
\pgfsetfillcolor{currentfill}%
\pgfsetlinewidth{1.003750pt}%
\definecolor{currentstroke}{rgb}{0.298039,0.447059,0.690196}%
\pgfsetstrokecolor{currentstroke}%
\pgfsetdash{}{0pt}%
\pgfpathmoveto{\pgfqpoint{4.978616in}{1.548017in}}%
\pgfpathcurveto{\pgfqpoint{4.986852in}{1.548017in}}{\pgfqpoint{4.994753in}{1.551290in}}{\pgfqpoint{5.000576in}{1.557114in}}%
\pgfpathcurveto{\pgfqpoint{5.006400in}{1.562938in}}{\pgfqpoint{5.009673in}{1.570838in}}{\pgfqpoint{5.009673in}{1.579074in}}%
\pgfpathcurveto{\pgfqpoint{5.009673in}{1.587310in}}{\pgfqpoint{5.006400in}{1.595210in}}{\pgfqpoint{5.000576in}{1.601034in}}%
\pgfpathcurveto{\pgfqpoint{4.994753in}{1.606858in}}{\pgfqpoint{4.986852in}{1.610130in}}{\pgfqpoint{4.978616in}{1.610130in}}%
\pgfpathcurveto{\pgfqpoint{4.970380in}{1.610130in}}{\pgfqpoint{4.962480in}{1.606858in}}{\pgfqpoint{4.956656in}{1.601034in}}%
\pgfpathcurveto{\pgfqpoint{4.950832in}{1.595210in}}{\pgfqpoint{4.947560in}{1.587310in}}{\pgfqpoint{4.947560in}{1.579074in}}%
\pgfpathcurveto{\pgfqpoint{4.947560in}{1.570838in}}{\pgfqpoint{4.950832in}{1.562938in}}{\pgfqpoint{4.956656in}{1.557114in}}%
\pgfpathcurveto{\pgfqpoint{4.962480in}{1.551290in}}{\pgfqpoint{4.970380in}{1.548017in}}{\pgfqpoint{4.978616in}{1.548017in}}%
\pgfpathclose%
\pgfusepath{stroke,fill}%
\end{pgfscope}%
\begin{pgfscope}%
\pgfpathrectangle{\pgfqpoint{3.793912in}{0.557870in}}{\pgfqpoint{2.446088in}{1.684734in}}%
\pgfusepath{clip}%
\pgfsetbuttcap%
\pgfsetroundjoin%
\definecolor{currentfill}{rgb}{0.298039,0.447059,0.690196}%
\pgfsetfillcolor{currentfill}%
\pgfsetlinewidth{1.003750pt}%
\definecolor{currentstroke}{rgb}{0.298039,0.447059,0.690196}%
\pgfsetstrokecolor{currentstroke}%
\pgfsetdash{}{0pt}%
\pgfpathmoveto{\pgfqpoint{3.905098in}{2.125798in}}%
\pgfpathcurveto{\pgfqpoint{3.913334in}{2.125798in}}{\pgfqpoint{3.921234in}{2.129070in}}{\pgfqpoint{3.927058in}{2.134894in}}%
\pgfpathcurveto{\pgfqpoint{3.932882in}{2.140718in}}{\pgfqpoint{3.936155in}{2.148618in}}{\pgfqpoint{3.936155in}{2.156854in}}%
\pgfpathcurveto{\pgfqpoint{3.936155in}{2.165091in}}{\pgfqpoint{3.932882in}{2.172991in}}{\pgfqpoint{3.927058in}{2.178814in}}%
\pgfpathcurveto{\pgfqpoint{3.921234in}{2.184638in}}{\pgfqpoint{3.913334in}{2.187911in}}{\pgfqpoint{3.905098in}{2.187911in}}%
\pgfpathcurveto{\pgfqpoint{3.896862in}{2.187911in}}{\pgfqpoint{3.888962in}{2.184638in}}{\pgfqpoint{3.883138in}{2.178814in}}%
\pgfpathcurveto{\pgfqpoint{3.877314in}{2.172991in}}{\pgfqpoint{3.874042in}{2.165091in}}{\pgfqpoint{3.874042in}{2.156854in}}%
\pgfpathcurveto{\pgfqpoint{3.874042in}{2.148618in}}{\pgfqpoint{3.877314in}{2.140718in}}{\pgfqpoint{3.883138in}{2.134894in}}%
\pgfpathcurveto{\pgfqpoint{3.888962in}{2.129070in}}{\pgfqpoint{3.896862in}{2.125798in}}{\pgfqpoint{3.905098in}{2.125798in}}%
\pgfpathclose%
\pgfusepath{stroke,fill}%
\end{pgfscope}%
\begin{pgfscope}%
\pgfpathrectangle{\pgfqpoint{3.793912in}{0.557870in}}{\pgfqpoint{2.446088in}{1.684734in}}%
\pgfusepath{clip}%
\pgfsetbuttcap%
\pgfsetroundjoin%
\definecolor{currentfill}{rgb}{0.298039,0.447059,0.690196}%
\pgfsetfillcolor{currentfill}%
\pgfsetlinewidth{1.003750pt}%
\definecolor{currentstroke}{rgb}{0.298039,0.447059,0.690196}%
\pgfsetstrokecolor{currentstroke}%
\pgfsetdash{}{0pt}%
\pgfpathmoveto{\pgfqpoint{3.905098in}{2.125798in}}%
\pgfpathcurveto{\pgfqpoint{3.913334in}{2.125798in}}{\pgfqpoint{3.921234in}{2.129070in}}{\pgfqpoint{3.927058in}{2.134894in}}%
\pgfpathcurveto{\pgfqpoint{3.932882in}{2.140718in}}{\pgfqpoint{3.936155in}{2.148618in}}{\pgfqpoint{3.936155in}{2.156854in}}%
\pgfpathcurveto{\pgfqpoint{3.936155in}{2.165091in}}{\pgfqpoint{3.932882in}{2.172991in}}{\pgfqpoint{3.927058in}{2.178814in}}%
\pgfpathcurveto{\pgfqpoint{3.921234in}{2.184638in}}{\pgfqpoint{3.913334in}{2.187911in}}{\pgfqpoint{3.905098in}{2.187911in}}%
\pgfpathcurveto{\pgfqpoint{3.896862in}{2.187911in}}{\pgfqpoint{3.888962in}{2.184638in}}{\pgfqpoint{3.883138in}{2.178814in}}%
\pgfpathcurveto{\pgfqpoint{3.877314in}{2.172991in}}{\pgfqpoint{3.874042in}{2.165091in}}{\pgfqpoint{3.874042in}{2.156854in}}%
\pgfpathcurveto{\pgfqpoint{3.874042in}{2.148618in}}{\pgfqpoint{3.877314in}{2.140718in}}{\pgfqpoint{3.883138in}{2.134894in}}%
\pgfpathcurveto{\pgfqpoint{3.888962in}{2.129070in}}{\pgfqpoint{3.896862in}{2.125798in}}{\pgfqpoint{3.905098in}{2.125798in}}%
\pgfpathclose%
\pgfusepath{stroke,fill}%
\end{pgfscope}%
\begin{pgfscope}%
\pgfpathrectangle{\pgfqpoint{3.793912in}{0.557870in}}{\pgfqpoint{2.446088in}{1.684734in}}%
\pgfusepath{clip}%
\pgfsetbuttcap%
\pgfsetroundjoin%
\definecolor{currentfill}{rgb}{0.298039,0.447059,0.690196}%
\pgfsetfillcolor{currentfill}%
\pgfsetlinewidth{1.003750pt}%
\definecolor{currentstroke}{rgb}{0.298039,0.447059,0.690196}%
\pgfsetstrokecolor{currentstroke}%
\pgfsetdash{}{0pt}%
\pgfpathmoveto{\pgfqpoint{3.905098in}{2.125798in}}%
\pgfpathcurveto{\pgfqpoint{3.913334in}{2.125798in}}{\pgfqpoint{3.921234in}{2.129070in}}{\pgfqpoint{3.927058in}{2.134894in}}%
\pgfpathcurveto{\pgfqpoint{3.932882in}{2.140718in}}{\pgfqpoint{3.936155in}{2.148618in}}{\pgfqpoint{3.936155in}{2.156854in}}%
\pgfpathcurveto{\pgfqpoint{3.936155in}{2.165091in}}{\pgfqpoint{3.932882in}{2.172991in}}{\pgfqpoint{3.927058in}{2.178814in}}%
\pgfpathcurveto{\pgfqpoint{3.921234in}{2.184638in}}{\pgfqpoint{3.913334in}{2.187911in}}{\pgfqpoint{3.905098in}{2.187911in}}%
\pgfpathcurveto{\pgfqpoint{3.896862in}{2.187911in}}{\pgfqpoint{3.888962in}{2.184638in}}{\pgfqpoint{3.883138in}{2.178814in}}%
\pgfpathcurveto{\pgfqpoint{3.877314in}{2.172991in}}{\pgfqpoint{3.874042in}{2.165091in}}{\pgfqpoint{3.874042in}{2.156854in}}%
\pgfpathcurveto{\pgfqpoint{3.874042in}{2.148618in}}{\pgfqpoint{3.877314in}{2.140718in}}{\pgfqpoint{3.883138in}{2.134894in}}%
\pgfpathcurveto{\pgfqpoint{3.888962in}{2.129070in}}{\pgfqpoint{3.896862in}{2.125798in}}{\pgfqpoint{3.905098in}{2.125798in}}%
\pgfpathclose%
\pgfusepath{stroke,fill}%
\end{pgfscope}%
\begin{pgfscope}%
\pgfpathrectangle{\pgfqpoint{3.793912in}{0.557870in}}{\pgfqpoint{2.446088in}{1.684734in}}%
\pgfusepath{clip}%
\pgfsetbuttcap%
\pgfsetroundjoin%
\definecolor{currentfill}{rgb}{0.298039,0.447059,0.690196}%
\pgfsetfillcolor{currentfill}%
\pgfsetlinewidth{1.003750pt}%
\definecolor{currentstroke}{rgb}{0.298039,0.447059,0.690196}%
\pgfsetstrokecolor{currentstroke}%
\pgfsetdash{}{0pt}%
\pgfpathmoveto{\pgfqpoint{4.058458in}{1.667242in}}%
\pgfpathcurveto{\pgfqpoint{4.066694in}{1.667242in}}{\pgfqpoint{4.074594in}{1.670514in}}{\pgfqpoint{4.080418in}{1.676338in}}%
\pgfpathcurveto{\pgfqpoint{4.086242in}{1.682162in}}{\pgfqpoint{4.089514in}{1.690062in}}{\pgfqpoint{4.089514in}{1.698298in}}%
\pgfpathcurveto{\pgfqpoint{4.089514in}{1.706535in}}{\pgfqpoint{4.086242in}{1.714435in}}{\pgfqpoint{4.080418in}{1.720259in}}%
\pgfpathcurveto{\pgfqpoint{4.074594in}{1.726083in}}{\pgfqpoint{4.066694in}{1.729355in}}{\pgfqpoint{4.058458in}{1.729355in}}%
\pgfpathcurveto{\pgfqpoint{4.050221in}{1.729355in}}{\pgfqpoint{4.042321in}{1.726083in}}{\pgfqpoint{4.036498in}{1.720259in}}%
\pgfpathcurveto{\pgfqpoint{4.030674in}{1.714435in}}{\pgfqpoint{4.027401in}{1.706535in}}{\pgfqpoint{4.027401in}{1.698298in}}%
\pgfpathcurveto{\pgfqpoint{4.027401in}{1.690062in}}{\pgfqpoint{4.030674in}{1.682162in}}{\pgfqpoint{4.036498in}{1.676338in}}%
\pgfpathcurveto{\pgfqpoint{4.042321in}{1.670514in}}{\pgfqpoint{4.050221in}{1.667242in}}{\pgfqpoint{4.058458in}{1.667242in}}%
\pgfpathclose%
\pgfusepath{stroke,fill}%
\end{pgfscope}%
\begin{pgfscope}%
\pgfpathrectangle{\pgfqpoint{3.793912in}{0.557870in}}{\pgfqpoint{2.446088in}{1.684734in}}%
\pgfusepath{clip}%
\pgfsetbuttcap%
\pgfsetroundjoin%
\definecolor{currentfill}{rgb}{0.298039,0.447059,0.690196}%
\pgfsetfillcolor{currentfill}%
\pgfsetlinewidth{1.003750pt}%
\definecolor{currentstroke}{rgb}{0.298039,0.447059,0.690196}%
\pgfsetstrokecolor{currentstroke}%
\pgfsetdash{}{0pt}%
\pgfpathmoveto{\pgfqpoint{5.822095in}{1.676413in}}%
\pgfpathcurveto{\pgfqpoint{5.830331in}{1.676413in}}{\pgfqpoint{5.838231in}{1.679685in}}{\pgfqpoint{5.844055in}{1.685509in}}%
\pgfpathcurveto{\pgfqpoint{5.849879in}{1.691333in}}{\pgfqpoint{5.853151in}{1.699233in}}{\pgfqpoint{5.853151in}{1.707470in}}%
\pgfpathcurveto{\pgfqpoint{5.853151in}{1.715706in}}{\pgfqpoint{5.849879in}{1.723606in}}{\pgfqpoint{5.844055in}{1.729430in}}%
\pgfpathcurveto{\pgfqpoint{5.838231in}{1.735254in}}{\pgfqpoint{5.830331in}{1.738526in}}{\pgfqpoint{5.822095in}{1.738526in}}%
\pgfpathcurveto{\pgfqpoint{5.813858in}{1.738526in}}{\pgfqpoint{5.805958in}{1.735254in}}{\pgfqpoint{5.800134in}{1.729430in}}%
\pgfpathcurveto{\pgfqpoint{5.794311in}{1.723606in}}{\pgfqpoint{5.791038in}{1.715706in}}{\pgfqpoint{5.791038in}{1.707470in}}%
\pgfpathcurveto{\pgfqpoint{5.791038in}{1.699233in}}{\pgfqpoint{5.794311in}{1.691333in}}{\pgfqpoint{5.800134in}{1.685509in}}%
\pgfpathcurveto{\pgfqpoint{5.805958in}{1.679685in}}{\pgfqpoint{5.813858in}{1.676413in}}{\pgfqpoint{5.822095in}{1.676413in}}%
\pgfpathclose%
\pgfusepath{stroke,fill}%
\end{pgfscope}%
\begin{pgfscope}%
\pgfpathrectangle{\pgfqpoint{3.793912in}{0.557870in}}{\pgfqpoint{2.446088in}{1.684734in}}%
\pgfusepath{clip}%
\pgfsetbuttcap%
\pgfsetroundjoin%
\definecolor{currentfill}{rgb}{0.298039,0.447059,0.690196}%
\pgfsetfillcolor{currentfill}%
\pgfsetlinewidth{1.003750pt}%
\definecolor{currentstroke}{rgb}{0.298039,0.447059,0.690196}%
\pgfsetstrokecolor{currentstroke}%
\pgfsetdash{}{0pt}%
\pgfpathmoveto{\pgfqpoint{5.975454in}{1.263713in}}%
\pgfpathcurveto{\pgfqpoint{5.983691in}{1.263713in}}{\pgfqpoint{5.991591in}{1.266985in}}{\pgfqpoint{5.997415in}{1.272809in}}%
\pgfpathcurveto{\pgfqpoint{6.003239in}{1.278633in}}{\pgfqpoint{6.006511in}{1.286533in}}{\pgfqpoint{6.006511in}{1.294769in}}%
\pgfpathcurveto{\pgfqpoint{6.006511in}{1.303006in}}{\pgfqpoint{6.003239in}{1.310906in}}{\pgfqpoint{5.997415in}{1.316730in}}%
\pgfpathcurveto{\pgfqpoint{5.991591in}{1.322554in}}{\pgfqpoint{5.983691in}{1.325826in}}{\pgfqpoint{5.975454in}{1.325826in}}%
\pgfpathcurveto{\pgfqpoint{5.967218in}{1.325826in}}{\pgfqpoint{5.959318in}{1.322554in}}{\pgfqpoint{5.953494in}{1.316730in}}%
\pgfpathcurveto{\pgfqpoint{5.947670in}{1.310906in}}{\pgfqpoint{5.944398in}{1.303006in}}{\pgfqpoint{5.944398in}{1.294769in}}%
\pgfpathcurveto{\pgfqpoint{5.944398in}{1.286533in}}{\pgfqpoint{5.947670in}{1.278633in}}{\pgfqpoint{5.953494in}{1.272809in}}%
\pgfpathcurveto{\pgfqpoint{5.959318in}{1.266985in}}{\pgfqpoint{5.967218in}{1.263713in}}{\pgfqpoint{5.975454in}{1.263713in}}%
\pgfpathclose%
\pgfusepath{stroke,fill}%
\end{pgfscope}%
\begin{pgfscope}%
\pgfpathrectangle{\pgfqpoint{3.793912in}{0.557870in}}{\pgfqpoint{2.446088in}{1.684734in}}%
\pgfusepath{clip}%
\pgfsetbuttcap%
\pgfsetroundjoin%
\definecolor{currentfill}{rgb}{0.298039,0.447059,0.690196}%
\pgfsetfillcolor{currentfill}%
\pgfsetlinewidth{1.003750pt}%
\definecolor{currentstroke}{rgb}{0.298039,0.447059,0.690196}%
\pgfsetstrokecolor{currentstroke}%
\pgfsetdash{}{0pt}%
\pgfpathmoveto{\pgfqpoint{5.975454in}{1.309568in}}%
\pgfpathcurveto{\pgfqpoint{5.983691in}{1.309568in}}{\pgfqpoint{5.991591in}{1.312841in}}{\pgfqpoint{5.997415in}{1.318665in}}%
\pgfpathcurveto{\pgfqpoint{6.003239in}{1.324489in}}{\pgfqpoint{6.006511in}{1.332389in}}{\pgfqpoint{6.006511in}{1.340625in}}%
\pgfpathcurveto{\pgfqpoint{6.006511in}{1.348861in}}{\pgfqpoint{6.003239in}{1.356761in}}{\pgfqpoint{5.997415in}{1.362585in}}%
\pgfpathcurveto{\pgfqpoint{5.991591in}{1.368409in}}{\pgfqpoint{5.983691in}{1.371681in}}{\pgfqpoint{5.975454in}{1.371681in}}%
\pgfpathcurveto{\pgfqpoint{5.967218in}{1.371681in}}{\pgfqpoint{5.959318in}{1.368409in}}{\pgfqpoint{5.953494in}{1.362585in}}%
\pgfpathcurveto{\pgfqpoint{5.947670in}{1.356761in}}{\pgfqpoint{5.944398in}{1.348861in}}{\pgfqpoint{5.944398in}{1.340625in}}%
\pgfpathcurveto{\pgfqpoint{5.944398in}{1.332389in}}{\pgfqpoint{5.947670in}{1.324489in}}{\pgfqpoint{5.953494in}{1.318665in}}%
\pgfpathcurveto{\pgfqpoint{5.959318in}{1.312841in}}{\pgfqpoint{5.967218in}{1.309568in}}{\pgfqpoint{5.975454in}{1.309568in}}%
\pgfpathclose%
\pgfusepath{stroke,fill}%
\end{pgfscope}%
\begin{pgfscope}%
\pgfpathrectangle{\pgfqpoint{3.793912in}{0.557870in}}{\pgfqpoint{2.446088in}{1.684734in}}%
\pgfusepath{clip}%
\pgfsetbuttcap%
\pgfsetroundjoin%
\definecolor{currentfill}{rgb}{0.298039,0.447059,0.690196}%
\pgfsetfillcolor{currentfill}%
\pgfsetlinewidth{1.003750pt}%
\definecolor{currentstroke}{rgb}{0.298039,0.447059,0.690196}%
\pgfsetstrokecolor{currentstroke}%
\pgfsetdash{}{0pt}%
\pgfpathmoveto{\pgfqpoint{3.905098in}{2.125798in}}%
\pgfpathcurveto{\pgfqpoint{3.913334in}{2.125798in}}{\pgfqpoint{3.921234in}{2.129070in}}{\pgfqpoint{3.927058in}{2.134894in}}%
\pgfpathcurveto{\pgfqpoint{3.932882in}{2.140718in}}{\pgfqpoint{3.936155in}{2.148618in}}{\pgfqpoint{3.936155in}{2.156854in}}%
\pgfpathcurveto{\pgfqpoint{3.936155in}{2.165091in}}{\pgfqpoint{3.932882in}{2.172991in}}{\pgfqpoint{3.927058in}{2.178814in}}%
\pgfpathcurveto{\pgfqpoint{3.921234in}{2.184638in}}{\pgfqpoint{3.913334in}{2.187911in}}{\pgfqpoint{3.905098in}{2.187911in}}%
\pgfpathcurveto{\pgfqpoint{3.896862in}{2.187911in}}{\pgfqpoint{3.888962in}{2.184638in}}{\pgfqpoint{3.883138in}{2.178814in}}%
\pgfpathcurveto{\pgfqpoint{3.877314in}{2.172991in}}{\pgfqpoint{3.874042in}{2.165091in}}{\pgfqpoint{3.874042in}{2.156854in}}%
\pgfpathcurveto{\pgfqpoint{3.874042in}{2.148618in}}{\pgfqpoint{3.877314in}{2.140718in}}{\pgfqpoint{3.883138in}{2.134894in}}%
\pgfpathcurveto{\pgfqpoint{3.888962in}{2.129070in}}{\pgfqpoint{3.896862in}{2.125798in}}{\pgfqpoint{3.905098in}{2.125798in}}%
\pgfpathclose%
\pgfusepath{stroke,fill}%
\end{pgfscope}%
\begin{pgfscope}%
\pgfpathrectangle{\pgfqpoint{3.793912in}{0.557870in}}{\pgfqpoint{2.446088in}{1.684734in}}%
\pgfusepath{clip}%
\pgfsetbuttcap%
\pgfsetroundjoin%
\definecolor{currentfill}{rgb}{0.298039,0.447059,0.690196}%
\pgfsetfillcolor{currentfill}%
\pgfsetlinewidth{1.003750pt}%
\definecolor{currentstroke}{rgb}{0.298039,0.447059,0.690196}%
\pgfsetstrokecolor{currentstroke}%
\pgfsetdash{}{0pt}%
\pgfpathmoveto{\pgfqpoint{5.515375in}{1.850664in}}%
\pgfpathcurveto{\pgfqpoint{5.523612in}{1.850664in}}{\pgfqpoint{5.531512in}{1.853937in}}{\pgfqpoint{5.537336in}{1.859761in}}%
\pgfpathcurveto{\pgfqpoint{5.543159in}{1.865584in}}{\pgfqpoint{5.546432in}{1.873484in}}{\pgfqpoint{5.546432in}{1.881721in}}%
\pgfpathcurveto{\pgfqpoint{5.546432in}{1.889957in}}{\pgfqpoint{5.543159in}{1.897857in}}{\pgfqpoint{5.537336in}{1.903681in}}%
\pgfpathcurveto{\pgfqpoint{5.531512in}{1.909505in}}{\pgfqpoint{5.523612in}{1.912777in}}{\pgfqpoint{5.515375in}{1.912777in}}%
\pgfpathcurveto{\pgfqpoint{5.507139in}{1.912777in}}{\pgfqpoint{5.499239in}{1.909505in}}{\pgfqpoint{5.493415in}{1.903681in}}%
\pgfpathcurveto{\pgfqpoint{5.487591in}{1.897857in}}{\pgfqpoint{5.484319in}{1.889957in}}{\pgfqpoint{5.484319in}{1.881721in}}%
\pgfpathcurveto{\pgfqpoint{5.484319in}{1.873484in}}{\pgfqpoint{5.487591in}{1.865584in}}{\pgfqpoint{5.493415in}{1.859761in}}%
\pgfpathcurveto{\pgfqpoint{5.499239in}{1.853937in}}{\pgfqpoint{5.507139in}{1.850664in}}{\pgfqpoint{5.515375in}{1.850664in}}%
\pgfpathclose%
\pgfusepath{stroke,fill}%
\end{pgfscope}%
\begin{pgfscope}%
\pgfsetrectcap%
\pgfsetmiterjoin%
\pgfsetlinewidth{1.254687pt}%
\definecolor{currentstroke}{rgb}{1.000000,1.000000,1.000000}%
\pgfsetstrokecolor{currentstroke}%
\pgfsetdash{}{0pt}%
\pgfpathmoveto{\pgfqpoint{3.793912in}{0.557870in}}%
\pgfpathlineto{\pgfqpoint{3.793912in}{2.242604in}}%
\pgfusepath{stroke}%
\end{pgfscope}%
\begin{pgfscope}%
\pgfsetrectcap%
\pgfsetmiterjoin%
\pgfsetlinewidth{1.254687pt}%
\definecolor{currentstroke}{rgb}{1.000000,1.000000,1.000000}%
\pgfsetstrokecolor{currentstroke}%
\pgfsetdash{}{0pt}%
\pgfpathmoveto{\pgfqpoint{6.240000in}{0.557870in}}%
\pgfpathlineto{\pgfqpoint{6.240000in}{2.242604in}}%
\pgfusepath{stroke}%
\end{pgfscope}%
\begin{pgfscope}%
\pgfsetrectcap%
\pgfsetmiterjoin%
\pgfsetlinewidth{1.254687pt}%
\definecolor{currentstroke}{rgb}{1.000000,1.000000,1.000000}%
\pgfsetstrokecolor{currentstroke}%
\pgfsetdash{}{0pt}%
\pgfpathmoveto{\pgfqpoint{3.793912in}{0.557870in}}%
\pgfpathlineto{\pgfqpoint{6.240000in}{0.557870in}}%
\pgfusepath{stroke}%
\end{pgfscope}%
\begin{pgfscope}%
\pgfsetrectcap%
\pgfsetmiterjoin%
\pgfsetlinewidth{1.254687pt}%
\definecolor{currentstroke}{rgb}{1.000000,1.000000,1.000000}%
\pgfsetstrokecolor{currentstroke}%
\pgfsetdash{}{0pt}%
\pgfpathmoveto{\pgfqpoint{3.793912in}{2.242604in}}%
\pgfpathlineto{\pgfqpoint{6.240000in}{2.242604in}}%
\pgfusepath{stroke}%
\end{pgfscope}%
\begin{pgfscope}%
\definecolor{textcolor}{rgb}{0.150000,0.150000,0.150000}%
\pgfsetstrokecolor{textcolor}%
\pgfsetfillcolor{textcolor}%
\pgftext[x=5.016956in,y=2.325938in,,base]{\color{textcolor}\sffamily\fontsize{11.000000}{13.200000}\selectfont (b)}%
\end{pgfscope}%
\end{pgfpicture}%
\makeatother%
\endgroup%

    \caption{(a) Distribution plot of \acrshort{dor} of all \acrshort{tsc} models evaluated at two cluster centers when applied to classify patient diagnosis.
             (b) Scatter plot of the same models sensitivity, and specificity.}
    \label{fig:tsc_ind_dor_sens_spec_dist}
\end{figure}

From the distribution plot in figure \ref{fig:tsc_ind_dor_sens_spec_dist}a one can see that the majority of \acrshort{dor} are close to zero, but there are some models that acheive a \acrshort{dor} above 30. In the scatter plot in figure \ref{fig:tsc_ind_dor_sens_spec_dist}b one can see that the specificity of the models and range from $0.5$ to 1, and the sensitivity scores range from 0 to $0.93$. As with heart failure, the \acrshort{tsc} models that perform best in terms of \acrshort{dor} use data from a single view. The \acrshort{2ch} view, and \acrshort{gls} curves are the only view and curve that are used among the models that achieve the five highest \acrshort{dor}. From the table of all the model results in the appendix \ref{tab:tsc_ind_raw_results} one can see that the highest performing model in terms of \acrshort{dor} to use a dataset other than \acrshort{gls} curves alone is \textit{gls-rls/2CH/scaled/ward/2} and it achieves a \acrshort{dor} of $26.76$. One can also note that the highest performing model in terms of \acrshort{dor} that uses a view other than only \acrshort{2ch} is \textit{rls/all-views/normalized/weighted/2} which achieves a \acrshort{dor} of 25.56. The \acrshort{tsc} models that achieve the highest \acrshort{dor} scores all use no preprocessing, or scaling. From table \ref{tab:tsc_ind_dor_sens_spec_dist} one can see that the \acrshort{tsc} models that acheive the highest \acrshort{dor} scores are \textit{gls/2CH/regular/centroid/2}, and \textit{gls/2CH/scaled/centroid/2} which are the same two models that achieve the highest \acrshort{dor} in the heart failure case study.

\begin{table*}
    \centering
    \ra{1.3}
    \begin{tabular}{lrrrr}
        \toprule
        Dataset-model             &  Accuracy &  Sensitivity &  Specificity &   \acrshort{dor} \\
        \midrule
        gls/2CH/regular/centroid/2 &      0.74 &         0.71 &         0.93 & 33.47 \\
        gls/2CH/scaled/centroid/2  &      0.74 &         0.71 &         0.93 & 33.47 \\
        gls/2CH/scaled/average/2   &      0.73 &         0.69 &         0.93 & 30.71 \\
        gls/2CH/regular/average/2  &      0.73 &         0.69 &         0.93 & 30.71 \\
        gls/2CH/scaled/ward/2      &      0.71 &         0.67 &         0.93 & 27.49 \\
        \bottomrule
    \end{tabular}
    \caption{The accuracy, \acrshort{dor}, sensitivity and specicity scores of the five best performing two-cluster-center \acrshort{tsc} models in terms of \acrshort{dor}, at detecting patient diagnoses.
             The \textbf{Dataset-model} column indicates 
             \textit{Dataset used}$/$\textit{View used}$/$\textit{Type of preprocessing used}$/$\textit{Linkage criteria of model}$/$\textit{Number of cluster centers}.}
    \label{tab:tsc_ind_dor_sens_spec_dist}
\end{table*}

\begin{table*}
    \centering
    \ra{1.3}
    \begin{tabular}{lr}
        \toprule
        Dataset-model                   &  \acrshort{ari} \\
        \midrule
        gls-rls/4CH/regular/complete/2   & 0.36 \\
        gls/all-views/regular/weighted/2 & 0.34 \\
        gls/all-views/scaled/weighted/4  & 0.33 \\
        gls/all-views/scaled/weighted/3  & 0.33 \\
        gls/APLAX/regular/single/10      & 0.32 \\
        \bottomrule
    \end{tabular}
    \caption{The five highest \acrshort{ari} scores attained when applying \acrshort{tsc} for detecting patient diagnoses.
             The \textbf{Dataset-model} column indicates \textit{Dataset used}$/$\textit{View used}$/$\textit{Linkage criteria of model}$/$\textit{Number of cluster centers}.}
    \label{tab:tsc_ind_ari}
\end{table*}

The majority of the \acrshort{ari} scorer for all the \acrshort{tsc} models evaluated at two to nine cluster centers are centered around zero. As with the \acrshort{tsc} models attaining the highest \acrshort{dor} the models using no preprocessing or scaling acheive the highest \acrshort{ari} indices when used to identify patient diagnoses. In addition, the \acrshort{gls} curves are also most often part of the dataset for the \acrshort{tsc} models receiving the highest \acrshort{ari} when used to identify patient diagnoses. From table \ref{tab:tsc_ind_ari} one can see that the \acrshort{tsc} models receiving the five highest \acrshort{ari} scores, are not among the \acrshort{tsc} models that receive the highest \acrshort{dor} scores. The \acrshort{tsc} model \textit{gls-rls/4CH/regular/complete/2} attains the highest \acrshort{ari} score when applied to identify patient diagnoses, and achieves an accuracy of $0.84$, a sensitivity of $0.87$ a specificity of $0.69$ and a \acrshort{dor} $14.65$. The \acrshort{tsc} model \textit{gls/all-views/regular/weighted/2} achieves the second highest \acrshort{ari} when applied to identify patient diagnoses, and achieves an accuracy of $0.82$, a sensitivity of $0.81$ a specificity of $0.83$ and a \acrshort{dor} $21.06$. What should also be noted is that the \acrshort{tsc} models achieving the two highest \acrshort{ari} when applied to identify patient diagnoses are models evaluated at two cluster centers, which means that none of the \acrshort{tsc} models evaluated at cluster centers between three and nine can perform better than the ones evaluated at two cluster centers. It may seem strange that the ordered lists of \acrshort{dor}, and \acrshort{ari} are so different. The reason for this is not because \acrshort{dor} inherently values sensitivity higher than specificity, but stems from how the \acrshort{dor} is defined. Recall that $\mathrm{DOR = ( TP \times TN )/ (FP \times FN)}$, since the patient diagnoses dataset is skewed in favour of positives \acrshort{tp} has the potential of being as high as 170 while \acrshort{tn} can be as high as 30. Therefore the \acrshort{dor} will be higher for models with a high sensitivity than for models with an equally high sensitivity. In figure \ref{fig:five_members_gls_rls_4CH_regular_complete_two} curves of five random cluster members assigned by the \textit{gls/all-views/regular/weighted/2} model are plotted. As with the observations made with regard to figure \ref{fig:tsc_hf_best_meth_5_samples} it is not possible to make any conclusive statements as to what the similarities are based on such a small sample size. However, based on the small sample size in \ref{fig:five_members_gls_rls_4CH_regular_complete_two} it seems as though the curves in cluster 2 (column (b)) are smoother in shape, than the curves in cluster 1 (column (a)). The \acrshort{tsc} model that is chosen as the best model for identifying patient diagnoses is \textit{gls/all-views/regular/weighted/2}, because it achieves the second highest \acrshort{ari}, and because it's sensitivity and specificity are more balanced than the model attaining the highest \acrshort{ari} and the models that achieve higher \acrshort{dor}.

\clearpage

\begin{figure}[ht]
    \centering
    %% Creator: Matplotlib, PGF backend
%%
%% To include the figure in your LaTeX document, write
%%   \input{<filename>.pgf}
%%
%% Make sure the required packages are loaded in your preamble
%%   \usepackage{pgf}
%%
%% Figures using additional raster images can only be included by \input if
%% they are in the same directory as the main LaTeX file. For loading figures
%% from other directories you can use the `import` package
%%   \usepackage{import}
%% and then include the figures with
%%   \import{<path to file>}{<filename>.pgf}
%%
%% Matplotlib used the following preamble
%%
\begingroup%
\makeatletter%
\begin{pgfpicture}%
\pgfpathrectangle{\pgfpointorigin}{\pgfqpoint{6.340000in}{8.840000in}}%
\pgfusepath{use as bounding box, clip}%
\begin{pgfscope}%
\pgfsetbuttcap%
\pgfsetmiterjoin%
\definecolor{currentfill}{rgb}{1.000000,1.000000,1.000000}%
\pgfsetfillcolor{currentfill}%
\pgfsetlinewidth{0.000000pt}%
\definecolor{currentstroke}{rgb}{1.000000,1.000000,1.000000}%
\pgfsetstrokecolor{currentstroke}%
\pgfsetdash{}{0pt}%
\pgfpathmoveto{\pgfqpoint{0.000000in}{-0.000000in}}%
\pgfpathlineto{\pgfqpoint{6.340000in}{-0.000000in}}%
\pgfpathlineto{\pgfqpoint{6.340000in}{8.840000in}}%
\pgfpathlineto{\pgfqpoint{0.000000in}{8.840000in}}%
\pgfpathclose%
\pgfusepath{fill}%
\end{pgfscope}%
\begin{pgfscope}%
\pgfsetbuttcap%
\pgfsetmiterjoin%
\definecolor{currentfill}{rgb}{0.917647,0.917647,0.949020}%
\pgfsetfillcolor{currentfill}%
\pgfsetlinewidth{0.000000pt}%
\definecolor{currentstroke}{rgb}{0.000000,0.000000,0.000000}%
\pgfsetstrokecolor{currentstroke}%
\pgfsetstrokeopacity{0.000000}%
\pgfsetdash{}{0pt}%
\pgfpathmoveto{\pgfqpoint{0.828241in}{6.184745in}}%
\pgfpathlineto{\pgfqpoint{3.242963in}{6.184745in}}%
\pgfpathlineto{\pgfqpoint{3.242963in}{8.542604in}}%
\pgfpathlineto{\pgfqpoint{0.828241in}{8.542604in}}%
\pgfpathclose%
\pgfusepath{fill}%
\end{pgfscope}%
\begin{pgfscope}%
\pgfpathrectangle{\pgfqpoint{0.828241in}{6.184745in}}{\pgfqpoint{2.414722in}{2.357859in}}%
\pgfusepath{clip}%
\pgfsetroundcap%
\pgfsetroundjoin%
\pgfsetlinewidth{1.003750pt}%
\definecolor{currentstroke}{rgb}{1.000000,1.000000,1.000000}%
\pgfsetstrokecolor{currentstroke}%
\pgfsetdash{}{0pt}%
\pgfpathmoveto{\pgfqpoint{0.938001in}{6.184745in}}%
\pgfpathlineto{\pgfqpoint{0.938001in}{8.542604in}}%
\pgfusepath{stroke}%
\end{pgfscope}%
\begin{pgfscope}%
\definecolor{textcolor}{rgb}{0.150000,0.150000,0.150000}%
\pgfsetstrokecolor{textcolor}%
\pgfsetfillcolor{textcolor}%
\pgftext[x=0.938001in,y=6.052801in,,top]{\color{textcolor}\sffamily\fontsize{11.000000}{13.200000}\selectfont \(\displaystyle 0.0\)}%
\end{pgfscope}%
\begin{pgfscope}%
\pgfpathrectangle{\pgfqpoint{0.828241in}{6.184745in}}{\pgfqpoint{2.414722in}{2.357859in}}%
\pgfusepath{clip}%
\pgfsetroundcap%
\pgfsetroundjoin%
\pgfsetlinewidth{1.003750pt}%
\definecolor{currentstroke}{rgb}{1.000000,1.000000,1.000000}%
\pgfsetstrokecolor{currentstroke}%
\pgfsetdash{}{0pt}%
\pgfpathmoveto{\pgfqpoint{1.787335in}{6.184745in}}%
\pgfpathlineto{\pgfqpoint{1.787335in}{8.542604in}}%
\pgfusepath{stroke}%
\end{pgfscope}%
\begin{pgfscope}%
\definecolor{textcolor}{rgb}{0.150000,0.150000,0.150000}%
\pgfsetstrokecolor{textcolor}%
\pgfsetfillcolor{textcolor}%
\pgftext[x=1.787335in,y=6.052801in,,top]{\color{textcolor}\sffamily\fontsize{11.000000}{13.200000}\selectfont \(\displaystyle 0.5\)}%
\end{pgfscope}%
\begin{pgfscope}%
\pgfpathrectangle{\pgfqpoint{0.828241in}{6.184745in}}{\pgfqpoint{2.414722in}{2.357859in}}%
\pgfusepath{clip}%
\pgfsetroundcap%
\pgfsetroundjoin%
\pgfsetlinewidth{1.003750pt}%
\definecolor{currentstroke}{rgb}{1.000000,1.000000,1.000000}%
\pgfsetstrokecolor{currentstroke}%
\pgfsetdash{}{0pt}%
\pgfpathmoveto{\pgfqpoint{2.636669in}{6.184745in}}%
\pgfpathlineto{\pgfqpoint{2.636669in}{8.542604in}}%
\pgfusepath{stroke}%
\end{pgfscope}%
\begin{pgfscope}%
\definecolor{textcolor}{rgb}{0.150000,0.150000,0.150000}%
\pgfsetstrokecolor{textcolor}%
\pgfsetfillcolor{textcolor}%
\pgftext[x=2.636669in,y=6.052801in,,top]{\color{textcolor}\sffamily\fontsize{11.000000}{13.200000}\selectfont \(\displaystyle 1.0\)}%
\end{pgfscope}%
\begin{pgfscope}%
\pgfpathrectangle{\pgfqpoint{0.828241in}{6.184745in}}{\pgfqpoint{2.414722in}{2.357859in}}%
\pgfusepath{clip}%
\pgfsetroundcap%
\pgfsetroundjoin%
\pgfsetlinewidth{1.003750pt}%
\definecolor{currentstroke}{rgb}{1.000000,1.000000,1.000000}%
\pgfsetstrokecolor{currentstroke}%
\pgfsetdash{}{0pt}%
\pgfpathmoveto{\pgfqpoint{0.828241in}{6.271864in}}%
\pgfpathlineto{\pgfqpoint{3.242963in}{6.271864in}}%
\pgfusepath{stroke}%
\end{pgfscope}%
\begin{pgfscope}%
\definecolor{textcolor}{rgb}{0.150000,0.150000,0.150000}%
\pgfsetstrokecolor{textcolor}%
\pgfsetfillcolor{textcolor}%
\pgftext[x=0.425926in,y=6.219057in,left,base]{\color{textcolor}\sffamily\fontsize{11.000000}{13.200000}\selectfont \(\displaystyle -12\)}%
\end{pgfscope}%
\begin{pgfscope}%
\pgfpathrectangle{\pgfqpoint{0.828241in}{6.184745in}}{\pgfqpoint{2.414722in}{2.357859in}}%
\pgfusepath{clip}%
\pgfsetroundcap%
\pgfsetroundjoin%
\pgfsetlinewidth{1.003750pt}%
\definecolor{currentstroke}{rgb}{1.000000,1.000000,1.000000}%
\pgfsetstrokecolor{currentstroke}%
\pgfsetdash{}{0pt}%
\pgfpathmoveto{\pgfqpoint{0.828241in}{6.620737in}}%
\pgfpathlineto{\pgfqpoint{3.242963in}{6.620737in}}%
\pgfusepath{stroke}%
\end{pgfscope}%
\begin{pgfscope}%
\definecolor{textcolor}{rgb}{0.150000,0.150000,0.150000}%
\pgfsetstrokecolor{textcolor}%
\pgfsetfillcolor{textcolor}%
\pgftext[x=0.425926in,y=6.567931in,left,base]{\color{textcolor}\sffamily\fontsize{11.000000}{13.200000}\selectfont \(\displaystyle -10\)}%
\end{pgfscope}%
\begin{pgfscope}%
\pgfpathrectangle{\pgfqpoint{0.828241in}{6.184745in}}{\pgfqpoint{2.414722in}{2.357859in}}%
\pgfusepath{clip}%
\pgfsetroundcap%
\pgfsetroundjoin%
\pgfsetlinewidth{1.003750pt}%
\definecolor{currentstroke}{rgb}{1.000000,1.000000,1.000000}%
\pgfsetstrokecolor{currentstroke}%
\pgfsetdash{}{0pt}%
\pgfpathmoveto{\pgfqpoint{0.828241in}{6.969611in}}%
\pgfpathlineto{\pgfqpoint{3.242963in}{6.969611in}}%
\pgfusepath{stroke}%
\end{pgfscope}%
\begin{pgfscope}%
\definecolor{textcolor}{rgb}{0.150000,0.150000,0.150000}%
\pgfsetstrokecolor{textcolor}%
\pgfsetfillcolor{textcolor}%
\pgftext[x=0.501968in,y=6.916804in,left,base]{\color{textcolor}\sffamily\fontsize{11.000000}{13.200000}\selectfont \(\displaystyle -8\)}%
\end{pgfscope}%
\begin{pgfscope}%
\pgfpathrectangle{\pgfqpoint{0.828241in}{6.184745in}}{\pgfqpoint{2.414722in}{2.357859in}}%
\pgfusepath{clip}%
\pgfsetroundcap%
\pgfsetroundjoin%
\pgfsetlinewidth{1.003750pt}%
\definecolor{currentstroke}{rgb}{1.000000,1.000000,1.000000}%
\pgfsetstrokecolor{currentstroke}%
\pgfsetdash{}{0pt}%
\pgfpathmoveto{\pgfqpoint{0.828241in}{7.318484in}}%
\pgfpathlineto{\pgfqpoint{3.242963in}{7.318484in}}%
\pgfusepath{stroke}%
\end{pgfscope}%
\begin{pgfscope}%
\definecolor{textcolor}{rgb}{0.150000,0.150000,0.150000}%
\pgfsetstrokecolor{textcolor}%
\pgfsetfillcolor{textcolor}%
\pgftext[x=0.501968in,y=7.265677in,left,base]{\color{textcolor}\sffamily\fontsize{11.000000}{13.200000}\selectfont \(\displaystyle -6\)}%
\end{pgfscope}%
\begin{pgfscope}%
\pgfpathrectangle{\pgfqpoint{0.828241in}{6.184745in}}{\pgfqpoint{2.414722in}{2.357859in}}%
\pgfusepath{clip}%
\pgfsetroundcap%
\pgfsetroundjoin%
\pgfsetlinewidth{1.003750pt}%
\definecolor{currentstroke}{rgb}{1.000000,1.000000,1.000000}%
\pgfsetstrokecolor{currentstroke}%
\pgfsetdash{}{0pt}%
\pgfpathmoveto{\pgfqpoint{0.828241in}{7.667357in}}%
\pgfpathlineto{\pgfqpoint{3.242963in}{7.667357in}}%
\pgfusepath{stroke}%
\end{pgfscope}%
\begin{pgfscope}%
\definecolor{textcolor}{rgb}{0.150000,0.150000,0.150000}%
\pgfsetstrokecolor{textcolor}%
\pgfsetfillcolor{textcolor}%
\pgftext[x=0.501968in,y=7.614551in,left,base]{\color{textcolor}\sffamily\fontsize{11.000000}{13.200000}\selectfont \(\displaystyle -4\)}%
\end{pgfscope}%
\begin{pgfscope}%
\pgfpathrectangle{\pgfqpoint{0.828241in}{6.184745in}}{\pgfqpoint{2.414722in}{2.357859in}}%
\pgfusepath{clip}%
\pgfsetroundcap%
\pgfsetroundjoin%
\pgfsetlinewidth{1.003750pt}%
\definecolor{currentstroke}{rgb}{1.000000,1.000000,1.000000}%
\pgfsetstrokecolor{currentstroke}%
\pgfsetdash{}{0pt}%
\pgfpathmoveto{\pgfqpoint{0.828241in}{8.016231in}}%
\pgfpathlineto{\pgfqpoint{3.242963in}{8.016231in}}%
\pgfusepath{stroke}%
\end{pgfscope}%
\begin{pgfscope}%
\definecolor{textcolor}{rgb}{0.150000,0.150000,0.150000}%
\pgfsetstrokecolor{textcolor}%
\pgfsetfillcolor{textcolor}%
\pgftext[x=0.501968in,y=7.963424in,left,base]{\color{textcolor}\sffamily\fontsize{11.000000}{13.200000}\selectfont \(\displaystyle -2\)}%
\end{pgfscope}%
\begin{pgfscope}%
\pgfpathrectangle{\pgfqpoint{0.828241in}{6.184745in}}{\pgfqpoint{2.414722in}{2.357859in}}%
\pgfusepath{clip}%
\pgfsetroundcap%
\pgfsetroundjoin%
\pgfsetlinewidth{1.003750pt}%
\definecolor{currentstroke}{rgb}{1.000000,1.000000,1.000000}%
\pgfsetstrokecolor{currentstroke}%
\pgfsetdash{}{0pt}%
\pgfpathmoveto{\pgfqpoint{0.828241in}{8.365104in}}%
\pgfpathlineto{\pgfqpoint{3.242963in}{8.365104in}}%
\pgfusepath{stroke}%
\end{pgfscope}%
\begin{pgfscope}%
\definecolor{textcolor}{rgb}{0.150000,0.150000,0.150000}%
\pgfsetstrokecolor{textcolor}%
\pgfsetfillcolor{textcolor}%
\pgftext[x=0.620255in,y=8.312297in,left,base]{\color{textcolor}\sffamily\fontsize{11.000000}{13.200000}\selectfont \(\displaystyle 0\)}%
\end{pgfscope}%
\begin{pgfscope}%
\definecolor{textcolor}{rgb}{0.150000,0.150000,0.150000}%
\pgfsetstrokecolor{textcolor}%
\pgfsetfillcolor{textcolor}%
\pgftext[x=0.370370in,y=7.363675in,,bottom,rotate=90.000000]{\color{textcolor}\sffamily\fontsize{11.000000}{13.200000}\selectfont 4CH/gls}%
\end{pgfscope}%
\begin{pgfscope}%
\pgfpathrectangle{\pgfqpoint{0.828241in}{6.184745in}}{\pgfqpoint{2.414722in}{2.357859in}}%
\pgfusepath{clip}%
\pgfsetroundcap%
\pgfsetroundjoin%
\pgfsetlinewidth{1.505625pt}%
\definecolor{currentstroke}{rgb}{0.298039,0.447059,0.690196}%
\pgfsetstrokecolor{currentstroke}%
\pgfsetdash{}{0pt}%
\pgfpathmoveto{\pgfqpoint{0.938001in}{7.401988in}}%
\pgfpathlineto{\pgfqpoint{0.965848in}{7.405310in}}%
\pgfpathlineto{\pgfqpoint{0.993695in}{7.427198in}}%
\pgfpathlineto{\pgfqpoint{1.021542in}{7.489106in}}%
\pgfpathlineto{\pgfqpoint{1.049389in}{7.607691in}}%
\pgfpathlineto{\pgfqpoint{1.077236in}{7.781760in}}%
\pgfpathlineto{\pgfqpoint{1.105083in}{7.982943in}}%
\pgfpathlineto{\pgfqpoint{1.132930in}{8.165701in}}%
\pgfpathlineto{\pgfqpoint{1.160777in}{8.293740in}}%
\pgfpathlineto{\pgfqpoint{1.188624in}{8.355952in}}%
\pgfpathlineto{\pgfqpoint{1.216471in}{8.365104in}}%
\pgfpathlineto{\pgfqpoint{1.244318in}{8.351383in}}%
\pgfpathlineto{\pgfqpoint{1.272166in}{8.343586in}}%
\pgfpathlineto{\pgfqpoint{1.300013in}{8.349267in}}%
\pgfpathlineto{\pgfqpoint{1.327860in}{8.351905in}}%
\pgfpathlineto{\pgfqpoint{1.355707in}{8.325741in}}%
\pgfpathlineto{\pgfqpoint{1.383554in}{8.251649in}}%
\pgfpathlineto{\pgfqpoint{1.411401in}{8.125814in}}%
\pgfpathlineto{\pgfqpoint{1.439248in}{7.960147in}}%
\pgfpathlineto{\pgfqpoint{1.467095in}{7.774275in}}%
\pgfpathlineto{\pgfqpoint{1.494942in}{7.583480in}}%
\pgfpathlineto{\pgfqpoint{1.522789in}{7.394911in}}%
\pgfpathlineto{\pgfqpoint{1.550636in}{7.215071in}}%
\pgfpathlineto{\pgfqpoint{1.578483in}{7.053001in}}%
\pgfpathlineto{\pgfqpoint{1.606330in}{6.913124in}}%
\pgfpathlineto{\pgfqpoint{1.634177in}{6.791128in}}%
\pgfpathlineto{\pgfqpoint{1.662024in}{6.679900in}}%
\pgfpathlineto{\pgfqpoint{1.689871in}{6.577519in}}%
\pgfpathlineto{\pgfqpoint{1.717718in}{6.488328in}}%
\pgfpathlineto{\pgfqpoint{1.745565in}{6.420646in}}%
\pgfpathlineto{\pgfqpoint{1.773412in}{6.379673in}}%
\pgfpathlineto{\pgfqpoint{1.801259in}{6.361350in}}%
\pgfpathlineto{\pgfqpoint{1.829106in}{6.352944in}}%
\pgfpathlineto{\pgfqpoint{1.856953in}{6.341419in}}%
\pgfpathlineto{\pgfqpoint{1.884800in}{6.322285in}}%
\pgfpathlineto{\pgfqpoint{1.912647in}{6.301919in}}%
\pgfpathlineto{\pgfqpoint{1.940494in}{6.291921in}}%
\pgfpathlineto{\pgfqpoint{1.968341in}{6.298354in}}%
\pgfpathlineto{\pgfqpoint{1.996188in}{6.325437in}}%
\pgfpathlineto{\pgfqpoint{2.024035in}{6.379544in}}%
\pgfpathlineto{\pgfqpoint{2.051882in}{6.464187in}}%
\pgfpathlineto{\pgfqpoint{2.079729in}{6.578834in}}%
\pgfpathlineto{\pgfqpoint{2.107576in}{6.715174in}}%
\pgfpathlineto{\pgfqpoint{2.135423in}{6.856110in}}%
\pgfpathlineto{\pgfqpoint{2.163270in}{6.983739in}}%
\pgfpathlineto{\pgfqpoint{2.191117in}{7.088658in}}%
\pgfpathlineto{\pgfqpoint{2.218964in}{7.170488in}}%
\pgfpathlineto{\pgfqpoint{2.246811in}{7.232193in}}%
\pgfpathlineto{\pgfqpoint{2.274658in}{7.275026in}}%
\pgfpathlineto{\pgfqpoint{2.302505in}{7.298524in}}%
\pgfpathlineto{\pgfqpoint{2.330352in}{7.305196in}}%
\pgfpathlineto{\pgfqpoint{2.358199in}{7.306276in}}%
\pgfpathlineto{\pgfqpoint{2.386046in}{7.324094in}}%
\pgfpathlineto{\pgfqpoint{2.413893in}{7.389664in}}%
\pgfpathlineto{\pgfqpoint{2.441740in}{7.531601in}}%
\pgfpathlineto{\pgfqpoint{2.469587in}{7.752841in}}%
\pgfpathlineto{\pgfqpoint{2.497434in}{8.011594in}}%
\pgfpathlineto{\pgfqpoint{2.525281in}{8.238520in}}%
\pgfpathlineto{\pgfqpoint{2.553128in}{8.381983in}}%
\pgfpathlineto{\pgfqpoint{2.580975in}{8.431680in}}%
\pgfpathlineto{\pgfqpoint{2.608822in}{8.411936in}}%
\pgfpathlineto{\pgfqpoint{2.636669in}{8.365104in}}%
\pgfpathlineto{\pgfqpoint{2.664516in}{8.331193in}}%
\pgfpathlineto{\pgfqpoint{2.692363in}{8.327780in}}%
\pgfpathlineto{\pgfqpoint{2.720210in}{8.341182in}}%
\pgfpathlineto{\pgfqpoint{2.748057in}{8.332819in}}%
\pgfpathlineto{\pgfqpoint{2.775904in}{8.265223in}}%
\pgfpathlineto{\pgfqpoint{2.803751in}{8.130158in}}%
\pgfpathlineto{\pgfqpoint{2.831598in}{7.948294in}}%
\pgfpathlineto{\pgfqpoint{2.859445in}{7.748114in}}%
\pgfpathlineto{\pgfqpoint{2.887292in}{7.551765in}}%
\pgfpathlineto{\pgfqpoint{2.915139in}{7.365921in}}%
\pgfusepath{stroke}%
\end{pgfscope}%
\begin{pgfscope}%
\pgfpathrectangle{\pgfqpoint{0.828241in}{6.184745in}}{\pgfqpoint{2.414722in}{2.357859in}}%
\pgfusepath{clip}%
\pgfsetroundcap%
\pgfsetroundjoin%
\pgfsetlinewidth{1.505625pt}%
\definecolor{currentstroke}{rgb}{0.866667,0.517647,0.321569}%
\pgfsetstrokecolor{currentstroke}%
\pgfsetdash{}{0pt}%
\pgfpathmoveto{\pgfqpoint{0.938001in}{8.092148in}}%
\pgfpathlineto{\pgfqpoint{0.965848in}{8.099020in}}%
\pgfpathlineto{\pgfqpoint{0.993695in}{8.115012in}}%
\pgfpathlineto{\pgfqpoint{1.021542in}{8.146761in}}%
\pgfpathlineto{\pgfqpoint{1.049389in}{8.194984in}}%
\pgfpathlineto{\pgfqpoint{1.077236in}{8.251747in}}%
\pgfpathlineto{\pgfqpoint{1.105083in}{8.305108in}}%
\pgfpathlineto{\pgfqpoint{1.132930in}{8.346833in}}%
\pgfpathlineto{\pgfqpoint{1.160777in}{8.374256in}}%
\pgfpathlineto{\pgfqpoint{1.188624in}{8.386212in}}%
\pgfpathlineto{\pgfqpoint{1.216471in}{8.382345in}}%
\pgfpathlineto{\pgfqpoint{1.244318in}{8.365104in}}%
\pgfpathlineto{\pgfqpoint{1.272166in}{8.340411in}}%
\pgfpathlineto{\pgfqpoint{1.300013in}{8.316978in}}%
\pgfpathlineto{\pgfqpoint{1.327860in}{8.302966in}}%
\pgfpathlineto{\pgfqpoint{1.355707in}{8.301494in}}%
\pgfpathlineto{\pgfqpoint{1.383554in}{8.307907in}}%
\pgfpathlineto{\pgfqpoint{1.411401in}{8.312072in}}%
\pgfpathlineto{\pgfqpoint{1.439248in}{8.305407in}}%
\pgfpathlineto{\pgfqpoint{1.467095in}{8.283771in}}%
\pgfpathlineto{\pgfqpoint{1.494942in}{8.246704in}}%
\pgfpathlineto{\pgfqpoint{1.522789in}{8.195753in}}%
\pgfpathlineto{\pgfqpoint{1.550636in}{8.133219in}}%
\pgfpathlineto{\pgfqpoint{1.578483in}{8.063442in}}%
\pgfpathlineto{\pgfqpoint{1.606330in}{7.990366in}}%
\pgfpathlineto{\pgfqpoint{1.634177in}{7.913154in}}%
\pgfpathlineto{\pgfqpoint{1.662024in}{7.828970in}}%
\pgfpathlineto{\pgfqpoint{1.689871in}{7.742079in}}%
\pgfpathlineto{\pgfqpoint{1.717718in}{7.661210in}}%
\pgfpathlineto{\pgfqpoint{1.745565in}{7.591759in}}%
\pgfpathlineto{\pgfqpoint{1.773412in}{7.536960in}}%
\pgfpathlineto{\pgfqpoint{1.801259in}{7.500650in}}%
\pgfpathlineto{\pgfqpoint{1.829106in}{7.484523in}}%
\pgfpathlineto{\pgfqpoint{1.856953in}{7.483551in}}%
\pgfpathlineto{\pgfqpoint{1.884800in}{7.486658in}}%
\pgfpathlineto{\pgfqpoint{1.912647in}{7.482245in}}%
\pgfpathlineto{\pgfqpoint{1.940494in}{7.465906in}}%
\pgfpathlineto{\pgfqpoint{1.968341in}{7.447207in}}%
\pgfpathlineto{\pgfqpoint{1.996188in}{7.449483in}}%
\pgfpathlineto{\pgfqpoint{2.024035in}{7.494394in}}%
\pgfpathlineto{\pgfqpoint{2.051882in}{7.582419in}}%
\pgfpathlineto{\pgfqpoint{2.079729in}{7.694915in}}%
\pgfpathlineto{\pgfqpoint{2.107576in}{7.809695in}}%
\pgfpathlineto{\pgfqpoint{2.135423in}{7.908010in}}%
\pgfpathlineto{\pgfqpoint{2.163270in}{7.979173in}}%
\pgfpathlineto{\pgfqpoint{2.191117in}{8.023955in}}%
\pgfpathlineto{\pgfqpoint{2.218964in}{8.049685in}}%
\pgfpathlineto{\pgfqpoint{2.246811in}{8.062850in}}%
\pgfpathlineto{\pgfqpoint{2.274658in}{8.068335in}}%
\pgfpathlineto{\pgfqpoint{2.302505in}{8.071923in}}%
\pgfpathlineto{\pgfqpoint{2.330352in}{8.078664in}}%
\pgfpathlineto{\pgfqpoint{2.358199in}{8.089163in}}%
\pgfpathlineto{\pgfqpoint{2.386046in}{8.101721in}}%
\pgfpathlineto{\pgfqpoint{2.413893in}{8.116258in}}%
\pgfpathlineto{\pgfqpoint{2.441740in}{8.132768in}}%
\pgfpathlineto{\pgfqpoint{2.469587in}{8.148344in}}%
\pgfpathlineto{\pgfqpoint{2.497434in}{8.159355in}}%
\pgfpathlineto{\pgfqpoint{2.525281in}{8.167045in}}%
\pgfpathlineto{\pgfqpoint{2.553128in}{8.179082in}}%
\pgfpathlineto{\pgfqpoint{2.580975in}{8.205474in}}%
\pgfpathlineto{\pgfqpoint{2.608822in}{8.250734in}}%
\pgfpathlineto{\pgfqpoint{2.636669in}{8.308106in}}%
\pgfpathlineto{\pgfqpoint{2.664516in}{8.361185in}}%
\pgfpathlineto{\pgfqpoint{2.692363in}{8.394904in}}%
\pgfpathlineto{\pgfqpoint{2.720210in}{8.406299in}}%
\pgfpathlineto{\pgfqpoint{2.748057in}{8.402181in}}%
\pgfpathlineto{\pgfqpoint{2.775904in}{8.388216in}}%
\pgfpathlineto{\pgfqpoint{2.803751in}{8.365104in}}%
\pgfpathlineto{\pgfqpoint{2.831598in}{8.333851in}}%
\pgfpathlineto{\pgfqpoint{2.859445in}{8.300077in}}%
\pgfpathlineto{\pgfqpoint{2.887292in}{8.270396in}}%
\pgfpathlineto{\pgfqpoint{2.915139in}{8.246705in}}%
\pgfpathlineto{\pgfqpoint{2.942986in}{8.225612in}}%
\pgfpathlineto{\pgfqpoint{2.970833in}{8.200809in}}%
\pgfpathlineto{\pgfqpoint{2.998680in}{8.166436in}}%
\pgfpathlineto{\pgfqpoint{3.026527in}{8.121134in}}%
\pgfpathlineto{\pgfqpoint{3.054374in}{8.066494in}}%
\pgfpathlineto{\pgfqpoint{3.082221in}{8.002725in}}%
\pgfpathlineto{\pgfqpoint{3.110068in}{7.931126in}}%
\pgfusepath{stroke}%
\end{pgfscope}%
\begin{pgfscope}%
\pgfpathrectangle{\pgfqpoint{0.828241in}{6.184745in}}{\pgfqpoint{2.414722in}{2.357859in}}%
\pgfusepath{clip}%
\pgfsetroundcap%
\pgfsetroundjoin%
\pgfsetlinewidth{1.505625pt}%
\definecolor{currentstroke}{rgb}{0.333333,0.658824,0.407843}%
\pgfsetstrokecolor{currentstroke}%
\pgfsetdash{}{0pt}%
\pgfpathmoveto{\pgfqpoint{0.938001in}{7.456894in}}%
\pgfpathlineto{\pgfqpoint{0.964543in}{7.460789in}}%
\pgfpathlineto{\pgfqpoint{0.991085in}{7.474047in}}%
\pgfpathlineto{\pgfqpoint{1.017626in}{7.516596in}}%
\pgfpathlineto{\pgfqpoint{1.044168in}{7.611053in}}%
\pgfpathlineto{\pgfqpoint{1.070710in}{7.767908in}}%
\pgfpathlineto{\pgfqpoint{1.097251in}{7.966149in}}%
\pgfpathlineto{\pgfqpoint{1.123793in}{8.159193in}}%
\pgfpathlineto{\pgfqpoint{1.150335in}{8.299615in}}%
\pgfpathlineto{\pgfqpoint{1.176877in}{8.365104in}}%
\pgfpathlineto{\pgfqpoint{1.203418in}{8.369691in}}%
\pgfpathlineto{\pgfqpoint{1.229960in}{8.342707in}}%
\pgfpathlineto{\pgfqpoint{1.256502in}{8.297593in}}%
\pgfpathlineto{\pgfqpoint{1.283043in}{8.226331in}}%
\pgfpathlineto{\pgfqpoint{1.309585in}{8.118036in}}%
\pgfpathlineto{\pgfqpoint{1.336127in}{7.970268in}}%
\pgfpathlineto{\pgfqpoint{1.362668in}{7.788188in}}%
\pgfpathlineto{\pgfqpoint{1.389210in}{7.587810in}}%
\pgfpathlineto{\pgfqpoint{1.415752in}{7.395655in}}%
\pgfpathlineto{\pgfqpoint{1.442293in}{7.234381in}}%
\pgfpathlineto{\pgfqpoint{1.468835in}{7.109710in}}%
\pgfpathlineto{\pgfqpoint{1.495377in}{7.017117in}}%
\pgfpathlineto{\pgfqpoint{1.521918in}{6.941865in}}%
\pgfpathlineto{\pgfqpoint{1.548460in}{6.862155in}}%
\pgfpathlineto{\pgfqpoint{1.575002in}{6.767257in}}%
\pgfpathlineto{\pgfqpoint{1.601543in}{6.672579in}}%
\pgfpathlineto{\pgfqpoint{1.628085in}{6.600394in}}%
\pgfpathlineto{\pgfqpoint{1.654627in}{6.563246in}}%
\pgfpathlineto{\pgfqpoint{1.681169in}{6.556904in}}%
\pgfpathlineto{\pgfqpoint{1.707710in}{6.564279in}}%
\pgfpathlineto{\pgfqpoint{1.734252in}{6.566465in}}%
\pgfpathlineto{\pgfqpoint{1.760794in}{6.553319in}}%
\pgfpathlineto{\pgfqpoint{1.787335in}{6.526179in}}%
\pgfpathlineto{\pgfqpoint{1.813877in}{6.496568in}}%
\pgfpathlineto{\pgfqpoint{1.840419in}{6.482051in}}%
\pgfpathlineto{\pgfqpoint{1.866960in}{6.495671in}}%
\pgfpathlineto{\pgfqpoint{1.893502in}{6.537004in}}%
\pgfpathlineto{\pgfqpoint{1.920044in}{6.600904in}}%
\pgfpathlineto{\pgfqpoint{1.946585in}{6.686219in}}%
\pgfpathlineto{\pgfqpoint{1.973127in}{6.785927in}}%
\pgfpathlineto{\pgfqpoint{1.999669in}{6.887842in}}%
\pgfpathlineto{\pgfqpoint{2.026210in}{6.982515in}}%
\pgfpathlineto{\pgfqpoint{2.052752in}{7.064889in}}%
\pgfpathlineto{\pgfqpoint{2.079294in}{7.134615in}}%
\pgfpathlineto{\pgfqpoint{2.105836in}{7.195011in}}%
\pgfpathlineto{\pgfqpoint{2.132377in}{7.245854in}}%
\pgfpathlineto{\pgfqpoint{2.158919in}{7.284149in}}%
\pgfpathlineto{\pgfqpoint{2.185461in}{7.310663in}}%
\pgfpathlineto{\pgfqpoint{2.212002in}{7.327712in}}%
\pgfpathlineto{\pgfqpoint{2.238544in}{7.336517in}}%
\pgfpathlineto{\pgfqpoint{2.265086in}{7.340850in}}%
\pgfpathlineto{\pgfqpoint{2.291627in}{7.352820in}}%
\pgfpathlineto{\pgfqpoint{2.318169in}{7.399096in}}%
\pgfpathlineto{\pgfqpoint{2.344711in}{7.514283in}}%
\pgfpathlineto{\pgfqpoint{2.371252in}{7.710643in}}%
\pgfpathlineto{\pgfqpoint{2.397794in}{7.954919in}}%
\pgfpathlineto{\pgfqpoint{2.424336in}{8.180209in}}%
\pgfpathlineto{\pgfqpoint{2.450877in}{8.325970in}}%
\pgfpathlineto{\pgfqpoint{2.477419in}{8.365104in}}%
\pgfpathlineto{\pgfqpoint{2.503961in}{8.323290in}}%
\pgfusepath{stroke}%
\end{pgfscope}%
\begin{pgfscope}%
\pgfpathrectangle{\pgfqpoint{0.828241in}{6.184745in}}{\pgfqpoint{2.414722in}{2.357859in}}%
\pgfusepath{clip}%
\pgfsetroundcap%
\pgfsetroundjoin%
\pgfsetlinewidth{1.505625pt}%
\definecolor{currentstroke}{rgb}{0.768627,0.305882,0.321569}%
\pgfsetstrokecolor{currentstroke}%
\pgfsetdash{}{0pt}%
\pgfpathmoveto{\pgfqpoint{0.938001in}{8.120946in}}%
\pgfpathlineto{\pgfqpoint{0.964135in}{8.136158in}}%
\pgfpathlineto{\pgfqpoint{0.990268in}{8.160371in}}%
\pgfpathlineto{\pgfqpoint{1.016401in}{8.195572in}}%
\pgfpathlineto{\pgfqpoint{1.042535in}{8.237495in}}%
\pgfpathlineto{\pgfqpoint{1.068668in}{8.280473in}}%
\pgfpathlineto{\pgfqpoint{1.094801in}{8.320898in}}%
\pgfpathlineto{\pgfqpoint{1.120935in}{8.356999in}}%
\pgfpathlineto{\pgfqpoint{1.147068in}{8.386221in}}%
\pgfpathlineto{\pgfqpoint{1.173201in}{8.402965in}}%
\pgfpathlineto{\pgfqpoint{1.199335in}{8.402954in}}%
\pgfpathlineto{\pgfqpoint{1.225468in}{8.387940in}}%
\pgfpathlineto{\pgfqpoint{1.251602in}{8.365104in}}%
\pgfpathlineto{\pgfqpoint{1.277735in}{8.342808in}}%
\pgfpathlineto{\pgfqpoint{1.303868in}{8.327303in}}%
\pgfpathlineto{\pgfqpoint{1.330002in}{8.319722in}}%
\pgfpathlineto{\pgfqpoint{1.356135in}{8.316856in}}%
\pgfpathlineto{\pgfqpoint{1.382268in}{8.313449in}}%
\pgfpathlineto{\pgfqpoint{1.408402in}{8.303224in}}%
\pgfpathlineto{\pgfqpoint{1.434535in}{8.279321in}}%
\pgfpathlineto{\pgfqpoint{1.460668in}{8.236368in}}%
\pgfpathlineto{\pgfqpoint{1.486802in}{8.175324in}}%
\pgfpathlineto{\pgfqpoint{1.512935in}{8.106205in}}%
\pgfpathlineto{\pgfqpoint{1.539068in}{8.040991in}}%
\pgfpathlineto{\pgfqpoint{1.565202in}{7.983038in}}%
\pgfpathlineto{\pgfqpoint{1.591335in}{7.926190in}}%
\pgfpathlineto{\pgfqpoint{1.617468in}{7.862483in}}%
\pgfpathlineto{\pgfqpoint{1.643602in}{7.788081in}}%
\pgfpathlineto{\pgfqpoint{1.669735in}{7.705617in}}%
\pgfpathlineto{\pgfqpoint{1.695869in}{7.625446in}}%
\pgfpathlineto{\pgfqpoint{1.722002in}{7.558022in}}%
\pgfpathlineto{\pgfqpoint{1.748135in}{7.509151in}}%
\pgfpathlineto{\pgfqpoint{1.774269in}{7.482131in}}%
\pgfpathlineto{\pgfqpoint{1.800402in}{7.480510in}}%
\pgfpathlineto{\pgfqpoint{1.826535in}{7.502246in}}%
\pgfpathlineto{\pgfqpoint{1.852669in}{7.533496in}}%
\pgfpathlineto{\pgfqpoint{1.878802in}{7.556804in}}%
\pgfpathlineto{\pgfqpoint{1.904935in}{7.564120in}}%
\pgfpathlineto{\pgfqpoint{1.931069in}{7.556462in}}%
\pgfpathlineto{\pgfqpoint{1.957202in}{7.530828in}}%
\pgfpathlineto{\pgfqpoint{1.983335in}{7.482788in}}%
\pgfpathlineto{\pgfqpoint{2.009469in}{7.424902in}}%
\pgfpathlineto{\pgfqpoint{2.035602in}{7.390495in}}%
\pgfpathlineto{\pgfqpoint{2.061735in}{7.412200in}}%
\pgfpathlineto{\pgfqpoint{2.087869in}{7.493663in}}%
\pgfpathlineto{\pgfqpoint{2.114002in}{7.612543in}}%
\pgfpathlineto{\pgfqpoint{2.140136in}{7.737979in}}%
\pgfpathlineto{\pgfqpoint{2.166269in}{7.847975in}}%
\pgfpathlineto{\pgfqpoint{2.192402in}{7.936788in}}%
\pgfpathlineto{\pgfqpoint{2.218536in}{8.006847in}}%
\pgfpathlineto{\pgfqpoint{2.244669in}{8.057832in}}%
\pgfpathlineto{\pgfqpoint{2.270802in}{8.088381in}}%
\pgfpathlineto{\pgfqpoint{2.296936in}{8.101817in}}%
\pgfpathlineto{\pgfqpoint{2.323069in}{8.106608in}}%
\pgfpathlineto{\pgfqpoint{2.349202in}{8.111943in}}%
\pgfpathlineto{\pgfqpoint{2.375336in}{8.123827in}}%
\pgfpathlineto{\pgfqpoint{2.401469in}{8.143022in}}%
\pgfpathlineto{\pgfqpoint{2.427602in}{8.163244in}}%
\pgfpathlineto{\pgfqpoint{2.453736in}{8.175995in}}%
\pgfpathlineto{\pgfqpoint{2.479869in}{8.179388in}}%
\pgfpathlineto{\pgfqpoint{2.506002in}{8.179562in}}%
\pgfpathlineto{\pgfqpoint{2.532136in}{8.182958in}}%
\pgfpathlineto{\pgfqpoint{2.558269in}{8.192769in}}%
\pgfpathlineto{\pgfqpoint{2.584403in}{8.213305in}}%
\pgfpathlineto{\pgfqpoint{2.610536in}{8.250242in}}%
\pgfpathlineto{\pgfqpoint{2.636669in}{8.302029in}}%
\pgfpathlineto{\pgfqpoint{2.662803in}{8.357131in}}%
\pgfpathlineto{\pgfqpoint{2.688936in}{8.401560in}}%
\pgfpathlineto{\pgfqpoint{2.715069in}{8.427658in}}%
\pgfpathlineto{\pgfqpoint{2.741203in}{8.435429in}}%
\pgfpathlineto{\pgfqpoint{2.767336in}{8.426565in}}%
\pgfpathlineto{\pgfqpoint{2.793469in}{8.402143in}}%
\pgfpathlineto{\pgfqpoint{2.819603in}{8.365104in}}%
\pgfpathlineto{\pgfqpoint{2.845736in}{8.321941in}}%
\pgfpathlineto{\pgfqpoint{2.871869in}{8.281239in}}%
\pgfpathlineto{\pgfqpoint{2.898003in}{8.248477in}}%
\pgfpathlineto{\pgfqpoint{2.924136in}{8.223303in}}%
\pgfpathlineto{\pgfqpoint{2.950269in}{8.198239in}}%
\pgfpathlineto{\pgfqpoint{2.976403in}{8.163598in}}%
\pgfpathlineto{\pgfqpoint{3.002536in}{8.113996in}}%
\pgfpathlineto{\pgfqpoint{3.028669in}{8.050668in}}%
\pgfpathlineto{\pgfqpoint{3.054803in}{7.977800in}}%
\pgfpathlineto{\pgfqpoint{3.080936in}{7.899503in}}%
\pgfpathlineto{\pgfqpoint{3.107070in}{7.817566in}}%
\pgfpathlineto{\pgfqpoint{3.133203in}{7.732011in}}%
\pgfusepath{stroke}%
\end{pgfscope}%
\begin{pgfscope}%
\pgfpathrectangle{\pgfqpoint{0.828241in}{6.184745in}}{\pgfqpoint{2.414722in}{2.357859in}}%
\pgfusepath{clip}%
\pgfsetroundcap%
\pgfsetroundjoin%
\pgfsetlinewidth{1.505625pt}%
\definecolor{currentstroke}{rgb}{0.505882,0.447059,0.701961}%
\pgfsetstrokecolor{currentstroke}%
\pgfsetdash{}{0pt}%
\pgfpathmoveto{\pgfqpoint{0.938001in}{7.401988in}}%
\pgfpathlineto{\pgfqpoint{0.965848in}{7.405310in}}%
\pgfpathlineto{\pgfqpoint{0.993695in}{7.427198in}}%
\pgfpathlineto{\pgfqpoint{1.021542in}{7.489106in}}%
\pgfpathlineto{\pgfqpoint{1.049389in}{7.607691in}}%
\pgfpathlineto{\pgfqpoint{1.077236in}{7.781760in}}%
\pgfpathlineto{\pgfqpoint{1.105083in}{7.982943in}}%
\pgfpathlineto{\pgfqpoint{1.132930in}{8.165701in}}%
\pgfpathlineto{\pgfqpoint{1.160777in}{8.293740in}}%
\pgfpathlineto{\pgfqpoint{1.188624in}{8.355952in}}%
\pgfpathlineto{\pgfqpoint{1.216471in}{8.365104in}}%
\pgfpathlineto{\pgfqpoint{1.244318in}{8.351383in}}%
\pgfpathlineto{\pgfqpoint{1.272166in}{8.343586in}}%
\pgfpathlineto{\pgfqpoint{1.300013in}{8.349267in}}%
\pgfpathlineto{\pgfqpoint{1.327860in}{8.351905in}}%
\pgfpathlineto{\pgfqpoint{1.355707in}{8.325741in}}%
\pgfpathlineto{\pgfqpoint{1.383554in}{8.251649in}}%
\pgfpathlineto{\pgfqpoint{1.411401in}{8.125814in}}%
\pgfpathlineto{\pgfqpoint{1.439248in}{7.960147in}}%
\pgfpathlineto{\pgfqpoint{1.467095in}{7.774275in}}%
\pgfpathlineto{\pgfqpoint{1.494942in}{7.583480in}}%
\pgfpathlineto{\pgfqpoint{1.522789in}{7.394911in}}%
\pgfpathlineto{\pgfqpoint{1.550636in}{7.215071in}}%
\pgfpathlineto{\pgfqpoint{1.578483in}{7.053001in}}%
\pgfpathlineto{\pgfqpoint{1.606330in}{6.913124in}}%
\pgfpathlineto{\pgfqpoint{1.634177in}{6.791128in}}%
\pgfpathlineto{\pgfqpoint{1.662024in}{6.679900in}}%
\pgfpathlineto{\pgfqpoint{1.689871in}{6.577519in}}%
\pgfpathlineto{\pgfqpoint{1.717718in}{6.488328in}}%
\pgfpathlineto{\pgfqpoint{1.745565in}{6.420646in}}%
\pgfpathlineto{\pgfqpoint{1.773412in}{6.379673in}}%
\pgfpathlineto{\pgfqpoint{1.801259in}{6.361350in}}%
\pgfpathlineto{\pgfqpoint{1.829106in}{6.352944in}}%
\pgfpathlineto{\pgfqpoint{1.856953in}{6.341419in}}%
\pgfpathlineto{\pgfqpoint{1.884800in}{6.322285in}}%
\pgfpathlineto{\pgfqpoint{1.912647in}{6.301919in}}%
\pgfpathlineto{\pgfqpoint{1.940494in}{6.291921in}}%
\pgfpathlineto{\pgfqpoint{1.968341in}{6.298354in}}%
\pgfpathlineto{\pgfqpoint{1.996188in}{6.325437in}}%
\pgfpathlineto{\pgfqpoint{2.024035in}{6.379544in}}%
\pgfpathlineto{\pgfqpoint{2.051882in}{6.464187in}}%
\pgfpathlineto{\pgfqpoint{2.079729in}{6.578834in}}%
\pgfpathlineto{\pgfqpoint{2.107576in}{6.715174in}}%
\pgfpathlineto{\pgfqpoint{2.135423in}{6.856110in}}%
\pgfpathlineto{\pgfqpoint{2.163270in}{6.983739in}}%
\pgfpathlineto{\pgfqpoint{2.191117in}{7.088658in}}%
\pgfpathlineto{\pgfqpoint{2.218964in}{7.170488in}}%
\pgfpathlineto{\pgfqpoint{2.246811in}{7.232193in}}%
\pgfpathlineto{\pgfqpoint{2.274658in}{7.275026in}}%
\pgfpathlineto{\pgfqpoint{2.302505in}{7.298524in}}%
\pgfpathlineto{\pgfqpoint{2.330352in}{7.305196in}}%
\pgfpathlineto{\pgfqpoint{2.358199in}{7.306276in}}%
\pgfpathlineto{\pgfqpoint{2.386046in}{7.324094in}}%
\pgfpathlineto{\pgfqpoint{2.413893in}{7.389664in}}%
\pgfpathlineto{\pgfqpoint{2.441740in}{7.531601in}}%
\pgfpathlineto{\pgfqpoint{2.469587in}{7.752841in}}%
\pgfpathlineto{\pgfqpoint{2.497434in}{8.011594in}}%
\pgfpathlineto{\pgfqpoint{2.525281in}{8.238520in}}%
\pgfpathlineto{\pgfqpoint{2.553128in}{8.381983in}}%
\pgfpathlineto{\pgfqpoint{2.580975in}{8.431680in}}%
\pgfpathlineto{\pgfqpoint{2.608822in}{8.411936in}}%
\pgfpathlineto{\pgfqpoint{2.636669in}{8.365104in}}%
\pgfpathlineto{\pgfqpoint{2.664516in}{8.331193in}}%
\pgfpathlineto{\pgfqpoint{2.692363in}{8.327780in}}%
\pgfpathlineto{\pgfqpoint{2.720210in}{8.341182in}}%
\pgfpathlineto{\pgfqpoint{2.748057in}{8.332819in}}%
\pgfpathlineto{\pgfqpoint{2.775904in}{8.265223in}}%
\pgfpathlineto{\pgfqpoint{2.803751in}{8.130158in}}%
\pgfpathlineto{\pgfqpoint{2.831598in}{7.948294in}}%
\pgfpathlineto{\pgfqpoint{2.859445in}{7.748114in}}%
\pgfpathlineto{\pgfqpoint{2.887292in}{7.551765in}}%
\pgfpathlineto{\pgfqpoint{2.915139in}{7.365921in}}%
\pgfusepath{stroke}%
\end{pgfscope}%
\begin{pgfscope}%
\pgfsetrectcap%
\pgfsetmiterjoin%
\pgfsetlinewidth{1.254687pt}%
\definecolor{currentstroke}{rgb}{1.000000,1.000000,1.000000}%
\pgfsetstrokecolor{currentstroke}%
\pgfsetdash{}{0pt}%
\pgfpathmoveto{\pgfqpoint{0.828241in}{6.184745in}}%
\pgfpathlineto{\pgfqpoint{0.828241in}{8.542604in}}%
\pgfusepath{stroke}%
\end{pgfscope}%
\begin{pgfscope}%
\pgfsetrectcap%
\pgfsetmiterjoin%
\pgfsetlinewidth{1.254687pt}%
\definecolor{currentstroke}{rgb}{1.000000,1.000000,1.000000}%
\pgfsetstrokecolor{currentstroke}%
\pgfsetdash{}{0pt}%
\pgfpathmoveto{\pgfqpoint{3.242963in}{6.184745in}}%
\pgfpathlineto{\pgfqpoint{3.242963in}{8.542604in}}%
\pgfusepath{stroke}%
\end{pgfscope}%
\begin{pgfscope}%
\pgfsetrectcap%
\pgfsetmiterjoin%
\pgfsetlinewidth{1.254687pt}%
\definecolor{currentstroke}{rgb}{1.000000,1.000000,1.000000}%
\pgfsetstrokecolor{currentstroke}%
\pgfsetdash{}{0pt}%
\pgfpathmoveto{\pgfqpoint{0.828241in}{6.184745in}}%
\pgfpathlineto{\pgfqpoint{3.242963in}{6.184745in}}%
\pgfusepath{stroke}%
\end{pgfscope}%
\begin{pgfscope}%
\pgfsetrectcap%
\pgfsetmiterjoin%
\pgfsetlinewidth{1.254687pt}%
\definecolor{currentstroke}{rgb}{1.000000,1.000000,1.000000}%
\pgfsetstrokecolor{currentstroke}%
\pgfsetdash{}{0pt}%
\pgfpathmoveto{\pgfqpoint{0.828241in}{8.542604in}}%
\pgfpathlineto{\pgfqpoint{3.242963in}{8.542604in}}%
\pgfusepath{stroke}%
\end{pgfscope}%
\begin{pgfscope}%
\definecolor{textcolor}{rgb}{0.150000,0.150000,0.150000}%
\pgfsetstrokecolor{textcolor}%
\pgfsetfillcolor{textcolor}%
\pgftext[x=2.035602in,y=8.625938in,,base]{\color{textcolor}\sffamily\fontsize{11.000000}{13.200000}\selectfont (a) Cluster 1 members}%
\end{pgfscope}%
\begin{pgfscope}%
\pgfsetbuttcap%
\pgfsetmiterjoin%
\definecolor{currentfill}{rgb}{0.917647,0.917647,0.949020}%
\pgfsetfillcolor{currentfill}%
\pgfsetlinewidth{0.000000pt}%
\definecolor{currentstroke}{rgb}{0.000000,0.000000,0.000000}%
\pgfsetstrokecolor{currentstroke}%
\pgfsetstrokeopacity{0.000000}%
\pgfsetdash{}{0pt}%
\pgfpathmoveto{\pgfqpoint{3.825278in}{6.184745in}}%
\pgfpathlineto{\pgfqpoint{6.240000in}{6.184745in}}%
\pgfpathlineto{\pgfqpoint{6.240000in}{8.542604in}}%
\pgfpathlineto{\pgfqpoint{3.825278in}{8.542604in}}%
\pgfpathclose%
\pgfusepath{fill}%
\end{pgfscope}%
\begin{pgfscope}%
\pgfpathrectangle{\pgfqpoint{3.825278in}{6.184745in}}{\pgfqpoint{2.414722in}{2.357859in}}%
\pgfusepath{clip}%
\pgfsetroundcap%
\pgfsetroundjoin%
\pgfsetlinewidth{1.003750pt}%
\definecolor{currentstroke}{rgb}{1.000000,1.000000,1.000000}%
\pgfsetstrokecolor{currentstroke}%
\pgfsetdash{}{0pt}%
\pgfpathmoveto{\pgfqpoint{3.935038in}{6.184745in}}%
\pgfpathlineto{\pgfqpoint{3.935038in}{8.542604in}}%
\pgfusepath{stroke}%
\end{pgfscope}%
\begin{pgfscope}%
\definecolor{textcolor}{rgb}{0.150000,0.150000,0.150000}%
\pgfsetstrokecolor{textcolor}%
\pgfsetfillcolor{textcolor}%
\pgftext[x=3.935038in,y=6.052801in,,top]{\color{textcolor}\sffamily\fontsize{11.000000}{13.200000}\selectfont \(\displaystyle 0.0\)}%
\end{pgfscope}%
\begin{pgfscope}%
\pgfpathrectangle{\pgfqpoint{3.825278in}{6.184745in}}{\pgfqpoint{2.414722in}{2.357859in}}%
\pgfusepath{clip}%
\pgfsetroundcap%
\pgfsetroundjoin%
\pgfsetlinewidth{1.003750pt}%
\definecolor{currentstroke}{rgb}{1.000000,1.000000,1.000000}%
\pgfsetstrokecolor{currentstroke}%
\pgfsetdash{}{0pt}%
\pgfpathmoveto{\pgfqpoint{4.566990in}{6.184745in}}%
\pgfpathlineto{\pgfqpoint{4.566990in}{8.542604in}}%
\pgfusepath{stroke}%
\end{pgfscope}%
\begin{pgfscope}%
\definecolor{textcolor}{rgb}{0.150000,0.150000,0.150000}%
\pgfsetstrokecolor{textcolor}%
\pgfsetfillcolor{textcolor}%
\pgftext[x=4.566990in,y=6.052801in,,top]{\color{textcolor}\sffamily\fontsize{11.000000}{13.200000}\selectfont \(\displaystyle 0.5\)}%
\end{pgfscope}%
\begin{pgfscope}%
\pgfpathrectangle{\pgfqpoint{3.825278in}{6.184745in}}{\pgfqpoint{2.414722in}{2.357859in}}%
\pgfusepath{clip}%
\pgfsetroundcap%
\pgfsetroundjoin%
\pgfsetlinewidth{1.003750pt}%
\definecolor{currentstroke}{rgb}{1.000000,1.000000,1.000000}%
\pgfsetstrokecolor{currentstroke}%
\pgfsetdash{}{0pt}%
\pgfpathmoveto{\pgfqpoint{5.198942in}{6.184745in}}%
\pgfpathlineto{\pgfqpoint{5.198942in}{8.542604in}}%
\pgfusepath{stroke}%
\end{pgfscope}%
\begin{pgfscope}%
\definecolor{textcolor}{rgb}{0.150000,0.150000,0.150000}%
\pgfsetstrokecolor{textcolor}%
\pgfsetfillcolor{textcolor}%
\pgftext[x=5.198942in,y=6.052801in,,top]{\color{textcolor}\sffamily\fontsize{11.000000}{13.200000}\selectfont \(\displaystyle 1.0\)}%
\end{pgfscope}%
\begin{pgfscope}%
\pgfpathrectangle{\pgfqpoint{3.825278in}{6.184745in}}{\pgfqpoint{2.414722in}{2.357859in}}%
\pgfusepath{clip}%
\pgfsetroundcap%
\pgfsetroundjoin%
\pgfsetlinewidth{1.003750pt}%
\definecolor{currentstroke}{rgb}{1.000000,1.000000,1.000000}%
\pgfsetstrokecolor{currentstroke}%
\pgfsetdash{}{0pt}%
\pgfpathmoveto{\pgfqpoint{5.830894in}{6.184745in}}%
\pgfpathlineto{\pgfqpoint{5.830894in}{8.542604in}}%
\pgfusepath{stroke}%
\end{pgfscope}%
\begin{pgfscope}%
\definecolor{textcolor}{rgb}{0.150000,0.150000,0.150000}%
\pgfsetstrokecolor{textcolor}%
\pgfsetfillcolor{textcolor}%
\pgftext[x=5.830894in,y=6.052801in,,top]{\color{textcolor}\sffamily\fontsize{11.000000}{13.200000}\selectfont \(\displaystyle 1.5\)}%
\end{pgfscope}%
\begin{pgfscope}%
\pgfpathrectangle{\pgfqpoint{3.825278in}{6.184745in}}{\pgfqpoint{2.414722in}{2.357859in}}%
\pgfusepath{clip}%
\pgfsetroundcap%
\pgfsetroundjoin%
\pgfsetlinewidth{1.003750pt}%
\definecolor{currentstroke}{rgb}{1.000000,1.000000,1.000000}%
\pgfsetstrokecolor{currentstroke}%
\pgfsetdash{}{0pt}%
\pgfpathmoveto{\pgfqpoint{3.825278in}{6.529809in}}%
\pgfpathlineto{\pgfqpoint{6.240000in}{6.529809in}}%
\pgfusepath{stroke}%
\end{pgfscope}%
\begin{pgfscope}%
\definecolor{textcolor}{rgb}{0.150000,0.150000,0.150000}%
\pgfsetstrokecolor{textcolor}%
\pgfsetfillcolor{textcolor}%
\pgftext[x=3.422963in,y=6.477002in,left,base]{\color{textcolor}\sffamily\fontsize{11.000000}{13.200000}\selectfont \(\displaystyle -20\)}%
\end{pgfscope}%
\begin{pgfscope}%
\pgfpathrectangle{\pgfqpoint{3.825278in}{6.184745in}}{\pgfqpoint{2.414722in}{2.357859in}}%
\pgfusepath{clip}%
\pgfsetroundcap%
\pgfsetroundjoin%
\pgfsetlinewidth{1.003750pt}%
\definecolor{currentstroke}{rgb}{1.000000,1.000000,1.000000}%
\pgfsetstrokecolor{currentstroke}%
\pgfsetdash{}{0pt}%
\pgfpathmoveto{\pgfqpoint{3.825278in}{6.989780in}}%
\pgfpathlineto{\pgfqpoint{6.240000in}{6.989780in}}%
\pgfusepath{stroke}%
\end{pgfscope}%
\begin{pgfscope}%
\definecolor{textcolor}{rgb}{0.150000,0.150000,0.150000}%
\pgfsetstrokecolor{textcolor}%
\pgfsetfillcolor{textcolor}%
\pgftext[x=3.422963in,y=6.936973in,left,base]{\color{textcolor}\sffamily\fontsize{11.000000}{13.200000}\selectfont \(\displaystyle -15\)}%
\end{pgfscope}%
\begin{pgfscope}%
\pgfpathrectangle{\pgfqpoint{3.825278in}{6.184745in}}{\pgfqpoint{2.414722in}{2.357859in}}%
\pgfusepath{clip}%
\pgfsetroundcap%
\pgfsetroundjoin%
\pgfsetlinewidth{1.003750pt}%
\definecolor{currentstroke}{rgb}{1.000000,1.000000,1.000000}%
\pgfsetstrokecolor{currentstroke}%
\pgfsetdash{}{0pt}%
\pgfpathmoveto{\pgfqpoint{3.825278in}{7.449750in}}%
\pgfpathlineto{\pgfqpoint{6.240000in}{7.449750in}}%
\pgfusepath{stroke}%
\end{pgfscope}%
\begin{pgfscope}%
\definecolor{textcolor}{rgb}{0.150000,0.150000,0.150000}%
\pgfsetstrokecolor{textcolor}%
\pgfsetfillcolor{textcolor}%
\pgftext[x=3.422963in,y=7.396943in,left,base]{\color{textcolor}\sffamily\fontsize{11.000000}{13.200000}\selectfont \(\displaystyle -10\)}%
\end{pgfscope}%
\begin{pgfscope}%
\pgfpathrectangle{\pgfqpoint{3.825278in}{6.184745in}}{\pgfqpoint{2.414722in}{2.357859in}}%
\pgfusepath{clip}%
\pgfsetroundcap%
\pgfsetroundjoin%
\pgfsetlinewidth{1.003750pt}%
\definecolor{currentstroke}{rgb}{1.000000,1.000000,1.000000}%
\pgfsetstrokecolor{currentstroke}%
\pgfsetdash{}{0pt}%
\pgfpathmoveto{\pgfqpoint{3.825278in}{7.909721in}}%
\pgfpathlineto{\pgfqpoint{6.240000in}{7.909721in}}%
\pgfusepath{stroke}%
\end{pgfscope}%
\begin{pgfscope}%
\definecolor{textcolor}{rgb}{0.150000,0.150000,0.150000}%
\pgfsetstrokecolor{textcolor}%
\pgfsetfillcolor{textcolor}%
\pgftext[x=3.499005in,y=7.856914in,left,base]{\color{textcolor}\sffamily\fontsize{11.000000}{13.200000}\selectfont \(\displaystyle -5\)}%
\end{pgfscope}%
\begin{pgfscope}%
\pgfpathrectangle{\pgfqpoint{3.825278in}{6.184745in}}{\pgfqpoint{2.414722in}{2.357859in}}%
\pgfusepath{clip}%
\pgfsetroundcap%
\pgfsetroundjoin%
\pgfsetlinewidth{1.003750pt}%
\definecolor{currentstroke}{rgb}{1.000000,1.000000,1.000000}%
\pgfsetstrokecolor{currentstroke}%
\pgfsetdash{}{0pt}%
\pgfpathmoveto{\pgfqpoint{3.825278in}{8.369691in}}%
\pgfpathlineto{\pgfqpoint{6.240000in}{8.369691in}}%
\pgfusepath{stroke}%
\end{pgfscope}%
\begin{pgfscope}%
\definecolor{textcolor}{rgb}{0.150000,0.150000,0.150000}%
\pgfsetstrokecolor{textcolor}%
\pgfsetfillcolor{textcolor}%
\pgftext[x=3.617292in,y=8.316884in,left,base]{\color{textcolor}\sffamily\fontsize{11.000000}{13.200000}\selectfont \(\displaystyle 0\)}%
\end{pgfscope}%
\begin{pgfscope}%
\pgfpathrectangle{\pgfqpoint{3.825278in}{6.184745in}}{\pgfqpoint{2.414722in}{2.357859in}}%
\pgfusepath{clip}%
\pgfsetroundcap%
\pgfsetroundjoin%
\pgfsetlinewidth{1.505625pt}%
\definecolor{currentstroke}{rgb}{0.298039,0.447059,0.690196}%
\pgfsetstrokecolor{currentstroke}%
\pgfsetdash{}{0pt}%
\pgfpathmoveto{\pgfqpoint{3.935038in}{7.754719in}}%
\pgfpathlineto{\pgfqpoint{3.950452in}{7.754811in}}%
\pgfpathlineto{\pgfqpoint{3.965865in}{7.762000in}}%
\pgfpathlineto{\pgfqpoint{3.981279in}{7.784571in}}%
\pgfpathlineto{\pgfqpoint{3.996692in}{7.828940in}}%
\pgfpathlineto{\pgfqpoint{4.012106in}{7.895076in}}%
\pgfpathlineto{\pgfqpoint{4.027519in}{7.977349in}}%
\pgfpathlineto{\pgfqpoint{4.042933in}{8.069678in}}%
\pgfpathlineto{\pgfqpoint{4.058346in}{8.166785in}}%
\pgfpathlineto{\pgfqpoint{4.073760in}{8.258876in}}%
\pgfpathlineto{\pgfqpoint{4.089173in}{8.330618in}}%
\pgfpathlineto{\pgfqpoint{4.104586in}{8.372267in}}%
\pgfpathlineto{\pgfqpoint{4.120000in}{8.383853in}}%
\pgfpathlineto{\pgfqpoint{4.135413in}{8.369691in}}%
\pgfpathlineto{\pgfqpoint{4.150827in}{8.337296in}}%
\pgfpathlineto{\pgfqpoint{4.166240in}{8.300681in}}%
\pgfpathlineto{\pgfqpoint{4.181654in}{8.273148in}}%
\pgfpathlineto{\pgfqpoint{4.197067in}{8.254667in}}%
\pgfpathlineto{\pgfqpoint{4.212481in}{8.235041in}}%
\pgfpathlineto{\pgfqpoint{4.227894in}{8.201936in}}%
\pgfpathlineto{\pgfqpoint{4.243308in}{8.145225in}}%
\pgfpathlineto{\pgfqpoint{4.258721in}{8.061968in}}%
\pgfpathlineto{\pgfqpoint{4.274135in}{7.959206in}}%
\pgfpathlineto{\pgfqpoint{4.289548in}{7.852098in}}%
\pgfpathlineto{\pgfqpoint{4.304961in}{7.754387in}}%
\pgfpathlineto{\pgfqpoint{4.320375in}{7.669762in}}%
\pgfpathlineto{\pgfqpoint{4.335788in}{7.593504in}}%
\pgfpathlineto{\pgfqpoint{4.351202in}{7.518503in}}%
\pgfpathlineto{\pgfqpoint{4.366615in}{7.441444in}}%
\pgfpathlineto{\pgfqpoint{4.382029in}{7.364626in}}%
\pgfpathlineto{\pgfqpoint{4.397442in}{7.290862in}}%
\pgfpathlineto{\pgfqpoint{4.412856in}{7.220627in}}%
\pgfpathlineto{\pgfqpoint{4.428269in}{7.150516in}}%
\pgfpathlineto{\pgfqpoint{4.443683in}{7.075931in}}%
\pgfpathlineto{\pgfqpoint{4.459096in}{7.002069in}}%
\pgfpathlineto{\pgfqpoint{4.474510in}{6.933378in}}%
\pgfpathlineto{\pgfqpoint{4.489923in}{6.870360in}}%
\pgfpathlineto{\pgfqpoint{4.505336in}{6.813215in}}%
\pgfpathlineto{\pgfqpoint{4.520750in}{6.763321in}}%
\pgfpathlineto{\pgfqpoint{4.536163in}{6.723980in}}%
\pgfpathlineto{\pgfqpoint{4.551577in}{6.699656in}}%
\pgfpathlineto{\pgfqpoint{4.566990in}{6.694854in}}%
\pgfpathlineto{\pgfqpoint{4.582404in}{6.711801in}}%
\pgfpathlineto{\pgfqpoint{4.597817in}{6.749985in}}%
\pgfpathlineto{\pgfqpoint{4.613231in}{6.807068in}}%
\pgfpathlineto{\pgfqpoint{4.628644in}{6.877043in}}%
\pgfpathlineto{\pgfqpoint{4.644058in}{6.948071in}}%
\pgfpathlineto{\pgfqpoint{4.659471in}{7.007655in}}%
\pgfpathlineto{\pgfqpoint{4.674885in}{7.050225in}}%
\pgfpathlineto{\pgfqpoint{4.690298in}{7.076570in}}%
\pgfpathlineto{\pgfqpoint{4.705711in}{7.094618in}}%
\pgfpathlineto{\pgfqpoint{4.721125in}{7.117730in}}%
\pgfpathlineto{\pgfqpoint{4.736538in}{7.154772in}}%
\pgfpathlineto{\pgfqpoint{4.751952in}{7.205718in}}%
\pgfpathlineto{\pgfqpoint{4.767365in}{7.265510in}}%
\pgfpathlineto{\pgfqpoint{4.782779in}{7.331122in}}%
\pgfpathlineto{\pgfqpoint{4.798192in}{7.401702in}}%
\pgfpathlineto{\pgfqpoint{4.813606in}{7.475684in}}%
\pgfpathlineto{\pgfqpoint{4.829019in}{7.548040in}}%
\pgfpathlineto{\pgfqpoint{4.844433in}{7.610699in}}%
\pgfpathlineto{\pgfqpoint{4.859846in}{7.655743in}}%
\pgfpathlineto{\pgfqpoint{4.875260in}{7.681523in}}%
\pgfpathlineto{\pgfqpoint{4.890673in}{7.693281in}}%
\pgfpathlineto{\pgfqpoint{4.906086in}{7.697432in}}%
\pgfpathlineto{\pgfqpoint{4.921500in}{7.698424in}}%
\pgfpathlineto{\pgfqpoint{4.936913in}{7.699347in}}%
\pgfpathlineto{\pgfqpoint{4.952327in}{7.702057in}}%
\pgfpathlineto{\pgfqpoint{4.967740in}{7.706967in}}%
\pgfpathlineto{\pgfqpoint{4.983154in}{7.714952in}}%
\pgfpathlineto{\pgfqpoint{4.998567in}{7.729900in}}%
\pgfpathlineto{\pgfqpoint{5.013981in}{7.758852in}}%
\pgfpathlineto{\pgfqpoint{5.029394in}{7.808909in}}%
\pgfpathlineto{\pgfqpoint{5.044808in}{7.882297in}}%
\pgfpathlineto{\pgfqpoint{5.060221in}{7.976182in}}%
\pgfpathlineto{\pgfqpoint{5.075635in}{8.084303in}}%
\pgfpathlineto{\pgfqpoint{5.091048in}{8.198207in}}%
\pgfpathlineto{\pgfqpoint{5.106462in}{8.307294in}}%
\pgfpathlineto{\pgfqpoint{5.121875in}{8.393775in}}%
\pgfpathlineto{\pgfqpoint{5.137288in}{8.435429in}}%
\pgfpathlineto{\pgfqpoint{5.152702in}{8.423333in}}%
\pgfpathlineto{\pgfqpoint{5.168115in}{8.369691in}}%
\pgfpathlineto{\pgfqpoint{5.183529in}{8.301621in}}%
\pgfpathlineto{\pgfqpoint{5.198942in}{8.248728in}}%
\pgfpathlineto{\pgfqpoint{5.214356in}{8.225459in}}%
\pgfpathlineto{\pgfqpoint{5.229769in}{8.220226in}}%
\pgfpathlineto{\pgfqpoint{5.245183in}{8.208926in}}%
\pgfpathlineto{\pgfqpoint{5.260596in}{8.174003in}}%
\pgfpathlineto{\pgfqpoint{5.276010in}{8.110881in}}%
\pgfpathlineto{\pgfqpoint{5.291423in}{8.025517in}}%
\pgfpathlineto{\pgfqpoint{5.306837in}{7.927863in}}%
\pgfpathlineto{\pgfqpoint{5.322250in}{7.825535in}}%
\pgfpathlineto{\pgfqpoint{5.337663in}{7.726734in}}%
\pgfpathlineto{\pgfqpoint{5.353077in}{7.639264in}}%
\pgfpathlineto{\pgfqpoint{5.368490in}{7.562475in}}%
\pgfusepath{stroke}%
\end{pgfscope}%
\begin{pgfscope}%
\pgfpathrectangle{\pgfqpoint{3.825278in}{6.184745in}}{\pgfqpoint{2.414722in}{2.357859in}}%
\pgfusepath{clip}%
\pgfsetroundcap%
\pgfsetroundjoin%
\pgfsetlinewidth{1.505625pt}%
\definecolor{currentstroke}{rgb}{0.866667,0.517647,0.321569}%
\pgfsetstrokecolor{currentstroke}%
\pgfsetdash{}{0pt}%
\pgfpathmoveto{\pgfqpoint{3.935038in}{7.493373in}}%
\pgfpathlineto{\pgfqpoint{3.950452in}{7.473102in}}%
\pgfpathlineto{\pgfqpoint{3.965865in}{7.466264in}}%
\pgfpathlineto{\pgfqpoint{3.981279in}{7.482355in}}%
\pgfpathlineto{\pgfqpoint{3.996692in}{7.527519in}}%
\pgfpathlineto{\pgfqpoint{4.012106in}{7.606506in}}%
\pgfpathlineto{\pgfqpoint{4.027519in}{7.721322in}}%
\pgfpathlineto{\pgfqpoint{4.042933in}{7.867145in}}%
\pgfpathlineto{\pgfqpoint{4.058346in}{8.028476in}}%
\pgfpathlineto{\pgfqpoint{4.073760in}{8.181115in}}%
\pgfpathlineto{\pgfqpoint{4.089173in}{8.298275in}}%
\pgfpathlineto{\pgfqpoint{4.104586in}{8.362642in}}%
\pgfpathlineto{\pgfqpoint{4.120000in}{8.379677in}}%
\pgfpathlineto{\pgfqpoint{4.135413in}{8.369691in}}%
\pgfpathlineto{\pgfqpoint{4.150827in}{8.353447in}}%
\pgfpathlineto{\pgfqpoint{4.166240in}{8.344358in}}%
\pgfpathlineto{\pgfqpoint{4.181654in}{8.346355in}}%
\pgfpathlineto{\pgfqpoint{4.197067in}{8.352026in}}%
\pgfpathlineto{\pgfqpoint{4.212481in}{8.344546in}}%
\pgfpathlineto{\pgfqpoint{4.227894in}{8.305161in}}%
\pgfpathlineto{\pgfqpoint{4.243308in}{8.223595in}}%
\pgfpathlineto{\pgfqpoint{4.258721in}{8.105522in}}%
\pgfpathlineto{\pgfqpoint{4.274135in}{7.969599in}}%
\pgfpathlineto{\pgfqpoint{4.289548in}{7.834921in}}%
\pgfpathlineto{\pgfqpoint{4.304961in}{7.712488in}}%
\pgfpathlineto{\pgfqpoint{4.320375in}{7.602268in}}%
\pgfpathlineto{\pgfqpoint{4.335788in}{7.499315in}}%
\pgfpathlineto{\pgfqpoint{4.351202in}{7.401676in}}%
\pgfpathlineto{\pgfqpoint{4.366615in}{7.309431in}}%
\pgfpathlineto{\pgfqpoint{4.382029in}{7.220392in}}%
\pgfpathlineto{\pgfqpoint{4.397442in}{7.129261in}}%
\pgfpathlineto{\pgfqpoint{4.412856in}{7.032897in}}%
\pgfpathlineto{\pgfqpoint{4.428269in}{6.932738in}}%
\pgfpathlineto{\pgfqpoint{4.443683in}{6.830944in}}%
\pgfpathlineto{\pgfqpoint{4.459096in}{6.731854in}}%
\pgfpathlineto{\pgfqpoint{4.474510in}{6.639933in}}%
\pgfpathlineto{\pgfqpoint{4.489923in}{6.556534in}}%
\pgfpathlineto{\pgfqpoint{4.505336in}{6.481848in}}%
\pgfpathlineto{\pgfqpoint{4.520750in}{6.416509in}}%
\pgfpathlineto{\pgfqpoint{4.536163in}{6.361613in}}%
\pgfpathlineto{\pgfqpoint{4.551577in}{6.319006in}}%
\pgfpathlineto{\pgfqpoint{4.566990in}{6.293861in}}%
\pgfpathlineto{\pgfqpoint{4.582404in}{6.291921in}}%
\pgfpathlineto{\pgfqpoint{4.597817in}{6.312722in}}%
\pgfpathlineto{\pgfqpoint{4.613231in}{6.347578in}}%
\pgfpathlineto{\pgfqpoint{4.628644in}{6.385717in}}%
\pgfpathlineto{\pgfqpoint{4.644058in}{6.421910in}}%
\pgfpathlineto{\pgfqpoint{4.659471in}{6.457071in}}%
\pgfpathlineto{\pgfqpoint{4.674885in}{6.493784in}}%
\pgfpathlineto{\pgfqpoint{4.690298in}{6.532976in}}%
\pgfpathlineto{\pgfqpoint{4.705711in}{6.574333in}}%
\pgfpathlineto{\pgfqpoint{4.721125in}{6.620040in}}%
\pgfpathlineto{\pgfqpoint{4.736538in}{6.676207in}}%
\pgfpathlineto{\pgfqpoint{4.751952in}{6.749546in}}%
\pgfpathlineto{\pgfqpoint{4.767365in}{6.843564in}}%
\pgfpathlineto{\pgfqpoint{4.782779in}{6.956250in}}%
\pgfpathlineto{\pgfqpoint{4.798192in}{7.078191in}}%
\pgfpathlineto{\pgfqpoint{4.813606in}{7.197679in}}%
\pgfpathlineto{\pgfqpoint{4.829019in}{7.309680in}}%
\pgfpathlineto{\pgfqpoint{4.844433in}{7.410902in}}%
\pgfpathlineto{\pgfqpoint{4.859846in}{7.494870in}}%
\pgfpathlineto{\pgfqpoint{4.875260in}{7.555785in}}%
\pgfpathlineto{\pgfqpoint{4.890673in}{7.592599in}}%
\pgfpathlineto{\pgfqpoint{4.906086in}{7.610631in}}%
\pgfpathlineto{\pgfqpoint{4.921500in}{7.617472in}}%
\pgfpathlineto{\pgfqpoint{4.936913in}{7.617612in}}%
\pgfpathlineto{\pgfqpoint{4.952327in}{7.611452in}}%
\pgfpathlineto{\pgfqpoint{4.967740in}{7.600180in}}%
\pgfpathlineto{\pgfqpoint{4.983154in}{7.588380in}}%
\pgfpathlineto{\pgfqpoint{4.998567in}{7.583869in}}%
\pgfpathlineto{\pgfqpoint{5.013981in}{7.597978in}}%
\pgfpathlineto{\pgfqpoint{5.029394in}{7.642334in}}%
\pgfpathlineto{\pgfqpoint{5.044808in}{7.722168in}}%
\pgfpathlineto{\pgfqpoint{5.060221in}{7.831928in}}%
\pgfpathlineto{\pgfqpoint{5.075635in}{7.956914in}}%
\pgfpathlineto{\pgfqpoint{5.091048in}{8.081744in}}%
\pgfpathlineto{\pgfqpoint{5.106462in}{8.196579in}}%
\pgfpathlineto{\pgfqpoint{5.121875in}{8.292657in}}%
\pgfpathlineto{\pgfqpoint{5.137288in}{8.357634in}}%
\pgfpathlineto{\pgfqpoint{5.152702in}{8.382859in}}%
\pgfpathlineto{\pgfqpoint{5.168115in}{8.369691in}}%
\pgfpathlineto{\pgfqpoint{5.183529in}{8.332002in}}%
\pgfpathlineto{\pgfqpoint{5.198942in}{8.293518in}}%
\pgfpathlineto{\pgfqpoint{5.214356in}{8.273105in}}%
\pgfpathlineto{\pgfqpoint{5.229769in}{8.270037in}}%
\pgfpathlineto{\pgfqpoint{5.245183in}{8.266118in}}%
\pgfpathlineto{\pgfqpoint{5.260596in}{8.239400in}}%
\pgfpathlineto{\pgfqpoint{5.276010in}{8.177179in}}%
\pgfpathlineto{\pgfqpoint{5.291423in}{8.079788in}}%
\pgfpathlineto{\pgfqpoint{5.306837in}{7.958755in}}%
\pgfpathlineto{\pgfqpoint{5.322250in}{7.830418in}}%
\pgfpathlineto{\pgfqpoint{5.337663in}{7.709284in}}%
\pgfpathlineto{\pgfqpoint{5.353077in}{7.600032in}}%
\pgfpathlineto{\pgfqpoint{5.368490in}{7.497404in}}%
\pgfusepath{stroke}%
\end{pgfscope}%
\begin{pgfscope}%
\pgfpathrectangle{\pgfqpoint{3.825278in}{6.184745in}}{\pgfqpoint{2.414722in}{2.357859in}}%
\pgfusepath{clip}%
\pgfsetroundcap%
\pgfsetroundjoin%
\pgfsetlinewidth{1.505625pt}%
\definecolor{currentstroke}{rgb}{0.333333,0.658824,0.407843}%
\pgfsetstrokecolor{currentstroke}%
\pgfsetdash{}{0pt}%
\pgfpathmoveto{\pgfqpoint{3.935038in}{7.789594in}}%
\pgfpathlineto{\pgfqpoint{3.957212in}{7.801809in}}%
\pgfpathlineto{\pgfqpoint{3.979386in}{7.807614in}}%
\pgfpathlineto{\pgfqpoint{4.001560in}{7.803827in}}%
\pgfpathlineto{\pgfqpoint{4.023733in}{7.791346in}}%
\pgfpathlineto{\pgfqpoint{4.045907in}{7.776003in}}%
\pgfpathlineto{\pgfqpoint{4.068081in}{7.768499in}}%
\pgfpathlineto{\pgfqpoint{4.090255in}{7.780042in}}%
\pgfpathlineto{\pgfqpoint{4.112428in}{7.817804in}}%
\pgfpathlineto{\pgfqpoint{4.134602in}{7.883035in}}%
\pgfpathlineto{\pgfqpoint{4.156776in}{7.973068in}}%
\pgfpathlineto{\pgfqpoint{4.178950in}{8.082116in}}%
\pgfpathlineto{\pgfqpoint{4.201123in}{8.197713in}}%
\pgfpathlineto{\pgfqpoint{4.223297in}{8.300081in}}%
\pgfpathlineto{\pgfqpoint{4.245471in}{8.369691in}}%
\pgfpathlineto{\pgfqpoint{4.267645in}{8.396530in}}%
\pgfpathlineto{\pgfqpoint{4.289818in}{8.385327in}}%
\pgfpathlineto{\pgfqpoint{4.311992in}{8.353050in}}%
\pgfpathlineto{\pgfqpoint{4.334166in}{8.319215in}}%
\pgfpathlineto{\pgfqpoint{4.356340in}{8.295989in}}%
\pgfpathlineto{\pgfqpoint{4.378513in}{8.284179in}}%
\pgfpathlineto{\pgfqpoint{4.400687in}{8.277007in}}%
\pgfpathlineto{\pgfqpoint{4.422861in}{8.265522in}}%
\pgfpathlineto{\pgfqpoint{4.445035in}{8.241599in}}%
\pgfpathlineto{\pgfqpoint{4.467208in}{8.202431in}}%
\pgfpathlineto{\pgfqpoint{4.489382in}{8.151534in}}%
\pgfpathlineto{\pgfqpoint{4.511556in}{8.096397in}}%
\pgfpathlineto{\pgfqpoint{4.533730in}{8.042941in}}%
\pgfpathlineto{\pgfqpoint{4.555903in}{7.992983in}}%
\pgfpathlineto{\pgfqpoint{4.578077in}{7.945162in}}%
\pgfpathlineto{\pgfqpoint{4.600251in}{7.898137in}}%
\pgfpathlineto{\pgfqpoint{4.622425in}{7.852013in}}%
\pgfpathlineto{\pgfqpoint{4.644598in}{7.808625in}}%
\pgfpathlineto{\pgfqpoint{4.666772in}{7.771660in}}%
\pgfpathlineto{\pgfqpoint{4.688946in}{7.746702in}}%
\pgfpathlineto{\pgfqpoint{4.711120in}{7.739274in}}%
\pgfpathlineto{\pgfqpoint{4.733293in}{7.751218in}}%
\pgfpathlineto{\pgfqpoint{4.755467in}{7.778322in}}%
\pgfpathlineto{\pgfqpoint{4.777641in}{7.811921in}}%
\pgfpathlineto{\pgfqpoint{4.799815in}{7.843863in}}%
\pgfpathlineto{\pgfqpoint{4.821988in}{7.870033in}}%
\pgfpathlineto{\pgfqpoint{4.844162in}{7.889213in}}%
\pgfpathlineto{\pgfqpoint{4.866336in}{7.900572in}}%
\pgfpathlineto{\pgfqpoint{4.888510in}{7.902025in}}%
\pgfpathlineto{\pgfqpoint{4.910683in}{7.890349in}}%
\pgfpathlineto{\pgfqpoint{4.932857in}{7.866547in}}%
\pgfpathlineto{\pgfqpoint{4.955031in}{7.842015in}}%
\pgfpathlineto{\pgfqpoint{4.977205in}{7.831220in}}%
\pgfpathlineto{\pgfqpoint{4.999379in}{7.836307in}}%
\pgfpathlineto{\pgfqpoint{5.021552in}{7.847766in}}%
\pgfpathlineto{\pgfqpoint{5.043726in}{7.855836in}}%
\pgfpathlineto{\pgfqpoint{5.065900in}{7.857941in}}%
\pgfpathlineto{\pgfqpoint{5.088074in}{7.857182in}}%
\pgfpathlineto{\pgfqpoint{5.110247in}{7.857889in}}%
\pgfpathlineto{\pgfqpoint{5.132421in}{7.864101in}}%
\pgfpathlineto{\pgfqpoint{5.154595in}{7.877841in}}%
\pgfpathlineto{\pgfqpoint{5.176769in}{7.897629in}}%
\pgfpathlineto{\pgfqpoint{5.198942in}{7.920491in}}%
\pgfpathlineto{\pgfqpoint{5.221116in}{7.944609in}}%
\pgfpathlineto{\pgfqpoint{5.243290in}{7.969352in}}%
\pgfpathlineto{\pgfqpoint{5.265464in}{7.994423in}}%
\pgfpathlineto{\pgfqpoint{5.287637in}{8.018489in}}%
\pgfpathlineto{\pgfqpoint{5.309811in}{8.040020in}}%
\pgfpathlineto{\pgfqpoint{5.331985in}{8.057659in}}%
\pgfpathlineto{\pgfqpoint{5.354159in}{8.070934in}}%
\pgfpathlineto{\pgfqpoint{5.376332in}{8.080781in}}%
\pgfpathlineto{\pgfqpoint{5.398506in}{8.088063in}}%
\pgfpathlineto{\pgfqpoint{5.420680in}{8.091971in}}%
\pgfpathlineto{\pgfqpoint{5.442854in}{8.091723in}}%
\pgfpathlineto{\pgfqpoint{5.465027in}{8.088725in}}%
\pgfpathlineto{\pgfqpoint{5.487201in}{8.084561in}}%
\pgfpathlineto{\pgfqpoint{5.509375in}{8.079856in}}%
\pgfpathlineto{\pgfqpoint{5.531549in}{8.075237in}}%
\pgfpathlineto{\pgfqpoint{5.553722in}{8.071135in}}%
\pgfpathlineto{\pgfqpoint{5.575896in}{8.067281in}}%
\pgfpathlineto{\pgfqpoint{5.598070in}{8.063411in}}%
\pgfpathlineto{\pgfqpoint{5.620244in}{8.060756in}}%
\pgfpathlineto{\pgfqpoint{5.642417in}{8.065259in}}%
\pgfpathlineto{\pgfqpoint{5.664591in}{8.087262in}}%
\pgfpathlineto{\pgfqpoint{5.686765in}{8.133516in}}%
\pgfpathlineto{\pgfqpoint{5.708939in}{8.199673in}}%
\pgfpathlineto{\pgfqpoint{5.731112in}{8.270625in}}%
\pgfpathlineto{\pgfqpoint{5.753286in}{8.329061in}}%
\pgfpathlineto{\pgfqpoint{5.775460in}{8.364508in}}%
\pgfpathlineto{\pgfqpoint{5.797634in}{8.376262in}}%
\pgfpathlineto{\pgfqpoint{5.819807in}{8.369691in}}%
\pgfpathlineto{\pgfqpoint{5.841981in}{8.350635in}}%
\pgfpathlineto{\pgfqpoint{5.864155in}{8.322883in}}%
\pgfpathlineto{\pgfqpoint{5.886329in}{8.288085in}}%
\pgfpathlineto{\pgfqpoint{5.908502in}{8.245631in}}%
\pgfpathlineto{\pgfqpoint{5.930676in}{8.193859in}}%
\pgfpathlineto{\pgfqpoint{5.952850in}{8.132823in}}%
\pgfpathlineto{\pgfqpoint{5.975024in}{8.065431in}}%
\pgfpathlineto{\pgfqpoint{5.997197in}{7.995428in}}%
\pgfpathlineto{\pgfqpoint{6.019371in}{7.924366in}}%
\pgfpathlineto{\pgfqpoint{6.041545in}{7.853031in}}%
\pgfpathlineto{\pgfqpoint{6.063719in}{7.783709in}}%
\pgfpathlineto{\pgfqpoint{6.085892in}{7.719232in}}%
\pgfpathlineto{\pgfqpoint{6.108066in}{7.659252in}}%
\pgfpathlineto{\pgfqpoint{6.130240in}{7.601103in}}%
\pgfusepath{stroke}%
\end{pgfscope}%
\begin{pgfscope}%
\pgfpathrectangle{\pgfqpoint{3.825278in}{6.184745in}}{\pgfqpoint{2.414722in}{2.357859in}}%
\pgfusepath{clip}%
\pgfsetroundcap%
\pgfsetroundjoin%
\pgfsetlinewidth{1.505625pt}%
\definecolor{currentstroke}{rgb}{0.768627,0.305882,0.321569}%
\pgfsetstrokecolor{currentstroke}%
\pgfsetdash{}{0pt}%
\pgfpathmoveto{\pgfqpoint{3.935038in}{8.108027in}}%
\pgfpathlineto{\pgfqpoint{3.954787in}{8.107245in}}%
\pgfpathlineto{\pgfqpoint{3.974535in}{8.106251in}}%
\pgfpathlineto{\pgfqpoint{3.994284in}{8.105444in}}%
\pgfpathlineto{\pgfqpoint{4.014032in}{8.106507in}}%
\pgfpathlineto{\pgfqpoint{4.033781in}{8.113017in}}%
\pgfpathlineto{\pgfqpoint{4.053529in}{8.130826in}}%
\pgfpathlineto{\pgfqpoint{4.073278in}{8.165705in}}%
\pgfpathlineto{\pgfqpoint{4.093026in}{8.216613in}}%
\pgfpathlineto{\pgfqpoint{4.112775in}{8.272241in}}%
\pgfpathlineto{\pgfqpoint{4.132523in}{8.317911in}}%
\pgfpathlineto{\pgfqpoint{4.152272in}{8.346634in}}%
\pgfpathlineto{\pgfqpoint{4.172020in}{8.361465in}}%
\pgfpathlineto{\pgfqpoint{4.191769in}{8.368610in}}%
\pgfpathlineto{\pgfqpoint{4.211517in}{8.369691in}}%
\pgfpathlineto{\pgfqpoint{4.231266in}{8.360774in}}%
\pgfpathlineto{\pgfqpoint{4.251014in}{8.337426in}}%
\pgfpathlineto{\pgfqpoint{4.270763in}{8.299447in}}%
\pgfpathlineto{\pgfqpoint{4.290511in}{8.249125in}}%
\pgfpathlineto{\pgfqpoint{4.310260in}{8.188115in}}%
\pgfpathlineto{\pgfqpoint{4.330008in}{8.115011in}}%
\pgfpathlineto{\pgfqpoint{4.349757in}{8.029554in}}%
\pgfpathlineto{\pgfqpoint{4.369505in}{7.934406in}}%
\pgfpathlineto{\pgfqpoint{4.389254in}{7.833676in}}%
\pgfpathlineto{\pgfqpoint{4.409002in}{7.729792in}}%
\pgfpathlineto{\pgfqpoint{4.428751in}{7.625960in}}%
\pgfpathlineto{\pgfqpoint{4.448499in}{7.529171in}}%
\pgfpathlineto{\pgfqpoint{4.468248in}{7.446033in}}%
\pgfpathlineto{\pgfqpoint{4.487996in}{7.377200in}}%
\pgfpathlineto{\pgfqpoint{4.507745in}{7.318589in}}%
\pgfpathlineto{\pgfqpoint{4.527493in}{7.266676in}}%
\pgfpathlineto{\pgfqpoint{4.547242in}{7.221462in}}%
\pgfpathlineto{\pgfqpoint{4.566990in}{7.186164in}}%
\pgfpathlineto{\pgfqpoint{4.586739in}{7.164195in}}%
\pgfpathlineto{\pgfqpoint{4.606487in}{7.155546in}}%
\pgfpathlineto{\pgfqpoint{4.626236in}{7.156865in}}%
\pgfpathlineto{\pgfqpoint{4.645984in}{7.163081in}}%
\pgfpathlineto{\pgfqpoint{4.665733in}{7.169836in}}%
\pgfpathlineto{\pgfqpoint{4.685481in}{7.175636in}}%
\pgfpathlineto{\pgfqpoint{4.705230in}{7.181334in}}%
\pgfpathlineto{\pgfqpoint{4.724978in}{7.187306in}}%
\pgfpathlineto{\pgfqpoint{4.744727in}{7.192329in}}%
\pgfpathlineto{\pgfqpoint{4.764475in}{7.195956in}}%
\pgfpathlineto{\pgfqpoint{4.784224in}{7.202767in}}%
\pgfpathlineto{\pgfqpoint{4.803972in}{7.224520in}}%
\pgfpathlineto{\pgfqpoint{4.823721in}{7.275375in}}%
\pgfpathlineto{\pgfqpoint{4.843469in}{7.362268in}}%
\pgfpathlineto{\pgfqpoint{4.863218in}{7.479816in}}%
\pgfpathlineto{\pgfqpoint{4.882966in}{7.611468in}}%
\pgfpathlineto{\pgfqpoint{4.902715in}{7.736438in}}%
\pgfpathlineto{\pgfqpoint{4.922463in}{7.839379in}}%
\pgfpathlineto{\pgfqpoint{4.942212in}{7.915645in}}%
\pgfpathlineto{\pgfqpoint{4.961960in}{7.969909in}}%
\pgfpathlineto{\pgfqpoint{4.981709in}{8.010624in}}%
\pgfpathlineto{\pgfqpoint{5.001457in}{8.043701in}}%
\pgfpathlineto{\pgfqpoint{5.021206in}{8.070660in}}%
\pgfpathlineto{\pgfqpoint{5.040954in}{8.092585in}}%
\pgfpathlineto{\pgfqpoint{5.060703in}{8.111615in}}%
\pgfpathlineto{\pgfqpoint{5.080451in}{8.129150in}}%
\pgfpathlineto{\pgfqpoint{5.100200in}{8.144545in}}%
\pgfpathlineto{\pgfqpoint{5.119948in}{8.156513in}}%
\pgfpathlineto{\pgfqpoint{5.139697in}{8.165800in}}%
\pgfpathlineto{\pgfqpoint{5.159445in}{8.175295in}}%
\pgfpathlineto{\pgfqpoint{5.179194in}{8.187024in}}%
\pgfpathlineto{\pgfqpoint{5.198942in}{8.199328in}}%
\pgfpathlineto{\pgfqpoint{5.218691in}{8.208629in}}%
\pgfpathlineto{\pgfqpoint{5.238439in}{8.212222in}}%
\pgfpathlineto{\pgfqpoint{5.258188in}{8.209348in}}%
\pgfpathlineto{\pgfqpoint{5.277936in}{8.201643in}}%
\pgfpathlineto{\pgfqpoint{5.297685in}{8.191727in}}%
\pgfpathlineto{\pgfqpoint{5.317433in}{8.181903in}}%
\pgfpathlineto{\pgfqpoint{5.337182in}{8.174623in}}%
\pgfpathlineto{\pgfqpoint{5.356930in}{8.172272in}}%
\pgfpathlineto{\pgfqpoint{5.376679in}{8.174286in}}%
\pgfpathlineto{\pgfqpoint{5.396427in}{8.178160in}}%
\pgfpathlineto{\pgfqpoint{5.416176in}{8.182498in}}%
\pgfpathlineto{\pgfqpoint{5.435924in}{8.188892in}}%
\pgfpathlineto{\pgfqpoint{5.455673in}{8.201212in}}%
\pgfpathlineto{\pgfqpoint{5.475421in}{8.222950in}}%
\pgfpathlineto{\pgfqpoint{5.495170in}{8.254010in}}%
\pgfpathlineto{\pgfqpoint{5.514918in}{8.289957in}}%
\pgfpathlineto{\pgfqpoint{5.534667in}{8.324646in}}%
\pgfpathlineto{\pgfqpoint{5.554415in}{8.352822in}}%
\pgfpathlineto{\pgfqpoint{5.574164in}{8.372162in}}%
\pgfpathlineto{\pgfqpoint{5.593912in}{8.382203in}}%
\pgfpathlineto{\pgfqpoint{5.613661in}{8.382148in}}%
\pgfpathlineto{\pgfqpoint{5.633409in}{8.369691in}}%
\pgfpathlineto{\pgfqpoint{5.653158in}{8.341716in}}%
\pgfpathlineto{\pgfqpoint{5.672906in}{8.295683in}}%
\pgfpathlineto{\pgfqpoint{5.692655in}{8.230707in}}%
\pgfpathlineto{\pgfqpoint{5.712403in}{8.149436in}}%
\pgfpathlineto{\pgfqpoint{5.732152in}{8.057868in}}%
\pgfpathlineto{\pgfqpoint{5.751900in}{7.961700in}}%
\pgfpathlineto{\pgfqpoint{5.771649in}{7.862369in}}%
\pgfpathlineto{\pgfqpoint{5.791397in}{7.759388in}}%
\pgfpathlineto{\pgfqpoint{5.811146in}{7.652730in}}%
\pgfpathlineto{\pgfqpoint{5.830894in}{7.543065in}}%
\pgfpathlineto{\pgfqpoint{5.850643in}{7.432200in}}%
\pgfpathlineto{\pgfqpoint{5.870391in}{7.325490in}}%
\pgfpathlineto{\pgfqpoint{5.890140in}{7.229618in}}%
\pgfpathlineto{\pgfqpoint{5.909888in}{7.144169in}}%
\pgfusepath{stroke}%
\end{pgfscope}%
\begin{pgfscope}%
\pgfpathrectangle{\pgfqpoint{3.825278in}{6.184745in}}{\pgfqpoint{2.414722in}{2.357859in}}%
\pgfusepath{clip}%
\pgfsetroundcap%
\pgfsetroundjoin%
\pgfsetlinewidth{1.505625pt}%
\definecolor{currentstroke}{rgb}{0.505882,0.447059,0.701961}%
\pgfsetstrokecolor{currentstroke}%
\pgfsetdash{}{0pt}%
\pgfpathmoveto{\pgfqpoint{3.935038in}{7.860150in}}%
\pgfpathlineto{\pgfqpoint{3.955758in}{8.004046in}}%
\pgfpathlineto{\pgfqpoint{3.976478in}{8.142676in}}%
\pgfpathlineto{\pgfqpoint{3.997198in}{8.260877in}}%
\pgfpathlineto{\pgfqpoint{4.017917in}{8.341613in}}%
\pgfpathlineto{\pgfqpoint{4.038637in}{8.379869in}}%
\pgfpathlineto{\pgfqpoint{4.059357in}{8.384325in}}%
\pgfpathlineto{\pgfqpoint{4.080076in}{8.369691in}}%
\pgfpathlineto{\pgfqpoint{4.100796in}{8.349514in}}%
\pgfpathlineto{\pgfqpoint{4.121516in}{8.331136in}}%
\pgfpathlineto{\pgfqpoint{4.142236in}{8.311919in}}%
\pgfpathlineto{\pgfqpoint{4.162955in}{8.279957in}}%
\pgfpathlineto{\pgfqpoint{4.183675in}{8.224419in}}%
\pgfpathlineto{\pgfqpoint{4.204395in}{8.143623in}}%
\pgfpathlineto{\pgfqpoint{4.225115in}{8.044080in}}%
\pgfpathlineto{\pgfqpoint{4.245834in}{7.934895in}}%
\pgfpathlineto{\pgfqpoint{4.266554in}{7.823699in}}%
\pgfpathlineto{\pgfqpoint{4.287274in}{7.716348in}}%
\pgfpathlineto{\pgfqpoint{4.307994in}{7.616118in}}%
\pgfpathlineto{\pgfqpoint{4.328713in}{7.523205in}}%
\pgfpathlineto{\pgfqpoint{4.349433in}{7.436205in}}%
\pgfpathlineto{\pgfqpoint{4.370153in}{7.352414in}}%
\pgfpathlineto{\pgfqpoint{4.390873in}{7.271036in}}%
\pgfpathlineto{\pgfqpoint{4.411592in}{7.195756in}}%
\pgfpathlineto{\pgfqpoint{4.432312in}{7.127916in}}%
\pgfpathlineto{\pgfqpoint{4.453032in}{7.069014in}}%
\pgfpathlineto{\pgfqpoint{4.473751in}{7.021449in}}%
\pgfpathlineto{\pgfqpoint{4.494471in}{6.985908in}}%
\pgfpathlineto{\pgfqpoint{4.515191in}{6.959517in}}%
\pgfpathlineto{\pgfqpoint{4.535911in}{6.937199in}}%
\pgfpathlineto{\pgfqpoint{4.556630in}{6.915679in}}%
\pgfpathlineto{\pgfqpoint{4.577350in}{6.895272in}}%
\pgfpathlineto{\pgfqpoint{4.598070in}{6.878656in}}%
\pgfpathlineto{\pgfqpoint{4.618790in}{6.872690in}}%
\pgfpathlineto{\pgfqpoint{4.639509in}{6.888419in}}%
\pgfpathlineto{\pgfqpoint{4.660229in}{6.938355in}}%
\pgfpathlineto{\pgfqpoint{4.680949in}{7.029716in}}%
\pgfpathlineto{\pgfqpoint{4.701669in}{7.157101in}}%
\pgfpathlineto{\pgfqpoint{4.722388in}{7.304040in}}%
\pgfpathlineto{\pgfqpoint{4.743108in}{7.450970in}}%
\pgfpathlineto{\pgfqpoint{4.763828in}{7.582646in}}%
\pgfpathlineto{\pgfqpoint{4.784548in}{7.689307in}}%
\pgfpathlineto{\pgfqpoint{4.805267in}{7.764756in}}%
\pgfpathlineto{\pgfqpoint{4.825987in}{7.807850in}}%
\pgfpathlineto{\pgfqpoint{4.846707in}{7.823032in}}%
\pgfpathlineto{\pgfqpoint{4.867426in}{7.817970in}}%
\pgfpathlineto{\pgfqpoint{4.888146in}{7.801618in}}%
\pgfpathlineto{\pgfqpoint{4.908866in}{7.782142in}}%
\pgfpathlineto{\pgfqpoint{4.929586in}{7.765361in}}%
\pgfpathlineto{\pgfqpoint{4.950305in}{7.754261in}}%
\pgfpathlineto{\pgfqpoint{4.971025in}{7.750986in}}%
\pgfpathlineto{\pgfqpoint{4.991745in}{7.760991in}}%
\pgfpathlineto{\pgfqpoint{5.012465in}{7.795401in}}%
\pgfpathlineto{\pgfqpoint{5.033184in}{7.865280in}}%
\pgfpathlineto{\pgfqpoint{5.053904in}{7.971098in}}%
\pgfpathlineto{\pgfqpoint{5.074624in}{8.099382in}}%
\pgfpathlineto{\pgfqpoint{5.095344in}{8.227409in}}%
\pgfpathlineto{\pgfqpoint{5.116063in}{8.330975in}}%
\pgfpathlineto{\pgfqpoint{5.136783in}{8.393845in}}%
\pgfpathlineto{\pgfqpoint{5.157503in}{8.413752in}}%
\pgfpathlineto{\pgfqpoint{5.178223in}{8.400687in}}%
\pgfpathlineto{\pgfqpoint{5.198942in}{8.369691in}}%
\pgfpathlineto{\pgfqpoint{5.219662in}{8.333751in}}%
\pgfpathlineto{\pgfqpoint{5.240382in}{8.298974in}}%
\pgfpathlineto{\pgfqpoint{5.261101in}{8.262846in}}%
\pgfpathlineto{\pgfqpoint{5.281821in}{8.216805in}}%
\pgfpathlineto{\pgfqpoint{5.302541in}{8.154059in}}%
\pgfpathlineto{\pgfqpoint{5.323261in}{8.077817in}}%
\pgfusepath{stroke}%
\end{pgfscope}%
\begin{pgfscope}%
\pgfsetrectcap%
\pgfsetmiterjoin%
\pgfsetlinewidth{1.254687pt}%
\definecolor{currentstroke}{rgb}{1.000000,1.000000,1.000000}%
\pgfsetstrokecolor{currentstroke}%
\pgfsetdash{}{0pt}%
\pgfpathmoveto{\pgfqpoint{3.825278in}{6.184745in}}%
\pgfpathlineto{\pgfqpoint{3.825278in}{8.542604in}}%
\pgfusepath{stroke}%
\end{pgfscope}%
\begin{pgfscope}%
\pgfsetrectcap%
\pgfsetmiterjoin%
\pgfsetlinewidth{1.254687pt}%
\definecolor{currentstroke}{rgb}{1.000000,1.000000,1.000000}%
\pgfsetstrokecolor{currentstroke}%
\pgfsetdash{}{0pt}%
\pgfpathmoveto{\pgfqpoint{6.240000in}{6.184745in}}%
\pgfpathlineto{\pgfqpoint{6.240000in}{8.542604in}}%
\pgfusepath{stroke}%
\end{pgfscope}%
\begin{pgfscope}%
\pgfsetrectcap%
\pgfsetmiterjoin%
\pgfsetlinewidth{1.254687pt}%
\definecolor{currentstroke}{rgb}{1.000000,1.000000,1.000000}%
\pgfsetstrokecolor{currentstroke}%
\pgfsetdash{}{0pt}%
\pgfpathmoveto{\pgfqpoint{3.825278in}{6.184745in}}%
\pgfpathlineto{\pgfqpoint{6.240000in}{6.184745in}}%
\pgfusepath{stroke}%
\end{pgfscope}%
\begin{pgfscope}%
\pgfsetrectcap%
\pgfsetmiterjoin%
\pgfsetlinewidth{1.254687pt}%
\definecolor{currentstroke}{rgb}{1.000000,1.000000,1.000000}%
\pgfsetstrokecolor{currentstroke}%
\pgfsetdash{}{0pt}%
\pgfpathmoveto{\pgfqpoint{3.825278in}{8.542604in}}%
\pgfpathlineto{\pgfqpoint{6.240000in}{8.542604in}}%
\pgfusepath{stroke}%
\end{pgfscope}%
\begin{pgfscope}%
\definecolor{textcolor}{rgb}{0.150000,0.150000,0.150000}%
\pgfsetstrokecolor{textcolor}%
\pgfsetfillcolor{textcolor}%
\pgftext[x=5.032639in,y=8.625938in,,base]{\color{textcolor}\sffamily\fontsize{11.000000}{13.200000}\selectfont (b) Cluster 2 members}%
\end{pgfscope}%
\begin{pgfscope}%
\pgfsetbuttcap%
\pgfsetmiterjoin%
\definecolor{currentfill}{rgb}{0.917647,0.917647,0.949020}%
\pgfsetfillcolor{currentfill}%
\pgfsetlinewidth{0.000000pt}%
\definecolor{currentstroke}{rgb}{0.000000,0.000000,0.000000}%
\pgfsetstrokecolor{currentstroke}%
\pgfsetstrokeopacity{0.000000}%
\pgfsetdash{}{0pt}%
\pgfpathmoveto{\pgfqpoint{0.828241in}{3.379757in}}%
\pgfpathlineto{\pgfqpoint{3.242963in}{3.379757in}}%
\pgfpathlineto{\pgfqpoint{3.242963in}{5.737616in}}%
\pgfpathlineto{\pgfqpoint{0.828241in}{5.737616in}}%
\pgfpathclose%
\pgfusepath{fill}%
\end{pgfscope}%
\begin{pgfscope}%
\pgfpathrectangle{\pgfqpoint{0.828241in}{3.379757in}}{\pgfqpoint{2.414722in}{2.357859in}}%
\pgfusepath{clip}%
\pgfsetroundcap%
\pgfsetroundjoin%
\pgfsetlinewidth{1.003750pt}%
\definecolor{currentstroke}{rgb}{1.000000,1.000000,1.000000}%
\pgfsetstrokecolor{currentstroke}%
\pgfsetdash{}{0pt}%
\pgfpathmoveto{\pgfqpoint{0.938001in}{3.379757in}}%
\pgfpathlineto{\pgfqpoint{0.938001in}{5.737616in}}%
\pgfusepath{stroke}%
\end{pgfscope}%
\begin{pgfscope}%
\definecolor{textcolor}{rgb}{0.150000,0.150000,0.150000}%
\pgfsetstrokecolor{textcolor}%
\pgfsetfillcolor{textcolor}%
\pgftext[x=0.938001in,y=3.247812in,,top]{\color{textcolor}\sffamily\fontsize{11.000000}{13.200000}\selectfont \(\displaystyle 0.0\)}%
\end{pgfscope}%
\begin{pgfscope}%
\pgfpathrectangle{\pgfqpoint{0.828241in}{3.379757in}}{\pgfqpoint{2.414722in}{2.357859in}}%
\pgfusepath{clip}%
\pgfsetroundcap%
\pgfsetroundjoin%
\pgfsetlinewidth{1.003750pt}%
\definecolor{currentstroke}{rgb}{1.000000,1.000000,1.000000}%
\pgfsetstrokecolor{currentstroke}%
\pgfsetdash{}{0pt}%
\pgfpathmoveto{\pgfqpoint{1.787335in}{3.379757in}}%
\pgfpathlineto{\pgfqpoint{1.787335in}{5.737616in}}%
\pgfusepath{stroke}%
\end{pgfscope}%
\begin{pgfscope}%
\definecolor{textcolor}{rgb}{0.150000,0.150000,0.150000}%
\pgfsetstrokecolor{textcolor}%
\pgfsetfillcolor{textcolor}%
\pgftext[x=1.787335in,y=3.247812in,,top]{\color{textcolor}\sffamily\fontsize{11.000000}{13.200000}\selectfont \(\displaystyle 0.5\)}%
\end{pgfscope}%
\begin{pgfscope}%
\pgfpathrectangle{\pgfqpoint{0.828241in}{3.379757in}}{\pgfqpoint{2.414722in}{2.357859in}}%
\pgfusepath{clip}%
\pgfsetroundcap%
\pgfsetroundjoin%
\pgfsetlinewidth{1.003750pt}%
\definecolor{currentstroke}{rgb}{1.000000,1.000000,1.000000}%
\pgfsetstrokecolor{currentstroke}%
\pgfsetdash{}{0pt}%
\pgfpathmoveto{\pgfqpoint{2.636669in}{3.379757in}}%
\pgfpathlineto{\pgfqpoint{2.636669in}{5.737616in}}%
\pgfusepath{stroke}%
\end{pgfscope}%
\begin{pgfscope}%
\definecolor{textcolor}{rgb}{0.150000,0.150000,0.150000}%
\pgfsetstrokecolor{textcolor}%
\pgfsetfillcolor{textcolor}%
\pgftext[x=2.636669in,y=3.247812in,,top]{\color{textcolor}\sffamily\fontsize{11.000000}{13.200000}\selectfont \(\displaystyle 1.0\)}%
\end{pgfscope}%
\begin{pgfscope}%
\pgfpathrectangle{\pgfqpoint{0.828241in}{3.379757in}}{\pgfqpoint{2.414722in}{2.357859in}}%
\pgfusepath{clip}%
\pgfsetroundcap%
\pgfsetroundjoin%
\pgfsetlinewidth{1.003750pt}%
\definecolor{currentstroke}{rgb}{1.000000,1.000000,1.000000}%
\pgfsetstrokecolor{currentstroke}%
\pgfsetdash{}{0pt}%
\pgfpathmoveto{\pgfqpoint{0.828241in}{3.603407in}}%
\pgfpathlineto{\pgfqpoint{3.242963in}{3.603407in}}%
\pgfusepath{stroke}%
\end{pgfscope}%
\begin{pgfscope}%
\definecolor{textcolor}{rgb}{0.150000,0.150000,0.150000}%
\pgfsetstrokecolor{textcolor}%
\pgfsetfillcolor{textcolor}%
\pgftext[x=0.425926in,y=3.550601in,left,base]{\color{textcolor}\sffamily\fontsize{11.000000}{13.200000}\selectfont \(\displaystyle -10\)}%
\end{pgfscope}%
\begin{pgfscope}%
\pgfpathrectangle{\pgfqpoint{0.828241in}{3.379757in}}{\pgfqpoint{2.414722in}{2.357859in}}%
\pgfusepath{clip}%
\pgfsetroundcap%
\pgfsetroundjoin%
\pgfsetlinewidth{1.003750pt}%
\definecolor{currentstroke}{rgb}{1.000000,1.000000,1.000000}%
\pgfsetstrokecolor{currentstroke}%
\pgfsetdash{}{0pt}%
\pgfpathmoveto{\pgfqpoint{0.828241in}{3.995976in}}%
\pgfpathlineto{\pgfqpoint{3.242963in}{3.995976in}}%
\pgfusepath{stroke}%
\end{pgfscope}%
\begin{pgfscope}%
\definecolor{textcolor}{rgb}{0.150000,0.150000,0.150000}%
\pgfsetstrokecolor{textcolor}%
\pgfsetfillcolor{textcolor}%
\pgftext[x=0.501968in,y=3.943169in,left,base]{\color{textcolor}\sffamily\fontsize{11.000000}{13.200000}\selectfont \(\displaystyle -8\)}%
\end{pgfscope}%
\begin{pgfscope}%
\pgfpathrectangle{\pgfqpoint{0.828241in}{3.379757in}}{\pgfqpoint{2.414722in}{2.357859in}}%
\pgfusepath{clip}%
\pgfsetroundcap%
\pgfsetroundjoin%
\pgfsetlinewidth{1.003750pt}%
\definecolor{currentstroke}{rgb}{1.000000,1.000000,1.000000}%
\pgfsetstrokecolor{currentstroke}%
\pgfsetdash{}{0pt}%
\pgfpathmoveto{\pgfqpoint{0.828241in}{4.388545in}}%
\pgfpathlineto{\pgfqpoint{3.242963in}{4.388545in}}%
\pgfusepath{stroke}%
\end{pgfscope}%
\begin{pgfscope}%
\definecolor{textcolor}{rgb}{0.150000,0.150000,0.150000}%
\pgfsetstrokecolor{textcolor}%
\pgfsetfillcolor{textcolor}%
\pgftext[x=0.501968in,y=4.335738in,left,base]{\color{textcolor}\sffamily\fontsize{11.000000}{13.200000}\selectfont \(\displaystyle -6\)}%
\end{pgfscope}%
\begin{pgfscope}%
\pgfpathrectangle{\pgfqpoint{0.828241in}{3.379757in}}{\pgfqpoint{2.414722in}{2.357859in}}%
\pgfusepath{clip}%
\pgfsetroundcap%
\pgfsetroundjoin%
\pgfsetlinewidth{1.003750pt}%
\definecolor{currentstroke}{rgb}{1.000000,1.000000,1.000000}%
\pgfsetstrokecolor{currentstroke}%
\pgfsetdash{}{0pt}%
\pgfpathmoveto{\pgfqpoint{0.828241in}{4.781114in}}%
\pgfpathlineto{\pgfqpoint{3.242963in}{4.781114in}}%
\pgfusepath{stroke}%
\end{pgfscope}%
\begin{pgfscope}%
\definecolor{textcolor}{rgb}{0.150000,0.150000,0.150000}%
\pgfsetstrokecolor{textcolor}%
\pgfsetfillcolor{textcolor}%
\pgftext[x=0.501968in,y=4.728307in,left,base]{\color{textcolor}\sffamily\fontsize{11.000000}{13.200000}\selectfont \(\displaystyle -4\)}%
\end{pgfscope}%
\begin{pgfscope}%
\pgfpathrectangle{\pgfqpoint{0.828241in}{3.379757in}}{\pgfqpoint{2.414722in}{2.357859in}}%
\pgfusepath{clip}%
\pgfsetroundcap%
\pgfsetroundjoin%
\pgfsetlinewidth{1.003750pt}%
\definecolor{currentstroke}{rgb}{1.000000,1.000000,1.000000}%
\pgfsetstrokecolor{currentstroke}%
\pgfsetdash{}{0pt}%
\pgfpathmoveto{\pgfqpoint{0.828241in}{5.173682in}}%
\pgfpathlineto{\pgfqpoint{3.242963in}{5.173682in}}%
\pgfusepath{stroke}%
\end{pgfscope}%
\begin{pgfscope}%
\definecolor{textcolor}{rgb}{0.150000,0.150000,0.150000}%
\pgfsetstrokecolor{textcolor}%
\pgfsetfillcolor{textcolor}%
\pgftext[x=0.501968in,y=5.120876in,left,base]{\color{textcolor}\sffamily\fontsize{11.000000}{13.200000}\selectfont \(\displaystyle -2\)}%
\end{pgfscope}%
\begin{pgfscope}%
\pgfpathrectangle{\pgfqpoint{0.828241in}{3.379757in}}{\pgfqpoint{2.414722in}{2.357859in}}%
\pgfusepath{clip}%
\pgfsetroundcap%
\pgfsetroundjoin%
\pgfsetlinewidth{1.003750pt}%
\definecolor{currentstroke}{rgb}{1.000000,1.000000,1.000000}%
\pgfsetstrokecolor{currentstroke}%
\pgfsetdash{}{0pt}%
\pgfpathmoveto{\pgfqpoint{0.828241in}{5.566251in}}%
\pgfpathlineto{\pgfqpoint{3.242963in}{5.566251in}}%
\pgfusepath{stroke}%
\end{pgfscope}%
\begin{pgfscope}%
\definecolor{textcolor}{rgb}{0.150000,0.150000,0.150000}%
\pgfsetstrokecolor{textcolor}%
\pgfsetfillcolor{textcolor}%
\pgftext[x=0.620255in,y=5.513445in,left,base]{\color{textcolor}\sffamily\fontsize{11.000000}{13.200000}\selectfont \(\displaystyle 0\)}%
\end{pgfscope}%
\begin{pgfscope}%
\definecolor{textcolor}{rgb}{0.150000,0.150000,0.150000}%
\pgfsetstrokecolor{textcolor}%
\pgfsetfillcolor{textcolor}%
\pgftext[x=0.370370in,y=4.558686in,,bottom,rotate=90.000000]{\color{textcolor}\sffamily\fontsize{11.000000}{13.200000}\selectfont 2CH/gls}%
\end{pgfscope}%
\begin{pgfscope}%
\pgfpathrectangle{\pgfqpoint{0.828241in}{3.379757in}}{\pgfqpoint{2.414722in}{2.357859in}}%
\pgfusepath{clip}%
\pgfsetroundcap%
\pgfsetroundjoin%
\pgfsetlinewidth{1.505625pt}%
\definecolor{currentstroke}{rgb}{0.298039,0.447059,0.690196}%
\pgfsetstrokecolor{currentstroke}%
\pgfsetdash{}{0pt}%
\pgfpathmoveto{\pgfqpoint{0.938001in}{4.326237in}}%
\pgfpathlineto{\pgfqpoint{0.965848in}{4.369539in}}%
\pgfpathlineto{\pgfqpoint{0.993695in}{4.451137in}}%
\pgfpathlineto{\pgfqpoint{1.021542in}{4.605988in}}%
\pgfpathlineto{\pgfqpoint{1.049389in}{4.843846in}}%
\pgfpathlineto{\pgfqpoint{1.077236in}{5.126689in}}%
\pgfpathlineto{\pgfqpoint{1.105083in}{5.381522in}}%
\pgfpathlineto{\pgfqpoint{1.132930in}{5.548284in}}%
\pgfpathlineto{\pgfqpoint{1.160777in}{5.614621in}}%
\pgfpathlineto{\pgfqpoint{1.188624in}{5.607360in}}%
\pgfpathlineto{\pgfqpoint{1.216471in}{5.566251in}}%
\pgfpathlineto{\pgfqpoint{1.244318in}{5.525192in}}%
\pgfpathlineto{\pgfqpoint{1.272166in}{5.503824in}}%
\pgfpathlineto{\pgfqpoint{1.300013in}{5.501200in}}%
\pgfpathlineto{\pgfqpoint{1.327860in}{5.494199in}}%
\pgfpathlineto{\pgfqpoint{1.355707in}{5.451112in}}%
\pgfpathlineto{\pgfqpoint{1.383554in}{5.352215in}}%
\pgfpathlineto{\pgfqpoint{1.411401in}{5.198709in}}%
\pgfpathlineto{\pgfqpoint{1.439248in}{5.007689in}}%
\pgfpathlineto{\pgfqpoint{1.467095in}{4.802632in}}%
\pgfpathlineto{\pgfqpoint{1.494942in}{4.606888in}}%
\pgfpathlineto{\pgfqpoint{1.522789in}{4.435897in}}%
\pgfpathlineto{\pgfqpoint{1.550636in}{4.289741in}}%
\pgfpathlineto{\pgfqpoint{1.578483in}{4.160036in}}%
\pgfpathlineto{\pgfqpoint{1.606330in}{4.038761in}}%
\pgfpathlineto{\pgfqpoint{1.634177in}{3.921869in}}%
\pgfpathlineto{\pgfqpoint{1.662024in}{3.811252in}}%
\pgfpathlineto{\pgfqpoint{1.689871in}{3.712418in}}%
\pgfpathlineto{\pgfqpoint{1.717718in}{3.631457in}}%
\pgfpathlineto{\pgfqpoint{1.745565in}{3.573325in}}%
\pgfpathlineto{\pgfqpoint{1.773412in}{3.540357in}}%
\pgfpathlineto{\pgfqpoint{1.801259in}{3.530694in}}%
\pgfpathlineto{\pgfqpoint{1.829106in}{3.536296in}}%
\pgfpathlineto{\pgfqpoint{1.856953in}{3.543183in}}%
\pgfpathlineto{\pgfqpoint{1.884800in}{3.538107in}}%
\pgfpathlineto{\pgfqpoint{1.912647in}{3.519111in}}%
\pgfpathlineto{\pgfqpoint{1.940494in}{3.496701in}}%
\pgfpathlineto{\pgfqpoint{1.968341in}{3.486932in}}%
\pgfpathlineto{\pgfqpoint{1.996188in}{3.504659in}}%
\pgfpathlineto{\pgfqpoint{2.024035in}{3.557932in}}%
\pgfpathlineto{\pgfqpoint{2.051882in}{3.645250in}}%
\pgfpathlineto{\pgfqpoint{2.079729in}{3.756442in}}%
\pgfpathlineto{\pgfqpoint{2.107576in}{3.876526in}}%
\pgfpathlineto{\pgfqpoint{2.135423in}{3.991261in}}%
\pgfpathlineto{\pgfqpoint{2.163270in}{4.091667in}}%
\pgfpathlineto{\pgfqpoint{2.191117in}{4.174050in}}%
\pgfpathlineto{\pgfqpoint{2.218964in}{4.237206in}}%
\pgfpathlineto{\pgfqpoint{2.246811in}{4.280091in}}%
\pgfpathlineto{\pgfqpoint{2.274658in}{4.302571in}}%
\pgfpathlineto{\pgfqpoint{2.302505in}{4.306580in}}%
\pgfpathlineto{\pgfqpoint{2.330352in}{4.300049in}}%
\pgfpathlineto{\pgfqpoint{2.358199in}{4.302329in}}%
\pgfpathlineto{\pgfqpoint{2.386046in}{4.345656in}}%
\pgfpathlineto{\pgfqpoint{2.413893in}{4.465567in}}%
\pgfpathlineto{\pgfqpoint{2.441740in}{4.681371in}}%
\pgfpathlineto{\pgfqpoint{2.469587in}{4.973792in}}%
\pgfpathlineto{\pgfqpoint{2.497434in}{5.274464in}}%
\pgfpathlineto{\pgfqpoint{2.525281in}{5.500132in}}%
\pgfpathlineto{\pgfqpoint{2.553128in}{5.608489in}}%
\pgfpathlineto{\pgfqpoint{2.580975in}{5.614928in}}%
\pgfpathlineto{\pgfqpoint{2.608822in}{5.566251in}}%
\pgfpathlineto{\pgfqpoint{2.636669in}{5.507284in}}%
\pgfpathlineto{\pgfqpoint{2.664516in}{5.464301in}}%
\pgfpathlineto{\pgfqpoint{2.692363in}{5.442439in}}%
\pgfpathlineto{\pgfqpoint{2.720210in}{5.424445in}}%
\pgfpathlineto{\pgfqpoint{2.748057in}{5.379901in}}%
\pgfpathlineto{\pgfqpoint{2.775904in}{5.290642in}}%
\pgfpathlineto{\pgfqpoint{2.803751in}{5.161881in}}%
\pgfpathlineto{\pgfqpoint{2.831598in}{5.008032in}}%
\pgfpathlineto{\pgfqpoint{2.859445in}{4.840497in}}%
\pgfpathlineto{\pgfqpoint{2.887292in}{4.666988in}}%
\pgfusepath{stroke}%
\end{pgfscope}%
\begin{pgfscope}%
\pgfpathrectangle{\pgfqpoint{0.828241in}{3.379757in}}{\pgfqpoint{2.414722in}{2.357859in}}%
\pgfusepath{clip}%
\pgfsetroundcap%
\pgfsetroundjoin%
\pgfsetlinewidth{1.505625pt}%
\definecolor{currentstroke}{rgb}{0.866667,0.517647,0.321569}%
\pgfsetstrokecolor{currentstroke}%
\pgfsetdash{}{0pt}%
\pgfpathmoveto{\pgfqpoint{0.938001in}{5.132251in}}%
\pgfpathlineto{\pgfqpoint{0.965848in}{5.197168in}}%
\pgfpathlineto{\pgfqpoint{0.993695in}{5.271548in}}%
\pgfpathlineto{\pgfqpoint{1.021542in}{5.355872in}}%
\pgfpathlineto{\pgfqpoint{1.049389in}{5.439626in}}%
\pgfpathlineto{\pgfqpoint{1.077236in}{5.510358in}}%
\pgfpathlineto{\pgfqpoint{1.105083in}{5.561908in}}%
\pgfpathlineto{\pgfqpoint{1.132930in}{5.592332in}}%
\pgfpathlineto{\pgfqpoint{1.160777in}{5.601076in}}%
\pgfpathlineto{\pgfqpoint{1.188624in}{5.589893in}}%
\pgfpathlineto{\pgfqpoint{1.216471in}{5.566251in}}%
\pgfpathlineto{\pgfqpoint{1.244318in}{5.542583in}}%
\pgfpathlineto{\pgfqpoint{1.272166in}{5.529712in}}%
\pgfpathlineto{\pgfqpoint{1.300013in}{5.530076in}}%
\pgfpathlineto{\pgfqpoint{1.327860in}{5.536651in}}%
\pgfpathlineto{\pgfqpoint{1.355707in}{5.538147in}}%
\pgfpathlineto{\pgfqpoint{1.383554in}{5.527219in}}%
\pgfpathlineto{\pgfqpoint{1.411401in}{5.502925in}}%
\pgfpathlineto{\pgfqpoint{1.439248in}{5.465686in}}%
\pgfpathlineto{\pgfqpoint{1.467095in}{5.412208in}}%
\pgfpathlineto{\pgfqpoint{1.494942in}{5.340694in}}%
\pgfpathlineto{\pgfqpoint{1.522789in}{5.257088in}}%
\pgfpathlineto{\pgfqpoint{1.550636in}{5.171816in}}%
\pgfpathlineto{\pgfqpoint{1.578483in}{5.090947in}}%
\pgfpathlineto{\pgfqpoint{1.606330in}{5.013363in}}%
\pgfpathlineto{\pgfqpoint{1.634177in}{4.934970in}}%
\pgfpathlineto{\pgfqpoint{1.662024in}{4.856020in}}%
\pgfpathlineto{\pgfqpoint{1.689871in}{4.783348in}}%
\pgfpathlineto{\pgfqpoint{1.717718in}{4.727410in}}%
\pgfpathlineto{\pgfqpoint{1.745565in}{4.695639in}}%
\pgfpathlineto{\pgfqpoint{1.773412in}{4.687081in}}%
\pgfpathlineto{\pgfqpoint{1.801259in}{4.691856in}}%
\pgfpathlineto{\pgfqpoint{1.829106in}{4.694085in}}%
\pgfpathlineto{\pgfqpoint{1.856953in}{4.679525in}}%
\pgfpathlineto{\pgfqpoint{1.884800in}{4.642133in}}%
\pgfpathlineto{\pgfqpoint{1.912647in}{4.588097in}}%
\pgfpathlineto{\pgfqpoint{1.940494in}{4.534818in}}%
\pgfpathlineto{\pgfqpoint{1.968341in}{4.507534in}}%
\pgfpathlineto{\pgfqpoint{1.996188in}{4.528946in}}%
\pgfpathlineto{\pgfqpoint{2.024035in}{4.606300in}}%
\pgfpathlineto{\pgfqpoint{2.051882in}{4.724379in}}%
\pgfpathlineto{\pgfqpoint{2.079729in}{4.850236in}}%
\pgfpathlineto{\pgfqpoint{2.107576in}{4.952317in}}%
\pgfpathlineto{\pgfqpoint{2.135423in}{5.018653in}}%
\pgfpathlineto{\pgfqpoint{2.163270in}{5.056147in}}%
\pgfpathlineto{\pgfqpoint{2.191117in}{5.078440in}}%
\pgfpathlineto{\pgfqpoint{2.218964in}{5.097727in}}%
\pgfpathlineto{\pgfqpoint{2.246811in}{5.120448in}}%
\pgfpathlineto{\pgfqpoint{2.274658in}{5.145971in}}%
\pgfpathlineto{\pgfqpoint{2.302505in}{5.169823in}}%
\pgfpathlineto{\pgfqpoint{2.330352in}{5.189998in}}%
\pgfpathlineto{\pgfqpoint{2.358199in}{5.210630in}}%
\pgfpathlineto{\pgfqpoint{2.386046in}{5.241262in}}%
\pgfpathlineto{\pgfqpoint{2.413893in}{5.292074in}}%
\pgfpathlineto{\pgfqpoint{2.441740in}{5.365659in}}%
\pgfpathlineto{\pgfqpoint{2.469587in}{5.451396in}}%
\pgfpathlineto{\pgfqpoint{2.497434in}{5.530578in}}%
\pgfpathlineto{\pgfqpoint{2.525281in}{5.588436in}}%
\pgfpathlineto{\pgfqpoint{2.553128in}{5.621002in}}%
\pgfpathlineto{\pgfqpoint{2.580975in}{5.630440in}}%
\pgfpathlineto{\pgfqpoint{2.608822in}{5.620153in}}%
\pgfpathlineto{\pgfqpoint{2.636669in}{5.595907in}}%
\pgfpathlineto{\pgfqpoint{2.664516in}{5.566251in}}%
\pgfpathlineto{\pgfqpoint{2.692363in}{5.539961in}}%
\pgfpathlineto{\pgfqpoint{2.720210in}{5.522548in}}%
\pgfpathlineto{\pgfqpoint{2.748057in}{5.513010in}}%
\pgfpathlineto{\pgfqpoint{2.775904in}{5.504944in}}%
\pgfpathlineto{\pgfqpoint{2.803751in}{5.490977in}}%
\pgfpathlineto{\pgfqpoint{2.831598in}{5.466470in}}%
\pgfpathlineto{\pgfqpoint{2.859445in}{5.429807in}}%
\pgfpathlineto{\pgfqpoint{2.887292in}{5.380275in}}%
\pgfpathlineto{\pgfqpoint{2.915139in}{5.319120in}}%
\pgfpathlineto{\pgfqpoint{2.942986in}{5.251013in}}%
\pgfusepath{stroke}%
\end{pgfscope}%
\begin{pgfscope}%
\pgfpathrectangle{\pgfqpoint{0.828241in}{3.379757in}}{\pgfqpoint{2.414722in}{2.357859in}}%
\pgfusepath{clip}%
\pgfsetroundcap%
\pgfsetroundjoin%
\pgfsetlinewidth{1.505625pt}%
\definecolor{currentstroke}{rgb}{0.333333,0.658824,0.407843}%
\pgfsetstrokecolor{currentstroke}%
\pgfsetdash{}{0pt}%
\pgfpathmoveto{\pgfqpoint{0.938001in}{4.116143in}}%
\pgfpathlineto{\pgfqpoint{0.964543in}{4.104850in}}%
\pgfpathlineto{\pgfqpoint{0.991085in}{4.133874in}}%
\pgfpathlineto{\pgfqpoint{1.017626in}{4.239177in}}%
\pgfpathlineto{\pgfqpoint{1.044168in}{4.444148in}}%
\pgfpathlineto{\pgfqpoint{1.070710in}{4.740144in}}%
\pgfpathlineto{\pgfqpoint{1.097251in}{5.065483in}}%
\pgfpathlineto{\pgfqpoint{1.123793in}{5.335218in}}%
\pgfpathlineto{\pgfqpoint{1.150335in}{5.498269in}}%
\pgfpathlineto{\pgfqpoint{1.176877in}{5.566251in}}%
\pgfpathlineto{\pgfqpoint{1.203418in}{5.579383in}}%
\pgfpathlineto{\pgfqpoint{1.229960in}{5.572161in}}%
\pgfpathlineto{\pgfqpoint{1.256502in}{5.556129in}}%
\pgfpathlineto{\pgfqpoint{1.283043in}{5.521996in}}%
\pgfpathlineto{\pgfqpoint{1.309585in}{5.453054in}}%
\pgfpathlineto{\pgfqpoint{1.336127in}{5.339167in}}%
\pgfpathlineto{\pgfqpoint{1.362668in}{5.187955in}}%
\pgfpathlineto{\pgfqpoint{1.389210in}{5.018942in}}%
\pgfpathlineto{\pgfqpoint{1.415752in}{4.850118in}}%
\pgfpathlineto{\pgfqpoint{1.442293in}{4.688961in}}%
\pgfpathlineto{\pgfqpoint{1.468835in}{4.535579in}}%
\pgfpathlineto{\pgfqpoint{1.495377in}{4.392793in}}%
\pgfpathlineto{\pgfqpoint{1.521918in}{4.259838in}}%
\pgfpathlineto{\pgfqpoint{1.548460in}{4.138501in}}%
\pgfpathlineto{\pgfqpoint{1.575002in}{4.037369in}}%
\pgfpathlineto{\pgfqpoint{1.601543in}{3.967844in}}%
\pgfpathlineto{\pgfqpoint{1.628085in}{3.932622in}}%
\pgfpathlineto{\pgfqpoint{1.654627in}{3.924473in}}%
\pgfpathlineto{\pgfqpoint{1.681169in}{3.933262in}}%
\pgfpathlineto{\pgfqpoint{1.707710in}{3.947238in}}%
\pgfpathlineto{\pgfqpoint{1.734252in}{3.956118in}}%
\pgfpathlineto{\pgfqpoint{1.760794in}{3.953645in}}%
\pgfpathlineto{\pgfqpoint{1.787335in}{3.940267in}}%
\pgfpathlineto{\pgfqpoint{1.813877in}{3.926910in}}%
\pgfpathlineto{\pgfqpoint{1.840419in}{3.925895in}}%
\pgfpathlineto{\pgfqpoint{1.866960in}{3.940871in}}%
\pgfpathlineto{\pgfqpoint{1.893502in}{3.974104in}}%
\pgfpathlineto{\pgfqpoint{1.920044in}{4.026953in}}%
\pgfpathlineto{\pgfqpoint{1.946585in}{4.099498in}}%
\pgfpathlineto{\pgfqpoint{1.973127in}{4.188583in}}%
\pgfpathlineto{\pgfqpoint{1.999669in}{4.286459in}}%
\pgfpathlineto{\pgfqpoint{2.026210in}{4.381519in}}%
\pgfpathlineto{\pgfqpoint{2.052752in}{4.463486in}}%
\pgfpathlineto{\pgfqpoint{2.079294in}{4.526652in}}%
\pgfpathlineto{\pgfqpoint{2.105836in}{4.569228in}}%
\pgfpathlineto{\pgfqpoint{2.132377in}{4.592066in}}%
\pgfpathlineto{\pgfqpoint{2.158919in}{4.598195in}}%
\pgfpathlineto{\pgfqpoint{2.185461in}{4.590633in}}%
\pgfpathlineto{\pgfqpoint{2.212002in}{4.579267in}}%
\pgfpathlineto{\pgfqpoint{2.238544in}{4.589568in}}%
\pgfpathlineto{\pgfqpoint{2.265086in}{4.652719in}}%
\pgfpathlineto{\pgfqpoint{2.291627in}{4.784208in}}%
\pgfpathlineto{\pgfqpoint{2.318169in}{4.972325in}}%
\pgfpathlineto{\pgfqpoint{2.344711in}{5.182503in}}%
\pgfpathlineto{\pgfqpoint{2.371252in}{5.369967in}}%
\pgfpathlineto{\pgfqpoint{2.397794in}{5.501093in}}%
\pgfpathlineto{\pgfqpoint{2.424336in}{5.566251in}}%
\pgfpathlineto{\pgfqpoint{2.450877in}{5.581018in}}%
\pgfpathlineto{\pgfqpoint{2.477419in}{5.571812in}}%
\pgfpathlineto{\pgfqpoint{2.503961in}{5.557039in}}%
\pgfusepath{stroke}%
\end{pgfscope}%
\begin{pgfscope}%
\pgfpathrectangle{\pgfqpoint{0.828241in}{3.379757in}}{\pgfqpoint{2.414722in}{2.357859in}}%
\pgfusepath{clip}%
\pgfsetroundcap%
\pgfsetroundjoin%
\pgfsetlinewidth{1.505625pt}%
\definecolor{currentstroke}{rgb}{0.768627,0.305882,0.321569}%
\pgfsetstrokecolor{currentstroke}%
\pgfsetdash{}{0pt}%
\pgfpathmoveto{\pgfqpoint{0.938001in}{4.978878in}}%
\pgfpathlineto{\pgfqpoint{0.964135in}{4.973810in}}%
\pgfpathlineto{\pgfqpoint{0.990268in}{4.982167in}}%
\pgfpathlineto{\pgfqpoint{1.016401in}{5.017661in}}%
\pgfpathlineto{\pgfqpoint{1.042535in}{5.089092in}}%
\pgfpathlineto{\pgfqpoint{1.068668in}{5.193192in}}%
\pgfpathlineto{\pgfqpoint{1.094801in}{5.313084in}}%
\pgfpathlineto{\pgfqpoint{1.120935in}{5.427014in}}%
\pgfpathlineto{\pgfqpoint{1.147068in}{5.517364in}}%
\pgfpathlineto{\pgfqpoint{1.173201in}{5.574369in}}%
\pgfpathlineto{\pgfqpoint{1.199335in}{5.597103in}}%
\pgfpathlineto{\pgfqpoint{1.225468in}{5.591215in}}%
\pgfpathlineto{\pgfqpoint{1.251602in}{5.566251in}}%
\pgfpathlineto{\pgfqpoint{1.277735in}{5.535784in}}%
\pgfpathlineto{\pgfqpoint{1.303868in}{5.516432in}}%
\pgfpathlineto{\pgfqpoint{1.330002in}{5.519463in}}%
\pgfpathlineto{\pgfqpoint{1.356135in}{5.540466in}}%
\pgfpathlineto{\pgfqpoint{1.382268in}{5.561099in}}%
\pgfpathlineto{\pgfqpoint{1.408402in}{5.563089in}}%
\pgfpathlineto{\pgfqpoint{1.434535in}{5.538543in}}%
\pgfpathlineto{\pgfqpoint{1.460668in}{5.489500in}}%
\pgfpathlineto{\pgfqpoint{1.486802in}{5.422321in}}%
\pgfpathlineto{\pgfqpoint{1.512935in}{5.344566in}}%
\pgfpathlineto{\pgfqpoint{1.539068in}{5.262422in}}%
\pgfpathlineto{\pgfqpoint{1.565202in}{5.178119in}}%
\pgfpathlineto{\pgfqpoint{1.591335in}{5.091850in}}%
\pgfpathlineto{\pgfqpoint{1.617468in}{5.005359in}}%
\pgfpathlineto{\pgfqpoint{1.643602in}{4.920545in}}%
\pgfpathlineto{\pgfqpoint{1.669735in}{4.836998in}}%
\pgfpathlineto{\pgfqpoint{1.695869in}{4.755394in}}%
\pgfpathlineto{\pgfqpoint{1.722002in}{4.680099in}}%
\pgfpathlineto{\pgfqpoint{1.748135in}{4.615389in}}%
\pgfpathlineto{\pgfqpoint{1.774269in}{4.565140in}}%
\pgfpathlineto{\pgfqpoint{1.800402in}{4.532810in}}%
\pgfpathlineto{\pgfqpoint{1.826535in}{4.520413in}}%
\pgfpathlineto{\pgfqpoint{1.852669in}{4.525624in}}%
\pgfpathlineto{\pgfqpoint{1.878802in}{4.539624in}}%
\pgfpathlineto{\pgfqpoint{1.904935in}{4.549474in}}%
\pgfpathlineto{\pgfqpoint{1.931069in}{4.539721in}}%
\pgfpathlineto{\pgfqpoint{1.957202in}{4.495975in}}%
\pgfpathlineto{\pgfqpoint{1.983335in}{4.414773in}}%
\pgfpathlineto{\pgfqpoint{2.009469in}{4.314328in}}%
\pgfpathlineto{\pgfqpoint{2.035602in}{4.230877in}}%
\pgfpathlineto{\pgfqpoint{2.061735in}{4.196762in}}%
\pgfpathlineto{\pgfqpoint{2.087869in}{4.224943in}}%
\pgfpathlineto{\pgfqpoint{2.114002in}{4.309675in}}%
\pgfpathlineto{\pgfqpoint{2.140136in}{4.434044in}}%
\pgfpathlineto{\pgfqpoint{2.166269in}{4.574879in}}%
\pgfpathlineto{\pgfqpoint{2.192402in}{4.705789in}}%
\pgfpathlineto{\pgfqpoint{2.218536in}{4.809416in}}%
\pgfpathlineto{\pgfqpoint{2.244669in}{4.885140in}}%
\pgfpathlineto{\pgfqpoint{2.270802in}{4.940834in}}%
\pgfpathlineto{\pgfqpoint{2.296936in}{4.984792in}}%
\pgfpathlineto{\pgfqpoint{2.323069in}{5.024726in}}%
\pgfpathlineto{\pgfqpoint{2.349202in}{5.066165in}}%
\pgfpathlineto{\pgfqpoint{2.375336in}{5.108123in}}%
\pgfpathlineto{\pgfqpoint{2.401469in}{5.145095in}}%
\pgfpathlineto{\pgfqpoint{2.427602in}{5.172677in}}%
\pgfpathlineto{\pgfqpoint{2.453736in}{5.189692in}}%
\pgfpathlineto{\pgfqpoint{2.479869in}{5.198874in}}%
\pgfpathlineto{\pgfqpoint{2.506002in}{5.205291in}}%
\pgfpathlineto{\pgfqpoint{2.532136in}{5.215016in}}%
\pgfpathlineto{\pgfqpoint{2.558269in}{5.233304in}}%
\pgfpathlineto{\pgfqpoint{2.584403in}{5.262270in}}%
\pgfpathlineto{\pgfqpoint{2.610536in}{5.303297in}}%
\pgfpathlineto{\pgfqpoint{2.636669in}{5.357575in}}%
\pgfpathlineto{\pgfqpoint{2.662803in}{5.422085in}}%
\pgfpathlineto{\pgfqpoint{2.688936in}{5.486795in}}%
\pgfpathlineto{\pgfqpoint{2.715069in}{5.539744in}}%
\pgfpathlineto{\pgfqpoint{2.741203in}{5.575200in}}%
\pgfpathlineto{\pgfqpoint{2.767336in}{5.593378in}}%
\pgfpathlineto{\pgfqpoint{2.793469in}{5.596272in}}%
\pgfpathlineto{\pgfqpoint{2.819603in}{5.585815in}}%
\pgfpathlineto{\pgfqpoint{2.845736in}{5.566251in}}%
\pgfpathlineto{\pgfqpoint{2.871869in}{5.546553in}}%
\pgfpathlineto{\pgfqpoint{2.898003in}{5.538070in}}%
\pgfpathlineto{\pgfqpoint{2.924136in}{5.546131in}}%
\pgfpathlineto{\pgfqpoint{2.950269in}{5.564099in}}%
\pgfpathlineto{\pgfqpoint{2.976403in}{5.575967in}}%
\pgfpathlineto{\pgfqpoint{3.002536in}{5.567644in}}%
\pgfpathlineto{\pgfqpoint{3.028669in}{5.534145in}}%
\pgfpathlineto{\pgfqpoint{3.054803in}{5.478970in}}%
\pgfpathlineto{\pgfqpoint{3.080936in}{5.409403in}}%
\pgfpathlineto{\pgfqpoint{3.107070in}{5.329839in}}%
\pgfpathlineto{\pgfqpoint{3.133203in}{5.240455in}}%
\pgfusepath{stroke}%
\end{pgfscope}%
\begin{pgfscope}%
\pgfpathrectangle{\pgfqpoint{0.828241in}{3.379757in}}{\pgfqpoint{2.414722in}{2.357859in}}%
\pgfusepath{clip}%
\pgfsetroundcap%
\pgfsetroundjoin%
\pgfsetlinewidth{1.505625pt}%
\definecolor{currentstroke}{rgb}{0.505882,0.447059,0.701961}%
\pgfsetstrokecolor{currentstroke}%
\pgfsetdash{}{0pt}%
\pgfpathmoveto{\pgfqpoint{0.938001in}{4.326237in}}%
\pgfpathlineto{\pgfqpoint{0.965848in}{4.369539in}}%
\pgfpathlineto{\pgfqpoint{0.993695in}{4.451137in}}%
\pgfpathlineto{\pgfqpoint{1.021542in}{4.605988in}}%
\pgfpathlineto{\pgfqpoint{1.049389in}{4.843846in}}%
\pgfpathlineto{\pgfqpoint{1.077236in}{5.126689in}}%
\pgfpathlineto{\pgfqpoint{1.105083in}{5.381522in}}%
\pgfpathlineto{\pgfqpoint{1.132930in}{5.548284in}}%
\pgfpathlineto{\pgfqpoint{1.160777in}{5.614621in}}%
\pgfpathlineto{\pgfqpoint{1.188624in}{5.607360in}}%
\pgfpathlineto{\pgfqpoint{1.216471in}{5.566251in}}%
\pgfpathlineto{\pgfqpoint{1.244318in}{5.525192in}}%
\pgfpathlineto{\pgfqpoint{1.272166in}{5.503824in}}%
\pgfpathlineto{\pgfqpoint{1.300013in}{5.501200in}}%
\pgfpathlineto{\pgfqpoint{1.327860in}{5.494199in}}%
\pgfpathlineto{\pgfqpoint{1.355707in}{5.451112in}}%
\pgfpathlineto{\pgfqpoint{1.383554in}{5.352215in}}%
\pgfpathlineto{\pgfqpoint{1.411401in}{5.198709in}}%
\pgfpathlineto{\pgfqpoint{1.439248in}{5.007689in}}%
\pgfpathlineto{\pgfqpoint{1.467095in}{4.802632in}}%
\pgfpathlineto{\pgfqpoint{1.494942in}{4.606888in}}%
\pgfpathlineto{\pgfqpoint{1.522789in}{4.435897in}}%
\pgfpathlineto{\pgfqpoint{1.550636in}{4.289741in}}%
\pgfpathlineto{\pgfqpoint{1.578483in}{4.160036in}}%
\pgfpathlineto{\pgfqpoint{1.606330in}{4.038761in}}%
\pgfpathlineto{\pgfqpoint{1.634177in}{3.921869in}}%
\pgfpathlineto{\pgfqpoint{1.662024in}{3.811252in}}%
\pgfpathlineto{\pgfqpoint{1.689871in}{3.712418in}}%
\pgfpathlineto{\pgfqpoint{1.717718in}{3.631457in}}%
\pgfpathlineto{\pgfqpoint{1.745565in}{3.573325in}}%
\pgfpathlineto{\pgfqpoint{1.773412in}{3.540357in}}%
\pgfpathlineto{\pgfqpoint{1.801259in}{3.530694in}}%
\pgfpathlineto{\pgfqpoint{1.829106in}{3.536296in}}%
\pgfpathlineto{\pgfqpoint{1.856953in}{3.543183in}}%
\pgfpathlineto{\pgfqpoint{1.884800in}{3.538107in}}%
\pgfpathlineto{\pgfqpoint{1.912647in}{3.519111in}}%
\pgfpathlineto{\pgfqpoint{1.940494in}{3.496701in}}%
\pgfpathlineto{\pgfqpoint{1.968341in}{3.486932in}}%
\pgfpathlineto{\pgfqpoint{1.996188in}{3.504659in}}%
\pgfpathlineto{\pgfqpoint{2.024035in}{3.557932in}}%
\pgfpathlineto{\pgfqpoint{2.051882in}{3.645250in}}%
\pgfpathlineto{\pgfqpoint{2.079729in}{3.756442in}}%
\pgfpathlineto{\pgfqpoint{2.107576in}{3.876526in}}%
\pgfpathlineto{\pgfqpoint{2.135423in}{3.991261in}}%
\pgfpathlineto{\pgfqpoint{2.163270in}{4.091667in}}%
\pgfpathlineto{\pgfqpoint{2.191117in}{4.174050in}}%
\pgfpathlineto{\pgfqpoint{2.218964in}{4.237206in}}%
\pgfpathlineto{\pgfqpoint{2.246811in}{4.280091in}}%
\pgfpathlineto{\pgfqpoint{2.274658in}{4.302571in}}%
\pgfpathlineto{\pgfqpoint{2.302505in}{4.306580in}}%
\pgfpathlineto{\pgfqpoint{2.330352in}{4.300049in}}%
\pgfpathlineto{\pgfqpoint{2.358199in}{4.302329in}}%
\pgfpathlineto{\pgfqpoint{2.386046in}{4.345656in}}%
\pgfpathlineto{\pgfqpoint{2.413893in}{4.465567in}}%
\pgfpathlineto{\pgfqpoint{2.441740in}{4.681371in}}%
\pgfpathlineto{\pgfqpoint{2.469587in}{4.973792in}}%
\pgfpathlineto{\pgfqpoint{2.497434in}{5.274464in}}%
\pgfpathlineto{\pgfqpoint{2.525281in}{5.500132in}}%
\pgfpathlineto{\pgfqpoint{2.553128in}{5.608489in}}%
\pgfpathlineto{\pgfqpoint{2.580975in}{5.614928in}}%
\pgfpathlineto{\pgfqpoint{2.608822in}{5.566251in}}%
\pgfpathlineto{\pgfqpoint{2.636669in}{5.507284in}}%
\pgfpathlineto{\pgfqpoint{2.664516in}{5.464301in}}%
\pgfpathlineto{\pgfqpoint{2.692363in}{5.442439in}}%
\pgfpathlineto{\pgfqpoint{2.720210in}{5.424445in}}%
\pgfpathlineto{\pgfqpoint{2.748057in}{5.379901in}}%
\pgfpathlineto{\pgfqpoint{2.775904in}{5.290642in}}%
\pgfpathlineto{\pgfqpoint{2.803751in}{5.161881in}}%
\pgfpathlineto{\pgfqpoint{2.831598in}{5.008032in}}%
\pgfpathlineto{\pgfqpoint{2.859445in}{4.840497in}}%
\pgfpathlineto{\pgfqpoint{2.887292in}{4.666988in}}%
\pgfusepath{stroke}%
\end{pgfscope}%
\begin{pgfscope}%
\pgfsetrectcap%
\pgfsetmiterjoin%
\pgfsetlinewidth{1.254687pt}%
\definecolor{currentstroke}{rgb}{1.000000,1.000000,1.000000}%
\pgfsetstrokecolor{currentstroke}%
\pgfsetdash{}{0pt}%
\pgfpathmoveto{\pgfqpoint{0.828241in}{3.379757in}}%
\pgfpathlineto{\pgfqpoint{0.828241in}{5.737616in}}%
\pgfusepath{stroke}%
\end{pgfscope}%
\begin{pgfscope}%
\pgfsetrectcap%
\pgfsetmiterjoin%
\pgfsetlinewidth{1.254687pt}%
\definecolor{currentstroke}{rgb}{1.000000,1.000000,1.000000}%
\pgfsetstrokecolor{currentstroke}%
\pgfsetdash{}{0pt}%
\pgfpathmoveto{\pgfqpoint{3.242963in}{3.379757in}}%
\pgfpathlineto{\pgfqpoint{3.242963in}{5.737616in}}%
\pgfusepath{stroke}%
\end{pgfscope}%
\begin{pgfscope}%
\pgfsetrectcap%
\pgfsetmiterjoin%
\pgfsetlinewidth{1.254687pt}%
\definecolor{currentstroke}{rgb}{1.000000,1.000000,1.000000}%
\pgfsetstrokecolor{currentstroke}%
\pgfsetdash{}{0pt}%
\pgfpathmoveto{\pgfqpoint{0.828241in}{3.379757in}}%
\pgfpathlineto{\pgfqpoint{3.242963in}{3.379757in}}%
\pgfusepath{stroke}%
\end{pgfscope}%
\begin{pgfscope}%
\pgfsetrectcap%
\pgfsetmiterjoin%
\pgfsetlinewidth{1.254687pt}%
\definecolor{currentstroke}{rgb}{1.000000,1.000000,1.000000}%
\pgfsetstrokecolor{currentstroke}%
\pgfsetdash{}{0pt}%
\pgfpathmoveto{\pgfqpoint{0.828241in}{5.737616in}}%
\pgfpathlineto{\pgfqpoint{3.242963in}{5.737616in}}%
\pgfusepath{stroke}%
\end{pgfscope}%
\begin{pgfscope}%
\pgfsetbuttcap%
\pgfsetmiterjoin%
\definecolor{currentfill}{rgb}{0.917647,0.917647,0.949020}%
\pgfsetfillcolor{currentfill}%
\pgfsetlinewidth{0.000000pt}%
\definecolor{currentstroke}{rgb}{0.000000,0.000000,0.000000}%
\pgfsetstrokecolor{currentstroke}%
\pgfsetstrokeopacity{0.000000}%
\pgfsetdash{}{0pt}%
\pgfpathmoveto{\pgfqpoint{3.825278in}{3.379757in}}%
\pgfpathlineto{\pgfqpoint{6.240000in}{3.379757in}}%
\pgfpathlineto{\pgfqpoint{6.240000in}{5.737616in}}%
\pgfpathlineto{\pgfqpoint{3.825278in}{5.737616in}}%
\pgfpathclose%
\pgfusepath{fill}%
\end{pgfscope}%
\begin{pgfscope}%
\pgfpathrectangle{\pgfqpoint{3.825278in}{3.379757in}}{\pgfqpoint{2.414722in}{2.357859in}}%
\pgfusepath{clip}%
\pgfsetroundcap%
\pgfsetroundjoin%
\pgfsetlinewidth{1.003750pt}%
\definecolor{currentstroke}{rgb}{1.000000,1.000000,1.000000}%
\pgfsetstrokecolor{currentstroke}%
\pgfsetdash{}{0pt}%
\pgfpathmoveto{\pgfqpoint{3.935038in}{3.379757in}}%
\pgfpathlineto{\pgfqpoint{3.935038in}{5.737616in}}%
\pgfusepath{stroke}%
\end{pgfscope}%
\begin{pgfscope}%
\definecolor{textcolor}{rgb}{0.150000,0.150000,0.150000}%
\pgfsetstrokecolor{textcolor}%
\pgfsetfillcolor{textcolor}%
\pgftext[x=3.935038in,y=3.247812in,,top]{\color{textcolor}\sffamily\fontsize{11.000000}{13.200000}\selectfont \(\displaystyle 0.0\)}%
\end{pgfscope}%
\begin{pgfscope}%
\pgfpathrectangle{\pgfqpoint{3.825278in}{3.379757in}}{\pgfqpoint{2.414722in}{2.357859in}}%
\pgfusepath{clip}%
\pgfsetroundcap%
\pgfsetroundjoin%
\pgfsetlinewidth{1.003750pt}%
\definecolor{currentstroke}{rgb}{1.000000,1.000000,1.000000}%
\pgfsetstrokecolor{currentstroke}%
\pgfsetdash{}{0pt}%
\pgfpathmoveto{\pgfqpoint{4.637503in}{3.379757in}}%
\pgfpathlineto{\pgfqpoint{4.637503in}{5.737616in}}%
\pgfusepath{stroke}%
\end{pgfscope}%
\begin{pgfscope}%
\definecolor{textcolor}{rgb}{0.150000,0.150000,0.150000}%
\pgfsetstrokecolor{textcolor}%
\pgfsetfillcolor{textcolor}%
\pgftext[x=4.637503in,y=3.247812in,,top]{\color{textcolor}\sffamily\fontsize{11.000000}{13.200000}\selectfont \(\displaystyle 0.5\)}%
\end{pgfscope}%
\begin{pgfscope}%
\pgfpathrectangle{\pgfqpoint{3.825278in}{3.379757in}}{\pgfqpoint{2.414722in}{2.357859in}}%
\pgfusepath{clip}%
\pgfsetroundcap%
\pgfsetroundjoin%
\pgfsetlinewidth{1.003750pt}%
\definecolor{currentstroke}{rgb}{1.000000,1.000000,1.000000}%
\pgfsetstrokecolor{currentstroke}%
\pgfsetdash{}{0pt}%
\pgfpathmoveto{\pgfqpoint{5.339967in}{3.379757in}}%
\pgfpathlineto{\pgfqpoint{5.339967in}{5.737616in}}%
\pgfusepath{stroke}%
\end{pgfscope}%
\begin{pgfscope}%
\definecolor{textcolor}{rgb}{0.150000,0.150000,0.150000}%
\pgfsetstrokecolor{textcolor}%
\pgfsetfillcolor{textcolor}%
\pgftext[x=5.339967in,y=3.247812in,,top]{\color{textcolor}\sffamily\fontsize{11.000000}{13.200000}\selectfont \(\displaystyle 1.0\)}%
\end{pgfscope}%
\begin{pgfscope}%
\pgfpathrectangle{\pgfqpoint{3.825278in}{3.379757in}}{\pgfqpoint{2.414722in}{2.357859in}}%
\pgfusepath{clip}%
\pgfsetroundcap%
\pgfsetroundjoin%
\pgfsetlinewidth{1.003750pt}%
\definecolor{currentstroke}{rgb}{1.000000,1.000000,1.000000}%
\pgfsetstrokecolor{currentstroke}%
\pgfsetdash{}{0pt}%
\pgfpathmoveto{\pgfqpoint{6.042432in}{3.379757in}}%
\pgfpathlineto{\pgfqpoint{6.042432in}{5.737616in}}%
\pgfusepath{stroke}%
\end{pgfscope}%
\begin{pgfscope}%
\definecolor{textcolor}{rgb}{0.150000,0.150000,0.150000}%
\pgfsetstrokecolor{textcolor}%
\pgfsetfillcolor{textcolor}%
\pgftext[x=6.042432in,y=3.247812in,,top]{\color{textcolor}\sffamily\fontsize{11.000000}{13.200000}\selectfont \(\displaystyle 1.5\)}%
\end{pgfscope}%
\begin{pgfscope}%
\pgfpathrectangle{\pgfqpoint{3.825278in}{3.379757in}}{\pgfqpoint{2.414722in}{2.357859in}}%
\pgfusepath{clip}%
\pgfsetroundcap%
\pgfsetroundjoin%
\pgfsetlinewidth{1.003750pt}%
\definecolor{currentstroke}{rgb}{1.000000,1.000000,1.000000}%
\pgfsetstrokecolor{currentstroke}%
\pgfsetdash{}{0pt}%
\pgfpathmoveto{\pgfqpoint{3.825278in}{3.544338in}}%
\pgfpathlineto{\pgfqpoint{6.240000in}{3.544338in}}%
\pgfusepath{stroke}%
\end{pgfscope}%
\begin{pgfscope}%
\definecolor{textcolor}{rgb}{0.150000,0.150000,0.150000}%
\pgfsetstrokecolor{textcolor}%
\pgfsetfillcolor{textcolor}%
\pgftext[x=3.422963in,y=3.491531in,left,base]{\color{textcolor}\sffamily\fontsize{11.000000}{13.200000}\selectfont \(\displaystyle -20\)}%
\end{pgfscope}%
\begin{pgfscope}%
\pgfpathrectangle{\pgfqpoint{3.825278in}{3.379757in}}{\pgfqpoint{2.414722in}{2.357859in}}%
\pgfusepath{clip}%
\pgfsetroundcap%
\pgfsetroundjoin%
\pgfsetlinewidth{1.003750pt}%
\definecolor{currentstroke}{rgb}{1.000000,1.000000,1.000000}%
\pgfsetstrokecolor{currentstroke}%
\pgfsetdash{}{0pt}%
\pgfpathmoveto{\pgfqpoint{3.825278in}{4.046733in}}%
\pgfpathlineto{\pgfqpoint{6.240000in}{4.046733in}}%
\pgfusepath{stroke}%
\end{pgfscope}%
\begin{pgfscope}%
\definecolor{textcolor}{rgb}{0.150000,0.150000,0.150000}%
\pgfsetstrokecolor{textcolor}%
\pgfsetfillcolor{textcolor}%
\pgftext[x=3.422963in,y=3.993927in,left,base]{\color{textcolor}\sffamily\fontsize{11.000000}{13.200000}\selectfont \(\displaystyle -15\)}%
\end{pgfscope}%
\begin{pgfscope}%
\pgfpathrectangle{\pgfqpoint{3.825278in}{3.379757in}}{\pgfqpoint{2.414722in}{2.357859in}}%
\pgfusepath{clip}%
\pgfsetroundcap%
\pgfsetroundjoin%
\pgfsetlinewidth{1.003750pt}%
\definecolor{currentstroke}{rgb}{1.000000,1.000000,1.000000}%
\pgfsetstrokecolor{currentstroke}%
\pgfsetdash{}{0pt}%
\pgfpathmoveto{\pgfqpoint{3.825278in}{4.549129in}}%
\pgfpathlineto{\pgfqpoint{6.240000in}{4.549129in}}%
\pgfusepath{stroke}%
\end{pgfscope}%
\begin{pgfscope}%
\definecolor{textcolor}{rgb}{0.150000,0.150000,0.150000}%
\pgfsetstrokecolor{textcolor}%
\pgfsetfillcolor{textcolor}%
\pgftext[x=3.422963in,y=4.496322in,left,base]{\color{textcolor}\sffamily\fontsize{11.000000}{13.200000}\selectfont \(\displaystyle -10\)}%
\end{pgfscope}%
\begin{pgfscope}%
\pgfpathrectangle{\pgfqpoint{3.825278in}{3.379757in}}{\pgfqpoint{2.414722in}{2.357859in}}%
\pgfusepath{clip}%
\pgfsetroundcap%
\pgfsetroundjoin%
\pgfsetlinewidth{1.003750pt}%
\definecolor{currentstroke}{rgb}{1.000000,1.000000,1.000000}%
\pgfsetstrokecolor{currentstroke}%
\pgfsetdash{}{0pt}%
\pgfpathmoveto{\pgfqpoint{3.825278in}{5.051524in}}%
\pgfpathlineto{\pgfqpoint{6.240000in}{5.051524in}}%
\pgfusepath{stroke}%
\end{pgfscope}%
\begin{pgfscope}%
\definecolor{textcolor}{rgb}{0.150000,0.150000,0.150000}%
\pgfsetstrokecolor{textcolor}%
\pgfsetfillcolor{textcolor}%
\pgftext[x=3.499005in,y=4.998718in,left,base]{\color{textcolor}\sffamily\fontsize{11.000000}{13.200000}\selectfont \(\displaystyle -5\)}%
\end{pgfscope}%
\begin{pgfscope}%
\pgfpathrectangle{\pgfqpoint{3.825278in}{3.379757in}}{\pgfqpoint{2.414722in}{2.357859in}}%
\pgfusepath{clip}%
\pgfsetroundcap%
\pgfsetroundjoin%
\pgfsetlinewidth{1.003750pt}%
\definecolor{currentstroke}{rgb}{1.000000,1.000000,1.000000}%
\pgfsetstrokecolor{currentstroke}%
\pgfsetdash{}{0pt}%
\pgfpathmoveto{\pgfqpoint{3.825278in}{5.553920in}}%
\pgfpathlineto{\pgfqpoint{6.240000in}{5.553920in}}%
\pgfusepath{stroke}%
\end{pgfscope}%
\begin{pgfscope}%
\definecolor{textcolor}{rgb}{0.150000,0.150000,0.150000}%
\pgfsetstrokecolor{textcolor}%
\pgfsetfillcolor{textcolor}%
\pgftext[x=3.617292in,y=5.501113in,left,base]{\color{textcolor}\sffamily\fontsize{11.000000}{13.200000}\selectfont \(\displaystyle 0\)}%
\end{pgfscope}%
\begin{pgfscope}%
\pgfpathrectangle{\pgfqpoint{3.825278in}{3.379757in}}{\pgfqpoint{2.414722in}{2.357859in}}%
\pgfusepath{clip}%
\pgfsetroundcap%
\pgfsetroundjoin%
\pgfsetlinewidth{1.505625pt}%
\definecolor{currentstroke}{rgb}{0.298039,0.447059,0.690196}%
\pgfsetstrokecolor{currentstroke}%
\pgfsetdash{}{0pt}%
\pgfpathmoveto{\pgfqpoint{3.935038in}{5.536127in}}%
\pgfpathlineto{\pgfqpoint{3.952172in}{5.556864in}}%
\pgfpathlineto{\pgfqpoint{3.969305in}{5.565171in}}%
\pgfpathlineto{\pgfqpoint{3.986438in}{5.553920in}}%
\pgfpathlineto{\pgfqpoint{4.003571in}{5.523636in}}%
\pgfpathlineto{\pgfqpoint{4.020705in}{5.481916in}}%
\pgfpathlineto{\pgfqpoint{4.037838in}{5.440040in}}%
\pgfpathlineto{\pgfqpoint{4.054971in}{5.404547in}}%
\pgfpathlineto{\pgfqpoint{4.072105in}{5.369020in}}%
\pgfpathlineto{\pgfqpoint{4.089238in}{5.318923in}}%
\pgfpathlineto{\pgfqpoint{4.106371in}{5.243963in}}%
\pgfpathlineto{\pgfqpoint{4.123504in}{5.143460in}}%
\pgfpathlineto{\pgfqpoint{4.140638in}{5.024856in}}%
\pgfpathlineto{\pgfqpoint{4.157771in}{4.897738in}}%
\pgfpathlineto{\pgfqpoint{4.174904in}{4.771482in}}%
\pgfpathlineto{\pgfqpoint{4.192038in}{4.654407in}}%
\pgfpathlineto{\pgfqpoint{4.209171in}{4.549189in}}%
\pgfpathlineto{\pgfqpoint{4.226304in}{4.452705in}}%
\pgfpathlineto{\pgfqpoint{4.243437in}{4.360609in}}%
\pgfpathlineto{\pgfqpoint{4.260571in}{4.270559in}}%
\pgfpathlineto{\pgfqpoint{4.277704in}{4.183268in}}%
\pgfpathlineto{\pgfqpoint{4.294837in}{4.099390in}}%
\pgfpathlineto{\pgfqpoint{4.311971in}{4.018420in}}%
\pgfpathlineto{\pgfqpoint{4.329104in}{3.941387in}}%
\pgfpathlineto{\pgfqpoint{4.346237in}{3.869777in}}%
\pgfpathlineto{\pgfqpoint{4.363370in}{3.803857in}}%
\pgfpathlineto{\pgfqpoint{4.380504in}{3.743605in}}%
\pgfpathlineto{\pgfqpoint{4.397637in}{3.687061in}}%
\pgfpathlineto{\pgfqpoint{4.414770in}{3.633185in}}%
\pgfpathlineto{\pgfqpoint{4.431903in}{3.585623in}}%
\pgfpathlineto{\pgfqpoint{4.449037in}{3.550533in}}%
\pgfpathlineto{\pgfqpoint{4.466170in}{3.534509in}}%
\pgfpathlineto{\pgfqpoint{4.483303in}{3.543815in}}%
\pgfpathlineto{\pgfqpoint{4.500437in}{3.580409in}}%
\pgfpathlineto{\pgfqpoint{4.517570in}{3.640435in}}%
\pgfpathlineto{\pgfqpoint{4.534703in}{3.715786in}}%
\pgfpathlineto{\pgfqpoint{4.551836in}{3.793659in}}%
\pgfpathlineto{\pgfqpoint{4.568970in}{3.859590in}}%
\pgfpathlineto{\pgfqpoint{4.586103in}{3.908103in}}%
\pgfpathlineto{\pgfqpoint{4.603236in}{3.946946in}}%
\pgfpathlineto{\pgfqpoint{4.620370in}{3.985510in}}%
\pgfpathlineto{\pgfqpoint{4.637503in}{4.024382in}}%
\pgfpathlineto{\pgfqpoint{4.654636in}{4.059922in}}%
\pgfpathlineto{\pgfqpoint{4.671769in}{4.092873in}}%
\pgfpathlineto{\pgfqpoint{4.688903in}{4.127181in}}%
\pgfpathlineto{\pgfqpoint{4.706036in}{4.165120in}}%
\pgfpathlineto{\pgfqpoint{4.723169in}{4.209366in}}%
\pgfpathlineto{\pgfqpoint{4.740303in}{4.260775in}}%
\pgfpathlineto{\pgfqpoint{4.757436in}{4.316631in}}%
\pgfpathlineto{\pgfqpoint{4.774569in}{4.371066in}}%
\pgfpathlineto{\pgfqpoint{4.791702in}{4.418899in}}%
\pgfpathlineto{\pgfqpoint{4.808836in}{4.459606in}}%
\pgfpathlineto{\pgfqpoint{4.825969in}{4.496861in}}%
\pgfpathlineto{\pgfqpoint{4.843102in}{4.536893in}}%
\pgfpathlineto{\pgfqpoint{4.860235in}{4.587763in}}%
\pgfpathlineto{\pgfqpoint{4.877369in}{4.656841in}}%
\pgfpathlineto{\pgfqpoint{4.894502in}{4.749331in}}%
\pgfpathlineto{\pgfqpoint{4.911635in}{4.869417in}}%
\pgfpathlineto{\pgfqpoint{4.928769in}{5.017172in}}%
\pgfpathlineto{\pgfqpoint{4.945902in}{5.182051in}}%
\pgfpathlineto{\pgfqpoint{4.963035in}{5.340210in}}%
\pgfpathlineto{\pgfqpoint{4.980168in}{5.465082in}}%
\pgfpathlineto{\pgfqpoint{4.997302in}{5.541395in}}%
\pgfpathlineto{\pgfqpoint{5.014435in}{5.568193in}}%
\pgfpathlineto{\pgfqpoint{5.031568in}{5.553920in}}%
\pgfpathlineto{\pgfqpoint{5.048702in}{5.509139in}}%
\pgfpathlineto{\pgfqpoint{5.065835in}{5.446333in}}%
\pgfusepath{stroke}%
\end{pgfscope}%
\begin{pgfscope}%
\pgfpathrectangle{\pgfqpoint{3.825278in}{3.379757in}}{\pgfqpoint{2.414722in}{2.357859in}}%
\pgfusepath{clip}%
\pgfsetroundcap%
\pgfsetroundjoin%
\pgfsetlinewidth{1.505625pt}%
\definecolor{currentstroke}{rgb}{0.866667,0.517647,0.321569}%
\pgfsetstrokecolor{currentstroke}%
\pgfsetdash{}{0pt}%
\pgfpathmoveto{\pgfqpoint{3.935038in}{4.660074in}}%
\pgfpathlineto{\pgfqpoint{3.952172in}{4.673089in}}%
\pgfpathlineto{\pgfqpoint{3.969305in}{4.701194in}}%
\pgfpathlineto{\pgfqpoint{3.986438in}{4.756481in}}%
\pgfpathlineto{\pgfqpoint{4.003571in}{4.841609in}}%
\pgfpathlineto{\pgfqpoint{4.020705in}{4.951587in}}%
\pgfpathlineto{\pgfqpoint{4.037838in}{5.078289in}}%
\pgfpathlineto{\pgfqpoint{4.054971in}{5.209579in}}%
\pgfpathlineto{\pgfqpoint{4.072105in}{5.329944in}}%
\pgfpathlineto{\pgfqpoint{4.089238in}{5.428892in}}%
\pgfpathlineto{\pgfqpoint{4.106371in}{5.501693in}}%
\pgfpathlineto{\pgfqpoint{4.123504in}{5.545782in}}%
\pgfpathlineto{\pgfqpoint{4.140638in}{5.560983in}}%
\pgfpathlineto{\pgfqpoint{4.157771in}{5.553920in}}%
\pgfpathlineto{\pgfqpoint{4.174904in}{5.538466in}}%
\pgfpathlineto{\pgfqpoint{4.192038in}{5.527677in}}%
\pgfpathlineto{\pgfqpoint{4.209171in}{5.523829in}}%
\pgfpathlineto{\pgfqpoint{4.226304in}{5.516579in}}%
\pgfpathlineto{\pgfqpoint{4.243437in}{5.491164in}}%
\pgfpathlineto{\pgfqpoint{4.260571in}{5.438601in}}%
\pgfpathlineto{\pgfqpoint{4.277704in}{5.357794in}}%
\pgfpathlineto{\pgfqpoint{4.294837in}{5.254661in}}%
\pgfpathlineto{\pgfqpoint{4.311971in}{5.140151in}}%
\pgfpathlineto{\pgfqpoint{4.329104in}{5.025899in}}%
\pgfpathlineto{\pgfqpoint{4.346237in}{4.920457in}}%
\pgfpathlineto{\pgfqpoint{4.363370in}{4.825738in}}%
\pgfpathlineto{\pgfqpoint{4.380504in}{4.736709in}}%
\pgfpathlineto{\pgfqpoint{4.397637in}{4.647208in}}%
\pgfpathlineto{\pgfqpoint{4.414770in}{4.553392in}}%
\pgfpathlineto{\pgfqpoint{4.431903in}{4.456709in}}%
\pgfpathlineto{\pgfqpoint{4.449037in}{4.363246in}}%
\pgfpathlineto{\pgfqpoint{4.466170in}{4.275397in}}%
\pgfpathlineto{\pgfqpoint{4.483303in}{4.186772in}}%
\pgfpathlineto{\pgfqpoint{4.500437in}{4.089933in}}%
\pgfpathlineto{\pgfqpoint{4.517570in}{3.990533in}}%
\pgfpathlineto{\pgfqpoint{4.534703in}{3.898356in}}%
\pgfpathlineto{\pgfqpoint{4.551836in}{3.818172in}}%
\pgfpathlineto{\pgfqpoint{4.568970in}{3.747952in}}%
\pgfpathlineto{\pgfqpoint{4.586103in}{3.684301in}}%
\pgfpathlineto{\pgfqpoint{4.603236in}{3.626911in}}%
\pgfpathlineto{\pgfqpoint{4.620370in}{3.575846in}}%
\pgfpathlineto{\pgfqpoint{4.637503in}{3.531074in}}%
\pgfpathlineto{\pgfqpoint{4.654636in}{3.498378in}}%
\pgfpathlineto{\pgfqpoint{4.671769in}{3.486932in}}%
\pgfpathlineto{\pgfqpoint{4.688903in}{3.497993in}}%
\pgfpathlineto{\pgfqpoint{4.706036in}{3.524401in}}%
\pgfpathlineto{\pgfqpoint{4.723169in}{3.559537in}}%
\pgfpathlineto{\pgfqpoint{4.740303in}{3.599921in}}%
\pgfpathlineto{\pgfqpoint{4.757436in}{3.642783in}}%
\pgfpathlineto{\pgfqpoint{4.774569in}{3.683807in}}%
\pgfpathlineto{\pgfqpoint{4.791702in}{3.720252in}}%
\pgfpathlineto{\pgfqpoint{4.808836in}{3.755045in}}%
\pgfpathlineto{\pgfqpoint{4.825969in}{3.795054in}}%
\pgfpathlineto{\pgfqpoint{4.843102in}{3.846366in}}%
\pgfpathlineto{\pgfqpoint{4.860235in}{3.911528in}}%
\pgfpathlineto{\pgfqpoint{4.877369in}{3.992085in}}%
\pgfpathlineto{\pgfqpoint{4.894502in}{4.089049in}}%
\pgfpathlineto{\pgfqpoint{4.911635in}{4.199986in}}%
\pgfpathlineto{\pgfqpoint{4.928769in}{4.317833in}}%
\pgfpathlineto{\pgfqpoint{4.945902in}{4.433057in}}%
\pgfpathlineto{\pgfqpoint{4.963035in}{4.539178in}}%
\pgfpathlineto{\pgfqpoint{4.980168in}{4.632587in}}%
\pgfpathlineto{\pgfqpoint{4.997302in}{4.710183in}}%
\pgfpathlineto{\pgfqpoint{5.014435in}{4.769755in}}%
\pgfpathlineto{\pgfqpoint{5.031568in}{4.811553in}}%
\pgfpathlineto{\pgfqpoint{5.048702in}{4.837212in}}%
\pgfpathlineto{\pgfqpoint{5.065835in}{4.848974in}}%
\pgfpathlineto{\pgfqpoint{5.082968in}{4.850231in}}%
\pgfpathlineto{\pgfqpoint{5.100101in}{4.846803in}}%
\pgfpathlineto{\pgfqpoint{5.117235in}{4.847805in}}%
\pgfpathlineto{\pgfqpoint{5.134368in}{4.865014in}}%
\pgfpathlineto{\pgfqpoint{5.151501in}{4.907696in}}%
\pgfpathlineto{\pgfqpoint{5.168635in}{4.977796in}}%
\pgfpathlineto{\pgfqpoint{5.185768in}{5.070028in}}%
\pgfpathlineto{\pgfqpoint{5.202901in}{5.175592in}}%
\pgfpathlineto{\pgfqpoint{5.220034in}{5.285694in}}%
\pgfpathlineto{\pgfqpoint{5.237168in}{5.392355in}}%
\pgfpathlineto{\pgfqpoint{5.254301in}{5.487647in}}%
\pgfpathlineto{\pgfqpoint{5.271434in}{5.562394in}}%
\pgfpathlineto{\pgfqpoint{5.288568in}{5.606853in}}%
\pgfpathlineto{\pgfqpoint{5.305701in}{5.615743in}}%
\pgfpathlineto{\pgfqpoint{5.322834in}{5.593074in}}%
\pgfpathlineto{\pgfqpoint{5.339967in}{5.553920in}}%
\pgfpathlineto{\pgfqpoint{5.357101in}{5.519437in}}%
\pgfpathlineto{\pgfqpoint{5.374234in}{5.502868in}}%
\pgfpathlineto{\pgfqpoint{5.391367in}{5.497691in}}%
\pgfpathlineto{\pgfqpoint{5.408500in}{5.483926in}}%
\pgfpathlineto{\pgfqpoint{5.425634in}{5.443157in}}%
\pgfpathlineto{\pgfqpoint{5.442767in}{5.367141in}}%
\pgfpathlineto{\pgfqpoint{5.459900in}{5.259248in}}%
\pgfpathlineto{\pgfqpoint{5.477034in}{5.134619in}}%
\pgfpathlineto{\pgfqpoint{5.494167in}{5.012221in}}%
\pgfpathlineto{\pgfqpoint{5.511300in}{4.903546in}}%
\pgfpathlineto{\pgfqpoint{5.528433in}{4.808860in}}%
\pgfusepath{stroke}%
\end{pgfscope}%
\begin{pgfscope}%
\pgfpathrectangle{\pgfqpoint{3.825278in}{3.379757in}}{\pgfqpoint{2.414722in}{2.357859in}}%
\pgfusepath{clip}%
\pgfsetroundcap%
\pgfsetroundjoin%
\pgfsetlinewidth{1.505625pt}%
\definecolor{currentstroke}{rgb}{0.333333,0.658824,0.407843}%
\pgfsetstrokecolor{currentstroke}%
\pgfsetdash{}{0pt}%
\pgfpathmoveto{\pgfqpoint{3.935038in}{4.892509in}}%
\pgfpathlineto{\pgfqpoint{3.959686in}{4.890869in}}%
\pgfpathlineto{\pgfqpoint{3.984334in}{4.889118in}}%
\pgfpathlineto{\pgfqpoint{4.008982in}{4.892492in}}%
\pgfpathlineto{\pgfqpoint{4.033630in}{4.913624in}}%
\pgfpathlineto{\pgfqpoint{4.058278in}{4.969060in}}%
\pgfpathlineto{\pgfqpoint{4.082926in}{5.069017in}}%
\pgfpathlineto{\pgfqpoint{4.107573in}{5.204350in}}%
\pgfpathlineto{\pgfqpoint{4.132221in}{5.346249in}}%
\pgfpathlineto{\pgfqpoint{4.156869in}{5.462092in}}%
\pgfpathlineto{\pgfqpoint{4.181517in}{5.532861in}}%
\pgfpathlineto{\pgfqpoint{4.206165in}{5.559078in}}%
\pgfpathlineto{\pgfqpoint{4.230813in}{5.553920in}}%
\pgfpathlineto{\pgfqpoint{4.255461in}{5.532506in}}%
\pgfpathlineto{\pgfqpoint{4.280109in}{5.505518in}}%
\pgfpathlineto{\pgfqpoint{4.304757in}{5.474525in}}%
\pgfpathlineto{\pgfqpoint{4.329404in}{5.432030in}}%
\pgfpathlineto{\pgfqpoint{4.354052in}{5.368111in}}%
\pgfpathlineto{\pgfqpoint{4.378700in}{5.277366in}}%
\pgfpathlineto{\pgfqpoint{4.403348in}{5.163006in}}%
\pgfpathlineto{\pgfqpoint{4.427996in}{5.038865in}}%
\pgfpathlineto{\pgfqpoint{4.452644in}{4.923116in}}%
\pgfpathlineto{\pgfqpoint{4.477292in}{4.825937in}}%
\pgfpathlineto{\pgfqpoint{4.501940in}{4.743895in}}%
\pgfpathlineto{\pgfqpoint{4.526587in}{4.667103in}}%
\pgfpathlineto{\pgfqpoint{4.551235in}{4.592025in}}%
\pgfpathlineto{\pgfqpoint{4.575883in}{4.520664in}}%
\pgfpathlineto{\pgfqpoint{4.600531in}{4.454638in}}%
\pgfpathlineto{\pgfqpoint{4.625179in}{4.393286in}}%
\pgfpathlineto{\pgfqpoint{4.649827in}{4.336229in}}%
\pgfpathlineto{\pgfqpoint{4.674475in}{4.286937in}}%
\pgfpathlineto{\pgfqpoint{4.699123in}{4.253056in}}%
\pgfpathlineto{\pgfqpoint{4.723770in}{4.241894in}}%
\pgfpathlineto{\pgfqpoint{4.748418in}{4.256041in}}%
\pgfpathlineto{\pgfqpoint{4.773066in}{4.291211in}}%
\pgfpathlineto{\pgfqpoint{4.797714in}{4.336676in}}%
\pgfpathlineto{\pgfqpoint{4.822362in}{4.380273in}}%
\pgfpathlineto{\pgfqpoint{4.847010in}{4.411664in}}%
\pgfpathlineto{\pgfqpoint{4.871658in}{4.423083in}}%
\pgfpathlineto{\pgfqpoint{4.896306in}{4.409516in}}%
\pgfpathlineto{\pgfqpoint{4.920953in}{4.371882in}}%
\pgfpathlineto{\pgfqpoint{4.945601in}{4.320085in}}%
\pgfpathlineto{\pgfqpoint{4.970249in}{4.271444in}}%
\pgfpathlineto{\pgfqpoint{4.994897in}{4.241623in}}%
\pgfpathlineto{\pgfqpoint{5.019545in}{4.236762in}}%
\pgfpathlineto{\pgfqpoint{5.044193in}{4.255769in}}%
\pgfpathlineto{\pgfqpoint{5.068841in}{4.296190in}}%
\pgfpathlineto{\pgfqpoint{5.093489in}{4.349146in}}%
\pgfpathlineto{\pgfqpoint{5.118136in}{4.401208in}}%
\pgfpathlineto{\pgfqpoint{5.142784in}{4.443561in}}%
\pgfpathlineto{\pgfqpoint{5.167432in}{4.475445in}}%
\pgfpathlineto{\pgfqpoint{5.192080in}{4.500387in}}%
\pgfpathlineto{\pgfqpoint{5.216728in}{4.523184in}}%
\pgfpathlineto{\pgfqpoint{5.241376in}{4.548563in}}%
\pgfpathlineto{\pgfqpoint{5.266024in}{4.578677in}}%
\pgfpathlineto{\pgfqpoint{5.290672in}{4.611387in}}%
\pgfpathlineto{\pgfqpoint{5.315319in}{4.643673in}}%
\pgfpathlineto{\pgfqpoint{5.339967in}{4.673973in}}%
\pgfpathlineto{\pgfqpoint{5.364615in}{4.703094in}}%
\pgfpathlineto{\pgfqpoint{5.389263in}{4.730521in}}%
\pgfpathlineto{\pgfqpoint{5.413911in}{4.753742in}}%
\pgfpathlineto{\pgfqpoint{5.438559in}{4.772397in}}%
\pgfpathlineto{\pgfqpoint{5.463207in}{4.789125in}}%
\pgfpathlineto{\pgfqpoint{5.487855in}{4.806029in}}%
\pgfpathlineto{\pgfqpoint{5.512502in}{4.822924in}}%
\pgfpathlineto{\pgfqpoint{5.537150in}{4.837955in}}%
\pgfpathlineto{\pgfqpoint{5.561798in}{4.850764in}}%
\pgfpathlineto{\pgfqpoint{5.586446in}{4.865700in}}%
\pgfpathlineto{\pgfqpoint{5.611094in}{4.892733in}}%
\pgfpathlineto{\pgfqpoint{5.635742in}{4.946715in}}%
\pgfpathlineto{\pgfqpoint{5.660390in}{5.040259in}}%
\pgfpathlineto{\pgfqpoint{5.685038in}{5.170817in}}%
\pgfpathlineto{\pgfqpoint{5.709686in}{5.317549in}}%
\pgfpathlineto{\pgfqpoint{5.734333in}{5.449558in}}%
\pgfpathlineto{\pgfqpoint{5.758981in}{5.538209in}}%
\pgfpathlineto{\pgfqpoint{5.783629in}{5.569981in}}%
\pgfpathlineto{\pgfqpoint{5.808277in}{5.553920in}}%
\pgfpathlineto{\pgfqpoint{5.832925in}{5.513746in}}%
\pgfpathlineto{\pgfqpoint{5.857573in}{5.472288in}}%
\pgfpathlineto{\pgfqpoint{5.882221in}{5.441054in}}%
\pgfpathlineto{\pgfqpoint{5.906869in}{5.417764in}}%
\pgfpathlineto{\pgfqpoint{5.931516in}{5.389044in}}%
\pgfpathlineto{\pgfqpoint{5.956164in}{5.338893in}}%
\pgfpathlineto{\pgfqpoint{5.980812in}{5.260557in}}%
\pgfpathlineto{\pgfqpoint{6.005460in}{5.161388in}}%
\pgfpathlineto{\pgfqpoint{6.030108in}{5.057004in}}%
\pgfpathlineto{\pgfqpoint{6.054756in}{4.962456in}}%
\pgfpathlineto{\pgfqpoint{6.079404in}{4.885717in}}%
\pgfpathlineto{\pgfqpoint{6.104052in}{4.822341in}}%
\pgfusepath{stroke}%
\end{pgfscope}%
\begin{pgfscope}%
\pgfpathrectangle{\pgfqpoint{3.825278in}{3.379757in}}{\pgfqpoint{2.414722in}{2.357859in}}%
\pgfusepath{clip}%
\pgfsetroundcap%
\pgfsetroundjoin%
\pgfsetlinewidth{1.505625pt}%
\definecolor{currentstroke}{rgb}{0.768627,0.305882,0.321569}%
\pgfsetstrokecolor{currentstroke}%
\pgfsetdash{}{0pt}%
\pgfpathmoveto{\pgfqpoint{3.935038in}{5.240198in}}%
\pgfpathlineto{\pgfqpoint{3.956990in}{5.256071in}}%
\pgfpathlineto{\pgfqpoint{3.978942in}{5.271620in}}%
\pgfpathlineto{\pgfqpoint{4.000894in}{5.284999in}}%
\pgfpathlineto{\pgfqpoint{4.022846in}{5.294633in}}%
\pgfpathlineto{\pgfqpoint{4.044798in}{5.302481in}}%
\pgfpathlineto{\pgfqpoint{4.066750in}{5.315328in}}%
\pgfpathlineto{\pgfqpoint{4.088702in}{5.342864in}}%
\pgfpathlineto{\pgfqpoint{4.110654in}{5.390588in}}%
\pgfpathlineto{\pgfqpoint{4.132606in}{5.451704in}}%
\pgfpathlineto{\pgfqpoint{4.154559in}{5.509770in}}%
\pgfpathlineto{\pgfqpoint{4.176511in}{5.551214in}}%
\pgfpathlineto{\pgfqpoint{4.198463in}{5.572295in}}%
\pgfpathlineto{\pgfqpoint{4.220415in}{5.576005in}}%
\pgfpathlineto{\pgfqpoint{4.242367in}{5.567826in}}%
\pgfpathlineto{\pgfqpoint{4.264319in}{5.553920in}}%
\pgfpathlineto{\pgfqpoint{4.286271in}{5.537024in}}%
\pgfpathlineto{\pgfqpoint{4.308223in}{5.514316in}}%
\pgfpathlineto{\pgfqpoint{4.330175in}{5.480045in}}%
\pgfpathlineto{\pgfqpoint{4.352127in}{5.428113in}}%
\pgfpathlineto{\pgfqpoint{4.374079in}{5.355264in}}%
\pgfpathlineto{\pgfqpoint{4.396031in}{5.263222in}}%
\pgfpathlineto{\pgfqpoint{4.417983in}{5.158360in}}%
\pgfpathlineto{\pgfqpoint{4.439935in}{5.048567in}}%
\pgfpathlineto{\pgfqpoint{4.461887in}{4.940092in}}%
\pgfpathlineto{\pgfqpoint{4.483839in}{4.836242in}}%
\pgfpathlineto{\pgfqpoint{4.505791in}{4.738910in}}%
\pgfpathlineto{\pgfqpoint{4.527743in}{4.649953in}}%
\pgfpathlineto{\pgfqpoint{4.549695in}{4.569925in}}%
\pgfpathlineto{\pgfqpoint{4.571647in}{4.497828in}}%
\pgfpathlineto{\pgfqpoint{4.593599in}{4.433895in}}%
\pgfpathlineto{\pgfqpoint{4.615551in}{4.380462in}}%
\pgfpathlineto{\pgfqpoint{4.637503in}{4.339817in}}%
\pgfpathlineto{\pgfqpoint{4.659455in}{4.312690in}}%
\pgfpathlineto{\pgfqpoint{4.681407in}{4.298287in}}%
\pgfpathlineto{\pgfqpoint{4.703359in}{4.294916in}}%
\pgfpathlineto{\pgfqpoint{4.725311in}{4.300147in}}%
\pgfpathlineto{\pgfqpoint{4.747263in}{4.310658in}}%
\pgfpathlineto{\pgfqpoint{4.769215in}{4.323026in}}%
\pgfpathlineto{\pgfqpoint{4.791167in}{4.334896in}}%
\pgfpathlineto{\pgfqpoint{4.813119in}{4.344455in}}%
\pgfpathlineto{\pgfqpoint{4.835071in}{4.350387in}}%
\pgfpathlineto{\pgfqpoint{4.857023in}{4.354301in}}%
\pgfpathlineto{\pgfqpoint{4.878975in}{4.362801in}}%
\pgfpathlineto{\pgfqpoint{4.900927in}{4.384645in}}%
\pgfpathlineto{\pgfqpoint{4.922879in}{4.425162in}}%
\pgfpathlineto{\pgfqpoint{4.944831in}{4.484435in}}%
\pgfpathlineto{\pgfqpoint{4.966783in}{4.559592in}}%
\pgfpathlineto{\pgfqpoint{4.988735in}{4.647665in}}%
\pgfpathlineto{\pgfqpoint{5.010687in}{4.745230in}}%
\pgfpathlineto{\pgfqpoint{5.032639in}{4.847080in}}%
\pgfpathlineto{\pgfqpoint{5.054591in}{4.945770in}}%
\pgfpathlineto{\pgfqpoint{5.076543in}{5.032950in}}%
\pgfpathlineto{\pgfqpoint{5.098495in}{5.101158in}}%
\pgfpathlineto{\pgfqpoint{5.120447in}{5.146720in}}%
\pgfpathlineto{\pgfqpoint{5.142399in}{5.172055in}}%
\pgfpathlineto{\pgfqpoint{5.164351in}{5.183399in}}%
\pgfpathlineto{\pgfqpoint{5.186303in}{5.187330in}}%
\pgfpathlineto{\pgfqpoint{5.208255in}{5.189716in}}%
\pgfpathlineto{\pgfqpoint{5.230207in}{5.194394in}}%
\pgfpathlineto{\pgfqpoint{5.252159in}{5.202357in}}%
\pgfpathlineto{\pgfqpoint{5.274111in}{5.212738in}}%
\pgfpathlineto{\pgfqpoint{5.296063in}{5.223964in}}%
\pgfpathlineto{\pgfqpoint{5.318015in}{5.233759in}}%
\pgfpathlineto{\pgfqpoint{5.339967in}{5.240588in}}%
\pgfpathlineto{\pgfqpoint{5.361919in}{5.244935in}}%
\pgfpathlineto{\pgfqpoint{5.383871in}{5.249086in}}%
\pgfpathlineto{\pgfqpoint{5.405823in}{5.255327in}}%
\pgfpathlineto{\pgfqpoint{5.427775in}{5.263965in}}%
\pgfpathlineto{\pgfqpoint{5.449727in}{5.273084in}}%
\pgfpathlineto{\pgfqpoint{5.471679in}{5.280371in}}%
\pgfpathlineto{\pgfqpoint{5.493631in}{5.284656in}}%
\pgfpathlineto{\pgfqpoint{5.515583in}{5.285357in}}%
\pgfpathlineto{\pgfqpoint{5.537535in}{5.282755in}}%
\pgfpathlineto{\pgfqpoint{5.559488in}{5.278635in}}%
\pgfpathlineto{\pgfqpoint{5.581440in}{5.275571in}}%
\pgfpathlineto{\pgfqpoint{5.603392in}{5.275343in}}%
\pgfpathlineto{\pgfqpoint{5.625344in}{5.277576in}}%
\pgfpathlineto{\pgfqpoint{5.647296in}{5.280615in}}%
\pgfpathlineto{\pgfqpoint{5.669248in}{5.283356in}}%
\pgfpathlineto{\pgfqpoint{5.691200in}{5.287631in}}%
\pgfpathlineto{\pgfqpoint{5.713152in}{5.300438in}}%
\pgfpathlineto{\pgfqpoint{5.735104in}{5.332755in}}%
\pgfpathlineto{\pgfqpoint{5.757056in}{5.389916in}}%
\pgfpathlineto{\pgfqpoint{5.779008in}{5.462341in}}%
\pgfpathlineto{\pgfqpoint{5.800960in}{5.529257in}}%
\pgfpathlineto{\pgfqpoint{5.822912in}{5.573919in}}%
\pgfpathlineto{\pgfqpoint{5.844864in}{5.592957in}}%
\pgfpathlineto{\pgfqpoint{5.866816in}{5.591996in}}%
\pgfpathlineto{\pgfqpoint{5.888768in}{5.577604in}}%
\pgfpathlineto{\pgfqpoint{5.910720in}{5.553920in}}%
\pgfpathlineto{\pgfqpoint{5.932672in}{5.522527in}}%
\pgfpathlineto{\pgfqpoint{5.954624in}{5.481925in}}%
\pgfpathlineto{\pgfqpoint{5.976576in}{5.426886in}}%
\pgfpathlineto{\pgfqpoint{5.998528in}{5.352150in}}%
\pgfpathlineto{\pgfqpoint{6.020480in}{5.257609in}}%
\pgfpathlineto{\pgfqpoint{6.042432in}{5.148363in}}%
\pgfpathlineto{\pgfqpoint{6.064384in}{5.029894in}}%
\pgfpathlineto{\pgfqpoint{6.086336in}{4.906595in}}%
\pgfpathlineto{\pgfqpoint{6.108288in}{4.784311in}}%
\pgfpathlineto{\pgfqpoint{6.130240in}{4.668144in}}%
\pgfusepath{stroke}%
\end{pgfscope}%
\begin{pgfscope}%
\pgfpathrectangle{\pgfqpoint{3.825278in}{3.379757in}}{\pgfqpoint{2.414722in}{2.357859in}}%
\pgfusepath{clip}%
\pgfsetroundcap%
\pgfsetroundjoin%
\pgfsetlinewidth{1.505625pt}%
\definecolor{currentstroke}{rgb}{0.505882,0.447059,0.701961}%
\pgfsetstrokecolor{currentstroke}%
\pgfsetdash{}{0pt}%
\pgfpathmoveto{\pgfqpoint{3.935038in}{5.124921in}}%
\pgfpathlineto{\pgfqpoint{3.958070in}{5.261241in}}%
\pgfpathlineto{\pgfqpoint{3.981102in}{5.385164in}}%
\pgfpathlineto{\pgfqpoint{4.004133in}{5.485540in}}%
\pgfpathlineto{\pgfqpoint{4.027165in}{5.554950in}}%
\pgfpathlineto{\pgfqpoint{4.050196in}{5.587236in}}%
\pgfpathlineto{\pgfqpoint{4.073228in}{5.582887in}}%
\pgfpathlineto{\pgfqpoint{4.096260in}{5.553920in}}%
\pgfpathlineto{\pgfqpoint{4.119291in}{5.515888in}}%
\pgfpathlineto{\pgfqpoint{4.142323in}{5.475979in}}%
\pgfpathlineto{\pgfqpoint{4.165355in}{5.430991in}}%
\pgfpathlineto{\pgfqpoint{4.188386in}{5.371405in}}%
\pgfpathlineto{\pgfqpoint{4.211418in}{5.289771in}}%
\pgfpathlineto{\pgfqpoint{4.234449in}{5.188448in}}%
\pgfpathlineto{\pgfqpoint{4.257481in}{5.077252in}}%
\pgfpathlineto{\pgfqpoint{4.280513in}{4.962618in}}%
\pgfpathlineto{\pgfqpoint{4.303544in}{4.846535in}}%
\pgfpathlineto{\pgfqpoint{4.326576in}{4.730048in}}%
\pgfpathlineto{\pgfqpoint{4.349608in}{4.614602in}}%
\pgfpathlineto{\pgfqpoint{4.372639in}{4.500673in}}%
\pgfpathlineto{\pgfqpoint{4.395671in}{4.390224in}}%
\pgfpathlineto{\pgfqpoint{4.418702in}{4.287907in}}%
\pgfpathlineto{\pgfqpoint{4.441734in}{4.194255in}}%
\pgfpathlineto{\pgfqpoint{4.464766in}{4.110484in}}%
\pgfpathlineto{\pgfqpoint{4.487797in}{4.037814in}}%
\pgfpathlineto{\pgfqpoint{4.510829in}{3.978703in}}%
\pgfpathlineto{\pgfqpoint{4.533861in}{3.938273in}}%
\pgfpathlineto{\pgfqpoint{4.556892in}{3.921490in}}%
\pgfpathlineto{\pgfqpoint{4.579924in}{3.927105in}}%
\pgfpathlineto{\pgfqpoint{4.602955in}{3.944963in}}%
\pgfpathlineto{\pgfqpoint{4.625987in}{3.963006in}}%
\pgfpathlineto{\pgfqpoint{4.649019in}{3.977397in}}%
\pgfpathlineto{\pgfqpoint{4.672050in}{3.993323in}}%
\pgfpathlineto{\pgfqpoint{4.695082in}{4.017545in}}%
\pgfpathlineto{\pgfqpoint{4.718114in}{4.054217in}}%
\pgfpathlineto{\pgfqpoint{4.741145in}{4.111002in}}%
\pgfpathlineto{\pgfqpoint{4.764177in}{4.195412in}}%
\pgfpathlineto{\pgfqpoint{4.787208in}{4.305741in}}%
\pgfpathlineto{\pgfqpoint{4.810240in}{4.432815in}}%
\pgfpathlineto{\pgfqpoint{4.833272in}{4.565843in}}%
\pgfpathlineto{\pgfqpoint{4.856303in}{4.693422in}}%
\pgfpathlineto{\pgfqpoint{4.879335in}{4.805420in}}%
\pgfpathlineto{\pgfqpoint{4.902367in}{4.898275in}}%
\pgfpathlineto{\pgfqpoint{4.925398in}{4.974327in}}%
\pgfpathlineto{\pgfqpoint{4.948430in}{5.036365in}}%
\pgfpathlineto{\pgfqpoint{4.971461in}{5.083110in}}%
\pgfpathlineto{\pgfqpoint{4.994493in}{5.110857in}}%
\pgfpathlineto{\pgfqpoint{5.017525in}{5.119235in}}%
\pgfpathlineto{\pgfqpoint{5.040556in}{5.113195in}}%
\pgfpathlineto{\pgfqpoint{5.063588in}{5.100023in}}%
\pgfpathlineto{\pgfqpoint{5.086619in}{5.086894in}}%
\pgfpathlineto{\pgfqpoint{5.109651in}{5.083561in}}%
\pgfpathlineto{\pgfqpoint{5.132683in}{5.106118in}}%
\pgfpathlineto{\pgfqpoint{5.155714in}{5.170963in}}%
\pgfpathlineto{\pgfqpoint{5.178746in}{5.279191in}}%
\pgfpathlineto{\pgfqpoint{5.201778in}{5.407407in}}%
\pgfpathlineto{\pgfqpoint{5.224809in}{5.521594in}}%
\pgfpathlineto{\pgfqpoint{5.247841in}{5.597742in}}%
\pgfpathlineto{\pgfqpoint{5.270872in}{5.630440in}}%
\pgfpathlineto{\pgfqpoint{5.293904in}{5.626564in}}%
\pgfpathlineto{\pgfqpoint{5.316936in}{5.597493in}}%
\pgfpathlineto{\pgfqpoint{5.339967in}{5.553920in}}%
\pgfpathlineto{\pgfqpoint{5.362999in}{5.502431in}}%
\pgfpathlineto{\pgfqpoint{5.386031in}{5.444201in}}%
\pgfpathlineto{\pgfqpoint{5.409062in}{5.375777in}}%
\pgfpathlineto{\pgfqpoint{5.432094in}{5.292837in}}%
\pgfpathlineto{\pgfqpoint{5.455125in}{5.196184in}}%
\pgfpathlineto{\pgfqpoint{5.478157in}{5.092391in}}%
\pgfusepath{stroke}%
\end{pgfscope}%
\begin{pgfscope}%
\pgfsetrectcap%
\pgfsetmiterjoin%
\pgfsetlinewidth{1.254687pt}%
\definecolor{currentstroke}{rgb}{1.000000,1.000000,1.000000}%
\pgfsetstrokecolor{currentstroke}%
\pgfsetdash{}{0pt}%
\pgfpathmoveto{\pgfqpoint{3.825278in}{3.379757in}}%
\pgfpathlineto{\pgfqpoint{3.825278in}{5.737616in}}%
\pgfusepath{stroke}%
\end{pgfscope}%
\begin{pgfscope}%
\pgfsetrectcap%
\pgfsetmiterjoin%
\pgfsetlinewidth{1.254687pt}%
\definecolor{currentstroke}{rgb}{1.000000,1.000000,1.000000}%
\pgfsetstrokecolor{currentstroke}%
\pgfsetdash{}{0pt}%
\pgfpathmoveto{\pgfqpoint{6.240000in}{3.379757in}}%
\pgfpathlineto{\pgfqpoint{6.240000in}{5.737616in}}%
\pgfusepath{stroke}%
\end{pgfscope}%
\begin{pgfscope}%
\pgfsetrectcap%
\pgfsetmiterjoin%
\pgfsetlinewidth{1.254687pt}%
\definecolor{currentstroke}{rgb}{1.000000,1.000000,1.000000}%
\pgfsetstrokecolor{currentstroke}%
\pgfsetdash{}{0pt}%
\pgfpathmoveto{\pgfqpoint{3.825278in}{3.379757in}}%
\pgfpathlineto{\pgfqpoint{6.240000in}{3.379757in}}%
\pgfusepath{stroke}%
\end{pgfscope}%
\begin{pgfscope}%
\pgfsetrectcap%
\pgfsetmiterjoin%
\pgfsetlinewidth{1.254687pt}%
\definecolor{currentstroke}{rgb}{1.000000,1.000000,1.000000}%
\pgfsetstrokecolor{currentstroke}%
\pgfsetdash{}{0pt}%
\pgfpathmoveto{\pgfqpoint{3.825278in}{5.737616in}}%
\pgfpathlineto{\pgfqpoint{6.240000in}{5.737616in}}%
\pgfusepath{stroke}%
\end{pgfscope}%
\begin{pgfscope}%
\pgfsetbuttcap%
\pgfsetmiterjoin%
\definecolor{currentfill}{rgb}{0.917647,0.917647,0.949020}%
\pgfsetfillcolor{currentfill}%
\pgfsetlinewidth{0.000000pt}%
\definecolor{currentstroke}{rgb}{0.000000,0.000000,0.000000}%
\pgfsetstrokecolor{currentstroke}%
\pgfsetstrokeopacity{0.000000}%
\pgfsetdash{}{0pt}%
\pgfpathmoveto{\pgfqpoint{0.828241in}{0.574768in}}%
\pgfpathlineto{\pgfqpoint{3.242963in}{0.574768in}}%
\pgfpathlineto{\pgfqpoint{3.242963in}{2.932627in}}%
\pgfpathlineto{\pgfqpoint{0.828241in}{2.932627in}}%
\pgfpathclose%
\pgfusepath{fill}%
\end{pgfscope}%
\begin{pgfscope}%
\pgfpathrectangle{\pgfqpoint{0.828241in}{0.574768in}}{\pgfqpoint{2.414722in}{2.357859in}}%
\pgfusepath{clip}%
\pgfsetroundcap%
\pgfsetroundjoin%
\pgfsetlinewidth{1.003750pt}%
\definecolor{currentstroke}{rgb}{1.000000,1.000000,1.000000}%
\pgfsetstrokecolor{currentstroke}%
\pgfsetdash{}{0pt}%
\pgfpathmoveto{\pgfqpoint{0.938001in}{0.574768in}}%
\pgfpathlineto{\pgfqpoint{0.938001in}{2.932627in}}%
\pgfusepath{stroke}%
\end{pgfscope}%
\begin{pgfscope}%
\definecolor{textcolor}{rgb}{0.150000,0.150000,0.150000}%
\pgfsetstrokecolor{textcolor}%
\pgfsetfillcolor{textcolor}%
\pgftext[x=0.938001in,y=0.442824in,,top]{\color{textcolor}\sffamily\fontsize{11.000000}{13.200000}\selectfont \(\displaystyle 0.0\)}%
\end{pgfscope}%
\begin{pgfscope}%
\pgfpathrectangle{\pgfqpoint{0.828241in}{0.574768in}}{\pgfqpoint{2.414722in}{2.357859in}}%
\pgfusepath{clip}%
\pgfsetroundcap%
\pgfsetroundjoin%
\pgfsetlinewidth{1.003750pt}%
\definecolor{currentstroke}{rgb}{1.000000,1.000000,1.000000}%
\pgfsetstrokecolor{currentstroke}%
\pgfsetdash{}{0pt}%
\pgfpathmoveto{\pgfqpoint{1.787335in}{0.574768in}}%
\pgfpathlineto{\pgfqpoint{1.787335in}{2.932627in}}%
\pgfusepath{stroke}%
\end{pgfscope}%
\begin{pgfscope}%
\definecolor{textcolor}{rgb}{0.150000,0.150000,0.150000}%
\pgfsetstrokecolor{textcolor}%
\pgfsetfillcolor{textcolor}%
\pgftext[x=1.787335in,y=0.442824in,,top]{\color{textcolor}\sffamily\fontsize{11.000000}{13.200000}\selectfont \(\displaystyle 0.5\)}%
\end{pgfscope}%
\begin{pgfscope}%
\pgfpathrectangle{\pgfqpoint{0.828241in}{0.574768in}}{\pgfqpoint{2.414722in}{2.357859in}}%
\pgfusepath{clip}%
\pgfsetroundcap%
\pgfsetroundjoin%
\pgfsetlinewidth{1.003750pt}%
\definecolor{currentstroke}{rgb}{1.000000,1.000000,1.000000}%
\pgfsetstrokecolor{currentstroke}%
\pgfsetdash{}{0pt}%
\pgfpathmoveto{\pgfqpoint{2.636669in}{0.574768in}}%
\pgfpathlineto{\pgfqpoint{2.636669in}{2.932627in}}%
\pgfusepath{stroke}%
\end{pgfscope}%
\begin{pgfscope}%
\definecolor{textcolor}{rgb}{0.150000,0.150000,0.150000}%
\pgfsetstrokecolor{textcolor}%
\pgfsetfillcolor{textcolor}%
\pgftext[x=2.636669in,y=0.442824in,,top]{\color{textcolor}\sffamily\fontsize{11.000000}{13.200000}\selectfont \(\displaystyle 1.0\)}%
\end{pgfscope}%
\begin{pgfscope}%
\definecolor{textcolor}{rgb}{0.150000,0.150000,0.150000}%
\pgfsetstrokecolor{textcolor}%
\pgfsetfillcolor{textcolor}%
\pgftext[x=2.035602in,y=0.252083in,,top]{\color{textcolor}\sffamily\fontsize{11.000000}{13.200000}\selectfont Time [s]}%
\end{pgfscope}%
\begin{pgfscope}%
\pgfpathrectangle{\pgfqpoint{0.828241in}{0.574768in}}{\pgfqpoint{2.414722in}{2.357859in}}%
\pgfusepath{clip}%
\pgfsetroundcap%
\pgfsetroundjoin%
\pgfsetlinewidth{1.003750pt}%
\definecolor{currentstroke}{rgb}{1.000000,1.000000,1.000000}%
\pgfsetstrokecolor{currentstroke}%
\pgfsetdash{}{0pt}%
\pgfpathmoveto{\pgfqpoint{0.828241in}{0.591647in}}%
\pgfpathlineto{\pgfqpoint{3.242963in}{0.591647in}}%
\pgfusepath{stroke}%
\end{pgfscope}%
\begin{pgfscope}%
\definecolor{textcolor}{rgb}{0.150000,0.150000,0.150000}%
\pgfsetstrokecolor{textcolor}%
\pgfsetfillcolor{textcolor}%
\pgftext[x=0.307639in,y=0.538841in,left,base]{\color{textcolor}\sffamily\fontsize{11.000000}{13.200000}\selectfont \(\displaystyle -15.0\)}%
\end{pgfscope}%
\begin{pgfscope}%
\pgfpathrectangle{\pgfqpoint{0.828241in}{0.574768in}}{\pgfqpoint{2.414722in}{2.357859in}}%
\pgfusepath{clip}%
\pgfsetroundcap%
\pgfsetroundjoin%
\pgfsetlinewidth{1.003750pt}%
\definecolor{currentstroke}{rgb}{1.000000,1.000000,1.000000}%
\pgfsetstrokecolor{currentstroke}%
\pgfsetdash{}{0pt}%
\pgfpathmoveto{\pgfqpoint{0.828241in}{0.949308in}}%
\pgfpathlineto{\pgfqpoint{3.242963in}{0.949308in}}%
\pgfusepath{stroke}%
\end{pgfscope}%
\begin{pgfscope}%
\definecolor{textcolor}{rgb}{0.150000,0.150000,0.150000}%
\pgfsetstrokecolor{textcolor}%
\pgfsetfillcolor{textcolor}%
\pgftext[x=0.307639in,y=0.896502in,left,base]{\color{textcolor}\sffamily\fontsize{11.000000}{13.200000}\selectfont \(\displaystyle -12.5\)}%
\end{pgfscope}%
\begin{pgfscope}%
\pgfpathrectangle{\pgfqpoint{0.828241in}{0.574768in}}{\pgfqpoint{2.414722in}{2.357859in}}%
\pgfusepath{clip}%
\pgfsetroundcap%
\pgfsetroundjoin%
\pgfsetlinewidth{1.003750pt}%
\definecolor{currentstroke}{rgb}{1.000000,1.000000,1.000000}%
\pgfsetstrokecolor{currentstroke}%
\pgfsetdash{}{0pt}%
\pgfpathmoveto{\pgfqpoint{0.828241in}{1.306969in}}%
\pgfpathlineto{\pgfqpoint{3.242963in}{1.306969in}}%
\pgfusepath{stroke}%
\end{pgfscope}%
\begin{pgfscope}%
\definecolor{textcolor}{rgb}{0.150000,0.150000,0.150000}%
\pgfsetstrokecolor{textcolor}%
\pgfsetfillcolor{textcolor}%
\pgftext[x=0.307639in,y=1.254163in,left,base]{\color{textcolor}\sffamily\fontsize{11.000000}{13.200000}\selectfont \(\displaystyle -10.0\)}%
\end{pgfscope}%
\begin{pgfscope}%
\pgfpathrectangle{\pgfqpoint{0.828241in}{0.574768in}}{\pgfqpoint{2.414722in}{2.357859in}}%
\pgfusepath{clip}%
\pgfsetroundcap%
\pgfsetroundjoin%
\pgfsetlinewidth{1.003750pt}%
\definecolor{currentstroke}{rgb}{1.000000,1.000000,1.000000}%
\pgfsetstrokecolor{currentstroke}%
\pgfsetdash{}{0pt}%
\pgfpathmoveto{\pgfqpoint{0.828241in}{1.664630in}}%
\pgfpathlineto{\pgfqpoint{3.242963in}{1.664630in}}%
\pgfusepath{stroke}%
\end{pgfscope}%
\begin{pgfscope}%
\definecolor{textcolor}{rgb}{0.150000,0.150000,0.150000}%
\pgfsetstrokecolor{textcolor}%
\pgfsetfillcolor{textcolor}%
\pgftext[x=0.383680in,y=1.611824in,left,base]{\color{textcolor}\sffamily\fontsize{11.000000}{13.200000}\selectfont \(\displaystyle -7.5\)}%
\end{pgfscope}%
\begin{pgfscope}%
\pgfpathrectangle{\pgfqpoint{0.828241in}{0.574768in}}{\pgfqpoint{2.414722in}{2.357859in}}%
\pgfusepath{clip}%
\pgfsetroundcap%
\pgfsetroundjoin%
\pgfsetlinewidth{1.003750pt}%
\definecolor{currentstroke}{rgb}{1.000000,1.000000,1.000000}%
\pgfsetstrokecolor{currentstroke}%
\pgfsetdash{}{0pt}%
\pgfpathmoveto{\pgfqpoint{0.828241in}{2.022291in}}%
\pgfpathlineto{\pgfqpoint{3.242963in}{2.022291in}}%
\pgfusepath{stroke}%
\end{pgfscope}%
\begin{pgfscope}%
\definecolor{textcolor}{rgb}{0.150000,0.150000,0.150000}%
\pgfsetstrokecolor{textcolor}%
\pgfsetfillcolor{textcolor}%
\pgftext[x=0.383680in,y=1.969485in,left,base]{\color{textcolor}\sffamily\fontsize{11.000000}{13.200000}\selectfont \(\displaystyle -5.0\)}%
\end{pgfscope}%
\begin{pgfscope}%
\pgfpathrectangle{\pgfqpoint{0.828241in}{0.574768in}}{\pgfqpoint{2.414722in}{2.357859in}}%
\pgfusepath{clip}%
\pgfsetroundcap%
\pgfsetroundjoin%
\pgfsetlinewidth{1.003750pt}%
\definecolor{currentstroke}{rgb}{1.000000,1.000000,1.000000}%
\pgfsetstrokecolor{currentstroke}%
\pgfsetdash{}{0pt}%
\pgfpathmoveto{\pgfqpoint{0.828241in}{2.379952in}}%
\pgfpathlineto{\pgfqpoint{3.242963in}{2.379952in}}%
\pgfusepath{stroke}%
\end{pgfscope}%
\begin{pgfscope}%
\definecolor{textcolor}{rgb}{0.150000,0.150000,0.150000}%
\pgfsetstrokecolor{textcolor}%
\pgfsetfillcolor{textcolor}%
\pgftext[x=0.383680in,y=2.327146in,left,base]{\color{textcolor}\sffamily\fontsize{11.000000}{13.200000}\selectfont \(\displaystyle -2.5\)}%
\end{pgfscope}%
\begin{pgfscope}%
\pgfpathrectangle{\pgfqpoint{0.828241in}{0.574768in}}{\pgfqpoint{2.414722in}{2.357859in}}%
\pgfusepath{clip}%
\pgfsetroundcap%
\pgfsetroundjoin%
\pgfsetlinewidth{1.003750pt}%
\definecolor{currentstroke}{rgb}{1.000000,1.000000,1.000000}%
\pgfsetstrokecolor{currentstroke}%
\pgfsetdash{}{0pt}%
\pgfpathmoveto{\pgfqpoint{0.828241in}{2.737613in}}%
\pgfpathlineto{\pgfqpoint{3.242963in}{2.737613in}}%
\pgfusepath{stroke}%
\end{pgfscope}%
\begin{pgfscope}%
\definecolor{textcolor}{rgb}{0.150000,0.150000,0.150000}%
\pgfsetstrokecolor{textcolor}%
\pgfsetfillcolor{textcolor}%
\pgftext[x=0.501968in,y=2.684807in,left,base]{\color{textcolor}\sffamily\fontsize{11.000000}{13.200000}\selectfont \(\displaystyle 0.0\)}%
\end{pgfscope}%
\begin{pgfscope}%
\definecolor{textcolor}{rgb}{0.150000,0.150000,0.150000}%
\pgfsetstrokecolor{textcolor}%
\pgfsetfillcolor{textcolor}%
\pgftext[x=0.252083in,y=1.753698in,,bottom,rotate=90.000000]{\color{textcolor}\sffamily\fontsize{11.000000}{13.200000}\selectfont APLAX/gls}%
\end{pgfscope}%
\begin{pgfscope}%
\pgfpathrectangle{\pgfqpoint{0.828241in}{0.574768in}}{\pgfqpoint{2.414722in}{2.357859in}}%
\pgfusepath{clip}%
\pgfsetroundcap%
\pgfsetroundjoin%
\pgfsetlinewidth{1.505625pt}%
\definecolor{currentstroke}{rgb}{0.298039,0.447059,0.690196}%
\pgfsetstrokecolor{currentstroke}%
\pgfsetdash{}{0pt}%
\pgfpathmoveto{\pgfqpoint{0.938001in}{1.848014in}}%
\pgfpathlineto{\pgfqpoint{0.964135in}{1.887173in}}%
\pgfpathlineto{\pgfqpoint{0.990268in}{1.963865in}}%
\pgfpathlineto{\pgfqpoint{1.016401in}{2.100659in}}%
\pgfpathlineto{\pgfqpoint{1.042535in}{2.284777in}}%
\pgfpathlineto{\pgfqpoint{1.068668in}{2.471666in}}%
\pgfpathlineto{\pgfqpoint{1.094801in}{2.613499in}}%
\pgfpathlineto{\pgfqpoint{1.120935in}{2.691464in}}%
\pgfpathlineto{\pgfqpoint{1.147068in}{2.720469in}}%
\pgfpathlineto{\pgfqpoint{1.173201in}{2.729155in}}%
\pgfpathlineto{\pgfqpoint{1.199335in}{2.737613in}}%
\pgfpathlineto{\pgfqpoint{1.225468in}{2.749788in}}%
\pgfpathlineto{\pgfqpoint{1.251602in}{2.759411in}}%
\pgfpathlineto{\pgfqpoint{1.277735in}{2.758272in}}%
\pgfpathlineto{\pgfqpoint{1.303868in}{2.737804in}}%
\pgfpathlineto{\pgfqpoint{1.330002in}{2.688670in}}%
\pgfpathlineto{\pgfqpoint{1.356135in}{2.604964in}}%
\pgfpathlineto{\pgfqpoint{1.382268in}{2.488723in}}%
\pgfpathlineto{\pgfqpoint{1.408402in}{2.348875in}}%
\pgfpathlineto{\pgfqpoint{1.434535in}{2.196372in}}%
\pgfpathlineto{\pgfqpoint{1.460668in}{2.037786in}}%
\pgfpathlineto{\pgfqpoint{1.486802in}{1.873633in}}%
\pgfpathlineto{\pgfqpoint{1.512935in}{1.703132in}}%
\pgfpathlineto{\pgfqpoint{1.539068in}{1.529329in}}%
\pgfpathlineto{\pgfqpoint{1.565202in}{1.358930in}}%
\pgfpathlineto{\pgfqpoint{1.591335in}{1.198362in}}%
\pgfpathlineto{\pgfqpoint{1.617468in}{1.052423in}}%
\pgfpathlineto{\pgfqpoint{1.643602in}{0.925447in}}%
\pgfpathlineto{\pgfqpoint{1.669735in}{0.820194in}}%
\pgfpathlineto{\pgfqpoint{1.695869in}{0.740318in}}%
\pgfpathlineto{\pgfqpoint{1.722002in}{0.692294in}}%
\pgfpathlineto{\pgfqpoint{1.748135in}{0.681944in}}%
\pgfpathlineto{\pgfqpoint{1.774269in}{0.707189in}}%
\pgfpathlineto{\pgfqpoint{1.800402in}{0.755123in}}%
\pgfpathlineto{\pgfqpoint{1.826535in}{0.807765in}}%
\pgfpathlineto{\pgfqpoint{1.852669in}{0.849046in}}%
\pgfpathlineto{\pgfqpoint{1.878802in}{0.870351in}}%
\pgfpathlineto{\pgfqpoint{1.904935in}{0.876511in}}%
\pgfpathlineto{\pgfqpoint{1.931069in}{0.886417in}}%
\pgfpathlineto{\pgfqpoint{1.957202in}{0.920951in}}%
\pgfpathlineto{\pgfqpoint{1.983335in}{0.992813in}}%
\pgfpathlineto{\pgfqpoint{2.009469in}{1.099796in}}%
\pgfpathlineto{\pgfqpoint{2.035602in}{1.226671in}}%
\pgfpathlineto{\pgfqpoint{2.061735in}{1.353648in}}%
\pgfpathlineto{\pgfqpoint{2.087869in}{1.466534in}}%
\pgfpathlineto{\pgfqpoint{2.114002in}{1.560280in}}%
\pgfpathlineto{\pgfqpoint{2.140136in}{1.636343in}}%
\pgfpathlineto{\pgfqpoint{2.166269in}{1.698248in}}%
\pgfpathlineto{\pgfqpoint{2.192402in}{1.748317in}}%
\pgfpathlineto{\pgfqpoint{2.218536in}{1.787080in}}%
\pgfpathlineto{\pgfqpoint{2.244669in}{1.813467in}}%
\pgfpathlineto{\pgfqpoint{2.270802in}{1.827636in}}%
\pgfpathlineto{\pgfqpoint{2.296936in}{1.837170in}}%
\pgfpathlineto{\pgfqpoint{2.323069in}{1.861296in}}%
\pgfpathlineto{\pgfqpoint{2.349202in}{1.928443in}}%
\pgfpathlineto{\pgfqpoint{2.375336in}{2.060477in}}%
\pgfpathlineto{\pgfqpoint{2.401469in}{2.248211in}}%
\pgfpathlineto{\pgfqpoint{2.427602in}{2.445947in}}%
\pgfpathlineto{\pgfqpoint{2.453736in}{2.600793in}}%
\pgfpathlineto{\pgfqpoint{2.479869in}{2.688672in}}%
\pgfpathlineto{\pgfqpoint{2.506002in}{2.722733in}}%
\pgfpathlineto{\pgfqpoint{2.532136in}{2.731696in}}%
\pgfpathlineto{\pgfqpoint{2.558269in}{2.737613in}}%
\pgfpathlineto{\pgfqpoint{2.584403in}{2.749858in}}%
\pgfpathlineto{\pgfqpoint{2.610536in}{2.765426in}}%
\pgfpathlineto{\pgfqpoint{2.636669in}{2.771576in}}%
\pgfpathlineto{\pgfqpoint{2.662803in}{2.752409in}}%
\pgfpathlineto{\pgfqpoint{2.688936in}{2.696724in}}%
\pgfpathlineto{\pgfqpoint{2.715069in}{2.605924in}}%
\pgfpathlineto{\pgfqpoint{2.741203in}{2.490240in}}%
\pgfpathlineto{\pgfqpoint{2.767336in}{2.358659in}}%
\pgfpathlineto{\pgfqpoint{2.793469in}{2.218516in}}%
\pgfusepath{stroke}%
\end{pgfscope}%
\begin{pgfscope}%
\pgfpathrectangle{\pgfqpoint{0.828241in}{0.574768in}}{\pgfqpoint{2.414722in}{2.357859in}}%
\pgfusepath{clip}%
\pgfsetroundcap%
\pgfsetroundjoin%
\pgfsetlinewidth{1.505625pt}%
\definecolor{currentstroke}{rgb}{0.866667,0.517647,0.321569}%
\pgfsetstrokecolor{currentstroke}%
\pgfsetdash{}{0pt}%
\pgfpathmoveto{\pgfqpoint{0.938001in}{2.294479in}}%
\pgfpathlineto{\pgfqpoint{0.965848in}{2.307680in}}%
\pgfpathlineto{\pgfqpoint{0.993695in}{2.324701in}}%
\pgfpathlineto{\pgfqpoint{1.021542in}{2.353736in}}%
\pgfpathlineto{\pgfqpoint{1.049389in}{2.402050in}}%
\pgfpathlineto{\pgfqpoint{1.077236in}{2.467995in}}%
\pgfpathlineto{\pgfqpoint{1.105083in}{2.541025in}}%
\pgfpathlineto{\pgfqpoint{1.132930in}{2.609569in}}%
\pgfpathlineto{\pgfqpoint{1.160777in}{2.665703in}}%
\pgfpathlineto{\pgfqpoint{1.188624in}{2.705936in}}%
\pgfpathlineto{\pgfqpoint{1.216471in}{2.729661in}}%
\pgfpathlineto{\pgfqpoint{1.244318in}{2.737613in}}%
\pgfpathlineto{\pgfqpoint{1.272166in}{2.732897in}}%
\pgfpathlineto{\pgfqpoint{1.300013in}{2.723949in}}%
\pgfpathlineto{\pgfqpoint{1.327860in}{2.722918in}}%
\pgfpathlineto{\pgfqpoint{1.355707in}{2.736927in}}%
\pgfpathlineto{\pgfqpoint{1.383554in}{2.762861in}}%
\pgfpathlineto{\pgfqpoint{1.411401in}{2.791690in}}%
\pgfpathlineto{\pgfqpoint{1.439248in}{2.814851in}}%
\pgfpathlineto{\pgfqpoint{1.467095in}{2.825452in}}%
\pgfpathlineto{\pgfqpoint{1.494942in}{2.817354in}}%
\pgfpathlineto{\pgfqpoint{1.522789in}{2.787462in}}%
\pgfpathlineto{\pgfqpoint{1.550636in}{2.737760in}}%
\pgfpathlineto{\pgfqpoint{1.578483in}{2.674467in}}%
\pgfpathlineto{\pgfqpoint{1.606330in}{2.603438in}}%
\pgfpathlineto{\pgfqpoint{1.634177in}{2.529585in}}%
\pgfpathlineto{\pgfqpoint{1.662024in}{2.458884in}}%
\pgfpathlineto{\pgfqpoint{1.689871in}{2.398033in}}%
\pgfpathlineto{\pgfqpoint{1.717718in}{2.349091in}}%
\pgfpathlineto{\pgfqpoint{1.745565in}{2.310822in}}%
\pgfpathlineto{\pgfqpoint{1.773412in}{2.285176in}}%
\pgfpathlineto{\pgfqpoint{1.801259in}{2.278017in}}%
\pgfpathlineto{\pgfqpoint{1.829106in}{2.290633in}}%
\pgfpathlineto{\pgfqpoint{1.856953in}{2.313165in}}%
\pgfpathlineto{\pgfqpoint{1.884800in}{2.329022in}}%
\pgfpathlineto{\pgfqpoint{1.912647in}{2.322496in}}%
\pgfpathlineto{\pgfqpoint{1.940494in}{2.283410in}}%
\pgfpathlineto{\pgfqpoint{1.968341in}{2.212489in}}%
\pgfpathlineto{\pgfqpoint{1.996188in}{2.125126in}}%
\pgfpathlineto{\pgfqpoint{2.024035in}{2.047460in}}%
\pgfpathlineto{\pgfqpoint{2.051882in}{2.005180in}}%
\pgfpathlineto{\pgfqpoint{2.079729in}{2.010960in}}%
\pgfpathlineto{\pgfqpoint{2.107576in}{2.059791in}}%
\pgfpathlineto{\pgfqpoint{2.135423in}{2.133631in}}%
\pgfpathlineto{\pgfqpoint{2.163270in}{2.211171in}}%
\pgfpathlineto{\pgfqpoint{2.191117in}{2.275211in}}%
\pgfpathlineto{\pgfqpoint{2.218964in}{2.316898in}}%
\pgfpathlineto{\pgfqpoint{2.246811in}{2.336796in}}%
\pgfpathlineto{\pgfqpoint{2.274658in}{2.342576in}}%
\pgfpathlineto{\pgfqpoint{2.302505in}{2.343203in}}%
\pgfpathlineto{\pgfqpoint{2.330352in}{2.343115in}}%
\pgfpathlineto{\pgfqpoint{2.358199in}{2.341872in}}%
\pgfpathlineto{\pgfqpoint{2.386046in}{2.339037in}}%
\pgfpathlineto{\pgfqpoint{2.413893in}{2.336463in}}%
\pgfpathlineto{\pgfqpoint{2.441740in}{2.336416in}}%
\pgfpathlineto{\pgfqpoint{2.469587in}{2.339822in}}%
\pgfpathlineto{\pgfqpoint{2.497434in}{2.348039in}}%
\pgfpathlineto{\pgfqpoint{2.525281in}{2.364973in}}%
\pgfpathlineto{\pgfqpoint{2.553128in}{2.400229in}}%
\pgfpathlineto{\pgfqpoint{2.580975in}{2.462378in}}%
\pgfpathlineto{\pgfqpoint{2.608822in}{2.545224in}}%
\pgfpathlineto{\pgfqpoint{2.636669in}{2.627719in}}%
\pgfpathlineto{\pgfqpoint{2.664516in}{2.690920in}}%
\pgfpathlineto{\pgfqpoint{2.692363in}{2.729928in}}%
\pgfpathlineto{\pgfqpoint{2.720210in}{2.748322in}}%
\pgfpathlineto{\pgfqpoint{2.748057in}{2.750089in}}%
\pgfpathlineto{\pgfqpoint{2.775904in}{2.737613in}}%
\pgfpathlineto{\pgfqpoint{2.803751in}{2.716486in}}%
\pgfpathlineto{\pgfqpoint{2.831598in}{2.697868in}}%
\pgfpathlineto{\pgfqpoint{2.859445in}{2.693203in}}%
\pgfpathlineto{\pgfqpoint{2.887292in}{2.707293in}}%
\pgfpathlineto{\pgfqpoint{2.915139in}{2.737471in}}%
\pgfpathlineto{\pgfqpoint{2.942986in}{2.774755in}}%
\pgfpathlineto{\pgfqpoint{2.970833in}{2.805809in}}%
\pgfpathlineto{\pgfqpoint{2.998680in}{2.816277in}}%
\pgfpathlineto{\pgfqpoint{3.026527in}{2.795592in}}%
\pgfpathlineto{\pgfqpoint{3.054374in}{2.743979in}}%
\pgfpathlineto{\pgfqpoint{3.082221in}{2.675001in}}%
\pgfusepath{stroke}%
\end{pgfscope}%
\begin{pgfscope}%
\pgfpathrectangle{\pgfqpoint{0.828241in}{0.574768in}}{\pgfqpoint{2.414722in}{2.357859in}}%
\pgfusepath{clip}%
\pgfsetroundcap%
\pgfsetroundjoin%
\pgfsetlinewidth{1.505625pt}%
\definecolor{currentstroke}{rgb}{0.333333,0.658824,0.407843}%
\pgfsetstrokecolor{currentstroke}%
\pgfsetdash{}{0pt}%
\pgfpathmoveto{\pgfqpoint{0.938001in}{2.136340in}}%
\pgfpathlineto{\pgfqpoint{0.964543in}{2.106410in}}%
\pgfpathlineto{\pgfqpoint{0.991085in}{2.077615in}}%
\pgfpathlineto{\pgfqpoint{1.017626in}{2.070933in}}%
\pgfpathlineto{\pgfqpoint{1.044168in}{2.117754in}}%
\pgfpathlineto{\pgfqpoint{1.070710in}{2.240004in}}%
\pgfpathlineto{\pgfqpoint{1.097251in}{2.423792in}}%
\pgfpathlineto{\pgfqpoint{1.123793in}{2.609910in}}%
\pgfpathlineto{\pgfqpoint{1.150335in}{2.737613in}}%
\pgfpathlineto{\pgfqpoint{1.176877in}{2.783258in}}%
\pgfpathlineto{\pgfqpoint{1.203418in}{2.762976in}}%
\pgfpathlineto{\pgfqpoint{1.229960in}{2.719507in}}%
\pgfpathlineto{\pgfqpoint{1.256502in}{2.691332in}}%
\pgfpathlineto{\pgfqpoint{1.283043in}{2.684642in}}%
\pgfpathlineto{\pgfqpoint{1.309585in}{2.674415in}}%
\pgfpathlineto{\pgfqpoint{1.336127in}{2.631961in}}%
\pgfpathlineto{\pgfqpoint{1.362668in}{2.551332in}}%
\pgfpathlineto{\pgfqpoint{1.389210in}{2.452009in}}%
\pgfpathlineto{\pgfqpoint{1.415752in}{2.355172in}}%
\pgfpathlineto{\pgfqpoint{1.442293in}{2.265838in}}%
\pgfpathlineto{\pgfqpoint{1.468835in}{2.179884in}}%
\pgfpathlineto{\pgfqpoint{1.495377in}{2.096896in}}%
\pgfpathlineto{\pgfqpoint{1.521918in}{2.021371in}}%
\pgfpathlineto{\pgfqpoint{1.548460in}{1.957853in}}%
\pgfpathlineto{\pgfqpoint{1.575002in}{1.911397in}}%
\pgfpathlineto{\pgfqpoint{1.601543in}{1.884504in}}%
\pgfpathlineto{\pgfqpoint{1.628085in}{1.873265in}}%
\pgfpathlineto{\pgfqpoint{1.654627in}{1.865673in}}%
\pgfpathlineto{\pgfqpoint{1.681169in}{1.848346in}}%
\pgfpathlineto{\pgfqpoint{1.707710in}{1.819545in}}%
\pgfpathlineto{\pgfqpoint{1.734252in}{1.791223in}}%
\pgfpathlineto{\pgfqpoint{1.760794in}{1.773897in}}%
\pgfpathlineto{\pgfqpoint{1.787335in}{1.762400in}}%
\pgfpathlineto{\pgfqpoint{1.813877in}{1.749633in}}%
\pgfpathlineto{\pgfqpoint{1.840419in}{1.742674in}}%
\pgfpathlineto{\pgfqpoint{1.866960in}{1.760650in}}%
\pgfpathlineto{\pgfqpoint{1.893502in}{1.820801in}}%
\pgfpathlineto{\pgfqpoint{1.920044in}{1.916947in}}%
\pgfpathlineto{\pgfqpoint{1.946585in}{2.016103in}}%
\pgfpathlineto{\pgfqpoint{1.973127in}{2.089337in}}%
\pgfpathlineto{\pgfqpoint{1.999669in}{2.131171in}}%
\pgfpathlineto{\pgfqpoint{2.026210in}{2.152805in}}%
\pgfpathlineto{\pgfqpoint{2.052752in}{2.165241in}}%
\pgfpathlineto{\pgfqpoint{2.079294in}{2.170336in}}%
\pgfpathlineto{\pgfqpoint{2.105836in}{2.165958in}}%
\pgfpathlineto{\pgfqpoint{2.132377in}{2.151207in}}%
\pgfpathlineto{\pgfqpoint{2.158919in}{2.136303in}}%
\pgfpathlineto{\pgfqpoint{2.185461in}{2.148750in}}%
\pgfpathlineto{\pgfqpoint{2.212002in}{2.219176in}}%
\pgfpathlineto{\pgfqpoint{2.238544in}{2.353923in}}%
\pgfpathlineto{\pgfqpoint{2.265086in}{2.521078in}}%
\pgfpathlineto{\pgfqpoint{2.291627in}{2.664905in}}%
\pgfpathlineto{\pgfqpoint{2.318169in}{2.737613in}}%
\pgfpathlineto{\pgfqpoint{2.344711in}{2.726446in}}%
\pgfpathlineto{\pgfqpoint{2.371252in}{2.665450in}}%
\pgfpathlineto{\pgfqpoint{2.397794in}{2.606091in}}%
\pgfpathlineto{\pgfqpoint{2.424336in}{2.575167in}}%
\pgfpathlineto{\pgfqpoint{2.450877in}{2.563231in}}%
\pgfpathlineto{\pgfqpoint{2.477419in}{2.543465in}}%
\pgfpathlineto{\pgfqpoint{2.503961in}{2.497334in}}%
\pgfusepath{stroke}%
\end{pgfscope}%
\begin{pgfscope}%
\pgfpathrectangle{\pgfqpoint{0.828241in}{0.574768in}}{\pgfqpoint{2.414722in}{2.357859in}}%
\pgfusepath{clip}%
\pgfsetroundcap%
\pgfsetroundjoin%
\pgfsetlinewidth{1.505625pt}%
\definecolor{currentstroke}{rgb}{0.768627,0.305882,0.321569}%
\pgfsetstrokecolor{currentstroke}%
\pgfsetdash{}{0pt}%
\pgfpathmoveto{\pgfqpoint{0.938001in}{2.337728in}}%
\pgfpathlineto{\pgfqpoint{0.964135in}{2.339310in}}%
\pgfpathlineto{\pgfqpoint{0.990268in}{2.342933in}}%
\pgfpathlineto{\pgfqpoint{1.016401in}{2.350773in}}%
\pgfpathlineto{\pgfqpoint{1.042535in}{2.367237in}}%
\pgfpathlineto{\pgfqpoint{1.068668in}{2.399273in}}%
\pgfpathlineto{\pgfqpoint{1.094801in}{2.453830in}}%
\pgfpathlineto{\pgfqpoint{1.120935in}{2.529314in}}%
\pgfpathlineto{\pgfqpoint{1.147068in}{2.609417in}}%
\pgfpathlineto{\pgfqpoint{1.173201in}{2.673000in}}%
\pgfpathlineto{\pgfqpoint{1.199335in}{2.709809in}}%
\pgfpathlineto{\pgfqpoint{1.225468in}{2.725255in}}%
\pgfpathlineto{\pgfqpoint{1.251602in}{2.730862in}}%
\pgfpathlineto{\pgfqpoint{1.277735in}{2.734462in}}%
\pgfpathlineto{\pgfqpoint{1.303868in}{2.737613in}}%
\pgfpathlineto{\pgfqpoint{1.330002in}{2.740198in}}%
\pgfpathlineto{\pgfqpoint{1.356135in}{2.745777in}}%
\pgfpathlineto{\pgfqpoint{1.382268in}{2.758966in}}%
\pgfpathlineto{\pgfqpoint{1.408402in}{2.777840in}}%
\pgfpathlineto{\pgfqpoint{1.434535in}{2.793569in}}%
\pgfpathlineto{\pgfqpoint{1.460668in}{2.794159in}}%
\pgfpathlineto{\pgfqpoint{1.486802in}{2.769962in}}%
\pgfpathlineto{\pgfqpoint{1.512935in}{2.720982in}}%
\pgfpathlineto{\pgfqpoint{1.539068in}{2.657320in}}%
\pgfpathlineto{\pgfqpoint{1.565202in}{2.588590in}}%
\pgfpathlineto{\pgfqpoint{1.591335in}{2.518051in}}%
\pgfpathlineto{\pgfqpoint{1.617468in}{2.445018in}}%
\pgfpathlineto{\pgfqpoint{1.643602in}{2.368887in}}%
\pgfpathlineto{\pgfqpoint{1.669735in}{2.291009in}}%
\pgfpathlineto{\pgfqpoint{1.695869in}{2.213615in}}%
\pgfpathlineto{\pgfqpoint{1.722002in}{2.137682in}}%
\pgfpathlineto{\pgfqpoint{1.748135in}{2.063654in}}%
\pgfpathlineto{\pgfqpoint{1.774269in}{1.994411in}}%
\pgfpathlineto{\pgfqpoint{1.800402in}{1.933384in}}%
\pgfpathlineto{\pgfqpoint{1.826535in}{1.884065in}}%
\pgfpathlineto{\pgfqpoint{1.852669in}{1.849312in}}%
\pgfpathlineto{\pgfqpoint{1.878802in}{1.829684in}}%
\pgfpathlineto{\pgfqpoint{1.904935in}{1.822008in}}%
\pgfpathlineto{\pgfqpoint{1.931069in}{1.820786in}}%
\pgfpathlineto{\pgfqpoint{1.957202in}{1.821566in}}%
\pgfpathlineto{\pgfqpoint{1.983335in}{1.821258in}}%
\pgfpathlineto{\pgfqpoint{2.009469in}{1.816727in}}%
\pgfpathlineto{\pgfqpoint{2.035602in}{1.804539in}}%
\pgfpathlineto{\pgfqpoint{2.061735in}{1.785784in}}%
\pgfpathlineto{\pgfqpoint{2.087869in}{1.769216in}}%
\pgfpathlineto{\pgfqpoint{2.114002in}{1.764785in}}%
\pgfpathlineto{\pgfqpoint{2.140136in}{1.775805in}}%
\pgfpathlineto{\pgfqpoint{2.166269in}{1.799204in}}%
\pgfpathlineto{\pgfqpoint{2.192402in}{1.830926in}}%
\pgfpathlineto{\pgfqpoint{2.218536in}{1.870244in}}%
\pgfpathlineto{\pgfqpoint{2.244669in}{1.920246in}}%
\pgfpathlineto{\pgfqpoint{2.270802in}{1.982637in}}%
\pgfpathlineto{\pgfqpoint{2.296936in}{2.053343in}}%
\pgfpathlineto{\pgfqpoint{2.323069in}{2.123295in}}%
\pgfpathlineto{\pgfqpoint{2.349202in}{2.183237in}}%
\pgfpathlineto{\pgfqpoint{2.375336in}{2.230024in}}%
\pgfpathlineto{\pgfqpoint{2.401469in}{2.266714in}}%
\pgfpathlineto{\pgfqpoint{2.427602in}{2.296208in}}%
\pgfpathlineto{\pgfqpoint{2.453736in}{2.318240in}}%
\pgfpathlineto{\pgfqpoint{2.479869in}{2.332936in}}%
\pgfpathlineto{\pgfqpoint{2.506002in}{2.343754in}}%
\pgfpathlineto{\pgfqpoint{2.532136in}{2.355440in}}%
\pgfpathlineto{\pgfqpoint{2.558269in}{2.369036in}}%
\pgfpathlineto{\pgfqpoint{2.584403in}{2.381173in}}%
\pgfpathlineto{\pgfqpoint{2.610536in}{2.387973in}}%
\pgfpathlineto{\pgfqpoint{2.636669in}{2.388936in}}%
\pgfpathlineto{\pgfqpoint{2.662803in}{2.386984in}}%
\pgfpathlineto{\pgfqpoint{2.688936in}{2.385673in}}%
\pgfpathlineto{\pgfqpoint{2.715069in}{2.386836in}}%
\pgfpathlineto{\pgfqpoint{2.741203in}{2.390170in}}%
\pgfpathlineto{\pgfqpoint{2.767336in}{2.394581in}}%
\pgfpathlineto{\pgfqpoint{2.793469in}{2.399391in}}%
\pgfpathlineto{\pgfqpoint{2.819603in}{2.405526in}}%
\pgfpathlineto{\pgfqpoint{2.845736in}{2.415674in}}%
\pgfpathlineto{\pgfqpoint{2.871869in}{2.434341in}}%
\pgfpathlineto{\pgfqpoint{2.898003in}{2.467210in}}%
\pgfpathlineto{\pgfqpoint{2.924136in}{2.516562in}}%
\pgfpathlineto{\pgfqpoint{2.950269in}{2.577014in}}%
\pgfpathlineto{\pgfqpoint{2.976403in}{2.637830in}}%
\pgfpathlineto{\pgfqpoint{3.002536in}{2.688798in}}%
\pgfpathlineto{\pgfqpoint{3.028669in}{2.723724in}}%
\pgfpathlineto{\pgfqpoint{3.054803in}{2.742125in}}%
\pgfpathlineto{\pgfqpoint{3.080936in}{2.747232in}}%
\pgfpathlineto{\pgfqpoint{3.107070in}{2.744029in}}%
\pgfpathlineto{\pgfqpoint{3.133203in}{2.737613in}}%
\pgfusepath{stroke}%
\end{pgfscope}%
\begin{pgfscope}%
\pgfpathrectangle{\pgfqpoint{0.828241in}{0.574768in}}{\pgfqpoint{2.414722in}{2.357859in}}%
\pgfusepath{clip}%
\pgfsetroundcap%
\pgfsetroundjoin%
\pgfsetlinewidth{1.505625pt}%
\definecolor{currentstroke}{rgb}{0.505882,0.447059,0.701961}%
\pgfsetstrokecolor{currentstroke}%
\pgfsetdash{}{0pt}%
\pgfpathmoveto{\pgfqpoint{0.938001in}{1.848014in}}%
\pgfpathlineto{\pgfqpoint{0.964135in}{1.887173in}}%
\pgfpathlineto{\pgfqpoint{0.990268in}{1.963865in}}%
\pgfpathlineto{\pgfqpoint{1.016401in}{2.100659in}}%
\pgfpathlineto{\pgfqpoint{1.042535in}{2.284777in}}%
\pgfpathlineto{\pgfqpoint{1.068668in}{2.471666in}}%
\pgfpathlineto{\pgfqpoint{1.094801in}{2.613499in}}%
\pgfpathlineto{\pgfqpoint{1.120935in}{2.691464in}}%
\pgfpathlineto{\pgfqpoint{1.147068in}{2.720469in}}%
\pgfpathlineto{\pgfqpoint{1.173201in}{2.729155in}}%
\pgfpathlineto{\pgfqpoint{1.199335in}{2.737613in}}%
\pgfpathlineto{\pgfqpoint{1.225468in}{2.749788in}}%
\pgfpathlineto{\pgfqpoint{1.251602in}{2.759411in}}%
\pgfpathlineto{\pgfqpoint{1.277735in}{2.758272in}}%
\pgfpathlineto{\pgfqpoint{1.303868in}{2.737804in}}%
\pgfpathlineto{\pgfqpoint{1.330002in}{2.688670in}}%
\pgfpathlineto{\pgfqpoint{1.356135in}{2.604964in}}%
\pgfpathlineto{\pgfqpoint{1.382268in}{2.488723in}}%
\pgfpathlineto{\pgfqpoint{1.408402in}{2.348875in}}%
\pgfpathlineto{\pgfqpoint{1.434535in}{2.196372in}}%
\pgfpathlineto{\pgfqpoint{1.460668in}{2.037786in}}%
\pgfpathlineto{\pgfqpoint{1.486802in}{1.873633in}}%
\pgfpathlineto{\pgfqpoint{1.512935in}{1.703132in}}%
\pgfpathlineto{\pgfqpoint{1.539068in}{1.529329in}}%
\pgfpathlineto{\pgfqpoint{1.565202in}{1.358930in}}%
\pgfpathlineto{\pgfqpoint{1.591335in}{1.198362in}}%
\pgfpathlineto{\pgfqpoint{1.617468in}{1.052423in}}%
\pgfpathlineto{\pgfqpoint{1.643602in}{0.925447in}}%
\pgfpathlineto{\pgfqpoint{1.669735in}{0.820194in}}%
\pgfpathlineto{\pgfqpoint{1.695869in}{0.740318in}}%
\pgfpathlineto{\pgfqpoint{1.722002in}{0.692294in}}%
\pgfpathlineto{\pgfqpoint{1.748135in}{0.681944in}}%
\pgfpathlineto{\pgfqpoint{1.774269in}{0.707189in}}%
\pgfpathlineto{\pgfqpoint{1.800402in}{0.755123in}}%
\pgfpathlineto{\pgfqpoint{1.826535in}{0.807765in}}%
\pgfpathlineto{\pgfqpoint{1.852669in}{0.849046in}}%
\pgfpathlineto{\pgfqpoint{1.878802in}{0.870351in}}%
\pgfpathlineto{\pgfqpoint{1.904935in}{0.876511in}}%
\pgfpathlineto{\pgfqpoint{1.931069in}{0.886417in}}%
\pgfpathlineto{\pgfqpoint{1.957202in}{0.920951in}}%
\pgfpathlineto{\pgfqpoint{1.983335in}{0.992813in}}%
\pgfpathlineto{\pgfqpoint{2.009469in}{1.099796in}}%
\pgfpathlineto{\pgfqpoint{2.035602in}{1.226671in}}%
\pgfpathlineto{\pgfqpoint{2.061735in}{1.353648in}}%
\pgfpathlineto{\pgfqpoint{2.087869in}{1.466534in}}%
\pgfpathlineto{\pgfqpoint{2.114002in}{1.560280in}}%
\pgfpathlineto{\pgfqpoint{2.140136in}{1.636343in}}%
\pgfpathlineto{\pgfqpoint{2.166269in}{1.698248in}}%
\pgfpathlineto{\pgfqpoint{2.192402in}{1.748317in}}%
\pgfpathlineto{\pgfqpoint{2.218536in}{1.787080in}}%
\pgfpathlineto{\pgfqpoint{2.244669in}{1.813467in}}%
\pgfpathlineto{\pgfqpoint{2.270802in}{1.827636in}}%
\pgfpathlineto{\pgfqpoint{2.296936in}{1.837170in}}%
\pgfpathlineto{\pgfqpoint{2.323069in}{1.861296in}}%
\pgfpathlineto{\pgfqpoint{2.349202in}{1.928443in}}%
\pgfpathlineto{\pgfqpoint{2.375336in}{2.060477in}}%
\pgfpathlineto{\pgfqpoint{2.401469in}{2.248211in}}%
\pgfpathlineto{\pgfqpoint{2.427602in}{2.445947in}}%
\pgfpathlineto{\pgfqpoint{2.453736in}{2.600793in}}%
\pgfpathlineto{\pgfqpoint{2.479869in}{2.688672in}}%
\pgfpathlineto{\pgfqpoint{2.506002in}{2.722733in}}%
\pgfpathlineto{\pgfqpoint{2.532136in}{2.731696in}}%
\pgfpathlineto{\pgfqpoint{2.558269in}{2.737613in}}%
\pgfpathlineto{\pgfqpoint{2.584403in}{2.749858in}}%
\pgfpathlineto{\pgfqpoint{2.610536in}{2.765426in}}%
\pgfpathlineto{\pgfqpoint{2.636669in}{2.771576in}}%
\pgfpathlineto{\pgfqpoint{2.662803in}{2.752409in}}%
\pgfpathlineto{\pgfqpoint{2.688936in}{2.696724in}}%
\pgfpathlineto{\pgfqpoint{2.715069in}{2.605924in}}%
\pgfpathlineto{\pgfqpoint{2.741203in}{2.490240in}}%
\pgfpathlineto{\pgfqpoint{2.767336in}{2.358659in}}%
\pgfpathlineto{\pgfqpoint{2.793469in}{2.218516in}}%
\pgfusepath{stroke}%
\end{pgfscope}%
\begin{pgfscope}%
\pgfsetrectcap%
\pgfsetmiterjoin%
\pgfsetlinewidth{1.254687pt}%
\definecolor{currentstroke}{rgb}{1.000000,1.000000,1.000000}%
\pgfsetstrokecolor{currentstroke}%
\pgfsetdash{}{0pt}%
\pgfpathmoveto{\pgfqpoint{0.828241in}{0.574768in}}%
\pgfpathlineto{\pgfqpoint{0.828241in}{2.932627in}}%
\pgfusepath{stroke}%
\end{pgfscope}%
\begin{pgfscope}%
\pgfsetrectcap%
\pgfsetmiterjoin%
\pgfsetlinewidth{1.254687pt}%
\definecolor{currentstroke}{rgb}{1.000000,1.000000,1.000000}%
\pgfsetstrokecolor{currentstroke}%
\pgfsetdash{}{0pt}%
\pgfpathmoveto{\pgfqpoint{3.242963in}{0.574768in}}%
\pgfpathlineto{\pgfqpoint{3.242963in}{2.932627in}}%
\pgfusepath{stroke}%
\end{pgfscope}%
\begin{pgfscope}%
\pgfsetrectcap%
\pgfsetmiterjoin%
\pgfsetlinewidth{1.254687pt}%
\definecolor{currentstroke}{rgb}{1.000000,1.000000,1.000000}%
\pgfsetstrokecolor{currentstroke}%
\pgfsetdash{}{0pt}%
\pgfpathmoveto{\pgfqpoint{0.828241in}{0.574768in}}%
\pgfpathlineto{\pgfqpoint{3.242963in}{0.574768in}}%
\pgfusepath{stroke}%
\end{pgfscope}%
\begin{pgfscope}%
\pgfsetrectcap%
\pgfsetmiterjoin%
\pgfsetlinewidth{1.254687pt}%
\definecolor{currentstroke}{rgb}{1.000000,1.000000,1.000000}%
\pgfsetstrokecolor{currentstroke}%
\pgfsetdash{}{0pt}%
\pgfpathmoveto{\pgfqpoint{0.828241in}{2.932627in}}%
\pgfpathlineto{\pgfqpoint{3.242963in}{2.932627in}}%
\pgfusepath{stroke}%
\end{pgfscope}%
\begin{pgfscope}%
\pgfsetbuttcap%
\pgfsetmiterjoin%
\definecolor{currentfill}{rgb}{0.917647,0.917647,0.949020}%
\pgfsetfillcolor{currentfill}%
\pgfsetlinewidth{0.000000pt}%
\definecolor{currentstroke}{rgb}{0.000000,0.000000,0.000000}%
\pgfsetstrokecolor{currentstroke}%
\pgfsetstrokeopacity{0.000000}%
\pgfsetdash{}{0pt}%
\pgfpathmoveto{\pgfqpoint{3.825278in}{0.574768in}}%
\pgfpathlineto{\pgfqpoint{6.240000in}{0.574768in}}%
\pgfpathlineto{\pgfqpoint{6.240000in}{2.932627in}}%
\pgfpathlineto{\pgfqpoint{3.825278in}{2.932627in}}%
\pgfpathclose%
\pgfusepath{fill}%
\end{pgfscope}%
\begin{pgfscope}%
\pgfpathrectangle{\pgfqpoint{3.825278in}{0.574768in}}{\pgfqpoint{2.414722in}{2.357859in}}%
\pgfusepath{clip}%
\pgfsetroundcap%
\pgfsetroundjoin%
\pgfsetlinewidth{1.003750pt}%
\definecolor{currentstroke}{rgb}{1.000000,1.000000,1.000000}%
\pgfsetstrokecolor{currentstroke}%
\pgfsetdash{}{0pt}%
\pgfpathmoveto{\pgfqpoint{3.935038in}{0.574768in}}%
\pgfpathlineto{\pgfqpoint{3.935038in}{2.932627in}}%
\pgfusepath{stroke}%
\end{pgfscope}%
\begin{pgfscope}%
\definecolor{textcolor}{rgb}{0.150000,0.150000,0.150000}%
\pgfsetstrokecolor{textcolor}%
\pgfsetfillcolor{textcolor}%
\pgftext[x=3.935038in,y=0.442824in,,top]{\color{textcolor}\sffamily\fontsize{11.000000}{13.200000}\selectfont \(\displaystyle 0.0\)}%
\end{pgfscope}%
\begin{pgfscope}%
\pgfpathrectangle{\pgfqpoint{3.825278in}{0.574768in}}{\pgfqpoint{2.414722in}{2.357859in}}%
\pgfusepath{clip}%
\pgfsetroundcap%
\pgfsetroundjoin%
\pgfsetlinewidth{1.003750pt}%
\definecolor{currentstroke}{rgb}{1.000000,1.000000,1.000000}%
\pgfsetstrokecolor{currentstroke}%
\pgfsetdash{}{0pt}%
\pgfpathmoveto{\pgfqpoint{4.637503in}{0.574768in}}%
\pgfpathlineto{\pgfqpoint{4.637503in}{2.932627in}}%
\pgfusepath{stroke}%
\end{pgfscope}%
\begin{pgfscope}%
\definecolor{textcolor}{rgb}{0.150000,0.150000,0.150000}%
\pgfsetstrokecolor{textcolor}%
\pgfsetfillcolor{textcolor}%
\pgftext[x=4.637503in,y=0.442824in,,top]{\color{textcolor}\sffamily\fontsize{11.000000}{13.200000}\selectfont \(\displaystyle 0.5\)}%
\end{pgfscope}%
\begin{pgfscope}%
\pgfpathrectangle{\pgfqpoint{3.825278in}{0.574768in}}{\pgfqpoint{2.414722in}{2.357859in}}%
\pgfusepath{clip}%
\pgfsetroundcap%
\pgfsetroundjoin%
\pgfsetlinewidth{1.003750pt}%
\definecolor{currentstroke}{rgb}{1.000000,1.000000,1.000000}%
\pgfsetstrokecolor{currentstroke}%
\pgfsetdash{}{0pt}%
\pgfpathmoveto{\pgfqpoint{5.339967in}{0.574768in}}%
\pgfpathlineto{\pgfqpoint{5.339967in}{2.932627in}}%
\pgfusepath{stroke}%
\end{pgfscope}%
\begin{pgfscope}%
\definecolor{textcolor}{rgb}{0.150000,0.150000,0.150000}%
\pgfsetstrokecolor{textcolor}%
\pgfsetfillcolor{textcolor}%
\pgftext[x=5.339967in,y=0.442824in,,top]{\color{textcolor}\sffamily\fontsize{11.000000}{13.200000}\selectfont \(\displaystyle 1.0\)}%
\end{pgfscope}%
\begin{pgfscope}%
\pgfpathrectangle{\pgfqpoint{3.825278in}{0.574768in}}{\pgfqpoint{2.414722in}{2.357859in}}%
\pgfusepath{clip}%
\pgfsetroundcap%
\pgfsetroundjoin%
\pgfsetlinewidth{1.003750pt}%
\definecolor{currentstroke}{rgb}{1.000000,1.000000,1.000000}%
\pgfsetstrokecolor{currentstroke}%
\pgfsetdash{}{0pt}%
\pgfpathmoveto{\pgfqpoint{6.042432in}{0.574768in}}%
\pgfpathlineto{\pgfqpoint{6.042432in}{2.932627in}}%
\pgfusepath{stroke}%
\end{pgfscope}%
\begin{pgfscope}%
\definecolor{textcolor}{rgb}{0.150000,0.150000,0.150000}%
\pgfsetstrokecolor{textcolor}%
\pgfsetfillcolor{textcolor}%
\pgftext[x=6.042432in,y=0.442824in,,top]{\color{textcolor}\sffamily\fontsize{11.000000}{13.200000}\selectfont \(\displaystyle 1.5\)}%
\end{pgfscope}%
\begin{pgfscope}%
\definecolor{textcolor}{rgb}{0.150000,0.150000,0.150000}%
\pgfsetstrokecolor{textcolor}%
\pgfsetfillcolor{textcolor}%
\pgftext[x=5.032639in,y=0.252083in,,top]{\color{textcolor}\sffamily\fontsize{11.000000}{13.200000}\selectfont Time [s]}%
\end{pgfscope}%
\begin{pgfscope}%
\pgfpathrectangle{\pgfqpoint{3.825278in}{0.574768in}}{\pgfqpoint{2.414722in}{2.357859in}}%
\pgfusepath{clip}%
\pgfsetroundcap%
\pgfsetroundjoin%
\pgfsetlinewidth{1.003750pt}%
\definecolor{currentstroke}{rgb}{1.000000,1.000000,1.000000}%
\pgfsetstrokecolor{currentstroke}%
\pgfsetdash{}{0pt}%
\pgfpathmoveto{\pgfqpoint{3.825278in}{0.884193in}}%
\pgfpathlineto{\pgfqpoint{6.240000in}{0.884193in}}%
\pgfusepath{stroke}%
\end{pgfscope}%
\begin{pgfscope}%
\definecolor{textcolor}{rgb}{0.150000,0.150000,0.150000}%
\pgfsetstrokecolor{textcolor}%
\pgfsetfillcolor{textcolor}%
\pgftext[x=3.422963in,y=0.831386in,left,base]{\color{textcolor}\sffamily\fontsize{11.000000}{13.200000}\selectfont \(\displaystyle -20\)}%
\end{pgfscope}%
\begin{pgfscope}%
\pgfpathrectangle{\pgfqpoint{3.825278in}{0.574768in}}{\pgfqpoint{2.414722in}{2.357859in}}%
\pgfusepath{clip}%
\pgfsetroundcap%
\pgfsetroundjoin%
\pgfsetlinewidth{1.003750pt}%
\definecolor{currentstroke}{rgb}{1.000000,1.000000,1.000000}%
\pgfsetstrokecolor{currentstroke}%
\pgfsetdash{}{0pt}%
\pgfpathmoveto{\pgfqpoint{3.825278in}{1.351024in}}%
\pgfpathlineto{\pgfqpoint{6.240000in}{1.351024in}}%
\pgfusepath{stroke}%
\end{pgfscope}%
\begin{pgfscope}%
\definecolor{textcolor}{rgb}{0.150000,0.150000,0.150000}%
\pgfsetstrokecolor{textcolor}%
\pgfsetfillcolor{textcolor}%
\pgftext[x=3.422963in,y=1.298218in,left,base]{\color{textcolor}\sffamily\fontsize{11.000000}{13.200000}\selectfont \(\displaystyle -15\)}%
\end{pgfscope}%
\begin{pgfscope}%
\pgfpathrectangle{\pgfqpoint{3.825278in}{0.574768in}}{\pgfqpoint{2.414722in}{2.357859in}}%
\pgfusepath{clip}%
\pgfsetroundcap%
\pgfsetroundjoin%
\pgfsetlinewidth{1.003750pt}%
\definecolor{currentstroke}{rgb}{1.000000,1.000000,1.000000}%
\pgfsetstrokecolor{currentstroke}%
\pgfsetdash{}{0pt}%
\pgfpathmoveto{\pgfqpoint{3.825278in}{1.817856in}}%
\pgfpathlineto{\pgfqpoint{6.240000in}{1.817856in}}%
\pgfusepath{stroke}%
\end{pgfscope}%
\begin{pgfscope}%
\definecolor{textcolor}{rgb}{0.150000,0.150000,0.150000}%
\pgfsetstrokecolor{textcolor}%
\pgfsetfillcolor{textcolor}%
\pgftext[x=3.422963in,y=1.765049in,left,base]{\color{textcolor}\sffamily\fontsize{11.000000}{13.200000}\selectfont \(\displaystyle -10\)}%
\end{pgfscope}%
\begin{pgfscope}%
\pgfpathrectangle{\pgfqpoint{3.825278in}{0.574768in}}{\pgfqpoint{2.414722in}{2.357859in}}%
\pgfusepath{clip}%
\pgfsetroundcap%
\pgfsetroundjoin%
\pgfsetlinewidth{1.003750pt}%
\definecolor{currentstroke}{rgb}{1.000000,1.000000,1.000000}%
\pgfsetstrokecolor{currentstroke}%
\pgfsetdash{}{0pt}%
\pgfpathmoveto{\pgfqpoint{3.825278in}{2.284688in}}%
\pgfpathlineto{\pgfqpoint{6.240000in}{2.284688in}}%
\pgfusepath{stroke}%
\end{pgfscope}%
\begin{pgfscope}%
\definecolor{textcolor}{rgb}{0.150000,0.150000,0.150000}%
\pgfsetstrokecolor{textcolor}%
\pgfsetfillcolor{textcolor}%
\pgftext[x=3.499005in,y=2.231881in,left,base]{\color{textcolor}\sffamily\fontsize{11.000000}{13.200000}\selectfont \(\displaystyle -5\)}%
\end{pgfscope}%
\begin{pgfscope}%
\pgfpathrectangle{\pgfqpoint{3.825278in}{0.574768in}}{\pgfqpoint{2.414722in}{2.357859in}}%
\pgfusepath{clip}%
\pgfsetroundcap%
\pgfsetroundjoin%
\pgfsetlinewidth{1.003750pt}%
\definecolor{currentstroke}{rgb}{1.000000,1.000000,1.000000}%
\pgfsetstrokecolor{currentstroke}%
\pgfsetdash{}{0pt}%
\pgfpathmoveto{\pgfqpoint{3.825278in}{2.751519in}}%
\pgfpathlineto{\pgfqpoint{6.240000in}{2.751519in}}%
\pgfusepath{stroke}%
\end{pgfscope}%
\begin{pgfscope}%
\definecolor{textcolor}{rgb}{0.150000,0.150000,0.150000}%
\pgfsetstrokecolor{textcolor}%
\pgfsetfillcolor{textcolor}%
\pgftext[x=3.617292in,y=2.698713in,left,base]{\color{textcolor}\sffamily\fontsize{11.000000}{13.200000}\selectfont \(\displaystyle 0\)}%
\end{pgfscope}%
\begin{pgfscope}%
\pgfpathrectangle{\pgfqpoint{3.825278in}{0.574768in}}{\pgfqpoint{2.414722in}{2.357859in}}%
\pgfusepath{clip}%
\pgfsetroundcap%
\pgfsetroundjoin%
\pgfsetlinewidth{1.505625pt}%
\definecolor{currentstroke}{rgb}{0.298039,0.447059,0.690196}%
\pgfsetstrokecolor{currentstroke}%
\pgfsetdash{}{0pt}%
\pgfpathmoveto{\pgfqpoint{3.935038in}{2.076589in}}%
\pgfpathlineto{\pgfqpoint{3.952172in}{2.080485in}}%
\pgfpathlineto{\pgfqpoint{3.969305in}{2.093272in}}%
\pgfpathlineto{\pgfqpoint{3.986438in}{2.125760in}}%
\pgfpathlineto{\pgfqpoint{4.003571in}{2.185376in}}%
\pgfpathlineto{\pgfqpoint{4.020705in}{2.273360in}}%
\pgfpathlineto{\pgfqpoint{4.037838in}{2.383437in}}%
\pgfpathlineto{\pgfqpoint{4.054971in}{2.502446in}}%
\pgfpathlineto{\pgfqpoint{4.072105in}{2.614148in}}%
\pgfpathlineto{\pgfqpoint{4.089238in}{2.703788in}}%
\pgfpathlineto{\pgfqpoint{4.106371in}{2.761291in}}%
\pgfpathlineto{\pgfqpoint{4.123504in}{2.783072in}}%
\pgfpathlineto{\pgfqpoint{4.140638in}{2.774959in}}%
\pgfpathlineto{\pgfqpoint{4.157771in}{2.751519in}}%
\pgfpathlineto{\pgfqpoint{4.174904in}{2.726551in}}%
\pgfpathlineto{\pgfqpoint{4.192038in}{2.704275in}}%
\pgfpathlineto{\pgfqpoint{4.209171in}{2.680379in}}%
\pgfpathlineto{\pgfqpoint{4.226304in}{2.648804in}}%
\pgfpathlineto{\pgfqpoint{4.243437in}{2.603806in}}%
\pgfpathlineto{\pgfqpoint{4.260571in}{2.541990in}}%
\pgfpathlineto{\pgfqpoint{4.277704in}{2.463009in}}%
\pgfpathlineto{\pgfqpoint{4.294837in}{2.371394in}}%
\pgfpathlineto{\pgfqpoint{4.311971in}{2.276680in}}%
\pgfpathlineto{\pgfqpoint{4.329104in}{2.185947in}}%
\pgfpathlineto{\pgfqpoint{4.346237in}{2.102211in}}%
\pgfpathlineto{\pgfqpoint{4.363370in}{2.028814in}}%
\pgfpathlineto{\pgfqpoint{4.380504in}{1.966517in}}%
\pgfpathlineto{\pgfqpoint{4.397637in}{1.909380in}}%
\pgfpathlineto{\pgfqpoint{4.414770in}{1.848893in}}%
\pgfpathlineto{\pgfqpoint{4.431903in}{1.779820in}}%
\pgfpathlineto{\pgfqpoint{4.449037in}{1.702455in}}%
\pgfpathlineto{\pgfqpoint{4.466170in}{1.621276in}}%
\pgfpathlineto{\pgfqpoint{4.483303in}{1.540449in}}%
\pgfpathlineto{\pgfqpoint{4.500437in}{1.461117in}}%
\pgfpathlineto{\pgfqpoint{4.517570in}{1.383193in}}%
\pgfpathlineto{\pgfqpoint{4.534703in}{1.307168in}}%
\pgfpathlineto{\pgfqpoint{4.551836in}{1.234773in}}%
\pgfpathlineto{\pgfqpoint{4.568970in}{1.168137in}}%
\pgfpathlineto{\pgfqpoint{4.586103in}{1.110071in}}%
\pgfpathlineto{\pgfqpoint{4.603236in}{1.063278in}}%
\pgfpathlineto{\pgfqpoint{4.620370in}{1.030277in}}%
\pgfpathlineto{\pgfqpoint{4.637503in}{1.015437in}}%
\pgfpathlineto{\pgfqpoint{4.654636in}{1.025158in}}%
\pgfpathlineto{\pgfqpoint{4.671769in}{1.062650in}}%
\pgfpathlineto{\pgfqpoint{4.688903in}{1.124628in}}%
\pgfpathlineto{\pgfqpoint{4.706036in}{1.203159in}}%
\pgfpathlineto{\pgfqpoint{4.723169in}{1.287600in}}%
\pgfpathlineto{\pgfqpoint{4.740303in}{1.366700in}}%
\pgfpathlineto{\pgfqpoint{4.757436in}{1.431115in}}%
\pgfpathlineto{\pgfqpoint{4.774569in}{1.474838in}}%
\pgfpathlineto{\pgfqpoint{4.791702in}{1.501526in}}%
\pgfpathlineto{\pgfqpoint{4.808836in}{1.525817in}}%
\pgfpathlineto{\pgfqpoint{4.825969in}{1.561283in}}%
\pgfpathlineto{\pgfqpoint{4.843102in}{1.612199in}}%
\pgfpathlineto{\pgfqpoint{4.860235in}{1.675711in}}%
\pgfpathlineto{\pgfqpoint{4.877369in}{1.744863in}}%
\pgfpathlineto{\pgfqpoint{4.894502in}{1.811629in}}%
\pgfpathlineto{\pgfqpoint{4.911635in}{1.871237in}}%
\pgfpathlineto{\pgfqpoint{4.928769in}{1.924051in}}%
\pgfpathlineto{\pgfqpoint{4.945902in}{1.972415in}}%
\pgfpathlineto{\pgfqpoint{4.963035in}{2.016833in}}%
\pgfpathlineto{\pgfqpoint{4.980168in}{2.055239in}}%
\pgfpathlineto{\pgfqpoint{4.997302in}{2.085140in}}%
\pgfpathlineto{\pgfqpoint{5.014435in}{2.106300in}}%
\pgfpathlineto{\pgfqpoint{5.031568in}{2.120487in}}%
\pgfpathlineto{\pgfqpoint{5.048702in}{2.129251in}}%
\pgfpathlineto{\pgfqpoint{5.065835in}{2.134034in}}%
\pgfpathlineto{\pgfqpoint{5.082968in}{2.136183in}}%
\pgfpathlineto{\pgfqpoint{5.100101in}{2.136182in}}%
\pgfpathlineto{\pgfqpoint{5.117235in}{2.134741in}}%
\pgfpathlineto{\pgfqpoint{5.134368in}{2.135269in}}%
\pgfpathlineto{\pgfqpoint{5.151501in}{2.143325in}}%
\pgfpathlineto{\pgfqpoint{5.168635in}{2.165401in}}%
\pgfpathlineto{\pgfqpoint{5.185768in}{2.208578in}}%
\pgfpathlineto{\pgfqpoint{5.202901in}{2.279344in}}%
\pgfpathlineto{\pgfqpoint{5.220034in}{2.378369in}}%
\pgfpathlineto{\pgfqpoint{5.237168in}{2.496352in}}%
\pgfpathlineto{\pgfqpoint{5.254301in}{2.615833in}}%
\pgfpathlineto{\pgfqpoint{5.271434in}{2.718975in}}%
\pgfpathlineto{\pgfqpoint{5.288568in}{2.792699in}}%
\pgfpathlineto{\pgfqpoint{5.305701in}{2.825452in}}%
\pgfpathlineto{\pgfqpoint{5.322834in}{2.809675in}}%
\pgfpathlineto{\pgfqpoint{5.339967in}{2.751519in}}%
\pgfpathlineto{\pgfqpoint{5.357101in}{2.673683in}}%
\pgfpathlineto{\pgfqpoint{5.374234in}{2.601788in}}%
\pgfpathlineto{\pgfqpoint{5.391367in}{2.547077in}}%
\pgfpathlineto{\pgfqpoint{5.408500in}{2.501470in}}%
\pgfpathlineto{\pgfqpoint{5.425634in}{2.449583in}}%
\pgfpathlineto{\pgfqpoint{5.442767in}{2.380792in}}%
\pgfpathlineto{\pgfqpoint{5.459900in}{2.293968in}}%
\pgfpathlineto{\pgfqpoint{5.477034in}{2.194756in}}%
\pgfpathlineto{\pgfqpoint{5.494167in}{2.090785in}}%
\pgfpathlineto{\pgfqpoint{5.511300in}{1.986779in}}%
\pgfpathlineto{\pgfqpoint{5.528433in}{1.885874in}}%
\pgfusepath{stroke}%
\end{pgfscope}%
\begin{pgfscope}%
\pgfpathrectangle{\pgfqpoint{3.825278in}{0.574768in}}{\pgfqpoint{2.414722in}{2.357859in}}%
\pgfusepath{clip}%
\pgfsetroundcap%
\pgfsetroundjoin%
\pgfsetlinewidth{1.505625pt}%
\definecolor{currentstroke}{rgb}{0.866667,0.517647,0.321569}%
\pgfsetstrokecolor{currentstroke}%
\pgfsetdash{}{0pt}%
\pgfpathmoveto{\pgfqpoint{3.935038in}{1.859189in}}%
\pgfpathlineto{\pgfqpoint{3.952172in}{1.857411in}}%
\pgfpathlineto{\pgfqpoint{3.969305in}{1.868257in}}%
\pgfpathlineto{\pgfqpoint{3.986438in}{1.902656in}}%
\pgfpathlineto{\pgfqpoint{4.003571in}{1.967432in}}%
\pgfpathlineto{\pgfqpoint{4.020705in}{2.061718in}}%
\pgfpathlineto{\pgfqpoint{4.037838in}{2.176694in}}%
\pgfpathlineto{\pgfqpoint{4.054971in}{2.301116in}}%
\pgfpathlineto{\pgfqpoint{4.072105in}{2.425389in}}%
\pgfpathlineto{\pgfqpoint{4.089238in}{2.536972in}}%
\pgfpathlineto{\pgfqpoint{4.106371in}{2.625042in}}%
\pgfpathlineto{\pgfqpoint{4.123504in}{2.688136in}}%
\pgfpathlineto{\pgfqpoint{4.140638in}{2.730028in}}%
\pgfpathlineto{\pgfqpoint{4.157771in}{2.751519in}}%
\pgfpathlineto{\pgfqpoint{4.174904in}{2.755517in}}%
\pgfpathlineto{\pgfqpoint{4.192038in}{2.752088in}}%
\pgfpathlineto{\pgfqpoint{4.209171in}{2.749837in}}%
\pgfpathlineto{\pgfqpoint{4.226304in}{2.745358in}}%
\pgfpathlineto{\pgfqpoint{4.243437in}{2.727829in}}%
\pgfpathlineto{\pgfqpoint{4.260571in}{2.686389in}}%
\pgfpathlineto{\pgfqpoint{4.277704in}{2.615482in}}%
\pgfpathlineto{\pgfqpoint{4.294837in}{2.521330in}}%
\pgfpathlineto{\pgfqpoint{4.311971in}{2.420421in}}%
\pgfpathlineto{\pgfqpoint{4.329104in}{2.325998in}}%
\pgfpathlineto{\pgfqpoint{4.346237in}{2.238367in}}%
\pgfpathlineto{\pgfqpoint{4.363370in}{2.150806in}}%
\pgfpathlineto{\pgfqpoint{4.380504in}{2.057443in}}%
\pgfpathlineto{\pgfqpoint{4.397637in}{1.955893in}}%
\pgfpathlineto{\pgfqpoint{4.414770in}{1.847525in}}%
\pgfpathlineto{\pgfqpoint{4.431903in}{1.735625in}}%
\pgfpathlineto{\pgfqpoint{4.449037in}{1.622956in}}%
\pgfpathlineto{\pgfqpoint{4.466170in}{1.510213in}}%
\pgfpathlineto{\pgfqpoint{4.483303in}{1.396087in}}%
\pgfpathlineto{\pgfqpoint{4.500437in}{1.282070in}}%
\pgfpathlineto{\pgfqpoint{4.517570in}{1.173223in}}%
\pgfpathlineto{\pgfqpoint{4.534703in}{1.073113in}}%
\pgfpathlineto{\pgfqpoint{4.551836in}{0.982586in}}%
\pgfpathlineto{\pgfqpoint{4.568970in}{0.901177in}}%
\pgfpathlineto{\pgfqpoint{4.586103in}{0.829964in}}%
\pgfpathlineto{\pgfqpoint{4.603236in}{0.770749in}}%
\pgfpathlineto{\pgfqpoint{4.620370in}{0.724097in}}%
\pgfpathlineto{\pgfqpoint{4.637503in}{0.692157in}}%
\pgfpathlineto{\pgfqpoint{4.654636in}{0.681944in}}%
\pgfpathlineto{\pgfqpoint{4.671769in}{0.699470in}}%
\pgfpathlineto{\pgfqpoint{4.688903in}{0.738327in}}%
\pgfpathlineto{\pgfqpoint{4.706036in}{0.780881in}}%
\pgfpathlineto{\pgfqpoint{4.723169in}{0.814150in}}%
\pgfpathlineto{\pgfqpoint{4.740303in}{0.838815in}}%
\pgfpathlineto{\pgfqpoint{4.757436in}{0.863786in}}%
\pgfpathlineto{\pgfqpoint{4.774569in}{0.896919in}}%
\pgfpathlineto{\pgfqpoint{4.791702in}{0.941732in}}%
\pgfpathlineto{\pgfqpoint{4.808836in}{1.000436in}}%
\pgfpathlineto{\pgfqpoint{4.825969in}{1.077246in}}%
\pgfpathlineto{\pgfqpoint{4.843102in}{1.176133in}}%
\pgfpathlineto{\pgfqpoint{4.860235in}{1.296079in}}%
\pgfpathlineto{\pgfqpoint{4.877369in}{1.428280in}}%
\pgfpathlineto{\pgfqpoint{4.894502in}{1.556165in}}%
\pgfpathlineto{\pgfqpoint{4.911635in}{1.664461in}}%
\pgfpathlineto{\pgfqpoint{4.928769in}{1.751943in}}%
\pgfpathlineto{\pgfqpoint{4.945902in}{1.829061in}}%
\pgfpathlineto{\pgfqpoint{4.963035in}{1.900970in}}%
\pgfpathlineto{\pgfqpoint{4.980168in}{1.961885in}}%
\pgfpathlineto{\pgfqpoint{4.997302in}{2.004566in}}%
\pgfpathlineto{\pgfqpoint{5.014435in}{2.028545in}}%
\pgfpathlineto{\pgfqpoint{5.031568in}{2.037806in}}%
\pgfpathlineto{\pgfqpoint{5.048702in}{2.036110in}}%
\pgfpathlineto{\pgfqpoint{5.065835in}{2.025233in}}%
\pgfpathlineto{\pgfqpoint{5.082968in}{2.006952in}}%
\pgfpathlineto{\pgfqpoint{5.100101in}{1.986174in}}%
\pgfpathlineto{\pgfqpoint{5.117235in}{1.970467in}}%
\pgfpathlineto{\pgfqpoint{5.134368in}{1.967595in}}%
\pgfpathlineto{\pgfqpoint{5.151501in}{1.984889in}}%
\pgfpathlineto{\pgfqpoint{5.168635in}{2.027530in}}%
\pgfpathlineto{\pgfqpoint{5.185768in}{2.096229in}}%
\pgfpathlineto{\pgfqpoint{5.202901in}{2.187454in}}%
\pgfpathlineto{\pgfqpoint{5.220034in}{2.295476in}}%
\pgfpathlineto{\pgfqpoint{5.237168in}{2.413374in}}%
\pgfpathlineto{\pgfqpoint{5.254301in}{2.529256in}}%
\pgfpathlineto{\pgfqpoint{5.271434in}{2.628574in}}%
\pgfpathlineto{\pgfqpoint{5.288568in}{2.700918in}}%
\pgfpathlineto{\pgfqpoint{5.305701in}{2.743321in}}%
\pgfpathlineto{\pgfqpoint{5.322834in}{2.758278in}}%
\pgfpathlineto{\pgfqpoint{5.339967in}{2.751519in}}%
\pgfpathlineto{\pgfqpoint{5.357101in}{2.734369in}}%
\pgfpathlineto{\pgfqpoint{5.374234in}{2.720356in}}%
\pgfpathlineto{\pgfqpoint{5.391367in}{2.713280in}}%
\pgfpathlineto{\pgfqpoint{5.408500in}{2.703798in}}%
\pgfpathlineto{\pgfqpoint{5.425634in}{2.678867in}}%
\pgfpathlineto{\pgfqpoint{5.442767in}{2.630985in}}%
\pgfpathlineto{\pgfqpoint{5.459900in}{2.560366in}}%
\pgfpathlineto{\pgfqpoint{5.477034in}{2.473014in}}%
\pgfpathlineto{\pgfqpoint{5.494167in}{2.378766in}}%
\pgfpathlineto{\pgfqpoint{5.511300in}{2.285208in}}%
\pgfpathlineto{\pgfqpoint{5.528433in}{2.193811in}}%
\pgfusepath{stroke}%
\end{pgfscope}%
\begin{pgfscope}%
\pgfpathrectangle{\pgfqpoint{3.825278in}{0.574768in}}{\pgfqpoint{2.414722in}{2.357859in}}%
\pgfusepath{clip}%
\pgfsetroundcap%
\pgfsetroundjoin%
\pgfsetlinewidth{1.505625pt}%
\definecolor{currentstroke}{rgb}{0.333333,0.658824,0.407843}%
\pgfsetstrokecolor{currentstroke}%
\pgfsetdash{}{0pt}%
\pgfpathmoveto{\pgfqpoint{3.935038in}{2.447114in}}%
\pgfpathlineto{\pgfqpoint{3.959686in}{2.517043in}}%
\pgfpathlineto{\pgfqpoint{3.984334in}{2.586475in}}%
\pgfpathlineto{\pgfqpoint{4.008982in}{2.650422in}}%
\pgfpathlineto{\pgfqpoint{4.033630in}{2.700845in}}%
\pgfpathlineto{\pgfqpoint{4.058278in}{2.733793in}}%
\pgfpathlineto{\pgfqpoint{4.082926in}{2.751519in}}%
\pgfpathlineto{\pgfqpoint{4.107573in}{2.757849in}}%
\pgfpathlineto{\pgfqpoint{4.132221in}{2.753249in}}%
\pgfpathlineto{\pgfqpoint{4.156869in}{2.735284in}}%
\pgfpathlineto{\pgfqpoint{4.181517in}{2.703230in}}%
\pgfpathlineto{\pgfqpoint{4.206165in}{2.658845in}}%
\pgfpathlineto{\pgfqpoint{4.230813in}{2.603344in}}%
\pgfpathlineto{\pgfqpoint{4.255461in}{2.537377in}}%
\pgfpathlineto{\pgfqpoint{4.280109in}{2.462450in}}%
\pgfpathlineto{\pgfqpoint{4.304757in}{2.381233in}}%
\pgfpathlineto{\pgfqpoint{4.329404in}{2.298722in}}%
\pgfpathlineto{\pgfqpoint{4.354052in}{2.220600in}}%
\pgfpathlineto{\pgfqpoint{4.378700in}{2.148395in}}%
\pgfpathlineto{\pgfqpoint{4.403348in}{2.080080in}}%
\pgfpathlineto{\pgfqpoint{4.427996in}{2.012454in}}%
\pgfpathlineto{\pgfqpoint{4.452644in}{1.943070in}}%
\pgfpathlineto{\pgfqpoint{4.477292in}{1.873951in}}%
\pgfpathlineto{\pgfqpoint{4.501940in}{1.811469in}}%
\pgfpathlineto{\pgfqpoint{4.526587in}{1.760993in}}%
\pgfpathlineto{\pgfqpoint{4.551235in}{1.724786in}}%
\pgfpathlineto{\pgfqpoint{4.575883in}{1.704124in}}%
\pgfpathlineto{\pgfqpoint{4.600531in}{1.700404in}}%
\pgfpathlineto{\pgfqpoint{4.625179in}{1.713905in}}%
\pgfpathlineto{\pgfqpoint{4.649827in}{1.741578in}}%
\pgfpathlineto{\pgfqpoint{4.674475in}{1.777123in}}%
\pgfpathlineto{\pgfqpoint{4.699123in}{1.814381in}}%
\pgfpathlineto{\pgfqpoint{4.723770in}{1.847998in}}%
\pgfpathlineto{\pgfqpoint{4.748418in}{1.872625in}}%
\pgfpathlineto{\pgfqpoint{4.773066in}{1.884861in}}%
\pgfpathlineto{\pgfqpoint{4.797714in}{1.883192in}}%
\pgfpathlineto{\pgfqpoint{4.822362in}{1.865219in}}%
\pgfpathlineto{\pgfqpoint{4.847010in}{1.830106in}}%
\pgfpathlineto{\pgfqpoint{4.871658in}{1.784449in}}%
\pgfpathlineto{\pgfqpoint{4.896306in}{1.743092in}}%
\pgfpathlineto{\pgfqpoint{4.920953in}{1.718529in}}%
\pgfpathlineto{\pgfqpoint{4.945601in}{1.713863in}}%
\pgfpathlineto{\pgfqpoint{4.970249in}{1.726596in}}%
\pgfpathlineto{\pgfqpoint{4.994897in}{1.754070in}}%
\pgfpathlineto{\pgfqpoint{5.019545in}{1.792943in}}%
\pgfpathlineto{\pgfqpoint{5.044193in}{1.838004in}}%
\pgfpathlineto{\pgfqpoint{5.068841in}{1.884665in}}%
\pgfpathlineto{\pgfqpoint{5.093489in}{1.931110in}}%
\pgfpathlineto{\pgfqpoint{5.118136in}{1.977328in}}%
\pgfpathlineto{\pgfqpoint{5.142784in}{2.022596in}}%
\pgfpathlineto{\pgfqpoint{5.167432in}{2.065043in}}%
\pgfpathlineto{\pgfqpoint{5.192080in}{2.102914in}}%
\pgfpathlineto{\pgfqpoint{5.216728in}{2.134761in}}%
\pgfpathlineto{\pgfqpoint{5.241376in}{2.159698in}}%
\pgfpathlineto{\pgfqpoint{5.266024in}{2.179086in}}%
\pgfpathlineto{\pgfqpoint{5.290672in}{2.196543in}}%
\pgfpathlineto{\pgfqpoint{5.315319in}{2.214131in}}%
\pgfpathlineto{\pgfqpoint{5.339967in}{2.230864in}}%
\pgfpathlineto{\pgfqpoint{5.364615in}{2.245530in}}%
\pgfpathlineto{\pgfqpoint{5.389263in}{2.258471in}}%
\pgfpathlineto{\pgfqpoint{5.413911in}{2.272183in}}%
\pgfpathlineto{\pgfqpoint{5.438559in}{2.291483in}}%
\pgfpathlineto{\pgfqpoint{5.463207in}{2.323178in}}%
\pgfpathlineto{\pgfqpoint{5.487855in}{2.374730in}}%
\pgfpathlineto{\pgfqpoint{5.512502in}{2.449076in}}%
\pgfpathlineto{\pgfqpoint{5.537150in}{2.537412in}}%
\pgfpathlineto{\pgfqpoint{5.561798in}{2.620431in}}%
\pgfpathlineto{\pgfqpoint{5.586446in}{2.683336in}}%
\pgfpathlineto{\pgfqpoint{5.611094in}{2.725189in}}%
\pgfpathlineto{\pgfqpoint{5.635742in}{2.751519in}}%
\pgfpathlineto{\pgfqpoint{5.660390in}{2.764160in}}%
\pgfpathlineto{\pgfqpoint{5.685038in}{2.761299in}}%
\pgfpathlineto{\pgfqpoint{5.709686in}{2.743916in}}%
\pgfpathlineto{\pgfqpoint{5.734333in}{2.717201in}}%
\pgfpathlineto{\pgfqpoint{5.758981in}{2.683511in}}%
\pgfpathlineto{\pgfqpoint{5.783629in}{2.640440in}}%
\pgfpathlineto{\pgfqpoint{5.808277in}{2.584854in}}%
\pgfpathlineto{\pgfqpoint{5.832925in}{2.517851in}}%
\pgfpathlineto{\pgfqpoint{5.857573in}{2.444553in}}%
\pgfpathlineto{\pgfqpoint{5.882221in}{2.369667in}}%
\pgfpathlineto{\pgfqpoint{5.906869in}{2.295928in}}%
\pgfpathlineto{\pgfqpoint{5.931516in}{2.223900in}}%
\pgfusepath{stroke}%
\end{pgfscope}%
\begin{pgfscope}%
\pgfpathrectangle{\pgfqpoint{3.825278in}{0.574768in}}{\pgfqpoint{2.414722in}{2.357859in}}%
\pgfusepath{clip}%
\pgfsetroundcap%
\pgfsetroundjoin%
\pgfsetlinewidth{1.505625pt}%
\definecolor{currentstroke}{rgb}{0.768627,0.305882,0.321569}%
\pgfsetstrokecolor{currentstroke}%
\pgfsetdash{}{0pt}%
\pgfpathmoveto{\pgfqpoint{3.935038in}{2.568674in}}%
\pgfpathlineto{\pgfqpoint{3.956990in}{2.569399in}}%
\pgfpathlineto{\pgfqpoint{3.978942in}{2.570556in}}%
\pgfpathlineto{\pgfqpoint{4.000894in}{2.572615in}}%
\pgfpathlineto{\pgfqpoint{4.022846in}{2.576272in}}%
\pgfpathlineto{\pgfqpoint{4.044798in}{2.583417in}}%
\pgfpathlineto{\pgfqpoint{4.066750in}{2.597158in}}%
\pgfpathlineto{\pgfqpoint{4.088702in}{2.619670in}}%
\pgfpathlineto{\pgfqpoint{4.110654in}{2.649029in}}%
\pgfpathlineto{\pgfqpoint{4.132606in}{2.679535in}}%
\pgfpathlineto{\pgfqpoint{4.154559in}{2.705422in}}%
\pgfpathlineto{\pgfqpoint{4.176511in}{2.723560in}}%
\pgfpathlineto{\pgfqpoint{4.198463in}{2.735285in}}%
\pgfpathlineto{\pgfqpoint{4.220415in}{2.744153in}}%
\pgfpathlineto{\pgfqpoint{4.242367in}{2.751519in}}%
\pgfpathlineto{\pgfqpoint{4.264319in}{2.754282in}}%
\pgfpathlineto{\pgfqpoint{4.286271in}{2.744861in}}%
\pgfpathlineto{\pgfqpoint{4.308223in}{2.712721in}}%
\pgfpathlineto{\pgfqpoint{4.330175in}{2.649547in}}%
\pgfpathlineto{\pgfqpoint{4.352127in}{2.557554in}}%
\pgfpathlineto{\pgfqpoint{4.374079in}{2.450333in}}%
\pgfpathlineto{\pgfqpoint{4.396031in}{2.343346in}}%
\pgfpathlineto{\pgfqpoint{4.417983in}{2.243852in}}%
\pgfpathlineto{\pgfqpoint{4.439935in}{2.149422in}}%
\pgfpathlineto{\pgfqpoint{4.461887in}{2.053921in}}%
\pgfpathlineto{\pgfqpoint{4.483839in}{1.954821in}}%
\pgfpathlineto{\pgfqpoint{4.505791in}{1.855808in}}%
\pgfpathlineto{\pgfqpoint{4.527743in}{1.764366in}}%
\pgfpathlineto{\pgfqpoint{4.549695in}{1.686430in}}%
\pgfpathlineto{\pgfqpoint{4.571647in}{1.623217in}}%
\pgfpathlineto{\pgfqpoint{4.593599in}{1.572158in}}%
\pgfpathlineto{\pgfqpoint{4.615551in}{1.531338in}}%
\pgfpathlineto{\pgfqpoint{4.637503in}{1.503108in}}%
\pgfpathlineto{\pgfqpoint{4.659455in}{1.490910in}}%
\pgfpathlineto{\pgfqpoint{4.681407in}{1.495663in}}%
\pgfpathlineto{\pgfqpoint{4.703359in}{1.514316in}}%
\pgfpathlineto{\pgfqpoint{4.725311in}{1.540677in}}%
\pgfpathlineto{\pgfqpoint{4.747263in}{1.567862in}}%
\pgfpathlineto{\pgfqpoint{4.769215in}{1.589643in}}%
\pgfpathlineto{\pgfqpoint{4.791167in}{1.600564in}}%
\pgfpathlineto{\pgfqpoint{4.813119in}{1.596995in}}%
\pgfpathlineto{\pgfqpoint{4.835071in}{1.579640in}}%
\pgfpathlineto{\pgfqpoint{4.857023in}{1.556384in}}%
\pgfpathlineto{\pgfqpoint{4.878975in}{1.540926in}}%
\pgfpathlineto{\pgfqpoint{4.900927in}{1.550011in}}%
\pgfpathlineto{\pgfqpoint{4.922879in}{1.599925in}}%
\pgfpathlineto{\pgfqpoint{4.944831in}{1.698232in}}%
\pgfpathlineto{\pgfqpoint{4.966783in}{1.836945in}}%
\pgfpathlineto{\pgfqpoint{4.988735in}{1.994427in}}%
\pgfpathlineto{\pgfqpoint{5.010687in}{2.143447in}}%
\pgfpathlineto{\pgfqpoint{5.032639in}{2.262524in}}%
\pgfpathlineto{\pgfqpoint{5.054591in}{2.342597in}}%
\pgfpathlineto{\pgfqpoint{5.076543in}{2.386648in}}%
\pgfpathlineto{\pgfqpoint{5.098495in}{2.405372in}}%
\pgfpathlineto{\pgfqpoint{5.120447in}{2.413104in}}%
\pgfpathlineto{\pgfqpoint{5.142399in}{2.423463in}}%
\pgfpathlineto{\pgfqpoint{5.164351in}{2.443406in}}%
\pgfpathlineto{\pgfqpoint{5.186303in}{2.471630in}}%
\pgfpathlineto{\pgfqpoint{5.208255in}{2.501932in}}%
\pgfpathlineto{\pgfqpoint{5.230207in}{2.527878in}}%
\pgfpathlineto{\pgfqpoint{5.252159in}{2.546501in}}%
\pgfpathlineto{\pgfqpoint{5.274111in}{2.558714in}}%
\pgfpathlineto{\pgfqpoint{5.296063in}{2.566817in}}%
\pgfpathlineto{\pgfqpoint{5.318015in}{2.572209in}}%
\pgfpathlineto{\pgfqpoint{5.339967in}{2.575261in}}%
\pgfpathlineto{\pgfqpoint{5.361919in}{2.575966in}}%
\pgfpathlineto{\pgfqpoint{5.383871in}{2.574616in}}%
\pgfpathlineto{\pgfqpoint{5.405823in}{2.572415in}}%
\pgfpathlineto{\pgfqpoint{5.427775in}{2.570652in}}%
\pgfpathlineto{\pgfqpoint{5.449727in}{2.569495in}}%
\pgfpathlineto{\pgfqpoint{5.471679in}{2.568116in}}%
\pgfpathlineto{\pgfqpoint{5.493631in}{2.566578in}}%
\pgfpathlineto{\pgfqpoint{5.515583in}{2.565888in}}%
\pgfpathlineto{\pgfqpoint{5.537535in}{2.566547in}}%
\pgfpathlineto{\pgfqpoint{5.559488in}{2.567798in}}%
\pgfpathlineto{\pgfqpoint{5.581440in}{2.568471in}}%
\pgfpathlineto{\pgfqpoint{5.603392in}{2.567797in}}%
\pgfpathlineto{\pgfqpoint{5.625344in}{2.567151in}}%
\pgfpathlineto{\pgfqpoint{5.647296in}{2.571622in}}%
\pgfpathlineto{\pgfqpoint{5.669248in}{2.588017in}}%
\pgfpathlineto{\pgfqpoint{5.691200in}{2.619633in}}%
\pgfpathlineto{\pgfqpoint{5.713152in}{2.662083in}}%
\pgfpathlineto{\pgfqpoint{5.735104in}{2.704505in}}%
\pgfpathlineto{\pgfqpoint{5.757056in}{2.737152in}}%
\pgfpathlineto{\pgfqpoint{5.779008in}{2.755789in}}%
\pgfpathlineto{\pgfqpoint{5.800960in}{2.762352in}}%
\pgfpathlineto{\pgfqpoint{5.822912in}{2.762246in}}%
\pgfpathlineto{\pgfqpoint{5.844864in}{2.759277in}}%
\pgfpathlineto{\pgfqpoint{5.866816in}{2.751519in}}%
\pgfpathlineto{\pgfqpoint{5.888768in}{2.731499in}}%
\pgfpathlineto{\pgfqpoint{5.910720in}{2.690481in}}%
\pgfpathlineto{\pgfqpoint{5.932672in}{2.622016in}}%
\pgfpathlineto{\pgfqpoint{5.954624in}{2.525799in}}%
\pgfpathlineto{\pgfqpoint{5.976576in}{2.410707in}}%
\pgfpathlineto{\pgfqpoint{5.998528in}{2.288944in}}%
\pgfpathlineto{\pgfqpoint{6.020480in}{2.168785in}}%
\pgfpathlineto{\pgfqpoint{6.042432in}{2.051633in}}%
\pgfpathlineto{\pgfqpoint{6.064384in}{1.933678in}}%
\pgfpathlineto{\pgfqpoint{6.086336in}{1.812983in}}%
\pgfpathlineto{\pgfqpoint{6.108288in}{1.693631in}}%
\pgfpathlineto{\pgfqpoint{6.130240in}{1.583627in}}%
\pgfusepath{stroke}%
\end{pgfscope}%
\begin{pgfscope}%
\pgfpathrectangle{\pgfqpoint{3.825278in}{0.574768in}}{\pgfqpoint{2.414722in}{2.357859in}}%
\pgfusepath{clip}%
\pgfsetroundcap%
\pgfsetroundjoin%
\pgfsetlinewidth{1.505625pt}%
\definecolor{currentstroke}{rgb}{0.505882,0.447059,0.701961}%
\pgfsetstrokecolor{currentstroke}%
\pgfsetdash{}{0pt}%
\pgfpathmoveto{\pgfqpoint{3.935038in}{2.023519in}}%
\pgfpathlineto{\pgfqpoint{3.958070in}{2.199541in}}%
\pgfpathlineto{\pgfqpoint{3.981102in}{2.363775in}}%
\pgfpathlineto{\pgfqpoint{4.004133in}{2.508114in}}%
\pgfpathlineto{\pgfqpoint{4.027165in}{2.624299in}}%
\pgfpathlineto{\pgfqpoint{4.050196in}{2.701648in}}%
\pgfpathlineto{\pgfqpoint{4.073228in}{2.739628in}}%
\pgfpathlineto{\pgfqpoint{4.096260in}{2.751519in}}%
\pgfpathlineto{\pgfqpoint{4.119291in}{2.751167in}}%
\pgfpathlineto{\pgfqpoint{4.142323in}{2.742228in}}%
\pgfpathlineto{\pgfqpoint{4.165355in}{2.720359in}}%
\pgfpathlineto{\pgfqpoint{4.188386in}{2.678692in}}%
\pgfpathlineto{\pgfqpoint{4.211418in}{2.611869in}}%
\pgfpathlineto{\pgfqpoint{4.234449in}{2.519467in}}%
\pgfpathlineto{\pgfqpoint{4.257481in}{2.407006in}}%
\pgfpathlineto{\pgfqpoint{4.280513in}{2.285007in}}%
\pgfpathlineto{\pgfqpoint{4.303544in}{2.163607in}}%
\pgfpathlineto{\pgfqpoint{4.326576in}{2.045673in}}%
\pgfpathlineto{\pgfqpoint{4.349608in}{1.929450in}}%
\pgfpathlineto{\pgfqpoint{4.372639in}{1.815676in}}%
\pgfpathlineto{\pgfqpoint{4.395671in}{1.706514in}}%
\pgfpathlineto{\pgfqpoint{4.418702in}{1.603297in}}%
\pgfpathlineto{\pgfqpoint{4.441734in}{1.506902in}}%
\pgfpathlineto{\pgfqpoint{4.464766in}{1.419090in}}%
\pgfpathlineto{\pgfqpoint{4.487797in}{1.341149in}}%
\pgfpathlineto{\pgfqpoint{4.510829in}{1.274104in}}%
\pgfpathlineto{\pgfqpoint{4.533861in}{1.220140in}}%
\pgfpathlineto{\pgfqpoint{4.556892in}{1.182460in}}%
\pgfpathlineto{\pgfqpoint{4.579924in}{1.162366in}}%
\pgfpathlineto{\pgfqpoint{4.602955in}{1.156061in}}%
\pgfpathlineto{\pgfqpoint{4.625987in}{1.156847in}}%
\pgfpathlineto{\pgfqpoint{4.649019in}{1.163758in}}%
\pgfpathlineto{\pgfqpoint{4.672050in}{1.187128in}}%
\pgfpathlineto{\pgfqpoint{4.695082in}{1.238202in}}%
\pgfpathlineto{\pgfqpoint{4.718114in}{1.314504in}}%
\pgfpathlineto{\pgfqpoint{4.741145in}{1.400221in}}%
\pgfpathlineto{\pgfqpoint{4.764177in}{1.480961in}}%
\pgfpathlineto{\pgfqpoint{4.787208in}{1.555021in}}%
\pgfpathlineto{\pgfqpoint{4.810240in}{1.628837in}}%
\pgfpathlineto{\pgfqpoint{4.833272in}{1.708225in}}%
\pgfpathlineto{\pgfqpoint{4.856303in}{1.793225in}}%
\pgfpathlineto{\pgfqpoint{4.879335in}{1.880576in}}%
\pgfpathlineto{\pgfqpoint{4.902367in}{1.965158in}}%
\pgfpathlineto{\pgfqpoint{4.925398in}{2.039800in}}%
\pgfpathlineto{\pgfqpoint{4.948430in}{2.098301in}}%
\pgfpathlineto{\pgfqpoint{4.971461in}{2.137726in}}%
\pgfpathlineto{\pgfqpoint{4.994493in}{2.158978in}}%
\pgfpathlineto{\pgfqpoint{5.017525in}{2.165003in}}%
\pgfpathlineto{\pgfqpoint{5.040556in}{2.159739in}}%
\pgfpathlineto{\pgfqpoint{5.063588in}{2.149778in}}%
\pgfpathlineto{\pgfqpoint{5.086619in}{2.147060in}}%
\pgfpathlineto{\pgfqpoint{5.109651in}{2.168311in}}%
\pgfpathlineto{\pgfqpoint{5.132683in}{2.230488in}}%
\pgfpathlineto{\pgfqpoint{5.155714in}{2.336753in}}%
\pgfpathlineto{\pgfqpoint{5.178746in}{2.468883in}}%
\pgfpathlineto{\pgfqpoint{5.201778in}{2.597971in}}%
\pgfpathlineto{\pgfqpoint{5.224809in}{2.697572in}}%
\pgfpathlineto{\pgfqpoint{5.247841in}{2.750258in}}%
\pgfpathlineto{\pgfqpoint{5.270872in}{2.760806in}}%
\pgfpathlineto{\pgfqpoint{5.293904in}{2.751519in}}%
\pgfpathlineto{\pgfqpoint{5.316936in}{2.742824in}}%
\pgfpathlineto{\pgfqpoint{5.339967in}{2.738214in}}%
\pgfpathlineto{\pgfqpoint{5.362999in}{2.725837in}}%
\pgfpathlineto{\pgfqpoint{5.386031in}{2.689762in}}%
\pgfpathlineto{\pgfqpoint{5.409062in}{2.621128in}}%
\pgfpathlineto{\pgfqpoint{5.432094in}{2.521586in}}%
\pgfpathlineto{\pgfqpoint{5.455125in}{2.401467in}}%
\pgfusepath{stroke}%
\end{pgfscope}%
\begin{pgfscope}%
\pgfsetrectcap%
\pgfsetmiterjoin%
\pgfsetlinewidth{1.254687pt}%
\definecolor{currentstroke}{rgb}{1.000000,1.000000,1.000000}%
\pgfsetstrokecolor{currentstroke}%
\pgfsetdash{}{0pt}%
\pgfpathmoveto{\pgfqpoint{3.825278in}{0.574768in}}%
\pgfpathlineto{\pgfqpoint{3.825278in}{2.932627in}}%
\pgfusepath{stroke}%
\end{pgfscope}%
\begin{pgfscope}%
\pgfsetrectcap%
\pgfsetmiterjoin%
\pgfsetlinewidth{1.254687pt}%
\definecolor{currentstroke}{rgb}{1.000000,1.000000,1.000000}%
\pgfsetstrokecolor{currentstroke}%
\pgfsetdash{}{0pt}%
\pgfpathmoveto{\pgfqpoint{6.240000in}{0.574768in}}%
\pgfpathlineto{\pgfqpoint{6.240000in}{2.932627in}}%
\pgfusepath{stroke}%
\end{pgfscope}%
\begin{pgfscope}%
\pgfsetrectcap%
\pgfsetmiterjoin%
\pgfsetlinewidth{1.254687pt}%
\definecolor{currentstroke}{rgb}{1.000000,1.000000,1.000000}%
\pgfsetstrokecolor{currentstroke}%
\pgfsetdash{}{0pt}%
\pgfpathmoveto{\pgfqpoint{3.825278in}{0.574768in}}%
\pgfpathlineto{\pgfqpoint{6.240000in}{0.574768in}}%
\pgfusepath{stroke}%
\end{pgfscope}%
\begin{pgfscope}%
\pgfsetrectcap%
\pgfsetmiterjoin%
\pgfsetlinewidth{1.254687pt}%
\definecolor{currentstroke}{rgb}{1.000000,1.000000,1.000000}%
\pgfsetstrokecolor{currentstroke}%
\pgfsetdash{}{0pt}%
\pgfpathmoveto{\pgfqpoint{3.825278in}{2.932627in}}%
\pgfpathlineto{\pgfqpoint{6.240000in}{2.932627in}}%
\pgfusepath{stroke}%
\end{pgfscope}%
\end{pgfpicture}%
\makeatother%
\endgroup%

    \caption{Here the curves of five random cluster members assigned by the \textit{gls/all-views/regular/weighted/2} model are plotted.
             Each row represents one of the seven possible strain curves in the \acrshort{4ch} view. Coloumn (a) and (b) represent cluster 1 and 2 respectively.
             To make it easier to visually separate the curves, only five random members from cluster 1 and 2 are included in the figure.}
    \label{fig:five_members_gls_rls_4CH_regular_complete_two}
\end{figure}

\clearpage

\subsection{Peak-value Clustering}

\begin{figure}[H]
    \centering
    % \includegraphics[width=\textwidth]{results/pvc_ind_dor_sens_spec_dist.png}
    %% Creator: Matplotlib, PGF backend
%%
%% To include the figure in your LaTeX document, write
%%   \input{<filename>.pgf}
%%
%% Make sure the required packages are loaded in your preamble
%%   \usepackage{pgf}
%%
%% Figures using additional raster images can only be included by \input if
%% they are in the same directory as the main LaTeX file. For loading figures
%% from other directories you can use the `import` package
%%   \usepackage{import}
%% and then include the figures with
%%   \import{<path to file>}{<filename>.pgf}
%%
%% Matplotlib used the following preamble
%%
\begingroup%
\makeatletter%
\begin{pgfpicture}%
\pgfpathrectangle{\pgfpointorigin}{\pgfqpoint{6.360708in}{2.540000in}}%
\pgfusepath{use as bounding box, clip}%
\begin{pgfscope}%
\pgfsetbuttcap%
\pgfsetmiterjoin%
\definecolor{currentfill}{rgb}{1.000000,1.000000,1.000000}%
\pgfsetfillcolor{currentfill}%
\pgfsetlinewidth{0.000000pt}%
\definecolor{currentstroke}{rgb}{1.000000,1.000000,1.000000}%
\pgfsetstrokecolor{currentstroke}%
\pgfsetdash{}{0pt}%
\pgfpathmoveto{\pgfqpoint{0.000000in}{0.000000in}}%
\pgfpathlineto{\pgfqpoint{6.360708in}{0.000000in}}%
\pgfpathlineto{\pgfqpoint{6.360708in}{2.540000in}}%
\pgfpathlineto{\pgfqpoint{0.000000in}{2.540000in}}%
\pgfpathclose%
\pgfusepath{fill}%
\end{pgfscope}%
\begin{pgfscope}%
\pgfsetbuttcap%
\pgfsetmiterjoin%
\definecolor{currentfill}{rgb}{0.917647,0.917647,0.949020}%
\pgfsetfillcolor{currentfill}%
\pgfsetlinewidth{0.000000pt}%
\definecolor{currentstroke}{rgb}{0.000000,0.000000,0.000000}%
\pgfsetstrokecolor{currentstroke}%
\pgfsetstrokeopacity{0.000000}%
\pgfsetdash{}{0pt}%
\pgfpathmoveto{\pgfqpoint{0.498727in}{0.557870in}}%
\pgfpathlineto{\pgfqpoint{3.017235in}{0.557870in}}%
\pgfpathlineto{\pgfqpoint{3.017235in}{2.242604in}}%
\pgfpathlineto{\pgfqpoint{0.498727in}{2.242604in}}%
\pgfpathclose%
\pgfusepath{fill}%
\end{pgfscope}%
\begin{pgfscope}%
\pgfpathrectangle{\pgfqpoint{0.498727in}{0.557870in}}{\pgfqpoint{2.518508in}{1.684734in}}%
\pgfusepath{clip}%
\pgfsetroundcap%
\pgfsetroundjoin%
\pgfsetlinewidth{1.003750pt}%
\definecolor{currentstroke}{rgb}{1.000000,1.000000,1.000000}%
\pgfsetstrokecolor{currentstroke}%
\pgfsetdash{}{0pt}%
\pgfpathmoveto{\pgfqpoint{0.613204in}{0.557870in}}%
\pgfpathlineto{\pgfqpoint{0.613204in}{2.242604in}}%
\pgfusepath{stroke}%
\end{pgfscope}%
\begin{pgfscope}%
\definecolor{textcolor}{rgb}{0.150000,0.150000,0.150000}%
\pgfsetstrokecolor{textcolor}%
\pgfsetfillcolor{textcolor}%
\pgftext[x=0.613204in,y=0.425926in,,top]{\color{textcolor}\sffamily\fontsize{11.000000}{13.200000}\selectfont \(\displaystyle 0\)}%
\end{pgfscope}%
\begin{pgfscope}%
\pgfpathrectangle{\pgfqpoint{0.498727in}{0.557870in}}{\pgfqpoint{2.518508in}{1.684734in}}%
\pgfusepath{clip}%
\pgfsetroundcap%
\pgfsetroundjoin%
\pgfsetlinewidth{1.003750pt}%
\definecolor{currentstroke}{rgb}{1.000000,1.000000,1.000000}%
\pgfsetstrokecolor{currentstroke}%
\pgfsetdash{}{0pt}%
\pgfpathmoveto{\pgfqpoint{1.195294in}{0.557870in}}%
\pgfpathlineto{\pgfqpoint{1.195294in}{2.242604in}}%
\pgfusepath{stroke}%
\end{pgfscope}%
\begin{pgfscope}%
\definecolor{textcolor}{rgb}{0.150000,0.150000,0.150000}%
\pgfsetstrokecolor{textcolor}%
\pgfsetfillcolor{textcolor}%
\pgftext[x=1.195294in,y=0.425926in,,top]{\color{textcolor}\sffamily\fontsize{11.000000}{13.200000}\selectfont \(\displaystyle 10\)}%
\end{pgfscope}%
\begin{pgfscope}%
\pgfpathrectangle{\pgfqpoint{0.498727in}{0.557870in}}{\pgfqpoint{2.518508in}{1.684734in}}%
\pgfusepath{clip}%
\pgfsetroundcap%
\pgfsetroundjoin%
\pgfsetlinewidth{1.003750pt}%
\definecolor{currentstroke}{rgb}{1.000000,1.000000,1.000000}%
\pgfsetstrokecolor{currentstroke}%
\pgfsetdash{}{0pt}%
\pgfpathmoveto{\pgfqpoint{1.777384in}{0.557870in}}%
\pgfpathlineto{\pgfqpoint{1.777384in}{2.242604in}}%
\pgfusepath{stroke}%
\end{pgfscope}%
\begin{pgfscope}%
\definecolor{textcolor}{rgb}{0.150000,0.150000,0.150000}%
\pgfsetstrokecolor{textcolor}%
\pgfsetfillcolor{textcolor}%
\pgftext[x=1.777384in,y=0.425926in,,top]{\color{textcolor}\sffamily\fontsize{11.000000}{13.200000}\selectfont \(\displaystyle 20\)}%
\end{pgfscope}%
\begin{pgfscope}%
\pgfpathrectangle{\pgfqpoint{0.498727in}{0.557870in}}{\pgfqpoint{2.518508in}{1.684734in}}%
\pgfusepath{clip}%
\pgfsetroundcap%
\pgfsetroundjoin%
\pgfsetlinewidth{1.003750pt}%
\definecolor{currentstroke}{rgb}{1.000000,1.000000,1.000000}%
\pgfsetstrokecolor{currentstroke}%
\pgfsetdash{}{0pt}%
\pgfpathmoveto{\pgfqpoint{2.359474in}{0.557870in}}%
\pgfpathlineto{\pgfqpoint{2.359474in}{2.242604in}}%
\pgfusepath{stroke}%
\end{pgfscope}%
\begin{pgfscope}%
\definecolor{textcolor}{rgb}{0.150000,0.150000,0.150000}%
\pgfsetstrokecolor{textcolor}%
\pgfsetfillcolor{textcolor}%
\pgftext[x=2.359474in,y=0.425926in,,top]{\color{textcolor}\sffamily\fontsize{11.000000}{13.200000}\selectfont \(\displaystyle 30\)}%
\end{pgfscope}%
\begin{pgfscope}%
\pgfpathrectangle{\pgfqpoint{0.498727in}{0.557870in}}{\pgfqpoint{2.518508in}{1.684734in}}%
\pgfusepath{clip}%
\pgfsetroundcap%
\pgfsetroundjoin%
\pgfsetlinewidth{1.003750pt}%
\definecolor{currentstroke}{rgb}{1.000000,1.000000,1.000000}%
\pgfsetstrokecolor{currentstroke}%
\pgfsetdash{}{0pt}%
\pgfpathmoveto{\pgfqpoint{2.941563in}{0.557870in}}%
\pgfpathlineto{\pgfqpoint{2.941563in}{2.242604in}}%
\pgfusepath{stroke}%
\end{pgfscope}%
\begin{pgfscope}%
\definecolor{textcolor}{rgb}{0.150000,0.150000,0.150000}%
\pgfsetstrokecolor{textcolor}%
\pgfsetfillcolor{textcolor}%
\pgftext[x=2.941563in,y=0.425926in,,top]{\color{textcolor}\sffamily\fontsize{11.000000}{13.200000}\selectfont \(\displaystyle 40\)}%
\end{pgfscope}%
\begin{pgfscope}%
\definecolor{textcolor}{rgb}{0.150000,0.150000,0.150000}%
\pgfsetstrokecolor{textcolor}%
\pgfsetfillcolor{textcolor}%
\pgftext[x=1.757981in,y=0.235185in,,top]{\color{textcolor}\sffamily\fontsize{11.000000}{13.200000}\selectfont DOR}%
\end{pgfscope}%
\begin{pgfscope}%
\pgfpathrectangle{\pgfqpoint{0.498727in}{0.557870in}}{\pgfqpoint{2.518508in}{1.684734in}}%
\pgfusepath{clip}%
\pgfsetroundcap%
\pgfsetroundjoin%
\pgfsetlinewidth{1.003750pt}%
\definecolor{currentstroke}{rgb}{1.000000,1.000000,1.000000}%
\pgfsetstrokecolor{currentstroke}%
\pgfsetdash{}{0pt}%
\pgfpathmoveto{\pgfqpoint{0.498727in}{0.557870in}}%
\pgfpathlineto{\pgfqpoint{3.017235in}{0.557870in}}%
\pgfusepath{stroke}%
\end{pgfscope}%
\begin{pgfscope}%
\definecolor{textcolor}{rgb}{0.150000,0.150000,0.150000}%
\pgfsetstrokecolor{textcolor}%
\pgfsetfillcolor{textcolor}%
\pgftext[x=0.290741in,y=0.505064in,left,base]{\color{textcolor}\sffamily\fontsize{11.000000}{13.200000}\selectfont \(\displaystyle 0\)}%
\end{pgfscope}%
\begin{pgfscope}%
\pgfpathrectangle{\pgfqpoint{0.498727in}{0.557870in}}{\pgfqpoint{2.518508in}{1.684734in}}%
\pgfusepath{clip}%
\pgfsetroundcap%
\pgfsetroundjoin%
\pgfsetlinewidth{1.003750pt}%
\definecolor{currentstroke}{rgb}{1.000000,1.000000,1.000000}%
\pgfsetstrokecolor{currentstroke}%
\pgfsetdash{}{0pt}%
\pgfpathmoveto{\pgfqpoint{0.498727in}{0.958997in}}%
\pgfpathlineto{\pgfqpoint{3.017235in}{0.958997in}}%
\pgfusepath{stroke}%
\end{pgfscope}%
\begin{pgfscope}%
\definecolor{textcolor}{rgb}{0.150000,0.150000,0.150000}%
\pgfsetstrokecolor{textcolor}%
\pgfsetfillcolor{textcolor}%
\pgftext[x=0.290741in,y=0.906191in,left,base]{\color{textcolor}\sffamily\fontsize{11.000000}{13.200000}\selectfont \(\displaystyle 2\)}%
\end{pgfscope}%
\begin{pgfscope}%
\pgfpathrectangle{\pgfqpoint{0.498727in}{0.557870in}}{\pgfqpoint{2.518508in}{1.684734in}}%
\pgfusepath{clip}%
\pgfsetroundcap%
\pgfsetroundjoin%
\pgfsetlinewidth{1.003750pt}%
\definecolor{currentstroke}{rgb}{1.000000,1.000000,1.000000}%
\pgfsetstrokecolor{currentstroke}%
\pgfsetdash{}{0pt}%
\pgfpathmoveto{\pgfqpoint{0.498727in}{1.360125in}}%
\pgfpathlineto{\pgfqpoint{3.017235in}{1.360125in}}%
\pgfusepath{stroke}%
\end{pgfscope}%
\begin{pgfscope}%
\definecolor{textcolor}{rgb}{0.150000,0.150000,0.150000}%
\pgfsetstrokecolor{textcolor}%
\pgfsetfillcolor{textcolor}%
\pgftext[x=0.290741in,y=1.307318in,left,base]{\color{textcolor}\sffamily\fontsize{11.000000}{13.200000}\selectfont \(\displaystyle 4\)}%
\end{pgfscope}%
\begin{pgfscope}%
\pgfpathrectangle{\pgfqpoint{0.498727in}{0.557870in}}{\pgfqpoint{2.518508in}{1.684734in}}%
\pgfusepath{clip}%
\pgfsetroundcap%
\pgfsetroundjoin%
\pgfsetlinewidth{1.003750pt}%
\definecolor{currentstroke}{rgb}{1.000000,1.000000,1.000000}%
\pgfsetstrokecolor{currentstroke}%
\pgfsetdash{}{0pt}%
\pgfpathmoveto{\pgfqpoint{0.498727in}{1.761252in}}%
\pgfpathlineto{\pgfqpoint{3.017235in}{1.761252in}}%
\pgfusepath{stroke}%
\end{pgfscope}%
\begin{pgfscope}%
\definecolor{textcolor}{rgb}{0.150000,0.150000,0.150000}%
\pgfsetstrokecolor{textcolor}%
\pgfsetfillcolor{textcolor}%
\pgftext[x=0.290741in,y=1.708445in,left,base]{\color{textcolor}\sffamily\fontsize{11.000000}{13.200000}\selectfont \(\displaystyle 6\)}%
\end{pgfscope}%
\begin{pgfscope}%
\pgfpathrectangle{\pgfqpoint{0.498727in}{0.557870in}}{\pgfqpoint{2.518508in}{1.684734in}}%
\pgfusepath{clip}%
\pgfsetroundcap%
\pgfsetroundjoin%
\pgfsetlinewidth{1.003750pt}%
\definecolor{currentstroke}{rgb}{1.000000,1.000000,1.000000}%
\pgfsetstrokecolor{currentstroke}%
\pgfsetdash{}{0pt}%
\pgfpathmoveto{\pgfqpoint{0.498727in}{2.162379in}}%
\pgfpathlineto{\pgfqpoint{3.017235in}{2.162379in}}%
\pgfusepath{stroke}%
\end{pgfscope}%
\begin{pgfscope}%
\definecolor{textcolor}{rgb}{0.150000,0.150000,0.150000}%
\pgfsetstrokecolor{textcolor}%
\pgfsetfillcolor{textcolor}%
\pgftext[x=0.290741in,y=2.109572in,left,base]{\color{textcolor}\sffamily\fontsize{11.000000}{13.200000}\selectfont \(\displaystyle 8\)}%
\end{pgfscope}%
\begin{pgfscope}%
\definecolor{textcolor}{rgb}{0.150000,0.150000,0.150000}%
\pgfsetstrokecolor{textcolor}%
\pgfsetfillcolor{textcolor}%
\pgftext[x=0.235185in,y=1.400237in,,bottom,rotate=90.000000]{\color{textcolor}\sffamily\fontsize{11.000000}{13.200000}\selectfont Occurance}%
\end{pgfscope}%
\begin{pgfscope}%
\pgfpathrectangle{\pgfqpoint{0.498727in}{0.557870in}}{\pgfqpoint{2.518508in}{1.684734in}}%
\pgfusepath{clip}%
\pgfsetbuttcap%
\pgfsetmiterjoin%
\definecolor{currentfill}{rgb}{0.298039,0.447059,0.690196}%
\pgfsetfillcolor{currentfill}%
\pgfsetfillopacity{0.400000}%
\pgfsetlinewidth{1.003750pt}%
\definecolor{currentstroke}{rgb}{1.000000,1.000000,1.000000}%
\pgfsetstrokecolor{currentstroke}%
\pgfsetstrokeopacity{0.400000}%
\pgfsetdash{}{0pt}%
\pgfpathmoveto{\pgfqpoint{0.613204in}{0.557870in}}%
\pgfpathlineto{\pgfqpoint{0.842160in}{0.557870in}}%
\pgfpathlineto{\pgfqpoint{0.842160in}{2.162379in}}%
\pgfpathlineto{\pgfqpoint{0.613204in}{2.162379in}}%
\pgfpathclose%
\pgfusepath{stroke,fill}%
\end{pgfscope}%
\begin{pgfscope}%
\pgfpathrectangle{\pgfqpoint{0.498727in}{0.557870in}}{\pgfqpoint{2.518508in}{1.684734in}}%
\pgfusepath{clip}%
\pgfsetbuttcap%
\pgfsetmiterjoin%
\definecolor{currentfill}{rgb}{0.298039,0.447059,0.690196}%
\pgfsetfillcolor{currentfill}%
\pgfsetfillopacity{0.400000}%
\pgfsetlinewidth{1.003750pt}%
\definecolor{currentstroke}{rgb}{1.000000,1.000000,1.000000}%
\pgfsetstrokecolor{currentstroke}%
\pgfsetstrokeopacity{0.400000}%
\pgfsetdash{}{0pt}%
\pgfpathmoveto{\pgfqpoint{0.842160in}{0.557870in}}%
\pgfpathlineto{\pgfqpoint{1.071115in}{0.557870in}}%
\pgfpathlineto{\pgfqpoint{1.071115in}{0.758434in}}%
\pgfpathlineto{\pgfqpoint{0.842160in}{0.758434in}}%
\pgfpathclose%
\pgfusepath{stroke,fill}%
\end{pgfscope}%
\begin{pgfscope}%
\pgfpathrectangle{\pgfqpoint{0.498727in}{0.557870in}}{\pgfqpoint{2.518508in}{1.684734in}}%
\pgfusepath{clip}%
\pgfsetbuttcap%
\pgfsetmiterjoin%
\definecolor{currentfill}{rgb}{0.298039,0.447059,0.690196}%
\pgfsetfillcolor{currentfill}%
\pgfsetfillopacity{0.400000}%
\pgfsetlinewidth{1.003750pt}%
\definecolor{currentstroke}{rgb}{1.000000,1.000000,1.000000}%
\pgfsetstrokecolor{currentstroke}%
\pgfsetstrokeopacity{0.400000}%
\pgfsetdash{}{0pt}%
\pgfpathmoveto{\pgfqpoint{1.071115in}{0.557870in}}%
\pgfpathlineto{\pgfqpoint{1.300070in}{0.557870in}}%
\pgfpathlineto{\pgfqpoint{1.300070in}{0.958997in}}%
\pgfpathlineto{\pgfqpoint{1.071115in}{0.958997in}}%
\pgfpathclose%
\pgfusepath{stroke,fill}%
\end{pgfscope}%
\begin{pgfscope}%
\pgfpathrectangle{\pgfqpoint{0.498727in}{0.557870in}}{\pgfqpoint{2.518508in}{1.684734in}}%
\pgfusepath{clip}%
\pgfsetbuttcap%
\pgfsetmiterjoin%
\definecolor{currentfill}{rgb}{0.298039,0.447059,0.690196}%
\pgfsetfillcolor{currentfill}%
\pgfsetfillopacity{0.400000}%
\pgfsetlinewidth{1.003750pt}%
\definecolor{currentstroke}{rgb}{1.000000,1.000000,1.000000}%
\pgfsetstrokecolor{currentstroke}%
\pgfsetstrokeopacity{0.400000}%
\pgfsetdash{}{0pt}%
\pgfpathmoveto{\pgfqpoint{1.300070in}{0.557870in}}%
\pgfpathlineto{\pgfqpoint{1.529026in}{0.557870in}}%
\pgfpathlineto{\pgfqpoint{1.529026in}{1.360125in}}%
\pgfpathlineto{\pgfqpoint{1.300070in}{1.360125in}}%
\pgfpathclose%
\pgfusepath{stroke,fill}%
\end{pgfscope}%
\begin{pgfscope}%
\pgfpathrectangle{\pgfqpoint{0.498727in}{0.557870in}}{\pgfqpoint{2.518508in}{1.684734in}}%
\pgfusepath{clip}%
\pgfsetbuttcap%
\pgfsetmiterjoin%
\definecolor{currentfill}{rgb}{0.298039,0.447059,0.690196}%
\pgfsetfillcolor{currentfill}%
\pgfsetfillopacity{0.400000}%
\pgfsetlinewidth{1.003750pt}%
\definecolor{currentstroke}{rgb}{1.000000,1.000000,1.000000}%
\pgfsetstrokecolor{currentstroke}%
\pgfsetstrokeopacity{0.400000}%
\pgfsetdash{}{0pt}%
\pgfpathmoveto{\pgfqpoint{1.529026in}{0.557870in}}%
\pgfpathlineto{\pgfqpoint{1.757981in}{0.557870in}}%
\pgfpathlineto{\pgfqpoint{1.757981in}{0.557870in}}%
\pgfpathlineto{\pgfqpoint{1.529026in}{0.557870in}}%
\pgfpathclose%
\pgfusepath{stroke,fill}%
\end{pgfscope}%
\begin{pgfscope}%
\pgfpathrectangle{\pgfqpoint{0.498727in}{0.557870in}}{\pgfqpoint{2.518508in}{1.684734in}}%
\pgfusepath{clip}%
\pgfsetbuttcap%
\pgfsetmiterjoin%
\definecolor{currentfill}{rgb}{0.298039,0.447059,0.690196}%
\pgfsetfillcolor{currentfill}%
\pgfsetfillopacity{0.400000}%
\pgfsetlinewidth{1.003750pt}%
\definecolor{currentstroke}{rgb}{1.000000,1.000000,1.000000}%
\pgfsetstrokecolor{currentstroke}%
\pgfsetstrokeopacity{0.400000}%
\pgfsetdash{}{0pt}%
\pgfpathmoveto{\pgfqpoint{1.757981in}{0.557870in}}%
\pgfpathlineto{\pgfqpoint{1.986936in}{0.557870in}}%
\pgfpathlineto{\pgfqpoint{1.986936in}{0.758434in}}%
\pgfpathlineto{\pgfqpoint{1.757981in}{0.758434in}}%
\pgfpathclose%
\pgfusepath{stroke,fill}%
\end{pgfscope}%
\begin{pgfscope}%
\pgfpathrectangle{\pgfqpoint{0.498727in}{0.557870in}}{\pgfqpoint{2.518508in}{1.684734in}}%
\pgfusepath{clip}%
\pgfsetbuttcap%
\pgfsetmiterjoin%
\definecolor{currentfill}{rgb}{0.298039,0.447059,0.690196}%
\pgfsetfillcolor{currentfill}%
\pgfsetfillopacity{0.400000}%
\pgfsetlinewidth{1.003750pt}%
\definecolor{currentstroke}{rgb}{1.000000,1.000000,1.000000}%
\pgfsetstrokecolor{currentstroke}%
\pgfsetstrokeopacity{0.400000}%
\pgfsetdash{}{0pt}%
\pgfpathmoveto{\pgfqpoint{1.986936in}{0.557870in}}%
\pgfpathlineto{\pgfqpoint{2.215892in}{0.557870in}}%
\pgfpathlineto{\pgfqpoint{2.215892in}{0.758434in}}%
\pgfpathlineto{\pgfqpoint{1.986936in}{0.758434in}}%
\pgfpathclose%
\pgfusepath{stroke,fill}%
\end{pgfscope}%
\begin{pgfscope}%
\pgfpathrectangle{\pgfqpoint{0.498727in}{0.557870in}}{\pgfqpoint{2.518508in}{1.684734in}}%
\pgfusepath{clip}%
\pgfsetbuttcap%
\pgfsetmiterjoin%
\definecolor{currentfill}{rgb}{0.298039,0.447059,0.690196}%
\pgfsetfillcolor{currentfill}%
\pgfsetfillopacity{0.400000}%
\pgfsetlinewidth{1.003750pt}%
\definecolor{currentstroke}{rgb}{1.000000,1.000000,1.000000}%
\pgfsetstrokecolor{currentstroke}%
\pgfsetstrokeopacity{0.400000}%
\pgfsetdash{}{0pt}%
\pgfpathmoveto{\pgfqpoint{2.215892in}{0.557870in}}%
\pgfpathlineto{\pgfqpoint{2.444847in}{0.557870in}}%
\pgfpathlineto{\pgfqpoint{2.444847in}{0.557870in}}%
\pgfpathlineto{\pgfqpoint{2.215892in}{0.557870in}}%
\pgfpathclose%
\pgfusepath{stroke,fill}%
\end{pgfscope}%
\begin{pgfscope}%
\pgfpathrectangle{\pgfqpoint{0.498727in}{0.557870in}}{\pgfqpoint{2.518508in}{1.684734in}}%
\pgfusepath{clip}%
\pgfsetbuttcap%
\pgfsetmiterjoin%
\definecolor{currentfill}{rgb}{0.298039,0.447059,0.690196}%
\pgfsetfillcolor{currentfill}%
\pgfsetfillopacity{0.400000}%
\pgfsetlinewidth{1.003750pt}%
\definecolor{currentstroke}{rgb}{1.000000,1.000000,1.000000}%
\pgfsetstrokecolor{currentstroke}%
\pgfsetstrokeopacity{0.400000}%
\pgfsetdash{}{0pt}%
\pgfpathmoveto{\pgfqpoint{2.444847in}{0.557870in}}%
\pgfpathlineto{\pgfqpoint{2.673802in}{0.557870in}}%
\pgfpathlineto{\pgfqpoint{2.673802in}{0.958997in}}%
\pgfpathlineto{\pgfqpoint{2.444847in}{0.958997in}}%
\pgfpathclose%
\pgfusepath{stroke,fill}%
\end{pgfscope}%
\begin{pgfscope}%
\pgfpathrectangle{\pgfqpoint{0.498727in}{0.557870in}}{\pgfqpoint{2.518508in}{1.684734in}}%
\pgfusepath{clip}%
\pgfsetbuttcap%
\pgfsetmiterjoin%
\definecolor{currentfill}{rgb}{0.298039,0.447059,0.690196}%
\pgfsetfillcolor{currentfill}%
\pgfsetfillopacity{0.400000}%
\pgfsetlinewidth{1.003750pt}%
\definecolor{currentstroke}{rgb}{1.000000,1.000000,1.000000}%
\pgfsetstrokecolor{currentstroke}%
\pgfsetstrokeopacity{0.400000}%
\pgfsetdash{}{0pt}%
\pgfpathmoveto{\pgfqpoint{2.673802in}{0.557870in}}%
\pgfpathlineto{\pgfqpoint{2.902757in}{0.557870in}}%
\pgfpathlineto{\pgfqpoint{2.902757in}{0.958997in}}%
\pgfpathlineto{\pgfqpoint{2.673802in}{0.958997in}}%
\pgfpathclose%
\pgfusepath{stroke,fill}%
\end{pgfscope}%
\begin{pgfscope}%
\pgfsetrectcap%
\pgfsetmiterjoin%
\pgfsetlinewidth{1.254687pt}%
\definecolor{currentstroke}{rgb}{1.000000,1.000000,1.000000}%
\pgfsetstrokecolor{currentstroke}%
\pgfsetdash{}{0pt}%
\pgfpathmoveto{\pgfqpoint{0.498727in}{0.557870in}}%
\pgfpathlineto{\pgfqpoint{0.498727in}{2.242604in}}%
\pgfusepath{stroke}%
\end{pgfscope}%
\begin{pgfscope}%
\pgfsetrectcap%
\pgfsetmiterjoin%
\pgfsetlinewidth{1.254687pt}%
\definecolor{currentstroke}{rgb}{1.000000,1.000000,1.000000}%
\pgfsetstrokecolor{currentstroke}%
\pgfsetdash{}{0pt}%
\pgfpathmoveto{\pgfqpoint{3.017235in}{0.557870in}}%
\pgfpathlineto{\pgfqpoint{3.017235in}{2.242604in}}%
\pgfusepath{stroke}%
\end{pgfscope}%
\begin{pgfscope}%
\pgfsetrectcap%
\pgfsetmiterjoin%
\pgfsetlinewidth{1.254687pt}%
\definecolor{currentstroke}{rgb}{1.000000,1.000000,1.000000}%
\pgfsetstrokecolor{currentstroke}%
\pgfsetdash{}{0pt}%
\pgfpathmoveto{\pgfqpoint{0.498727in}{0.557870in}}%
\pgfpathlineto{\pgfqpoint{3.017235in}{0.557870in}}%
\pgfusepath{stroke}%
\end{pgfscope}%
\begin{pgfscope}%
\pgfsetrectcap%
\pgfsetmiterjoin%
\pgfsetlinewidth{1.254687pt}%
\definecolor{currentstroke}{rgb}{1.000000,1.000000,1.000000}%
\pgfsetstrokecolor{currentstroke}%
\pgfsetdash{}{0pt}%
\pgfpathmoveto{\pgfqpoint{0.498727in}{2.242604in}}%
\pgfpathlineto{\pgfqpoint{3.017235in}{2.242604in}}%
\pgfusepath{stroke}%
\end{pgfscope}%
\begin{pgfscope}%
\definecolor{textcolor}{rgb}{0.150000,0.150000,0.150000}%
\pgfsetstrokecolor{textcolor}%
\pgfsetfillcolor{textcolor}%
\pgftext[x=1.757981in,y=2.325938in,,base]{\color{textcolor}\sffamily\fontsize{11.000000}{13.200000}\selectfont (a)}%
\end{pgfscope}%
\begin{pgfscope}%
\pgfsetbuttcap%
\pgfsetmiterjoin%
\definecolor{currentfill}{rgb}{0.917647,0.917647,0.949020}%
\pgfsetfillcolor{currentfill}%
\pgfsetlinewidth{0.000000pt}%
\definecolor{currentstroke}{rgb}{0.000000,0.000000,0.000000}%
\pgfsetstrokecolor{currentstroke}%
\pgfsetstrokeopacity{0.000000}%
\pgfsetdash{}{0pt}%
\pgfpathmoveto{\pgfqpoint{3.721492in}{0.557870in}}%
\pgfpathlineto{\pgfqpoint{6.240000in}{0.557870in}}%
\pgfpathlineto{\pgfqpoint{6.240000in}{2.242604in}}%
\pgfpathlineto{\pgfqpoint{3.721492in}{2.242604in}}%
\pgfpathclose%
\pgfusepath{fill}%
\end{pgfscope}%
\begin{pgfscope}%
\pgfpathrectangle{\pgfqpoint{3.721492in}{0.557870in}}{\pgfqpoint{2.518508in}{1.684734in}}%
\pgfusepath{clip}%
\pgfsetroundcap%
\pgfsetroundjoin%
\pgfsetlinewidth{1.003750pt}%
\definecolor{currentstroke}{rgb}{1.000000,1.000000,1.000000}%
\pgfsetstrokecolor{currentstroke}%
\pgfsetdash{}{0pt}%
\pgfpathmoveto{\pgfqpoint{3.835969in}{0.557870in}}%
\pgfpathlineto{\pgfqpoint{3.835969in}{2.242604in}}%
\pgfusepath{stroke}%
\end{pgfscope}%
\begin{pgfscope}%
\definecolor{textcolor}{rgb}{0.150000,0.150000,0.150000}%
\pgfsetstrokecolor{textcolor}%
\pgfsetfillcolor{textcolor}%
\pgftext[x=3.835969in,y=0.425926in,,top]{\color{textcolor}\sffamily\fontsize{11.000000}{13.200000}\selectfont \(\displaystyle 0.00\)}%
\end{pgfscope}%
\begin{pgfscope}%
\pgfpathrectangle{\pgfqpoint{3.721492in}{0.557870in}}{\pgfqpoint{2.518508in}{1.684734in}}%
\pgfusepath{clip}%
\pgfsetroundcap%
\pgfsetroundjoin%
\pgfsetlinewidth{1.003750pt}%
\definecolor{currentstroke}{rgb}{1.000000,1.000000,1.000000}%
\pgfsetstrokecolor{currentstroke}%
\pgfsetdash{}{0pt}%
\pgfpathmoveto{\pgfqpoint{4.408358in}{0.557870in}}%
\pgfpathlineto{\pgfqpoint{4.408358in}{2.242604in}}%
\pgfusepath{stroke}%
\end{pgfscope}%
\begin{pgfscope}%
\definecolor{textcolor}{rgb}{0.150000,0.150000,0.150000}%
\pgfsetstrokecolor{textcolor}%
\pgfsetfillcolor{textcolor}%
\pgftext[x=4.408358in,y=0.425926in,,top]{\color{textcolor}\sffamily\fontsize{11.000000}{13.200000}\selectfont \(\displaystyle 0.25\)}%
\end{pgfscope}%
\begin{pgfscope}%
\pgfpathrectangle{\pgfqpoint{3.721492in}{0.557870in}}{\pgfqpoint{2.518508in}{1.684734in}}%
\pgfusepath{clip}%
\pgfsetroundcap%
\pgfsetroundjoin%
\pgfsetlinewidth{1.003750pt}%
\definecolor{currentstroke}{rgb}{1.000000,1.000000,1.000000}%
\pgfsetstrokecolor{currentstroke}%
\pgfsetdash{}{0pt}%
\pgfpathmoveto{\pgfqpoint{4.980746in}{0.557870in}}%
\pgfpathlineto{\pgfqpoint{4.980746in}{2.242604in}}%
\pgfusepath{stroke}%
\end{pgfscope}%
\begin{pgfscope}%
\definecolor{textcolor}{rgb}{0.150000,0.150000,0.150000}%
\pgfsetstrokecolor{textcolor}%
\pgfsetfillcolor{textcolor}%
\pgftext[x=4.980746in,y=0.425926in,,top]{\color{textcolor}\sffamily\fontsize{11.000000}{13.200000}\selectfont \(\displaystyle 0.50\)}%
\end{pgfscope}%
\begin{pgfscope}%
\pgfpathrectangle{\pgfqpoint{3.721492in}{0.557870in}}{\pgfqpoint{2.518508in}{1.684734in}}%
\pgfusepath{clip}%
\pgfsetroundcap%
\pgfsetroundjoin%
\pgfsetlinewidth{1.003750pt}%
\definecolor{currentstroke}{rgb}{1.000000,1.000000,1.000000}%
\pgfsetstrokecolor{currentstroke}%
\pgfsetdash{}{0pt}%
\pgfpathmoveto{\pgfqpoint{5.553134in}{0.557870in}}%
\pgfpathlineto{\pgfqpoint{5.553134in}{2.242604in}}%
\pgfusepath{stroke}%
\end{pgfscope}%
\begin{pgfscope}%
\definecolor{textcolor}{rgb}{0.150000,0.150000,0.150000}%
\pgfsetstrokecolor{textcolor}%
\pgfsetfillcolor{textcolor}%
\pgftext[x=5.553134in,y=0.425926in,,top]{\color{textcolor}\sffamily\fontsize{11.000000}{13.200000}\selectfont \(\displaystyle 0.75\)}%
\end{pgfscope}%
\begin{pgfscope}%
\pgfpathrectangle{\pgfqpoint{3.721492in}{0.557870in}}{\pgfqpoint{2.518508in}{1.684734in}}%
\pgfusepath{clip}%
\pgfsetroundcap%
\pgfsetroundjoin%
\pgfsetlinewidth{1.003750pt}%
\definecolor{currentstroke}{rgb}{1.000000,1.000000,1.000000}%
\pgfsetstrokecolor{currentstroke}%
\pgfsetdash{}{0pt}%
\pgfpathmoveto{\pgfqpoint{6.125522in}{0.557870in}}%
\pgfpathlineto{\pgfqpoint{6.125522in}{2.242604in}}%
\pgfusepath{stroke}%
\end{pgfscope}%
\begin{pgfscope}%
\definecolor{textcolor}{rgb}{0.150000,0.150000,0.150000}%
\pgfsetstrokecolor{textcolor}%
\pgfsetfillcolor{textcolor}%
\pgftext[x=6.125522in,y=0.425926in,,top]{\color{textcolor}\sffamily\fontsize{11.000000}{13.200000}\selectfont \(\displaystyle 1.00\)}%
\end{pgfscope}%
\begin{pgfscope}%
\definecolor{textcolor}{rgb}{0.150000,0.150000,0.150000}%
\pgfsetstrokecolor{textcolor}%
\pgfsetfillcolor{textcolor}%
\pgftext[x=4.980746in,y=0.235185in,,top]{\color{textcolor}\sffamily\fontsize{11.000000}{13.200000}\selectfont Specificity}%
\end{pgfscope}%
\begin{pgfscope}%
\pgfpathrectangle{\pgfqpoint{3.721492in}{0.557870in}}{\pgfqpoint{2.518508in}{1.684734in}}%
\pgfusepath{clip}%
\pgfsetroundcap%
\pgfsetroundjoin%
\pgfsetlinewidth{1.003750pt}%
\definecolor{currentstroke}{rgb}{1.000000,1.000000,1.000000}%
\pgfsetstrokecolor{currentstroke}%
\pgfsetdash{}{0pt}%
\pgfpathmoveto{\pgfqpoint{3.721492in}{0.634449in}}%
\pgfpathlineto{\pgfqpoint{6.240000in}{0.634449in}}%
\pgfusepath{stroke}%
\end{pgfscope}%
\begin{pgfscope}%
\definecolor{textcolor}{rgb}{0.150000,0.150000,0.150000}%
\pgfsetstrokecolor{textcolor}%
\pgfsetfillcolor{textcolor}%
\pgftext[x=3.319177in,y=0.581642in,left,base]{\color{textcolor}\sffamily\fontsize{11.000000}{13.200000}\selectfont \(\displaystyle 0.00\)}%
\end{pgfscope}%
\begin{pgfscope}%
\pgfpathrectangle{\pgfqpoint{3.721492in}{0.557870in}}{\pgfqpoint{2.518508in}{1.684734in}}%
\pgfusepath{clip}%
\pgfsetroundcap%
\pgfsetroundjoin%
\pgfsetlinewidth{1.003750pt}%
\definecolor{currentstroke}{rgb}{1.000000,1.000000,1.000000}%
\pgfsetstrokecolor{currentstroke}%
\pgfsetdash{}{0pt}%
\pgfpathmoveto{\pgfqpoint{3.721492in}{1.017343in}}%
\pgfpathlineto{\pgfqpoint{6.240000in}{1.017343in}}%
\pgfusepath{stroke}%
\end{pgfscope}%
\begin{pgfscope}%
\definecolor{textcolor}{rgb}{0.150000,0.150000,0.150000}%
\pgfsetstrokecolor{textcolor}%
\pgfsetfillcolor{textcolor}%
\pgftext[x=3.319177in,y=0.964536in,left,base]{\color{textcolor}\sffamily\fontsize{11.000000}{13.200000}\selectfont \(\displaystyle 0.25\)}%
\end{pgfscope}%
\begin{pgfscope}%
\pgfpathrectangle{\pgfqpoint{3.721492in}{0.557870in}}{\pgfqpoint{2.518508in}{1.684734in}}%
\pgfusepath{clip}%
\pgfsetroundcap%
\pgfsetroundjoin%
\pgfsetlinewidth{1.003750pt}%
\definecolor{currentstroke}{rgb}{1.000000,1.000000,1.000000}%
\pgfsetstrokecolor{currentstroke}%
\pgfsetdash{}{0pt}%
\pgfpathmoveto{\pgfqpoint{3.721492in}{1.400237in}}%
\pgfpathlineto{\pgfqpoint{6.240000in}{1.400237in}}%
\pgfusepath{stroke}%
\end{pgfscope}%
\begin{pgfscope}%
\definecolor{textcolor}{rgb}{0.150000,0.150000,0.150000}%
\pgfsetstrokecolor{textcolor}%
\pgfsetfillcolor{textcolor}%
\pgftext[x=3.319177in,y=1.347431in,left,base]{\color{textcolor}\sffamily\fontsize{11.000000}{13.200000}\selectfont \(\displaystyle 0.50\)}%
\end{pgfscope}%
\begin{pgfscope}%
\pgfpathrectangle{\pgfqpoint{3.721492in}{0.557870in}}{\pgfqpoint{2.518508in}{1.684734in}}%
\pgfusepath{clip}%
\pgfsetroundcap%
\pgfsetroundjoin%
\pgfsetlinewidth{1.003750pt}%
\definecolor{currentstroke}{rgb}{1.000000,1.000000,1.000000}%
\pgfsetstrokecolor{currentstroke}%
\pgfsetdash{}{0pt}%
\pgfpathmoveto{\pgfqpoint{3.721492in}{1.783131in}}%
\pgfpathlineto{\pgfqpoint{6.240000in}{1.783131in}}%
\pgfusepath{stroke}%
\end{pgfscope}%
\begin{pgfscope}%
\definecolor{textcolor}{rgb}{0.150000,0.150000,0.150000}%
\pgfsetstrokecolor{textcolor}%
\pgfsetfillcolor{textcolor}%
\pgftext[x=3.319177in,y=1.730325in,left,base]{\color{textcolor}\sffamily\fontsize{11.000000}{13.200000}\selectfont \(\displaystyle 0.75\)}%
\end{pgfscope}%
\begin{pgfscope}%
\pgfpathrectangle{\pgfqpoint{3.721492in}{0.557870in}}{\pgfqpoint{2.518508in}{1.684734in}}%
\pgfusepath{clip}%
\pgfsetroundcap%
\pgfsetroundjoin%
\pgfsetlinewidth{1.003750pt}%
\definecolor{currentstroke}{rgb}{1.000000,1.000000,1.000000}%
\pgfsetstrokecolor{currentstroke}%
\pgfsetdash{}{0pt}%
\pgfpathmoveto{\pgfqpoint{3.721492in}{2.166025in}}%
\pgfpathlineto{\pgfqpoint{6.240000in}{2.166025in}}%
\pgfusepath{stroke}%
\end{pgfscope}%
\begin{pgfscope}%
\definecolor{textcolor}{rgb}{0.150000,0.150000,0.150000}%
\pgfsetstrokecolor{textcolor}%
\pgfsetfillcolor{textcolor}%
\pgftext[x=3.319177in,y=2.113219in,left,base]{\color{textcolor}\sffamily\fontsize{11.000000}{13.200000}\selectfont \(\displaystyle 1.00\)}%
\end{pgfscope}%
\begin{pgfscope}%
\definecolor{textcolor}{rgb}{0.150000,0.150000,0.150000}%
\pgfsetstrokecolor{textcolor}%
\pgfsetfillcolor{textcolor}%
\pgftext[x=3.263621in,y=1.400237in,,bottom,rotate=90.000000]{\color{textcolor}\sffamily\fontsize{11.000000}{13.200000}\selectfont Sensitivity}%
\end{pgfscope}%
\begin{pgfscope}%
\pgfpathrectangle{\pgfqpoint{3.721492in}{0.557870in}}{\pgfqpoint{2.518508in}{1.684734in}}%
\pgfusepath{clip}%
\pgfsetbuttcap%
\pgfsetroundjoin%
\definecolor{currentfill}{rgb}{0.298039,0.447059,0.690196}%
\pgfsetfillcolor{currentfill}%
\pgfsetlinewidth{1.003750pt}%
\definecolor{currentstroke}{rgb}{0.298039,0.447059,0.690196}%
\pgfsetstrokecolor{currentstroke}%
\pgfsetdash{}{0pt}%
\pgfpathmoveto{\pgfqpoint{4.336809in}{1.890668in}}%
\pgfpathcurveto{\pgfqpoint{4.345045in}{1.890668in}}{\pgfqpoint{4.352945in}{1.893941in}}{\pgfqpoint{4.358769in}{1.899765in}}%
\pgfpathcurveto{\pgfqpoint{4.364593in}{1.905589in}}{\pgfqpoint{4.367866in}{1.913489in}}{\pgfqpoint{4.367866in}{1.921725in}}%
\pgfpathcurveto{\pgfqpoint{4.367866in}{1.929961in}}{\pgfqpoint{4.364593in}{1.937861in}}{\pgfqpoint{4.358769in}{1.943685in}}%
\pgfpathcurveto{\pgfqpoint{4.352945in}{1.949509in}}{\pgfqpoint{4.345045in}{1.952781in}}{\pgfqpoint{4.336809in}{1.952781in}}%
\pgfpathcurveto{\pgfqpoint{4.328573in}{1.952781in}}{\pgfqpoint{4.320673in}{1.949509in}}{\pgfqpoint{4.314849in}{1.943685in}}%
\pgfpathcurveto{\pgfqpoint{4.309025in}{1.937861in}}{\pgfqpoint{4.305753in}{1.929961in}}{\pgfqpoint{4.305753in}{1.921725in}}%
\pgfpathcurveto{\pgfqpoint{4.305753in}{1.913489in}}{\pgfqpoint{4.309025in}{1.905589in}}{\pgfqpoint{4.314849in}{1.899765in}}%
\pgfpathcurveto{\pgfqpoint{4.320673in}{1.893941in}}{\pgfqpoint{4.328573in}{1.890668in}}{\pgfqpoint{4.336809in}{1.890668in}}%
\pgfpathclose%
\pgfusepath{stroke,fill}%
\end{pgfscope}%
\begin{pgfscope}%
\pgfpathrectangle{\pgfqpoint{3.721492in}{0.557870in}}{\pgfqpoint{2.518508in}{1.684734in}}%
\pgfusepath{clip}%
\pgfsetbuttcap%
\pgfsetroundjoin%
\definecolor{currentfill}{rgb}{0.298039,0.447059,0.690196}%
\pgfsetfillcolor{currentfill}%
\pgfsetlinewidth{1.003750pt}%
\definecolor{currentstroke}{rgb}{0.298039,0.447059,0.690196}%
\pgfsetstrokecolor{currentstroke}%
\pgfsetdash{}{0pt}%
\pgfpathmoveto{\pgfqpoint{5.910877in}{1.373879in}}%
\pgfpathcurveto{\pgfqpoint{5.919113in}{1.373879in}}{\pgfqpoint{5.927013in}{1.377151in}}{\pgfqpoint{5.932837in}{1.382975in}}%
\pgfpathcurveto{\pgfqpoint{5.938661in}{1.388799in}}{\pgfqpoint{5.941933in}{1.396699in}}{\pgfqpoint{5.941933in}{1.404935in}}%
\pgfpathcurveto{\pgfqpoint{5.941933in}{1.413172in}}{\pgfqpoint{5.938661in}{1.421072in}}{\pgfqpoint{5.932837in}{1.426896in}}%
\pgfpathcurveto{\pgfqpoint{5.927013in}{1.432719in}}{\pgfqpoint{5.919113in}{1.435992in}}{\pgfqpoint{5.910877in}{1.435992in}}%
\pgfpathcurveto{\pgfqpoint{5.902640in}{1.435992in}}{\pgfqpoint{5.894740in}{1.432719in}}{\pgfqpoint{5.888917in}{1.426896in}}%
\pgfpathcurveto{\pgfqpoint{5.883093in}{1.421072in}}{\pgfqpoint{5.879820in}{1.413172in}}{\pgfqpoint{5.879820in}{1.404935in}}%
\pgfpathcurveto{\pgfqpoint{5.879820in}{1.396699in}}{\pgfqpoint{5.883093in}{1.388799in}}{\pgfqpoint{5.888917in}{1.382975in}}%
\pgfpathcurveto{\pgfqpoint{5.894740in}{1.377151in}}{\pgfqpoint{5.902640in}{1.373879in}}{\pgfqpoint{5.910877in}{1.373879in}}%
\pgfpathclose%
\pgfusepath{stroke,fill}%
\end{pgfscope}%
\begin{pgfscope}%
\pgfpathrectangle{\pgfqpoint{3.721492in}{0.557870in}}{\pgfqpoint{2.518508in}{1.684734in}}%
\pgfusepath{clip}%
\pgfsetbuttcap%
\pgfsetroundjoin%
\definecolor{currentfill}{rgb}{0.298039,0.447059,0.690196}%
\pgfsetfillcolor{currentfill}%
\pgfsetlinewidth{1.003750pt}%
\definecolor{currentstroke}{rgb}{0.298039,0.447059,0.690196}%
\pgfsetstrokecolor{currentstroke}%
\pgfsetdash{}{0pt}%
\pgfpathmoveto{\pgfqpoint{3.835969in}{2.125336in}}%
\pgfpathcurveto{\pgfqpoint{3.844206in}{2.125336in}}{\pgfqpoint{3.852106in}{2.128609in}}{\pgfqpoint{3.857930in}{2.134433in}}%
\pgfpathcurveto{\pgfqpoint{3.863754in}{2.140256in}}{\pgfqpoint{3.867026in}{2.148157in}}{\pgfqpoint{3.867026in}{2.156393in}}%
\pgfpathcurveto{\pgfqpoint{3.867026in}{2.164629in}}{\pgfqpoint{3.863754in}{2.172529in}}{\pgfqpoint{3.857930in}{2.178353in}}%
\pgfpathcurveto{\pgfqpoint{3.852106in}{2.184177in}}{\pgfqpoint{3.844206in}{2.187449in}}{\pgfqpoint{3.835969in}{2.187449in}}%
\pgfpathcurveto{\pgfqpoint{3.827733in}{2.187449in}}{\pgfqpoint{3.819833in}{2.184177in}}{\pgfqpoint{3.814009in}{2.178353in}}%
\pgfpathcurveto{\pgfqpoint{3.808185in}{2.172529in}}{\pgfqpoint{3.804913in}{2.164629in}}{\pgfqpoint{3.804913in}{2.156393in}}%
\pgfpathcurveto{\pgfqpoint{3.804913in}{2.148157in}}{\pgfqpoint{3.808185in}{2.140256in}}{\pgfqpoint{3.814009in}{2.134433in}}%
\pgfpathcurveto{\pgfqpoint{3.819833in}{2.128609in}}{\pgfqpoint{3.827733in}{2.125336in}}{\pgfqpoint{3.835969in}{2.125336in}}%
\pgfpathclose%
\pgfusepath{stroke,fill}%
\end{pgfscope}%
\begin{pgfscope}%
\pgfpathrectangle{\pgfqpoint{3.721492in}{0.557870in}}{\pgfqpoint{2.518508in}{1.684734in}}%
\pgfusepath{clip}%
\pgfsetbuttcap%
\pgfsetroundjoin%
\definecolor{currentfill}{rgb}{0.298039,0.447059,0.690196}%
\pgfsetfillcolor{currentfill}%
\pgfsetlinewidth{1.003750pt}%
\definecolor{currentstroke}{rgb}{0.298039,0.447059,0.690196}%
\pgfsetstrokecolor{currentstroke}%
\pgfsetdash{}{0pt}%
\pgfpathmoveto{\pgfqpoint{5.967622in}{1.383630in}}%
\pgfpathcurveto{\pgfqpoint{5.975858in}{1.383630in}}{\pgfqpoint{5.983758in}{1.386902in}}{\pgfqpoint{5.989582in}{1.392726in}}%
\pgfpathcurveto{\pgfqpoint{5.995406in}{1.398550in}}{\pgfqpoint{5.998679in}{1.406450in}}{\pgfqpoint{5.998679in}{1.414686in}}%
\pgfpathcurveto{\pgfqpoint{5.998679in}{1.422922in}}{\pgfqpoint{5.995406in}{1.430822in}}{\pgfqpoint{5.989582in}{1.436646in}}%
\pgfpathcurveto{\pgfqpoint{5.983758in}{1.442470in}}{\pgfqpoint{5.975858in}{1.445743in}}{\pgfqpoint{5.967622in}{1.445743in}}%
\pgfpathcurveto{\pgfqpoint{5.959386in}{1.445743in}}{\pgfqpoint{5.951486in}{1.442470in}}{\pgfqpoint{5.945662in}{1.436646in}}%
\pgfpathcurveto{\pgfqpoint{5.939838in}{1.430822in}}{\pgfqpoint{5.936566in}{1.422922in}}{\pgfqpoint{5.936566in}{1.414686in}}%
\pgfpathcurveto{\pgfqpoint{5.936566in}{1.406450in}}{\pgfqpoint{5.939838in}{1.398550in}}{\pgfqpoint{5.945662in}{1.392726in}}%
\pgfpathcurveto{\pgfqpoint{5.951486in}{1.386902in}}{\pgfqpoint{5.959386in}{1.383630in}}{\pgfqpoint{5.967622in}{1.383630in}}%
\pgfpathclose%
\pgfusepath{stroke,fill}%
\end{pgfscope}%
\begin{pgfscope}%
\pgfpathrectangle{\pgfqpoint{3.721492in}{0.557870in}}{\pgfqpoint{2.518508in}{1.684734in}}%
\pgfusepath{clip}%
\pgfsetbuttcap%
\pgfsetroundjoin%
\definecolor{currentfill}{rgb}{0.298039,0.447059,0.690196}%
\pgfsetfillcolor{currentfill}%
\pgfsetlinewidth{1.003750pt}%
\definecolor{currentstroke}{rgb}{0.298039,0.447059,0.690196}%
\pgfsetstrokecolor{currentstroke}%
\pgfsetdash{}{0pt}%
\pgfpathmoveto{\pgfqpoint{5.967622in}{1.354732in}}%
\pgfpathcurveto{\pgfqpoint{5.975858in}{1.354732in}}{\pgfqpoint{5.983758in}{1.358004in}}{\pgfqpoint{5.989582in}{1.363828in}}%
\pgfpathcurveto{\pgfqpoint{5.995406in}{1.369652in}}{\pgfqpoint{5.998679in}{1.377552in}}{\pgfqpoint{5.998679in}{1.385788in}}%
\pgfpathcurveto{\pgfqpoint{5.998679in}{1.394025in}}{\pgfqpoint{5.995406in}{1.401925in}}{\pgfqpoint{5.989582in}{1.407749in}}%
\pgfpathcurveto{\pgfqpoint{5.983758in}{1.413573in}}{\pgfqpoint{5.975858in}{1.416845in}}{\pgfqpoint{5.967622in}{1.416845in}}%
\pgfpathcurveto{\pgfqpoint{5.959386in}{1.416845in}}{\pgfqpoint{5.951486in}{1.413573in}}{\pgfqpoint{5.945662in}{1.407749in}}%
\pgfpathcurveto{\pgfqpoint{5.939838in}{1.401925in}}{\pgfqpoint{5.936566in}{1.394025in}}{\pgfqpoint{5.936566in}{1.385788in}}%
\pgfpathcurveto{\pgfqpoint{5.936566in}{1.377552in}}{\pgfqpoint{5.939838in}{1.369652in}}{\pgfqpoint{5.945662in}{1.363828in}}%
\pgfpathcurveto{\pgfqpoint{5.951486in}{1.358004in}}{\pgfqpoint{5.959386in}{1.354732in}}{\pgfqpoint{5.967622in}{1.354732in}}%
\pgfpathclose%
\pgfusepath{stroke,fill}%
\end{pgfscope}%
\begin{pgfscope}%
\pgfpathrectangle{\pgfqpoint{3.721492in}{0.557870in}}{\pgfqpoint{2.518508in}{1.684734in}}%
\pgfusepath{clip}%
\pgfsetbuttcap%
\pgfsetroundjoin%
\definecolor{currentfill}{rgb}{0.298039,0.447059,0.690196}%
\pgfsetfillcolor{currentfill}%
\pgfsetlinewidth{1.003750pt}%
\definecolor{currentstroke}{rgb}{0.298039,0.447059,0.690196}%
\pgfsetstrokecolor{currentstroke}%
\pgfsetdash{}{0pt}%
\pgfpathmoveto{\pgfqpoint{5.967622in}{1.325834in}}%
\pgfpathcurveto{\pgfqpoint{5.975858in}{1.325834in}}{\pgfqpoint{5.983758in}{1.329107in}}{\pgfqpoint{5.989582in}{1.334930in}}%
\pgfpathcurveto{\pgfqpoint{5.995406in}{1.340754in}}{\pgfqpoint{5.998679in}{1.348654in}}{\pgfqpoint{5.998679in}{1.356891in}}%
\pgfpathcurveto{\pgfqpoint{5.998679in}{1.365127in}}{\pgfqpoint{5.995406in}{1.373027in}}{\pgfqpoint{5.989582in}{1.378851in}}%
\pgfpathcurveto{\pgfqpoint{5.983758in}{1.384675in}}{\pgfqpoint{5.975858in}{1.387947in}}{\pgfqpoint{5.967622in}{1.387947in}}%
\pgfpathcurveto{\pgfqpoint{5.959386in}{1.387947in}}{\pgfqpoint{5.951486in}{1.384675in}}{\pgfqpoint{5.945662in}{1.378851in}}%
\pgfpathcurveto{\pgfqpoint{5.939838in}{1.373027in}}{\pgfqpoint{5.936566in}{1.365127in}}{\pgfqpoint{5.936566in}{1.356891in}}%
\pgfpathcurveto{\pgfqpoint{5.936566in}{1.348654in}}{\pgfqpoint{5.939838in}{1.340754in}}{\pgfqpoint{5.945662in}{1.334930in}}%
\pgfpathcurveto{\pgfqpoint{5.951486in}{1.329107in}}{\pgfqpoint{5.959386in}{1.325834in}}{\pgfqpoint{5.967622in}{1.325834in}}%
\pgfpathclose%
\pgfusepath{stroke,fill}%
\end{pgfscope}%
\begin{pgfscope}%
\pgfpathrectangle{\pgfqpoint{3.721492in}{0.557870in}}{\pgfqpoint{2.518508in}{1.684734in}}%
\pgfusepath{clip}%
\pgfsetbuttcap%
\pgfsetroundjoin%
\definecolor{currentfill}{rgb}{0.298039,0.447059,0.690196}%
\pgfsetfillcolor{currentfill}%
\pgfsetlinewidth{1.003750pt}%
\definecolor{currentstroke}{rgb}{0.298039,0.447059,0.690196}%
\pgfsetstrokecolor{currentstroke}%
\pgfsetdash{}{0pt}%
\pgfpathmoveto{\pgfqpoint{3.835969in}{2.085563in}}%
\pgfpathcurveto{\pgfqpoint{3.844206in}{2.085563in}}{\pgfqpoint{3.852106in}{2.088835in}}{\pgfqpoint{3.857930in}{2.094659in}}%
\pgfpathcurveto{\pgfqpoint{3.863754in}{2.100483in}}{\pgfqpoint{3.867026in}{2.108383in}}{\pgfqpoint{3.867026in}{2.116620in}}%
\pgfpathcurveto{\pgfqpoint{3.867026in}{2.124856in}}{\pgfqpoint{3.863754in}{2.132756in}}{\pgfqpoint{3.857930in}{2.138580in}}%
\pgfpathcurveto{\pgfqpoint{3.852106in}{2.144404in}}{\pgfqpoint{3.844206in}{2.147676in}}{\pgfqpoint{3.835969in}{2.147676in}}%
\pgfpathcurveto{\pgfqpoint{3.827733in}{2.147676in}}{\pgfqpoint{3.819833in}{2.144404in}}{\pgfqpoint{3.814009in}{2.138580in}}%
\pgfpathcurveto{\pgfqpoint{3.808185in}{2.132756in}}{\pgfqpoint{3.804913in}{2.124856in}}{\pgfqpoint{3.804913in}{2.116620in}}%
\pgfpathcurveto{\pgfqpoint{3.804913in}{2.108383in}}{\pgfqpoint{3.808185in}{2.100483in}}{\pgfqpoint{3.814009in}{2.094659in}}%
\pgfpathcurveto{\pgfqpoint{3.819833in}{2.088835in}}{\pgfqpoint{3.827733in}{2.085563in}}{\pgfqpoint{3.835969in}{2.085563in}}%
\pgfpathclose%
\pgfusepath{stroke,fill}%
\end{pgfscope}%
\begin{pgfscope}%
\pgfpathrectangle{\pgfqpoint{3.721492in}{0.557870in}}{\pgfqpoint{2.518508in}{1.684734in}}%
\pgfusepath{clip}%
\pgfsetbuttcap%
\pgfsetroundjoin%
\definecolor{currentfill}{rgb}{0.298039,0.447059,0.690196}%
\pgfsetfillcolor{currentfill}%
\pgfsetlinewidth{1.003750pt}%
\definecolor{currentstroke}{rgb}{0.298039,0.447059,0.690196}%
\pgfsetstrokecolor{currentstroke}%
\pgfsetdash{}{0pt}%
\pgfpathmoveto{\pgfqpoint{3.993870in}{2.115207in}}%
\pgfpathcurveto{\pgfqpoint{4.002106in}{2.115207in}}{\pgfqpoint{4.010006in}{2.118479in}}{\pgfqpoint{4.015830in}{2.124303in}}%
\pgfpathcurveto{\pgfqpoint{4.021654in}{2.130127in}}{\pgfqpoint{4.024926in}{2.138027in}}{\pgfqpoint{4.024926in}{2.146263in}}%
\pgfpathcurveto{\pgfqpoint{4.024926in}{2.154499in}}{\pgfqpoint{4.021654in}{2.162399in}}{\pgfqpoint{4.015830in}{2.168223in}}%
\pgfpathcurveto{\pgfqpoint{4.010006in}{2.174047in}}{\pgfqpoint{4.002106in}{2.177320in}}{\pgfqpoint{3.993870in}{2.177320in}}%
\pgfpathcurveto{\pgfqpoint{3.985633in}{2.177320in}}{\pgfqpoint{3.977733in}{2.174047in}}{\pgfqpoint{3.971909in}{2.168223in}}%
\pgfpathcurveto{\pgfqpoint{3.966085in}{2.162399in}}{\pgfqpoint{3.962813in}{2.154499in}}{\pgfqpoint{3.962813in}{2.146263in}}%
\pgfpathcurveto{\pgfqpoint{3.962813in}{2.138027in}}{\pgfqpoint{3.966085in}{2.130127in}}{\pgfqpoint{3.971909in}{2.124303in}}%
\pgfpathcurveto{\pgfqpoint{3.977733in}{2.118479in}}{\pgfqpoint{3.985633in}{2.115207in}}{\pgfqpoint{3.993870in}{2.115207in}}%
\pgfpathclose%
\pgfusepath{stroke,fill}%
\end{pgfscope}%
\begin{pgfscope}%
\pgfpathrectangle{\pgfqpoint{3.721492in}{0.557870in}}{\pgfqpoint{2.518508in}{1.684734in}}%
\pgfusepath{clip}%
\pgfsetbuttcap%
\pgfsetroundjoin%
\definecolor{currentfill}{rgb}{0.298039,0.447059,0.690196}%
\pgfsetfillcolor{currentfill}%
\pgfsetlinewidth{1.003750pt}%
\definecolor{currentstroke}{rgb}{0.298039,0.447059,0.690196}%
\pgfsetstrokecolor{currentstroke}%
\pgfsetdash{}{0pt}%
\pgfpathmoveto{\pgfqpoint{5.967622in}{1.334597in}}%
\pgfpathcurveto{\pgfqpoint{5.975858in}{1.334597in}}{\pgfqpoint{5.983758in}{1.337869in}}{\pgfqpoint{5.989582in}{1.343693in}}%
\pgfpathcurveto{\pgfqpoint{5.995406in}{1.349517in}}{\pgfqpoint{5.998679in}{1.357417in}}{\pgfqpoint{5.998679in}{1.365653in}}%
\pgfpathcurveto{\pgfqpoint{5.998679in}{1.373890in}}{\pgfqpoint{5.995406in}{1.381790in}}{\pgfqpoint{5.989582in}{1.387614in}}%
\pgfpathcurveto{\pgfqpoint{5.983758in}{1.393437in}}{\pgfqpoint{5.975858in}{1.396710in}}{\pgfqpoint{5.967622in}{1.396710in}}%
\pgfpathcurveto{\pgfqpoint{5.959386in}{1.396710in}}{\pgfqpoint{5.951486in}{1.393437in}}{\pgfqpoint{5.945662in}{1.387614in}}%
\pgfpathcurveto{\pgfqpoint{5.939838in}{1.381790in}}{\pgfqpoint{5.936566in}{1.373890in}}{\pgfqpoint{5.936566in}{1.365653in}}%
\pgfpathcurveto{\pgfqpoint{5.936566in}{1.357417in}}{\pgfqpoint{5.939838in}{1.349517in}}{\pgfqpoint{5.945662in}{1.343693in}}%
\pgfpathcurveto{\pgfqpoint{5.951486in}{1.337869in}}{\pgfqpoint{5.959386in}{1.334597in}}{\pgfqpoint{5.967622in}{1.334597in}}%
\pgfpathclose%
\pgfusepath{stroke,fill}%
\end{pgfscope}%
\begin{pgfscope}%
\pgfpathrectangle{\pgfqpoint{3.721492in}{0.557870in}}{\pgfqpoint{2.518508in}{1.684734in}}%
\pgfusepath{clip}%
\pgfsetbuttcap%
\pgfsetroundjoin%
\definecolor{currentfill}{rgb}{0.298039,0.447059,0.690196}%
\pgfsetfillcolor{currentfill}%
\pgfsetlinewidth{1.003750pt}%
\definecolor{currentstroke}{rgb}{0.298039,0.447059,0.690196}%
\pgfsetstrokecolor{currentstroke}%
\pgfsetdash{}{0pt}%
\pgfpathmoveto{\pgfqpoint{3.835969in}{2.125573in}}%
\pgfpathcurveto{\pgfqpoint{3.844206in}{2.125573in}}{\pgfqpoint{3.852106in}{2.128845in}}{\pgfqpoint{3.857930in}{2.134669in}}%
\pgfpathcurveto{\pgfqpoint{3.863754in}{2.140493in}}{\pgfqpoint{3.867026in}{2.148393in}}{\pgfqpoint{3.867026in}{2.156629in}}%
\pgfpathcurveto{\pgfqpoint{3.867026in}{2.164865in}}{\pgfqpoint{3.863754in}{2.172766in}}{\pgfqpoint{3.857930in}{2.178589in}}%
\pgfpathcurveto{\pgfqpoint{3.852106in}{2.184413in}}{\pgfqpoint{3.844206in}{2.187686in}}{\pgfqpoint{3.835969in}{2.187686in}}%
\pgfpathcurveto{\pgfqpoint{3.827733in}{2.187686in}}{\pgfqpoint{3.819833in}{2.184413in}}{\pgfqpoint{3.814009in}{2.178589in}}%
\pgfpathcurveto{\pgfqpoint{3.808185in}{2.172766in}}{\pgfqpoint{3.804913in}{2.164865in}}{\pgfqpoint{3.804913in}{2.156629in}}%
\pgfpathcurveto{\pgfqpoint{3.804913in}{2.148393in}}{\pgfqpoint{3.808185in}{2.140493in}}{\pgfqpoint{3.814009in}{2.134669in}}%
\pgfpathcurveto{\pgfqpoint{3.819833in}{2.128845in}}{\pgfqpoint{3.827733in}{2.125573in}}{\pgfqpoint{3.835969in}{2.125573in}}%
\pgfpathclose%
\pgfusepath{stroke,fill}%
\end{pgfscope}%
\begin{pgfscope}%
\pgfpathrectangle{\pgfqpoint{3.721492in}{0.557870in}}{\pgfqpoint{2.518508in}{1.684734in}}%
\pgfusepath{clip}%
\pgfsetbuttcap%
\pgfsetroundjoin%
\definecolor{currentfill}{rgb}{0.298039,0.447059,0.690196}%
\pgfsetfillcolor{currentfill}%
\pgfsetlinewidth{1.003750pt}%
\definecolor{currentstroke}{rgb}{0.298039,0.447059,0.690196}%
\pgfsetstrokecolor{currentstroke}%
\pgfsetdash{}{0pt}%
\pgfpathmoveto{\pgfqpoint{5.982425in}{1.571198in}}%
\pgfpathcurveto{\pgfqpoint{5.990662in}{1.571198in}}{\pgfqpoint{5.998562in}{1.574471in}}{\pgfqpoint{6.004386in}{1.580295in}}%
\pgfpathcurveto{\pgfqpoint{6.010209in}{1.586119in}}{\pgfqpoint{6.013482in}{1.594019in}}{\pgfqpoint{6.013482in}{1.602255in}}%
\pgfpathcurveto{\pgfqpoint{6.013482in}{1.610491in}}{\pgfqpoint{6.010209in}{1.618391in}}{\pgfqpoint{6.004386in}{1.624215in}}%
\pgfpathcurveto{\pgfqpoint{5.998562in}{1.630039in}}{\pgfqpoint{5.990662in}{1.633311in}}{\pgfqpoint{5.982425in}{1.633311in}}%
\pgfpathcurveto{\pgfqpoint{5.974189in}{1.633311in}}{\pgfqpoint{5.966289in}{1.630039in}}{\pgfqpoint{5.960465in}{1.624215in}}%
\pgfpathcurveto{\pgfqpoint{5.954641in}{1.618391in}}{\pgfqpoint{5.951369in}{1.610491in}}{\pgfqpoint{5.951369in}{1.602255in}}%
\pgfpathcurveto{\pgfqpoint{5.951369in}{1.594019in}}{\pgfqpoint{5.954641in}{1.586119in}}{\pgfqpoint{5.960465in}{1.580295in}}%
\pgfpathcurveto{\pgfqpoint{5.966289in}{1.574471in}}{\pgfqpoint{5.974189in}{1.571198in}}{\pgfqpoint{5.982425in}{1.571198in}}%
\pgfpathclose%
\pgfusepath{stroke,fill}%
\end{pgfscope}%
\begin{pgfscope}%
\pgfpathrectangle{\pgfqpoint{3.721492in}{0.557870in}}{\pgfqpoint{2.518508in}{1.684734in}}%
\pgfusepath{clip}%
\pgfsetbuttcap%
\pgfsetroundjoin%
\definecolor{currentfill}{rgb}{0.298039,0.447059,0.690196}%
\pgfsetfillcolor{currentfill}%
\pgfsetlinewidth{1.003750pt}%
\definecolor{currentstroke}{rgb}{0.298039,0.447059,0.690196}%
\pgfsetstrokecolor{currentstroke}%
\pgfsetdash{}{0pt}%
\pgfpathmoveto{\pgfqpoint{5.982425in}{1.674556in}}%
\pgfpathcurveto{\pgfqpoint{5.990662in}{1.674556in}}{\pgfqpoint{5.998562in}{1.677829in}}{\pgfqpoint{6.004386in}{1.683653in}}%
\pgfpathcurveto{\pgfqpoint{6.010209in}{1.689477in}}{\pgfqpoint{6.013482in}{1.697377in}}{\pgfqpoint{6.013482in}{1.705613in}}%
\pgfpathcurveto{\pgfqpoint{6.013482in}{1.713849in}}{\pgfqpoint{6.010209in}{1.721749in}}{\pgfqpoint{6.004386in}{1.727573in}}%
\pgfpathcurveto{\pgfqpoint{5.998562in}{1.733397in}}{\pgfqpoint{5.990662in}{1.736669in}}{\pgfqpoint{5.982425in}{1.736669in}}%
\pgfpathcurveto{\pgfqpoint{5.974189in}{1.736669in}}{\pgfqpoint{5.966289in}{1.733397in}}{\pgfqpoint{5.960465in}{1.727573in}}%
\pgfpathcurveto{\pgfqpoint{5.954641in}{1.721749in}}{\pgfqpoint{5.951369in}{1.713849in}}{\pgfqpoint{5.951369in}{1.705613in}}%
\pgfpathcurveto{\pgfqpoint{5.951369in}{1.697377in}}{\pgfqpoint{5.954641in}{1.689477in}}{\pgfqpoint{5.960465in}{1.683653in}}%
\pgfpathcurveto{\pgfqpoint{5.966289in}{1.677829in}}{\pgfqpoint{5.974189in}{1.674556in}}{\pgfqpoint{5.982425in}{1.674556in}}%
\pgfpathclose%
\pgfusepath{stroke,fill}%
\end{pgfscope}%
\begin{pgfscope}%
\pgfpathrectangle{\pgfqpoint{3.721492in}{0.557870in}}{\pgfqpoint{2.518508in}{1.684734in}}%
\pgfusepath{clip}%
\pgfsetbuttcap%
\pgfsetroundjoin%
\definecolor{currentfill}{rgb}{0.298039,0.447059,0.690196}%
\pgfsetfillcolor{currentfill}%
\pgfsetlinewidth{1.003750pt}%
\definecolor{currentstroke}{rgb}{0.298039,0.447059,0.690196}%
\pgfsetstrokecolor{currentstroke}%
\pgfsetdash{}{0pt}%
\pgfpathmoveto{\pgfqpoint{5.982425in}{1.712141in}}%
\pgfpathcurveto{\pgfqpoint{5.990662in}{1.712141in}}{\pgfqpoint{5.998562in}{1.715413in}}{\pgfqpoint{6.004386in}{1.721237in}}%
\pgfpathcurveto{\pgfqpoint{6.010209in}{1.727061in}}{\pgfqpoint{6.013482in}{1.734961in}}{\pgfqpoint{6.013482in}{1.743198in}}%
\pgfpathcurveto{\pgfqpoint{6.013482in}{1.751434in}}{\pgfqpoint{6.010209in}{1.759334in}}{\pgfqpoint{6.004386in}{1.765158in}}%
\pgfpathcurveto{\pgfqpoint{5.998562in}{1.770982in}}{\pgfqpoint{5.990662in}{1.774254in}}{\pgfqpoint{5.982425in}{1.774254in}}%
\pgfpathcurveto{\pgfqpoint{5.974189in}{1.774254in}}{\pgfqpoint{5.966289in}{1.770982in}}{\pgfqpoint{5.960465in}{1.765158in}}%
\pgfpathcurveto{\pgfqpoint{5.954641in}{1.759334in}}{\pgfqpoint{5.951369in}{1.751434in}}{\pgfqpoint{5.951369in}{1.743198in}}%
\pgfpathcurveto{\pgfqpoint{5.951369in}{1.734961in}}{\pgfqpoint{5.954641in}{1.727061in}}{\pgfqpoint{5.960465in}{1.721237in}}%
\pgfpathcurveto{\pgfqpoint{5.966289in}{1.715413in}}{\pgfqpoint{5.974189in}{1.712141in}}{\pgfqpoint{5.982425in}{1.712141in}}%
\pgfpathclose%
\pgfusepath{stroke,fill}%
\end{pgfscope}%
\begin{pgfscope}%
\pgfpathrectangle{\pgfqpoint{3.721492in}{0.557870in}}{\pgfqpoint{2.518508in}{1.684734in}}%
\pgfusepath{clip}%
\pgfsetbuttcap%
\pgfsetroundjoin%
\definecolor{currentfill}{rgb}{0.298039,0.447059,0.690196}%
\pgfsetfillcolor{currentfill}%
\pgfsetlinewidth{1.003750pt}%
\definecolor{currentstroke}{rgb}{0.298039,0.447059,0.690196}%
\pgfsetstrokecolor{currentstroke}%
\pgfsetdash{}{0pt}%
\pgfpathmoveto{\pgfqpoint{3.835969in}{2.125336in}}%
\pgfpathcurveto{\pgfqpoint{3.844206in}{2.125336in}}{\pgfqpoint{3.852106in}{2.128609in}}{\pgfqpoint{3.857930in}{2.134433in}}%
\pgfpathcurveto{\pgfqpoint{3.863754in}{2.140256in}}{\pgfqpoint{3.867026in}{2.148157in}}{\pgfqpoint{3.867026in}{2.156393in}}%
\pgfpathcurveto{\pgfqpoint{3.867026in}{2.164629in}}{\pgfqpoint{3.863754in}{2.172529in}}{\pgfqpoint{3.857930in}{2.178353in}}%
\pgfpathcurveto{\pgfqpoint{3.852106in}{2.184177in}}{\pgfqpoint{3.844206in}{2.187449in}}{\pgfqpoint{3.835969in}{2.187449in}}%
\pgfpathcurveto{\pgfqpoint{3.827733in}{2.187449in}}{\pgfqpoint{3.819833in}{2.184177in}}{\pgfqpoint{3.814009in}{2.178353in}}%
\pgfpathcurveto{\pgfqpoint{3.808185in}{2.172529in}}{\pgfqpoint{3.804913in}{2.164629in}}{\pgfqpoint{3.804913in}{2.156393in}}%
\pgfpathcurveto{\pgfqpoint{3.804913in}{2.148157in}}{\pgfqpoint{3.808185in}{2.140256in}}{\pgfqpoint{3.814009in}{2.134433in}}%
\pgfpathcurveto{\pgfqpoint{3.819833in}{2.128609in}}{\pgfqpoint{3.827733in}{2.125336in}}{\pgfqpoint{3.835969in}{2.125336in}}%
\pgfpathclose%
\pgfusepath{stroke,fill}%
\end{pgfscope}%
\begin{pgfscope}%
\pgfpathrectangle{\pgfqpoint{3.721492in}{0.557870in}}{\pgfqpoint{2.518508in}{1.684734in}}%
\pgfusepath{clip}%
\pgfsetbuttcap%
\pgfsetroundjoin%
\definecolor{currentfill}{rgb}{0.298039,0.447059,0.690196}%
\pgfsetfillcolor{currentfill}%
\pgfsetlinewidth{1.003750pt}%
\definecolor{currentstroke}{rgb}{0.298039,0.447059,0.690196}%
\pgfsetstrokecolor{currentstroke}%
\pgfsetdash{}{0pt}%
\pgfpathmoveto{\pgfqpoint{5.967622in}{1.730402in}}%
\pgfpathcurveto{\pgfqpoint{5.975858in}{1.730402in}}{\pgfqpoint{5.983758in}{1.733674in}}{\pgfqpoint{5.989582in}{1.739498in}}%
\pgfpathcurveto{\pgfqpoint{5.995406in}{1.745322in}}{\pgfqpoint{5.998679in}{1.753222in}}{\pgfqpoint{5.998679in}{1.761458in}}%
\pgfpathcurveto{\pgfqpoint{5.998679in}{1.769694in}}{\pgfqpoint{5.995406in}{1.777594in}}{\pgfqpoint{5.989582in}{1.783418in}}%
\pgfpathcurveto{\pgfqpoint{5.983758in}{1.789242in}}{\pgfqpoint{5.975858in}{1.792515in}}{\pgfqpoint{5.967622in}{1.792515in}}%
\pgfpathcurveto{\pgfqpoint{5.959386in}{1.792515in}}{\pgfqpoint{5.951486in}{1.789242in}}{\pgfqpoint{5.945662in}{1.783418in}}%
\pgfpathcurveto{\pgfqpoint{5.939838in}{1.777594in}}{\pgfqpoint{5.936566in}{1.769694in}}{\pgfqpoint{5.936566in}{1.761458in}}%
\pgfpathcurveto{\pgfqpoint{5.936566in}{1.753222in}}{\pgfqpoint{5.939838in}{1.745322in}}{\pgfqpoint{5.945662in}{1.739498in}}%
\pgfpathcurveto{\pgfqpoint{5.951486in}{1.733674in}}{\pgfqpoint{5.959386in}{1.730402in}}{\pgfqpoint{5.967622in}{1.730402in}}%
\pgfpathclose%
\pgfusepath{stroke,fill}%
\end{pgfscope}%
\begin{pgfscope}%
\pgfpathrectangle{\pgfqpoint{3.721492in}{0.557870in}}{\pgfqpoint{2.518508in}{1.684734in}}%
\pgfusepath{clip}%
\pgfsetbuttcap%
\pgfsetroundjoin%
\definecolor{currentfill}{rgb}{0.298039,0.447059,0.690196}%
\pgfsetfillcolor{currentfill}%
\pgfsetlinewidth{1.003750pt}%
\definecolor{currentstroke}{rgb}{0.298039,0.447059,0.690196}%
\pgfsetstrokecolor{currentstroke}%
\pgfsetdash{}{0pt}%
\pgfpathmoveto{\pgfqpoint{3.835969in}{2.125336in}}%
\pgfpathcurveto{\pgfqpoint{3.844206in}{2.125336in}}{\pgfqpoint{3.852106in}{2.128609in}}{\pgfqpoint{3.857930in}{2.134433in}}%
\pgfpathcurveto{\pgfqpoint{3.863754in}{2.140256in}}{\pgfqpoint{3.867026in}{2.148157in}}{\pgfqpoint{3.867026in}{2.156393in}}%
\pgfpathcurveto{\pgfqpoint{3.867026in}{2.164629in}}{\pgfqpoint{3.863754in}{2.172529in}}{\pgfqpoint{3.857930in}{2.178353in}}%
\pgfpathcurveto{\pgfqpoint{3.852106in}{2.184177in}}{\pgfqpoint{3.844206in}{2.187449in}}{\pgfqpoint{3.835969in}{2.187449in}}%
\pgfpathcurveto{\pgfqpoint{3.827733in}{2.187449in}}{\pgfqpoint{3.819833in}{2.184177in}}{\pgfqpoint{3.814009in}{2.178353in}}%
\pgfpathcurveto{\pgfqpoint{3.808185in}{2.172529in}}{\pgfqpoint{3.804913in}{2.164629in}}{\pgfqpoint{3.804913in}{2.156393in}}%
\pgfpathcurveto{\pgfqpoint{3.804913in}{2.148157in}}{\pgfqpoint{3.808185in}{2.140256in}}{\pgfqpoint{3.814009in}{2.134433in}}%
\pgfpathcurveto{\pgfqpoint{3.819833in}{2.128609in}}{\pgfqpoint{3.827733in}{2.125336in}}{\pgfqpoint{3.835969in}{2.125336in}}%
\pgfpathclose%
\pgfusepath{stroke,fill}%
\end{pgfscope}%
\begin{pgfscope}%
\pgfpathrectangle{\pgfqpoint{3.721492in}{0.557870in}}{\pgfqpoint{2.518508in}{1.684734in}}%
\pgfusepath{clip}%
\pgfsetbuttcap%
\pgfsetroundjoin%
\definecolor{currentfill}{rgb}{0.298039,0.447059,0.690196}%
\pgfsetfillcolor{currentfill}%
\pgfsetlinewidth{1.003750pt}%
\definecolor{currentstroke}{rgb}{0.298039,0.447059,0.690196}%
\pgfsetstrokecolor{currentstroke}%
\pgfsetdash{}{0pt}%
\pgfpathmoveto{\pgfqpoint{5.967622in}{1.306569in}}%
\pgfpathcurveto{\pgfqpoint{5.975858in}{1.306569in}}{\pgfqpoint{5.983758in}{1.309841in}}{\pgfqpoint{5.989582in}{1.315665in}}%
\pgfpathcurveto{\pgfqpoint{5.995406in}{1.321489in}}{\pgfqpoint{5.998679in}{1.329389in}}{\pgfqpoint{5.998679in}{1.337626in}}%
\pgfpathcurveto{\pgfqpoint{5.998679in}{1.345862in}}{\pgfqpoint{5.995406in}{1.353762in}}{\pgfqpoint{5.989582in}{1.359586in}}%
\pgfpathcurveto{\pgfqpoint{5.983758in}{1.365410in}}{\pgfqpoint{5.975858in}{1.368682in}}{\pgfqpoint{5.967622in}{1.368682in}}%
\pgfpathcurveto{\pgfqpoint{5.959386in}{1.368682in}}{\pgfqpoint{5.951486in}{1.365410in}}{\pgfqpoint{5.945662in}{1.359586in}}%
\pgfpathcurveto{\pgfqpoint{5.939838in}{1.353762in}}{\pgfqpoint{5.936566in}{1.345862in}}{\pgfqpoint{5.936566in}{1.337626in}}%
\pgfpathcurveto{\pgfqpoint{5.936566in}{1.329389in}}{\pgfqpoint{5.939838in}{1.321489in}}{\pgfqpoint{5.945662in}{1.315665in}}%
\pgfpathcurveto{\pgfqpoint{5.951486in}{1.309841in}}{\pgfqpoint{5.959386in}{1.306569in}}{\pgfqpoint{5.967622in}{1.306569in}}%
\pgfpathclose%
\pgfusepath{stroke,fill}%
\end{pgfscope}%
\begin{pgfscope}%
\pgfpathrectangle{\pgfqpoint{3.721492in}{0.557870in}}{\pgfqpoint{2.518508in}{1.684734in}}%
\pgfusepath{clip}%
\pgfsetbuttcap%
\pgfsetroundjoin%
\definecolor{currentfill}{rgb}{0.298039,0.447059,0.690196}%
\pgfsetfillcolor{currentfill}%
\pgfsetlinewidth{1.003750pt}%
\definecolor{currentstroke}{rgb}{0.298039,0.447059,0.690196}%
\pgfsetstrokecolor{currentstroke}%
\pgfsetdash{}{0pt}%
\pgfpathmoveto{\pgfqpoint{3.835969in}{2.125088in}}%
\pgfpathcurveto{\pgfqpoint{3.844206in}{2.125088in}}{\pgfqpoint{3.852106in}{2.128360in}}{\pgfqpoint{3.857930in}{2.134184in}}%
\pgfpathcurveto{\pgfqpoint{3.863754in}{2.140008in}}{\pgfqpoint{3.867026in}{2.147908in}}{\pgfqpoint{3.867026in}{2.156144in}}%
\pgfpathcurveto{\pgfqpoint{3.867026in}{2.164380in}}{\pgfqpoint{3.863754in}{2.172281in}}{\pgfqpoint{3.857930in}{2.178104in}}%
\pgfpathcurveto{\pgfqpoint{3.852106in}{2.183928in}}{\pgfqpoint{3.844206in}{2.187201in}}{\pgfqpoint{3.835969in}{2.187201in}}%
\pgfpathcurveto{\pgfqpoint{3.827733in}{2.187201in}}{\pgfqpoint{3.819833in}{2.183928in}}{\pgfqpoint{3.814009in}{2.178104in}}%
\pgfpathcurveto{\pgfqpoint{3.808185in}{2.172281in}}{\pgfqpoint{3.804913in}{2.164380in}}{\pgfqpoint{3.804913in}{2.156144in}}%
\pgfpathcurveto{\pgfqpoint{3.804913in}{2.147908in}}{\pgfqpoint{3.808185in}{2.140008in}}{\pgfqpoint{3.814009in}{2.134184in}}%
\pgfpathcurveto{\pgfqpoint{3.819833in}{2.128360in}}{\pgfqpoint{3.827733in}{2.125088in}}{\pgfqpoint{3.835969in}{2.125088in}}%
\pgfpathclose%
\pgfusepath{stroke,fill}%
\end{pgfscope}%
\begin{pgfscope}%
\pgfpathrectangle{\pgfqpoint{3.721492in}{0.557870in}}{\pgfqpoint{2.518508in}{1.684734in}}%
\pgfusepath{clip}%
\pgfsetbuttcap%
\pgfsetroundjoin%
\definecolor{currentfill}{rgb}{0.298039,0.447059,0.690196}%
\pgfsetfillcolor{currentfill}%
\pgfsetlinewidth{1.003750pt}%
\definecolor{currentstroke}{rgb}{0.298039,0.447059,0.690196}%
\pgfsetstrokecolor{currentstroke}%
\pgfsetdash{}{0pt}%
\pgfpathmoveto{\pgfqpoint{5.967622in}{1.561863in}}%
\pgfpathcurveto{\pgfqpoint{5.975858in}{1.561863in}}{\pgfqpoint{5.983758in}{1.565135in}}{\pgfqpoint{5.989582in}{1.570959in}}%
\pgfpathcurveto{\pgfqpoint{5.995406in}{1.576783in}}{\pgfqpoint{5.998679in}{1.584683in}}{\pgfqpoint{5.998679in}{1.592919in}}%
\pgfpathcurveto{\pgfqpoint{5.998679in}{1.601156in}}{\pgfqpoint{5.995406in}{1.609056in}}{\pgfqpoint{5.989582in}{1.614880in}}%
\pgfpathcurveto{\pgfqpoint{5.983758in}{1.620704in}}{\pgfqpoint{5.975858in}{1.623976in}}{\pgfqpoint{5.967622in}{1.623976in}}%
\pgfpathcurveto{\pgfqpoint{5.959386in}{1.623976in}}{\pgfqpoint{5.951486in}{1.620704in}}{\pgfqpoint{5.945662in}{1.614880in}}%
\pgfpathcurveto{\pgfqpoint{5.939838in}{1.609056in}}{\pgfqpoint{5.936566in}{1.601156in}}{\pgfqpoint{5.936566in}{1.592919in}}%
\pgfpathcurveto{\pgfqpoint{5.936566in}{1.584683in}}{\pgfqpoint{5.939838in}{1.576783in}}{\pgfqpoint{5.945662in}{1.570959in}}%
\pgfpathcurveto{\pgfqpoint{5.951486in}{1.565135in}}{\pgfqpoint{5.959386in}{1.561863in}}{\pgfqpoint{5.967622in}{1.561863in}}%
\pgfpathclose%
\pgfusepath{stroke,fill}%
\end{pgfscope}%
\begin{pgfscope}%
\pgfpathrectangle{\pgfqpoint{3.721492in}{0.557870in}}{\pgfqpoint{2.518508in}{1.684734in}}%
\pgfusepath{clip}%
\pgfsetbuttcap%
\pgfsetroundjoin%
\definecolor{currentfill}{rgb}{0.298039,0.447059,0.690196}%
\pgfsetfillcolor{currentfill}%
\pgfsetlinewidth{1.003750pt}%
\definecolor{currentstroke}{rgb}{0.298039,0.447059,0.690196}%
\pgfsetstrokecolor{currentstroke}%
\pgfsetdash{}{0pt}%
\pgfpathmoveto{\pgfqpoint{3.835969in}{2.125088in}}%
\pgfpathcurveto{\pgfqpoint{3.844206in}{2.125088in}}{\pgfqpoint{3.852106in}{2.128360in}}{\pgfqpoint{3.857930in}{2.134184in}}%
\pgfpathcurveto{\pgfqpoint{3.863754in}{2.140008in}}{\pgfqpoint{3.867026in}{2.147908in}}{\pgfqpoint{3.867026in}{2.156144in}}%
\pgfpathcurveto{\pgfqpoint{3.867026in}{2.164380in}}{\pgfqpoint{3.863754in}{2.172281in}}{\pgfqpoint{3.857930in}{2.178104in}}%
\pgfpathcurveto{\pgfqpoint{3.852106in}{2.183928in}}{\pgfqpoint{3.844206in}{2.187201in}}{\pgfqpoint{3.835969in}{2.187201in}}%
\pgfpathcurveto{\pgfqpoint{3.827733in}{2.187201in}}{\pgfqpoint{3.819833in}{2.183928in}}{\pgfqpoint{3.814009in}{2.178104in}}%
\pgfpathcurveto{\pgfqpoint{3.808185in}{2.172281in}}{\pgfqpoint{3.804913in}{2.164380in}}{\pgfqpoint{3.804913in}{2.156144in}}%
\pgfpathcurveto{\pgfqpoint{3.804913in}{2.147908in}}{\pgfqpoint{3.808185in}{2.140008in}}{\pgfqpoint{3.814009in}{2.134184in}}%
\pgfpathcurveto{\pgfqpoint{3.819833in}{2.128360in}}{\pgfqpoint{3.827733in}{2.125088in}}{\pgfqpoint{3.835969in}{2.125088in}}%
\pgfpathclose%
\pgfusepath{stroke,fill}%
\end{pgfscope}%
\begin{pgfscope}%
\pgfpathrectangle{\pgfqpoint{3.721492in}{0.557870in}}{\pgfqpoint{2.518508in}{1.684734in}}%
\pgfusepath{clip}%
\pgfsetbuttcap%
\pgfsetroundjoin%
\definecolor{currentfill}{rgb}{0.298039,0.447059,0.690196}%
\pgfsetfillcolor{currentfill}%
\pgfsetlinewidth{1.003750pt}%
\definecolor{currentstroke}{rgb}{0.298039,0.447059,0.690196}%
\pgfsetstrokecolor{currentstroke}%
\pgfsetdash{}{0pt}%
\pgfpathmoveto{\pgfqpoint{5.967622in}{1.710080in}}%
\pgfpathcurveto{\pgfqpoint{5.975858in}{1.710080in}}{\pgfqpoint{5.983758in}{1.713352in}}{\pgfqpoint{5.989582in}{1.719176in}}%
\pgfpathcurveto{\pgfqpoint{5.995406in}{1.725000in}}{\pgfqpoint{5.998679in}{1.732900in}}{\pgfqpoint{5.998679in}{1.741136in}}%
\pgfpathcurveto{\pgfqpoint{5.998679in}{1.749373in}}{\pgfqpoint{5.995406in}{1.757273in}}{\pgfqpoint{5.989582in}{1.763097in}}%
\pgfpathcurveto{\pgfqpoint{5.983758in}{1.768921in}}{\pgfqpoint{5.975858in}{1.772193in}}{\pgfqpoint{5.967622in}{1.772193in}}%
\pgfpathcurveto{\pgfqpoint{5.959386in}{1.772193in}}{\pgfqpoint{5.951486in}{1.768921in}}{\pgfqpoint{5.945662in}{1.763097in}}%
\pgfpathcurveto{\pgfqpoint{5.939838in}{1.757273in}}{\pgfqpoint{5.936566in}{1.749373in}}{\pgfqpoint{5.936566in}{1.741136in}}%
\pgfpathcurveto{\pgfqpoint{5.936566in}{1.732900in}}{\pgfqpoint{5.939838in}{1.725000in}}{\pgfqpoint{5.945662in}{1.719176in}}%
\pgfpathcurveto{\pgfqpoint{5.951486in}{1.713352in}}{\pgfqpoint{5.959386in}{1.710080in}}{\pgfqpoint{5.967622in}{1.710080in}}%
\pgfpathclose%
\pgfusepath{stroke,fill}%
\end{pgfscope}%
\begin{pgfscope}%
\pgfsetrectcap%
\pgfsetmiterjoin%
\pgfsetlinewidth{1.254687pt}%
\definecolor{currentstroke}{rgb}{1.000000,1.000000,1.000000}%
\pgfsetstrokecolor{currentstroke}%
\pgfsetdash{}{0pt}%
\pgfpathmoveto{\pgfqpoint{3.721492in}{0.557870in}}%
\pgfpathlineto{\pgfqpoint{3.721492in}{2.242604in}}%
\pgfusepath{stroke}%
\end{pgfscope}%
\begin{pgfscope}%
\pgfsetrectcap%
\pgfsetmiterjoin%
\pgfsetlinewidth{1.254687pt}%
\definecolor{currentstroke}{rgb}{1.000000,1.000000,1.000000}%
\pgfsetstrokecolor{currentstroke}%
\pgfsetdash{}{0pt}%
\pgfpathmoveto{\pgfqpoint{6.240000in}{0.557870in}}%
\pgfpathlineto{\pgfqpoint{6.240000in}{2.242604in}}%
\pgfusepath{stroke}%
\end{pgfscope}%
\begin{pgfscope}%
\pgfsetrectcap%
\pgfsetmiterjoin%
\pgfsetlinewidth{1.254687pt}%
\definecolor{currentstroke}{rgb}{1.000000,1.000000,1.000000}%
\pgfsetstrokecolor{currentstroke}%
\pgfsetdash{}{0pt}%
\pgfpathmoveto{\pgfqpoint{3.721492in}{0.557870in}}%
\pgfpathlineto{\pgfqpoint{6.240000in}{0.557870in}}%
\pgfusepath{stroke}%
\end{pgfscope}%
\begin{pgfscope}%
\pgfsetrectcap%
\pgfsetmiterjoin%
\pgfsetlinewidth{1.254687pt}%
\definecolor{currentstroke}{rgb}{1.000000,1.000000,1.000000}%
\pgfsetstrokecolor{currentstroke}%
\pgfsetdash{}{0pt}%
\pgfpathmoveto{\pgfqpoint{3.721492in}{2.242604in}}%
\pgfpathlineto{\pgfqpoint{6.240000in}{2.242604in}}%
\pgfusepath{stroke}%
\end{pgfscope}%
\begin{pgfscope}%
\definecolor{textcolor}{rgb}{0.150000,0.150000,0.150000}%
\pgfsetstrokecolor{textcolor}%
\pgfsetfillcolor{textcolor}%
\pgftext[x=4.980746in,y=2.325938in,,base]{\color{textcolor}\sffamily\fontsize{11.000000}{13.200000}\selectfont (b)}%
\end{pgfscope}%
\end{pgfpicture}%
\makeatother%
\endgroup%

    \caption{(a) Distribution plot of \acrshort{dor} of all PVC models evaluated at two cluster centers when applied to classify patient diagnosis.
             (b) Scatter plot of the same models sensitivity, and specificity.}
    \label{fig:pvc_ind_dor_sens_spec_dist}
\end{figure}

From the distribution plot in figure \ref{fig:pvc_ind_dor_sens_spec_dist}a one can see that the majority of the PVC models get \acrshort{dor} close to zero, but there are a few models that attain \acrshort{dor} above 30, and close to 40. From the scatter plot in \ref{fig:pvc_ind_dor_sens_spec_dist}b one can see that almost all the sensitivity scores are above $0.5$, while the specificity scores are concentrated in the areas $0$ to $0.25$ and $0.95$. As with the heart failure case study the PVC models that perform high in terms of \acrshort{dor} use a dataset that is a combination of peak systolic strain values and \acrshort{ef}. From table \ref{tab:pvc_ind_dor_sens_spec_dist} one can see that \textit{gls-EF/ward/2} and \textit{rls-EF/complete/2} are the two top performers in terms of \acrshort{dor}. \textit{gls-EF/ward/2} achieves a slightly higher specificity score, where as \textit{rls-EF/complete/2} attains a slightly higher specificity score.

\begin{table*}
    \centering
    \ra{1.3}
    \begin{tabular}{lrrrr}
        \toprule
        Dataset-model    &  Accuracy &  Sensitivity &  Specificity &   \acrshort{dor} \\
        \midrule
        gls-EF/ward/2     &      0.76 &         0.72 &         0.94 & 39.33 \\
        rls-EF/complete/2 &      0.77 &         0.74 &         0.93 & 37.61 \\
        gls-rls-EF/ward/2 &      0.76 &         0.72 &         0.93 & 35.16 \\
        gls-EF/average/2  &      0.74 &         0.70 &         0.94 & 34.90 \\
        gls-EF/complete/2 &      0.68 &         0.63 &         0.94 & 25.75 \\
        \bottomrule
    \end{tabular}
    \caption{The accuracy, \acrshort{dor}, sensitivity and specicity scores of the five best performing two-cluster-center PVC models in terms of \acrshort{dor}, at detecting patient diagnoses.
             The \textbf{Dataset-model} column indicates \textit{Dataset used}$/$\textit{Linkage criteria of model}$/$\textit{Number of cluster centers}.}
    \label{tab:pvc_ind_dor_sens_spec_dist}
\end{table*}

\begin{table*}
    \centering
    \ra{1.3}
    \begin{tabular}{lr}
        \toprule
        Dataset-model     &  \acrshort{ari} \\
        \midrule
        gls/average/6      & 0.29 \\
        gls/average/7      & 0.29 \\
        gls-rls/complete/3 & 0.28 \\
        rls-EF/complete/2  & 0.26 \\
        gls-EF/ward/2      & 0.25 \\
        \bottomrule
    \end{tabular}
    \caption{The five highest \acrshort{ari} scores attained when applying PVC for detecting patient diagnoses.
             The \textbf{Dataset-model} column indicates \textit{Dataset used}$/$\textit{Linkage criteria of model}$/$\textit{Number of cluster centers}.}
    \label{tab:pvc_ind_ari}
\end{table*}

The majority of the \acrshort{ari} scores of PVC models applied to identify patient diagnoses are centered around zero, but as one can see from table \ref{tab:pvc_ind_ari} there are a few models that acieve an \acrshort{ari} above $0.2$ close to $0.3$. For a change, the PVC models that perform best in terms of \acrshort{ari}, are neither models evaluated at two cluster centers, or models that are applied on a combination of peak systolic strain values and \acrshort{ef}. In contrast to the heart failure case study, the PVC models that achieve the highest \acrshort{ari}, when applied to identify patient diagnoses, are not the same models that achieve the highest \acrshort{dor}. The two PVC models that achieve the highest \acrshort{ari} are the \textit{gls/average} model evaluated at 6 and 7 cluster centers respectively. To get a better idea of why \textit{gls/average/6} and \textit{gls/average/7} attain the \acrshort{ari} they do, scatter plots of these two models, and \textit{gls-EF/ward/2} have been given in figure \ref{fig:scatter_gls_ef_hf_cluster_assignments}. A scatter plot of the target variable patient diagnosis is also given for comparison. The dimensions used are peak systolic \acrshort{gls} in all three views as these are the dimensions that are common to all three models. From the scatter plot in plot \ref{fig:scatter_gls_pd} one can see that the healthy patients are in the minority, and are concentrated in the corner with low peak systolic \acrshort{gls} values in the \acrshort{4ch}, \acrshort{2ch} and \acrshort{aplax} views. There are also some healthy patients with low-medium peak systolic \acrshort{gls} values, and very few healthy patients with high peak systolic \acrshort{gls} values. From plot \ref{fig:scatter_gls_ef_ward2_ind} one can see that \textit{gls-EF/ward/2} is able to isolate the concentration of healthy patients with low peak systolic \acrshort{gls}, but at the cost of many \acrshort{fn}. In plot \ref{fig:scatter_gls_average6} and \ref{fig:scatter_gls_average7} one can see that cluster 1 of model \textit{gls/average/6}, and cluster 2 of model \textit{gls/average/7} capture the healthy patients with low peak systolic \acrshort{gls}, but are unable of capturing the healthy patients with medium to high values. If one combines clusters 1 and 5 of \textit{gls/average/6}, and lets them represent healthy patients, and let the remaining clusters represent unhealthy patients the model attains an accuracy of $0.74$, a sensitivity of $0.70$, a specificity of $0.94$ and a \acrshort{dor} of $34.90$. If one combines clusters 2 and 5 of \textit{gls/average/7}, and lets them represent healthy patients, and let the remaining clusters represent unhealthy patients this model attains an accuracy of $0.74$, a sensitivity of $0.70$, a specificity of $0.94$ and a \acrshort{dor} of $35.94$. While the performance of the revised \textit{gls/average/6} and \textit{gls/average/6} models are good, they are still not as good as the performance of the top three performers in terms of \acrshort{dor}, which attain higher accuracy, sensitivity and \acrshort{dor}. Therefore, \textit{rls-EF/complete/2} is chosen as the best of the PVC models at identifying patient diagnosis, as it achieves the second highest \acrshort{dor}, and a more balanced sensitivity/specificity than \textit{gls-EF/ward/2} that attains the highest \acrshort{dor} score.

\begin{figure}[H]
    \centering
    \begin{subfigure}[b]{0.49\textwidth}
        \centering
        \includegraphics[width=0.99\textwidth]{results/pd/scatter_gls_indication_bin.png}
        \caption{Patient Diagnosis. \textbf{H} stands for \textbf{Healthy}, and \textbf{U} stands for \textbf{Unhealthy}}
        \label{fig:scatter_gls_pd}
    \end{subfigure}
    \begin{subfigure}[b]{0.49\textwidth}
        \centering
        \includegraphics[width=0.99\textwidth]{results/pd/scatter_gls_EF_ward2.png}
        \caption{\textit{GLS-EF Ward/2} cluster assignments.}
        \label{fig:scatter_gls_ef_ward2_ind}
    \end{subfigure}\\
    \begin{subfigure}[b]{0.49\textwidth}
        \centering
        \includegraphics[width=0.99\textwidth]{results/pd/scatter_gls_average6.png}
        \caption{\textit{GLS Average/6} cluster assignments.}
        \label{fig:scatter_gls_average6}
    \end{subfigure}
    \begin{subfigure}[b]{0.49\textwidth}
        \centering
        \includegraphics[width=0.99\textwidth]{results/pd/scatter_gls_average7.png}
        \caption{\textit{GLS Average/7} cluster assignments.}
        \label{fig:scatter_gls_average7}
    \end{subfigure}
    \caption{Scatterplot of peak \acrshort{gls} values in each view. Colors in the of the different dots are given by heart failure diagnosis, and cluster assignments of 
             \textit{gls-EF/ward/2}, \textit{average/6} and \textit{average/7} models. Numbers are not included on the axes because the point of the figure is to illustrate the separability 
             of clusters, and patient diagnosis.}
             \label{fig:scatter_gls_ind_cluster_assignments}
\end{figure}

\newpage

\subsection{Deep Neural Network}

\begin{figure}[H]
    \centering
    %% Creator: Matplotlib, PGF backend
%%
%% To include the figure in your LaTeX document, write
%%   \input{<filename>.pgf}
%%
%% Make sure the required packages are loaded in your preamble
%%   \usepackage{pgf}
%%
%% Figures using additional raster images can only be included by \input if
%% they are in the same directory as the main LaTeX file. For loading figures
%% from other directories you can use the `import` package
%%   \usepackage{import}
%% and then include the figures with
%%   \import{<path to file>}{<filename>.pgf}
%%
%% Matplotlib used the following preamble
%%
\begingroup%
\makeatletter%
\begin{pgfpicture}%
\pgfpathrectangle{\pgfpointorigin}{\pgfqpoint{6.439273in}{2.540000in}}%
\pgfusepath{use as bounding box, clip}%
\begin{pgfscope}%
\pgfsetbuttcap%
\pgfsetmiterjoin%
\definecolor{currentfill}{rgb}{1.000000,1.000000,1.000000}%
\pgfsetfillcolor{currentfill}%
\pgfsetlinewidth{0.000000pt}%
\definecolor{currentstroke}{rgb}{1.000000,1.000000,1.000000}%
\pgfsetstrokecolor{currentstroke}%
\pgfsetdash{}{0pt}%
\pgfpathmoveto{\pgfqpoint{0.000000in}{0.000000in}}%
\pgfpathlineto{\pgfqpoint{6.439273in}{0.000000in}}%
\pgfpathlineto{\pgfqpoint{6.439273in}{2.540000in}}%
\pgfpathlineto{\pgfqpoint{0.000000in}{2.540000in}}%
\pgfpathclose%
\pgfusepath{fill}%
\end{pgfscope}%
\begin{pgfscope}%
\pgfsetbuttcap%
\pgfsetmiterjoin%
\definecolor{currentfill}{rgb}{0.917647,0.917647,0.949020}%
\pgfsetfillcolor{currentfill}%
\pgfsetlinewidth{0.000000pt}%
\definecolor{currentstroke}{rgb}{0.000000,0.000000,0.000000}%
\pgfsetstrokecolor{currentstroke}%
\pgfsetstrokeopacity{0.000000}%
\pgfsetdash{}{0pt}%
\pgfpathmoveto{\pgfqpoint{0.693056in}{0.557870in}}%
\pgfpathlineto{\pgfqpoint{3.156042in}{0.557870in}}%
\pgfpathlineto{\pgfqpoint{3.156042in}{2.242604in}}%
\pgfpathlineto{\pgfqpoint{0.693056in}{2.242604in}}%
\pgfpathclose%
\pgfusepath{fill}%
\end{pgfscope}%
\begin{pgfscope}%
\pgfpathrectangle{\pgfqpoint{0.693056in}{0.557870in}}{\pgfqpoint{2.462986in}{1.684734in}}%
\pgfusepath{clip}%
\pgfsetroundcap%
\pgfsetroundjoin%
\pgfsetlinewidth{1.003750pt}%
\definecolor{currentstroke}{rgb}{1.000000,1.000000,1.000000}%
\pgfsetstrokecolor{currentstroke}%
\pgfsetdash{}{0pt}%
\pgfpathmoveto{\pgfqpoint{0.805010in}{0.557870in}}%
\pgfpathlineto{\pgfqpoint{0.805010in}{2.242604in}}%
\pgfusepath{stroke}%
\end{pgfscope}%
\begin{pgfscope}%
\definecolor{textcolor}{rgb}{0.150000,0.150000,0.150000}%
\pgfsetstrokecolor{textcolor}%
\pgfsetfillcolor{textcolor}%
\pgftext[x=0.805010in,y=0.425926in,,top]{\color{textcolor}\sffamily\fontsize{11.000000}{13.200000}\selectfont \(\displaystyle -0.50\)}%
\end{pgfscope}%
\begin{pgfscope}%
\pgfpathrectangle{\pgfqpoint{0.693056in}{0.557870in}}{\pgfqpoint{2.462986in}{1.684734in}}%
\pgfusepath{clip}%
\pgfsetroundcap%
\pgfsetroundjoin%
\pgfsetlinewidth{1.003750pt}%
\definecolor{currentstroke}{rgb}{1.000000,1.000000,1.000000}%
\pgfsetstrokecolor{currentstroke}%
\pgfsetdash{}{0pt}%
\pgfpathmoveto{\pgfqpoint{1.364779in}{0.557870in}}%
\pgfpathlineto{\pgfqpoint{1.364779in}{2.242604in}}%
\pgfusepath{stroke}%
\end{pgfscope}%
\begin{pgfscope}%
\definecolor{textcolor}{rgb}{0.150000,0.150000,0.150000}%
\pgfsetstrokecolor{textcolor}%
\pgfsetfillcolor{textcolor}%
\pgftext[x=1.364779in,y=0.425926in,,top]{\color{textcolor}\sffamily\fontsize{11.000000}{13.200000}\selectfont \(\displaystyle -0.25\)}%
\end{pgfscope}%
\begin{pgfscope}%
\pgfpathrectangle{\pgfqpoint{0.693056in}{0.557870in}}{\pgfqpoint{2.462986in}{1.684734in}}%
\pgfusepath{clip}%
\pgfsetroundcap%
\pgfsetroundjoin%
\pgfsetlinewidth{1.003750pt}%
\definecolor{currentstroke}{rgb}{1.000000,1.000000,1.000000}%
\pgfsetstrokecolor{currentstroke}%
\pgfsetdash{}{0pt}%
\pgfpathmoveto{\pgfqpoint{1.924549in}{0.557870in}}%
\pgfpathlineto{\pgfqpoint{1.924549in}{2.242604in}}%
\pgfusepath{stroke}%
\end{pgfscope}%
\begin{pgfscope}%
\definecolor{textcolor}{rgb}{0.150000,0.150000,0.150000}%
\pgfsetstrokecolor{textcolor}%
\pgfsetfillcolor{textcolor}%
\pgftext[x=1.924549in,y=0.425926in,,top]{\color{textcolor}\sffamily\fontsize{11.000000}{13.200000}\selectfont \(\displaystyle 0.00\)}%
\end{pgfscope}%
\begin{pgfscope}%
\pgfpathrectangle{\pgfqpoint{0.693056in}{0.557870in}}{\pgfqpoint{2.462986in}{1.684734in}}%
\pgfusepath{clip}%
\pgfsetroundcap%
\pgfsetroundjoin%
\pgfsetlinewidth{1.003750pt}%
\definecolor{currentstroke}{rgb}{1.000000,1.000000,1.000000}%
\pgfsetstrokecolor{currentstroke}%
\pgfsetdash{}{0pt}%
\pgfpathmoveto{\pgfqpoint{2.484318in}{0.557870in}}%
\pgfpathlineto{\pgfqpoint{2.484318in}{2.242604in}}%
\pgfusepath{stroke}%
\end{pgfscope}%
\begin{pgfscope}%
\definecolor{textcolor}{rgb}{0.150000,0.150000,0.150000}%
\pgfsetstrokecolor{textcolor}%
\pgfsetfillcolor{textcolor}%
\pgftext[x=2.484318in,y=0.425926in,,top]{\color{textcolor}\sffamily\fontsize{11.000000}{13.200000}\selectfont \(\displaystyle 0.25\)}%
\end{pgfscope}%
\begin{pgfscope}%
\pgfpathrectangle{\pgfqpoint{0.693056in}{0.557870in}}{\pgfqpoint{2.462986in}{1.684734in}}%
\pgfusepath{clip}%
\pgfsetroundcap%
\pgfsetroundjoin%
\pgfsetlinewidth{1.003750pt}%
\definecolor{currentstroke}{rgb}{1.000000,1.000000,1.000000}%
\pgfsetstrokecolor{currentstroke}%
\pgfsetdash{}{0pt}%
\pgfpathmoveto{\pgfqpoint{3.044088in}{0.557870in}}%
\pgfpathlineto{\pgfqpoint{3.044088in}{2.242604in}}%
\pgfusepath{stroke}%
\end{pgfscope}%
\begin{pgfscope}%
\definecolor{textcolor}{rgb}{0.150000,0.150000,0.150000}%
\pgfsetstrokecolor{textcolor}%
\pgfsetfillcolor{textcolor}%
\pgftext[x=3.044088in,y=0.425926in,,top]{\color{textcolor}\sffamily\fontsize{11.000000}{13.200000}\selectfont \(\displaystyle 0.50\)}%
\end{pgfscope}%
\begin{pgfscope}%
\definecolor{textcolor}{rgb}{0.150000,0.150000,0.150000}%
\pgfsetstrokecolor{textcolor}%
\pgfsetfillcolor{textcolor}%
\pgftext[x=1.924549in,y=0.235185in,,top]{\color{textcolor}\sffamily\fontsize{11.000000}{13.200000}\selectfont DOR}%
\end{pgfscope}%
\begin{pgfscope}%
\pgfpathrectangle{\pgfqpoint{0.693056in}{0.557870in}}{\pgfqpoint{2.462986in}{1.684734in}}%
\pgfusepath{clip}%
\pgfsetroundcap%
\pgfsetroundjoin%
\pgfsetlinewidth{1.003750pt}%
\definecolor{currentstroke}{rgb}{1.000000,1.000000,1.000000}%
\pgfsetstrokecolor{currentstroke}%
\pgfsetdash{}{0pt}%
\pgfpathmoveto{\pgfqpoint{0.693056in}{0.557870in}}%
\pgfpathlineto{\pgfqpoint{3.156042in}{0.557870in}}%
\pgfusepath{stroke}%
\end{pgfscope}%
\begin{pgfscope}%
\definecolor{textcolor}{rgb}{0.150000,0.150000,0.150000}%
\pgfsetstrokecolor{textcolor}%
\pgfsetfillcolor{textcolor}%
\pgftext[x=0.290741in,y=0.505064in,left,base]{\color{textcolor}\sffamily\fontsize{11.000000}{13.200000}\selectfont \(\displaystyle 0.00\)}%
\end{pgfscope}%
\begin{pgfscope}%
\pgfpathrectangle{\pgfqpoint{0.693056in}{0.557870in}}{\pgfqpoint{2.462986in}{1.684734in}}%
\pgfusepath{clip}%
\pgfsetroundcap%
\pgfsetroundjoin%
\pgfsetlinewidth{1.003750pt}%
\definecolor{currentstroke}{rgb}{1.000000,1.000000,1.000000}%
\pgfsetstrokecolor{currentstroke}%
\pgfsetdash{}{0pt}%
\pgfpathmoveto{\pgfqpoint{0.693056in}{0.958997in}}%
\pgfpathlineto{\pgfqpoint{3.156042in}{0.958997in}}%
\pgfusepath{stroke}%
\end{pgfscope}%
\begin{pgfscope}%
\definecolor{textcolor}{rgb}{0.150000,0.150000,0.150000}%
\pgfsetstrokecolor{textcolor}%
\pgfsetfillcolor{textcolor}%
\pgftext[x=0.290741in,y=0.906191in,left,base]{\color{textcolor}\sffamily\fontsize{11.000000}{13.200000}\selectfont \(\displaystyle 0.25\)}%
\end{pgfscope}%
\begin{pgfscope}%
\pgfpathrectangle{\pgfqpoint{0.693056in}{0.557870in}}{\pgfqpoint{2.462986in}{1.684734in}}%
\pgfusepath{clip}%
\pgfsetroundcap%
\pgfsetroundjoin%
\pgfsetlinewidth{1.003750pt}%
\definecolor{currentstroke}{rgb}{1.000000,1.000000,1.000000}%
\pgfsetstrokecolor{currentstroke}%
\pgfsetdash{}{0pt}%
\pgfpathmoveto{\pgfqpoint{0.693056in}{1.360125in}}%
\pgfpathlineto{\pgfqpoint{3.156042in}{1.360125in}}%
\pgfusepath{stroke}%
\end{pgfscope}%
\begin{pgfscope}%
\definecolor{textcolor}{rgb}{0.150000,0.150000,0.150000}%
\pgfsetstrokecolor{textcolor}%
\pgfsetfillcolor{textcolor}%
\pgftext[x=0.290741in,y=1.307318in,left,base]{\color{textcolor}\sffamily\fontsize{11.000000}{13.200000}\selectfont \(\displaystyle 0.50\)}%
\end{pgfscope}%
\begin{pgfscope}%
\pgfpathrectangle{\pgfqpoint{0.693056in}{0.557870in}}{\pgfqpoint{2.462986in}{1.684734in}}%
\pgfusepath{clip}%
\pgfsetroundcap%
\pgfsetroundjoin%
\pgfsetlinewidth{1.003750pt}%
\definecolor{currentstroke}{rgb}{1.000000,1.000000,1.000000}%
\pgfsetstrokecolor{currentstroke}%
\pgfsetdash{}{0pt}%
\pgfpathmoveto{\pgfqpoint{0.693056in}{1.761252in}}%
\pgfpathlineto{\pgfqpoint{3.156042in}{1.761252in}}%
\pgfusepath{stroke}%
\end{pgfscope}%
\begin{pgfscope}%
\definecolor{textcolor}{rgb}{0.150000,0.150000,0.150000}%
\pgfsetstrokecolor{textcolor}%
\pgfsetfillcolor{textcolor}%
\pgftext[x=0.290741in,y=1.708445in,left,base]{\color{textcolor}\sffamily\fontsize{11.000000}{13.200000}\selectfont \(\displaystyle 0.75\)}%
\end{pgfscope}%
\begin{pgfscope}%
\pgfpathrectangle{\pgfqpoint{0.693056in}{0.557870in}}{\pgfqpoint{2.462986in}{1.684734in}}%
\pgfusepath{clip}%
\pgfsetroundcap%
\pgfsetroundjoin%
\pgfsetlinewidth{1.003750pt}%
\definecolor{currentstroke}{rgb}{1.000000,1.000000,1.000000}%
\pgfsetstrokecolor{currentstroke}%
\pgfsetdash{}{0pt}%
\pgfpathmoveto{\pgfqpoint{0.693056in}{2.162379in}}%
\pgfpathlineto{\pgfqpoint{3.156042in}{2.162379in}}%
\pgfusepath{stroke}%
\end{pgfscope}%
\begin{pgfscope}%
\definecolor{textcolor}{rgb}{0.150000,0.150000,0.150000}%
\pgfsetstrokecolor{textcolor}%
\pgfsetfillcolor{textcolor}%
\pgftext[x=0.290741in,y=2.109572in,left,base]{\color{textcolor}\sffamily\fontsize{11.000000}{13.200000}\selectfont \(\displaystyle 1.00\)}%
\end{pgfscope}%
\begin{pgfscope}%
\definecolor{textcolor}{rgb}{0.150000,0.150000,0.150000}%
\pgfsetstrokecolor{textcolor}%
\pgfsetfillcolor{textcolor}%
\pgftext[x=0.235185in,y=1.400237in,,bottom,rotate=90.000000]{\color{textcolor}\sffamily\fontsize{11.000000}{13.200000}\selectfont Occurance}%
\end{pgfscope}%
\begin{pgfscope}%
\pgfpathrectangle{\pgfqpoint{0.693056in}{0.557870in}}{\pgfqpoint{2.462986in}{1.684734in}}%
\pgfusepath{clip}%
\pgfsetbuttcap%
\pgfsetmiterjoin%
\definecolor{currentfill}{rgb}{0.298039,0.447059,0.690196}%
\pgfsetfillcolor{currentfill}%
\pgfsetfillopacity{0.400000}%
\pgfsetlinewidth{1.003750pt}%
\definecolor{currentstroke}{rgb}{1.000000,1.000000,1.000000}%
\pgfsetstrokecolor{currentstroke}%
\pgfsetstrokeopacity{0.400000}%
\pgfsetdash{}{0pt}%
\pgfpathmoveto{\pgfqpoint{0.805010in}{0.557870in}}%
\pgfpathlineto{\pgfqpoint{1.028918in}{0.557870in}}%
\pgfpathlineto{\pgfqpoint{1.028918in}{0.557870in}}%
\pgfpathlineto{\pgfqpoint{0.805010in}{0.557870in}}%
\pgfpathclose%
\pgfusepath{stroke,fill}%
\end{pgfscope}%
\begin{pgfscope}%
\pgfpathrectangle{\pgfqpoint{0.693056in}{0.557870in}}{\pgfqpoint{2.462986in}{1.684734in}}%
\pgfusepath{clip}%
\pgfsetbuttcap%
\pgfsetmiterjoin%
\definecolor{currentfill}{rgb}{0.298039,0.447059,0.690196}%
\pgfsetfillcolor{currentfill}%
\pgfsetfillopacity{0.400000}%
\pgfsetlinewidth{1.003750pt}%
\definecolor{currentstroke}{rgb}{1.000000,1.000000,1.000000}%
\pgfsetstrokecolor{currentstroke}%
\pgfsetstrokeopacity{0.400000}%
\pgfsetdash{}{0pt}%
\pgfpathmoveto{\pgfqpoint{1.028918in}{0.557870in}}%
\pgfpathlineto{\pgfqpoint{1.252825in}{0.557870in}}%
\pgfpathlineto{\pgfqpoint{1.252825in}{0.557870in}}%
\pgfpathlineto{\pgfqpoint{1.028918in}{0.557870in}}%
\pgfpathclose%
\pgfusepath{stroke,fill}%
\end{pgfscope}%
\begin{pgfscope}%
\pgfpathrectangle{\pgfqpoint{0.693056in}{0.557870in}}{\pgfqpoint{2.462986in}{1.684734in}}%
\pgfusepath{clip}%
\pgfsetbuttcap%
\pgfsetmiterjoin%
\definecolor{currentfill}{rgb}{0.298039,0.447059,0.690196}%
\pgfsetfillcolor{currentfill}%
\pgfsetfillopacity{0.400000}%
\pgfsetlinewidth{1.003750pt}%
\definecolor{currentstroke}{rgb}{1.000000,1.000000,1.000000}%
\pgfsetstrokecolor{currentstroke}%
\pgfsetstrokeopacity{0.400000}%
\pgfsetdash{}{0pt}%
\pgfpathmoveto{\pgfqpoint{1.252825in}{0.557870in}}%
\pgfpathlineto{\pgfqpoint{1.476733in}{0.557870in}}%
\pgfpathlineto{\pgfqpoint{1.476733in}{0.557870in}}%
\pgfpathlineto{\pgfqpoint{1.252825in}{0.557870in}}%
\pgfpathclose%
\pgfusepath{stroke,fill}%
\end{pgfscope}%
\begin{pgfscope}%
\pgfpathrectangle{\pgfqpoint{0.693056in}{0.557870in}}{\pgfqpoint{2.462986in}{1.684734in}}%
\pgfusepath{clip}%
\pgfsetbuttcap%
\pgfsetmiterjoin%
\definecolor{currentfill}{rgb}{0.298039,0.447059,0.690196}%
\pgfsetfillcolor{currentfill}%
\pgfsetfillopacity{0.400000}%
\pgfsetlinewidth{1.003750pt}%
\definecolor{currentstroke}{rgb}{1.000000,1.000000,1.000000}%
\pgfsetstrokecolor{currentstroke}%
\pgfsetstrokeopacity{0.400000}%
\pgfsetdash{}{0pt}%
\pgfpathmoveto{\pgfqpoint{1.476733in}{0.557870in}}%
\pgfpathlineto{\pgfqpoint{1.700641in}{0.557870in}}%
\pgfpathlineto{\pgfqpoint{1.700641in}{0.557870in}}%
\pgfpathlineto{\pgfqpoint{1.476733in}{0.557870in}}%
\pgfpathclose%
\pgfusepath{stroke,fill}%
\end{pgfscope}%
\begin{pgfscope}%
\pgfpathrectangle{\pgfqpoint{0.693056in}{0.557870in}}{\pgfqpoint{2.462986in}{1.684734in}}%
\pgfusepath{clip}%
\pgfsetbuttcap%
\pgfsetmiterjoin%
\definecolor{currentfill}{rgb}{0.298039,0.447059,0.690196}%
\pgfsetfillcolor{currentfill}%
\pgfsetfillopacity{0.400000}%
\pgfsetlinewidth{1.003750pt}%
\definecolor{currentstroke}{rgb}{1.000000,1.000000,1.000000}%
\pgfsetstrokecolor{currentstroke}%
\pgfsetstrokeopacity{0.400000}%
\pgfsetdash{}{0pt}%
\pgfpathmoveto{\pgfqpoint{1.700641in}{0.557870in}}%
\pgfpathlineto{\pgfqpoint{1.924549in}{0.557870in}}%
\pgfpathlineto{\pgfqpoint{1.924549in}{0.557870in}}%
\pgfpathlineto{\pgfqpoint{1.700641in}{0.557870in}}%
\pgfpathclose%
\pgfusepath{stroke,fill}%
\end{pgfscope}%
\begin{pgfscope}%
\pgfpathrectangle{\pgfqpoint{0.693056in}{0.557870in}}{\pgfqpoint{2.462986in}{1.684734in}}%
\pgfusepath{clip}%
\pgfsetbuttcap%
\pgfsetmiterjoin%
\definecolor{currentfill}{rgb}{0.298039,0.447059,0.690196}%
\pgfsetfillcolor{currentfill}%
\pgfsetfillopacity{0.400000}%
\pgfsetlinewidth{1.003750pt}%
\definecolor{currentstroke}{rgb}{1.000000,1.000000,1.000000}%
\pgfsetstrokecolor{currentstroke}%
\pgfsetstrokeopacity{0.400000}%
\pgfsetdash{}{0pt}%
\pgfpathmoveto{\pgfqpoint{1.924549in}{0.557870in}}%
\pgfpathlineto{\pgfqpoint{2.148457in}{0.557870in}}%
\pgfpathlineto{\pgfqpoint{2.148457in}{2.162379in}}%
\pgfpathlineto{\pgfqpoint{1.924549in}{2.162379in}}%
\pgfpathclose%
\pgfusepath{stroke,fill}%
\end{pgfscope}%
\begin{pgfscope}%
\pgfpathrectangle{\pgfqpoint{0.693056in}{0.557870in}}{\pgfqpoint{2.462986in}{1.684734in}}%
\pgfusepath{clip}%
\pgfsetbuttcap%
\pgfsetmiterjoin%
\definecolor{currentfill}{rgb}{0.298039,0.447059,0.690196}%
\pgfsetfillcolor{currentfill}%
\pgfsetfillopacity{0.400000}%
\pgfsetlinewidth{1.003750pt}%
\definecolor{currentstroke}{rgb}{1.000000,1.000000,1.000000}%
\pgfsetstrokecolor{currentstroke}%
\pgfsetstrokeopacity{0.400000}%
\pgfsetdash{}{0pt}%
\pgfpathmoveto{\pgfqpoint{2.148457in}{0.557870in}}%
\pgfpathlineto{\pgfqpoint{2.372364in}{0.557870in}}%
\pgfpathlineto{\pgfqpoint{2.372364in}{0.557870in}}%
\pgfpathlineto{\pgfqpoint{2.148457in}{0.557870in}}%
\pgfpathclose%
\pgfusepath{stroke,fill}%
\end{pgfscope}%
\begin{pgfscope}%
\pgfpathrectangle{\pgfqpoint{0.693056in}{0.557870in}}{\pgfqpoint{2.462986in}{1.684734in}}%
\pgfusepath{clip}%
\pgfsetbuttcap%
\pgfsetmiterjoin%
\definecolor{currentfill}{rgb}{0.298039,0.447059,0.690196}%
\pgfsetfillcolor{currentfill}%
\pgfsetfillopacity{0.400000}%
\pgfsetlinewidth{1.003750pt}%
\definecolor{currentstroke}{rgb}{1.000000,1.000000,1.000000}%
\pgfsetstrokecolor{currentstroke}%
\pgfsetstrokeopacity{0.400000}%
\pgfsetdash{}{0pt}%
\pgfpathmoveto{\pgfqpoint{2.372364in}{0.557870in}}%
\pgfpathlineto{\pgfqpoint{2.596272in}{0.557870in}}%
\pgfpathlineto{\pgfqpoint{2.596272in}{0.557870in}}%
\pgfpathlineto{\pgfqpoint{2.372364in}{0.557870in}}%
\pgfpathclose%
\pgfusepath{stroke,fill}%
\end{pgfscope}%
\begin{pgfscope}%
\pgfpathrectangle{\pgfqpoint{0.693056in}{0.557870in}}{\pgfqpoint{2.462986in}{1.684734in}}%
\pgfusepath{clip}%
\pgfsetbuttcap%
\pgfsetmiterjoin%
\definecolor{currentfill}{rgb}{0.298039,0.447059,0.690196}%
\pgfsetfillcolor{currentfill}%
\pgfsetfillopacity{0.400000}%
\pgfsetlinewidth{1.003750pt}%
\definecolor{currentstroke}{rgb}{1.000000,1.000000,1.000000}%
\pgfsetstrokecolor{currentstroke}%
\pgfsetstrokeopacity{0.400000}%
\pgfsetdash{}{0pt}%
\pgfpathmoveto{\pgfqpoint{2.596272in}{0.557870in}}%
\pgfpathlineto{\pgfqpoint{2.820180in}{0.557870in}}%
\pgfpathlineto{\pgfqpoint{2.820180in}{0.557870in}}%
\pgfpathlineto{\pgfqpoint{2.596272in}{0.557870in}}%
\pgfpathclose%
\pgfusepath{stroke,fill}%
\end{pgfscope}%
\begin{pgfscope}%
\pgfpathrectangle{\pgfqpoint{0.693056in}{0.557870in}}{\pgfqpoint{2.462986in}{1.684734in}}%
\pgfusepath{clip}%
\pgfsetbuttcap%
\pgfsetmiterjoin%
\definecolor{currentfill}{rgb}{0.298039,0.447059,0.690196}%
\pgfsetfillcolor{currentfill}%
\pgfsetfillopacity{0.400000}%
\pgfsetlinewidth{1.003750pt}%
\definecolor{currentstroke}{rgb}{1.000000,1.000000,1.000000}%
\pgfsetstrokecolor{currentstroke}%
\pgfsetstrokeopacity{0.400000}%
\pgfsetdash{}{0pt}%
\pgfpathmoveto{\pgfqpoint{2.820180in}{0.557870in}}%
\pgfpathlineto{\pgfqpoint{3.044088in}{0.557870in}}%
\pgfpathlineto{\pgfqpoint{3.044088in}{0.557870in}}%
\pgfpathlineto{\pgfqpoint{2.820180in}{0.557870in}}%
\pgfpathclose%
\pgfusepath{stroke,fill}%
\end{pgfscope}%
\begin{pgfscope}%
\pgfsetrectcap%
\pgfsetmiterjoin%
\pgfsetlinewidth{1.254687pt}%
\definecolor{currentstroke}{rgb}{1.000000,1.000000,1.000000}%
\pgfsetstrokecolor{currentstroke}%
\pgfsetdash{}{0pt}%
\pgfpathmoveto{\pgfqpoint{0.693056in}{0.557870in}}%
\pgfpathlineto{\pgfqpoint{0.693056in}{2.242604in}}%
\pgfusepath{stroke}%
\end{pgfscope}%
\begin{pgfscope}%
\pgfsetrectcap%
\pgfsetmiterjoin%
\pgfsetlinewidth{1.254687pt}%
\definecolor{currentstroke}{rgb}{1.000000,1.000000,1.000000}%
\pgfsetstrokecolor{currentstroke}%
\pgfsetdash{}{0pt}%
\pgfpathmoveto{\pgfqpoint{3.156042in}{0.557870in}}%
\pgfpathlineto{\pgfqpoint{3.156042in}{2.242604in}}%
\pgfusepath{stroke}%
\end{pgfscope}%
\begin{pgfscope}%
\pgfsetrectcap%
\pgfsetmiterjoin%
\pgfsetlinewidth{1.254687pt}%
\definecolor{currentstroke}{rgb}{1.000000,1.000000,1.000000}%
\pgfsetstrokecolor{currentstroke}%
\pgfsetdash{}{0pt}%
\pgfpathmoveto{\pgfqpoint{0.693056in}{0.557870in}}%
\pgfpathlineto{\pgfqpoint{3.156042in}{0.557870in}}%
\pgfusepath{stroke}%
\end{pgfscope}%
\begin{pgfscope}%
\pgfsetrectcap%
\pgfsetmiterjoin%
\pgfsetlinewidth{1.254687pt}%
\definecolor{currentstroke}{rgb}{1.000000,1.000000,1.000000}%
\pgfsetstrokecolor{currentstroke}%
\pgfsetdash{}{0pt}%
\pgfpathmoveto{\pgfqpoint{0.693056in}{2.242604in}}%
\pgfpathlineto{\pgfqpoint{3.156042in}{2.242604in}}%
\pgfusepath{stroke}%
\end{pgfscope}%
\begin{pgfscope}%
\definecolor{textcolor}{rgb}{0.150000,0.150000,0.150000}%
\pgfsetstrokecolor{textcolor}%
\pgfsetfillcolor{textcolor}%
\pgftext[x=1.924549in,y=2.325938in,,base]{\color{textcolor}\sffamily\fontsize{11.000000}{13.200000}\selectfont (a)}%
\end{pgfscope}%
\begin{pgfscope}%
\pgfsetbuttcap%
\pgfsetmiterjoin%
\definecolor{currentfill}{rgb}{0.917647,0.917647,0.949020}%
\pgfsetfillcolor{currentfill}%
\pgfsetlinewidth{0.000000pt}%
\definecolor{currentstroke}{rgb}{0.000000,0.000000,0.000000}%
\pgfsetstrokecolor{currentstroke}%
\pgfsetstrokeopacity{0.000000}%
\pgfsetdash{}{0pt}%
\pgfpathmoveto{\pgfqpoint{3.853056in}{0.557870in}}%
\pgfpathlineto{\pgfqpoint{6.316042in}{0.557870in}}%
\pgfpathlineto{\pgfqpoint{6.316042in}{2.242604in}}%
\pgfpathlineto{\pgfqpoint{3.853056in}{2.242604in}}%
\pgfpathclose%
\pgfusepath{fill}%
\end{pgfscope}%
\begin{pgfscope}%
\pgfpathrectangle{\pgfqpoint{3.853056in}{0.557870in}}{\pgfqpoint{2.462986in}{1.684734in}}%
\pgfusepath{clip}%
\pgfsetroundcap%
\pgfsetroundjoin%
\pgfsetlinewidth{1.003750pt}%
\definecolor{currentstroke}{rgb}{1.000000,1.000000,1.000000}%
\pgfsetstrokecolor{currentstroke}%
\pgfsetdash{}{0pt}%
\pgfpathmoveto{\pgfqpoint{3.965010in}{0.557870in}}%
\pgfpathlineto{\pgfqpoint{3.965010in}{2.242604in}}%
\pgfusepath{stroke}%
\end{pgfscope}%
\begin{pgfscope}%
\definecolor{textcolor}{rgb}{0.150000,0.150000,0.150000}%
\pgfsetstrokecolor{textcolor}%
\pgfsetfillcolor{textcolor}%
\pgftext[x=3.965010in,y=0.425926in,,top]{\color{textcolor}\sffamily\fontsize{11.000000}{13.200000}\selectfont \(\displaystyle 0.00\)}%
\end{pgfscope}%
\begin{pgfscope}%
\pgfpathrectangle{\pgfqpoint{3.853056in}{0.557870in}}{\pgfqpoint{2.462986in}{1.684734in}}%
\pgfusepath{clip}%
\pgfsetroundcap%
\pgfsetroundjoin%
\pgfsetlinewidth{1.003750pt}%
\definecolor{currentstroke}{rgb}{1.000000,1.000000,1.000000}%
\pgfsetstrokecolor{currentstroke}%
\pgfsetdash{}{0pt}%
\pgfpathmoveto{\pgfqpoint{4.524779in}{0.557870in}}%
\pgfpathlineto{\pgfqpoint{4.524779in}{2.242604in}}%
\pgfusepath{stroke}%
\end{pgfscope}%
\begin{pgfscope}%
\definecolor{textcolor}{rgb}{0.150000,0.150000,0.150000}%
\pgfsetstrokecolor{textcolor}%
\pgfsetfillcolor{textcolor}%
\pgftext[x=4.524779in,y=0.425926in,,top]{\color{textcolor}\sffamily\fontsize{11.000000}{13.200000}\selectfont \(\displaystyle 0.25\)}%
\end{pgfscope}%
\begin{pgfscope}%
\pgfpathrectangle{\pgfqpoint{3.853056in}{0.557870in}}{\pgfqpoint{2.462986in}{1.684734in}}%
\pgfusepath{clip}%
\pgfsetroundcap%
\pgfsetroundjoin%
\pgfsetlinewidth{1.003750pt}%
\definecolor{currentstroke}{rgb}{1.000000,1.000000,1.000000}%
\pgfsetstrokecolor{currentstroke}%
\pgfsetdash{}{0pt}%
\pgfpathmoveto{\pgfqpoint{5.084549in}{0.557870in}}%
\pgfpathlineto{\pgfqpoint{5.084549in}{2.242604in}}%
\pgfusepath{stroke}%
\end{pgfscope}%
\begin{pgfscope}%
\definecolor{textcolor}{rgb}{0.150000,0.150000,0.150000}%
\pgfsetstrokecolor{textcolor}%
\pgfsetfillcolor{textcolor}%
\pgftext[x=5.084549in,y=0.425926in,,top]{\color{textcolor}\sffamily\fontsize{11.000000}{13.200000}\selectfont \(\displaystyle 0.50\)}%
\end{pgfscope}%
\begin{pgfscope}%
\pgfpathrectangle{\pgfqpoint{3.853056in}{0.557870in}}{\pgfqpoint{2.462986in}{1.684734in}}%
\pgfusepath{clip}%
\pgfsetroundcap%
\pgfsetroundjoin%
\pgfsetlinewidth{1.003750pt}%
\definecolor{currentstroke}{rgb}{1.000000,1.000000,1.000000}%
\pgfsetstrokecolor{currentstroke}%
\pgfsetdash{}{0pt}%
\pgfpathmoveto{\pgfqpoint{5.644318in}{0.557870in}}%
\pgfpathlineto{\pgfqpoint{5.644318in}{2.242604in}}%
\pgfusepath{stroke}%
\end{pgfscope}%
\begin{pgfscope}%
\definecolor{textcolor}{rgb}{0.150000,0.150000,0.150000}%
\pgfsetstrokecolor{textcolor}%
\pgfsetfillcolor{textcolor}%
\pgftext[x=5.644318in,y=0.425926in,,top]{\color{textcolor}\sffamily\fontsize{11.000000}{13.200000}\selectfont \(\displaystyle 0.75\)}%
\end{pgfscope}%
\begin{pgfscope}%
\pgfpathrectangle{\pgfqpoint{3.853056in}{0.557870in}}{\pgfqpoint{2.462986in}{1.684734in}}%
\pgfusepath{clip}%
\pgfsetroundcap%
\pgfsetroundjoin%
\pgfsetlinewidth{1.003750pt}%
\definecolor{currentstroke}{rgb}{1.000000,1.000000,1.000000}%
\pgfsetstrokecolor{currentstroke}%
\pgfsetdash{}{0pt}%
\pgfpathmoveto{\pgfqpoint{6.204088in}{0.557870in}}%
\pgfpathlineto{\pgfqpoint{6.204088in}{2.242604in}}%
\pgfusepath{stroke}%
\end{pgfscope}%
\begin{pgfscope}%
\definecolor{textcolor}{rgb}{0.150000,0.150000,0.150000}%
\pgfsetstrokecolor{textcolor}%
\pgfsetfillcolor{textcolor}%
\pgftext[x=6.204088in,y=0.425926in,,top]{\color{textcolor}\sffamily\fontsize{11.000000}{13.200000}\selectfont \(\displaystyle 1.00\)}%
\end{pgfscope}%
\begin{pgfscope}%
\definecolor{textcolor}{rgb}{0.150000,0.150000,0.150000}%
\pgfsetstrokecolor{textcolor}%
\pgfsetfillcolor{textcolor}%
\pgftext[x=5.084549in,y=0.235185in,,top]{\color{textcolor}\sffamily\fontsize{11.000000}{13.200000}\selectfont Specificity}%
\end{pgfscope}%
\begin{pgfscope}%
\pgfpathrectangle{\pgfqpoint{3.853056in}{0.557870in}}{\pgfqpoint{2.462986in}{1.684734in}}%
\pgfusepath{clip}%
\pgfsetroundcap%
\pgfsetroundjoin%
\pgfsetlinewidth{1.003750pt}%
\definecolor{currentstroke}{rgb}{1.000000,1.000000,1.000000}%
\pgfsetstrokecolor{currentstroke}%
\pgfsetdash{}{0pt}%
\pgfpathmoveto{\pgfqpoint{3.853056in}{0.634449in}}%
\pgfpathlineto{\pgfqpoint{6.316042in}{0.634449in}}%
\pgfusepath{stroke}%
\end{pgfscope}%
\begin{pgfscope}%
\definecolor{textcolor}{rgb}{0.150000,0.150000,0.150000}%
\pgfsetstrokecolor{textcolor}%
\pgfsetfillcolor{textcolor}%
\pgftext[x=3.450741in,y=0.581642in,left,base]{\color{textcolor}\sffamily\fontsize{11.000000}{13.200000}\selectfont \(\displaystyle 0.00\)}%
\end{pgfscope}%
\begin{pgfscope}%
\pgfpathrectangle{\pgfqpoint{3.853056in}{0.557870in}}{\pgfqpoint{2.462986in}{1.684734in}}%
\pgfusepath{clip}%
\pgfsetroundcap%
\pgfsetroundjoin%
\pgfsetlinewidth{1.003750pt}%
\definecolor{currentstroke}{rgb}{1.000000,1.000000,1.000000}%
\pgfsetstrokecolor{currentstroke}%
\pgfsetdash{}{0pt}%
\pgfpathmoveto{\pgfqpoint{3.853056in}{1.017343in}}%
\pgfpathlineto{\pgfqpoint{6.316042in}{1.017343in}}%
\pgfusepath{stroke}%
\end{pgfscope}%
\begin{pgfscope}%
\definecolor{textcolor}{rgb}{0.150000,0.150000,0.150000}%
\pgfsetstrokecolor{textcolor}%
\pgfsetfillcolor{textcolor}%
\pgftext[x=3.450741in,y=0.964536in,left,base]{\color{textcolor}\sffamily\fontsize{11.000000}{13.200000}\selectfont \(\displaystyle 0.25\)}%
\end{pgfscope}%
\begin{pgfscope}%
\pgfpathrectangle{\pgfqpoint{3.853056in}{0.557870in}}{\pgfqpoint{2.462986in}{1.684734in}}%
\pgfusepath{clip}%
\pgfsetroundcap%
\pgfsetroundjoin%
\pgfsetlinewidth{1.003750pt}%
\definecolor{currentstroke}{rgb}{1.000000,1.000000,1.000000}%
\pgfsetstrokecolor{currentstroke}%
\pgfsetdash{}{0pt}%
\pgfpathmoveto{\pgfqpoint{3.853056in}{1.400237in}}%
\pgfpathlineto{\pgfqpoint{6.316042in}{1.400237in}}%
\pgfusepath{stroke}%
\end{pgfscope}%
\begin{pgfscope}%
\definecolor{textcolor}{rgb}{0.150000,0.150000,0.150000}%
\pgfsetstrokecolor{textcolor}%
\pgfsetfillcolor{textcolor}%
\pgftext[x=3.450741in,y=1.347431in,left,base]{\color{textcolor}\sffamily\fontsize{11.000000}{13.200000}\selectfont \(\displaystyle 0.50\)}%
\end{pgfscope}%
\begin{pgfscope}%
\pgfpathrectangle{\pgfqpoint{3.853056in}{0.557870in}}{\pgfqpoint{2.462986in}{1.684734in}}%
\pgfusepath{clip}%
\pgfsetroundcap%
\pgfsetroundjoin%
\pgfsetlinewidth{1.003750pt}%
\definecolor{currentstroke}{rgb}{1.000000,1.000000,1.000000}%
\pgfsetstrokecolor{currentstroke}%
\pgfsetdash{}{0pt}%
\pgfpathmoveto{\pgfqpoint{3.853056in}{1.783131in}}%
\pgfpathlineto{\pgfqpoint{6.316042in}{1.783131in}}%
\pgfusepath{stroke}%
\end{pgfscope}%
\begin{pgfscope}%
\definecolor{textcolor}{rgb}{0.150000,0.150000,0.150000}%
\pgfsetstrokecolor{textcolor}%
\pgfsetfillcolor{textcolor}%
\pgftext[x=3.450741in,y=1.730325in,left,base]{\color{textcolor}\sffamily\fontsize{11.000000}{13.200000}\selectfont \(\displaystyle 0.75\)}%
\end{pgfscope}%
\begin{pgfscope}%
\pgfpathrectangle{\pgfqpoint{3.853056in}{0.557870in}}{\pgfqpoint{2.462986in}{1.684734in}}%
\pgfusepath{clip}%
\pgfsetroundcap%
\pgfsetroundjoin%
\pgfsetlinewidth{1.003750pt}%
\definecolor{currentstroke}{rgb}{1.000000,1.000000,1.000000}%
\pgfsetstrokecolor{currentstroke}%
\pgfsetdash{}{0pt}%
\pgfpathmoveto{\pgfqpoint{3.853056in}{2.166025in}}%
\pgfpathlineto{\pgfqpoint{6.316042in}{2.166025in}}%
\pgfusepath{stroke}%
\end{pgfscope}%
\begin{pgfscope}%
\definecolor{textcolor}{rgb}{0.150000,0.150000,0.150000}%
\pgfsetstrokecolor{textcolor}%
\pgfsetfillcolor{textcolor}%
\pgftext[x=3.450741in,y=2.113219in,left,base]{\color{textcolor}\sffamily\fontsize{11.000000}{13.200000}\selectfont \(\displaystyle 1.00\)}%
\end{pgfscope}%
\begin{pgfscope}%
\definecolor{textcolor}{rgb}{0.150000,0.150000,0.150000}%
\pgfsetstrokecolor{textcolor}%
\pgfsetfillcolor{textcolor}%
\pgftext[x=3.395185in,y=1.400237in,,bottom,rotate=90.000000]{\color{textcolor}\sffamily\fontsize{11.000000}{13.200000}\selectfont Sensitivity}%
\end{pgfscope}%
\begin{pgfscope}%
\pgfpathrectangle{\pgfqpoint{3.853056in}{0.557870in}}{\pgfqpoint{2.462986in}{1.684734in}}%
\pgfusepath{clip}%
\pgfsetbuttcap%
\pgfsetroundjoin%
\definecolor{currentfill}{rgb}{0.298039,0.447059,0.690196}%
\pgfsetfillcolor{currentfill}%
\pgfsetlinewidth{1.003750pt}%
\definecolor{currentstroke}{rgb}{0.298039,0.447059,0.690196}%
\pgfsetstrokecolor{currentstroke}%
\pgfsetdash{}{0pt}%
\pgfpathmoveto{\pgfqpoint{3.965010in}{2.125798in}}%
\pgfpathcurveto{\pgfqpoint{3.973246in}{2.125798in}}{\pgfqpoint{3.981146in}{2.129070in}}{\pgfqpoint{3.986970in}{2.134894in}}%
\pgfpathcurveto{\pgfqpoint{3.992794in}{2.140718in}}{\pgfqpoint{3.996066in}{2.148618in}}{\pgfqpoint{3.996066in}{2.156854in}}%
\pgfpathcurveto{\pgfqpoint{3.996066in}{2.165091in}}{\pgfqpoint{3.992794in}{2.172991in}}{\pgfqpoint{3.986970in}{2.178814in}}%
\pgfpathcurveto{\pgfqpoint{3.981146in}{2.184638in}}{\pgfqpoint{3.973246in}{2.187911in}}{\pgfqpoint{3.965010in}{2.187911in}}%
\pgfpathcurveto{\pgfqpoint{3.956773in}{2.187911in}}{\pgfqpoint{3.948873in}{2.184638in}}{\pgfqpoint{3.943049in}{2.178814in}}%
\pgfpathcurveto{\pgfqpoint{3.937226in}{2.172991in}}{\pgfqpoint{3.933953in}{2.165091in}}{\pgfqpoint{3.933953in}{2.156854in}}%
\pgfpathcurveto{\pgfqpoint{3.933953in}{2.148618in}}{\pgfqpoint{3.937226in}{2.140718in}}{\pgfqpoint{3.943049in}{2.134894in}}%
\pgfpathcurveto{\pgfqpoint{3.948873in}{2.129070in}}{\pgfqpoint{3.956773in}{2.125798in}}{\pgfqpoint{3.965010in}{2.125798in}}%
\pgfpathclose%
\pgfusepath{stroke,fill}%
\end{pgfscope}%
\begin{pgfscope}%
\pgfsetrectcap%
\pgfsetmiterjoin%
\pgfsetlinewidth{1.254687pt}%
\definecolor{currentstroke}{rgb}{1.000000,1.000000,1.000000}%
\pgfsetstrokecolor{currentstroke}%
\pgfsetdash{}{0pt}%
\pgfpathmoveto{\pgfqpoint{3.853056in}{0.557870in}}%
\pgfpathlineto{\pgfqpoint{3.853056in}{2.242604in}}%
\pgfusepath{stroke}%
\end{pgfscope}%
\begin{pgfscope}%
\pgfsetrectcap%
\pgfsetmiterjoin%
\pgfsetlinewidth{1.254687pt}%
\definecolor{currentstroke}{rgb}{1.000000,1.000000,1.000000}%
\pgfsetstrokecolor{currentstroke}%
\pgfsetdash{}{0pt}%
\pgfpathmoveto{\pgfqpoint{6.316042in}{0.557870in}}%
\pgfpathlineto{\pgfqpoint{6.316042in}{2.242604in}}%
\pgfusepath{stroke}%
\end{pgfscope}%
\begin{pgfscope}%
\pgfsetrectcap%
\pgfsetmiterjoin%
\pgfsetlinewidth{1.254687pt}%
\definecolor{currentstroke}{rgb}{1.000000,1.000000,1.000000}%
\pgfsetstrokecolor{currentstroke}%
\pgfsetdash{}{0pt}%
\pgfpathmoveto{\pgfqpoint{3.853056in}{0.557870in}}%
\pgfpathlineto{\pgfqpoint{6.316042in}{0.557870in}}%
\pgfusepath{stroke}%
\end{pgfscope}%
\begin{pgfscope}%
\pgfsetrectcap%
\pgfsetmiterjoin%
\pgfsetlinewidth{1.254687pt}%
\definecolor{currentstroke}{rgb}{1.000000,1.000000,1.000000}%
\pgfsetstrokecolor{currentstroke}%
\pgfsetdash{}{0pt}%
\pgfpathmoveto{\pgfqpoint{3.853056in}{2.242604in}}%
\pgfpathlineto{\pgfqpoint{6.316042in}{2.242604in}}%
\pgfusepath{stroke}%
\end{pgfscope}%
\begin{pgfscope}%
\definecolor{textcolor}{rgb}{0.150000,0.150000,0.150000}%
\pgfsetstrokecolor{textcolor}%
\pgfsetfillcolor{textcolor}%
\pgftext[x=5.084549in,y=2.325938in,,base]{\color{textcolor}\sffamily\fontsize{11.000000}{13.200000}\selectfont (b)}%
\end{pgfscope}%
\end{pgfpicture}%
\makeatother%
\endgroup%

    \caption{(a) Distribution plot of \acrshort{dor} of all \acrshort{ann} models when trained to classify patient diagnosis.
             (b) Scatter plot of the same models sensitivity, and specificity.}
    \label{fig:dl_ind_dor_sens_spec_dist}
\end{figure}

\begin{table*}
    \centering
    \ra{1.3}
    \begin{tabular}{lrrrr}
        \toprule
        Dataset-model              &  Accuracy &  Sensitivity &  Specificity &  \acrshort{dor} \\
        \midrule
        all-strain/4CH/upsampled   &      0.83 &         0.99 &         0.00 & 0.00 \\
        all-strain/2CH/regular     &      0.85 &         1.00 &         0.00 &  NaN \\
        gls/2CH/regular            &      0.85 &         1.00 &         0.00 &  NaN \\
        rls/2CH/regular            &      0.85 &         1.00 &         0.00 &  NaN \\
        all-strain/2CH/downsampled &      0.85 &         1.00 &         0.00 &  NaN \\
        \bottomrule
    \end{tabular}
    \caption{The accuracy, \acrshort{dor}, sensitivity and specicity scores of the five best performing variations of the \acrshort{ann} in terms of \acrshort{dor}, when trained to predict patient diagnoses.
             The \textbf{Dataset-model} column indicates \textit{Dataset used}$/$\textit{View used}$/$\textit{Whether curve has been upsampled, downsampled or is regular}.}
    \label{tab:dl_hf_dor_sens_spec_dist}
\end{table*}

From the distribution plot in figure \ref{fig:dl_ind_dor_sens_spec_dist} one can see that the collective performance of the different variations of the \acrshort{ann} trained to predict patient diagnosis is terrible. The \acrshort{dor} of all the models are either zero because the number of \acrshort{tn} attained are zero, or not defined because the number of \acrshort{fn} are zero. The sensitivities are all 1, or close to 1, and the specificities are all 0. It is evident that the \acrshort{ann} are not able to generalize the traits of the healthy patients from such a small dataset. The \acrshort{ann} models are will therefore not be discussed further with relation to prediction of patient diagnosis, and are not included in the comparison of the four model groups.

\newpage

\subsection{Peak-value Classifiers}

\begin{figure}[H]
    \centering
    % \includegraphics[width=\textwidth]{results/pvmlc_ind_dor_sens_spec_dist.png}
    %% Creator: Matplotlib, PGF backend
%%
%% To include the figure in your LaTeX document, write
%%   \input{<filename>.pgf}
%%
%% Make sure the required packages are loaded in your preamble
%%   \usepackage{pgf}
%%
%% Figures using additional raster images can only be included by \input if
%% they are in the same directory as the main LaTeX file. For loading figures
%% from other directories you can use the `import` package
%%   \usepackage{import}
%% and then include the figures with
%%   \import{<path to file>}{<filename>.pgf}
%%
%% Matplotlib used the following preamble
%%
\begingroup%
\makeatletter%
\begin{pgfpicture}%
\pgfpathrectangle{\pgfpointorigin}{\pgfqpoint{6.362271in}{2.540000in}}%
\pgfusepath{use as bounding box, clip}%
\begin{pgfscope}%
\pgfsetbuttcap%
\pgfsetmiterjoin%
\definecolor{currentfill}{rgb}{1.000000,1.000000,1.000000}%
\pgfsetfillcolor{currentfill}%
\pgfsetlinewidth{0.000000pt}%
\definecolor{currentstroke}{rgb}{1.000000,1.000000,1.000000}%
\pgfsetstrokecolor{currentstroke}%
\pgfsetdash{}{0pt}%
\pgfpathmoveto{\pgfqpoint{0.000000in}{0.000000in}}%
\pgfpathlineto{\pgfqpoint{6.362271in}{0.000000in}}%
\pgfpathlineto{\pgfqpoint{6.362271in}{2.540000in}}%
\pgfpathlineto{\pgfqpoint{0.000000in}{2.540000in}}%
\pgfpathclose%
\pgfusepath{fill}%
\end{pgfscope}%
\begin{pgfscope}%
\pgfsetbuttcap%
\pgfsetmiterjoin%
\definecolor{currentfill}{rgb}{0.917647,0.917647,0.949020}%
\pgfsetfillcolor{currentfill}%
\pgfsetlinewidth{0.000000pt}%
\definecolor{currentstroke}{rgb}{0.000000,0.000000,0.000000}%
\pgfsetstrokecolor{currentstroke}%
\pgfsetstrokeopacity{0.000000}%
\pgfsetdash{}{0pt}%
\pgfpathmoveto{\pgfqpoint{0.574769in}{0.557870in}}%
\pgfpathlineto{\pgfqpoint{3.058877in}{0.557870in}}%
\pgfpathlineto{\pgfqpoint{3.058877in}{2.242604in}}%
\pgfpathlineto{\pgfqpoint{0.574769in}{2.242604in}}%
\pgfpathclose%
\pgfusepath{fill}%
\end{pgfscope}%
\begin{pgfscope}%
\pgfpathrectangle{\pgfqpoint{0.574769in}{0.557870in}}{\pgfqpoint{2.484109in}{1.684734in}}%
\pgfusepath{clip}%
\pgfsetroundcap%
\pgfsetroundjoin%
\pgfsetlinewidth{1.003750pt}%
\definecolor{currentstroke}{rgb}{1.000000,1.000000,1.000000}%
\pgfsetstrokecolor{currentstroke}%
\pgfsetdash{}{0pt}%
\pgfpathmoveto{\pgfqpoint{0.626035in}{0.557870in}}%
\pgfpathlineto{\pgfqpoint{0.626035in}{2.242604in}}%
\pgfusepath{stroke}%
\end{pgfscope}%
\begin{pgfscope}%
\definecolor{textcolor}{rgb}{0.150000,0.150000,0.150000}%
\pgfsetstrokecolor{textcolor}%
\pgfsetfillcolor{textcolor}%
\pgftext[x=0.626035in,y=0.425926in,,top]{\color{textcolor}\sffamily\fontsize{11.000000}{13.200000}\selectfont \(\displaystyle 0\)}%
\end{pgfscope}%
\begin{pgfscope}%
\pgfpathrectangle{\pgfqpoint{0.574769in}{0.557870in}}{\pgfqpoint{2.484109in}{1.684734in}}%
\pgfusepath{clip}%
\pgfsetroundcap%
\pgfsetroundjoin%
\pgfsetlinewidth{1.003750pt}%
\definecolor{currentstroke}{rgb}{1.000000,1.000000,1.000000}%
\pgfsetstrokecolor{currentstroke}%
\pgfsetdash{}{0pt}%
\pgfpathmoveto{\pgfqpoint{1.464058in}{0.557870in}}%
\pgfpathlineto{\pgfqpoint{1.464058in}{2.242604in}}%
\pgfusepath{stroke}%
\end{pgfscope}%
\begin{pgfscope}%
\definecolor{textcolor}{rgb}{0.150000,0.150000,0.150000}%
\pgfsetstrokecolor{textcolor}%
\pgfsetfillcolor{textcolor}%
\pgftext[x=1.464058in,y=0.425926in,,top]{\color{textcolor}\sffamily\fontsize{11.000000}{13.200000}\selectfont \(\displaystyle 50\)}%
\end{pgfscope}%
\begin{pgfscope}%
\pgfpathrectangle{\pgfqpoint{0.574769in}{0.557870in}}{\pgfqpoint{2.484109in}{1.684734in}}%
\pgfusepath{clip}%
\pgfsetroundcap%
\pgfsetroundjoin%
\pgfsetlinewidth{1.003750pt}%
\definecolor{currentstroke}{rgb}{1.000000,1.000000,1.000000}%
\pgfsetstrokecolor{currentstroke}%
\pgfsetdash{}{0pt}%
\pgfpathmoveto{\pgfqpoint{2.302082in}{0.557870in}}%
\pgfpathlineto{\pgfqpoint{2.302082in}{2.242604in}}%
\pgfusepath{stroke}%
\end{pgfscope}%
\begin{pgfscope}%
\definecolor{textcolor}{rgb}{0.150000,0.150000,0.150000}%
\pgfsetstrokecolor{textcolor}%
\pgfsetfillcolor{textcolor}%
\pgftext[x=2.302082in,y=0.425926in,,top]{\color{textcolor}\sffamily\fontsize{11.000000}{13.200000}\selectfont \(\displaystyle 100\)}%
\end{pgfscope}%
\begin{pgfscope}%
\definecolor{textcolor}{rgb}{0.150000,0.150000,0.150000}%
\pgfsetstrokecolor{textcolor}%
\pgfsetfillcolor{textcolor}%
\pgftext[x=1.816823in,y=0.235185in,,top]{\color{textcolor}\sffamily\fontsize{11.000000}{13.200000}\selectfont DOR}%
\end{pgfscope}%
\begin{pgfscope}%
\pgfpathrectangle{\pgfqpoint{0.574769in}{0.557870in}}{\pgfqpoint{2.484109in}{1.684734in}}%
\pgfusepath{clip}%
\pgfsetroundcap%
\pgfsetroundjoin%
\pgfsetlinewidth{1.003750pt}%
\definecolor{currentstroke}{rgb}{1.000000,1.000000,1.000000}%
\pgfsetstrokecolor{currentstroke}%
\pgfsetdash{}{0pt}%
\pgfpathmoveto{\pgfqpoint{0.574769in}{0.557870in}}%
\pgfpathlineto{\pgfqpoint{3.058877in}{0.557870in}}%
\pgfusepath{stroke}%
\end{pgfscope}%
\begin{pgfscope}%
\definecolor{textcolor}{rgb}{0.150000,0.150000,0.150000}%
\pgfsetstrokecolor{textcolor}%
\pgfsetfillcolor{textcolor}%
\pgftext[x=0.366783in,y=0.505064in,left,base]{\color{textcolor}\sffamily\fontsize{11.000000}{13.200000}\selectfont \(\displaystyle 0\)}%
\end{pgfscope}%
\begin{pgfscope}%
\pgfpathrectangle{\pgfqpoint{0.574769in}{0.557870in}}{\pgfqpoint{2.484109in}{1.684734in}}%
\pgfusepath{clip}%
\pgfsetroundcap%
\pgfsetroundjoin%
\pgfsetlinewidth{1.003750pt}%
\definecolor{currentstroke}{rgb}{1.000000,1.000000,1.000000}%
\pgfsetstrokecolor{currentstroke}%
\pgfsetdash{}{0pt}%
\pgfpathmoveto{\pgfqpoint{0.574769in}{0.922531in}}%
\pgfpathlineto{\pgfqpoint{3.058877in}{0.922531in}}%
\pgfusepath{stroke}%
\end{pgfscope}%
\begin{pgfscope}%
\definecolor{textcolor}{rgb}{0.150000,0.150000,0.150000}%
\pgfsetstrokecolor{textcolor}%
\pgfsetfillcolor{textcolor}%
\pgftext[x=0.366783in,y=0.869725in,left,base]{\color{textcolor}\sffamily\fontsize{11.000000}{13.200000}\selectfont \(\displaystyle 5\)}%
\end{pgfscope}%
\begin{pgfscope}%
\pgfpathrectangle{\pgfqpoint{0.574769in}{0.557870in}}{\pgfqpoint{2.484109in}{1.684734in}}%
\pgfusepath{clip}%
\pgfsetroundcap%
\pgfsetroundjoin%
\pgfsetlinewidth{1.003750pt}%
\definecolor{currentstroke}{rgb}{1.000000,1.000000,1.000000}%
\pgfsetstrokecolor{currentstroke}%
\pgfsetdash{}{0pt}%
\pgfpathmoveto{\pgfqpoint{0.574769in}{1.287192in}}%
\pgfpathlineto{\pgfqpoint{3.058877in}{1.287192in}}%
\pgfusepath{stroke}%
\end{pgfscope}%
\begin{pgfscope}%
\definecolor{textcolor}{rgb}{0.150000,0.150000,0.150000}%
\pgfsetstrokecolor{textcolor}%
\pgfsetfillcolor{textcolor}%
\pgftext[x=0.290741in,y=1.234386in,left,base]{\color{textcolor}\sffamily\fontsize{11.000000}{13.200000}\selectfont \(\displaystyle 10\)}%
\end{pgfscope}%
\begin{pgfscope}%
\pgfpathrectangle{\pgfqpoint{0.574769in}{0.557870in}}{\pgfqpoint{2.484109in}{1.684734in}}%
\pgfusepath{clip}%
\pgfsetroundcap%
\pgfsetroundjoin%
\pgfsetlinewidth{1.003750pt}%
\definecolor{currentstroke}{rgb}{1.000000,1.000000,1.000000}%
\pgfsetstrokecolor{currentstroke}%
\pgfsetdash{}{0pt}%
\pgfpathmoveto{\pgfqpoint{0.574769in}{1.651853in}}%
\pgfpathlineto{\pgfqpoint{3.058877in}{1.651853in}}%
\pgfusepath{stroke}%
\end{pgfscope}%
\begin{pgfscope}%
\definecolor{textcolor}{rgb}{0.150000,0.150000,0.150000}%
\pgfsetstrokecolor{textcolor}%
\pgfsetfillcolor{textcolor}%
\pgftext[x=0.290741in,y=1.599047in,left,base]{\color{textcolor}\sffamily\fontsize{11.000000}{13.200000}\selectfont \(\displaystyle 15\)}%
\end{pgfscope}%
\begin{pgfscope}%
\pgfpathrectangle{\pgfqpoint{0.574769in}{0.557870in}}{\pgfqpoint{2.484109in}{1.684734in}}%
\pgfusepath{clip}%
\pgfsetroundcap%
\pgfsetroundjoin%
\pgfsetlinewidth{1.003750pt}%
\definecolor{currentstroke}{rgb}{1.000000,1.000000,1.000000}%
\pgfsetstrokecolor{currentstroke}%
\pgfsetdash{}{0pt}%
\pgfpathmoveto{\pgfqpoint{0.574769in}{2.016514in}}%
\pgfpathlineto{\pgfqpoint{3.058877in}{2.016514in}}%
\pgfusepath{stroke}%
\end{pgfscope}%
\begin{pgfscope}%
\definecolor{textcolor}{rgb}{0.150000,0.150000,0.150000}%
\pgfsetstrokecolor{textcolor}%
\pgfsetfillcolor{textcolor}%
\pgftext[x=0.290741in,y=1.963708in,left,base]{\color{textcolor}\sffamily\fontsize{11.000000}{13.200000}\selectfont \(\displaystyle 20\)}%
\end{pgfscope}%
\begin{pgfscope}%
\definecolor{textcolor}{rgb}{0.150000,0.150000,0.150000}%
\pgfsetstrokecolor{textcolor}%
\pgfsetfillcolor{textcolor}%
\pgftext[x=0.235185in,y=1.400237in,,bottom,rotate=90.000000]{\color{textcolor}\sffamily\fontsize{11.000000}{13.200000}\selectfont Occurance}%
\end{pgfscope}%
\begin{pgfscope}%
\pgfpathrectangle{\pgfqpoint{0.574769in}{0.557870in}}{\pgfqpoint{2.484109in}{1.684734in}}%
\pgfusepath{clip}%
\pgfsetbuttcap%
\pgfsetmiterjoin%
\definecolor{currentfill}{rgb}{0.298039,0.447059,0.690196}%
\pgfsetfillcolor{currentfill}%
\pgfsetfillopacity{0.400000}%
\pgfsetlinewidth{1.003750pt}%
\definecolor{currentstroke}{rgb}{1.000000,1.000000,1.000000}%
\pgfsetstrokecolor{currentstroke}%
\pgfsetstrokeopacity{0.400000}%
\pgfsetdash{}{0pt}%
\pgfpathmoveto{\pgfqpoint{0.687683in}{0.557870in}}%
\pgfpathlineto{\pgfqpoint{0.913511in}{0.557870in}}%
\pgfpathlineto{\pgfqpoint{0.913511in}{2.162379in}}%
\pgfpathlineto{\pgfqpoint{0.687683in}{2.162379in}}%
\pgfpathclose%
\pgfusepath{stroke,fill}%
\end{pgfscope}%
\begin{pgfscope}%
\pgfpathrectangle{\pgfqpoint{0.574769in}{0.557870in}}{\pgfqpoint{2.484109in}{1.684734in}}%
\pgfusepath{clip}%
\pgfsetbuttcap%
\pgfsetmiterjoin%
\definecolor{currentfill}{rgb}{0.298039,0.447059,0.690196}%
\pgfsetfillcolor{currentfill}%
\pgfsetfillopacity{0.400000}%
\pgfsetlinewidth{1.003750pt}%
\definecolor{currentstroke}{rgb}{1.000000,1.000000,1.000000}%
\pgfsetstrokecolor{currentstroke}%
\pgfsetstrokeopacity{0.400000}%
\pgfsetdash{}{0pt}%
\pgfpathmoveto{\pgfqpoint{0.913511in}{0.557870in}}%
\pgfpathlineto{\pgfqpoint{1.139339in}{0.557870in}}%
\pgfpathlineto{\pgfqpoint{1.139339in}{1.651853in}}%
\pgfpathlineto{\pgfqpoint{0.913511in}{1.651853in}}%
\pgfpathclose%
\pgfusepath{stroke,fill}%
\end{pgfscope}%
\begin{pgfscope}%
\pgfpathrectangle{\pgfqpoint{0.574769in}{0.557870in}}{\pgfqpoint{2.484109in}{1.684734in}}%
\pgfusepath{clip}%
\pgfsetbuttcap%
\pgfsetmiterjoin%
\definecolor{currentfill}{rgb}{0.298039,0.447059,0.690196}%
\pgfsetfillcolor{currentfill}%
\pgfsetfillopacity{0.400000}%
\pgfsetlinewidth{1.003750pt}%
\definecolor{currentstroke}{rgb}{1.000000,1.000000,1.000000}%
\pgfsetstrokecolor{currentstroke}%
\pgfsetstrokeopacity{0.400000}%
\pgfsetdash{}{0pt}%
\pgfpathmoveto{\pgfqpoint{1.139339in}{0.557870in}}%
\pgfpathlineto{\pgfqpoint{1.365167in}{0.557870in}}%
\pgfpathlineto{\pgfqpoint{1.365167in}{0.776667in}}%
\pgfpathlineto{\pgfqpoint{1.139339in}{0.776667in}}%
\pgfpathclose%
\pgfusepath{stroke,fill}%
\end{pgfscope}%
\begin{pgfscope}%
\pgfpathrectangle{\pgfqpoint{0.574769in}{0.557870in}}{\pgfqpoint{2.484109in}{1.684734in}}%
\pgfusepath{clip}%
\pgfsetbuttcap%
\pgfsetmiterjoin%
\definecolor{currentfill}{rgb}{0.298039,0.447059,0.690196}%
\pgfsetfillcolor{currentfill}%
\pgfsetfillopacity{0.400000}%
\pgfsetlinewidth{1.003750pt}%
\definecolor{currentstroke}{rgb}{1.000000,1.000000,1.000000}%
\pgfsetstrokecolor{currentstroke}%
\pgfsetstrokeopacity{0.400000}%
\pgfsetdash{}{0pt}%
\pgfpathmoveto{\pgfqpoint{1.365167in}{0.557870in}}%
\pgfpathlineto{\pgfqpoint{1.590995in}{0.557870in}}%
\pgfpathlineto{\pgfqpoint{1.590995in}{1.068396in}}%
\pgfpathlineto{\pgfqpoint{1.365167in}{1.068396in}}%
\pgfpathclose%
\pgfusepath{stroke,fill}%
\end{pgfscope}%
\begin{pgfscope}%
\pgfpathrectangle{\pgfqpoint{0.574769in}{0.557870in}}{\pgfqpoint{2.484109in}{1.684734in}}%
\pgfusepath{clip}%
\pgfsetbuttcap%
\pgfsetmiterjoin%
\definecolor{currentfill}{rgb}{0.298039,0.447059,0.690196}%
\pgfsetfillcolor{currentfill}%
\pgfsetfillopacity{0.400000}%
\pgfsetlinewidth{1.003750pt}%
\definecolor{currentstroke}{rgb}{1.000000,1.000000,1.000000}%
\pgfsetstrokecolor{currentstroke}%
\pgfsetstrokeopacity{0.400000}%
\pgfsetdash{}{0pt}%
\pgfpathmoveto{\pgfqpoint{1.590995in}{0.557870in}}%
\pgfpathlineto{\pgfqpoint{1.816823in}{0.557870in}}%
\pgfpathlineto{\pgfqpoint{1.816823in}{0.922531in}}%
\pgfpathlineto{\pgfqpoint{1.590995in}{0.922531in}}%
\pgfpathclose%
\pgfusepath{stroke,fill}%
\end{pgfscope}%
\begin{pgfscope}%
\pgfpathrectangle{\pgfqpoint{0.574769in}{0.557870in}}{\pgfqpoint{2.484109in}{1.684734in}}%
\pgfusepath{clip}%
\pgfsetbuttcap%
\pgfsetmiterjoin%
\definecolor{currentfill}{rgb}{0.298039,0.447059,0.690196}%
\pgfsetfillcolor{currentfill}%
\pgfsetfillopacity{0.400000}%
\pgfsetlinewidth{1.003750pt}%
\definecolor{currentstroke}{rgb}{1.000000,1.000000,1.000000}%
\pgfsetstrokecolor{currentstroke}%
\pgfsetstrokeopacity{0.400000}%
\pgfsetdash{}{0pt}%
\pgfpathmoveto{\pgfqpoint{1.816823in}{0.557870in}}%
\pgfpathlineto{\pgfqpoint{2.042651in}{0.557870in}}%
\pgfpathlineto{\pgfqpoint{2.042651in}{0.849599in}}%
\pgfpathlineto{\pgfqpoint{1.816823in}{0.849599in}}%
\pgfpathclose%
\pgfusepath{stroke,fill}%
\end{pgfscope}%
\begin{pgfscope}%
\pgfpathrectangle{\pgfqpoint{0.574769in}{0.557870in}}{\pgfqpoint{2.484109in}{1.684734in}}%
\pgfusepath{clip}%
\pgfsetbuttcap%
\pgfsetmiterjoin%
\definecolor{currentfill}{rgb}{0.298039,0.447059,0.690196}%
\pgfsetfillcolor{currentfill}%
\pgfsetfillopacity{0.400000}%
\pgfsetlinewidth{1.003750pt}%
\definecolor{currentstroke}{rgb}{1.000000,1.000000,1.000000}%
\pgfsetstrokecolor{currentstroke}%
\pgfsetstrokeopacity{0.400000}%
\pgfsetdash{}{0pt}%
\pgfpathmoveto{\pgfqpoint{2.042651in}{0.557870in}}%
\pgfpathlineto{\pgfqpoint{2.268479in}{0.557870in}}%
\pgfpathlineto{\pgfqpoint{2.268479in}{0.630802in}}%
\pgfpathlineto{\pgfqpoint{2.042651in}{0.630802in}}%
\pgfpathclose%
\pgfusepath{stroke,fill}%
\end{pgfscope}%
\begin{pgfscope}%
\pgfpathrectangle{\pgfqpoint{0.574769in}{0.557870in}}{\pgfqpoint{2.484109in}{1.684734in}}%
\pgfusepath{clip}%
\pgfsetbuttcap%
\pgfsetmiterjoin%
\definecolor{currentfill}{rgb}{0.298039,0.447059,0.690196}%
\pgfsetfillcolor{currentfill}%
\pgfsetfillopacity{0.400000}%
\pgfsetlinewidth{1.003750pt}%
\definecolor{currentstroke}{rgb}{1.000000,1.000000,1.000000}%
\pgfsetstrokecolor{currentstroke}%
\pgfsetstrokeopacity{0.400000}%
\pgfsetdash{}{0pt}%
\pgfpathmoveto{\pgfqpoint{2.268479in}{0.557870in}}%
\pgfpathlineto{\pgfqpoint{2.494307in}{0.557870in}}%
\pgfpathlineto{\pgfqpoint{2.494307in}{0.557870in}}%
\pgfpathlineto{\pgfqpoint{2.268479in}{0.557870in}}%
\pgfpathclose%
\pgfusepath{stroke,fill}%
\end{pgfscope}%
\begin{pgfscope}%
\pgfpathrectangle{\pgfqpoint{0.574769in}{0.557870in}}{\pgfqpoint{2.484109in}{1.684734in}}%
\pgfusepath{clip}%
\pgfsetbuttcap%
\pgfsetmiterjoin%
\definecolor{currentfill}{rgb}{0.298039,0.447059,0.690196}%
\pgfsetfillcolor{currentfill}%
\pgfsetfillopacity{0.400000}%
\pgfsetlinewidth{1.003750pt}%
\definecolor{currentstroke}{rgb}{1.000000,1.000000,1.000000}%
\pgfsetstrokecolor{currentstroke}%
\pgfsetstrokeopacity{0.400000}%
\pgfsetdash{}{0pt}%
\pgfpathmoveto{\pgfqpoint{2.494307in}{0.557870in}}%
\pgfpathlineto{\pgfqpoint{2.720135in}{0.557870in}}%
\pgfpathlineto{\pgfqpoint{2.720135in}{0.557870in}}%
\pgfpathlineto{\pgfqpoint{2.494307in}{0.557870in}}%
\pgfpathclose%
\pgfusepath{stroke,fill}%
\end{pgfscope}%
\begin{pgfscope}%
\pgfpathrectangle{\pgfqpoint{0.574769in}{0.557870in}}{\pgfqpoint{2.484109in}{1.684734in}}%
\pgfusepath{clip}%
\pgfsetbuttcap%
\pgfsetmiterjoin%
\definecolor{currentfill}{rgb}{0.298039,0.447059,0.690196}%
\pgfsetfillcolor{currentfill}%
\pgfsetfillopacity{0.400000}%
\pgfsetlinewidth{1.003750pt}%
\definecolor{currentstroke}{rgb}{1.000000,1.000000,1.000000}%
\pgfsetstrokecolor{currentstroke}%
\pgfsetstrokeopacity{0.400000}%
\pgfsetdash{}{0pt}%
\pgfpathmoveto{\pgfqpoint{2.720135in}{0.557870in}}%
\pgfpathlineto{\pgfqpoint{2.945963in}{0.557870in}}%
\pgfpathlineto{\pgfqpoint{2.945963in}{0.630802in}}%
\pgfpathlineto{\pgfqpoint{2.720135in}{0.630802in}}%
\pgfpathclose%
\pgfusepath{stroke,fill}%
\end{pgfscope}%
\begin{pgfscope}%
\pgfsetrectcap%
\pgfsetmiterjoin%
\pgfsetlinewidth{1.254687pt}%
\definecolor{currentstroke}{rgb}{1.000000,1.000000,1.000000}%
\pgfsetstrokecolor{currentstroke}%
\pgfsetdash{}{0pt}%
\pgfpathmoveto{\pgfqpoint{0.574769in}{0.557870in}}%
\pgfpathlineto{\pgfqpoint{0.574769in}{2.242604in}}%
\pgfusepath{stroke}%
\end{pgfscope}%
\begin{pgfscope}%
\pgfsetrectcap%
\pgfsetmiterjoin%
\pgfsetlinewidth{1.254687pt}%
\definecolor{currentstroke}{rgb}{1.000000,1.000000,1.000000}%
\pgfsetstrokecolor{currentstroke}%
\pgfsetdash{}{0pt}%
\pgfpathmoveto{\pgfqpoint{3.058877in}{0.557870in}}%
\pgfpathlineto{\pgfqpoint{3.058877in}{2.242604in}}%
\pgfusepath{stroke}%
\end{pgfscope}%
\begin{pgfscope}%
\pgfsetrectcap%
\pgfsetmiterjoin%
\pgfsetlinewidth{1.254687pt}%
\definecolor{currentstroke}{rgb}{1.000000,1.000000,1.000000}%
\pgfsetstrokecolor{currentstroke}%
\pgfsetdash{}{0pt}%
\pgfpathmoveto{\pgfqpoint{0.574769in}{0.557870in}}%
\pgfpathlineto{\pgfqpoint{3.058877in}{0.557870in}}%
\pgfusepath{stroke}%
\end{pgfscope}%
\begin{pgfscope}%
\pgfsetrectcap%
\pgfsetmiterjoin%
\pgfsetlinewidth{1.254687pt}%
\definecolor{currentstroke}{rgb}{1.000000,1.000000,1.000000}%
\pgfsetstrokecolor{currentstroke}%
\pgfsetdash{}{0pt}%
\pgfpathmoveto{\pgfqpoint{0.574769in}{2.242604in}}%
\pgfpathlineto{\pgfqpoint{3.058877in}{2.242604in}}%
\pgfusepath{stroke}%
\end{pgfscope}%
\begin{pgfscope}%
\definecolor{textcolor}{rgb}{0.150000,0.150000,0.150000}%
\pgfsetstrokecolor{textcolor}%
\pgfsetfillcolor{textcolor}%
\pgftext[x=1.816823in,y=2.325938in,,base]{\color{textcolor}\sffamily\fontsize{11.000000}{13.200000}\selectfont (a)}%
\end{pgfscope}%
\begin{pgfscope}%
\pgfsetbuttcap%
\pgfsetmiterjoin%
\definecolor{currentfill}{rgb}{0.917647,0.917647,0.949020}%
\pgfsetfillcolor{currentfill}%
\pgfsetlinewidth{0.000000pt}%
\definecolor{currentstroke}{rgb}{0.000000,0.000000,0.000000}%
\pgfsetstrokecolor{currentstroke}%
\pgfsetstrokeopacity{0.000000}%
\pgfsetdash{}{0pt}%
\pgfpathmoveto{\pgfqpoint{3.755891in}{0.557870in}}%
\pgfpathlineto{\pgfqpoint{6.240000in}{0.557870in}}%
\pgfpathlineto{\pgfqpoint{6.240000in}{2.242604in}}%
\pgfpathlineto{\pgfqpoint{3.755891in}{2.242604in}}%
\pgfpathclose%
\pgfusepath{fill}%
\end{pgfscope}%
\begin{pgfscope}%
\pgfpathrectangle{\pgfqpoint{3.755891in}{0.557870in}}{\pgfqpoint{2.484109in}{1.684734in}}%
\pgfusepath{clip}%
\pgfsetroundcap%
\pgfsetroundjoin%
\pgfsetlinewidth{1.003750pt}%
\definecolor{currentstroke}{rgb}{1.000000,1.000000,1.000000}%
\pgfsetstrokecolor{currentstroke}%
\pgfsetdash{}{0pt}%
\pgfpathmoveto{\pgfqpoint{3.868805in}{0.557870in}}%
\pgfpathlineto{\pgfqpoint{3.868805in}{2.242604in}}%
\pgfusepath{stroke}%
\end{pgfscope}%
\begin{pgfscope}%
\definecolor{textcolor}{rgb}{0.150000,0.150000,0.150000}%
\pgfsetstrokecolor{textcolor}%
\pgfsetfillcolor{textcolor}%
\pgftext[x=3.868805in,y=0.425926in,,top]{\color{textcolor}\sffamily\fontsize{11.000000}{13.200000}\selectfont \(\displaystyle 0.00\)}%
\end{pgfscope}%
\begin{pgfscope}%
\pgfpathrectangle{\pgfqpoint{3.755891in}{0.557870in}}{\pgfqpoint{2.484109in}{1.684734in}}%
\pgfusepath{clip}%
\pgfsetroundcap%
\pgfsetroundjoin%
\pgfsetlinewidth{1.003750pt}%
\definecolor{currentstroke}{rgb}{1.000000,1.000000,1.000000}%
\pgfsetstrokecolor{currentstroke}%
\pgfsetdash{}{0pt}%
\pgfpathmoveto{\pgfqpoint{4.433376in}{0.557870in}}%
\pgfpathlineto{\pgfqpoint{4.433376in}{2.242604in}}%
\pgfusepath{stroke}%
\end{pgfscope}%
\begin{pgfscope}%
\definecolor{textcolor}{rgb}{0.150000,0.150000,0.150000}%
\pgfsetstrokecolor{textcolor}%
\pgfsetfillcolor{textcolor}%
\pgftext[x=4.433376in,y=0.425926in,,top]{\color{textcolor}\sffamily\fontsize{11.000000}{13.200000}\selectfont \(\displaystyle 0.25\)}%
\end{pgfscope}%
\begin{pgfscope}%
\pgfpathrectangle{\pgfqpoint{3.755891in}{0.557870in}}{\pgfqpoint{2.484109in}{1.684734in}}%
\pgfusepath{clip}%
\pgfsetroundcap%
\pgfsetroundjoin%
\pgfsetlinewidth{1.003750pt}%
\definecolor{currentstroke}{rgb}{1.000000,1.000000,1.000000}%
\pgfsetstrokecolor{currentstroke}%
\pgfsetdash{}{0pt}%
\pgfpathmoveto{\pgfqpoint{4.997946in}{0.557870in}}%
\pgfpathlineto{\pgfqpoint{4.997946in}{2.242604in}}%
\pgfusepath{stroke}%
\end{pgfscope}%
\begin{pgfscope}%
\definecolor{textcolor}{rgb}{0.150000,0.150000,0.150000}%
\pgfsetstrokecolor{textcolor}%
\pgfsetfillcolor{textcolor}%
\pgftext[x=4.997946in,y=0.425926in,,top]{\color{textcolor}\sffamily\fontsize{11.000000}{13.200000}\selectfont \(\displaystyle 0.50\)}%
\end{pgfscope}%
\begin{pgfscope}%
\pgfpathrectangle{\pgfqpoint{3.755891in}{0.557870in}}{\pgfqpoint{2.484109in}{1.684734in}}%
\pgfusepath{clip}%
\pgfsetroundcap%
\pgfsetroundjoin%
\pgfsetlinewidth{1.003750pt}%
\definecolor{currentstroke}{rgb}{1.000000,1.000000,1.000000}%
\pgfsetstrokecolor{currentstroke}%
\pgfsetdash{}{0pt}%
\pgfpathmoveto{\pgfqpoint{5.562516in}{0.557870in}}%
\pgfpathlineto{\pgfqpoint{5.562516in}{2.242604in}}%
\pgfusepath{stroke}%
\end{pgfscope}%
\begin{pgfscope}%
\definecolor{textcolor}{rgb}{0.150000,0.150000,0.150000}%
\pgfsetstrokecolor{textcolor}%
\pgfsetfillcolor{textcolor}%
\pgftext[x=5.562516in,y=0.425926in,,top]{\color{textcolor}\sffamily\fontsize{11.000000}{13.200000}\selectfont \(\displaystyle 0.75\)}%
\end{pgfscope}%
\begin{pgfscope}%
\pgfpathrectangle{\pgfqpoint{3.755891in}{0.557870in}}{\pgfqpoint{2.484109in}{1.684734in}}%
\pgfusepath{clip}%
\pgfsetroundcap%
\pgfsetroundjoin%
\pgfsetlinewidth{1.003750pt}%
\definecolor{currentstroke}{rgb}{1.000000,1.000000,1.000000}%
\pgfsetstrokecolor{currentstroke}%
\pgfsetdash{}{0pt}%
\pgfpathmoveto{\pgfqpoint{6.127086in}{0.557870in}}%
\pgfpathlineto{\pgfqpoint{6.127086in}{2.242604in}}%
\pgfusepath{stroke}%
\end{pgfscope}%
\begin{pgfscope}%
\definecolor{textcolor}{rgb}{0.150000,0.150000,0.150000}%
\pgfsetstrokecolor{textcolor}%
\pgfsetfillcolor{textcolor}%
\pgftext[x=6.127086in,y=0.425926in,,top]{\color{textcolor}\sffamily\fontsize{11.000000}{13.200000}\selectfont \(\displaystyle 1.00\)}%
\end{pgfscope}%
\begin{pgfscope}%
\definecolor{textcolor}{rgb}{0.150000,0.150000,0.150000}%
\pgfsetstrokecolor{textcolor}%
\pgfsetfillcolor{textcolor}%
\pgftext[x=4.997946in,y=0.235185in,,top]{\color{textcolor}\sffamily\fontsize{11.000000}{13.200000}\selectfont Specificity}%
\end{pgfscope}%
\begin{pgfscope}%
\pgfpathrectangle{\pgfqpoint{3.755891in}{0.557870in}}{\pgfqpoint{2.484109in}{1.684734in}}%
\pgfusepath{clip}%
\pgfsetroundcap%
\pgfsetroundjoin%
\pgfsetlinewidth{1.003750pt}%
\definecolor{currentstroke}{rgb}{1.000000,1.000000,1.000000}%
\pgfsetstrokecolor{currentstroke}%
\pgfsetdash{}{0pt}%
\pgfpathmoveto{\pgfqpoint{3.755891in}{0.634449in}}%
\pgfpathlineto{\pgfqpoint{6.240000in}{0.634449in}}%
\pgfusepath{stroke}%
\end{pgfscope}%
\begin{pgfscope}%
\definecolor{textcolor}{rgb}{0.150000,0.150000,0.150000}%
\pgfsetstrokecolor{textcolor}%
\pgfsetfillcolor{textcolor}%
\pgftext[x=3.353576in,y=0.581642in,left,base]{\color{textcolor}\sffamily\fontsize{11.000000}{13.200000}\selectfont \(\displaystyle 0.00\)}%
\end{pgfscope}%
\begin{pgfscope}%
\pgfpathrectangle{\pgfqpoint{3.755891in}{0.557870in}}{\pgfqpoint{2.484109in}{1.684734in}}%
\pgfusepath{clip}%
\pgfsetroundcap%
\pgfsetroundjoin%
\pgfsetlinewidth{1.003750pt}%
\definecolor{currentstroke}{rgb}{1.000000,1.000000,1.000000}%
\pgfsetstrokecolor{currentstroke}%
\pgfsetdash{}{0pt}%
\pgfpathmoveto{\pgfqpoint{3.755891in}{1.017343in}}%
\pgfpathlineto{\pgfqpoint{6.240000in}{1.017343in}}%
\pgfusepath{stroke}%
\end{pgfscope}%
\begin{pgfscope}%
\definecolor{textcolor}{rgb}{0.150000,0.150000,0.150000}%
\pgfsetstrokecolor{textcolor}%
\pgfsetfillcolor{textcolor}%
\pgftext[x=3.353576in,y=0.964536in,left,base]{\color{textcolor}\sffamily\fontsize{11.000000}{13.200000}\selectfont \(\displaystyle 0.25\)}%
\end{pgfscope}%
\begin{pgfscope}%
\pgfpathrectangle{\pgfqpoint{3.755891in}{0.557870in}}{\pgfqpoint{2.484109in}{1.684734in}}%
\pgfusepath{clip}%
\pgfsetroundcap%
\pgfsetroundjoin%
\pgfsetlinewidth{1.003750pt}%
\definecolor{currentstroke}{rgb}{1.000000,1.000000,1.000000}%
\pgfsetstrokecolor{currentstroke}%
\pgfsetdash{}{0pt}%
\pgfpathmoveto{\pgfqpoint{3.755891in}{1.400237in}}%
\pgfpathlineto{\pgfqpoint{6.240000in}{1.400237in}}%
\pgfusepath{stroke}%
\end{pgfscope}%
\begin{pgfscope}%
\definecolor{textcolor}{rgb}{0.150000,0.150000,0.150000}%
\pgfsetstrokecolor{textcolor}%
\pgfsetfillcolor{textcolor}%
\pgftext[x=3.353576in,y=1.347431in,left,base]{\color{textcolor}\sffamily\fontsize{11.000000}{13.200000}\selectfont \(\displaystyle 0.50\)}%
\end{pgfscope}%
\begin{pgfscope}%
\pgfpathrectangle{\pgfqpoint{3.755891in}{0.557870in}}{\pgfqpoint{2.484109in}{1.684734in}}%
\pgfusepath{clip}%
\pgfsetroundcap%
\pgfsetroundjoin%
\pgfsetlinewidth{1.003750pt}%
\definecolor{currentstroke}{rgb}{1.000000,1.000000,1.000000}%
\pgfsetstrokecolor{currentstroke}%
\pgfsetdash{}{0pt}%
\pgfpathmoveto{\pgfqpoint{3.755891in}{1.783131in}}%
\pgfpathlineto{\pgfqpoint{6.240000in}{1.783131in}}%
\pgfusepath{stroke}%
\end{pgfscope}%
\begin{pgfscope}%
\definecolor{textcolor}{rgb}{0.150000,0.150000,0.150000}%
\pgfsetstrokecolor{textcolor}%
\pgfsetfillcolor{textcolor}%
\pgftext[x=3.353576in,y=1.730325in,left,base]{\color{textcolor}\sffamily\fontsize{11.000000}{13.200000}\selectfont \(\displaystyle 0.75\)}%
\end{pgfscope}%
\begin{pgfscope}%
\pgfpathrectangle{\pgfqpoint{3.755891in}{0.557870in}}{\pgfqpoint{2.484109in}{1.684734in}}%
\pgfusepath{clip}%
\pgfsetroundcap%
\pgfsetroundjoin%
\pgfsetlinewidth{1.003750pt}%
\definecolor{currentstroke}{rgb}{1.000000,1.000000,1.000000}%
\pgfsetstrokecolor{currentstroke}%
\pgfsetdash{}{0pt}%
\pgfpathmoveto{\pgfqpoint{3.755891in}{2.166025in}}%
\pgfpathlineto{\pgfqpoint{6.240000in}{2.166025in}}%
\pgfusepath{stroke}%
\end{pgfscope}%
\begin{pgfscope}%
\definecolor{textcolor}{rgb}{0.150000,0.150000,0.150000}%
\pgfsetstrokecolor{textcolor}%
\pgfsetfillcolor{textcolor}%
\pgftext[x=3.353576in,y=2.113219in,left,base]{\color{textcolor}\sffamily\fontsize{11.000000}{13.200000}\selectfont \(\displaystyle 1.00\)}%
\end{pgfscope}%
\begin{pgfscope}%
\definecolor{textcolor}{rgb}{0.150000,0.150000,0.150000}%
\pgfsetstrokecolor{textcolor}%
\pgfsetfillcolor{textcolor}%
\pgftext[x=3.298021in,y=1.400237in,,bottom,rotate=90.000000]{\color{textcolor}\sffamily\fontsize{11.000000}{13.200000}\selectfont Sensitivity}%
\end{pgfscope}%
\begin{pgfscope}%
\pgfpathrectangle{\pgfqpoint{3.755891in}{0.557870in}}{\pgfqpoint{2.484109in}{1.684734in}}%
\pgfusepath{clip}%
\pgfsetbuttcap%
\pgfsetroundjoin%
\definecolor{currentfill}{rgb}{0.298039,0.447059,0.690196}%
\pgfsetfillcolor{currentfill}%
\pgfsetlinewidth{1.003750pt}%
\definecolor{currentstroke}{rgb}{0.298039,0.447059,0.690196}%
\pgfsetstrokecolor{currentstroke}%
\pgfsetdash{}{0pt}%
\pgfpathmoveto{\pgfqpoint{4.160196in}{2.106780in}}%
\pgfpathcurveto{\pgfqpoint{4.168433in}{2.106780in}}{\pgfqpoint{4.176333in}{2.110053in}}{\pgfqpoint{4.182157in}{2.115877in}}%
\pgfpathcurveto{\pgfqpoint{4.187981in}{2.121701in}}{\pgfqpoint{4.191253in}{2.129601in}}{\pgfqpoint{4.191253in}{2.137837in}}%
\pgfpathcurveto{\pgfqpoint{4.191253in}{2.146073in}}{\pgfqpoint{4.187981in}{2.153973in}}{\pgfqpoint{4.182157in}{2.159797in}}%
\pgfpathcurveto{\pgfqpoint{4.176333in}{2.165621in}}{\pgfqpoint{4.168433in}{2.168893in}}{\pgfqpoint{4.160196in}{2.168893in}}%
\pgfpathcurveto{\pgfqpoint{4.151960in}{2.168893in}}{\pgfqpoint{4.144060in}{2.165621in}}{\pgfqpoint{4.138236in}{2.159797in}}%
\pgfpathcurveto{\pgfqpoint{4.132412in}{2.153973in}}{\pgfqpoint{4.129140in}{2.146073in}}{\pgfqpoint{4.129140in}{2.137837in}}%
\pgfpathcurveto{\pgfqpoint{4.129140in}{2.129601in}}{\pgfqpoint{4.132412in}{2.121701in}}{\pgfqpoint{4.138236in}{2.115877in}}%
\pgfpathcurveto{\pgfqpoint{4.144060in}{2.110053in}}{\pgfqpoint{4.151960in}{2.106780in}}{\pgfqpoint{4.160196in}{2.106780in}}%
\pgfpathclose%
\pgfusepath{stroke,fill}%
\end{pgfscope}%
\begin{pgfscope}%
\pgfpathrectangle{\pgfqpoint{3.755891in}{0.557870in}}{\pgfqpoint{2.484109in}{1.684734in}}%
\pgfusepath{clip}%
\pgfsetbuttcap%
\pgfsetroundjoin%
\definecolor{currentfill}{rgb}{0.298039,0.447059,0.690196}%
\pgfsetfillcolor{currentfill}%
\pgfsetlinewidth{1.003750pt}%
\definecolor{currentstroke}{rgb}{0.298039,0.447059,0.690196}%
\pgfsetstrokecolor{currentstroke}%
\pgfsetdash{}{0pt}%
\pgfpathmoveto{\pgfqpoint{4.815826in}{2.031611in}}%
\pgfpathcurveto{\pgfqpoint{4.824063in}{2.031611in}}{\pgfqpoint{4.831963in}{2.034883in}}{\pgfqpoint{4.837787in}{2.040707in}}%
\pgfpathcurveto{\pgfqpoint{4.843610in}{2.046531in}}{\pgfqpoint{4.846883in}{2.054431in}}{\pgfqpoint{4.846883in}{2.062667in}}%
\pgfpathcurveto{\pgfqpoint{4.846883in}{2.070904in}}{\pgfqpoint{4.843610in}{2.078804in}}{\pgfqpoint{4.837787in}{2.084628in}}%
\pgfpathcurveto{\pgfqpoint{4.831963in}{2.090452in}}{\pgfqpoint{4.824063in}{2.093724in}}{\pgfqpoint{4.815826in}{2.093724in}}%
\pgfpathcurveto{\pgfqpoint{4.807590in}{2.093724in}}{\pgfqpoint{4.799690in}{2.090452in}}{\pgfqpoint{4.793866in}{2.084628in}}%
\pgfpathcurveto{\pgfqpoint{4.788042in}{2.078804in}}{\pgfqpoint{4.784770in}{2.070904in}}{\pgfqpoint{4.784770in}{2.062667in}}%
\pgfpathcurveto{\pgfqpoint{4.784770in}{2.054431in}}{\pgfqpoint{4.788042in}{2.046531in}}{\pgfqpoint{4.793866in}{2.040707in}}%
\pgfpathcurveto{\pgfqpoint{4.799690in}{2.034883in}}{\pgfqpoint{4.807590in}{2.031611in}}{\pgfqpoint{4.815826in}{2.031611in}}%
\pgfpathclose%
\pgfusepath{stroke,fill}%
\end{pgfscope}%
\begin{pgfscope}%
\pgfpathrectangle{\pgfqpoint{3.755891in}{0.557870in}}{\pgfqpoint{2.484109in}{1.684734in}}%
\pgfusepath{clip}%
\pgfsetbuttcap%
\pgfsetroundjoin%
\definecolor{currentfill}{rgb}{0.298039,0.447059,0.690196}%
\pgfsetfillcolor{currentfill}%
\pgfsetlinewidth{1.003750pt}%
\definecolor{currentstroke}{rgb}{0.298039,0.447059,0.690196}%
\pgfsetstrokecolor{currentstroke}%
\pgfsetdash{}{0pt}%
\pgfpathmoveto{\pgfqpoint{4.014501in}{2.116177in}}%
\pgfpathcurveto{\pgfqpoint{4.022737in}{2.116177in}}{\pgfqpoint{4.030637in}{2.119449in}}{\pgfqpoint{4.036461in}{2.125273in}}%
\pgfpathcurveto{\pgfqpoint{4.042285in}{2.131097in}}{\pgfqpoint{4.045557in}{2.138997in}}{\pgfqpoint{4.045557in}{2.147233in}}%
\pgfpathcurveto{\pgfqpoint{4.045557in}{2.155469in}}{\pgfqpoint{4.042285in}{2.163369in}}{\pgfqpoint{4.036461in}{2.169193in}}%
\pgfpathcurveto{\pgfqpoint{4.030637in}{2.175017in}}{\pgfqpoint{4.022737in}{2.178290in}}{\pgfqpoint{4.014501in}{2.178290in}}%
\pgfpathcurveto{\pgfqpoint{4.006265in}{2.178290in}}{\pgfqpoint{3.998365in}{2.175017in}}{\pgfqpoint{3.992541in}{2.169193in}}%
\pgfpathcurveto{\pgfqpoint{3.986717in}{2.163369in}}{\pgfqpoint{3.983444in}{2.155469in}}{\pgfqpoint{3.983444in}{2.147233in}}%
\pgfpathcurveto{\pgfqpoint{3.983444in}{2.138997in}}{\pgfqpoint{3.986717in}{2.131097in}}{\pgfqpoint{3.992541in}{2.125273in}}%
\pgfpathcurveto{\pgfqpoint{3.998365in}{2.119449in}}{\pgfqpoint{4.006265in}{2.116177in}}{\pgfqpoint{4.014501in}{2.116177in}}%
\pgfpathclose%
\pgfusepath{stroke,fill}%
\end{pgfscope}%
\begin{pgfscope}%
\pgfpathrectangle{\pgfqpoint{3.755891in}{0.557870in}}{\pgfqpoint{2.484109in}{1.684734in}}%
\pgfusepath{clip}%
\pgfsetbuttcap%
\pgfsetroundjoin%
\definecolor{currentfill}{rgb}{0.298039,0.447059,0.690196}%
\pgfsetfillcolor{currentfill}%
\pgfsetlinewidth{1.003750pt}%
\definecolor{currentstroke}{rgb}{0.298039,0.447059,0.690196}%
\pgfsetstrokecolor{currentstroke}%
\pgfsetdash{}{0pt}%
\pgfpathmoveto{\pgfqpoint{4.014501in}{2.106780in}}%
\pgfpathcurveto{\pgfqpoint{4.022737in}{2.106780in}}{\pgfqpoint{4.030637in}{2.110053in}}{\pgfqpoint{4.036461in}{2.115877in}}%
\pgfpathcurveto{\pgfqpoint{4.042285in}{2.121701in}}{\pgfqpoint{4.045557in}{2.129601in}}{\pgfqpoint{4.045557in}{2.137837in}}%
\pgfpathcurveto{\pgfqpoint{4.045557in}{2.146073in}}{\pgfqpoint{4.042285in}{2.153973in}}{\pgfqpoint{4.036461in}{2.159797in}}%
\pgfpathcurveto{\pgfqpoint{4.030637in}{2.165621in}}{\pgfqpoint{4.022737in}{2.168893in}}{\pgfqpoint{4.014501in}{2.168893in}}%
\pgfpathcurveto{\pgfqpoint{4.006265in}{2.168893in}}{\pgfqpoint{3.998365in}{2.165621in}}{\pgfqpoint{3.992541in}{2.159797in}}%
\pgfpathcurveto{\pgfqpoint{3.986717in}{2.153973in}}{\pgfqpoint{3.983444in}{2.146073in}}{\pgfqpoint{3.983444in}{2.137837in}}%
\pgfpathcurveto{\pgfqpoint{3.983444in}{2.129601in}}{\pgfqpoint{3.986717in}{2.121701in}}{\pgfqpoint{3.992541in}{2.115877in}}%
\pgfpathcurveto{\pgfqpoint{3.998365in}{2.110053in}}{\pgfqpoint{4.006265in}{2.106780in}}{\pgfqpoint{4.014501in}{2.106780in}}%
\pgfpathclose%
\pgfusepath{stroke,fill}%
\end{pgfscope}%
\begin{pgfscope}%
\pgfpathrectangle{\pgfqpoint{3.755891in}{0.557870in}}{\pgfqpoint{2.484109in}{1.684734in}}%
\pgfusepath{clip}%
\pgfsetbuttcap%
\pgfsetroundjoin%
\definecolor{currentfill}{rgb}{0.298039,0.447059,0.690196}%
\pgfsetfillcolor{currentfill}%
\pgfsetlinewidth{1.003750pt}%
\definecolor{currentstroke}{rgb}{0.298039,0.447059,0.690196}%
\pgfsetstrokecolor{currentstroke}%
\pgfsetdash{}{0pt}%
\pgfpathmoveto{\pgfqpoint{4.888674in}{1.984630in}}%
\pgfpathcurveto{\pgfqpoint{4.896910in}{1.984630in}}{\pgfqpoint{4.904810in}{1.987902in}}{\pgfqpoint{4.910634in}{1.993726in}}%
\pgfpathcurveto{\pgfqpoint{4.916458in}{1.999550in}}{\pgfqpoint{4.919731in}{2.007450in}}{\pgfqpoint{4.919731in}{2.015687in}}%
\pgfpathcurveto{\pgfqpoint{4.919731in}{2.023923in}}{\pgfqpoint{4.916458in}{2.031823in}}{\pgfqpoint{4.910634in}{2.037647in}}%
\pgfpathcurveto{\pgfqpoint{4.904810in}{2.043471in}}{\pgfqpoint{4.896910in}{2.046743in}}{\pgfqpoint{4.888674in}{2.046743in}}%
\pgfpathcurveto{\pgfqpoint{4.880438in}{2.046743in}}{\pgfqpoint{4.872538in}{2.043471in}}{\pgfqpoint{4.866714in}{2.037647in}}%
\pgfpathcurveto{\pgfqpoint{4.860890in}{2.031823in}}{\pgfqpoint{4.857618in}{2.023923in}}{\pgfqpoint{4.857618in}{2.015687in}}%
\pgfpathcurveto{\pgfqpoint{4.857618in}{2.007450in}}{\pgfqpoint{4.860890in}{1.999550in}}{\pgfqpoint{4.866714in}{1.993726in}}%
\pgfpathcurveto{\pgfqpoint{4.872538in}{1.987902in}}{\pgfqpoint{4.880438in}{1.984630in}}{\pgfqpoint{4.888674in}{1.984630in}}%
\pgfpathclose%
\pgfusepath{stroke,fill}%
\end{pgfscope}%
\begin{pgfscope}%
\pgfpathrectangle{\pgfqpoint{3.755891in}{0.557870in}}{\pgfqpoint{2.484109in}{1.684734in}}%
\pgfusepath{clip}%
\pgfsetbuttcap%
\pgfsetroundjoin%
\definecolor{currentfill}{rgb}{0.298039,0.447059,0.690196}%
\pgfsetfillcolor{currentfill}%
\pgfsetlinewidth{1.003750pt}%
\definecolor{currentstroke}{rgb}{0.298039,0.447059,0.690196}%
\pgfsetstrokecolor{currentstroke}%
\pgfsetdash{}{0pt}%
\pgfpathmoveto{\pgfqpoint{4.815826in}{1.984630in}}%
\pgfpathcurveto{\pgfqpoint{4.824063in}{1.984630in}}{\pgfqpoint{4.831963in}{1.987902in}}{\pgfqpoint{4.837787in}{1.993726in}}%
\pgfpathcurveto{\pgfqpoint{4.843610in}{1.999550in}}{\pgfqpoint{4.846883in}{2.007450in}}{\pgfqpoint{4.846883in}{2.015687in}}%
\pgfpathcurveto{\pgfqpoint{4.846883in}{2.023923in}}{\pgfqpoint{4.843610in}{2.031823in}}{\pgfqpoint{4.837787in}{2.037647in}}%
\pgfpathcurveto{\pgfqpoint{4.831963in}{2.043471in}}{\pgfqpoint{4.824063in}{2.046743in}}{\pgfqpoint{4.815826in}{2.046743in}}%
\pgfpathcurveto{\pgfqpoint{4.807590in}{2.046743in}}{\pgfqpoint{4.799690in}{2.043471in}}{\pgfqpoint{4.793866in}{2.037647in}}%
\pgfpathcurveto{\pgfqpoint{4.788042in}{2.031823in}}{\pgfqpoint{4.784770in}{2.023923in}}{\pgfqpoint{4.784770in}{2.015687in}}%
\pgfpathcurveto{\pgfqpoint{4.784770in}{2.007450in}}{\pgfqpoint{4.788042in}{1.999550in}}{\pgfqpoint{4.793866in}{1.993726in}}%
\pgfpathcurveto{\pgfqpoint{4.799690in}{1.987902in}}{\pgfqpoint{4.807590in}{1.984630in}}{\pgfqpoint{4.815826in}{1.984630in}}%
\pgfpathclose%
\pgfusepath{stroke,fill}%
\end{pgfscope}%
\begin{pgfscope}%
\pgfpathrectangle{\pgfqpoint{3.755891in}{0.557870in}}{\pgfqpoint{2.484109in}{1.684734in}}%
\pgfusepath{clip}%
\pgfsetbuttcap%
\pgfsetroundjoin%
\definecolor{currentfill}{rgb}{0.298039,0.447059,0.690196}%
\pgfsetfillcolor{currentfill}%
\pgfsetlinewidth{1.003750pt}%
\definecolor{currentstroke}{rgb}{0.298039,0.447059,0.690196}%
\pgfsetstrokecolor{currentstroke}%
\pgfsetdash{}{0pt}%
\pgfpathmoveto{\pgfqpoint{4.815826in}{2.041007in}}%
\pgfpathcurveto{\pgfqpoint{4.824063in}{2.041007in}}{\pgfqpoint{4.831963in}{2.044279in}}{\pgfqpoint{4.837787in}{2.050103in}}%
\pgfpathcurveto{\pgfqpoint{4.843610in}{2.055927in}}{\pgfqpoint{4.846883in}{2.063827in}}{\pgfqpoint{4.846883in}{2.072064in}}%
\pgfpathcurveto{\pgfqpoint{4.846883in}{2.080300in}}{\pgfqpoint{4.843610in}{2.088200in}}{\pgfqpoint{4.837787in}{2.094024in}}%
\pgfpathcurveto{\pgfqpoint{4.831963in}{2.099848in}}{\pgfqpoint{4.824063in}{2.103120in}}{\pgfqpoint{4.815826in}{2.103120in}}%
\pgfpathcurveto{\pgfqpoint{4.807590in}{2.103120in}}{\pgfqpoint{4.799690in}{2.099848in}}{\pgfqpoint{4.793866in}{2.094024in}}%
\pgfpathcurveto{\pgfqpoint{4.788042in}{2.088200in}}{\pgfqpoint{4.784770in}{2.080300in}}{\pgfqpoint{4.784770in}{2.072064in}}%
\pgfpathcurveto{\pgfqpoint{4.784770in}{2.063827in}}{\pgfqpoint{4.788042in}{2.055927in}}{\pgfqpoint{4.793866in}{2.050103in}}%
\pgfpathcurveto{\pgfqpoint{4.799690in}{2.044279in}}{\pgfqpoint{4.807590in}{2.041007in}}{\pgfqpoint{4.815826in}{2.041007in}}%
\pgfpathclose%
\pgfusepath{stroke,fill}%
\end{pgfscope}%
\begin{pgfscope}%
\pgfpathrectangle{\pgfqpoint{3.755891in}{0.557870in}}{\pgfqpoint{2.484109in}{1.684734in}}%
\pgfusepath{clip}%
\pgfsetbuttcap%
\pgfsetroundjoin%
\definecolor{currentfill}{rgb}{0.298039,0.447059,0.690196}%
\pgfsetfillcolor{currentfill}%
\pgfsetlinewidth{1.003750pt}%
\definecolor{currentstroke}{rgb}{0.298039,0.447059,0.690196}%
\pgfsetstrokecolor{currentstroke}%
\pgfsetdash{}{0pt}%
\pgfpathmoveto{\pgfqpoint{4.888674in}{2.031611in}}%
\pgfpathcurveto{\pgfqpoint{4.896910in}{2.031611in}}{\pgfqpoint{4.904810in}{2.034883in}}{\pgfqpoint{4.910634in}{2.040707in}}%
\pgfpathcurveto{\pgfqpoint{4.916458in}{2.046531in}}{\pgfqpoint{4.919731in}{2.054431in}}{\pgfqpoint{4.919731in}{2.062667in}}%
\pgfpathcurveto{\pgfqpoint{4.919731in}{2.070904in}}{\pgfqpoint{4.916458in}{2.078804in}}{\pgfqpoint{4.910634in}{2.084628in}}%
\pgfpathcurveto{\pgfqpoint{4.904810in}{2.090452in}}{\pgfqpoint{4.896910in}{2.093724in}}{\pgfqpoint{4.888674in}{2.093724in}}%
\pgfpathcurveto{\pgfqpoint{4.880438in}{2.093724in}}{\pgfqpoint{4.872538in}{2.090452in}}{\pgfqpoint{4.866714in}{2.084628in}}%
\pgfpathcurveto{\pgfqpoint{4.860890in}{2.078804in}}{\pgfqpoint{4.857618in}{2.070904in}}{\pgfqpoint{4.857618in}{2.062667in}}%
\pgfpathcurveto{\pgfqpoint{4.857618in}{2.054431in}}{\pgfqpoint{4.860890in}{2.046531in}}{\pgfqpoint{4.866714in}{2.040707in}}%
\pgfpathcurveto{\pgfqpoint{4.872538in}{2.034883in}}{\pgfqpoint{4.880438in}{2.031611in}}{\pgfqpoint{4.888674in}{2.031611in}}%
\pgfpathclose%
\pgfusepath{stroke,fill}%
\end{pgfscope}%
\begin{pgfscope}%
\pgfpathrectangle{\pgfqpoint{3.755891in}{0.557870in}}{\pgfqpoint{2.484109in}{1.684734in}}%
\pgfusepath{clip}%
\pgfsetbuttcap%
\pgfsetroundjoin%
\definecolor{currentfill}{rgb}{0.298039,0.447059,0.690196}%
\pgfsetfillcolor{currentfill}%
\pgfsetlinewidth{1.003750pt}%
\definecolor{currentstroke}{rgb}{0.298039,0.447059,0.690196}%
\pgfsetstrokecolor{currentstroke}%
\pgfsetdash{}{0pt}%
\pgfpathmoveto{\pgfqpoint{5.689999in}{1.890668in}}%
\pgfpathcurveto{\pgfqpoint{5.698236in}{1.890668in}}{\pgfqpoint{5.706136in}{1.893941in}}{\pgfqpoint{5.711960in}{1.899765in}}%
\pgfpathcurveto{\pgfqpoint{5.717784in}{1.905589in}}{\pgfqpoint{5.721056in}{1.913489in}}{\pgfqpoint{5.721056in}{1.921725in}}%
\pgfpathcurveto{\pgfqpoint{5.721056in}{1.929961in}}{\pgfqpoint{5.717784in}{1.937861in}}{\pgfqpoint{5.711960in}{1.943685in}}%
\pgfpathcurveto{\pgfqpoint{5.706136in}{1.949509in}}{\pgfqpoint{5.698236in}{1.952781in}}{\pgfqpoint{5.689999in}{1.952781in}}%
\pgfpathcurveto{\pgfqpoint{5.681763in}{1.952781in}}{\pgfqpoint{5.673863in}{1.949509in}}{\pgfqpoint{5.668039in}{1.943685in}}%
\pgfpathcurveto{\pgfqpoint{5.662215in}{1.937861in}}{\pgfqpoint{5.658943in}{1.929961in}}{\pgfqpoint{5.658943in}{1.921725in}}%
\pgfpathcurveto{\pgfqpoint{5.658943in}{1.913489in}}{\pgfqpoint{5.662215in}{1.905589in}}{\pgfqpoint{5.668039in}{1.899765in}}%
\pgfpathcurveto{\pgfqpoint{5.673863in}{1.893941in}}{\pgfqpoint{5.681763in}{1.890668in}}{\pgfqpoint{5.689999in}{1.890668in}}%
\pgfpathclose%
\pgfusepath{stroke,fill}%
\end{pgfscope}%
\begin{pgfscope}%
\pgfpathrectangle{\pgfqpoint{3.755891in}{0.557870in}}{\pgfqpoint{2.484109in}{1.684734in}}%
\pgfusepath{clip}%
\pgfsetbuttcap%
\pgfsetroundjoin%
\definecolor{currentfill}{rgb}{0.298039,0.447059,0.690196}%
\pgfsetfillcolor{currentfill}%
\pgfsetlinewidth{1.003750pt}%
\definecolor{currentstroke}{rgb}{0.298039,0.447059,0.690196}%
\pgfsetstrokecolor{currentstroke}%
\pgfsetdash{}{0pt}%
\pgfpathmoveto{\pgfqpoint{4.597283in}{2.041007in}}%
\pgfpathcurveto{\pgfqpoint{4.605519in}{2.041007in}}{\pgfqpoint{4.613419in}{2.044279in}}{\pgfqpoint{4.619243in}{2.050103in}}%
\pgfpathcurveto{\pgfqpoint{4.625067in}{2.055927in}}{\pgfqpoint{4.628339in}{2.063827in}}{\pgfqpoint{4.628339in}{2.072064in}}%
\pgfpathcurveto{\pgfqpoint{4.628339in}{2.080300in}}{\pgfqpoint{4.625067in}{2.088200in}}{\pgfqpoint{4.619243in}{2.094024in}}%
\pgfpathcurveto{\pgfqpoint{4.613419in}{2.099848in}}{\pgfqpoint{4.605519in}{2.103120in}}{\pgfqpoint{4.597283in}{2.103120in}}%
\pgfpathcurveto{\pgfqpoint{4.589047in}{2.103120in}}{\pgfqpoint{4.581147in}{2.099848in}}{\pgfqpoint{4.575323in}{2.094024in}}%
\pgfpathcurveto{\pgfqpoint{4.569499in}{2.088200in}}{\pgfqpoint{4.566226in}{2.080300in}}{\pgfqpoint{4.566226in}{2.072064in}}%
\pgfpathcurveto{\pgfqpoint{4.566226in}{2.063827in}}{\pgfqpoint{4.569499in}{2.055927in}}{\pgfqpoint{4.575323in}{2.050103in}}%
\pgfpathcurveto{\pgfqpoint{4.581147in}{2.044279in}}{\pgfqpoint{4.589047in}{2.041007in}}{\pgfqpoint{4.597283in}{2.041007in}}%
\pgfpathclose%
\pgfusepath{stroke,fill}%
\end{pgfscope}%
\begin{pgfscope}%
\pgfpathrectangle{\pgfqpoint{3.755891in}{0.557870in}}{\pgfqpoint{2.484109in}{1.684734in}}%
\pgfusepath{clip}%
\pgfsetbuttcap%
\pgfsetroundjoin%
\definecolor{currentfill}{rgb}{0.298039,0.447059,0.690196}%
\pgfsetfillcolor{currentfill}%
\pgfsetlinewidth{1.003750pt}%
\definecolor{currentstroke}{rgb}{0.298039,0.447059,0.690196}%
\pgfsetstrokecolor{currentstroke}%
\pgfsetdash{}{0pt}%
\pgfpathmoveto{\pgfqpoint{4.755987in}{2.029011in}}%
\pgfpathcurveto{\pgfqpoint{4.764223in}{2.029011in}}{\pgfqpoint{4.772123in}{2.032283in}}{\pgfqpoint{4.777947in}{2.038107in}}%
\pgfpathcurveto{\pgfqpoint{4.783771in}{2.043931in}}{\pgfqpoint{4.787044in}{2.051831in}}{\pgfqpoint{4.787044in}{2.060067in}}%
\pgfpathcurveto{\pgfqpoint{4.787044in}{2.068304in}}{\pgfqpoint{4.783771in}{2.076204in}}{\pgfqpoint{4.777947in}{2.082028in}}%
\pgfpathcurveto{\pgfqpoint{4.772123in}{2.087851in}}{\pgfqpoint{4.764223in}{2.091124in}}{\pgfqpoint{4.755987in}{2.091124in}}%
\pgfpathcurveto{\pgfqpoint{4.747751in}{2.091124in}}{\pgfqpoint{4.739851in}{2.087851in}}{\pgfqpoint{4.734027in}{2.082028in}}%
\pgfpathcurveto{\pgfqpoint{4.728203in}{2.076204in}}{\pgfqpoint{4.724931in}{2.068304in}}{\pgfqpoint{4.724931in}{2.060067in}}%
\pgfpathcurveto{\pgfqpoint{4.724931in}{2.051831in}}{\pgfqpoint{4.728203in}{2.043931in}}{\pgfqpoint{4.734027in}{2.038107in}}%
\pgfpathcurveto{\pgfqpoint{4.739851in}{2.032283in}}{\pgfqpoint{4.747751in}{2.029011in}}{\pgfqpoint{4.755987in}{2.029011in}}%
\pgfpathclose%
\pgfusepath{stroke,fill}%
\end{pgfscope}%
\begin{pgfscope}%
\pgfpathrectangle{\pgfqpoint{3.755891in}{0.557870in}}{\pgfqpoint{2.484109in}{1.684734in}}%
\pgfusepath{clip}%
\pgfsetbuttcap%
\pgfsetroundjoin%
\definecolor{currentfill}{rgb}{0.298039,0.447059,0.690196}%
\pgfsetfillcolor{currentfill}%
\pgfsetlinewidth{1.003750pt}%
\definecolor{currentstroke}{rgb}{0.298039,0.447059,0.690196}%
\pgfsetstrokecolor{currentstroke}%
\pgfsetdash{}{0pt}%
\pgfpathmoveto{\pgfqpoint{5.481863in}{2.057908in}}%
\pgfpathcurveto{\pgfqpoint{5.490099in}{2.057908in}}{\pgfqpoint{5.497999in}{2.061181in}}{\pgfqpoint{5.503823in}{2.067005in}}%
\pgfpathcurveto{\pgfqpoint{5.509647in}{2.072829in}}{\pgfqpoint{5.512919in}{2.080729in}}{\pgfqpoint{5.512919in}{2.088965in}}%
\pgfpathcurveto{\pgfqpoint{5.512919in}{2.097201in}}{\pgfqpoint{5.509647in}{2.105101in}}{\pgfqpoint{5.503823in}{2.110925in}}%
\pgfpathcurveto{\pgfqpoint{5.497999in}{2.116749in}}{\pgfqpoint{5.490099in}{2.120021in}}{\pgfqpoint{5.481863in}{2.120021in}}%
\pgfpathcurveto{\pgfqpoint{5.473627in}{2.120021in}}{\pgfqpoint{5.465727in}{2.116749in}}{\pgfqpoint{5.459903in}{2.110925in}}%
\pgfpathcurveto{\pgfqpoint{5.454079in}{2.105101in}}{\pgfqpoint{5.450806in}{2.097201in}}{\pgfqpoint{5.450806in}{2.088965in}}%
\pgfpathcurveto{\pgfqpoint{5.450806in}{2.080729in}}{\pgfqpoint{5.454079in}{2.072829in}}{\pgfqpoint{5.459903in}{2.067005in}}%
\pgfpathcurveto{\pgfqpoint{5.465727in}{2.061181in}}{\pgfqpoint{5.473627in}{2.057908in}}{\pgfqpoint{5.481863in}{2.057908in}}%
\pgfpathclose%
\pgfusepath{stroke,fill}%
\end{pgfscope}%
\begin{pgfscope}%
\pgfpathrectangle{\pgfqpoint{3.755891in}{0.557870in}}{\pgfqpoint{2.484109in}{1.684734in}}%
\pgfusepath{clip}%
\pgfsetbuttcap%
\pgfsetroundjoin%
\definecolor{currentfill}{rgb}{0.298039,0.447059,0.690196}%
\pgfsetfillcolor{currentfill}%
\pgfsetlinewidth{1.003750pt}%
\definecolor{currentstroke}{rgb}{0.298039,0.447059,0.690196}%
\pgfsetstrokecolor{currentstroke}%
\pgfsetdash{}{0pt}%
\pgfpathmoveto{\pgfqpoint{5.320557in}{2.096439in}}%
\pgfpathcurveto{\pgfqpoint{5.328793in}{2.096439in}}{\pgfqpoint{5.336694in}{2.099711in}}{\pgfqpoint{5.342517in}{2.105535in}}%
\pgfpathcurveto{\pgfqpoint{5.348341in}{2.111359in}}{\pgfqpoint{5.351614in}{2.119259in}}{\pgfqpoint{5.351614in}{2.127495in}}%
\pgfpathcurveto{\pgfqpoint{5.351614in}{2.135731in}}{\pgfqpoint{5.348341in}{2.143631in}}{\pgfqpoint{5.342517in}{2.149455in}}%
\pgfpathcurveto{\pgfqpoint{5.336694in}{2.155279in}}{\pgfqpoint{5.328793in}{2.158552in}}{\pgfqpoint{5.320557in}{2.158552in}}%
\pgfpathcurveto{\pgfqpoint{5.312321in}{2.158552in}}{\pgfqpoint{5.304421in}{2.155279in}}{\pgfqpoint{5.298597in}{2.149455in}}%
\pgfpathcurveto{\pgfqpoint{5.292773in}{2.143631in}}{\pgfqpoint{5.289501in}{2.135731in}}{\pgfqpoint{5.289501in}{2.127495in}}%
\pgfpathcurveto{\pgfqpoint{5.289501in}{2.119259in}}{\pgfqpoint{5.292773in}{2.111359in}}{\pgfqpoint{5.298597in}{2.105535in}}%
\pgfpathcurveto{\pgfqpoint{5.304421in}{2.099711in}}{\pgfqpoint{5.312321in}{2.096439in}}{\pgfqpoint{5.320557in}{2.096439in}}%
\pgfpathclose%
\pgfusepath{stroke,fill}%
\end{pgfscope}%
\begin{pgfscope}%
\pgfpathrectangle{\pgfqpoint{3.755891in}{0.557870in}}{\pgfqpoint{2.484109in}{1.684734in}}%
\pgfusepath{clip}%
\pgfsetbuttcap%
\pgfsetroundjoin%
\definecolor{currentfill}{rgb}{0.298039,0.447059,0.690196}%
\pgfsetfillcolor{currentfill}%
\pgfsetlinewidth{1.003750pt}%
\definecolor{currentstroke}{rgb}{0.298039,0.447059,0.690196}%
\pgfsetstrokecolor{currentstroke}%
\pgfsetdash{}{0pt}%
\pgfpathmoveto{\pgfqpoint{5.481863in}{2.019378in}}%
\pgfpathcurveto{\pgfqpoint{5.490099in}{2.019378in}}{\pgfqpoint{5.497999in}{2.022651in}}{\pgfqpoint{5.503823in}{2.028474in}}%
\pgfpathcurveto{\pgfqpoint{5.509647in}{2.034298in}}{\pgfqpoint{5.512919in}{2.042198in}}{\pgfqpoint{5.512919in}{2.050435in}}%
\pgfpathcurveto{\pgfqpoint{5.512919in}{2.058671in}}{\pgfqpoint{5.509647in}{2.066571in}}{\pgfqpoint{5.503823in}{2.072395in}}%
\pgfpathcurveto{\pgfqpoint{5.497999in}{2.078219in}}{\pgfqpoint{5.490099in}{2.081491in}}{\pgfqpoint{5.481863in}{2.081491in}}%
\pgfpathcurveto{\pgfqpoint{5.473627in}{2.081491in}}{\pgfqpoint{5.465727in}{2.078219in}}{\pgfqpoint{5.459903in}{2.072395in}}%
\pgfpathcurveto{\pgfqpoint{5.454079in}{2.066571in}}{\pgfqpoint{5.450806in}{2.058671in}}{\pgfqpoint{5.450806in}{2.050435in}}%
\pgfpathcurveto{\pgfqpoint{5.450806in}{2.042198in}}{\pgfqpoint{5.454079in}{2.034298in}}{\pgfqpoint{5.459903in}{2.028474in}}%
\pgfpathcurveto{\pgfqpoint{5.465727in}{2.022651in}}{\pgfqpoint{5.473627in}{2.019378in}}{\pgfqpoint{5.481863in}{2.019378in}}%
\pgfpathclose%
\pgfusepath{stroke,fill}%
\end{pgfscope}%
\begin{pgfscope}%
\pgfpathrectangle{\pgfqpoint{3.755891in}{0.557870in}}{\pgfqpoint{2.484109in}{1.684734in}}%
\pgfusepath{clip}%
\pgfsetbuttcap%
\pgfsetroundjoin%
\definecolor{currentfill}{rgb}{0.298039,0.447059,0.690196}%
\pgfsetfillcolor{currentfill}%
\pgfsetlinewidth{1.003750pt}%
\definecolor{currentstroke}{rgb}{0.298039,0.447059,0.690196}%
\pgfsetstrokecolor{currentstroke}%
\pgfsetdash{}{0pt}%
\pgfpathmoveto{\pgfqpoint{5.078599in}{2.038643in}}%
\pgfpathcurveto{\pgfqpoint{5.086835in}{2.038643in}}{\pgfqpoint{5.094735in}{2.041916in}}{\pgfqpoint{5.100559in}{2.047740in}}%
\pgfpathcurveto{\pgfqpoint{5.106383in}{2.053563in}}{\pgfqpoint{5.109655in}{2.061464in}}{\pgfqpoint{5.109655in}{2.069700in}}%
\pgfpathcurveto{\pgfqpoint{5.109655in}{2.077936in}}{\pgfqpoint{5.106383in}{2.085836in}}{\pgfqpoint{5.100559in}{2.091660in}}%
\pgfpathcurveto{\pgfqpoint{5.094735in}{2.097484in}}{\pgfqpoint{5.086835in}{2.100756in}}{\pgfqpoint{5.078599in}{2.100756in}}%
\pgfpathcurveto{\pgfqpoint{5.070362in}{2.100756in}}{\pgfqpoint{5.062462in}{2.097484in}}{\pgfqpoint{5.056638in}{2.091660in}}%
\pgfpathcurveto{\pgfqpoint{5.050814in}{2.085836in}}{\pgfqpoint{5.047542in}{2.077936in}}{\pgfqpoint{5.047542in}{2.069700in}}%
\pgfpathcurveto{\pgfqpoint{5.047542in}{2.061464in}}{\pgfqpoint{5.050814in}{2.053563in}}{\pgfqpoint{5.056638in}{2.047740in}}%
\pgfpathcurveto{\pgfqpoint{5.062462in}{2.041916in}}{\pgfqpoint{5.070362in}{2.038643in}}{\pgfqpoint{5.078599in}{2.038643in}}%
\pgfpathclose%
\pgfusepath{stroke,fill}%
\end{pgfscope}%
\begin{pgfscope}%
\pgfpathrectangle{\pgfqpoint{3.755891in}{0.557870in}}{\pgfqpoint{2.484109in}{1.684734in}}%
\pgfusepath{clip}%
\pgfsetbuttcap%
\pgfsetroundjoin%
\definecolor{currentfill}{rgb}{0.298039,0.447059,0.690196}%
\pgfsetfillcolor{currentfill}%
\pgfsetlinewidth{1.003750pt}%
\definecolor{currentstroke}{rgb}{0.298039,0.447059,0.690196}%
\pgfsetstrokecolor{currentstroke}%
\pgfsetdash{}{0pt}%
\pgfpathmoveto{\pgfqpoint{5.320557in}{2.057908in}}%
\pgfpathcurveto{\pgfqpoint{5.328793in}{2.057908in}}{\pgfqpoint{5.336694in}{2.061181in}}{\pgfqpoint{5.342517in}{2.067005in}}%
\pgfpathcurveto{\pgfqpoint{5.348341in}{2.072829in}}{\pgfqpoint{5.351614in}{2.080729in}}{\pgfqpoint{5.351614in}{2.088965in}}%
\pgfpathcurveto{\pgfqpoint{5.351614in}{2.097201in}}{\pgfqpoint{5.348341in}{2.105101in}}{\pgfqpoint{5.342517in}{2.110925in}}%
\pgfpathcurveto{\pgfqpoint{5.336694in}{2.116749in}}{\pgfqpoint{5.328793in}{2.120021in}}{\pgfqpoint{5.320557in}{2.120021in}}%
\pgfpathcurveto{\pgfqpoint{5.312321in}{2.120021in}}{\pgfqpoint{5.304421in}{2.116749in}}{\pgfqpoint{5.298597in}{2.110925in}}%
\pgfpathcurveto{\pgfqpoint{5.292773in}{2.105101in}}{\pgfqpoint{5.289501in}{2.097201in}}{\pgfqpoint{5.289501in}{2.088965in}}%
\pgfpathcurveto{\pgfqpoint{5.289501in}{2.080729in}}{\pgfqpoint{5.292773in}{2.072829in}}{\pgfqpoint{5.298597in}{2.067005in}}%
\pgfpathcurveto{\pgfqpoint{5.304421in}{2.061181in}}{\pgfqpoint{5.312321in}{2.057908in}}{\pgfqpoint{5.320557in}{2.057908in}}%
\pgfpathclose%
\pgfusepath{stroke,fill}%
\end{pgfscope}%
\begin{pgfscope}%
\pgfpathrectangle{\pgfqpoint{3.755891in}{0.557870in}}{\pgfqpoint{2.484109in}{1.684734in}}%
\pgfusepath{clip}%
\pgfsetbuttcap%
\pgfsetroundjoin%
\definecolor{currentfill}{rgb}{0.298039,0.447059,0.690196}%
\pgfsetfillcolor{currentfill}%
\pgfsetlinewidth{1.003750pt}%
\definecolor{currentstroke}{rgb}{0.298039,0.447059,0.690196}%
\pgfsetstrokecolor{currentstroke}%
\pgfsetdash{}{0pt}%
\pgfpathmoveto{\pgfqpoint{5.078599in}{2.086806in}}%
\pgfpathcurveto{\pgfqpoint{5.086835in}{2.086806in}}{\pgfqpoint{5.094735in}{2.090078in}}{\pgfqpoint{5.100559in}{2.095902in}}%
\pgfpathcurveto{\pgfqpoint{5.106383in}{2.101726in}}{\pgfqpoint{5.109655in}{2.109626in}}{\pgfqpoint{5.109655in}{2.117863in}}%
\pgfpathcurveto{\pgfqpoint{5.109655in}{2.126099in}}{\pgfqpoint{5.106383in}{2.133999in}}{\pgfqpoint{5.100559in}{2.139823in}}%
\pgfpathcurveto{\pgfqpoint{5.094735in}{2.145647in}}{\pgfqpoint{5.086835in}{2.148919in}}{\pgfqpoint{5.078599in}{2.148919in}}%
\pgfpathcurveto{\pgfqpoint{5.070362in}{2.148919in}}{\pgfqpoint{5.062462in}{2.145647in}}{\pgfqpoint{5.056638in}{2.139823in}}%
\pgfpathcurveto{\pgfqpoint{5.050814in}{2.133999in}}{\pgfqpoint{5.047542in}{2.126099in}}{\pgfqpoint{5.047542in}{2.117863in}}%
\pgfpathcurveto{\pgfqpoint{5.047542in}{2.109626in}}{\pgfqpoint{5.050814in}{2.101726in}}{\pgfqpoint{5.056638in}{2.095902in}}%
\pgfpathcurveto{\pgfqpoint{5.062462in}{2.090078in}}{\pgfqpoint{5.070362in}{2.086806in}}{\pgfqpoint{5.078599in}{2.086806in}}%
\pgfpathclose%
\pgfusepath{stroke,fill}%
\end{pgfscope}%
\begin{pgfscope}%
\pgfpathrectangle{\pgfqpoint{3.755891in}{0.557870in}}{\pgfqpoint{2.484109in}{1.684734in}}%
\pgfusepath{clip}%
\pgfsetbuttcap%
\pgfsetroundjoin%
\definecolor{currentfill}{rgb}{0.298039,0.447059,0.690196}%
\pgfsetfillcolor{currentfill}%
\pgfsetlinewidth{1.003750pt}%
\definecolor{currentstroke}{rgb}{0.298039,0.447059,0.690196}%
\pgfsetstrokecolor{currentstroke}%
\pgfsetdash{}{0pt}%
\pgfpathmoveto{\pgfqpoint{5.401210in}{2.086806in}}%
\pgfpathcurveto{\pgfqpoint{5.409446in}{2.086806in}}{\pgfqpoint{5.417346in}{2.090078in}}{\pgfqpoint{5.423170in}{2.095902in}}%
\pgfpathcurveto{\pgfqpoint{5.428994in}{2.101726in}}{\pgfqpoint{5.432267in}{2.109626in}}{\pgfqpoint{5.432267in}{2.117863in}}%
\pgfpathcurveto{\pgfqpoint{5.432267in}{2.126099in}}{\pgfqpoint{5.428994in}{2.133999in}}{\pgfqpoint{5.423170in}{2.139823in}}%
\pgfpathcurveto{\pgfqpoint{5.417346in}{2.145647in}}{\pgfqpoint{5.409446in}{2.148919in}}{\pgfqpoint{5.401210in}{2.148919in}}%
\pgfpathcurveto{\pgfqpoint{5.392974in}{2.148919in}}{\pgfqpoint{5.385074in}{2.145647in}}{\pgfqpoint{5.379250in}{2.139823in}}%
\pgfpathcurveto{\pgfqpoint{5.373426in}{2.133999in}}{\pgfqpoint{5.370154in}{2.126099in}}{\pgfqpoint{5.370154in}{2.117863in}}%
\pgfpathcurveto{\pgfqpoint{5.370154in}{2.109626in}}{\pgfqpoint{5.373426in}{2.101726in}}{\pgfqpoint{5.379250in}{2.095902in}}%
\pgfpathcurveto{\pgfqpoint{5.385074in}{2.090078in}}{\pgfqpoint{5.392974in}{2.086806in}}{\pgfqpoint{5.401210in}{2.086806in}}%
\pgfpathclose%
\pgfusepath{stroke,fill}%
\end{pgfscope}%
\begin{pgfscope}%
\pgfpathrectangle{\pgfqpoint{3.755891in}{0.557870in}}{\pgfqpoint{2.484109in}{1.684734in}}%
\pgfusepath{clip}%
\pgfsetbuttcap%
\pgfsetroundjoin%
\definecolor{currentfill}{rgb}{0.298039,0.447059,0.690196}%
\pgfsetfillcolor{currentfill}%
\pgfsetlinewidth{1.003750pt}%
\definecolor{currentstroke}{rgb}{0.298039,0.447059,0.690196}%
\pgfsetstrokecolor{currentstroke}%
\pgfsetdash{}{0pt}%
\pgfpathmoveto{\pgfqpoint{5.885127in}{1.768932in}}%
\pgfpathcurveto{\pgfqpoint{5.893364in}{1.768932in}}{\pgfqpoint{5.901264in}{1.772204in}}{\pgfqpoint{5.907088in}{1.778028in}}%
\pgfpathcurveto{\pgfqpoint{5.912912in}{1.783852in}}{\pgfqpoint{5.916184in}{1.791752in}}{\pgfqpoint{5.916184in}{1.799988in}}%
\pgfpathcurveto{\pgfqpoint{5.916184in}{1.808225in}}{\pgfqpoint{5.912912in}{1.816125in}}{\pgfqpoint{5.907088in}{1.821949in}}%
\pgfpathcurveto{\pgfqpoint{5.901264in}{1.827772in}}{\pgfqpoint{5.893364in}{1.831045in}}{\pgfqpoint{5.885127in}{1.831045in}}%
\pgfpathcurveto{\pgfqpoint{5.876891in}{1.831045in}}{\pgfqpoint{5.868991in}{1.827772in}}{\pgfqpoint{5.863167in}{1.821949in}}%
\pgfpathcurveto{\pgfqpoint{5.857343in}{1.816125in}}{\pgfqpoint{5.854071in}{1.808225in}}{\pgfqpoint{5.854071in}{1.799988in}}%
\pgfpathcurveto{\pgfqpoint{5.854071in}{1.791752in}}{\pgfqpoint{5.857343in}{1.783852in}}{\pgfqpoint{5.863167in}{1.778028in}}%
\pgfpathcurveto{\pgfqpoint{5.868991in}{1.772204in}}{\pgfqpoint{5.876891in}{1.768932in}}{\pgfqpoint{5.885127in}{1.768932in}}%
\pgfpathclose%
\pgfusepath{stroke,fill}%
\end{pgfscope}%
\begin{pgfscope}%
\pgfpathrectangle{\pgfqpoint{3.755891in}{0.557870in}}{\pgfqpoint{2.484109in}{1.684734in}}%
\pgfusepath{clip}%
\pgfsetbuttcap%
\pgfsetroundjoin%
\definecolor{currentfill}{rgb}{0.298039,0.447059,0.690196}%
\pgfsetfillcolor{currentfill}%
\pgfsetlinewidth{1.003750pt}%
\definecolor{currentstroke}{rgb}{0.298039,0.447059,0.690196}%
\pgfsetstrokecolor{currentstroke}%
\pgfsetdash{}{0pt}%
\pgfpathmoveto{\pgfqpoint{4.110764in}{2.115704in}}%
\pgfpathcurveto{\pgfqpoint{4.119000in}{2.115704in}}{\pgfqpoint{4.126900in}{2.118976in}}{\pgfqpoint{4.132724in}{2.124800in}}%
\pgfpathcurveto{\pgfqpoint{4.138548in}{2.130624in}}{\pgfqpoint{4.141820in}{2.138524in}}{\pgfqpoint{4.141820in}{2.146760in}}%
\pgfpathcurveto{\pgfqpoint{4.141820in}{2.154997in}}{\pgfqpoint{4.138548in}{2.162897in}}{\pgfqpoint{4.132724in}{2.168721in}}%
\pgfpathcurveto{\pgfqpoint{4.126900in}{2.174544in}}{\pgfqpoint{4.119000in}{2.177817in}}{\pgfqpoint{4.110764in}{2.177817in}}%
\pgfpathcurveto{\pgfqpoint{4.102528in}{2.177817in}}{\pgfqpoint{4.094628in}{2.174544in}}{\pgfqpoint{4.088804in}{2.168721in}}%
\pgfpathcurveto{\pgfqpoint{4.082980in}{2.162897in}}{\pgfqpoint{4.079707in}{2.154997in}}{\pgfqpoint{4.079707in}{2.146760in}}%
\pgfpathcurveto{\pgfqpoint{4.079707in}{2.138524in}}{\pgfqpoint{4.082980in}{2.130624in}}{\pgfqpoint{4.088804in}{2.124800in}}%
\pgfpathcurveto{\pgfqpoint{4.094628in}{2.118976in}}{\pgfqpoint{4.102528in}{2.115704in}}{\pgfqpoint{4.110764in}{2.115704in}}%
\pgfpathclose%
\pgfusepath{stroke,fill}%
\end{pgfscope}%
\begin{pgfscope}%
\pgfpathrectangle{\pgfqpoint{3.755891in}{0.557870in}}{\pgfqpoint{2.484109in}{1.684734in}}%
\pgfusepath{clip}%
\pgfsetbuttcap%
\pgfsetroundjoin%
\definecolor{currentfill}{rgb}{0.298039,0.447059,0.690196}%
\pgfsetfillcolor{currentfill}%
\pgfsetlinewidth{1.003750pt}%
\definecolor{currentstroke}{rgb}{0.298039,0.447059,0.690196}%
\pgfsetstrokecolor{currentstroke}%
\pgfsetdash{}{0pt}%
\pgfpathmoveto{\pgfqpoint{5.159251in}{2.036157in}}%
\pgfpathcurveto{\pgfqpoint{5.167488in}{2.036157in}}{\pgfqpoint{5.175388in}{2.039430in}}{\pgfqpoint{5.181212in}{2.045254in}}%
\pgfpathcurveto{\pgfqpoint{5.187036in}{2.051078in}}{\pgfqpoint{5.190308in}{2.058978in}}{\pgfqpoint{5.190308in}{2.067214in}}%
\pgfpathcurveto{\pgfqpoint{5.190308in}{2.075450in}}{\pgfqpoint{5.187036in}{2.083350in}}{\pgfqpoint{5.181212in}{2.089174in}}%
\pgfpathcurveto{\pgfqpoint{5.175388in}{2.094998in}}{\pgfqpoint{5.167488in}{2.098270in}}{\pgfqpoint{5.159251in}{2.098270in}}%
\pgfpathcurveto{\pgfqpoint{5.151015in}{2.098270in}}{\pgfqpoint{5.143115in}{2.094998in}}{\pgfqpoint{5.137291in}{2.089174in}}%
\pgfpathcurveto{\pgfqpoint{5.131467in}{2.083350in}}{\pgfqpoint{5.128195in}{2.075450in}}{\pgfqpoint{5.128195in}{2.067214in}}%
\pgfpathcurveto{\pgfqpoint{5.128195in}{2.058978in}}{\pgfqpoint{5.131467in}{2.051078in}}{\pgfqpoint{5.137291in}{2.045254in}}%
\pgfpathcurveto{\pgfqpoint{5.143115in}{2.039430in}}{\pgfqpoint{5.151015in}{2.036157in}}{\pgfqpoint{5.159251in}{2.036157in}}%
\pgfpathclose%
\pgfusepath{stroke,fill}%
\end{pgfscope}%
\begin{pgfscope}%
\pgfpathrectangle{\pgfqpoint{3.755891in}{0.557870in}}{\pgfqpoint{2.484109in}{1.684734in}}%
\pgfusepath{clip}%
\pgfsetbuttcap%
\pgfsetroundjoin%
\definecolor{currentfill}{rgb}{0.298039,0.447059,0.690196}%
\pgfsetfillcolor{currentfill}%
\pgfsetlinewidth{1.003750pt}%
\definecolor{currentstroke}{rgb}{0.298039,0.447059,0.690196}%
\pgfsetstrokecolor{currentstroke}%
\pgfsetdash{}{0pt}%
\pgfpathmoveto{\pgfqpoint{5.723822in}{2.055920in}}%
\pgfpathcurveto{\pgfqpoint{5.732058in}{2.055920in}}{\pgfqpoint{5.739958in}{2.059192in}}{\pgfqpoint{5.745782in}{2.065016in}}%
\pgfpathcurveto{\pgfqpoint{5.751606in}{2.070840in}}{\pgfqpoint{5.754878in}{2.078740in}}{\pgfqpoint{5.754878in}{2.086976in}}%
\pgfpathcurveto{\pgfqpoint{5.754878in}{2.095213in}}{\pgfqpoint{5.751606in}{2.103113in}}{\pgfqpoint{5.745782in}{2.108937in}}%
\pgfpathcurveto{\pgfqpoint{5.739958in}{2.114760in}}{\pgfqpoint{5.732058in}{2.118033in}}{\pgfqpoint{5.723822in}{2.118033in}}%
\pgfpathcurveto{\pgfqpoint{5.715585in}{2.118033in}}{\pgfqpoint{5.707685in}{2.114760in}}{\pgfqpoint{5.701861in}{2.108937in}}%
\pgfpathcurveto{\pgfqpoint{5.696037in}{2.103113in}}{\pgfqpoint{5.692765in}{2.095213in}}{\pgfqpoint{5.692765in}{2.086976in}}%
\pgfpathcurveto{\pgfqpoint{5.692765in}{2.078740in}}{\pgfqpoint{5.696037in}{2.070840in}}{\pgfqpoint{5.701861in}{2.065016in}}%
\pgfpathcurveto{\pgfqpoint{5.707685in}{2.059192in}}{\pgfqpoint{5.715585in}{2.055920in}}{\pgfqpoint{5.723822in}{2.055920in}}%
\pgfpathclose%
\pgfusepath{stroke,fill}%
\end{pgfscope}%
\begin{pgfscope}%
\pgfpathrectangle{\pgfqpoint{3.755891in}{0.557870in}}{\pgfqpoint{2.484109in}{1.684734in}}%
\pgfusepath{clip}%
\pgfsetbuttcap%
\pgfsetroundjoin%
\definecolor{currentfill}{rgb}{0.298039,0.447059,0.690196}%
\pgfsetfillcolor{currentfill}%
\pgfsetlinewidth{1.003750pt}%
\definecolor{currentstroke}{rgb}{0.298039,0.447059,0.690196}%
\pgfsetstrokecolor{currentstroke}%
\pgfsetdash{}{0pt}%
\pgfpathmoveto{\pgfqpoint{5.320557in}{2.075682in}}%
\pgfpathcurveto{\pgfqpoint{5.328793in}{2.075682in}}{\pgfqpoint{5.336694in}{2.078954in}}{\pgfqpoint{5.342517in}{2.084778in}}%
\pgfpathcurveto{\pgfqpoint{5.348341in}{2.090602in}}{\pgfqpoint{5.351614in}{2.098502in}}{\pgfqpoint{5.351614in}{2.106739in}}%
\pgfpathcurveto{\pgfqpoint{5.351614in}{2.114975in}}{\pgfqpoint{5.348341in}{2.122875in}}{\pgfqpoint{5.342517in}{2.128699in}}%
\pgfpathcurveto{\pgfqpoint{5.336694in}{2.134523in}}{\pgfqpoint{5.328793in}{2.137795in}}{\pgfqpoint{5.320557in}{2.137795in}}%
\pgfpathcurveto{\pgfqpoint{5.312321in}{2.137795in}}{\pgfqpoint{5.304421in}{2.134523in}}{\pgfqpoint{5.298597in}{2.128699in}}%
\pgfpathcurveto{\pgfqpoint{5.292773in}{2.122875in}}{\pgfqpoint{5.289501in}{2.114975in}}{\pgfqpoint{5.289501in}{2.106739in}}%
\pgfpathcurveto{\pgfqpoint{5.289501in}{2.098502in}}{\pgfqpoint{5.292773in}{2.090602in}}{\pgfqpoint{5.298597in}{2.084778in}}%
\pgfpathcurveto{\pgfqpoint{5.304421in}{2.078954in}}{\pgfqpoint{5.312321in}{2.075682in}}{\pgfqpoint{5.320557in}{2.075682in}}%
\pgfpathclose%
\pgfusepath{stroke,fill}%
\end{pgfscope}%
\begin{pgfscope}%
\pgfpathrectangle{\pgfqpoint{3.755891in}{0.557870in}}{\pgfqpoint{2.484109in}{1.684734in}}%
\pgfusepath{clip}%
\pgfsetbuttcap%
\pgfsetroundjoin%
\definecolor{currentfill}{rgb}{0.298039,0.447059,0.690196}%
\pgfsetfillcolor{currentfill}%
\pgfsetlinewidth{1.003750pt}%
\definecolor{currentstroke}{rgb}{0.298039,0.447059,0.690196}%
\pgfsetstrokecolor{currentstroke}%
\pgfsetdash{}{0pt}%
\pgfpathmoveto{\pgfqpoint{5.320557in}{2.006514in}}%
\pgfpathcurveto{\pgfqpoint{5.328793in}{2.006514in}}{\pgfqpoint{5.336694in}{2.009786in}}{\pgfqpoint{5.342517in}{2.015610in}}%
\pgfpathcurveto{\pgfqpoint{5.348341in}{2.021434in}}{\pgfqpoint{5.351614in}{2.029334in}}{\pgfqpoint{5.351614in}{2.037571in}}%
\pgfpathcurveto{\pgfqpoint{5.351614in}{2.045807in}}{\pgfqpoint{5.348341in}{2.053707in}}{\pgfqpoint{5.342517in}{2.059531in}}%
\pgfpathcurveto{\pgfqpoint{5.336694in}{2.065355in}}{\pgfqpoint{5.328793in}{2.068627in}}{\pgfqpoint{5.320557in}{2.068627in}}%
\pgfpathcurveto{\pgfqpoint{5.312321in}{2.068627in}}{\pgfqpoint{5.304421in}{2.065355in}}{\pgfqpoint{5.298597in}{2.059531in}}%
\pgfpathcurveto{\pgfqpoint{5.292773in}{2.053707in}}{\pgfqpoint{5.289501in}{2.045807in}}{\pgfqpoint{5.289501in}{2.037571in}}%
\pgfpathcurveto{\pgfqpoint{5.289501in}{2.029334in}}{\pgfqpoint{5.292773in}{2.021434in}}{\pgfqpoint{5.298597in}{2.015610in}}%
\pgfpathcurveto{\pgfqpoint{5.304421in}{2.009786in}}{\pgfqpoint{5.312321in}{2.006514in}}{\pgfqpoint{5.320557in}{2.006514in}}%
\pgfpathclose%
\pgfusepath{stroke,fill}%
\end{pgfscope}%
\begin{pgfscope}%
\pgfpathrectangle{\pgfqpoint{3.755891in}{0.557870in}}{\pgfqpoint{2.484109in}{1.684734in}}%
\pgfusepath{clip}%
\pgfsetbuttcap%
\pgfsetroundjoin%
\definecolor{currentfill}{rgb}{0.298039,0.447059,0.690196}%
\pgfsetfillcolor{currentfill}%
\pgfsetlinewidth{1.003750pt}%
\definecolor{currentstroke}{rgb}{0.298039,0.447059,0.690196}%
\pgfsetstrokecolor{currentstroke}%
\pgfsetdash{}{0pt}%
\pgfpathmoveto{\pgfqpoint{5.481863in}{2.055920in}}%
\pgfpathcurveto{\pgfqpoint{5.490099in}{2.055920in}}{\pgfqpoint{5.497999in}{2.059192in}}{\pgfqpoint{5.503823in}{2.065016in}}%
\pgfpathcurveto{\pgfqpoint{5.509647in}{2.070840in}}{\pgfqpoint{5.512919in}{2.078740in}}{\pgfqpoint{5.512919in}{2.086976in}}%
\pgfpathcurveto{\pgfqpoint{5.512919in}{2.095213in}}{\pgfqpoint{5.509647in}{2.103113in}}{\pgfqpoint{5.503823in}{2.108937in}}%
\pgfpathcurveto{\pgfqpoint{5.497999in}{2.114760in}}{\pgfqpoint{5.490099in}{2.118033in}}{\pgfqpoint{5.481863in}{2.118033in}}%
\pgfpathcurveto{\pgfqpoint{5.473627in}{2.118033in}}{\pgfqpoint{5.465727in}{2.114760in}}{\pgfqpoint{5.459903in}{2.108937in}}%
\pgfpathcurveto{\pgfqpoint{5.454079in}{2.103113in}}{\pgfqpoint{5.450806in}{2.095213in}}{\pgfqpoint{5.450806in}{2.086976in}}%
\pgfpathcurveto{\pgfqpoint{5.450806in}{2.078740in}}{\pgfqpoint{5.454079in}{2.070840in}}{\pgfqpoint{5.459903in}{2.065016in}}%
\pgfpathcurveto{\pgfqpoint{5.465727in}{2.059192in}}{\pgfqpoint{5.473627in}{2.055920in}}{\pgfqpoint{5.481863in}{2.055920in}}%
\pgfpathclose%
\pgfusepath{stroke,fill}%
\end{pgfscope}%
\begin{pgfscope}%
\pgfpathrectangle{\pgfqpoint{3.755891in}{0.557870in}}{\pgfqpoint{2.484109in}{1.684734in}}%
\pgfusepath{clip}%
\pgfsetbuttcap%
\pgfsetroundjoin%
\definecolor{currentfill}{rgb}{0.298039,0.447059,0.690196}%
\pgfsetfillcolor{currentfill}%
\pgfsetlinewidth{1.003750pt}%
\definecolor{currentstroke}{rgb}{0.298039,0.447059,0.690196}%
\pgfsetstrokecolor{currentstroke}%
\pgfsetdash{}{0pt}%
\pgfpathmoveto{\pgfqpoint{5.078599in}{2.075682in}}%
\pgfpathcurveto{\pgfqpoint{5.086835in}{2.075682in}}{\pgfqpoint{5.094735in}{2.078954in}}{\pgfqpoint{5.100559in}{2.084778in}}%
\pgfpathcurveto{\pgfqpoint{5.106383in}{2.090602in}}{\pgfqpoint{5.109655in}{2.098502in}}{\pgfqpoint{5.109655in}{2.106739in}}%
\pgfpathcurveto{\pgfqpoint{5.109655in}{2.114975in}}{\pgfqpoint{5.106383in}{2.122875in}}{\pgfqpoint{5.100559in}{2.128699in}}%
\pgfpathcurveto{\pgfqpoint{5.094735in}{2.134523in}}{\pgfqpoint{5.086835in}{2.137795in}}{\pgfqpoint{5.078599in}{2.137795in}}%
\pgfpathcurveto{\pgfqpoint{5.070362in}{2.137795in}}{\pgfqpoint{5.062462in}{2.134523in}}{\pgfqpoint{5.056638in}{2.128699in}}%
\pgfpathcurveto{\pgfqpoint{5.050814in}{2.122875in}}{\pgfqpoint{5.047542in}{2.114975in}}{\pgfqpoint{5.047542in}{2.106739in}}%
\pgfpathcurveto{\pgfqpoint{5.047542in}{2.098502in}}{\pgfqpoint{5.050814in}{2.090602in}}{\pgfqpoint{5.056638in}{2.084778in}}%
\pgfpathcurveto{\pgfqpoint{5.062462in}{2.078954in}}{\pgfqpoint{5.070362in}{2.075682in}}{\pgfqpoint{5.078599in}{2.075682in}}%
\pgfpathclose%
\pgfusepath{stroke,fill}%
\end{pgfscope}%
\begin{pgfscope}%
\pgfpathrectangle{\pgfqpoint{3.755891in}{0.557870in}}{\pgfqpoint{2.484109in}{1.684734in}}%
\pgfusepath{clip}%
\pgfsetbuttcap%
\pgfsetroundjoin%
\definecolor{currentfill}{rgb}{0.298039,0.447059,0.690196}%
\pgfsetfillcolor{currentfill}%
\pgfsetlinewidth{1.003750pt}%
\definecolor{currentstroke}{rgb}{0.298039,0.447059,0.690196}%
\pgfsetstrokecolor{currentstroke}%
\pgfsetdash{}{0pt}%
\pgfpathmoveto{\pgfqpoint{5.320557in}{2.065801in}}%
\pgfpathcurveto{\pgfqpoint{5.328793in}{2.065801in}}{\pgfqpoint{5.336694in}{2.069073in}}{\pgfqpoint{5.342517in}{2.074897in}}%
\pgfpathcurveto{\pgfqpoint{5.348341in}{2.080721in}}{\pgfqpoint{5.351614in}{2.088621in}}{\pgfqpoint{5.351614in}{2.096857in}}%
\pgfpathcurveto{\pgfqpoint{5.351614in}{2.105094in}}{\pgfqpoint{5.348341in}{2.112994in}}{\pgfqpoint{5.342517in}{2.118818in}}%
\pgfpathcurveto{\pgfqpoint{5.336694in}{2.124642in}}{\pgfqpoint{5.328793in}{2.127914in}}{\pgfqpoint{5.320557in}{2.127914in}}%
\pgfpathcurveto{\pgfqpoint{5.312321in}{2.127914in}}{\pgfqpoint{5.304421in}{2.124642in}}{\pgfqpoint{5.298597in}{2.118818in}}%
\pgfpathcurveto{\pgfqpoint{5.292773in}{2.112994in}}{\pgfqpoint{5.289501in}{2.105094in}}{\pgfqpoint{5.289501in}{2.096857in}}%
\pgfpathcurveto{\pgfqpoint{5.289501in}{2.088621in}}{\pgfqpoint{5.292773in}{2.080721in}}{\pgfqpoint{5.298597in}{2.074897in}}%
\pgfpathcurveto{\pgfqpoint{5.304421in}{2.069073in}}{\pgfqpoint{5.312321in}{2.065801in}}{\pgfqpoint{5.320557in}{2.065801in}}%
\pgfpathclose%
\pgfusepath{stroke,fill}%
\end{pgfscope}%
\begin{pgfscope}%
\pgfpathrectangle{\pgfqpoint{3.755891in}{0.557870in}}{\pgfqpoint{2.484109in}{1.684734in}}%
\pgfusepath{clip}%
\pgfsetbuttcap%
\pgfsetroundjoin%
\definecolor{currentfill}{rgb}{0.298039,0.447059,0.690196}%
\pgfsetfillcolor{currentfill}%
\pgfsetlinewidth{1.003750pt}%
\definecolor{currentstroke}{rgb}{0.298039,0.447059,0.690196}%
\pgfsetstrokecolor{currentstroke}%
\pgfsetdash{}{0pt}%
\pgfpathmoveto{\pgfqpoint{5.481863in}{2.085563in}}%
\pgfpathcurveto{\pgfqpoint{5.490099in}{2.085563in}}{\pgfqpoint{5.497999in}{2.088835in}}{\pgfqpoint{5.503823in}{2.094659in}}%
\pgfpathcurveto{\pgfqpoint{5.509647in}{2.100483in}}{\pgfqpoint{5.512919in}{2.108383in}}{\pgfqpoint{5.512919in}{2.116620in}}%
\pgfpathcurveto{\pgfqpoint{5.512919in}{2.124856in}}{\pgfqpoint{5.509647in}{2.132756in}}{\pgfqpoint{5.503823in}{2.138580in}}%
\pgfpathcurveto{\pgfqpoint{5.497999in}{2.144404in}}{\pgfqpoint{5.490099in}{2.147676in}}{\pgfqpoint{5.481863in}{2.147676in}}%
\pgfpathcurveto{\pgfqpoint{5.473627in}{2.147676in}}{\pgfqpoint{5.465727in}{2.144404in}}{\pgfqpoint{5.459903in}{2.138580in}}%
\pgfpathcurveto{\pgfqpoint{5.454079in}{2.132756in}}{\pgfqpoint{5.450806in}{2.124856in}}{\pgfqpoint{5.450806in}{2.116620in}}%
\pgfpathcurveto{\pgfqpoint{5.450806in}{2.108383in}}{\pgfqpoint{5.454079in}{2.100483in}}{\pgfqpoint{5.459903in}{2.094659in}}%
\pgfpathcurveto{\pgfqpoint{5.465727in}{2.088835in}}{\pgfqpoint{5.473627in}{2.085563in}}{\pgfqpoint{5.481863in}{2.085563in}}%
\pgfpathclose%
\pgfusepath{stroke,fill}%
\end{pgfscope}%
\begin{pgfscope}%
\pgfpathrectangle{\pgfqpoint{3.755891in}{0.557870in}}{\pgfqpoint{2.484109in}{1.684734in}}%
\pgfusepath{clip}%
\pgfsetbuttcap%
\pgfsetroundjoin%
\definecolor{currentfill}{rgb}{0.298039,0.447059,0.690196}%
\pgfsetfillcolor{currentfill}%
\pgfsetlinewidth{1.003750pt}%
\definecolor{currentstroke}{rgb}{0.298039,0.447059,0.690196}%
\pgfsetstrokecolor{currentstroke}%
\pgfsetdash{}{0pt}%
\pgfpathmoveto{\pgfqpoint{5.885127in}{1.759486in}}%
\pgfpathcurveto{\pgfqpoint{5.893364in}{1.759486in}}{\pgfqpoint{5.901264in}{1.762758in}}{\pgfqpoint{5.907088in}{1.768582in}}%
\pgfpathcurveto{\pgfqpoint{5.912912in}{1.774406in}}{\pgfqpoint{5.916184in}{1.782306in}}{\pgfqpoint{5.916184in}{1.790542in}}%
\pgfpathcurveto{\pgfqpoint{5.916184in}{1.798778in}}{\pgfqpoint{5.912912in}{1.806678in}}{\pgfqpoint{5.907088in}{1.812502in}}%
\pgfpathcurveto{\pgfqpoint{5.901264in}{1.818326in}}{\pgfqpoint{5.893364in}{1.821599in}}{\pgfqpoint{5.885127in}{1.821599in}}%
\pgfpathcurveto{\pgfqpoint{5.876891in}{1.821599in}}{\pgfqpoint{5.868991in}{1.818326in}}{\pgfqpoint{5.863167in}{1.812502in}}%
\pgfpathcurveto{\pgfqpoint{5.857343in}{1.806678in}}{\pgfqpoint{5.854071in}{1.798778in}}{\pgfqpoint{5.854071in}{1.790542in}}%
\pgfpathcurveto{\pgfqpoint{5.854071in}{1.782306in}}{\pgfqpoint{5.857343in}{1.774406in}}{\pgfqpoint{5.863167in}{1.768582in}}%
\pgfpathcurveto{\pgfqpoint{5.868991in}{1.762758in}}{\pgfqpoint{5.876891in}{1.759486in}}{\pgfqpoint{5.885127in}{1.759486in}}%
\pgfpathclose%
\pgfusepath{stroke,fill}%
\end{pgfscope}%
\begin{pgfscope}%
\pgfpathrectangle{\pgfqpoint{3.755891in}{0.557870in}}{\pgfqpoint{2.484109in}{1.684734in}}%
\pgfusepath{clip}%
\pgfsetbuttcap%
\pgfsetroundjoin%
\definecolor{currentfill}{rgb}{0.298039,0.447059,0.690196}%
\pgfsetfillcolor{currentfill}%
\pgfsetlinewidth{1.003750pt}%
\definecolor{currentstroke}{rgb}{0.298039,0.447059,0.690196}%
\pgfsetstrokecolor{currentstroke}%
\pgfsetdash{}{0pt}%
\pgfpathmoveto{\pgfqpoint{4.233044in}{2.087988in}}%
\pgfpathcurveto{\pgfqpoint{4.241280in}{2.087988in}}{\pgfqpoint{4.249180in}{2.091260in}}{\pgfqpoint{4.255004in}{2.097084in}}%
\pgfpathcurveto{\pgfqpoint{4.260828in}{2.102908in}}{\pgfqpoint{4.264101in}{2.110808in}}{\pgfqpoint{4.264101in}{2.119044in}}%
\pgfpathcurveto{\pgfqpoint{4.264101in}{2.127281in}}{\pgfqpoint{4.260828in}{2.135181in}}{\pgfqpoint{4.255004in}{2.141005in}}%
\pgfpathcurveto{\pgfqpoint{4.249180in}{2.146829in}}{\pgfqpoint{4.241280in}{2.150101in}}{\pgfqpoint{4.233044in}{2.150101in}}%
\pgfpathcurveto{\pgfqpoint{4.224808in}{2.150101in}}{\pgfqpoint{4.216908in}{2.146829in}}{\pgfqpoint{4.211084in}{2.141005in}}%
\pgfpathcurveto{\pgfqpoint{4.205260in}{2.135181in}}{\pgfqpoint{4.201988in}{2.127281in}}{\pgfqpoint{4.201988in}{2.119044in}}%
\pgfpathcurveto{\pgfqpoint{4.201988in}{2.110808in}}{\pgfqpoint{4.205260in}{2.102908in}}{\pgfqpoint{4.211084in}{2.097084in}}%
\pgfpathcurveto{\pgfqpoint{4.216908in}{2.091260in}}{\pgfqpoint{4.224808in}{2.087988in}}{\pgfqpoint{4.233044in}{2.087988in}}%
\pgfpathclose%
\pgfusepath{stroke,fill}%
\end{pgfscope}%
\begin{pgfscope}%
\pgfpathrectangle{\pgfqpoint{3.755891in}{0.557870in}}{\pgfqpoint{2.484109in}{1.684734in}}%
\pgfusepath{clip}%
\pgfsetbuttcap%
\pgfsetroundjoin%
\definecolor{currentfill}{rgb}{0.298039,0.447059,0.690196}%
\pgfsetfillcolor{currentfill}%
\pgfsetlinewidth{1.003750pt}%
\definecolor{currentstroke}{rgb}{0.298039,0.447059,0.690196}%
\pgfsetstrokecolor{currentstroke}%
\pgfsetdash{}{0pt}%
\pgfpathmoveto{\pgfqpoint{5.180065in}{1.994026in}}%
\pgfpathcurveto{\pgfqpoint{5.188301in}{1.994026in}}{\pgfqpoint{5.196201in}{1.997299in}}{\pgfqpoint{5.202025in}{2.003122in}}%
\pgfpathcurveto{\pgfqpoint{5.207849in}{2.008946in}}{\pgfqpoint{5.211122in}{2.016846in}}{\pgfqpoint{5.211122in}{2.025083in}}%
\pgfpathcurveto{\pgfqpoint{5.211122in}{2.033319in}}{\pgfqpoint{5.207849in}{2.041219in}}{\pgfqpoint{5.202025in}{2.047043in}}%
\pgfpathcurveto{\pgfqpoint{5.196201in}{2.052867in}}{\pgfqpoint{5.188301in}{2.056139in}}{\pgfqpoint{5.180065in}{2.056139in}}%
\pgfpathcurveto{\pgfqpoint{5.171829in}{2.056139in}}{\pgfqpoint{5.163929in}{2.052867in}}{\pgfqpoint{5.158105in}{2.047043in}}%
\pgfpathcurveto{\pgfqpoint{5.152281in}{2.041219in}}{\pgfqpoint{5.149009in}{2.033319in}}{\pgfqpoint{5.149009in}{2.025083in}}%
\pgfpathcurveto{\pgfqpoint{5.149009in}{2.016846in}}{\pgfqpoint{5.152281in}{2.008946in}}{\pgfqpoint{5.158105in}{2.003122in}}%
\pgfpathcurveto{\pgfqpoint{5.163929in}{1.997299in}}{\pgfqpoint{5.171829in}{1.994026in}}{\pgfqpoint{5.180065in}{1.994026in}}%
\pgfpathclose%
\pgfusepath{stroke,fill}%
\end{pgfscope}%
\begin{pgfscope}%
\pgfpathrectangle{\pgfqpoint{3.755891in}{0.557870in}}{\pgfqpoint{2.484109in}{1.684734in}}%
\pgfusepath{clip}%
\pgfsetbuttcap%
\pgfsetroundjoin%
\definecolor{currentfill}{rgb}{0.298039,0.447059,0.690196}%
\pgfsetfillcolor{currentfill}%
\pgfsetlinewidth{1.003750pt}%
\definecolor{currentstroke}{rgb}{0.298039,0.447059,0.690196}%
\pgfsetstrokecolor{currentstroke}%
\pgfsetdash{}{0pt}%
\pgfpathmoveto{\pgfqpoint{4.014501in}{2.116177in}}%
\pgfpathcurveto{\pgfqpoint{4.022737in}{2.116177in}}{\pgfqpoint{4.030637in}{2.119449in}}{\pgfqpoint{4.036461in}{2.125273in}}%
\pgfpathcurveto{\pgfqpoint{4.042285in}{2.131097in}}{\pgfqpoint{4.045557in}{2.138997in}}{\pgfqpoint{4.045557in}{2.147233in}}%
\pgfpathcurveto{\pgfqpoint{4.045557in}{2.155469in}}{\pgfqpoint{4.042285in}{2.163369in}}{\pgfqpoint{4.036461in}{2.169193in}}%
\pgfpathcurveto{\pgfqpoint{4.030637in}{2.175017in}}{\pgfqpoint{4.022737in}{2.178290in}}{\pgfqpoint{4.014501in}{2.178290in}}%
\pgfpathcurveto{\pgfqpoint{4.006265in}{2.178290in}}{\pgfqpoint{3.998365in}{2.175017in}}{\pgfqpoint{3.992541in}{2.169193in}}%
\pgfpathcurveto{\pgfqpoint{3.986717in}{2.163369in}}{\pgfqpoint{3.983444in}{2.155469in}}{\pgfqpoint{3.983444in}{2.147233in}}%
\pgfpathcurveto{\pgfqpoint{3.983444in}{2.138997in}}{\pgfqpoint{3.986717in}{2.131097in}}{\pgfqpoint{3.992541in}{2.125273in}}%
\pgfpathcurveto{\pgfqpoint{3.998365in}{2.119449in}}{\pgfqpoint{4.006265in}{2.116177in}}{\pgfqpoint{4.014501in}{2.116177in}}%
\pgfpathclose%
\pgfusepath{stroke,fill}%
\end{pgfscope}%
\begin{pgfscope}%
\pgfpathrectangle{\pgfqpoint{3.755891in}{0.557870in}}{\pgfqpoint{2.484109in}{1.684734in}}%
\pgfusepath{clip}%
\pgfsetbuttcap%
\pgfsetroundjoin%
\definecolor{currentfill}{rgb}{0.298039,0.447059,0.690196}%
\pgfsetfillcolor{currentfill}%
\pgfsetlinewidth{1.003750pt}%
\definecolor{currentstroke}{rgb}{0.298039,0.447059,0.690196}%
\pgfsetstrokecolor{currentstroke}%
\pgfsetdash{}{0pt}%
\pgfpathmoveto{\pgfqpoint{4.014501in}{2.125573in}}%
\pgfpathcurveto{\pgfqpoint{4.022737in}{2.125573in}}{\pgfqpoint{4.030637in}{2.128845in}}{\pgfqpoint{4.036461in}{2.134669in}}%
\pgfpathcurveto{\pgfqpoint{4.042285in}{2.140493in}}{\pgfqpoint{4.045557in}{2.148393in}}{\pgfqpoint{4.045557in}{2.156629in}}%
\pgfpathcurveto{\pgfqpoint{4.045557in}{2.164865in}}{\pgfqpoint{4.042285in}{2.172766in}}{\pgfqpoint{4.036461in}{2.178589in}}%
\pgfpathcurveto{\pgfqpoint{4.030637in}{2.184413in}}{\pgfqpoint{4.022737in}{2.187686in}}{\pgfqpoint{4.014501in}{2.187686in}}%
\pgfpathcurveto{\pgfqpoint{4.006265in}{2.187686in}}{\pgfqpoint{3.998365in}{2.184413in}}{\pgfqpoint{3.992541in}{2.178589in}}%
\pgfpathcurveto{\pgfqpoint{3.986717in}{2.172766in}}{\pgfqpoint{3.983444in}{2.164865in}}{\pgfqpoint{3.983444in}{2.156629in}}%
\pgfpathcurveto{\pgfqpoint{3.983444in}{2.148393in}}{\pgfqpoint{3.986717in}{2.140493in}}{\pgfqpoint{3.992541in}{2.134669in}}%
\pgfpathcurveto{\pgfqpoint{3.998365in}{2.128845in}}{\pgfqpoint{4.006265in}{2.125573in}}{\pgfqpoint{4.014501in}{2.125573in}}%
\pgfpathclose%
\pgfusepath{stroke,fill}%
\end{pgfscope}%
\begin{pgfscope}%
\pgfpathrectangle{\pgfqpoint{3.755891in}{0.557870in}}{\pgfqpoint{2.484109in}{1.684734in}}%
\pgfusepath{clip}%
\pgfsetbuttcap%
\pgfsetroundjoin%
\definecolor{currentfill}{rgb}{0.298039,0.447059,0.690196}%
\pgfsetfillcolor{currentfill}%
\pgfsetlinewidth{1.003750pt}%
\definecolor{currentstroke}{rgb}{0.298039,0.447059,0.690196}%
\pgfsetstrokecolor{currentstroke}%
\pgfsetdash{}{0pt}%
\pgfpathmoveto{\pgfqpoint{5.107217in}{1.984630in}}%
\pgfpathcurveto{\pgfqpoint{5.115454in}{1.984630in}}{\pgfqpoint{5.123354in}{1.987902in}}{\pgfqpoint{5.129178in}{1.993726in}}%
\pgfpathcurveto{\pgfqpoint{5.135001in}{1.999550in}}{\pgfqpoint{5.138274in}{2.007450in}}{\pgfqpoint{5.138274in}{2.015687in}}%
\pgfpathcurveto{\pgfqpoint{5.138274in}{2.023923in}}{\pgfqpoint{5.135001in}{2.031823in}}{\pgfqpoint{5.129178in}{2.037647in}}%
\pgfpathcurveto{\pgfqpoint{5.123354in}{2.043471in}}{\pgfqpoint{5.115454in}{2.046743in}}{\pgfqpoint{5.107217in}{2.046743in}}%
\pgfpathcurveto{\pgfqpoint{5.098981in}{2.046743in}}{\pgfqpoint{5.091081in}{2.043471in}}{\pgfqpoint{5.085257in}{2.037647in}}%
\pgfpathcurveto{\pgfqpoint{5.079433in}{2.031823in}}{\pgfqpoint{5.076161in}{2.023923in}}{\pgfqpoint{5.076161in}{2.015687in}}%
\pgfpathcurveto{\pgfqpoint{5.076161in}{2.007450in}}{\pgfqpoint{5.079433in}{1.999550in}}{\pgfqpoint{5.085257in}{1.993726in}}%
\pgfpathcurveto{\pgfqpoint{5.091081in}{1.987902in}}{\pgfqpoint{5.098981in}{1.984630in}}{\pgfqpoint{5.107217in}{1.984630in}}%
\pgfpathclose%
\pgfusepath{stroke,fill}%
\end{pgfscope}%
\begin{pgfscope}%
\pgfpathrectangle{\pgfqpoint{3.755891in}{0.557870in}}{\pgfqpoint{2.484109in}{1.684734in}}%
\pgfusepath{clip}%
\pgfsetbuttcap%
\pgfsetroundjoin%
\definecolor{currentfill}{rgb}{0.298039,0.447059,0.690196}%
\pgfsetfillcolor{currentfill}%
\pgfsetlinewidth{1.003750pt}%
\definecolor{currentstroke}{rgb}{0.298039,0.447059,0.690196}%
\pgfsetstrokecolor{currentstroke}%
\pgfsetdash{}{0pt}%
\pgfpathmoveto{\pgfqpoint{4.888674in}{1.984630in}}%
\pgfpathcurveto{\pgfqpoint{4.896910in}{1.984630in}}{\pgfqpoint{4.904810in}{1.987902in}}{\pgfqpoint{4.910634in}{1.993726in}}%
\pgfpathcurveto{\pgfqpoint{4.916458in}{1.999550in}}{\pgfqpoint{4.919731in}{2.007450in}}{\pgfqpoint{4.919731in}{2.015687in}}%
\pgfpathcurveto{\pgfqpoint{4.919731in}{2.023923in}}{\pgfqpoint{4.916458in}{2.031823in}}{\pgfqpoint{4.910634in}{2.037647in}}%
\pgfpathcurveto{\pgfqpoint{4.904810in}{2.043471in}}{\pgfqpoint{4.896910in}{2.046743in}}{\pgfqpoint{4.888674in}{2.046743in}}%
\pgfpathcurveto{\pgfqpoint{4.880438in}{2.046743in}}{\pgfqpoint{4.872538in}{2.043471in}}{\pgfqpoint{4.866714in}{2.037647in}}%
\pgfpathcurveto{\pgfqpoint{4.860890in}{2.031823in}}{\pgfqpoint{4.857618in}{2.023923in}}{\pgfqpoint{4.857618in}{2.015687in}}%
\pgfpathcurveto{\pgfqpoint{4.857618in}{2.007450in}}{\pgfqpoint{4.860890in}{1.999550in}}{\pgfqpoint{4.866714in}{1.993726in}}%
\pgfpathcurveto{\pgfqpoint{4.872538in}{1.987902in}}{\pgfqpoint{4.880438in}{1.984630in}}{\pgfqpoint{4.888674in}{1.984630in}}%
\pgfpathclose%
\pgfusepath{stroke,fill}%
\end{pgfscope}%
\begin{pgfscope}%
\pgfpathrectangle{\pgfqpoint{3.755891in}{0.557870in}}{\pgfqpoint{2.484109in}{1.684734in}}%
\pgfusepath{clip}%
\pgfsetbuttcap%
\pgfsetroundjoin%
\definecolor{currentfill}{rgb}{0.298039,0.447059,0.690196}%
\pgfsetfillcolor{currentfill}%
\pgfsetlinewidth{1.003750pt}%
\definecolor{currentstroke}{rgb}{0.298039,0.447059,0.690196}%
\pgfsetstrokecolor{currentstroke}%
\pgfsetdash{}{0pt}%
\pgfpathmoveto{\pgfqpoint{5.034370in}{2.031611in}}%
\pgfpathcurveto{\pgfqpoint{5.042606in}{2.031611in}}{\pgfqpoint{5.050506in}{2.034883in}}{\pgfqpoint{5.056330in}{2.040707in}}%
\pgfpathcurveto{\pgfqpoint{5.062154in}{2.046531in}}{\pgfqpoint{5.065426in}{2.054431in}}{\pgfqpoint{5.065426in}{2.062667in}}%
\pgfpathcurveto{\pgfqpoint{5.065426in}{2.070904in}}{\pgfqpoint{5.062154in}{2.078804in}}{\pgfqpoint{5.056330in}{2.084628in}}%
\pgfpathcurveto{\pgfqpoint{5.050506in}{2.090452in}}{\pgfqpoint{5.042606in}{2.093724in}}{\pgfqpoint{5.034370in}{2.093724in}}%
\pgfpathcurveto{\pgfqpoint{5.026133in}{2.093724in}}{\pgfqpoint{5.018233in}{2.090452in}}{\pgfqpoint{5.012409in}{2.084628in}}%
\pgfpathcurveto{\pgfqpoint{5.006585in}{2.078804in}}{\pgfqpoint{5.003313in}{2.070904in}}{\pgfqpoint{5.003313in}{2.062667in}}%
\pgfpathcurveto{\pgfqpoint{5.003313in}{2.054431in}}{\pgfqpoint{5.006585in}{2.046531in}}{\pgfqpoint{5.012409in}{2.040707in}}%
\pgfpathcurveto{\pgfqpoint{5.018233in}{2.034883in}}{\pgfqpoint{5.026133in}{2.031611in}}{\pgfqpoint{5.034370in}{2.031611in}}%
\pgfpathclose%
\pgfusepath{stroke,fill}%
\end{pgfscope}%
\begin{pgfscope}%
\pgfpathrectangle{\pgfqpoint{3.755891in}{0.557870in}}{\pgfqpoint{2.484109in}{1.684734in}}%
\pgfusepath{clip}%
\pgfsetbuttcap%
\pgfsetroundjoin%
\definecolor{currentfill}{rgb}{0.298039,0.447059,0.690196}%
\pgfsetfillcolor{currentfill}%
\pgfsetlinewidth{1.003750pt}%
\definecolor{currentstroke}{rgb}{0.298039,0.447059,0.690196}%
\pgfsetstrokecolor{currentstroke}%
\pgfsetdash{}{0pt}%
\pgfpathmoveto{\pgfqpoint{5.180065in}{2.050403in}}%
\pgfpathcurveto{\pgfqpoint{5.188301in}{2.050403in}}{\pgfqpoint{5.196201in}{2.053676in}}{\pgfqpoint{5.202025in}{2.059500in}}%
\pgfpathcurveto{\pgfqpoint{5.207849in}{2.065323in}}{\pgfqpoint{5.211122in}{2.073224in}}{\pgfqpoint{5.211122in}{2.081460in}}%
\pgfpathcurveto{\pgfqpoint{5.211122in}{2.089696in}}{\pgfqpoint{5.207849in}{2.097596in}}{\pgfqpoint{5.202025in}{2.103420in}}%
\pgfpathcurveto{\pgfqpoint{5.196201in}{2.109244in}}{\pgfqpoint{5.188301in}{2.112516in}}{\pgfqpoint{5.180065in}{2.112516in}}%
\pgfpathcurveto{\pgfqpoint{5.171829in}{2.112516in}}{\pgfqpoint{5.163929in}{2.109244in}}{\pgfqpoint{5.158105in}{2.103420in}}%
\pgfpathcurveto{\pgfqpoint{5.152281in}{2.097596in}}{\pgfqpoint{5.149009in}{2.089696in}}{\pgfqpoint{5.149009in}{2.081460in}}%
\pgfpathcurveto{\pgfqpoint{5.149009in}{2.073224in}}{\pgfqpoint{5.152281in}{2.065323in}}{\pgfqpoint{5.158105in}{2.059500in}}%
\pgfpathcurveto{\pgfqpoint{5.163929in}{2.053676in}}{\pgfqpoint{5.171829in}{2.050403in}}{\pgfqpoint{5.180065in}{2.050403in}}%
\pgfpathclose%
\pgfusepath{stroke,fill}%
\end{pgfscope}%
\begin{pgfscope}%
\pgfpathrectangle{\pgfqpoint{3.755891in}{0.557870in}}{\pgfqpoint{2.484109in}{1.684734in}}%
\pgfusepath{clip}%
\pgfsetbuttcap%
\pgfsetroundjoin%
\definecolor{currentfill}{rgb}{0.298039,0.447059,0.690196}%
\pgfsetfillcolor{currentfill}%
\pgfsetlinewidth{1.003750pt}%
\definecolor{currentstroke}{rgb}{0.298039,0.447059,0.690196}%
\pgfsetstrokecolor{currentstroke}%
\pgfsetdash{}{0pt}%
\pgfpathmoveto{\pgfqpoint{5.762847in}{1.843687in}}%
\pgfpathcurveto{\pgfqpoint{5.771083in}{1.843687in}}{\pgfqpoint{5.778983in}{1.846960in}}{\pgfqpoint{5.784807in}{1.852784in}}%
\pgfpathcurveto{\pgfqpoint{5.790631in}{1.858608in}}{\pgfqpoint{5.793904in}{1.866508in}}{\pgfqpoint{5.793904in}{1.874744in}}%
\pgfpathcurveto{\pgfqpoint{5.793904in}{1.882980in}}{\pgfqpoint{5.790631in}{1.890880in}}{\pgfqpoint{5.784807in}{1.896704in}}%
\pgfpathcurveto{\pgfqpoint{5.778983in}{1.902528in}}{\pgfqpoint{5.771083in}{1.905800in}}{\pgfqpoint{5.762847in}{1.905800in}}%
\pgfpathcurveto{\pgfqpoint{5.754611in}{1.905800in}}{\pgfqpoint{5.746711in}{1.902528in}}{\pgfqpoint{5.740887in}{1.896704in}}%
\pgfpathcurveto{\pgfqpoint{5.735063in}{1.890880in}}{\pgfqpoint{5.731791in}{1.882980in}}{\pgfqpoint{5.731791in}{1.874744in}}%
\pgfpathcurveto{\pgfqpoint{5.731791in}{1.866508in}}{\pgfqpoint{5.735063in}{1.858608in}}{\pgfqpoint{5.740887in}{1.852784in}}%
\pgfpathcurveto{\pgfqpoint{5.746711in}{1.846960in}}{\pgfqpoint{5.754611in}{1.843687in}}{\pgfqpoint{5.762847in}{1.843687in}}%
\pgfpathclose%
\pgfusepath{stroke,fill}%
\end{pgfscope}%
\begin{pgfscope}%
\pgfpathrectangle{\pgfqpoint{3.755891in}{0.557870in}}{\pgfqpoint{2.484109in}{1.684734in}}%
\pgfusepath{clip}%
\pgfsetbuttcap%
\pgfsetroundjoin%
\definecolor{currentfill}{rgb}{0.298039,0.447059,0.690196}%
\pgfsetfillcolor{currentfill}%
\pgfsetlinewidth{1.003750pt}%
\definecolor{currentstroke}{rgb}{0.298039,0.447059,0.690196}%
\pgfsetstrokecolor{currentstroke}%
\pgfsetdash{}{0pt}%
\pgfpathmoveto{\pgfqpoint{4.597283in}{2.041007in}}%
\pgfpathcurveto{\pgfqpoint{4.605519in}{2.041007in}}{\pgfqpoint{4.613419in}{2.044279in}}{\pgfqpoint{4.619243in}{2.050103in}}%
\pgfpathcurveto{\pgfqpoint{4.625067in}{2.055927in}}{\pgfqpoint{4.628339in}{2.063827in}}{\pgfqpoint{4.628339in}{2.072064in}}%
\pgfpathcurveto{\pgfqpoint{4.628339in}{2.080300in}}{\pgfqpoint{4.625067in}{2.088200in}}{\pgfqpoint{4.619243in}{2.094024in}}%
\pgfpathcurveto{\pgfqpoint{4.613419in}{2.099848in}}{\pgfqpoint{4.605519in}{2.103120in}}{\pgfqpoint{4.597283in}{2.103120in}}%
\pgfpathcurveto{\pgfqpoint{4.589047in}{2.103120in}}{\pgfqpoint{4.581147in}{2.099848in}}{\pgfqpoint{4.575323in}{2.094024in}}%
\pgfpathcurveto{\pgfqpoint{4.569499in}{2.088200in}}{\pgfqpoint{4.566226in}{2.080300in}}{\pgfqpoint{4.566226in}{2.072064in}}%
\pgfpathcurveto{\pgfqpoint{4.566226in}{2.063827in}}{\pgfqpoint{4.569499in}{2.055927in}}{\pgfqpoint{4.575323in}{2.050103in}}%
\pgfpathcurveto{\pgfqpoint{4.581147in}{2.044279in}}{\pgfqpoint{4.589047in}{2.041007in}}{\pgfqpoint{4.597283in}{2.041007in}}%
\pgfpathclose%
\pgfusepath{stroke,fill}%
\end{pgfscope}%
\begin{pgfscope}%
\pgfpathrectangle{\pgfqpoint{3.755891in}{0.557870in}}{\pgfqpoint{2.484109in}{1.684734in}}%
\pgfusepath{clip}%
\pgfsetbuttcap%
\pgfsetroundjoin%
\definecolor{currentfill}{rgb}{0.298039,0.447059,0.690196}%
\pgfsetfillcolor{currentfill}%
\pgfsetlinewidth{1.003750pt}%
\definecolor{currentstroke}{rgb}{0.298039,0.447059,0.690196}%
\pgfsetstrokecolor{currentstroke}%
\pgfsetdash{}{0pt}%
\pgfpathmoveto{\pgfqpoint{4.755987in}{2.057908in}}%
\pgfpathcurveto{\pgfqpoint{4.764223in}{2.057908in}}{\pgfqpoint{4.772123in}{2.061181in}}{\pgfqpoint{4.777947in}{2.067005in}}%
\pgfpathcurveto{\pgfqpoint{4.783771in}{2.072829in}}{\pgfqpoint{4.787044in}{2.080729in}}{\pgfqpoint{4.787044in}{2.088965in}}%
\pgfpathcurveto{\pgfqpoint{4.787044in}{2.097201in}}{\pgfqpoint{4.783771in}{2.105101in}}{\pgfqpoint{4.777947in}{2.110925in}}%
\pgfpathcurveto{\pgfqpoint{4.772123in}{2.116749in}}{\pgfqpoint{4.764223in}{2.120021in}}{\pgfqpoint{4.755987in}{2.120021in}}%
\pgfpathcurveto{\pgfqpoint{4.747751in}{2.120021in}}{\pgfqpoint{4.739851in}{2.116749in}}{\pgfqpoint{4.734027in}{2.110925in}}%
\pgfpathcurveto{\pgfqpoint{4.728203in}{2.105101in}}{\pgfqpoint{4.724931in}{2.097201in}}{\pgfqpoint{4.724931in}{2.088965in}}%
\pgfpathcurveto{\pgfqpoint{4.724931in}{2.080729in}}{\pgfqpoint{4.728203in}{2.072829in}}{\pgfqpoint{4.734027in}{2.067005in}}%
\pgfpathcurveto{\pgfqpoint{4.739851in}{2.061181in}}{\pgfqpoint{4.747751in}{2.057908in}}{\pgfqpoint{4.755987in}{2.057908in}}%
\pgfpathclose%
\pgfusepath{stroke,fill}%
\end{pgfscope}%
\begin{pgfscope}%
\pgfpathrectangle{\pgfqpoint{3.755891in}{0.557870in}}{\pgfqpoint{2.484109in}{1.684734in}}%
\pgfusepath{clip}%
\pgfsetbuttcap%
\pgfsetroundjoin%
\definecolor{currentfill}{rgb}{0.298039,0.447059,0.690196}%
\pgfsetfillcolor{currentfill}%
\pgfsetlinewidth{1.003750pt}%
\definecolor{currentstroke}{rgb}{0.298039,0.447059,0.690196}%
\pgfsetstrokecolor{currentstroke}%
\pgfsetdash{}{0pt}%
\pgfpathmoveto{\pgfqpoint{5.643169in}{2.038643in}}%
\pgfpathcurveto{\pgfqpoint{5.651405in}{2.038643in}}{\pgfqpoint{5.659305in}{2.041916in}}{\pgfqpoint{5.665129in}{2.047740in}}%
\pgfpathcurveto{\pgfqpoint{5.670953in}{2.053563in}}{\pgfqpoint{5.674225in}{2.061464in}}{\pgfqpoint{5.674225in}{2.069700in}}%
\pgfpathcurveto{\pgfqpoint{5.674225in}{2.077936in}}{\pgfqpoint{5.670953in}{2.085836in}}{\pgfqpoint{5.665129in}{2.091660in}}%
\pgfpathcurveto{\pgfqpoint{5.659305in}{2.097484in}}{\pgfqpoint{5.651405in}{2.100756in}}{\pgfqpoint{5.643169in}{2.100756in}}%
\pgfpathcurveto{\pgfqpoint{5.634932in}{2.100756in}}{\pgfqpoint{5.627032in}{2.097484in}}{\pgfqpoint{5.621208in}{2.091660in}}%
\pgfpathcurveto{\pgfqpoint{5.615385in}{2.085836in}}{\pgfqpoint{5.612112in}{2.077936in}}{\pgfqpoint{5.612112in}{2.069700in}}%
\pgfpathcurveto{\pgfqpoint{5.612112in}{2.061464in}}{\pgfqpoint{5.615385in}{2.053563in}}{\pgfqpoint{5.621208in}{2.047740in}}%
\pgfpathcurveto{\pgfqpoint{5.627032in}{2.041916in}}{\pgfqpoint{5.634932in}{2.038643in}}{\pgfqpoint{5.643169in}{2.038643in}}%
\pgfpathclose%
\pgfusepath{stroke,fill}%
\end{pgfscope}%
\begin{pgfscope}%
\pgfpathrectangle{\pgfqpoint{3.755891in}{0.557870in}}{\pgfqpoint{2.484109in}{1.684734in}}%
\pgfusepath{clip}%
\pgfsetbuttcap%
\pgfsetroundjoin%
\definecolor{currentfill}{rgb}{0.298039,0.447059,0.690196}%
\pgfsetfillcolor{currentfill}%
\pgfsetlinewidth{1.003750pt}%
\definecolor{currentstroke}{rgb}{0.298039,0.447059,0.690196}%
\pgfsetstrokecolor{currentstroke}%
\pgfsetdash{}{0pt}%
\pgfpathmoveto{\pgfqpoint{5.320557in}{2.086806in}}%
\pgfpathcurveto{\pgfqpoint{5.328793in}{2.086806in}}{\pgfqpoint{5.336694in}{2.090078in}}{\pgfqpoint{5.342517in}{2.095902in}}%
\pgfpathcurveto{\pgfqpoint{5.348341in}{2.101726in}}{\pgfqpoint{5.351614in}{2.109626in}}{\pgfqpoint{5.351614in}{2.117863in}}%
\pgfpathcurveto{\pgfqpoint{5.351614in}{2.126099in}}{\pgfqpoint{5.348341in}{2.133999in}}{\pgfqpoint{5.342517in}{2.139823in}}%
\pgfpathcurveto{\pgfqpoint{5.336694in}{2.145647in}}{\pgfqpoint{5.328793in}{2.148919in}}{\pgfqpoint{5.320557in}{2.148919in}}%
\pgfpathcurveto{\pgfqpoint{5.312321in}{2.148919in}}{\pgfqpoint{5.304421in}{2.145647in}}{\pgfqpoint{5.298597in}{2.139823in}}%
\pgfpathcurveto{\pgfqpoint{5.292773in}{2.133999in}}{\pgfqpoint{5.289501in}{2.126099in}}{\pgfqpoint{5.289501in}{2.117863in}}%
\pgfpathcurveto{\pgfqpoint{5.289501in}{2.109626in}}{\pgfqpoint{5.292773in}{2.101726in}}{\pgfqpoint{5.298597in}{2.095902in}}%
\pgfpathcurveto{\pgfqpoint{5.304421in}{2.090078in}}{\pgfqpoint{5.312321in}{2.086806in}}{\pgfqpoint{5.320557in}{2.086806in}}%
\pgfpathclose%
\pgfusepath{stroke,fill}%
\end{pgfscope}%
\begin{pgfscope}%
\pgfpathrectangle{\pgfqpoint{3.755891in}{0.557870in}}{\pgfqpoint{2.484109in}{1.684734in}}%
\pgfusepath{clip}%
\pgfsetbuttcap%
\pgfsetroundjoin%
\definecolor{currentfill}{rgb}{0.298039,0.447059,0.690196}%
\pgfsetfillcolor{currentfill}%
\pgfsetlinewidth{1.003750pt}%
\definecolor{currentstroke}{rgb}{0.298039,0.447059,0.690196}%
\pgfsetstrokecolor{currentstroke}%
\pgfsetdash{}{0pt}%
\pgfpathmoveto{\pgfqpoint{5.643169in}{2.048276in}}%
\pgfpathcurveto{\pgfqpoint{5.651405in}{2.048276in}}{\pgfqpoint{5.659305in}{2.051548in}}{\pgfqpoint{5.665129in}{2.057372in}}%
\pgfpathcurveto{\pgfqpoint{5.670953in}{2.063196in}}{\pgfqpoint{5.674225in}{2.071096in}}{\pgfqpoint{5.674225in}{2.079332in}}%
\pgfpathcurveto{\pgfqpoint{5.674225in}{2.087569in}}{\pgfqpoint{5.670953in}{2.095469in}}{\pgfqpoint{5.665129in}{2.101293in}}%
\pgfpathcurveto{\pgfqpoint{5.659305in}{2.107117in}}{\pgfqpoint{5.651405in}{2.110389in}}{\pgfqpoint{5.643169in}{2.110389in}}%
\pgfpathcurveto{\pgfqpoint{5.634932in}{2.110389in}}{\pgfqpoint{5.627032in}{2.107117in}}{\pgfqpoint{5.621208in}{2.101293in}}%
\pgfpathcurveto{\pgfqpoint{5.615385in}{2.095469in}}{\pgfqpoint{5.612112in}{2.087569in}}{\pgfqpoint{5.612112in}{2.079332in}}%
\pgfpathcurveto{\pgfqpoint{5.612112in}{2.071096in}}{\pgfqpoint{5.615385in}{2.063196in}}{\pgfqpoint{5.621208in}{2.057372in}}%
\pgfpathcurveto{\pgfqpoint{5.627032in}{2.051548in}}{\pgfqpoint{5.634932in}{2.048276in}}{\pgfqpoint{5.643169in}{2.048276in}}%
\pgfpathclose%
\pgfusepath{stroke,fill}%
\end{pgfscope}%
\begin{pgfscope}%
\pgfpathrectangle{\pgfqpoint{3.755891in}{0.557870in}}{\pgfqpoint{2.484109in}{1.684734in}}%
\pgfusepath{clip}%
\pgfsetbuttcap%
\pgfsetroundjoin%
\definecolor{currentfill}{rgb}{0.298039,0.447059,0.690196}%
\pgfsetfillcolor{currentfill}%
\pgfsetlinewidth{1.003750pt}%
\definecolor{currentstroke}{rgb}{0.298039,0.447059,0.690196}%
\pgfsetstrokecolor{currentstroke}%
\pgfsetdash{}{0pt}%
\pgfpathmoveto{\pgfqpoint{5.078599in}{2.057908in}}%
\pgfpathcurveto{\pgfqpoint{5.086835in}{2.057908in}}{\pgfqpoint{5.094735in}{2.061181in}}{\pgfqpoint{5.100559in}{2.067005in}}%
\pgfpathcurveto{\pgfqpoint{5.106383in}{2.072829in}}{\pgfqpoint{5.109655in}{2.080729in}}{\pgfqpoint{5.109655in}{2.088965in}}%
\pgfpathcurveto{\pgfqpoint{5.109655in}{2.097201in}}{\pgfqpoint{5.106383in}{2.105101in}}{\pgfqpoint{5.100559in}{2.110925in}}%
\pgfpathcurveto{\pgfqpoint{5.094735in}{2.116749in}}{\pgfqpoint{5.086835in}{2.120021in}}{\pgfqpoint{5.078599in}{2.120021in}}%
\pgfpathcurveto{\pgfqpoint{5.070362in}{2.120021in}}{\pgfqpoint{5.062462in}{2.116749in}}{\pgfqpoint{5.056638in}{2.110925in}}%
\pgfpathcurveto{\pgfqpoint{5.050814in}{2.105101in}}{\pgfqpoint{5.047542in}{2.097201in}}{\pgfqpoint{5.047542in}{2.088965in}}%
\pgfpathcurveto{\pgfqpoint{5.047542in}{2.080729in}}{\pgfqpoint{5.050814in}{2.072829in}}{\pgfqpoint{5.056638in}{2.067005in}}%
\pgfpathcurveto{\pgfqpoint{5.062462in}{2.061181in}}{\pgfqpoint{5.070362in}{2.057908in}}{\pgfqpoint{5.078599in}{2.057908in}}%
\pgfpathclose%
\pgfusepath{stroke,fill}%
\end{pgfscope}%
\begin{pgfscope}%
\pgfpathrectangle{\pgfqpoint{3.755891in}{0.557870in}}{\pgfqpoint{2.484109in}{1.684734in}}%
\pgfusepath{clip}%
\pgfsetbuttcap%
\pgfsetroundjoin%
\definecolor{currentfill}{rgb}{0.298039,0.447059,0.690196}%
\pgfsetfillcolor{currentfill}%
\pgfsetlinewidth{1.003750pt}%
\definecolor{currentstroke}{rgb}{0.298039,0.447059,0.690196}%
\pgfsetstrokecolor{currentstroke}%
\pgfsetdash{}{0pt}%
\pgfpathmoveto{\pgfqpoint{5.401210in}{2.077174in}}%
\pgfpathcurveto{\pgfqpoint{5.409446in}{2.077174in}}{\pgfqpoint{5.417346in}{2.080446in}}{\pgfqpoint{5.423170in}{2.086270in}}%
\pgfpathcurveto{\pgfqpoint{5.428994in}{2.092094in}}{\pgfqpoint{5.432267in}{2.099994in}}{\pgfqpoint{5.432267in}{2.108230in}}%
\pgfpathcurveto{\pgfqpoint{5.432267in}{2.116466in}}{\pgfqpoint{5.428994in}{2.124366in}}{\pgfqpoint{5.423170in}{2.130190in}}%
\pgfpathcurveto{\pgfqpoint{5.417346in}{2.136014in}}{\pgfqpoint{5.409446in}{2.139287in}}{\pgfqpoint{5.401210in}{2.139287in}}%
\pgfpathcurveto{\pgfqpoint{5.392974in}{2.139287in}}{\pgfqpoint{5.385074in}{2.136014in}}{\pgfqpoint{5.379250in}{2.130190in}}%
\pgfpathcurveto{\pgfqpoint{5.373426in}{2.124366in}}{\pgfqpoint{5.370154in}{2.116466in}}{\pgfqpoint{5.370154in}{2.108230in}}%
\pgfpathcurveto{\pgfqpoint{5.370154in}{2.099994in}}{\pgfqpoint{5.373426in}{2.092094in}}{\pgfqpoint{5.379250in}{2.086270in}}%
\pgfpathcurveto{\pgfqpoint{5.385074in}{2.080446in}}{\pgfqpoint{5.392974in}{2.077174in}}{\pgfqpoint{5.401210in}{2.077174in}}%
\pgfpathclose%
\pgfusepath{stroke,fill}%
\end{pgfscope}%
\begin{pgfscope}%
\pgfpathrectangle{\pgfqpoint{3.755891in}{0.557870in}}{\pgfqpoint{2.484109in}{1.684734in}}%
\pgfusepath{clip}%
\pgfsetbuttcap%
\pgfsetroundjoin%
\definecolor{currentfill}{rgb}{0.298039,0.447059,0.690196}%
\pgfsetfillcolor{currentfill}%
\pgfsetlinewidth{1.003750pt}%
\definecolor{currentstroke}{rgb}{0.298039,0.447059,0.690196}%
\pgfsetstrokecolor{currentstroke}%
\pgfsetdash{}{0pt}%
\pgfpathmoveto{\pgfqpoint{5.320557in}{2.096439in}}%
\pgfpathcurveto{\pgfqpoint{5.328793in}{2.096439in}}{\pgfqpoint{5.336694in}{2.099711in}}{\pgfqpoint{5.342517in}{2.105535in}}%
\pgfpathcurveto{\pgfqpoint{5.348341in}{2.111359in}}{\pgfqpoint{5.351614in}{2.119259in}}{\pgfqpoint{5.351614in}{2.127495in}}%
\pgfpathcurveto{\pgfqpoint{5.351614in}{2.135731in}}{\pgfqpoint{5.348341in}{2.143631in}}{\pgfqpoint{5.342517in}{2.149455in}}%
\pgfpathcurveto{\pgfqpoint{5.336694in}{2.155279in}}{\pgfqpoint{5.328793in}{2.158552in}}{\pgfqpoint{5.320557in}{2.158552in}}%
\pgfpathcurveto{\pgfqpoint{5.312321in}{2.158552in}}{\pgfqpoint{5.304421in}{2.155279in}}{\pgfqpoint{5.298597in}{2.149455in}}%
\pgfpathcurveto{\pgfqpoint{5.292773in}{2.143631in}}{\pgfqpoint{5.289501in}{2.135731in}}{\pgfqpoint{5.289501in}{2.127495in}}%
\pgfpathcurveto{\pgfqpoint{5.289501in}{2.119259in}}{\pgfqpoint{5.292773in}{2.111359in}}{\pgfqpoint{5.298597in}{2.105535in}}%
\pgfpathcurveto{\pgfqpoint{5.304421in}{2.099711in}}{\pgfqpoint{5.312321in}{2.096439in}}{\pgfqpoint{5.320557in}{2.096439in}}%
\pgfpathclose%
\pgfusepath{stroke,fill}%
\end{pgfscope}%
\begin{pgfscope}%
\pgfpathrectangle{\pgfqpoint{3.755891in}{0.557870in}}{\pgfqpoint{2.484109in}{1.684734in}}%
\pgfusepath{clip}%
\pgfsetbuttcap%
\pgfsetroundjoin%
\definecolor{currentfill}{rgb}{0.298039,0.447059,0.690196}%
\pgfsetfillcolor{currentfill}%
\pgfsetlinewidth{1.003750pt}%
\definecolor{currentstroke}{rgb}{0.298039,0.447059,0.690196}%
\pgfsetstrokecolor{currentstroke}%
\pgfsetdash{}{0pt}%
\pgfpathmoveto{\pgfqpoint{5.562516in}{2.077174in}}%
\pgfpathcurveto{\pgfqpoint{5.570752in}{2.077174in}}{\pgfqpoint{5.578652in}{2.080446in}}{\pgfqpoint{5.584476in}{2.086270in}}%
\pgfpathcurveto{\pgfqpoint{5.590300in}{2.092094in}}{\pgfqpoint{5.593572in}{2.099994in}}{\pgfqpoint{5.593572in}{2.108230in}}%
\pgfpathcurveto{\pgfqpoint{5.593572in}{2.116466in}}{\pgfqpoint{5.590300in}{2.124366in}}{\pgfqpoint{5.584476in}{2.130190in}}%
\pgfpathcurveto{\pgfqpoint{5.578652in}{2.136014in}}{\pgfqpoint{5.570752in}{2.139287in}}{\pgfqpoint{5.562516in}{2.139287in}}%
\pgfpathcurveto{\pgfqpoint{5.554280in}{2.139287in}}{\pgfqpoint{5.546379in}{2.136014in}}{\pgfqpoint{5.540556in}{2.130190in}}%
\pgfpathcurveto{\pgfqpoint{5.534732in}{2.124366in}}{\pgfqpoint{5.531459in}{2.116466in}}{\pgfqpoint{5.531459in}{2.108230in}}%
\pgfpathcurveto{\pgfqpoint{5.531459in}{2.099994in}}{\pgfqpoint{5.534732in}{2.092094in}}{\pgfqpoint{5.540556in}{2.086270in}}%
\pgfpathcurveto{\pgfqpoint{5.546379in}{2.080446in}}{\pgfqpoint{5.554280in}{2.077174in}}{\pgfqpoint{5.562516in}{2.077174in}}%
\pgfpathclose%
\pgfusepath{stroke,fill}%
\end{pgfscope}%
\begin{pgfscope}%
\pgfpathrectangle{\pgfqpoint{3.755891in}{0.557870in}}{\pgfqpoint{2.484109in}{1.684734in}}%
\pgfusepath{clip}%
\pgfsetbuttcap%
\pgfsetroundjoin%
\definecolor{currentfill}{rgb}{0.298039,0.447059,0.690196}%
\pgfsetfillcolor{currentfill}%
\pgfsetlinewidth{1.003750pt}%
\definecolor{currentstroke}{rgb}{0.298039,0.447059,0.690196}%
\pgfsetstrokecolor{currentstroke}%
\pgfsetdash{}{0pt}%
\pgfpathmoveto{\pgfqpoint{5.885127in}{1.759299in}}%
\pgfpathcurveto{\pgfqpoint{5.893364in}{1.759299in}}{\pgfqpoint{5.901264in}{1.762572in}}{\pgfqpoint{5.907088in}{1.768395in}}%
\pgfpathcurveto{\pgfqpoint{5.912912in}{1.774219in}}{\pgfqpoint{5.916184in}{1.782119in}}{\pgfqpoint{5.916184in}{1.790356in}}%
\pgfpathcurveto{\pgfqpoint{5.916184in}{1.798592in}}{\pgfqpoint{5.912912in}{1.806492in}}{\pgfqpoint{5.907088in}{1.812316in}}%
\pgfpathcurveto{\pgfqpoint{5.901264in}{1.818140in}}{\pgfqpoint{5.893364in}{1.821412in}}{\pgfqpoint{5.885127in}{1.821412in}}%
\pgfpathcurveto{\pgfqpoint{5.876891in}{1.821412in}}{\pgfqpoint{5.868991in}{1.818140in}}{\pgfqpoint{5.863167in}{1.812316in}}%
\pgfpathcurveto{\pgfqpoint{5.857343in}{1.806492in}}{\pgfqpoint{5.854071in}{1.798592in}}{\pgfqpoint{5.854071in}{1.790356in}}%
\pgfpathcurveto{\pgfqpoint{5.854071in}{1.782119in}}{\pgfqpoint{5.857343in}{1.774219in}}{\pgfqpoint{5.863167in}{1.768395in}}%
\pgfpathcurveto{\pgfqpoint{5.868991in}{1.762572in}}{\pgfqpoint{5.876891in}{1.759299in}}{\pgfqpoint{5.885127in}{1.759299in}}%
\pgfpathclose%
\pgfusepath{stroke,fill}%
\end{pgfscope}%
\begin{pgfscope}%
\pgfpathrectangle{\pgfqpoint{3.755891in}{0.557870in}}{\pgfqpoint{2.484109in}{1.684734in}}%
\pgfusepath{clip}%
\pgfsetbuttcap%
\pgfsetroundjoin%
\definecolor{currentfill}{rgb}{0.298039,0.447059,0.690196}%
\pgfsetfillcolor{currentfill}%
\pgfsetlinewidth{1.003750pt}%
\definecolor{currentstroke}{rgb}{0.298039,0.447059,0.690196}%
\pgfsetstrokecolor{currentstroke}%
\pgfsetdash{}{0pt}%
\pgfpathmoveto{\pgfqpoint{4.110764in}{2.115704in}}%
\pgfpathcurveto{\pgfqpoint{4.119000in}{2.115704in}}{\pgfqpoint{4.126900in}{2.118976in}}{\pgfqpoint{4.132724in}{2.124800in}}%
\pgfpathcurveto{\pgfqpoint{4.138548in}{2.130624in}}{\pgfqpoint{4.141820in}{2.138524in}}{\pgfqpoint{4.141820in}{2.146760in}}%
\pgfpathcurveto{\pgfqpoint{4.141820in}{2.154997in}}{\pgfqpoint{4.138548in}{2.162897in}}{\pgfqpoint{4.132724in}{2.168721in}}%
\pgfpathcurveto{\pgfqpoint{4.126900in}{2.174544in}}{\pgfqpoint{4.119000in}{2.177817in}}{\pgfqpoint{4.110764in}{2.177817in}}%
\pgfpathcurveto{\pgfqpoint{4.102528in}{2.177817in}}{\pgfqpoint{4.094628in}{2.174544in}}{\pgfqpoint{4.088804in}{2.168721in}}%
\pgfpathcurveto{\pgfqpoint{4.082980in}{2.162897in}}{\pgfqpoint{4.079707in}{2.154997in}}{\pgfqpoint{4.079707in}{2.146760in}}%
\pgfpathcurveto{\pgfqpoint{4.079707in}{2.138524in}}{\pgfqpoint{4.082980in}{2.130624in}}{\pgfqpoint{4.088804in}{2.124800in}}%
\pgfpathcurveto{\pgfqpoint{4.094628in}{2.118976in}}{\pgfqpoint{4.102528in}{2.115704in}}{\pgfqpoint{4.110764in}{2.115704in}}%
\pgfpathclose%
\pgfusepath{stroke,fill}%
\end{pgfscope}%
\begin{pgfscope}%
\pgfpathrectangle{\pgfqpoint{3.755891in}{0.557870in}}{\pgfqpoint{2.484109in}{1.684734in}}%
\pgfusepath{clip}%
\pgfsetbuttcap%
\pgfsetroundjoin%
\definecolor{currentfill}{rgb}{0.298039,0.447059,0.690196}%
\pgfsetfillcolor{currentfill}%
\pgfsetlinewidth{1.003750pt}%
\definecolor{currentstroke}{rgb}{0.298039,0.447059,0.690196}%
\pgfsetstrokecolor{currentstroke}%
\pgfsetdash{}{0pt}%
\pgfpathmoveto{\pgfqpoint{4.675334in}{2.046039in}}%
\pgfpathcurveto{\pgfqpoint{4.683570in}{2.046039in}}{\pgfqpoint{4.691470in}{2.049311in}}{\pgfqpoint{4.697294in}{2.055135in}}%
\pgfpathcurveto{\pgfqpoint{4.703118in}{2.060959in}}{\pgfqpoint{4.706391in}{2.068859in}}{\pgfqpoint{4.706391in}{2.077095in}}%
\pgfpathcurveto{\pgfqpoint{4.706391in}{2.085331in}}{\pgfqpoint{4.703118in}{2.093231in}}{\pgfqpoint{4.697294in}{2.099055in}}%
\pgfpathcurveto{\pgfqpoint{4.691470in}{2.104879in}}{\pgfqpoint{4.683570in}{2.108152in}}{\pgfqpoint{4.675334in}{2.108152in}}%
\pgfpathcurveto{\pgfqpoint{4.667098in}{2.108152in}}{\pgfqpoint{4.659198in}{2.104879in}}{\pgfqpoint{4.653374in}{2.099055in}}%
\pgfpathcurveto{\pgfqpoint{4.647550in}{2.093231in}}{\pgfqpoint{4.644278in}{2.085331in}}{\pgfqpoint{4.644278in}{2.077095in}}%
\pgfpathcurveto{\pgfqpoint{4.644278in}{2.068859in}}{\pgfqpoint{4.647550in}{2.060959in}}{\pgfqpoint{4.653374in}{2.055135in}}%
\pgfpathcurveto{\pgfqpoint{4.659198in}{2.049311in}}{\pgfqpoint{4.667098in}{2.046039in}}{\pgfqpoint{4.675334in}{2.046039in}}%
\pgfpathclose%
\pgfusepath{stroke,fill}%
\end{pgfscope}%
\begin{pgfscope}%
\pgfpathrectangle{\pgfqpoint{3.755891in}{0.557870in}}{\pgfqpoint{2.484109in}{1.684734in}}%
\pgfusepath{clip}%
\pgfsetbuttcap%
\pgfsetroundjoin%
\definecolor{currentfill}{rgb}{0.298039,0.447059,0.690196}%
\pgfsetfillcolor{currentfill}%
\pgfsetlinewidth{1.003750pt}%
\definecolor{currentstroke}{rgb}{0.298039,0.447059,0.690196}%
\pgfsetstrokecolor{currentstroke}%
\pgfsetdash{}{0pt}%
\pgfpathmoveto{\pgfqpoint{5.723822in}{2.046039in}}%
\pgfpathcurveto{\pgfqpoint{5.732058in}{2.046039in}}{\pgfqpoint{5.739958in}{2.049311in}}{\pgfqpoint{5.745782in}{2.055135in}}%
\pgfpathcurveto{\pgfqpoint{5.751606in}{2.060959in}}{\pgfqpoint{5.754878in}{2.068859in}}{\pgfqpoint{5.754878in}{2.077095in}}%
\pgfpathcurveto{\pgfqpoint{5.754878in}{2.085331in}}{\pgfqpoint{5.751606in}{2.093231in}}{\pgfqpoint{5.745782in}{2.099055in}}%
\pgfpathcurveto{\pgfqpoint{5.739958in}{2.104879in}}{\pgfqpoint{5.732058in}{2.108152in}}{\pgfqpoint{5.723822in}{2.108152in}}%
\pgfpathcurveto{\pgfqpoint{5.715585in}{2.108152in}}{\pgfqpoint{5.707685in}{2.104879in}}{\pgfqpoint{5.701861in}{2.099055in}}%
\pgfpathcurveto{\pgfqpoint{5.696037in}{2.093231in}}{\pgfqpoint{5.692765in}{2.085331in}}{\pgfqpoint{5.692765in}{2.077095in}}%
\pgfpathcurveto{\pgfqpoint{5.692765in}{2.068859in}}{\pgfqpoint{5.696037in}{2.060959in}}{\pgfqpoint{5.701861in}{2.055135in}}%
\pgfpathcurveto{\pgfqpoint{5.707685in}{2.049311in}}{\pgfqpoint{5.715585in}{2.046039in}}{\pgfqpoint{5.723822in}{2.046039in}}%
\pgfpathclose%
\pgfusepath{stroke,fill}%
\end{pgfscope}%
\begin{pgfscope}%
\pgfpathrectangle{\pgfqpoint{3.755891in}{0.557870in}}{\pgfqpoint{2.484109in}{1.684734in}}%
\pgfusepath{clip}%
\pgfsetbuttcap%
\pgfsetroundjoin%
\definecolor{currentfill}{rgb}{0.298039,0.447059,0.690196}%
\pgfsetfillcolor{currentfill}%
\pgfsetlinewidth{1.003750pt}%
\definecolor{currentstroke}{rgb}{0.298039,0.447059,0.690196}%
\pgfsetstrokecolor{currentstroke}%
\pgfsetdash{}{0pt}%
\pgfpathmoveto{\pgfqpoint{5.401210in}{2.085563in}}%
\pgfpathcurveto{\pgfqpoint{5.409446in}{2.085563in}}{\pgfqpoint{5.417346in}{2.088835in}}{\pgfqpoint{5.423170in}{2.094659in}}%
\pgfpathcurveto{\pgfqpoint{5.428994in}{2.100483in}}{\pgfqpoint{5.432267in}{2.108383in}}{\pgfqpoint{5.432267in}{2.116620in}}%
\pgfpathcurveto{\pgfqpoint{5.432267in}{2.124856in}}{\pgfqpoint{5.428994in}{2.132756in}}{\pgfqpoint{5.423170in}{2.138580in}}%
\pgfpathcurveto{\pgfqpoint{5.417346in}{2.144404in}}{\pgfqpoint{5.409446in}{2.147676in}}{\pgfqpoint{5.401210in}{2.147676in}}%
\pgfpathcurveto{\pgfqpoint{5.392974in}{2.147676in}}{\pgfqpoint{5.385074in}{2.144404in}}{\pgfqpoint{5.379250in}{2.138580in}}%
\pgfpathcurveto{\pgfqpoint{5.373426in}{2.132756in}}{\pgfqpoint{5.370154in}{2.124856in}}{\pgfqpoint{5.370154in}{2.116620in}}%
\pgfpathcurveto{\pgfqpoint{5.370154in}{2.108383in}}{\pgfqpoint{5.373426in}{2.100483in}}{\pgfqpoint{5.379250in}{2.094659in}}%
\pgfpathcurveto{\pgfqpoint{5.385074in}{2.088835in}}{\pgfqpoint{5.392974in}{2.085563in}}{\pgfqpoint{5.401210in}{2.085563in}}%
\pgfpathclose%
\pgfusepath{stroke,fill}%
\end{pgfscope}%
\begin{pgfscope}%
\pgfpathrectangle{\pgfqpoint{3.755891in}{0.557870in}}{\pgfqpoint{2.484109in}{1.684734in}}%
\pgfusepath{clip}%
\pgfsetbuttcap%
\pgfsetroundjoin%
\definecolor{currentfill}{rgb}{0.298039,0.447059,0.690196}%
\pgfsetfillcolor{currentfill}%
\pgfsetlinewidth{1.003750pt}%
\definecolor{currentstroke}{rgb}{0.298039,0.447059,0.690196}%
\pgfsetstrokecolor{currentstroke}%
\pgfsetdash{}{0pt}%
\pgfpathmoveto{\pgfqpoint{5.643169in}{2.026276in}}%
\pgfpathcurveto{\pgfqpoint{5.651405in}{2.026276in}}{\pgfqpoint{5.659305in}{2.029549in}}{\pgfqpoint{5.665129in}{2.035373in}}%
\pgfpathcurveto{\pgfqpoint{5.670953in}{2.041197in}}{\pgfqpoint{5.674225in}{2.049097in}}{\pgfqpoint{5.674225in}{2.057333in}}%
\pgfpathcurveto{\pgfqpoint{5.674225in}{2.065569in}}{\pgfqpoint{5.670953in}{2.073469in}}{\pgfqpoint{5.665129in}{2.079293in}}%
\pgfpathcurveto{\pgfqpoint{5.659305in}{2.085117in}}{\pgfqpoint{5.651405in}{2.088389in}}{\pgfqpoint{5.643169in}{2.088389in}}%
\pgfpathcurveto{\pgfqpoint{5.634932in}{2.088389in}}{\pgfqpoint{5.627032in}{2.085117in}}{\pgfqpoint{5.621208in}{2.079293in}}%
\pgfpathcurveto{\pgfqpoint{5.615385in}{2.073469in}}{\pgfqpoint{5.612112in}{2.065569in}}{\pgfqpoint{5.612112in}{2.057333in}}%
\pgfpathcurveto{\pgfqpoint{5.612112in}{2.049097in}}{\pgfqpoint{5.615385in}{2.041197in}}{\pgfqpoint{5.621208in}{2.035373in}}%
\pgfpathcurveto{\pgfqpoint{5.627032in}{2.029549in}}{\pgfqpoint{5.634932in}{2.026276in}}{\pgfqpoint{5.643169in}{2.026276in}}%
\pgfpathclose%
\pgfusepath{stroke,fill}%
\end{pgfscope}%
\begin{pgfscope}%
\pgfpathrectangle{\pgfqpoint{3.755891in}{0.557870in}}{\pgfqpoint{2.484109in}{1.684734in}}%
\pgfusepath{clip}%
\pgfsetbuttcap%
\pgfsetroundjoin%
\definecolor{currentfill}{rgb}{0.298039,0.447059,0.690196}%
\pgfsetfillcolor{currentfill}%
\pgfsetlinewidth{1.003750pt}%
\definecolor{currentstroke}{rgb}{0.298039,0.447059,0.690196}%
\pgfsetstrokecolor{currentstroke}%
\pgfsetdash{}{0pt}%
\pgfpathmoveto{\pgfqpoint{5.481863in}{2.016395in}}%
\pgfpathcurveto{\pgfqpoint{5.490099in}{2.016395in}}{\pgfqpoint{5.497999in}{2.019668in}}{\pgfqpoint{5.503823in}{2.025491in}}%
\pgfpathcurveto{\pgfqpoint{5.509647in}{2.031315in}}{\pgfqpoint{5.512919in}{2.039215in}}{\pgfqpoint{5.512919in}{2.047452in}}%
\pgfpathcurveto{\pgfqpoint{5.512919in}{2.055688in}}{\pgfqpoint{5.509647in}{2.063588in}}{\pgfqpoint{5.503823in}{2.069412in}}%
\pgfpathcurveto{\pgfqpoint{5.497999in}{2.075236in}}{\pgfqpoint{5.490099in}{2.078508in}}{\pgfqpoint{5.481863in}{2.078508in}}%
\pgfpathcurveto{\pgfqpoint{5.473627in}{2.078508in}}{\pgfqpoint{5.465727in}{2.075236in}}{\pgfqpoint{5.459903in}{2.069412in}}%
\pgfpathcurveto{\pgfqpoint{5.454079in}{2.063588in}}{\pgfqpoint{5.450806in}{2.055688in}}{\pgfqpoint{5.450806in}{2.047452in}}%
\pgfpathcurveto{\pgfqpoint{5.450806in}{2.039215in}}{\pgfqpoint{5.454079in}{2.031315in}}{\pgfqpoint{5.459903in}{2.025491in}}%
\pgfpathcurveto{\pgfqpoint{5.465727in}{2.019668in}}{\pgfqpoint{5.473627in}{2.016395in}}{\pgfqpoint{5.481863in}{2.016395in}}%
\pgfpathclose%
\pgfusepath{stroke,fill}%
\end{pgfscope}%
\begin{pgfscope}%
\pgfpathrectangle{\pgfqpoint{3.755891in}{0.557870in}}{\pgfqpoint{2.484109in}{1.684734in}}%
\pgfusepath{clip}%
\pgfsetbuttcap%
\pgfsetroundjoin%
\definecolor{currentfill}{rgb}{0.298039,0.447059,0.690196}%
\pgfsetfillcolor{currentfill}%
\pgfsetlinewidth{1.003750pt}%
\definecolor{currentstroke}{rgb}{0.298039,0.447059,0.690196}%
\pgfsetstrokecolor{currentstroke}%
\pgfsetdash{}{0pt}%
\pgfpathmoveto{\pgfqpoint{5.643169in}{2.095444in}}%
\pgfpathcurveto{\pgfqpoint{5.651405in}{2.095444in}}{\pgfqpoint{5.659305in}{2.098717in}}{\pgfqpoint{5.665129in}{2.104541in}}%
\pgfpathcurveto{\pgfqpoint{5.670953in}{2.110364in}}{\pgfqpoint{5.674225in}{2.118265in}}{\pgfqpoint{5.674225in}{2.126501in}}%
\pgfpathcurveto{\pgfqpoint{5.674225in}{2.134737in}}{\pgfqpoint{5.670953in}{2.142637in}}{\pgfqpoint{5.665129in}{2.148461in}}%
\pgfpathcurveto{\pgfqpoint{5.659305in}{2.154285in}}{\pgfqpoint{5.651405in}{2.157557in}}{\pgfqpoint{5.643169in}{2.157557in}}%
\pgfpathcurveto{\pgfqpoint{5.634932in}{2.157557in}}{\pgfqpoint{5.627032in}{2.154285in}}{\pgfqpoint{5.621208in}{2.148461in}}%
\pgfpathcurveto{\pgfqpoint{5.615385in}{2.142637in}}{\pgfqpoint{5.612112in}{2.134737in}}{\pgfqpoint{5.612112in}{2.126501in}}%
\pgfpathcurveto{\pgfqpoint{5.612112in}{2.118265in}}{\pgfqpoint{5.615385in}{2.110364in}}{\pgfqpoint{5.621208in}{2.104541in}}%
\pgfpathcurveto{\pgfqpoint{5.627032in}{2.098717in}}{\pgfqpoint{5.634932in}{2.095444in}}{\pgfqpoint{5.643169in}{2.095444in}}%
\pgfpathclose%
\pgfusepath{stroke,fill}%
\end{pgfscope}%
\begin{pgfscope}%
\pgfpathrectangle{\pgfqpoint{3.755891in}{0.557870in}}{\pgfqpoint{2.484109in}{1.684734in}}%
\pgfusepath{clip}%
\pgfsetbuttcap%
\pgfsetroundjoin%
\definecolor{currentfill}{rgb}{0.298039,0.447059,0.690196}%
\pgfsetfillcolor{currentfill}%
\pgfsetlinewidth{1.003750pt}%
\definecolor{currentstroke}{rgb}{0.298039,0.447059,0.690196}%
\pgfsetstrokecolor{currentstroke}%
\pgfsetdash{}{0pt}%
\pgfpathmoveto{\pgfqpoint{5.078599in}{2.055920in}}%
\pgfpathcurveto{\pgfqpoint{5.086835in}{2.055920in}}{\pgfqpoint{5.094735in}{2.059192in}}{\pgfqpoint{5.100559in}{2.065016in}}%
\pgfpathcurveto{\pgfqpoint{5.106383in}{2.070840in}}{\pgfqpoint{5.109655in}{2.078740in}}{\pgfqpoint{5.109655in}{2.086976in}}%
\pgfpathcurveto{\pgfqpoint{5.109655in}{2.095213in}}{\pgfqpoint{5.106383in}{2.103113in}}{\pgfqpoint{5.100559in}{2.108937in}}%
\pgfpathcurveto{\pgfqpoint{5.094735in}{2.114760in}}{\pgfqpoint{5.086835in}{2.118033in}}{\pgfqpoint{5.078599in}{2.118033in}}%
\pgfpathcurveto{\pgfqpoint{5.070362in}{2.118033in}}{\pgfqpoint{5.062462in}{2.114760in}}{\pgfqpoint{5.056638in}{2.108937in}}%
\pgfpathcurveto{\pgfqpoint{5.050814in}{2.103113in}}{\pgfqpoint{5.047542in}{2.095213in}}{\pgfqpoint{5.047542in}{2.086976in}}%
\pgfpathcurveto{\pgfqpoint{5.047542in}{2.078740in}}{\pgfqpoint{5.050814in}{2.070840in}}{\pgfqpoint{5.056638in}{2.065016in}}%
\pgfpathcurveto{\pgfqpoint{5.062462in}{2.059192in}}{\pgfqpoint{5.070362in}{2.055920in}}{\pgfqpoint{5.078599in}{2.055920in}}%
\pgfpathclose%
\pgfusepath{stroke,fill}%
\end{pgfscope}%
\begin{pgfscope}%
\pgfpathrectangle{\pgfqpoint{3.755891in}{0.557870in}}{\pgfqpoint{2.484109in}{1.684734in}}%
\pgfusepath{clip}%
\pgfsetbuttcap%
\pgfsetroundjoin%
\definecolor{currentfill}{rgb}{0.298039,0.447059,0.690196}%
\pgfsetfillcolor{currentfill}%
\pgfsetlinewidth{1.003750pt}%
\definecolor{currentstroke}{rgb}{0.298039,0.447059,0.690196}%
\pgfsetstrokecolor{currentstroke}%
\pgfsetdash{}{0pt}%
\pgfpathmoveto{\pgfqpoint{5.481863in}{2.085563in}}%
\pgfpathcurveto{\pgfqpoint{5.490099in}{2.085563in}}{\pgfqpoint{5.497999in}{2.088835in}}{\pgfqpoint{5.503823in}{2.094659in}}%
\pgfpathcurveto{\pgfqpoint{5.509647in}{2.100483in}}{\pgfqpoint{5.512919in}{2.108383in}}{\pgfqpoint{5.512919in}{2.116620in}}%
\pgfpathcurveto{\pgfqpoint{5.512919in}{2.124856in}}{\pgfqpoint{5.509647in}{2.132756in}}{\pgfqpoint{5.503823in}{2.138580in}}%
\pgfpathcurveto{\pgfqpoint{5.497999in}{2.144404in}}{\pgfqpoint{5.490099in}{2.147676in}}{\pgfqpoint{5.481863in}{2.147676in}}%
\pgfpathcurveto{\pgfqpoint{5.473627in}{2.147676in}}{\pgfqpoint{5.465727in}{2.144404in}}{\pgfqpoint{5.459903in}{2.138580in}}%
\pgfpathcurveto{\pgfqpoint{5.454079in}{2.132756in}}{\pgfqpoint{5.450806in}{2.124856in}}{\pgfqpoint{5.450806in}{2.116620in}}%
\pgfpathcurveto{\pgfqpoint{5.450806in}{2.108383in}}{\pgfqpoint{5.454079in}{2.100483in}}{\pgfqpoint{5.459903in}{2.094659in}}%
\pgfpathcurveto{\pgfqpoint{5.465727in}{2.088835in}}{\pgfqpoint{5.473627in}{2.085563in}}{\pgfqpoint{5.481863in}{2.085563in}}%
\pgfpathclose%
\pgfusepath{stroke,fill}%
\end{pgfscope}%
\begin{pgfscope}%
\pgfpathrectangle{\pgfqpoint{3.755891in}{0.557870in}}{\pgfqpoint{2.484109in}{1.684734in}}%
\pgfusepath{clip}%
\pgfsetbuttcap%
\pgfsetroundjoin%
\definecolor{currentfill}{rgb}{0.298039,0.447059,0.690196}%
\pgfsetfillcolor{currentfill}%
\pgfsetlinewidth{1.003750pt}%
\definecolor{currentstroke}{rgb}{0.298039,0.447059,0.690196}%
\pgfsetstrokecolor{currentstroke}%
\pgfsetdash{}{0pt}%
\pgfpathmoveto{\pgfqpoint{5.885127in}{1.749605in}}%
\pgfpathcurveto{\pgfqpoint{5.893364in}{1.749605in}}{\pgfqpoint{5.901264in}{1.752877in}}{\pgfqpoint{5.907088in}{1.758701in}}%
\pgfpathcurveto{\pgfqpoint{5.912912in}{1.764525in}}{\pgfqpoint{5.916184in}{1.772425in}}{\pgfqpoint{5.916184in}{1.780661in}}%
\pgfpathcurveto{\pgfqpoint{5.916184in}{1.788897in}}{\pgfqpoint{5.912912in}{1.796797in}}{\pgfqpoint{5.907088in}{1.802621in}}%
\pgfpathcurveto{\pgfqpoint{5.901264in}{1.808445in}}{\pgfqpoint{5.893364in}{1.811718in}}{\pgfqpoint{5.885127in}{1.811718in}}%
\pgfpathcurveto{\pgfqpoint{5.876891in}{1.811718in}}{\pgfqpoint{5.868991in}{1.808445in}}{\pgfqpoint{5.863167in}{1.802621in}}%
\pgfpathcurveto{\pgfqpoint{5.857343in}{1.796797in}}{\pgfqpoint{5.854071in}{1.788897in}}{\pgfqpoint{5.854071in}{1.780661in}}%
\pgfpathcurveto{\pgfqpoint{5.854071in}{1.772425in}}{\pgfqpoint{5.857343in}{1.764525in}}{\pgfqpoint{5.863167in}{1.758701in}}%
\pgfpathcurveto{\pgfqpoint{5.868991in}{1.752877in}}{\pgfqpoint{5.876891in}{1.749605in}}{\pgfqpoint{5.885127in}{1.749605in}}%
\pgfpathclose%
\pgfusepath{stroke,fill}%
\end{pgfscope}%
\begin{pgfscope}%
\pgfsetrectcap%
\pgfsetmiterjoin%
\pgfsetlinewidth{1.254687pt}%
\definecolor{currentstroke}{rgb}{1.000000,1.000000,1.000000}%
\pgfsetstrokecolor{currentstroke}%
\pgfsetdash{}{0pt}%
\pgfpathmoveto{\pgfqpoint{3.755891in}{0.557870in}}%
\pgfpathlineto{\pgfqpoint{3.755891in}{2.242604in}}%
\pgfusepath{stroke}%
\end{pgfscope}%
\begin{pgfscope}%
\pgfsetrectcap%
\pgfsetmiterjoin%
\pgfsetlinewidth{1.254687pt}%
\definecolor{currentstroke}{rgb}{1.000000,1.000000,1.000000}%
\pgfsetstrokecolor{currentstroke}%
\pgfsetdash{}{0pt}%
\pgfpathmoveto{\pgfqpoint{6.240000in}{0.557870in}}%
\pgfpathlineto{\pgfqpoint{6.240000in}{2.242604in}}%
\pgfusepath{stroke}%
\end{pgfscope}%
\begin{pgfscope}%
\pgfsetrectcap%
\pgfsetmiterjoin%
\pgfsetlinewidth{1.254687pt}%
\definecolor{currentstroke}{rgb}{1.000000,1.000000,1.000000}%
\pgfsetstrokecolor{currentstroke}%
\pgfsetdash{}{0pt}%
\pgfpathmoveto{\pgfqpoint{3.755891in}{0.557870in}}%
\pgfpathlineto{\pgfqpoint{6.240000in}{0.557870in}}%
\pgfusepath{stroke}%
\end{pgfscope}%
\begin{pgfscope}%
\pgfsetrectcap%
\pgfsetmiterjoin%
\pgfsetlinewidth{1.254687pt}%
\definecolor{currentstroke}{rgb}{1.000000,1.000000,1.000000}%
\pgfsetstrokecolor{currentstroke}%
\pgfsetdash{}{0pt}%
\pgfpathmoveto{\pgfqpoint{3.755891in}{2.242604in}}%
\pgfpathlineto{\pgfqpoint{6.240000in}{2.242604in}}%
\pgfusepath{stroke}%
\end{pgfscope}%
\begin{pgfscope}%
\definecolor{textcolor}{rgb}{0.150000,0.150000,0.150000}%
\pgfsetstrokecolor{textcolor}%
\pgfsetfillcolor{textcolor}%
\pgftext[x=4.997946in,y=2.325938in,,base]{\color{textcolor}\sffamily\fontsize{11.000000}{13.200000}\selectfont (b)}%
\end{pgfscope}%
\end{pgfpicture}%
\makeatother%
\endgroup%

    \caption{(a) Distribution plot of \acrshort{dor} of all PVSC models when trained to classify patient diagnosis.
             (b) Scatter plot of the same models sensitivity, and specificity.}
    \label{fig:pvmlc_ind_dor_sens_spec_dist}
\end{figure}

\begin{table*}
    \centering
    \ra{1.3}
    \begin{tabular}{lrrrr}
        \toprule
        Dataset-model          &  Accuracy &  Sensitivity &  Specificity &  \acrshort{dor} \\
        \midrule
        gls-rls-EF/Ada-Boost   &      0.95 &         0.97 &         0.79 & 138.42 \\
        gls-rls/KNN            &      0.93 &         0.95 &         0.82 &  84.53 \\
        rls-EF/Extra-Trees     &      0.93 &         0.96 &         0.75 &  76.50 \\
        gls-rls-EF/Extra-Trees &      0.93 &         0.97 &         0.71 &  75.00 \\
        gls-rls/Extra-Trees    &      0.93 &         0.97 &         0.71 &  75.00 \\
        \bottomrule
    \end{tabular}
    \caption{The accuracy, \acrshort{dor}, sensitivity and specicity scores of the five best performing PVSC models in terms of \acrshort{dor}, when trained to predict patient diagnosis.
             The \textbf{Dataset-model} column indicates \textit{Dataset used}$/$\textit{Specific machine learning model used}.}
    \label{tab:pvsc_hf_dor_sens_spec_dist}
\end{table*}

From the distribution plot in figure \ref{fig:pvmlc_ind_dor_sens_spec_dist} it might seem like the majority of the \acrshort{dor} scores are close to zero, but in that is due to the shear spread of \acrshort{dor} scores so it should be said explicitly that the lowest \acrshort{dor} score of a PVSC model is 3.68 and is attained by the \textit{gls/Gaussian-Process} model. The spread of \acrshort{dor} is so great that some models attain a \acrshort{dor} close to 100, and one model attains a \acrshort{dor} close to 150. From the scatter plot in figure \ref{fig:pvmlc_ind_dor_sens_spec_dist} one can see that the sensitivity ranges from $0.75$ to 1, and the specificity ranges from close to zero to approximately $0.95$. Among the top five PVSC models in terms of \acrshort{dor} are many different combinations of models, and datasets. Three of the five highest \acrshort{dor} scores are attained by Extra-Trees models, and the top two scores are attained by KNN and Ada Boost classifiers. \textit{gls-rls-EF/Ada-Boost} and \textit{gls-rls/KNN} are the two top PVSC performers with regard to \acrshort{dor}. \textit{gls-rls-EF/Ada-Boost} achieves the highest sensitivity of the two by two points, and \textit{gls-rls/KNN} achieves the highest specificity of the two by three points. Since sensitivity and specificity is weighted equally in this study \textit{gls-rls/KNN} is chosen as the best of the PVSC models trained to identify patient diagnoses.

\newpage

\subsection{Comparisons}

\begin{table*}
    \centering
    \ra{1.3}
    \begin{tabular}{lcccc}
        \toprule
        Dataset-model                                 &  Accuracy &  Sensitivity &  Specificity &  \acrshort{dor} \\
        \midrule
        \textbf{TSC}-gls/all-views/regular/weighted/2 &      0.82 &         0.81 &         0.83 & 21.06 \\
        \textbf{PVC}-rls-EF/complete/2                &      0.77 &         0.74 &         0.93 & 37.61 \\
        \textbf{PVSC}-gls-rls/KNN                     &      0.93 &         0.95 &         0.82 & 84.53 \\
        \midrule
        Dataset-model                                 &  \acrshort{tp} &  \acrshort{tn} &  \acrshort{fp} &  \acrshort{fn} \\
        \midrule
        \textbf{TSC}-gls/all-views/regular/weighted/2 & 136 &  24 &   5 &  31 \\
        \textbf{PVC}-rls-EF/complete/2                & 117 &  27 &   2 &  42 \\
        \textbf{PVSC}-gls-rls/KNN                     & 147 &  23 &  5  &   4 \\
        \bottomrule
    \end{tabular}
    \caption{A table comparing the best contenders within each model group for predicting patient diagnoses. 
             The top table compare the models by their accuracy, sensitivity, specificity and \acrshort{dor}, 
             and the bottom table shows the number of \acrshort{tp}, \acrshort{tn}, \acrshort{fp} and \acrshort{fn} that the different models attain on their respective datasets.}
    \label{tab:pd_compare}
\end{table*}

From the top table in \ref{tab:pd_compare} one can see that there is a significant difference in performance between the three models included for comparison. The \acrshort{tsc} model \textit{gls/all-views/regular/weighted/2} attains the second highest accuracy, sensitivity and specificity of the three models, but also attains the lowest \acrshort{dor}. The \acrshort{tsc} model can also be said to attain the most balanced scores in terms of sensitivity and specificity. The PVC model \textit{rls-EF/complete/2} attains the highest specificity, second highest \acrshort{dor}, but lowest sensitivity and accuracy of the three models. The PVSC model \textit{gls-rls/KNN} attains the highest accuracy, sensitivity and \acrshort{dor} of all the models, but it also achieves the lowest specificity of all the models. However, since the PVSC model is so close to the \acrshort{tsc} model in terms of specificity, and is so much better than the other two models in all other metrics, it is chosen as the best model of identifying patient diagnoses. This can be confirmed from the bottom table in \ref{tab:pd_compare}, where one can see that the \acrshort{pvsc} model only gets one \acrshort{tn} less than the \acrshort{tsc} model, but attains 11 more \acrshort{tp}. 



\newpage
\section{Case Study: Segment Indication}

\subsection{Time-series Clustering}

\begin{figure}[htb]
    \centering
    %% Creator: Matplotlib, PGF backend
%%
%% To include the figure in your LaTeX document, write
%%   \input{<filename>.pgf}
%%
%% Make sure the required packages are loaded in your preamble
%%   \usepackage{pgf}
%%
%% Figures using additional raster images can only be included by \input if
%% they are in the same directory as the main LaTeX file. For loading figures
%% from other directories you can use the `import` package
%%   \usepackage{import}
%% and then include the figures with
%%   \import{<path to file>}{<filename>.pgf}
%%
%% Matplotlib used the following preamble
%%
\begingroup%
\makeatletter%
\begin{pgfpicture}%
\pgfpathrectangle{\pgfpointorigin}{\pgfqpoint{6.480568in}{2.540000in}}%
\pgfusepath{use as bounding box, clip}%
\begin{pgfscope}%
\pgfsetbuttcap%
\pgfsetmiterjoin%
\definecolor{currentfill}{rgb}{1.000000,1.000000,1.000000}%
\pgfsetfillcolor{currentfill}%
\pgfsetlinewidth{0.000000pt}%
\definecolor{currentstroke}{rgb}{1.000000,1.000000,1.000000}%
\pgfsetstrokecolor{currentstroke}%
\pgfsetdash{}{0pt}%
\pgfpathmoveto{\pgfqpoint{0.000000in}{0.000000in}}%
\pgfpathlineto{\pgfqpoint{6.480568in}{0.000000in}}%
\pgfpathlineto{\pgfqpoint{6.480568in}{2.540000in}}%
\pgfpathlineto{\pgfqpoint{0.000000in}{2.540000in}}%
\pgfpathclose%
\pgfusepath{fill}%
\end{pgfscope}%
\begin{pgfscope}%
\pgfsetbuttcap%
\pgfsetmiterjoin%
\definecolor{currentfill}{rgb}{0.917647,0.917647,0.949020}%
\pgfsetfillcolor{currentfill}%
\pgfsetlinewidth{0.000000pt}%
\definecolor{currentstroke}{rgb}{0.000000,0.000000,0.000000}%
\pgfsetstrokecolor{currentstroke}%
\pgfsetstrokeopacity{0.000000}%
\pgfsetdash{}{0pt}%
\pgfpathmoveto{\pgfqpoint{0.693056in}{0.557870in}}%
\pgfpathlineto{\pgfqpoint{3.176962in}{0.557870in}}%
\pgfpathlineto{\pgfqpoint{3.176962in}{2.242604in}}%
\pgfpathlineto{\pgfqpoint{0.693056in}{2.242604in}}%
\pgfpathclose%
\pgfusepath{fill}%
\end{pgfscope}%
\begin{pgfscope}%
\pgfpathrectangle{\pgfqpoint{0.693056in}{0.557870in}}{\pgfqpoint{2.483906in}{1.684734in}}%
\pgfusepath{clip}%
\pgfsetroundcap%
\pgfsetroundjoin%
\pgfsetlinewidth{1.003750pt}%
\definecolor{currentstroke}{rgb}{1.000000,1.000000,1.000000}%
\pgfsetstrokecolor{currentstroke}%
\pgfsetdash{}{0pt}%
\pgfpathmoveto{\pgfqpoint{0.805961in}{0.557870in}}%
\pgfpathlineto{\pgfqpoint{0.805961in}{2.242604in}}%
\pgfusepath{stroke}%
\end{pgfscope}%
\begin{pgfscope}%
\definecolor{textcolor}{rgb}{0.150000,0.150000,0.150000}%
\pgfsetstrokecolor{textcolor}%
\pgfsetfillcolor{textcolor}%
\pgftext[x=0.805961in,y=0.425926in,,top]{\color{textcolor}\sffamily\fontsize{11.000000}{13.200000}\selectfont \(\displaystyle 0\)}%
\end{pgfscope}%
\begin{pgfscope}%
\pgfpathrectangle{\pgfqpoint{0.693056in}{0.557870in}}{\pgfqpoint{2.483906in}{1.684734in}}%
\pgfusepath{clip}%
\pgfsetroundcap%
\pgfsetroundjoin%
\pgfsetlinewidth{1.003750pt}%
\definecolor{currentstroke}{rgb}{1.000000,1.000000,1.000000}%
\pgfsetstrokecolor{currentstroke}%
\pgfsetdash{}{0pt}%
\pgfpathmoveto{\pgfqpoint{1.568945in}{0.557870in}}%
\pgfpathlineto{\pgfqpoint{1.568945in}{2.242604in}}%
\pgfusepath{stroke}%
\end{pgfscope}%
\begin{pgfscope}%
\definecolor{textcolor}{rgb}{0.150000,0.150000,0.150000}%
\pgfsetstrokecolor{textcolor}%
\pgfsetfillcolor{textcolor}%
\pgftext[x=1.568945in,y=0.425926in,,top]{\color{textcolor}\sffamily\fontsize{11.000000}{13.200000}\selectfont \(\displaystyle 5\)}%
\end{pgfscope}%
\begin{pgfscope}%
\pgfpathrectangle{\pgfqpoint{0.693056in}{0.557870in}}{\pgfqpoint{2.483906in}{1.684734in}}%
\pgfusepath{clip}%
\pgfsetroundcap%
\pgfsetroundjoin%
\pgfsetlinewidth{1.003750pt}%
\definecolor{currentstroke}{rgb}{1.000000,1.000000,1.000000}%
\pgfsetstrokecolor{currentstroke}%
\pgfsetdash{}{0pt}%
\pgfpathmoveto{\pgfqpoint{2.331929in}{0.557870in}}%
\pgfpathlineto{\pgfqpoint{2.331929in}{2.242604in}}%
\pgfusepath{stroke}%
\end{pgfscope}%
\begin{pgfscope}%
\definecolor{textcolor}{rgb}{0.150000,0.150000,0.150000}%
\pgfsetstrokecolor{textcolor}%
\pgfsetfillcolor{textcolor}%
\pgftext[x=2.331929in,y=0.425926in,,top]{\color{textcolor}\sffamily\fontsize{11.000000}{13.200000}\selectfont \(\displaystyle 10\)}%
\end{pgfscope}%
\begin{pgfscope}%
\pgfpathrectangle{\pgfqpoint{0.693056in}{0.557870in}}{\pgfqpoint{2.483906in}{1.684734in}}%
\pgfusepath{clip}%
\pgfsetroundcap%
\pgfsetroundjoin%
\pgfsetlinewidth{1.003750pt}%
\definecolor{currentstroke}{rgb}{1.000000,1.000000,1.000000}%
\pgfsetstrokecolor{currentstroke}%
\pgfsetdash{}{0pt}%
\pgfpathmoveto{\pgfqpoint{3.094913in}{0.557870in}}%
\pgfpathlineto{\pgfqpoint{3.094913in}{2.242604in}}%
\pgfusepath{stroke}%
\end{pgfscope}%
\begin{pgfscope}%
\definecolor{textcolor}{rgb}{0.150000,0.150000,0.150000}%
\pgfsetstrokecolor{textcolor}%
\pgfsetfillcolor{textcolor}%
\pgftext[x=3.094913in,y=0.425926in,,top]{\color{textcolor}\sffamily\fontsize{11.000000}{13.200000}\selectfont \(\displaystyle 15\)}%
\end{pgfscope}%
\begin{pgfscope}%
\definecolor{textcolor}{rgb}{0.150000,0.150000,0.150000}%
\pgfsetstrokecolor{textcolor}%
\pgfsetfillcolor{textcolor}%
\pgftext[x=1.935009in,y=0.235185in,,top]{\color{textcolor}\sffamily\fontsize{11.000000}{13.200000}\selectfont DOR}%
\end{pgfscope}%
\begin{pgfscope}%
\pgfpathrectangle{\pgfqpoint{0.693056in}{0.557870in}}{\pgfqpoint{2.483906in}{1.684734in}}%
\pgfusepath{clip}%
\pgfsetroundcap%
\pgfsetroundjoin%
\pgfsetlinewidth{1.003750pt}%
\definecolor{currentstroke}{rgb}{1.000000,1.000000,1.000000}%
\pgfsetstrokecolor{currentstroke}%
\pgfsetdash{}{0pt}%
\pgfpathmoveto{\pgfqpoint{0.693056in}{0.557870in}}%
\pgfpathlineto{\pgfqpoint{3.176962in}{0.557870in}}%
\pgfusepath{stroke}%
\end{pgfscope}%
\begin{pgfscope}%
\definecolor{textcolor}{rgb}{0.150000,0.150000,0.150000}%
\pgfsetstrokecolor{textcolor}%
\pgfsetfillcolor{textcolor}%
\pgftext[x=0.366783in,y=0.505064in,left,base]{\color{textcolor}\sffamily\fontsize{11.000000}{13.200000}\selectfont \(\displaystyle 0.0\)}%
\end{pgfscope}%
\begin{pgfscope}%
\pgfpathrectangle{\pgfqpoint{0.693056in}{0.557870in}}{\pgfqpoint{2.483906in}{1.684734in}}%
\pgfusepath{clip}%
\pgfsetroundcap%
\pgfsetroundjoin%
\pgfsetlinewidth{1.003750pt}%
\definecolor{currentstroke}{rgb}{1.000000,1.000000,1.000000}%
\pgfsetstrokecolor{currentstroke}%
\pgfsetdash{}{0pt}%
\pgfpathmoveto{\pgfqpoint{0.693056in}{0.958997in}}%
\pgfpathlineto{\pgfqpoint{3.176962in}{0.958997in}}%
\pgfusepath{stroke}%
\end{pgfscope}%
\begin{pgfscope}%
\definecolor{textcolor}{rgb}{0.150000,0.150000,0.150000}%
\pgfsetstrokecolor{textcolor}%
\pgfsetfillcolor{textcolor}%
\pgftext[x=0.366783in,y=0.906191in,left,base]{\color{textcolor}\sffamily\fontsize{11.000000}{13.200000}\selectfont \(\displaystyle 2.5\)}%
\end{pgfscope}%
\begin{pgfscope}%
\pgfpathrectangle{\pgfqpoint{0.693056in}{0.557870in}}{\pgfqpoint{2.483906in}{1.684734in}}%
\pgfusepath{clip}%
\pgfsetroundcap%
\pgfsetroundjoin%
\pgfsetlinewidth{1.003750pt}%
\definecolor{currentstroke}{rgb}{1.000000,1.000000,1.000000}%
\pgfsetstrokecolor{currentstroke}%
\pgfsetdash{}{0pt}%
\pgfpathmoveto{\pgfqpoint{0.693056in}{1.360125in}}%
\pgfpathlineto{\pgfqpoint{3.176962in}{1.360125in}}%
\pgfusepath{stroke}%
\end{pgfscope}%
\begin{pgfscope}%
\definecolor{textcolor}{rgb}{0.150000,0.150000,0.150000}%
\pgfsetstrokecolor{textcolor}%
\pgfsetfillcolor{textcolor}%
\pgftext[x=0.366783in,y=1.307318in,left,base]{\color{textcolor}\sffamily\fontsize{11.000000}{13.200000}\selectfont \(\displaystyle 5.0\)}%
\end{pgfscope}%
\begin{pgfscope}%
\pgfpathrectangle{\pgfqpoint{0.693056in}{0.557870in}}{\pgfqpoint{2.483906in}{1.684734in}}%
\pgfusepath{clip}%
\pgfsetroundcap%
\pgfsetroundjoin%
\pgfsetlinewidth{1.003750pt}%
\definecolor{currentstroke}{rgb}{1.000000,1.000000,1.000000}%
\pgfsetstrokecolor{currentstroke}%
\pgfsetdash{}{0pt}%
\pgfpathmoveto{\pgfqpoint{0.693056in}{1.761252in}}%
\pgfpathlineto{\pgfqpoint{3.176962in}{1.761252in}}%
\pgfusepath{stroke}%
\end{pgfscope}%
\begin{pgfscope}%
\definecolor{textcolor}{rgb}{0.150000,0.150000,0.150000}%
\pgfsetstrokecolor{textcolor}%
\pgfsetfillcolor{textcolor}%
\pgftext[x=0.366783in,y=1.708445in,left,base]{\color{textcolor}\sffamily\fontsize{11.000000}{13.200000}\selectfont \(\displaystyle 7.5\)}%
\end{pgfscope}%
\begin{pgfscope}%
\pgfpathrectangle{\pgfqpoint{0.693056in}{0.557870in}}{\pgfqpoint{2.483906in}{1.684734in}}%
\pgfusepath{clip}%
\pgfsetroundcap%
\pgfsetroundjoin%
\pgfsetlinewidth{1.003750pt}%
\definecolor{currentstroke}{rgb}{1.000000,1.000000,1.000000}%
\pgfsetstrokecolor{currentstroke}%
\pgfsetdash{}{0pt}%
\pgfpathmoveto{\pgfqpoint{0.693056in}{2.162379in}}%
\pgfpathlineto{\pgfqpoint{3.176962in}{2.162379in}}%
\pgfusepath{stroke}%
\end{pgfscope}%
\begin{pgfscope}%
\definecolor{textcolor}{rgb}{0.150000,0.150000,0.150000}%
\pgfsetstrokecolor{textcolor}%
\pgfsetfillcolor{textcolor}%
\pgftext[x=0.290741in,y=2.109572in,left,base]{\color{textcolor}\sffamily\fontsize{11.000000}{13.200000}\selectfont \(\displaystyle 10.0\)}%
\end{pgfscope}%
\begin{pgfscope}%
\definecolor{textcolor}{rgb}{0.150000,0.150000,0.150000}%
\pgfsetstrokecolor{textcolor}%
\pgfsetfillcolor{textcolor}%
\pgftext[x=0.235185in,y=1.400237in,,bottom,rotate=90.000000]{\color{textcolor}\sffamily\fontsize{11.000000}{13.200000}\selectfont Occurance}%
\end{pgfscope}%
\begin{pgfscope}%
\pgfpathrectangle{\pgfqpoint{0.693056in}{0.557870in}}{\pgfqpoint{2.483906in}{1.684734in}}%
\pgfusepath{clip}%
\pgfsetbuttcap%
\pgfsetmiterjoin%
\definecolor{currentfill}{rgb}{0.298039,0.447059,0.690196}%
\pgfsetfillcolor{currentfill}%
\pgfsetfillopacity{0.400000}%
\pgfsetlinewidth{1.003750pt}%
\definecolor{currentstroke}{rgb}{1.000000,1.000000,1.000000}%
\pgfsetstrokecolor{currentstroke}%
\pgfsetstrokeopacity{0.400000}%
\pgfsetdash{}{0pt}%
\pgfpathmoveto{\pgfqpoint{0.805961in}{0.557870in}}%
\pgfpathlineto{\pgfqpoint{1.031770in}{0.557870in}}%
\pgfpathlineto{\pgfqpoint{1.031770in}{2.162379in}}%
\pgfpathlineto{\pgfqpoint{0.805961in}{2.162379in}}%
\pgfpathclose%
\pgfusepath{stroke,fill}%
\end{pgfscope}%
\begin{pgfscope}%
\pgfpathrectangle{\pgfqpoint{0.693056in}{0.557870in}}{\pgfqpoint{2.483906in}{1.684734in}}%
\pgfusepath{clip}%
\pgfsetbuttcap%
\pgfsetmiterjoin%
\definecolor{currentfill}{rgb}{0.298039,0.447059,0.690196}%
\pgfsetfillcolor{currentfill}%
\pgfsetfillopacity{0.400000}%
\pgfsetlinewidth{1.003750pt}%
\definecolor{currentstroke}{rgb}{1.000000,1.000000,1.000000}%
\pgfsetstrokecolor{currentstroke}%
\pgfsetstrokeopacity{0.400000}%
\pgfsetdash{}{0pt}%
\pgfpathmoveto{\pgfqpoint{1.031770in}{0.557870in}}%
\pgfpathlineto{\pgfqpoint{1.257580in}{0.557870in}}%
\pgfpathlineto{\pgfqpoint{1.257580in}{0.718321in}}%
\pgfpathlineto{\pgfqpoint{1.031770in}{0.718321in}}%
\pgfpathclose%
\pgfusepath{stroke,fill}%
\end{pgfscope}%
\begin{pgfscope}%
\pgfpathrectangle{\pgfqpoint{0.693056in}{0.557870in}}{\pgfqpoint{2.483906in}{1.684734in}}%
\pgfusepath{clip}%
\pgfsetbuttcap%
\pgfsetmiterjoin%
\definecolor{currentfill}{rgb}{0.298039,0.447059,0.690196}%
\pgfsetfillcolor{currentfill}%
\pgfsetfillopacity{0.400000}%
\pgfsetlinewidth{1.003750pt}%
\definecolor{currentstroke}{rgb}{1.000000,1.000000,1.000000}%
\pgfsetstrokecolor{currentstroke}%
\pgfsetstrokeopacity{0.400000}%
\pgfsetdash{}{0pt}%
\pgfpathmoveto{\pgfqpoint{1.257580in}{0.557870in}}%
\pgfpathlineto{\pgfqpoint{1.483390in}{0.557870in}}%
\pgfpathlineto{\pgfqpoint{1.483390in}{0.557870in}}%
\pgfpathlineto{\pgfqpoint{1.257580in}{0.557870in}}%
\pgfpathclose%
\pgfusepath{stroke,fill}%
\end{pgfscope}%
\begin{pgfscope}%
\pgfpathrectangle{\pgfqpoint{0.693056in}{0.557870in}}{\pgfqpoint{2.483906in}{1.684734in}}%
\pgfusepath{clip}%
\pgfsetbuttcap%
\pgfsetmiterjoin%
\definecolor{currentfill}{rgb}{0.298039,0.447059,0.690196}%
\pgfsetfillcolor{currentfill}%
\pgfsetfillopacity{0.400000}%
\pgfsetlinewidth{1.003750pt}%
\definecolor{currentstroke}{rgb}{1.000000,1.000000,1.000000}%
\pgfsetstrokecolor{currentstroke}%
\pgfsetstrokeopacity{0.400000}%
\pgfsetdash{}{0pt}%
\pgfpathmoveto{\pgfqpoint{1.483390in}{0.557870in}}%
\pgfpathlineto{\pgfqpoint{1.709199in}{0.557870in}}%
\pgfpathlineto{\pgfqpoint{1.709199in}{0.718321in}}%
\pgfpathlineto{\pgfqpoint{1.483390in}{0.718321in}}%
\pgfpathclose%
\pgfusepath{stroke,fill}%
\end{pgfscope}%
\begin{pgfscope}%
\pgfpathrectangle{\pgfqpoint{0.693056in}{0.557870in}}{\pgfqpoint{2.483906in}{1.684734in}}%
\pgfusepath{clip}%
\pgfsetbuttcap%
\pgfsetmiterjoin%
\definecolor{currentfill}{rgb}{0.298039,0.447059,0.690196}%
\pgfsetfillcolor{currentfill}%
\pgfsetfillopacity{0.400000}%
\pgfsetlinewidth{1.003750pt}%
\definecolor{currentstroke}{rgb}{1.000000,1.000000,1.000000}%
\pgfsetstrokecolor{currentstroke}%
\pgfsetstrokeopacity{0.400000}%
\pgfsetdash{}{0pt}%
\pgfpathmoveto{\pgfqpoint{1.709199in}{0.557870in}}%
\pgfpathlineto{\pgfqpoint{1.935009in}{0.557870in}}%
\pgfpathlineto{\pgfqpoint{1.935009in}{0.718321in}}%
\pgfpathlineto{\pgfqpoint{1.709199in}{0.718321in}}%
\pgfpathclose%
\pgfusepath{stroke,fill}%
\end{pgfscope}%
\begin{pgfscope}%
\pgfpathrectangle{\pgfqpoint{0.693056in}{0.557870in}}{\pgfqpoint{2.483906in}{1.684734in}}%
\pgfusepath{clip}%
\pgfsetbuttcap%
\pgfsetmiterjoin%
\definecolor{currentfill}{rgb}{0.298039,0.447059,0.690196}%
\pgfsetfillcolor{currentfill}%
\pgfsetfillopacity{0.400000}%
\pgfsetlinewidth{1.003750pt}%
\definecolor{currentstroke}{rgb}{1.000000,1.000000,1.000000}%
\pgfsetstrokecolor{currentstroke}%
\pgfsetstrokeopacity{0.400000}%
\pgfsetdash{}{0pt}%
\pgfpathmoveto{\pgfqpoint{1.935009in}{0.557870in}}%
\pgfpathlineto{\pgfqpoint{2.160818in}{0.557870in}}%
\pgfpathlineto{\pgfqpoint{2.160818in}{0.557870in}}%
\pgfpathlineto{\pgfqpoint{1.935009in}{0.557870in}}%
\pgfpathclose%
\pgfusepath{stroke,fill}%
\end{pgfscope}%
\begin{pgfscope}%
\pgfpathrectangle{\pgfqpoint{0.693056in}{0.557870in}}{\pgfqpoint{2.483906in}{1.684734in}}%
\pgfusepath{clip}%
\pgfsetbuttcap%
\pgfsetmiterjoin%
\definecolor{currentfill}{rgb}{0.298039,0.447059,0.690196}%
\pgfsetfillcolor{currentfill}%
\pgfsetfillopacity{0.400000}%
\pgfsetlinewidth{1.003750pt}%
\definecolor{currentstroke}{rgb}{1.000000,1.000000,1.000000}%
\pgfsetstrokecolor{currentstroke}%
\pgfsetstrokeopacity{0.400000}%
\pgfsetdash{}{0pt}%
\pgfpathmoveto{\pgfqpoint{2.160818in}{0.557870in}}%
\pgfpathlineto{\pgfqpoint{2.386628in}{0.557870in}}%
\pgfpathlineto{\pgfqpoint{2.386628in}{0.557870in}}%
\pgfpathlineto{\pgfqpoint{2.160818in}{0.557870in}}%
\pgfpathclose%
\pgfusepath{stroke,fill}%
\end{pgfscope}%
\begin{pgfscope}%
\pgfpathrectangle{\pgfqpoint{0.693056in}{0.557870in}}{\pgfqpoint{2.483906in}{1.684734in}}%
\pgfusepath{clip}%
\pgfsetbuttcap%
\pgfsetmiterjoin%
\definecolor{currentfill}{rgb}{0.298039,0.447059,0.690196}%
\pgfsetfillcolor{currentfill}%
\pgfsetfillopacity{0.400000}%
\pgfsetlinewidth{1.003750pt}%
\definecolor{currentstroke}{rgb}{1.000000,1.000000,1.000000}%
\pgfsetstrokecolor{currentstroke}%
\pgfsetstrokeopacity{0.400000}%
\pgfsetdash{}{0pt}%
\pgfpathmoveto{\pgfqpoint{2.386628in}{0.557870in}}%
\pgfpathlineto{\pgfqpoint{2.612438in}{0.557870in}}%
\pgfpathlineto{\pgfqpoint{2.612438in}{0.557870in}}%
\pgfpathlineto{\pgfqpoint{2.386628in}{0.557870in}}%
\pgfpathclose%
\pgfusepath{stroke,fill}%
\end{pgfscope}%
\begin{pgfscope}%
\pgfpathrectangle{\pgfqpoint{0.693056in}{0.557870in}}{\pgfqpoint{2.483906in}{1.684734in}}%
\pgfusepath{clip}%
\pgfsetbuttcap%
\pgfsetmiterjoin%
\definecolor{currentfill}{rgb}{0.298039,0.447059,0.690196}%
\pgfsetfillcolor{currentfill}%
\pgfsetfillopacity{0.400000}%
\pgfsetlinewidth{1.003750pt}%
\definecolor{currentstroke}{rgb}{1.000000,1.000000,1.000000}%
\pgfsetstrokecolor{currentstroke}%
\pgfsetstrokeopacity{0.400000}%
\pgfsetdash{}{0pt}%
\pgfpathmoveto{\pgfqpoint{2.612438in}{0.557870in}}%
\pgfpathlineto{\pgfqpoint{2.838247in}{0.557870in}}%
\pgfpathlineto{\pgfqpoint{2.838247in}{1.199674in}}%
\pgfpathlineto{\pgfqpoint{2.612438in}{1.199674in}}%
\pgfpathclose%
\pgfusepath{stroke,fill}%
\end{pgfscope}%
\begin{pgfscope}%
\pgfpathrectangle{\pgfqpoint{0.693056in}{0.557870in}}{\pgfqpoint{2.483906in}{1.684734in}}%
\pgfusepath{clip}%
\pgfsetbuttcap%
\pgfsetmiterjoin%
\definecolor{currentfill}{rgb}{0.298039,0.447059,0.690196}%
\pgfsetfillcolor{currentfill}%
\pgfsetfillopacity{0.400000}%
\pgfsetlinewidth{1.003750pt}%
\definecolor{currentstroke}{rgb}{1.000000,1.000000,1.000000}%
\pgfsetstrokecolor{currentstroke}%
\pgfsetstrokeopacity{0.400000}%
\pgfsetdash{}{0pt}%
\pgfpathmoveto{\pgfqpoint{2.838247in}{0.557870in}}%
\pgfpathlineto{\pgfqpoint{3.064057in}{0.557870in}}%
\pgfpathlineto{\pgfqpoint{3.064057in}{0.878772in}}%
\pgfpathlineto{\pgfqpoint{2.838247in}{0.878772in}}%
\pgfpathclose%
\pgfusepath{stroke,fill}%
\end{pgfscope}%
\begin{pgfscope}%
\pgfsetrectcap%
\pgfsetmiterjoin%
\pgfsetlinewidth{1.254687pt}%
\definecolor{currentstroke}{rgb}{1.000000,1.000000,1.000000}%
\pgfsetstrokecolor{currentstroke}%
\pgfsetdash{}{0pt}%
\pgfpathmoveto{\pgfqpoint{0.693056in}{0.557870in}}%
\pgfpathlineto{\pgfqpoint{0.693056in}{2.242604in}}%
\pgfusepath{stroke}%
\end{pgfscope}%
\begin{pgfscope}%
\pgfsetrectcap%
\pgfsetmiterjoin%
\pgfsetlinewidth{1.254687pt}%
\definecolor{currentstroke}{rgb}{1.000000,1.000000,1.000000}%
\pgfsetstrokecolor{currentstroke}%
\pgfsetdash{}{0pt}%
\pgfpathmoveto{\pgfqpoint{3.176962in}{0.557870in}}%
\pgfpathlineto{\pgfqpoint{3.176962in}{2.242604in}}%
\pgfusepath{stroke}%
\end{pgfscope}%
\begin{pgfscope}%
\pgfsetrectcap%
\pgfsetmiterjoin%
\pgfsetlinewidth{1.254687pt}%
\definecolor{currentstroke}{rgb}{1.000000,1.000000,1.000000}%
\pgfsetstrokecolor{currentstroke}%
\pgfsetdash{}{0pt}%
\pgfpathmoveto{\pgfqpoint{0.693056in}{0.557870in}}%
\pgfpathlineto{\pgfqpoint{3.176962in}{0.557870in}}%
\pgfusepath{stroke}%
\end{pgfscope}%
\begin{pgfscope}%
\pgfsetrectcap%
\pgfsetmiterjoin%
\pgfsetlinewidth{1.254687pt}%
\definecolor{currentstroke}{rgb}{1.000000,1.000000,1.000000}%
\pgfsetstrokecolor{currentstroke}%
\pgfsetdash{}{0pt}%
\pgfpathmoveto{\pgfqpoint{0.693056in}{2.242604in}}%
\pgfpathlineto{\pgfqpoint{3.176962in}{2.242604in}}%
\pgfusepath{stroke}%
\end{pgfscope}%
\begin{pgfscope}%
\definecolor{textcolor}{rgb}{0.150000,0.150000,0.150000}%
\pgfsetstrokecolor{textcolor}%
\pgfsetfillcolor{textcolor}%
\pgftext[x=1.935009in,y=2.325938in,,base]{\color{textcolor}\sffamily\fontsize{11.000000}{13.200000}\selectfont (a)}%
\end{pgfscope}%
\begin{pgfscope}%
\pgfsetbuttcap%
\pgfsetmiterjoin%
\definecolor{currentfill}{rgb}{0.917647,0.917647,0.949020}%
\pgfsetfillcolor{currentfill}%
\pgfsetlinewidth{0.000000pt}%
\definecolor{currentstroke}{rgb}{0.000000,0.000000,0.000000}%
\pgfsetstrokecolor{currentstroke}%
\pgfsetstrokeopacity{0.000000}%
\pgfsetdash{}{0pt}%
\pgfpathmoveto{\pgfqpoint{3.874381in}{0.557870in}}%
\pgfpathlineto{\pgfqpoint{6.358287in}{0.557870in}}%
\pgfpathlineto{\pgfqpoint{6.358287in}{2.242604in}}%
\pgfpathlineto{\pgfqpoint{3.874381in}{2.242604in}}%
\pgfpathclose%
\pgfusepath{fill}%
\end{pgfscope}%
\begin{pgfscope}%
\pgfpathrectangle{\pgfqpoint{3.874381in}{0.557870in}}{\pgfqpoint{2.483906in}{1.684734in}}%
\pgfusepath{clip}%
\pgfsetroundcap%
\pgfsetroundjoin%
\pgfsetlinewidth{1.003750pt}%
\definecolor{currentstroke}{rgb}{1.000000,1.000000,1.000000}%
\pgfsetstrokecolor{currentstroke}%
\pgfsetdash{}{0pt}%
\pgfpathmoveto{\pgfqpoint{3.987286in}{0.557870in}}%
\pgfpathlineto{\pgfqpoint{3.987286in}{2.242604in}}%
\pgfusepath{stroke}%
\end{pgfscope}%
\begin{pgfscope}%
\definecolor{textcolor}{rgb}{0.150000,0.150000,0.150000}%
\pgfsetstrokecolor{textcolor}%
\pgfsetfillcolor{textcolor}%
\pgftext[x=3.987286in,y=0.425926in,,top]{\color{textcolor}\sffamily\fontsize{11.000000}{13.200000}\selectfont \(\displaystyle 0.00\)}%
\end{pgfscope}%
\begin{pgfscope}%
\pgfpathrectangle{\pgfqpoint{3.874381in}{0.557870in}}{\pgfqpoint{2.483906in}{1.684734in}}%
\pgfusepath{clip}%
\pgfsetroundcap%
\pgfsetroundjoin%
\pgfsetlinewidth{1.003750pt}%
\definecolor{currentstroke}{rgb}{1.000000,1.000000,1.000000}%
\pgfsetstrokecolor{currentstroke}%
\pgfsetdash{}{0pt}%
\pgfpathmoveto{\pgfqpoint{4.551810in}{0.557870in}}%
\pgfpathlineto{\pgfqpoint{4.551810in}{2.242604in}}%
\pgfusepath{stroke}%
\end{pgfscope}%
\begin{pgfscope}%
\definecolor{textcolor}{rgb}{0.150000,0.150000,0.150000}%
\pgfsetstrokecolor{textcolor}%
\pgfsetfillcolor{textcolor}%
\pgftext[x=4.551810in,y=0.425926in,,top]{\color{textcolor}\sffamily\fontsize{11.000000}{13.200000}\selectfont \(\displaystyle 0.25\)}%
\end{pgfscope}%
\begin{pgfscope}%
\pgfpathrectangle{\pgfqpoint{3.874381in}{0.557870in}}{\pgfqpoint{2.483906in}{1.684734in}}%
\pgfusepath{clip}%
\pgfsetroundcap%
\pgfsetroundjoin%
\pgfsetlinewidth{1.003750pt}%
\definecolor{currentstroke}{rgb}{1.000000,1.000000,1.000000}%
\pgfsetstrokecolor{currentstroke}%
\pgfsetdash{}{0pt}%
\pgfpathmoveto{\pgfqpoint{5.116334in}{0.557870in}}%
\pgfpathlineto{\pgfqpoint{5.116334in}{2.242604in}}%
\pgfusepath{stroke}%
\end{pgfscope}%
\begin{pgfscope}%
\definecolor{textcolor}{rgb}{0.150000,0.150000,0.150000}%
\pgfsetstrokecolor{textcolor}%
\pgfsetfillcolor{textcolor}%
\pgftext[x=5.116334in,y=0.425926in,,top]{\color{textcolor}\sffamily\fontsize{11.000000}{13.200000}\selectfont \(\displaystyle 0.50\)}%
\end{pgfscope}%
\begin{pgfscope}%
\pgfpathrectangle{\pgfqpoint{3.874381in}{0.557870in}}{\pgfqpoint{2.483906in}{1.684734in}}%
\pgfusepath{clip}%
\pgfsetroundcap%
\pgfsetroundjoin%
\pgfsetlinewidth{1.003750pt}%
\definecolor{currentstroke}{rgb}{1.000000,1.000000,1.000000}%
\pgfsetstrokecolor{currentstroke}%
\pgfsetdash{}{0pt}%
\pgfpathmoveto{\pgfqpoint{5.680858in}{0.557870in}}%
\pgfpathlineto{\pgfqpoint{5.680858in}{2.242604in}}%
\pgfusepath{stroke}%
\end{pgfscope}%
\begin{pgfscope}%
\definecolor{textcolor}{rgb}{0.150000,0.150000,0.150000}%
\pgfsetstrokecolor{textcolor}%
\pgfsetfillcolor{textcolor}%
\pgftext[x=5.680858in,y=0.425926in,,top]{\color{textcolor}\sffamily\fontsize{11.000000}{13.200000}\selectfont \(\displaystyle 0.75\)}%
\end{pgfscope}%
\begin{pgfscope}%
\pgfpathrectangle{\pgfqpoint{3.874381in}{0.557870in}}{\pgfqpoint{2.483906in}{1.684734in}}%
\pgfusepath{clip}%
\pgfsetroundcap%
\pgfsetroundjoin%
\pgfsetlinewidth{1.003750pt}%
\definecolor{currentstroke}{rgb}{1.000000,1.000000,1.000000}%
\pgfsetstrokecolor{currentstroke}%
\pgfsetdash{}{0pt}%
\pgfpathmoveto{\pgfqpoint{6.245382in}{0.557870in}}%
\pgfpathlineto{\pgfqpoint{6.245382in}{2.242604in}}%
\pgfusepath{stroke}%
\end{pgfscope}%
\begin{pgfscope}%
\definecolor{textcolor}{rgb}{0.150000,0.150000,0.150000}%
\pgfsetstrokecolor{textcolor}%
\pgfsetfillcolor{textcolor}%
\pgftext[x=6.245382in,y=0.425926in,,top]{\color{textcolor}\sffamily\fontsize{11.000000}{13.200000}\selectfont \(\displaystyle 1.00\)}%
\end{pgfscope}%
\begin{pgfscope}%
\definecolor{textcolor}{rgb}{0.150000,0.150000,0.150000}%
\pgfsetstrokecolor{textcolor}%
\pgfsetfillcolor{textcolor}%
\pgftext[x=5.116334in,y=0.235185in,,top]{\color{textcolor}\sffamily\fontsize{11.000000}{13.200000}\selectfont Specificity}%
\end{pgfscope}%
\begin{pgfscope}%
\pgfpathrectangle{\pgfqpoint{3.874381in}{0.557870in}}{\pgfqpoint{2.483906in}{1.684734in}}%
\pgfusepath{clip}%
\pgfsetroundcap%
\pgfsetroundjoin%
\pgfsetlinewidth{1.003750pt}%
\definecolor{currentstroke}{rgb}{1.000000,1.000000,1.000000}%
\pgfsetstrokecolor{currentstroke}%
\pgfsetdash{}{0pt}%
\pgfpathmoveto{\pgfqpoint{3.874381in}{0.634449in}}%
\pgfpathlineto{\pgfqpoint{6.358287in}{0.634449in}}%
\pgfusepath{stroke}%
\end{pgfscope}%
\begin{pgfscope}%
\definecolor{textcolor}{rgb}{0.150000,0.150000,0.150000}%
\pgfsetstrokecolor{textcolor}%
\pgfsetfillcolor{textcolor}%
\pgftext[x=3.472066in,y=0.581642in,left,base]{\color{textcolor}\sffamily\fontsize{11.000000}{13.200000}\selectfont \(\displaystyle 0.00\)}%
\end{pgfscope}%
\begin{pgfscope}%
\pgfpathrectangle{\pgfqpoint{3.874381in}{0.557870in}}{\pgfqpoint{2.483906in}{1.684734in}}%
\pgfusepath{clip}%
\pgfsetroundcap%
\pgfsetroundjoin%
\pgfsetlinewidth{1.003750pt}%
\definecolor{currentstroke}{rgb}{1.000000,1.000000,1.000000}%
\pgfsetstrokecolor{currentstroke}%
\pgfsetdash{}{0pt}%
\pgfpathmoveto{\pgfqpoint{3.874381in}{1.017343in}}%
\pgfpathlineto{\pgfqpoint{6.358287in}{1.017343in}}%
\pgfusepath{stroke}%
\end{pgfscope}%
\begin{pgfscope}%
\definecolor{textcolor}{rgb}{0.150000,0.150000,0.150000}%
\pgfsetstrokecolor{textcolor}%
\pgfsetfillcolor{textcolor}%
\pgftext[x=3.472066in,y=0.964536in,left,base]{\color{textcolor}\sffamily\fontsize{11.000000}{13.200000}\selectfont \(\displaystyle 0.25\)}%
\end{pgfscope}%
\begin{pgfscope}%
\pgfpathrectangle{\pgfqpoint{3.874381in}{0.557870in}}{\pgfqpoint{2.483906in}{1.684734in}}%
\pgfusepath{clip}%
\pgfsetroundcap%
\pgfsetroundjoin%
\pgfsetlinewidth{1.003750pt}%
\definecolor{currentstroke}{rgb}{1.000000,1.000000,1.000000}%
\pgfsetstrokecolor{currentstroke}%
\pgfsetdash{}{0pt}%
\pgfpathmoveto{\pgfqpoint{3.874381in}{1.400237in}}%
\pgfpathlineto{\pgfqpoint{6.358287in}{1.400237in}}%
\pgfusepath{stroke}%
\end{pgfscope}%
\begin{pgfscope}%
\definecolor{textcolor}{rgb}{0.150000,0.150000,0.150000}%
\pgfsetstrokecolor{textcolor}%
\pgfsetfillcolor{textcolor}%
\pgftext[x=3.472066in,y=1.347431in,left,base]{\color{textcolor}\sffamily\fontsize{11.000000}{13.200000}\selectfont \(\displaystyle 0.50\)}%
\end{pgfscope}%
\begin{pgfscope}%
\pgfpathrectangle{\pgfqpoint{3.874381in}{0.557870in}}{\pgfqpoint{2.483906in}{1.684734in}}%
\pgfusepath{clip}%
\pgfsetroundcap%
\pgfsetroundjoin%
\pgfsetlinewidth{1.003750pt}%
\definecolor{currentstroke}{rgb}{1.000000,1.000000,1.000000}%
\pgfsetstrokecolor{currentstroke}%
\pgfsetdash{}{0pt}%
\pgfpathmoveto{\pgfqpoint{3.874381in}{1.783131in}}%
\pgfpathlineto{\pgfqpoint{6.358287in}{1.783131in}}%
\pgfusepath{stroke}%
\end{pgfscope}%
\begin{pgfscope}%
\definecolor{textcolor}{rgb}{0.150000,0.150000,0.150000}%
\pgfsetstrokecolor{textcolor}%
\pgfsetfillcolor{textcolor}%
\pgftext[x=3.472066in,y=1.730325in,left,base]{\color{textcolor}\sffamily\fontsize{11.000000}{13.200000}\selectfont \(\displaystyle 0.75\)}%
\end{pgfscope}%
\begin{pgfscope}%
\pgfpathrectangle{\pgfqpoint{3.874381in}{0.557870in}}{\pgfqpoint{2.483906in}{1.684734in}}%
\pgfusepath{clip}%
\pgfsetroundcap%
\pgfsetroundjoin%
\pgfsetlinewidth{1.003750pt}%
\definecolor{currentstroke}{rgb}{1.000000,1.000000,1.000000}%
\pgfsetstrokecolor{currentstroke}%
\pgfsetdash{}{0pt}%
\pgfpathmoveto{\pgfqpoint{3.874381in}{2.166025in}}%
\pgfpathlineto{\pgfqpoint{6.358287in}{2.166025in}}%
\pgfusepath{stroke}%
\end{pgfscope}%
\begin{pgfscope}%
\definecolor{textcolor}{rgb}{0.150000,0.150000,0.150000}%
\pgfsetstrokecolor{textcolor}%
\pgfsetfillcolor{textcolor}%
\pgftext[x=3.472066in,y=2.113219in,left,base]{\color{textcolor}\sffamily\fontsize{11.000000}{13.200000}\selectfont \(\displaystyle 1.00\)}%
\end{pgfscope}%
\begin{pgfscope}%
\definecolor{textcolor}{rgb}{0.150000,0.150000,0.150000}%
\pgfsetstrokecolor{textcolor}%
\pgfsetfillcolor{textcolor}%
\pgftext[x=3.416511in,y=1.400237in,,bottom,rotate=90.000000]{\color{textcolor}\sffamily\fontsize{11.000000}{13.200000}\selectfont Sensitivity}%
\end{pgfscope}%
\begin{pgfscope}%
\pgfpathrectangle{\pgfqpoint{3.874381in}{0.557870in}}{\pgfqpoint{2.483906in}{1.684734in}}%
\pgfusepath{clip}%
\pgfsetbuttcap%
\pgfsetroundjoin%
\definecolor{currentfill}{rgb}{0.298039,0.447059,0.690196}%
\pgfsetfillcolor{currentfill}%
\pgfsetlinewidth{1.003750pt}%
\definecolor{currentstroke}{rgb}{0.298039,0.447059,0.690196}%
\pgfsetstrokecolor{currentstroke}%
\pgfsetdash{}{0pt}%
\pgfpathmoveto{\pgfqpoint{6.132145in}{1.275137in}}%
\pgfpathcurveto{\pgfqpoint{6.140381in}{1.275137in}}{\pgfqpoint{6.148281in}{1.278409in}}{\pgfqpoint{6.154105in}{1.284233in}}%
\pgfpathcurveto{\pgfqpoint{6.159929in}{1.290057in}}{\pgfqpoint{6.163201in}{1.297957in}}{\pgfqpoint{6.163201in}{1.306193in}}%
\pgfpathcurveto{\pgfqpoint{6.163201in}{1.314429in}}{\pgfqpoint{6.159929in}{1.322329in}}{\pgfqpoint{6.154105in}{1.328153in}}%
\pgfpathcurveto{\pgfqpoint{6.148281in}{1.333977in}}{\pgfqpoint{6.140381in}{1.337250in}}{\pgfqpoint{6.132145in}{1.337250in}}%
\pgfpathcurveto{\pgfqpoint{6.123908in}{1.337250in}}{\pgfqpoint{6.116008in}{1.333977in}}{\pgfqpoint{6.110184in}{1.328153in}}%
\pgfpathcurveto{\pgfqpoint{6.104360in}{1.322329in}}{\pgfqpoint{6.101088in}{1.314429in}}{\pgfqpoint{6.101088in}{1.306193in}}%
\pgfpathcurveto{\pgfqpoint{6.101088in}{1.297957in}}{\pgfqpoint{6.104360in}{1.290057in}}{\pgfqpoint{6.110184in}{1.284233in}}%
\pgfpathcurveto{\pgfqpoint{6.116008in}{1.278409in}}{\pgfqpoint{6.123908in}{1.275137in}}{\pgfqpoint{6.132145in}{1.275137in}}%
\pgfpathclose%
\pgfusepath{stroke,fill}%
\end{pgfscope}%
\begin{pgfscope}%
\pgfpathrectangle{\pgfqpoint{3.874381in}{0.557870in}}{\pgfqpoint{2.483906in}{1.684734in}}%
\pgfusepath{clip}%
\pgfsetbuttcap%
\pgfsetroundjoin%
\definecolor{currentfill}{rgb}{0.298039,0.447059,0.690196}%
\pgfsetfillcolor{currentfill}%
\pgfsetlinewidth{1.003750pt}%
\definecolor{currentstroke}{rgb}{0.298039,0.447059,0.690196}%
\pgfsetstrokecolor{currentstroke}%
\pgfsetdash{}{0pt}%
\pgfpathmoveto{\pgfqpoint{6.132145in}{1.275137in}}%
\pgfpathcurveto{\pgfqpoint{6.140381in}{1.275137in}}{\pgfqpoint{6.148281in}{1.278409in}}{\pgfqpoint{6.154105in}{1.284233in}}%
\pgfpathcurveto{\pgfqpoint{6.159929in}{1.290057in}}{\pgfqpoint{6.163201in}{1.297957in}}{\pgfqpoint{6.163201in}{1.306193in}}%
\pgfpathcurveto{\pgfqpoint{6.163201in}{1.314429in}}{\pgfqpoint{6.159929in}{1.322329in}}{\pgfqpoint{6.154105in}{1.328153in}}%
\pgfpathcurveto{\pgfqpoint{6.148281in}{1.333977in}}{\pgfqpoint{6.140381in}{1.337250in}}{\pgfqpoint{6.132145in}{1.337250in}}%
\pgfpathcurveto{\pgfqpoint{6.123908in}{1.337250in}}{\pgfqpoint{6.116008in}{1.333977in}}{\pgfqpoint{6.110184in}{1.328153in}}%
\pgfpathcurveto{\pgfqpoint{6.104360in}{1.322329in}}{\pgfqpoint{6.101088in}{1.314429in}}{\pgfqpoint{6.101088in}{1.306193in}}%
\pgfpathcurveto{\pgfqpoint{6.101088in}{1.297957in}}{\pgfqpoint{6.104360in}{1.290057in}}{\pgfqpoint{6.110184in}{1.284233in}}%
\pgfpathcurveto{\pgfqpoint{6.116008in}{1.278409in}}{\pgfqpoint{6.123908in}{1.275137in}}{\pgfqpoint{6.132145in}{1.275137in}}%
\pgfpathclose%
\pgfusepath{stroke,fill}%
\end{pgfscope}%
\begin{pgfscope}%
\pgfpathrectangle{\pgfqpoint{3.874381in}{0.557870in}}{\pgfqpoint{2.483906in}{1.684734in}}%
\pgfusepath{clip}%
\pgfsetbuttcap%
\pgfsetroundjoin%
\definecolor{currentfill}{rgb}{0.298039,0.447059,0.690196}%
\pgfsetfillcolor{currentfill}%
\pgfsetlinewidth{1.003750pt}%
\definecolor{currentstroke}{rgb}{0.298039,0.447059,0.690196}%
\pgfsetstrokecolor{currentstroke}%
\pgfsetdash{}{0pt}%
\pgfpathmoveto{\pgfqpoint{5.973612in}{1.587803in}}%
\pgfpathcurveto{\pgfqpoint{5.981848in}{1.587803in}}{\pgfqpoint{5.989748in}{1.591075in}}{\pgfqpoint{5.995572in}{1.596899in}}%
\pgfpathcurveto{\pgfqpoint{6.001396in}{1.602723in}}{\pgfqpoint{6.004668in}{1.610623in}}{\pgfqpoint{6.004668in}{1.618859in}}%
\pgfpathcurveto{\pgfqpoint{6.004668in}{1.627096in}}{\pgfqpoint{6.001396in}{1.634996in}}{\pgfqpoint{5.995572in}{1.640820in}}%
\pgfpathcurveto{\pgfqpoint{5.989748in}{1.646644in}}{\pgfqpoint{5.981848in}{1.649916in}}{\pgfqpoint{5.973612in}{1.649916in}}%
\pgfpathcurveto{\pgfqpoint{5.965375in}{1.649916in}}{\pgfqpoint{5.957475in}{1.646644in}}{\pgfqpoint{5.951651in}{1.640820in}}%
\pgfpathcurveto{\pgfqpoint{5.945827in}{1.634996in}}{\pgfqpoint{5.942555in}{1.627096in}}{\pgfqpoint{5.942555in}{1.618859in}}%
\pgfpathcurveto{\pgfqpoint{5.942555in}{1.610623in}}{\pgfqpoint{5.945827in}{1.602723in}}{\pgfqpoint{5.951651in}{1.596899in}}%
\pgfpathcurveto{\pgfqpoint{5.957475in}{1.591075in}}{\pgfqpoint{5.965375in}{1.587803in}}{\pgfqpoint{5.973612in}{1.587803in}}%
\pgfpathclose%
\pgfusepath{stroke,fill}%
\end{pgfscope}%
\begin{pgfscope}%
\pgfpathrectangle{\pgfqpoint{3.874381in}{0.557870in}}{\pgfqpoint{2.483906in}{1.684734in}}%
\pgfusepath{clip}%
\pgfsetbuttcap%
\pgfsetroundjoin%
\definecolor{currentfill}{rgb}{0.298039,0.447059,0.690196}%
\pgfsetfillcolor{currentfill}%
\pgfsetlinewidth{1.003750pt}%
\definecolor{currentstroke}{rgb}{0.298039,0.447059,0.690196}%
\pgfsetstrokecolor{currentstroke}%
\pgfsetdash{}{0pt}%
\pgfpathmoveto{\pgfqpoint{5.973612in}{1.587803in}}%
\pgfpathcurveto{\pgfqpoint{5.981848in}{1.587803in}}{\pgfqpoint{5.989748in}{1.591075in}}{\pgfqpoint{5.995572in}{1.596899in}}%
\pgfpathcurveto{\pgfqpoint{6.001396in}{1.602723in}}{\pgfqpoint{6.004668in}{1.610623in}}{\pgfqpoint{6.004668in}{1.618859in}}%
\pgfpathcurveto{\pgfqpoint{6.004668in}{1.627096in}}{\pgfqpoint{6.001396in}{1.634996in}}{\pgfqpoint{5.995572in}{1.640820in}}%
\pgfpathcurveto{\pgfqpoint{5.989748in}{1.646644in}}{\pgfqpoint{5.981848in}{1.649916in}}{\pgfqpoint{5.973612in}{1.649916in}}%
\pgfpathcurveto{\pgfqpoint{5.965375in}{1.649916in}}{\pgfqpoint{5.957475in}{1.646644in}}{\pgfqpoint{5.951651in}{1.640820in}}%
\pgfpathcurveto{\pgfqpoint{5.945827in}{1.634996in}}{\pgfqpoint{5.942555in}{1.627096in}}{\pgfqpoint{5.942555in}{1.618859in}}%
\pgfpathcurveto{\pgfqpoint{5.942555in}{1.610623in}}{\pgfqpoint{5.945827in}{1.602723in}}{\pgfqpoint{5.951651in}{1.596899in}}%
\pgfpathcurveto{\pgfqpoint{5.957475in}{1.591075in}}{\pgfqpoint{5.965375in}{1.587803in}}{\pgfqpoint{5.973612in}{1.587803in}}%
\pgfpathclose%
\pgfusepath{stroke,fill}%
\end{pgfscope}%
\begin{pgfscope}%
\pgfpathrectangle{\pgfqpoint{3.874381in}{0.557870in}}{\pgfqpoint{2.483906in}{1.684734in}}%
\pgfusepath{clip}%
\pgfsetbuttcap%
\pgfsetroundjoin%
\definecolor{currentfill}{rgb}{0.298039,0.447059,0.690196}%
\pgfsetfillcolor{currentfill}%
\pgfsetlinewidth{1.003750pt}%
\definecolor{currentstroke}{rgb}{0.298039,0.447059,0.690196}%
\pgfsetstrokecolor{currentstroke}%
\pgfsetdash{}{0pt}%
\pgfpathmoveto{\pgfqpoint{6.005585in}{1.529992in}}%
\pgfpathcurveto{\pgfqpoint{6.013821in}{1.529992in}}{\pgfqpoint{6.021721in}{1.533264in}}{\pgfqpoint{6.027545in}{1.539088in}}%
\pgfpathcurveto{\pgfqpoint{6.033369in}{1.544912in}}{\pgfqpoint{6.036641in}{1.552812in}}{\pgfqpoint{6.036641in}{1.561049in}}%
\pgfpathcurveto{\pgfqpoint{6.036641in}{1.569285in}}{\pgfqpoint{6.033369in}{1.577185in}}{\pgfqpoint{6.027545in}{1.583009in}}%
\pgfpathcurveto{\pgfqpoint{6.021721in}{1.588833in}}{\pgfqpoint{6.013821in}{1.592105in}}{\pgfqpoint{6.005585in}{1.592105in}}%
\pgfpathcurveto{\pgfqpoint{5.997348in}{1.592105in}}{\pgfqpoint{5.989448in}{1.588833in}}{\pgfqpoint{5.983624in}{1.583009in}}%
\pgfpathcurveto{\pgfqpoint{5.977800in}{1.577185in}}{\pgfqpoint{5.974528in}{1.569285in}}{\pgfqpoint{5.974528in}{1.561049in}}%
\pgfpathcurveto{\pgfqpoint{5.974528in}{1.552812in}}{\pgfqpoint{5.977800in}{1.544912in}}{\pgfqpoint{5.983624in}{1.539088in}}%
\pgfpathcurveto{\pgfqpoint{5.989448in}{1.533264in}}{\pgfqpoint{5.997348in}{1.529992in}}{\pgfqpoint{6.005585in}{1.529992in}}%
\pgfpathclose%
\pgfusepath{stroke,fill}%
\end{pgfscope}%
\begin{pgfscope}%
\pgfpathrectangle{\pgfqpoint{3.874381in}{0.557870in}}{\pgfqpoint{2.483906in}{1.684734in}}%
\pgfusepath{clip}%
\pgfsetbuttcap%
\pgfsetroundjoin%
\definecolor{currentfill}{rgb}{0.298039,0.447059,0.690196}%
\pgfsetfillcolor{currentfill}%
\pgfsetlinewidth{1.003750pt}%
\definecolor{currentstroke}{rgb}{0.298039,0.447059,0.690196}%
\pgfsetstrokecolor{currentstroke}%
\pgfsetdash{}{0pt}%
\pgfpathmoveto{\pgfqpoint{6.005585in}{1.529992in}}%
\pgfpathcurveto{\pgfqpoint{6.013821in}{1.529992in}}{\pgfqpoint{6.021721in}{1.533264in}}{\pgfqpoint{6.027545in}{1.539088in}}%
\pgfpathcurveto{\pgfqpoint{6.033369in}{1.544912in}}{\pgfqpoint{6.036641in}{1.552812in}}{\pgfqpoint{6.036641in}{1.561049in}}%
\pgfpathcurveto{\pgfqpoint{6.036641in}{1.569285in}}{\pgfqpoint{6.033369in}{1.577185in}}{\pgfqpoint{6.027545in}{1.583009in}}%
\pgfpathcurveto{\pgfqpoint{6.021721in}{1.588833in}}{\pgfqpoint{6.013821in}{1.592105in}}{\pgfqpoint{6.005585in}{1.592105in}}%
\pgfpathcurveto{\pgfqpoint{5.997348in}{1.592105in}}{\pgfqpoint{5.989448in}{1.588833in}}{\pgfqpoint{5.983624in}{1.583009in}}%
\pgfpathcurveto{\pgfqpoint{5.977800in}{1.577185in}}{\pgfqpoint{5.974528in}{1.569285in}}{\pgfqpoint{5.974528in}{1.561049in}}%
\pgfpathcurveto{\pgfqpoint{5.974528in}{1.552812in}}{\pgfqpoint{5.977800in}{1.544912in}}{\pgfqpoint{5.983624in}{1.539088in}}%
\pgfpathcurveto{\pgfqpoint{5.989448in}{1.533264in}}{\pgfqpoint{5.997348in}{1.529992in}}{\pgfqpoint{6.005585in}{1.529992in}}%
\pgfpathclose%
\pgfusepath{stroke,fill}%
\end{pgfscope}%
\begin{pgfscope}%
\pgfpathrectangle{\pgfqpoint{3.874381in}{0.557870in}}{\pgfqpoint{2.483906in}{1.684734in}}%
\pgfusepath{clip}%
\pgfsetbuttcap%
\pgfsetroundjoin%
\definecolor{currentfill}{rgb}{0.298039,0.447059,0.690196}%
\pgfsetfillcolor{currentfill}%
\pgfsetlinewidth{1.003750pt}%
\definecolor{currentstroke}{rgb}{0.298039,0.447059,0.690196}%
\pgfsetstrokecolor{currentstroke}%
\pgfsetdash{}{0pt}%
\pgfpathmoveto{\pgfqpoint{6.124151in}{0.988526in}}%
\pgfpathcurveto{\pgfqpoint{6.132388in}{0.988526in}}{\pgfqpoint{6.140288in}{0.991798in}}{\pgfqpoint{6.146112in}{0.997622in}}%
\pgfpathcurveto{\pgfqpoint{6.151935in}{1.003446in}}{\pgfqpoint{6.155208in}{1.011346in}}{\pgfqpoint{6.155208in}{1.019582in}}%
\pgfpathcurveto{\pgfqpoint{6.155208in}{1.027819in}}{\pgfqpoint{6.151935in}{1.035719in}}{\pgfqpoint{6.146112in}{1.041543in}}%
\pgfpathcurveto{\pgfqpoint{6.140288in}{1.047366in}}{\pgfqpoint{6.132388in}{1.050639in}}{\pgfqpoint{6.124151in}{1.050639in}}%
\pgfpathcurveto{\pgfqpoint{6.115915in}{1.050639in}}{\pgfqpoint{6.108015in}{1.047366in}}{\pgfqpoint{6.102191in}{1.041543in}}%
\pgfpathcurveto{\pgfqpoint{6.096367in}{1.035719in}}{\pgfqpoint{6.093095in}{1.027819in}}{\pgfqpoint{6.093095in}{1.019582in}}%
\pgfpathcurveto{\pgfqpoint{6.093095in}{1.011346in}}{\pgfqpoint{6.096367in}{1.003446in}}{\pgfqpoint{6.102191in}{0.997622in}}%
\pgfpathcurveto{\pgfqpoint{6.108015in}{0.991798in}}{\pgfqpoint{6.115915in}{0.988526in}}{\pgfqpoint{6.124151in}{0.988526in}}%
\pgfpathclose%
\pgfusepath{stroke,fill}%
\end{pgfscope}%
\begin{pgfscope}%
\pgfpathrectangle{\pgfqpoint{3.874381in}{0.557870in}}{\pgfqpoint{2.483906in}{1.684734in}}%
\pgfusepath{clip}%
\pgfsetbuttcap%
\pgfsetroundjoin%
\definecolor{currentfill}{rgb}{0.298039,0.447059,0.690196}%
\pgfsetfillcolor{currentfill}%
\pgfsetlinewidth{1.003750pt}%
\definecolor{currentstroke}{rgb}{0.298039,0.447059,0.690196}%
\pgfsetstrokecolor{currentstroke}%
\pgfsetdash{}{0pt}%
\pgfpathmoveto{\pgfqpoint{6.056209in}{1.082163in}}%
\pgfpathcurveto{\pgfqpoint{6.064445in}{1.082163in}}{\pgfqpoint{6.072345in}{1.085435in}}{\pgfqpoint{6.078169in}{1.091259in}}%
\pgfpathcurveto{\pgfqpoint{6.083993in}{1.097083in}}{\pgfqpoint{6.087265in}{1.104983in}}{\pgfqpoint{6.087265in}{1.113219in}}%
\pgfpathcurveto{\pgfqpoint{6.087265in}{1.121456in}}{\pgfqpoint{6.083993in}{1.129356in}}{\pgfqpoint{6.078169in}{1.135180in}}%
\pgfpathcurveto{\pgfqpoint{6.072345in}{1.141004in}}{\pgfqpoint{6.064445in}{1.144276in}}{\pgfqpoint{6.056209in}{1.144276in}}%
\pgfpathcurveto{\pgfqpoint{6.047972in}{1.144276in}}{\pgfqpoint{6.040072in}{1.141004in}}{\pgfqpoint{6.034248in}{1.135180in}}%
\pgfpathcurveto{\pgfqpoint{6.028424in}{1.129356in}}{\pgfqpoint{6.025152in}{1.121456in}}{\pgfqpoint{6.025152in}{1.113219in}}%
\pgfpathcurveto{\pgfqpoint{6.025152in}{1.104983in}}{\pgfqpoint{6.028424in}{1.097083in}}{\pgfqpoint{6.034248in}{1.091259in}}%
\pgfpathcurveto{\pgfqpoint{6.040072in}{1.085435in}}{\pgfqpoint{6.047972in}{1.082163in}}{\pgfqpoint{6.056209in}{1.082163in}}%
\pgfpathclose%
\pgfusepath{stroke,fill}%
\end{pgfscope}%
\begin{pgfscope}%
\pgfpathrectangle{\pgfqpoint{3.874381in}{0.557870in}}{\pgfqpoint{2.483906in}{1.684734in}}%
\pgfusepath{clip}%
\pgfsetbuttcap%
\pgfsetroundjoin%
\definecolor{currentfill}{rgb}{0.298039,0.447059,0.690196}%
\pgfsetfillcolor{currentfill}%
\pgfsetlinewidth{1.003750pt}%
\definecolor{currentstroke}{rgb}{0.298039,0.447059,0.690196}%
\pgfsetstrokecolor{currentstroke}%
\pgfsetdash{}{0pt}%
\pgfpathmoveto{\pgfqpoint{5.384775in}{1.353303in}}%
\pgfpathcurveto{\pgfqpoint{5.393011in}{1.353303in}}{\pgfqpoint{5.400911in}{1.356575in}}{\pgfqpoint{5.406735in}{1.362399in}}%
\pgfpathcurveto{\pgfqpoint{5.412559in}{1.368223in}}{\pgfqpoint{5.415831in}{1.376123in}}{\pgfqpoint{5.415831in}{1.384360in}}%
\pgfpathcurveto{\pgfqpoint{5.415831in}{1.392596in}}{\pgfqpoint{5.412559in}{1.400496in}}{\pgfqpoint{5.406735in}{1.406320in}}%
\pgfpathcurveto{\pgfqpoint{5.400911in}{1.412144in}}{\pgfqpoint{5.393011in}{1.415416in}}{\pgfqpoint{5.384775in}{1.415416in}}%
\pgfpathcurveto{\pgfqpoint{5.376538in}{1.415416in}}{\pgfqpoint{5.368638in}{1.412144in}}{\pgfqpoint{5.362814in}{1.406320in}}%
\pgfpathcurveto{\pgfqpoint{5.356990in}{1.400496in}}{\pgfqpoint{5.353718in}{1.392596in}}{\pgfqpoint{5.353718in}{1.384360in}}%
\pgfpathcurveto{\pgfqpoint{5.353718in}{1.376123in}}{\pgfqpoint{5.356990in}{1.368223in}}{\pgfqpoint{5.362814in}{1.362399in}}%
\pgfpathcurveto{\pgfqpoint{5.368638in}{1.356575in}}{\pgfqpoint{5.376538in}{1.353303in}}{\pgfqpoint{5.384775in}{1.353303in}}%
\pgfpathclose%
\pgfusepath{stroke,fill}%
\end{pgfscope}%
\begin{pgfscope}%
\pgfpathrectangle{\pgfqpoint{3.874381in}{0.557870in}}{\pgfqpoint{2.483906in}{1.684734in}}%
\pgfusepath{clip}%
\pgfsetbuttcap%
\pgfsetroundjoin%
\definecolor{currentfill}{rgb}{0.298039,0.447059,0.690196}%
\pgfsetfillcolor{currentfill}%
\pgfsetlinewidth{1.003750pt}%
\definecolor{currentstroke}{rgb}{0.298039,0.447059,0.690196}%
\pgfsetstrokecolor{currentstroke}%
\pgfsetdash{}{0pt}%
\pgfpathmoveto{\pgfqpoint{5.113004in}{1.514522in}}%
\pgfpathcurveto{\pgfqpoint{5.121240in}{1.514522in}}{\pgfqpoint{5.129140in}{1.517794in}}{\pgfqpoint{5.134964in}{1.523618in}}%
\pgfpathcurveto{\pgfqpoint{5.140788in}{1.529442in}}{\pgfqpoint{5.144060in}{1.537342in}}{\pgfqpoint{5.144060in}{1.545578in}}%
\pgfpathcurveto{\pgfqpoint{5.144060in}{1.553814in}}{\pgfqpoint{5.140788in}{1.561715in}}{\pgfqpoint{5.134964in}{1.567538in}}%
\pgfpathcurveto{\pgfqpoint{5.129140in}{1.573362in}}{\pgfqpoint{5.121240in}{1.576635in}}{\pgfqpoint{5.113004in}{1.576635in}}%
\pgfpathcurveto{\pgfqpoint{5.104767in}{1.576635in}}{\pgfqpoint{5.096867in}{1.573362in}}{\pgfqpoint{5.091043in}{1.567538in}}%
\pgfpathcurveto{\pgfqpoint{5.085220in}{1.561715in}}{\pgfqpoint{5.081947in}{1.553814in}}{\pgfqpoint{5.081947in}{1.545578in}}%
\pgfpathcurveto{\pgfqpoint{5.081947in}{1.537342in}}{\pgfqpoint{5.085220in}{1.529442in}}{\pgfqpoint{5.091043in}{1.523618in}}%
\pgfpathcurveto{\pgfqpoint{5.096867in}{1.517794in}}{\pgfqpoint{5.104767in}{1.514522in}}{\pgfqpoint{5.113004in}{1.514522in}}%
\pgfpathclose%
\pgfusepath{stroke,fill}%
\end{pgfscope}%
\begin{pgfscope}%
\pgfpathrectangle{\pgfqpoint{3.874381in}{0.557870in}}{\pgfqpoint{2.483906in}{1.684734in}}%
\pgfusepath{clip}%
\pgfsetbuttcap%
\pgfsetroundjoin%
\definecolor{currentfill}{rgb}{0.298039,0.447059,0.690196}%
\pgfsetfillcolor{currentfill}%
\pgfsetlinewidth{1.003750pt}%
\definecolor{currentstroke}{rgb}{0.298039,0.447059,0.690196}%
\pgfsetstrokecolor{currentstroke}%
\pgfsetdash{}{0pt}%
\pgfpathmoveto{\pgfqpoint{4.064554in}{2.036446in}}%
\pgfpathcurveto{\pgfqpoint{4.072791in}{2.036446in}}{\pgfqpoint{4.080691in}{2.039719in}}{\pgfqpoint{4.086515in}{2.045543in}}%
\pgfpathcurveto{\pgfqpoint{4.092339in}{2.051367in}}{\pgfqpoint{4.095611in}{2.059267in}}{\pgfqpoint{4.095611in}{2.067503in}}%
\pgfpathcurveto{\pgfqpoint{4.095611in}{2.075739in}}{\pgfqpoint{4.092339in}{2.083639in}}{\pgfqpoint{4.086515in}{2.089463in}}%
\pgfpathcurveto{\pgfqpoint{4.080691in}{2.095287in}}{\pgfqpoint{4.072791in}{2.098559in}}{\pgfqpoint{4.064554in}{2.098559in}}%
\pgfpathcurveto{\pgfqpoint{4.056318in}{2.098559in}}{\pgfqpoint{4.048418in}{2.095287in}}{\pgfqpoint{4.042594in}{2.089463in}}%
\pgfpathcurveto{\pgfqpoint{4.036770in}{2.083639in}}{\pgfqpoint{4.033498in}{2.075739in}}{\pgfqpoint{4.033498in}{2.067503in}}%
\pgfpathcurveto{\pgfqpoint{4.033498in}{2.059267in}}{\pgfqpoint{4.036770in}{2.051367in}}{\pgfqpoint{4.042594in}{2.045543in}}%
\pgfpathcurveto{\pgfqpoint{4.048418in}{2.039719in}}{\pgfqpoint{4.056318in}{2.036446in}}{\pgfqpoint{4.064554in}{2.036446in}}%
\pgfpathclose%
\pgfusepath{stroke,fill}%
\end{pgfscope}%
\begin{pgfscope}%
\pgfpathrectangle{\pgfqpoint{3.874381in}{0.557870in}}{\pgfqpoint{2.483906in}{1.684734in}}%
\pgfusepath{clip}%
\pgfsetbuttcap%
\pgfsetroundjoin%
\definecolor{currentfill}{rgb}{0.298039,0.447059,0.690196}%
\pgfsetfillcolor{currentfill}%
\pgfsetlinewidth{1.003750pt}%
\definecolor{currentstroke}{rgb}{0.298039,0.447059,0.690196}%
\pgfsetstrokecolor{currentstroke}%
\pgfsetdash{}{0pt}%
\pgfpathmoveto{\pgfqpoint{3.992615in}{2.084486in}}%
\pgfpathcurveto{\pgfqpoint{4.000851in}{2.084486in}}{\pgfqpoint{4.008751in}{2.087759in}}{\pgfqpoint{4.014575in}{2.093583in}}%
\pgfpathcurveto{\pgfqpoint{4.020399in}{2.099406in}}{\pgfqpoint{4.023671in}{2.107306in}}{\pgfqpoint{4.023671in}{2.115543in}}%
\pgfpathcurveto{\pgfqpoint{4.023671in}{2.123779in}}{\pgfqpoint{4.020399in}{2.131679in}}{\pgfqpoint{4.014575in}{2.137503in}}%
\pgfpathcurveto{\pgfqpoint{4.008751in}{2.143327in}}{\pgfqpoint{4.000851in}{2.146599in}}{\pgfqpoint{3.992615in}{2.146599in}}%
\pgfpathcurveto{\pgfqpoint{3.984379in}{2.146599in}}{\pgfqpoint{3.976479in}{2.143327in}}{\pgfqpoint{3.970655in}{2.137503in}}%
\pgfpathcurveto{\pgfqpoint{3.964831in}{2.131679in}}{\pgfqpoint{3.961558in}{2.123779in}}{\pgfqpoint{3.961558in}{2.115543in}}%
\pgfpathcurveto{\pgfqpoint{3.961558in}{2.107306in}}{\pgfqpoint{3.964831in}{2.099406in}}{\pgfqpoint{3.970655in}{2.093583in}}%
\pgfpathcurveto{\pgfqpoint{3.976479in}{2.087759in}}{\pgfqpoint{3.984379in}{2.084486in}}{\pgfqpoint{3.992615in}{2.084486in}}%
\pgfpathclose%
\pgfusepath{stroke,fill}%
\end{pgfscope}%
\begin{pgfscope}%
\pgfpathrectangle{\pgfqpoint{3.874381in}{0.557870in}}{\pgfqpoint{2.483906in}{1.684734in}}%
\pgfusepath{clip}%
\pgfsetbuttcap%
\pgfsetroundjoin%
\definecolor{currentfill}{rgb}{0.298039,0.447059,0.690196}%
\pgfsetfillcolor{currentfill}%
\pgfsetlinewidth{1.003750pt}%
\definecolor{currentstroke}{rgb}{0.298039,0.447059,0.690196}%
\pgfsetstrokecolor{currentstroke}%
\pgfsetdash{}{0pt}%
\pgfpathmoveto{\pgfqpoint{3.987286in}{2.133340in}}%
\pgfpathcurveto{\pgfqpoint{3.995522in}{2.133340in}}{\pgfqpoint{4.003422in}{2.136613in}}{\pgfqpoint{4.009246in}{2.142437in}}%
\pgfpathcurveto{\pgfqpoint{4.015070in}{2.148261in}}{\pgfqpoint{4.018343in}{2.156161in}}{\pgfqpoint{4.018343in}{2.164397in}}%
\pgfpathcurveto{\pgfqpoint{4.018343in}{2.172633in}}{\pgfqpoint{4.015070in}{2.180533in}}{\pgfqpoint{4.009246in}{2.186357in}}%
\pgfpathcurveto{\pgfqpoint{4.003422in}{2.192181in}}{\pgfqpoint{3.995522in}{2.195453in}}{\pgfqpoint{3.987286in}{2.195453in}}%
\pgfpathcurveto{\pgfqpoint{3.979050in}{2.195453in}}{\pgfqpoint{3.971150in}{2.192181in}}{\pgfqpoint{3.965326in}{2.186357in}}%
\pgfpathcurveto{\pgfqpoint{3.959502in}{2.180533in}}{\pgfqpoint{3.956230in}{2.172633in}}{\pgfqpoint{3.956230in}{2.164397in}}%
\pgfpathcurveto{\pgfqpoint{3.956230in}{2.156161in}}{\pgfqpoint{3.959502in}{2.148261in}}{\pgfqpoint{3.965326in}{2.142437in}}%
\pgfpathcurveto{\pgfqpoint{3.971150in}{2.136613in}}{\pgfqpoint{3.979050in}{2.133340in}}{\pgfqpoint{3.987286in}{2.133340in}}%
\pgfpathclose%
\pgfusepath{stroke,fill}%
\end{pgfscope}%
\begin{pgfscope}%
\pgfpathrectangle{\pgfqpoint{3.874381in}{0.557870in}}{\pgfqpoint{2.483906in}{1.684734in}}%
\pgfusepath{clip}%
\pgfsetbuttcap%
\pgfsetroundjoin%
\definecolor{currentfill}{rgb}{0.298039,0.447059,0.690196}%
\pgfsetfillcolor{currentfill}%
\pgfsetlinewidth{1.003750pt}%
\definecolor{currentstroke}{rgb}{0.298039,0.447059,0.690196}%
\pgfsetstrokecolor{currentstroke}%
\pgfsetdash{}{0pt}%
\pgfpathmoveto{\pgfqpoint{3.987286in}{2.133340in}}%
\pgfpathcurveto{\pgfqpoint{3.995522in}{2.133340in}}{\pgfqpoint{4.003422in}{2.136613in}}{\pgfqpoint{4.009246in}{2.142437in}}%
\pgfpathcurveto{\pgfqpoint{4.015070in}{2.148261in}}{\pgfqpoint{4.018343in}{2.156161in}}{\pgfqpoint{4.018343in}{2.164397in}}%
\pgfpathcurveto{\pgfqpoint{4.018343in}{2.172633in}}{\pgfqpoint{4.015070in}{2.180533in}}{\pgfqpoint{4.009246in}{2.186357in}}%
\pgfpathcurveto{\pgfqpoint{4.003422in}{2.192181in}}{\pgfqpoint{3.995522in}{2.195453in}}{\pgfqpoint{3.987286in}{2.195453in}}%
\pgfpathcurveto{\pgfqpoint{3.979050in}{2.195453in}}{\pgfqpoint{3.971150in}{2.192181in}}{\pgfqpoint{3.965326in}{2.186357in}}%
\pgfpathcurveto{\pgfqpoint{3.959502in}{2.180533in}}{\pgfqpoint{3.956230in}{2.172633in}}{\pgfqpoint{3.956230in}{2.164397in}}%
\pgfpathcurveto{\pgfqpoint{3.956230in}{2.156161in}}{\pgfqpoint{3.959502in}{2.148261in}}{\pgfqpoint{3.965326in}{2.142437in}}%
\pgfpathcurveto{\pgfqpoint{3.971150in}{2.136613in}}{\pgfqpoint{3.979050in}{2.133340in}}{\pgfqpoint{3.987286in}{2.133340in}}%
\pgfpathclose%
\pgfusepath{stroke,fill}%
\end{pgfscope}%
\begin{pgfscope}%
\pgfpathrectangle{\pgfqpoint{3.874381in}{0.557870in}}{\pgfqpoint{2.483906in}{1.684734in}}%
\pgfusepath{clip}%
\pgfsetbuttcap%
\pgfsetroundjoin%
\definecolor{currentfill}{rgb}{0.298039,0.447059,0.690196}%
\pgfsetfillcolor{currentfill}%
\pgfsetlinewidth{1.003750pt}%
\definecolor{currentstroke}{rgb}{0.298039,0.447059,0.690196}%
\pgfsetstrokecolor{currentstroke}%
\pgfsetdash{}{0pt}%
\pgfpathmoveto{\pgfqpoint{3.987286in}{2.132526in}}%
\pgfpathcurveto{\pgfqpoint{3.995522in}{2.132526in}}{\pgfqpoint{4.003422in}{2.135798in}}{\pgfqpoint{4.009246in}{2.141622in}}%
\pgfpathcurveto{\pgfqpoint{4.015070in}{2.147446in}}{\pgfqpoint{4.018343in}{2.155346in}}{\pgfqpoint{4.018343in}{2.163583in}}%
\pgfpathcurveto{\pgfqpoint{4.018343in}{2.171819in}}{\pgfqpoint{4.015070in}{2.179719in}}{\pgfqpoint{4.009246in}{2.185543in}}%
\pgfpathcurveto{\pgfqpoint{4.003422in}{2.191367in}}{\pgfqpoint{3.995522in}{2.194639in}}{\pgfqpoint{3.987286in}{2.194639in}}%
\pgfpathcurveto{\pgfqpoint{3.979050in}{2.194639in}}{\pgfqpoint{3.971150in}{2.191367in}}{\pgfqpoint{3.965326in}{2.185543in}}%
\pgfpathcurveto{\pgfqpoint{3.959502in}{2.179719in}}{\pgfqpoint{3.956230in}{2.171819in}}{\pgfqpoint{3.956230in}{2.163583in}}%
\pgfpathcurveto{\pgfqpoint{3.956230in}{2.155346in}}{\pgfqpoint{3.959502in}{2.147446in}}{\pgfqpoint{3.965326in}{2.141622in}}%
\pgfpathcurveto{\pgfqpoint{3.971150in}{2.135798in}}{\pgfqpoint{3.979050in}{2.132526in}}{\pgfqpoint{3.987286in}{2.132526in}}%
\pgfpathclose%
\pgfusepath{stroke,fill}%
\end{pgfscope}%
\begin{pgfscope}%
\pgfpathrectangle{\pgfqpoint{3.874381in}{0.557870in}}{\pgfqpoint{2.483906in}{1.684734in}}%
\pgfusepath{clip}%
\pgfsetbuttcap%
\pgfsetroundjoin%
\definecolor{currentfill}{rgb}{0.298039,0.447059,0.690196}%
\pgfsetfillcolor{currentfill}%
\pgfsetlinewidth{1.003750pt}%
\definecolor{currentstroke}{rgb}{0.298039,0.447059,0.690196}%
\pgfsetstrokecolor{currentstroke}%
\pgfsetdash{}{0pt}%
\pgfpathmoveto{\pgfqpoint{3.987286in}{2.132526in}}%
\pgfpathcurveto{\pgfqpoint{3.995522in}{2.132526in}}{\pgfqpoint{4.003422in}{2.135798in}}{\pgfqpoint{4.009246in}{2.141622in}}%
\pgfpathcurveto{\pgfqpoint{4.015070in}{2.147446in}}{\pgfqpoint{4.018343in}{2.155346in}}{\pgfqpoint{4.018343in}{2.163583in}}%
\pgfpathcurveto{\pgfqpoint{4.018343in}{2.171819in}}{\pgfqpoint{4.015070in}{2.179719in}}{\pgfqpoint{4.009246in}{2.185543in}}%
\pgfpathcurveto{\pgfqpoint{4.003422in}{2.191367in}}{\pgfqpoint{3.995522in}{2.194639in}}{\pgfqpoint{3.987286in}{2.194639in}}%
\pgfpathcurveto{\pgfqpoint{3.979050in}{2.194639in}}{\pgfqpoint{3.971150in}{2.191367in}}{\pgfqpoint{3.965326in}{2.185543in}}%
\pgfpathcurveto{\pgfqpoint{3.959502in}{2.179719in}}{\pgfqpoint{3.956230in}{2.171819in}}{\pgfqpoint{3.956230in}{2.163583in}}%
\pgfpathcurveto{\pgfqpoint{3.956230in}{2.155346in}}{\pgfqpoint{3.959502in}{2.147446in}}{\pgfqpoint{3.965326in}{2.141622in}}%
\pgfpathcurveto{\pgfqpoint{3.971150in}{2.135798in}}{\pgfqpoint{3.979050in}{2.132526in}}{\pgfqpoint{3.987286in}{2.132526in}}%
\pgfpathclose%
\pgfusepath{stroke,fill}%
\end{pgfscope}%
\begin{pgfscope}%
\pgfpathrectangle{\pgfqpoint{3.874381in}{0.557870in}}{\pgfqpoint{2.483906in}{1.684734in}}%
\pgfusepath{clip}%
\pgfsetbuttcap%
\pgfsetroundjoin%
\definecolor{currentfill}{rgb}{0.298039,0.447059,0.690196}%
\pgfsetfillcolor{currentfill}%
\pgfsetlinewidth{1.003750pt}%
\definecolor{currentstroke}{rgb}{0.298039,0.447059,0.690196}%
\pgfsetstrokecolor{currentstroke}%
\pgfsetdash{}{0pt}%
\pgfpathmoveto{\pgfqpoint{3.987286in}{2.131712in}}%
\pgfpathcurveto{\pgfqpoint{3.995522in}{2.131712in}}{\pgfqpoint{4.003422in}{2.134984in}}{\pgfqpoint{4.009246in}{2.140808in}}%
\pgfpathcurveto{\pgfqpoint{4.015070in}{2.146632in}}{\pgfqpoint{4.018343in}{2.154532in}}{\pgfqpoint{4.018343in}{2.162768in}}%
\pgfpathcurveto{\pgfqpoint{4.018343in}{2.171005in}}{\pgfqpoint{4.015070in}{2.178905in}}{\pgfqpoint{4.009246in}{2.184729in}}%
\pgfpathcurveto{\pgfqpoint{4.003422in}{2.190553in}}{\pgfqpoint{3.995522in}{2.193825in}}{\pgfqpoint{3.987286in}{2.193825in}}%
\pgfpathcurveto{\pgfqpoint{3.979050in}{2.193825in}}{\pgfqpoint{3.971150in}{2.190553in}}{\pgfqpoint{3.965326in}{2.184729in}}%
\pgfpathcurveto{\pgfqpoint{3.959502in}{2.178905in}}{\pgfqpoint{3.956230in}{2.171005in}}{\pgfqpoint{3.956230in}{2.162768in}}%
\pgfpathcurveto{\pgfqpoint{3.956230in}{2.154532in}}{\pgfqpoint{3.959502in}{2.146632in}}{\pgfqpoint{3.965326in}{2.140808in}}%
\pgfpathcurveto{\pgfqpoint{3.971150in}{2.134984in}}{\pgfqpoint{3.979050in}{2.131712in}}{\pgfqpoint{3.987286in}{2.131712in}}%
\pgfpathclose%
\pgfusepath{stroke,fill}%
\end{pgfscope}%
\begin{pgfscope}%
\pgfpathrectangle{\pgfqpoint{3.874381in}{0.557870in}}{\pgfqpoint{2.483906in}{1.684734in}}%
\pgfusepath{clip}%
\pgfsetbuttcap%
\pgfsetroundjoin%
\definecolor{currentfill}{rgb}{0.298039,0.447059,0.690196}%
\pgfsetfillcolor{currentfill}%
\pgfsetlinewidth{1.003750pt}%
\definecolor{currentstroke}{rgb}{0.298039,0.447059,0.690196}%
\pgfsetstrokecolor{currentstroke}%
\pgfsetdash{}{0pt}%
\pgfpathmoveto{\pgfqpoint{3.987286in}{2.131712in}}%
\pgfpathcurveto{\pgfqpoint{3.995522in}{2.131712in}}{\pgfqpoint{4.003422in}{2.134984in}}{\pgfqpoint{4.009246in}{2.140808in}}%
\pgfpathcurveto{\pgfqpoint{4.015070in}{2.146632in}}{\pgfqpoint{4.018343in}{2.154532in}}{\pgfqpoint{4.018343in}{2.162768in}}%
\pgfpathcurveto{\pgfqpoint{4.018343in}{2.171005in}}{\pgfqpoint{4.015070in}{2.178905in}}{\pgfqpoint{4.009246in}{2.184729in}}%
\pgfpathcurveto{\pgfqpoint{4.003422in}{2.190553in}}{\pgfqpoint{3.995522in}{2.193825in}}{\pgfqpoint{3.987286in}{2.193825in}}%
\pgfpathcurveto{\pgfqpoint{3.979050in}{2.193825in}}{\pgfqpoint{3.971150in}{2.190553in}}{\pgfqpoint{3.965326in}{2.184729in}}%
\pgfpathcurveto{\pgfqpoint{3.959502in}{2.178905in}}{\pgfqpoint{3.956230in}{2.171005in}}{\pgfqpoint{3.956230in}{2.162768in}}%
\pgfpathcurveto{\pgfqpoint{3.956230in}{2.154532in}}{\pgfqpoint{3.959502in}{2.146632in}}{\pgfqpoint{3.965326in}{2.140808in}}%
\pgfpathcurveto{\pgfqpoint{3.971150in}{2.134984in}}{\pgfqpoint{3.979050in}{2.131712in}}{\pgfqpoint{3.987286in}{2.131712in}}%
\pgfpathclose%
\pgfusepath{stroke,fill}%
\end{pgfscope}%
\begin{pgfscope}%
\pgfpathrectangle{\pgfqpoint{3.874381in}{0.557870in}}{\pgfqpoint{2.483906in}{1.684734in}}%
\pgfusepath{clip}%
\pgfsetbuttcap%
\pgfsetroundjoin%
\definecolor{currentfill}{rgb}{0.298039,0.447059,0.690196}%
\pgfsetfillcolor{currentfill}%
\pgfsetlinewidth{1.003750pt}%
\definecolor{currentstroke}{rgb}{0.298039,0.447059,0.690196}%
\pgfsetstrokecolor{currentstroke}%
\pgfsetdash{}{0pt}%
\pgfpathmoveto{\pgfqpoint{3.987286in}{2.127641in}}%
\pgfpathcurveto{\pgfqpoint{3.995522in}{2.127641in}}{\pgfqpoint{4.003422in}{2.130913in}}{\pgfqpoint{4.009246in}{2.136737in}}%
\pgfpathcurveto{\pgfqpoint{4.015070in}{2.142561in}}{\pgfqpoint{4.018343in}{2.150461in}}{\pgfqpoint{4.018343in}{2.158697in}}%
\pgfpathcurveto{\pgfqpoint{4.018343in}{2.166934in}}{\pgfqpoint{4.015070in}{2.174834in}}{\pgfqpoint{4.009246in}{2.180657in}}%
\pgfpathcurveto{\pgfqpoint{4.003422in}{2.186481in}}{\pgfqpoint{3.995522in}{2.189754in}}{\pgfqpoint{3.987286in}{2.189754in}}%
\pgfpathcurveto{\pgfqpoint{3.979050in}{2.189754in}}{\pgfqpoint{3.971150in}{2.186481in}}{\pgfqpoint{3.965326in}{2.180657in}}%
\pgfpathcurveto{\pgfqpoint{3.959502in}{2.174834in}}{\pgfqpoint{3.956230in}{2.166934in}}{\pgfqpoint{3.956230in}{2.158697in}}%
\pgfpathcurveto{\pgfqpoint{3.956230in}{2.150461in}}{\pgfqpoint{3.959502in}{2.142561in}}{\pgfqpoint{3.965326in}{2.136737in}}%
\pgfpathcurveto{\pgfqpoint{3.971150in}{2.130913in}}{\pgfqpoint{3.979050in}{2.127641in}}{\pgfqpoint{3.987286in}{2.127641in}}%
\pgfpathclose%
\pgfusepath{stroke,fill}%
\end{pgfscope}%
\begin{pgfscope}%
\pgfsetrectcap%
\pgfsetmiterjoin%
\pgfsetlinewidth{1.254687pt}%
\definecolor{currentstroke}{rgb}{1.000000,1.000000,1.000000}%
\pgfsetstrokecolor{currentstroke}%
\pgfsetdash{}{0pt}%
\pgfpathmoveto{\pgfqpoint{3.874381in}{0.557870in}}%
\pgfpathlineto{\pgfqpoint{3.874381in}{2.242604in}}%
\pgfusepath{stroke}%
\end{pgfscope}%
\begin{pgfscope}%
\pgfsetrectcap%
\pgfsetmiterjoin%
\pgfsetlinewidth{1.254687pt}%
\definecolor{currentstroke}{rgb}{1.000000,1.000000,1.000000}%
\pgfsetstrokecolor{currentstroke}%
\pgfsetdash{}{0pt}%
\pgfpathmoveto{\pgfqpoint{6.358287in}{0.557870in}}%
\pgfpathlineto{\pgfqpoint{6.358287in}{2.242604in}}%
\pgfusepath{stroke}%
\end{pgfscope}%
\begin{pgfscope}%
\pgfsetrectcap%
\pgfsetmiterjoin%
\pgfsetlinewidth{1.254687pt}%
\definecolor{currentstroke}{rgb}{1.000000,1.000000,1.000000}%
\pgfsetstrokecolor{currentstroke}%
\pgfsetdash{}{0pt}%
\pgfpathmoveto{\pgfqpoint{3.874381in}{0.557870in}}%
\pgfpathlineto{\pgfqpoint{6.358287in}{0.557870in}}%
\pgfusepath{stroke}%
\end{pgfscope}%
\begin{pgfscope}%
\pgfsetrectcap%
\pgfsetmiterjoin%
\pgfsetlinewidth{1.254687pt}%
\definecolor{currentstroke}{rgb}{1.000000,1.000000,1.000000}%
\pgfsetstrokecolor{currentstroke}%
\pgfsetdash{}{0pt}%
\pgfpathmoveto{\pgfqpoint{3.874381in}{2.242604in}}%
\pgfpathlineto{\pgfqpoint{6.358287in}{2.242604in}}%
\pgfusepath{stroke}%
\end{pgfscope}%
\begin{pgfscope}%
\definecolor{textcolor}{rgb}{0.150000,0.150000,0.150000}%
\pgfsetstrokecolor{textcolor}%
\pgfsetfillcolor{textcolor}%
\pgftext[x=5.116334in,y=2.325938in,,base]{\color{textcolor}\sffamily\fontsize{11.000000}{13.200000}\selectfont (b)}%
\end{pgfscope}%
\end{pgfpicture}%
\makeatother%
\endgroup%

    % \includegraphics[width=\textwidth]{results/tsc_segm_ind_dor_sens_spec_dist.png}
    \caption{Distribution of \acrshort{dor}, sensitivity and specificity for the different \acrshort{tsc} models when classifying left ventricle segment indication.}
    \label{fig:tsc_segm_ind_dor_sens_spec_dist}
\end{figure}

\begin{table*}[htb]
    \centering
    \ra{1.3}
    \begin{tabular}{lrrrr}
        \toprule
        Dataset-model     &  Accuracy &  Sensitivity &  Specificity &  \acrshort{dor} \\
        \midrule
        regular/weighted/2 &      0.69 &         0.45 &         0.95 & 15.63 \\
        scaled/weighted/2  &      0.69 &         0.45 &         0.95 & 15.63 \\
        regular/ward/2     &      0.77 &         0.66 &         0.88 & 14.26 \\
        scaled/ward/2      &      0.77 &         0.66 &         0.88 & 14.26 \\
        regular/complete/2 &      0.75 &         0.62 &         0.89 & 13.92 \\
        \bottomrule
    \end{tabular}
    \caption{The accuracy, \acrshort{dor}, sensitivity and specicity scores of the five best performing two-cluster-center \acrshort{tsc} models in terms of \acrshort{dor}, at detecting segment indication.
             The \textbf{Dataset-model} column indicates \textit{Type of preprocessing used}$/$\textit{Linkage criteria of model}$/$\textit{Number of cluster centers}.}
    \label{tab:tsc_segm_ind_dor_sens_spec_dist}
\end{table*}

From the distribution plot in figure \ref{fig:tsc_segm_ind_dor_sens_spec_dist}a one can see that the majority of the \acrshort{dor} are close to zero, but a few models are able to achieve \acrshort{dor} above 12, and some models attain a \acrshort{dor} close to 15 when applied to identify segment indication. From the scatter plot in figure \ref{fig:tsc_segm_ind_dor_sens_spec_dist}b one can see that the sensitivity of the \acrshort{tsc} models range from $0.25$ to 1, and the specificity of the \acrshort{tsc} models range from 0 to approximately 1. The spread in both sensitivity and specificity is quite large, and there are very few models that are able to a attain a high sensitivity while at the same time attaining a high specificity, and vice versa. Common to the high performing \acrshort{tsc} models in terms of \acrshort{dor} is that they all use either no preprocessing at all, or scaling. \textit{z-norm/complete/2} is the seventh best \acrshort{tsc} model in terms of \acrshort{dor}, and attains a \acrshort{dor} of 5.92 when applied to identify segment indication. \textit{norm/ward/2} is the ninth best models in terms of \acrshort{dor}, and attains a \acrshort{dor} of 1.56, when applied to identify segment indication. This can be comfirmed from table \ref{tab:tsc_segm_ind_raw_results}. The two \acrshort{tsc} models attaining the highest \acrshort{dor} \textit{regular/weighted/2}, and \textit{scaled/weighted/2} differ only in type of preprocessing used. From table \ref{tab:tsc_segm_ind_dor_sens_spec_dist} and table \ref{tab:tsc_segm_ind_raw_results} one can see that the two models attain the same scores in all metrics, this is because they yield the exact same cluster assignments to the individual segment strain curves. The same goes for the next two \acrshort{tsc} models in line \textit{regular/ward/2} \textit{scaled/ward/2}, these two models are also the models that attain the highest accarcy of all the \acrshort{tsc} models. Of the two \acrshort{tsc} models \textit{regular/weighted/2}, and \textit{regular/ward/2} the latter is preferred for predicting segment indication because \textit{regular/ward/2} has a more persistent performance in both sensitivity and specificity, where as \textit{regular/weighted/2} has a high specificity, but a very low sensitivity.

\begin{table*}
    \centering
    \ra{1.3}
    \begin{tabular}{lr}
        \toprule
        Dataset-model     &  \acrshort{ari} \\
        \midrule
        scaled/centroid/5  & 0.286 \\
        regular/centroid/5 & 0.286 \\
        regular/ward/2     & 0.284 \\
        scaled/ward/2      & 0.284 \\
        scaled/centroid/6  & 0.271 \\
        \bottomrule
    \end{tabular}
    \caption{The five highest \acrshort{ari} scores attained when applying \acrshort{tsc} for detecting segmend indication.
             The \textbf{Dataset-model} column indicates \textit{Type of preprocessing used}$/$\textit{Linkage criteria of model}$/$\textit{Number of cluster centers}.}
    \label{tab:tsc_segm_ind_ari}
\end{table*}

The majority of the \acrshort{ari} of \acrshort{tsc} models applied to identify segment indication, but as one can see from table \ref{tab:tsc_segm_ind_ari} some models are able to attain \acrshort{ari} above 25. As with the other case studies, the \acrshort{tsc} models that attain the highest \acrshort{ari} are models that use either no preprocessing at all or scaling. Puzzlingly enough the top two \acrshort{tsc} models for classifying segment indication in terms of \acrshort{ari}, are models evaluated at five cluster centers, not two. TSC models \textit{scaled/centroid/5}, and \textit{regular/centroid/5} differ only in type of preprocessing used, and they yield the exact same cluster assignments, and evaluations scores. The next two models in order of \acrshort{ari} \textit{regular/ward/2}, and \textit{scaled/ward/2} are familiar from the list of \acrshort{tsc} models attaining the highest \acrshort{dor} when applied to identify segment indication. From table \ref{tab:tsc_segm_ind_ari} one can also see that the difference in \acrshort{ari} between \textit{regular/centroid/5}, and \textit{regular/ward/2} is only 0.002 Since the \textit{regular/ward/2} model will be considered the best of the \acrshort{tsc} models at classifying segment indication. It attains the third highest \acrshort{ari} of all the \acrshort{tsc} models applied to identify segment indication, and is the preferred model among the \acrshort{tsc} models evaluated at two cluster centers.

\newpage

\subsection{Artificial Neural Network}

\begin{table*}[htb]
    \centering
    \ra{1.3}
    \begin{tabular}{lrrrr}
        \toprule
        model      &  Accuracy &  Sensitivity &  Specificity &  \acrshort{dor} \\
        \midrule
        regular     &      0.74 &         0.80 &         0.68 & 8.65 \\
        downsampled &      0.74 &         0.74 &         0.75 & 8.38 \\
        upsampled   &      0.65 &         0.55 &         0.73 & 3.36 \\
        \bottomrule
    \end{tabular}
    \caption{Evaluation metrics of the \acrshort{ann} for classifying the binary indication of individual segments in the left ventricle.}
    \label{tab:ANN_segm_ind_perf}
\end{table*}

Of the three variations of the \acrshort{ann} model, the one that uses no resampling, and the one that downsamples all signals to the lowest sample rate achieve relatively similar \acrshort{dor} scores. The variation that upsamples the sample rate of all the curves to the highest sample rate performs significantly worse than the other two in terms of \acrshort{dor} and sensitivity. Of the three variations the model that uses downsampling is the preferred model of the three since its sensitivity and specificity are more balanced than the model that uses no resampling, and accuracy is higher than the model that uses upsampling.

\subsection{Comparisons}

\begin{table*}
    \centering
    \ra{1.3}
    \begin{tabular}{lcccc}
        \toprule
        Dataset-model               & Accuracy & Sensitivity & Specificity & \acrshort{dor} \\
        \midrule
        \textbf{TSC}-regular/ward/2 &     0.76 &        0.64 &        0.88 & 13.15 \\
        \textbf{ANN}-downsampled    &     0.74 &        0.74 &        0.75 & 8.38 \\
        \midrule
        Dataset-model               &  TP  &  TN  &  FP  &  FN \\
        \midrule
        \textbf{TSC}-regular/ward/2 & 1202 & 1491 &  204 &  616 \\
        \textbf{ANN}-downsampled    & 1255 & 1390 &  473 &  440 \\
        \bottomrule
    \end{tabular}
    \caption{A table comparing the best contenders within each model group for predicting segment indication. 
             The top table compare the models by their accuracy, sensitivity, specificity and \acrshort{dor}, 
             and the bottom table shows the number of TPs, TNs, FPs and FNs that the different models attain.}
    \label{tab:segm_ind_compare}
\end{table*}

From table \ref{tab:segm_ind_compare} one can see that the performances of the \acrshort{ann}, and \acrshort{tsc} models are quite close in terms of accuracy, but differ significantly in the other metrics. The \acrshort{tsc} model \textit{regular/ward/2} attains a higher accuracy, specificity and \acrshort{dor} than the \acrshort{ann} model \textit{downsampled}. This can also be confirmed by the fact that the \acrshort{tsc} model attains more TN, and fewer FP than the \acrshort{ann} model.  The \acrshort{ann} model attains the highest sensitivity, which can be confirmed by the fact that it attains more TP and fewer FN than the \acrshort{tsc} model. The \acrshort{ann} model is also the model that attains the most balanced scores of sensitivity and specificity. Therefore the \acrshort{ann} model is chosen as the best performer at predicting the segment indication. 

\newpage
\section*{Chapter Summary}

In the heart failure case study the PVC model was found to be the best performer, by a narrow margin. 
The TSC, and PVSC models also performed well, but the NN did not. In fact, the performance of the NN was not much better 
than what could be achieved by randomly guessing the binary label with equal probability of choosing one or zero. 
The PVC model that performed best at identifying heart failure among patients is \textit{gls-EF/complete/2}, and it attains an 
accuracy of 0.76, sensitivity of 0.81, specificity of 0.72 and DOR of 10.85. \medskip

In the patient diagnosis case study the PVSC model is regarded as the top performer.
Here too, it was a close call between the PVSC, PVC and TSC models.
The patient diagnosis dataset was skewed as there were 170 patients with a heart disease, and only 30 healthy patients. 
For this reason it is probable that the NN was unable to generalize the feature of the healthy patients, 
because almost all the variations of the NN ended up always making the prediction that the patient was diseased yielding a score of 0 in specificity. \smallskip
The PVSC model that performed best at predicting patient diagnosis is \textit{gls-rls/KNN}, and it attains an
accuracy of 0.93, sensitivity of 0.95, specificity of 0.82 and DOR of 84.53. \medskip

In the segment indication case study only the TSC and NN models were compared, and for a change of pace it was only the NN that was chosen as the best performer. 
The TSC model did not perform much worse, in fact it performed better than the NN in many respects. 
The key reason for why the NN was preferred was because it had a more balanced sensitivity, and specificity scores than the TSC model. 
The NN model that performed best at predicting segment indication is \textit{downsampled}, and it attains an
accuracy of 0.74, sensitivity of 0.74, specificity of 0.72 and DOR of 8.38. 
