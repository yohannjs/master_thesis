\chapter{Results}

In this chapter the results will be presented in the form of three case studies. 
Each case study will focus on a single target variable, and aims to find which model group performs best at predicting the target variable in question.
Recall that the three target variables that will be considered in this thesis are: Heart failure, patient diagnosis, and the indication of individual left ventricle segments.
As mentioned earlier in the chapter, four model groups will be tested. 
The case studies will first deal with each model group individually, where variants of the models with different hypermarameters will be tested on the different datasets. 
Then, the best performing model within each model group will be used to compare the four model groups.

\section{Case Study: Heart Failure}

\subsection{Time-series Clustering}

\begin{figure}[htb]
    \centering
    % \includegraphics[width=\textwidth]{results/tsc_hf_dor_sens_spec_dist.png}
    %% Creator: Matplotlib, PGF backend
%%
%% To include the figure in your LaTeX document, write
%%   \input{<filename>.pgf}
%%
%% Make sure the required packages are loaded in your preamble
%%   \usepackage{pgf}
%%
%% Figures using additional raster images can only be included by \input if
%% they are in the same directory as the main LaTeX file. For loading figures
%% from other directories you can use the `import` package
%%   \usepackage{import}
%% and then include the figures with
%%   \import{<path to file>}{<filename>.pgf}
%%
%% Matplotlib used the following preamble
%%
\begingroup%
\makeatletter%
\begin{pgfpicture}%
\pgfpathrectangle{\pgfpointorigin}{\pgfqpoint{6.364000in}{2.340000in}}%
\pgfusepath{use as bounding box, clip}%
\begin{pgfscope}%
\pgfsetbuttcap%
\pgfsetmiterjoin%
\definecolor{currentfill}{rgb}{1.000000,1.000000,1.000000}%
\pgfsetfillcolor{currentfill}%
\pgfsetlinewidth{0.000000pt}%
\definecolor{currentstroke}{rgb}{1.000000,1.000000,1.000000}%
\pgfsetstrokecolor{currentstroke}%
\pgfsetdash{}{0pt}%
\pgfpathmoveto{\pgfqpoint{0.000000in}{-0.000000in}}%
\pgfpathlineto{\pgfqpoint{6.364000in}{-0.000000in}}%
\pgfpathlineto{\pgfqpoint{6.364000in}{2.340000in}}%
\pgfpathlineto{\pgfqpoint{0.000000in}{2.340000in}}%
\pgfpathclose%
\pgfusepath{fill}%
\end{pgfscope}%
\begin{pgfscope}%
\pgfsetbuttcap%
\pgfsetmiterjoin%
\definecolor{currentfill}{rgb}{0.917647,0.917647,0.949020}%
\pgfsetfillcolor{currentfill}%
\pgfsetlinewidth{0.000000pt}%
\definecolor{currentstroke}{rgb}{0.000000,0.000000,0.000000}%
\pgfsetstrokecolor{currentstroke}%
\pgfsetstrokeopacity{0.000000}%
\pgfsetdash{}{0pt}%
\pgfpathmoveto{\pgfqpoint{0.650810in}{0.557870in}}%
\pgfpathlineto{\pgfqpoint{3.096898in}{0.557870in}}%
\pgfpathlineto{\pgfqpoint{3.096898in}{2.042604in}}%
\pgfpathlineto{\pgfqpoint{0.650810in}{2.042604in}}%
\pgfpathclose%
\pgfusepath{fill}%
\end{pgfscope}%
\begin{pgfscope}%
\pgfpathrectangle{\pgfqpoint{0.650810in}{0.557870in}}{\pgfqpoint{2.446088in}{1.484734in}}%
\pgfusepath{clip}%
\pgfsetroundcap%
\pgfsetroundjoin%
\pgfsetlinewidth{1.003750pt}%
\definecolor{currentstroke}{rgb}{1.000000,1.000000,1.000000}%
\pgfsetstrokecolor{currentstroke}%
\pgfsetdash{}{0pt}%
\pgfpathmoveto{\pgfqpoint{0.761996in}{0.557870in}}%
\pgfpathlineto{\pgfqpoint{0.761996in}{2.042604in}}%
\pgfusepath{stroke}%
\end{pgfscope}%
\begin{pgfscope}%
\definecolor{textcolor}{rgb}{0.150000,0.150000,0.150000}%
\pgfsetstrokecolor{textcolor}%
\pgfsetfillcolor{textcolor}%
\pgftext[x=0.761996in,y=0.425926in,,top]{\color{textcolor}\sffamily\fontsize{11.000000}{13.200000}\selectfont \(\displaystyle -0.50\)}%
\end{pgfscope}%
\begin{pgfscope}%
\pgfpathrectangle{\pgfqpoint{0.650810in}{0.557870in}}{\pgfqpoint{2.446088in}{1.484734in}}%
\pgfusepath{clip}%
\pgfsetroundcap%
\pgfsetroundjoin%
\pgfsetlinewidth{1.003750pt}%
\definecolor{currentstroke}{rgb}{1.000000,1.000000,1.000000}%
\pgfsetstrokecolor{currentstroke}%
\pgfsetdash{}{0pt}%
\pgfpathmoveto{\pgfqpoint{1.317925in}{0.557870in}}%
\pgfpathlineto{\pgfqpoint{1.317925in}{2.042604in}}%
\pgfusepath{stroke}%
\end{pgfscope}%
\begin{pgfscope}%
\definecolor{textcolor}{rgb}{0.150000,0.150000,0.150000}%
\pgfsetstrokecolor{textcolor}%
\pgfsetfillcolor{textcolor}%
\pgftext[x=1.317925in,y=0.425926in,,top]{\color{textcolor}\sffamily\fontsize{11.000000}{13.200000}\selectfont \(\displaystyle -0.25\)}%
\end{pgfscope}%
\begin{pgfscope}%
\pgfpathrectangle{\pgfqpoint{0.650810in}{0.557870in}}{\pgfqpoint{2.446088in}{1.484734in}}%
\pgfusepath{clip}%
\pgfsetroundcap%
\pgfsetroundjoin%
\pgfsetlinewidth{1.003750pt}%
\definecolor{currentstroke}{rgb}{1.000000,1.000000,1.000000}%
\pgfsetstrokecolor{currentstroke}%
\pgfsetdash{}{0pt}%
\pgfpathmoveto{\pgfqpoint{1.873854in}{0.557870in}}%
\pgfpathlineto{\pgfqpoint{1.873854in}{2.042604in}}%
\pgfusepath{stroke}%
\end{pgfscope}%
\begin{pgfscope}%
\definecolor{textcolor}{rgb}{0.150000,0.150000,0.150000}%
\pgfsetstrokecolor{textcolor}%
\pgfsetfillcolor{textcolor}%
\pgftext[x=1.873854in,y=0.425926in,,top]{\color{textcolor}\sffamily\fontsize{11.000000}{13.200000}\selectfont \(\displaystyle 0.00\)}%
\end{pgfscope}%
\begin{pgfscope}%
\pgfpathrectangle{\pgfqpoint{0.650810in}{0.557870in}}{\pgfqpoint{2.446088in}{1.484734in}}%
\pgfusepath{clip}%
\pgfsetroundcap%
\pgfsetroundjoin%
\pgfsetlinewidth{1.003750pt}%
\definecolor{currentstroke}{rgb}{1.000000,1.000000,1.000000}%
\pgfsetstrokecolor{currentstroke}%
\pgfsetdash{}{0pt}%
\pgfpathmoveto{\pgfqpoint{2.429783in}{0.557870in}}%
\pgfpathlineto{\pgfqpoint{2.429783in}{2.042604in}}%
\pgfusepath{stroke}%
\end{pgfscope}%
\begin{pgfscope}%
\definecolor{textcolor}{rgb}{0.150000,0.150000,0.150000}%
\pgfsetstrokecolor{textcolor}%
\pgfsetfillcolor{textcolor}%
\pgftext[x=2.429783in,y=0.425926in,,top]{\color{textcolor}\sffamily\fontsize{11.000000}{13.200000}\selectfont \(\displaystyle 0.25\)}%
\end{pgfscope}%
\begin{pgfscope}%
\pgfpathrectangle{\pgfqpoint{0.650810in}{0.557870in}}{\pgfqpoint{2.446088in}{1.484734in}}%
\pgfusepath{clip}%
\pgfsetroundcap%
\pgfsetroundjoin%
\pgfsetlinewidth{1.003750pt}%
\definecolor{currentstroke}{rgb}{1.000000,1.000000,1.000000}%
\pgfsetstrokecolor{currentstroke}%
\pgfsetdash{}{0pt}%
\pgfpathmoveto{\pgfqpoint{2.985712in}{0.557870in}}%
\pgfpathlineto{\pgfqpoint{2.985712in}{2.042604in}}%
\pgfusepath{stroke}%
\end{pgfscope}%
\begin{pgfscope}%
\definecolor{textcolor}{rgb}{0.150000,0.150000,0.150000}%
\pgfsetstrokecolor{textcolor}%
\pgfsetfillcolor{textcolor}%
\pgftext[x=2.985712in,y=0.425926in,,top]{\color{textcolor}\sffamily\fontsize{11.000000}{13.200000}\selectfont \(\displaystyle 0.50\)}%
\end{pgfscope}%
\begin{pgfscope}%
\definecolor{textcolor}{rgb}{0.150000,0.150000,0.150000}%
\pgfsetstrokecolor{textcolor}%
\pgfsetfillcolor{textcolor}%
\pgftext[x=1.873854in,y=0.235185in,,top]{\color{textcolor}\sffamily\fontsize{11.000000}{13.200000}\selectfont DOR}%
\end{pgfscope}%
\begin{pgfscope}%
\pgfpathrectangle{\pgfqpoint{0.650810in}{0.557870in}}{\pgfqpoint{2.446088in}{1.484734in}}%
\pgfusepath{clip}%
\pgfsetroundcap%
\pgfsetroundjoin%
\pgfsetlinewidth{1.003750pt}%
\definecolor{currentstroke}{rgb}{1.000000,1.000000,1.000000}%
\pgfsetstrokecolor{currentstroke}%
\pgfsetdash{}{0pt}%
\pgfpathmoveto{\pgfqpoint{0.650810in}{0.557870in}}%
\pgfpathlineto{\pgfqpoint{3.096898in}{0.557870in}}%
\pgfusepath{stroke}%
\end{pgfscope}%
\begin{pgfscope}%
\definecolor{textcolor}{rgb}{0.150000,0.150000,0.150000}%
\pgfsetstrokecolor{textcolor}%
\pgfsetfillcolor{textcolor}%
\pgftext[x=0.442824in,y=0.505064in,left,base]{\color{textcolor}\sffamily\fontsize{11.000000}{13.200000}\selectfont \(\displaystyle 0\)}%
\end{pgfscope}%
\begin{pgfscope}%
\pgfpathrectangle{\pgfqpoint{0.650810in}{0.557870in}}{\pgfqpoint{2.446088in}{1.484734in}}%
\pgfusepath{clip}%
\pgfsetroundcap%
\pgfsetroundjoin%
\pgfsetlinewidth{1.003750pt}%
\definecolor{currentstroke}{rgb}{1.000000,1.000000,1.000000}%
\pgfsetstrokecolor{currentstroke}%
\pgfsetdash{}{0pt}%
\pgfpathmoveto{\pgfqpoint{0.650810in}{1.032378in}}%
\pgfpathlineto{\pgfqpoint{3.096898in}{1.032378in}}%
\pgfusepath{stroke}%
\end{pgfscope}%
\begin{pgfscope}%
\definecolor{textcolor}{rgb}{0.150000,0.150000,0.150000}%
\pgfsetstrokecolor{textcolor}%
\pgfsetfillcolor{textcolor}%
\pgftext[x=0.290741in,y=0.979571in,left,base]{\color{textcolor}\sffamily\fontsize{11.000000}{13.200000}\selectfont \(\displaystyle 100\)}%
\end{pgfscope}%
\begin{pgfscope}%
\pgfpathrectangle{\pgfqpoint{0.650810in}{0.557870in}}{\pgfqpoint{2.446088in}{1.484734in}}%
\pgfusepath{clip}%
\pgfsetroundcap%
\pgfsetroundjoin%
\pgfsetlinewidth{1.003750pt}%
\definecolor{currentstroke}{rgb}{1.000000,1.000000,1.000000}%
\pgfsetstrokecolor{currentstroke}%
\pgfsetdash{}{0pt}%
\pgfpathmoveto{\pgfqpoint{0.650810in}{1.506885in}}%
\pgfpathlineto{\pgfqpoint{3.096898in}{1.506885in}}%
\pgfusepath{stroke}%
\end{pgfscope}%
\begin{pgfscope}%
\definecolor{textcolor}{rgb}{0.150000,0.150000,0.150000}%
\pgfsetstrokecolor{textcolor}%
\pgfsetfillcolor{textcolor}%
\pgftext[x=0.290741in,y=1.454079in,left,base]{\color{textcolor}\sffamily\fontsize{11.000000}{13.200000}\selectfont \(\displaystyle 200\)}%
\end{pgfscope}%
\begin{pgfscope}%
\pgfpathrectangle{\pgfqpoint{0.650810in}{0.557870in}}{\pgfqpoint{2.446088in}{1.484734in}}%
\pgfusepath{clip}%
\pgfsetroundcap%
\pgfsetroundjoin%
\pgfsetlinewidth{1.003750pt}%
\definecolor{currentstroke}{rgb}{1.000000,1.000000,1.000000}%
\pgfsetstrokecolor{currentstroke}%
\pgfsetdash{}{0pt}%
\pgfpathmoveto{\pgfqpoint{0.650810in}{1.981393in}}%
\pgfpathlineto{\pgfqpoint{3.096898in}{1.981393in}}%
\pgfusepath{stroke}%
\end{pgfscope}%
\begin{pgfscope}%
\definecolor{textcolor}{rgb}{0.150000,0.150000,0.150000}%
\pgfsetstrokecolor{textcolor}%
\pgfsetfillcolor{textcolor}%
\pgftext[x=0.290741in,y=1.928586in,left,base]{\color{textcolor}\sffamily\fontsize{11.000000}{13.200000}\selectfont \(\displaystyle 300\)}%
\end{pgfscope}%
\begin{pgfscope}%
\definecolor{textcolor}{rgb}{0.150000,0.150000,0.150000}%
\pgfsetstrokecolor{textcolor}%
\pgfsetfillcolor{textcolor}%
\pgftext[x=0.235185in,y=1.300237in,,bottom,rotate=90.000000]{\color{textcolor}\sffamily\fontsize{11.000000}{13.200000}\selectfont Occurance}%
\end{pgfscope}%
\begin{pgfscope}%
\pgfpathrectangle{\pgfqpoint{0.650810in}{0.557870in}}{\pgfqpoint{2.446088in}{1.484734in}}%
\pgfusepath{clip}%
\pgfsetbuttcap%
\pgfsetmiterjoin%
\definecolor{currentfill}{rgb}{0.298039,0.447059,0.690196}%
\pgfsetfillcolor{currentfill}%
\pgfsetfillopacity{0.400000}%
\pgfsetlinewidth{1.003750pt}%
\definecolor{currentstroke}{rgb}{1.000000,1.000000,1.000000}%
\pgfsetstrokecolor{currentstroke}%
\pgfsetstrokeopacity{0.400000}%
\pgfsetdash{}{0pt}%
\pgfpathmoveto{\pgfqpoint{0.761996in}{0.557870in}}%
\pgfpathlineto{\pgfqpoint{0.984368in}{0.557870in}}%
\pgfpathlineto{\pgfqpoint{0.984368in}{0.557870in}}%
\pgfpathlineto{\pgfqpoint{0.761996in}{0.557870in}}%
\pgfpathclose%
\pgfusepath{stroke,fill}%
\end{pgfscope}%
\begin{pgfscope}%
\pgfpathrectangle{\pgfqpoint{0.650810in}{0.557870in}}{\pgfqpoint{2.446088in}{1.484734in}}%
\pgfusepath{clip}%
\pgfsetbuttcap%
\pgfsetmiterjoin%
\definecolor{currentfill}{rgb}{0.298039,0.447059,0.690196}%
\pgfsetfillcolor{currentfill}%
\pgfsetfillopacity{0.400000}%
\pgfsetlinewidth{1.003750pt}%
\definecolor{currentstroke}{rgb}{1.000000,1.000000,1.000000}%
\pgfsetstrokecolor{currentstroke}%
\pgfsetstrokeopacity{0.400000}%
\pgfsetdash{}{0pt}%
\pgfpathmoveto{\pgfqpoint{0.984368in}{0.557870in}}%
\pgfpathlineto{\pgfqpoint{1.206739in}{0.557870in}}%
\pgfpathlineto{\pgfqpoint{1.206739in}{0.557870in}}%
\pgfpathlineto{\pgfqpoint{0.984368in}{0.557870in}}%
\pgfpathclose%
\pgfusepath{stroke,fill}%
\end{pgfscope}%
\begin{pgfscope}%
\pgfpathrectangle{\pgfqpoint{0.650810in}{0.557870in}}{\pgfqpoint{2.446088in}{1.484734in}}%
\pgfusepath{clip}%
\pgfsetbuttcap%
\pgfsetmiterjoin%
\definecolor{currentfill}{rgb}{0.298039,0.447059,0.690196}%
\pgfsetfillcolor{currentfill}%
\pgfsetfillopacity{0.400000}%
\pgfsetlinewidth{1.003750pt}%
\definecolor{currentstroke}{rgb}{1.000000,1.000000,1.000000}%
\pgfsetstrokecolor{currentstroke}%
\pgfsetstrokeopacity{0.400000}%
\pgfsetdash{}{0pt}%
\pgfpathmoveto{\pgfqpoint{1.206739in}{0.557870in}}%
\pgfpathlineto{\pgfqpoint{1.429111in}{0.557870in}}%
\pgfpathlineto{\pgfqpoint{1.429111in}{0.557870in}}%
\pgfpathlineto{\pgfqpoint{1.206739in}{0.557870in}}%
\pgfpathclose%
\pgfusepath{stroke,fill}%
\end{pgfscope}%
\begin{pgfscope}%
\pgfpathrectangle{\pgfqpoint{0.650810in}{0.557870in}}{\pgfqpoint{2.446088in}{1.484734in}}%
\pgfusepath{clip}%
\pgfsetbuttcap%
\pgfsetmiterjoin%
\definecolor{currentfill}{rgb}{0.298039,0.447059,0.690196}%
\pgfsetfillcolor{currentfill}%
\pgfsetfillopacity{0.400000}%
\pgfsetlinewidth{1.003750pt}%
\definecolor{currentstroke}{rgb}{1.000000,1.000000,1.000000}%
\pgfsetstrokecolor{currentstroke}%
\pgfsetstrokeopacity{0.400000}%
\pgfsetdash{}{0pt}%
\pgfpathmoveto{\pgfqpoint{1.429111in}{0.557870in}}%
\pgfpathlineto{\pgfqpoint{1.651483in}{0.557870in}}%
\pgfpathlineto{\pgfqpoint{1.651483in}{0.557870in}}%
\pgfpathlineto{\pgfqpoint{1.429111in}{0.557870in}}%
\pgfpathclose%
\pgfusepath{stroke,fill}%
\end{pgfscope}%
\begin{pgfscope}%
\pgfpathrectangle{\pgfqpoint{0.650810in}{0.557870in}}{\pgfqpoint{2.446088in}{1.484734in}}%
\pgfusepath{clip}%
\pgfsetbuttcap%
\pgfsetmiterjoin%
\definecolor{currentfill}{rgb}{0.298039,0.447059,0.690196}%
\pgfsetfillcolor{currentfill}%
\pgfsetfillopacity{0.400000}%
\pgfsetlinewidth{1.003750pt}%
\definecolor{currentstroke}{rgb}{1.000000,1.000000,1.000000}%
\pgfsetstrokecolor{currentstroke}%
\pgfsetstrokeopacity{0.400000}%
\pgfsetdash{}{0pt}%
\pgfpathmoveto{\pgfqpoint{1.651483in}{0.557870in}}%
\pgfpathlineto{\pgfqpoint{1.873854in}{0.557870in}}%
\pgfpathlineto{\pgfqpoint{1.873854in}{0.557870in}}%
\pgfpathlineto{\pgfqpoint{1.651483in}{0.557870in}}%
\pgfpathclose%
\pgfusepath{stroke,fill}%
\end{pgfscope}%
\begin{pgfscope}%
\pgfpathrectangle{\pgfqpoint{0.650810in}{0.557870in}}{\pgfqpoint{2.446088in}{1.484734in}}%
\pgfusepath{clip}%
\pgfsetbuttcap%
\pgfsetmiterjoin%
\definecolor{currentfill}{rgb}{0.298039,0.447059,0.690196}%
\pgfsetfillcolor{currentfill}%
\pgfsetfillopacity{0.400000}%
\pgfsetlinewidth{1.003750pt}%
\definecolor{currentstroke}{rgb}{1.000000,1.000000,1.000000}%
\pgfsetstrokecolor{currentstroke}%
\pgfsetstrokeopacity{0.400000}%
\pgfsetdash{}{0pt}%
\pgfpathmoveto{\pgfqpoint{1.873854in}{0.557870in}}%
\pgfpathlineto{\pgfqpoint{2.096226in}{0.557870in}}%
\pgfpathlineto{\pgfqpoint{2.096226in}{1.971903in}}%
\pgfpathlineto{\pgfqpoint{1.873854in}{1.971903in}}%
\pgfpathclose%
\pgfusepath{stroke,fill}%
\end{pgfscope}%
\begin{pgfscope}%
\pgfpathrectangle{\pgfqpoint{0.650810in}{0.557870in}}{\pgfqpoint{2.446088in}{1.484734in}}%
\pgfusepath{clip}%
\pgfsetbuttcap%
\pgfsetmiterjoin%
\definecolor{currentfill}{rgb}{0.298039,0.447059,0.690196}%
\pgfsetfillcolor{currentfill}%
\pgfsetfillopacity{0.400000}%
\pgfsetlinewidth{1.003750pt}%
\definecolor{currentstroke}{rgb}{1.000000,1.000000,1.000000}%
\pgfsetstrokecolor{currentstroke}%
\pgfsetstrokeopacity{0.400000}%
\pgfsetdash{}{0pt}%
\pgfpathmoveto{\pgfqpoint{2.096226in}{0.557870in}}%
\pgfpathlineto{\pgfqpoint{2.318598in}{0.557870in}}%
\pgfpathlineto{\pgfqpoint{2.318598in}{0.557870in}}%
\pgfpathlineto{\pgfqpoint{2.096226in}{0.557870in}}%
\pgfpathclose%
\pgfusepath{stroke,fill}%
\end{pgfscope}%
\begin{pgfscope}%
\pgfpathrectangle{\pgfqpoint{0.650810in}{0.557870in}}{\pgfqpoint{2.446088in}{1.484734in}}%
\pgfusepath{clip}%
\pgfsetbuttcap%
\pgfsetmiterjoin%
\definecolor{currentfill}{rgb}{0.298039,0.447059,0.690196}%
\pgfsetfillcolor{currentfill}%
\pgfsetfillopacity{0.400000}%
\pgfsetlinewidth{1.003750pt}%
\definecolor{currentstroke}{rgb}{1.000000,1.000000,1.000000}%
\pgfsetstrokecolor{currentstroke}%
\pgfsetstrokeopacity{0.400000}%
\pgfsetdash{}{0pt}%
\pgfpathmoveto{\pgfqpoint{2.318598in}{0.557870in}}%
\pgfpathlineto{\pgfqpoint{2.540969in}{0.557870in}}%
\pgfpathlineto{\pgfqpoint{2.540969in}{0.557870in}}%
\pgfpathlineto{\pgfqpoint{2.318598in}{0.557870in}}%
\pgfpathclose%
\pgfusepath{stroke,fill}%
\end{pgfscope}%
\begin{pgfscope}%
\pgfpathrectangle{\pgfqpoint{0.650810in}{0.557870in}}{\pgfqpoint{2.446088in}{1.484734in}}%
\pgfusepath{clip}%
\pgfsetbuttcap%
\pgfsetmiterjoin%
\definecolor{currentfill}{rgb}{0.298039,0.447059,0.690196}%
\pgfsetfillcolor{currentfill}%
\pgfsetfillopacity{0.400000}%
\pgfsetlinewidth{1.003750pt}%
\definecolor{currentstroke}{rgb}{1.000000,1.000000,1.000000}%
\pgfsetstrokecolor{currentstroke}%
\pgfsetstrokeopacity{0.400000}%
\pgfsetdash{}{0pt}%
\pgfpathmoveto{\pgfqpoint{2.540969in}{0.557870in}}%
\pgfpathlineto{\pgfqpoint{2.763341in}{0.557870in}}%
\pgfpathlineto{\pgfqpoint{2.763341in}{0.557870in}}%
\pgfpathlineto{\pgfqpoint{2.540969in}{0.557870in}}%
\pgfpathclose%
\pgfusepath{stroke,fill}%
\end{pgfscope}%
\begin{pgfscope}%
\pgfpathrectangle{\pgfqpoint{0.650810in}{0.557870in}}{\pgfqpoint{2.446088in}{1.484734in}}%
\pgfusepath{clip}%
\pgfsetbuttcap%
\pgfsetmiterjoin%
\definecolor{currentfill}{rgb}{0.298039,0.447059,0.690196}%
\pgfsetfillcolor{currentfill}%
\pgfsetfillopacity{0.400000}%
\pgfsetlinewidth{1.003750pt}%
\definecolor{currentstroke}{rgb}{1.000000,1.000000,1.000000}%
\pgfsetstrokecolor{currentstroke}%
\pgfsetstrokeopacity{0.400000}%
\pgfsetdash{}{0pt}%
\pgfpathmoveto{\pgfqpoint{2.763341in}{0.557870in}}%
\pgfpathlineto{\pgfqpoint{2.985712in}{0.557870in}}%
\pgfpathlineto{\pgfqpoint{2.985712in}{0.557870in}}%
\pgfpathlineto{\pgfqpoint{2.763341in}{0.557870in}}%
\pgfpathclose%
\pgfusepath{stroke,fill}%
\end{pgfscope}%
\begin{pgfscope}%
\pgfsetrectcap%
\pgfsetmiterjoin%
\pgfsetlinewidth{1.254687pt}%
\definecolor{currentstroke}{rgb}{1.000000,1.000000,1.000000}%
\pgfsetstrokecolor{currentstroke}%
\pgfsetdash{}{0pt}%
\pgfpathmoveto{\pgfqpoint{0.650810in}{0.557870in}}%
\pgfpathlineto{\pgfqpoint{0.650810in}{2.042604in}}%
\pgfusepath{stroke}%
\end{pgfscope}%
\begin{pgfscope}%
\pgfsetrectcap%
\pgfsetmiterjoin%
\pgfsetlinewidth{1.254687pt}%
\definecolor{currentstroke}{rgb}{1.000000,1.000000,1.000000}%
\pgfsetstrokecolor{currentstroke}%
\pgfsetdash{}{0pt}%
\pgfpathmoveto{\pgfqpoint{3.096898in}{0.557870in}}%
\pgfpathlineto{\pgfqpoint{3.096898in}{2.042604in}}%
\pgfusepath{stroke}%
\end{pgfscope}%
\begin{pgfscope}%
\pgfsetrectcap%
\pgfsetmiterjoin%
\pgfsetlinewidth{1.254687pt}%
\definecolor{currentstroke}{rgb}{1.000000,1.000000,1.000000}%
\pgfsetstrokecolor{currentstroke}%
\pgfsetdash{}{0pt}%
\pgfpathmoveto{\pgfqpoint{0.650810in}{0.557870in}}%
\pgfpathlineto{\pgfqpoint{3.096898in}{0.557870in}}%
\pgfusepath{stroke}%
\end{pgfscope}%
\begin{pgfscope}%
\pgfsetrectcap%
\pgfsetmiterjoin%
\pgfsetlinewidth{1.254687pt}%
\definecolor{currentstroke}{rgb}{1.000000,1.000000,1.000000}%
\pgfsetstrokecolor{currentstroke}%
\pgfsetdash{}{0pt}%
\pgfpathmoveto{\pgfqpoint{0.650810in}{2.042604in}}%
\pgfpathlineto{\pgfqpoint{3.096898in}{2.042604in}}%
\pgfusepath{stroke}%
\end{pgfscope}%
\begin{pgfscope}%
\definecolor{textcolor}{rgb}{0.150000,0.150000,0.150000}%
\pgfsetstrokecolor{textcolor}%
\pgfsetfillcolor{textcolor}%
\pgftext[x=1.873854in,y=2.125938in,,base]{\color{textcolor}\sffamily\fontsize{11.000000}{13.200000}\selectfont (a)}%
\end{pgfscope}%
\begin{pgfscope}%
\pgfsetbuttcap%
\pgfsetmiterjoin%
\definecolor{currentfill}{rgb}{0.917647,0.917647,0.949020}%
\pgfsetfillcolor{currentfill}%
\pgfsetlinewidth{0.000000pt}%
\definecolor{currentstroke}{rgb}{0.000000,0.000000,0.000000}%
\pgfsetstrokecolor{currentstroke}%
\pgfsetstrokeopacity{0.000000}%
\pgfsetdash{}{0pt}%
\pgfpathmoveto{\pgfqpoint{3.793912in}{0.557870in}}%
\pgfpathlineto{\pgfqpoint{6.240000in}{0.557870in}}%
\pgfpathlineto{\pgfqpoint{6.240000in}{2.042604in}}%
\pgfpathlineto{\pgfqpoint{3.793912in}{2.042604in}}%
\pgfpathclose%
\pgfusepath{fill}%
\end{pgfscope}%
\begin{pgfscope}%
\pgfpathrectangle{\pgfqpoint{3.793912in}{0.557870in}}{\pgfqpoint{2.446088in}{1.484734in}}%
\pgfusepath{clip}%
\pgfsetroundcap%
\pgfsetroundjoin%
\pgfsetlinewidth{1.003750pt}%
\definecolor{currentstroke}{rgb}{1.000000,1.000000,1.000000}%
\pgfsetstrokecolor{currentstroke}%
\pgfsetdash{}{0pt}%
\pgfpathmoveto{\pgfqpoint{3.905098in}{0.557870in}}%
\pgfpathlineto{\pgfqpoint{3.905098in}{2.042604in}}%
\pgfusepath{stroke}%
\end{pgfscope}%
\begin{pgfscope}%
\definecolor{textcolor}{rgb}{0.150000,0.150000,0.150000}%
\pgfsetstrokecolor{textcolor}%
\pgfsetfillcolor{textcolor}%
\pgftext[x=3.905098in,y=0.425926in,,top]{\color{textcolor}\sffamily\fontsize{11.000000}{13.200000}\selectfont \(\displaystyle 0.00\)}%
\end{pgfscope}%
\begin{pgfscope}%
\pgfpathrectangle{\pgfqpoint{3.793912in}{0.557870in}}{\pgfqpoint{2.446088in}{1.484734in}}%
\pgfusepath{clip}%
\pgfsetroundcap%
\pgfsetroundjoin%
\pgfsetlinewidth{1.003750pt}%
\definecolor{currentstroke}{rgb}{1.000000,1.000000,1.000000}%
\pgfsetstrokecolor{currentstroke}%
\pgfsetdash{}{0pt}%
\pgfpathmoveto{\pgfqpoint{4.461027in}{0.557870in}}%
\pgfpathlineto{\pgfqpoint{4.461027in}{2.042604in}}%
\pgfusepath{stroke}%
\end{pgfscope}%
\begin{pgfscope}%
\definecolor{textcolor}{rgb}{0.150000,0.150000,0.150000}%
\pgfsetstrokecolor{textcolor}%
\pgfsetfillcolor{textcolor}%
\pgftext[x=4.461027in,y=0.425926in,,top]{\color{textcolor}\sffamily\fontsize{11.000000}{13.200000}\selectfont \(\displaystyle 0.25\)}%
\end{pgfscope}%
\begin{pgfscope}%
\pgfpathrectangle{\pgfqpoint{3.793912in}{0.557870in}}{\pgfqpoint{2.446088in}{1.484734in}}%
\pgfusepath{clip}%
\pgfsetroundcap%
\pgfsetroundjoin%
\pgfsetlinewidth{1.003750pt}%
\definecolor{currentstroke}{rgb}{1.000000,1.000000,1.000000}%
\pgfsetstrokecolor{currentstroke}%
\pgfsetdash{}{0pt}%
\pgfpathmoveto{\pgfqpoint{5.016956in}{0.557870in}}%
\pgfpathlineto{\pgfqpoint{5.016956in}{2.042604in}}%
\pgfusepath{stroke}%
\end{pgfscope}%
\begin{pgfscope}%
\definecolor{textcolor}{rgb}{0.150000,0.150000,0.150000}%
\pgfsetstrokecolor{textcolor}%
\pgfsetfillcolor{textcolor}%
\pgftext[x=5.016956in,y=0.425926in,,top]{\color{textcolor}\sffamily\fontsize{11.000000}{13.200000}\selectfont \(\displaystyle 0.50\)}%
\end{pgfscope}%
\begin{pgfscope}%
\pgfpathrectangle{\pgfqpoint{3.793912in}{0.557870in}}{\pgfqpoint{2.446088in}{1.484734in}}%
\pgfusepath{clip}%
\pgfsetroundcap%
\pgfsetroundjoin%
\pgfsetlinewidth{1.003750pt}%
\definecolor{currentstroke}{rgb}{1.000000,1.000000,1.000000}%
\pgfsetstrokecolor{currentstroke}%
\pgfsetdash{}{0pt}%
\pgfpathmoveto{\pgfqpoint{5.572885in}{0.557870in}}%
\pgfpathlineto{\pgfqpoint{5.572885in}{2.042604in}}%
\pgfusepath{stroke}%
\end{pgfscope}%
\begin{pgfscope}%
\definecolor{textcolor}{rgb}{0.150000,0.150000,0.150000}%
\pgfsetstrokecolor{textcolor}%
\pgfsetfillcolor{textcolor}%
\pgftext[x=5.572885in,y=0.425926in,,top]{\color{textcolor}\sffamily\fontsize{11.000000}{13.200000}\selectfont \(\displaystyle 0.75\)}%
\end{pgfscope}%
\begin{pgfscope}%
\pgfpathrectangle{\pgfqpoint{3.793912in}{0.557870in}}{\pgfqpoint{2.446088in}{1.484734in}}%
\pgfusepath{clip}%
\pgfsetroundcap%
\pgfsetroundjoin%
\pgfsetlinewidth{1.003750pt}%
\definecolor{currentstroke}{rgb}{1.000000,1.000000,1.000000}%
\pgfsetstrokecolor{currentstroke}%
\pgfsetdash{}{0pt}%
\pgfpathmoveto{\pgfqpoint{6.128814in}{0.557870in}}%
\pgfpathlineto{\pgfqpoint{6.128814in}{2.042604in}}%
\pgfusepath{stroke}%
\end{pgfscope}%
\begin{pgfscope}%
\definecolor{textcolor}{rgb}{0.150000,0.150000,0.150000}%
\pgfsetstrokecolor{textcolor}%
\pgfsetfillcolor{textcolor}%
\pgftext[x=6.128814in,y=0.425926in,,top]{\color{textcolor}\sffamily\fontsize{11.000000}{13.200000}\selectfont \(\displaystyle 1.00\)}%
\end{pgfscope}%
\begin{pgfscope}%
\definecolor{textcolor}{rgb}{0.150000,0.150000,0.150000}%
\pgfsetstrokecolor{textcolor}%
\pgfsetfillcolor{textcolor}%
\pgftext[x=5.016956in,y=0.235185in,,top]{\color{textcolor}\sffamily\fontsize{11.000000}{13.200000}\selectfont Specificity}%
\end{pgfscope}%
\begin{pgfscope}%
\pgfpathrectangle{\pgfqpoint{3.793912in}{0.557870in}}{\pgfqpoint{2.446088in}{1.484734in}}%
\pgfusepath{clip}%
\pgfsetroundcap%
\pgfsetroundjoin%
\pgfsetlinewidth{1.003750pt}%
\definecolor{currentstroke}{rgb}{1.000000,1.000000,1.000000}%
\pgfsetstrokecolor{currentstroke}%
\pgfsetdash{}{0pt}%
\pgfpathmoveto{\pgfqpoint{3.793912in}{0.625358in}}%
\pgfpathlineto{\pgfqpoint{6.240000in}{0.625358in}}%
\pgfusepath{stroke}%
\end{pgfscope}%
\begin{pgfscope}%
\definecolor{textcolor}{rgb}{0.150000,0.150000,0.150000}%
\pgfsetstrokecolor{textcolor}%
\pgfsetfillcolor{textcolor}%
\pgftext[x=3.467639in,y=0.572552in,left,base]{\color{textcolor}\sffamily\fontsize{11.000000}{13.200000}\selectfont \(\displaystyle 0.0\)}%
\end{pgfscope}%
\begin{pgfscope}%
\pgfpathrectangle{\pgfqpoint{3.793912in}{0.557870in}}{\pgfqpoint{2.446088in}{1.484734in}}%
\pgfusepath{clip}%
\pgfsetroundcap%
\pgfsetroundjoin%
\pgfsetlinewidth{1.003750pt}%
\definecolor{currentstroke}{rgb}{1.000000,1.000000,1.000000}%
\pgfsetstrokecolor{currentstroke}%
\pgfsetdash{}{0pt}%
\pgfpathmoveto{\pgfqpoint{3.793912in}{1.300237in}}%
\pgfpathlineto{\pgfqpoint{6.240000in}{1.300237in}}%
\pgfusepath{stroke}%
\end{pgfscope}%
\begin{pgfscope}%
\definecolor{textcolor}{rgb}{0.150000,0.150000,0.150000}%
\pgfsetstrokecolor{textcolor}%
\pgfsetfillcolor{textcolor}%
\pgftext[x=3.467639in,y=1.247431in,left,base]{\color{textcolor}\sffamily\fontsize{11.000000}{13.200000}\selectfont \(\displaystyle 0.5\)}%
\end{pgfscope}%
\begin{pgfscope}%
\pgfpathrectangle{\pgfqpoint{3.793912in}{0.557870in}}{\pgfqpoint{2.446088in}{1.484734in}}%
\pgfusepath{clip}%
\pgfsetroundcap%
\pgfsetroundjoin%
\pgfsetlinewidth{1.003750pt}%
\definecolor{currentstroke}{rgb}{1.000000,1.000000,1.000000}%
\pgfsetstrokecolor{currentstroke}%
\pgfsetdash{}{0pt}%
\pgfpathmoveto{\pgfqpoint{3.793912in}{1.975116in}}%
\pgfpathlineto{\pgfqpoint{6.240000in}{1.975116in}}%
\pgfusepath{stroke}%
\end{pgfscope}%
\begin{pgfscope}%
\definecolor{textcolor}{rgb}{0.150000,0.150000,0.150000}%
\pgfsetstrokecolor{textcolor}%
\pgfsetfillcolor{textcolor}%
\pgftext[x=3.467639in,y=1.922310in,left,base]{\color{textcolor}\sffamily\fontsize{11.000000}{13.200000}\selectfont \(\displaystyle 1.0\)}%
\end{pgfscope}%
\begin{pgfscope}%
\definecolor{textcolor}{rgb}{0.150000,0.150000,0.150000}%
\pgfsetstrokecolor{textcolor}%
\pgfsetfillcolor{textcolor}%
\pgftext[x=3.412083in,y=1.300237in,,bottom,rotate=90.000000]{\color{textcolor}\sffamily\fontsize{11.000000}{13.200000}\selectfont Sensitivity}%
\end{pgfscope}%
\begin{pgfscope}%
\pgfpathrectangle{\pgfqpoint{3.793912in}{0.557870in}}{\pgfqpoint{2.446088in}{1.484734in}}%
\pgfusepath{clip}%
\pgfsetbuttcap%
\pgfsetroundjoin%
\definecolor{currentfill}{rgb}{0.298039,0.447059,0.690196}%
\pgfsetfillcolor{currentfill}%
\pgfsetlinewidth{1.003750pt}%
\definecolor{currentstroke}{rgb}{0.298039,0.447059,0.690196}%
\pgfsetstrokecolor{currentstroke}%
\pgfsetdash{}{0pt}%
\pgfpathmoveto{\pgfqpoint{3.905098in}{1.439605in}}%
\pgfpathcurveto{\pgfqpoint{3.913334in}{1.439605in}}{\pgfqpoint{3.921234in}{1.442877in}}{\pgfqpoint{3.927058in}{1.448701in}}%
\pgfpathcurveto{\pgfqpoint{3.932882in}{1.454525in}}{\pgfqpoint{3.936155in}{1.462425in}}{\pgfqpoint{3.936155in}{1.470661in}}%
\pgfpathcurveto{\pgfqpoint{3.936155in}{1.478898in}}{\pgfqpoint{3.932882in}{1.486798in}}{\pgfqpoint{3.927058in}{1.492621in}}%
\pgfpathcurveto{\pgfqpoint{3.921234in}{1.498445in}}{\pgfqpoint{3.913334in}{1.501718in}}{\pgfqpoint{3.905098in}{1.501718in}}%
\pgfpathcurveto{\pgfqpoint{3.896862in}{1.501718in}}{\pgfqpoint{3.888962in}{1.498445in}}{\pgfqpoint{3.883138in}{1.492621in}}%
\pgfpathcurveto{\pgfqpoint{3.877314in}{1.486798in}}{\pgfqpoint{3.874042in}{1.478898in}}{\pgfqpoint{3.874042in}{1.470661in}}%
\pgfpathcurveto{\pgfqpoint{3.874042in}{1.462425in}}{\pgfqpoint{3.877314in}{1.454525in}}{\pgfqpoint{3.883138in}{1.448701in}}%
\pgfpathcurveto{\pgfqpoint{3.888962in}{1.442877in}}{\pgfqpoint{3.896862in}{1.439605in}}{\pgfqpoint{3.905098in}{1.439605in}}%
\pgfpathclose%
\pgfusepath{stroke,fill}%
\end{pgfscope}%
\begin{pgfscope}%
\pgfpathrectangle{\pgfqpoint{3.793912in}{0.557870in}}{\pgfqpoint{2.446088in}{1.484734in}}%
\pgfusepath{clip}%
\pgfsetbuttcap%
\pgfsetroundjoin%
\definecolor{currentfill}{rgb}{0.298039,0.447059,0.690196}%
\pgfsetfillcolor{currentfill}%
\pgfsetlinewidth{1.003750pt}%
\definecolor{currentstroke}{rgb}{0.298039,0.447059,0.690196}%
\pgfsetstrokecolor{currentstroke}%
\pgfsetdash{}{0pt}%
\pgfpathmoveto{\pgfqpoint{3.905098in}{1.439605in}}%
\pgfpathcurveto{\pgfqpoint{3.913334in}{1.439605in}}{\pgfqpoint{3.921234in}{1.442877in}}{\pgfqpoint{3.927058in}{1.448701in}}%
\pgfpathcurveto{\pgfqpoint{3.932882in}{1.454525in}}{\pgfqpoint{3.936155in}{1.462425in}}{\pgfqpoint{3.936155in}{1.470661in}}%
\pgfpathcurveto{\pgfqpoint{3.936155in}{1.478898in}}{\pgfqpoint{3.932882in}{1.486798in}}{\pgfqpoint{3.927058in}{1.492621in}}%
\pgfpathcurveto{\pgfqpoint{3.921234in}{1.498445in}}{\pgfqpoint{3.913334in}{1.501718in}}{\pgfqpoint{3.905098in}{1.501718in}}%
\pgfpathcurveto{\pgfqpoint{3.896862in}{1.501718in}}{\pgfqpoint{3.888962in}{1.498445in}}{\pgfqpoint{3.883138in}{1.492621in}}%
\pgfpathcurveto{\pgfqpoint{3.877314in}{1.486798in}}{\pgfqpoint{3.874042in}{1.478898in}}{\pgfqpoint{3.874042in}{1.470661in}}%
\pgfpathcurveto{\pgfqpoint{3.874042in}{1.462425in}}{\pgfqpoint{3.877314in}{1.454525in}}{\pgfqpoint{3.883138in}{1.448701in}}%
\pgfpathcurveto{\pgfqpoint{3.888962in}{1.442877in}}{\pgfqpoint{3.896862in}{1.439605in}}{\pgfqpoint{3.905098in}{1.439605in}}%
\pgfpathclose%
\pgfusepath{stroke,fill}%
\end{pgfscope}%
\begin{pgfscope}%
\pgfpathrectangle{\pgfqpoint{3.793912in}{0.557870in}}{\pgfqpoint{2.446088in}{1.484734in}}%
\pgfusepath{clip}%
\pgfsetbuttcap%
\pgfsetroundjoin%
\definecolor{currentfill}{rgb}{0.298039,0.447059,0.690196}%
\pgfsetfillcolor{currentfill}%
\pgfsetlinewidth{1.003750pt}%
\definecolor{currentstroke}{rgb}{0.298039,0.447059,0.690196}%
\pgfsetstrokecolor{currentstroke}%
\pgfsetdash{}{0pt}%
\pgfpathmoveto{\pgfqpoint{3.905098in}{1.507774in}}%
\pgfpathcurveto{\pgfqpoint{3.913334in}{1.507774in}}{\pgfqpoint{3.921234in}{1.511047in}}{\pgfqpoint{3.927058in}{1.516871in}}%
\pgfpathcurveto{\pgfqpoint{3.932882in}{1.522694in}}{\pgfqpoint{3.936155in}{1.530595in}}{\pgfqpoint{3.936155in}{1.538831in}}%
\pgfpathcurveto{\pgfqpoint{3.936155in}{1.547067in}}{\pgfqpoint{3.932882in}{1.554967in}}{\pgfqpoint{3.927058in}{1.560791in}}%
\pgfpathcurveto{\pgfqpoint{3.921234in}{1.566615in}}{\pgfqpoint{3.913334in}{1.569887in}}{\pgfqpoint{3.905098in}{1.569887in}}%
\pgfpathcurveto{\pgfqpoint{3.896862in}{1.569887in}}{\pgfqpoint{3.888962in}{1.566615in}}{\pgfqpoint{3.883138in}{1.560791in}}%
\pgfpathcurveto{\pgfqpoint{3.877314in}{1.554967in}}{\pgfqpoint{3.874042in}{1.547067in}}{\pgfqpoint{3.874042in}{1.538831in}}%
\pgfpathcurveto{\pgfqpoint{3.874042in}{1.530595in}}{\pgfqpoint{3.877314in}{1.522694in}}{\pgfqpoint{3.883138in}{1.516871in}}%
\pgfpathcurveto{\pgfqpoint{3.888962in}{1.511047in}}{\pgfqpoint{3.896862in}{1.507774in}}{\pgfqpoint{3.905098in}{1.507774in}}%
\pgfpathclose%
\pgfusepath{stroke,fill}%
\end{pgfscope}%
\begin{pgfscope}%
\pgfpathrectangle{\pgfqpoint{3.793912in}{0.557870in}}{\pgfqpoint{2.446088in}{1.484734in}}%
\pgfusepath{clip}%
\pgfsetbuttcap%
\pgfsetroundjoin%
\definecolor{currentfill}{rgb}{0.298039,0.447059,0.690196}%
\pgfsetfillcolor{currentfill}%
\pgfsetlinewidth{1.003750pt}%
\definecolor{currentstroke}{rgb}{0.298039,0.447059,0.690196}%
\pgfsetstrokecolor{currentstroke}%
\pgfsetdash{}{0pt}%
\pgfpathmoveto{\pgfqpoint{5.433903in}{0.594302in}}%
\pgfpathcurveto{\pgfqpoint{5.442139in}{0.594302in}}{\pgfqpoint{5.450039in}{0.597574in}}{\pgfqpoint{5.455863in}{0.603398in}}%
\pgfpathcurveto{\pgfqpoint{5.461687in}{0.609222in}}{\pgfqpoint{5.464959in}{0.617122in}}{\pgfqpoint{5.464959in}{0.625358in}}%
\pgfpathcurveto{\pgfqpoint{5.464959in}{0.633594in}}{\pgfqpoint{5.461687in}{0.641495in}}{\pgfqpoint{5.455863in}{0.647318in}}%
\pgfpathcurveto{\pgfqpoint{5.450039in}{0.653142in}}{\pgfqpoint{5.442139in}{0.656415in}}{\pgfqpoint{5.433903in}{0.656415in}}%
\pgfpathcurveto{\pgfqpoint{5.425667in}{0.656415in}}{\pgfqpoint{5.417767in}{0.653142in}}{\pgfqpoint{5.411943in}{0.647318in}}%
\pgfpathcurveto{\pgfqpoint{5.406119in}{0.641495in}}{\pgfqpoint{5.402846in}{0.633594in}}{\pgfqpoint{5.402846in}{0.625358in}}%
\pgfpathcurveto{\pgfqpoint{5.402846in}{0.617122in}}{\pgfqpoint{5.406119in}{0.609222in}}{\pgfqpoint{5.411943in}{0.603398in}}%
\pgfpathcurveto{\pgfqpoint{5.417767in}{0.597574in}}{\pgfqpoint{5.425667in}{0.594302in}}{\pgfqpoint{5.433903in}{0.594302in}}%
\pgfpathclose%
\pgfusepath{stroke,fill}%
\end{pgfscope}%
\begin{pgfscope}%
\pgfpathrectangle{\pgfqpoint{3.793912in}{0.557870in}}{\pgfqpoint{2.446088in}{1.484734in}}%
\pgfusepath{clip}%
\pgfsetbuttcap%
\pgfsetroundjoin%
\definecolor{currentfill}{rgb}{0.298039,0.447059,0.690196}%
\pgfsetfillcolor{currentfill}%
\pgfsetlinewidth{1.003750pt}%
\definecolor{currentstroke}{rgb}{0.298039,0.447059,0.690196}%
\pgfsetstrokecolor{currentstroke}%
\pgfsetdash{}{0pt}%
\pgfpathmoveto{\pgfqpoint{3.905098in}{0.880614in}}%
\pgfpathcurveto{\pgfqpoint{3.913334in}{0.880614in}}{\pgfqpoint{3.921234in}{0.883886in}}{\pgfqpoint{3.927058in}{0.889710in}}%
\pgfpathcurveto{\pgfqpoint{3.932882in}{0.895534in}}{\pgfqpoint{3.936155in}{0.903434in}}{\pgfqpoint{3.936155in}{0.911671in}}%
\pgfpathcurveto{\pgfqpoint{3.936155in}{0.919907in}}{\pgfqpoint{3.932882in}{0.927807in}}{\pgfqpoint{3.927058in}{0.933631in}}%
\pgfpathcurveto{\pgfqpoint{3.921234in}{0.939455in}}{\pgfqpoint{3.913334in}{0.942727in}}{\pgfqpoint{3.905098in}{0.942727in}}%
\pgfpathcurveto{\pgfqpoint{3.896862in}{0.942727in}}{\pgfqpoint{3.888962in}{0.939455in}}{\pgfqpoint{3.883138in}{0.933631in}}%
\pgfpathcurveto{\pgfqpoint{3.877314in}{0.927807in}}{\pgfqpoint{3.874042in}{0.919907in}}{\pgfqpoint{3.874042in}{0.911671in}}%
\pgfpathcurveto{\pgfqpoint{3.874042in}{0.903434in}}{\pgfqpoint{3.877314in}{0.895534in}}{\pgfqpoint{3.883138in}{0.889710in}}%
\pgfpathcurveto{\pgfqpoint{3.888962in}{0.883886in}}{\pgfqpoint{3.896862in}{0.880614in}}{\pgfqpoint{3.905098in}{0.880614in}}%
\pgfpathclose%
\pgfusepath{stroke,fill}%
\end{pgfscope}%
\begin{pgfscope}%
\pgfpathrectangle{\pgfqpoint{3.793912in}{0.557870in}}{\pgfqpoint{2.446088in}{1.484734in}}%
\pgfusepath{clip}%
\pgfsetbuttcap%
\pgfsetroundjoin%
\definecolor{currentfill}{rgb}{0.298039,0.447059,0.690196}%
\pgfsetfillcolor{currentfill}%
\pgfsetlinewidth{1.003750pt}%
\definecolor{currentstroke}{rgb}{0.298039,0.447059,0.690196}%
\pgfsetstrokecolor{currentstroke}%
\pgfsetdash{}{0pt}%
\pgfpathmoveto{\pgfqpoint{4.901138in}{0.594302in}}%
\pgfpathcurveto{\pgfqpoint{4.909374in}{0.594302in}}{\pgfqpoint{4.917274in}{0.597574in}}{\pgfqpoint{4.923098in}{0.603398in}}%
\pgfpathcurveto{\pgfqpoint{4.928922in}{0.609222in}}{\pgfqpoint{4.932194in}{0.617122in}}{\pgfqpoint{4.932194in}{0.625358in}}%
\pgfpathcurveto{\pgfqpoint{4.932194in}{0.633594in}}{\pgfqpoint{4.928922in}{0.641495in}}{\pgfqpoint{4.923098in}{0.647318in}}%
\pgfpathcurveto{\pgfqpoint{4.917274in}{0.653142in}}{\pgfqpoint{4.909374in}{0.656415in}}{\pgfqpoint{4.901138in}{0.656415in}}%
\pgfpathcurveto{\pgfqpoint{4.892901in}{0.656415in}}{\pgfqpoint{4.885001in}{0.653142in}}{\pgfqpoint{4.879177in}{0.647318in}}%
\pgfpathcurveto{\pgfqpoint{4.873353in}{0.641495in}}{\pgfqpoint{4.870081in}{0.633594in}}{\pgfqpoint{4.870081in}{0.625358in}}%
\pgfpathcurveto{\pgfqpoint{4.870081in}{0.617122in}}{\pgfqpoint{4.873353in}{0.609222in}}{\pgfqpoint{4.879177in}{0.603398in}}%
\pgfpathcurveto{\pgfqpoint{4.885001in}{0.597574in}}{\pgfqpoint{4.892901in}{0.594302in}}{\pgfqpoint{4.901138in}{0.594302in}}%
\pgfpathclose%
\pgfusepath{stroke,fill}%
\end{pgfscope}%
\begin{pgfscope}%
\pgfpathrectangle{\pgfqpoint{3.793912in}{0.557870in}}{\pgfqpoint{2.446088in}{1.484734in}}%
\pgfusepath{clip}%
\pgfsetbuttcap%
\pgfsetroundjoin%
\definecolor{currentfill}{rgb}{0.298039,0.447059,0.690196}%
\pgfsetfillcolor{currentfill}%
\pgfsetlinewidth{1.003750pt}%
\definecolor{currentstroke}{rgb}{0.298039,0.447059,0.690196}%
\pgfsetstrokecolor{currentstroke}%
\pgfsetdash{}{0pt}%
\pgfpathmoveto{\pgfqpoint{3.905098in}{1.930426in}}%
\pgfpathcurveto{\pgfqpoint{3.913334in}{1.930426in}}{\pgfqpoint{3.921234in}{1.933698in}}{\pgfqpoint{3.927058in}{1.939522in}}%
\pgfpathcurveto{\pgfqpoint{3.932882in}{1.945346in}}{\pgfqpoint{3.936155in}{1.953246in}}{\pgfqpoint{3.936155in}{1.961482in}}%
\pgfpathcurveto{\pgfqpoint{3.936155in}{1.969719in}}{\pgfqpoint{3.932882in}{1.977619in}}{\pgfqpoint{3.927058in}{1.983443in}}%
\pgfpathcurveto{\pgfqpoint{3.921234in}{1.989267in}}{\pgfqpoint{3.913334in}{1.992539in}}{\pgfqpoint{3.905098in}{1.992539in}}%
\pgfpathcurveto{\pgfqpoint{3.896862in}{1.992539in}}{\pgfqpoint{3.888962in}{1.989267in}}{\pgfqpoint{3.883138in}{1.983443in}}%
\pgfpathcurveto{\pgfqpoint{3.877314in}{1.977619in}}{\pgfqpoint{3.874042in}{1.969719in}}{\pgfqpoint{3.874042in}{1.961482in}}%
\pgfpathcurveto{\pgfqpoint{3.874042in}{1.953246in}}{\pgfqpoint{3.877314in}{1.945346in}}{\pgfqpoint{3.883138in}{1.939522in}}%
\pgfpathcurveto{\pgfqpoint{3.888962in}{1.933698in}}{\pgfqpoint{3.896862in}{1.930426in}}{\pgfqpoint{3.905098in}{1.930426in}}%
\pgfpathclose%
\pgfusepath{stroke,fill}%
\end{pgfscope}%
\begin{pgfscope}%
\pgfpathrectangle{\pgfqpoint{3.793912in}{0.557870in}}{\pgfqpoint{2.446088in}{1.484734in}}%
\pgfusepath{clip}%
\pgfsetbuttcap%
\pgfsetroundjoin%
\definecolor{currentfill}{rgb}{0.298039,0.447059,0.690196}%
\pgfsetfillcolor{currentfill}%
\pgfsetlinewidth{1.003750pt}%
\definecolor{currentstroke}{rgb}{0.298039,0.447059,0.690196}%
\pgfsetstrokecolor{currentstroke}%
\pgfsetdash{}{0pt}%
\pgfpathmoveto{\pgfqpoint{3.905098in}{1.739551in}}%
\pgfpathcurveto{\pgfqpoint{3.913334in}{1.739551in}}{\pgfqpoint{3.921234in}{1.742823in}}{\pgfqpoint{3.927058in}{1.748647in}}%
\pgfpathcurveto{\pgfqpoint{3.932882in}{1.754471in}}{\pgfqpoint{3.936155in}{1.762371in}}{\pgfqpoint{3.936155in}{1.770607in}}%
\pgfpathcurveto{\pgfqpoint{3.936155in}{1.778844in}}{\pgfqpoint{3.932882in}{1.786744in}}{\pgfqpoint{3.927058in}{1.792568in}}%
\pgfpathcurveto{\pgfqpoint{3.921234in}{1.798392in}}{\pgfqpoint{3.913334in}{1.801664in}}{\pgfqpoint{3.905098in}{1.801664in}}%
\pgfpathcurveto{\pgfqpoint{3.896862in}{1.801664in}}{\pgfqpoint{3.888962in}{1.798392in}}{\pgfqpoint{3.883138in}{1.792568in}}%
\pgfpathcurveto{\pgfqpoint{3.877314in}{1.786744in}}{\pgfqpoint{3.874042in}{1.778844in}}{\pgfqpoint{3.874042in}{1.770607in}}%
\pgfpathcurveto{\pgfqpoint{3.874042in}{1.762371in}}{\pgfqpoint{3.877314in}{1.754471in}}{\pgfqpoint{3.883138in}{1.748647in}}%
\pgfpathcurveto{\pgfqpoint{3.888962in}{1.742823in}}{\pgfqpoint{3.896862in}{1.739551in}}{\pgfqpoint{3.905098in}{1.739551in}}%
\pgfpathclose%
\pgfusepath{stroke,fill}%
\end{pgfscope}%
\begin{pgfscope}%
\pgfpathrectangle{\pgfqpoint{3.793912in}{0.557870in}}{\pgfqpoint{2.446088in}{1.484734in}}%
\pgfusepath{clip}%
\pgfsetbuttcap%
\pgfsetroundjoin%
\definecolor{currentfill}{rgb}{0.298039,0.447059,0.690196}%
\pgfsetfillcolor{currentfill}%
\pgfsetlinewidth{1.003750pt}%
\definecolor{currentstroke}{rgb}{0.298039,0.447059,0.690196}%
\pgfsetstrokecolor{currentstroke}%
\pgfsetdash{}{0pt}%
\pgfpathmoveto{\pgfqpoint{3.905098in}{1.930426in}}%
\pgfpathcurveto{\pgfqpoint{3.913334in}{1.930426in}}{\pgfqpoint{3.921234in}{1.933698in}}{\pgfqpoint{3.927058in}{1.939522in}}%
\pgfpathcurveto{\pgfqpoint{3.932882in}{1.945346in}}{\pgfqpoint{3.936155in}{1.953246in}}{\pgfqpoint{3.936155in}{1.961482in}}%
\pgfpathcurveto{\pgfqpoint{3.936155in}{1.969719in}}{\pgfqpoint{3.932882in}{1.977619in}}{\pgfqpoint{3.927058in}{1.983443in}}%
\pgfpathcurveto{\pgfqpoint{3.921234in}{1.989267in}}{\pgfqpoint{3.913334in}{1.992539in}}{\pgfqpoint{3.905098in}{1.992539in}}%
\pgfpathcurveto{\pgfqpoint{3.896862in}{1.992539in}}{\pgfqpoint{3.888962in}{1.989267in}}{\pgfqpoint{3.883138in}{1.983443in}}%
\pgfpathcurveto{\pgfqpoint{3.877314in}{1.977619in}}{\pgfqpoint{3.874042in}{1.969719in}}{\pgfqpoint{3.874042in}{1.961482in}}%
\pgfpathcurveto{\pgfqpoint{3.874042in}{1.953246in}}{\pgfqpoint{3.877314in}{1.945346in}}{\pgfqpoint{3.883138in}{1.939522in}}%
\pgfpathcurveto{\pgfqpoint{3.888962in}{1.933698in}}{\pgfqpoint{3.896862in}{1.930426in}}{\pgfqpoint{3.905098in}{1.930426in}}%
\pgfpathclose%
\pgfusepath{stroke,fill}%
\end{pgfscope}%
\begin{pgfscope}%
\pgfpathrectangle{\pgfqpoint{3.793912in}{0.557870in}}{\pgfqpoint{2.446088in}{1.484734in}}%
\pgfusepath{clip}%
\pgfsetbuttcap%
\pgfsetroundjoin%
\definecolor{currentfill}{rgb}{0.298039,0.447059,0.690196}%
\pgfsetfillcolor{currentfill}%
\pgfsetlinewidth{1.003750pt}%
\definecolor{currentstroke}{rgb}{0.298039,0.447059,0.690196}%
\pgfsetstrokecolor{currentstroke}%
\pgfsetdash{}{0pt}%
\pgfpathmoveto{\pgfqpoint{5.457067in}{0.594302in}}%
\pgfpathcurveto{\pgfqpoint{5.465303in}{0.594302in}}{\pgfqpoint{5.473203in}{0.597574in}}{\pgfqpoint{5.479027in}{0.603398in}}%
\pgfpathcurveto{\pgfqpoint{5.484851in}{0.609222in}}{\pgfqpoint{5.488123in}{0.617122in}}{\pgfqpoint{5.488123in}{0.625358in}}%
\pgfpathcurveto{\pgfqpoint{5.488123in}{0.633594in}}{\pgfqpoint{5.484851in}{0.641495in}}{\pgfqpoint{5.479027in}{0.647318in}}%
\pgfpathcurveto{\pgfqpoint{5.473203in}{0.653142in}}{\pgfqpoint{5.465303in}{0.656415in}}{\pgfqpoint{5.457067in}{0.656415in}}%
\pgfpathcurveto{\pgfqpoint{5.448830in}{0.656415in}}{\pgfqpoint{5.440930in}{0.653142in}}{\pgfqpoint{5.435106in}{0.647318in}}%
\pgfpathcurveto{\pgfqpoint{5.429282in}{0.641495in}}{\pgfqpoint{5.426010in}{0.633594in}}{\pgfqpoint{5.426010in}{0.625358in}}%
\pgfpathcurveto{\pgfqpoint{5.426010in}{0.617122in}}{\pgfqpoint{5.429282in}{0.609222in}}{\pgfqpoint{5.435106in}{0.603398in}}%
\pgfpathcurveto{\pgfqpoint{5.440930in}{0.597574in}}{\pgfqpoint{5.448830in}{0.594302in}}{\pgfqpoint{5.457067in}{0.594302in}}%
\pgfpathclose%
\pgfusepath{stroke,fill}%
\end{pgfscope}%
\begin{pgfscope}%
\pgfpathrectangle{\pgfqpoint{3.793912in}{0.557870in}}{\pgfqpoint{2.446088in}{1.484734in}}%
\pgfusepath{clip}%
\pgfsetbuttcap%
\pgfsetroundjoin%
\definecolor{currentfill}{rgb}{0.298039,0.447059,0.690196}%
\pgfsetfillcolor{currentfill}%
\pgfsetlinewidth{1.003750pt}%
\definecolor{currentstroke}{rgb}{0.298039,0.447059,0.690196}%
\pgfsetstrokecolor{currentstroke}%
\pgfsetdash{}{0pt}%
\pgfpathmoveto{\pgfqpoint{3.905098in}{1.930426in}}%
\pgfpathcurveto{\pgfqpoint{3.913334in}{1.930426in}}{\pgfqpoint{3.921234in}{1.933698in}}{\pgfqpoint{3.927058in}{1.939522in}}%
\pgfpathcurveto{\pgfqpoint{3.932882in}{1.945346in}}{\pgfqpoint{3.936155in}{1.953246in}}{\pgfqpoint{3.936155in}{1.961482in}}%
\pgfpathcurveto{\pgfqpoint{3.936155in}{1.969719in}}{\pgfqpoint{3.932882in}{1.977619in}}{\pgfqpoint{3.927058in}{1.983443in}}%
\pgfpathcurveto{\pgfqpoint{3.921234in}{1.989267in}}{\pgfqpoint{3.913334in}{1.992539in}}{\pgfqpoint{3.905098in}{1.992539in}}%
\pgfpathcurveto{\pgfqpoint{3.896862in}{1.992539in}}{\pgfqpoint{3.888962in}{1.989267in}}{\pgfqpoint{3.883138in}{1.983443in}}%
\pgfpathcurveto{\pgfqpoint{3.877314in}{1.977619in}}{\pgfqpoint{3.874042in}{1.969719in}}{\pgfqpoint{3.874042in}{1.961482in}}%
\pgfpathcurveto{\pgfqpoint{3.874042in}{1.953246in}}{\pgfqpoint{3.877314in}{1.945346in}}{\pgfqpoint{3.883138in}{1.939522in}}%
\pgfpathcurveto{\pgfqpoint{3.888962in}{1.933698in}}{\pgfqpoint{3.896862in}{1.930426in}}{\pgfqpoint{3.905098in}{1.930426in}}%
\pgfpathclose%
\pgfusepath{stroke,fill}%
\end{pgfscope}%
\begin{pgfscope}%
\pgfpathrectangle{\pgfqpoint{3.793912in}{0.557870in}}{\pgfqpoint{2.446088in}{1.484734in}}%
\pgfusepath{clip}%
\pgfsetbuttcap%
\pgfsetroundjoin%
\definecolor{currentfill}{rgb}{0.298039,0.447059,0.690196}%
\pgfsetfillcolor{currentfill}%
\pgfsetlinewidth{1.003750pt}%
\definecolor{currentstroke}{rgb}{0.298039,0.447059,0.690196}%
\pgfsetstrokecolor{currentstroke}%
\pgfsetdash{}{0pt}%
\pgfpathmoveto{\pgfqpoint{3.905098in}{1.930426in}}%
\pgfpathcurveto{\pgfqpoint{3.913334in}{1.930426in}}{\pgfqpoint{3.921234in}{1.933698in}}{\pgfqpoint{3.927058in}{1.939522in}}%
\pgfpathcurveto{\pgfqpoint{3.932882in}{1.945346in}}{\pgfqpoint{3.936155in}{1.953246in}}{\pgfqpoint{3.936155in}{1.961482in}}%
\pgfpathcurveto{\pgfqpoint{3.936155in}{1.969719in}}{\pgfqpoint{3.932882in}{1.977619in}}{\pgfqpoint{3.927058in}{1.983443in}}%
\pgfpathcurveto{\pgfqpoint{3.921234in}{1.989267in}}{\pgfqpoint{3.913334in}{1.992539in}}{\pgfqpoint{3.905098in}{1.992539in}}%
\pgfpathcurveto{\pgfqpoint{3.896862in}{1.992539in}}{\pgfqpoint{3.888962in}{1.989267in}}{\pgfqpoint{3.883138in}{1.983443in}}%
\pgfpathcurveto{\pgfqpoint{3.877314in}{1.977619in}}{\pgfqpoint{3.874042in}{1.969719in}}{\pgfqpoint{3.874042in}{1.961482in}}%
\pgfpathcurveto{\pgfqpoint{3.874042in}{1.953246in}}{\pgfqpoint{3.877314in}{1.945346in}}{\pgfqpoint{3.883138in}{1.939522in}}%
\pgfpathcurveto{\pgfqpoint{3.888962in}{1.933698in}}{\pgfqpoint{3.896862in}{1.930426in}}{\pgfqpoint{3.905098in}{1.930426in}}%
\pgfpathclose%
\pgfusepath{stroke,fill}%
\end{pgfscope}%
\begin{pgfscope}%
\pgfpathrectangle{\pgfqpoint{3.793912in}{0.557870in}}{\pgfqpoint{2.446088in}{1.484734in}}%
\pgfusepath{clip}%
\pgfsetbuttcap%
\pgfsetroundjoin%
\definecolor{currentfill}{rgb}{0.298039,0.447059,0.690196}%
\pgfsetfillcolor{currentfill}%
\pgfsetlinewidth{1.003750pt}%
\definecolor{currentstroke}{rgb}{0.298039,0.447059,0.690196}%
\pgfsetstrokecolor{currentstroke}%
\pgfsetdash{}{0pt}%
\pgfpathmoveto{\pgfqpoint{3.905098in}{1.930426in}}%
\pgfpathcurveto{\pgfqpoint{3.913334in}{1.930426in}}{\pgfqpoint{3.921234in}{1.933698in}}{\pgfqpoint{3.927058in}{1.939522in}}%
\pgfpathcurveto{\pgfqpoint{3.932882in}{1.945346in}}{\pgfqpoint{3.936155in}{1.953246in}}{\pgfqpoint{3.936155in}{1.961482in}}%
\pgfpathcurveto{\pgfqpoint{3.936155in}{1.969719in}}{\pgfqpoint{3.932882in}{1.977619in}}{\pgfqpoint{3.927058in}{1.983443in}}%
\pgfpathcurveto{\pgfqpoint{3.921234in}{1.989267in}}{\pgfqpoint{3.913334in}{1.992539in}}{\pgfqpoint{3.905098in}{1.992539in}}%
\pgfpathcurveto{\pgfqpoint{3.896862in}{1.992539in}}{\pgfqpoint{3.888962in}{1.989267in}}{\pgfqpoint{3.883138in}{1.983443in}}%
\pgfpathcurveto{\pgfqpoint{3.877314in}{1.977619in}}{\pgfqpoint{3.874042in}{1.969719in}}{\pgfqpoint{3.874042in}{1.961482in}}%
\pgfpathcurveto{\pgfqpoint{3.874042in}{1.953246in}}{\pgfqpoint{3.877314in}{1.945346in}}{\pgfqpoint{3.883138in}{1.939522in}}%
\pgfpathcurveto{\pgfqpoint{3.888962in}{1.933698in}}{\pgfqpoint{3.896862in}{1.930426in}}{\pgfqpoint{3.905098in}{1.930426in}}%
\pgfpathclose%
\pgfusepath{stroke,fill}%
\end{pgfscope}%
\begin{pgfscope}%
\pgfpathrectangle{\pgfqpoint{3.793912in}{0.557870in}}{\pgfqpoint{2.446088in}{1.484734in}}%
\pgfusepath{clip}%
\pgfsetbuttcap%
\pgfsetroundjoin%
\definecolor{currentfill}{rgb}{0.298039,0.447059,0.690196}%
\pgfsetfillcolor{currentfill}%
\pgfsetlinewidth{1.003750pt}%
\definecolor{currentstroke}{rgb}{0.298039,0.447059,0.690196}%
\pgfsetstrokecolor{currentstroke}%
\pgfsetdash{}{0pt}%
\pgfpathmoveto{\pgfqpoint{3.905098in}{1.930426in}}%
\pgfpathcurveto{\pgfqpoint{3.913334in}{1.930426in}}{\pgfqpoint{3.921234in}{1.933698in}}{\pgfqpoint{3.927058in}{1.939522in}}%
\pgfpathcurveto{\pgfqpoint{3.932882in}{1.945346in}}{\pgfqpoint{3.936155in}{1.953246in}}{\pgfqpoint{3.936155in}{1.961482in}}%
\pgfpathcurveto{\pgfqpoint{3.936155in}{1.969719in}}{\pgfqpoint{3.932882in}{1.977619in}}{\pgfqpoint{3.927058in}{1.983443in}}%
\pgfpathcurveto{\pgfqpoint{3.921234in}{1.989267in}}{\pgfqpoint{3.913334in}{1.992539in}}{\pgfqpoint{3.905098in}{1.992539in}}%
\pgfpathcurveto{\pgfqpoint{3.896862in}{1.992539in}}{\pgfqpoint{3.888962in}{1.989267in}}{\pgfqpoint{3.883138in}{1.983443in}}%
\pgfpathcurveto{\pgfqpoint{3.877314in}{1.977619in}}{\pgfqpoint{3.874042in}{1.969719in}}{\pgfqpoint{3.874042in}{1.961482in}}%
\pgfpathcurveto{\pgfqpoint{3.874042in}{1.953246in}}{\pgfqpoint{3.877314in}{1.945346in}}{\pgfqpoint{3.883138in}{1.939522in}}%
\pgfpathcurveto{\pgfqpoint{3.888962in}{1.933698in}}{\pgfqpoint{3.896862in}{1.930426in}}{\pgfqpoint{3.905098in}{1.930426in}}%
\pgfpathclose%
\pgfusepath{stroke,fill}%
\end{pgfscope}%
\begin{pgfscope}%
\pgfpathrectangle{\pgfqpoint{3.793912in}{0.557870in}}{\pgfqpoint{2.446088in}{1.484734in}}%
\pgfusepath{clip}%
\pgfsetbuttcap%
\pgfsetroundjoin%
\definecolor{currentfill}{rgb}{0.298039,0.447059,0.690196}%
\pgfsetfillcolor{currentfill}%
\pgfsetlinewidth{1.003750pt}%
\definecolor{currentstroke}{rgb}{0.298039,0.447059,0.690196}%
\pgfsetstrokecolor{currentstroke}%
\pgfsetdash{}{0pt}%
\pgfpathmoveto{\pgfqpoint{3.905098in}{1.071489in}}%
\pgfpathcurveto{\pgfqpoint{3.913334in}{1.071489in}}{\pgfqpoint{3.921234in}{1.074761in}}{\pgfqpoint{3.927058in}{1.080585in}}%
\pgfpathcurveto{\pgfqpoint{3.932882in}{1.086409in}}{\pgfqpoint{3.936155in}{1.094309in}}{\pgfqpoint{3.936155in}{1.102545in}}%
\pgfpathcurveto{\pgfqpoint{3.936155in}{1.110782in}}{\pgfqpoint{3.932882in}{1.118682in}}{\pgfqpoint{3.927058in}{1.124506in}}%
\pgfpathcurveto{\pgfqpoint{3.921234in}{1.130330in}}{\pgfqpoint{3.913334in}{1.133602in}}{\pgfqpoint{3.905098in}{1.133602in}}%
\pgfpathcurveto{\pgfqpoint{3.896862in}{1.133602in}}{\pgfqpoint{3.888962in}{1.130330in}}{\pgfqpoint{3.883138in}{1.124506in}}%
\pgfpathcurveto{\pgfqpoint{3.877314in}{1.118682in}}{\pgfqpoint{3.874042in}{1.110782in}}{\pgfqpoint{3.874042in}{1.102545in}}%
\pgfpathcurveto{\pgfqpoint{3.874042in}{1.094309in}}{\pgfqpoint{3.877314in}{1.086409in}}{\pgfqpoint{3.883138in}{1.080585in}}%
\pgfpathcurveto{\pgfqpoint{3.888962in}{1.074761in}}{\pgfqpoint{3.896862in}{1.071489in}}{\pgfqpoint{3.905098in}{1.071489in}}%
\pgfpathclose%
\pgfusepath{stroke,fill}%
\end{pgfscope}%
\begin{pgfscope}%
\pgfpathrectangle{\pgfqpoint{3.793912in}{0.557870in}}{\pgfqpoint{2.446088in}{1.484734in}}%
\pgfusepath{clip}%
\pgfsetbuttcap%
\pgfsetroundjoin%
\definecolor{currentfill}{rgb}{0.298039,0.447059,0.690196}%
\pgfsetfillcolor{currentfill}%
\pgfsetlinewidth{1.003750pt}%
\definecolor{currentstroke}{rgb}{0.298039,0.447059,0.690196}%
\pgfsetstrokecolor{currentstroke}%
\pgfsetdash{}{0pt}%
\pgfpathmoveto{\pgfqpoint{3.905098in}{1.930426in}}%
\pgfpathcurveto{\pgfqpoint{3.913334in}{1.930426in}}{\pgfqpoint{3.921234in}{1.933698in}}{\pgfqpoint{3.927058in}{1.939522in}}%
\pgfpathcurveto{\pgfqpoint{3.932882in}{1.945346in}}{\pgfqpoint{3.936155in}{1.953246in}}{\pgfqpoint{3.936155in}{1.961482in}}%
\pgfpathcurveto{\pgfqpoint{3.936155in}{1.969719in}}{\pgfqpoint{3.932882in}{1.977619in}}{\pgfqpoint{3.927058in}{1.983443in}}%
\pgfpathcurveto{\pgfqpoint{3.921234in}{1.989267in}}{\pgfqpoint{3.913334in}{1.992539in}}{\pgfqpoint{3.905098in}{1.992539in}}%
\pgfpathcurveto{\pgfqpoint{3.896862in}{1.992539in}}{\pgfqpoint{3.888962in}{1.989267in}}{\pgfqpoint{3.883138in}{1.983443in}}%
\pgfpathcurveto{\pgfqpoint{3.877314in}{1.977619in}}{\pgfqpoint{3.874042in}{1.969719in}}{\pgfqpoint{3.874042in}{1.961482in}}%
\pgfpathcurveto{\pgfqpoint{3.874042in}{1.953246in}}{\pgfqpoint{3.877314in}{1.945346in}}{\pgfqpoint{3.883138in}{1.939522in}}%
\pgfpathcurveto{\pgfqpoint{3.888962in}{1.933698in}}{\pgfqpoint{3.896862in}{1.930426in}}{\pgfqpoint{3.905098in}{1.930426in}}%
\pgfpathclose%
\pgfusepath{stroke,fill}%
\end{pgfscope}%
\begin{pgfscope}%
\pgfpathrectangle{\pgfqpoint{3.793912in}{0.557870in}}{\pgfqpoint{2.446088in}{1.484734in}}%
\pgfusepath{clip}%
\pgfsetbuttcap%
\pgfsetroundjoin%
\definecolor{currentfill}{rgb}{0.298039,0.447059,0.690196}%
\pgfsetfillcolor{currentfill}%
\pgfsetlinewidth{1.003750pt}%
\definecolor{currentstroke}{rgb}{0.298039,0.447059,0.690196}%
\pgfsetstrokecolor{currentstroke}%
\pgfsetdash{}{0pt}%
\pgfpathmoveto{\pgfqpoint{5.179102in}{0.594302in}}%
\pgfpathcurveto{\pgfqpoint{5.187338in}{0.594302in}}{\pgfqpoint{5.195238in}{0.597574in}}{\pgfqpoint{5.201062in}{0.603398in}}%
\pgfpathcurveto{\pgfqpoint{5.206886in}{0.609222in}}{\pgfqpoint{5.210159in}{0.617122in}}{\pgfqpoint{5.210159in}{0.625358in}}%
\pgfpathcurveto{\pgfqpoint{5.210159in}{0.633594in}}{\pgfqpoint{5.206886in}{0.641495in}}{\pgfqpoint{5.201062in}{0.647318in}}%
\pgfpathcurveto{\pgfqpoint{5.195238in}{0.653142in}}{\pgfqpoint{5.187338in}{0.656415in}}{\pgfqpoint{5.179102in}{0.656415in}}%
\pgfpathcurveto{\pgfqpoint{5.170866in}{0.656415in}}{\pgfqpoint{5.162966in}{0.653142in}}{\pgfqpoint{5.157142in}{0.647318in}}%
\pgfpathcurveto{\pgfqpoint{5.151318in}{0.641495in}}{\pgfqpoint{5.148046in}{0.633594in}}{\pgfqpoint{5.148046in}{0.625358in}}%
\pgfpathcurveto{\pgfqpoint{5.148046in}{0.617122in}}{\pgfqpoint{5.151318in}{0.609222in}}{\pgfqpoint{5.157142in}{0.603398in}}%
\pgfpathcurveto{\pgfqpoint{5.162966in}{0.597574in}}{\pgfqpoint{5.170866in}{0.594302in}}{\pgfqpoint{5.179102in}{0.594302in}}%
\pgfpathclose%
\pgfusepath{stroke,fill}%
\end{pgfscope}%
\begin{pgfscope}%
\pgfpathrectangle{\pgfqpoint{3.793912in}{0.557870in}}{\pgfqpoint{2.446088in}{1.484734in}}%
\pgfusepath{clip}%
\pgfsetbuttcap%
\pgfsetroundjoin%
\definecolor{currentfill}{rgb}{0.298039,0.447059,0.690196}%
\pgfsetfillcolor{currentfill}%
\pgfsetlinewidth{1.003750pt}%
\definecolor{currentstroke}{rgb}{0.298039,0.447059,0.690196}%
\pgfsetstrokecolor{currentstroke}%
\pgfsetdash{}{0pt}%
\pgfpathmoveto{\pgfqpoint{3.905098in}{1.930426in}}%
\pgfpathcurveto{\pgfqpoint{3.913334in}{1.930426in}}{\pgfqpoint{3.921234in}{1.933698in}}{\pgfqpoint{3.927058in}{1.939522in}}%
\pgfpathcurveto{\pgfqpoint{3.932882in}{1.945346in}}{\pgfqpoint{3.936155in}{1.953246in}}{\pgfqpoint{3.936155in}{1.961482in}}%
\pgfpathcurveto{\pgfqpoint{3.936155in}{1.969719in}}{\pgfqpoint{3.932882in}{1.977619in}}{\pgfqpoint{3.927058in}{1.983443in}}%
\pgfpathcurveto{\pgfqpoint{3.921234in}{1.989267in}}{\pgfqpoint{3.913334in}{1.992539in}}{\pgfqpoint{3.905098in}{1.992539in}}%
\pgfpathcurveto{\pgfqpoint{3.896862in}{1.992539in}}{\pgfqpoint{3.888962in}{1.989267in}}{\pgfqpoint{3.883138in}{1.983443in}}%
\pgfpathcurveto{\pgfqpoint{3.877314in}{1.977619in}}{\pgfqpoint{3.874042in}{1.969719in}}{\pgfqpoint{3.874042in}{1.961482in}}%
\pgfpathcurveto{\pgfqpoint{3.874042in}{1.953246in}}{\pgfqpoint{3.877314in}{1.945346in}}{\pgfqpoint{3.883138in}{1.939522in}}%
\pgfpathcurveto{\pgfqpoint{3.888962in}{1.933698in}}{\pgfqpoint{3.896862in}{1.930426in}}{\pgfqpoint{3.905098in}{1.930426in}}%
\pgfpathclose%
\pgfusepath{stroke,fill}%
\end{pgfscope}%
\begin{pgfscope}%
\pgfpathrectangle{\pgfqpoint{3.793912in}{0.557870in}}{\pgfqpoint{2.446088in}{1.484734in}}%
\pgfusepath{clip}%
\pgfsetbuttcap%
\pgfsetroundjoin%
\definecolor{currentfill}{rgb}{0.298039,0.447059,0.690196}%
\pgfsetfillcolor{currentfill}%
\pgfsetlinewidth{1.003750pt}%
\definecolor{currentstroke}{rgb}{0.298039,0.447059,0.690196}%
\pgfsetstrokecolor{currentstroke}%
\pgfsetdash{}{0pt}%
\pgfpathmoveto{\pgfqpoint{3.905098in}{1.930426in}}%
\pgfpathcurveto{\pgfqpoint{3.913334in}{1.930426in}}{\pgfqpoint{3.921234in}{1.933698in}}{\pgfqpoint{3.927058in}{1.939522in}}%
\pgfpathcurveto{\pgfqpoint{3.932882in}{1.945346in}}{\pgfqpoint{3.936155in}{1.953246in}}{\pgfqpoint{3.936155in}{1.961482in}}%
\pgfpathcurveto{\pgfqpoint{3.936155in}{1.969719in}}{\pgfqpoint{3.932882in}{1.977619in}}{\pgfqpoint{3.927058in}{1.983443in}}%
\pgfpathcurveto{\pgfqpoint{3.921234in}{1.989267in}}{\pgfqpoint{3.913334in}{1.992539in}}{\pgfqpoint{3.905098in}{1.992539in}}%
\pgfpathcurveto{\pgfqpoint{3.896862in}{1.992539in}}{\pgfqpoint{3.888962in}{1.989267in}}{\pgfqpoint{3.883138in}{1.983443in}}%
\pgfpathcurveto{\pgfqpoint{3.877314in}{1.977619in}}{\pgfqpoint{3.874042in}{1.969719in}}{\pgfqpoint{3.874042in}{1.961482in}}%
\pgfpathcurveto{\pgfqpoint{3.874042in}{1.953246in}}{\pgfqpoint{3.877314in}{1.945346in}}{\pgfqpoint{3.883138in}{1.939522in}}%
\pgfpathcurveto{\pgfqpoint{3.888962in}{1.933698in}}{\pgfqpoint{3.896862in}{1.930426in}}{\pgfqpoint{3.905098in}{1.930426in}}%
\pgfpathclose%
\pgfusepath{stroke,fill}%
\end{pgfscope}%
\begin{pgfscope}%
\pgfpathrectangle{\pgfqpoint{3.793912in}{0.557870in}}{\pgfqpoint{2.446088in}{1.484734in}}%
\pgfusepath{clip}%
\pgfsetbuttcap%
\pgfsetroundjoin%
\definecolor{currentfill}{rgb}{0.298039,0.447059,0.690196}%
\pgfsetfillcolor{currentfill}%
\pgfsetlinewidth{1.003750pt}%
\definecolor{currentstroke}{rgb}{0.298039,0.447059,0.690196}%
\pgfsetstrokecolor{currentstroke}%
\pgfsetdash{}{0pt}%
\pgfpathmoveto{\pgfqpoint{3.905098in}{1.930426in}}%
\pgfpathcurveto{\pgfqpoint{3.913334in}{1.930426in}}{\pgfqpoint{3.921234in}{1.933698in}}{\pgfqpoint{3.927058in}{1.939522in}}%
\pgfpathcurveto{\pgfqpoint{3.932882in}{1.945346in}}{\pgfqpoint{3.936155in}{1.953246in}}{\pgfqpoint{3.936155in}{1.961482in}}%
\pgfpathcurveto{\pgfqpoint{3.936155in}{1.969719in}}{\pgfqpoint{3.932882in}{1.977619in}}{\pgfqpoint{3.927058in}{1.983443in}}%
\pgfpathcurveto{\pgfqpoint{3.921234in}{1.989267in}}{\pgfqpoint{3.913334in}{1.992539in}}{\pgfqpoint{3.905098in}{1.992539in}}%
\pgfpathcurveto{\pgfqpoint{3.896862in}{1.992539in}}{\pgfqpoint{3.888962in}{1.989267in}}{\pgfqpoint{3.883138in}{1.983443in}}%
\pgfpathcurveto{\pgfqpoint{3.877314in}{1.977619in}}{\pgfqpoint{3.874042in}{1.969719in}}{\pgfqpoint{3.874042in}{1.961482in}}%
\pgfpathcurveto{\pgfqpoint{3.874042in}{1.953246in}}{\pgfqpoint{3.877314in}{1.945346in}}{\pgfqpoint{3.883138in}{1.939522in}}%
\pgfpathcurveto{\pgfqpoint{3.888962in}{1.933698in}}{\pgfqpoint{3.896862in}{1.930426in}}{\pgfqpoint{3.905098in}{1.930426in}}%
\pgfpathclose%
\pgfusepath{stroke,fill}%
\end{pgfscope}%
\begin{pgfscope}%
\pgfpathrectangle{\pgfqpoint{3.793912in}{0.557870in}}{\pgfqpoint{2.446088in}{1.484734in}}%
\pgfusepath{clip}%
\pgfsetbuttcap%
\pgfsetroundjoin%
\definecolor{currentfill}{rgb}{0.298039,0.447059,0.690196}%
\pgfsetfillcolor{currentfill}%
\pgfsetlinewidth{1.003750pt}%
\definecolor{currentstroke}{rgb}{0.298039,0.447059,0.690196}%
\pgfsetstrokecolor{currentstroke}%
\pgfsetdash{}{0pt}%
\pgfpathmoveto{\pgfqpoint{5.526558in}{0.594302in}}%
\pgfpathcurveto{\pgfqpoint{5.534794in}{0.594302in}}{\pgfqpoint{5.542694in}{0.597574in}}{\pgfqpoint{5.548518in}{0.603398in}}%
\pgfpathcurveto{\pgfqpoint{5.554342in}{0.609222in}}{\pgfqpoint{5.557614in}{0.617122in}}{\pgfqpoint{5.557614in}{0.625358in}}%
\pgfpathcurveto{\pgfqpoint{5.557614in}{0.633594in}}{\pgfqpoint{5.554342in}{0.641495in}}{\pgfqpoint{5.548518in}{0.647318in}}%
\pgfpathcurveto{\pgfqpoint{5.542694in}{0.653142in}}{\pgfqpoint{5.534794in}{0.656415in}}{\pgfqpoint{5.526558in}{0.656415in}}%
\pgfpathcurveto{\pgfqpoint{5.518321in}{0.656415in}}{\pgfqpoint{5.510421in}{0.653142in}}{\pgfqpoint{5.504597in}{0.647318in}}%
\pgfpathcurveto{\pgfqpoint{5.498774in}{0.641495in}}{\pgfqpoint{5.495501in}{0.633594in}}{\pgfqpoint{5.495501in}{0.625358in}}%
\pgfpathcurveto{\pgfqpoint{5.495501in}{0.617122in}}{\pgfqpoint{5.498774in}{0.609222in}}{\pgfqpoint{5.504597in}{0.603398in}}%
\pgfpathcurveto{\pgfqpoint{5.510421in}{0.597574in}}{\pgfqpoint{5.518321in}{0.594302in}}{\pgfqpoint{5.526558in}{0.594302in}}%
\pgfpathclose%
\pgfusepath{stroke,fill}%
\end{pgfscope}%
\begin{pgfscope}%
\pgfpathrectangle{\pgfqpoint{3.793912in}{0.557870in}}{\pgfqpoint{2.446088in}{1.484734in}}%
\pgfusepath{clip}%
\pgfsetbuttcap%
\pgfsetroundjoin%
\definecolor{currentfill}{rgb}{0.298039,0.447059,0.690196}%
\pgfsetfillcolor{currentfill}%
\pgfsetlinewidth{1.003750pt}%
\definecolor{currentstroke}{rgb}{0.298039,0.447059,0.690196}%
\pgfsetstrokecolor{currentstroke}%
\pgfsetdash{}{0pt}%
\pgfpathmoveto{\pgfqpoint{3.905098in}{1.589578in}}%
\pgfpathcurveto{\pgfqpoint{3.913334in}{1.589578in}}{\pgfqpoint{3.921234in}{1.592850in}}{\pgfqpoint{3.927058in}{1.598674in}}%
\pgfpathcurveto{\pgfqpoint{3.932882in}{1.604498in}}{\pgfqpoint{3.936155in}{1.612398in}}{\pgfqpoint{3.936155in}{1.620634in}}%
\pgfpathcurveto{\pgfqpoint{3.936155in}{1.628871in}}{\pgfqpoint{3.932882in}{1.636771in}}{\pgfqpoint{3.927058in}{1.642595in}}%
\pgfpathcurveto{\pgfqpoint{3.921234in}{1.648419in}}{\pgfqpoint{3.913334in}{1.651691in}}{\pgfqpoint{3.905098in}{1.651691in}}%
\pgfpathcurveto{\pgfqpoint{3.896862in}{1.651691in}}{\pgfqpoint{3.888962in}{1.648419in}}{\pgfqpoint{3.883138in}{1.642595in}}%
\pgfpathcurveto{\pgfqpoint{3.877314in}{1.636771in}}{\pgfqpoint{3.874042in}{1.628871in}}{\pgfqpoint{3.874042in}{1.620634in}}%
\pgfpathcurveto{\pgfqpoint{3.874042in}{1.612398in}}{\pgfqpoint{3.877314in}{1.604498in}}{\pgfqpoint{3.883138in}{1.598674in}}%
\pgfpathcurveto{\pgfqpoint{3.888962in}{1.592850in}}{\pgfqpoint{3.896862in}{1.589578in}}{\pgfqpoint{3.905098in}{1.589578in}}%
\pgfpathclose%
\pgfusepath{stroke,fill}%
\end{pgfscope}%
\begin{pgfscope}%
\pgfpathrectangle{\pgfqpoint{3.793912in}{0.557870in}}{\pgfqpoint{2.446088in}{1.484734in}}%
\pgfusepath{clip}%
\pgfsetbuttcap%
\pgfsetroundjoin%
\definecolor{currentfill}{rgb}{0.298039,0.447059,0.690196}%
\pgfsetfillcolor{currentfill}%
\pgfsetlinewidth{1.003750pt}%
\definecolor{currentstroke}{rgb}{0.298039,0.447059,0.690196}%
\pgfsetstrokecolor{currentstroke}%
\pgfsetdash{}{0pt}%
\pgfpathmoveto{\pgfqpoint{3.905098in}{1.507774in}}%
\pgfpathcurveto{\pgfqpoint{3.913334in}{1.507774in}}{\pgfqpoint{3.921234in}{1.511047in}}{\pgfqpoint{3.927058in}{1.516871in}}%
\pgfpathcurveto{\pgfqpoint{3.932882in}{1.522694in}}{\pgfqpoint{3.936155in}{1.530595in}}{\pgfqpoint{3.936155in}{1.538831in}}%
\pgfpathcurveto{\pgfqpoint{3.936155in}{1.547067in}}{\pgfqpoint{3.932882in}{1.554967in}}{\pgfqpoint{3.927058in}{1.560791in}}%
\pgfpathcurveto{\pgfqpoint{3.921234in}{1.566615in}}{\pgfqpoint{3.913334in}{1.569887in}}{\pgfqpoint{3.905098in}{1.569887in}}%
\pgfpathcurveto{\pgfqpoint{3.896862in}{1.569887in}}{\pgfqpoint{3.888962in}{1.566615in}}{\pgfqpoint{3.883138in}{1.560791in}}%
\pgfpathcurveto{\pgfqpoint{3.877314in}{1.554967in}}{\pgfqpoint{3.874042in}{1.547067in}}{\pgfqpoint{3.874042in}{1.538831in}}%
\pgfpathcurveto{\pgfqpoint{3.874042in}{1.530595in}}{\pgfqpoint{3.877314in}{1.522694in}}{\pgfqpoint{3.883138in}{1.516871in}}%
\pgfpathcurveto{\pgfqpoint{3.888962in}{1.511047in}}{\pgfqpoint{3.896862in}{1.507774in}}{\pgfqpoint{3.905098in}{1.507774in}}%
\pgfpathclose%
\pgfusepath{stroke,fill}%
\end{pgfscope}%
\begin{pgfscope}%
\pgfpathrectangle{\pgfqpoint{3.793912in}{0.557870in}}{\pgfqpoint{2.446088in}{1.484734in}}%
\pgfusepath{clip}%
\pgfsetbuttcap%
\pgfsetroundjoin%
\definecolor{currentfill}{rgb}{0.298039,0.447059,0.690196}%
\pgfsetfillcolor{currentfill}%
\pgfsetlinewidth{1.003750pt}%
\definecolor{currentstroke}{rgb}{0.298039,0.447059,0.690196}%
\pgfsetstrokecolor{currentstroke}%
\pgfsetdash{}{0pt}%
\pgfpathmoveto{\pgfqpoint{3.905098in}{1.425971in}}%
\pgfpathcurveto{\pgfqpoint{3.913334in}{1.425971in}}{\pgfqpoint{3.921234in}{1.429243in}}{\pgfqpoint{3.927058in}{1.435067in}}%
\pgfpathcurveto{\pgfqpoint{3.932882in}{1.440891in}}{\pgfqpoint{3.936155in}{1.448791in}}{\pgfqpoint{3.936155in}{1.457027in}}%
\pgfpathcurveto{\pgfqpoint{3.936155in}{1.465264in}}{\pgfqpoint{3.932882in}{1.473164in}}{\pgfqpoint{3.927058in}{1.478988in}}%
\pgfpathcurveto{\pgfqpoint{3.921234in}{1.484811in}}{\pgfqpoint{3.913334in}{1.488084in}}{\pgfqpoint{3.905098in}{1.488084in}}%
\pgfpathcurveto{\pgfqpoint{3.896862in}{1.488084in}}{\pgfqpoint{3.888962in}{1.484811in}}{\pgfqpoint{3.883138in}{1.478988in}}%
\pgfpathcurveto{\pgfqpoint{3.877314in}{1.473164in}}{\pgfqpoint{3.874042in}{1.465264in}}{\pgfqpoint{3.874042in}{1.457027in}}%
\pgfpathcurveto{\pgfqpoint{3.874042in}{1.448791in}}{\pgfqpoint{3.877314in}{1.440891in}}{\pgfqpoint{3.883138in}{1.435067in}}%
\pgfpathcurveto{\pgfqpoint{3.888962in}{1.429243in}}{\pgfqpoint{3.896862in}{1.425971in}}{\pgfqpoint{3.905098in}{1.425971in}}%
\pgfpathclose%
\pgfusepath{stroke,fill}%
\end{pgfscope}%
\begin{pgfscope}%
\pgfpathrectangle{\pgfqpoint{3.793912in}{0.557870in}}{\pgfqpoint{2.446088in}{1.484734in}}%
\pgfusepath{clip}%
\pgfsetbuttcap%
\pgfsetroundjoin%
\definecolor{currentfill}{rgb}{0.298039,0.447059,0.690196}%
\pgfsetfillcolor{currentfill}%
\pgfsetlinewidth{1.003750pt}%
\definecolor{currentstroke}{rgb}{0.298039,0.447059,0.690196}%
\pgfsetstrokecolor{currentstroke}%
\pgfsetdash{}{0pt}%
\pgfpathmoveto{\pgfqpoint{3.905098in}{1.480506in}}%
\pgfpathcurveto{\pgfqpoint{3.913334in}{1.480506in}}{\pgfqpoint{3.921234in}{1.483779in}}{\pgfqpoint{3.927058in}{1.489603in}}%
\pgfpathcurveto{\pgfqpoint{3.932882in}{1.495427in}}{\pgfqpoint{3.936155in}{1.503327in}}{\pgfqpoint{3.936155in}{1.511563in}}%
\pgfpathcurveto{\pgfqpoint{3.936155in}{1.519799in}}{\pgfqpoint{3.932882in}{1.527699in}}{\pgfqpoint{3.927058in}{1.533523in}}%
\pgfpathcurveto{\pgfqpoint{3.921234in}{1.539347in}}{\pgfqpoint{3.913334in}{1.542619in}}{\pgfqpoint{3.905098in}{1.542619in}}%
\pgfpathcurveto{\pgfqpoint{3.896862in}{1.542619in}}{\pgfqpoint{3.888962in}{1.539347in}}{\pgfqpoint{3.883138in}{1.533523in}}%
\pgfpathcurveto{\pgfqpoint{3.877314in}{1.527699in}}{\pgfqpoint{3.874042in}{1.519799in}}{\pgfqpoint{3.874042in}{1.511563in}}%
\pgfpathcurveto{\pgfqpoint{3.874042in}{1.503327in}}{\pgfqpoint{3.877314in}{1.495427in}}{\pgfqpoint{3.883138in}{1.489603in}}%
\pgfpathcurveto{\pgfqpoint{3.888962in}{1.483779in}}{\pgfqpoint{3.896862in}{1.480506in}}{\pgfqpoint{3.905098in}{1.480506in}}%
\pgfpathclose%
\pgfusepath{stroke,fill}%
\end{pgfscope}%
\begin{pgfscope}%
\pgfpathrectangle{\pgfqpoint{3.793912in}{0.557870in}}{\pgfqpoint{2.446088in}{1.484734in}}%
\pgfusepath{clip}%
\pgfsetbuttcap%
\pgfsetroundjoin%
\definecolor{currentfill}{rgb}{0.298039,0.447059,0.690196}%
\pgfsetfillcolor{currentfill}%
\pgfsetlinewidth{1.003750pt}%
\definecolor{currentstroke}{rgb}{0.298039,0.447059,0.690196}%
\pgfsetstrokecolor{currentstroke}%
\pgfsetdash{}{0pt}%
\pgfpathmoveto{\pgfqpoint{3.905098in}{0.771543in}}%
\pgfpathcurveto{\pgfqpoint{3.913334in}{0.771543in}}{\pgfqpoint{3.921234in}{0.774815in}}{\pgfqpoint{3.927058in}{0.780639in}}%
\pgfpathcurveto{\pgfqpoint{3.932882in}{0.786463in}}{\pgfqpoint{3.936155in}{0.794363in}}{\pgfqpoint{3.936155in}{0.802599in}}%
\pgfpathcurveto{\pgfqpoint{3.936155in}{0.810835in}}{\pgfqpoint{3.932882in}{0.818735in}}{\pgfqpoint{3.927058in}{0.824559in}}%
\pgfpathcurveto{\pgfqpoint{3.921234in}{0.830383in}}{\pgfqpoint{3.913334in}{0.833656in}}{\pgfqpoint{3.905098in}{0.833656in}}%
\pgfpathcurveto{\pgfqpoint{3.896862in}{0.833656in}}{\pgfqpoint{3.888962in}{0.830383in}}{\pgfqpoint{3.883138in}{0.824559in}}%
\pgfpathcurveto{\pgfqpoint{3.877314in}{0.818735in}}{\pgfqpoint{3.874042in}{0.810835in}}{\pgfqpoint{3.874042in}{0.802599in}}%
\pgfpathcurveto{\pgfqpoint{3.874042in}{0.794363in}}{\pgfqpoint{3.877314in}{0.786463in}}{\pgfqpoint{3.883138in}{0.780639in}}%
\pgfpathcurveto{\pgfqpoint{3.888962in}{0.774815in}}{\pgfqpoint{3.896862in}{0.771543in}}{\pgfqpoint{3.905098in}{0.771543in}}%
\pgfpathclose%
\pgfusepath{stroke,fill}%
\end{pgfscope}%
\begin{pgfscope}%
\pgfpathrectangle{\pgfqpoint{3.793912in}{0.557870in}}{\pgfqpoint{2.446088in}{1.484734in}}%
\pgfusepath{clip}%
\pgfsetbuttcap%
\pgfsetroundjoin%
\definecolor{currentfill}{rgb}{0.298039,0.447059,0.690196}%
\pgfsetfillcolor{currentfill}%
\pgfsetlinewidth{1.003750pt}%
\definecolor{currentstroke}{rgb}{0.298039,0.447059,0.690196}%
\pgfsetstrokecolor{currentstroke}%
\pgfsetdash{}{0pt}%
\pgfpathmoveto{\pgfqpoint{3.905098in}{1.603212in}}%
\pgfpathcurveto{\pgfqpoint{3.913334in}{1.603212in}}{\pgfqpoint{3.921234in}{1.606484in}}{\pgfqpoint{3.927058in}{1.612308in}}%
\pgfpathcurveto{\pgfqpoint{3.932882in}{1.618132in}}{\pgfqpoint{3.936155in}{1.626032in}}{\pgfqpoint{3.936155in}{1.634268in}}%
\pgfpathcurveto{\pgfqpoint{3.936155in}{1.642505in}}{\pgfqpoint{3.932882in}{1.650405in}}{\pgfqpoint{3.927058in}{1.656229in}}%
\pgfpathcurveto{\pgfqpoint{3.921234in}{1.662052in}}{\pgfqpoint{3.913334in}{1.665325in}}{\pgfqpoint{3.905098in}{1.665325in}}%
\pgfpathcurveto{\pgfqpoint{3.896862in}{1.665325in}}{\pgfqpoint{3.888962in}{1.662052in}}{\pgfqpoint{3.883138in}{1.656229in}}%
\pgfpathcurveto{\pgfqpoint{3.877314in}{1.650405in}}{\pgfqpoint{3.874042in}{1.642505in}}{\pgfqpoint{3.874042in}{1.634268in}}%
\pgfpathcurveto{\pgfqpoint{3.874042in}{1.626032in}}{\pgfqpoint{3.877314in}{1.618132in}}{\pgfqpoint{3.883138in}{1.612308in}}%
\pgfpathcurveto{\pgfqpoint{3.888962in}{1.606484in}}{\pgfqpoint{3.896862in}{1.603212in}}{\pgfqpoint{3.905098in}{1.603212in}}%
\pgfpathclose%
\pgfusepath{stroke,fill}%
\end{pgfscope}%
\begin{pgfscope}%
\pgfpathrectangle{\pgfqpoint{3.793912in}{0.557870in}}{\pgfqpoint{2.446088in}{1.484734in}}%
\pgfusepath{clip}%
\pgfsetbuttcap%
\pgfsetroundjoin%
\definecolor{currentfill}{rgb}{0.298039,0.447059,0.690196}%
\pgfsetfillcolor{currentfill}%
\pgfsetlinewidth{1.003750pt}%
\definecolor{currentstroke}{rgb}{0.298039,0.447059,0.690196}%
\pgfsetstrokecolor{currentstroke}%
\pgfsetdash{}{0pt}%
\pgfpathmoveto{\pgfqpoint{3.905098in}{1.180560in}}%
\pgfpathcurveto{\pgfqpoint{3.913334in}{1.180560in}}{\pgfqpoint{3.921234in}{1.183833in}}{\pgfqpoint{3.927058in}{1.189656in}}%
\pgfpathcurveto{\pgfqpoint{3.932882in}{1.195480in}}{\pgfqpoint{3.936155in}{1.203380in}}{\pgfqpoint{3.936155in}{1.211617in}}%
\pgfpathcurveto{\pgfqpoint{3.936155in}{1.219853in}}{\pgfqpoint{3.932882in}{1.227753in}}{\pgfqpoint{3.927058in}{1.233577in}}%
\pgfpathcurveto{\pgfqpoint{3.921234in}{1.239401in}}{\pgfqpoint{3.913334in}{1.242673in}}{\pgfqpoint{3.905098in}{1.242673in}}%
\pgfpathcurveto{\pgfqpoint{3.896862in}{1.242673in}}{\pgfqpoint{3.888962in}{1.239401in}}{\pgfqpoint{3.883138in}{1.233577in}}%
\pgfpathcurveto{\pgfqpoint{3.877314in}{1.227753in}}{\pgfqpoint{3.874042in}{1.219853in}}{\pgfqpoint{3.874042in}{1.211617in}}%
\pgfpathcurveto{\pgfqpoint{3.874042in}{1.203380in}}{\pgfqpoint{3.877314in}{1.195480in}}{\pgfqpoint{3.883138in}{1.189656in}}%
\pgfpathcurveto{\pgfqpoint{3.888962in}{1.183833in}}{\pgfqpoint{3.896862in}{1.180560in}}{\pgfqpoint{3.905098in}{1.180560in}}%
\pgfpathclose%
\pgfusepath{stroke,fill}%
\end{pgfscope}%
\begin{pgfscope}%
\pgfpathrectangle{\pgfqpoint{3.793912in}{0.557870in}}{\pgfqpoint{2.446088in}{1.484734in}}%
\pgfusepath{clip}%
\pgfsetbuttcap%
\pgfsetroundjoin%
\definecolor{currentfill}{rgb}{0.298039,0.447059,0.690196}%
\pgfsetfillcolor{currentfill}%
\pgfsetlinewidth{1.003750pt}%
\definecolor{currentstroke}{rgb}{0.298039,0.447059,0.690196}%
\pgfsetstrokecolor{currentstroke}%
\pgfsetdash{}{0pt}%
\pgfpathmoveto{\pgfqpoint{3.905098in}{1.644114in}}%
\pgfpathcurveto{\pgfqpoint{3.913334in}{1.644114in}}{\pgfqpoint{3.921234in}{1.647386in}}{\pgfqpoint{3.927058in}{1.653210in}}%
\pgfpathcurveto{\pgfqpoint{3.932882in}{1.659034in}}{\pgfqpoint{3.936155in}{1.666934in}}{\pgfqpoint{3.936155in}{1.675170in}}%
\pgfpathcurveto{\pgfqpoint{3.936155in}{1.683406in}}{\pgfqpoint{3.932882in}{1.691306in}}{\pgfqpoint{3.927058in}{1.697130in}}%
\pgfpathcurveto{\pgfqpoint{3.921234in}{1.702954in}}{\pgfqpoint{3.913334in}{1.706227in}}{\pgfqpoint{3.905098in}{1.706227in}}%
\pgfpathcurveto{\pgfqpoint{3.896862in}{1.706227in}}{\pgfqpoint{3.888962in}{1.702954in}}{\pgfqpoint{3.883138in}{1.697130in}}%
\pgfpathcurveto{\pgfqpoint{3.877314in}{1.691306in}}{\pgfqpoint{3.874042in}{1.683406in}}{\pgfqpoint{3.874042in}{1.675170in}}%
\pgfpathcurveto{\pgfqpoint{3.874042in}{1.666934in}}{\pgfqpoint{3.877314in}{1.659034in}}{\pgfqpoint{3.883138in}{1.653210in}}%
\pgfpathcurveto{\pgfqpoint{3.888962in}{1.647386in}}{\pgfqpoint{3.896862in}{1.644114in}}{\pgfqpoint{3.905098in}{1.644114in}}%
\pgfpathclose%
\pgfusepath{stroke,fill}%
\end{pgfscope}%
\begin{pgfscope}%
\pgfpathrectangle{\pgfqpoint{3.793912in}{0.557870in}}{\pgfqpoint{2.446088in}{1.484734in}}%
\pgfusepath{clip}%
\pgfsetbuttcap%
\pgfsetroundjoin%
\definecolor{currentfill}{rgb}{0.298039,0.447059,0.690196}%
\pgfsetfillcolor{currentfill}%
\pgfsetlinewidth{1.003750pt}%
\definecolor{currentstroke}{rgb}{0.298039,0.447059,0.690196}%
\pgfsetstrokecolor{currentstroke}%
\pgfsetdash{}{0pt}%
\pgfpathmoveto{\pgfqpoint{3.905098in}{0.921516in}}%
\pgfpathcurveto{\pgfqpoint{3.913334in}{0.921516in}}{\pgfqpoint{3.921234in}{0.924788in}}{\pgfqpoint{3.927058in}{0.930612in}}%
\pgfpathcurveto{\pgfqpoint{3.932882in}{0.936436in}}{\pgfqpoint{3.936155in}{0.944336in}}{\pgfqpoint{3.936155in}{0.952572in}}%
\pgfpathcurveto{\pgfqpoint{3.936155in}{0.960809in}}{\pgfqpoint{3.932882in}{0.968709in}}{\pgfqpoint{3.927058in}{0.974533in}}%
\pgfpathcurveto{\pgfqpoint{3.921234in}{0.980356in}}{\pgfqpoint{3.913334in}{0.983629in}}{\pgfqpoint{3.905098in}{0.983629in}}%
\pgfpathcurveto{\pgfqpoint{3.896862in}{0.983629in}}{\pgfqpoint{3.888962in}{0.980356in}}{\pgfqpoint{3.883138in}{0.974533in}}%
\pgfpathcurveto{\pgfqpoint{3.877314in}{0.968709in}}{\pgfqpoint{3.874042in}{0.960809in}}{\pgfqpoint{3.874042in}{0.952572in}}%
\pgfpathcurveto{\pgfqpoint{3.874042in}{0.944336in}}{\pgfqpoint{3.877314in}{0.936436in}}{\pgfqpoint{3.883138in}{0.930612in}}%
\pgfpathcurveto{\pgfqpoint{3.888962in}{0.924788in}}{\pgfqpoint{3.896862in}{0.921516in}}{\pgfqpoint{3.905098in}{0.921516in}}%
\pgfpathclose%
\pgfusepath{stroke,fill}%
\end{pgfscope}%
\begin{pgfscope}%
\pgfpathrectangle{\pgfqpoint{3.793912in}{0.557870in}}{\pgfqpoint{2.446088in}{1.484734in}}%
\pgfusepath{clip}%
\pgfsetbuttcap%
\pgfsetroundjoin%
\definecolor{currentfill}{rgb}{0.298039,0.447059,0.690196}%
\pgfsetfillcolor{currentfill}%
\pgfsetlinewidth{1.003750pt}%
\definecolor{currentstroke}{rgb}{0.298039,0.447059,0.690196}%
\pgfsetstrokecolor{currentstroke}%
\pgfsetdash{}{0pt}%
\pgfpathmoveto{\pgfqpoint{3.905098in}{0.921516in}}%
\pgfpathcurveto{\pgfqpoint{3.913334in}{0.921516in}}{\pgfqpoint{3.921234in}{0.924788in}}{\pgfqpoint{3.927058in}{0.930612in}}%
\pgfpathcurveto{\pgfqpoint{3.932882in}{0.936436in}}{\pgfqpoint{3.936155in}{0.944336in}}{\pgfqpoint{3.936155in}{0.952572in}}%
\pgfpathcurveto{\pgfqpoint{3.936155in}{0.960809in}}{\pgfqpoint{3.932882in}{0.968709in}}{\pgfqpoint{3.927058in}{0.974533in}}%
\pgfpathcurveto{\pgfqpoint{3.921234in}{0.980356in}}{\pgfqpoint{3.913334in}{0.983629in}}{\pgfqpoint{3.905098in}{0.983629in}}%
\pgfpathcurveto{\pgfqpoint{3.896862in}{0.983629in}}{\pgfqpoint{3.888962in}{0.980356in}}{\pgfqpoint{3.883138in}{0.974533in}}%
\pgfpathcurveto{\pgfqpoint{3.877314in}{0.968709in}}{\pgfqpoint{3.874042in}{0.960809in}}{\pgfqpoint{3.874042in}{0.952572in}}%
\pgfpathcurveto{\pgfqpoint{3.874042in}{0.944336in}}{\pgfqpoint{3.877314in}{0.936436in}}{\pgfqpoint{3.883138in}{0.930612in}}%
\pgfpathcurveto{\pgfqpoint{3.888962in}{0.924788in}}{\pgfqpoint{3.896862in}{0.921516in}}{\pgfqpoint{3.905098in}{0.921516in}}%
\pgfpathclose%
\pgfusepath{stroke,fill}%
\end{pgfscope}%
\begin{pgfscope}%
\pgfpathrectangle{\pgfqpoint{3.793912in}{0.557870in}}{\pgfqpoint{2.446088in}{1.484734in}}%
\pgfusepath{clip}%
\pgfsetbuttcap%
\pgfsetroundjoin%
\definecolor{currentfill}{rgb}{0.298039,0.447059,0.690196}%
\pgfsetfillcolor{currentfill}%
\pgfsetlinewidth{1.003750pt}%
\definecolor{currentstroke}{rgb}{0.298039,0.447059,0.690196}%
\pgfsetstrokecolor{currentstroke}%
\pgfsetdash{}{0pt}%
\pgfpathmoveto{\pgfqpoint{3.905098in}{1.930426in}}%
\pgfpathcurveto{\pgfqpoint{3.913334in}{1.930426in}}{\pgfqpoint{3.921234in}{1.933698in}}{\pgfqpoint{3.927058in}{1.939522in}}%
\pgfpathcurveto{\pgfqpoint{3.932882in}{1.945346in}}{\pgfqpoint{3.936155in}{1.953246in}}{\pgfqpoint{3.936155in}{1.961482in}}%
\pgfpathcurveto{\pgfqpoint{3.936155in}{1.969719in}}{\pgfqpoint{3.932882in}{1.977619in}}{\pgfqpoint{3.927058in}{1.983443in}}%
\pgfpathcurveto{\pgfqpoint{3.921234in}{1.989267in}}{\pgfqpoint{3.913334in}{1.992539in}}{\pgfqpoint{3.905098in}{1.992539in}}%
\pgfpathcurveto{\pgfqpoint{3.896862in}{1.992539in}}{\pgfqpoint{3.888962in}{1.989267in}}{\pgfqpoint{3.883138in}{1.983443in}}%
\pgfpathcurveto{\pgfqpoint{3.877314in}{1.977619in}}{\pgfqpoint{3.874042in}{1.969719in}}{\pgfqpoint{3.874042in}{1.961482in}}%
\pgfpathcurveto{\pgfqpoint{3.874042in}{1.953246in}}{\pgfqpoint{3.877314in}{1.945346in}}{\pgfqpoint{3.883138in}{1.939522in}}%
\pgfpathcurveto{\pgfqpoint{3.888962in}{1.933698in}}{\pgfqpoint{3.896862in}{1.930426in}}{\pgfqpoint{3.905098in}{1.930426in}}%
\pgfpathclose%
\pgfusepath{stroke,fill}%
\end{pgfscope}%
\begin{pgfscope}%
\pgfpathrectangle{\pgfqpoint{3.793912in}{0.557870in}}{\pgfqpoint{2.446088in}{1.484734in}}%
\pgfusepath{clip}%
\pgfsetbuttcap%
\pgfsetroundjoin%
\definecolor{currentfill}{rgb}{0.298039,0.447059,0.690196}%
\pgfsetfillcolor{currentfill}%
\pgfsetlinewidth{1.003750pt}%
\definecolor{currentstroke}{rgb}{0.298039,0.447059,0.690196}%
\pgfsetstrokecolor{currentstroke}%
\pgfsetdash{}{0pt}%
\pgfpathmoveto{\pgfqpoint{3.905098in}{1.930426in}}%
\pgfpathcurveto{\pgfqpoint{3.913334in}{1.930426in}}{\pgfqpoint{3.921234in}{1.933698in}}{\pgfqpoint{3.927058in}{1.939522in}}%
\pgfpathcurveto{\pgfqpoint{3.932882in}{1.945346in}}{\pgfqpoint{3.936155in}{1.953246in}}{\pgfqpoint{3.936155in}{1.961482in}}%
\pgfpathcurveto{\pgfqpoint{3.936155in}{1.969719in}}{\pgfqpoint{3.932882in}{1.977619in}}{\pgfqpoint{3.927058in}{1.983443in}}%
\pgfpathcurveto{\pgfqpoint{3.921234in}{1.989267in}}{\pgfqpoint{3.913334in}{1.992539in}}{\pgfqpoint{3.905098in}{1.992539in}}%
\pgfpathcurveto{\pgfqpoint{3.896862in}{1.992539in}}{\pgfqpoint{3.888962in}{1.989267in}}{\pgfqpoint{3.883138in}{1.983443in}}%
\pgfpathcurveto{\pgfqpoint{3.877314in}{1.977619in}}{\pgfqpoint{3.874042in}{1.969719in}}{\pgfqpoint{3.874042in}{1.961482in}}%
\pgfpathcurveto{\pgfqpoint{3.874042in}{1.953246in}}{\pgfqpoint{3.877314in}{1.945346in}}{\pgfqpoint{3.883138in}{1.939522in}}%
\pgfpathcurveto{\pgfqpoint{3.888962in}{1.933698in}}{\pgfqpoint{3.896862in}{1.930426in}}{\pgfqpoint{3.905098in}{1.930426in}}%
\pgfpathclose%
\pgfusepath{stroke,fill}%
\end{pgfscope}%
\begin{pgfscope}%
\pgfpathrectangle{\pgfqpoint{3.793912in}{0.557870in}}{\pgfqpoint{2.446088in}{1.484734in}}%
\pgfusepath{clip}%
\pgfsetbuttcap%
\pgfsetroundjoin%
\definecolor{currentfill}{rgb}{0.298039,0.447059,0.690196}%
\pgfsetfillcolor{currentfill}%
\pgfsetlinewidth{1.003750pt}%
\definecolor{currentstroke}{rgb}{0.298039,0.447059,0.690196}%
\pgfsetstrokecolor{currentstroke}%
\pgfsetdash{}{0pt}%
\pgfpathmoveto{\pgfqpoint{3.905098in}{1.930426in}}%
\pgfpathcurveto{\pgfqpoint{3.913334in}{1.930426in}}{\pgfqpoint{3.921234in}{1.933698in}}{\pgfqpoint{3.927058in}{1.939522in}}%
\pgfpathcurveto{\pgfqpoint{3.932882in}{1.945346in}}{\pgfqpoint{3.936155in}{1.953246in}}{\pgfqpoint{3.936155in}{1.961482in}}%
\pgfpathcurveto{\pgfqpoint{3.936155in}{1.969719in}}{\pgfqpoint{3.932882in}{1.977619in}}{\pgfqpoint{3.927058in}{1.983443in}}%
\pgfpathcurveto{\pgfqpoint{3.921234in}{1.989267in}}{\pgfqpoint{3.913334in}{1.992539in}}{\pgfqpoint{3.905098in}{1.992539in}}%
\pgfpathcurveto{\pgfqpoint{3.896862in}{1.992539in}}{\pgfqpoint{3.888962in}{1.989267in}}{\pgfqpoint{3.883138in}{1.983443in}}%
\pgfpathcurveto{\pgfqpoint{3.877314in}{1.977619in}}{\pgfqpoint{3.874042in}{1.969719in}}{\pgfqpoint{3.874042in}{1.961482in}}%
\pgfpathcurveto{\pgfqpoint{3.874042in}{1.953246in}}{\pgfqpoint{3.877314in}{1.945346in}}{\pgfqpoint{3.883138in}{1.939522in}}%
\pgfpathcurveto{\pgfqpoint{3.888962in}{1.933698in}}{\pgfqpoint{3.896862in}{1.930426in}}{\pgfqpoint{3.905098in}{1.930426in}}%
\pgfpathclose%
\pgfusepath{stroke,fill}%
\end{pgfscope}%
\begin{pgfscope}%
\pgfpathrectangle{\pgfqpoint{3.793912in}{0.557870in}}{\pgfqpoint{2.446088in}{1.484734in}}%
\pgfusepath{clip}%
\pgfsetbuttcap%
\pgfsetroundjoin%
\definecolor{currentfill}{rgb}{0.298039,0.447059,0.690196}%
\pgfsetfillcolor{currentfill}%
\pgfsetlinewidth{1.003750pt}%
\definecolor{currentstroke}{rgb}{0.298039,0.447059,0.690196}%
\pgfsetstrokecolor{currentstroke}%
\pgfsetdash{}{0pt}%
\pgfpathmoveto{\pgfqpoint{5.457067in}{0.594302in}}%
\pgfpathcurveto{\pgfqpoint{5.465303in}{0.594302in}}{\pgfqpoint{5.473203in}{0.597574in}}{\pgfqpoint{5.479027in}{0.603398in}}%
\pgfpathcurveto{\pgfqpoint{5.484851in}{0.609222in}}{\pgfqpoint{5.488123in}{0.617122in}}{\pgfqpoint{5.488123in}{0.625358in}}%
\pgfpathcurveto{\pgfqpoint{5.488123in}{0.633594in}}{\pgfqpoint{5.484851in}{0.641495in}}{\pgfqpoint{5.479027in}{0.647318in}}%
\pgfpathcurveto{\pgfqpoint{5.473203in}{0.653142in}}{\pgfqpoint{5.465303in}{0.656415in}}{\pgfqpoint{5.457067in}{0.656415in}}%
\pgfpathcurveto{\pgfqpoint{5.448830in}{0.656415in}}{\pgfqpoint{5.440930in}{0.653142in}}{\pgfqpoint{5.435106in}{0.647318in}}%
\pgfpathcurveto{\pgfqpoint{5.429282in}{0.641495in}}{\pgfqpoint{5.426010in}{0.633594in}}{\pgfqpoint{5.426010in}{0.625358in}}%
\pgfpathcurveto{\pgfqpoint{5.426010in}{0.617122in}}{\pgfqpoint{5.429282in}{0.609222in}}{\pgfqpoint{5.435106in}{0.603398in}}%
\pgfpathcurveto{\pgfqpoint{5.440930in}{0.597574in}}{\pgfqpoint{5.448830in}{0.594302in}}{\pgfqpoint{5.457067in}{0.594302in}}%
\pgfpathclose%
\pgfusepath{stroke,fill}%
\end{pgfscope}%
\begin{pgfscope}%
\pgfpathrectangle{\pgfqpoint{3.793912in}{0.557870in}}{\pgfqpoint{2.446088in}{1.484734in}}%
\pgfusepath{clip}%
\pgfsetbuttcap%
\pgfsetroundjoin%
\definecolor{currentfill}{rgb}{0.298039,0.447059,0.690196}%
\pgfsetfillcolor{currentfill}%
\pgfsetlinewidth{1.003750pt}%
\definecolor{currentstroke}{rgb}{0.298039,0.447059,0.690196}%
\pgfsetstrokecolor{currentstroke}%
\pgfsetdash{}{0pt}%
\pgfpathmoveto{\pgfqpoint{3.905098in}{1.930426in}}%
\pgfpathcurveto{\pgfqpoint{3.913334in}{1.930426in}}{\pgfqpoint{3.921234in}{1.933698in}}{\pgfqpoint{3.927058in}{1.939522in}}%
\pgfpathcurveto{\pgfqpoint{3.932882in}{1.945346in}}{\pgfqpoint{3.936155in}{1.953246in}}{\pgfqpoint{3.936155in}{1.961482in}}%
\pgfpathcurveto{\pgfqpoint{3.936155in}{1.969719in}}{\pgfqpoint{3.932882in}{1.977619in}}{\pgfqpoint{3.927058in}{1.983443in}}%
\pgfpathcurveto{\pgfqpoint{3.921234in}{1.989267in}}{\pgfqpoint{3.913334in}{1.992539in}}{\pgfqpoint{3.905098in}{1.992539in}}%
\pgfpathcurveto{\pgfqpoint{3.896862in}{1.992539in}}{\pgfqpoint{3.888962in}{1.989267in}}{\pgfqpoint{3.883138in}{1.983443in}}%
\pgfpathcurveto{\pgfqpoint{3.877314in}{1.977619in}}{\pgfqpoint{3.874042in}{1.969719in}}{\pgfqpoint{3.874042in}{1.961482in}}%
\pgfpathcurveto{\pgfqpoint{3.874042in}{1.953246in}}{\pgfqpoint{3.877314in}{1.945346in}}{\pgfqpoint{3.883138in}{1.939522in}}%
\pgfpathcurveto{\pgfqpoint{3.888962in}{1.933698in}}{\pgfqpoint{3.896862in}{1.930426in}}{\pgfqpoint{3.905098in}{1.930426in}}%
\pgfpathclose%
\pgfusepath{stroke,fill}%
\end{pgfscope}%
\begin{pgfscope}%
\pgfpathrectangle{\pgfqpoint{3.793912in}{0.557870in}}{\pgfqpoint{2.446088in}{1.484734in}}%
\pgfusepath{clip}%
\pgfsetbuttcap%
\pgfsetroundjoin%
\definecolor{currentfill}{rgb}{0.298039,0.447059,0.690196}%
\pgfsetfillcolor{currentfill}%
\pgfsetlinewidth{1.003750pt}%
\definecolor{currentstroke}{rgb}{0.298039,0.447059,0.690196}%
\pgfsetstrokecolor{currentstroke}%
\pgfsetdash{}{0pt}%
\pgfpathmoveto{\pgfqpoint{3.905098in}{1.930426in}}%
\pgfpathcurveto{\pgfqpoint{3.913334in}{1.930426in}}{\pgfqpoint{3.921234in}{1.933698in}}{\pgfqpoint{3.927058in}{1.939522in}}%
\pgfpathcurveto{\pgfqpoint{3.932882in}{1.945346in}}{\pgfqpoint{3.936155in}{1.953246in}}{\pgfqpoint{3.936155in}{1.961482in}}%
\pgfpathcurveto{\pgfqpoint{3.936155in}{1.969719in}}{\pgfqpoint{3.932882in}{1.977619in}}{\pgfqpoint{3.927058in}{1.983443in}}%
\pgfpathcurveto{\pgfqpoint{3.921234in}{1.989267in}}{\pgfqpoint{3.913334in}{1.992539in}}{\pgfqpoint{3.905098in}{1.992539in}}%
\pgfpathcurveto{\pgfqpoint{3.896862in}{1.992539in}}{\pgfqpoint{3.888962in}{1.989267in}}{\pgfqpoint{3.883138in}{1.983443in}}%
\pgfpathcurveto{\pgfqpoint{3.877314in}{1.977619in}}{\pgfqpoint{3.874042in}{1.969719in}}{\pgfqpoint{3.874042in}{1.961482in}}%
\pgfpathcurveto{\pgfqpoint{3.874042in}{1.953246in}}{\pgfqpoint{3.877314in}{1.945346in}}{\pgfqpoint{3.883138in}{1.939522in}}%
\pgfpathcurveto{\pgfqpoint{3.888962in}{1.933698in}}{\pgfqpoint{3.896862in}{1.930426in}}{\pgfqpoint{3.905098in}{1.930426in}}%
\pgfpathclose%
\pgfusepath{stroke,fill}%
\end{pgfscope}%
\begin{pgfscope}%
\pgfpathrectangle{\pgfqpoint{3.793912in}{0.557870in}}{\pgfqpoint{2.446088in}{1.484734in}}%
\pgfusepath{clip}%
\pgfsetbuttcap%
\pgfsetroundjoin%
\definecolor{currentfill}{rgb}{0.298039,0.447059,0.690196}%
\pgfsetfillcolor{currentfill}%
\pgfsetlinewidth{1.003750pt}%
\definecolor{currentstroke}{rgb}{0.298039,0.447059,0.690196}%
\pgfsetstrokecolor{currentstroke}%
\pgfsetdash{}{0pt}%
\pgfpathmoveto{\pgfqpoint{3.905098in}{1.930426in}}%
\pgfpathcurveto{\pgfqpoint{3.913334in}{1.930426in}}{\pgfqpoint{3.921234in}{1.933698in}}{\pgfqpoint{3.927058in}{1.939522in}}%
\pgfpathcurveto{\pgfqpoint{3.932882in}{1.945346in}}{\pgfqpoint{3.936155in}{1.953246in}}{\pgfqpoint{3.936155in}{1.961482in}}%
\pgfpathcurveto{\pgfqpoint{3.936155in}{1.969719in}}{\pgfqpoint{3.932882in}{1.977619in}}{\pgfqpoint{3.927058in}{1.983443in}}%
\pgfpathcurveto{\pgfqpoint{3.921234in}{1.989267in}}{\pgfqpoint{3.913334in}{1.992539in}}{\pgfqpoint{3.905098in}{1.992539in}}%
\pgfpathcurveto{\pgfqpoint{3.896862in}{1.992539in}}{\pgfqpoint{3.888962in}{1.989267in}}{\pgfqpoint{3.883138in}{1.983443in}}%
\pgfpathcurveto{\pgfqpoint{3.877314in}{1.977619in}}{\pgfqpoint{3.874042in}{1.969719in}}{\pgfqpoint{3.874042in}{1.961482in}}%
\pgfpathcurveto{\pgfqpoint{3.874042in}{1.953246in}}{\pgfqpoint{3.877314in}{1.945346in}}{\pgfqpoint{3.883138in}{1.939522in}}%
\pgfpathcurveto{\pgfqpoint{3.888962in}{1.933698in}}{\pgfqpoint{3.896862in}{1.930426in}}{\pgfqpoint{3.905098in}{1.930426in}}%
\pgfpathclose%
\pgfusepath{stroke,fill}%
\end{pgfscope}%
\begin{pgfscope}%
\pgfpathrectangle{\pgfqpoint{3.793912in}{0.557870in}}{\pgfqpoint{2.446088in}{1.484734in}}%
\pgfusepath{clip}%
\pgfsetbuttcap%
\pgfsetroundjoin%
\definecolor{currentfill}{rgb}{0.298039,0.447059,0.690196}%
\pgfsetfillcolor{currentfill}%
\pgfsetlinewidth{1.003750pt}%
\definecolor{currentstroke}{rgb}{0.298039,0.447059,0.690196}%
\pgfsetstrokecolor{currentstroke}%
\pgfsetdash{}{0pt}%
\pgfpathmoveto{\pgfqpoint{3.905098in}{1.930426in}}%
\pgfpathcurveto{\pgfqpoint{3.913334in}{1.930426in}}{\pgfqpoint{3.921234in}{1.933698in}}{\pgfqpoint{3.927058in}{1.939522in}}%
\pgfpathcurveto{\pgfqpoint{3.932882in}{1.945346in}}{\pgfqpoint{3.936155in}{1.953246in}}{\pgfqpoint{3.936155in}{1.961482in}}%
\pgfpathcurveto{\pgfqpoint{3.936155in}{1.969719in}}{\pgfqpoint{3.932882in}{1.977619in}}{\pgfqpoint{3.927058in}{1.983443in}}%
\pgfpathcurveto{\pgfqpoint{3.921234in}{1.989267in}}{\pgfqpoint{3.913334in}{1.992539in}}{\pgfqpoint{3.905098in}{1.992539in}}%
\pgfpathcurveto{\pgfqpoint{3.896862in}{1.992539in}}{\pgfqpoint{3.888962in}{1.989267in}}{\pgfqpoint{3.883138in}{1.983443in}}%
\pgfpathcurveto{\pgfqpoint{3.877314in}{1.977619in}}{\pgfqpoint{3.874042in}{1.969719in}}{\pgfqpoint{3.874042in}{1.961482in}}%
\pgfpathcurveto{\pgfqpoint{3.874042in}{1.953246in}}{\pgfqpoint{3.877314in}{1.945346in}}{\pgfqpoint{3.883138in}{1.939522in}}%
\pgfpathcurveto{\pgfqpoint{3.888962in}{1.933698in}}{\pgfqpoint{3.896862in}{1.930426in}}{\pgfqpoint{3.905098in}{1.930426in}}%
\pgfpathclose%
\pgfusepath{stroke,fill}%
\end{pgfscope}%
\begin{pgfscope}%
\pgfpathrectangle{\pgfqpoint{3.793912in}{0.557870in}}{\pgfqpoint{2.446088in}{1.484734in}}%
\pgfusepath{clip}%
\pgfsetbuttcap%
\pgfsetroundjoin%
\definecolor{currentfill}{rgb}{0.298039,0.447059,0.690196}%
\pgfsetfillcolor{currentfill}%
\pgfsetlinewidth{1.003750pt}%
\definecolor{currentstroke}{rgb}{0.298039,0.447059,0.690196}%
\pgfsetstrokecolor{currentstroke}%
\pgfsetdash{}{0pt}%
\pgfpathmoveto{\pgfqpoint{3.905098in}{1.235096in}}%
\pgfpathcurveto{\pgfqpoint{3.913334in}{1.235096in}}{\pgfqpoint{3.921234in}{1.238368in}}{\pgfqpoint{3.927058in}{1.244192in}}%
\pgfpathcurveto{\pgfqpoint{3.932882in}{1.250016in}}{\pgfqpoint{3.936155in}{1.257916in}}{\pgfqpoint{3.936155in}{1.266152in}}%
\pgfpathcurveto{\pgfqpoint{3.936155in}{1.274389in}}{\pgfqpoint{3.932882in}{1.282289in}}{\pgfqpoint{3.927058in}{1.288113in}}%
\pgfpathcurveto{\pgfqpoint{3.921234in}{1.293937in}}{\pgfqpoint{3.913334in}{1.297209in}}{\pgfqpoint{3.905098in}{1.297209in}}%
\pgfpathcurveto{\pgfqpoint{3.896862in}{1.297209in}}{\pgfqpoint{3.888962in}{1.293937in}}{\pgfqpoint{3.883138in}{1.288113in}}%
\pgfpathcurveto{\pgfqpoint{3.877314in}{1.282289in}}{\pgfqpoint{3.874042in}{1.274389in}}{\pgfqpoint{3.874042in}{1.266152in}}%
\pgfpathcurveto{\pgfqpoint{3.874042in}{1.257916in}}{\pgfqpoint{3.877314in}{1.250016in}}{\pgfqpoint{3.883138in}{1.244192in}}%
\pgfpathcurveto{\pgfqpoint{3.888962in}{1.238368in}}{\pgfqpoint{3.896862in}{1.235096in}}{\pgfqpoint{3.905098in}{1.235096in}}%
\pgfpathclose%
\pgfusepath{stroke,fill}%
\end{pgfscope}%
\begin{pgfscope}%
\pgfpathrectangle{\pgfqpoint{3.793912in}{0.557870in}}{\pgfqpoint{2.446088in}{1.484734in}}%
\pgfusepath{clip}%
\pgfsetbuttcap%
\pgfsetroundjoin%
\definecolor{currentfill}{rgb}{0.298039,0.447059,0.690196}%
\pgfsetfillcolor{currentfill}%
\pgfsetlinewidth{1.003750pt}%
\definecolor{currentstroke}{rgb}{0.298039,0.447059,0.690196}%
\pgfsetstrokecolor{currentstroke}%
\pgfsetdash{}{0pt}%
\pgfpathmoveto{\pgfqpoint{3.905098in}{1.930426in}}%
\pgfpathcurveto{\pgfqpoint{3.913334in}{1.930426in}}{\pgfqpoint{3.921234in}{1.933698in}}{\pgfqpoint{3.927058in}{1.939522in}}%
\pgfpathcurveto{\pgfqpoint{3.932882in}{1.945346in}}{\pgfqpoint{3.936155in}{1.953246in}}{\pgfqpoint{3.936155in}{1.961482in}}%
\pgfpathcurveto{\pgfqpoint{3.936155in}{1.969719in}}{\pgfqpoint{3.932882in}{1.977619in}}{\pgfqpoint{3.927058in}{1.983443in}}%
\pgfpathcurveto{\pgfqpoint{3.921234in}{1.989267in}}{\pgfqpoint{3.913334in}{1.992539in}}{\pgfqpoint{3.905098in}{1.992539in}}%
\pgfpathcurveto{\pgfqpoint{3.896862in}{1.992539in}}{\pgfqpoint{3.888962in}{1.989267in}}{\pgfqpoint{3.883138in}{1.983443in}}%
\pgfpathcurveto{\pgfqpoint{3.877314in}{1.977619in}}{\pgfqpoint{3.874042in}{1.969719in}}{\pgfqpoint{3.874042in}{1.961482in}}%
\pgfpathcurveto{\pgfqpoint{3.874042in}{1.953246in}}{\pgfqpoint{3.877314in}{1.945346in}}{\pgfqpoint{3.883138in}{1.939522in}}%
\pgfpathcurveto{\pgfqpoint{3.888962in}{1.933698in}}{\pgfqpoint{3.896862in}{1.930426in}}{\pgfqpoint{3.905098in}{1.930426in}}%
\pgfpathclose%
\pgfusepath{stroke,fill}%
\end{pgfscope}%
\begin{pgfscope}%
\pgfpathrectangle{\pgfqpoint{3.793912in}{0.557870in}}{\pgfqpoint{2.446088in}{1.484734in}}%
\pgfusepath{clip}%
\pgfsetbuttcap%
\pgfsetroundjoin%
\definecolor{currentfill}{rgb}{0.298039,0.447059,0.690196}%
\pgfsetfillcolor{currentfill}%
\pgfsetlinewidth{1.003750pt}%
\definecolor{currentstroke}{rgb}{0.298039,0.447059,0.690196}%
\pgfsetstrokecolor{currentstroke}%
\pgfsetdash{}{0pt}%
\pgfpathmoveto{\pgfqpoint{3.905098in}{1.044221in}}%
\pgfpathcurveto{\pgfqpoint{3.913334in}{1.044221in}}{\pgfqpoint{3.921234in}{1.047493in}}{\pgfqpoint{3.927058in}{1.053317in}}%
\pgfpathcurveto{\pgfqpoint{3.932882in}{1.059141in}}{\pgfqpoint{3.936155in}{1.067041in}}{\pgfqpoint{3.936155in}{1.075278in}}%
\pgfpathcurveto{\pgfqpoint{3.936155in}{1.083514in}}{\pgfqpoint{3.932882in}{1.091414in}}{\pgfqpoint{3.927058in}{1.097238in}}%
\pgfpathcurveto{\pgfqpoint{3.921234in}{1.103062in}}{\pgfqpoint{3.913334in}{1.106334in}}{\pgfqpoint{3.905098in}{1.106334in}}%
\pgfpathcurveto{\pgfqpoint{3.896862in}{1.106334in}}{\pgfqpoint{3.888962in}{1.103062in}}{\pgfqpoint{3.883138in}{1.097238in}}%
\pgfpathcurveto{\pgfqpoint{3.877314in}{1.091414in}}{\pgfqpoint{3.874042in}{1.083514in}}{\pgfqpoint{3.874042in}{1.075278in}}%
\pgfpathcurveto{\pgfqpoint{3.874042in}{1.067041in}}{\pgfqpoint{3.877314in}{1.059141in}}{\pgfqpoint{3.883138in}{1.053317in}}%
\pgfpathcurveto{\pgfqpoint{3.888962in}{1.047493in}}{\pgfqpoint{3.896862in}{1.044221in}}{\pgfqpoint{3.905098in}{1.044221in}}%
\pgfpathclose%
\pgfusepath{stroke,fill}%
\end{pgfscope}%
\begin{pgfscope}%
\pgfpathrectangle{\pgfqpoint{3.793912in}{0.557870in}}{\pgfqpoint{2.446088in}{1.484734in}}%
\pgfusepath{clip}%
\pgfsetbuttcap%
\pgfsetroundjoin%
\definecolor{currentfill}{rgb}{0.298039,0.447059,0.690196}%
\pgfsetfillcolor{currentfill}%
\pgfsetlinewidth{1.003750pt}%
\definecolor{currentstroke}{rgb}{0.298039,0.447059,0.690196}%
\pgfsetstrokecolor{currentstroke}%
\pgfsetdash{}{0pt}%
\pgfpathmoveto{\pgfqpoint{3.905098in}{1.930426in}}%
\pgfpathcurveto{\pgfqpoint{3.913334in}{1.930426in}}{\pgfqpoint{3.921234in}{1.933698in}}{\pgfqpoint{3.927058in}{1.939522in}}%
\pgfpathcurveto{\pgfqpoint{3.932882in}{1.945346in}}{\pgfqpoint{3.936155in}{1.953246in}}{\pgfqpoint{3.936155in}{1.961482in}}%
\pgfpathcurveto{\pgfqpoint{3.936155in}{1.969719in}}{\pgfqpoint{3.932882in}{1.977619in}}{\pgfqpoint{3.927058in}{1.983443in}}%
\pgfpathcurveto{\pgfqpoint{3.921234in}{1.989267in}}{\pgfqpoint{3.913334in}{1.992539in}}{\pgfqpoint{3.905098in}{1.992539in}}%
\pgfpathcurveto{\pgfqpoint{3.896862in}{1.992539in}}{\pgfqpoint{3.888962in}{1.989267in}}{\pgfqpoint{3.883138in}{1.983443in}}%
\pgfpathcurveto{\pgfqpoint{3.877314in}{1.977619in}}{\pgfqpoint{3.874042in}{1.969719in}}{\pgfqpoint{3.874042in}{1.961482in}}%
\pgfpathcurveto{\pgfqpoint{3.874042in}{1.953246in}}{\pgfqpoint{3.877314in}{1.945346in}}{\pgfqpoint{3.883138in}{1.939522in}}%
\pgfpathcurveto{\pgfqpoint{3.888962in}{1.933698in}}{\pgfqpoint{3.896862in}{1.930426in}}{\pgfqpoint{3.905098in}{1.930426in}}%
\pgfpathclose%
\pgfusepath{stroke,fill}%
\end{pgfscope}%
\begin{pgfscope}%
\pgfpathrectangle{\pgfqpoint{3.793912in}{0.557870in}}{\pgfqpoint{2.446088in}{1.484734in}}%
\pgfusepath{clip}%
\pgfsetbuttcap%
\pgfsetroundjoin%
\definecolor{currentfill}{rgb}{0.298039,0.447059,0.690196}%
\pgfsetfillcolor{currentfill}%
\pgfsetlinewidth{1.003750pt}%
\definecolor{currentstroke}{rgb}{0.298039,0.447059,0.690196}%
\pgfsetstrokecolor{currentstroke}%
\pgfsetdash{}{0pt}%
\pgfpathmoveto{\pgfqpoint{3.905098in}{1.930426in}}%
\pgfpathcurveto{\pgfqpoint{3.913334in}{1.930426in}}{\pgfqpoint{3.921234in}{1.933698in}}{\pgfqpoint{3.927058in}{1.939522in}}%
\pgfpathcurveto{\pgfqpoint{3.932882in}{1.945346in}}{\pgfqpoint{3.936155in}{1.953246in}}{\pgfqpoint{3.936155in}{1.961482in}}%
\pgfpathcurveto{\pgfqpoint{3.936155in}{1.969719in}}{\pgfqpoint{3.932882in}{1.977619in}}{\pgfqpoint{3.927058in}{1.983443in}}%
\pgfpathcurveto{\pgfqpoint{3.921234in}{1.989267in}}{\pgfqpoint{3.913334in}{1.992539in}}{\pgfqpoint{3.905098in}{1.992539in}}%
\pgfpathcurveto{\pgfqpoint{3.896862in}{1.992539in}}{\pgfqpoint{3.888962in}{1.989267in}}{\pgfqpoint{3.883138in}{1.983443in}}%
\pgfpathcurveto{\pgfqpoint{3.877314in}{1.977619in}}{\pgfqpoint{3.874042in}{1.969719in}}{\pgfqpoint{3.874042in}{1.961482in}}%
\pgfpathcurveto{\pgfqpoint{3.874042in}{1.953246in}}{\pgfqpoint{3.877314in}{1.945346in}}{\pgfqpoint{3.883138in}{1.939522in}}%
\pgfpathcurveto{\pgfqpoint{3.888962in}{1.933698in}}{\pgfqpoint{3.896862in}{1.930426in}}{\pgfqpoint{3.905098in}{1.930426in}}%
\pgfpathclose%
\pgfusepath{stroke,fill}%
\end{pgfscope}%
\begin{pgfscope}%
\pgfpathrectangle{\pgfqpoint{3.793912in}{0.557870in}}{\pgfqpoint{2.446088in}{1.484734in}}%
\pgfusepath{clip}%
\pgfsetbuttcap%
\pgfsetroundjoin%
\definecolor{currentfill}{rgb}{0.298039,0.447059,0.690196}%
\pgfsetfillcolor{currentfill}%
\pgfsetlinewidth{1.003750pt}%
\definecolor{currentstroke}{rgb}{0.298039,0.447059,0.690196}%
\pgfsetstrokecolor{currentstroke}%
\pgfsetdash{}{0pt}%
\pgfpathmoveto{\pgfqpoint{3.905098in}{1.930426in}}%
\pgfpathcurveto{\pgfqpoint{3.913334in}{1.930426in}}{\pgfqpoint{3.921234in}{1.933698in}}{\pgfqpoint{3.927058in}{1.939522in}}%
\pgfpathcurveto{\pgfqpoint{3.932882in}{1.945346in}}{\pgfqpoint{3.936155in}{1.953246in}}{\pgfqpoint{3.936155in}{1.961482in}}%
\pgfpathcurveto{\pgfqpoint{3.936155in}{1.969719in}}{\pgfqpoint{3.932882in}{1.977619in}}{\pgfqpoint{3.927058in}{1.983443in}}%
\pgfpathcurveto{\pgfqpoint{3.921234in}{1.989267in}}{\pgfqpoint{3.913334in}{1.992539in}}{\pgfqpoint{3.905098in}{1.992539in}}%
\pgfpathcurveto{\pgfqpoint{3.896862in}{1.992539in}}{\pgfqpoint{3.888962in}{1.989267in}}{\pgfqpoint{3.883138in}{1.983443in}}%
\pgfpathcurveto{\pgfqpoint{3.877314in}{1.977619in}}{\pgfqpoint{3.874042in}{1.969719in}}{\pgfqpoint{3.874042in}{1.961482in}}%
\pgfpathcurveto{\pgfqpoint{3.874042in}{1.953246in}}{\pgfqpoint{3.877314in}{1.945346in}}{\pgfqpoint{3.883138in}{1.939522in}}%
\pgfpathcurveto{\pgfqpoint{3.888962in}{1.933698in}}{\pgfqpoint{3.896862in}{1.930426in}}{\pgfqpoint{3.905098in}{1.930426in}}%
\pgfpathclose%
\pgfusepath{stroke,fill}%
\end{pgfscope}%
\begin{pgfscope}%
\pgfpathrectangle{\pgfqpoint{3.793912in}{0.557870in}}{\pgfqpoint{2.446088in}{1.484734in}}%
\pgfusepath{clip}%
\pgfsetbuttcap%
\pgfsetroundjoin%
\definecolor{currentfill}{rgb}{0.298039,0.447059,0.690196}%
\pgfsetfillcolor{currentfill}%
\pgfsetlinewidth{1.003750pt}%
\definecolor{currentstroke}{rgb}{0.298039,0.447059,0.690196}%
\pgfsetstrokecolor{currentstroke}%
\pgfsetdash{}{0pt}%
\pgfpathmoveto{\pgfqpoint{3.905098in}{1.603212in}}%
\pgfpathcurveto{\pgfqpoint{3.913334in}{1.603212in}}{\pgfqpoint{3.921234in}{1.606484in}}{\pgfqpoint{3.927058in}{1.612308in}}%
\pgfpathcurveto{\pgfqpoint{3.932882in}{1.618132in}}{\pgfqpoint{3.936155in}{1.626032in}}{\pgfqpoint{3.936155in}{1.634268in}}%
\pgfpathcurveto{\pgfqpoint{3.936155in}{1.642505in}}{\pgfqpoint{3.932882in}{1.650405in}}{\pgfqpoint{3.927058in}{1.656229in}}%
\pgfpathcurveto{\pgfqpoint{3.921234in}{1.662052in}}{\pgfqpoint{3.913334in}{1.665325in}}{\pgfqpoint{3.905098in}{1.665325in}}%
\pgfpathcurveto{\pgfqpoint{3.896862in}{1.665325in}}{\pgfqpoint{3.888962in}{1.662052in}}{\pgfqpoint{3.883138in}{1.656229in}}%
\pgfpathcurveto{\pgfqpoint{3.877314in}{1.650405in}}{\pgfqpoint{3.874042in}{1.642505in}}{\pgfqpoint{3.874042in}{1.634268in}}%
\pgfpathcurveto{\pgfqpoint{3.874042in}{1.626032in}}{\pgfqpoint{3.877314in}{1.618132in}}{\pgfqpoint{3.883138in}{1.612308in}}%
\pgfpathcurveto{\pgfqpoint{3.888962in}{1.606484in}}{\pgfqpoint{3.896862in}{1.603212in}}{\pgfqpoint{3.905098in}{1.603212in}}%
\pgfpathclose%
\pgfusepath{stroke,fill}%
\end{pgfscope}%
\begin{pgfscope}%
\pgfpathrectangle{\pgfqpoint{3.793912in}{0.557870in}}{\pgfqpoint{2.446088in}{1.484734in}}%
\pgfusepath{clip}%
\pgfsetbuttcap%
\pgfsetroundjoin%
\definecolor{currentfill}{rgb}{0.298039,0.447059,0.690196}%
\pgfsetfillcolor{currentfill}%
\pgfsetlinewidth{1.003750pt}%
\definecolor{currentstroke}{rgb}{0.298039,0.447059,0.690196}%
\pgfsetstrokecolor{currentstroke}%
\pgfsetdash{}{0pt}%
\pgfpathmoveto{\pgfqpoint{3.905098in}{1.180560in}}%
\pgfpathcurveto{\pgfqpoint{3.913334in}{1.180560in}}{\pgfqpoint{3.921234in}{1.183833in}}{\pgfqpoint{3.927058in}{1.189656in}}%
\pgfpathcurveto{\pgfqpoint{3.932882in}{1.195480in}}{\pgfqpoint{3.936155in}{1.203380in}}{\pgfqpoint{3.936155in}{1.211617in}}%
\pgfpathcurveto{\pgfqpoint{3.936155in}{1.219853in}}{\pgfqpoint{3.932882in}{1.227753in}}{\pgfqpoint{3.927058in}{1.233577in}}%
\pgfpathcurveto{\pgfqpoint{3.921234in}{1.239401in}}{\pgfqpoint{3.913334in}{1.242673in}}{\pgfqpoint{3.905098in}{1.242673in}}%
\pgfpathcurveto{\pgfqpoint{3.896862in}{1.242673in}}{\pgfqpoint{3.888962in}{1.239401in}}{\pgfqpoint{3.883138in}{1.233577in}}%
\pgfpathcurveto{\pgfqpoint{3.877314in}{1.227753in}}{\pgfqpoint{3.874042in}{1.219853in}}{\pgfqpoint{3.874042in}{1.211617in}}%
\pgfpathcurveto{\pgfqpoint{3.874042in}{1.203380in}}{\pgfqpoint{3.877314in}{1.195480in}}{\pgfqpoint{3.883138in}{1.189656in}}%
\pgfpathcurveto{\pgfqpoint{3.888962in}{1.183833in}}{\pgfqpoint{3.896862in}{1.180560in}}{\pgfqpoint{3.905098in}{1.180560in}}%
\pgfpathclose%
\pgfusepath{stroke,fill}%
\end{pgfscope}%
\begin{pgfscope}%
\pgfpathrectangle{\pgfqpoint{3.793912in}{0.557870in}}{\pgfqpoint{2.446088in}{1.484734in}}%
\pgfusepath{clip}%
\pgfsetbuttcap%
\pgfsetroundjoin%
\definecolor{currentfill}{rgb}{0.298039,0.447059,0.690196}%
\pgfsetfillcolor{currentfill}%
\pgfsetlinewidth{1.003750pt}%
\definecolor{currentstroke}{rgb}{0.298039,0.447059,0.690196}%
\pgfsetstrokecolor{currentstroke}%
\pgfsetdash{}{0pt}%
\pgfpathmoveto{\pgfqpoint{3.905098in}{1.644114in}}%
\pgfpathcurveto{\pgfqpoint{3.913334in}{1.644114in}}{\pgfqpoint{3.921234in}{1.647386in}}{\pgfqpoint{3.927058in}{1.653210in}}%
\pgfpathcurveto{\pgfqpoint{3.932882in}{1.659034in}}{\pgfqpoint{3.936155in}{1.666934in}}{\pgfqpoint{3.936155in}{1.675170in}}%
\pgfpathcurveto{\pgfqpoint{3.936155in}{1.683406in}}{\pgfqpoint{3.932882in}{1.691306in}}{\pgfqpoint{3.927058in}{1.697130in}}%
\pgfpathcurveto{\pgfqpoint{3.921234in}{1.702954in}}{\pgfqpoint{3.913334in}{1.706227in}}{\pgfqpoint{3.905098in}{1.706227in}}%
\pgfpathcurveto{\pgfqpoint{3.896862in}{1.706227in}}{\pgfqpoint{3.888962in}{1.702954in}}{\pgfqpoint{3.883138in}{1.697130in}}%
\pgfpathcurveto{\pgfqpoint{3.877314in}{1.691306in}}{\pgfqpoint{3.874042in}{1.683406in}}{\pgfqpoint{3.874042in}{1.675170in}}%
\pgfpathcurveto{\pgfqpoint{3.874042in}{1.666934in}}{\pgfqpoint{3.877314in}{1.659034in}}{\pgfqpoint{3.883138in}{1.653210in}}%
\pgfpathcurveto{\pgfqpoint{3.888962in}{1.647386in}}{\pgfqpoint{3.896862in}{1.644114in}}{\pgfqpoint{3.905098in}{1.644114in}}%
\pgfpathclose%
\pgfusepath{stroke,fill}%
\end{pgfscope}%
\begin{pgfscope}%
\pgfpathrectangle{\pgfqpoint{3.793912in}{0.557870in}}{\pgfqpoint{2.446088in}{1.484734in}}%
\pgfusepath{clip}%
\pgfsetbuttcap%
\pgfsetroundjoin%
\definecolor{currentfill}{rgb}{0.298039,0.447059,0.690196}%
\pgfsetfillcolor{currentfill}%
\pgfsetlinewidth{1.003750pt}%
\definecolor{currentstroke}{rgb}{0.298039,0.447059,0.690196}%
\pgfsetstrokecolor{currentstroke}%
\pgfsetdash{}{0pt}%
\pgfpathmoveto{\pgfqpoint{3.905098in}{0.921516in}}%
\pgfpathcurveto{\pgfqpoint{3.913334in}{0.921516in}}{\pgfqpoint{3.921234in}{0.924788in}}{\pgfqpoint{3.927058in}{0.930612in}}%
\pgfpathcurveto{\pgfqpoint{3.932882in}{0.936436in}}{\pgfqpoint{3.936155in}{0.944336in}}{\pgfqpoint{3.936155in}{0.952572in}}%
\pgfpathcurveto{\pgfqpoint{3.936155in}{0.960809in}}{\pgfqpoint{3.932882in}{0.968709in}}{\pgfqpoint{3.927058in}{0.974533in}}%
\pgfpathcurveto{\pgfqpoint{3.921234in}{0.980356in}}{\pgfqpoint{3.913334in}{0.983629in}}{\pgfqpoint{3.905098in}{0.983629in}}%
\pgfpathcurveto{\pgfqpoint{3.896862in}{0.983629in}}{\pgfqpoint{3.888962in}{0.980356in}}{\pgfqpoint{3.883138in}{0.974533in}}%
\pgfpathcurveto{\pgfqpoint{3.877314in}{0.968709in}}{\pgfqpoint{3.874042in}{0.960809in}}{\pgfqpoint{3.874042in}{0.952572in}}%
\pgfpathcurveto{\pgfqpoint{3.874042in}{0.944336in}}{\pgfqpoint{3.877314in}{0.936436in}}{\pgfqpoint{3.883138in}{0.930612in}}%
\pgfpathcurveto{\pgfqpoint{3.888962in}{0.924788in}}{\pgfqpoint{3.896862in}{0.921516in}}{\pgfqpoint{3.905098in}{0.921516in}}%
\pgfpathclose%
\pgfusepath{stroke,fill}%
\end{pgfscope}%
\begin{pgfscope}%
\pgfpathrectangle{\pgfqpoint{3.793912in}{0.557870in}}{\pgfqpoint{2.446088in}{1.484734in}}%
\pgfusepath{clip}%
\pgfsetbuttcap%
\pgfsetroundjoin%
\definecolor{currentfill}{rgb}{0.298039,0.447059,0.690196}%
\pgfsetfillcolor{currentfill}%
\pgfsetlinewidth{1.003750pt}%
\definecolor{currentstroke}{rgb}{0.298039,0.447059,0.690196}%
\pgfsetstrokecolor{currentstroke}%
\pgfsetdash{}{0pt}%
\pgfpathmoveto{\pgfqpoint{3.905098in}{0.921516in}}%
\pgfpathcurveto{\pgfqpoint{3.913334in}{0.921516in}}{\pgfqpoint{3.921234in}{0.924788in}}{\pgfqpoint{3.927058in}{0.930612in}}%
\pgfpathcurveto{\pgfqpoint{3.932882in}{0.936436in}}{\pgfqpoint{3.936155in}{0.944336in}}{\pgfqpoint{3.936155in}{0.952572in}}%
\pgfpathcurveto{\pgfqpoint{3.936155in}{0.960809in}}{\pgfqpoint{3.932882in}{0.968709in}}{\pgfqpoint{3.927058in}{0.974533in}}%
\pgfpathcurveto{\pgfqpoint{3.921234in}{0.980356in}}{\pgfqpoint{3.913334in}{0.983629in}}{\pgfqpoint{3.905098in}{0.983629in}}%
\pgfpathcurveto{\pgfqpoint{3.896862in}{0.983629in}}{\pgfqpoint{3.888962in}{0.980356in}}{\pgfqpoint{3.883138in}{0.974533in}}%
\pgfpathcurveto{\pgfqpoint{3.877314in}{0.968709in}}{\pgfqpoint{3.874042in}{0.960809in}}{\pgfqpoint{3.874042in}{0.952572in}}%
\pgfpathcurveto{\pgfqpoint{3.874042in}{0.944336in}}{\pgfqpoint{3.877314in}{0.936436in}}{\pgfqpoint{3.883138in}{0.930612in}}%
\pgfpathcurveto{\pgfqpoint{3.888962in}{0.924788in}}{\pgfqpoint{3.896862in}{0.921516in}}{\pgfqpoint{3.905098in}{0.921516in}}%
\pgfpathclose%
\pgfusepath{stroke,fill}%
\end{pgfscope}%
\begin{pgfscope}%
\pgfpathrectangle{\pgfqpoint{3.793912in}{0.557870in}}{\pgfqpoint{2.446088in}{1.484734in}}%
\pgfusepath{clip}%
\pgfsetbuttcap%
\pgfsetroundjoin%
\definecolor{currentfill}{rgb}{0.298039,0.447059,0.690196}%
\pgfsetfillcolor{currentfill}%
\pgfsetlinewidth{1.003750pt}%
\definecolor{currentstroke}{rgb}{0.298039,0.447059,0.690196}%
\pgfsetstrokecolor{currentstroke}%
\pgfsetdash{}{0pt}%
\pgfpathmoveto{\pgfqpoint{5.457067in}{0.594302in}}%
\pgfpathcurveto{\pgfqpoint{5.465303in}{0.594302in}}{\pgfqpoint{5.473203in}{0.597574in}}{\pgfqpoint{5.479027in}{0.603398in}}%
\pgfpathcurveto{\pgfqpoint{5.484851in}{0.609222in}}{\pgfqpoint{5.488123in}{0.617122in}}{\pgfqpoint{5.488123in}{0.625358in}}%
\pgfpathcurveto{\pgfqpoint{5.488123in}{0.633594in}}{\pgfqpoint{5.484851in}{0.641495in}}{\pgfqpoint{5.479027in}{0.647318in}}%
\pgfpathcurveto{\pgfqpoint{5.473203in}{0.653142in}}{\pgfqpoint{5.465303in}{0.656415in}}{\pgfqpoint{5.457067in}{0.656415in}}%
\pgfpathcurveto{\pgfqpoint{5.448830in}{0.656415in}}{\pgfqpoint{5.440930in}{0.653142in}}{\pgfqpoint{5.435106in}{0.647318in}}%
\pgfpathcurveto{\pgfqpoint{5.429282in}{0.641495in}}{\pgfqpoint{5.426010in}{0.633594in}}{\pgfqpoint{5.426010in}{0.625358in}}%
\pgfpathcurveto{\pgfqpoint{5.426010in}{0.617122in}}{\pgfqpoint{5.429282in}{0.609222in}}{\pgfqpoint{5.435106in}{0.603398in}}%
\pgfpathcurveto{\pgfqpoint{5.440930in}{0.597574in}}{\pgfqpoint{5.448830in}{0.594302in}}{\pgfqpoint{5.457067in}{0.594302in}}%
\pgfpathclose%
\pgfusepath{stroke,fill}%
\end{pgfscope}%
\begin{pgfscope}%
\pgfpathrectangle{\pgfqpoint{3.793912in}{0.557870in}}{\pgfqpoint{2.446088in}{1.484734in}}%
\pgfusepath{clip}%
\pgfsetbuttcap%
\pgfsetroundjoin%
\definecolor{currentfill}{rgb}{0.298039,0.447059,0.690196}%
\pgfsetfillcolor{currentfill}%
\pgfsetlinewidth{1.003750pt}%
\definecolor{currentstroke}{rgb}{0.298039,0.447059,0.690196}%
\pgfsetstrokecolor{currentstroke}%
\pgfsetdash{}{0pt}%
\pgfpathmoveto{\pgfqpoint{3.905098in}{1.739551in}}%
\pgfpathcurveto{\pgfqpoint{3.913334in}{1.739551in}}{\pgfqpoint{3.921234in}{1.742823in}}{\pgfqpoint{3.927058in}{1.748647in}}%
\pgfpathcurveto{\pgfqpoint{3.932882in}{1.754471in}}{\pgfqpoint{3.936155in}{1.762371in}}{\pgfqpoint{3.936155in}{1.770607in}}%
\pgfpathcurveto{\pgfqpoint{3.936155in}{1.778844in}}{\pgfqpoint{3.932882in}{1.786744in}}{\pgfqpoint{3.927058in}{1.792568in}}%
\pgfpathcurveto{\pgfqpoint{3.921234in}{1.798392in}}{\pgfqpoint{3.913334in}{1.801664in}}{\pgfqpoint{3.905098in}{1.801664in}}%
\pgfpathcurveto{\pgfqpoint{3.896862in}{1.801664in}}{\pgfqpoint{3.888962in}{1.798392in}}{\pgfqpoint{3.883138in}{1.792568in}}%
\pgfpathcurveto{\pgfqpoint{3.877314in}{1.786744in}}{\pgfqpoint{3.874042in}{1.778844in}}{\pgfqpoint{3.874042in}{1.770607in}}%
\pgfpathcurveto{\pgfqpoint{3.874042in}{1.762371in}}{\pgfqpoint{3.877314in}{1.754471in}}{\pgfqpoint{3.883138in}{1.748647in}}%
\pgfpathcurveto{\pgfqpoint{3.888962in}{1.742823in}}{\pgfqpoint{3.896862in}{1.739551in}}{\pgfqpoint{3.905098in}{1.739551in}}%
\pgfpathclose%
\pgfusepath{stroke,fill}%
\end{pgfscope}%
\begin{pgfscope}%
\pgfpathrectangle{\pgfqpoint{3.793912in}{0.557870in}}{\pgfqpoint{2.446088in}{1.484734in}}%
\pgfusepath{clip}%
\pgfsetbuttcap%
\pgfsetroundjoin%
\definecolor{currentfill}{rgb}{0.298039,0.447059,0.690196}%
\pgfsetfillcolor{currentfill}%
\pgfsetlinewidth{1.003750pt}%
\definecolor{currentstroke}{rgb}{0.298039,0.447059,0.690196}%
\pgfsetstrokecolor{currentstroke}%
\pgfsetdash{}{0pt}%
\pgfpathmoveto{\pgfqpoint{5.341248in}{0.594302in}}%
\pgfpathcurveto{\pgfqpoint{5.349484in}{0.594302in}}{\pgfqpoint{5.357384in}{0.597574in}}{\pgfqpoint{5.363208in}{0.603398in}}%
\pgfpathcurveto{\pgfqpoint{5.369032in}{0.609222in}}{\pgfqpoint{5.372305in}{0.617122in}}{\pgfqpoint{5.372305in}{0.625358in}}%
\pgfpathcurveto{\pgfqpoint{5.372305in}{0.633594in}}{\pgfqpoint{5.369032in}{0.641495in}}{\pgfqpoint{5.363208in}{0.647318in}}%
\pgfpathcurveto{\pgfqpoint{5.357384in}{0.653142in}}{\pgfqpoint{5.349484in}{0.656415in}}{\pgfqpoint{5.341248in}{0.656415in}}%
\pgfpathcurveto{\pgfqpoint{5.333012in}{0.656415in}}{\pgfqpoint{5.325112in}{0.653142in}}{\pgfqpoint{5.319288in}{0.647318in}}%
\pgfpathcurveto{\pgfqpoint{5.313464in}{0.641495in}}{\pgfqpoint{5.310192in}{0.633594in}}{\pgfqpoint{5.310192in}{0.625358in}}%
\pgfpathcurveto{\pgfqpoint{5.310192in}{0.617122in}}{\pgfqpoint{5.313464in}{0.609222in}}{\pgfqpoint{5.319288in}{0.603398in}}%
\pgfpathcurveto{\pgfqpoint{5.325112in}{0.597574in}}{\pgfqpoint{5.333012in}{0.594302in}}{\pgfqpoint{5.341248in}{0.594302in}}%
\pgfpathclose%
\pgfusepath{stroke,fill}%
\end{pgfscope}%
\begin{pgfscope}%
\pgfpathrectangle{\pgfqpoint{3.793912in}{0.557870in}}{\pgfqpoint{2.446088in}{1.484734in}}%
\pgfusepath{clip}%
\pgfsetbuttcap%
\pgfsetroundjoin%
\definecolor{currentfill}{rgb}{0.298039,0.447059,0.690196}%
\pgfsetfillcolor{currentfill}%
\pgfsetlinewidth{1.003750pt}%
\definecolor{currentstroke}{rgb}{0.298039,0.447059,0.690196}%
\pgfsetstrokecolor{currentstroke}%
\pgfsetdash{}{0pt}%
\pgfpathmoveto{\pgfqpoint{3.905098in}{1.766819in}}%
\pgfpathcurveto{\pgfqpoint{3.913334in}{1.766819in}}{\pgfqpoint{3.921234in}{1.770091in}}{\pgfqpoint{3.927058in}{1.775915in}}%
\pgfpathcurveto{\pgfqpoint{3.932882in}{1.781739in}}{\pgfqpoint{3.936155in}{1.789639in}}{\pgfqpoint{3.936155in}{1.797875in}}%
\pgfpathcurveto{\pgfqpoint{3.936155in}{1.806112in}}{\pgfqpoint{3.932882in}{1.814012in}}{\pgfqpoint{3.927058in}{1.819836in}}%
\pgfpathcurveto{\pgfqpoint{3.921234in}{1.825659in}}{\pgfqpoint{3.913334in}{1.828932in}}{\pgfqpoint{3.905098in}{1.828932in}}%
\pgfpathcurveto{\pgfqpoint{3.896862in}{1.828932in}}{\pgfqpoint{3.888962in}{1.825659in}}{\pgfqpoint{3.883138in}{1.819836in}}%
\pgfpathcurveto{\pgfqpoint{3.877314in}{1.814012in}}{\pgfqpoint{3.874042in}{1.806112in}}{\pgfqpoint{3.874042in}{1.797875in}}%
\pgfpathcurveto{\pgfqpoint{3.874042in}{1.789639in}}{\pgfqpoint{3.877314in}{1.781739in}}{\pgfqpoint{3.883138in}{1.775915in}}%
\pgfpathcurveto{\pgfqpoint{3.888962in}{1.770091in}}{\pgfqpoint{3.896862in}{1.766819in}}{\pgfqpoint{3.905098in}{1.766819in}}%
\pgfpathclose%
\pgfusepath{stroke,fill}%
\end{pgfscope}%
\begin{pgfscope}%
\pgfpathrectangle{\pgfqpoint{3.793912in}{0.557870in}}{\pgfqpoint{2.446088in}{1.484734in}}%
\pgfusepath{clip}%
\pgfsetbuttcap%
\pgfsetroundjoin%
\definecolor{currentfill}{rgb}{0.298039,0.447059,0.690196}%
\pgfsetfillcolor{currentfill}%
\pgfsetlinewidth{1.003750pt}%
\definecolor{currentstroke}{rgb}{0.298039,0.447059,0.690196}%
\pgfsetstrokecolor{currentstroke}%
\pgfsetdash{}{0pt}%
\pgfpathmoveto{\pgfqpoint{3.905098in}{1.371435in}}%
\pgfpathcurveto{\pgfqpoint{3.913334in}{1.371435in}}{\pgfqpoint{3.921234in}{1.374707in}}{\pgfqpoint{3.927058in}{1.380531in}}%
\pgfpathcurveto{\pgfqpoint{3.932882in}{1.386355in}}{\pgfqpoint{3.936155in}{1.394255in}}{\pgfqpoint{3.936155in}{1.402492in}}%
\pgfpathcurveto{\pgfqpoint{3.936155in}{1.410728in}}{\pgfqpoint{3.932882in}{1.418628in}}{\pgfqpoint{3.927058in}{1.424452in}}%
\pgfpathcurveto{\pgfqpoint{3.921234in}{1.430276in}}{\pgfqpoint{3.913334in}{1.433548in}}{\pgfqpoint{3.905098in}{1.433548in}}%
\pgfpathcurveto{\pgfqpoint{3.896862in}{1.433548in}}{\pgfqpoint{3.888962in}{1.430276in}}{\pgfqpoint{3.883138in}{1.424452in}}%
\pgfpathcurveto{\pgfqpoint{3.877314in}{1.418628in}}{\pgfqpoint{3.874042in}{1.410728in}}{\pgfqpoint{3.874042in}{1.402492in}}%
\pgfpathcurveto{\pgfqpoint{3.874042in}{1.394255in}}{\pgfqpoint{3.877314in}{1.386355in}}{\pgfqpoint{3.883138in}{1.380531in}}%
\pgfpathcurveto{\pgfqpoint{3.888962in}{1.374707in}}{\pgfqpoint{3.896862in}{1.371435in}}{\pgfqpoint{3.905098in}{1.371435in}}%
\pgfpathclose%
\pgfusepath{stroke,fill}%
\end{pgfscope}%
\begin{pgfscope}%
\pgfpathrectangle{\pgfqpoint{3.793912in}{0.557870in}}{\pgfqpoint{2.446088in}{1.484734in}}%
\pgfusepath{clip}%
\pgfsetbuttcap%
\pgfsetroundjoin%
\definecolor{currentfill}{rgb}{0.298039,0.447059,0.690196}%
\pgfsetfillcolor{currentfill}%
\pgfsetlinewidth{1.003750pt}%
\definecolor{currentstroke}{rgb}{0.298039,0.447059,0.690196}%
\pgfsetstrokecolor{currentstroke}%
\pgfsetdash{}{0pt}%
\pgfpathmoveto{\pgfqpoint{3.905098in}{1.371435in}}%
\pgfpathcurveto{\pgfqpoint{3.913334in}{1.371435in}}{\pgfqpoint{3.921234in}{1.374707in}}{\pgfqpoint{3.927058in}{1.380531in}}%
\pgfpathcurveto{\pgfqpoint{3.932882in}{1.386355in}}{\pgfqpoint{3.936155in}{1.394255in}}{\pgfqpoint{3.936155in}{1.402492in}}%
\pgfpathcurveto{\pgfqpoint{3.936155in}{1.410728in}}{\pgfqpoint{3.932882in}{1.418628in}}{\pgfqpoint{3.927058in}{1.424452in}}%
\pgfpathcurveto{\pgfqpoint{3.921234in}{1.430276in}}{\pgfqpoint{3.913334in}{1.433548in}}{\pgfqpoint{3.905098in}{1.433548in}}%
\pgfpathcurveto{\pgfqpoint{3.896862in}{1.433548in}}{\pgfqpoint{3.888962in}{1.430276in}}{\pgfqpoint{3.883138in}{1.424452in}}%
\pgfpathcurveto{\pgfqpoint{3.877314in}{1.418628in}}{\pgfqpoint{3.874042in}{1.410728in}}{\pgfqpoint{3.874042in}{1.402492in}}%
\pgfpathcurveto{\pgfqpoint{3.874042in}{1.394255in}}{\pgfqpoint{3.877314in}{1.386355in}}{\pgfqpoint{3.883138in}{1.380531in}}%
\pgfpathcurveto{\pgfqpoint{3.888962in}{1.374707in}}{\pgfqpoint{3.896862in}{1.371435in}}{\pgfqpoint{3.905098in}{1.371435in}}%
\pgfpathclose%
\pgfusepath{stroke,fill}%
\end{pgfscope}%
\begin{pgfscope}%
\pgfpathrectangle{\pgfqpoint{3.793912in}{0.557870in}}{\pgfqpoint{2.446088in}{1.484734in}}%
\pgfusepath{clip}%
\pgfsetbuttcap%
\pgfsetroundjoin%
\definecolor{currentfill}{rgb}{0.298039,0.447059,0.690196}%
\pgfsetfillcolor{currentfill}%
\pgfsetlinewidth{1.003750pt}%
\definecolor{currentstroke}{rgb}{0.298039,0.447059,0.690196}%
\pgfsetstrokecolor{currentstroke}%
\pgfsetdash{}{0pt}%
\pgfpathmoveto{\pgfqpoint{3.905098in}{1.930426in}}%
\pgfpathcurveto{\pgfqpoint{3.913334in}{1.930426in}}{\pgfqpoint{3.921234in}{1.933698in}}{\pgfqpoint{3.927058in}{1.939522in}}%
\pgfpathcurveto{\pgfqpoint{3.932882in}{1.945346in}}{\pgfqpoint{3.936155in}{1.953246in}}{\pgfqpoint{3.936155in}{1.961482in}}%
\pgfpathcurveto{\pgfqpoint{3.936155in}{1.969719in}}{\pgfqpoint{3.932882in}{1.977619in}}{\pgfqpoint{3.927058in}{1.983443in}}%
\pgfpathcurveto{\pgfqpoint{3.921234in}{1.989267in}}{\pgfqpoint{3.913334in}{1.992539in}}{\pgfqpoint{3.905098in}{1.992539in}}%
\pgfpathcurveto{\pgfqpoint{3.896862in}{1.992539in}}{\pgfqpoint{3.888962in}{1.989267in}}{\pgfqpoint{3.883138in}{1.983443in}}%
\pgfpathcurveto{\pgfqpoint{3.877314in}{1.977619in}}{\pgfqpoint{3.874042in}{1.969719in}}{\pgfqpoint{3.874042in}{1.961482in}}%
\pgfpathcurveto{\pgfqpoint{3.874042in}{1.953246in}}{\pgfqpoint{3.877314in}{1.945346in}}{\pgfqpoint{3.883138in}{1.939522in}}%
\pgfpathcurveto{\pgfqpoint{3.888962in}{1.933698in}}{\pgfqpoint{3.896862in}{1.930426in}}{\pgfqpoint{3.905098in}{1.930426in}}%
\pgfpathclose%
\pgfusepath{stroke,fill}%
\end{pgfscope}%
\begin{pgfscope}%
\pgfpathrectangle{\pgfqpoint{3.793912in}{0.557870in}}{\pgfqpoint{2.446088in}{1.484734in}}%
\pgfusepath{clip}%
\pgfsetbuttcap%
\pgfsetroundjoin%
\definecolor{currentfill}{rgb}{0.298039,0.447059,0.690196}%
\pgfsetfillcolor{currentfill}%
\pgfsetlinewidth{1.003750pt}%
\definecolor{currentstroke}{rgb}{0.298039,0.447059,0.690196}%
\pgfsetstrokecolor{currentstroke}%
\pgfsetdash{}{0pt}%
\pgfpathmoveto{\pgfqpoint{3.905098in}{1.930426in}}%
\pgfpathcurveto{\pgfqpoint{3.913334in}{1.930426in}}{\pgfqpoint{3.921234in}{1.933698in}}{\pgfqpoint{3.927058in}{1.939522in}}%
\pgfpathcurveto{\pgfqpoint{3.932882in}{1.945346in}}{\pgfqpoint{3.936155in}{1.953246in}}{\pgfqpoint{3.936155in}{1.961482in}}%
\pgfpathcurveto{\pgfqpoint{3.936155in}{1.969719in}}{\pgfqpoint{3.932882in}{1.977619in}}{\pgfqpoint{3.927058in}{1.983443in}}%
\pgfpathcurveto{\pgfqpoint{3.921234in}{1.989267in}}{\pgfqpoint{3.913334in}{1.992539in}}{\pgfqpoint{3.905098in}{1.992539in}}%
\pgfpathcurveto{\pgfqpoint{3.896862in}{1.992539in}}{\pgfqpoint{3.888962in}{1.989267in}}{\pgfqpoint{3.883138in}{1.983443in}}%
\pgfpathcurveto{\pgfqpoint{3.877314in}{1.977619in}}{\pgfqpoint{3.874042in}{1.969719in}}{\pgfqpoint{3.874042in}{1.961482in}}%
\pgfpathcurveto{\pgfqpoint{3.874042in}{1.953246in}}{\pgfqpoint{3.877314in}{1.945346in}}{\pgfqpoint{3.883138in}{1.939522in}}%
\pgfpathcurveto{\pgfqpoint{3.888962in}{1.933698in}}{\pgfqpoint{3.896862in}{1.930426in}}{\pgfqpoint{3.905098in}{1.930426in}}%
\pgfpathclose%
\pgfusepath{stroke,fill}%
\end{pgfscope}%
\begin{pgfscope}%
\pgfpathrectangle{\pgfqpoint{3.793912in}{0.557870in}}{\pgfqpoint{2.446088in}{1.484734in}}%
\pgfusepath{clip}%
\pgfsetbuttcap%
\pgfsetroundjoin%
\definecolor{currentfill}{rgb}{0.298039,0.447059,0.690196}%
\pgfsetfillcolor{currentfill}%
\pgfsetlinewidth{1.003750pt}%
\definecolor{currentstroke}{rgb}{0.298039,0.447059,0.690196}%
\pgfsetstrokecolor{currentstroke}%
\pgfsetdash{}{0pt}%
\pgfpathmoveto{\pgfqpoint{3.905098in}{1.930426in}}%
\pgfpathcurveto{\pgfqpoint{3.913334in}{1.930426in}}{\pgfqpoint{3.921234in}{1.933698in}}{\pgfqpoint{3.927058in}{1.939522in}}%
\pgfpathcurveto{\pgfqpoint{3.932882in}{1.945346in}}{\pgfqpoint{3.936155in}{1.953246in}}{\pgfqpoint{3.936155in}{1.961482in}}%
\pgfpathcurveto{\pgfqpoint{3.936155in}{1.969719in}}{\pgfqpoint{3.932882in}{1.977619in}}{\pgfqpoint{3.927058in}{1.983443in}}%
\pgfpathcurveto{\pgfqpoint{3.921234in}{1.989267in}}{\pgfqpoint{3.913334in}{1.992539in}}{\pgfqpoint{3.905098in}{1.992539in}}%
\pgfpathcurveto{\pgfqpoint{3.896862in}{1.992539in}}{\pgfqpoint{3.888962in}{1.989267in}}{\pgfqpoint{3.883138in}{1.983443in}}%
\pgfpathcurveto{\pgfqpoint{3.877314in}{1.977619in}}{\pgfqpoint{3.874042in}{1.969719in}}{\pgfqpoint{3.874042in}{1.961482in}}%
\pgfpathcurveto{\pgfqpoint{3.874042in}{1.953246in}}{\pgfqpoint{3.877314in}{1.945346in}}{\pgfqpoint{3.883138in}{1.939522in}}%
\pgfpathcurveto{\pgfqpoint{3.888962in}{1.933698in}}{\pgfqpoint{3.896862in}{1.930426in}}{\pgfqpoint{3.905098in}{1.930426in}}%
\pgfpathclose%
\pgfusepath{stroke,fill}%
\end{pgfscope}%
\begin{pgfscope}%
\pgfpathrectangle{\pgfqpoint{3.793912in}{0.557870in}}{\pgfqpoint{2.446088in}{1.484734in}}%
\pgfusepath{clip}%
\pgfsetbuttcap%
\pgfsetroundjoin%
\definecolor{currentfill}{rgb}{0.298039,0.447059,0.690196}%
\pgfsetfillcolor{currentfill}%
\pgfsetlinewidth{1.003750pt}%
\definecolor{currentstroke}{rgb}{0.298039,0.447059,0.690196}%
\pgfsetstrokecolor{currentstroke}%
\pgfsetdash{}{0pt}%
\pgfpathmoveto{\pgfqpoint{4.854810in}{0.594302in}}%
\pgfpathcurveto{\pgfqpoint{4.863046in}{0.594302in}}{\pgfqpoint{4.870946in}{0.597574in}}{\pgfqpoint{4.876770in}{0.603398in}}%
\pgfpathcurveto{\pgfqpoint{4.882594in}{0.609222in}}{\pgfqpoint{4.885867in}{0.617122in}}{\pgfqpoint{4.885867in}{0.625358in}}%
\pgfpathcurveto{\pgfqpoint{4.885867in}{0.633594in}}{\pgfqpoint{4.882594in}{0.641495in}}{\pgfqpoint{4.876770in}{0.647318in}}%
\pgfpathcurveto{\pgfqpoint{4.870946in}{0.653142in}}{\pgfqpoint{4.863046in}{0.656415in}}{\pgfqpoint{4.854810in}{0.656415in}}%
\pgfpathcurveto{\pgfqpoint{4.846574in}{0.656415in}}{\pgfqpoint{4.838674in}{0.653142in}}{\pgfqpoint{4.832850in}{0.647318in}}%
\pgfpathcurveto{\pgfqpoint{4.827026in}{0.641495in}}{\pgfqpoint{4.823754in}{0.633594in}}{\pgfqpoint{4.823754in}{0.625358in}}%
\pgfpathcurveto{\pgfqpoint{4.823754in}{0.617122in}}{\pgfqpoint{4.827026in}{0.609222in}}{\pgfqpoint{4.832850in}{0.603398in}}%
\pgfpathcurveto{\pgfqpoint{4.838674in}{0.597574in}}{\pgfqpoint{4.846574in}{0.594302in}}{\pgfqpoint{4.854810in}{0.594302in}}%
\pgfpathclose%
\pgfusepath{stroke,fill}%
\end{pgfscope}%
\begin{pgfscope}%
\pgfpathrectangle{\pgfqpoint{3.793912in}{0.557870in}}{\pgfqpoint{2.446088in}{1.484734in}}%
\pgfusepath{clip}%
\pgfsetbuttcap%
\pgfsetroundjoin%
\definecolor{currentfill}{rgb}{0.298039,0.447059,0.690196}%
\pgfsetfillcolor{currentfill}%
\pgfsetlinewidth{1.003750pt}%
\definecolor{currentstroke}{rgb}{0.298039,0.447059,0.690196}%
\pgfsetstrokecolor{currentstroke}%
\pgfsetdash{}{0pt}%
\pgfpathmoveto{\pgfqpoint{3.905098in}{1.930426in}}%
\pgfpathcurveto{\pgfqpoint{3.913334in}{1.930426in}}{\pgfqpoint{3.921234in}{1.933698in}}{\pgfqpoint{3.927058in}{1.939522in}}%
\pgfpathcurveto{\pgfqpoint{3.932882in}{1.945346in}}{\pgfqpoint{3.936155in}{1.953246in}}{\pgfqpoint{3.936155in}{1.961482in}}%
\pgfpathcurveto{\pgfqpoint{3.936155in}{1.969719in}}{\pgfqpoint{3.932882in}{1.977619in}}{\pgfqpoint{3.927058in}{1.983443in}}%
\pgfpathcurveto{\pgfqpoint{3.921234in}{1.989267in}}{\pgfqpoint{3.913334in}{1.992539in}}{\pgfqpoint{3.905098in}{1.992539in}}%
\pgfpathcurveto{\pgfqpoint{3.896862in}{1.992539in}}{\pgfqpoint{3.888962in}{1.989267in}}{\pgfqpoint{3.883138in}{1.983443in}}%
\pgfpathcurveto{\pgfqpoint{3.877314in}{1.977619in}}{\pgfqpoint{3.874042in}{1.969719in}}{\pgfqpoint{3.874042in}{1.961482in}}%
\pgfpathcurveto{\pgfqpoint{3.874042in}{1.953246in}}{\pgfqpoint{3.877314in}{1.945346in}}{\pgfqpoint{3.883138in}{1.939522in}}%
\pgfpathcurveto{\pgfqpoint{3.888962in}{1.933698in}}{\pgfqpoint{3.896862in}{1.930426in}}{\pgfqpoint{3.905098in}{1.930426in}}%
\pgfpathclose%
\pgfusepath{stroke,fill}%
\end{pgfscope}%
\begin{pgfscope}%
\pgfpathrectangle{\pgfqpoint{3.793912in}{0.557870in}}{\pgfqpoint{2.446088in}{1.484734in}}%
\pgfusepath{clip}%
\pgfsetbuttcap%
\pgfsetroundjoin%
\definecolor{currentfill}{rgb}{0.298039,0.447059,0.690196}%
\pgfsetfillcolor{currentfill}%
\pgfsetlinewidth{1.003750pt}%
\definecolor{currentstroke}{rgb}{0.298039,0.447059,0.690196}%
\pgfsetstrokecolor{currentstroke}%
\pgfsetdash{}{0pt}%
\pgfpathmoveto{\pgfqpoint{3.905098in}{1.930426in}}%
\pgfpathcurveto{\pgfqpoint{3.913334in}{1.930426in}}{\pgfqpoint{3.921234in}{1.933698in}}{\pgfqpoint{3.927058in}{1.939522in}}%
\pgfpathcurveto{\pgfqpoint{3.932882in}{1.945346in}}{\pgfqpoint{3.936155in}{1.953246in}}{\pgfqpoint{3.936155in}{1.961482in}}%
\pgfpathcurveto{\pgfqpoint{3.936155in}{1.969719in}}{\pgfqpoint{3.932882in}{1.977619in}}{\pgfqpoint{3.927058in}{1.983443in}}%
\pgfpathcurveto{\pgfqpoint{3.921234in}{1.989267in}}{\pgfqpoint{3.913334in}{1.992539in}}{\pgfqpoint{3.905098in}{1.992539in}}%
\pgfpathcurveto{\pgfqpoint{3.896862in}{1.992539in}}{\pgfqpoint{3.888962in}{1.989267in}}{\pgfqpoint{3.883138in}{1.983443in}}%
\pgfpathcurveto{\pgfqpoint{3.877314in}{1.977619in}}{\pgfqpoint{3.874042in}{1.969719in}}{\pgfqpoint{3.874042in}{1.961482in}}%
\pgfpathcurveto{\pgfqpoint{3.874042in}{1.953246in}}{\pgfqpoint{3.877314in}{1.945346in}}{\pgfqpoint{3.883138in}{1.939522in}}%
\pgfpathcurveto{\pgfqpoint{3.888962in}{1.933698in}}{\pgfqpoint{3.896862in}{1.930426in}}{\pgfqpoint{3.905098in}{1.930426in}}%
\pgfpathclose%
\pgfusepath{stroke,fill}%
\end{pgfscope}%
\begin{pgfscope}%
\pgfpathrectangle{\pgfqpoint{3.793912in}{0.557870in}}{\pgfqpoint{2.446088in}{1.484734in}}%
\pgfusepath{clip}%
\pgfsetbuttcap%
\pgfsetroundjoin%
\definecolor{currentfill}{rgb}{0.298039,0.447059,0.690196}%
\pgfsetfillcolor{currentfill}%
\pgfsetlinewidth{1.003750pt}%
\definecolor{currentstroke}{rgb}{0.298039,0.447059,0.690196}%
\pgfsetstrokecolor{currentstroke}%
\pgfsetdash{}{0pt}%
\pgfpathmoveto{\pgfqpoint{3.905098in}{1.930426in}}%
\pgfpathcurveto{\pgfqpoint{3.913334in}{1.930426in}}{\pgfqpoint{3.921234in}{1.933698in}}{\pgfqpoint{3.927058in}{1.939522in}}%
\pgfpathcurveto{\pgfqpoint{3.932882in}{1.945346in}}{\pgfqpoint{3.936155in}{1.953246in}}{\pgfqpoint{3.936155in}{1.961482in}}%
\pgfpathcurveto{\pgfqpoint{3.936155in}{1.969719in}}{\pgfqpoint{3.932882in}{1.977619in}}{\pgfqpoint{3.927058in}{1.983443in}}%
\pgfpathcurveto{\pgfqpoint{3.921234in}{1.989267in}}{\pgfqpoint{3.913334in}{1.992539in}}{\pgfqpoint{3.905098in}{1.992539in}}%
\pgfpathcurveto{\pgfqpoint{3.896862in}{1.992539in}}{\pgfqpoint{3.888962in}{1.989267in}}{\pgfqpoint{3.883138in}{1.983443in}}%
\pgfpathcurveto{\pgfqpoint{3.877314in}{1.977619in}}{\pgfqpoint{3.874042in}{1.969719in}}{\pgfqpoint{3.874042in}{1.961482in}}%
\pgfpathcurveto{\pgfqpoint{3.874042in}{1.953246in}}{\pgfqpoint{3.877314in}{1.945346in}}{\pgfqpoint{3.883138in}{1.939522in}}%
\pgfpathcurveto{\pgfqpoint{3.888962in}{1.933698in}}{\pgfqpoint{3.896862in}{1.930426in}}{\pgfqpoint{3.905098in}{1.930426in}}%
\pgfpathclose%
\pgfusepath{stroke,fill}%
\end{pgfscope}%
\begin{pgfscope}%
\pgfpathrectangle{\pgfqpoint{3.793912in}{0.557870in}}{\pgfqpoint{2.446088in}{1.484734in}}%
\pgfusepath{clip}%
\pgfsetbuttcap%
\pgfsetroundjoin%
\definecolor{currentfill}{rgb}{0.298039,0.447059,0.690196}%
\pgfsetfillcolor{currentfill}%
\pgfsetlinewidth{1.003750pt}%
\definecolor{currentstroke}{rgb}{0.298039,0.447059,0.690196}%
\pgfsetstrokecolor{currentstroke}%
\pgfsetdash{}{0pt}%
\pgfpathmoveto{\pgfqpoint{3.905098in}{1.930426in}}%
\pgfpathcurveto{\pgfqpoint{3.913334in}{1.930426in}}{\pgfqpoint{3.921234in}{1.933698in}}{\pgfqpoint{3.927058in}{1.939522in}}%
\pgfpathcurveto{\pgfqpoint{3.932882in}{1.945346in}}{\pgfqpoint{3.936155in}{1.953246in}}{\pgfqpoint{3.936155in}{1.961482in}}%
\pgfpathcurveto{\pgfqpoint{3.936155in}{1.969719in}}{\pgfqpoint{3.932882in}{1.977619in}}{\pgfqpoint{3.927058in}{1.983443in}}%
\pgfpathcurveto{\pgfqpoint{3.921234in}{1.989267in}}{\pgfqpoint{3.913334in}{1.992539in}}{\pgfqpoint{3.905098in}{1.992539in}}%
\pgfpathcurveto{\pgfqpoint{3.896862in}{1.992539in}}{\pgfqpoint{3.888962in}{1.989267in}}{\pgfqpoint{3.883138in}{1.983443in}}%
\pgfpathcurveto{\pgfqpoint{3.877314in}{1.977619in}}{\pgfqpoint{3.874042in}{1.969719in}}{\pgfqpoint{3.874042in}{1.961482in}}%
\pgfpathcurveto{\pgfqpoint{3.874042in}{1.953246in}}{\pgfqpoint{3.877314in}{1.945346in}}{\pgfqpoint{3.883138in}{1.939522in}}%
\pgfpathcurveto{\pgfqpoint{3.888962in}{1.933698in}}{\pgfqpoint{3.896862in}{1.930426in}}{\pgfqpoint{3.905098in}{1.930426in}}%
\pgfpathclose%
\pgfusepath{stroke,fill}%
\end{pgfscope}%
\begin{pgfscope}%
\pgfpathrectangle{\pgfqpoint{3.793912in}{0.557870in}}{\pgfqpoint{2.446088in}{1.484734in}}%
\pgfusepath{clip}%
\pgfsetbuttcap%
\pgfsetroundjoin%
\definecolor{currentfill}{rgb}{0.298039,0.447059,0.690196}%
\pgfsetfillcolor{currentfill}%
\pgfsetlinewidth{1.003750pt}%
\definecolor{currentstroke}{rgb}{0.298039,0.447059,0.690196}%
\pgfsetstrokecolor{currentstroke}%
\pgfsetdash{}{0pt}%
\pgfpathmoveto{\pgfqpoint{3.905098in}{1.930426in}}%
\pgfpathcurveto{\pgfqpoint{3.913334in}{1.930426in}}{\pgfqpoint{3.921234in}{1.933698in}}{\pgfqpoint{3.927058in}{1.939522in}}%
\pgfpathcurveto{\pgfqpoint{3.932882in}{1.945346in}}{\pgfqpoint{3.936155in}{1.953246in}}{\pgfqpoint{3.936155in}{1.961482in}}%
\pgfpathcurveto{\pgfqpoint{3.936155in}{1.969719in}}{\pgfqpoint{3.932882in}{1.977619in}}{\pgfqpoint{3.927058in}{1.983443in}}%
\pgfpathcurveto{\pgfqpoint{3.921234in}{1.989267in}}{\pgfqpoint{3.913334in}{1.992539in}}{\pgfqpoint{3.905098in}{1.992539in}}%
\pgfpathcurveto{\pgfqpoint{3.896862in}{1.992539in}}{\pgfqpoint{3.888962in}{1.989267in}}{\pgfqpoint{3.883138in}{1.983443in}}%
\pgfpathcurveto{\pgfqpoint{3.877314in}{1.977619in}}{\pgfqpoint{3.874042in}{1.969719in}}{\pgfqpoint{3.874042in}{1.961482in}}%
\pgfpathcurveto{\pgfqpoint{3.874042in}{1.953246in}}{\pgfqpoint{3.877314in}{1.945346in}}{\pgfqpoint{3.883138in}{1.939522in}}%
\pgfpathcurveto{\pgfqpoint{3.888962in}{1.933698in}}{\pgfqpoint{3.896862in}{1.930426in}}{\pgfqpoint{3.905098in}{1.930426in}}%
\pgfpathclose%
\pgfusepath{stroke,fill}%
\end{pgfscope}%
\begin{pgfscope}%
\pgfpathrectangle{\pgfqpoint{3.793912in}{0.557870in}}{\pgfqpoint{2.446088in}{1.484734in}}%
\pgfusepath{clip}%
\pgfsetbuttcap%
\pgfsetroundjoin%
\definecolor{currentfill}{rgb}{0.298039,0.447059,0.690196}%
\pgfsetfillcolor{currentfill}%
\pgfsetlinewidth{1.003750pt}%
\definecolor{currentstroke}{rgb}{0.298039,0.447059,0.690196}%
\pgfsetstrokecolor{currentstroke}%
\pgfsetdash{}{0pt}%
\pgfpathmoveto{\pgfqpoint{3.905098in}{1.930426in}}%
\pgfpathcurveto{\pgfqpoint{3.913334in}{1.930426in}}{\pgfqpoint{3.921234in}{1.933698in}}{\pgfqpoint{3.927058in}{1.939522in}}%
\pgfpathcurveto{\pgfqpoint{3.932882in}{1.945346in}}{\pgfqpoint{3.936155in}{1.953246in}}{\pgfqpoint{3.936155in}{1.961482in}}%
\pgfpathcurveto{\pgfqpoint{3.936155in}{1.969719in}}{\pgfqpoint{3.932882in}{1.977619in}}{\pgfqpoint{3.927058in}{1.983443in}}%
\pgfpathcurveto{\pgfqpoint{3.921234in}{1.989267in}}{\pgfqpoint{3.913334in}{1.992539in}}{\pgfqpoint{3.905098in}{1.992539in}}%
\pgfpathcurveto{\pgfqpoint{3.896862in}{1.992539in}}{\pgfqpoint{3.888962in}{1.989267in}}{\pgfqpoint{3.883138in}{1.983443in}}%
\pgfpathcurveto{\pgfqpoint{3.877314in}{1.977619in}}{\pgfqpoint{3.874042in}{1.969719in}}{\pgfqpoint{3.874042in}{1.961482in}}%
\pgfpathcurveto{\pgfqpoint{3.874042in}{1.953246in}}{\pgfqpoint{3.877314in}{1.945346in}}{\pgfqpoint{3.883138in}{1.939522in}}%
\pgfpathcurveto{\pgfqpoint{3.888962in}{1.933698in}}{\pgfqpoint{3.896862in}{1.930426in}}{\pgfqpoint{3.905098in}{1.930426in}}%
\pgfpathclose%
\pgfusepath{stroke,fill}%
\end{pgfscope}%
\begin{pgfscope}%
\pgfpathrectangle{\pgfqpoint{3.793912in}{0.557870in}}{\pgfqpoint{2.446088in}{1.484734in}}%
\pgfusepath{clip}%
\pgfsetbuttcap%
\pgfsetroundjoin%
\definecolor{currentfill}{rgb}{0.298039,0.447059,0.690196}%
\pgfsetfillcolor{currentfill}%
\pgfsetlinewidth{1.003750pt}%
\definecolor{currentstroke}{rgb}{0.298039,0.447059,0.690196}%
\pgfsetstrokecolor{currentstroke}%
\pgfsetdash{}{0pt}%
\pgfpathmoveto{\pgfqpoint{4.877974in}{0.594302in}}%
\pgfpathcurveto{\pgfqpoint{4.886210in}{0.594302in}}{\pgfqpoint{4.894110in}{0.597574in}}{\pgfqpoint{4.899934in}{0.603398in}}%
\pgfpathcurveto{\pgfqpoint{4.905758in}{0.609222in}}{\pgfqpoint{4.909030in}{0.617122in}}{\pgfqpoint{4.909030in}{0.625358in}}%
\pgfpathcurveto{\pgfqpoint{4.909030in}{0.633594in}}{\pgfqpoint{4.905758in}{0.641495in}}{\pgfqpoint{4.899934in}{0.647318in}}%
\pgfpathcurveto{\pgfqpoint{4.894110in}{0.653142in}}{\pgfqpoint{4.886210in}{0.656415in}}{\pgfqpoint{4.877974in}{0.656415in}}%
\pgfpathcurveto{\pgfqpoint{4.869738in}{0.656415in}}{\pgfqpoint{4.861838in}{0.653142in}}{\pgfqpoint{4.856014in}{0.647318in}}%
\pgfpathcurveto{\pgfqpoint{4.850190in}{0.641495in}}{\pgfqpoint{4.846917in}{0.633594in}}{\pgfqpoint{4.846917in}{0.625358in}}%
\pgfpathcurveto{\pgfqpoint{4.846917in}{0.617122in}}{\pgfqpoint{4.850190in}{0.609222in}}{\pgfqpoint{4.856014in}{0.603398in}}%
\pgfpathcurveto{\pgfqpoint{4.861838in}{0.597574in}}{\pgfqpoint{4.869738in}{0.594302in}}{\pgfqpoint{4.877974in}{0.594302in}}%
\pgfpathclose%
\pgfusepath{stroke,fill}%
\end{pgfscope}%
\begin{pgfscope}%
\pgfpathrectangle{\pgfqpoint{3.793912in}{0.557870in}}{\pgfqpoint{2.446088in}{1.484734in}}%
\pgfusepath{clip}%
\pgfsetbuttcap%
\pgfsetroundjoin%
\definecolor{currentfill}{rgb}{0.298039,0.447059,0.690196}%
\pgfsetfillcolor{currentfill}%
\pgfsetlinewidth{1.003750pt}%
\definecolor{currentstroke}{rgb}{0.298039,0.447059,0.690196}%
\pgfsetstrokecolor{currentstroke}%
\pgfsetdash{}{0pt}%
\pgfpathmoveto{\pgfqpoint{3.905098in}{1.930426in}}%
\pgfpathcurveto{\pgfqpoint{3.913334in}{1.930426in}}{\pgfqpoint{3.921234in}{1.933698in}}{\pgfqpoint{3.927058in}{1.939522in}}%
\pgfpathcurveto{\pgfqpoint{3.932882in}{1.945346in}}{\pgfqpoint{3.936155in}{1.953246in}}{\pgfqpoint{3.936155in}{1.961482in}}%
\pgfpathcurveto{\pgfqpoint{3.936155in}{1.969719in}}{\pgfqpoint{3.932882in}{1.977619in}}{\pgfqpoint{3.927058in}{1.983443in}}%
\pgfpathcurveto{\pgfqpoint{3.921234in}{1.989267in}}{\pgfqpoint{3.913334in}{1.992539in}}{\pgfqpoint{3.905098in}{1.992539in}}%
\pgfpathcurveto{\pgfqpoint{3.896862in}{1.992539in}}{\pgfqpoint{3.888962in}{1.989267in}}{\pgfqpoint{3.883138in}{1.983443in}}%
\pgfpathcurveto{\pgfqpoint{3.877314in}{1.977619in}}{\pgfqpoint{3.874042in}{1.969719in}}{\pgfqpoint{3.874042in}{1.961482in}}%
\pgfpathcurveto{\pgfqpoint{3.874042in}{1.953246in}}{\pgfqpoint{3.877314in}{1.945346in}}{\pgfqpoint{3.883138in}{1.939522in}}%
\pgfpathcurveto{\pgfqpoint{3.888962in}{1.933698in}}{\pgfqpoint{3.896862in}{1.930426in}}{\pgfqpoint{3.905098in}{1.930426in}}%
\pgfpathclose%
\pgfusepath{stroke,fill}%
\end{pgfscope}%
\begin{pgfscope}%
\pgfpathrectangle{\pgfqpoint{3.793912in}{0.557870in}}{\pgfqpoint{2.446088in}{1.484734in}}%
\pgfusepath{clip}%
\pgfsetbuttcap%
\pgfsetroundjoin%
\definecolor{currentfill}{rgb}{0.298039,0.447059,0.690196}%
\pgfsetfillcolor{currentfill}%
\pgfsetlinewidth{1.003750pt}%
\definecolor{currentstroke}{rgb}{0.298039,0.447059,0.690196}%
\pgfsetstrokecolor{currentstroke}%
\pgfsetdash{}{0pt}%
\pgfpathmoveto{\pgfqpoint{3.905098in}{1.930426in}}%
\pgfpathcurveto{\pgfqpoint{3.913334in}{1.930426in}}{\pgfqpoint{3.921234in}{1.933698in}}{\pgfqpoint{3.927058in}{1.939522in}}%
\pgfpathcurveto{\pgfqpoint{3.932882in}{1.945346in}}{\pgfqpoint{3.936155in}{1.953246in}}{\pgfqpoint{3.936155in}{1.961482in}}%
\pgfpathcurveto{\pgfqpoint{3.936155in}{1.969719in}}{\pgfqpoint{3.932882in}{1.977619in}}{\pgfqpoint{3.927058in}{1.983443in}}%
\pgfpathcurveto{\pgfqpoint{3.921234in}{1.989267in}}{\pgfqpoint{3.913334in}{1.992539in}}{\pgfqpoint{3.905098in}{1.992539in}}%
\pgfpathcurveto{\pgfqpoint{3.896862in}{1.992539in}}{\pgfqpoint{3.888962in}{1.989267in}}{\pgfqpoint{3.883138in}{1.983443in}}%
\pgfpathcurveto{\pgfqpoint{3.877314in}{1.977619in}}{\pgfqpoint{3.874042in}{1.969719in}}{\pgfqpoint{3.874042in}{1.961482in}}%
\pgfpathcurveto{\pgfqpoint{3.874042in}{1.953246in}}{\pgfqpoint{3.877314in}{1.945346in}}{\pgfqpoint{3.883138in}{1.939522in}}%
\pgfpathcurveto{\pgfqpoint{3.888962in}{1.933698in}}{\pgfqpoint{3.896862in}{1.930426in}}{\pgfqpoint{3.905098in}{1.930426in}}%
\pgfpathclose%
\pgfusepath{stroke,fill}%
\end{pgfscope}%
\begin{pgfscope}%
\pgfpathrectangle{\pgfqpoint{3.793912in}{0.557870in}}{\pgfqpoint{2.446088in}{1.484734in}}%
\pgfusepath{clip}%
\pgfsetbuttcap%
\pgfsetroundjoin%
\definecolor{currentfill}{rgb}{0.298039,0.447059,0.690196}%
\pgfsetfillcolor{currentfill}%
\pgfsetlinewidth{1.003750pt}%
\definecolor{currentstroke}{rgb}{0.298039,0.447059,0.690196}%
\pgfsetstrokecolor{currentstroke}%
\pgfsetdash{}{0pt}%
\pgfpathmoveto{\pgfqpoint{3.905098in}{1.930426in}}%
\pgfpathcurveto{\pgfqpoint{3.913334in}{1.930426in}}{\pgfqpoint{3.921234in}{1.933698in}}{\pgfqpoint{3.927058in}{1.939522in}}%
\pgfpathcurveto{\pgfqpoint{3.932882in}{1.945346in}}{\pgfqpoint{3.936155in}{1.953246in}}{\pgfqpoint{3.936155in}{1.961482in}}%
\pgfpathcurveto{\pgfqpoint{3.936155in}{1.969719in}}{\pgfqpoint{3.932882in}{1.977619in}}{\pgfqpoint{3.927058in}{1.983443in}}%
\pgfpathcurveto{\pgfqpoint{3.921234in}{1.989267in}}{\pgfqpoint{3.913334in}{1.992539in}}{\pgfqpoint{3.905098in}{1.992539in}}%
\pgfpathcurveto{\pgfqpoint{3.896862in}{1.992539in}}{\pgfqpoint{3.888962in}{1.989267in}}{\pgfqpoint{3.883138in}{1.983443in}}%
\pgfpathcurveto{\pgfqpoint{3.877314in}{1.977619in}}{\pgfqpoint{3.874042in}{1.969719in}}{\pgfqpoint{3.874042in}{1.961482in}}%
\pgfpathcurveto{\pgfqpoint{3.874042in}{1.953246in}}{\pgfqpoint{3.877314in}{1.945346in}}{\pgfqpoint{3.883138in}{1.939522in}}%
\pgfpathcurveto{\pgfqpoint{3.888962in}{1.933698in}}{\pgfqpoint{3.896862in}{1.930426in}}{\pgfqpoint{3.905098in}{1.930426in}}%
\pgfpathclose%
\pgfusepath{stroke,fill}%
\end{pgfscope}%
\begin{pgfscope}%
\pgfpathrectangle{\pgfqpoint{3.793912in}{0.557870in}}{\pgfqpoint{2.446088in}{1.484734in}}%
\pgfusepath{clip}%
\pgfsetbuttcap%
\pgfsetroundjoin%
\definecolor{currentfill}{rgb}{0.298039,0.447059,0.690196}%
\pgfsetfillcolor{currentfill}%
\pgfsetlinewidth{1.003750pt}%
\definecolor{currentstroke}{rgb}{0.298039,0.447059,0.690196}%
\pgfsetstrokecolor{currentstroke}%
\pgfsetdash{}{0pt}%
\pgfpathmoveto{\pgfqpoint{5.457067in}{0.594302in}}%
\pgfpathcurveto{\pgfqpoint{5.465303in}{0.594302in}}{\pgfqpoint{5.473203in}{0.597574in}}{\pgfqpoint{5.479027in}{0.603398in}}%
\pgfpathcurveto{\pgfqpoint{5.484851in}{0.609222in}}{\pgfqpoint{5.488123in}{0.617122in}}{\pgfqpoint{5.488123in}{0.625358in}}%
\pgfpathcurveto{\pgfqpoint{5.488123in}{0.633594in}}{\pgfqpoint{5.484851in}{0.641495in}}{\pgfqpoint{5.479027in}{0.647318in}}%
\pgfpathcurveto{\pgfqpoint{5.473203in}{0.653142in}}{\pgfqpoint{5.465303in}{0.656415in}}{\pgfqpoint{5.457067in}{0.656415in}}%
\pgfpathcurveto{\pgfqpoint{5.448830in}{0.656415in}}{\pgfqpoint{5.440930in}{0.653142in}}{\pgfqpoint{5.435106in}{0.647318in}}%
\pgfpathcurveto{\pgfqpoint{5.429282in}{0.641495in}}{\pgfqpoint{5.426010in}{0.633594in}}{\pgfqpoint{5.426010in}{0.625358in}}%
\pgfpathcurveto{\pgfqpoint{5.426010in}{0.617122in}}{\pgfqpoint{5.429282in}{0.609222in}}{\pgfqpoint{5.435106in}{0.603398in}}%
\pgfpathcurveto{\pgfqpoint{5.440930in}{0.597574in}}{\pgfqpoint{5.448830in}{0.594302in}}{\pgfqpoint{5.457067in}{0.594302in}}%
\pgfpathclose%
\pgfusepath{stroke,fill}%
\end{pgfscope}%
\begin{pgfscope}%
\pgfpathrectangle{\pgfqpoint{3.793912in}{0.557870in}}{\pgfqpoint{2.446088in}{1.484734in}}%
\pgfusepath{clip}%
\pgfsetbuttcap%
\pgfsetroundjoin%
\definecolor{currentfill}{rgb}{0.298039,0.447059,0.690196}%
\pgfsetfillcolor{currentfill}%
\pgfsetlinewidth{1.003750pt}%
\definecolor{currentstroke}{rgb}{0.298039,0.447059,0.690196}%
\pgfsetstrokecolor{currentstroke}%
\pgfsetdash{}{0pt}%
\pgfpathmoveto{\pgfqpoint{3.905098in}{1.739551in}}%
\pgfpathcurveto{\pgfqpoint{3.913334in}{1.739551in}}{\pgfqpoint{3.921234in}{1.742823in}}{\pgfqpoint{3.927058in}{1.748647in}}%
\pgfpathcurveto{\pgfqpoint{3.932882in}{1.754471in}}{\pgfqpoint{3.936155in}{1.762371in}}{\pgfqpoint{3.936155in}{1.770607in}}%
\pgfpathcurveto{\pgfqpoint{3.936155in}{1.778844in}}{\pgfqpoint{3.932882in}{1.786744in}}{\pgfqpoint{3.927058in}{1.792568in}}%
\pgfpathcurveto{\pgfqpoint{3.921234in}{1.798392in}}{\pgfqpoint{3.913334in}{1.801664in}}{\pgfqpoint{3.905098in}{1.801664in}}%
\pgfpathcurveto{\pgfqpoint{3.896862in}{1.801664in}}{\pgfqpoint{3.888962in}{1.798392in}}{\pgfqpoint{3.883138in}{1.792568in}}%
\pgfpathcurveto{\pgfqpoint{3.877314in}{1.786744in}}{\pgfqpoint{3.874042in}{1.778844in}}{\pgfqpoint{3.874042in}{1.770607in}}%
\pgfpathcurveto{\pgfqpoint{3.874042in}{1.762371in}}{\pgfqpoint{3.877314in}{1.754471in}}{\pgfqpoint{3.883138in}{1.748647in}}%
\pgfpathcurveto{\pgfqpoint{3.888962in}{1.742823in}}{\pgfqpoint{3.896862in}{1.739551in}}{\pgfqpoint{3.905098in}{1.739551in}}%
\pgfpathclose%
\pgfusepath{stroke,fill}%
\end{pgfscope}%
\begin{pgfscope}%
\pgfpathrectangle{\pgfqpoint{3.793912in}{0.557870in}}{\pgfqpoint{2.446088in}{1.484734in}}%
\pgfusepath{clip}%
\pgfsetbuttcap%
\pgfsetroundjoin%
\definecolor{currentfill}{rgb}{0.298039,0.447059,0.690196}%
\pgfsetfillcolor{currentfill}%
\pgfsetlinewidth{1.003750pt}%
\definecolor{currentstroke}{rgb}{0.298039,0.447059,0.690196}%
\pgfsetstrokecolor{currentstroke}%
\pgfsetdash{}{0pt}%
\pgfpathmoveto{\pgfqpoint{5.341248in}{0.594302in}}%
\pgfpathcurveto{\pgfqpoint{5.349484in}{0.594302in}}{\pgfqpoint{5.357384in}{0.597574in}}{\pgfqpoint{5.363208in}{0.603398in}}%
\pgfpathcurveto{\pgfqpoint{5.369032in}{0.609222in}}{\pgfqpoint{5.372305in}{0.617122in}}{\pgfqpoint{5.372305in}{0.625358in}}%
\pgfpathcurveto{\pgfqpoint{5.372305in}{0.633594in}}{\pgfqpoint{5.369032in}{0.641495in}}{\pgfqpoint{5.363208in}{0.647318in}}%
\pgfpathcurveto{\pgfqpoint{5.357384in}{0.653142in}}{\pgfqpoint{5.349484in}{0.656415in}}{\pgfqpoint{5.341248in}{0.656415in}}%
\pgfpathcurveto{\pgfqpoint{5.333012in}{0.656415in}}{\pgfqpoint{5.325112in}{0.653142in}}{\pgfqpoint{5.319288in}{0.647318in}}%
\pgfpathcurveto{\pgfqpoint{5.313464in}{0.641495in}}{\pgfqpoint{5.310192in}{0.633594in}}{\pgfqpoint{5.310192in}{0.625358in}}%
\pgfpathcurveto{\pgfqpoint{5.310192in}{0.617122in}}{\pgfqpoint{5.313464in}{0.609222in}}{\pgfqpoint{5.319288in}{0.603398in}}%
\pgfpathcurveto{\pgfqpoint{5.325112in}{0.597574in}}{\pgfqpoint{5.333012in}{0.594302in}}{\pgfqpoint{5.341248in}{0.594302in}}%
\pgfpathclose%
\pgfusepath{stroke,fill}%
\end{pgfscope}%
\begin{pgfscope}%
\pgfpathrectangle{\pgfqpoint{3.793912in}{0.557870in}}{\pgfqpoint{2.446088in}{1.484734in}}%
\pgfusepath{clip}%
\pgfsetbuttcap%
\pgfsetroundjoin%
\definecolor{currentfill}{rgb}{0.298039,0.447059,0.690196}%
\pgfsetfillcolor{currentfill}%
\pgfsetlinewidth{1.003750pt}%
\definecolor{currentstroke}{rgb}{0.298039,0.447059,0.690196}%
\pgfsetstrokecolor{currentstroke}%
\pgfsetdash{}{0pt}%
\pgfpathmoveto{\pgfqpoint{3.905098in}{1.766819in}}%
\pgfpathcurveto{\pgfqpoint{3.913334in}{1.766819in}}{\pgfqpoint{3.921234in}{1.770091in}}{\pgfqpoint{3.927058in}{1.775915in}}%
\pgfpathcurveto{\pgfqpoint{3.932882in}{1.781739in}}{\pgfqpoint{3.936155in}{1.789639in}}{\pgfqpoint{3.936155in}{1.797875in}}%
\pgfpathcurveto{\pgfqpoint{3.936155in}{1.806112in}}{\pgfqpoint{3.932882in}{1.814012in}}{\pgfqpoint{3.927058in}{1.819836in}}%
\pgfpathcurveto{\pgfqpoint{3.921234in}{1.825659in}}{\pgfqpoint{3.913334in}{1.828932in}}{\pgfqpoint{3.905098in}{1.828932in}}%
\pgfpathcurveto{\pgfqpoint{3.896862in}{1.828932in}}{\pgfqpoint{3.888962in}{1.825659in}}{\pgfqpoint{3.883138in}{1.819836in}}%
\pgfpathcurveto{\pgfqpoint{3.877314in}{1.814012in}}{\pgfqpoint{3.874042in}{1.806112in}}{\pgfqpoint{3.874042in}{1.797875in}}%
\pgfpathcurveto{\pgfqpoint{3.874042in}{1.789639in}}{\pgfqpoint{3.877314in}{1.781739in}}{\pgfqpoint{3.883138in}{1.775915in}}%
\pgfpathcurveto{\pgfqpoint{3.888962in}{1.770091in}}{\pgfqpoint{3.896862in}{1.766819in}}{\pgfqpoint{3.905098in}{1.766819in}}%
\pgfpathclose%
\pgfusepath{stroke,fill}%
\end{pgfscope}%
\begin{pgfscope}%
\pgfpathrectangle{\pgfqpoint{3.793912in}{0.557870in}}{\pgfqpoint{2.446088in}{1.484734in}}%
\pgfusepath{clip}%
\pgfsetbuttcap%
\pgfsetroundjoin%
\definecolor{currentfill}{rgb}{0.298039,0.447059,0.690196}%
\pgfsetfillcolor{currentfill}%
\pgfsetlinewidth{1.003750pt}%
\definecolor{currentstroke}{rgb}{0.298039,0.447059,0.690196}%
\pgfsetstrokecolor{currentstroke}%
\pgfsetdash{}{0pt}%
\pgfpathmoveto{\pgfqpoint{3.905098in}{1.371435in}}%
\pgfpathcurveto{\pgfqpoint{3.913334in}{1.371435in}}{\pgfqpoint{3.921234in}{1.374707in}}{\pgfqpoint{3.927058in}{1.380531in}}%
\pgfpathcurveto{\pgfqpoint{3.932882in}{1.386355in}}{\pgfqpoint{3.936155in}{1.394255in}}{\pgfqpoint{3.936155in}{1.402492in}}%
\pgfpathcurveto{\pgfqpoint{3.936155in}{1.410728in}}{\pgfqpoint{3.932882in}{1.418628in}}{\pgfqpoint{3.927058in}{1.424452in}}%
\pgfpathcurveto{\pgfqpoint{3.921234in}{1.430276in}}{\pgfqpoint{3.913334in}{1.433548in}}{\pgfqpoint{3.905098in}{1.433548in}}%
\pgfpathcurveto{\pgfqpoint{3.896862in}{1.433548in}}{\pgfqpoint{3.888962in}{1.430276in}}{\pgfqpoint{3.883138in}{1.424452in}}%
\pgfpathcurveto{\pgfqpoint{3.877314in}{1.418628in}}{\pgfqpoint{3.874042in}{1.410728in}}{\pgfqpoint{3.874042in}{1.402492in}}%
\pgfpathcurveto{\pgfqpoint{3.874042in}{1.394255in}}{\pgfqpoint{3.877314in}{1.386355in}}{\pgfqpoint{3.883138in}{1.380531in}}%
\pgfpathcurveto{\pgfqpoint{3.888962in}{1.374707in}}{\pgfqpoint{3.896862in}{1.371435in}}{\pgfqpoint{3.905098in}{1.371435in}}%
\pgfpathclose%
\pgfusepath{stroke,fill}%
\end{pgfscope}%
\begin{pgfscope}%
\pgfpathrectangle{\pgfqpoint{3.793912in}{0.557870in}}{\pgfqpoint{2.446088in}{1.484734in}}%
\pgfusepath{clip}%
\pgfsetbuttcap%
\pgfsetroundjoin%
\definecolor{currentfill}{rgb}{0.298039,0.447059,0.690196}%
\pgfsetfillcolor{currentfill}%
\pgfsetlinewidth{1.003750pt}%
\definecolor{currentstroke}{rgb}{0.298039,0.447059,0.690196}%
\pgfsetstrokecolor{currentstroke}%
\pgfsetdash{}{0pt}%
\pgfpathmoveto{\pgfqpoint{3.905098in}{1.371435in}}%
\pgfpathcurveto{\pgfqpoint{3.913334in}{1.371435in}}{\pgfqpoint{3.921234in}{1.374707in}}{\pgfqpoint{3.927058in}{1.380531in}}%
\pgfpathcurveto{\pgfqpoint{3.932882in}{1.386355in}}{\pgfqpoint{3.936155in}{1.394255in}}{\pgfqpoint{3.936155in}{1.402492in}}%
\pgfpathcurveto{\pgfqpoint{3.936155in}{1.410728in}}{\pgfqpoint{3.932882in}{1.418628in}}{\pgfqpoint{3.927058in}{1.424452in}}%
\pgfpathcurveto{\pgfqpoint{3.921234in}{1.430276in}}{\pgfqpoint{3.913334in}{1.433548in}}{\pgfqpoint{3.905098in}{1.433548in}}%
\pgfpathcurveto{\pgfqpoint{3.896862in}{1.433548in}}{\pgfqpoint{3.888962in}{1.430276in}}{\pgfqpoint{3.883138in}{1.424452in}}%
\pgfpathcurveto{\pgfqpoint{3.877314in}{1.418628in}}{\pgfqpoint{3.874042in}{1.410728in}}{\pgfqpoint{3.874042in}{1.402492in}}%
\pgfpathcurveto{\pgfqpoint{3.874042in}{1.394255in}}{\pgfqpoint{3.877314in}{1.386355in}}{\pgfqpoint{3.883138in}{1.380531in}}%
\pgfpathcurveto{\pgfqpoint{3.888962in}{1.374707in}}{\pgfqpoint{3.896862in}{1.371435in}}{\pgfqpoint{3.905098in}{1.371435in}}%
\pgfpathclose%
\pgfusepath{stroke,fill}%
\end{pgfscope}%
\begin{pgfscope}%
\pgfpathrectangle{\pgfqpoint{3.793912in}{0.557870in}}{\pgfqpoint{2.446088in}{1.484734in}}%
\pgfusepath{clip}%
\pgfsetbuttcap%
\pgfsetroundjoin%
\definecolor{currentfill}{rgb}{0.298039,0.447059,0.690196}%
\pgfsetfillcolor{currentfill}%
\pgfsetlinewidth{1.003750pt}%
\definecolor{currentstroke}{rgb}{0.298039,0.447059,0.690196}%
\pgfsetstrokecolor{currentstroke}%
\pgfsetdash{}{0pt}%
\pgfpathmoveto{\pgfqpoint{3.905098in}{1.235096in}}%
\pgfpathcurveto{\pgfqpoint{3.913334in}{1.235096in}}{\pgfqpoint{3.921234in}{1.238368in}}{\pgfqpoint{3.927058in}{1.244192in}}%
\pgfpathcurveto{\pgfqpoint{3.932882in}{1.250016in}}{\pgfqpoint{3.936155in}{1.257916in}}{\pgfqpoint{3.936155in}{1.266152in}}%
\pgfpathcurveto{\pgfqpoint{3.936155in}{1.274389in}}{\pgfqpoint{3.932882in}{1.282289in}}{\pgfqpoint{3.927058in}{1.288113in}}%
\pgfpathcurveto{\pgfqpoint{3.921234in}{1.293937in}}{\pgfqpoint{3.913334in}{1.297209in}}{\pgfqpoint{3.905098in}{1.297209in}}%
\pgfpathcurveto{\pgfqpoint{3.896862in}{1.297209in}}{\pgfqpoint{3.888962in}{1.293937in}}{\pgfqpoint{3.883138in}{1.288113in}}%
\pgfpathcurveto{\pgfqpoint{3.877314in}{1.282289in}}{\pgfqpoint{3.874042in}{1.274389in}}{\pgfqpoint{3.874042in}{1.266152in}}%
\pgfpathcurveto{\pgfqpoint{3.874042in}{1.257916in}}{\pgfqpoint{3.877314in}{1.250016in}}{\pgfqpoint{3.883138in}{1.244192in}}%
\pgfpathcurveto{\pgfqpoint{3.888962in}{1.238368in}}{\pgfqpoint{3.896862in}{1.235096in}}{\pgfqpoint{3.905098in}{1.235096in}}%
\pgfpathclose%
\pgfusepath{stroke,fill}%
\end{pgfscope}%
\begin{pgfscope}%
\pgfpathrectangle{\pgfqpoint{3.793912in}{0.557870in}}{\pgfqpoint{2.446088in}{1.484734in}}%
\pgfusepath{clip}%
\pgfsetbuttcap%
\pgfsetroundjoin%
\definecolor{currentfill}{rgb}{0.298039,0.447059,0.690196}%
\pgfsetfillcolor{currentfill}%
\pgfsetlinewidth{1.003750pt}%
\definecolor{currentstroke}{rgb}{0.298039,0.447059,0.690196}%
\pgfsetstrokecolor{currentstroke}%
\pgfsetdash{}{0pt}%
\pgfpathmoveto{\pgfqpoint{3.905098in}{1.453239in}}%
\pgfpathcurveto{\pgfqpoint{3.913334in}{1.453239in}}{\pgfqpoint{3.921234in}{1.456511in}}{\pgfqpoint{3.927058in}{1.462335in}}%
\pgfpathcurveto{\pgfqpoint{3.932882in}{1.468159in}}{\pgfqpoint{3.936155in}{1.476059in}}{\pgfqpoint{3.936155in}{1.484295in}}%
\pgfpathcurveto{\pgfqpoint{3.936155in}{1.492531in}}{\pgfqpoint{3.932882in}{1.500431in}}{\pgfqpoint{3.927058in}{1.506255in}}%
\pgfpathcurveto{\pgfqpoint{3.921234in}{1.512079in}}{\pgfqpoint{3.913334in}{1.515352in}}{\pgfqpoint{3.905098in}{1.515352in}}%
\pgfpathcurveto{\pgfqpoint{3.896862in}{1.515352in}}{\pgfqpoint{3.888962in}{1.512079in}}{\pgfqpoint{3.883138in}{1.506255in}}%
\pgfpathcurveto{\pgfqpoint{3.877314in}{1.500431in}}{\pgfqpoint{3.874042in}{1.492531in}}{\pgfqpoint{3.874042in}{1.484295in}}%
\pgfpathcurveto{\pgfqpoint{3.874042in}{1.476059in}}{\pgfqpoint{3.877314in}{1.468159in}}{\pgfqpoint{3.883138in}{1.462335in}}%
\pgfpathcurveto{\pgfqpoint{3.888962in}{1.456511in}}{\pgfqpoint{3.896862in}{1.453239in}}{\pgfqpoint{3.905098in}{1.453239in}}%
\pgfpathclose%
\pgfusepath{stroke,fill}%
\end{pgfscope}%
\begin{pgfscope}%
\pgfpathrectangle{\pgfqpoint{3.793912in}{0.557870in}}{\pgfqpoint{2.446088in}{1.484734in}}%
\pgfusepath{clip}%
\pgfsetbuttcap%
\pgfsetroundjoin%
\definecolor{currentfill}{rgb}{0.298039,0.447059,0.690196}%
\pgfsetfillcolor{currentfill}%
\pgfsetlinewidth{1.003750pt}%
\definecolor{currentstroke}{rgb}{0.298039,0.447059,0.690196}%
\pgfsetstrokecolor{currentstroke}%
\pgfsetdash{}{0pt}%
\pgfpathmoveto{\pgfqpoint{3.905098in}{1.480506in}}%
\pgfpathcurveto{\pgfqpoint{3.913334in}{1.480506in}}{\pgfqpoint{3.921234in}{1.483779in}}{\pgfqpoint{3.927058in}{1.489603in}}%
\pgfpathcurveto{\pgfqpoint{3.932882in}{1.495427in}}{\pgfqpoint{3.936155in}{1.503327in}}{\pgfqpoint{3.936155in}{1.511563in}}%
\pgfpathcurveto{\pgfqpoint{3.936155in}{1.519799in}}{\pgfqpoint{3.932882in}{1.527699in}}{\pgfqpoint{3.927058in}{1.533523in}}%
\pgfpathcurveto{\pgfqpoint{3.921234in}{1.539347in}}{\pgfqpoint{3.913334in}{1.542619in}}{\pgfqpoint{3.905098in}{1.542619in}}%
\pgfpathcurveto{\pgfqpoint{3.896862in}{1.542619in}}{\pgfqpoint{3.888962in}{1.539347in}}{\pgfqpoint{3.883138in}{1.533523in}}%
\pgfpathcurveto{\pgfqpoint{3.877314in}{1.527699in}}{\pgfqpoint{3.874042in}{1.519799in}}{\pgfqpoint{3.874042in}{1.511563in}}%
\pgfpathcurveto{\pgfqpoint{3.874042in}{1.503327in}}{\pgfqpoint{3.877314in}{1.495427in}}{\pgfqpoint{3.883138in}{1.489603in}}%
\pgfpathcurveto{\pgfqpoint{3.888962in}{1.483779in}}{\pgfqpoint{3.896862in}{1.480506in}}{\pgfqpoint{3.905098in}{1.480506in}}%
\pgfpathclose%
\pgfusepath{stroke,fill}%
\end{pgfscope}%
\begin{pgfscope}%
\pgfpathrectangle{\pgfqpoint{3.793912in}{0.557870in}}{\pgfqpoint{2.446088in}{1.484734in}}%
\pgfusepath{clip}%
\pgfsetbuttcap%
\pgfsetroundjoin%
\definecolor{currentfill}{rgb}{0.298039,0.447059,0.690196}%
\pgfsetfillcolor{currentfill}%
\pgfsetlinewidth{1.003750pt}%
\definecolor{currentstroke}{rgb}{0.298039,0.447059,0.690196}%
\pgfsetstrokecolor{currentstroke}%
\pgfsetdash{}{0pt}%
\pgfpathmoveto{\pgfqpoint{3.905098in}{1.480506in}}%
\pgfpathcurveto{\pgfqpoint{3.913334in}{1.480506in}}{\pgfqpoint{3.921234in}{1.483779in}}{\pgfqpoint{3.927058in}{1.489603in}}%
\pgfpathcurveto{\pgfqpoint{3.932882in}{1.495427in}}{\pgfqpoint{3.936155in}{1.503327in}}{\pgfqpoint{3.936155in}{1.511563in}}%
\pgfpathcurveto{\pgfqpoint{3.936155in}{1.519799in}}{\pgfqpoint{3.932882in}{1.527699in}}{\pgfqpoint{3.927058in}{1.533523in}}%
\pgfpathcurveto{\pgfqpoint{3.921234in}{1.539347in}}{\pgfqpoint{3.913334in}{1.542619in}}{\pgfqpoint{3.905098in}{1.542619in}}%
\pgfpathcurveto{\pgfqpoint{3.896862in}{1.542619in}}{\pgfqpoint{3.888962in}{1.539347in}}{\pgfqpoint{3.883138in}{1.533523in}}%
\pgfpathcurveto{\pgfqpoint{3.877314in}{1.527699in}}{\pgfqpoint{3.874042in}{1.519799in}}{\pgfqpoint{3.874042in}{1.511563in}}%
\pgfpathcurveto{\pgfqpoint{3.874042in}{1.503327in}}{\pgfqpoint{3.877314in}{1.495427in}}{\pgfqpoint{3.883138in}{1.489603in}}%
\pgfpathcurveto{\pgfqpoint{3.888962in}{1.483779in}}{\pgfqpoint{3.896862in}{1.480506in}}{\pgfqpoint{3.905098in}{1.480506in}}%
\pgfpathclose%
\pgfusepath{stroke,fill}%
\end{pgfscope}%
\begin{pgfscope}%
\pgfpathrectangle{\pgfqpoint{3.793912in}{0.557870in}}{\pgfqpoint{2.446088in}{1.484734in}}%
\pgfusepath{clip}%
\pgfsetbuttcap%
\pgfsetroundjoin%
\definecolor{currentfill}{rgb}{0.298039,0.447059,0.690196}%
\pgfsetfillcolor{currentfill}%
\pgfsetlinewidth{1.003750pt}%
\definecolor{currentstroke}{rgb}{0.298039,0.447059,0.690196}%
\pgfsetstrokecolor{currentstroke}%
\pgfsetdash{}{0pt}%
\pgfpathmoveto{\pgfqpoint{5.526558in}{0.594302in}}%
\pgfpathcurveto{\pgfqpoint{5.534794in}{0.594302in}}{\pgfqpoint{5.542694in}{0.597574in}}{\pgfqpoint{5.548518in}{0.603398in}}%
\pgfpathcurveto{\pgfqpoint{5.554342in}{0.609222in}}{\pgfqpoint{5.557614in}{0.617122in}}{\pgfqpoint{5.557614in}{0.625358in}}%
\pgfpathcurveto{\pgfqpoint{5.557614in}{0.633594in}}{\pgfqpoint{5.554342in}{0.641495in}}{\pgfqpoint{5.548518in}{0.647318in}}%
\pgfpathcurveto{\pgfqpoint{5.542694in}{0.653142in}}{\pgfqpoint{5.534794in}{0.656415in}}{\pgfqpoint{5.526558in}{0.656415in}}%
\pgfpathcurveto{\pgfqpoint{5.518321in}{0.656415in}}{\pgfqpoint{5.510421in}{0.653142in}}{\pgfqpoint{5.504597in}{0.647318in}}%
\pgfpathcurveto{\pgfqpoint{5.498774in}{0.641495in}}{\pgfqpoint{5.495501in}{0.633594in}}{\pgfqpoint{5.495501in}{0.625358in}}%
\pgfpathcurveto{\pgfqpoint{5.495501in}{0.617122in}}{\pgfqpoint{5.498774in}{0.609222in}}{\pgfqpoint{5.504597in}{0.603398in}}%
\pgfpathcurveto{\pgfqpoint{5.510421in}{0.597574in}}{\pgfqpoint{5.518321in}{0.594302in}}{\pgfqpoint{5.526558in}{0.594302in}}%
\pgfpathclose%
\pgfusepath{stroke,fill}%
\end{pgfscope}%
\begin{pgfscope}%
\pgfpathrectangle{\pgfqpoint{3.793912in}{0.557870in}}{\pgfqpoint{2.446088in}{1.484734in}}%
\pgfusepath{clip}%
\pgfsetbuttcap%
\pgfsetroundjoin%
\definecolor{currentfill}{rgb}{0.298039,0.447059,0.690196}%
\pgfsetfillcolor{currentfill}%
\pgfsetlinewidth{1.003750pt}%
\definecolor{currentstroke}{rgb}{0.298039,0.447059,0.690196}%
\pgfsetstrokecolor{currentstroke}%
\pgfsetdash{}{0pt}%
\pgfpathmoveto{\pgfqpoint{3.905098in}{1.085123in}}%
\pgfpathcurveto{\pgfqpoint{3.913334in}{1.085123in}}{\pgfqpoint{3.921234in}{1.088395in}}{\pgfqpoint{3.927058in}{1.094219in}}%
\pgfpathcurveto{\pgfqpoint{3.932882in}{1.100043in}}{\pgfqpoint{3.936155in}{1.107943in}}{\pgfqpoint{3.936155in}{1.116179in}}%
\pgfpathcurveto{\pgfqpoint{3.936155in}{1.124416in}}{\pgfqpoint{3.932882in}{1.132316in}}{\pgfqpoint{3.927058in}{1.138140in}}%
\pgfpathcurveto{\pgfqpoint{3.921234in}{1.143963in}}{\pgfqpoint{3.913334in}{1.147236in}}{\pgfqpoint{3.905098in}{1.147236in}}%
\pgfpathcurveto{\pgfqpoint{3.896862in}{1.147236in}}{\pgfqpoint{3.888962in}{1.143963in}}{\pgfqpoint{3.883138in}{1.138140in}}%
\pgfpathcurveto{\pgfqpoint{3.877314in}{1.132316in}}{\pgfqpoint{3.874042in}{1.124416in}}{\pgfqpoint{3.874042in}{1.116179in}}%
\pgfpathcurveto{\pgfqpoint{3.874042in}{1.107943in}}{\pgfqpoint{3.877314in}{1.100043in}}{\pgfqpoint{3.883138in}{1.094219in}}%
\pgfpathcurveto{\pgfqpoint{3.888962in}{1.088395in}}{\pgfqpoint{3.896862in}{1.085123in}}{\pgfqpoint{3.905098in}{1.085123in}}%
\pgfpathclose%
\pgfusepath{stroke,fill}%
\end{pgfscope}%
\begin{pgfscope}%
\pgfpathrectangle{\pgfqpoint{3.793912in}{0.557870in}}{\pgfqpoint{2.446088in}{1.484734in}}%
\pgfusepath{clip}%
\pgfsetbuttcap%
\pgfsetroundjoin%
\definecolor{currentfill}{rgb}{0.298039,0.447059,0.690196}%
\pgfsetfillcolor{currentfill}%
\pgfsetlinewidth{1.003750pt}%
\definecolor{currentstroke}{rgb}{0.298039,0.447059,0.690196}%
\pgfsetstrokecolor{currentstroke}%
\pgfsetdash{}{0pt}%
\pgfpathmoveto{\pgfqpoint{3.905098in}{1.930426in}}%
\pgfpathcurveto{\pgfqpoint{3.913334in}{1.930426in}}{\pgfqpoint{3.921234in}{1.933698in}}{\pgfqpoint{3.927058in}{1.939522in}}%
\pgfpathcurveto{\pgfqpoint{3.932882in}{1.945346in}}{\pgfqpoint{3.936155in}{1.953246in}}{\pgfqpoint{3.936155in}{1.961482in}}%
\pgfpathcurveto{\pgfqpoint{3.936155in}{1.969719in}}{\pgfqpoint{3.932882in}{1.977619in}}{\pgfqpoint{3.927058in}{1.983443in}}%
\pgfpathcurveto{\pgfqpoint{3.921234in}{1.989267in}}{\pgfqpoint{3.913334in}{1.992539in}}{\pgfqpoint{3.905098in}{1.992539in}}%
\pgfpathcurveto{\pgfqpoint{3.896862in}{1.992539in}}{\pgfqpoint{3.888962in}{1.989267in}}{\pgfqpoint{3.883138in}{1.983443in}}%
\pgfpathcurveto{\pgfqpoint{3.877314in}{1.977619in}}{\pgfqpoint{3.874042in}{1.969719in}}{\pgfqpoint{3.874042in}{1.961482in}}%
\pgfpathcurveto{\pgfqpoint{3.874042in}{1.953246in}}{\pgfqpoint{3.877314in}{1.945346in}}{\pgfqpoint{3.883138in}{1.939522in}}%
\pgfpathcurveto{\pgfqpoint{3.888962in}{1.933698in}}{\pgfqpoint{3.896862in}{1.930426in}}{\pgfqpoint{3.905098in}{1.930426in}}%
\pgfpathclose%
\pgfusepath{stroke,fill}%
\end{pgfscope}%
\begin{pgfscope}%
\pgfpathrectangle{\pgfqpoint{3.793912in}{0.557870in}}{\pgfqpoint{2.446088in}{1.484734in}}%
\pgfusepath{clip}%
\pgfsetbuttcap%
\pgfsetroundjoin%
\definecolor{currentfill}{rgb}{0.298039,0.447059,0.690196}%
\pgfsetfillcolor{currentfill}%
\pgfsetlinewidth{1.003750pt}%
\definecolor{currentstroke}{rgb}{0.298039,0.447059,0.690196}%
\pgfsetstrokecolor{currentstroke}%
\pgfsetdash{}{0pt}%
\pgfpathmoveto{\pgfqpoint{3.905098in}{0.894248in}}%
\pgfpathcurveto{\pgfqpoint{3.913334in}{0.894248in}}{\pgfqpoint{3.921234in}{0.897520in}}{\pgfqpoint{3.927058in}{0.903344in}}%
\pgfpathcurveto{\pgfqpoint{3.932882in}{0.909168in}}{\pgfqpoint{3.936155in}{0.917068in}}{\pgfqpoint{3.936155in}{0.925304in}}%
\pgfpathcurveto{\pgfqpoint{3.936155in}{0.933541in}}{\pgfqpoint{3.932882in}{0.941441in}}{\pgfqpoint{3.927058in}{0.947265in}}%
\pgfpathcurveto{\pgfqpoint{3.921234in}{0.953089in}}{\pgfqpoint{3.913334in}{0.956361in}}{\pgfqpoint{3.905098in}{0.956361in}}%
\pgfpathcurveto{\pgfqpoint{3.896862in}{0.956361in}}{\pgfqpoint{3.888962in}{0.953089in}}{\pgfqpoint{3.883138in}{0.947265in}}%
\pgfpathcurveto{\pgfqpoint{3.877314in}{0.941441in}}{\pgfqpoint{3.874042in}{0.933541in}}{\pgfqpoint{3.874042in}{0.925304in}}%
\pgfpathcurveto{\pgfqpoint{3.874042in}{0.917068in}}{\pgfqpoint{3.877314in}{0.909168in}}{\pgfqpoint{3.883138in}{0.903344in}}%
\pgfpathcurveto{\pgfqpoint{3.888962in}{0.897520in}}{\pgfqpoint{3.896862in}{0.894248in}}{\pgfqpoint{3.905098in}{0.894248in}}%
\pgfpathclose%
\pgfusepath{stroke,fill}%
\end{pgfscope}%
\begin{pgfscope}%
\pgfpathrectangle{\pgfqpoint{3.793912in}{0.557870in}}{\pgfqpoint{2.446088in}{1.484734in}}%
\pgfusepath{clip}%
\pgfsetbuttcap%
\pgfsetroundjoin%
\definecolor{currentfill}{rgb}{0.298039,0.447059,0.690196}%
\pgfsetfillcolor{currentfill}%
\pgfsetlinewidth{1.003750pt}%
\definecolor{currentstroke}{rgb}{0.298039,0.447059,0.690196}%
\pgfsetstrokecolor{currentstroke}%
\pgfsetdash{}{0pt}%
\pgfpathmoveto{\pgfqpoint{3.905098in}{1.930426in}}%
\pgfpathcurveto{\pgfqpoint{3.913334in}{1.930426in}}{\pgfqpoint{3.921234in}{1.933698in}}{\pgfqpoint{3.927058in}{1.939522in}}%
\pgfpathcurveto{\pgfqpoint{3.932882in}{1.945346in}}{\pgfqpoint{3.936155in}{1.953246in}}{\pgfqpoint{3.936155in}{1.961482in}}%
\pgfpathcurveto{\pgfqpoint{3.936155in}{1.969719in}}{\pgfqpoint{3.932882in}{1.977619in}}{\pgfqpoint{3.927058in}{1.983443in}}%
\pgfpathcurveto{\pgfqpoint{3.921234in}{1.989267in}}{\pgfqpoint{3.913334in}{1.992539in}}{\pgfqpoint{3.905098in}{1.992539in}}%
\pgfpathcurveto{\pgfqpoint{3.896862in}{1.992539in}}{\pgfqpoint{3.888962in}{1.989267in}}{\pgfqpoint{3.883138in}{1.983443in}}%
\pgfpathcurveto{\pgfqpoint{3.877314in}{1.977619in}}{\pgfqpoint{3.874042in}{1.969719in}}{\pgfqpoint{3.874042in}{1.961482in}}%
\pgfpathcurveto{\pgfqpoint{3.874042in}{1.953246in}}{\pgfqpoint{3.877314in}{1.945346in}}{\pgfqpoint{3.883138in}{1.939522in}}%
\pgfpathcurveto{\pgfqpoint{3.888962in}{1.933698in}}{\pgfqpoint{3.896862in}{1.930426in}}{\pgfqpoint{3.905098in}{1.930426in}}%
\pgfpathclose%
\pgfusepath{stroke,fill}%
\end{pgfscope}%
\begin{pgfscope}%
\pgfpathrectangle{\pgfqpoint{3.793912in}{0.557870in}}{\pgfqpoint{2.446088in}{1.484734in}}%
\pgfusepath{clip}%
\pgfsetbuttcap%
\pgfsetroundjoin%
\definecolor{currentfill}{rgb}{0.298039,0.447059,0.690196}%
\pgfsetfillcolor{currentfill}%
\pgfsetlinewidth{1.003750pt}%
\definecolor{currentstroke}{rgb}{0.298039,0.447059,0.690196}%
\pgfsetstrokecolor{currentstroke}%
\pgfsetdash{}{0pt}%
\pgfpathmoveto{\pgfqpoint{3.905098in}{0.907882in}}%
\pgfpathcurveto{\pgfqpoint{3.913334in}{0.907882in}}{\pgfqpoint{3.921234in}{0.911154in}}{\pgfqpoint{3.927058in}{0.916978in}}%
\pgfpathcurveto{\pgfqpoint{3.932882in}{0.922802in}}{\pgfqpoint{3.936155in}{0.930702in}}{\pgfqpoint{3.936155in}{0.938938in}}%
\pgfpathcurveto{\pgfqpoint{3.936155in}{0.947175in}}{\pgfqpoint{3.932882in}{0.955075in}}{\pgfqpoint{3.927058in}{0.960899in}}%
\pgfpathcurveto{\pgfqpoint{3.921234in}{0.966723in}}{\pgfqpoint{3.913334in}{0.969995in}}{\pgfqpoint{3.905098in}{0.969995in}}%
\pgfpathcurveto{\pgfqpoint{3.896862in}{0.969995in}}{\pgfqpoint{3.888962in}{0.966723in}}{\pgfqpoint{3.883138in}{0.960899in}}%
\pgfpathcurveto{\pgfqpoint{3.877314in}{0.955075in}}{\pgfqpoint{3.874042in}{0.947175in}}{\pgfqpoint{3.874042in}{0.938938in}}%
\pgfpathcurveto{\pgfqpoint{3.874042in}{0.930702in}}{\pgfqpoint{3.877314in}{0.922802in}}{\pgfqpoint{3.883138in}{0.916978in}}%
\pgfpathcurveto{\pgfqpoint{3.888962in}{0.911154in}}{\pgfqpoint{3.896862in}{0.907882in}}{\pgfqpoint{3.905098in}{0.907882in}}%
\pgfpathclose%
\pgfusepath{stroke,fill}%
\end{pgfscope}%
\begin{pgfscope}%
\pgfpathrectangle{\pgfqpoint{3.793912in}{0.557870in}}{\pgfqpoint{2.446088in}{1.484734in}}%
\pgfusepath{clip}%
\pgfsetbuttcap%
\pgfsetroundjoin%
\definecolor{currentfill}{rgb}{0.298039,0.447059,0.690196}%
\pgfsetfillcolor{currentfill}%
\pgfsetlinewidth{1.003750pt}%
\definecolor{currentstroke}{rgb}{0.298039,0.447059,0.690196}%
\pgfsetstrokecolor{currentstroke}%
\pgfsetdash{}{0pt}%
\pgfpathmoveto{\pgfqpoint{3.905098in}{1.930426in}}%
\pgfpathcurveto{\pgfqpoint{3.913334in}{1.930426in}}{\pgfqpoint{3.921234in}{1.933698in}}{\pgfqpoint{3.927058in}{1.939522in}}%
\pgfpathcurveto{\pgfqpoint{3.932882in}{1.945346in}}{\pgfqpoint{3.936155in}{1.953246in}}{\pgfqpoint{3.936155in}{1.961482in}}%
\pgfpathcurveto{\pgfqpoint{3.936155in}{1.969719in}}{\pgfqpoint{3.932882in}{1.977619in}}{\pgfqpoint{3.927058in}{1.983443in}}%
\pgfpathcurveto{\pgfqpoint{3.921234in}{1.989267in}}{\pgfqpoint{3.913334in}{1.992539in}}{\pgfqpoint{3.905098in}{1.992539in}}%
\pgfpathcurveto{\pgfqpoint{3.896862in}{1.992539in}}{\pgfqpoint{3.888962in}{1.989267in}}{\pgfqpoint{3.883138in}{1.983443in}}%
\pgfpathcurveto{\pgfqpoint{3.877314in}{1.977619in}}{\pgfqpoint{3.874042in}{1.969719in}}{\pgfqpoint{3.874042in}{1.961482in}}%
\pgfpathcurveto{\pgfqpoint{3.874042in}{1.953246in}}{\pgfqpoint{3.877314in}{1.945346in}}{\pgfqpoint{3.883138in}{1.939522in}}%
\pgfpathcurveto{\pgfqpoint{3.888962in}{1.933698in}}{\pgfqpoint{3.896862in}{1.930426in}}{\pgfqpoint{3.905098in}{1.930426in}}%
\pgfpathclose%
\pgfusepath{stroke,fill}%
\end{pgfscope}%
\begin{pgfscope}%
\pgfpathrectangle{\pgfqpoint{3.793912in}{0.557870in}}{\pgfqpoint{2.446088in}{1.484734in}}%
\pgfusepath{clip}%
\pgfsetbuttcap%
\pgfsetroundjoin%
\definecolor{currentfill}{rgb}{0.298039,0.447059,0.690196}%
\pgfsetfillcolor{currentfill}%
\pgfsetlinewidth{1.003750pt}%
\definecolor{currentstroke}{rgb}{0.298039,0.447059,0.690196}%
\pgfsetstrokecolor{currentstroke}%
\pgfsetdash{}{0pt}%
\pgfpathmoveto{\pgfqpoint{3.905098in}{1.930426in}}%
\pgfpathcurveto{\pgfqpoint{3.913334in}{1.930426in}}{\pgfqpoint{3.921234in}{1.933698in}}{\pgfqpoint{3.927058in}{1.939522in}}%
\pgfpathcurveto{\pgfqpoint{3.932882in}{1.945346in}}{\pgfqpoint{3.936155in}{1.953246in}}{\pgfqpoint{3.936155in}{1.961482in}}%
\pgfpathcurveto{\pgfqpoint{3.936155in}{1.969719in}}{\pgfqpoint{3.932882in}{1.977619in}}{\pgfqpoint{3.927058in}{1.983443in}}%
\pgfpathcurveto{\pgfqpoint{3.921234in}{1.989267in}}{\pgfqpoint{3.913334in}{1.992539in}}{\pgfqpoint{3.905098in}{1.992539in}}%
\pgfpathcurveto{\pgfqpoint{3.896862in}{1.992539in}}{\pgfqpoint{3.888962in}{1.989267in}}{\pgfqpoint{3.883138in}{1.983443in}}%
\pgfpathcurveto{\pgfqpoint{3.877314in}{1.977619in}}{\pgfqpoint{3.874042in}{1.969719in}}{\pgfqpoint{3.874042in}{1.961482in}}%
\pgfpathcurveto{\pgfqpoint{3.874042in}{1.953246in}}{\pgfqpoint{3.877314in}{1.945346in}}{\pgfqpoint{3.883138in}{1.939522in}}%
\pgfpathcurveto{\pgfqpoint{3.888962in}{1.933698in}}{\pgfqpoint{3.896862in}{1.930426in}}{\pgfqpoint{3.905098in}{1.930426in}}%
\pgfpathclose%
\pgfusepath{stroke,fill}%
\end{pgfscope}%
\begin{pgfscope}%
\pgfpathrectangle{\pgfqpoint{3.793912in}{0.557870in}}{\pgfqpoint{2.446088in}{1.484734in}}%
\pgfusepath{clip}%
\pgfsetbuttcap%
\pgfsetroundjoin%
\definecolor{currentfill}{rgb}{0.298039,0.447059,0.690196}%
\pgfsetfillcolor{currentfill}%
\pgfsetlinewidth{1.003750pt}%
\definecolor{currentstroke}{rgb}{0.298039,0.447059,0.690196}%
\pgfsetstrokecolor{currentstroke}%
\pgfsetdash{}{0pt}%
\pgfpathmoveto{\pgfqpoint{3.905098in}{1.930426in}}%
\pgfpathcurveto{\pgfqpoint{3.913334in}{1.930426in}}{\pgfqpoint{3.921234in}{1.933698in}}{\pgfqpoint{3.927058in}{1.939522in}}%
\pgfpathcurveto{\pgfqpoint{3.932882in}{1.945346in}}{\pgfqpoint{3.936155in}{1.953246in}}{\pgfqpoint{3.936155in}{1.961482in}}%
\pgfpathcurveto{\pgfqpoint{3.936155in}{1.969719in}}{\pgfqpoint{3.932882in}{1.977619in}}{\pgfqpoint{3.927058in}{1.983443in}}%
\pgfpathcurveto{\pgfqpoint{3.921234in}{1.989267in}}{\pgfqpoint{3.913334in}{1.992539in}}{\pgfqpoint{3.905098in}{1.992539in}}%
\pgfpathcurveto{\pgfqpoint{3.896862in}{1.992539in}}{\pgfqpoint{3.888962in}{1.989267in}}{\pgfqpoint{3.883138in}{1.983443in}}%
\pgfpathcurveto{\pgfqpoint{3.877314in}{1.977619in}}{\pgfqpoint{3.874042in}{1.969719in}}{\pgfqpoint{3.874042in}{1.961482in}}%
\pgfpathcurveto{\pgfqpoint{3.874042in}{1.953246in}}{\pgfqpoint{3.877314in}{1.945346in}}{\pgfqpoint{3.883138in}{1.939522in}}%
\pgfpathcurveto{\pgfqpoint{3.888962in}{1.933698in}}{\pgfqpoint{3.896862in}{1.930426in}}{\pgfqpoint{3.905098in}{1.930426in}}%
\pgfpathclose%
\pgfusepath{stroke,fill}%
\end{pgfscope}%
\begin{pgfscope}%
\pgfpathrectangle{\pgfqpoint{3.793912in}{0.557870in}}{\pgfqpoint{2.446088in}{1.484734in}}%
\pgfusepath{clip}%
\pgfsetbuttcap%
\pgfsetroundjoin%
\definecolor{currentfill}{rgb}{0.298039,0.447059,0.690196}%
\pgfsetfillcolor{currentfill}%
\pgfsetlinewidth{1.003750pt}%
\definecolor{currentstroke}{rgb}{0.298039,0.447059,0.690196}%
\pgfsetstrokecolor{currentstroke}%
\pgfsetdash{}{0pt}%
\pgfpathmoveto{\pgfqpoint{3.905098in}{1.930426in}}%
\pgfpathcurveto{\pgfqpoint{3.913334in}{1.930426in}}{\pgfqpoint{3.921234in}{1.933698in}}{\pgfqpoint{3.927058in}{1.939522in}}%
\pgfpathcurveto{\pgfqpoint{3.932882in}{1.945346in}}{\pgfqpoint{3.936155in}{1.953246in}}{\pgfqpoint{3.936155in}{1.961482in}}%
\pgfpathcurveto{\pgfqpoint{3.936155in}{1.969719in}}{\pgfqpoint{3.932882in}{1.977619in}}{\pgfqpoint{3.927058in}{1.983443in}}%
\pgfpathcurveto{\pgfqpoint{3.921234in}{1.989267in}}{\pgfqpoint{3.913334in}{1.992539in}}{\pgfqpoint{3.905098in}{1.992539in}}%
\pgfpathcurveto{\pgfqpoint{3.896862in}{1.992539in}}{\pgfqpoint{3.888962in}{1.989267in}}{\pgfqpoint{3.883138in}{1.983443in}}%
\pgfpathcurveto{\pgfqpoint{3.877314in}{1.977619in}}{\pgfqpoint{3.874042in}{1.969719in}}{\pgfqpoint{3.874042in}{1.961482in}}%
\pgfpathcurveto{\pgfqpoint{3.874042in}{1.953246in}}{\pgfqpoint{3.877314in}{1.945346in}}{\pgfqpoint{3.883138in}{1.939522in}}%
\pgfpathcurveto{\pgfqpoint{3.888962in}{1.933698in}}{\pgfqpoint{3.896862in}{1.930426in}}{\pgfqpoint{3.905098in}{1.930426in}}%
\pgfpathclose%
\pgfusepath{stroke,fill}%
\end{pgfscope}%
\begin{pgfscope}%
\pgfpathrectangle{\pgfqpoint{3.793912in}{0.557870in}}{\pgfqpoint{2.446088in}{1.484734in}}%
\pgfusepath{clip}%
\pgfsetbuttcap%
\pgfsetroundjoin%
\definecolor{currentfill}{rgb}{0.298039,0.447059,0.690196}%
\pgfsetfillcolor{currentfill}%
\pgfsetlinewidth{1.003750pt}%
\definecolor{currentstroke}{rgb}{0.298039,0.447059,0.690196}%
\pgfsetstrokecolor{currentstroke}%
\pgfsetdash{}{0pt}%
\pgfpathmoveto{\pgfqpoint{5.410739in}{0.594302in}}%
\pgfpathcurveto{\pgfqpoint{5.418975in}{0.594302in}}{\pgfqpoint{5.426876in}{0.597574in}}{\pgfqpoint{5.432699in}{0.603398in}}%
\pgfpathcurveto{\pgfqpoint{5.438523in}{0.609222in}}{\pgfqpoint{5.441796in}{0.617122in}}{\pgfqpoint{5.441796in}{0.625358in}}%
\pgfpathcurveto{\pgfqpoint{5.441796in}{0.633594in}}{\pgfqpoint{5.438523in}{0.641495in}}{\pgfqpoint{5.432699in}{0.647318in}}%
\pgfpathcurveto{\pgfqpoint{5.426876in}{0.653142in}}{\pgfqpoint{5.418975in}{0.656415in}}{\pgfqpoint{5.410739in}{0.656415in}}%
\pgfpathcurveto{\pgfqpoint{5.402503in}{0.656415in}}{\pgfqpoint{5.394603in}{0.653142in}}{\pgfqpoint{5.388779in}{0.647318in}}%
\pgfpathcurveto{\pgfqpoint{5.382955in}{0.641495in}}{\pgfqpoint{5.379683in}{0.633594in}}{\pgfqpoint{5.379683in}{0.625358in}}%
\pgfpathcurveto{\pgfqpoint{5.379683in}{0.617122in}}{\pgfqpoint{5.382955in}{0.609222in}}{\pgfqpoint{5.388779in}{0.603398in}}%
\pgfpathcurveto{\pgfqpoint{5.394603in}{0.597574in}}{\pgfqpoint{5.402503in}{0.594302in}}{\pgfqpoint{5.410739in}{0.594302in}}%
\pgfpathclose%
\pgfusepath{stroke,fill}%
\end{pgfscope}%
\begin{pgfscope}%
\pgfpathrectangle{\pgfqpoint{3.793912in}{0.557870in}}{\pgfqpoint{2.446088in}{1.484734in}}%
\pgfusepath{clip}%
\pgfsetbuttcap%
\pgfsetroundjoin%
\definecolor{currentfill}{rgb}{0.298039,0.447059,0.690196}%
\pgfsetfillcolor{currentfill}%
\pgfsetlinewidth{1.003750pt}%
\definecolor{currentstroke}{rgb}{0.298039,0.447059,0.690196}%
\pgfsetstrokecolor{currentstroke}%
\pgfsetdash{}{0pt}%
\pgfpathmoveto{\pgfqpoint{3.905098in}{1.930426in}}%
\pgfpathcurveto{\pgfqpoint{3.913334in}{1.930426in}}{\pgfqpoint{3.921234in}{1.933698in}}{\pgfqpoint{3.927058in}{1.939522in}}%
\pgfpathcurveto{\pgfqpoint{3.932882in}{1.945346in}}{\pgfqpoint{3.936155in}{1.953246in}}{\pgfqpoint{3.936155in}{1.961482in}}%
\pgfpathcurveto{\pgfqpoint{3.936155in}{1.969719in}}{\pgfqpoint{3.932882in}{1.977619in}}{\pgfqpoint{3.927058in}{1.983443in}}%
\pgfpathcurveto{\pgfqpoint{3.921234in}{1.989267in}}{\pgfqpoint{3.913334in}{1.992539in}}{\pgfqpoint{3.905098in}{1.992539in}}%
\pgfpathcurveto{\pgfqpoint{3.896862in}{1.992539in}}{\pgfqpoint{3.888962in}{1.989267in}}{\pgfqpoint{3.883138in}{1.983443in}}%
\pgfpathcurveto{\pgfqpoint{3.877314in}{1.977619in}}{\pgfqpoint{3.874042in}{1.969719in}}{\pgfqpoint{3.874042in}{1.961482in}}%
\pgfpathcurveto{\pgfqpoint{3.874042in}{1.953246in}}{\pgfqpoint{3.877314in}{1.945346in}}{\pgfqpoint{3.883138in}{1.939522in}}%
\pgfpathcurveto{\pgfqpoint{3.888962in}{1.933698in}}{\pgfqpoint{3.896862in}{1.930426in}}{\pgfqpoint{3.905098in}{1.930426in}}%
\pgfpathclose%
\pgfusepath{stroke,fill}%
\end{pgfscope}%
\begin{pgfscope}%
\pgfpathrectangle{\pgfqpoint{3.793912in}{0.557870in}}{\pgfqpoint{2.446088in}{1.484734in}}%
\pgfusepath{clip}%
\pgfsetbuttcap%
\pgfsetroundjoin%
\definecolor{currentfill}{rgb}{0.298039,0.447059,0.690196}%
\pgfsetfillcolor{currentfill}%
\pgfsetlinewidth{1.003750pt}%
\definecolor{currentstroke}{rgb}{0.298039,0.447059,0.690196}%
\pgfsetstrokecolor{currentstroke}%
\pgfsetdash{}{0pt}%
\pgfpathmoveto{\pgfqpoint{5.433903in}{0.594302in}}%
\pgfpathcurveto{\pgfqpoint{5.442139in}{0.594302in}}{\pgfqpoint{5.450039in}{0.597574in}}{\pgfqpoint{5.455863in}{0.603398in}}%
\pgfpathcurveto{\pgfqpoint{5.461687in}{0.609222in}}{\pgfqpoint{5.464959in}{0.617122in}}{\pgfqpoint{5.464959in}{0.625358in}}%
\pgfpathcurveto{\pgfqpoint{5.464959in}{0.633594in}}{\pgfqpoint{5.461687in}{0.641495in}}{\pgfqpoint{5.455863in}{0.647318in}}%
\pgfpathcurveto{\pgfqpoint{5.450039in}{0.653142in}}{\pgfqpoint{5.442139in}{0.656415in}}{\pgfqpoint{5.433903in}{0.656415in}}%
\pgfpathcurveto{\pgfqpoint{5.425667in}{0.656415in}}{\pgfqpoint{5.417767in}{0.653142in}}{\pgfqpoint{5.411943in}{0.647318in}}%
\pgfpathcurveto{\pgfqpoint{5.406119in}{0.641495in}}{\pgfqpoint{5.402846in}{0.633594in}}{\pgfqpoint{5.402846in}{0.625358in}}%
\pgfpathcurveto{\pgfqpoint{5.402846in}{0.617122in}}{\pgfqpoint{5.406119in}{0.609222in}}{\pgfqpoint{5.411943in}{0.603398in}}%
\pgfpathcurveto{\pgfqpoint{5.417767in}{0.597574in}}{\pgfqpoint{5.425667in}{0.594302in}}{\pgfqpoint{5.433903in}{0.594302in}}%
\pgfpathclose%
\pgfusepath{stroke,fill}%
\end{pgfscope}%
\begin{pgfscope}%
\pgfpathrectangle{\pgfqpoint{3.793912in}{0.557870in}}{\pgfqpoint{2.446088in}{1.484734in}}%
\pgfusepath{clip}%
\pgfsetbuttcap%
\pgfsetroundjoin%
\definecolor{currentfill}{rgb}{0.298039,0.447059,0.690196}%
\pgfsetfillcolor{currentfill}%
\pgfsetlinewidth{1.003750pt}%
\definecolor{currentstroke}{rgb}{0.298039,0.447059,0.690196}%
\pgfsetstrokecolor{currentstroke}%
\pgfsetdash{}{0pt}%
\pgfpathmoveto{\pgfqpoint{3.905098in}{1.930426in}}%
\pgfpathcurveto{\pgfqpoint{3.913334in}{1.930426in}}{\pgfqpoint{3.921234in}{1.933698in}}{\pgfqpoint{3.927058in}{1.939522in}}%
\pgfpathcurveto{\pgfqpoint{3.932882in}{1.945346in}}{\pgfqpoint{3.936155in}{1.953246in}}{\pgfqpoint{3.936155in}{1.961482in}}%
\pgfpathcurveto{\pgfqpoint{3.936155in}{1.969719in}}{\pgfqpoint{3.932882in}{1.977619in}}{\pgfqpoint{3.927058in}{1.983443in}}%
\pgfpathcurveto{\pgfqpoint{3.921234in}{1.989267in}}{\pgfqpoint{3.913334in}{1.992539in}}{\pgfqpoint{3.905098in}{1.992539in}}%
\pgfpathcurveto{\pgfqpoint{3.896862in}{1.992539in}}{\pgfqpoint{3.888962in}{1.989267in}}{\pgfqpoint{3.883138in}{1.983443in}}%
\pgfpathcurveto{\pgfqpoint{3.877314in}{1.977619in}}{\pgfqpoint{3.874042in}{1.969719in}}{\pgfqpoint{3.874042in}{1.961482in}}%
\pgfpathcurveto{\pgfqpoint{3.874042in}{1.953246in}}{\pgfqpoint{3.877314in}{1.945346in}}{\pgfqpoint{3.883138in}{1.939522in}}%
\pgfpathcurveto{\pgfqpoint{3.888962in}{1.933698in}}{\pgfqpoint{3.896862in}{1.930426in}}{\pgfqpoint{3.905098in}{1.930426in}}%
\pgfpathclose%
\pgfusepath{stroke,fill}%
\end{pgfscope}%
\begin{pgfscope}%
\pgfpathrectangle{\pgfqpoint{3.793912in}{0.557870in}}{\pgfqpoint{2.446088in}{1.484734in}}%
\pgfusepath{clip}%
\pgfsetbuttcap%
\pgfsetroundjoin%
\definecolor{currentfill}{rgb}{0.298039,0.447059,0.690196}%
\pgfsetfillcolor{currentfill}%
\pgfsetlinewidth{1.003750pt}%
\definecolor{currentstroke}{rgb}{0.298039,0.447059,0.690196}%
\pgfsetstrokecolor{currentstroke}%
\pgfsetdash{}{0pt}%
\pgfpathmoveto{\pgfqpoint{3.905098in}{1.930426in}}%
\pgfpathcurveto{\pgfqpoint{3.913334in}{1.930426in}}{\pgfqpoint{3.921234in}{1.933698in}}{\pgfqpoint{3.927058in}{1.939522in}}%
\pgfpathcurveto{\pgfqpoint{3.932882in}{1.945346in}}{\pgfqpoint{3.936155in}{1.953246in}}{\pgfqpoint{3.936155in}{1.961482in}}%
\pgfpathcurveto{\pgfqpoint{3.936155in}{1.969719in}}{\pgfqpoint{3.932882in}{1.977619in}}{\pgfqpoint{3.927058in}{1.983443in}}%
\pgfpathcurveto{\pgfqpoint{3.921234in}{1.989267in}}{\pgfqpoint{3.913334in}{1.992539in}}{\pgfqpoint{3.905098in}{1.992539in}}%
\pgfpathcurveto{\pgfqpoint{3.896862in}{1.992539in}}{\pgfqpoint{3.888962in}{1.989267in}}{\pgfqpoint{3.883138in}{1.983443in}}%
\pgfpathcurveto{\pgfqpoint{3.877314in}{1.977619in}}{\pgfqpoint{3.874042in}{1.969719in}}{\pgfqpoint{3.874042in}{1.961482in}}%
\pgfpathcurveto{\pgfqpoint{3.874042in}{1.953246in}}{\pgfqpoint{3.877314in}{1.945346in}}{\pgfqpoint{3.883138in}{1.939522in}}%
\pgfpathcurveto{\pgfqpoint{3.888962in}{1.933698in}}{\pgfqpoint{3.896862in}{1.930426in}}{\pgfqpoint{3.905098in}{1.930426in}}%
\pgfpathclose%
\pgfusepath{stroke,fill}%
\end{pgfscope}%
\begin{pgfscope}%
\pgfpathrectangle{\pgfqpoint{3.793912in}{0.557870in}}{\pgfqpoint{2.446088in}{1.484734in}}%
\pgfusepath{clip}%
\pgfsetbuttcap%
\pgfsetroundjoin%
\definecolor{currentfill}{rgb}{0.298039,0.447059,0.690196}%
\pgfsetfillcolor{currentfill}%
\pgfsetlinewidth{1.003750pt}%
\definecolor{currentstroke}{rgb}{0.298039,0.447059,0.690196}%
\pgfsetstrokecolor{currentstroke}%
\pgfsetdash{}{0pt}%
\pgfpathmoveto{\pgfqpoint{3.905098in}{1.930426in}}%
\pgfpathcurveto{\pgfqpoint{3.913334in}{1.930426in}}{\pgfqpoint{3.921234in}{1.933698in}}{\pgfqpoint{3.927058in}{1.939522in}}%
\pgfpathcurveto{\pgfqpoint{3.932882in}{1.945346in}}{\pgfqpoint{3.936155in}{1.953246in}}{\pgfqpoint{3.936155in}{1.961482in}}%
\pgfpathcurveto{\pgfqpoint{3.936155in}{1.969719in}}{\pgfqpoint{3.932882in}{1.977619in}}{\pgfqpoint{3.927058in}{1.983443in}}%
\pgfpathcurveto{\pgfqpoint{3.921234in}{1.989267in}}{\pgfqpoint{3.913334in}{1.992539in}}{\pgfqpoint{3.905098in}{1.992539in}}%
\pgfpathcurveto{\pgfqpoint{3.896862in}{1.992539in}}{\pgfqpoint{3.888962in}{1.989267in}}{\pgfqpoint{3.883138in}{1.983443in}}%
\pgfpathcurveto{\pgfqpoint{3.877314in}{1.977619in}}{\pgfqpoint{3.874042in}{1.969719in}}{\pgfqpoint{3.874042in}{1.961482in}}%
\pgfpathcurveto{\pgfqpoint{3.874042in}{1.953246in}}{\pgfqpoint{3.877314in}{1.945346in}}{\pgfqpoint{3.883138in}{1.939522in}}%
\pgfpathcurveto{\pgfqpoint{3.888962in}{1.933698in}}{\pgfqpoint{3.896862in}{1.930426in}}{\pgfqpoint{3.905098in}{1.930426in}}%
\pgfpathclose%
\pgfusepath{stroke,fill}%
\end{pgfscope}%
\begin{pgfscope}%
\pgfpathrectangle{\pgfqpoint{3.793912in}{0.557870in}}{\pgfqpoint{2.446088in}{1.484734in}}%
\pgfusepath{clip}%
\pgfsetbuttcap%
\pgfsetroundjoin%
\definecolor{currentfill}{rgb}{0.298039,0.447059,0.690196}%
\pgfsetfillcolor{currentfill}%
\pgfsetlinewidth{1.003750pt}%
\definecolor{currentstroke}{rgb}{0.298039,0.447059,0.690196}%
\pgfsetstrokecolor{currentstroke}%
\pgfsetdash{}{0pt}%
\pgfpathmoveto{\pgfqpoint{3.905098in}{1.235096in}}%
\pgfpathcurveto{\pgfqpoint{3.913334in}{1.235096in}}{\pgfqpoint{3.921234in}{1.238368in}}{\pgfqpoint{3.927058in}{1.244192in}}%
\pgfpathcurveto{\pgfqpoint{3.932882in}{1.250016in}}{\pgfqpoint{3.936155in}{1.257916in}}{\pgfqpoint{3.936155in}{1.266152in}}%
\pgfpathcurveto{\pgfqpoint{3.936155in}{1.274389in}}{\pgfqpoint{3.932882in}{1.282289in}}{\pgfqpoint{3.927058in}{1.288113in}}%
\pgfpathcurveto{\pgfqpoint{3.921234in}{1.293937in}}{\pgfqpoint{3.913334in}{1.297209in}}{\pgfqpoint{3.905098in}{1.297209in}}%
\pgfpathcurveto{\pgfqpoint{3.896862in}{1.297209in}}{\pgfqpoint{3.888962in}{1.293937in}}{\pgfqpoint{3.883138in}{1.288113in}}%
\pgfpathcurveto{\pgfqpoint{3.877314in}{1.282289in}}{\pgfqpoint{3.874042in}{1.274389in}}{\pgfqpoint{3.874042in}{1.266152in}}%
\pgfpathcurveto{\pgfqpoint{3.874042in}{1.257916in}}{\pgfqpoint{3.877314in}{1.250016in}}{\pgfqpoint{3.883138in}{1.244192in}}%
\pgfpathcurveto{\pgfqpoint{3.888962in}{1.238368in}}{\pgfqpoint{3.896862in}{1.235096in}}{\pgfqpoint{3.905098in}{1.235096in}}%
\pgfpathclose%
\pgfusepath{stroke,fill}%
\end{pgfscope}%
\begin{pgfscope}%
\pgfpathrectangle{\pgfqpoint{3.793912in}{0.557870in}}{\pgfqpoint{2.446088in}{1.484734in}}%
\pgfusepath{clip}%
\pgfsetbuttcap%
\pgfsetroundjoin%
\definecolor{currentfill}{rgb}{0.298039,0.447059,0.690196}%
\pgfsetfillcolor{currentfill}%
\pgfsetlinewidth{1.003750pt}%
\definecolor{currentstroke}{rgb}{0.298039,0.447059,0.690196}%
\pgfsetstrokecolor{currentstroke}%
\pgfsetdash{}{0pt}%
\pgfpathmoveto{\pgfqpoint{3.905098in}{1.453239in}}%
\pgfpathcurveto{\pgfqpoint{3.913334in}{1.453239in}}{\pgfqpoint{3.921234in}{1.456511in}}{\pgfqpoint{3.927058in}{1.462335in}}%
\pgfpathcurveto{\pgfqpoint{3.932882in}{1.468159in}}{\pgfqpoint{3.936155in}{1.476059in}}{\pgfqpoint{3.936155in}{1.484295in}}%
\pgfpathcurveto{\pgfqpoint{3.936155in}{1.492531in}}{\pgfqpoint{3.932882in}{1.500431in}}{\pgfqpoint{3.927058in}{1.506255in}}%
\pgfpathcurveto{\pgfqpoint{3.921234in}{1.512079in}}{\pgfqpoint{3.913334in}{1.515352in}}{\pgfqpoint{3.905098in}{1.515352in}}%
\pgfpathcurveto{\pgfqpoint{3.896862in}{1.515352in}}{\pgfqpoint{3.888962in}{1.512079in}}{\pgfqpoint{3.883138in}{1.506255in}}%
\pgfpathcurveto{\pgfqpoint{3.877314in}{1.500431in}}{\pgfqpoint{3.874042in}{1.492531in}}{\pgfqpoint{3.874042in}{1.484295in}}%
\pgfpathcurveto{\pgfqpoint{3.874042in}{1.476059in}}{\pgfqpoint{3.877314in}{1.468159in}}{\pgfqpoint{3.883138in}{1.462335in}}%
\pgfpathcurveto{\pgfqpoint{3.888962in}{1.456511in}}{\pgfqpoint{3.896862in}{1.453239in}}{\pgfqpoint{3.905098in}{1.453239in}}%
\pgfpathclose%
\pgfusepath{stroke,fill}%
\end{pgfscope}%
\begin{pgfscope}%
\pgfpathrectangle{\pgfqpoint{3.793912in}{0.557870in}}{\pgfqpoint{2.446088in}{1.484734in}}%
\pgfusepath{clip}%
\pgfsetbuttcap%
\pgfsetroundjoin%
\definecolor{currentfill}{rgb}{0.298039,0.447059,0.690196}%
\pgfsetfillcolor{currentfill}%
\pgfsetlinewidth{1.003750pt}%
\definecolor{currentstroke}{rgb}{0.298039,0.447059,0.690196}%
\pgfsetstrokecolor{currentstroke}%
\pgfsetdash{}{0pt}%
\pgfpathmoveto{\pgfqpoint{3.905098in}{1.480506in}}%
\pgfpathcurveto{\pgfqpoint{3.913334in}{1.480506in}}{\pgfqpoint{3.921234in}{1.483779in}}{\pgfqpoint{3.927058in}{1.489603in}}%
\pgfpathcurveto{\pgfqpoint{3.932882in}{1.495427in}}{\pgfqpoint{3.936155in}{1.503327in}}{\pgfqpoint{3.936155in}{1.511563in}}%
\pgfpathcurveto{\pgfqpoint{3.936155in}{1.519799in}}{\pgfqpoint{3.932882in}{1.527699in}}{\pgfqpoint{3.927058in}{1.533523in}}%
\pgfpathcurveto{\pgfqpoint{3.921234in}{1.539347in}}{\pgfqpoint{3.913334in}{1.542619in}}{\pgfqpoint{3.905098in}{1.542619in}}%
\pgfpathcurveto{\pgfqpoint{3.896862in}{1.542619in}}{\pgfqpoint{3.888962in}{1.539347in}}{\pgfqpoint{3.883138in}{1.533523in}}%
\pgfpathcurveto{\pgfqpoint{3.877314in}{1.527699in}}{\pgfqpoint{3.874042in}{1.519799in}}{\pgfqpoint{3.874042in}{1.511563in}}%
\pgfpathcurveto{\pgfqpoint{3.874042in}{1.503327in}}{\pgfqpoint{3.877314in}{1.495427in}}{\pgfqpoint{3.883138in}{1.489603in}}%
\pgfpathcurveto{\pgfqpoint{3.888962in}{1.483779in}}{\pgfqpoint{3.896862in}{1.480506in}}{\pgfqpoint{3.905098in}{1.480506in}}%
\pgfpathclose%
\pgfusepath{stroke,fill}%
\end{pgfscope}%
\begin{pgfscope}%
\pgfpathrectangle{\pgfqpoint{3.793912in}{0.557870in}}{\pgfqpoint{2.446088in}{1.484734in}}%
\pgfusepath{clip}%
\pgfsetbuttcap%
\pgfsetroundjoin%
\definecolor{currentfill}{rgb}{0.298039,0.447059,0.690196}%
\pgfsetfillcolor{currentfill}%
\pgfsetlinewidth{1.003750pt}%
\definecolor{currentstroke}{rgb}{0.298039,0.447059,0.690196}%
\pgfsetstrokecolor{currentstroke}%
\pgfsetdash{}{0pt}%
\pgfpathmoveto{\pgfqpoint{3.905098in}{1.480506in}}%
\pgfpathcurveto{\pgfqpoint{3.913334in}{1.480506in}}{\pgfqpoint{3.921234in}{1.483779in}}{\pgfqpoint{3.927058in}{1.489603in}}%
\pgfpathcurveto{\pgfqpoint{3.932882in}{1.495427in}}{\pgfqpoint{3.936155in}{1.503327in}}{\pgfqpoint{3.936155in}{1.511563in}}%
\pgfpathcurveto{\pgfqpoint{3.936155in}{1.519799in}}{\pgfqpoint{3.932882in}{1.527699in}}{\pgfqpoint{3.927058in}{1.533523in}}%
\pgfpathcurveto{\pgfqpoint{3.921234in}{1.539347in}}{\pgfqpoint{3.913334in}{1.542619in}}{\pgfqpoint{3.905098in}{1.542619in}}%
\pgfpathcurveto{\pgfqpoint{3.896862in}{1.542619in}}{\pgfqpoint{3.888962in}{1.539347in}}{\pgfqpoint{3.883138in}{1.533523in}}%
\pgfpathcurveto{\pgfqpoint{3.877314in}{1.527699in}}{\pgfqpoint{3.874042in}{1.519799in}}{\pgfqpoint{3.874042in}{1.511563in}}%
\pgfpathcurveto{\pgfqpoint{3.874042in}{1.503327in}}{\pgfqpoint{3.877314in}{1.495427in}}{\pgfqpoint{3.883138in}{1.489603in}}%
\pgfpathcurveto{\pgfqpoint{3.888962in}{1.483779in}}{\pgfqpoint{3.896862in}{1.480506in}}{\pgfqpoint{3.905098in}{1.480506in}}%
\pgfpathclose%
\pgfusepath{stroke,fill}%
\end{pgfscope}%
\begin{pgfscope}%
\pgfpathrectangle{\pgfqpoint{3.793912in}{0.557870in}}{\pgfqpoint{2.446088in}{1.484734in}}%
\pgfusepath{clip}%
\pgfsetbuttcap%
\pgfsetroundjoin%
\definecolor{currentfill}{rgb}{0.298039,0.447059,0.690196}%
\pgfsetfillcolor{currentfill}%
\pgfsetlinewidth{1.003750pt}%
\definecolor{currentstroke}{rgb}{0.298039,0.447059,0.690196}%
\pgfsetstrokecolor{currentstroke}%
\pgfsetdash{}{0pt}%
\pgfpathmoveto{\pgfqpoint{5.526558in}{0.594302in}}%
\pgfpathcurveto{\pgfqpoint{5.534794in}{0.594302in}}{\pgfqpoint{5.542694in}{0.597574in}}{\pgfqpoint{5.548518in}{0.603398in}}%
\pgfpathcurveto{\pgfqpoint{5.554342in}{0.609222in}}{\pgfqpoint{5.557614in}{0.617122in}}{\pgfqpoint{5.557614in}{0.625358in}}%
\pgfpathcurveto{\pgfqpoint{5.557614in}{0.633594in}}{\pgfqpoint{5.554342in}{0.641495in}}{\pgfqpoint{5.548518in}{0.647318in}}%
\pgfpathcurveto{\pgfqpoint{5.542694in}{0.653142in}}{\pgfqpoint{5.534794in}{0.656415in}}{\pgfqpoint{5.526558in}{0.656415in}}%
\pgfpathcurveto{\pgfqpoint{5.518321in}{0.656415in}}{\pgfqpoint{5.510421in}{0.653142in}}{\pgfqpoint{5.504597in}{0.647318in}}%
\pgfpathcurveto{\pgfqpoint{5.498774in}{0.641495in}}{\pgfqpoint{5.495501in}{0.633594in}}{\pgfqpoint{5.495501in}{0.625358in}}%
\pgfpathcurveto{\pgfqpoint{5.495501in}{0.617122in}}{\pgfqpoint{5.498774in}{0.609222in}}{\pgfqpoint{5.504597in}{0.603398in}}%
\pgfpathcurveto{\pgfqpoint{5.510421in}{0.597574in}}{\pgfqpoint{5.518321in}{0.594302in}}{\pgfqpoint{5.526558in}{0.594302in}}%
\pgfpathclose%
\pgfusepath{stroke,fill}%
\end{pgfscope}%
\begin{pgfscope}%
\pgfpathrectangle{\pgfqpoint{3.793912in}{0.557870in}}{\pgfqpoint{2.446088in}{1.484734in}}%
\pgfusepath{clip}%
\pgfsetbuttcap%
\pgfsetroundjoin%
\definecolor{currentfill}{rgb}{0.298039,0.447059,0.690196}%
\pgfsetfillcolor{currentfill}%
\pgfsetlinewidth{1.003750pt}%
\definecolor{currentstroke}{rgb}{0.298039,0.447059,0.690196}%
\pgfsetstrokecolor{currentstroke}%
\pgfsetdash{}{0pt}%
\pgfpathmoveto{\pgfqpoint{3.905098in}{1.085123in}}%
\pgfpathcurveto{\pgfqpoint{3.913334in}{1.085123in}}{\pgfqpoint{3.921234in}{1.088395in}}{\pgfqpoint{3.927058in}{1.094219in}}%
\pgfpathcurveto{\pgfqpoint{3.932882in}{1.100043in}}{\pgfqpoint{3.936155in}{1.107943in}}{\pgfqpoint{3.936155in}{1.116179in}}%
\pgfpathcurveto{\pgfqpoint{3.936155in}{1.124416in}}{\pgfqpoint{3.932882in}{1.132316in}}{\pgfqpoint{3.927058in}{1.138140in}}%
\pgfpathcurveto{\pgfqpoint{3.921234in}{1.143963in}}{\pgfqpoint{3.913334in}{1.147236in}}{\pgfqpoint{3.905098in}{1.147236in}}%
\pgfpathcurveto{\pgfqpoint{3.896862in}{1.147236in}}{\pgfqpoint{3.888962in}{1.143963in}}{\pgfqpoint{3.883138in}{1.138140in}}%
\pgfpathcurveto{\pgfqpoint{3.877314in}{1.132316in}}{\pgfqpoint{3.874042in}{1.124416in}}{\pgfqpoint{3.874042in}{1.116179in}}%
\pgfpathcurveto{\pgfqpoint{3.874042in}{1.107943in}}{\pgfqpoint{3.877314in}{1.100043in}}{\pgfqpoint{3.883138in}{1.094219in}}%
\pgfpathcurveto{\pgfqpoint{3.888962in}{1.088395in}}{\pgfqpoint{3.896862in}{1.085123in}}{\pgfqpoint{3.905098in}{1.085123in}}%
\pgfpathclose%
\pgfusepath{stroke,fill}%
\end{pgfscope}%
\begin{pgfscope}%
\pgfpathrectangle{\pgfqpoint{3.793912in}{0.557870in}}{\pgfqpoint{2.446088in}{1.484734in}}%
\pgfusepath{clip}%
\pgfsetbuttcap%
\pgfsetroundjoin%
\definecolor{currentfill}{rgb}{0.298039,0.447059,0.690196}%
\pgfsetfillcolor{currentfill}%
\pgfsetlinewidth{1.003750pt}%
\definecolor{currentstroke}{rgb}{0.298039,0.447059,0.690196}%
\pgfsetstrokecolor{currentstroke}%
\pgfsetdash{}{0pt}%
\pgfpathmoveto{\pgfqpoint{3.905098in}{1.930426in}}%
\pgfpathcurveto{\pgfqpoint{3.913334in}{1.930426in}}{\pgfqpoint{3.921234in}{1.933698in}}{\pgfqpoint{3.927058in}{1.939522in}}%
\pgfpathcurveto{\pgfqpoint{3.932882in}{1.945346in}}{\pgfqpoint{3.936155in}{1.953246in}}{\pgfqpoint{3.936155in}{1.961482in}}%
\pgfpathcurveto{\pgfqpoint{3.936155in}{1.969719in}}{\pgfqpoint{3.932882in}{1.977619in}}{\pgfqpoint{3.927058in}{1.983443in}}%
\pgfpathcurveto{\pgfqpoint{3.921234in}{1.989267in}}{\pgfqpoint{3.913334in}{1.992539in}}{\pgfqpoint{3.905098in}{1.992539in}}%
\pgfpathcurveto{\pgfqpoint{3.896862in}{1.992539in}}{\pgfqpoint{3.888962in}{1.989267in}}{\pgfqpoint{3.883138in}{1.983443in}}%
\pgfpathcurveto{\pgfqpoint{3.877314in}{1.977619in}}{\pgfqpoint{3.874042in}{1.969719in}}{\pgfqpoint{3.874042in}{1.961482in}}%
\pgfpathcurveto{\pgfqpoint{3.874042in}{1.953246in}}{\pgfqpoint{3.877314in}{1.945346in}}{\pgfqpoint{3.883138in}{1.939522in}}%
\pgfpathcurveto{\pgfqpoint{3.888962in}{1.933698in}}{\pgfqpoint{3.896862in}{1.930426in}}{\pgfqpoint{3.905098in}{1.930426in}}%
\pgfpathclose%
\pgfusepath{stroke,fill}%
\end{pgfscope}%
\begin{pgfscope}%
\pgfpathrectangle{\pgfqpoint{3.793912in}{0.557870in}}{\pgfqpoint{2.446088in}{1.484734in}}%
\pgfusepath{clip}%
\pgfsetbuttcap%
\pgfsetroundjoin%
\definecolor{currentfill}{rgb}{0.298039,0.447059,0.690196}%
\pgfsetfillcolor{currentfill}%
\pgfsetlinewidth{1.003750pt}%
\definecolor{currentstroke}{rgb}{0.298039,0.447059,0.690196}%
\pgfsetstrokecolor{currentstroke}%
\pgfsetdash{}{0pt}%
\pgfpathmoveto{\pgfqpoint{5.016956in}{0.594302in}}%
\pgfpathcurveto{\pgfqpoint{5.025192in}{0.594302in}}{\pgfqpoint{5.033092in}{0.597574in}}{\pgfqpoint{5.038916in}{0.603398in}}%
\pgfpathcurveto{\pgfqpoint{5.044740in}{0.609222in}}{\pgfqpoint{5.048013in}{0.617122in}}{\pgfqpoint{5.048013in}{0.625358in}}%
\pgfpathcurveto{\pgfqpoint{5.048013in}{0.633594in}}{\pgfqpoint{5.044740in}{0.641495in}}{\pgfqpoint{5.038916in}{0.647318in}}%
\pgfpathcurveto{\pgfqpoint{5.033092in}{0.653142in}}{\pgfqpoint{5.025192in}{0.656415in}}{\pgfqpoint{5.016956in}{0.656415in}}%
\pgfpathcurveto{\pgfqpoint{5.008720in}{0.656415in}}{\pgfqpoint{5.000820in}{0.653142in}}{\pgfqpoint{4.994996in}{0.647318in}}%
\pgfpathcurveto{\pgfqpoint{4.989172in}{0.641495in}}{\pgfqpoint{4.985900in}{0.633594in}}{\pgfqpoint{4.985900in}{0.625358in}}%
\pgfpathcurveto{\pgfqpoint{4.985900in}{0.617122in}}{\pgfqpoint{4.989172in}{0.609222in}}{\pgfqpoint{4.994996in}{0.603398in}}%
\pgfpathcurveto{\pgfqpoint{5.000820in}{0.597574in}}{\pgfqpoint{5.008720in}{0.594302in}}{\pgfqpoint{5.016956in}{0.594302in}}%
\pgfpathclose%
\pgfusepath{stroke,fill}%
\end{pgfscope}%
\begin{pgfscope}%
\pgfpathrectangle{\pgfqpoint{3.793912in}{0.557870in}}{\pgfqpoint{2.446088in}{1.484734in}}%
\pgfusepath{clip}%
\pgfsetbuttcap%
\pgfsetroundjoin%
\definecolor{currentfill}{rgb}{0.298039,0.447059,0.690196}%
\pgfsetfillcolor{currentfill}%
\pgfsetlinewidth{1.003750pt}%
\definecolor{currentstroke}{rgb}{0.298039,0.447059,0.690196}%
\pgfsetstrokecolor{currentstroke}%
\pgfsetdash{}{0pt}%
\pgfpathmoveto{\pgfqpoint{3.905098in}{1.330533in}}%
\pgfpathcurveto{\pgfqpoint{3.913334in}{1.330533in}}{\pgfqpoint{3.921234in}{1.333806in}}{\pgfqpoint{3.927058in}{1.339630in}}%
\pgfpathcurveto{\pgfqpoint{3.932882in}{1.345454in}}{\pgfqpoint{3.936155in}{1.353354in}}{\pgfqpoint{3.936155in}{1.361590in}}%
\pgfpathcurveto{\pgfqpoint{3.936155in}{1.369826in}}{\pgfqpoint{3.932882in}{1.377726in}}{\pgfqpoint{3.927058in}{1.383550in}}%
\pgfpathcurveto{\pgfqpoint{3.921234in}{1.389374in}}{\pgfqpoint{3.913334in}{1.392646in}}{\pgfqpoint{3.905098in}{1.392646in}}%
\pgfpathcurveto{\pgfqpoint{3.896862in}{1.392646in}}{\pgfqpoint{3.888962in}{1.389374in}}{\pgfqpoint{3.883138in}{1.383550in}}%
\pgfpathcurveto{\pgfqpoint{3.877314in}{1.377726in}}{\pgfqpoint{3.874042in}{1.369826in}}{\pgfqpoint{3.874042in}{1.361590in}}%
\pgfpathcurveto{\pgfqpoint{3.874042in}{1.353354in}}{\pgfqpoint{3.877314in}{1.345454in}}{\pgfqpoint{3.883138in}{1.339630in}}%
\pgfpathcurveto{\pgfqpoint{3.888962in}{1.333806in}}{\pgfqpoint{3.896862in}{1.330533in}}{\pgfqpoint{3.905098in}{1.330533in}}%
\pgfpathclose%
\pgfusepath{stroke,fill}%
\end{pgfscope}%
\begin{pgfscope}%
\pgfpathrectangle{\pgfqpoint{3.793912in}{0.557870in}}{\pgfqpoint{2.446088in}{1.484734in}}%
\pgfusepath{clip}%
\pgfsetbuttcap%
\pgfsetroundjoin%
\definecolor{currentfill}{rgb}{0.298039,0.447059,0.690196}%
\pgfsetfillcolor{currentfill}%
\pgfsetlinewidth{1.003750pt}%
\definecolor{currentstroke}{rgb}{0.298039,0.447059,0.690196}%
\pgfsetstrokecolor{currentstroke}%
\pgfsetdash{}{0pt}%
\pgfpathmoveto{\pgfqpoint{3.905098in}{1.330533in}}%
\pgfpathcurveto{\pgfqpoint{3.913334in}{1.330533in}}{\pgfqpoint{3.921234in}{1.333806in}}{\pgfqpoint{3.927058in}{1.339630in}}%
\pgfpathcurveto{\pgfqpoint{3.932882in}{1.345454in}}{\pgfqpoint{3.936155in}{1.353354in}}{\pgfqpoint{3.936155in}{1.361590in}}%
\pgfpathcurveto{\pgfqpoint{3.936155in}{1.369826in}}{\pgfqpoint{3.932882in}{1.377726in}}{\pgfqpoint{3.927058in}{1.383550in}}%
\pgfpathcurveto{\pgfqpoint{3.921234in}{1.389374in}}{\pgfqpoint{3.913334in}{1.392646in}}{\pgfqpoint{3.905098in}{1.392646in}}%
\pgfpathcurveto{\pgfqpoint{3.896862in}{1.392646in}}{\pgfqpoint{3.888962in}{1.389374in}}{\pgfqpoint{3.883138in}{1.383550in}}%
\pgfpathcurveto{\pgfqpoint{3.877314in}{1.377726in}}{\pgfqpoint{3.874042in}{1.369826in}}{\pgfqpoint{3.874042in}{1.361590in}}%
\pgfpathcurveto{\pgfqpoint{3.874042in}{1.353354in}}{\pgfqpoint{3.877314in}{1.345454in}}{\pgfqpoint{3.883138in}{1.339630in}}%
\pgfpathcurveto{\pgfqpoint{3.888962in}{1.333806in}}{\pgfqpoint{3.896862in}{1.330533in}}{\pgfqpoint{3.905098in}{1.330533in}}%
\pgfpathclose%
\pgfusepath{stroke,fill}%
\end{pgfscope}%
\begin{pgfscope}%
\pgfpathrectangle{\pgfqpoint{3.793912in}{0.557870in}}{\pgfqpoint{2.446088in}{1.484734in}}%
\pgfusepath{clip}%
\pgfsetbuttcap%
\pgfsetroundjoin%
\definecolor{currentfill}{rgb}{0.298039,0.447059,0.690196}%
\pgfsetfillcolor{currentfill}%
\pgfsetlinewidth{1.003750pt}%
\definecolor{currentstroke}{rgb}{0.298039,0.447059,0.690196}%
\pgfsetstrokecolor{currentstroke}%
\pgfsetdash{}{0pt}%
\pgfpathmoveto{\pgfqpoint{3.905098in}{0.662471in}}%
\pgfpathcurveto{\pgfqpoint{3.913334in}{0.662471in}}{\pgfqpoint{3.921234in}{0.665744in}}{\pgfqpoint{3.927058in}{0.671568in}}%
\pgfpathcurveto{\pgfqpoint{3.932882in}{0.677391in}}{\pgfqpoint{3.936155in}{0.685292in}}{\pgfqpoint{3.936155in}{0.693528in}}%
\pgfpathcurveto{\pgfqpoint{3.936155in}{0.701764in}}{\pgfqpoint{3.932882in}{0.709664in}}{\pgfqpoint{3.927058in}{0.715488in}}%
\pgfpathcurveto{\pgfqpoint{3.921234in}{0.721312in}}{\pgfqpoint{3.913334in}{0.724584in}}{\pgfqpoint{3.905098in}{0.724584in}}%
\pgfpathcurveto{\pgfqpoint{3.896862in}{0.724584in}}{\pgfqpoint{3.888962in}{0.721312in}}{\pgfqpoint{3.883138in}{0.715488in}}%
\pgfpathcurveto{\pgfqpoint{3.877314in}{0.709664in}}{\pgfqpoint{3.874042in}{0.701764in}}{\pgfqpoint{3.874042in}{0.693528in}}%
\pgfpathcurveto{\pgfqpoint{3.874042in}{0.685292in}}{\pgfqpoint{3.877314in}{0.677391in}}{\pgfqpoint{3.883138in}{0.671568in}}%
\pgfpathcurveto{\pgfqpoint{3.888962in}{0.665744in}}{\pgfqpoint{3.896862in}{0.662471in}}{\pgfqpoint{3.905098in}{0.662471in}}%
\pgfpathclose%
\pgfusepath{stroke,fill}%
\end{pgfscope}%
\begin{pgfscope}%
\pgfpathrectangle{\pgfqpoint{3.793912in}{0.557870in}}{\pgfqpoint{2.446088in}{1.484734in}}%
\pgfusepath{clip}%
\pgfsetbuttcap%
\pgfsetroundjoin%
\definecolor{currentfill}{rgb}{0.298039,0.447059,0.690196}%
\pgfsetfillcolor{currentfill}%
\pgfsetlinewidth{1.003750pt}%
\definecolor{currentstroke}{rgb}{0.298039,0.447059,0.690196}%
\pgfsetstrokecolor{currentstroke}%
\pgfsetdash{}{0pt}%
\pgfpathmoveto{\pgfqpoint{3.905098in}{1.930426in}}%
\pgfpathcurveto{\pgfqpoint{3.913334in}{1.930426in}}{\pgfqpoint{3.921234in}{1.933698in}}{\pgfqpoint{3.927058in}{1.939522in}}%
\pgfpathcurveto{\pgfqpoint{3.932882in}{1.945346in}}{\pgfqpoint{3.936155in}{1.953246in}}{\pgfqpoint{3.936155in}{1.961482in}}%
\pgfpathcurveto{\pgfqpoint{3.936155in}{1.969719in}}{\pgfqpoint{3.932882in}{1.977619in}}{\pgfqpoint{3.927058in}{1.983443in}}%
\pgfpathcurveto{\pgfqpoint{3.921234in}{1.989267in}}{\pgfqpoint{3.913334in}{1.992539in}}{\pgfqpoint{3.905098in}{1.992539in}}%
\pgfpathcurveto{\pgfqpoint{3.896862in}{1.992539in}}{\pgfqpoint{3.888962in}{1.989267in}}{\pgfqpoint{3.883138in}{1.983443in}}%
\pgfpathcurveto{\pgfqpoint{3.877314in}{1.977619in}}{\pgfqpoint{3.874042in}{1.969719in}}{\pgfqpoint{3.874042in}{1.961482in}}%
\pgfpathcurveto{\pgfqpoint{3.874042in}{1.953246in}}{\pgfqpoint{3.877314in}{1.945346in}}{\pgfqpoint{3.883138in}{1.939522in}}%
\pgfpathcurveto{\pgfqpoint{3.888962in}{1.933698in}}{\pgfqpoint{3.896862in}{1.930426in}}{\pgfqpoint{3.905098in}{1.930426in}}%
\pgfpathclose%
\pgfusepath{stroke,fill}%
\end{pgfscope}%
\begin{pgfscope}%
\pgfpathrectangle{\pgfqpoint{3.793912in}{0.557870in}}{\pgfqpoint{2.446088in}{1.484734in}}%
\pgfusepath{clip}%
\pgfsetbuttcap%
\pgfsetroundjoin%
\definecolor{currentfill}{rgb}{0.298039,0.447059,0.690196}%
\pgfsetfillcolor{currentfill}%
\pgfsetlinewidth{1.003750pt}%
\definecolor{currentstroke}{rgb}{0.298039,0.447059,0.690196}%
\pgfsetstrokecolor{currentstroke}%
\pgfsetdash{}{0pt}%
\pgfpathmoveto{\pgfqpoint{3.905098in}{1.930426in}}%
\pgfpathcurveto{\pgfqpoint{3.913334in}{1.930426in}}{\pgfqpoint{3.921234in}{1.933698in}}{\pgfqpoint{3.927058in}{1.939522in}}%
\pgfpathcurveto{\pgfqpoint{3.932882in}{1.945346in}}{\pgfqpoint{3.936155in}{1.953246in}}{\pgfqpoint{3.936155in}{1.961482in}}%
\pgfpathcurveto{\pgfqpoint{3.936155in}{1.969719in}}{\pgfqpoint{3.932882in}{1.977619in}}{\pgfqpoint{3.927058in}{1.983443in}}%
\pgfpathcurveto{\pgfqpoint{3.921234in}{1.989267in}}{\pgfqpoint{3.913334in}{1.992539in}}{\pgfqpoint{3.905098in}{1.992539in}}%
\pgfpathcurveto{\pgfqpoint{3.896862in}{1.992539in}}{\pgfqpoint{3.888962in}{1.989267in}}{\pgfqpoint{3.883138in}{1.983443in}}%
\pgfpathcurveto{\pgfqpoint{3.877314in}{1.977619in}}{\pgfqpoint{3.874042in}{1.969719in}}{\pgfqpoint{3.874042in}{1.961482in}}%
\pgfpathcurveto{\pgfqpoint{3.874042in}{1.953246in}}{\pgfqpoint{3.877314in}{1.945346in}}{\pgfqpoint{3.883138in}{1.939522in}}%
\pgfpathcurveto{\pgfqpoint{3.888962in}{1.933698in}}{\pgfqpoint{3.896862in}{1.930426in}}{\pgfqpoint{3.905098in}{1.930426in}}%
\pgfpathclose%
\pgfusepath{stroke,fill}%
\end{pgfscope}%
\begin{pgfscope}%
\pgfpathrectangle{\pgfqpoint{3.793912in}{0.557870in}}{\pgfqpoint{2.446088in}{1.484734in}}%
\pgfusepath{clip}%
\pgfsetbuttcap%
\pgfsetroundjoin%
\definecolor{currentfill}{rgb}{0.298039,0.447059,0.690196}%
\pgfsetfillcolor{currentfill}%
\pgfsetlinewidth{1.003750pt}%
\definecolor{currentstroke}{rgb}{0.298039,0.447059,0.690196}%
\pgfsetstrokecolor{currentstroke}%
\pgfsetdash{}{0pt}%
\pgfpathmoveto{\pgfqpoint{5.735031in}{0.594302in}}%
\pgfpathcurveto{\pgfqpoint{5.743267in}{0.594302in}}{\pgfqpoint{5.751167in}{0.597574in}}{\pgfqpoint{5.756991in}{0.603398in}}%
\pgfpathcurveto{\pgfqpoint{5.762815in}{0.609222in}}{\pgfqpoint{5.766088in}{0.617122in}}{\pgfqpoint{5.766088in}{0.625358in}}%
\pgfpathcurveto{\pgfqpoint{5.766088in}{0.633594in}}{\pgfqpoint{5.762815in}{0.641495in}}{\pgfqpoint{5.756991in}{0.647318in}}%
\pgfpathcurveto{\pgfqpoint{5.751167in}{0.653142in}}{\pgfqpoint{5.743267in}{0.656415in}}{\pgfqpoint{5.735031in}{0.656415in}}%
\pgfpathcurveto{\pgfqpoint{5.726795in}{0.656415in}}{\pgfqpoint{5.718895in}{0.653142in}}{\pgfqpoint{5.713071in}{0.647318in}}%
\pgfpathcurveto{\pgfqpoint{5.707247in}{0.641495in}}{\pgfqpoint{5.703975in}{0.633594in}}{\pgfqpoint{5.703975in}{0.625358in}}%
\pgfpathcurveto{\pgfqpoint{5.703975in}{0.617122in}}{\pgfqpoint{5.707247in}{0.609222in}}{\pgfqpoint{5.713071in}{0.603398in}}%
\pgfpathcurveto{\pgfqpoint{5.718895in}{0.597574in}}{\pgfqpoint{5.726795in}{0.594302in}}{\pgfqpoint{5.735031in}{0.594302in}}%
\pgfpathclose%
\pgfusepath{stroke,fill}%
\end{pgfscope}%
\begin{pgfscope}%
\pgfpathrectangle{\pgfqpoint{3.793912in}{0.557870in}}{\pgfqpoint{2.446088in}{1.484734in}}%
\pgfusepath{clip}%
\pgfsetbuttcap%
\pgfsetroundjoin%
\definecolor{currentfill}{rgb}{0.298039,0.447059,0.690196}%
\pgfsetfillcolor{currentfill}%
\pgfsetlinewidth{1.003750pt}%
\definecolor{currentstroke}{rgb}{0.298039,0.447059,0.690196}%
\pgfsetstrokecolor{currentstroke}%
\pgfsetdash{}{0pt}%
\pgfpathmoveto{\pgfqpoint{5.202266in}{0.594302in}}%
\pgfpathcurveto{\pgfqpoint{5.210502in}{0.594302in}}{\pgfqpoint{5.218402in}{0.597574in}}{\pgfqpoint{5.224226in}{0.603398in}}%
\pgfpathcurveto{\pgfqpoint{5.230050in}{0.609222in}}{\pgfqpoint{5.233322in}{0.617122in}}{\pgfqpoint{5.233322in}{0.625358in}}%
\pgfpathcurveto{\pgfqpoint{5.233322in}{0.633594in}}{\pgfqpoint{5.230050in}{0.641495in}}{\pgfqpoint{5.224226in}{0.647318in}}%
\pgfpathcurveto{\pgfqpoint{5.218402in}{0.653142in}}{\pgfqpoint{5.210502in}{0.656415in}}{\pgfqpoint{5.202266in}{0.656415in}}%
\pgfpathcurveto{\pgfqpoint{5.194030in}{0.656415in}}{\pgfqpoint{5.186129in}{0.653142in}}{\pgfqpoint{5.180306in}{0.647318in}}%
\pgfpathcurveto{\pgfqpoint{5.174482in}{0.641495in}}{\pgfqpoint{5.171209in}{0.633594in}}{\pgfqpoint{5.171209in}{0.625358in}}%
\pgfpathcurveto{\pgfqpoint{5.171209in}{0.617122in}}{\pgfqpoint{5.174482in}{0.609222in}}{\pgfqpoint{5.180306in}{0.603398in}}%
\pgfpathcurveto{\pgfqpoint{5.186129in}{0.597574in}}{\pgfqpoint{5.194030in}{0.594302in}}{\pgfqpoint{5.202266in}{0.594302in}}%
\pgfpathclose%
\pgfusepath{stroke,fill}%
\end{pgfscope}%
\begin{pgfscope}%
\pgfpathrectangle{\pgfqpoint{3.793912in}{0.557870in}}{\pgfqpoint{2.446088in}{1.484734in}}%
\pgfusepath{clip}%
\pgfsetbuttcap%
\pgfsetroundjoin%
\definecolor{currentfill}{rgb}{0.298039,0.447059,0.690196}%
\pgfsetfillcolor{currentfill}%
\pgfsetlinewidth{1.003750pt}%
\definecolor{currentstroke}{rgb}{0.298039,0.447059,0.690196}%
\pgfsetstrokecolor{currentstroke}%
\pgfsetdash{}{0pt}%
\pgfpathmoveto{\pgfqpoint{3.905098in}{1.930426in}}%
\pgfpathcurveto{\pgfqpoint{3.913334in}{1.930426in}}{\pgfqpoint{3.921234in}{1.933698in}}{\pgfqpoint{3.927058in}{1.939522in}}%
\pgfpathcurveto{\pgfqpoint{3.932882in}{1.945346in}}{\pgfqpoint{3.936155in}{1.953246in}}{\pgfqpoint{3.936155in}{1.961482in}}%
\pgfpathcurveto{\pgfqpoint{3.936155in}{1.969719in}}{\pgfqpoint{3.932882in}{1.977619in}}{\pgfqpoint{3.927058in}{1.983443in}}%
\pgfpathcurveto{\pgfqpoint{3.921234in}{1.989267in}}{\pgfqpoint{3.913334in}{1.992539in}}{\pgfqpoint{3.905098in}{1.992539in}}%
\pgfpathcurveto{\pgfqpoint{3.896862in}{1.992539in}}{\pgfqpoint{3.888962in}{1.989267in}}{\pgfqpoint{3.883138in}{1.983443in}}%
\pgfpathcurveto{\pgfqpoint{3.877314in}{1.977619in}}{\pgfqpoint{3.874042in}{1.969719in}}{\pgfqpoint{3.874042in}{1.961482in}}%
\pgfpathcurveto{\pgfqpoint{3.874042in}{1.953246in}}{\pgfqpoint{3.877314in}{1.945346in}}{\pgfqpoint{3.883138in}{1.939522in}}%
\pgfpathcurveto{\pgfqpoint{3.888962in}{1.933698in}}{\pgfqpoint{3.896862in}{1.930426in}}{\pgfqpoint{3.905098in}{1.930426in}}%
\pgfpathclose%
\pgfusepath{stroke,fill}%
\end{pgfscope}%
\begin{pgfscope}%
\pgfpathrectangle{\pgfqpoint{3.793912in}{0.557870in}}{\pgfqpoint{2.446088in}{1.484734in}}%
\pgfusepath{clip}%
\pgfsetbuttcap%
\pgfsetroundjoin%
\definecolor{currentfill}{rgb}{0.298039,0.447059,0.690196}%
\pgfsetfillcolor{currentfill}%
\pgfsetlinewidth{1.003750pt}%
\definecolor{currentstroke}{rgb}{0.298039,0.447059,0.690196}%
\pgfsetstrokecolor{currentstroke}%
\pgfsetdash{}{0pt}%
\pgfpathmoveto{\pgfqpoint{3.905098in}{1.930426in}}%
\pgfpathcurveto{\pgfqpoint{3.913334in}{1.930426in}}{\pgfqpoint{3.921234in}{1.933698in}}{\pgfqpoint{3.927058in}{1.939522in}}%
\pgfpathcurveto{\pgfqpoint{3.932882in}{1.945346in}}{\pgfqpoint{3.936155in}{1.953246in}}{\pgfqpoint{3.936155in}{1.961482in}}%
\pgfpathcurveto{\pgfqpoint{3.936155in}{1.969719in}}{\pgfqpoint{3.932882in}{1.977619in}}{\pgfqpoint{3.927058in}{1.983443in}}%
\pgfpathcurveto{\pgfqpoint{3.921234in}{1.989267in}}{\pgfqpoint{3.913334in}{1.992539in}}{\pgfqpoint{3.905098in}{1.992539in}}%
\pgfpathcurveto{\pgfqpoint{3.896862in}{1.992539in}}{\pgfqpoint{3.888962in}{1.989267in}}{\pgfqpoint{3.883138in}{1.983443in}}%
\pgfpathcurveto{\pgfqpoint{3.877314in}{1.977619in}}{\pgfqpoint{3.874042in}{1.969719in}}{\pgfqpoint{3.874042in}{1.961482in}}%
\pgfpathcurveto{\pgfqpoint{3.874042in}{1.953246in}}{\pgfqpoint{3.877314in}{1.945346in}}{\pgfqpoint{3.883138in}{1.939522in}}%
\pgfpathcurveto{\pgfqpoint{3.888962in}{1.933698in}}{\pgfqpoint{3.896862in}{1.930426in}}{\pgfqpoint{3.905098in}{1.930426in}}%
\pgfpathclose%
\pgfusepath{stroke,fill}%
\end{pgfscope}%
\begin{pgfscope}%
\pgfpathrectangle{\pgfqpoint{3.793912in}{0.557870in}}{\pgfqpoint{2.446088in}{1.484734in}}%
\pgfusepath{clip}%
\pgfsetbuttcap%
\pgfsetroundjoin%
\definecolor{currentfill}{rgb}{0.298039,0.447059,0.690196}%
\pgfsetfillcolor{currentfill}%
\pgfsetlinewidth{1.003750pt}%
\definecolor{currentstroke}{rgb}{0.298039,0.447059,0.690196}%
\pgfsetstrokecolor{currentstroke}%
\pgfsetdash{}{0pt}%
\pgfpathmoveto{\pgfqpoint{3.997753in}{0.594302in}}%
\pgfpathcurveto{\pgfqpoint{4.005989in}{0.594302in}}{\pgfqpoint{4.013889in}{0.597574in}}{\pgfqpoint{4.019713in}{0.603398in}}%
\pgfpathcurveto{\pgfqpoint{4.025537in}{0.609222in}}{\pgfqpoint{4.028809in}{0.617122in}}{\pgfqpoint{4.028809in}{0.625358in}}%
\pgfpathcurveto{\pgfqpoint{4.028809in}{0.633594in}}{\pgfqpoint{4.025537in}{0.641495in}}{\pgfqpoint{4.019713in}{0.647318in}}%
\pgfpathcurveto{\pgfqpoint{4.013889in}{0.653142in}}{\pgfqpoint{4.005989in}{0.656415in}}{\pgfqpoint{3.997753in}{0.656415in}}%
\pgfpathcurveto{\pgfqpoint{3.989517in}{0.656415in}}{\pgfqpoint{3.981617in}{0.653142in}}{\pgfqpoint{3.975793in}{0.647318in}}%
\pgfpathcurveto{\pgfqpoint{3.969969in}{0.641495in}}{\pgfqpoint{3.966696in}{0.633594in}}{\pgfqpoint{3.966696in}{0.625358in}}%
\pgfpathcurveto{\pgfqpoint{3.966696in}{0.617122in}}{\pgfqpoint{3.969969in}{0.609222in}}{\pgfqpoint{3.975793in}{0.603398in}}%
\pgfpathcurveto{\pgfqpoint{3.981617in}{0.597574in}}{\pgfqpoint{3.989517in}{0.594302in}}{\pgfqpoint{3.997753in}{0.594302in}}%
\pgfpathclose%
\pgfusepath{stroke,fill}%
\end{pgfscope}%
\begin{pgfscope}%
\pgfpathrectangle{\pgfqpoint{3.793912in}{0.557870in}}{\pgfqpoint{2.446088in}{1.484734in}}%
\pgfusepath{clip}%
\pgfsetbuttcap%
\pgfsetroundjoin%
\definecolor{currentfill}{rgb}{0.298039,0.447059,0.690196}%
\pgfsetfillcolor{currentfill}%
\pgfsetlinewidth{1.003750pt}%
\definecolor{currentstroke}{rgb}{0.298039,0.447059,0.690196}%
\pgfsetstrokecolor{currentstroke}%
\pgfsetdash{}{0pt}%
\pgfpathmoveto{\pgfqpoint{3.905098in}{1.930426in}}%
\pgfpathcurveto{\pgfqpoint{3.913334in}{1.930426in}}{\pgfqpoint{3.921234in}{1.933698in}}{\pgfqpoint{3.927058in}{1.939522in}}%
\pgfpathcurveto{\pgfqpoint{3.932882in}{1.945346in}}{\pgfqpoint{3.936155in}{1.953246in}}{\pgfqpoint{3.936155in}{1.961482in}}%
\pgfpathcurveto{\pgfqpoint{3.936155in}{1.969719in}}{\pgfqpoint{3.932882in}{1.977619in}}{\pgfqpoint{3.927058in}{1.983443in}}%
\pgfpathcurveto{\pgfqpoint{3.921234in}{1.989267in}}{\pgfqpoint{3.913334in}{1.992539in}}{\pgfqpoint{3.905098in}{1.992539in}}%
\pgfpathcurveto{\pgfqpoint{3.896862in}{1.992539in}}{\pgfqpoint{3.888962in}{1.989267in}}{\pgfqpoint{3.883138in}{1.983443in}}%
\pgfpathcurveto{\pgfqpoint{3.877314in}{1.977619in}}{\pgfqpoint{3.874042in}{1.969719in}}{\pgfqpoint{3.874042in}{1.961482in}}%
\pgfpathcurveto{\pgfqpoint{3.874042in}{1.953246in}}{\pgfqpoint{3.877314in}{1.945346in}}{\pgfqpoint{3.883138in}{1.939522in}}%
\pgfpathcurveto{\pgfqpoint{3.888962in}{1.933698in}}{\pgfqpoint{3.896862in}{1.930426in}}{\pgfqpoint{3.905098in}{1.930426in}}%
\pgfpathclose%
\pgfusepath{stroke,fill}%
\end{pgfscope}%
\begin{pgfscope}%
\pgfpathrectangle{\pgfqpoint{3.793912in}{0.557870in}}{\pgfqpoint{2.446088in}{1.484734in}}%
\pgfusepath{clip}%
\pgfsetbuttcap%
\pgfsetroundjoin%
\definecolor{currentfill}{rgb}{0.298039,0.447059,0.690196}%
\pgfsetfillcolor{currentfill}%
\pgfsetlinewidth{1.003750pt}%
\definecolor{currentstroke}{rgb}{0.298039,0.447059,0.690196}%
\pgfsetstrokecolor{currentstroke}%
\pgfsetdash{}{0pt}%
\pgfpathmoveto{\pgfqpoint{5.202266in}{0.594302in}}%
\pgfpathcurveto{\pgfqpoint{5.210502in}{0.594302in}}{\pgfqpoint{5.218402in}{0.597574in}}{\pgfqpoint{5.224226in}{0.603398in}}%
\pgfpathcurveto{\pgfqpoint{5.230050in}{0.609222in}}{\pgfqpoint{5.233322in}{0.617122in}}{\pgfqpoint{5.233322in}{0.625358in}}%
\pgfpathcurveto{\pgfqpoint{5.233322in}{0.633594in}}{\pgfqpoint{5.230050in}{0.641495in}}{\pgfqpoint{5.224226in}{0.647318in}}%
\pgfpathcurveto{\pgfqpoint{5.218402in}{0.653142in}}{\pgfqpoint{5.210502in}{0.656415in}}{\pgfqpoint{5.202266in}{0.656415in}}%
\pgfpathcurveto{\pgfqpoint{5.194030in}{0.656415in}}{\pgfqpoint{5.186129in}{0.653142in}}{\pgfqpoint{5.180306in}{0.647318in}}%
\pgfpathcurveto{\pgfqpoint{5.174482in}{0.641495in}}{\pgfqpoint{5.171209in}{0.633594in}}{\pgfqpoint{5.171209in}{0.625358in}}%
\pgfpathcurveto{\pgfqpoint{5.171209in}{0.617122in}}{\pgfqpoint{5.174482in}{0.609222in}}{\pgfqpoint{5.180306in}{0.603398in}}%
\pgfpathcurveto{\pgfqpoint{5.186129in}{0.597574in}}{\pgfqpoint{5.194030in}{0.594302in}}{\pgfqpoint{5.202266in}{0.594302in}}%
\pgfpathclose%
\pgfusepath{stroke,fill}%
\end{pgfscope}%
\begin{pgfscope}%
\pgfpathrectangle{\pgfqpoint{3.793912in}{0.557870in}}{\pgfqpoint{2.446088in}{1.484734in}}%
\pgfusepath{clip}%
\pgfsetbuttcap%
\pgfsetroundjoin%
\definecolor{currentfill}{rgb}{0.298039,0.447059,0.690196}%
\pgfsetfillcolor{currentfill}%
\pgfsetlinewidth{1.003750pt}%
\definecolor{currentstroke}{rgb}{0.298039,0.447059,0.690196}%
\pgfsetstrokecolor{currentstroke}%
\pgfsetdash{}{0pt}%
\pgfpathmoveto{\pgfqpoint{3.905098in}{0.621570in}}%
\pgfpathcurveto{\pgfqpoint{3.913334in}{0.621570in}}{\pgfqpoint{3.921234in}{0.624842in}}{\pgfqpoint{3.927058in}{0.630666in}}%
\pgfpathcurveto{\pgfqpoint{3.932882in}{0.636490in}}{\pgfqpoint{3.936155in}{0.644390in}}{\pgfqpoint{3.936155in}{0.652626in}}%
\pgfpathcurveto{\pgfqpoint{3.936155in}{0.660862in}}{\pgfqpoint{3.932882in}{0.668762in}}{\pgfqpoint{3.927058in}{0.674586in}}%
\pgfpathcurveto{\pgfqpoint{3.921234in}{0.680410in}}{\pgfqpoint{3.913334in}{0.683683in}}{\pgfqpoint{3.905098in}{0.683683in}}%
\pgfpathcurveto{\pgfqpoint{3.896862in}{0.683683in}}{\pgfqpoint{3.888962in}{0.680410in}}{\pgfqpoint{3.883138in}{0.674586in}}%
\pgfpathcurveto{\pgfqpoint{3.877314in}{0.668762in}}{\pgfqpoint{3.874042in}{0.660862in}}{\pgfqpoint{3.874042in}{0.652626in}}%
\pgfpathcurveto{\pgfqpoint{3.874042in}{0.644390in}}{\pgfqpoint{3.877314in}{0.636490in}}{\pgfqpoint{3.883138in}{0.630666in}}%
\pgfpathcurveto{\pgfqpoint{3.888962in}{0.624842in}}{\pgfqpoint{3.896862in}{0.621570in}}{\pgfqpoint{3.905098in}{0.621570in}}%
\pgfpathclose%
\pgfusepath{stroke,fill}%
\end{pgfscope}%
\begin{pgfscope}%
\pgfpathrectangle{\pgfqpoint{3.793912in}{0.557870in}}{\pgfqpoint{2.446088in}{1.484734in}}%
\pgfusepath{clip}%
\pgfsetbuttcap%
\pgfsetroundjoin%
\definecolor{currentfill}{rgb}{0.298039,0.447059,0.690196}%
\pgfsetfillcolor{currentfill}%
\pgfsetlinewidth{1.003750pt}%
\definecolor{currentstroke}{rgb}{0.298039,0.447059,0.690196}%
\pgfsetstrokecolor{currentstroke}%
\pgfsetdash{}{0pt}%
\pgfpathmoveto{\pgfqpoint{5.318084in}{0.594302in}}%
\pgfpathcurveto{\pgfqpoint{5.326321in}{0.594302in}}{\pgfqpoint{5.334221in}{0.597574in}}{\pgfqpoint{5.340045in}{0.603398in}}%
\pgfpathcurveto{\pgfqpoint{5.345869in}{0.609222in}}{\pgfqpoint{5.349141in}{0.617122in}}{\pgfqpoint{5.349141in}{0.625358in}}%
\pgfpathcurveto{\pgfqpoint{5.349141in}{0.633594in}}{\pgfqpoint{5.345869in}{0.641495in}}{\pgfqpoint{5.340045in}{0.647318in}}%
\pgfpathcurveto{\pgfqpoint{5.334221in}{0.653142in}}{\pgfqpoint{5.326321in}{0.656415in}}{\pgfqpoint{5.318084in}{0.656415in}}%
\pgfpathcurveto{\pgfqpoint{5.309848in}{0.656415in}}{\pgfqpoint{5.301948in}{0.653142in}}{\pgfqpoint{5.296124in}{0.647318in}}%
\pgfpathcurveto{\pgfqpoint{5.290300in}{0.641495in}}{\pgfqpoint{5.287028in}{0.633594in}}{\pgfqpoint{5.287028in}{0.625358in}}%
\pgfpathcurveto{\pgfqpoint{5.287028in}{0.617122in}}{\pgfqpoint{5.290300in}{0.609222in}}{\pgfqpoint{5.296124in}{0.603398in}}%
\pgfpathcurveto{\pgfqpoint{5.301948in}{0.597574in}}{\pgfqpoint{5.309848in}{0.594302in}}{\pgfqpoint{5.318084in}{0.594302in}}%
\pgfpathclose%
\pgfusepath{stroke,fill}%
\end{pgfscope}%
\begin{pgfscope}%
\pgfpathrectangle{\pgfqpoint{3.793912in}{0.557870in}}{\pgfqpoint{2.446088in}{1.484734in}}%
\pgfusepath{clip}%
\pgfsetbuttcap%
\pgfsetroundjoin%
\definecolor{currentfill}{rgb}{0.298039,0.447059,0.690196}%
\pgfsetfillcolor{currentfill}%
\pgfsetlinewidth{1.003750pt}%
\definecolor{currentstroke}{rgb}{0.298039,0.447059,0.690196}%
\pgfsetstrokecolor{currentstroke}%
\pgfsetdash{}{0pt}%
\pgfpathmoveto{\pgfqpoint{3.905098in}{1.930426in}}%
\pgfpathcurveto{\pgfqpoint{3.913334in}{1.930426in}}{\pgfqpoint{3.921234in}{1.933698in}}{\pgfqpoint{3.927058in}{1.939522in}}%
\pgfpathcurveto{\pgfqpoint{3.932882in}{1.945346in}}{\pgfqpoint{3.936155in}{1.953246in}}{\pgfqpoint{3.936155in}{1.961482in}}%
\pgfpathcurveto{\pgfqpoint{3.936155in}{1.969719in}}{\pgfqpoint{3.932882in}{1.977619in}}{\pgfqpoint{3.927058in}{1.983443in}}%
\pgfpathcurveto{\pgfqpoint{3.921234in}{1.989267in}}{\pgfqpoint{3.913334in}{1.992539in}}{\pgfqpoint{3.905098in}{1.992539in}}%
\pgfpathcurveto{\pgfqpoint{3.896862in}{1.992539in}}{\pgfqpoint{3.888962in}{1.989267in}}{\pgfqpoint{3.883138in}{1.983443in}}%
\pgfpathcurveto{\pgfqpoint{3.877314in}{1.977619in}}{\pgfqpoint{3.874042in}{1.969719in}}{\pgfqpoint{3.874042in}{1.961482in}}%
\pgfpathcurveto{\pgfqpoint{3.874042in}{1.953246in}}{\pgfqpoint{3.877314in}{1.945346in}}{\pgfqpoint{3.883138in}{1.939522in}}%
\pgfpathcurveto{\pgfqpoint{3.888962in}{1.933698in}}{\pgfqpoint{3.896862in}{1.930426in}}{\pgfqpoint{3.905098in}{1.930426in}}%
\pgfpathclose%
\pgfusepath{stroke,fill}%
\end{pgfscope}%
\begin{pgfscope}%
\pgfpathrectangle{\pgfqpoint{3.793912in}{0.557870in}}{\pgfqpoint{2.446088in}{1.484734in}}%
\pgfusepath{clip}%
\pgfsetbuttcap%
\pgfsetroundjoin%
\definecolor{currentfill}{rgb}{0.298039,0.447059,0.690196}%
\pgfsetfillcolor{currentfill}%
\pgfsetlinewidth{1.003750pt}%
\definecolor{currentstroke}{rgb}{0.298039,0.447059,0.690196}%
\pgfsetstrokecolor{currentstroke}%
\pgfsetdash{}{0pt}%
\pgfpathmoveto{\pgfqpoint{3.905098in}{1.930426in}}%
\pgfpathcurveto{\pgfqpoint{3.913334in}{1.930426in}}{\pgfqpoint{3.921234in}{1.933698in}}{\pgfqpoint{3.927058in}{1.939522in}}%
\pgfpathcurveto{\pgfqpoint{3.932882in}{1.945346in}}{\pgfqpoint{3.936155in}{1.953246in}}{\pgfqpoint{3.936155in}{1.961482in}}%
\pgfpathcurveto{\pgfqpoint{3.936155in}{1.969719in}}{\pgfqpoint{3.932882in}{1.977619in}}{\pgfqpoint{3.927058in}{1.983443in}}%
\pgfpathcurveto{\pgfqpoint{3.921234in}{1.989267in}}{\pgfqpoint{3.913334in}{1.992539in}}{\pgfqpoint{3.905098in}{1.992539in}}%
\pgfpathcurveto{\pgfqpoint{3.896862in}{1.992539in}}{\pgfqpoint{3.888962in}{1.989267in}}{\pgfqpoint{3.883138in}{1.983443in}}%
\pgfpathcurveto{\pgfqpoint{3.877314in}{1.977619in}}{\pgfqpoint{3.874042in}{1.969719in}}{\pgfqpoint{3.874042in}{1.961482in}}%
\pgfpathcurveto{\pgfqpoint{3.874042in}{1.953246in}}{\pgfqpoint{3.877314in}{1.945346in}}{\pgfqpoint{3.883138in}{1.939522in}}%
\pgfpathcurveto{\pgfqpoint{3.888962in}{1.933698in}}{\pgfqpoint{3.896862in}{1.930426in}}{\pgfqpoint{3.905098in}{1.930426in}}%
\pgfpathclose%
\pgfusepath{stroke,fill}%
\end{pgfscope}%
\begin{pgfscope}%
\pgfpathrectangle{\pgfqpoint{3.793912in}{0.557870in}}{\pgfqpoint{2.446088in}{1.484734in}}%
\pgfusepath{clip}%
\pgfsetbuttcap%
\pgfsetroundjoin%
\definecolor{currentfill}{rgb}{0.298039,0.447059,0.690196}%
\pgfsetfillcolor{currentfill}%
\pgfsetlinewidth{1.003750pt}%
\definecolor{currentstroke}{rgb}{0.298039,0.447059,0.690196}%
\pgfsetstrokecolor{currentstroke}%
\pgfsetdash{}{0pt}%
\pgfpathmoveto{\pgfqpoint{3.905098in}{0.621570in}}%
\pgfpathcurveto{\pgfqpoint{3.913334in}{0.621570in}}{\pgfqpoint{3.921234in}{0.624842in}}{\pgfqpoint{3.927058in}{0.630666in}}%
\pgfpathcurveto{\pgfqpoint{3.932882in}{0.636490in}}{\pgfqpoint{3.936155in}{0.644390in}}{\pgfqpoint{3.936155in}{0.652626in}}%
\pgfpathcurveto{\pgfqpoint{3.936155in}{0.660862in}}{\pgfqpoint{3.932882in}{0.668762in}}{\pgfqpoint{3.927058in}{0.674586in}}%
\pgfpathcurveto{\pgfqpoint{3.921234in}{0.680410in}}{\pgfqpoint{3.913334in}{0.683683in}}{\pgfqpoint{3.905098in}{0.683683in}}%
\pgfpathcurveto{\pgfqpoint{3.896862in}{0.683683in}}{\pgfqpoint{3.888962in}{0.680410in}}{\pgfqpoint{3.883138in}{0.674586in}}%
\pgfpathcurveto{\pgfqpoint{3.877314in}{0.668762in}}{\pgfqpoint{3.874042in}{0.660862in}}{\pgfqpoint{3.874042in}{0.652626in}}%
\pgfpathcurveto{\pgfqpoint{3.874042in}{0.644390in}}{\pgfqpoint{3.877314in}{0.636490in}}{\pgfqpoint{3.883138in}{0.630666in}}%
\pgfpathcurveto{\pgfqpoint{3.888962in}{0.624842in}}{\pgfqpoint{3.896862in}{0.621570in}}{\pgfqpoint{3.905098in}{0.621570in}}%
\pgfpathclose%
\pgfusepath{stroke,fill}%
\end{pgfscope}%
\begin{pgfscope}%
\pgfpathrectangle{\pgfqpoint{3.793912in}{0.557870in}}{\pgfqpoint{2.446088in}{1.484734in}}%
\pgfusepath{clip}%
\pgfsetbuttcap%
\pgfsetroundjoin%
\definecolor{currentfill}{rgb}{0.298039,0.447059,0.690196}%
\pgfsetfillcolor{currentfill}%
\pgfsetlinewidth{1.003750pt}%
\definecolor{currentstroke}{rgb}{0.298039,0.447059,0.690196}%
\pgfsetstrokecolor{currentstroke}%
\pgfsetdash{}{0pt}%
\pgfpathmoveto{\pgfqpoint{3.905098in}{1.930426in}}%
\pgfpathcurveto{\pgfqpoint{3.913334in}{1.930426in}}{\pgfqpoint{3.921234in}{1.933698in}}{\pgfqpoint{3.927058in}{1.939522in}}%
\pgfpathcurveto{\pgfqpoint{3.932882in}{1.945346in}}{\pgfqpoint{3.936155in}{1.953246in}}{\pgfqpoint{3.936155in}{1.961482in}}%
\pgfpathcurveto{\pgfqpoint{3.936155in}{1.969719in}}{\pgfqpoint{3.932882in}{1.977619in}}{\pgfqpoint{3.927058in}{1.983443in}}%
\pgfpathcurveto{\pgfqpoint{3.921234in}{1.989267in}}{\pgfqpoint{3.913334in}{1.992539in}}{\pgfqpoint{3.905098in}{1.992539in}}%
\pgfpathcurveto{\pgfqpoint{3.896862in}{1.992539in}}{\pgfqpoint{3.888962in}{1.989267in}}{\pgfqpoint{3.883138in}{1.983443in}}%
\pgfpathcurveto{\pgfqpoint{3.877314in}{1.977619in}}{\pgfqpoint{3.874042in}{1.969719in}}{\pgfqpoint{3.874042in}{1.961482in}}%
\pgfpathcurveto{\pgfqpoint{3.874042in}{1.953246in}}{\pgfqpoint{3.877314in}{1.945346in}}{\pgfqpoint{3.883138in}{1.939522in}}%
\pgfpathcurveto{\pgfqpoint{3.888962in}{1.933698in}}{\pgfqpoint{3.896862in}{1.930426in}}{\pgfqpoint{3.905098in}{1.930426in}}%
\pgfpathclose%
\pgfusepath{stroke,fill}%
\end{pgfscope}%
\begin{pgfscope}%
\pgfpathrectangle{\pgfqpoint{3.793912in}{0.557870in}}{\pgfqpoint{2.446088in}{1.484734in}}%
\pgfusepath{clip}%
\pgfsetbuttcap%
\pgfsetroundjoin%
\definecolor{currentfill}{rgb}{0.298039,0.447059,0.690196}%
\pgfsetfillcolor{currentfill}%
\pgfsetlinewidth{1.003750pt}%
\definecolor{currentstroke}{rgb}{0.298039,0.447059,0.690196}%
\pgfsetstrokecolor{currentstroke}%
\pgfsetdash{}{0pt}%
\pgfpathmoveto{\pgfqpoint{3.905098in}{1.398703in}}%
\pgfpathcurveto{\pgfqpoint{3.913334in}{1.398703in}}{\pgfqpoint{3.921234in}{1.401975in}}{\pgfqpoint{3.927058in}{1.407799in}}%
\pgfpathcurveto{\pgfqpoint{3.932882in}{1.413623in}}{\pgfqpoint{3.936155in}{1.421523in}}{\pgfqpoint{3.936155in}{1.429759in}}%
\pgfpathcurveto{\pgfqpoint{3.936155in}{1.437996in}}{\pgfqpoint{3.932882in}{1.445896in}}{\pgfqpoint{3.927058in}{1.451720in}}%
\pgfpathcurveto{\pgfqpoint{3.921234in}{1.457544in}}{\pgfqpoint{3.913334in}{1.460816in}}{\pgfqpoint{3.905098in}{1.460816in}}%
\pgfpathcurveto{\pgfqpoint{3.896862in}{1.460816in}}{\pgfqpoint{3.888962in}{1.457544in}}{\pgfqpoint{3.883138in}{1.451720in}}%
\pgfpathcurveto{\pgfqpoint{3.877314in}{1.445896in}}{\pgfqpoint{3.874042in}{1.437996in}}{\pgfqpoint{3.874042in}{1.429759in}}%
\pgfpathcurveto{\pgfqpoint{3.874042in}{1.421523in}}{\pgfqpoint{3.877314in}{1.413623in}}{\pgfqpoint{3.883138in}{1.407799in}}%
\pgfpathcurveto{\pgfqpoint{3.888962in}{1.401975in}}{\pgfqpoint{3.896862in}{1.398703in}}{\pgfqpoint{3.905098in}{1.398703in}}%
\pgfpathclose%
\pgfusepath{stroke,fill}%
\end{pgfscope}%
\begin{pgfscope}%
\pgfpathrectangle{\pgfqpoint{3.793912in}{0.557870in}}{\pgfqpoint{2.446088in}{1.484734in}}%
\pgfusepath{clip}%
\pgfsetbuttcap%
\pgfsetroundjoin%
\definecolor{currentfill}{rgb}{0.298039,0.447059,0.690196}%
\pgfsetfillcolor{currentfill}%
\pgfsetlinewidth{1.003750pt}%
\definecolor{currentstroke}{rgb}{0.298039,0.447059,0.690196}%
\pgfsetstrokecolor{currentstroke}%
\pgfsetdash{}{0pt}%
\pgfpathmoveto{\pgfqpoint{3.905098in}{1.330533in}}%
\pgfpathcurveto{\pgfqpoint{3.913334in}{1.330533in}}{\pgfqpoint{3.921234in}{1.333806in}}{\pgfqpoint{3.927058in}{1.339630in}}%
\pgfpathcurveto{\pgfqpoint{3.932882in}{1.345454in}}{\pgfqpoint{3.936155in}{1.353354in}}{\pgfqpoint{3.936155in}{1.361590in}}%
\pgfpathcurveto{\pgfqpoint{3.936155in}{1.369826in}}{\pgfqpoint{3.932882in}{1.377726in}}{\pgfqpoint{3.927058in}{1.383550in}}%
\pgfpathcurveto{\pgfqpoint{3.921234in}{1.389374in}}{\pgfqpoint{3.913334in}{1.392646in}}{\pgfqpoint{3.905098in}{1.392646in}}%
\pgfpathcurveto{\pgfqpoint{3.896862in}{1.392646in}}{\pgfqpoint{3.888962in}{1.389374in}}{\pgfqpoint{3.883138in}{1.383550in}}%
\pgfpathcurveto{\pgfqpoint{3.877314in}{1.377726in}}{\pgfqpoint{3.874042in}{1.369826in}}{\pgfqpoint{3.874042in}{1.361590in}}%
\pgfpathcurveto{\pgfqpoint{3.874042in}{1.353354in}}{\pgfqpoint{3.877314in}{1.345454in}}{\pgfqpoint{3.883138in}{1.339630in}}%
\pgfpathcurveto{\pgfqpoint{3.888962in}{1.333806in}}{\pgfqpoint{3.896862in}{1.330533in}}{\pgfqpoint{3.905098in}{1.330533in}}%
\pgfpathclose%
\pgfusepath{stroke,fill}%
\end{pgfscope}%
\begin{pgfscope}%
\pgfpathrectangle{\pgfqpoint{3.793912in}{0.557870in}}{\pgfqpoint{2.446088in}{1.484734in}}%
\pgfusepath{clip}%
\pgfsetbuttcap%
\pgfsetroundjoin%
\definecolor{currentfill}{rgb}{0.298039,0.447059,0.690196}%
\pgfsetfillcolor{currentfill}%
\pgfsetlinewidth{1.003750pt}%
\definecolor{currentstroke}{rgb}{0.298039,0.447059,0.690196}%
\pgfsetstrokecolor{currentstroke}%
\pgfsetdash{}{0pt}%
\pgfpathmoveto{\pgfqpoint{3.905098in}{1.398703in}}%
\pgfpathcurveto{\pgfqpoint{3.913334in}{1.398703in}}{\pgfqpoint{3.921234in}{1.401975in}}{\pgfqpoint{3.927058in}{1.407799in}}%
\pgfpathcurveto{\pgfqpoint{3.932882in}{1.413623in}}{\pgfqpoint{3.936155in}{1.421523in}}{\pgfqpoint{3.936155in}{1.429759in}}%
\pgfpathcurveto{\pgfqpoint{3.936155in}{1.437996in}}{\pgfqpoint{3.932882in}{1.445896in}}{\pgfqpoint{3.927058in}{1.451720in}}%
\pgfpathcurveto{\pgfqpoint{3.921234in}{1.457544in}}{\pgfqpoint{3.913334in}{1.460816in}}{\pgfqpoint{3.905098in}{1.460816in}}%
\pgfpathcurveto{\pgfqpoint{3.896862in}{1.460816in}}{\pgfqpoint{3.888962in}{1.457544in}}{\pgfqpoint{3.883138in}{1.451720in}}%
\pgfpathcurveto{\pgfqpoint{3.877314in}{1.445896in}}{\pgfqpoint{3.874042in}{1.437996in}}{\pgfqpoint{3.874042in}{1.429759in}}%
\pgfpathcurveto{\pgfqpoint{3.874042in}{1.421523in}}{\pgfqpoint{3.877314in}{1.413623in}}{\pgfqpoint{3.883138in}{1.407799in}}%
\pgfpathcurveto{\pgfqpoint{3.888962in}{1.401975in}}{\pgfqpoint{3.896862in}{1.398703in}}{\pgfqpoint{3.905098in}{1.398703in}}%
\pgfpathclose%
\pgfusepath{stroke,fill}%
\end{pgfscope}%
\begin{pgfscope}%
\pgfpathrectangle{\pgfqpoint{3.793912in}{0.557870in}}{\pgfqpoint{2.446088in}{1.484734in}}%
\pgfusepath{clip}%
\pgfsetbuttcap%
\pgfsetroundjoin%
\definecolor{currentfill}{rgb}{0.298039,0.447059,0.690196}%
\pgfsetfillcolor{currentfill}%
\pgfsetlinewidth{1.003750pt}%
\definecolor{currentstroke}{rgb}{0.298039,0.447059,0.690196}%
\pgfsetstrokecolor{currentstroke}%
\pgfsetdash{}{0pt}%
\pgfpathmoveto{\pgfqpoint{3.951425in}{0.594302in}}%
\pgfpathcurveto{\pgfqpoint{3.959662in}{0.594302in}}{\pgfqpoint{3.967562in}{0.597574in}}{\pgfqpoint{3.973386in}{0.603398in}}%
\pgfpathcurveto{\pgfqpoint{3.979210in}{0.609222in}}{\pgfqpoint{3.982482in}{0.617122in}}{\pgfqpoint{3.982482in}{0.625358in}}%
\pgfpathcurveto{\pgfqpoint{3.982482in}{0.633594in}}{\pgfqpoint{3.979210in}{0.641495in}}{\pgfqpoint{3.973386in}{0.647318in}}%
\pgfpathcurveto{\pgfqpoint{3.967562in}{0.653142in}}{\pgfqpoint{3.959662in}{0.656415in}}{\pgfqpoint{3.951425in}{0.656415in}}%
\pgfpathcurveto{\pgfqpoint{3.943189in}{0.656415in}}{\pgfqpoint{3.935289in}{0.653142in}}{\pgfqpoint{3.929465in}{0.647318in}}%
\pgfpathcurveto{\pgfqpoint{3.923641in}{0.641495in}}{\pgfqpoint{3.920369in}{0.633594in}}{\pgfqpoint{3.920369in}{0.625358in}}%
\pgfpathcurveto{\pgfqpoint{3.920369in}{0.617122in}}{\pgfqpoint{3.923641in}{0.609222in}}{\pgfqpoint{3.929465in}{0.603398in}}%
\pgfpathcurveto{\pgfqpoint{3.935289in}{0.597574in}}{\pgfqpoint{3.943189in}{0.594302in}}{\pgfqpoint{3.951425in}{0.594302in}}%
\pgfpathclose%
\pgfusepath{stroke,fill}%
\end{pgfscope}%
\begin{pgfscope}%
\pgfpathrectangle{\pgfqpoint{3.793912in}{0.557870in}}{\pgfqpoint{2.446088in}{1.484734in}}%
\pgfusepath{clip}%
\pgfsetbuttcap%
\pgfsetroundjoin%
\definecolor{currentfill}{rgb}{0.298039,0.447059,0.690196}%
\pgfsetfillcolor{currentfill}%
\pgfsetlinewidth{1.003750pt}%
\definecolor{currentstroke}{rgb}{0.298039,0.447059,0.690196}%
\pgfsetstrokecolor{currentstroke}%
\pgfsetdash{}{0pt}%
\pgfpathmoveto{\pgfqpoint{3.905098in}{1.385069in}}%
\pgfpathcurveto{\pgfqpoint{3.913334in}{1.385069in}}{\pgfqpoint{3.921234in}{1.388341in}}{\pgfqpoint{3.927058in}{1.394165in}}%
\pgfpathcurveto{\pgfqpoint{3.932882in}{1.399989in}}{\pgfqpoint{3.936155in}{1.407889in}}{\pgfqpoint{3.936155in}{1.416126in}}%
\pgfpathcurveto{\pgfqpoint{3.936155in}{1.424362in}}{\pgfqpoint{3.932882in}{1.432262in}}{\pgfqpoint{3.927058in}{1.438086in}}%
\pgfpathcurveto{\pgfqpoint{3.921234in}{1.443910in}}{\pgfqpoint{3.913334in}{1.447182in}}{\pgfqpoint{3.905098in}{1.447182in}}%
\pgfpathcurveto{\pgfqpoint{3.896862in}{1.447182in}}{\pgfqpoint{3.888962in}{1.443910in}}{\pgfqpoint{3.883138in}{1.438086in}}%
\pgfpathcurveto{\pgfqpoint{3.877314in}{1.432262in}}{\pgfqpoint{3.874042in}{1.424362in}}{\pgfqpoint{3.874042in}{1.416126in}}%
\pgfpathcurveto{\pgfqpoint{3.874042in}{1.407889in}}{\pgfqpoint{3.877314in}{1.399989in}}{\pgfqpoint{3.883138in}{1.394165in}}%
\pgfpathcurveto{\pgfqpoint{3.888962in}{1.388341in}}{\pgfqpoint{3.896862in}{1.385069in}}{\pgfqpoint{3.905098in}{1.385069in}}%
\pgfpathclose%
\pgfusepath{stroke,fill}%
\end{pgfscope}%
\begin{pgfscope}%
\pgfpathrectangle{\pgfqpoint{3.793912in}{0.557870in}}{\pgfqpoint{2.446088in}{1.484734in}}%
\pgfusepath{clip}%
\pgfsetbuttcap%
\pgfsetroundjoin%
\definecolor{currentfill}{rgb}{0.298039,0.447059,0.690196}%
\pgfsetfillcolor{currentfill}%
\pgfsetlinewidth{1.003750pt}%
\definecolor{currentstroke}{rgb}{0.298039,0.447059,0.690196}%
\pgfsetstrokecolor{currentstroke}%
\pgfsetdash{}{0pt}%
\pgfpathmoveto{\pgfqpoint{3.905098in}{1.930426in}}%
\pgfpathcurveto{\pgfqpoint{3.913334in}{1.930426in}}{\pgfqpoint{3.921234in}{1.933698in}}{\pgfqpoint{3.927058in}{1.939522in}}%
\pgfpathcurveto{\pgfqpoint{3.932882in}{1.945346in}}{\pgfqpoint{3.936155in}{1.953246in}}{\pgfqpoint{3.936155in}{1.961482in}}%
\pgfpathcurveto{\pgfqpoint{3.936155in}{1.969719in}}{\pgfqpoint{3.932882in}{1.977619in}}{\pgfqpoint{3.927058in}{1.983443in}}%
\pgfpathcurveto{\pgfqpoint{3.921234in}{1.989267in}}{\pgfqpoint{3.913334in}{1.992539in}}{\pgfqpoint{3.905098in}{1.992539in}}%
\pgfpathcurveto{\pgfqpoint{3.896862in}{1.992539in}}{\pgfqpoint{3.888962in}{1.989267in}}{\pgfqpoint{3.883138in}{1.983443in}}%
\pgfpathcurveto{\pgfqpoint{3.877314in}{1.977619in}}{\pgfqpoint{3.874042in}{1.969719in}}{\pgfqpoint{3.874042in}{1.961482in}}%
\pgfpathcurveto{\pgfqpoint{3.874042in}{1.953246in}}{\pgfqpoint{3.877314in}{1.945346in}}{\pgfqpoint{3.883138in}{1.939522in}}%
\pgfpathcurveto{\pgfqpoint{3.888962in}{1.933698in}}{\pgfqpoint{3.896862in}{1.930426in}}{\pgfqpoint{3.905098in}{1.930426in}}%
\pgfpathclose%
\pgfusepath{stroke,fill}%
\end{pgfscope}%
\begin{pgfscope}%
\pgfpathrectangle{\pgfqpoint{3.793912in}{0.557870in}}{\pgfqpoint{2.446088in}{1.484734in}}%
\pgfusepath{clip}%
\pgfsetbuttcap%
\pgfsetroundjoin%
\definecolor{currentfill}{rgb}{0.298039,0.447059,0.690196}%
\pgfsetfillcolor{currentfill}%
\pgfsetlinewidth{1.003750pt}%
\definecolor{currentstroke}{rgb}{0.298039,0.447059,0.690196}%
\pgfsetstrokecolor{currentstroke}%
\pgfsetdash{}{0pt}%
\pgfpathmoveto{\pgfqpoint{3.905098in}{1.466873in}}%
\pgfpathcurveto{\pgfqpoint{3.913334in}{1.466873in}}{\pgfqpoint{3.921234in}{1.470145in}}{\pgfqpoint{3.927058in}{1.475969in}}%
\pgfpathcurveto{\pgfqpoint{3.932882in}{1.481793in}}{\pgfqpoint{3.936155in}{1.489693in}}{\pgfqpoint{3.936155in}{1.497929in}}%
\pgfpathcurveto{\pgfqpoint{3.936155in}{1.506165in}}{\pgfqpoint{3.932882in}{1.514065in}}{\pgfqpoint{3.927058in}{1.519889in}}%
\pgfpathcurveto{\pgfqpoint{3.921234in}{1.525713in}}{\pgfqpoint{3.913334in}{1.528986in}}{\pgfqpoint{3.905098in}{1.528986in}}%
\pgfpathcurveto{\pgfqpoint{3.896862in}{1.528986in}}{\pgfqpoint{3.888962in}{1.525713in}}{\pgfqpoint{3.883138in}{1.519889in}}%
\pgfpathcurveto{\pgfqpoint{3.877314in}{1.514065in}}{\pgfqpoint{3.874042in}{1.506165in}}{\pgfqpoint{3.874042in}{1.497929in}}%
\pgfpathcurveto{\pgfqpoint{3.874042in}{1.489693in}}{\pgfqpoint{3.877314in}{1.481793in}}{\pgfqpoint{3.883138in}{1.475969in}}%
\pgfpathcurveto{\pgfqpoint{3.888962in}{1.470145in}}{\pgfqpoint{3.896862in}{1.466873in}}{\pgfqpoint{3.905098in}{1.466873in}}%
\pgfpathclose%
\pgfusepath{stroke,fill}%
\end{pgfscope}%
\begin{pgfscope}%
\pgfpathrectangle{\pgfqpoint{3.793912in}{0.557870in}}{\pgfqpoint{2.446088in}{1.484734in}}%
\pgfusepath{clip}%
\pgfsetbuttcap%
\pgfsetroundjoin%
\definecolor{currentfill}{rgb}{0.298039,0.447059,0.690196}%
\pgfsetfillcolor{currentfill}%
\pgfsetlinewidth{1.003750pt}%
\definecolor{currentstroke}{rgb}{0.298039,0.447059,0.690196}%
\pgfsetstrokecolor{currentstroke}%
\pgfsetdash{}{0pt}%
\pgfpathmoveto{\pgfqpoint{3.905098in}{1.930426in}}%
\pgfpathcurveto{\pgfqpoint{3.913334in}{1.930426in}}{\pgfqpoint{3.921234in}{1.933698in}}{\pgfqpoint{3.927058in}{1.939522in}}%
\pgfpathcurveto{\pgfqpoint{3.932882in}{1.945346in}}{\pgfqpoint{3.936155in}{1.953246in}}{\pgfqpoint{3.936155in}{1.961482in}}%
\pgfpathcurveto{\pgfqpoint{3.936155in}{1.969719in}}{\pgfqpoint{3.932882in}{1.977619in}}{\pgfqpoint{3.927058in}{1.983443in}}%
\pgfpathcurveto{\pgfqpoint{3.921234in}{1.989267in}}{\pgfqpoint{3.913334in}{1.992539in}}{\pgfqpoint{3.905098in}{1.992539in}}%
\pgfpathcurveto{\pgfqpoint{3.896862in}{1.992539in}}{\pgfqpoint{3.888962in}{1.989267in}}{\pgfqpoint{3.883138in}{1.983443in}}%
\pgfpathcurveto{\pgfqpoint{3.877314in}{1.977619in}}{\pgfqpoint{3.874042in}{1.969719in}}{\pgfqpoint{3.874042in}{1.961482in}}%
\pgfpathcurveto{\pgfqpoint{3.874042in}{1.953246in}}{\pgfqpoint{3.877314in}{1.945346in}}{\pgfqpoint{3.883138in}{1.939522in}}%
\pgfpathcurveto{\pgfqpoint{3.888962in}{1.933698in}}{\pgfqpoint{3.896862in}{1.930426in}}{\pgfqpoint{3.905098in}{1.930426in}}%
\pgfpathclose%
\pgfusepath{stroke,fill}%
\end{pgfscope}%
\begin{pgfscope}%
\pgfpathrectangle{\pgfqpoint{3.793912in}{0.557870in}}{\pgfqpoint{2.446088in}{1.484734in}}%
\pgfusepath{clip}%
\pgfsetbuttcap%
\pgfsetroundjoin%
\definecolor{currentfill}{rgb}{0.298039,0.447059,0.690196}%
\pgfsetfillcolor{currentfill}%
\pgfsetlinewidth{1.003750pt}%
\definecolor{currentstroke}{rgb}{0.298039,0.447059,0.690196}%
\pgfsetstrokecolor{currentstroke}%
\pgfsetdash{}{0pt}%
\pgfpathmoveto{\pgfqpoint{3.905098in}{1.303266in}}%
\pgfpathcurveto{\pgfqpoint{3.913334in}{1.303266in}}{\pgfqpoint{3.921234in}{1.306538in}}{\pgfqpoint{3.927058in}{1.312362in}}%
\pgfpathcurveto{\pgfqpoint{3.932882in}{1.318186in}}{\pgfqpoint{3.936155in}{1.326086in}}{\pgfqpoint{3.936155in}{1.334322in}}%
\pgfpathcurveto{\pgfqpoint{3.936155in}{1.342558in}}{\pgfqpoint{3.932882in}{1.350458in}}{\pgfqpoint{3.927058in}{1.356282in}}%
\pgfpathcurveto{\pgfqpoint{3.921234in}{1.362106in}}{\pgfqpoint{3.913334in}{1.365379in}}{\pgfqpoint{3.905098in}{1.365379in}}%
\pgfpathcurveto{\pgfqpoint{3.896862in}{1.365379in}}{\pgfqpoint{3.888962in}{1.362106in}}{\pgfqpoint{3.883138in}{1.356282in}}%
\pgfpathcurveto{\pgfqpoint{3.877314in}{1.350458in}}{\pgfqpoint{3.874042in}{1.342558in}}{\pgfqpoint{3.874042in}{1.334322in}}%
\pgfpathcurveto{\pgfqpoint{3.874042in}{1.326086in}}{\pgfqpoint{3.877314in}{1.318186in}}{\pgfqpoint{3.883138in}{1.312362in}}%
\pgfpathcurveto{\pgfqpoint{3.888962in}{1.306538in}}{\pgfqpoint{3.896862in}{1.303266in}}{\pgfqpoint{3.905098in}{1.303266in}}%
\pgfpathclose%
\pgfusepath{stroke,fill}%
\end{pgfscope}%
\begin{pgfscope}%
\pgfpathrectangle{\pgfqpoint{3.793912in}{0.557870in}}{\pgfqpoint{2.446088in}{1.484734in}}%
\pgfusepath{clip}%
\pgfsetbuttcap%
\pgfsetroundjoin%
\definecolor{currentfill}{rgb}{0.298039,0.447059,0.690196}%
\pgfsetfillcolor{currentfill}%
\pgfsetlinewidth{1.003750pt}%
\definecolor{currentstroke}{rgb}{0.298039,0.447059,0.690196}%
\pgfsetstrokecolor{currentstroke}%
\pgfsetdash{}{0pt}%
\pgfpathmoveto{\pgfqpoint{3.905098in}{1.930426in}}%
\pgfpathcurveto{\pgfqpoint{3.913334in}{1.930426in}}{\pgfqpoint{3.921234in}{1.933698in}}{\pgfqpoint{3.927058in}{1.939522in}}%
\pgfpathcurveto{\pgfqpoint{3.932882in}{1.945346in}}{\pgfqpoint{3.936155in}{1.953246in}}{\pgfqpoint{3.936155in}{1.961482in}}%
\pgfpathcurveto{\pgfqpoint{3.936155in}{1.969719in}}{\pgfqpoint{3.932882in}{1.977619in}}{\pgfqpoint{3.927058in}{1.983443in}}%
\pgfpathcurveto{\pgfqpoint{3.921234in}{1.989267in}}{\pgfqpoint{3.913334in}{1.992539in}}{\pgfqpoint{3.905098in}{1.992539in}}%
\pgfpathcurveto{\pgfqpoint{3.896862in}{1.992539in}}{\pgfqpoint{3.888962in}{1.989267in}}{\pgfqpoint{3.883138in}{1.983443in}}%
\pgfpathcurveto{\pgfqpoint{3.877314in}{1.977619in}}{\pgfqpoint{3.874042in}{1.969719in}}{\pgfqpoint{3.874042in}{1.961482in}}%
\pgfpathcurveto{\pgfqpoint{3.874042in}{1.953246in}}{\pgfqpoint{3.877314in}{1.945346in}}{\pgfqpoint{3.883138in}{1.939522in}}%
\pgfpathcurveto{\pgfqpoint{3.888962in}{1.933698in}}{\pgfqpoint{3.896862in}{1.930426in}}{\pgfqpoint{3.905098in}{1.930426in}}%
\pgfpathclose%
\pgfusepath{stroke,fill}%
\end{pgfscope}%
\begin{pgfscope}%
\pgfpathrectangle{\pgfqpoint{3.793912in}{0.557870in}}{\pgfqpoint{2.446088in}{1.484734in}}%
\pgfusepath{clip}%
\pgfsetbuttcap%
\pgfsetroundjoin%
\definecolor{currentfill}{rgb}{0.298039,0.447059,0.690196}%
\pgfsetfillcolor{currentfill}%
\pgfsetlinewidth{1.003750pt}%
\definecolor{currentstroke}{rgb}{0.298039,0.447059,0.690196}%
\pgfsetstrokecolor{currentstroke}%
\pgfsetdash{}{0pt}%
\pgfpathmoveto{\pgfqpoint{4.044080in}{0.594302in}}%
\pgfpathcurveto{\pgfqpoint{4.052317in}{0.594302in}}{\pgfqpoint{4.060217in}{0.597574in}}{\pgfqpoint{4.066041in}{0.603398in}}%
\pgfpathcurveto{\pgfqpoint{4.071864in}{0.609222in}}{\pgfqpoint{4.075137in}{0.617122in}}{\pgfqpoint{4.075137in}{0.625358in}}%
\pgfpathcurveto{\pgfqpoint{4.075137in}{0.633594in}}{\pgfqpoint{4.071864in}{0.641495in}}{\pgfqpoint{4.066041in}{0.647318in}}%
\pgfpathcurveto{\pgfqpoint{4.060217in}{0.653142in}}{\pgfqpoint{4.052317in}{0.656415in}}{\pgfqpoint{4.044080in}{0.656415in}}%
\pgfpathcurveto{\pgfqpoint{4.035844in}{0.656415in}}{\pgfqpoint{4.027944in}{0.653142in}}{\pgfqpoint{4.022120in}{0.647318in}}%
\pgfpathcurveto{\pgfqpoint{4.016296in}{0.641495in}}{\pgfqpoint{4.013024in}{0.633594in}}{\pgfqpoint{4.013024in}{0.625358in}}%
\pgfpathcurveto{\pgfqpoint{4.013024in}{0.617122in}}{\pgfqpoint{4.016296in}{0.609222in}}{\pgfqpoint{4.022120in}{0.603398in}}%
\pgfpathcurveto{\pgfqpoint{4.027944in}{0.597574in}}{\pgfqpoint{4.035844in}{0.594302in}}{\pgfqpoint{4.044080in}{0.594302in}}%
\pgfpathclose%
\pgfusepath{stroke,fill}%
\end{pgfscope}%
\begin{pgfscope}%
\pgfpathrectangle{\pgfqpoint{3.793912in}{0.557870in}}{\pgfqpoint{2.446088in}{1.484734in}}%
\pgfusepath{clip}%
\pgfsetbuttcap%
\pgfsetroundjoin%
\definecolor{currentfill}{rgb}{0.298039,0.447059,0.690196}%
\pgfsetfillcolor{currentfill}%
\pgfsetlinewidth{1.003750pt}%
\definecolor{currentstroke}{rgb}{0.298039,0.447059,0.690196}%
\pgfsetstrokecolor{currentstroke}%
\pgfsetdash{}{0pt}%
\pgfpathmoveto{\pgfqpoint{3.905098in}{1.930426in}}%
\pgfpathcurveto{\pgfqpoint{3.913334in}{1.930426in}}{\pgfqpoint{3.921234in}{1.933698in}}{\pgfqpoint{3.927058in}{1.939522in}}%
\pgfpathcurveto{\pgfqpoint{3.932882in}{1.945346in}}{\pgfqpoint{3.936155in}{1.953246in}}{\pgfqpoint{3.936155in}{1.961482in}}%
\pgfpathcurveto{\pgfqpoint{3.936155in}{1.969719in}}{\pgfqpoint{3.932882in}{1.977619in}}{\pgfqpoint{3.927058in}{1.983443in}}%
\pgfpathcurveto{\pgfqpoint{3.921234in}{1.989267in}}{\pgfqpoint{3.913334in}{1.992539in}}{\pgfqpoint{3.905098in}{1.992539in}}%
\pgfpathcurveto{\pgfqpoint{3.896862in}{1.992539in}}{\pgfqpoint{3.888962in}{1.989267in}}{\pgfqpoint{3.883138in}{1.983443in}}%
\pgfpathcurveto{\pgfqpoint{3.877314in}{1.977619in}}{\pgfqpoint{3.874042in}{1.969719in}}{\pgfqpoint{3.874042in}{1.961482in}}%
\pgfpathcurveto{\pgfqpoint{3.874042in}{1.953246in}}{\pgfqpoint{3.877314in}{1.945346in}}{\pgfqpoint{3.883138in}{1.939522in}}%
\pgfpathcurveto{\pgfqpoint{3.888962in}{1.933698in}}{\pgfqpoint{3.896862in}{1.930426in}}{\pgfqpoint{3.905098in}{1.930426in}}%
\pgfpathclose%
\pgfusepath{stroke,fill}%
\end{pgfscope}%
\begin{pgfscope}%
\pgfpathrectangle{\pgfqpoint{3.793912in}{0.557870in}}{\pgfqpoint{2.446088in}{1.484734in}}%
\pgfusepath{clip}%
\pgfsetbuttcap%
\pgfsetroundjoin%
\definecolor{currentfill}{rgb}{0.298039,0.447059,0.690196}%
\pgfsetfillcolor{currentfill}%
\pgfsetlinewidth{1.003750pt}%
\definecolor{currentstroke}{rgb}{0.298039,0.447059,0.690196}%
\pgfsetstrokecolor{currentstroke}%
\pgfsetdash{}{0pt}%
\pgfpathmoveto{\pgfqpoint{3.905098in}{1.644114in}}%
\pgfpathcurveto{\pgfqpoint{3.913334in}{1.644114in}}{\pgfqpoint{3.921234in}{1.647386in}}{\pgfqpoint{3.927058in}{1.653210in}}%
\pgfpathcurveto{\pgfqpoint{3.932882in}{1.659034in}}{\pgfqpoint{3.936155in}{1.666934in}}{\pgfqpoint{3.936155in}{1.675170in}}%
\pgfpathcurveto{\pgfqpoint{3.936155in}{1.683406in}}{\pgfqpoint{3.932882in}{1.691306in}}{\pgfqpoint{3.927058in}{1.697130in}}%
\pgfpathcurveto{\pgfqpoint{3.921234in}{1.702954in}}{\pgfqpoint{3.913334in}{1.706227in}}{\pgfqpoint{3.905098in}{1.706227in}}%
\pgfpathcurveto{\pgfqpoint{3.896862in}{1.706227in}}{\pgfqpoint{3.888962in}{1.702954in}}{\pgfqpoint{3.883138in}{1.697130in}}%
\pgfpathcurveto{\pgfqpoint{3.877314in}{1.691306in}}{\pgfqpoint{3.874042in}{1.683406in}}{\pgfqpoint{3.874042in}{1.675170in}}%
\pgfpathcurveto{\pgfqpoint{3.874042in}{1.666934in}}{\pgfqpoint{3.877314in}{1.659034in}}{\pgfqpoint{3.883138in}{1.653210in}}%
\pgfpathcurveto{\pgfqpoint{3.888962in}{1.647386in}}{\pgfqpoint{3.896862in}{1.644114in}}{\pgfqpoint{3.905098in}{1.644114in}}%
\pgfpathclose%
\pgfusepath{stroke,fill}%
\end{pgfscope}%
\begin{pgfscope}%
\pgfpathrectangle{\pgfqpoint{3.793912in}{0.557870in}}{\pgfqpoint{2.446088in}{1.484734in}}%
\pgfusepath{clip}%
\pgfsetbuttcap%
\pgfsetroundjoin%
\definecolor{currentfill}{rgb}{0.298039,0.447059,0.690196}%
\pgfsetfillcolor{currentfill}%
\pgfsetlinewidth{1.003750pt}%
\definecolor{currentstroke}{rgb}{0.298039,0.447059,0.690196}%
\pgfsetstrokecolor{currentstroke}%
\pgfsetdash{}{0pt}%
\pgfpathmoveto{\pgfqpoint{3.905098in}{1.930426in}}%
\pgfpathcurveto{\pgfqpoint{3.913334in}{1.930426in}}{\pgfqpoint{3.921234in}{1.933698in}}{\pgfqpoint{3.927058in}{1.939522in}}%
\pgfpathcurveto{\pgfqpoint{3.932882in}{1.945346in}}{\pgfqpoint{3.936155in}{1.953246in}}{\pgfqpoint{3.936155in}{1.961482in}}%
\pgfpathcurveto{\pgfqpoint{3.936155in}{1.969719in}}{\pgfqpoint{3.932882in}{1.977619in}}{\pgfqpoint{3.927058in}{1.983443in}}%
\pgfpathcurveto{\pgfqpoint{3.921234in}{1.989267in}}{\pgfqpoint{3.913334in}{1.992539in}}{\pgfqpoint{3.905098in}{1.992539in}}%
\pgfpathcurveto{\pgfqpoint{3.896862in}{1.992539in}}{\pgfqpoint{3.888962in}{1.989267in}}{\pgfqpoint{3.883138in}{1.983443in}}%
\pgfpathcurveto{\pgfqpoint{3.877314in}{1.977619in}}{\pgfqpoint{3.874042in}{1.969719in}}{\pgfqpoint{3.874042in}{1.961482in}}%
\pgfpathcurveto{\pgfqpoint{3.874042in}{1.953246in}}{\pgfqpoint{3.877314in}{1.945346in}}{\pgfqpoint{3.883138in}{1.939522in}}%
\pgfpathcurveto{\pgfqpoint{3.888962in}{1.933698in}}{\pgfqpoint{3.896862in}{1.930426in}}{\pgfqpoint{3.905098in}{1.930426in}}%
\pgfpathclose%
\pgfusepath{stroke,fill}%
\end{pgfscope}%
\begin{pgfscope}%
\pgfpathrectangle{\pgfqpoint{3.793912in}{0.557870in}}{\pgfqpoint{2.446088in}{1.484734in}}%
\pgfusepath{clip}%
\pgfsetbuttcap%
\pgfsetroundjoin%
\definecolor{currentfill}{rgb}{0.298039,0.447059,0.690196}%
\pgfsetfillcolor{currentfill}%
\pgfsetlinewidth{1.003750pt}%
\definecolor{currentstroke}{rgb}{0.298039,0.447059,0.690196}%
\pgfsetstrokecolor{currentstroke}%
\pgfsetdash{}{0pt}%
\pgfpathmoveto{\pgfqpoint{4.414700in}{0.594302in}}%
\pgfpathcurveto{\pgfqpoint{4.422936in}{0.594302in}}{\pgfqpoint{4.430836in}{0.597574in}}{\pgfqpoint{4.436660in}{0.603398in}}%
\pgfpathcurveto{\pgfqpoint{4.442484in}{0.609222in}}{\pgfqpoint{4.445756in}{0.617122in}}{\pgfqpoint{4.445756in}{0.625358in}}%
\pgfpathcurveto{\pgfqpoint{4.445756in}{0.633594in}}{\pgfqpoint{4.442484in}{0.641495in}}{\pgfqpoint{4.436660in}{0.647318in}}%
\pgfpathcurveto{\pgfqpoint{4.430836in}{0.653142in}}{\pgfqpoint{4.422936in}{0.656415in}}{\pgfqpoint{4.414700in}{0.656415in}}%
\pgfpathcurveto{\pgfqpoint{4.406463in}{0.656415in}}{\pgfqpoint{4.398563in}{0.653142in}}{\pgfqpoint{4.392739in}{0.647318in}}%
\pgfpathcurveto{\pgfqpoint{4.386915in}{0.641495in}}{\pgfqpoint{4.383643in}{0.633594in}}{\pgfqpoint{4.383643in}{0.625358in}}%
\pgfpathcurveto{\pgfqpoint{4.383643in}{0.617122in}}{\pgfqpoint{4.386915in}{0.609222in}}{\pgfqpoint{4.392739in}{0.603398in}}%
\pgfpathcurveto{\pgfqpoint{4.398563in}{0.597574in}}{\pgfqpoint{4.406463in}{0.594302in}}{\pgfqpoint{4.414700in}{0.594302in}}%
\pgfpathclose%
\pgfusepath{stroke,fill}%
\end{pgfscope}%
\begin{pgfscope}%
\pgfpathrectangle{\pgfqpoint{3.793912in}{0.557870in}}{\pgfqpoint{2.446088in}{1.484734in}}%
\pgfusepath{clip}%
\pgfsetbuttcap%
\pgfsetroundjoin%
\definecolor{currentfill}{rgb}{0.298039,0.447059,0.690196}%
\pgfsetfillcolor{currentfill}%
\pgfsetlinewidth{1.003750pt}%
\definecolor{currentstroke}{rgb}{0.298039,0.447059,0.690196}%
\pgfsetstrokecolor{currentstroke}%
\pgfsetdash{}{0pt}%
\pgfpathmoveto{\pgfqpoint{3.905098in}{1.930426in}}%
\pgfpathcurveto{\pgfqpoint{3.913334in}{1.930426in}}{\pgfqpoint{3.921234in}{1.933698in}}{\pgfqpoint{3.927058in}{1.939522in}}%
\pgfpathcurveto{\pgfqpoint{3.932882in}{1.945346in}}{\pgfqpoint{3.936155in}{1.953246in}}{\pgfqpoint{3.936155in}{1.961482in}}%
\pgfpathcurveto{\pgfqpoint{3.936155in}{1.969719in}}{\pgfqpoint{3.932882in}{1.977619in}}{\pgfqpoint{3.927058in}{1.983443in}}%
\pgfpathcurveto{\pgfqpoint{3.921234in}{1.989267in}}{\pgfqpoint{3.913334in}{1.992539in}}{\pgfqpoint{3.905098in}{1.992539in}}%
\pgfpathcurveto{\pgfqpoint{3.896862in}{1.992539in}}{\pgfqpoint{3.888962in}{1.989267in}}{\pgfqpoint{3.883138in}{1.983443in}}%
\pgfpathcurveto{\pgfqpoint{3.877314in}{1.977619in}}{\pgfqpoint{3.874042in}{1.969719in}}{\pgfqpoint{3.874042in}{1.961482in}}%
\pgfpathcurveto{\pgfqpoint{3.874042in}{1.953246in}}{\pgfqpoint{3.877314in}{1.945346in}}{\pgfqpoint{3.883138in}{1.939522in}}%
\pgfpathcurveto{\pgfqpoint{3.888962in}{1.933698in}}{\pgfqpoint{3.896862in}{1.930426in}}{\pgfqpoint{3.905098in}{1.930426in}}%
\pgfpathclose%
\pgfusepath{stroke,fill}%
\end{pgfscope}%
\begin{pgfscope}%
\pgfpathrectangle{\pgfqpoint{3.793912in}{0.557870in}}{\pgfqpoint{2.446088in}{1.484734in}}%
\pgfusepath{clip}%
\pgfsetbuttcap%
\pgfsetroundjoin%
\definecolor{currentfill}{rgb}{0.298039,0.447059,0.690196}%
\pgfsetfillcolor{currentfill}%
\pgfsetlinewidth{1.003750pt}%
\definecolor{currentstroke}{rgb}{0.298039,0.447059,0.690196}%
\pgfsetstrokecolor{currentstroke}%
\pgfsetdash{}{0pt}%
\pgfpathmoveto{\pgfqpoint{3.905098in}{1.930426in}}%
\pgfpathcurveto{\pgfqpoint{3.913334in}{1.930426in}}{\pgfqpoint{3.921234in}{1.933698in}}{\pgfqpoint{3.927058in}{1.939522in}}%
\pgfpathcurveto{\pgfqpoint{3.932882in}{1.945346in}}{\pgfqpoint{3.936155in}{1.953246in}}{\pgfqpoint{3.936155in}{1.961482in}}%
\pgfpathcurveto{\pgfqpoint{3.936155in}{1.969719in}}{\pgfqpoint{3.932882in}{1.977619in}}{\pgfqpoint{3.927058in}{1.983443in}}%
\pgfpathcurveto{\pgfqpoint{3.921234in}{1.989267in}}{\pgfqpoint{3.913334in}{1.992539in}}{\pgfqpoint{3.905098in}{1.992539in}}%
\pgfpathcurveto{\pgfqpoint{3.896862in}{1.992539in}}{\pgfqpoint{3.888962in}{1.989267in}}{\pgfqpoint{3.883138in}{1.983443in}}%
\pgfpathcurveto{\pgfqpoint{3.877314in}{1.977619in}}{\pgfqpoint{3.874042in}{1.969719in}}{\pgfqpoint{3.874042in}{1.961482in}}%
\pgfpathcurveto{\pgfqpoint{3.874042in}{1.953246in}}{\pgfqpoint{3.877314in}{1.945346in}}{\pgfqpoint{3.883138in}{1.939522in}}%
\pgfpathcurveto{\pgfqpoint{3.888962in}{1.933698in}}{\pgfqpoint{3.896862in}{1.930426in}}{\pgfqpoint{3.905098in}{1.930426in}}%
\pgfpathclose%
\pgfusepath{stroke,fill}%
\end{pgfscope}%
\begin{pgfscope}%
\pgfpathrectangle{\pgfqpoint{3.793912in}{0.557870in}}{\pgfqpoint{2.446088in}{1.484734in}}%
\pgfusepath{clip}%
\pgfsetbuttcap%
\pgfsetroundjoin%
\definecolor{currentfill}{rgb}{0.298039,0.447059,0.690196}%
\pgfsetfillcolor{currentfill}%
\pgfsetlinewidth{1.003750pt}%
\definecolor{currentstroke}{rgb}{0.298039,0.447059,0.690196}%
\pgfsetstrokecolor{currentstroke}%
\pgfsetdash{}{0pt}%
\pgfpathmoveto{\pgfqpoint{3.905098in}{1.930426in}}%
\pgfpathcurveto{\pgfqpoint{3.913334in}{1.930426in}}{\pgfqpoint{3.921234in}{1.933698in}}{\pgfqpoint{3.927058in}{1.939522in}}%
\pgfpathcurveto{\pgfqpoint{3.932882in}{1.945346in}}{\pgfqpoint{3.936155in}{1.953246in}}{\pgfqpoint{3.936155in}{1.961482in}}%
\pgfpathcurveto{\pgfqpoint{3.936155in}{1.969719in}}{\pgfqpoint{3.932882in}{1.977619in}}{\pgfqpoint{3.927058in}{1.983443in}}%
\pgfpathcurveto{\pgfqpoint{3.921234in}{1.989267in}}{\pgfqpoint{3.913334in}{1.992539in}}{\pgfqpoint{3.905098in}{1.992539in}}%
\pgfpathcurveto{\pgfqpoint{3.896862in}{1.992539in}}{\pgfqpoint{3.888962in}{1.989267in}}{\pgfqpoint{3.883138in}{1.983443in}}%
\pgfpathcurveto{\pgfqpoint{3.877314in}{1.977619in}}{\pgfqpoint{3.874042in}{1.969719in}}{\pgfqpoint{3.874042in}{1.961482in}}%
\pgfpathcurveto{\pgfqpoint{3.874042in}{1.953246in}}{\pgfqpoint{3.877314in}{1.945346in}}{\pgfqpoint{3.883138in}{1.939522in}}%
\pgfpathcurveto{\pgfqpoint{3.888962in}{1.933698in}}{\pgfqpoint{3.896862in}{1.930426in}}{\pgfqpoint{3.905098in}{1.930426in}}%
\pgfpathclose%
\pgfusepath{stroke,fill}%
\end{pgfscope}%
\begin{pgfscope}%
\pgfpathrectangle{\pgfqpoint{3.793912in}{0.557870in}}{\pgfqpoint{2.446088in}{1.484734in}}%
\pgfusepath{clip}%
\pgfsetbuttcap%
\pgfsetroundjoin%
\definecolor{currentfill}{rgb}{0.298039,0.447059,0.690196}%
\pgfsetfillcolor{currentfill}%
\pgfsetlinewidth{1.003750pt}%
\definecolor{currentstroke}{rgb}{0.298039,0.447059,0.690196}%
\pgfsetstrokecolor{currentstroke}%
\pgfsetdash{}{0pt}%
\pgfpathmoveto{\pgfqpoint{3.905098in}{1.930426in}}%
\pgfpathcurveto{\pgfqpoint{3.913334in}{1.930426in}}{\pgfqpoint{3.921234in}{1.933698in}}{\pgfqpoint{3.927058in}{1.939522in}}%
\pgfpathcurveto{\pgfqpoint{3.932882in}{1.945346in}}{\pgfqpoint{3.936155in}{1.953246in}}{\pgfqpoint{3.936155in}{1.961482in}}%
\pgfpathcurveto{\pgfqpoint{3.936155in}{1.969719in}}{\pgfqpoint{3.932882in}{1.977619in}}{\pgfqpoint{3.927058in}{1.983443in}}%
\pgfpathcurveto{\pgfqpoint{3.921234in}{1.989267in}}{\pgfqpoint{3.913334in}{1.992539in}}{\pgfqpoint{3.905098in}{1.992539in}}%
\pgfpathcurveto{\pgfqpoint{3.896862in}{1.992539in}}{\pgfqpoint{3.888962in}{1.989267in}}{\pgfqpoint{3.883138in}{1.983443in}}%
\pgfpathcurveto{\pgfqpoint{3.877314in}{1.977619in}}{\pgfqpoint{3.874042in}{1.969719in}}{\pgfqpoint{3.874042in}{1.961482in}}%
\pgfpathcurveto{\pgfqpoint{3.874042in}{1.953246in}}{\pgfqpoint{3.877314in}{1.945346in}}{\pgfqpoint{3.883138in}{1.939522in}}%
\pgfpathcurveto{\pgfqpoint{3.888962in}{1.933698in}}{\pgfqpoint{3.896862in}{1.930426in}}{\pgfqpoint{3.905098in}{1.930426in}}%
\pgfpathclose%
\pgfusepath{stroke,fill}%
\end{pgfscope}%
\begin{pgfscope}%
\pgfpathrectangle{\pgfqpoint{3.793912in}{0.557870in}}{\pgfqpoint{2.446088in}{1.484734in}}%
\pgfusepath{clip}%
\pgfsetbuttcap%
\pgfsetroundjoin%
\definecolor{currentfill}{rgb}{0.298039,0.447059,0.690196}%
\pgfsetfillcolor{currentfill}%
\pgfsetlinewidth{1.003750pt}%
\definecolor{currentstroke}{rgb}{0.298039,0.447059,0.690196}%
\pgfsetstrokecolor{currentstroke}%
\pgfsetdash{}{0pt}%
\pgfpathmoveto{\pgfqpoint{3.905098in}{1.930426in}}%
\pgfpathcurveto{\pgfqpoint{3.913334in}{1.930426in}}{\pgfqpoint{3.921234in}{1.933698in}}{\pgfqpoint{3.927058in}{1.939522in}}%
\pgfpathcurveto{\pgfqpoint{3.932882in}{1.945346in}}{\pgfqpoint{3.936155in}{1.953246in}}{\pgfqpoint{3.936155in}{1.961482in}}%
\pgfpathcurveto{\pgfqpoint{3.936155in}{1.969719in}}{\pgfqpoint{3.932882in}{1.977619in}}{\pgfqpoint{3.927058in}{1.983443in}}%
\pgfpathcurveto{\pgfqpoint{3.921234in}{1.989267in}}{\pgfqpoint{3.913334in}{1.992539in}}{\pgfqpoint{3.905098in}{1.992539in}}%
\pgfpathcurveto{\pgfqpoint{3.896862in}{1.992539in}}{\pgfqpoint{3.888962in}{1.989267in}}{\pgfqpoint{3.883138in}{1.983443in}}%
\pgfpathcurveto{\pgfqpoint{3.877314in}{1.977619in}}{\pgfqpoint{3.874042in}{1.969719in}}{\pgfqpoint{3.874042in}{1.961482in}}%
\pgfpathcurveto{\pgfqpoint{3.874042in}{1.953246in}}{\pgfqpoint{3.877314in}{1.945346in}}{\pgfqpoint{3.883138in}{1.939522in}}%
\pgfpathcurveto{\pgfqpoint{3.888962in}{1.933698in}}{\pgfqpoint{3.896862in}{1.930426in}}{\pgfqpoint{3.905098in}{1.930426in}}%
\pgfpathclose%
\pgfusepath{stroke,fill}%
\end{pgfscope}%
\begin{pgfscope}%
\pgfpathrectangle{\pgfqpoint{3.793912in}{0.557870in}}{\pgfqpoint{2.446088in}{1.484734in}}%
\pgfusepath{clip}%
\pgfsetbuttcap%
\pgfsetroundjoin%
\definecolor{currentfill}{rgb}{0.298039,0.447059,0.690196}%
\pgfsetfillcolor{currentfill}%
\pgfsetlinewidth{1.003750pt}%
\definecolor{currentstroke}{rgb}{0.298039,0.447059,0.690196}%
\pgfsetstrokecolor{currentstroke}%
\pgfsetdash{}{0pt}%
\pgfpathmoveto{\pgfqpoint{3.905098in}{1.930426in}}%
\pgfpathcurveto{\pgfqpoint{3.913334in}{1.930426in}}{\pgfqpoint{3.921234in}{1.933698in}}{\pgfqpoint{3.927058in}{1.939522in}}%
\pgfpathcurveto{\pgfqpoint{3.932882in}{1.945346in}}{\pgfqpoint{3.936155in}{1.953246in}}{\pgfqpoint{3.936155in}{1.961482in}}%
\pgfpathcurveto{\pgfqpoint{3.936155in}{1.969719in}}{\pgfqpoint{3.932882in}{1.977619in}}{\pgfqpoint{3.927058in}{1.983443in}}%
\pgfpathcurveto{\pgfqpoint{3.921234in}{1.989267in}}{\pgfqpoint{3.913334in}{1.992539in}}{\pgfqpoint{3.905098in}{1.992539in}}%
\pgfpathcurveto{\pgfqpoint{3.896862in}{1.992539in}}{\pgfqpoint{3.888962in}{1.989267in}}{\pgfqpoint{3.883138in}{1.983443in}}%
\pgfpathcurveto{\pgfqpoint{3.877314in}{1.977619in}}{\pgfqpoint{3.874042in}{1.969719in}}{\pgfqpoint{3.874042in}{1.961482in}}%
\pgfpathcurveto{\pgfqpoint{3.874042in}{1.953246in}}{\pgfqpoint{3.877314in}{1.945346in}}{\pgfqpoint{3.883138in}{1.939522in}}%
\pgfpathcurveto{\pgfqpoint{3.888962in}{1.933698in}}{\pgfqpoint{3.896862in}{1.930426in}}{\pgfqpoint{3.905098in}{1.930426in}}%
\pgfpathclose%
\pgfusepath{stroke,fill}%
\end{pgfscope}%
\begin{pgfscope}%
\pgfpathrectangle{\pgfqpoint{3.793912in}{0.557870in}}{\pgfqpoint{2.446088in}{1.484734in}}%
\pgfusepath{clip}%
\pgfsetbuttcap%
\pgfsetroundjoin%
\definecolor{currentfill}{rgb}{0.298039,0.447059,0.690196}%
\pgfsetfillcolor{currentfill}%
\pgfsetlinewidth{1.003750pt}%
\definecolor{currentstroke}{rgb}{0.298039,0.447059,0.690196}%
\pgfsetstrokecolor{currentstroke}%
\pgfsetdash{}{0pt}%
\pgfpathmoveto{\pgfqpoint{5.202266in}{0.594302in}}%
\pgfpathcurveto{\pgfqpoint{5.210502in}{0.594302in}}{\pgfqpoint{5.218402in}{0.597574in}}{\pgfqpoint{5.224226in}{0.603398in}}%
\pgfpathcurveto{\pgfqpoint{5.230050in}{0.609222in}}{\pgfqpoint{5.233322in}{0.617122in}}{\pgfqpoint{5.233322in}{0.625358in}}%
\pgfpathcurveto{\pgfqpoint{5.233322in}{0.633594in}}{\pgfqpoint{5.230050in}{0.641495in}}{\pgfqpoint{5.224226in}{0.647318in}}%
\pgfpathcurveto{\pgfqpoint{5.218402in}{0.653142in}}{\pgfqpoint{5.210502in}{0.656415in}}{\pgfqpoint{5.202266in}{0.656415in}}%
\pgfpathcurveto{\pgfqpoint{5.194030in}{0.656415in}}{\pgfqpoint{5.186129in}{0.653142in}}{\pgfqpoint{5.180306in}{0.647318in}}%
\pgfpathcurveto{\pgfqpoint{5.174482in}{0.641495in}}{\pgfqpoint{5.171209in}{0.633594in}}{\pgfqpoint{5.171209in}{0.625358in}}%
\pgfpathcurveto{\pgfqpoint{5.171209in}{0.617122in}}{\pgfqpoint{5.174482in}{0.609222in}}{\pgfqpoint{5.180306in}{0.603398in}}%
\pgfpathcurveto{\pgfqpoint{5.186129in}{0.597574in}}{\pgfqpoint{5.194030in}{0.594302in}}{\pgfqpoint{5.202266in}{0.594302in}}%
\pgfpathclose%
\pgfusepath{stroke,fill}%
\end{pgfscope}%
\begin{pgfscope}%
\pgfpathrectangle{\pgfqpoint{3.793912in}{0.557870in}}{\pgfqpoint{2.446088in}{1.484734in}}%
\pgfusepath{clip}%
\pgfsetbuttcap%
\pgfsetroundjoin%
\definecolor{currentfill}{rgb}{0.298039,0.447059,0.690196}%
\pgfsetfillcolor{currentfill}%
\pgfsetlinewidth{1.003750pt}%
\definecolor{currentstroke}{rgb}{0.298039,0.447059,0.690196}%
\pgfsetstrokecolor{currentstroke}%
\pgfsetdash{}{0pt}%
\pgfpathmoveto{\pgfqpoint{3.905098in}{1.930426in}}%
\pgfpathcurveto{\pgfqpoint{3.913334in}{1.930426in}}{\pgfqpoint{3.921234in}{1.933698in}}{\pgfqpoint{3.927058in}{1.939522in}}%
\pgfpathcurveto{\pgfqpoint{3.932882in}{1.945346in}}{\pgfqpoint{3.936155in}{1.953246in}}{\pgfqpoint{3.936155in}{1.961482in}}%
\pgfpathcurveto{\pgfqpoint{3.936155in}{1.969719in}}{\pgfqpoint{3.932882in}{1.977619in}}{\pgfqpoint{3.927058in}{1.983443in}}%
\pgfpathcurveto{\pgfqpoint{3.921234in}{1.989267in}}{\pgfqpoint{3.913334in}{1.992539in}}{\pgfqpoint{3.905098in}{1.992539in}}%
\pgfpathcurveto{\pgfqpoint{3.896862in}{1.992539in}}{\pgfqpoint{3.888962in}{1.989267in}}{\pgfqpoint{3.883138in}{1.983443in}}%
\pgfpathcurveto{\pgfqpoint{3.877314in}{1.977619in}}{\pgfqpoint{3.874042in}{1.969719in}}{\pgfqpoint{3.874042in}{1.961482in}}%
\pgfpathcurveto{\pgfqpoint{3.874042in}{1.953246in}}{\pgfqpoint{3.877314in}{1.945346in}}{\pgfqpoint{3.883138in}{1.939522in}}%
\pgfpathcurveto{\pgfqpoint{3.888962in}{1.933698in}}{\pgfqpoint{3.896862in}{1.930426in}}{\pgfqpoint{3.905098in}{1.930426in}}%
\pgfpathclose%
\pgfusepath{stroke,fill}%
\end{pgfscope}%
\begin{pgfscope}%
\pgfpathrectangle{\pgfqpoint{3.793912in}{0.557870in}}{\pgfqpoint{2.446088in}{1.484734in}}%
\pgfusepath{clip}%
\pgfsetbuttcap%
\pgfsetroundjoin%
\definecolor{currentfill}{rgb}{0.298039,0.447059,0.690196}%
\pgfsetfillcolor{currentfill}%
\pgfsetlinewidth{1.003750pt}%
\definecolor{currentstroke}{rgb}{0.298039,0.447059,0.690196}%
\pgfsetstrokecolor{currentstroke}%
\pgfsetdash{}{0pt}%
\pgfpathmoveto{\pgfqpoint{3.905098in}{1.930426in}}%
\pgfpathcurveto{\pgfqpoint{3.913334in}{1.930426in}}{\pgfqpoint{3.921234in}{1.933698in}}{\pgfqpoint{3.927058in}{1.939522in}}%
\pgfpathcurveto{\pgfqpoint{3.932882in}{1.945346in}}{\pgfqpoint{3.936155in}{1.953246in}}{\pgfqpoint{3.936155in}{1.961482in}}%
\pgfpathcurveto{\pgfqpoint{3.936155in}{1.969719in}}{\pgfqpoint{3.932882in}{1.977619in}}{\pgfqpoint{3.927058in}{1.983443in}}%
\pgfpathcurveto{\pgfqpoint{3.921234in}{1.989267in}}{\pgfqpoint{3.913334in}{1.992539in}}{\pgfqpoint{3.905098in}{1.992539in}}%
\pgfpathcurveto{\pgfqpoint{3.896862in}{1.992539in}}{\pgfqpoint{3.888962in}{1.989267in}}{\pgfqpoint{3.883138in}{1.983443in}}%
\pgfpathcurveto{\pgfqpoint{3.877314in}{1.977619in}}{\pgfqpoint{3.874042in}{1.969719in}}{\pgfqpoint{3.874042in}{1.961482in}}%
\pgfpathcurveto{\pgfqpoint{3.874042in}{1.953246in}}{\pgfqpoint{3.877314in}{1.945346in}}{\pgfqpoint{3.883138in}{1.939522in}}%
\pgfpathcurveto{\pgfqpoint{3.888962in}{1.933698in}}{\pgfqpoint{3.896862in}{1.930426in}}{\pgfqpoint{3.905098in}{1.930426in}}%
\pgfpathclose%
\pgfusepath{stroke,fill}%
\end{pgfscope}%
\begin{pgfscope}%
\pgfpathrectangle{\pgfqpoint{3.793912in}{0.557870in}}{\pgfqpoint{2.446088in}{1.484734in}}%
\pgfusepath{clip}%
\pgfsetbuttcap%
\pgfsetroundjoin%
\definecolor{currentfill}{rgb}{0.298039,0.447059,0.690196}%
\pgfsetfillcolor{currentfill}%
\pgfsetlinewidth{1.003750pt}%
\definecolor{currentstroke}{rgb}{0.298039,0.447059,0.690196}%
\pgfsetstrokecolor{currentstroke}%
\pgfsetdash{}{0pt}%
\pgfpathmoveto{\pgfqpoint{3.905098in}{1.930426in}}%
\pgfpathcurveto{\pgfqpoint{3.913334in}{1.930426in}}{\pgfqpoint{3.921234in}{1.933698in}}{\pgfqpoint{3.927058in}{1.939522in}}%
\pgfpathcurveto{\pgfqpoint{3.932882in}{1.945346in}}{\pgfqpoint{3.936155in}{1.953246in}}{\pgfqpoint{3.936155in}{1.961482in}}%
\pgfpathcurveto{\pgfqpoint{3.936155in}{1.969719in}}{\pgfqpoint{3.932882in}{1.977619in}}{\pgfqpoint{3.927058in}{1.983443in}}%
\pgfpathcurveto{\pgfqpoint{3.921234in}{1.989267in}}{\pgfqpoint{3.913334in}{1.992539in}}{\pgfqpoint{3.905098in}{1.992539in}}%
\pgfpathcurveto{\pgfqpoint{3.896862in}{1.992539in}}{\pgfqpoint{3.888962in}{1.989267in}}{\pgfqpoint{3.883138in}{1.983443in}}%
\pgfpathcurveto{\pgfqpoint{3.877314in}{1.977619in}}{\pgfqpoint{3.874042in}{1.969719in}}{\pgfqpoint{3.874042in}{1.961482in}}%
\pgfpathcurveto{\pgfqpoint{3.874042in}{1.953246in}}{\pgfqpoint{3.877314in}{1.945346in}}{\pgfqpoint{3.883138in}{1.939522in}}%
\pgfpathcurveto{\pgfqpoint{3.888962in}{1.933698in}}{\pgfqpoint{3.896862in}{1.930426in}}{\pgfqpoint{3.905098in}{1.930426in}}%
\pgfpathclose%
\pgfusepath{stroke,fill}%
\end{pgfscope}%
\begin{pgfscope}%
\pgfpathrectangle{\pgfqpoint{3.793912in}{0.557870in}}{\pgfqpoint{2.446088in}{1.484734in}}%
\pgfusepath{clip}%
\pgfsetbuttcap%
\pgfsetroundjoin%
\definecolor{currentfill}{rgb}{0.298039,0.447059,0.690196}%
\pgfsetfillcolor{currentfill}%
\pgfsetlinewidth{1.003750pt}%
\definecolor{currentstroke}{rgb}{0.298039,0.447059,0.690196}%
\pgfsetstrokecolor{currentstroke}%
\pgfsetdash{}{0pt}%
\pgfpathmoveto{\pgfqpoint{3.905098in}{1.930426in}}%
\pgfpathcurveto{\pgfqpoint{3.913334in}{1.930426in}}{\pgfqpoint{3.921234in}{1.933698in}}{\pgfqpoint{3.927058in}{1.939522in}}%
\pgfpathcurveto{\pgfqpoint{3.932882in}{1.945346in}}{\pgfqpoint{3.936155in}{1.953246in}}{\pgfqpoint{3.936155in}{1.961482in}}%
\pgfpathcurveto{\pgfqpoint{3.936155in}{1.969719in}}{\pgfqpoint{3.932882in}{1.977619in}}{\pgfqpoint{3.927058in}{1.983443in}}%
\pgfpathcurveto{\pgfqpoint{3.921234in}{1.989267in}}{\pgfqpoint{3.913334in}{1.992539in}}{\pgfqpoint{3.905098in}{1.992539in}}%
\pgfpathcurveto{\pgfqpoint{3.896862in}{1.992539in}}{\pgfqpoint{3.888962in}{1.989267in}}{\pgfqpoint{3.883138in}{1.983443in}}%
\pgfpathcurveto{\pgfqpoint{3.877314in}{1.977619in}}{\pgfqpoint{3.874042in}{1.969719in}}{\pgfqpoint{3.874042in}{1.961482in}}%
\pgfpathcurveto{\pgfqpoint{3.874042in}{1.953246in}}{\pgfqpoint{3.877314in}{1.945346in}}{\pgfqpoint{3.883138in}{1.939522in}}%
\pgfpathcurveto{\pgfqpoint{3.888962in}{1.933698in}}{\pgfqpoint{3.896862in}{1.930426in}}{\pgfqpoint{3.905098in}{1.930426in}}%
\pgfpathclose%
\pgfusepath{stroke,fill}%
\end{pgfscope}%
\begin{pgfscope}%
\pgfpathrectangle{\pgfqpoint{3.793912in}{0.557870in}}{\pgfqpoint{2.446088in}{1.484734in}}%
\pgfusepath{clip}%
\pgfsetbuttcap%
\pgfsetroundjoin%
\definecolor{currentfill}{rgb}{0.298039,0.447059,0.690196}%
\pgfsetfillcolor{currentfill}%
\pgfsetlinewidth{1.003750pt}%
\definecolor{currentstroke}{rgb}{0.298039,0.447059,0.690196}%
\pgfsetstrokecolor{currentstroke}%
\pgfsetdash{}{0pt}%
\pgfpathmoveto{\pgfqpoint{3.905098in}{1.385069in}}%
\pgfpathcurveto{\pgfqpoint{3.913334in}{1.385069in}}{\pgfqpoint{3.921234in}{1.388341in}}{\pgfqpoint{3.927058in}{1.394165in}}%
\pgfpathcurveto{\pgfqpoint{3.932882in}{1.399989in}}{\pgfqpoint{3.936155in}{1.407889in}}{\pgfqpoint{3.936155in}{1.416126in}}%
\pgfpathcurveto{\pgfqpoint{3.936155in}{1.424362in}}{\pgfqpoint{3.932882in}{1.432262in}}{\pgfqpoint{3.927058in}{1.438086in}}%
\pgfpathcurveto{\pgfqpoint{3.921234in}{1.443910in}}{\pgfqpoint{3.913334in}{1.447182in}}{\pgfqpoint{3.905098in}{1.447182in}}%
\pgfpathcurveto{\pgfqpoint{3.896862in}{1.447182in}}{\pgfqpoint{3.888962in}{1.443910in}}{\pgfqpoint{3.883138in}{1.438086in}}%
\pgfpathcurveto{\pgfqpoint{3.877314in}{1.432262in}}{\pgfqpoint{3.874042in}{1.424362in}}{\pgfqpoint{3.874042in}{1.416126in}}%
\pgfpathcurveto{\pgfqpoint{3.874042in}{1.407889in}}{\pgfqpoint{3.877314in}{1.399989in}}{\pgfqpoint{3.883138in}{1.394165in}}%
\pgfpathcurveto{\pgfqpoint{3.888962in}{1.388341in}}{\pgfqpoint{3.896862in}{1.385069in}}{\pgfqpoint{3.905098in}{1.385069in}}%
\pgfpathclose%
\pgfusepath{stroke,fill}%
\end{pgfscope}%
\begin{pgfscope}%
\pgfpathrectangle{\pgfqpoint{3.793912in}{0.557870in}}{\pgfqpoint{2.446088in}{1.484734in}}%
\pgfusepath{clip}%
\pgfsetbuttcap%
\pgfsetroundjoin%
\definecolor{currentfill}{rgb}{0.298039,0.447059,0.690196}%
\pgfsetfillcolor{currentfill}%
\pgfsetlinewidth{1.003750pt}%
\definecolor{currentstroke}{rgb}{0.298039,0.447059,0.690196}%
\pgfsetstrokecolor{currentstroke}%
\pgfsetdash{}{0pt}%
\pgfpathmoveto{\pgfqpoint{3.905098in}{1.357801in}}%
\pgfpathcurveto{\pgfqpoint{3.913334in}{1.357801in}}{\pgfqpoint{3.921234in}{1.361074in}}{\pgfqpoint{3.927058in}{1.366897in}}%
\pgfpathcurveto{\pgfqpoint{3.932882in}{1.372721in}}{\pgfqpoint{3.936155in}{1.380621in}}{\pgfqpoint{3.936155in}{1.388858in}}%
\pgfpathcurveto{\pgfqpoint{3.936155in}{1.397094in}}{\pgfqpoint{3.932882in}{1.404994in}}{\pgfqpoint{3.927058in}{1.410818in}}%
\pgfpathcurveto{\pgfqpoint{3.921234in}{1.416642in}}{\pgfqpoint{3.913334in}{1.419914in}}{\pgfqpoint{3.905098in}{1.419914in}}%
\pgfpathcurveto{\pgfqpoint{3.896862in}{1.419914in}}{\pgfqpoint{3.888962in}{1.416642in}}{\pgfqpoint{3.883138in}{1.410818in}}%
\pgfpathcurveto{\pgfqpoint{3.877314in}{1.404994in}}{\pgfqpoint{3.874042in}{1.397094in}}{\pgfqpoint{3.874042in}{1.388858in}}%
\pgfpathcurveto{\pgfqpoint{3.874042in}{1.380621in}}{\pgfqpoint{3.877314in}{1.372721in}}{\pgfqpoint{3.883138in}{1.366897in}}%
\pgfpathcurveto{\pgfqpoint{3.888962in}{1.361074in}}{\pgfqpoint{3.896862in}{1.357801in}}{\pgfqpoint{3.905098in}{1.357801in}}%
\pgfpathclose%
\pgfusepath{stroke,fill}%
\end{pgfscope}%
\begin{pgfscope}%
\pgfpathrectangle{\pgfqpoint{3.793912in}{0.557870in}}{\pgfqpoint{2.446088in}{1.484734in}}%
\pgfusepath{clip}%
\pgfsetbuttcap%
\pgfsetroundjoin%
\definecolor{currentfill}{rgb}{0.298039,0.447059,0.690196}%
\pgfsetfillcolor{currentfill}%
\pgfsetlinewidth{1.003750pt}%
\definecolor{currentstroke}{rgb}{0.298039,0.447059,0.690196}%
\pgfsetstrokecolor{currentstroke}%
\pgfsetdash{}{0pt}%
\pgfpathmoveto{\pgfqpoint{3.905098in}{1.303266in}}%
\pgfpathcurveto{\pgfqpoint{3.913334in}{1.303266in}}{\pgfqpoint{3.921234in}{1.306538in}}{\pgfqpoint{3.927058in}{1.312362in}}%
\pgfpathcurveto{\pgfqpoint{3.932882in}{1.318186in}}{\pgfqpoint{3.936155in}{1.326086in}}{\pgfqpoint{3.936155in}{1.334322in}}%
\pgfpathcurveto{\pgfqpoint{3.936155in}{1.342558in}}{\pgfqpoint{3.932882in}{1.350458in}}{\pgfqpoint{3.927058in}{1.356282in}}%
\pgfpathcurveto{\pgfqpoint{3.921234in}{1.362106in}}{\pgfqpoint{3.913334in}{1.365379in}}{\pgfqpoint{3.905098in}{1.365379in}}%
\pgfpathcurveto{\pgfqpoint{3.896862in}{1.365379in}}{\pgfqpoint{3.888962in}{1.362106in}}{\pgfqpoint{3.883138in}{1.356282in}}%
\pgfpathcurveto{\pgfqpoint{3.877314in}{1.350458in}}{\pgfqpoint{3.874042in}{1.342558in}}{\pgfqpoint{3.874042in}{1.334322in}}%
\pgfpathcurveto{\pgfqpoint{3.874042in}{1.326086in}}{\pgfqpoint{3.877314in}{1.318186in}}{\pgfqpoint{3.883138in}{1.312362in}}%
\pgfpathcurveto{\pgfqpoint{3.888962in}{1.306538in}}{\pgfqpoint{3.896862in}{1.303266in}}{\pgfqpoint{3.905098in}{1.303266in}}%
\pgfpathclose%
\pgfusepath{stroke,fill}%
\end{pgfscope}%
\begin{pgfscope}%
\pgfpathrectangle{\pgfqpoint{3.793912in}{0.557870in}}{\pgfqpoint{2.446088in}{1.484734in}}%
\pgfusepath{clip}%
\pgfsetbuttcap%
\pgfsetroundjoin%
\definecolor{currentfill}{rgb}{0.298039,0.447059,0.690196}%
\pgfsetfillcolor{currentfill}%
\pgfsetlinewidth{1.003750pt}%
\definecolor{currentstroke}{rgb}{0.298039,0.447059,0.690196}%
\pgfsetstrokecolor{currentstroke}%
\pgfsetdash{}{0pt}%
\pgfpathmoveto{\pgfqpoint{3.905098in}{1.930426in}}%
\pgfpathcurveto{\pgfqpoint{3.913334in}{1.930426in}}{\pgfqpoint{3.921234in}{1.933698in}}{\pgfqpoint{3.927058in}{1.939522in}}%
\pgfpathcurveto{\pgfqpoint{3.932882in}{1.945346in}}{\pgfqpoint{3.936155in}{1.953246in}}{\pgfqpoint{3.936155in}{1.961482in}}%
\pgfpathcurveto{\pgfqpoint{3.936155in}{1.969719in}}{\pgfqpoint{3.932882in}{1.977619in}}{\pgfqpoint{3.927058in}{1.983443in}}%
\pgfpathcurveto{\pgfqpoint{3.921234in}{1.989267in}}{\pgfqpoint{3.913334in}{1.992539in}}{\pgfqpoint{3.905098in}{1.992539in}}%
\pgfpathcurveto{\pgfqpoint{3.896862in}{1.992539in}}{\pgfqpoint{3.888962in}{1.989267in}}{\pgfqpoint{3.883138in}{1.983443in}}%
\pgfpathcurveto{\pgfqpoint{3.877314in}{1.977619in}}{\pgfqpoint{3.874042in}{1.969719in}}{\pgfqpoint{3.874042in}{1.961482in}}%
\pgfpathcurveto{\pgfqpoint{3.874042in}{1.953246in}}{\pgfqpoint{3.877314in}{1.945346in}}{\pgfqpoint{3.883138in}{1.939522in}}%
\pgfpathcurveto{\pgfqpoint{3.888962in}{1.933698in}}{\pgfqpoint{3.896862in}{1.930426in}}{\pgfqpoint{3.905098in}{1.930426in}}%
\pgfpathclose%
\pgfusepath{stroke,fill}%
\end{pgfscope}%
\begin{pgfscope}%
\pgfpathrectangle{\pgfqpoint{3.793912in}{0.557870in}}{\pgfqpoint{2.446088in}{1.484734in}}%
\pgfusepath{clip}%
\pgfsetbuttcap%
\pgfsetroundjoin%
\definecolor{currentfill}{rgb}{0.298039,0.447059,0.690196}%
\pgfsetfillcolor{currentfill}%
\pgfsetlinewidth{1.003750pt}%
\definecolor{currentstroke}{rgb}{0.298039,0.447059,0.690196}%
\pgfsetstrokecolor{currentstroke}%
\pgfsetdash{}{0pt}%
\pgfpathmoveto{\pgfqpoint{3.905098in}{1.930426in}}%
\pgfpathcurveto{\pgfqpoint{3.913334in}{1.930426in}}{\pgfqpoint{3.921234in}{1.933698in}}{\pgfqpoint{3.927058in}{1.939522in}}%
\pgfpathcurveto{\pgfqpoint{3.932882in}{1.945346in}}{\pgfqpoint{3.936155in}{1.953246in}}{\pgfqpoint{3.936155in}{1.961482in}}%
\pgfpathcurveto{\pgfqpoint{3.936155in}{1.969719in}}{\pgfqpoint{3.932882in}{1.977619in}}{\pgfqpoint{3.927058in}{1.983443in}}%
\pgfpathcurveto{\pgfqpoint{3.921234in}{1.989267in}}{\pgfqpoint{3.913334in}{1.992539in}}{\pgfqpoint{3.905098in}{1.992539in}}%
\pgfpathcurveto{\pgfqpoint{3.896862in}{1.992539in}}{\pgfqpoint{3.888962in}{1.989267in}}{\pgfqpoint{3.883138in}{1.983443in}}%
\pgfpathcurveto{\pgfqpoint{3.877314in}{1.977619in}}{\pgfqpoint{3.874042in}{1.969719in}}{\pgfqpoint{3.874042in}{1.961482in}}%
\pgfpathcurveto{\pgfqpoint{3.874042in}{1.953246in}}{\pgfqpoint{3.877314in}{1.945346in}}{\pgfqpoint{3.883138in}{1.939522in}}%
\pgfpathcurveto{\pgfqpoint{3.888962in}{1.933698in}}{\pgfqpoint{3.896862in}{1.930426in}}{\pgfqpoint{3.905098in}{1.930426in}}%
\pgfpathclose%
\pgfusepath{stroke,fill}%
\end{pgfscope}%
\begin{pgfscope}%
\pgfpathrectangle{\pgfqpoint{3.793912in}{0.557870in}}{\pgfqpoint{2.446088in}{1.484734in}}%
\pgfusepath{clip}%
\pgfsetbuttcap%
\pgfsetroundjoin%
\definecolor{currentfill}{rgb}{0.298039,0.447059,0.690196}%
\pgfsetfillcolor{currentfill}%
\pgfsetlinewidth{1.003750pt}%
\definecolor{currentstroke}{rgb}{0.298039,0.447059,0.690196}%
\pgfsetstrokecolor{currentstroke}%
\pgfsetdash{}{0pt}%
\pgfpathmoveto{\pgfqpoint{3.905098in}{1.507774in}}%
\pgfpathcurveto{\pgfqpoint{3.913334in}{1.507774in}}{\pgfqpoint{3.921234in}{1.511047in}}{\pgfqpoint{3.927058in}{1.516871in}}%
\pgfpathcurveto{\pgfqpoint{3.932882in}{1.522694in}}{\pgfqpoint{3.936155in}{1.530595in}}{\pgfqpoint{3.936155in}{1.538831in}}%
\pgfpathcurveto{\pgfqpoint{3.936155in}{1.547067in}}{\pgfqpoint{3.932882in}{1.554967in}}{\pgfqpoint{3.927058in}{1.560791in}}%
\pgfpathcurveto{\pgfqpoint{3.921234in}{1.566615in}}{\pgfqpoint{3.913334in}{1.569887in}}{\pgfqpoint{3.905098in}{1.569887in}}%
\pgfpathcurveto{\pgfqpoint{3.896862in}{1.569887in}}{\pgfqpoint{3.888962in}{1.566615in}}{\pgfqpoint{3.883138in}{1.560791in}}%
\pgfpathcurveto{\pgfqpoint{3.877314in}{1.554967in}}{\pgfqpoint{3.874042in}{1.547067in}}{\pgfqpoint{3.874042in}{1.538831in}}%
\pgfpathcurveto{\pgfqpoint{3.874042in}{1.530595in}}{\pgfqpoint{3.877314in}{1.522694in}}{\pgfqpoint{3.883138in}{1.516871in}}%
\pgfpathcurveto{\pgfqpoint{3.888962in}{1.511047in}}{\pgfqpoint{3.896862in}{1.507774in}}{\pgfqpoint{3.905098in}{1.507774in}}%
\pgfpathclose%
\pgfusepath{stroke,fill}%
\end{pgfscope}%
\begin{pgfscope}%
\pgfpathrectangle{\pgfqpoint{3.793912in}{0.557870in}}{\pgfqpoint{2.446088in}{1.484734in}}%
\pgfusepath{clip}%
\pgfsetbuttcap%
\pgfsetroundjoin%
\definecolor{currentfill}{rgb}{0.298039,0.447059,0.690196}%
\pgfsetfillcolor{currentfill}%
\pgfsetlinewidth{1.003750pt}%
\definecolor{currentstroke}{rgb}{0.298039,0.447059,0.690196}%
\pgfsetstrokecolor{currentstroke}%
\pgfsetdash{}{0pt}%
\pgfpathmoveto{\pgfqpoint{3.905098in}{1.930426in}}%
\pgfpathcurveto{\pgfqpoint{3.913334in}{1.930426in}}{\pgfqpoint{3.921234in}{1.933698in}}{\pgfqpoint{3.927058in}{1.939522in}}%
\pgfpathcurveto{\pgfqpoint{3.932882in}{1.945346in}}{\pgfqpoint{3.936155in}{1.953246in}}{\pgfqpoint{3.936155in}{1.961482in}}%
\pgfpathcurveto{\pgfqpoint{3.936155in}{1.969719in}}{\pgfqpoint{3.932882in}{1.977619in}}{\pgfqpoint{3.927058in}{1.983443in}}%
\pgfpathcurveto{\pgfqpoint{3.921234in}{1.989267in}}{\pgfqpoint{3.913334in}{1.992539in}}{\pgfqpoint{3.905098in}{1.992539in}}%
\pgfpathcurveto{\pgfqpoint{3.896862in}{1.992539in}}{\pgfqpoint{3.888962in}{1.989267in}}{\pgfqpoint{3.883138in}{1.983443in}}%
\pgfpathcurveto{\pgfqpoint{3.877314in}{1.977619in}}{\pgfqpoint{3.874042in}{1.969719in}}{\pgfqpoint{3.874042in}{1.961482in}}%
\pgfpathcurveto{\pgfqpoint{3.874042in}{1.953246in}}{\pgfqpoint{3.877314in}{1.945346in}}{\pgfqpoint{3.883138in}{1.939522in}}%
\pgfpathcurveto{\pgfqpoint{3.888962in}{1.933698in}}{\pgfqpoint{3.896862in}{1.930426in}}{\pgfqpoint{3.905098in}{1.930426in}}%
\pgfpathclose%
\pgfusepath{stroke,fill}%
\end{pgfscope}%
\begin{pgfscope}%
\pgfpathrectangle{\pgfqpoint{3.793912in}{0.557870in}}{\pgfqpoint{2.446088in}{1.484734in}}%
\pgfusepath{clip}%
\pgfsetbuttcap%
\pgfsetroundjoin%
\definecolor{currentfill}{rgb}{0.298039,0.447059,0.690196}%
\pgfsetfillcolor{currentfill}%
\pgfsetlinewidth{1.003750pt}%
\definecolor{currentstroke}{rgb}{0.298039,0.447059,0.690196}%
\pgfsetstrokecolor{currentstroke}%
\pgfsetdash{}{0pt}%
\pgfpathmoveto{\pgfqpoint{3.905098in}{1.207828in}}%
\pgfpathcurveto{\pgfqpoint{3.913334in}{1.207828in}}{\pgfqpoint{3.921234in}{1.211100in}}{\pgfqpoint{3.927058in}{1.216924in}}%
\pgfpathcurveto{\pgfqpoint{3.932882in}{1.222748in}}{\pgfqpoint{3.936155in}{1.230648in}}{\pgfqpoint{3.936155in}{1.238885in}}%
\pgfpathcurveto{\pgfqpoint{3.936155in}{1.247121in}}{\pgfqpoint{3.932882in}{1.255021in}}{\pgfqpoint{3.927058in}{1.260845in}}%
\pgfpathcurveto{\pgfqpoint{3.921234in}{1.266669in}}{\pgfqpoint{3.913334in}{1.269941in}}{\pgfqpoint{3.905098in}{1.269941in}}%
\pgfpathcurveto{\pgfqpoint{3.896862in}{1.269941in}}{\pgfqpoint{3.888962in}{1.266669in}}{\pgfqpoint{3.883138in}{1.260845in}}%
\pgfpathcurveto{\pgfqpoint{3.877314in}{1.255021in}}{\pgfqpoint{3.874042in}{1.247121in}}{\pgfqpoint{3.874042in}{1.238885in}}%
\pgfpathcurveto{\pgfqpoint{3.874042in}{1.230648in}}{\pgfqpoint{3.877314in}{1.222748in}}{\pgfqpoint{3.883138in}{1.216924in}}%
\pgfpathcurveto{\pgfqpoint{3.888962in}{1.211100in}}{\pgfqpoint{3.896862in}{1.207828in}}{\pgfqpoint{3.905098in}{1.207828in}}%
\pgfpathclose%
\pgfusepath{stroke,fill}%
\end{pgfscope}%
\begin{pgfscope}%
\pgfpathrectangle{\pgfqpoint{3.793912in}{0.557870in}}{\pgfqpoint{2.446088in}{1.484734in}}%
\pgfusepath{clip}%
\pgfsetbuttcap%
\pgfsetroundjoin%
\definecolor{currentfill}{rgb}{0.298039,0.447059,0.690196}%
\pgfsetfillcolor{currentfill}%
\pgfsetlinewidth{1.003750pt}%
\definecolor{currentstroke}{rgb}{0.298039,0.447059,0.690196}%
\pgfsetstrokecolor{currentstroke}%
\pgfsetdash{}{0pt}%
\pgfpathmoveto{\pgfqpoint{3.905098in}{1.930426in}}%
\pgfpathcurveto{\pgfqpoint{3.913334in}{1.930426in}}{\pgfqpoint{3.921234in}{1.933698in}}{\pgfqpoint{3.927058in}{1.939522in}}%
\pgfpathcurveto{\pgfqpoint{3.932882in}{1.945346in}}{\pgfqpoint{3.936155in}{1.953246in}}{\pgfqpoint{3.936155in}{1.961482in}}%
\pgfpathcurveto{\pgfqpoint{3.936155in}{1.969719in}}{\pgfqpoint{3.932882in}{1.977619in}}{\pgfqpoint{3.927058in}{1.983443in}}%
\pgfpathcurveto{\pgfqpoint{3.921234in}{1.989267in}}{\pgfqpoint{3.913334in}{1.992539in}}{\pgfqpoint{3.905098in}{1.992539in}}%
\pgfpathcurveto{\pgfqpoint{3.896862in}{1.992539in}}{\pgfqpoint{3.888962in}{1.989267in}}{\pgfqpoint{3.883138in}{1.983443in}}%
\pgfpathcurveto{\pgfqpoint{3.877314in}{1.977619in}}{\pgfqpoint{3.874042in}{1.969719in}}{\pgfqpoint{3.874042in}{1.961482in}}%
\pgfpathcurveto{\pgfqpoint{3.874042in}{1.953246in}}{\pgfqpoint{3.877314in}{1.945346in}}{\pgfqpoint{3.883138in}{1.939522in}}%
\pgfpathcurveto{\pgfqpoint{3.888962in}{1.933698in}}{\pgfqpoint{3.896862in}{1.930426in}}{\pgfqpoint{3.905098in}{1.930426in}}%
\pgfpathclose%
\pgfusepath{stroke,fill}%
\end{pgfscope}%
\begin{pgfscope}%
\pgfpathrectangle{\pgfqpoint{3.793912in}{0.557870in}}{\pgfqpoint{2.446088in}{1.484734in}}%
\pgfusepath{clip}%
\pgfsetbuttcap%
\pgfsetroundjoin%
\definecolor{currentfill}{rgb}{0.298039,0.447059,0.690196}%
\pgfsetfillcolor{currentfill}%
\pgfsetlinewidth{1.003750pt}%
\definecolor{currentstroke}{rgb}{0.298039,0.447059,0.690196}%
\pgfsetstrokecolor{currentstroke}%
\pgfsetdash{}{0pt}%
\pgfpathmoveto{\pgfqpoint{3.905098in}{1.930426in}}%
\pgfpathcurveto{\pgfqpoint{3.913334in}{1.930426in}}{\pgfqpoint{3.921234in}{1.933698in}}{\pgfqpoint{3.927058in}{1.939522in}}%
\pgfpathcurveto{\pgfqpoint{3.932882in}{1.945346in}}{\pgfqpoint{3.936155in}{1.953246in}}{\pgfqpoint{3.936155in}{1.961482in}}%
\pgfpathcurveto{\pgfqpoint{3.936155in}{1.969719in}}{\pgfqpoint{3.932882in}{1.977619in}}{\pgfqpoint{3.927058in}{1.983443in}}%
\pgfpathcurveto{\pgfqpoint{3.921234in}{1.989267in}}{\pgfqpoint{3.913334in}{1.992539in}}{\pgfqpoint{3.905098in}{1.992539in}}%
\pgfpathcurveto{\pgfqpoint{3.896862in}{1.992539in}}{\pgfqpoint{3.888962in}{1.989267in}}{\pgfqpoint{3.883138in}{1.983443in}}%
\pgfpathcurveto{\pgfqpoint{3.877314in}{1.977619in}}{\pgfqpoint{3.874042in}{1.969719in}}{\pgfqpoint{3.874042in}{1.961482in}}%
\pgfpathcurveto{\pgfqpoint{3.874042in}{1.953246in}}{\pgfqpoint{3.877314in}{1.945346in}}{\pgfqpoint{3.883138in}{1.939522in}}%
\pgfpathcurveto{\pgfqpoint{3.888962in}{1.933698in}}{\pgfqpoint{3.896862in}{1.930426in}}{\pgfqpoint{3.905098in}{1.930426in}}%
\pgfpathclose%
\pgfusepath{stroke,fill}%
\end{pgfscope}%
\begin{pgfscope}%
\pgfpathrectangle{\pgfqpoint{3.793912in}{0.557870in}}{\pgfqpoint{2.446088in}{1.484734in}}%
\pgfusepath{clip}%
\pgfsetbuttcap%
\pgfsetroundjoin%
\definecolor{currentfill}{rgb}{0.298039,0.447059,0.690196}%
\pgfsetfillcolor{currentfill}%
\pgfsetlinewidth{1.003750pt}%
\definecolor{currentstroke}{rgb}{0.298039,0.447059,0.690196}%
\pgfsetstrokecolor{currentstroke}%
\pgfsetdash{}{0pt}%
\pgfpathmoveto{\pgfqpoint{3.905098in}{1.930426in}}%
\pgfpathcurveto{\pgfqpoint{3.913334in}{1.930426in}}{\pgfqpoint{3.921234in}{1.933698in}}{\pgfqpoint{3.927058in}{1.939522in}}%
\pgfpathcurveto{\pgfqpoint{3.932882in}{1.945346in}}{\pgfqpoint{3.936155in}{1.953246in}}{\pgfqpoint{3.936155in}{1.961482in}}%
\pgfpathcurveto{\pgfqpoint{3.936155in}{1.969719in}}{\pgfqpoint{3.932882in}{1.977619in}}{\pgfqpoint{3.927058in}{1.983443in}}%
\pgfpathcurveto{\pgfqpoint{3.921234in}{1.989267in}}{\pgfqpoint{3.913334in}{1.992539in}}{\pgfqpoint{3.905098in}{1.992539in}}%
\pgfpathcurveto{\pgfqpoint{3.896862in}{1.992539in}}{\pgfqpoint{3.888962in}{1.989267in}}{\pgfqpoint{3.883138in}{1.983443in}}%
\pgfpathcurveto{\pgfqpoint{3.877314in}{1.977619in}}{\pgfqpoint{3.874042in}{1.969719in}}{\pgfqpoint{3.874042in}{1.961482in}}%
\pgfpathcurveto{\pgfqpoint{3.874042in}{1.953246in}}{\pgfqpoint{3.877314in}{1.945346in}}{\pgfqpoint{3.883138in}{1.939522in}}%
\pgfpathcurveto{\pgfqpoint{3.888962in}{1.933698in}}{\pgfqpoint{3.896862in}{1.930426in}}{\pgfqpoint{3.905098in}{1.930426in}}%
\pgfpathclose%
\pgfusepath{stroke,fill}%
\end{pgfscope}%
\begin{pgfscope}%
\pgfpathrectangle{\pgfqpoint{3.793912in}{0.557870in}}{\pgfqpoint{2.446088in}{1.484734in}}%
\pgfusepath{clip}%
\pgfsetbuttcap%
\pgfsetroundjoin%
\definecolor{currentfill}{rgb}{0.298039,0.447059,0.690196}%
\pgfsetfillcolor{currentfill}%
\pgfsetlinewidth{1.003750pt}%
\definecolor{currentstroke}{rgb}{0.298039,0.447059,0.690196}%
\pgfsetstrokecolor{currentstroke}%
\pgfsetdash{}{0pt}%
\pgfpathmoveto{\pgfqpoint{4.553682in}{0.594302in}}%
\pgfpathcurveto{\pgfqpoint{4.561918in}{0.594302in}}{\pgfqpoint{4.569818in}{0.597574in}}{\pgfqpoint{4.575642in}{0.603398in}}%
\pgfpathcurveto{\pgfqpoint{4.581466in}{0.609222in}}{\pgfqpoint{4.584738in}{0.617122in}}{\pgfqpoint{4.584738in}{0.625358in}}%
\pgfpathcurveto{\pgfqpoint{4.584738in}{0.633594in}}{\pgfqpoint{4.581466in}{0.641495in}}{\pgfqpoint{4.575642in}{0.647318in}}%
\pgfpathcurveto{\pgfqpoint{4.569818in}{0.653142in}}{\pgfqpoint{4.561918in}{0.656415in}}{\pgfqpoint{4.553682in}{0.656415in}}%
\pgfpathcurveto{\pgfqpoint{4.545446in}{0.656415in}}{\pgfqpoint{4.537546in}{0.653142in}}{\pgfqpoint{4.531722in}{0.647318in}}%
\pgfpathcurveto{\pgfqpoint{4.525898in}{0.641495in}}{\pgfqpoint{4.522625in}{0.633594in}}{\pgfqpoint{4.522625in}{0.625358in}}%
\pgfpathcurveto{\pgfqpoint{4.522625in}{0.617122in}}{\pgfqpoint{4.525898in}{0.609222in}}{\pgfqpoint{4.531722in}{0.603398in}}%
\pgfpathcurveto{\pgfqpoint{4.537546in}{0.597574in}}{\pgfqpoint{4.545446in}{0.594302in}}{\pgfqpoint{4.553682in}{0.594302in}}%
\pgfpathclose%
\pgfusepath{stroke,fill}%
\end{pgfscope}%
\begin{pgfscope}%
\pgfpathrectangle{\pgfqpoint{3.793912in}{0.557870in}}{\pgfqpoint{2.446088in}{1.484734in}}%
\pgfusepath{clip}%
\pgfsetbuttcap%
\pgfsetroundjoin%
\definecolor{currentfill}{rgb}{0.298039,0.447059,0.690196}%
\pgfsetfillcolor{currentfill}%
\pgfsetlinewidth{1.003750pt}%
\definecolor{currentstroke}{rgb}{0.298039,0.447059,0.690196}%
\pgfsetstrokecolor{currentstroke}%
\pgfsetdash{}{0pt}%
\pgfpathmoveto{\pgfqpoint{3.905098in}{1.930426in}}%
\pgfpathcurveto{\pgfqpoint{3.913334in}{1.930426in}}{\pgfqpoint{3.921234in}{1.933698in}}{\pgfqpoint{3.927058in}{1.939522in}}%
\pgfpathcurveto{\pgfqpoint{3.932882in}{1.945346in}}{\pgfqpoint{3.936155in}{1.953246in}}{\pgfqpoint{3.936155in}{1.961482in}}%
\pgfpathcurveto{\pgfqpoint{3.936155in}{1.969719in}}{\pgfqpoint{3.932882in}{1.977619in}}{\pgfqpoint{3.927058in}{1.983443in}}%
\pgfpathcurveto{\pgfqpoint{3.921234in}{1.989267in}}{\pgfqpoint{3.913334in}{1.992539in}}{\pgfqpoint{3.905098in}{1.992539in}}%
\pgfpathcurveto{\pgfqpoint{3.896862in}{1.992539in}}{\pgfqpoint{3.888962in}{1.989267in}}{\pgfqpoint{3.883138in}{1.983443in}}%
\pgfpathcurveto{\pgfqpoint{3.877314in}{1.977619in}}{\pgfqpoint{3.874042in}{1.969719in}}{\pgfqpoint{3.874042in}{1.961482in}}%
\pgfpathcurveto{\pgfqpoint{3.874042in}{1.953246in}}{\pgfqpoint{3.877314in}{1.945346in}}{\pgfqpoint{3.883138in}{1.939522in}}%
\pgfpathcurveto{\pgfqpoint{3.888962in}{1.933698in}}{\pgfqpoint{3.896862in}{1.930426in}}{\pgfqpoint{3.905098in}{1.930426in}}%
\pgfpathclose%
\pgfusepath{stroke,fill}%
\end{pgfscope}%
\begin{pgfscope}%
\pgfpathrectangle{\pgfqpoint{3.793912in}{0.557870in}}{\pgfqpoint{2.446088in}{1.484734in}}%
\pgfusepath{clip}%
\pgfsetbuttcap%
\pgfsetroundjoin%
\definecolor{currentfill}{rgb}{0.298039,0.447059,0.690196}%
\pgfsetfillcolor{currentfill}%
\pgfsetlinewidth{1.003750pt}%
\definecolor{currentstroke}{rgb}{0.298039,0.447059,0.690196}%
\pgfsetstrokecolor{currentstroke}%
\pgfsetdash{}{0pt}%
\pgfpathmoveto{\pgfqpoint{4.623173in}{0.594302in}}%
\pgfpathcurveto{\pgfqpoint{4.631409in}{0.594302in}}{\pgfqpoint{4.639309in}{0.597574in}}{\pgfqpoint{4.645133in}{0.603398in}}%
\pgfpathcurveto{\pgfqpoint{4.650957in}{0.609222in}}{\pgfqpoint{4.654230in}{0.617122in}}{\pgfqpoint{4.654230in}{0.625358in}}%
\pgfpathcurveto{\pgfqpoint{4.654230in}{0.633594in}}{\pgfqpoint{4.650957in}{0.641495in}}{\pgfqpoint{4.645133in}{0.647318in}}%
\pgfpathcurveto{\pgfqpoint{4.639309in}{0.653142in}}{\pgfqpoint{4.631409in}{0.656415in}}{\pgfqpoint{4.623173in}{0.656415in}}%
\pgfpathcurveto{\pgfqpoint{4.614937in}{0.656415in}}{\pgfqpoint{4.607037in}{0.653142in}}{\pgfqpoint{4.601213in}{0.647318in}}%
\pgfpathcurveto{\pgfqpoint{4.595389in}{0.641495in}}{\pgfqpoint{4.592117in}{0.633594in}}{\pgfqpoint{4.592117in}{0.625358in}}%
\pgfpathcurveto{\pgfqpoint{4.592117in}{0.617122in}}{\pgfqpoint{4.595389in}{0.609222in}}{\pgfqpoint{4.601213in}{0.603398in}}%
\pgfpathcurveto{\pgfqpoint{4.607037in}{0.597574in}}{\pgfqpoint{4.614937in}{0.594302in}}{\pgfqpoint{4.623173in}{0.594302in}}%
\pgfpathclose%
\pgfusepath{stroke,fill}%
\end{pgfscope}%
\begin{pgfscope}%
\pgfpathrectangle{\pgfqpoint{3.793912in}{0.557870in}}{\pgfqpoint{2.446088in}{1.484734in}}%
\pgfusepath{clip}%
\pgfsetbuttcap%
\pgfsetroundjoin%
\definecolor{currentfill}{rgb}{0.298039,0.447059,0.690196}%
\pgfsetfillcolor{currentfill}%
\pgfsetlinewidth{1.003750pt}%
\definecolor{currentstroke}{rgb}{0.298039,0.447059,0.690196}%
\pgfsetstrokecolor{currentstroke}%
\pgfsetdash{}{0pt}%
\pgfpathmoveto{\pgfqpoint{3.905098in}{1.930426in}}%
\pgfpathcurveto{\pgfqpoint{3.913334in}{1.930426in}}{\pgfqpoint{3.921234in}{1.933698in}}{\pgfqpoint{3.927058in}{1.939522in}}%
\pgfpathcurveto{\pgfqpoint{3.932882in}{1.945346in}}{\pgfqpoint{3.936155in}{1.953246in}}{\pgfqpoint{3.936155in}{1.961482in}}%
\pgfpathcurveto{\pgfqpoint{3.936155in}{1.969719in}}{\pgfqpoint{3.932882in}{1.977619in}}{\pgfqpoint{3.927058in}{1.983443in}}%
\pgfpathcurveto{\pgfqpoint{3.921234in}{1.989267in}}{\pgfqpoint{3.913334in}{1.992539in}}{\pgfqpoint{3.905098in}{1.992539in}}%
\pgfpathcurveto{\pgfqpoint{3.896862in}{1.992539in}}{\pgfqpoint{3.888962in}{1.989267in}}{\pgfqpoint{3.883138in}{1.983443in}}%
\pgfpathcurveto{\pgfqpoint{3.877314in}{1.977619in}}{\pgfqpoint{3.874042in}{1.969719in}}{\pgfqpoint{3.874042in}{1.961482in}}%
\pgfpathcurveto{\pgfqpoint{3.874042in}{1.953246in}}{\pgfqpoint{3.877314in}{1.945346in}}{\pgfqpoint{3.883138in}{1.939522in}}%
\pgfpathcurveto{\pgfqpoint{3.888962in}{1.933698in}}{\pgfqpoint{3.896862in}{1.930426in}}{\pgfqpoint{3.905098in}{1.930426in}}%
\pgfpathclose%
\pgfusepath{stroke,fill}%
\end{pgfscope}%
\begin{pgfscope}%
\pgfpathrectangle{\pgfqpoint{3.793912in}{0.557870in}}{\pgfqpoint{2.446088in}{1.484734in}}%
\pgfusepath{clip}%
\pgfsetbuttcap%
\pgfsetroundjoin%
\definecolor{currentfill}{rgb}{0.298039,0.447059,0.690196}%
\pgfsetfillcolor{currentfill}%
\pgfsetlinewidth{1.003750pt}%
\definecolor{currentstroke}{rgb}{0.298039,0.447059,0.690196}%
\pgfsetstrokecolor{currentstroke}%
\pgfsetdash{}{0pt}%
\pgfpathmoveto{\pgfqpoint{5.920341in}{0.594302in}}%
\pgfpathcurveto{\pgfqpoint{5.928577in}{0.594302in}}{\pgfqpoint{5.936477in}{0.597574in}}{\pgfqpoint{5.942301in}{0.603398in}}%
\pgfpathcurveto{\pgfqpoint{5.948125in}{0.609222in}}{\pgfqpoint{5.951397in}{0.617122in}}{\pgfqpoint{5.951397in}{0.625358in}}%
\pgfpathcurveto{\pgfqpoint{5.951397in}{0.633594in}}{\pgfqpoint{5.948125in}{0.641495in}}{\pgfqpoint{5.942301in}{0.647318in}}%
\pgfpathcurveto{\pgfqpoint{5.936477in}{0.653142in}}{\pgfqpoint{5.928577in}{0.656415in}}{\pgfqpoint{5.920341in}{0.656415in}}%
\pgfpathcurveto{\pgfqpoint{5.912105in}{0.656415in}}{\pgfqpoint{5.904204in}{0.653142in}}{\pgfqpoint{5.898381in}{0.647318in}}%
\pgfpathcurveto{\pgfqpoint{5.892557in}{0.641495in}}{\pgfqpoint{5.889284in}{0.633594in}}{\pgfqpoint{5.889284in}{0.625358in}}%
\pgfpathcurveto{\pgfqpoint{5.889284in}{0.617122in}}{\pgfqpoint{5.892557in}{0.609222in}}{\pgfqpoint{5.898381in}{0.603398in}}%
\pgfpathcurveto{\pgfqpoint{5.904204in}{0.597574in}}{\pgfqpoint{5.912105in}{0.594302in}}{\pgfqpoint{5.920341in}{0.594302in}}%
\pgfpathclose%
\pgfusepath{stroke,fill}%
\end{pgfscope}%
\begin{pgfscope}%
\pgfpathrectangle{\pgfqpoint{3.793912in}{0.557870in}}{\pgfqpoint{2.446088in}{1.484734in}}%
\pgfusepath{clip}%
\pgfsetbuttcap%
\pgfsetroundjoin%
\definecolor{currentfill}{rgb}{0.298039,0.447059,0.690196}%
\pgfsetfillcolor{currentfill}%
\pgfsetlinewidth{1.003750pt}%
\definecolor{currentstroke}{rgb}{0.298039,0.447059,0.690196}%
\pgfsetstrokecolor{currentstroke}%
\pgfsetdash{}{0pt}%
\pgfpathmoveto{\pgfqpoint{3.905098in}{1.930426in}}%
\pgfpathcurveto{\pgfqpoint{3.913334in}{1.930426in}}{\pgfqpoint{3.921234in}{1.933698in}}{\pgfqpoint{3.927058in}{1.939522in}}%
\pgfpathcurveto{\pgfqpoint{3.932882in}{1.945346in}}{\pgfqpoint{3.936155in}{1.953246in}}{\pgfqpoint{3.936155in}{1.961482in}}%
\pgfpathcurveto{\pgfqpoint{3.936155in}{1.969719in}}{\pgfqpoint{3.932882in}{1.977619in}}{\pgfqpoint{3.927058in}{1.983443in}}%
\pgfpathcurveto{\pgfqpoint{3.921234in}{1.989267in}}{\pgfqpoint{3.913334in}{1.992539in}}{\pgfqpoint{3.905098in}{1.992539in}}%
\pgfpathcurveto{\pgfqpoint{3.896862in}{1.992539in}}{\pgfqpoint{3.888962in}{1.989267in}}{\pgfqpoint{3.883138in}{1.983443in}}%
\pgfpathcurveto{\pgfqpoint{3.877314in}{1.977619in}}{\pgfqpoint{3.874042in}{1.969719in}}{\pgfqpoint{3.874042in}{1.961482in}}%
\pgfpathcurveto{\pgfqpoint{3.874042in}{1.953246in}}{\pgfqpoint{3.877314in}{1.945346in}}{\pgfqpoint{3.883138in}{1.939522in}}%
\pgfpathcurveto{\pgfqpoint{3.888962in}{1.933698in}}{\pgfqpoint{3.896862in}{1.930426in}}{\pgfqpoint{3.905098in}{1.930426in}}%
\pgfpathclose%
\pgfusepath{stroke,fill}%
\end{pgfscope}%
\begin{pgfscope}%
\pgfpathrectangle{\pgfqpoint{3.793912in}{0.557870in}}{\pgfqpoint{2.446088in}{1.484734in}}%
\pgfusepath{clip}%
\pgfsetbuttcap%
\pgfsetroundjoin%
\definecolor{currentfill}{rgb}{0.298039,0.447059,0.690196}%
\pgfsetfillcolor{currentfill}%
\pgfsetlinewidth{1.003750pt}%
\definecolor{currentstroke}{rgb}{0.298039,0.447059,0.690196}%
\pgfsetstrokecolor{currentstroke}%
\pgfsetdash{}{0pt}%
\pgfpathmoveto{\pgfqpoint{3.905098in}{0.635203in}}%
\pgfpathcurveto{\pgfqpoint{3.913334in}{0.635203in}}{\pgfqpoint{3.921234in}{0.638476in}}{\pgfqpoint{3.927058in}{0.644300in}}%
\pgfpathcurveto{\pgfqpoint{3.932882in}{0.650124in}}{\pgfqpoint{3.936155in}{0.658024in}}{\pgfqpoint{3.936155in}{0.666260in}}%
\pgfpathcurveto{\pgfqpoint{3.936155in}{0.674496in}}{\pgfqpoint{3.932882in}{0.682396in}}{\pgfqpoint{3.927058in}{0.688220in}}%
\pgfpathcurveto{\pgfqpoint{3.921234in}{0.694044in}}{\pgfqpoint{3.913334in}{0.697316in}}{\pgfqpoint{3.905098in}{0.697316in}}%
\pgfpathcurveto{\pgfqpoint{3.896862in}{0.697316in}}{\pgfqpoint{3.888962in}{0.694044in}}{\pgfqpoint{3.883138in}{0.688220in}}%
\pgfpathcurveto{\pgfqpoint{3.877314in}{0.682396in}}{\pgfqpoint{3.874042in}{0.674496in}}{\pgfqpoint{3.874042in}{0.666260in}}%
\pgfpathcurveto{\pgfqpoint{3.874042in}{0.658024in}}{\pgfqpoint{3.877314in}{0.650124in}}{\pgfqpoint{3.883138in}{0.644300in}}%
\pgfpathcurveto{\pgfqpoint{3.888962in}{0.638476in}}{\pgfqpoint{3.896862in}{0.635203in}}{\pgfqpoint{3.905098in}{0.635203in}}%
\pgfpathclose%
\pgfusepath{stroke,fill}%
\end{pgfscope}%
\begin{pgfscope}%
\pgfpathrectangle{\pgfqpoint{3.793912in}{0.557870in}}{\pgfqpoint{2.446088in}{1.484734in}}%
\pgfusepath{clip}%
\pgfsetbuttcap%
\pgfsetroundjoin%
\definecolor{currentfill}{rgb}{0.298039,0.447059,0.690196}%
\pgfsetfillcolor{currentfill}%
\pgfsetlinewidth{1.003750pt}%
\definecolor{currentstroke}{rgb}{0.298039,0.447059,0.690196}%
\pgfsetstrokecolor{currentstroke}%
\pgfsetdash{}{0pt}%
\pgfpathmoveto{\pgfqpoint{3.905098in}{1.930426in}}%
\pgfpathcurveto{\pgfqpoint{3.913334in}{1.930426in}}{\pgfqpoint{3.921234in}{1.933698in}}{\pgfqpoint{3.927058in}{1.939522in}}%
\pgfpathcurveto{\pgfqpoint{3.932882in}{1.945346in}}{\pgfqpoint{3.936155in}{1.953246in}}{\pgfqpoint{3.936155in}{1.961482in}}%
\pgfpathcurveto{\pgfqpoint{3.936155in}{1.969719in}}{\pgfqpoint{3.932882in}{1.977619in}}{\pgfqpoint{3.927058in}{1.983443in}}%
\pgfpathcurveto{\pgfqpoint{3.921234in}{1.989267in}}{\pgfqpoint{3.913334in}{1.992539in}}{\pgfqpoint{3.905098in}{1.992539in}}%
\pgfpathcurveto{\pgfqpoint{3.896862in}{1.992539in}}{\pgfqpoint{3.888962in}{1.989267in}}{\pgfqpoint{3.883138in}{1.983443in}}%
\pgfpathcurveto{\pgfqpoint{3.877314in}{1.977619in}}{\pgfqpoint{3.874042in}{1.969719in}}{\pgfqpoint{3.874042in}{1.961482in}}%
\pgfpathcurveto{\pgfqpoint{3.874042in}{1.953246in}}{\pgfqpoint{3.877314in}{1.945346in}}{\pgfqpoint{3.883138in}{1.939522in}}%
\pgfpathcurveto{\pgfqpoint{3.888962in}{1.933698in}}{\pgfqpoint{3.896862in}{1.930426in}}{\pgfqpoint{3.905098in}{1.930426in}}%
\pgfpathclose%
\pgfusepath{stroke,fill}%
\end{pgfscope}%
\begin{pgfscope}%
\pgfpathrectangle{\pgfqpoint{3.793912in}{0.557870in}}{\pgfqpoint{2.446088in}{1.484734in}}%
\pgfusepath{clip}%
\pgfsetbuttcap%
\pgfsetroundjoin%
\definecolor{currentfill}{rgb}{0.298039,0.447059,0.690196}%
\pgfsetfillcolor{currentfill}%
\pgfsetlinewidth{1.003750pt}%
\definecolor{currentstroke}{rgb}{0.298039,0.447059,0.690196}%
\pgfsetstrokecolor{currentstroke}%
\pgfsetdash{}{0pt}%
\pgfpathmoveto{\pgfqpoint{5.179102in}{0.594302in}}%
\pgfpathcurveto{\pgfqpoint{5.187338in}{0.594302in}}{\pgfqpoint{5.195238in}{0.597574in}}{\pgfqpoint{5.201062in}{0.603398in}}%
\pgfpathcurveto{\pgfqpoint{5.206886in}{0.609222in}}{\pgfqpoint{5.210159in}{0.617122in}}{\pgfqpoint{5.210159in}{0.625358in}}%
\pgfpathcurveto{\pgfqpoint{5.210159in}{0.633594in}}{\pgfqpoint{5.206886in}{0.641495in}}{\pgfqpoint{5.201062in}{0.647318in}}%
\pgfpathcurveto{\pgfqpoint{5.195238in}{0.653142in}}{\pgfqpoint{5.187338in}{0.656415in}}{\pgfqpoint{5.179102in}{0.656415in}}%
\pgfpathcurveto{\pgfqpoint{5.170866in}{0.656415in}}{\pgfqpoint{5.162966in}{0.653142in}}{\pgfqpoint{5.157142in}{0.647318in}}%
\pgfpathcurveto{\pgfqpoint{5.151318in}{0.641495in}}{\pgfqpoint{5.148046in}{0.633594in}}{\pgfqpoint{5.148046in}{0.625358in}}%
\pgfpathcurveto{\pgfqpoint{5.148046in}{0.617122in}}{\pgfqpoint{5.151318in}{0.609222in}}{\pgfqpoint{5.157142in}{0.603398in}}%
\pgfpathcurveto{\pgfqpoint{5.162966in}{0.597574in}}{\pgfqpoint{5.170866in}{0.594302in}}{\pgfqpoint{5.179102in}{0.594302in}}%
\pgfpathclose%
\pgfusepath{stroke,fill}%
\end{pgfscope}%
\begin{pgfscope}%
\pgfpathrectangle{\pgfqpoint{3.793912in}{0.557870in}}{\pgfqpoint{2.446088in}{1.484734in}}%
\pgfusepath{clip}%
\pgfsetbuttcap%
\pgfsetroundjoin%
\definecolor{currentfill}{rgb}{0.298039,0.447059,0.690196}%
\pgfsetfillcolor{currentfill}%
\pgfsetlinewidth{1.003750pt}%
\definecolor{currentstroke}{rgb}{0.298039,0.447059,0.690196}%
\pgfsetstrokecolor{currentstroke}%
\pgfsetdash{}{0pt}%
\pgfpathmoveto{\pgfqpoint{3.905098in}{1.930426in}}%
\pgfpathcurveto{\pgfqpoint{3.913334in}{1.930426in}}{\pgfqpoint{3.921234in}{1.933698in}}{\pgfqpoint{3.927058in}{1.939522in}}%
\pgfpathcurveto{\pgfqpoint{3.932882in}{1.945346in}}{\pgfqpoint{3.936155in}{1.953246in}}{\pgfqpoint{3.936155in}{1.961482in}}%
\pgfpathcurveto{\pgfqpoint{3.936155in}{1.969719in}}{\pgfqpoint{3.932882in}{1.977619in}}{\pgfqpoint{3.927058in}{1.983443in}}%
\pgfpathcurveto{\pgfqpoint{3.921234in}{1.989267in}}{\pgfqpoint{3.913334in}{1.992539in}}{\pgfqpoint{3.905098in}{1.992539in}}%
\pgfpathcurveto{\pgfqpoint{3.896862in}{1.992539in}}{\pgfqpoint{3.888962in}{1.989267in}}{\pgfqpoint{3.883138in}{1.983443in}}%
\pgfpathcurveto{\pgfqpoint{3.877314in}{1.977619in}}{\pgfqpoint{3.874042in}{1.969719in}}{\pgfqpoint{3.874042in}{1.961482in}}%
\pgfpathcurveto{\pgfqpoint{3.874042in}{1.953246in}}{\pgfqpoint{3.877314in}{1.945346in}}{\pgfqpoint{3.883138in}{1.939522in}}%
\pgfpathcurveto{\pgfqpoint{3.888962in}{1.933698in}}{\pgfqpoint{3.896862in}{1.930426in}}{\pgfqpoint{3.905098in}{1.930426in}}%
\pgfpathclose%
\pgfusepath{stroke,fill}%
\end{pgfscope}%
\begin{pgfscope}%
\pgfpathrectangle{\pgfqpoint{3.793912in}{0.557870in}}{\pgfqpoint{2.446088in}{1.484734in}}%
\pgfusepath{clip}%
\pgfsetbuttcap%
\pgfsetroundjoin%
\definecolor{currentfill}{rgb}{0.298039,0.447059,0.690196}%
\pgfsetfillcolor{currentfill}%
\pgfsetlinewidth{1.003750pt}%
\definecolor{currentstroke}{rgb}{0.298039,0.447059,0.690196}%
\pgfsetstrokecolor{currentstroke}%
\pgfsetdash{}{0pt}%
\pgfpathmoveto{\pgfqpoint{3.951425in}{0.594302in}}%
\pgfpathcurveto{\pgfqpoint{3.959662in}{0.594302in}}{\pgfqpoint{3.967562in}{0.597574in}}{\pgfqpoint{3.973386in}{0.603398in}}%
\pgfpathcurveto{\pgfqpoint{3.979210in}{0.609222in}}{\pgfqpoint{3.982482in}{0.617122in}}{\pgfqpoint{3.982482in}{0.625358in}}%
\pgfpathcurveto{\pgfqpoint{3.982482in}{0.633594in}}{\pgfqpoint{3.979210in}{0.641495in}}{\pgfqpoint{3.973386in}{0.647318in}}%
\pgfpathcurveto{\pgfqpoint{3.967562in}{0.653142in}}{\pgfqpoint{3.959662in}{0.656415in}}{\pgfqpoint{3.951425in}{0.656415in}}%
\pgfpathcurveto{\pgfqpoint{3.943189in}{0.656415in}}{\pgfqpoint{3.935289in}{0.653142in}}{\pgfqpoint{3.929465in}{0.647318in}}%
\pgfpathcurveto{\pgfqpoint{3.923641in}{0.641495in}}{\pgfqpoint{3.920369in}{0.633594in}}{\pgfqpoint{3.920369in}{0.625358in}}%
\pgfpathcurveto{\pgfqpoint{3.920369in}{0.617122in}}{\pgfqpoint{3.923641in}{0.609222in}}{\pgfqpoint{3.929465in}{0.603398in}}%
\pgfpathcurveto{\pgfqpoint{3.935289in}{0.597574in}}{\pgfqpoint{3.943189in}{0.594302in}}{\pgfqpoint{3.951425in}{0.594302in}}%
\pgfpathclose%
\pgfusepath{stroke,fill}%
\end{pgfscope}%
\begin{pgfscope}%
\pgfpathrectangle{\pgfqpoint{3.793912in}{0.557870in}}{\pgfqpoint{2.446088in}{1.484734in}}%
\pgfusepath{clip}%
\pgfsetbuttcap%
\pgfsetroundjoin%
\definecolor{currentfill}{rgb}{0.298039,0.447059,0.690196}%
\pgfsetfillcolor{currentfill}%
\pgfsetlinewidth{1.003750pt}%
\definecolor{currentstroke}{rgb}{0.298039,0.447059,0.690196}%
\pgfsetstrokecolor{currentstroke}%
\pgfsetdash{}{0pt}%
\pgfpathmoveto{\pgfqpoint{3.951425in}{0.594302in}}%
\pgfpathcurveto{\pgfqpoint{3.959662in}{0.594302in}}{\pgfqpoint{3.967562in}{0.597574in}}{\pgfqpoint{3.973386in}{0.603398in}}%
\pgfpathcurveto{\pgfqpoint{3.979210in}{0.609222in}}{\pgfqpoint{3.982482in}{0.617122in}}{\pgfqpoint{3.982482in}{0.625358in}}%
\pgfpathcurveto{\pgfqpoint{3.982482in}{0.633594in}}{\pgfqpoint{3.979210in}{0.641495in}}{\pgfqpoint{3.973386in}{0.647318in}}%
\pgfpathcurveto{\pgfqpoint{3.967562in}{0.653142in}}{\pgfqpoint{3.959662in}{0.656415in}}{\pgfqpoint{3.951425in}{0.656415in}}%
\pgfpathcurveto{\pgfqpoint{3.943189in}{0.656415in}}{\pgfqpoint{3.935289in}{0.653142in}}{\pgfqpoint{3.929465in}{0.647318in}}%
\pgfpathcurveto{\pgfqpoint{3.923641in}{0.641495in}}{\pgfqpoint{3.920369in}{0.633594in}}{\pgfqpoint{3.920369in}{0.625358in}}%
\pgfpathcurveto{\pgfqpoint{3.920369in}{0.617122in}}{\pgfqpoint{3.923641in}{0.609222in}}{\pgfqpoint{3.929465in}{0.603398in}}%
\pgfpathcurveto{\pgfqpoint{3.935289in}{0.597574in}}{\pgfqpoint{3.943189in}{0.594302in}}{\pgfqpoint{3.951425in}{0.594302in}}%
\pgfpathclose%
\pgfusepath{stroke,fill}%
\end{pgfscope}%
\begin{pgfscope}%
\pgfpathrectangle{\pgfqpoint{3.793912in}{0.557870in}}{\pgfqpoint{2.446088in}{1.484734in}}%
\pgfusepath{clip}%
\pgfsetbuttcap%
\pgfsetroundjoin%
\definecolor{currentfill}{rgb}{0.298039,0.447059,0.690196}%
\pgfsetfillcolor{currentfill}%
\pgfsetlinewidth{1.003750pt}%
\definecolor{currentstroke}{rgb}{0.298039,0.447059,0.690196}%
\pgfsetstrokecolor{currentstroke}%
\pgfsetdash{}{0pt}%
\pgfpathmoveto{\pgfqpoint{3.905098in}{1.930426in}}%
\pgfpathcurveto{\pgfqpoint{3.913334in}{1.930426in}}{\pgfqpoint{3.921234in}{1.933698in}}{\pgfqpoint{3.927058in}{1.939522in}}%
\pgfpathcurveto{\pgfqpoint{3.932882in}{1.945346in}}{\pgfqpoint{3.936155in}{1.953246in}}{\pgfqpoint{3.936155in}{1.961482in}}%
\pgfpathcurveto{\pgfqpoint{3.936155in}{1.969719in}}{\pgfqpoint{3.932882in}{1.977619in}}{\pgfqpoint{3.927058in}{1.983443in}}%
\pgfpathcurveto{\pgfqpoint{3.921234in}{1.989267in}}{\pgfqpoint{3.913334in}{1.992539in}}{\pgfqpoint{3.905098in}{1.992539in}}%
\pgfpathcurveto{\pgfqpoint{3.896862in}{1.992539in}}{\pgfqpoint{3.888962in}{1.989267in}}{\pgfqpoint{3.883138in}{1.983443in}}%
\pgfpathcurveto{\pgfqpoint{3.877314in}{1.977619in}}{\pgfqpoint{3.874042in}{1.969719in}}{\pgfqpoint{3.874042in}{1.961482in}}%
\pgfpathcurveto{\pgfqpoint{3.874042in}{1.953246in}}{\pgfqpoint{3.877314in}{1.945346in}}{\pgfqpoint{3.883138in}{1.939522in}}%
\pgfpathcurveto{\pgfqpoint{3.888962in}{1.933698in}}{\pgfqpoint{3.896862in}{1.930426in}}{\pgfqpoint{3.905098in}{1.930426in}}%
\pgfpathclose%
\pgfusepath{stroke,fill}%
\end{pgfscope}%
\begin{pgfscope}%
\pgfpathrectangle{\pgfqpoint{3.793912in}{0.557870in}}{\pgfqpoint{2.446088in}{1.484734in}}%
\pgfusepath{clip}%
\pgfsetbuttcap%
\pgfsetroundjoin%
\definecolor{currentfill}{rgb}{0.298039,0.447059,0.690196}%
\pgfsetfillcolor{currentfill}%
\pgfsetlinewidth{1.003750pt}%
\definecolor{currentstroke}{rgb}{0.298039,0.447059,0.690196}%
\pgfsetstrokecolor{currentstroke}%
\pgfsetdash{}{0pt}%
\pgfpathmoveto{\pgfqpoint{3.905098in}{1.412337in}}%
\pgfpathcurveto{\pgfqpoint{3.913334in}{1.412337in}}{\pgfqpoint{3.921234in}{1.415609in}}{\pgfqpoint{3.927058in}{1.421433in}}%
\pgfpathcurveto{\pgfqpoint{3.932882in}{1.427257in}}{\pgfqpoint{3.936155in}{1.435157in}}{\pgfqpoint{3.936155in}{1.443393in}}%
\pgfpathcurveto{\pgfqpoint{3.936155in}{1.451630in}}{\pgfqpoint{3.932882in}{1.459530in}}{\pgfqpoint{3.927058in}{1.465354in}}%
\pgfpathcurveto{\pgfqpoint{3.921234in}{1.471178in}}{\pgfqpoint{3.913334in}{1.474450in}}{\pgfqpoint{3.905098in}{1.474450in}}%
\pgfpathcurveto{\pgfqpoint{3.896862in}{1.474450in}}{\pgfqpoint{3.888962in}{1.471178in}}{\pgfqpoint{3.883138in}{1.465354in}}%
\pgfpathcurveto{\pgfqpoint{3.877314in}{1.459530in}}{\pgfqpoint{3.874042in}{1.451630in}}{\pgfqpoint{3.874042in}{1.443393in}}%
\pgfpathcurveto{\pgfqpoint{3.874042in}{1.435157in}}{\pgfqpoint{3.877314in}{1.427257in}}{\pgfqpoint{3.883138in}{1.421433in}}%
\pgfpathcurveto{\pgfqpoint{3.888962in}{1.415609in}}{\pgfqpoint{3.896862in}{1.412337in}}{\pgfqpoint{3.905098in}{1.412337in}}%
\pgfpathclose%
\pgfusepath{stroke,fill}%
\end{pgfscope}%
\begin{pgfscope}%
\pgfpathrectangle{\pgfqpoint{3.793912in}{0.557870in}}{\pgfqpoint{2.446088in}{1.484734in}}%
\pgfusepath{clip}%
\pgfsetbuttcap%
\pgfsetroundjoin%
\definecolor{currentfill}{rgb}{0.298039,0.447059,0.690196}%
\pgfsetfillcolor{currentfill}%
\pgfsetlinewidth{1.003750pt}%
\definecolor{currentstroke}{rgb}{0.298039,0.447059,0.690196}%
\pgfsetstrokecolor{currentstroke}%
\pgfsetdash{}{0pt}%
\pgfpathmoveto{\pgfqpoint{3.905098in}{1.930426in}}%
\pgfpathcurveto{\pgfqpoint{3.913334in}{1.930426in}}{\pgfqpoint{3.921234in}{1.933698in}}{\pgfqpoint{3.927058in}{1.939522in}}%
\pgfpathcurveto{\pgfqpoint{3.932882in}{1.945346in}}{\pgfqpoint{3.936155in}{1.953246in}}{\pgfqpoint{3.936155in}{1.961482in}}%
\pgfpathcurveto{\pgfqpoint{3.936155in}{1.969719in}}{\pgfqpoint{3.932882in}{1.977619in}}{\pgfqpoint{3.927058in}{1.983443in}}%
\pgfpathcurveto{\pgfqpoint{3.921234in}{1.989267in}}{\pgfqpoint{3.913334in}{1.992539in}}{\pgfqpoint{3.905098in}{1.992539in}}%
\pgfpathcurveto{\pgfqpoint{3.896862in}{1.992539in}}{\pgfqpoint{3.888962in}{1.989267in}}{\pgfqpoint{3.883138in}{1.983443in}}%
\pgfpathcurveto{\pgfqpoint{3.877314in}{1.977619in}}{\pgfqpoint{3.874042in}{1.969719in}}{\pgfqpoint{3.874042in}{1.961482in}}%
\pgfpathcurveto{\pgfqpoint{3.874042in}{1.953246in}}{\pgfqpoint{3.877314in}{1.945346in}}{\pgfqpoint{3.883138in}{1.939522in}}%
\pgfpathcurveto{\pgfqpoint{3.888962in}{1.933698in}}{\pgfqpoint{3.896862in}{1.930426in}}{\pgfqpoint{3.905098in}{1.930426in}}%
\pgfpathclose%
\pgfusepath{stroke,fill}%
\end{pgfscope}%
\begin{pgfscope}%
\pgfpathrectangle{\pgfqpoint{3.793912in}{0.557870in}}{\pgfqpoint{2.446088in}{1.484734in}}%
\pgfusepath{clip}%
\pgfsetbuttcap%
\pgfsetroundjoin%
\definecolor{currentfill}{rgb}{0.298039,0.447059,0.690196}%
\pgfsetfillcolor{currentfill}%
\pgfsetlinewidth{1.003750pt}%
\definecolor{currentstroke}{rgb}{0.298039,0.447059,0.690196}%
\pgfsetstrokecolor{currentstroke}%
\pgfsetdash{}{0pt}%
\pgfpathmoveto{\pgfqpoint{5.410739in}{0.594302in}}%
\pgfpathcurveto{\pgfqpoint{5.418975in}{0.594302in}}{\pgfqpoint{5.426876in}{0.597574in}}{\pgfqpoint{5.432699in}{0.603398in}}%
\pgfpathcurveto{\pgfqpoint{5.438523in}{0.609222in}}{\pgfqpoint{5.441796in}{0.617122in}}{\pgfqpoint{5.441796in}{0.625358in}}%
\pgfpathcurveto{\pgfqpoint{5.441796in}{0.633594in}}{\pgfqpoint{5.438523in}{0.641495in}}{\pgfqpoint{5.432699in}{0.647318in}}%
\pgfpathcurveto{\pgfqpoint{5.426876in}{0.653142in}}{\pgfqpoint{5.418975in}{0.656415in}}{\pgfqpoint{5.410739in}{0.656415in}}%
\pgfpathcurveto{\pgfqpoint{5.402503in}{0.656415in}}{\pgfqpoint{5.394603in}{0.653142in}}{\pgfqpoint{5.388779in}{0.647318in}}%
\pgfpathcurveto{\pgfqpoint{5.382955in}{0.641495in}}{\pgfqpoint{5.379683in}{0.633594in}}{\pgfqpoint{5.379683in}{0.625358in}}%
\pgfpathcurveto{\pgfqpoint{5.379683in}{0.617122in}}{\pgfqpoint{5.382955in}{0.609222in}}{\pgfqpoint{5.388779in}{0.603398in}}%
\pgfpathcurveto{\pgfqpoint{5.394603in}{0.597574in}}{\pgfqpoint{5.402503in}{0.594302in}}{\pgfqpoint{5.410739in}{0.594302in}}%
\pgfpathclose%
\pgfusepath{stroke,fill}%
\end{pgfscope}%
\begin{pgfscope}%
\pgfpathrectangle{\pgfqpoint{3.793912in}{0.557870in}}{\pgfqpoint{2.446088in}{1.484734in}}%
\pgfusepath{clip}%
\pgfsetbuttcap%
\pgfsetroundjoin%
\definecolor{currentfill}{rgb}{0.298039,0.447059,0.690196}%
\pgfsetfillcolor{currentfill}%
\pgfsetlinewidth{1.003750pt}%
\definecolor{currentstroke}{rgb}{0.298039,0.447059,0.690196}%
\pgfsetstrokecolor{currentstroke}%
\pgfsetdash{}{0pt}%
\pgfpathmoveto{\pgfqpoint{3.905098in}{1.930426in}}%
\pgfpathcurveto{\pgfqpoint{3.913334in}{1.930426in}}{\pgfqpoint{3.921234in}{1.933698in}}{\pgfqpoint{3.927058in}{1.939522in}}%
\pgfpathcurveto{\pgfqpoint{3.932882in}{1.945346in}}{\pgfqpoint{3.936155in}{1.953246in}}{\pgfqpoint{3.936155in}{1.961482in}}%
\pgfpathcurveto{\pgfqpoint{3.936155in}{1.969719in}}{\pgfqpoint{3.932882in}{1.977619in}}{\pgfqpoint{3.927058in}{1.983443in}}%
\pgfpathcurveto{\pgfqpoint{3.921234in}{1.989267in}}{\pgfqpoint{3.913334in}{1.992539in}}{\pgfqpoint{3.905098in}{1.992539in}}%
\pgfpathcurveto{\pgfqpoint{3.896862in}{1.992539in}}{\pgfqpoint{3.888962in}{1.989267in}}{\pgfqpoint{3.883138in}{1.983443in}}%
\pgfpathcurveto{\pgfqpoint{3.877314in}{1.977619in}}{\pgfqpoint{3.874042in}{1.969719in}}{\pgfqpoint{3.874042in}{1.961482in}}%
\pgfpathcurveto{\pgfqpoint{3.874042in}{1.953246in}}{\pgfqpoint{3.877314in}{1.945346in}}{\pgfqpoint{3.883138in}{1.939522in}}%
\pgfpathcurveto{\pgfqpoint{3.888962in}{1.933698in}}{\pgfqpoint{3.896862in}{1.930426in}}{\pgfqpoint{3.905098in}{1.930426in}}%
\pgfpathclose%
\pgfusepath{stroke,fill}%
\end{pgfscope}%
\begin{pgfscope}%
\pgfpathrectangle{\pgfqpoint{3.793912in}{0.557870in}}{\pgfqpoint{2.446088in}{1.484734in}}%
\pgfusepath{clip}%
\pgfsetbuttcap%
\pgfsetroundjoin%
\definecolor{currentfill}{rgb}{0.298039,0.447059,0.690196}%
\pgfsetfillcolor{currentfill}%
\pgfsetlinewidth{1.003750pt}%
\definecolor{currentstroke}{rgb}{0.298039,0.447059,0.690196}%
\pgfsetstrokecolor{currentstroke}%
\pgfsetdash{}{0pt}%
\pgfpathmoveto{\pgfqpoint{3.905098in}{1.930426in}}%
\pgfpathcurveto{\pgfqpoint{3.913334in}{1.930426in}}{\pgfqpoint{3.921234in}{1.933698in}}{\pgfqpoint{3.927058in}{1.939522in}}%
\pgfpathcurveto{\pgfqpoint{3.932882in}{1.945346in}}{\pgfqpoint{3.936155in}{1.953246in}}{\pgfqpoint{3.936155in}{1.961482in}}%
\pgfpathcurveto{\pgfqpoint{3.936155in}{1.969719in}}{\pgfqpoint{3.932882in}{1.977619in}}{\pgfqpoint{3.927058in}{1.983443in}}%
\pgfpathcurveto{\pgfqpoint{3.921234in}{1.989267in}}{\pgfqpoint{3.913334in}{1.992539in}}{\pgfqpoint{3.905098in}{1.992539in}}%
\pgfpathcurveto{\pgfqpoint{3.896862in}{1.992539in}}{\pgfqpoint{3.888962in}{1.989267in}}{\pgfqpoint{3.883138in}{1.983443in}}%
\pgfpathcurveto{\pgfqpoint{3.877314in}{1.977619in}}{\pgfqpoint{3.874042in}{1.969719in}}{\pgfqpoint{3.874042in}{1.961482in}}%
\pgfpathcurveto{\pgfqpoint{3.874042in}{1.953246in}}{\pgfqpoint{3.877314in}{1.945346in}}{\pgfqpoint{3.883138in}{1.939522in}}%
\pgfpathcurveto{\pgfqpoint{3.888962in}{1.933698in}}{\pgfqpoint{3.896862in}{1.930426in}}{\pgfqpoint{3.905098in}{1.930426in}}%
\pgfpathclose%
\pgfusepath{stroke,fill}%
\end{pgfscope}%
\begin{pgfscope}%
\pgfpathrectangle{\pgfqpoint{3.793912in}{0.557870in}}{\pgfqpoint{2.446088in}{1.484734in}}%
\pgfusepath{clip}%
\pgfsetbuttcap%
\pgfsetroundjoin%
\definecolor{currentfill}{rgb}{0.298039,0.447059,0.690196}%
\pgfsetfillcolor{currentfill}%
\pgfsetlinewidth{1.003750pt}%
\definecolor{currentstroke}{rgb}{0.298039,0.447059,0.690196}%
\pgfsetstrokecolor{currentstroke}%
\pgfsetdash{}{0pt}%
\pgfpathmoveto{\pgfqpoint{3.905098in}{1.930426in}}%
\pgfpathcurveto{\pgfqpoint{3.913334in}{1.930426in}}{\pgfqpoint{3.921234in}{1.933698in}}{\pgfqpoint{3.927058in}{1.939522in}}%
\pgfpathcurveto{\pgfqpoint{3.932882in}{1.945346in}}{\pgfqpoint{3.936155in}{1.953246in}}{\pgfqpoint{3.936155in}{1.961482in}}%
\pgfpathcurveto{\pgfqpoint{3.936155in}{1.969719in}}{\pgfqpoint{3.932882in}{1.977619in}}{\pgfqpoint{3.927058in}{1.983443in}}%
\pgfpathcurveto{\pgfqpoint{3.921234in}{1.989267in}}{\pgfqpoint{3.913334in}{1.992539in}}{\pgfqpoint{3.905098in}{1.992539in}}%
\pgfpathcurveto{\pgfqpoint{3.896862in}{1.992539in}}{\pgfqpoint{3.888962in}{1.989267in}}{\pgfqpoint{3.883138in}{1.983443in}}%
\pgfpathcurveto{\pgfqpoint{3.877314in}{1.977619in}}{\pgfqpoint{3.874042in}{1.969719in}}{\pgfqpoint{3.874042in}{1.961482in}}%
\pgfpathcurveto{\pgfqpoint{3.874042in}{1.953246in}}{\pgfqpoint{3.877314in}{1.945346in}}{\pgfqpoint{3.883138in}{1.939522in}}%
\pgfpathcurveto{\pgfqpoint{3.888962in}{1.933698in}}{\pgfqpoint{3.896862in}{1.930426in}}{\pgfqpoint{3.905098in}{1.930426in}}%
\pgfpathclose%
\pgfusepath{stroke,fill}%
\end{pgfscope}%
\begin{pgfscope}%
\pgfpathrectangle{\pgfqpoint{3.793912in}{0.557870in}}{\pgfqpoint{2.446088in}{1.484734in}}%
\pgfusepath{clip}%
\pgfsetbuttcap%
\pgfsetroundjoin%
\definecolor{currentfill}{rgb}{0.298039,0.447059,0.690196}%
\pgfsetfillcolor{currentfill}%
\pgfsetlinewidth{1.003750pt}%
\definecolor{currentstroke}{rgb}{0.298039,0.447059,0.690196}%
\pgfsetstrokecolor{currentstroke}%
\pgfsetdash{}{0pt}%
\pgfpathmoveto{\pgfqpoint{3.905098in}{1.330533in}}%
\pgfpathcurveto{\pgfqpoint{3.913334in}{1.330533in}}{\pgfqpoint{3.921234in}{1.333806in}}{\pgfqpoint{3.927058in}{1.339630in}}%
\pgfpathcurveto{\pgfqpoint{3.932882in}{1.345454in}}{\pgfqpoint{3.936155in}{1.353354in}}{\pgfqpoint{3.936155in}{1.361590in}}%
\pgfpathcurveto{\pgfqpoint{3.936155in}{1.369826in}}{\pgfqpoint{3.932882in}{1.377726in}}{\pgfqpoint{3.927058in}{1.383550in}}%
\pgfpathcurveto{\pgfqpoint{3.921234in}{1.389374in}}{\pgfqpoint{3.913334in}{1.392646in}}{\pgfqpoint{3.905098in}{1.392646in}}%
\pgfpathcurveto{\pgfqpoint{3.896862in}{1.392646in}}{\pgfqpoint{3.888962in}{1.389374in}}{\pgfqpoint{3.883138in}{1.383550in}}%
\pgfpathcurveto{\pgfqpoint{3.877314in}{1.377726in}}{\pgfqpoint{3.874042in}{1.369826in}}{\pgfqpoint{3.874042in}{1.361590in}}%
\pgfpathcurveto{\pgfqpoint{3.874042in}{1.353354in}}{\pgfqpoint{3.877314in}{1.345454in}}{\pgfqpoint{3.883138in}{1.339630in}}%
\pgfpathcurveto{\pgfqpoint{3.888962in}{1.333806in}}{\pgfqpoint{3.896862in}{1.330533in}}{\pgfqpoint{3.905098in}{1.330533in}}%
\pgfpathclose%
\pgfusepath{stroke,fill}%
\end{pgfscope}%
\begin{pgfscope}%
\pgfpathrectangle{\pgfqpoint{3.793912in}{0.557870in}}{\pgfqpoint{2.446088in}{1.484734in}}%
\pgfusepath{clip}%
\pgfsetbuttcap%
\pgfsetroundjoin%
\definecolor{currentfill}{rgb}{0.298039,0.447059,0.690196}%
\pgfsetfillcolor{currentfill}%
\pgfsetlinewidth{1.003750pt}%
\definecolor{currentstroke}{rgb}{0.298039,0.447059,0.690196}%
\pgfsetstrokecolor{currentstroke}%
\pgfsetdash{}{0pt}%
\pgfpathmoveto{\pgfqpoint{3.905098in}{1.385069in}}%
\pgfpathcurveto{\pgfqpoint{3.913334in}{1.385069in}}{\pgfqpoint{3.921234in}{1.388341in}}{\pgfqpoint{3.927058in}{1.394165in}}%
\pgfpathcurveto{\pgfqpoint{3.932882in}{1.399989in}}{\pgfqpoint{3.936155in}{1.407889in}}{\pgfqpoint{3.936155in}{1.416126in}}%
\pgfpathcurveto{\pgfqpoint{3.936155in}{1.424362in}}{\pgfqpoint{3.932882in}{1.432262in}}{\pgfqpoint{3.927058in}{1.438086in}}%
\pgfpathcurveto{\pgfqpoint{3.921234in}{1.443910in}}{\pgfqpoint{3.913334in}{1.447182in}}{\pgfqpoint{3.905098in}{1.447182in}}%
\pgfpathcurveto{\pgfqpoint{3.896862in}{1.447182in}}{\pgfqpoint{3.888962in}{1.443910in}}{\pgfqpoint{3.883138in}{1.438086in}}%
\pgfpathcurveto{\pgfqpoint{3.877314in}{1.432262in}}{\pgfqpoint{3.874042in}{1.424362in}}{\pgfqpoint{3.874042in}{1.416126in}}%
\pgfpathcurveto{\pgfqpoint{3.874042in}{1.407889in}}{\pgfqpoint{3.877314in}{1.399989in}}{\pgfqpoint{3.883138in}{1.394165in}}%
\pgfpathcurveto{\pgfqpoint{3.888962in}{1.388341in}}{\pgfqpoint{3.896862in}{1.385069in}}{\pgfqpoint{3.905098in}{1.385069in}}%
\pgfpathclose%
\pgfusepath{stroke,fill}%
\end{pgfscope}%
\begin{pgfscope}%
\pgfpathrectangle{\pgfqpoint{3.793912in}{0.557870in}}{\pgfqpoint{2.446088in}{1.484734in}}%
\pgfusepath{clip}%
\pgfsetbuttcap%
\pgfsetroundjoin%
\definecolor{currentfill}{rgb}{0.298039,0.447059,0.690196}%
\pgfsetfillcolor{currentfill}%
\pgfsetlinewidth{1.003750pt}%
\definecolor{currentstroke}{rgb}{0.298039,0.447059,0.690196}%
\pgfsetstrokecolor{currentstroke}%
\pgfsetdash{}{0pt}%
\pgfpathmoveto{\pgfqpoint{5.781359in}{0.594302in}}%
\pgfpathcurveto{\pgfqpoint{5.789595in}{0.594302in}}{\pgfqpoint{5.797495in}{0.597574in}}{\pgfqpoint{5.803319in}{0.603398in}}%
\pgfpathcurveto{\pgfqpoint{5.809143in}{0.609222in}}{\pgfqpoint{5.812415in}{0.617122in}}{\pgfqpoint{5.812415in}{0.625358in}}%
\pgfpathcurveto{\pgfqpoint{5.812415in}{0.633594in}}{\pgfqpoint{5.809143in}{0.641495in}}{\pgfqpoint{5.803319in}{0.647318in}}%
\pgfpathcurveto{\pgfqpoint{5.797495in}{0.653142in}}{\pgfqpoint{5.789595in}{0.656415in}}{\pgfqpoint{5.781359in}{0.656415in}}%
\pgfpathcurveto{\pgfqpoint{5.773122in}{0.656415in}}{\pgfqpoint{5.765222in}{0.653142in}}{\pgfqpoint{5.759398in}{0.647318in}}%
\pgfpathcurveto{\pgfqpoint{5.753574in}{0.641495in}}{\pgfqpoint{5.750302in}{0.633594in}}{\pgfqpoint{5.750302in}{0.625358in}}%
\pgfpathcurveto{\pgfqpoint{5.750302in}{0.617122in}}{\pgfqpoint{5.753574in}{0.609222in}}{\pgfqpoint{5.759398in}{0.603398in}}%
\pgfpathcurveto{\pgfqpoint{5.765222in}{0.597574in}}{\pgfqpoint{5.773122in}{0.594302in}}{\pgfqpoint{5.781359in}{0.594302in}}%
\pgfpathclose%
\pgfusepath{stroke,fill}%
\end{pgfscope}%
\begin{pgfscope}%
\pgfpathrectangle{\pgfqpoint{3.793912in}{0.557870in}}{\pgfqpoint{2.446088in}{1.484734in}}%
\pgfusepath{clip}%
\pgfsetbuttcap%
\pgfsetroundjoin%
\definecolor{currentfill}{rgb}{0.298039,0.447059,0.690196}%
\pgfsetfillcolor{currentfill}%
\pgfsetlinewidth{1.003750pt}%
\definecolor{currentstroke}{rgb}{0.298039,0.447059,0.690196}%
\pgfsetstrokecolor{currentstroke}%
\pgfsetdash{}{0pt}%
\pgfpathmoveto{\pgfqpoint{3.905098in}{1.930426in}}%
\pgfpathcurveto{\pgfqpoint{3.913334in}{1.930426in}}{\pgfqpoint{3.921234in}{1.933698in}}{\pgfqpoint{3.927058in}{1.939522in}}%
\pgfpathcurveto{\pgfqpoint{3.932882in}{1.945346in}}{\pgfqpoint{3.936155in}{1.953246in}}{\pgfqpoint{3.936155in}{1.961482in}}%
\pgfpathcurveto{\pgfqpoint{3.936155in}{1.969719in}}{\pgfqpoint{3.932882in}{1.977619in}}{\pgfqpoint{3.927058in}{1.983443in}}%
\pgfpathcurveto{\pgfqpoint{3.921234in}{1.989267in}}{\pgfqpoint{3.913334in}{1.992539in}}{\pgfqpoint{3.905098in}{1.992539in}}%
\pgfpathcurveto{\pgfqpoint{3.896862in}{1.992539in}}{\pgfqpoint{3.888962in}{1.989267in}}{\pgfqpoint{3.883138in}{1.983443in}}%
\pgfpathcurveto{\pgfqpoint{3.877314in}{1.977619in}}{\pgfqpoint{3.874042in}{1.969719in}}{\pgfqpoint{3.874042in}{1.961482in}}%
\pgfpathcurveto{\pgfqpoint{3.874042in}{1.953246in}}{\pgfqpoint{3.877314in}{1.945346in}}{\pgfqpoint{3.883138in}{1.939522in}}%
\pgfpathcurveto{\pgfqpoint{3.888962in}{1.933698in}}{\pgfqpoint{3.896862in}{1.930426in}}{\pgfqpoint{3.905098in}{1.930426in}}%
\pgfpathclose%
\pgfusepath{stroke,fill}%
\end{pgfscope}%
\begin{pgfscope}%
\pgfpathrectangle{\pgfqpoint{3.793912in}{0.557870in}}{\pgfqpoint{2.446088in}{1.484734in}}%
\pgfusepath{clip}%
\pgfsetbuttcap%
\pgfsetroundjoin%
\definecolor{currentfill}{rgb}{0.298039,0.447059,0.690196}%
\pgfsetfillcolor{currentfill}%
\pgfsetlinewidth{1.003750pt}%
\definecolor{currentstroke}{rgb}{0.298039,0.447059,0.690196}%
\pgfsetstrokecolor{currentstroke}%
\pgfsetdash{}{0pt}%
\pgfpathmoveto{\pgfqpoint{3.905098in}{1.057855in}}%
\pgfpathcurveto{\pgfqpoint{3.913334in}{1.057855in}}{\pgfqpoint{3.921234in}{1.061127in}}{\pgfqpoint{3.927058in}{1.066951in}}%
\pgfpathcurveto{\pgfqpoint{3.932882in}{1.072775in}}{\pgfqpoint{3.936155in}{1.080675in}}{\pgfqpoint{3.936155in}{1.088911in}}%
\pgfpathcurveto{\pgfqpoint{3.936155in}{1.097148in}}{\pgfqpoint{3.932882in}{1.105048in}}{\pgfqpoint{3.927058in}{1.110872in}}%
\pgfpathcurveto{\pgfqpoint{3.921234in}{1.116696in}}{\pgfqpoint{3.913334in}{1.119968in}}{\pgfqpoint{3.905098in}{1.119968in}}%
\pgfpathcurveto{\pgfqpoint{3.896862in}{1.119968in}}{\pgfqpoint{3.888962in}{1.116696in}}{\pgfqpoint{3.883138in}{1.110872in}}%
\pgfpathcurveto{\pgfqpoint{3.877314in}{1.105048in}}{\pgfqpoint{3.874042in}{1.097148in}}{\pgfqpoint{3.874042in}{1.088911in}}%
\pgfpathcurveto{\pgfqpoint{3.874042in}{1.080675in}}{\pgfqpoint{3.877314in}{1.072775in}}{\pgfqpoint{3.883138in}{1.066951in}}%
\pgfpathcurveto{\pgfqpoint{3.888962in}{1.061127in}}{\pgfqpoint{3.896862in}{1.057855in}}{\pgfqpoint{3.905098in}{1.057855in}}%
\pgfpathclose%
\pgfusepath{stroke,fill}%
\end{pgfscope}%
\begin{pgfscope}%
\pgfpathrectangle{\pgfqpoint{3.793912in}{0.557870in}}{\pgfqpoint{2.446088in}{1.484734in}}%
\pgfusepath{clip}%
\pgfsetbuttcap%
\pgfsetroundjoin%
\definecolor{currentfill}{rgb}{0.298039,0.447059,0.690196}%
\pgfsetfillcolor{currentfill}%
\pgfsetlinewidth{1.003750pt}%
\definecolor{currentstroke}{rgb}{0.298039,0.447059,0.690196}%
\pgfsetstrokecolor{currentstroke}%
\pgfsetdash{}{0pt}%
\pgfpathmoveto{\pgfqpoint{3.951425in}{0.594302in}}%
\pgfpathcurveto{\pgfqpoint{3.959662in}{0.594302in}}{\pgfqpoint{3.967562in}{0.597574in}}{\pgfqpoint{3.973386in}{0.603398in}}%
\pgfpathcurveto{\pgfqpoint{3.979210in}{0.609222in}}{\pgfqpoint{3.982482in}{0.617122in}}{\pgfqpoint{3.982482in}{0.625358in}}%
\pgfpathcurveto{\pgfqpoint{3.982482in}{0.633594in}}{\pgfqpoint{3.979210in}{0.641495in}}{\pgfqpoint{3.973386in}{0.647318in}}%
\pgfpathcurveto{\pgfqpoint{3.967562in}{0.653142in}}{\pgfqpoint{3.959662in}{0.656415in}}{\pgfqpoint{3.951425in}{0.656415in}}%
\pgfpathcurveto{\pgfqpoint{3.943189in}{0.656415in}}{\pgfqpoint{3.935289in}{0.653142in}}{\pgfqpoint{3.929465in}{0.647318in}}%
\pgfpathcurveto{\pgfqpoint{3.923641in}{0.641495in}}{\pgfqpoint{3.920369in}{0.633594in}}{\pgfqpoint{3.920369in}{0.625358in}}%
\pgfpathcurveto{\pgfqpoint{3.920369in}{0.617122in}}{\pgfqpoint{3.923641in}{0.609222in}}{\pgfqpoint{3.929465in}{0.603398in}}%
\pgfpathcurveto{\pgfqpoint{3.935289in}{0.597574in}}{\pgfqpoint{3.943189in}{0.594302in}}{\pgfqpoint{3.951425in}{0.594302in}}%
\pgfpathclose%
\pgfusepath{stroke,fill}%
\end{pgfscope}%
\begin{pgfscope}%
\pgfpathrectangle{\pgfqpoint{3.793912in}{0.557870in}}{\pgfqpoint{2.446088in}{1.484734in}}%
\pgfusepath{clip}%
\pgfsetbuttcap%
\pgfsetroundjoin%
\definecolor{currentfill}{rgb}{0.298039,0.447059,0.690196}%
\pgfsetfillcolor{currentfill}%
\pgfsetlinewidth{1.003750pt}%
\definecolor{currentstroke}{rgb}{0.298039,0.447059,0.690196}%
\pgfsetstrokecolor{currentstroke}%
\pgfsetdash{}{0pt}%
\pgfpathmoveto{\pgfqpoint{3.905098in}{0.921516in}}%
\pgfpathcurveto{\pgfqpoint{3.913334in}{0.921516in}}{\pgfqpoint{3.921234in}{0.924788in}}{\pgfqpoint{3.927058in}{0.930612in}}%
\pgfpathcurveto{\pgfqpoint{3.932882in}{0.936436in}}{\pgfqpoint{3.936155in}{0.944336in}}{\pgfqpoint{3.936155in}{0.952572in}}%
\pgfpathcurveto{\pgfqpoint{3.936155in}{0.960809in}}{\pgfqpoint{3.932882in}{0.968709in}}{\pgfqpoint{3.927058in}{0.974533in}}%
\pgfpathcurveto{\pgfqpoint{3.921234in}{0.980356in}}{\pgfqpoint{3.913334in}{0.983629in}}{\pgfqpoint{3.905098in}{0.983629in}}%
\pgfpathcurveto{\pgfqpoint{3.896862in}{0.983629in}}{\pgfqpoint{3.888962in}{0.980356in}}{\pgfqpoint{3.883138in}{0.974533in}}%
\pgfpathcurveto{\pgfqpoint{3.877314in}{0.968709in}}{\pgfqpoint{3.874042in}{0.960809in}}{\pgfqpoint{3.874042in}{0.952572in}}%
\pgfpathcurveto{\pgfqpoint{3.874042in}{0.944336in}}{\pgfqpoint{3.877314in}{0.936436in}}{\pgfqpoint{3.883138in}{0.930612in}}%
\pgfpathcurveto{\pgfqpoint{3.888962in}{0.924788in}}{\pgfqpoint{3.896862in}{0.921516in}}{\pgfqpoint{3.905098in}{0.921516in}}%
\pgfpathclose%
\pgfusepath{stroke,fill}%
\end{pgfscope}%
\begin{pgfscope}%
\pgfpathrectangle{\pgfqpoint{3.793912in}{0.557870in}}{\pgfqpoint{2.446088in}{1.484734in}}%
\pgfusepath{clip}%
\pgfsetbuttcap%
\pgfsetroundjoin%
\definecolor{currentfill}{rgb}{0.298039,0.447059,0.690196}%
\pgfsetfillcolor{currentfill}%
\pgfsetlinewidth{1.003750pt}%
\definecolor{currentstroke}{rgb}{0.298039,0.447059,0.690196}%
\pgfsetstrokecolor{currentstroke}%
\pgfsetdash{}{0pt}%
\pgfpathmoveto{\pgfqpoint{3.905098in}{1.930426in}}%
\pgfpathcurveto{\pgfqpoint{3.913334in}{1.930426in}}{\pgfqpoint{3.921234in}{1.933698in}}{\pgfqpoint{3.927058in}{1.939522in}}%
\pgfpathcurveto{\pgfqpoint{3.932882in}{1.945346in}}{\pgfqpoint{3.936155in}{1.953246in}}{\pgfqpoint{3.936155in}{1.961482in}}%
\pgfpathcurveto{\pgfqpoint{3.936155in}{1.969719in}}{\pgfqpoint{3.932882in}{1.977619in}}{\pgfqpoint{3.927058in}{1.983443in}}%
\pgfpathcurveto{\pgfqpoint{3.921234in}{1.989267in}}{\pgfqpoint{3.913334in}{1.992539in}}{\pgfqpoint{3.905098in}{1.992539in}}%
\pgfpathcurveto{\pgfqpoint{3.896862in}{1.992539in}}{\pgfqpoint{3.888962in}{1.989267in}}{\pgfqpoint{3.883138in}{1.983443in}}%
\pgfpathcurveto{\pgfqpoint{3.877314in}{1.977619in}}{\pgfqpoint{3.874042in}{1.969719in}}{\pgfqpoint{3.874042in}{1.961482in}}%
\pgfpathcurveto{\pgfqpoint{3.874042in}{1.953246in}}{\pgfqpoint{3.877314in}{1.945346in}}{\pgfqpoint{3.883138in}{1.939522in}}%
\pgfpathcurveto{\pgfqpoint{3.888962in}{1.933698in}}{\pgfqpoint{3.896862in}{1.930426in}}{\pgfqpoint{3.905098in}{1.930426in}}%
\pgfpathclose%
\pgfusepath{stroke,fill}%
\end{pgfscope}%
\begin{pgfscope}%
\pgfpathrectangle{\pgfqpoint{3.793912in}{0.557870in}}{\pgfqpoint{2.446088in}{1.484734in}}%
\pgfusepath{clip}%
\pgfsetbuttcap%
\pgfsetroundjoin%
\definecolor{currentfill}{rgb}{0.298039,0.447059,0.690196}%
\pgfsetfillcolor{currentfill}%
\pgfsetlinewidth{1.003750pt}%
\definecolor{currentstroke}{rgb}{0.298039,0.447059,0.690196}%
\pgfsetstrokecolor{currentstroke}%
\pgfsetdash{}{0pt}%
\pgfpathmoveto{\pgfqpoint{3.997753in}{0.594302in}}%
\pgfpathcurveto{\pgfqpoint{4.005989in}{0.594302in}}{\pgfqpoint{4.013889in}{0.597574in}}{\pgfqpoint{4.019713in}{0.603398in}}%
\pgfpathcurveto{\pgfqpoint{4.025537in}{0.609222in}}{\pgfqpoint{4.028809in}{0.617122in}}{\pgfqpoint{4.028809in}{0.625358in}}%
\pgfpathcurveto{\pgfqpoint{4.028809in}{0.633594in}}{\pgfqpoint{4.025537in}{0.641495in}}{\pgfqpoint{4.019713in}{0.647318in}}%
\pgfpathcurveto{\pgfqpoint{4.013889in}{0.653142in}}{\pgfqpoint{4.005989in}{0.656415in}}{\pgfqpoint{3.997753in}{0.656415in}}%
\pgfpathcurveto{\pgfqpoint{3.989517in}{0.656415in}}{\pgfqpoint{3.981617in}{0.653142in}}{\pgfqpoint{3.975793in}{0.647318in}}%
\pgfpathcurveto{\pgfqpoint{3.969969in}{0.641495in}}{\pgfqpoint{3.966696in}{0.633594in}}{\pgfqpoint{3.966696in}{0.625358in}}%
\pgfpathcurveto{\pgfqpoint{3.966696in}{0.617122in}}{\pgfqpoint{3.969969in}{0.609222in}}{\pgfqpoint{3.975793in}{0.603398in}}%
\pgfpathcurveto{\pgfqpoint{3.981617in}{0.597574in}}{\pgfqpoint{3.989517in}{0.594302in}}{\pgfqpoint{3.997753in}{0.594302in}}%
\pgfpathclose%
\pgfusepath{stroke,fill}%
\end{pgfscope}%
\begin{pgfscope}%
\pgfpathrectangle{\pgfqpoint{3.793912in}{0.557870in}}{\pgfqpoint{2.446088in}{1.484734in}}%
\pgfusepath{clip}%
\pgfsetbuttcap%
\pgfsetroundjoin%
\definecolor{currentfill}{rgb}{0.298039,0.447059,0.690196}%
\pgfsetfillcolor{currentfill}%
\pgfsetlinewidth{1.003750pt}%
\definecolor{currentstroke}{rgb}{0.298039,0.447059,0.690196}%
\pgfsetstrokecolor{currentstroke}%
\pgfsetdash{}{0pt}%
\pgfpathmoveto{\pgfqpoint{3.905098in}{1.930426in}}%
\pgfpathcurveto{\pgfqpoint{3.913334in}{1.930426in}}{\pgfqpoint{3.921234in}{1.933698in}}{\pgfqpoint{3.927058in}{1.939522in}}%
\pgfpathcurveto{\pgfqpoint{3.932882in}{1.945346in}}{\pgfqpoint{3.936155in}{1.953246in}}{\pgfqpoint{3.936155in}{1.961482in}}%
\pgfpathcurveto{\pgfqpoint{3.936155in}{1.969719in}}{\pgfqpoint{3.932882in}{1.977619in}}{\pgfqpoint{3.927058in}{1.983443in}}%
\pgfpathcurveto{\pgfqpoint{3.921234in}{1.989267in}}{\pgfqpoint{3.913334in}{1.992539in}}{\pgfqpoint{3.905098in}{1.992539in}}%
\pgfpathcurveto{\pgfqpoint{3.896862in}{1.992539in}}{\pgfqpoint{3.888962in}{1.989267in}}{\pgfqpoint{3.883138in}{1.983443in}}%
\pgfpathcurveto{\pgfqpoint{3.877314in}{1.977619in}}{\pgfqpoint{3.874042in}{1.969719in}}{\pgfqpoint{3.874042in}{1.961482in}}%
\pgfpathcurveto{\pgfqpoint{3.874042in}{1.953246in}}{\pgfqpoint{3.877314in}{1.945346in}}{\pgfqpoint{3.883138in}{1.939522in}}%
\pgfpathcurveto{\pgfqpoint{3.888962in}{1.933698in}}{\pgfqpoint{3.896862in}{1.930426in}}{\pgfqpoint{3.905098in}{1.930426in}}%
\pgfpathclose%
\pgfusepath{stroke,fill}%
\end{pgfscope}%
\begin{pgfscope}%
\pgfpathrectangle{\pgfqpoint{3.793912in}{0.557870in}}{\pgfqpoint{2.446088in}{1.484734in}}%
\pgfusepath{clip}%
\pgfsetbuttcap%
\pgfsetroundjoin%
\definecolor{currentfill}{rgb}{0.298039,0.447059,0.690196}%
\pgfsetfillcolor{currentfill}%
\pgfsetlinewidth{1.003750pt}%
\definecolor{currentstroke}{rgb}{0.298039,0.447059,0.690196}%
\pgfsetstrokecolor{currentstroke}%
\pgfsetdash{}{0pt}%
\pgfpathmoveto{\pgfqpoint{3.905098in}{1.930426in}}%
\pgfpathcurveto{\pgfqpoint{3.913334in}{1.930426in}}{\pgfqpoint{3.921234in}{1.933698in}}{\pgfqpoint{3.927058in}{1.939522in}}%
\pgfpathcurveto{\pgfqpoint{3.932882in}{1.945346in}}{\pgfqpoint{3.936155in}{1.953246in}}{\pgfqpoint{3.936155in}{1.961482in}}%
\pgfpathcurveto{\pgfqpoint{3.936155in}{1.969719in}}{\pgfqpoint{3.932882in}{1.977619in}}{\pgfqpoint{3.927058in}{1.983443in}}%
\pgfpathcurveto{\pgfqpoint{3.921234in}{1.989267in}}{\pgfqpoint{3.913334in}{1.992539in}}{\pgfqpoint{3.905098in}{1.992539in}}%
\pgfpathcurveto{\pgfqpoint{3.896862in}{1.992539in}}{\pgfqpoint{3.888962in}{1.989267in}}{\pgfqpoint{3.883138in}{1.983443in}}%
\pgfpathcurveto{\pgfqpoint{3.877314in}{1.977619in}}{\pgfqpoint{3.874042in}{1.969719in}}{\pgfqpoint{3.874042in}{1.961482in}}%
\pgfpathcurveto{\pgfqpoint{3.874042in}{1.953246in}}{\pgfqpoint{3.877314in}{1.945346in}}{\pgfqpoint{3.883138in}{1.939522in}}%
\pgfpathcurveto{\pgfqpoint{3.888962in}{1.933698in}}{\pgfqpoint{3.896862in}{1.930426in}}{\pgfqpoint{3.905098in}{1.930426in}}%
\pgfpathclose%
\pgfusepath{stroke,fill}%
\end{pgfscope}%
\begin{pgfscope}%
\pgfpathrectangle{\pgfqpoint{3.793912in}{0.557870in}}{\pgfqpoint{2.446088in}{1.484734in}}%
\pgfusepath{clip}%
\pgfsetbuttcap%
\pgfsetroundjoin%
\definecolor{currentfill}{rgb}{0.298039,0.447059,0.690196}%
\pgfsetfillcolor{currentfill}%
\pgfsetlinewidth{1.003750pt}%
\definecolor{currentstroke}{rgb}{0.298039,0.447059,0.690196}%
\pgfsetstrokecolor{currentstroke}%
\pgfsetdash{}{0pt}%
\pgfpathmoveto{\pgfqpoint{3.905098in}{1.412337in}}%
\pgfpathcurveto{\pgfqpoint{3.913334in}{1.412337in}}{\pgfqpoint{3.921234in}{1.415609in}}{\pgfqpoint{3.927058in}{1.421433in}}%
\pgfpathcurveto{\pgfqpoint{3.932882in}{1.427257in}}{\pgfqpoint{3.936155in}{1.435157in}}{\pgfqpoint{3.936155in}{1.443393in}}%
\pgfpathcurveto{\pgfqpoint{3.936155in}{1.451630in}}{\pgfqpoint{3.932882in}{1.459530in}}{\pgfqpoint{3.927058in}{1.465354in}}%
\pgfpathcurveto{\pgfqpoint{3.921234in}{1.471178in}}{\pgfqpoint{3.913334in}{1.474450in}}{\pgfqpoint{3.905098in}{1.474450in}}%
\pgfpathcurveto{\pgfqpoint{3.896862in}{1.474450in}}{\pgfqpoint{3.888962in}{1.471178in}}{\pgfqpoint{3.883138in}{1.465354in}}%
\pgfpathcurveto{\pgfqpoint{3.877314in}{1.459530in}}{\pgfqpoint{3.874042in}{1.451630in}}{\pgfqpoint{3.874042in}{1.443393in}}%
\pgfpathcurveto{\pgfqpoint{3.874042in}{1.435157in}}{\pgfqpoint{3.877314in}{1.427257in}}{\pgfqpoint{3.883138in}{1.421433in}}%
\pgfpathcurveto{\pgfqpoint{3.888962in}{1.415609in}}{\pgfqpoint{3.896862in}{1.412337in}}{\pgfqpoint{3.905098in}{1.412337in}}%
\pgfpathclose%
\pgfusepath{stroke,fill}%
\end{pgfscope}%
\begin{pgfscope}%
\pgfpathrectangle{\pgfqpoint{3.793912in}{0.557870in}}{\pgfqpoint{2.446088in}{1.484734in}}%
\pgfusepath{clip}%
\pgfsetbuttcap%
\pgfsetroundjoin%
\definecolor{currentfill}{rgb}{0.298039,0.447059,0.690196}%
\pgfsetfillcolor{currentfill}%
\pgfsetlinewidth{1.003750pt}%
\definecolor{currentstroke}{rgb}{0.298039,0.447059,0.690196}%
\pgfsetstrokecolor{currentstroke}%
\pgfsetdash{}{0pt}%
\pgfpathmoveto{\pgfqpoint{3.905098in}{0.621570in}}%
\pgfpathcurveto{\pgfqpoint{3.913334in}{0.621570in}}{\pgfqpoint{3.921234in}{0.624842in}}{\pgfqpoint{3.927058in}{0.630666in}}%
\pgfpathcurveto{\pgfqpoint{3.932882in}{0.636490in}}{\pgfqpoint{3.936155in}{0.644390in}}{\pgfqpoint{3.936155in}{0.652626in}}%
\pgfpathcurveto{\pgfqpoint{3.936155in}{0.660862in}}{\pgfqpoint{3.932882in}{0.668762in}}{\pgfqpoint{3.927058in}{0.674586in}}%
\pgfpathcurveto{\pgfqpoint{3.921234in}{0.680410in}}{\pgfqpoint{3.913334in}{0.683683in}}{\pgfqpoint{3.905098in}{0.683683in}}%
\pgfpathcurveto{\pgfqpoint{3.896862in}{0.683683in}}{\pgfqpoint{3.888962in}{0.680410in}}{\pgfqpoint{3.883138in}{0.674586in}}%
\pgfpathcurveto{\pgfqpoint{3.877314in}{0.668762in}}{\pgfqpoint{3.874042in}{0.660862in}}{\pgfqpoint{3.874042in}{0.652626in}}%
\pgfpathcurveto{\pgfqpoint{3.874042in}{0.644390in}}{\pgfqpoint{3.877314in}{0.636490in}}{\pgfqpoint{3.883138in}{0.630666in}}%
\pgfpathcurveto{\pgfqpoint{3.888962in}{0.624842in}}{\pgfqpoint{3.896862in}{0.621570in}}{\pgfqpoint{3.905098in}{0.621570in}}%
\pgfpathclose%
\pgfusepath{stroke,fill}%
\end{pgfscope}%
\begin{pgfscope}%
\pgfpathrectangle{\pgfqpoint{3.793912in}{0.557870in}}{\pgfqpoint{2.446088in}{1.484734in}}%
\pgfusepath{clip}%
\pgfsetbuttcap%
\pgfsetroundjoin%
\definecolor{currentfill}{rgb}{0.298039,0.447059,0.690196}%
\pgfsetfillcolor{currentfill}%
\pgfsetlinewidth{1.003750pt}%
\definecolor{currentstroke}{rgb}{0.298039,0.447059,0.690196}%
\pgfsetstrokecolor{currentstroke}%
\pgfsetdash{}{0pt}%
\pgfpathmoveto{\pgfqpoint{3.905098in}{1.003319in}}%
\pgfpathcurveto{\pgfqpoint{3.913334in}{1.003319in}}{\pgfqpoint{3.921234in}{1.006592in}}{\pgfqpoint{3.927058in}{1.012416in}}%
\pgfpathcurveto{\pgfqpoint{3.932882in}{1.018239in}}{\pgfqpoint{3.936155in}{1.026140in}}{\pgfqpoint{3.936155in}{1.034376in}}%
\pgfpathcurveto{\pgfqpoint{3.936155in}{1.042612in}}{\pgfqpoint{3.932882in}{1.050512in}}{\pgfqpoint{3.927058in}{1.056336in}}%
\pgfpathcurveto{\pgfqpoint{3.921234in}{1.062160in}}{\pgfqpoint{3.913334in}{1.065432in}}{\pgfqpoint{3.905098in}{1.065432in}}%
\pgfpathcurveto{\pgfqpoint{3.896862in}{1.065432in}}{\pgfqpoint{3.888962in}{1.062160in}}{\pgfqpoint{3.883138in}{1.056336in}}%
\pgfpathcurveto{\pgfqpoint{3.877314in}{1.050512in}}{\pgfqpoint{3.874042in}{1.042612in}}{\pgfqpoint{3.874042in}{1.034376in}}%
\pgfpathcurveto{\pgfqpoint{3.874042in}{1.026140in}}{\pgfqpoint{3.877314in}{1.018239in}}{\pgfqpoint{3.883138in}{1.012416in}}%
\pgfpathcurveto{\pgfqpoint{3.888962in}{1.006592in}}{\pgfqpoint{3.896862in}{1.003319in}}{\pgfqpoint{3.905098in}{1.003319in}}%
\pgfpathclose%
\pgfusepath{stroke,fill}%
\end{pgfscope}%
\begin{pgfscope}%
\pgfpathrectangle{\pgfqpoint{3.793912in}{0.557870in}}{\pgfqpoint{2.446088in}{1.484734in}}%
\pgfusepath{clip}%
\pgfsetbuttcap%
\pgfsetroundjoin%
\definecolor{currentfill}{rgb}{0.298039,0.447059,0.690196}%
\pgfsetfillcolor{currentfill}%
\pgfsetlinewidth{1.003750pt}%
\definecolor{currentstroke}{rgb}{0.298039,0.447059,0.690196}%
\pgfsetstrokecolor{currentstroke}%
\pgfsetdash{}{0pt}%
\pgfpathmoveto{\pgfqpoint{3.905098in}{1.930426in}}%
\pgfpathcurveto{\pgfqpoint{3.913334in}{1.930426in}}{\pgfqpoint{3.921234in}{1.933698in}}{\pgfqpoint{3.927058in}{1.939522in}}%
\pgfpathcurveto{\pgfqpoint{3.932882in}{1.945346in}}{\pgfqpoint{3.936155in}{1.953246in}}{\pgfqpoint{3.936155in}{1.961482in}}%
\pgfpathcurveto{\pgfqpoint{3.936155in}{1.969719in}}{\pgfqpoint{3.932882in}{1.977619in}}{\pgfqpoint{3.927058in}{1.983443in}}%
\pgfpathcurveto{\pgfqpoint{3.921234in}{1.989267in}}{\pgfqpoint{3.913334in}{1.992539in}}{\pgfqpoint{3.905098in}{1.992539in}}%
\pgfpathcurveto{\pgfqpoint{3.896862in}{1.992539in}}{\pgfqpoint{3.888962in}{1.989267in}}{\pgfqpoint{3.883138in}{1.983443in}}%
\pgfpathcurveto{\pgfqpoint{3.877314in}{1.977619in}}{\pgfqpoint{3.874042in}{1.969719in}}{\pgfqpoint{3.874042in}{1.961482in}}%
\pgfpathcurveto{\pgfqpoint{3.874042in}{1.953246in}}{\pgfqpoint{3.877314in}{1.945346in}}{\pgfqpoint{3.883138in}{1.939522in}}%
\pgfpathcurveto{\pgfqpoint{3.888962in}{1.933698in}}{\pgfqpoint{3.896862in}{1.930426in}}{\pgfqpoint{3.905098in}{1.930426in}}%
\pgfpathclose%
\pgfusepath{stroke,fill}%
\end{pgfscope}%
\begin{pgfscope}%
\pgfpathrectangle{\pgfqpoint{3.793912in}{0.557870in}}{\pgfqpoint{2.446088in}{1.484734in}}%
\pgfusepath{clip}%
\pgfsetbuttcap%
\pgfsetroundjoin%
\definecolor{currentfill}{rgb}{0.298039,0.447059,0.690196}%
\pgfsetfillcolor{currentfill}%
\pgfsetlinewidth{1.003750pt}%
\definecolor{currentstroke}{rgb}{0.298039,0.447059,0.690196}%
\pgfsetstrokecolor{currentstroke}%
\pgfsetdash{}{0pt}%
\pgfpathmoveto{\pgfqpoint{3.905098in}{1.930426in}}%
\pgfpathcurveto{\pgfqpoint{3.913334in}{1.930426in}}{\pgfqpoint{3.921234in}{1.933698in}}{\pgfqpoint{3.927058in}{1.939522in}}%
\pgfpathcurveto{\pgfqpoint{3.932882in}{1.945346in}}{\pgfqpoint{3.936155in}{1.953246in}}{\pgfqpoint{3.936155in}{1.961482in}}%
\pgfpathcurveto{\pgfqpoint{3.936155in}{1.969719in}}{\pgfqpoint{3.932882in}{1.977619in}}{\pgfqpoint{3.927058in}{1.983443in}}%
\pgfpathcurveto{\pgfqpoint{3.921234in}{1.989267in}}{\pgfqpoint{3.913334in}{1.992539in}}{\pgfqpoint{3.905098in}{1.992539in}}%
\pgfpathcurveto{\pgfqpoint{3.896862in}{1.992539in}}{\pgfqpoint{3.888962in}{1.989267in}}{\pgfqpoint{3.883138in}{1.983443in}}%
\pgfpathcurveto{\pgfqpoint{3.877314in}{1.977619in}}{\pgfqpoint{3.874042in}{1.969719in}}{\pgfqpoint{3.874042in}{1.961482in}}%
\pgfpathcurveto{\pgfqpoint{3.874042in}{1.953246in}}{\pgfqpoint{3.877314in}{1.945346in}}{\pgfqpoint{3.883138in}{1.939522in}}%
\pgfpathcurveto{\pgfqpoint{3.888962in}{1.933698in}}{\pgfqpoint{3.896862in}{1.930426in}}{\pgfqpoint{3.905098in}{1.930426in}}%
\pgfpathclose%
\pgfusepath{stroke,fill}%
\end{pgfscope}%
\begin{pgfscope}%
\pgfpathrectangle{\pgfqpoint{3.793912in}{0.557870in}}{\pgfqpoint{2.446088in}{1.484734in}}%
\pgfusepath{clip}%
\pgfsetbuttcap%
\pgfsetroundjoin%
\definecolor{currentfill}{rgb}{0.298039,0.447059,0.690196}%
\pgfsetfillcolor{currentfill}%
\pgfsetlinewidth{1.003750pt}%
\definecolor{currentstroke}{rgb}{0.298039,0.447059,0.690196}%
\pgfsetstrokecolor{currentstroke}%
\pgfsetdash{}{0pt}%
\pgfpathmoveto{\pgfqpoint{3.905098in}{1.385069in}}%
\pgfpathcurveto{\pgfqpoint{3.913334in}{1.385069in}}{\pgfqpoint{3.921234in}{1.388341in}}{\pgfqpoint{3.927058in}{1.394165in}}%
\pgfpathcurveto{\pgfqpoint{3.932882in}{1.399989in}}{\pgfqpoint{3.936155in}{1.407889in}}{\pgfqpoint{3.936155in}{1.416126in}}%
\pgfpathcurveto{\pgfqpoint{3.936155in}{1.424362in}}{\pgfqpoint{3.932882in}{1.432262in}}{\pgfqpoint{3.927058in}{1.438086in}}%
\pgfpathcurveto{\pgfqpoint{3.921234in}{1.443910in}}{\pgfqpoint{3.913334in}{1.447182in}}{\pgfqpoint{3.905098in}{1.447182in}}%
\pgfpathcurveto{\pgfqpoint{3.896862in}{1.447182in}}{\pgfqpoint{3.888962in}{1.443910in}}{\pgfqpoint{3.883138in}{1.438086in}}%
\pgfpathcurveto{\pgfqpoint{3.877314in}{1.432262in}}{\pgfqpoint{3.874042in}{1.424362in}}{\pgfqpoint{3.874042in}{1.416126in}}%
\pgfpathcurveto{\pgfqpoint{3.874042in}{1.407889in}}{\pgfqpoint{3.877314in}{1.399989in}}{\pgfqpoint{3.883138in}{1.394165in}}%
\pgfpathcurveto{\pgfqpoint{3.888962in}{1.388341in}}{\pgfqpoint{3.896862in}{1.385069in}}{\pgfqpoint{3.905098in}{1.385069in}}%
\pgfpathclose%
\pgfusepath{stroke,fill}%
\end{pgfscope}%
\begin{pgfscope}%
\pgfpathrectangle{\pgfqpoint{3.793912in}{0.557870in}}{\pgfqpoint{2.446088in}{1.484734in}}%
\pgfusepath{clip}%
\pgfsetbuttcap%
\pgfsetroundjoin%
\definecolor{currentfill}{rgb}{0.298039,0.447059,0.690196}%
\pgfsetfillcolor{currentfill}%
\pgfsetlinewidth{1.003750pt}%
\definecolor{currentstroke}{rgb}{0.298039,0.447059,0.690196}%
\pgfsetstrokecolor{currentstroke}%
\pgfsetdash{}{0pt}%
\pgfpathmoveto{\pgfqpoint{3.905098in}{1.712283in}}%
\pgfpathcurveto{\pgfqpoint{3.913334in}{1.712283in}}{\pgfqpoint{3.921234in}{1.715555in}}{\pgfqpoint{3.927058in}{1.721379in}}%
\pgfpathcurveto{\pgfqpoint{3.932882in}{1.727203in}}{\pgfqpoint{3.936155in}{1.735103in}}{\pgfqpoint{3.936155in}{1.743340in}}%
\pgfpathcurveto{\pgfqpoint{3.936155in}{1.751576in}}{\pgfqpoint{3.932882in}{1.759476in}}{\pgfqpoint{3.927058in}{1.765300in}}%
\pgfpathcurveto{\pgfqpoint{3.921234in}{1.771124in}}{\pgfqpoint{3.913334in}{1.774396in}}{\pgfqpoint{3.905098in}{1.774396in}}%
\pgfpathcurveto{\pgfqpoint{3.896862in}{1.774396in}}{\pgfqpoint{3.888962in}{1.771124in}}{\pgfqpoint{3.883138in}{1.765300in}}%
\pgfpathcurveto{\pgfqpoint{3.877314in}{1.759476in}}{\pgfqpoint{3.874042in}{1.751576in}}{\pgfqpoint{3.874042in}{1.743340in}}%
\pgfpathcurveto{\pgfqpoint{3.874042in}{1.735103in}}{\pgfqpoint{3.877314in}{1.727203in}}{\pgfqpoint{3.883138in}{1.721379in}}%
\pgfpathcurveto{\pgfqpoint{3.888962in}{1.715555in}}{\pgfqpoint{3.896862in}{1.712283in}}{\pgfqpoint{3.905098in}{1.712283in}}%
\pgfpathclose%
\pgfusepath{stroke,fill}%
\end{pgfscope}%
\begin{pgfscope}%
\pgfpathrectangle{\pgfqpoint{3.793912in}{0.557870in}}{\pgfqpoint{2.446088in}{1.484734in}}%
\pgfusepath{clip}%
\pgfsetbuttcap%
\pgfsetroundjoin%
\definecolor{currentfill}{rgb}{0.298039,0.447059,0.690196}%
\pgfsetfillcolor{currentfill}%
\pgfsetlinewidth{1.003750pt}%
\definecolor{currentstroke}{rgb}{0.298039,0.447059,0.690196}%
\pgfsetstrokecolor{currentstroke}%
\pgfsetdash{}{0pt}%
\pgfpathmoveto{\pgfqpoint{3.905098in}{1.398703in}}%
\pgfpathcurveto{\pgfqpoint{3.913334in}{1.398703in}}{\pgfqpoint{3.921234in}{1.401975in}}{\pgfqpoint{3.927058in}{1.407799in}}%
\pgfpathcurveto{\pgfqpoint{3.932882in}{1.413623in}}{\pgfqpoint{3.936155in}{1.421523in}}{\pgfqpoint{3.936155in}{1.429759in}}%
\pgfpathcurveto{\pgfqpoint{3.936155in}{1.437996in}}{\pgfqpoint{3.932882in}{1.445896in}}{\pgfqpoint{3.927058in}{1.451720in}}%
\pgfpathcurveto{\pgfqpoint{3.921234in}{1.457544in}}{\pgfqpoint{3.913334in}{1.460816in}}{\pgfqpoint{3.905098in}{1.460816in}}%
\pgfpathcurveto{\pgfqpoint{3.896862in}{1.460816in}}{\pgfqpoint{3.888962in}{1.457544in}}{\pgfqpoint{3.883138in}{1.451720in}}%
\pgfpathcurveto{\pgfqpoint{3.877314in}{1.445896in}}{\pgfqpoint{3.874042in}{1.437996in}}{\pgfqpoint{3.874042in}{1.429759in}}%
\pgfpathcurveto{\pgfqpoint{3.874042in}{1.421523in}}{\pgfqpoint{3.877314in}{1.413623in}}{\pgfqpoint{3.883138in}{1.407799in}}%
\pgfpathcurveto{\pgfqpoint{3.888962in}{1.401975in}}{\pgfqpoint{3.896862in}{1.398703in}}{\pgfqpoint{3.905098in}{1.398703in}}%
\pgfpathclose%
\pgfusepath{stroke,fill}%
\end{pgfscope}%
\begin{pgfscope}%
\pgfpathrectangle{\pgfqpoint{3.793912in}{0.557870in}}{\pgfqpoint{2.446088in}{1.484734in}}%
\pgfusepath{clip}%
\pgfsetbuttcap%
\pgfsetroundjoin%
\definecolor{currentfill}{rgb}{0.298039,0.447059,0.690196}%
\pgfsetfillcolor{currentfill}%
\pgfsetlinewidth{1.003750pt}%
\definecolor{currentstroke}{rgb}{0.298039,0.447059,0.690196}%
\pgfsetstrokecolor{currentstroke}%
\pgfsetdash{}{0pt}%
\pgfpathmoveto{\pgfqpoint{4.924301in}{0.594302in}}%
\pgfpathcurveto{\pgfqpoint{4.932538in}{0.594302in}}{\pgfqpoint{4.940438in}{0.597574in}}{\pgfqpoint{4.946262in}{0.603398in}}%
\pgfpathcurveto{\pgfqpoint{4.952085in}{0.609222in}}{\pgfqpoint{4.955358in}{0.617122in}}{\pgfqpoint{4.955358in}{0.625358in}}%
\pgfpathcurveto{\pgfqpoint{4.955358in}{0.633594in}}{\pgfqpoint{4.952085in}{0.641495in}}{\pgfqpoint{4.946262in}{0.647318in}}%
\pgfpathcurveto{\pgfqpoint{4.940438in}{0.653142in}}{\pgfqpoint{4.932538in}{0.656415in}}{\pgfqpoint{4.924301in}{0.656415in}}%
\pgfpathcurveto{\pgfqpoint{4.916065in}{0.656415in}}{\pgfqpoint{4.908165in}{0.653142in}}{\pgfqpoint{4.902341in}{0.647318in}}%
\pgfpathcurveto{\pgfqpoint{4.896517in}{0.641495in}}{\pgfqpoint{4.893245in}{0.633594in}}{\pgfqpoint{4.893245in}{0.625358in}}%
\pgfpathcurveto{\pgfqpoint{4.893245in}{0.617122in}}{\pgfqpoint{4.896517in}{0.609222in}}{\pgfqpoint{4.902341in}{0.603398in}}%
\pgfpathcurveto{\pgfqpoint{4.908165in}{0.597574in}}{\pgfqpoint{4.916065in}{0.594302in}}{\pgfqpoint{4.924301in}{0.594302in}}%
\pgfpathclose%
\pgfusepath{stroke,fill}%
\end{pgfscope}%
\begin{pgfscope}%
\pgfpathrectangle{\pgfqpoint{3.793912in}{0.557870in}}{\pgfqpoint{2.446088in}{1.484734in}}%
\pgfusepath{clip}%
\pgfsetbuttcap%
\pgfsetroundjoin%
\definecolor{currentfill}{rgb}{0.298039,0.447059,0.690196}%
\pgfsetfillcolor{currentfill}%
\pgfsetlinewidth{1.003750pt}%
\definecolor{currentstroke}{rgb}{0.298039,0.447059,0.690196}%
\pgfsetstrokecolor{currentstroke}%
\pgfsetdash{}{0pt}%
\pgfpathmoveto{\pgfqpoint{3.905098in}{1.930426in}}%
\pgfpathcurveto{\pgfqpoint{3.913334in}{1.930426in}}{\pgfqpoint{3.921234in}{1.933698in}}{\pgfqpoint{3.927058in}{1.939522in}}%
\pgfpathcurveto{\pgfqpoint{3.932882in}{1.945346in}}{\pgfqpoint{3.936155in}{1.953246in}}{\pgfqpoint{3.936155in}{1.961482in}}%
\pgfpathcurveto{\pgfqpoint{3.936155in}{1.969719in}}{\pgfqpoint{3.932882in}{1.977619in}}{\pgfqpoint{3.927058in}{1.983443in}}%
\pgfpathcurveto{\pgfqpoint{3.921234in}{1.989267in}}{\pgfqpoint{3.913334in}{1.992539in}}{\pgfqpoint{3.905098in}{1.992539in}}%
\pgfpathcurveto{\pgfqpoint{3.896862in}{1.992539in}}{\pgfqpoint{3.888962in}{1.989267in}}{\pgfqpoint{3.883138in}{1.983443in}}%
\pgfpathcurveto{\pgfqpoint{3.877314in}{1.977619in}}{\pgfqpoint{3.874042in}{1.969719in}}{\pgfqpoint{3.874042in}{1.961482in}}%
\pgfpathcurveto{\pgfqpoint{3.874042in}{1.953246in}}{\pgfqpoint{3.877314in}{1.945346in}}{\pgfqpoint{3.883138in}{1.939522in}}%
\pgfpathcurveto{\pgfqpoint{3.888962in}{1.933698in}}{\pgfqpoint{3.896862in}{1.930426in}}{\pgfqpoint{3.905098in}{1.930426in}}%
\pgfpathclose%
\pgfusepath{stroke,fill}%
\end{pgfscope}%
\begin{pgfscope}%
\pgfpathrectangle{\pgfqpoint{3.793912in}{0.557870in}}{\pgfqpoint{2.446088in}{1.484734in}}%
\pgfusepath{clip}%
\pgfsetbuttcap%
\pgfsetroundjoin%
\definecolor{currentfill}{rgb}{0.298039,0.447059,0.690196}%
\pgfsetfillcolor{currentfill}%
\pgfsetlinewidth{1.003750pt}%
\definecolor{currentstroke}{rgb}{0.298039,0.447059,0.690196}%
\pgfsetstrokecolor{currentstroke}%
\pgfsetdash{}{0pt}%
\pgfpathmoveto{\pgfqpoint{3.905098in}{1.412337in}}%
\pgfpathcurveto{\pgfqpoint{3.913334in}{1.412337in}}{\pgfqpoint{3.921234in}{1.415609in}}{\pgfqpoint{3.927058in}{1.421433in}}%
\pgfpathcurveto{\pgfqpoint{3.932882in}{1.427257in}}{\pgfqpoint{3.936155in}{1.435157in}}{\pgfqpoint{3.936155in}{1.443393in}}%
\pgfpathcurveto{\pgfqpoint{3.936155in}{1.451630in}}{\pgfqpoint{3.932882in}{1.459530in}}{\pgfqpoint{3.927058in}{1.465354in}}%
\pgfpathcurveto{\pgfqpoint{3.921234in}{1.471178in}}{\pgfqpoint{3.913334in}{1.474450in}}{\pgfqpoint{3.905098in}{1.474450in}}%
\pgfpathcurveto{\pgfqpoint{3.896862in}{1.474450in}}{\pgfqpoint{3.888962in}{1.471178in}}{\pgfqpoint{3.883138in}{1.465354in}}%
\pgfpathcurveto{\pgfqpoint{3.877314in}{1.459530in}}{\pgfqpoint{3.874042in}{1.451630in}}{\pgfqpoint{3.874042in}{1.443393in}}%
\pgfpathcurveto{\pgfqpoint{3.874042in}{1.435157in}}{\pgfqpoint{3.877314in}{1.427257in}}{\pgfqpoint{3.883138in}{1.421433in}}%
\pgfpathcurveto{\pgfqpoint{3.888962in}{1.415609in}}{\pgfqpoint{3.896862in}{1.412337in}}{\pgfqpoint{3.905098in}{1.412337in}}%
\pgfpathclose%
\pgfusepath{stroke,fill}%
\end{pgfscope}%
\begin{pgfscope}%
\pgfpathrectangle{\pgfqpoint{3.793912in}{0.557870in}}{\pgfqpoint{2.446088in}{1.484734in}}%
\pgfusepath{clip}%
\pgfsetbuttcap%
\pgfsetroundjoin%
\definecolor{currentfill}{rgb}{0.298039,0.447059,0.690196}%
\pgfsetfillcolor{currentfill}%
\pgfsetlinewidth{1.003750pt}%
\definecolor{currentstroke}{rgb}{0.298039,0.447059,0.690196}%
\pgfsetstrokecolor{currentstroke}%
\pgfsetdash{}{0pt}%
\pgfpathmoveto{\pgfqpoint{3.905098in}{1.385069in}}%
\pgfpathcurveto{\pgfqpoint{3.913334in}{1.385069in}}{\pgfqpoint{3.921234in}{1.388341in}}{\pgfqpoint{3.927058in}{1.394165in}}%
\pgfpathcurveto{\pgfqpoint{3.932882in}{1.399989in}}{\pgfqpoint{3.936155in}{1.407889in}}{\pgfqpoint{3.936155in}{1.416126in}}%
\pgfpathcurveto{\pgfqpoint{3.936155in}{1.424362in}}{\pgfqpoint{3.932882in}{1.432262in}}{\pgfqpoint{3.927058in}{1.438086in}}%
\pgfpathcurveto{\pgfqpoint{3.921234in}{1.443910in}}{\pgfqpoint{3.913334in}{1.447182in}}{\pgfqpoint{3.905098in}{1.447182in}}%
\pgfpathcurveto{\pgfqpoint{3.896862in}{1.447182in}}{\pgfqpoint{3.888962in}{1.443910in}}{\pgfqpoint{3.883138in}{1.438086in}}%
\pgfpathcurveto{\pgfqpoint{3.877314in}{1.432262in}}{\pgfqpoint{3.874042in}{1.424362in}}{\pgfqpoint{3.874042in}{1.416126in}}%
\pgfpathcurveto{\pgfqpoint{3.874042in}{1.407889in}}{\pgfqpoint{3.877314in}{1.399989in}}{\pgfqpoint{3.883138in}{1.394165in}}%
\pgfpathcurveto{\pgfqpoint{3.888962in}{1.388341in}}{\pgfqpoint{3.896862in}{1.385069in}}{\pgfqpoint{3.905098in}{1.385069in}}%
\pgfpathclose%
\pgfusepath{stroke,fill}%
\end{pgfscope}%
\begin{pgfscope}%
\pgfpathrectangle{\pgfqpoint{3.793912in}{0.557870in}}{\pgfqpoint{2.446088in}{1.484734in}}%
\pgfusepath{clip}%
\pgfsetbuttcap%
\pgfsetroundjoin%
\definecolor{currentfill}{rgb}{0.298039,0.447059,0.690196}%
\pgfsetfillcolor{currentfill}%
\pgfsetlinewidth{1.003750pt}%
\definecolor{currentstroke}{rgb}{0.298039,0.447059,0.690196}%
\pgfsetstrokecolor{currentstroke}%
\pgfsetdash{}{0pt}%
\pgfpathmoveto{\pgfqpoint{3.905098in}{1.425971in}}%
\pgfpathcurveto{\pgfqpoint{3.913334in}{1.425971in}}{\pgfqpoint{3.921234in}{1.429243in}}{\pgfqpoint{3.927058in}{1.435067in}}%
\pgfpathcurveto{\pgfqpoint{3.932882in}{1.440891in}}{\pgfqpoint{3.936155in}{1.448791in}}{\pgfqpoint{3.936155in}{1.457027in}}%
\pgfpathcurveto{\pgfqpoint{3.936155in}{1.465264in}}{\pgfqpoint{3.932882in}{1.473164in}}{\pgfqpoint{3.927058in}{1.478988in}}%
\pgfpathcurveto{\pgfqpoint{3.921234in}{1.484811in}}{\pgfqpoint{3.913334in}{1.488084in}}{\pgfqpoint{3.905098in}{1.488084in}}%
\pgfpathcurveto{\pgfqpoint{3.896862in}{1.488084in}}{\pgfqpoint{3.888962in}{1.484811in}}{\pgfqpoint{3.883138in}{1.478988in}}%
\pgfpathcurveto{\pgfqpoint{3.877314in}{1.473164in}}{\pgfqpoint{3.874042in}{1.465264in}}{\pgfqpoint{3.874042in}{1.457027in}}%
\pgfpathcurveto{\pgfqpoint{3.874042in}{1.448791in}}{\pgfqpoint{3.877314in}{1.440891in}}{\pgfqpoint{3.883138in}{1.435067in}}%
\pgfpathcurveto{\pgfqpoint{3.888962in}{1.429243in}}{\pgfqpoint{3.896862in}{1.425971in}}{\pgfqpoint{3.905098in}{1.425971in}}%
\pgfpathclose%
\pgfusepath{stroke,fill}%
\end{pgfscope}%
\begin{pgfscope}%
\pgfpathrectangle{\pgfqpoint{3.793912in}{0.557870in}}{\pgfqpoint{2.446088in}{1.484734in}}%
\pgfusepath{clip}%
\pgfsetbuttcap%
\pgfsetroundjoin%
\definecolor{currentfill}{rgb}{0.298039,0.447059,0.690196}%
\pgfsetfillcolor{currentfill}%
\pgfsetlinewidth{1.003750pt}%
\definecolor{currentstroke}{rgb}{0.298039,0.447059,0.690196}%
\pgfsetstrokecolor{currentstroke}%
\pgfsetdash{}{0pt}%
\pgfpathmoveto{\pgfqpoint{3.905098in}{1.439605in}}%
\pgfpathcurveto{\pgfqpoint{3.913334in}{1.439605in}}{\pgfqpoint{3.921234in}{1.442877in}}{\pgfqpoint{3.927058in}{1.448701in}}%
\pgfpathcurveto{\pgfqpoint{3.932882in}{1.454525in}}{\pgfqpoint{3.936155in}{1.462425in}}{\pgfqpoint{3.936155in}{1.470661in}}%
\pgfpathcurveto{\pgfqpoint{3.936155in}{1.478898in}}{\pgfqpoint{3.932882in}{1.486798in}}{\pgfqpoint{3.927058in}{1.492621in}}%
\pgfpathcurveto{\pgfqpoint{3.921234in}{1.498445in}}{\pgfqpoint{3.913334in}{1.501718in}}{\pgfqpoint{3.905098in}{1.501718in}}%
\pgfpathcurveto{\pgfqpoint{3.896862in}{1.501718in}}{\pgfqpoint{3.888962in}{1.498445in}}{\pgfqpoint{3.883138in}{1.492621in}}%
\pgfpathcurveto{\pgfqpoint{3.877314in}{1.486798in}}{\pgfqpoint{3.874042in}{1.478898in}}{\pgfqpoint{3.874042in}{1.470661in}}%
\pgfpathcurveto{\pgfqpoint{3.874042in}{1.462425in}}{\pgfqpoint{3.877314in}{1.454525in}}{\pgfqpoint{3.883138in}{1.448701in}}%
\pgfpathcurveto{\pgfqpoint{3.888962in}{1.442877in}}{\pgfqpoint{3.896862in}{1.439605in}}{\pgfqpoint{3.905098in}{1.439605in}}%
\pgfpathclose%
\pgfusepath{stroke,fill}%
\end{pgfscope}%
\begin{pgfscope}%
\pgfpathrectangle{\pgfqpoint{3.793912in}{0.557870in}}{\pgfqpoint{2.446088in}{1.484734in}}%
\pgfusepath{clip}%
\pgfsetbuttcap%
\pgfsetroundjoin%
\definecolor{currentfill}{rgb}{0.298039,0.447059,0.690196}%
\pgfsetfillcolor{currentfill}%
\pgfsetlinewidth{1.003750pt}%
\definecolor{currentstroke}{rgb}{0.298039,0.447059,0.690196}%
\pgfsetstrokecolor{currentstroke}%
\pgfsetdash{}{0pt}%
\pgfpathmoveto{\pgfqpoint{3.905098in}{1.930426in}}%
\pgfpathcurveto{\pgfqpoint{3.913334in}{1.930426in}}{\pgfqpoint{3.921234in}{1.933698in}}{\pgfqpoint{3.927058in}{1.939522in}}%
\pgfpathcurveto{\pgfqpoint{3.932882in}{1.945346in}}{\pgfqpoint{3.936155in}{1.953246in}}{\pgfqpoint{3.936155in}{1.961482in}}%
\pgfpathcurveto{\pgfqpoint{3.936155in}{1.969719in}}{\pgfqpoint{3.932882in}{1.977619in}}{\pgfqpoint{3.927058in}{1.983443in}}%
\pgfpathcurveto{\pgfqpoint{3.921234in}{1.989267in}}{\pgfqpoint{3.913334in}{1.992539in}}{\pgfqpoint{3.905098in}{1.992539in}}%
\pgfpathcurveto{\pgfqpoint{3.896862in}{1.992539in}}{\pgfqpoint{3.888962in}{1.989267in}}{\pgfqpoint{3.883138in}{1.983443in}}%
\pgfpathcurveto{\pgfqpoint{3.877314in}{1.977619in}}{\pgfqpoint{3.874042in}{1.969719in}}{\pgfqpoint{3.874042in}{1.961482in}}%
\pgfpathcurveto{\pgfqpoint{3.874042in}{1.953246in}}{\pgfqpoint{3.877314in}{1.945346in}}{\pgfqpoint{3.883138in}{1.939522in}}%
\pgfpathcurveto{\pgfqpoint{3.888962in}{1.933698in}}{\pgfqpoint{3.896862in}{1.930426in}}{\pgfqpoint{3.905098in}{1.930426in}}%
\pgfpathclose%
\pgfusepath{stroke,fill}%
\end{pgfscope}%
\begin{pgfscope}%
\pgfpathrectangle{\pgfqpoint{3.793912in}{0.557870in}}{\pgfqpoint{2.446088in}{1.484734in}}%
\pgfusepath{clip}%
\pgfsetbuttcap%
\pgfsetroundjoin%
\definecolor{currentfill}{rgb}{0.298039,0.447059,0.690196}%
\pgfsetfillcolor{currentfill}%
\pgfsetlinewidth{1.003750pt}%
\definecolor{currentstroke}{rgb}{0.298039,0.447059,0.690196}%
\pgfsetstrokecolor{currentstroke}%
\pgfsetdash{}{0pt}%
\pgfpathmoveto{\pgfqpoint{3.905098in}{1.221462in}}%
\pgfpathcurveto{\pgfqpoint{3.913334in}{1.221462in}}{\pgfqpoint{3.921234in}{1.224734in}}{\pgfqpoint{3.927058in}{1.230558in}}%
\pgfpathcurveto{\pgfqpoint{3.932882in}{1.236382in}}{\pgfqpoint{3.936155in}{1.244282in}}{\pgfqpoint{3.936155in}{1.252519in}}%
\pgfpathcurveto{\pgfqpoint{3.936155in}{1.260755in}}{\pgfqpoint{3.932882in}{1.268655in}}{\pgfqpoint{3.927058in}{1.274479in}}%
\pgfpathcurveto{\pgfqpoint{3.921234in}{1.280303in}}{\pgfqpoint{3.913334in}{1.283575in}}{\pgfqpoint{3.905098in}{1.283575in}}%
\pgfpathcurveto{\pgfqpoint{3.896862in}{1.283575in}}{\pgfqpoint{3.888962in}{1.280303in}}{\pgfqpoint{3.883138in}{1.274479in}}%
\pgfpathcurveto{\pgfqpoint{3.877314in}{1.268655in}}{\pgfqpoint{3.874042in}{1.260755in}}{\pgfqpoint{3.874042in}{1.252519in}}%
\pgfpathcurveto{\pgfqpoint{3.874042in}{1.244282in}}{\pgfqpoint{3.877314in}{1.236382in}}{\pgfqpoint{3.883138in}{1.230558in}}%
\pgfpathcurveto{\pgfqpoint{3.888962in}{1.224734in}}{\pgfqpoint{3.896862in}{1.221462in}}{\pgfqpoint{3.905098in}{1.221462in}}%
\pgfpathclose%
\pgfusepath{stroke,fill}%
\end{pgfscope}%
\begin{pgfscope}%
\pgfpathrectangle{\pgfqpoint{3.793912in}{0.557870in}}{\pgfqpoint{2.446088in}{1.484734in}}%
\pgfusepath{clip}%
\pgfsetbuttcap%
\pgfsetroundjoin%
\definecolor{currentfill}{rgb}{0.298039,0.447059,0.690196}%
\pgfsetfillcolor{currentfill}%
\pgfsetlinewidth{1.003750pt}%
\definecolor{currentstroke}{rgb}{0.298039,0.447059,0.690196}%
\pgfsetstrokecolor{currentstroke}%
\pgfsetdash{}{0pt}%
\pgfpathmoveto{\pgfqpoint{3.905098in}{1.930426in}}%
\pgfpathcurveto{\pgfqpoint{3.913334in}{1.930426in}}{\pgfqpoint{3.921234in}{1.933698in}}{\pgfqpoint{3.927058in}{1.939522in}}%
\pgfpathcurveto{\pgfqpoint{3.932882in}{1.945346in}}{\pgfqpoint{3.936155in}{1.953246in}}{\pgfqpoint{3.936155in}{1.961482in}}%
\pgfpathcurveto{\pgfqpoint{3.936155in}{1.969719in}}{\pgfqpoint{3.932882in}{1.977619in}}{\pgfqpoint{3.927058in}{1.983443in}}%
\pgfpathcurveto{\pgfqpoint{3.921234in}{1.989267in}}{\pgfqpoint{3.913334in}{1.992539in}}{\pgfqpoint{3.905098in}{1.992539in}}%
\pgfpathcurveto{\pgfqpoint{3.896862in}{1.992539in}}{\pgfqpoint{3.888962in}{1.989267in}}{\pgfqpoint{3.883138in}{1.983443in}}%
\pgfpathcurveto{\pgfqpoint{3.877314in}{1.977619in}}{\pgfqpoint{3.874042in}{1.969719in}}{\pgfqpoint{3.874042in}{1.961482in}}%
\pgfpathcurveto{\pgfqpoint{3.874042in}{1.953246in}}{\pgfqpoint{3.877314in}{1.945346in}}{\pgfqpoint{3.883138in}{1.939522in}}%
\pgfpathcurveto{\pgfqpoint{3.888962in}{1.933698in}}{\pgfqpoint{3.896862in}{1.930426in}}{\pgfqpoint{3.905098in}{1.930426in}}%
\pgfpathclose%
\pgfusepath{stroke,fill}%
\end{pgfscope}%
\begin{pgfscope}%
\pgfpathrectangle{\pgfqpoint{3.793912in}{0.557870in}}{\pgfqpoint{2.446088in}{1.484734in}}%
\pgfusepath{clip}%
\pgfsetbuttcap%
\pgfsetroundjoin%
\definecolor{currentfill}{rgb}{0.298039,0.447059,0.690196}%
\pgfsetfillcolor{currentfill}%
\pgfsetlinewidth{1.003750pt}%
\definecolor{currentstroke}{rgb}{0.298039,0.447059,0.690196}%
\pgfsetstrokecolor{currentstroke}%
\pgfsetdash{}{0pt}%
\pgfpathmoveto{\pgfqpoint{3.905098in}{0.798810in}}%
\pgfpathcurveto{\pgfqpoint{3.913334in}{0.798810in}}{\pgfqpoint{3.921234in}{0.802083in}}{\pgfqpoint{3.927058in}{0.807907in}}%
\pgfpathcurveto{\pgfqpoint{3.932882in}{0.813731in}}{\pgfqpoint{3.936155in}{0.821631in}}{\pgfqpoint{3.936155in}{0.829867in}}%
\pgfpathcurveto{\pgfqpoint{3.936155in}{0.838103in}}{\pgfqpoint{3.932882in}{0.846003in}}{\pgfqpoint{3.927058in}{0.851827in}}%
\pgfpathcurveto{\pgfqpoint{3.921234in}{0.857651in}}{\pgfqpoint{3.913334in}{0.860923in}}{\pgfqpoint{3.905098in}{0.860923in}}%
\pgfpathcurveto{\pgfqpoint{3.896862in}{0.860923in}}{\pgfqpoint{3.888962in}{0.857651in}}{\pgfqpoint{3.883138in}{0.851827in}}%
\pgfpathcurveto{\pgfqpoint{3.877314in}{0.846003in}}{\pgfqpoint{3.874042in}{0.838103in}}{\pgfqpoint{3.874042in}{0.829867in}}%
\pgfpathcurveto{\pgfqpoint{3.874042in}{0.821631in}}{\pgfqpoint{3.877314in}{0.813731in}}{\pgfqpoint{3.883138in}{0.807907in}}%
\pgfpathcurveto{\pgfqpoint{3.888962in}{0.802083in}}{\pgfqpoint{3.896862in}{0.798810in}}{\pgfqpoint{3.905098in}{0.798810in}}%
\pgfpathclose%
\pgfusepath{stroke,fill}%
\end{pgfscope}%
\begin{pgfscope}%
\pgfpathrectangle{\pgfqpoint{3.793912in}{0.557870in}}{\pgfqpoint{2.446088in}{1.484734in}}%
\pgfusepath{clip}%
\pgfsetbuttcap%
\pgfsetroundjoin%
\definecolor{currentfill}{rgb}{0.298039,0.447059,0.690196}%
\pgfsetfillcolor{currentfill}%
\pgfsetlinewidth{1.003750pt}%
\definecolor{currentstroke}{rgb}{0.298039,0.447059,0.690196}%
\pgfsetstrokecolor{currentstroke}%
\pgfsetdash{}{0pt}%
\pgfpathmoveto{\pgfqpoint{3.905098in}{0.621570in}}%
\pgfpathcurveto{\pgfqpoint{3.913334in}{0.621570in}}{\pgfqpoint{3.921234in}{0.624842in}}{\pgfqpoint{3.927058in}{0.630666in}}%
\pgfpathcurveto{\pgfqpoint{3.932882in}{0.636490in}}{\pgfqpoint{3.936155in}{0.644390in}}{\pgfqpoint{3.936155in}{0.652626in}}%
\pgfpathcurveto{\pgfqpoint{3.936155in}{0.660862in}}{\pgfqpoint{3.932882in}{0.668762in}}{\pgfqpoint{3.927058in}{0.674586in}}%
\pgfpathcurveto{\pgfqpoint{3.921234in}{0.680410in}}{\pgfqpoint{3.913334in}{0.683683in}}{\pgfqpoint{3.905098in}{0.683683in}}%
\pgfpathcurveto{\pgfqpoint{3.896862in}{0.683683in}}{\pgfqpoint{3.888962in}{0.680410in}}{\pgfqpoint{3.883138in}{0.674586in}}%
\pgfpathcurveto{\pgfqpoint{3.877314in}{0.668762in}}{\pgfqpoint{3.874042in}{0.660862in}}{\pgfqpoint{3.874042in}{0.652626in}}%
\pgfpathcurveto{\pgfqpoint{3.874042in}{0.644390in}}{\pgfqpoint{3.877314in}{0.636490in}}{\pgfqpoint{3.883138in}{0.630666in}}%
\pgfpathcurveto{\pgfqpoint{3.888962in}{0.624842in}}{\pgfqpoint{3.896862in}{0.621570in}}{\pgfqpoint{3.905098in}{0.621570in}}%
\pgfpathclose%
\pgfusepath{stroke,fill}%
\end{pgfscope}%
\begin{pgfscope}%
\pgfpathrectangle{\pgfqpoint{3.793912in}{0.557870in}}{\pgfqpoint{2.446088in}{1.484734in}}%
\pgfusepath{clip}%
\pgfsetbuttcap%
\pgfsetroundjoin%
\definecolor{currentfill}{rgb}{0.298039,0.447059,0.690196}%
\pgfsetfillcolor{currentfill}%
\pgfsetlinewidth{1.003750pt}%
\definecolor{currentstroke}{rgb}{0.298039,0.447059,0.690196}%
\pgfsetstrokecolor{currentstroke}%
\pgfsetdash{}{0pt}%
\pgfpathmoveto{\pgfqpoint{3.905098in}{0.962418in}}%
\pgfpathcurveto{\pgfqpoint{3.913334in}{0.962418in}}{\pgfqpoint{3.921234in}{0.965690in}}{\pgfqpoint{3.927058in}{0.971514in}}%
\pgfpathcurveto{\pgfqpoint{3.932882in}{0.977338in}}{\pgfqpoint{3.936155in}{0.985238in}}{\pgfqpoint{3.936155in}{0.993474in}}%
\pgfpathcurveto{\pgfqpoint{3.936155in}{1.001710in}}{\pgfqpoint{3.932882in}{1.009610in}}{\pgfqpoint{3.927058in}{1.015434in}}%
\pgfpathcurveto{\pgfqpoint{3.921234in}{1.021258in}}{\pgfqpoint{3.913334in}{1.024531in}}{\pgfqpoint{3.905098in}{1.024531in}}%
\pgfpathcurveto{\pgfqpoint{3.896862in}{1.024531in}}{\pgfqpoint{3.888962in}{1.021258in}}{\pgfqpoint{3.883138in}{1.015434in}}%
\pgfpathcurveto{\pgfqpoint{3.877314in}{1.009610in}}{\pgfqpoint{3.874042in}{1.001710in}}{\pgfqpoint{3.874042in}{0.993474in}}%
\pgfpathcurveto{\pgfqpoint{3.874042in}{0.985238in}}{\pgfqpoint{3.877314in}{0.977338in}}{\pgfqpoint{3.883138in}{0.971514in}}%
\pgfpathcurveto{\pgfqpoint{3.888962in}{0.965690in}}{\pgfqpoint{3.896862in}{0.962418in}}{\pgfqpoint{3.905098in}{0.962418in}}%
\pgfpathclose%
\pgfusepath{stroke,fill}%
\end{pgfscope}%
\begin{pgfscope}%
\pgfpathrectangle{\pgfqpoint{3.793912in}{0.557870in}}{\pgfqpoint{2.446088in}{1.484734in}}%
\pgfusepath{clip}%
\pgfsetbuttcap%
\pgfsetroundjoin%
\definecolor{currentfill}{rgb}{0.298039,0.447059,0.690196}%
\pgfsetfillcolor{currentfill}%
\pgfsetlinewidth{1.003750pt}%
\definecolor{currentstroke}{rgb}{0.298039,0.447059,0.690196}%
\pgfsetstrokecolor{currentstroke}%
\pgfsetdash{}{0pt}%
\pgfpathmoveto{\pgfqpoint{3.905098in}{1.930426in}}%
\pgfpathcurveto{\pgfqpoint{3.913334in}{1.930426in}}{\pgfqpoint{3.921234in}{1.933698in}}{\pgfqpoint{3.927058in}{1.939522in}}%
\pgfpathcurveto{\pgfqpoint{3.932882in}{1.945346in}}{\pgfqpoint{3.936155in}{1.953246in}}{\pgfqpoint{3.936155in}{1.961482in}}%
\pgfpathcurveto{\pgfqpoint{3.936155in}{1.969719in}}{\pgfqpoint{3.932882in}{1.977619in}}{\pgfqpoint{3.927058in}{1.983443in}}%
\pgfpathcurveto{\pgfqpoint{3.921234in}{1.989267in}}{\pgfqpoint{3.913334in}{1.992539in}}{\pgfqpoint{3.905098in}{1.992539in}}%
\pgfpathcurveto{\pgfqpoint{3.896862in}{1.992539in}}{\pgfqpoint{3.888962in}{1.989267in}}{\pgfqpoint{3.883138in}{1.983443in}}%
\pgfpathcurveto{\pgfqpoint{3.877314in}{1.977619in}}{\pgfqpoint{3.874042in}{1.969719in}}{\pgfqpoint{3.874042in}{1.961482in}}%
\pgfpathcurveto{\pgfqpoint{3.874042in}{1.953246in}}{\pgfqpoint{3.877314in}{1.945346in}}{\pgfqpoint{3.883138in}{1.939522in}}%
\pgfpathcurveto{\pgfqpoint{3.888962in}{1.933698in}}{\pgfqpoint{3.896862in}{1.930426in}}{\pgfqpoint{3.905098in}{1.930426in}}%
\pgfpathclose%
\pgfusepath{stroke,fill}%
\end{pgfscope}%
\begin{pgfscope}%
\pgfpathrectangle{\pgfqpoint{3.793912in}{0.557870in}}{\pgfqpoint{2.446088in}{1.484734in}}%
\pgfusepath{clip}%
\pgfsetbuttcap%
\pgfsetroundjoin%
\definecolor{currentfill}{rgb}{0.298039,0.447059,0.690196}%
\pgfsetfillcolor{currentfill}%
\pgfsetlinewidth{1.003750pt}%
\definecolor{currentstroke}{rgb}{0.298039,0.447059,0.690196}%
\pgfsetstrokecolor{currentstroke}%
\pgfsetdash{}{0pt}%
\pgfpathmoveto{\pgfqpoint{3.905098in}{1.930426in}}%
\pgfpathcurveto{\pgfqpoint{3.913334in}{1.930426in}}{\pgfqpoint{3.921234in}{1.933698in}}{\pgfqpoint{3.927058in}{1.939522in}}%
\pgfpathcurveto{\pgfqpoint{3.932882in}{1.945346in}}{\pgfqpoint{3.936155in}{1.953246in}}{\pgfqpoint{3.936155in}{1.961482in}}%
\pgfpathcurveto{\pgfqpoint{3.936155in}{1.969719in}}{\pgfqpoint{3.932882in}{1.977619in}}{\pgfqpoint{3.927058in}{1.983443in}}%
\pgfpathcurveto{\pgfqpoint{3.921234in}{1.989267in}}{\pgfqpoint{3.913334in}{1.992539in}}{\pgfqpoint{3.905098in}{1.992539in}}%
\pgfpathcurveto{\pgfqpoint{3.896862in}{1.992539in}}{\pgfqpoint{3.888962in}{1.989267in}}{\pgfqpoint{3.883138in}{1.983443in}}%
\pgfpathcurveto{\pgfqpoint{3.877314in}{1.977619in}}{\pgfqpoint{3.874042in}{1.969719in}}{\pgfqpoint{3.874042in}{1.961482in}}%
\pgfpathcurveto{\pgfqpoint{3.874042in}{1.953246in}}{\pgfqpoint{3.877314in}{1.945346in}}{\pgfqpoint{3.883138in}{1.939522in}}%
\pgfpathcurveto{\pgfqpoint{3.888962in}{1.933698in}}{\pgfqpoint{3.896862in}{1.930426in}}{\pgfqpoint{3.905098in}{1.930426in}}%
\pgfpathclose%
\pgfusepath{stroke,fill}%
\end{pgfscope}%
\begin{pgfscope}%
\pgfpathrectangle{\pgfqpoint{3.793912in}{0.557870in}}{\pgfqpoint{2.446088in}{1.484734in}}%
\pgfusepath{clip}%
\pgfsetbuttcap%
\pgfsetroundjoin%
\definecolor{currentfill}{rgb}{0.298039,0.447059,0.690196}%
\pgfsetfillcolor{currentfill}%
\pgfsetlinewidth{1.003750pt}%
\definecolor{currentstroke}{rgb}{0.298039,0.447059,0.690196}%
\pgfsetstrokecolor{currentstroke}%
\pgfsetdash{}{0pt}%
\pgfpathmoveto{\pgfqpoint{5.248593in}{0.594302in}}%
\pgfpathcurveto{\pgfqpoint{5.256829in}{0.594302in}}{\pgfqpoint{5.264730in}{0.597574in}}{\pgfqpoint{5.270553in}{0.603398in}}%
\pgfpathcurveto{\pgfqpoint{5.276377in}{0.609222in}}{\pgfqpoint{5.279650in}{0.617122in}}{\pgfqpoint{5.279650in}{0.625358in}}%
\pgfpathcurveto{\pgfqpoint{5.279650in}{0.633594in}}{\pgfqpoint{5.276377in}{0.641495in}}{\pgfqpoint{5.270553in}{0.647318in}}%
\pgfpathcurveto{\pgfqpoint{5.264730in}{0.653142in}}{\pgfqpoint{5.256829in}{0.656415in}}{\pgfqpoint{5.248593in}{0.656415in}}%
\pgfpathcurveto{\pgfqpoint{5.240357in}{0.656415in}}{\pgfqpoint{5.232457in}{0.653142in}}{\pgfqpoint{5.226633in}{0.647318in}}%
\pgfpathcurveto{\pgfqpoint{5.220809in}{0.641495in}}{\pgfqpoint{5.217537in}{0.633594in}}{\pgfqpoint{5.217537in}{0.625358in}}%
\pgfpathcurveto{\pgfqpoint{5.217537in}{0.617122in}}{\pgfqpoint{5.220809in}{0.609222in}}{\pgfqpoint{5.226633in}{0.603398in}}%
\pgfpathcurveto{\pgfqpoint{5.232457in}{0.597574in}}{\pgfqpoint{5.240357in}{0.594302in}}{\pgfqpoint{5.248593in}{0.594302in}}%
\pgfpathclose%
\pgfusepath{stroke,fill}%
\end{pgfscope}%
\begin{pgfscope}%
\pgfpathrectangle{\pgfqpoint{3.793912in}{0.557870in}}{\pgfqpoint{2.446088in}{1.484734in}}%
\pgfusepath{clip}%
\pgfsetbuttcap%
\pgfsetroundjoin%
\definecolor{currentfill}{rgb}{0.298039,0.447059,0.690196}%
\pgfsetfillcolor{currentfill}%
\pgfsetlinewidth{1.003750pt}%
\definecolor{currentstroke}{rgb}{0.298039,0.447059,0.690196}%
\pgfsetstrokecolor{currentstroke}%
\pgfsetdash{}{0pt}%
\pgfpathmoveto{\pgfqpoint{3.905098in}{0.621570in}}%
\pgfpathcurveto{\pgfqpoint{3.913334in}{0.621570in}}{\pgfqpoint{3.921234in}{0.624842in}}{\pgfqpoint{3.927058in}{0.630666in}}%
\pgfpathcurveto{\pgfqpoint{3.932882in}{0.636490in}}{\pgfqpoint{3.936155in}{0.644390in}}{\pgfqpoint{3.936155in}{0.652626in}}%
\pgfpathcurveto{\pgfqpoint{3.936155in}{0.660862in}}{\pgfqpoint{3.932882in}{0.668762in}}{\pgfqpoint{3.927058in}{0.674586in}}%
\pgfpathcurveto{\pgfqpoint{3.921234in}{0.680410in}}{\pgfqpoint{3.913334in}{0.683683in}}{\pgfqpoint{3.905098in}{0.683683in}}%
\pgfpathcurveto{\pgfqpoint{3.896862in}{0.683683in}}{\pgfqpoint{3.888962in}{0.680410in}}{\pgfqpoint{3.883138in}{0.674586in}}%
\pgfpathcurveto{\pgfqpoint{3.877314in}{0.668762in}}{\pgfqpoint{3.874042in}{0.660862in}}{\pgfqpoint{3.874042in}{0.652626in}}%
\pgfpathcurveto{\pgfqpoint{3.874042in}{0.644390in}}{\pgfqpoint{3.877314in}{0.636490in}}{\pgfqpoint{3.883138in}{0.630666in}}%
\pgfpathcurveto{\pgfqpoint{3.888962in}{0.624842in}}{\pgfqpoint{3.896862in}{0.621570in}}{\pgfqpoint{3.905098in}{0.621570in}}%
\pgfpathclose%
\pgfusepath{stroke,fill}%
\end{pgfscope}%
\begin{pgfscope}%
\pgfpathrectangle{\pgfqpoint{3.793912in}{0.557870in}}{\pgfqpoint{2.446088in}{1.484734in}}%
\pgfusepath{clip}%
\pgfsetbuttcap%
\pgfsetroundjoin%
\definecolor{currentfill}{rgb}{0.298039,0.447059,0.690196}%
\pgfsetfillcolor{currentfill}%
\pgfsetlinewidth{1.003750pt}%
\definecolor{currentstroke}{rgb}{0.298039,0.447059,0.690196}%
\pgfsetstrokecolor{currentstroke}%
\pgfsetdash{}{0pt}%
\pgfpathmoveto{\pgfqpoint{5.271757in}{0.594302in}}%
\pgfpathcurveto{\pgfqpoint{5.279993in}{0.594302in}}{\pgfqpoint{5.287893in}{0.597574in}}{\pgfqpoint{5.293717in}{0.603398in}}%
\pgfpathcurveto{\pgfqpoint{5.299541in}{0.609222in}}{\pgfqpoint{5.302813in}{0.617122in}}{\pgfqpoint{5.302813in}{0.625358in}}%
\pgfpathcurveto{\pgfqpoint{5.302813in}{0.633594in}}{\pgfqpoint{5.299541in}{0.641495in}}{\pgfqpoint{5.293717in}{0.647318in}}%
\pgfpathcurveto{\pgfqpoint{5.287893in}{0.653142in}}{\pgfqpoint{5.279993in}{0.656415in}}{\pgfqpoint{5.271757in}{0.656415in}}%
\pgfpathcurveto{\pgfqpoint{5.263521in}{0.656415in}}{\pgfqpoint{5.255621in}{0.653142in}}{\pgfqpoint{5.249797in}{0.647318in}}%
\pgfpathcurveto{\pgfqpoint{5.243973in}{0.641495in}}{\pgfqpoint{5.240700in}{0.633594in}}{\pgfqpoint{5.240700in}{0.625358in}}%
\pgfpathcurveto{\pgfqpoint{5.240700in}{0.617122in}}{\pgfqpoint{5.243973in}{0.609222in}}{\pgfqpoint{5.249797in}{0.603398in}}%
\pgfpathcurveto{\pgfqpoint{5.255621in}{0.597574in}}{\pgfqpoint{5.263521in}{0.594302in}}{\pgfqpoint{5.271757in}{0.594302in}}%
\pgfpathclose%
\pgfusepath{stroke,fill}%
\end{pgfscope}%
\begin{pgfscope}%
\pgfpathrectangle{\pgfqpoint{3.793912in}{0.557870in}}{\pgfqpoint{2.446088in}{1.484734in}}%
\pgfusepath{clip}%
\pgfsetbuttcap%
\pgfsetroundjoin%
\definecolor{currentfill}{rgb}{0.298039,0.447059,0.690196}%
\pgfsetfillcolor{currentfill}%
\pgfsetlinewidth{1.003750pt}%
\definecolor{currentstroke}{rgb}{0.298039,0.447059,0.690196}%
\pgfsetstrokecolor{currentstroke}%
\pgfsetdash{}{0pt}%
\pgfpathmoveto{\pgfqpoint{3.905098in}{1.930426in}}%
\pgfpathcurveto{\pgfqpoint{3.913334in}{1.930426in}}{\pgfqpoint{3.921234in}{1.933698in}}{\pgfqpoint{3.927058in}{1.939522in}}%
\pgfpathcurveto{\pgfqpoint{3.932882in}{1.945346in}}{\pgfqpoint{3.936155in}{1.953246in}}{\pgfqpoint{3.936155in}{1.961482in}}%
\pgfpathcurveto{\pgfqpoint{3.936155in}{1.969719in}}{\pgfqpoint{3.932882in}{1.977619in}}{\pgfqpoint{3.927058in}{1.983443in}}%
\pgfpathcurveto{\pgfqpoint{3.921234in}{1.989267in}}{\pgfqpoint{3.913334in}{1.992539in}}{\pgfqpoint{3.905098in}{1.992539in}}%
\pgfpathcurveto{\pgfqpoint{3.896862in}{1.992539in}}{\pgfqpoint{3.888962in}{1.989267in}}{\pgfqpoint{3.883138in}{1.983443in}}%
\pgfpathcurveto{\pgfqpoint{3.877314in}{1.977619in}}{\pgfqpoint{3.874042in}{1.969719in}}{\pgfqpoint{3.874042in}{1.961482in}}%
\pgfpathcurveto{\pgfqpoint{3.874042in}{1.953246in}}{\pgfqpoint{3.877314in}{1.945346in}}{\pgfqpoint{3.883138in}{1.939522in}}%
\pgfpathcurveto{\pgfqpoint{3.888962in}{1.933698in}}{\pgfqpoint{3.896862in}{1.930426in}}{\pgfqpoint{3.905098in}{1.930426in}}%
\pgfpathclose%
\pgfusepath{stroke,fill}%
\end{pgfscope}%
\begin{pgfscope}%
\pgfpathrectangle{\pgfqpoint{3.793912in}{0.557870in}}{\pgfqpoint{2.446088in}{1.484734in}}%
\pgfusepath{clip}%
\pgfsetbuttcap%
\pgfsetroundjoin%
\definecolor{currentfill}{rgb}{0.298039,0.447059,0.690196}%
\pgfsetfillcolor{currentfill}%
\pgfsetlinewidth{1.003750pt}%
\definecolor{currentstroke}{rgb}{0.298039,0.447059,0.690196}%
\pgfsetstrokecolor{currentstroke}%
\pgfsetdash{}{0pt}%
\pgfpathmoveto{\pgfqpoint{3.905098in}{0.621570in}}%
\pgfpathcurveto{\pgfqpoint{3.913334in}{0.621570in}}{\pgfqpoint{3.921234in}{0.624842in}}{\pgfqpoint{3.927058in}{0.630666in}}%
\pgfpathcurveto{\pgfqpoint{3.932882in}{0.636490in}}{\pgfqpoint{3.936155in}{0.644390in}}{\pgfqpoint{3.936155in}{0.652626in}}%
\pgfpathcurveto{\pgfqpoint{3.936155in}{0.660862in}}{\pgfqpoint{3.932882in}{0.668762in}}{\pgfqpoint{3.927058in}{0.674586in}}%
\pgfpathcurveto{\pgfqpoint{3.921234in}{0.680410in}}{\pgfqpoint{3.913334in}{0.683683in}}{\pgfqpoint{3.905098in}{0.683683in}}%
\pgfpathcurveto{\pgfqpoint{3.896862in}{0.683683in}}{\pgfqpoint{3.888962in}{0.680410in}}{\pgfqpoint{3.883138in}{0.674586in}}%
\pgfpathcurveto{\pgfqpoint{3.877314in}{0.668762in}}{\pgfqpoint{3.874042in}{0.660862in}}{\pgfqpoint{3.874042in}{0.652626in}}%
\pgfpathcurveto{\pgfqpoint{3.874042in}{0.644390in}}{\pgfqpoint{3.877314in}{0.636490in}}{\pgfqpoint{3.883138in}{0.630666in}}%
\pgfpathcurveto{\pgfqpoint{3.888962in}{0.624842in}}{\pgfqpoint{3.896862in}{0.621570in}}{\pgfqpoint{3.905098in}{0.621570in}}%
\pgfpathclose%
\pgfusepath{stroke,fill}%
\end{pgfscope}%
\begin{pgfscope}%
\pgfpathrectangle{\pgfqpoint{3.793912in}{0.557870in}}{\pgfqpoint{2.446088in}{1.484734in}}%
\pgfusepath{clip}%
\pgfsetbuttcap%
\pgfsetroundjoin%
\definecolor{currentfill}{rgb}{0.298039,0.447059,0.690196}%
\pgfsetfillcolor{currentfill}%
\pgfsetlinewidth{1.003750pt}%
\definecolor{currentstroke}{rgb}{0.298039,0.447059,0.690196}%
\pgfsetstrokecolor{currentstroke}%
\pgfsetdash{}{0pt}%
\pgfpathmoveto{\pgfqpoint{3.905098in}{1.930426in}}%
\pgfpathcurveto{\pgfqpoint{3.913334in}{1.930426in}}{\pgfqpoint{3.921234in}{1.933698in}}{\pgfqpoint{3.927058in}{1.939522in}}%
\pgfpathcurveto{\pgfqpoint{3.932882in}{1.945346in}}{\pgfqpoint{3.936155in}{1.953246in}}{\pgfqpoint{3.936155in}{1.961482in}}%
\pgfpathcurveto{\pgfqpoint{3.936155in}{1.969719in}}{\pgfqpoint{3.932882in}{1.977619in}}{\pgfqpoint{3.927058in}{1.983443in}}%
\pgfpathcurveto{\pgfqpoint{3.921234in}{1.989267in}}{\pgfqpoint{3.913334in}{1.992539in}}{\pgfqpoint{3.905098in}{1.992539in}}%
\pgfpathcurveto{\pgfqpoint{3.896862in}{1.992539in}}{\pgfqpoint{3.888962in}{1.989267in}}{\pgfqpoint{3.883138in}{1.983443in}}%
\pgfpathcurveto{\pgfqpoint{3.877314in}{1.977619in}}{\pgfqpoint{3.874042in}{1.969719in}}{\pgfqpoint{3.874042in}{1.961482in}}%
\pgfpathcurveto{\pgfqpoint{3.874042in}{1.953246in}}{\pgfqpoint{3.877314in}{1.945346in}}{\pgfqpoint{3.883138in}{1.939522in}}%
\pgfpathcurveto{\pgfqpoint{3.888962in}{1.933698in}}{\pgfqpoint{3.896862in}{1.930426in}}{\pgfqpoint{3.905098in}{1.930426in}}%
\pgfpathclose%
\pgfusepath{stroke,fill}%
\end{pgfscope}%
\begin{pgfscope}%
\pgfpathrectangle{\pgfqpoint{3.793912in}{0.557870in}}{\pgfqpoint{2.446088in}{1.484734in}}%
\pgfusepath{clip}%
\pgfsetbuttcap%
\pgfsetroundjoin%
\definecolor{currentfill}{rgb}{0.298039,0.447059,0.690196}%
\pgfsetfillcolor{currentfill}%
\pgfsetlinewidth{1.003750pt}%
\definecolor{currentstroke}{rgb}{0.298039,0.447059,0.690196}%
\pgfsetstrokecolor{currentstroke}%
\pgfsetdash{}{0pt}%
\pgfpathmoveto{\pgfqpoint{3.905098in}{1.330533in}}%
\pgfpathcurveto{\pgfqpoint{3.913334in}{1.330533in}}{\pgfqpoint{3.921234in}{1.333806in}}{\pgfqpoint{3.927058in}{1.339630in}}%
\pgfpathcurveto{\pgfqpoint{3.932882in}{1.345454in}}{\pgfqpoint{3.936155in}{1.353354in}}{\pgfqpoint{3.936155in}{1.361590in}}%
\pgfpathcurveto{\pgfqpoint{3.936155in}{1.369826in}}{\pgfqpoint{3.932882in}{1.377726in}}{\pgfqpoint{3.927058in}{1.383550in}}%
\pgfpathcurveto{\pgfqpoint{3.921234in}{1.389374in}}{\pgfqpoint{3.913334in}{1.392646in}}{\pgfqpoint{3.905098in}{1.392646in}}%
\pgfpathcurveto{\pgfqpoint{3.896862in}{1.392646in}}{\pgfqpoint{3.888962in}{1.389374in}}{\pgfqpoint{3.883138in}{1.383550in}}%
\pgfpathcurveto{\pgfqpoint{3.877314in}{1.377726in}}{\pgfqpoint{3.874042in}{1.369826in}}{\pgfqpoint{3.874042in}{1.361590in}}%
\pgfpathcurveto{\pgfqpoint{3.874042in}{1.353354in}}{\pgfqpoint{3.877314in}{1.345454in}}{\pgfqpoint{3.883138in}{1.339630in}}%
\pgfpathcurveto{\pgfqpoint{3.888962in}{1.333806in}}{\pgfqpoint{3.896862in}{1.330533in}}{\pgfqpoint{3.905098in}{1.330533in}}%
\pgfpathclose%
\pgfusepath{stroke,fill}%
\end{pgfscope}%
\begin{pgfscope}%
\pgfpathrectangle{\pgfqpoint{3.793912in}{0.557870in}}{\pgfqpoint{2.446088in}{1.484734in}}%
\pgfusepath{clip}%
\pgfsetbuttcap%
\pgfsetroundjoin%
\definecolor{currentfill}{rgb}{0.298039,0.447059,0.690196}%
\pgfsetfillcolor{currentfill}%
\pgfsetlinewidth{1.003750pt}%
\definecolor{currentstroke}{rgb}{0.298039,0.447059,0.690196}%
\pgfsetstrokecolor{currentstroke}%
\pgfsetdash{}{0pt}%
\pgfpathmoveto{\pgfqpoint{3.905098in}{1.412337in}}%
\pgfpathcurveto{\pgfqpoint{3.913334in}{1.412337in}}{\pgfqpoint{3.921234in}{1.415609in}}{\pgfqpoint{3.927058in}{1.421433in}}%
\pgfpathcurveto{\pgfqpoint{3.932882in}{1.427257in}}{\pgfqpoint{3.936155in}{1.435157in}}{\pgfqpoint{3.936155in}{1.443393in}}%
\pgfpathcurveto{\pgfqpoint{3.936155in}{1.451630in}}{\pgfqpoint{3.932882in}{1.459530in}}{\pgfqpoint{3.927058in}{1.465354in}}%
\pgfpathcurveto{\pgfqpoint{3.921234in}{1.471178in}}{\pgfqpoint{3.913334in}{1.474450in}}{\pgfqpoint{3.905098in}{1.474450in}}%
\pgfpathcurveto{\pgfqpoint{3.896862in}{1.474450in}}{\pgfqpoint{3.888962in}{1.471178in}}{\pgfqpoint{3.883138in}{1.465354in}}%
\pgfpathcurveto{\pgfqpoint{3.877314in}{1.459530in}}{\pgfqpoint{3.874042in}{1.451630in}}{\pgfqpoint{3.874042in}{1.443393in}}%
\pgfpathcurveto{\pgfqpoint{3.874042in}{1.435157in}}{\pgfqpoint{3.877314in}{1.427257in}}{\pgfqpoint{3.883138in}{1.421433in}}%
\pgfpathcurveto{\pgfqpoint{3.888962in}{1.415609in}}{\pgfqpoint{3.896862in}{1.412337in}}{\pgfqpoint{3.905098in}{1.412337in}}%
\pgfpathclose%
\pgfusepath{stroke,fill}%
\end{pgfscope}%
\begin{pgfscope}%
\pgfpathrectangle{\pgfqpoint{3.793912in}{0.557870in}}{\pgfqpoint{2.446088in}{1.484734in}}%
\pgfusepath{clip}%
\pgfsetbuttcap%
\pgfsetroundjoin%
\definecolor{currentfill}{rgb}{0.298039,0.447059,0.690196}%
\pgfsetfillcolor{currentfill}%
\pgfsetlinewidth{1.003750pt}%
\definecolor{currentstroke}{rgb}{0.298039,0.447059,0.690196}%
\pgfsetstrokecolor{currentstroke}%
\pgfsetdash{}{0pt}%
\pgfpathmoveto{\pgfqpoint{3.905098in}{1.344167in}}%
\pgfpathcurveto{\pgfqpoint{3.913334in}{1.344167in}}{\pgfqpoint{3.921234in}{1.347440in}}{\pgfqpoint{3.927058in}{1.353264in}}%
\pgfpathcurveto{\pgfqpoint{3.932882in}{1.359087in}}{\pgfqpoint{3.936155in}{1.366988in}}{\pgfqpoint{3.936155in}{1.375224in}}%
\pgfpathcurveto{\pgfqpoint{3.936155in}{1.383460in}}{\pgfqpoint{3.932882in}{1.391360in}}{\pgfqpoint{3.927058in}{1.397184in}}%
\pgfpathcurveto{\pgfqpoint{3.921234in}{1.403008in}}{\pgfqpoint{3.913334in}{1.406280in}}{\pgfqpoint{3.905098in}{1.406280in}}%
\pgfpathcurveto{\pgfqpoint{3.896862in}{1.406280in}}{\pgfqpoint{3.888962in}{1.403008in}}{\pgfqpoint{3.883138in}{1.397184in}}%
\pgfpathcurveto{\pgfqpoint{3.877314in}{1.391360in}}{\pgfqpoint{3.874042in}{1.383460in}}{\pgfqpoint{3.874042in}{1.375224in}}%
\pgfpathcurveto{\pgfqpoint{3.874042in}{1.366988in}}{\pgfqpoint{3.877314in}{1.359087in}}{\pgfqpoint{3.883138in}{1.353264in}}%
\pgfpathcurveto{\pgfqpoint{3.888962in}{1.347440in}}{\pgfqpoint{3.896862in}{1.344167in}}{\pgfqpoint{3.905098in}{1.344167in}}%
\pgfpathclose%
\pgfusepath{stroke,fill}%
\end{pgfscope}%
\begin{pgfscope}%
\pgfpathrectangle{\pgfqpoint{3.793912in}{0.557870in}}{\pgfqpoint{2.446088in}{1.484734in}}%
\pgfusepath{clip}%
\pgfsetbuttcap%
\pgfsetroundjoin%
\definecolor{currentfill}{rgb}{0.298039,0.447059,0.690196}%
\pgfsetfillcolor{currentfill}%
\pgfsetlinewidth{1.003750pt}%
\definecolor{currentstroke}{rgb}{0.298039,0.447059,0.690196}%
\pgfsetstrokecolor{currentstroke}%
\pgfsetdash{}{0pt}%
\pgfpathmoveto{\pgfqpoint{3.905098in}{1.425971in}}%
\pgfpathcurveto{\pgfqpoint{3.913334in}{1.425971in}}{\pgfqpoint{3.921234in}{1.429243in}}{\pgfqpoint{3.927058in}{1.435067in}}%
\pgfpathcurveto{\pgfqpoint{3.932882in}{1.440891in}}{\pgfqpoint{3.936155in}{1.448791in}}{\pgfqpoint{3.936155in}{1.457027in}}%
\pgfpathcurveto{\pgfqpoint{3.936155in}{1.465264in}}{\pgfqpoint{3.932882in}{1.473164in}}{\pgfqpoint{3.927058in}{1.478988in}}%
\pgfpathcurveto{\pgfqpoint{3.921234in}{1.484811in}}{\pgfqpoint{3.913334in}{1.488084in}}{\pgfqpoint{3.905098in}{1.488084in}}%
\pgfpathcurveto{\pgfqpoint{3.896862in}{1.488084in}}{\pgfqpoint{3.888962in}{1.484811in}}{\pgfqpoint{3.883138in}{1.478988in}}%
\pgfpathcurveto{\pgfqpoint{3.877314in}{1.473164in}}{\pgfqpoint{3.874042in}{1.465264in}}{\pgfqpoint{3.874042in}{1.457027in}}%
\pgfpathcurveto{\pgfqpoint{3.874042in}{1.448791in}}{\pgfqpoint{3.877314in}{1.440891in}}{\pgfqpoint{3.883138in}{1.435067in}}%
\pgfpathcurveto{\pgfqpoint{3.888962in}{1.429243in}}{\pgfqpoint{3.896862in}{1.425971in}}{\pgfqpoint{3.905098in}{1.425971in}}%
\pgfpathclose%
\pgfusepath{stroke,fill}%
\end{pgfscope}%
\begin{pgfscope}%
\pgfpathrectangle{\pgfqpoint{3.793912in}{0.557870in}}{\pgfqpoint{2.446088in}{1.484734in}}%
\pgfusepath{clip}%
\pgfsetbuttcap%
\pgfsetroundjoin%
\definecolor{currentfill}{rgb}{0.298039,0.447059,0.690196}%
\pgfsetfillcolor{currentfill}%
\pgfsetlinewidth{1.003750pt}%
\definecolor{currentstroke}{rgb}{0.298039,0.447059,0.690196}%
\pgfsetstrokecolor{currentstroke}%
\pgfsetdash{}{0pt}%
\pgfpathmoveto{\pgfqpoint{3.905098in}{1.930426in}}%
\pgfpathcurveto{\pgfqpoint{3.913334in}{1.930426in}}{\pgfqpoint{3.921234in}{1.933698in}}{\pgfqpoint{3.927058in}{1.939522in}}%
\pgfpathcurveto{\pgfqpoint{3.932882in}{1.945346in}}{\pgfqpoint{3.936155in}{1.953246in}}{\pgfqpoint{3.936155in}{1.961482in}}%
\pgfpathcurveto{\pgfqpoint{3.936155in}{1.969719in}}{\pgfqpoint{3.932882in}{1.977619in}}{\pgfqpoint{3.927058in}{1.983443in}}%
\pgfpathcurveto{\pgfqpoint{3.921234in}{1.989267in}}{\pgfqpoint{3.913334in}{1.992539in}}{\pgfqpoint{3.905098in}{1.992539in}}%
\pgfpathcurveto{\pgfqpoint{3.896862in}{1.992539in}}{\pgfqpoint{3.888962in}{1.989267in}}{\pgfqpoint{3.883138in}{1.983443in}}%
\pgfpathcurveto{\pgfqpoint{3.877314in}{1.977619in}}{\pgfqpoint{3.874042in}{1.969719in}}{\pgfqpoint{3.874042in}{1.961482in}}%
\pgfpathcurveto{\pgfqpoint{3.874042in}{1.953246in}}{\pgfqpoint{3.877314in}{1.945346in}}{\pgfqpoint{3.883138in}{1.939522in}}%
\pgfpathcurveto{\pgfqpoint{3.888962in}{1.933698in}}{\pgfqpoint{3.896862in}{1.930426in}}{\pgfqpoint{3.905098in}{1.930426in}}%
\pgfpathclose%
\pgfusepath{stroke,fill}%
\end{pgfscope}%
\begin{pgfscope}%
\pgfpathrectangle{\pgfqpoint{3.793912in}{0.557870in}}{\pgfqpoint{2.446088in}{1.484734in}}%
\pgfusepath{clip}%
\pgfsetbuttcap%
\pgfsetroundjoin%
\definecolor{currentfill}{rgb}{0.298039,0.447059,0.690196}%
\pgfsetfillcolor{currentfill}%
\pgfsetlinewidth{1.003750pt}%
\definecolor{currentstroke}{rgb}{0.298039,0.447059,0.690196}%
\pgfsetstrokecolor{currentstroke}%
\pgfsetdash{}{0pt}%
\pgfpathmoveto{\pgfqpoint{3.905098in}{1.412337in}}%
\pgfpathcurveto{\pgfqpoint{3.913334in}{1.412337in}}{\pgfqpoint{3.921234in}{1.415609in}}{\pgfqpoint{3.927058in}{1.421433in}}%
\pgfpathcurveto{\pgfqpoint{3.932882in}{1.427257in}}{\pgfqpoint{3.936155in}{1.435157in}}{\pgfqpoint{3.936155in}{1.443393in}}%
\pgfpathcurveto{\pgfqpoint{3.936155in}{1.451630in}}{\pgfqpoint{3.932882in}{1.459530in}}{\pgfqpoint{3.927058in}{1.465354in}}%
\pgfpathcurveto{\pgfqpoint{3.921234in}{1.471178in}}{\pgfqpoint{3.913334in}{1.474450in}}{\pgfqpoint{3.905098in}{1.474450in}}%
\pgfpathcurveto{\pgfqpoint{3.896862in}{1.474450in}}{\pgfqpoint{3.888962in}{1.471178in}}{\pgfqpoint{3.883138in}{1.465354in}}%
\pgfpathcurveto{\pgfqpoint{3.877314in}{1.459530in}}{\pgfqpoint{3.874042in}{1.451630in}}{\pgfqpoint{3.874042in}{1.443393in}}%
\pgfpathcurveto{\pgfqpoint{3.874042in}{1.435157in}}{\pgfqpoint{3.877314in}{1.427257in}}{\pgfqpoint{3.883138in}{1.421433in}}%
\pgfpathcurveto{\pgfqpoint{3.888962in}{1.415609in}}{\pgfqpoint{3.896862in}{1.412337in}}{\pgfqpoint{3.905098in}{1.412337in}}%
\pgfpathclose%
\pgfusepath{stroke,fill}%
\end{pgfscope}%
\begin{pgfscope}%
\pgfpathrectangle{\pgfqpoint{3.793912in}{0.557870in}}{\pgfqpoint{2.446088in}{1.484734in}}%
\pgfusepath{clip}%
\pgfsetbuttcap%
\pgfsetroundjoin%
\definecolor{currentfill}{rgb}{0.298039,0.447059,0.690196}%
\pgfsetfillcolor{currentfill}%
\pgfsetlinewidth{1.003750pt}%
\definecolor{currentstroke}{rgb}{0.298039,0.447059,0.690196}%
\pgfsetstrokecolor{currentstroke}%
\pgfsetdash{}{0pt}%
\pgfpathmoveto{\pgfqpoint{3.905098in}{1.930426in}}%
\pgfpathcurveto{\pgfqpoint{3.913334in}{1.930426in}}{\pgfqpoint{3.921234in}{1.933698in}}{\pgfqpoint{3.927058in}{1.939522in}}%
\pgfpathcurveto{\pgfqpoint{3.932882in}{1.945346in}}{\pgfqpoint{3.936155in}{1.953246in}}{\pgfqpoint{3.936155in}{1.961482in}}%
\pgfpathcurveto{\pgfqpoint{3.936155in}{1.969719in}}{\pgfqpoint{3.932882in}{1.977619in}}{\pgfqpoint{3.927058in}{1.983443in}}%
\pgfpathcurveto{\pgfqpoint{3.921234in}{1.989267in}}{\pgfqpoint{3.913334in}{1.992539in}}{\pgfqpoint{3.905098in}{1.992539in}}%
\pgfpathcurveto{\pgfqpoint{3.896862in}{1.992539in}}{\pgfqpoint{3.888962in}{1.989267in}}{\pgfqpoint{3.883138in}{1.983443in}}%
\pgfpathcurveto{\pgfqpoint{3.877314in}{1.977619in}}{\pgfqpoint{3.874042in}{1.969719in}}{\pgfqpoint{3.874042in}{1.961482in}}%
\pgfpathcurveto{\pgfqpoint{3.874042in}{1.953246in}}{\pgfqpoint{3.877314in}{1.945346in}}{\pgfqpoint{3.883138in}{1.939522in}}%
\pgfpathcurveto{\pgfqpoint{3.888962in}{1.933698in}}{\pgfqpoint{3.896862in}{1.930426in}}{\pgfqpoint{3.905098in}{1.930426in}}%
\pgfpathclose%
\pgfusepath{stroke,fill}%
\end{pgfscope}%
\begin{pgfscope}%
\pgfpathrectangle{\pgfqpoint{3.793912in}{0.557870in}}{\pgfqpoint{2.446088in}{1.484734in}}%
\pgfusepath{clip}%
\pgfsetbuttcap%
\pgfsetroundjoin%
\definecolor{currentfill}{rgb}{0.298039,0.447059,0.690196}%
\pgfsetfillcolor{currentfill}%
\pgfsetlinewidth{1.003750pt}%
\definecolor{currentstroke}{rgb}{0.298039,0.447059,0.690196}%
\pgfsetstrokecolor{currentstroke}%
\pgfsetdash{}{0pt}%
\pgfpathmoveto{\pgfqpoint{4.669500in}{0.594302in}}%
\pgfpathcurveto{\pgfqpoint{4.677737in}{0.594302in}}{\pgfqpoint{4.685637in}{0.597574in}}{\pgfqpoint{4.691461in}{0.603398in}}%
\pgfpathcurveto{\pgfqpoint{4.697285in}{0.609222in}}{\pgfqpoint{4.700557in}{0.617122in}}{\pgfqpoint{4.700557in}{0.625358in}}%
\pgfpathcurveto{\pgfqpoint{4.700557in}{0.633594in}}{\pgfqpoint{4.697285in}{0.641495in}}{\pgfqpoint{4.691461in}{0.647318in}}%
\pgfpathcurveto{\pgfqpoint{4.685637in}{0.653142in}}{\pgfqpoint{4.677737in}{0.656415in}}{\pgfqpoint{4.669500in}{0.656415in}}%
\pgfpathcurveto{\pgfqpoint{4.661264in}{0.656415in}}{\pgfqpoint{4.653364in}{0.653142in}}{\pgfqpoint{4.647540in}{0.647318in}}%
\pgfpathcurveto{\pgfqpoint{4.641716in}{0.641495in}}{\pgfqpoint{4.638444in}{0.633594in}}{\pgfqpoint{4.638444in}{0.625358in}}%
\pgfpathcurveto{\pgfqpoint{4.638444in}{0.617122in}}{\pgfqpoint{4.641716in}{0.609222in}}{\pgfqpoint{4.647540in}{0.603398in}}%
\pgfpathcurveto{\pgfqpoint{4.653364in}{0.597574in}}{\pgfqpoint{4.661264in}{0.594302in}}{\pgfqpoint{4.669500in}{0.594302in}}%
\pgfpathclose%
\pgfusepath{stroke,fill}%
\end{pgfscope}%
\begin{pgfscope}%
\pgfpathrectangle{\pgfqpoint{3.793912in}{0.557870in}}{\pgfqpoint{2.446088in}{1.484734in}}%
\pgfusepath{clip}%
\pgfsetbuttcap%
\pgfsetroundjoin%
\definecolor{currentfill}{rgb}{0.298039,0.447059,0.690196}%
\pgfsetfillcolor{currentfill}%
\pgfsetlinewidth{1.003750pt}%
\definecolor{currentstroke}{rgb}{0.298039,0.447059,0.690196}%
\pgfsetstrokecolor{currentstroke}%
\pgfsetdash{}{0pt}%
\pgfpathmoveto{\pgfqpoint{3.905098in}{1.330533in}}%
\pgfpathcurveto{\pgfqpoint{3.913334in}{1.330533in}}{\pgfqpoint{3.921234in}{1.333806in}}{\pgfqpoint{3.927058in}{1.339630in}}%
\pgfpathcurveto{\pgfqpoint{3.932882in}{1.345454in}}{\pgfqpoint{3.936155in}{1.353354in}}{\pgfqpoint{3.936155in}{1.361590in}}%
\pgfpathcurveto{\pgfqpoint{3.936155in}{1.369826in}}{\pgfqpoint{3.932882in}{1.377726in}}{\pgfqpoint{3.927058in}{1.383550in}}%
\pgfpathcurveto{\pgfqpoint{3.921234in}{1.389374in}}{\pgfqpoint{3.913334in}{1.392646in}}{\pgfqpoint{3.905098in}{1.392646in}}%
\pgfpathcurveto{\pgfqpoint{3.896862in}{1.392646in}}{\pgfqpoint{3.888962in}{1.389374in}}{\pgfqpoint{3.883138in}{1.383550in}}%
\pgfpathcurveto{\pgfqpoint{3.877314in}{1.377726in}}{\pgfqpoint{3.874042in}{1.369826in}}{\pgfqpoint{3.874042in}{1.361590in}}%
\pgfpathcurveto{\pgfqpoint{3.874042in}{1.353354in}}{\pgfqpoint{3.877314in}{1.345454in}}{\pgfqpoint{3.883138in}{1.339630in}}%
\pgfpathcurveto{\pgfqpoint{3.888962in}{1.333806in}}{\pgfqpoint{3.896862in}{1.330533in}}{\pgfqpoint{3.905098in}{1.330533in}}%
\pgfpathclose%
\pgfusepath{stroke,fill}%
\end{pgfscope}%
\begin{pgfscope}%
\pgfpathrectangle{\pgfqpoint{3.793912in}{0.557870in}}{\pgfqpoint{2.446088in}{1.484734in}}%
\pgfusepath{clip}%
\pgfsetbuttcap%
\pgfsetroundjoin%
\definecolor{currentfill}{rgb}{0.298039,0.447059,0.690196}%
\pgfsetfillcolor{currentfill}%
\pgfsetlinewidth{1.003750pt}%
\definecolor{currentstroke}{rgb}{0.298039,0.447059,0.690196}%
\pgfsetstrokecolor{currentstroke}%
\pgfsetdash{}{0pt}%
\pgfpathmoveto{\pgfqpoint{3.905098in}{1.330533in}}%
\pgfpathcurveto{\pgfqpoint{3.913334in}{1.330533in}}{\pgfqpoint{3.921234in}{1.333806in}}{\pgfqpoint{3.927058in}{1.339630in}}%
\pgfpathcurveto{\pgfqpoint{3.932882in}{1.345454in}}{\pgfqpoint{3.936155in}{1.353354in}}{\pgfqpoint{3.936155in}{1.361590in}}%
\pgfpathcurveto{\pgfqpoint{3.936155in}{1.369826in}}{\pgfqpoint{3.932882in}{1.377726in}}{\pgfqpoint{3.927058in}{1.383550in}}%
\pgfpathcurveto{\pgfqpoint{3.921234in}{1.389374in}}{\pgfqpoint{3.913334in}{1.392646in}}{\pgfqpoint{3.905098in}{1.392646in}}%
\pgfpathcurveto{\pgfqpoint{3.896862in}{1.392646in}}{\pgfqpoint{3.888962in}{1.389374in}}{\pgfqpoint{3.883138in}{1.383550in}}%
\pgfpathcurveto{\pgfqpoint{3.877314in}{1.377726in}}{\pgfqpoint{3.874042in}{1.369826in}}{\pgfqpoint{3.874042in}{1.361590in}}%
\pgfpathcurveto{\pgfqpoint{3.874042in}{1.353354in}}{\pgfqpoint{3.877314in}{1.345454in}}{\pgfqpoint{3.883138in}{1.339630in}}%
\pgfpathcurveto{\pgfqpoint{3.888962in}{1.333806in}}{\pgfqpoint{3.896862in}{1.330533in}}{\pgfqpoint{3.905098in}{1.330533in}}%
\pgfpathclose%
\pgfusepath{stroke,fill}%
\end{pgfscope}%
\begin{pgfscope}%
\pgfpathrectangle{\pgfqpoint{3.793912in}{0.557870in}}{\pgfqpoint{2.446088in}{1.484734in}}%
\pgfusepath{clip}%
\pgfsetbuttcap%
\pgfsetroundjoin%
\definecolor{currentfill}{rgb}{0.298039,0.447059,0.690196}%
\pgfsetfillcolor{currentfill}%
\pgfsetlinewidth{1.003750pt}%
\definecolor{currentstroke}{rgb}{0.298039,0.447059,0.690196}%
\pgfsetstrokecolor{currentstroke}%
\pgfsetdash{}{0pt}%
\pgfpathmoveto{\pgfqpoint{4.067244in}{0.594302in}}%
\pgfpathcurveto{\pgfqpoint{4.075480in}{0.594302in}}{\pgfqpoint{4.083380in}{0.597574in}}{\pgfqpoint{4.089204in}{0.603398in}}%
\pgfpathcurveto{\pgfqpoint{4.095028in}{0.609222in}}{\pgfqpoint{4.098301in}{0.617122in}}{\pgfqpoint{4.098301in}{0.625358in}}%
\pgfpathcurveto{\pgfqpoint{4.098301in}{0.633594in}}{\pgfqpoint{4.095028in}{0.641495in}}{\pgfqpoint{4.089204in}{0.647318in}}%
\pgfpathcurveto{\pgfqpoint{4.083380in}{0.653142in}}{\pgfqpoint{4.075480in}{0.656415in}}{\pgfqpoint{4.067244in}{0.656415in}}%
\pgfpathcurveto{\pgfqpoint{4.059008in}{0.656415in}}{\pgfqpoint{4.051108in}{0.653142in}}{\pgfqpoint{4.045284in}{0.647318in}}%
\pgfpathcurveto{\pgfqpoint{4.039460in}{0.641495in}}{\pgfqpoint{4.036188in}{0.633594in}}{\pgfqpoint{4.036188in}{0.625358in}}%
\pgfpathcurveto{\pgfqpoint{4.036188in}{0.617122in}}{\pgfqpoint{4.039460in}{0.609222in}}{\pgfqpoint{4.045284in}{0.603398in}}%
\pgfpathcurveto{\pgfqpoint{4.051108in}{0.597574in}}{\pgfqpoint{4.059008in}{0.594302in}}{\pgfqpoint{4.067244in}{0.594302in}}%
\pgfpathclose%
\pgfusepath{stroke,fill}%
\end{pgfscope}%
\begin{pgfscope}%
\pgfpathrectangle{\pgfqpoint{3.793912in}{0.557870in}}{\pgfqpoint{2.446088in}{1.484734in}}%
\pgfusepath{clip}%
\pgfsetbuttcap%
\pgfsetroundjoin%
\definecolor{currentfill}{rgb}{0.298039,0.447059,0.690196}%
\pgfsetfillcolor{currentfill}%
\pgfsetlinewidth{1.003750pt}%
\definecolor{currentstroke}{rgb}{0.298039,0.447059,0.690196}%
\pgfsetstrokecolor{currentstroke}%
\pgfsetdash{}{0pt}%
\pgfpathmoveto{\pgfqpoint{3.905098in}{1.930426in}}%
\pgfpathcurveto{\pgfqpoint{3.913334in}{1.930426in}}{\pgfqpoint{3.921234in}{1.933698in}}{\pgfqpoint{3.927058in}{1.939522in}}%
\pgfpathcurveto{\pgfqpoint{3.932882in}{1.945346in}}{\pgfqpoint{3.936155in}{1.953246in}}{\pgfqpoint{3.936155in}{1.961482in}}%
\pgfpathcurveto{\pgfqpoint{3.936155in}{1.969719in}}{\pgfqpoint{3.932882in}{1.977619in}}{\pgfqpoint{3.927058in}{1.983443in}}%
\pgfpathcurveto{\pgfqpoint{3.921234in}{1.989267in}}{\pgfqpoint{3.913334in}{1.992539in}}{\pgfqpoint{3.905098in}{1.992539in}}%
\pgfpathcurveto{\pgfqpoint{3.896862in}{1.992539in}}{\pgfqpoint{3.888962in}{1.989267in}}{\pgfqpoint{3.883138in}{1.983443in}}%
\pgfpathcurveto{\pgfqpoint{3.877314in}{1.977619in}}{\pgfqpoint{3.874042in}{1.969719in}}{\pgfqpoint{3.874042in}{1.961482in}}%
\pgfpathcurveto{\pgfqpoint{3.874042in}{1.953246in}}{\pgfqpoint{3.877314in}{1.945346in}}{\pgfqpoint{3.883138in}{1.939522in}}%
\pgfpathcurveto{\pgfqpoint{3.888962in}{1.933698in}}{\pgfqpoint{3.896862in}{1.930426in}}{\pgfqpoint{3.905098in}{1.930426in}}%
\pgfpathclose%
\pgfusepath{stroke,fill}%
\end{pgfscope}%
\begin{pgfscope}%
\pgfpathrectangle{\pgfqpoint{3.793912in}{0.557870in}}{\pgfqpoint{2.446088in}{1.484734in}}%
\pgfusepath{clip}%
\pgfsetbuttcap%
\pgfsetroundjoin%
\definecolor{currentfill}{rgb}{0.298039,0.447059,0.690196}%
\pgfsetfillcolor{currentfill}%
\pgfsetlinewidth{1.003750pt}%
\definecolor{currentstroke}{rgb}{0.298039,0.447059,0.690196}%
\pgfsetstrokecolor{currentstroke}%
\pgfsetdash{}{0pt}%
\pgfpathmoveto{\pgfqpoint{5.457067in}{0.594302in}}%
\pgfpathcurveto{\pgfqpoint{5.465303in}{0.594302in}}{\pgfqpoint{5.473203in}{0.597574in}}{\pgfqpoint{5.479027in}{0.603398in}}%
\pgfpathcurveto{\pgfqpoint{5.484851in}{0.609222in}}{\pgfqpoint{5.488123in}{0.617122in}}{\pgfqpoint{5.488123in}{0.625358in}}%
\pgfpathcurveto{\pgfqpoint{5.488123in}{0.633594in}}{\pgfqpoint{5.484851in}{0.641495in}}{\pgfqpoint{5.479027in}{0.647318in}}%
\pgfpathcurveto{\pgfqpoint{5.473203in}{0.653142in}}{\pgfqpoint{5.465303in}{0.656415in}}{\pgfqpoint{5.457067in}{0.656415in}}%
\pgfpathcurveto{\pgfqpoint{5.448830in}{0.656415in}}{\pgfqpoint{5.440930in}{0.653142in}}{\pgfqpoint{5.435106in}{0.647318in}}%
\pgfpathcurveto{\pgfqpoint{5.429282in}{0.641495in}}{\pgfqpoint{5.426010in}{0.633594in}}{\pgfqpoint{5.426010in}{0.625358in}}%
\pgfpathcurveto{\pgfqpoint{5.426010in}{0.617122in}}{\pgfqpoint{5.429282in}{0.609222in}}{\pgfqpoint{5.435106in}{0.603398in}}%
\pgfpathcurveto{\pgfqpoint{5.440930in}{0.597574in}}{\pgfqpoint{5.448830in}{0.594302in}}{\pgfqpoint{5.457067in}{0.594302in}}%
\pgfpathclose%
\pgfusepath{stroke,fill}%
\end{pgfscope}%
\begin{pgfscope}%
\pgfpathrectangle{\pgfqpoint{3.793912in}{0.557870in}}{\pgfqpoint{2.446088in}{1.484734in}}%
\pgfusepath{clip}%
\pgfsetbuttcap%
\pgfsetroundjoin%
\definecolor{currentfill}{rgb}{0.298039,0.447059,0.690196}%
\pgfsetfillcolor{currentfill}%
\pgfsetlinewidth{1.003750pt}%
\definecolor{currentstroke}{rgb}{0.298039,0.447059,0.690196}%
\pgfsetstrokecolor{currentstroke}%
\pgfsetdash{}{0pt}%
\pgfpathmoveto{\pgfqpoint{3.905098in}{1.930426in}}%
\pgfpathcurveto{\pgfqpoint{3.913334in}{1.930426in}}{\pgfqpoint{3.921234in}{1.933698in}}{\pgfqpoint{3.927058in}{1.939522in}}%
\pgfpathcurveto{\pgfqpoint{3.932882in}{1.945346in}}{\pgfqpoint{3.936155in}{1.953246in}}{\pgfqpoint{3.936155in}{1.961482in}}%
\pgfpathcurveto{\pgfqpoint{3.936155in}{1.969719in}}{\pgfqpoint{3.932882in}{1.977619in}}{\pgfqpoint{3.927058in}{1.983443in}}%
\pgfpathcurveto{\pgfqpoint{3.921234in}{1.989267in}}{\pgfqpoint{3.913334in}{1.992539in}}{\pgfqpoint{3.905098in}{1.992539in}}%
\pgfpathcurveto{\pgfqpoint{3.896862in}{1.992539in}}{\pgfqpoint{3.888962in}{1.989267in}}{\pgfqpoint{3.883138in}{1.983443in}}%
\pgfpathcurveto{\pgfqpoint{3.877314in}{1.977619in}}{\pgfqpoint{3.874042in}{1.969719in}}{\pgfqpoint{3.874042in}{1.961482in}}%
\pgfpathcurveto{\pgfqpoint{3.874042in}{1.953246in}}{\pgfqpoint{3.877314in}{1.945346in}}{\pgfqpoint{3.883138in}{1.939522in}}%
\pgfpathcurveto{\pgfqpoint{3.888962in}{1.933698in}}{\pgfqpoint{3.896862in}{1.930426in}}{\pgfqpoint{3.905098in}{1.930426in}}%
\pgfpathclose%
\pgfusepath{stroke,fill}%
\end{pgfscope}%
\begin{pgfscope}%
\pgfpathrectangle{\pgfqpoint{3.793912in}{0.557870in}}{\pgfqpoint{2.446088in}{1.484734in}}%
\pgfusepath{clip}%
\pgfsetbuttcap%
\pgfsetroundjoin%
\definecolor{currentfill}{rgb}{0.298039,0.447059,0.690196}%
\pgfsetfillcolor{currentfill}%
\pgfsetlinewidth{1.003750pt}%
\definecolor{currentstroke}{rgb}{0.298039,0.447059,0.690196}%
\pgfsetstrokecolor{currentstroke}%
\pgfsetdash{}{0pt}%
\pgfpathmoveto{\pgfqpoint{5.410739in}{0.594302in}}%
\pgfpathcurveto{\pgfqpoint{5.418975in}{0.594302in}}{\pgfqpoint{5.426876in}{0.597574in}}{\pgfqpoint{5.432699in}{0.603398in}}%
\pgfpathcurveto{\pgfqpoint{5.438523in}{0.609222in}}{\pgfqpoint{5.441796in}{0.617122in}}{\pgfqpoint{5.441796in}{0.625358in}}%
\pgfpathcurveto{\pgfqpoint{5.441796in}{0.633594in}}{\pgfqpoint{5.438523in}{0.641495in}}{\pgfqpoint{5.432699in}{0.647318in}}%
\pgfpathcurveto{\pgfqpoint{5.426876in}{0.653142in}}{\pgfqpoint{5.418975in}{0.656415in}}{\pgfqpoint{5.410739in}{0.656415in}}%
\pgfpathcurveto{\pgfqpoint{5.402503in}{0.656415in}}{\pgfqpoint{5.394603in}{0.653142in}}{\pgfqpoint{5.388779in}{0.647318in}}%
\pgfpathcurveto{\pgfqpoint{5.382955in}{0.641495in}}{\pgfqpoint{5.379683in}{0.633594in}}{\pgfqpoint{5.379683in}{0.625358in}}%
\pgfpathcurveto{\pgfqpoint{5.379683in}{0.617122in}}{\pgfqpoint{5.382955in}{0.609222in}}{\pgfqpoint{5.388779in}{0.603398in}}%
\pgfpathcurveto{\pgfqpoint{5.394603in}{0.597574in}}{\pgfqpoint{5.402503in}{0.594302in}}{\pgfqpoint{5.410739in}{0.594302in}}%
\pgfpathclose%
\pgfusepath{stroke,fill}%
\end{pgfscope}%
\begin{pgfscope}%
\pgfpathrectangle{\pgfqpoint{3.793912in}{0.557870in}}{\pgfqpoint{2.446088in}{1.484734in}}%
\pgfusepath{clip}%
\pgfsetbuttcap%
\pgfsetroundjoin%
\definecolor{currentfill}{rgb}{0.298039,0.447059,0.690196}%
\pgfsetfillcolor{currentfill}%
\pgfsetlinewidth{1.003750pt}%
\definecolor{currentstroke}{rgb}{0.298039,0.447059,0.690196}%
\pgfsetstrokecolor{currentstroke}%
\pgfsetdash{}{0pt}%
\pgfpathmoveto{\pgfqpoint{3.905098in}{1.930426in}}%
\pgfpathcurveto{\pgfqpoint{3.913334in}{1.930426in}}{\pgfqpoint{3.921234in}{1.933698in}}{\pgfqpoint{3.927058in}{1.939522in}}%
\pgfpathcurveto{\pgfqpoint{3.932882in}{1.945346in}}{\pgfqpoint{3.936155in}{1.953246in}}{\pgfqpoint{3.936155in}{1.961482in}}%
\pgfpathcurveto{\pgfqpoint{3.936155in}{1.969719in}}{\pgfqpoint{3.932882in}{1.977619in}}{\pgfqpoint{3.927058in}{1.983443in}}%
\pgfpathcurveto{\pgfqpoint{3.921234in}{1.989267in}}{\pgfqpoint{3.913334in}{1.992539in}}{\pgfqpoint{3.905098in}{1.992539in}}%
\pgfpathcurveto{\pgfqpoint{3.896862in}{1.992539in}}{\pgfqpoint{3.888962in}{1.989267in}}{\pgfqpoint{3.883138in}{1.983443in}}%
\pgfpathcurveto{\pgfqpoint{3.877314in}{1.977619in}}{\pgfqpoint{3.874042in}{1.969719in}}{\pgfqpoint{3.874042in}{1.961482in}}%
\pgfpathcurveto{\pgfqpoint{3.874042in}{1.953246in}}{\pgfqpoint{3.877314in}{1.945346in}}{\pgfqpoint{3.883138in}{1.939522in}}%
\pgfpathcurveto{\pgfqpoint{3.888962in}{1.933698in}}{\pgfqpoint{3.896862in}{1.930426in}}{\pgfqpoint{3.905098in}{1.930426in}}%
\pgfpathclose%
\pgfusepath{stroke,fill}%
\end{pgfscope}%
\begin{pgfscope}%
\pgfpathrectangle{\pgfqpoint{3.793912in}{0.557870in}}{\pgfqpoint{2.446088in}{1.484734in}}%
\pgfusepath{clip}%
\pgfsetbuttcap%
\pgfsetroundjoin%
\definecolor{currentfill}{rgb}{0.298039,0.447059,0.690196}%
\pgfsetfillcolor{currentfill}%
\pgfsetlinewidth{1.003750pt}%
\definecolor{currentstroke}{rgb}{0.298039,0.447059,0.690196}%
\pgfsetstrokecolor{currentstroke}%
\pgfsetdash{}{0pt}%
\pgfpathmoveto{\pgfqpoint{3.905098in}{1.930426in}}%
\pgfpathcurveto{\pgfqpoint{3.913334in}{1.930426in}}{\pgfqpoint{3.921234in}{1.933698in}}{\pgfqpoint{3.927058in}{1.939522in}}%
\pgfpathcurveto{\pgfqpoint{3.932882in}{1.945346in}}{\pgfqpoint{3.936155in}{1.953246in}}{\pgfqpoint{3.936155in}{1.961482in}}%
\pgfpathcurveto{\pgfqpoint{3.936155in}{1.969719in}}{\pgfqpoint{3.932882in}{1.977619in}}{\pgfqpoint{3.927058in}{1.983443in}}%
\pgfpathcurveto{\pgfqpoint{3.921234in}{1.989267in}}{\pgfqpoint{3.913334in}{1.992539in}}{\pgfqpoint{3.905098in}{1.992539in}}%
\pgfpathcurveto{\pgfqpoint{3.896862in}{1.992539in}}{\pgfqpoint{3.888962in}{1.989267in}}{\pgfqpoint{3.883138in}{1.983443in}}%
\pgfpathcurveto{\pgfqpoint{3.877314in}{1.977619in}}{\pgfqpoint{3.874042in}{1.969719in}}{\pgfqpoint{3.874042in}{1.961482in}}%
\pgfpathcurveto{\pgfqpoint{3.874042in}{1.953246in}}{\pgfqpoint{3.877314in}{1.945346in}}{\pgfqpoint{3.883138in}{1.939522in}}%
\pgfpathcurveto{\pgfqpoint{3.888962in}{1.933698in}}{\pgfqpoint{3.896862in}{1.930426in}}{\pgfqpoint{3.905098in}{1.930426in}}%
\pgfpathclose%
\pgfusepath{stroke,fill}%
\end{pgfscope}%
\begin{pgfscope}%
\pgfpathrectangle{\pgfqpoint{3.793912in}{0.557870in}}{\pgfqpoint{2.446088in}{1.484734in}}%
\pgfusepath{clip}%
\pgfsetbuttcap%
\pgfsetroundjoin%
\definecolor{currentfill}{rgb}{0.298039,0.447059,0.690196}%
\pgfsetfillcolor{currentfill}%
\pgfsetlinewidth{1.003750pt}%
\definecolor{currentstroke}{rgb}{0.298039,0.447059,0.690196}%
\pgfsetstrokecolor{currentstroke}%
\pgfsetdash{}{0pt}%
\pgfpathmoveto{\pgfqpoint{3.905098in}{1.930426in}}%
\pgfpathcurveto{\pgfqpoint{3.913334in}{1.930426in}}{\pgfqpoint{3.921234in}{1.933698in}}{\pgfqpoint{3.927058in}{1.939522in}}%
\pgfpathcurveto{\pgfqpoint{3.932882in}{1.945346in}}{\pgfqpoint{3.936155in}{1.953246in}}{\pgfqpoint{3.936155in}{1.961482in}}%
\pgfpathcurveto{\pgfqpoint{3.936155in}{1.969719in}}{\pgfqpoint{3.932882in}{1.977619in}}{\pgfqpoint{3.927058in}{1.983443in}}%
\pgfpathcurveto{\pgfqpoint{3.921234in}{1.989267in}}{\pgfqpoint{3.913334in}{1.992539in}}{\pgfqpoint{3.905098in}{1.992539in}}%
\pgfpathcurveto{\pgfqpoint{3.896862in}{1.992539in}}{\pgfqpoint{3.888962in}{1.989267in}}{\pgfqpoint{3.883138in}{1.983443in}}%
\pgfpathcurveto{\pgfqpoint{3.877314in}{1.977619in}}{\pgfqpoint{3.874042in}{1.969719in}}{\pgfqpoint{3.874042in}{1.961482in}}%
\pgfpathcurveto{\pgfqpoint{3.874042in}{1.953246in}}{\pgfqpoint{3.877314in}{1.945346in}}{\pgfqpoint{3.883138in}{1.939522in}}%
\pgfpathcurveto{\pgfqpoint{3.888962in}{1.933698in}}{\pgfqpoint{3.896862in}{1.930426in}}{\pgfqpoint{3.905098in}{1.930426in}}%
\pgfpathclose%
\pgfusepath{stroke,fill}%
\end{pgfscope}%
\begin{pgfscope}%
\pgfpathrectangle{\pgfqpoint{3.793912in}{0.557870in}}{\pgfqpoint{2.446088in}{1.484734in}}%
\pgfusepath{clip}%
\pgfsetbuttcap%
\pgfsetroundjoin%
\definecolor{currentfill}{rgb}{0.298039,0.447059,0.690196}%
\pgfsetfillcolor{currentfill}%
\pgfsetlinewidth{1.003750pt}%
\definecolor{currentstroke}{rgb}{0.298039,0.447059,0.690196}%
\pgfsetstrokecolor{currentstroke}%
\pgfsetdash{}{0pt}%
\pgfpathmoveto{\pgfqpoint{3.905098in}{1.930426in}}%
\pgfpathcurveto{\pgfqpoint{3.913334in}{1.930426in}}{\pgfqpoint{3.921234in}{1.933698in}}{\pgfqpoint{3.927058in}{1.939522in}}%
\pgfpathcurveto{\pgfqpoint{3.932882in}{1.945346in}}{\pgfqpoint{3.936155in}{1.953246in}}{\pgfqpoint{3.936155in}{1.961482in}}%
\pgfpathcurveto{\pgfqpoint{3.936155in}{1.969719in}}{\pgfqpoint{3.932882in}{1.977619in}}{\pgfqpoint{3.927058in}{1.983443in}}%
\pgfpathcurveto{\pgfqpoint{3.921234in}{1.989267in}}{\pgfqpoint{3.913334in}{1.992539in}}{\pgfqpoint{3.905098in}{1.992539in}}%
\pgfpathcurveto{\pgfqpoint{3.896862in}{1.992539in}}{\pgfqpoint{3.888962in}{1.989267in}}{\pgfqpoint{3.883138in}{1.983443in}}%
\pgfpathcurveto{\pgfqpoint{3.877314in}{1.977619in}}{\pgfqpoint{3.874042in}{1.969719in}}{\pgfqpoint{3.874042in}{1.961482in}}%
\pgfpathcurveto{\pgfqpoint{3.874042in}{1.953246in}}{\pgfqpoint{3.877314in}{1.945346in}}{\pgfqpoint{3.883138in}{1.939522in}}%
\pgfpathcurveto{\pgfqpoint{3.888962in}{1.933698in}}{\pgfqpoint{3.896862in}{1.930426in}}{\pgfqpoint{3.905098in}{1.930426in}}%
\pgfpathclose%
\pgfusepath{stroke,fill}%
\end{pgfscope}%
\begin{pgfscope}%
\pgfpathrectangle{\pgfqpoint{3.793912in}{0.557870in}}{\pgfqpoint{2.446088in}{1.484734in}}%
\pgfusepath{clip}%
\pgfsetbuttcap%
\pgfsetroundjoin%
\definecolor{currentfill}{rgb}{0.298039,0.447059,0.690196}%
\pgfsetfillcolor{currentfill}%
\pgfsetlinewidth{1.003750pt}%
\definecolor{currentstroke}{rgb}{0.298039,0.447059,0.690196}%
\pgfsetstrokecolor{currentstroke}%
\pgfsetdash{}{0pt}%
\pgfpathmoveto{\pgfqpoint{3.905098in}{1.930426in}}%
\pgfpathcurveto{\pgfqpoint{3.913334in}{1.930426in}}{\pgfqpoint{3.921234in}{1.933698in}}{\pgfqpoint{3.927058in}{1.939522in}}%
\pgfpathcurveto{\pgfqpoint{3.932882in}{1.945346in}}{\pgfqpoint{3.936155in}{1.953246in}}{\pgfqpoint{3.936155in}{1.961482in}}%
\pgfpathcurveto{\pgfqpoint{3.936155in}{1.969719in}}{\pgfqpoint{3.932882in}{1.977619in}}{\pgfqpoint{3.927058in}{1.983443in}}%
\pgfpathcurveto{\pgfqpoint{3.921234in}{1.989267in}}{\pgfqpoint{3.913334in}{1.992539in}}{\pgfqpoint{3.905098in}{1.992539in}}%
\pgfpathcurveto{\pgfqpoint{3.896862in}{1.992539in}}{\pgfqpoint{3.888962in}{1.989267in}}{\pgfqpoint{3.883138in}{1.983443in}}%
\pgfpathcurveto{\pgfqpoint{3.877314in}{1.977619in}}{\pgfqpoint{3.874042in}{1.969719in}}{\pgfqpoint{3.874042in}{1.961482in}}%
\pgfpathcurveto{\pgfqpoint{3.874042in}{1.953246in}}{\pgfqpoint{3.877314in}{1.945346in}}{\pgfqpoint{3.883138in}{1.939522in}}%
\pgfpathcurveto{\pgfqpoint{3.888962in}{1.933698in}}{\pgfqpoint{3.896862in}{1.930426in}}{\pgfqpoint{3.905098in}{1.930426in}}%
\pgfpathclose%
\pgfusepath{stroke,fill}%
\end{pgfscope}%
\begin{pgfscope}%
\pgfpathrectangle{\pgfqpoint{3.793912in}{0.557870in}}{\pgfqpoint{2.446088in}{1.484734in}}%
\pgfusepath{clip}%
\pgfsetbuttcap%
\pgfsetroundjoin%
\definecolor{currentfill}{rgb}{0.298039,0.447059,0.690196}%
\pgfsetfillcolor{currentfill}%
\pgfsetlinewidth{1.003750pt}%
\definecolor{currentstroke}{rgb}{0.298039,0.447059,0.690196}%
\pgfsetstrokecolor{currentstroke}%
\pgfsetdash{}{0pt}%
\pgfpathmoveto{\pgfqpoint{3.905098in}{1.930426in}}%
\pgfpathcurveto{\pgfqpoint{3.913334in}{1.930426in}}{\pgfqpoint{3.921234in}{1.933698in}}{\pgfqpoint{3.927058in}{1.939522in}}%
\pgfpathcurveto{\pgfqpoint{3.932882in}{1.945346in}}{\pgfqpoint{3.936155in}{1.953246in}}{\pgfqpoint{3.936155in}{1.961482in}}%
\pgfpathcurveto{\pgfqpoint{3.936155in}{1.969719in}}{\pgfqpoint{3.932882in}{1.977619in}}{\pgfqpoint{3.927058in}{1.983443in}}%
\pgfpathcurveto{\pgfqpoint{3.921234in}{1.989267in}}{\pgfqpoint{3.913334in}{1.992539in}}{\pgfqpoint{3.905098in}{1.992539in}}%
\pgfpathcurveto{\pgfqpoint{3.896862in}{1.992539in}}{\pgfqpoint{3.888962in}{1.989267in}}{\pgfqpoint{3.883138in}{1.983443in}}%
\pgfpathcurveto{\pgfqpoint{3.877314in}{1.977619in}}{\pgfqpoint{3.874042in}{1.969719in}}{\pgfqpoint{3.874042in}{1.961482in}}%
\pgfpathcurveto{\pgfqpoint{3.874042in}{1.953246in}}{\pgfqpoint{3.877314in}{1.945346in}}{\pgfqpoint{3.883138in}{1.939522in}}%
\pgfpathcurveto{\pgfqpoint{3.888962in}{1.933698in}}{\pgfqpoint{3.896862in}{1.930426in}}{\pgfqpoint{3.905098in}{1.930426in}}%
\pgfpathclose%
\pgfusepath{stroke,fill}%
\end{pgfscope}%
\begin{pgfscope}%
\pgfpathrectangle{\pgfqpoint{3.793912in}{0.557870in}}{\pgfqpoint{2.446088in}{1.484734in}}%
\pgfusepath{clip}%
\pgfsetbuttcap%
\pgfsetroundjoin%
\definecolor{currentfill}{rgb}{0.298039,0.447059,0.690196}%
\pgfsetfillcolor{currentfill}%
\pgfsetlinewidth{1.003750pt}%
\definecolor{currentstroke}{rgb}{0.298039,0.447059,0.690196}%
\pgfsetstrokecolor{currentstroke}%
\pgfsetdash{}{0pt}%
\pgfpathmoveto{\pgfqpoint{5.248593in}{0.594302in}}%
\pgfpathcurveto{\pgfqpoint{5.256829in}{0.594302in}}{\pgfqpoint{5.264730in}{0.597574in}}{\pgfqpoint{5.270553in}{0.603398in}}%
\pgfpathcurveto{\pgfqpoint{5.276377in}{0.609222in}}{\pgfqpoint{5.279650in}{0.617122in}}{\pgfqpoint{5.279650in}{0.625358in}}%
\pgfpathcurveto{\pgfqpoint{5.279650in}{0.633594in}}{\pgfqpoint{5.276377in}{0.641495in}}{\pgfqpoint{5.270553in}{0.647318in}}%
\pgfpathcurveto{\pgfqpoint{5.264730in}{0.653142in}}{\pgfqpoint{5.256829in}{0.656415in}}{\pgfqpoint{5.248593in}{0.656415in}}%
\pgfpathcurveto{\pgfqpoint{5.240357in}{0.656415in}}{\pgfqpoint{5.232457in}{0.653142in}}{\pgfqpoint{5.226633in}{0.647318in}}%
\pgfpathcurveto{\pgfqpoint{5.220809in}{0.641495in}}{\pgfqpoint{5.217537in}{0.633594in}}{\pgfqpoint{5.217537in}{0.625358in}}%
\pgfpathcurveto{\pgfqpoint{5.217537in}{0.617122in}}{\pgfqpoint{5.220809in}{0.609222in}}{\pgfqpoint{5.226633in}{0.603398in}}%
\pgfpathcurveto{\pgfqpoint{5.232457in}{0.597574in}}{\pgfqpoint{5.240357in}{0.594302in}}{\pgfqpoint{5.248593in}{0.594302in}}%
\pgfpathclose%
\pgfusepath{stroke,fill}%
\end{pgfscope}%
\begin{pgfscope}%
\pgfpathrectangle{\pgfqpoint{3.793912in}{0.557870in}}{\pgfqpoint{2.446088in}{1.484734in}}%
\pgfusepath{clip}%
\pgfsetbuttcap%
\pgfsetroundjoin%
\definecolor{currentfill}{rgb}{0.298039,0.447059,0.690196}%
\pgfsetfillcolor{currentfill}%
\pgfsetlinewidth{1.003750pt}%
\definecolor{currentstroke}{rgb}{0.298039,0.447059,0.690196}%
\pgfsetstrokecolor{currentstroke}%
\pgfsetdash{}{0pt}%
\pgfpathmoveto{\pgfqpoint{3.905098in}{1.930426in}}%
\pgfpathcurveto{\pgfqpoint{3.913334in}{1.930426in}}{\pgfqpoint{3.921234in}{1.933698in}}{\pgfqpoint{3.927058in}{1.939522in}}%
\pgfpathcurveto{\pgfqpoint{3.932882in}{1.945346in}}{\pgfqpoint{3.936155in}{1.953246in}}{\pgfqpoint{3.936155in}{1.961482in}}%
\pgfpathcurveto{\pgfqpoint{3.936155in}{1.969719in}}{\pgfqpoint{3.932882in}{1.977619in}}{\pgfqpoint{3.927058in}{1.983443in}}%
\pgfpathcurveto{\pgfqpoint{3.921234in}{1.989267in}}{\pgfqpoint{3.913334in}{1.992539in}}{\pgfqpoint{3.905098in}{1.992539in}}%
\pgfpathcurveto{\pgfqpoint{3.896862in}{1.992539in}}{\pgfqpoint{3.888962in}{1.989267in}}{\pgfqpoint{3.883138in}{1.983443in}}%
\pgfpathcurveto{\pgfqpoint{3.877314in}{1.977619in}}{\pgfqpoint{3.874042in}{1.969719in}}{\pgfqpoint{3.874042in}{1.961482in}}%
\pgfpathcurveto{\pgfqpoint{3.874042in}{1.953246in}}{\pgfqpoint{3.877314in}{1.945346in}}{\pgfqpoint{3.883138in}{1.939522in}}%
\pgfpathcurveto{\pgfqpoint{3.888962in}{1.933698in}}{\pgfqpoint{3.896862in}{1.930426in}}{\pgfqpoint{3.905098in}{1.930426in}}%
\pgfpathclose%
\pgfusepath{stroke,fill}%
\end{pgfscope}%
\begin{pgfscope}%
\pgfpathrectangle{\pgfqpoint{3.793912in}{0.557870in}}{\pgfqpoint{2.446088in}{1.484734in}}%
\pgfusepath{clip}%
\pgfsetbuttcap%
\pgfsetroundjoin%
\definecolor{currentfill}{rgb}{0.298039,0.447059,0.690196}%
\pgfsetfillcolor{currentfill}%
\pgfsetlinewidth{1.003750pt}%
\definecolor{currentstroke}{rgb}{0.298039,0.447059,0.690196}%
\pgfsetstrokecolor{currentstroke}%
\pgfsetdash{}{0pt}%
\pgfpathmoveto{\pgfqpoint{3.905098in}{1.930426in}}%
\pgfpathcurveto{\pgfqpoint{3.913334in}{1.930426in}}{\pgfqpoint{3.921234in}{1.933698in}}{\pgfqpoint{3.927058in}{1.939522in}}%
\pgfpathcurveto{\pgfqpoint{3.932882in}{1.945346in}}{\pgfqpoint{3.936155in}{1.953246in}}{\pgfqpoint{3.936155in}{1.961482in}}%
\pgfpathcurveto{\pgfqpoint{3.936155in}{1.969719in}}{\pgfqpoint{3.932882in}{1.977619in}}{\pgfqpoint{3.927058in}{1.983443in}}%
\pgfpathcurveto{\pgfqpoint{3.921234in}{1.989267in}}{\pgfqpoint{3.913334in}{1.992539in}}{\pgfqpoint{3.905098in}{1.992539in}}%
\pgfpathcurveto{\pgfqpoint{3.896862in}{1.992539in}}{\pgfqpoint{3.888962in}{1.989267in}}{\pgfqpoint{3.883138in}{1.983443in}}%
\pgfpathcurveto{\pgfqpoint{3.877314in}{1.977619in}}{\pgfqpoint{3.874042in}{1.969719in}}{\pgfqpoint{3.874042in}{1.961482in}}%
\pgfpathcurveto{\pgfqpoint{3.874042in}{1.953246in}}{\pgfqpoint{3.877314in}{1.945346in}}{\pgfqpoint{3.883138in}{1.939522in}}%
\pgfpathcurveto{\pgfqpoint{3.888962in}{1.933698in}}{\pgfqpoint{3.896862in}{1.930426in}}{\pgfqpoint{3.905098in}{1.930426in}}%
\pgfpathclose%
\pgfusepath{stroke,fill}%
\end{pgfscope}%
\begin{pgfscope}%
\pgfpathrectangle{\pgfqpoint{3.793912in}{0.557870in}}{\pgfqpoint{2.446088in}{1.484734in}}%
\pgfusepath{clip}%
\pgfsetbuttcap%
\pgfsetroundjoin%
\definecolor{currentfill}{rgb}{0.298039,0.447059,0.690196}%
\pgfsetfillcolor{currentfill}%
\pgfsetlinewidth{1.003750pt}%
\definecolor{currentstroke}{rgb}{0.298039,0.447059,0.690196}%
\pgfsetstrokecolor{currentstroke}%
\pgfsetdash{}{0pt}%
\pgfpathmoveto{\pgfqpoint{3.905098in}{1.930426in}}%
\pgfpathcurveto{\pgfqpoint{3.913334in}{1.930426in}}{\pgfqpoint{3.921234in}{1.933698in}}{\pgfqpoint{3.927058in}{1.939522in}}%
\pgfpathcurveto{\pgfqpoint{3.932882in}{1.945346in}}{\pgfqpoint{3.936155in}{1.953246in}}{\pgfqpoint{3.936155in}{1.961482in}}%
\pgfpathcurveto{\pgfqpoint{3.936155in}{1.969719in}}{\pgfqpoint{3.932882in}{1.977619in}}{\pgfqpoint{3.927058in}{1.983443in}}%
\pgfpathcurveto{\pgfqpoint{3.921234in}{1.989267in}}{\pgfqpoint{3.913334in}{1.992539in}}{\pgfqpoint{3.905098in}{1.992539in}}%
\pgfpathcurveto{\pgfqpoint{3.896862in}{1.992539in}}{\pgfqpoint{3.888962in}{1.989267in}}{\pgfqpoint{3.883138in}{1.983443in}}%
\pgfpathcurveto{\pgfqpoint{3.877314in}{1.977619in}}{\pgfqpoint{3.874042in}{1.969719in}}{\pgfqpoint{3.874042in}{1.961482in}}%
\pgfpathcurveto{\pgfqpoint{3.874042in}{1.953246in}}{\pgfqpoint{3.877314in}{1.945346in}}{\pgfqpoint{3.883138in}{1.939522in}}%
\pgfpathcurveto{\pgfqpoint{3.888962in}{1.933698in}}{\pgfqpoint{3.896862in}{1.930426in}}{\pgfqpoint{3.905098in}{1.930426in}}%
\pgfpathclose%
\pgfusepath{stroke,fill}%
\end{pgfscope}%
\begin{pgfscope}%
\pgfpathrectangle{\pgfqpoint{3.793912in}{0.557870in}}{\pgfqpoint{2.446088in}{1.484734in}}%
\pgfusepath{clip}%
\pgfsetbuttcap%
\pgfsetroundjoin%
\definecolor{currentfill}{rgb}{0.298039,0.447059,0.690196}%
\pgfsetfillcolor{currentfill}%
\pgfsetlinewidth{1.003750pt}%
\definecolor{currentstroke}{rgb}{0.298039,0.447059,0.690196}%
\pgfsetstrokecolor{currentstroke}%
\pgfsetdash{}{0pt}%
\pgfpathmoveto{\pgfqpoint{3.905098in}{1.930426in}}%
\pgfpathcurveto{\pgfqpoint{3.913334in}{1.930426in}}{\pgfqpoint{3.921234in}{1.933698in}}{\pgfqpoint{3.927058in}{1.939522in}}%
\pgfpathcurveto{\pgfqpoint{3.932882in}{1.945346in}}{\pgfqpoint{3.936155in}{1.953246in}}{\pgfqpoint{3.936155in}{1.961482in}}%
\pgfpathcurveto{\pgfqpoint{3.936155in}{1.969719in}}{\pgfqpoint{3.932882in}{1.977619in}}{\pgfqpoint{3.927058in}{1.983443in}}%
\pgfpathcurveto{\pgfqpoint{3.921234in}{1.989267in}}{\pgfqpoint{3.913334in}{1.992539in}}{\pgfqpoint{3.905098in}{1.992539in}}%
\pgfpathcurveto{\pgfqpoint{3.896862in}{1.992539in}}{\pgfqpoint{3.888962in}{1.989267in}}{\pgfqpoint{3.883138in}{1.983443in}}%
\pgfpathcurveto{\pgfqpoint{3.877314in}{1.977619in}}{\pgfqpoint{3.874042in}{1.969719in}}{\pgfqpoint{3.874042in}{1.961482in}}%
\pgfpathcurveto{\pgfqpoint{3.874042in}{1.953246in}}{\pgfqpoint{3.877314in}{1.945346in}}{\pgfqpoint{3.883138in}{1.939522in}}%
\pgfpathcurveto{\pgfqpoint{3.888962in}{1.933698in}}{\pgfqpoint{3.896862in}{1.930426in}}{\pgfqpoint{3.905098in}{1.930426in}}%
\pgfpathclose%
\pgfusepath{stroke,fill}%
\end{pgfscope}%
\begin{pgfscope}%
\pgfpathrectangle{\pgfqpoint{3.793912in}{0.557870in}}{\pgfqpoint{2.446088in}{1.484734in}}%
\pgfusepath{clip}%
\pgfsetbuttcap%
\pgfsetroundjoin%
\definecolor{currentfill}{rgb}{0.298039,0.447059,0.690196}%
\pgfsetfillcolor{currentfill}%
\pgfsetlinewidth{1.003750pt}%
\definecolor{currentstroke}{rgb}{0.298039,0.447059,0.690196}%
\pgfsetstrokecolor{currentstroke}%
\pgfsetdash{}{0pt}%
\pgfpathmoveto{\pgfqpoint{3.905098in}{1.330533in}}%
\pgfpathcurveto{\pgfqpoint{3.913334in}{1.330533in}}{\pgfqpoint{3.921234in}{1.333806in}}{\pgfqpoint{3.927058in}{1.339630in}}%
\pgfpathcurveto{\pgfqpoint{3.932882in}{1.345454in}}{\pgfqpoint{3.936155in}{1.353354in}}{\pgfqpoint{3.936155in}{1.361590in}}%
\pgfpathcurveto{\pgfqpoint{3.936155in}{1.369826in}}{\pgfqpoint{3.932882in}{1.377726in}}{\pgfqpoint{3.927058in}{1.383550in}}%
\pgfpathcurveto{\pgfqpoint{3.921234in}{1.389374in}}{\pgfqpoint{3.913334in}{1.392646in}}{\pgfqpoint{3.905098in}{1.392646in}}%
\pgfpathcurveto{\pgfqpoint{3.896862in}{1.392646in}}{\pgfqpoint{3.888962in}{1.389374in}}{\pgfqpoint{3.883138in}{1.383550in}}%
\pgfpathcurveto{\pgfqpoint{3.877314in}{1.377726in}}{\pgfqpoint{3.874042in}{1.369826in}}{\pgfqpoint{3.874042in}{1.361590in}}%
\pgfpathcurveto{\pgfqpoint{3.874042in}{1.353354in}}{\pgfqpoint{3.877314in}{1.345454in}}{\pgfqpoint{3.883138in}{1.339630in}}%
\pgfpathcurveto{\pgfqpoint{3.888962in}{1.333806in}}{\pgfqpoint{3.896862in}{1.330533in}}{\pgfqpoint{3.905098in}{1.330533in}}%
\pgfpathclose%
\pgfusepath{stroke,fill}%
\end{pgfscope}%
\begin{pgfscope}%
\pgfpathrectangle{\pgfqpoint{3.793912in}{0.557870in}}{\pgfqpoint{2.446088in}{1.484734in}}%
\pgfusepath{clip}%
\pgfsetbuttcap%
\pgfsetroundjoin%
\definecolor{currentfill}{rgb}{0.298039,0.447059,0.690196}%
\pgfsetfillcolor{currentfill}%
\pgfsetlinewidth{1.003750pt}%
\definecolor{currentstroke}{rgb}{0.298039,0.447059,0.690196}%
\pgfsetstrokecolor{currentstroke}%
\pgfsetdash{}{0pt}%
\pgfpathmoveto{\pgfqpoint{5.619213in}{0.594302in}}%
\pgfpathcurveto{\pgfqpoint{5.627449in}{0.594302in}}{\pgfqpoint{5.635349in}{0.597574in}}{\pgfqpoint{5.641173in}{0.603398in}}%
\pgfpathcurveto{\pgfqpoint{5.646997in}{0.609222in}}{\pgfqpoint{5.650269in}{0.617122in}}{\pgfqpoint{5.650269in}{0.625358in}}%
\pgfpathcurveto{\pgfqpoint{5.650269in}{0.633594in}}{\pgfqpoint{5.646997in}{0.641495in}}{\pgfqpoint{5.641173in}{0.647318in}}%
\pgfpathcurveto{\pgfqpoint{5.635349in}{0.653142in}}{\pgfqpoint{5.627449in}{0.656415in}}{\pgfqpoint{5.619213in}{0.656415in}}%
\pgfpathcurveto{\pgfqpoint{5.610976in}{0.656415in}}{\pgfqpoint{5.603076in}{0.653142in}}{\pgfqpoint{5.597252in}{0.647318in}}%
\pgfpathcurveto{\pgfqpoint{5.591428in}{0.641495in}}{\pgfqpoint{5.588156in}{0.633594in}}{\pgfqpoint{5.588156in}{0.625358in}}%
\pgfpathcurveto{\pgfqpoint{5.588156in}{0.617122in}}{\pgfqpoint{5.591428in}{0.609222in}}{\pgfqpoint{5.597252in}{0.603398in}}%
\pgfpathcurveto{\pgfqpoint{5.603076in}{0.597574in}}{\pgfqpoint{5.610976in}{0.594302in}}{\pgfqpoint{5.619213in}{0.594302in}}%
\pgfpathclose%
\pgfusepath{stroke,fill}%
\end{pgfscope}%
\begin{pgfscope}%
\pgfpathrectangle{\pgfqpoint{3.793912in}{0.557870in}}{\pgfqpoint{2.446088in}{1.484734in}}%
\pgfusepath{clip}%
\pgfsetbuttcap%
\pgfsetroundjoin%
\definecolor{currentfill}{rgb}{0.298039,0.447059,0.690196}%
\pgfsetfillcolor{currentfill}%
\pgfsetlinewidth{1.003750pt}%
\definecolor{currentstroke}{rgb}{0.298039,0.447059,0.690196}%
\pgfsetstrokecolor{currentstroke}%
\pgfsetdash{}{0pt}%
\pgfpathmoveto{\pgfqpoint{3.905098in}{1.357801in}}%
\pgfpathcurveto{\pgfqpoint{3.913334in}{1.357801in}}{\pgfqpoint{3.921234in}{1.361074in}}{\pgfqpoint{3.927058in}{1.366897in}}%
\pgfpathcurveto{\pgfqpoint{3.932882in}{1.372721in}}{\pgfqpoint{3.936155in}{1.380621in}}{\pgfqpoint{3.936155in}{1.388858in}}%
\pgfpathcurveto{\pgfqpoint{3.936155in}{1.397094in}}{\pgfqpoint{3.932882in}{1.404994in}}{\pgfqpoint{3.927058in}{1.410818in}}%
\pgfpathcurveto{\pgfqpoint{3.921234in}{1.416642in}}{\pgfqpoint{3.913334in}{1.419914in}}{\pgfqpoint{3.905098in}{1.419914in}}%
\pgfpathcurveto{\pgfqpoint{3.896862in}{1.419914in}}{\pgfqpoint{3.888962in}{1.416642in}}{\pgfqpoint{3.883138in}{1.410818in}}%
\pgfpathcurveto{\pgfqpoint{3.877314in}{1.404994in}}{\pgfqpoint{3.874042in}{1.397094in}}{\pgfqpoint{3.874042in}{1.388858in}}%
\pgfpathcurveto{\pgfqpoint{3.874042in}{1.380621in}}{\pgfqpoint{3.877314in}{1.372721in}}{\pgfqpoint{3.883138in}{1.366897in}}%
\pgfpathcurveto{\pgfqpoint{3.888962in}{1.361074in}}{\pgfqpoint{3.896862in}{1.357801in}}{\pgfqpoint{3.905098in}{1.357801in}}%
\pgfpathclose%
\pgfusepath{stroke,fill}%
\end{pgfscope}%
\begin{pgfscope}%
\pgfpathrectangle{\pgfqpoint{3.793912in}{0.557870in}}{\pgfqpoint{2.446088in}{1.484734in}}%
\pgfusepath{clip}%
\pgfsetbuttcap%
\pgfsetroundjoin%
\definecolor{currentfill}{rgb}{0.298039,0.447059,0.690196}%
\pgfsetfillcolor{currentfill}%
\pgfsetlinewidth{1.003750pt}%
\definecolor{currentstroke}{rgb}{0.298039,0.447059,0.690196}%
\pgfsetstrokecolor{currentstroke}%
\pgfsetdash{}{0pt}%
\pgfpathmoveto{\pgfqpoint{3.905098in}{1.903158in}}%
\pgfpathcurveto{\pgfqpoint{3.913334in}{1.903158in}}{\pgfqpoint{3.921234in}{1.906430in}}{\pgfqpoint{3.927058in}{1.912254in}}%
\pgfpathcurveto{\pgfqpoint{3.932882in}{1.918078in}}{\pgfqpoint{3.936155in}{1.925978in}}{\pgfqpoint{3.936155in}{1.934215in}}%
\pgfpathcurveto{\pgfqpoint{3.936155in}{1.942451in}}{\pgfqpoint{3.932882in}{1.950351in}}{\pgfqpoint{3.927058in}{1.956175in}}%
\pgfpathcurveto{\pgfqpoint{3.921234in}{1.961999in}}{\pgfqpoint{3.913334in}{1.965271in}}{\pgfqpoint{3.905098in}{1.965271in}}%
\pgfpathcurveto{\pgfqpoint{3.896862in}{1.965271in}}{\pgfqpoint{3.888962in}{1.961999in}}{\pgfqpoint{3.883138in}{1.956175in}}%
\pgfpathcurveto{\pgfqpoint{3.877314in}{1.950351in}}{\pgfqpoint{3.874042in}{1.942451in}}{\pgfqpoint{3.874042in}{1.934215in}}%
\pgfpathcurveto{\pgfqpoint{3.874042in}{1.925978in}}{\pgfqpoint{3.877314in}{1.918078in}}{\pgfqpoint{3.883138in}{1.912254in}}%
\pgfpathcurveto{\pgfqpoint{3.888962in}{1.906430in}}{\pgfqpoint{3.896862in}{1.903158in}}{\pgfqpoint{3.905098in}{1.903158in}}%
\pgfpathclose%
\pgfusepath{stroke,fill}%
\end{pgfscope}%
\begin{pgfscope}%
\pgfpathrectangle{\pgfqpoint{3.793912in}{0.557870in}}{\pgfqpoint{2.446088in}{1.484734in}}%
\pgfusepath{clip}%
\pgfsetbuttcap%
\pgfsetroundjoin%
\definecolor{currentfill}{rgb}{0.298039,0.447059,0.690196}%
\pgfsetfillcolor{currentfill}%
\pgfsetlinewidth{1.003750pt}%
\definecolor{currentstroke}{rgb}{0.298039,0.447059,0.690196}%
\pgfsetstrokecolor{currentstroke}%
\pgfsetdash{}{0pt}%
\pgfpathmoveto{\pgfqpoint{4.970629in}{0.594302in}}%
\pgfpathcurveto{\pgfqpoint{4.978865in}{0.594302in}}{\pgfqpoint{4.986765in}{0.597574in}}{\pgfqpoint{4.992589in}{0.603398in}}%
\pgfpathcurveto{\pgfqpoint{4.998413in}{0.609222in}}{\pgfqpoint{5.001685in}{0.617122in}}{\pgfqpoint{5.001685in}{0.625358in}}%
\pgfpathcurveto{\pgfqpoint{5.001685in}{0.633594in}}{\pgfqpoint{4.998413in}{0.641495in}}{\pgfqpoint{4.992589in}{0.647318in}}%
\pgfpathcurveto{\pgfqpoint{4.986765in}{0.653142in}}{\pgfqpoint{4.978865in}{0.656415in}}{\pgfqpoint{4.970629in}{0.656415in}}%
\pgfpathcurveto{\pgfqpoint{4.962392in}{0.656415in}}{\pgfqpoint{4.954492in}{0.653142in}}{\pgfqpoint{4.948668in}{0.647318in}}%
\pgfpathcurveto{\pgfqpoint{4.942845in}{0.641495in}}{\pgfqpoint{4.939572in}{0.633594in}}{\pgfqpoint{4.939572in}{0.625358in}}%
\pgfpathcurveto{\pgfqpoint{4.939572in}{0.617122in}}{\pgfqpoint{4.942845in}{0.609222in}}{\pgfqpoint{4.948668in}{0.603398in}}%
\pgfpathcurveto{\pgfqpoint{4.954492in}{0.597574in}}{\pgfqpoint{4.962392in}{0.594302in}}{\pgfqpoint{4.970629in}{0.594302in}}%
\pgfpathclose%
\pgfusepath{stroke,fill}%
\end{pgfscope}%
\begin{pgfscope}%
\pgfpathrectangle{\pgfqpoint{3.793912in}{0.557870in}}{\pgfqpoint{2.446088in}{1.484734in}}%
\pgfusepath{clip}%
\pgfsetbuttcap%
\pgfsetroundjoin%
\definecolor{currentfill}{rgb}{0.298039,0.447059,0.690196}%
\pgfsetfillcolor{currentfill}%
\pgfsetlinewidth{1.003750pt}%
\definecolor{currentstroke}{rgb}{0.298039,0.447059,0.690196}%
\pgfsetstrokecolor{currentstroke}%
\pgfsetdash{}{0pt}%
\pgfpathmoveto{\pgfqpoint{3.905098in}{1.930426in}}%
\pgfpathcurveto{\pgfqpoint{3.913334in}{1.930426in}}{\pgfqpoint{3.921234in}{1.933698in}}{\pgfqpoint{3.927058in}{1.939522in}}%
\pgfpathcurveto{\pgfqpoint{3.932882in}{1.945346in}}{\pgfqpoint{3.936155in}{1.953246in}}{\pgfqpoint{3.936155in}{1.961482in}}%
\pgfpathcurveto{\pgfqpoint{3.936155in}{1.969719in}}{\pgfqpoint{3.932882in}{1.977619in}}{\pgfqpoint{3.927058in}{1.983443in}}%
\pgfpathcurveto{\pgfqpoint{3.921234in}{1.989267in}}{\pgfqpoint{3.913334in}{1.992539in}}{\pgfqpoint{3.905098in}{1.992539in}}%
\pgfpathcurveto{\pgfqpoint{3.896862in}{1.992539in}}{\pgfqpoint{3.888962in}{1.989267in}}{\pgfqpoint{3.883138in}{1.983443in}}%
\pgfpathcurveto{\pgfqpoint{3.877314in}{1.977619in}}{\pgfqpoint{3.874042in}{1.969719in}}{\pgfqpoint{3.874042in}{1.961482in}}%
\pgfpathcurveto{\pgfqpoint{3.874042in}{1.953246in}}{\pgfqpoint{3.877314in}{1.945346in}}{\pgfqpoint{3.883138in}{1.939522in}}%
\pgfpathcurveto{\pgfqpoint{3.888962in}{1.933698in}}{\pgfqpoint{3.896862in}{1.930426in}}{\pgfqpoint{3.905098in}{1.930426in}}%
\pgfpathclose%
\pgfusepath{stroke,fill}%
\end{pgfscope}%
\begin{pgfscope}%
\pgfpathrectangle{\pgfqpoint{3.793912in}{0.557870in}}{\pgfqpoint{2.446088in}{1.484734in}}%
\pgfusepath{clip}%
\pgfsetbuttcap%
\pgfsetroundjoin%
\definecolor{currentfill}{rgb}{0.298039,0.447059,0.690196}%
\pgfsetfillcolor{currentfill}%
\pgfsetlinewidth{1.003750pt}%
\definecolor{currentstroke}{rgb}{0.298039,0.447059,0.690196}%
\pgfsetstrokecolor{currentstroke}%
\pgfsetdash{}{0pt}%
\pgfpathmoveto{\pgfqpoint{5.457067in}{0.594302in}}%
\pgfpathcurveto{\pgfqpoint{5.465303in}{0.594302in}}{\pgfqpoint{5.473203in}{0.597574in}}{\pgfqpoint{5.479027in}{0.603398in}}%
\pgfpathcurveto{\pgfqpoint{5.484851in}{0.609222in}}{\pgfqpoint{5.488123in}{0.617122in}}{\pgfqpoint{5.488123in}{0.625358in}}%
\pgfpathcurveto{\pgfqpoint{5.488123in}{0.633594in}}{\pgfqpoint{5.484851in}{0.641495in}}{\pgfqpoint{5.479027in}{0.647318in}}%
\pgfpathcurveto{\pgfqpoint{5.473203in}{0.653142in}}{\pgfqpoint{5.465303in}{0.656415in}}{\pgfqpoint{5.457067in}{0.656415in}}%
\pgfpathcurveto{\pgfqpoint{5.448830in}{0.656415in}}{\pgfqpoint{5.440930in}{0.653142in}}{\pgfqpoint{5.435106in}{0.647318in}}%
\pgfpathcurveto{\pgfqpoint{5.429282in}{0.641495in}}{\pgfqpoint{5.426010in}{0.633594in}}{\pgfqpoint{5.426010in}{0.625358in}}%
\pgfpathcurveto{\pgfqpoint{5.426010in}{0.617122in}}{\pgfqpoint{5.429282in}{0.609222in}}{\pgfqpoint{5.435106in}{0.603398in}}%
\pgfpathcurveto{\pgfqpoint{5.440930in}{0.597574in}}{\pgfqpoint{5.448830in}{0.594302in}}{\pgfqpoint{5.457067in}{0.594302in}}%
\pgfpathclose%
\pgfusepath{stroke,fill}%
\end{pgfscope}%
\begin{pgfscope}%
\pgfpathrectangle{\pgfqpoint{3.793912in}{0.557870in}}{\pgfqpoint{2.446088in}{1.484734in}}%
\pgfusepath{clip}%
\pgfsetbuttcap%
\pgfsetroundjoin%
\definecolor{currentfill}{rgb}{0.298039,0.447059,0.690196}%
\pgfsetfillcolor{currentfill}%
\pgfsetlinewidth{1.003750pt}%
\definecolor{currentstroke}{rgb}{0.298039,0.447059,0.690196}%
\pgfsetstrokecolor{currentstroke}%
\pgfsetdash{}{0pt}%
\pgfpathmoveto{\pgfqpoint{3.905098in}{1.930426in}}%
\pgfpathcurveto{\pgfqpoint{3.913334in}{1.930426in}}{\pgfqpoint{3.921234in}{1.933698in}}{\pgfqpoint{3.927058in}{1.939522in}}%
\pgfpathcurveto{\pgfqpoint{3.932882in}{1.945346in}}{\pgfqpoint{3.936155in}{1.953246in}}{\pgfqpoint{3.936155in}{1.961482in}}%
\pgfpathcurveto{\pgfqpoint{3.936155in}{1.969719in}}{\pgfqpoint{3.932882in}{1.977619in}}{\pgfqpoint{3.927058in}{1.983443in}}%
\pgfpathcurveto{\pgfqpoint{3.921234in}{1.989267in}}{\pgfqpoint{3.913334in}{1.992539in}}{\pgfqpoint{3.905098in}{1.992539in}}%
\pgfpathcurveto{\pgfqpoint{3.896862in}{1.992539in}}{\pgfqpoint{3.888962in}{1.989267in}}{\pgfqpoint{3.883138in}{1.983443in}}%
\pgfpathcurveto{\pgfqpoint{3.877314in}{1.977619in}}{\pgfqpoint{3.874042in}{1.969719in}}{\pgfqpoint{3.874042in}{1.961482in}}%
\pgfpathcurveto{\pgfqpoint{3.874042in}{1.953246in}}{\pgfqpoint{3.877314in}{1.945346in}}{\pgfqpoint{3.883138in}{1.939522in}}%
\pgfpathcurveto{\pgfqpoint{3.888962in}{1.933698in}}{\pgfqpoint{3.896862in}{1.930426in}}{\pgfqpoint{3.905098in}{1.930426in}}%
\pgfpathclose%
\pgfusepath{stroke,fill}%
\end{pgfscope}%
\begin{pgfscope}%
\pgfpathrectangle{\pgfqpoint{3.793912in}{0.557870in}}{\pgfqpoint{2.446088in}{1.484734in}}%
\pgfusepath{clip}%
\pgfsetbuttcap%
\pgfsetroundjoin%
\definecolor{currentfill}{rgb}{0.298039,0.447059,0.690196}%
\pgfsetfillcolor{currentfill}%
\pgfsetlinewidth{1.003750pt}%
\definecolor{currentstroke}{rgb}{0.298039,0.447059,0.690196}%
\pgfsetstrokecolor{currentstroke}%
\pgfsetdash{}{0pt}%
\pgfpathmoveto{\pgfqpoint{5.318084in}{0.594302in}}%
\pgfpathcurveto{\pgfqpoint{5.326321in}{0.594302in}}{\pgfqpoint{5.334221in}{0.597574in}}{\pgfqpoint{5.340045in}{0.603398in}}%
\pgfpathcurveto{\pgfqpoint{5.345869in}{0.609222in}}{\pgfqpoint{5.349141in}{0.617122in}}{\pgfqpoint{5.349141in}{0.625358in}}%
\pgfpathcurveto{\pgfqpoint{5.349141in}{0.633594in}}{\pgfqpoint{5.345869in}{0.641495in}}{\pgfqpoint{5.340045in}{0.647318in}}%
\pgfpathcurveto{\pgfqpoint{5.334221in}{0.653142in}}{\pgfqpoint{5.326321in}{0.656415in}}{\pgfqpoint{5.318084in}{0.656415in}}%
\pgfpathcurveto{\pgfqpoint{5.309848in}{0.656415in}}{\pgfqpoint{5.301948in}{0.653142in}}{\pgfqpoint{5.296124in}{0.647318in}}%
\pgfpathcurveto{\pgfqpoint{5.290300in}{0.641495in}}{\pgfqpoint{5.287028in}{0.633594in}}{\pgfqpoint{5.287028in}{0.625358in}}%
\pgfpathcurveto{\pgfqpoint{5.287028in}{0.617122in}}{\pgfqpoint{5.290300in}{0.609222in}}{\pgfqpoint{5.296124in}{0.603398in}}%
\pgfpathcurveto{\pgfqpoint{5.301948in}{0.597574in}}{\pgfqpoint{5.309848in}{0.594302in}}{\pgfqpoint{5.318084in}{0.594302in}}%
\pgfpathclose%
\pgfusepath{stroke,fill}%
\end{pgfscope}%
\begin{pgfscope}%
\pgfpathrectangle{\pgfqpoint{3.793912in}{0.557870in}}{\pgfqpoint{2.446088in}{1.484734in}}%
\pgfusepath{clip}%
\pgfsetbuttcap%
\pgfsetroundjoin%
\definecolor{currentfill}{rgb}{0.298039,0.447059,0.690196}%
\pgfsetfillcolor{currentfill}%
\pgfsetlinewidth{1.003750pt}%
\definecolor{currentstroke}{rgb}{0.298039,0.447059,0.690196}%
\pgfsetstrokecolor{currentstroke}%
\pgfsetdash{}{0pt}%
\pgfpathmoveto{\pgfqpoint{3.905098in}{1.930426in}}%
\pgfpathcurveto{\pgfqpoint{3.913334in}{1.930426in}}{\pgfqpoint{3.921234in}{1.933698in}}{\pgfqpoint{3.927058in}{1.939522in}}%
\pgfpathcurveto{\pgfqpoint{3.932882in}{1.945346in}}{\pgfqpoint{3.936155in}{1.953246in}}{\pgfqpoint{3.936155in}{1.961482in}}%
\pgfpathcurveto{\pgfqpoint{3.936155in}{1.969719in}}{\pgfqpoint{3.932882in}{1.977619in}}{\pgfqpoint{3.927058in}{1.983443in}}%
\pgfpathcurveto{\pgfqpoint{3.921234in}{1.989267in}}{\pgfqpoint{3.913334in}{1.992539in}}{\pgfqpoint{3.905098in}{1.992539in}}%
\pgfpathcurveto{\pgfqpoint{3.896862in}{1.992539in}}{\pgfqpoint{3.888962in}{1.989267in}}{\pgfqpoint{3.883138in}{1.983443in}}%
\pgfpathcurveto{\pgfqpoint{3.877314in}{1.977619in}}{\pgfqpoint{3.874042in}{1.969719in}}{\pgfqpoint{3.874042in}{1.961482in}}%
\pgfpathcurveto{\pgfqpoint{3.874042in}{1.953246in}}{\pgfqpoint{3.877314in}{1.945346in}}{\pgfqpoint{3.883138in}{1.939522in}}%
\pgfpathcurveto{\pgfqpoint{3.888962in}{1.933698in}}{\pgfqpoint{3.896862in}{1.930426in}}{\pgfqpoint{3.905098in}{1.930426in}}%
\pgfpathclose%
\pgfusepath{stroke,fill}%
\end{pgfscope}%
\begin{pgfscope}%
\pgfpathrectangle{\pgfqpoint{3.793912in}{0.557870in}}{\pgfqpoint{2.446088in}{1.484734in}}%
\pgfusepath{clip}%
\pgfsetbuttcap%
\pgfsetroundjoin%
\definecolor{currentfill}{rgb}{0.298039,0.447059,0.690196}%
\pgfsetfillcolor{currentfill}%
\pgfsetlinewidth{1.003750pt}%
\definecolor{currentstroke}{rgb}{0.298039,0.447059,0.690196}%
\pgfsetstrokecolor{currentstroke}%
\pgfsetdash{}{0pt}%
\pgfpathmoveto{\pgfqpoint{3.905098in}{1.930426in}}%
\pgfpathcurveto{\pgfqpoint{3.913334in}{1.930426in}}{\pgfqpoint{3.921234in}{1.933698in}}{\pgfqpoint{3.927058in}{1.939522in}}%
\pgfpathcurveto{\pgfqpoint{3.932882in}{1.945346in}}{\pgfqpoint{3.936155in}{1.953246in}}{\pgfqpoint{3.936155in}{1.961482in}}%
\pgfpathcurveto{\pgfqpoint{3.936155in}{1.969719in}}{\pgfqpoint{3.932882in}{1.977619in}}{\pgfqpoint{3.927058in}{1.983443in}}%
\pgfpathcurveto{\pgfqpoint{3.921234in}{1.989267in}}{\pgfqpoint{3.913334in}{1.992539in}}{\pgfqpoint{3.905098in}{1.992539in}}%
\pgfpathcurveto{\pgfqpoint{3.896862in}{1.992539in}}{\pgfqpoint{3.888962in}{1.989267in}}{\pgfqpoint{3.883138in}{1.983443in}}%
\pgfpathcurveto{\pgfqpoint{3.877314in}{1.977619in}}{\pgfqpoint{3.874042in}{1.969719in}}{\pgfqpoint{3.874042in}{1.961482in}}%
\pgfpathcurveto{\pgfqpoint{3.874042in}{1.953246in}}{\pgfqpoint{3.877314in}{1.945346in}}{\pgfqpoint{3.883138in}{1.939522in}}%
\pgfpathcurveto{\pgfqpoint{3.888962in}{1.933698in}}{\pgfqpoint{3.896862in}{1.930426in}}{\pgfqpoint{3.905098in}{1.930426in}}%
\pgfpathclose%
\pgfusepath{stroke,fill}%
\end{pgfscope}%
\begin{pgfscope}%
\pgfpathrectangle{\pgfqpoint{3.793912in}{0.557870in}}{\pgfqpoint{2.446088in}{1.484734in}}%
\pgfusepath{clip}%
\pgfsetbuttcap%
\pgfsetroundjoin%
\definecolor{currentfill}{rgb}{0.298039,0.447059,0.690196}%
\pgfsetfillcolor{currentfill}%
\pgfsetlinewidth{1.003750pt}%
\definecolor{currentstroke}{rgb}{0.298039,0.447059,0.690196}%
\pgfsetstrokecolor{currentstroke}%
\pgfsetdash{}{0pt}%
\pgfpathmoveto{\pgfqpoint{3.905098in}{1.930426in}}%
\pgfpathcurveto{\pgfqpoint{3.913334in}{1.930426in}}{\pgfqpoint{3.921234in}{1.933698in}}{\pgfqpoint{3.927058in}{1.939522in}}%
\pgfpathcurveto{\pgfqpoint{3.932882in}{1.945346in}}{\pgfqpoint{3.936155in}{1.953246in}}{\pgfqpoint{3.936155in}{1.961482in}}%
\pgfpathcurveto{\pgfqpoint{3.936155in}{1.969719in}}{\pgfqpoint{3.932882in}{1.977619in}}{\pgfqpoint{3.927058in}{1.983443in}}%
\pgfpathcurveto{\pgfqpoint{3.921234in}{1.989267in}}{\pgfqpoint{3.913334in}{1.992539in}}{\pgfqpoint{3.905098in}{1.992539in}}%
\pgfpathcurveto{\pgfqpoint{3.896862in}{1.992539in}}{\pgfqpoint{3.888962in}{1.989267in}}{\pgfqpoint{3.883138in}{1.983443in}}%
\pgfpathcurveto{\pgfqpoint{3.877314in}{1.977619in}}{\pgfqpoint{3.874042in}{1.969719in}}{\pgfqpoint{3.874042in}{1.961482in}}%
\pgfpathcurveto{\pgfqpoint{3.874042in}{1.953246in}}{\pgfqpoint{3.877314in}{1.945346in}}{\pgfqpoint{3.883138in}{1.939522in}}%
\pgfpathcurveto{\pgfqpoint{3.888962in}{1.933698in}}{\pgfqpoint{3.896862in}{1.930426in}}{\pgfqpoint{3.905098in}{1.930426in}}%
\pgfpathclose%
\pgfusepath{stroke,fill}%
\end{pgfscope}%
\begin{pgfscope}%
\pgfpathrectangle{\pgfqpoint{3.793912in}{0.557870in}}{\pgfqpoint{2.446088in}{1.484734in}}%
\pgfusepath{clip}%
\pgfsetbuttcap%
\pgfsetroundjoin%
\definecolor{currentfill}{rgb}{0.298039,0.447059,0.690196}%
\pgfsetfillcolor{currentfill}%
\pgfsetlinewidth{1.003750pt}%
\definecolor{currentstroke}{rgb}{0.298039,0.447059,0.690196}%
\pgfsetstrokecolor{currentstroke}%
\pgfsetdash{}{0pt}%
\pgfpathmoveto{\pgfqpoint{3.905098in}{1.930426in}}%
\pgfpathcurveto{\pgfqpoint{3.913334in}{1.930426in}}{\pgfqpoint{3.921234in}{1.933698in}}{\pgfqpoint{3.927058in}{1.939522in}}%
\pgfpathcurveto{\pgfqpoint{3.932882in}{1.945346in}}{\pgfqpoint{3.936155in}{1.953246in}}{\pgfqpoint{3.936155in}{1.961482in}}%
\pgfpathcurveto{\pgfqpoint{3.936155in}{1.969719in}}{\pgfqpoint{3.932882in}{1.977619in}}{\pgfqpoint{3.927058in}{1.983443in}}%
\pgfpathcurveto{\pgfqpoint{3.921234in}{1.989267in}}{\pgfqpoint{3.913334in}{1.992539in}}{\pgfqpoint{3.905098in}{1.992539in}}%
\pgfpathcurveto{\pgfqpoint{3.896862in}{1.992539in}}{\pgfqpoint{3.888962in}{1.989267in}}{\pgfqpoint{3.883138in}{1.983443in}}%
\pgfpathcurveto{\pgfqpoint{3.877314in}{1.977619in}}{\pgfqpoint{3.874042in}{1.969719in}}{\pgfqpoint{3.874042in}{1.961482in}}%
\pgfpathcurveto{\pgfqpoint{3.874042in}{1.953246in}}{\pgfqpoint{3.877314in}{1.945346in}}{\pgfqpoint{3.883138in}{1.939522in}}%
\pgfpathcurveto{\pgfqpoint{3.888962in}{1.933698in}}{\pgfqpoint{3.896862in}{1.930426in}}{\pgfqpoint{3.905098in}{1.930426in}}%
\pgfpathclose%
\pgfusepath{stroke,fill}%
\end{pgfscope}%
\begin{pgfscope}%
\pgfpathrectangle{\pgfqpoint{3.793912in}{0.557870in}}{\pgfqpoint{2.446088in}{1.484734in}}%
\pgfusepath{clip}%
\pgfsetbuttcap%
\pgfsetroundjoin%
\definecolor{currentfill}{rgb}{0.298039,0.447059,0.690196}%
\pgfsetfillcolor{currentfill}%
\pgfsetlinewidth{1.003750pt}%
\definecolor{currentstroke}{rgb}{0.298039,0.447059,0.690196}%
\pgfsetstrokecolor{currentstroke}%
\pgfsetdash{}{0pt}%
\pgfpathmoveto{\pgfqpoint{4.646337in}{0.594302in}}%
\pgfpathcurveto{\pgfqpoint{4.654573in}{0.594302in}}{\pgfqpoint{4.662473in}{0.597574in}}{\pgfqpoint{4.668297in}{0.603398in}}%
\pgfpathcurveto{\pgfqpoint{4.674121in}{0.609222in}}{\pgfqpoint{4.677393in}{0.617122in}}{\pgfqpoint{4.677393in}{0.625358in}}%
\pgfpathcurveto{\pgfqpoint{4.677393in}{0.633594in}}{\pgfqpoint{4.674121in}{0.641495in}}{\pgfqpoint{4.668297in}{0.647318in}}%
\pgfpathcurveto{\pgfqpoint{4.662473in}{0.653142in}}{\pgfqpoint{4.654573in}{0.656415in}}{\pgfqpoint{4.646337in}{0.656415in}}%
\pgfpathcurveto{\pgfqpoint{4.638100in}{0.656415in}}{\pgfqpoint{4.630200in}{0.653142in}}{\pgfqpoint{4.624376in}{0.647318in}}%
\pgfpathcurveto{\pgfqpoint{4.618553in}{0.641495in}}{\pgfqpoint{4.615280in}{0.633594in}}{\pgfqpoint{4.615280in}{0.625358in}}%
\pgfpathcurveto{\pgfqpoint{4.615280in}{0.617122in}}{\pgfqpoint{4.618553in}{0.609222in}}{\pgfqpoint{4.624376in}{0.603398in}}%
\pgfpathcurveto{\pgfqpoint{4.630200in}{0.597574in}}{\pgfqpoint{4.638100in}{0.594302in}}{\pgfqpoint{4.646337in}{0.594302in}}%
\pgfpathclose%
\pgfusepath{stroke,fill}%
\end{pgfscope}%
\begin{pgfscope}%
\pgfpathrectangle{\pgfqpoint{3.793912in}{0.557870in}}{\pgfqpoint{2.446088in}{1.484734in}}%
\pgfusepath{clip}%
\pgfsetbuttcap%
\pgfsetroundjoin%
\definecolor{currentfill}{rgb}{0.298039,0.447059,0.690196}%
\pgfsetfillcolor{currentfill}%
\pgfsetlinewidth{1.003750pt}%
\definecolor{currentstroke}{rgb}{0.298039,0.447059,0.690196}%
\pgfsetstrokecolor{currentstroke}%
\pgfsetdash{}{0pt}%
\pgfpathmoveto{\pgfqpoint{3.905098in}{1.930426in}}%
\pgfpathcurveto{\pgfqpoint{3.913334in}{1.930426in}}{\pgfqpoint{3.921234in}{1.933698in}}{\pgfqpoint{3.927058in}{1.939522in}}%
\pgfpathcurveto{\pgfqpoint{3.932882in}{1.945346in}}{\pgfqpoint{3.936155in}{1.953246in}}{\pgfqpoint{3.936155in}{1.961482in}}%
\pgfpathcurveto{\pgfqpoint{3.936155in}{1.969719in}}{\pgfqpoint{3.932882in}{1.977619in}}{\pgfqpoint{3.927058in}{1.983443in}}%
\pgfpathcurveto{\pgfqpoint{3.921234in}{1.989267in}}{\pgfqpoint{3.913334in}{1.992539in}}{\pgfqpoint{3.905098in}{1.992539in}}%
\pgfpathcurveto{\pgfqpoint{3.896862in}{1.992539in}}{\pgfqpoint{3.888962in}{1.989267in}}{\pgfqpoint{3.883138in}{1.983443in}}%
\pgfpathcurveto{\pgfqpoint{3.877314in}{1.977619in}}{\pgfqpoint{3.874042in}{1.969719in}}{\pgfqpoint{3.874042in}{1.961482in}}%
\pgfpathcurveto{\pgfqpoint{3.874042in}{1.953246in}}{\pgfqpoint{3.877314in}{1.945346in}}{\pgfqpoint{3.883138in}{1.939522in}}%
\pgfpathcurveto{\pgfqpoint{3.888962in}{1.933698in}}{\pgfqpoint{3.896862in}{1.930426in}}{\pgfqpoint{3.905098in}{1.930426in}}%
\pgfpathclose%
\pgfusepath{stroke,fill}%
\end{pgfscope}%
\begin{pgfscope}%
\pgfpathrectangle{\pgfqpoint{3.793912in}{0.557870in}}{\pgfqpoint{2.446088in}{1.484734in}}%
\pgfusepath{clip}%
\pgfsetbuttcap%
\pgfsetroundjoin%
\definecolor{currentfill}{rgb}{0.298039,0.447059,0.690196}%
\pgfsetfillcolor{currentfill}%
\pgfsetlinewidth{1.003750pt}%
\definecolor{currentstroke}{rgb}{0.298039,0.447059,0.690196}%
\pgfsetstrokecolor{currentstroke}%
\pgfsetdash{}{0pt}%
\pgfpathmoveto{\pgfqpoint{4.785319in}{0.594302in}}%
\pgfpathcurveto{\pgfqpoint{4.793555in}{0.594302in}}{\pgfqpoint{4.801455in}{0.597574in}}{\pgfqpoint{4.807279in}{0.603398in}}%
\pgfpathcurveto{\pgfqpoint{4.813103in}{0.609222in}}{\pgfqpoint{4.816376in}{0.617122in}}{\pgfqpoint{4.816376in}{0.625358in}}%
\pgfpathcurveto{\pgfqpoint{4.816376in}{0.633594in}}{\pgfqpoint{4.813103in}{0.641495in}}{\pgfqpoint{4.807279in}{0.647318in}}%
\pgfpathcurveto{\pgfqpoint{4.801455in}{0.653142in}}{\pgfqpoint{4.793555in}{0.656415in}}{\pgfqpoint{4.785319in}{0.656415in}}%
\pgfpathcurveto{\pgfqpoint{4.777083in}{0.656415in}}{\pgfqpoint{4.769183in}{0.653142in}}{\pgfqpoint{4.763359in}{0.647318in}}%
\pgfpathcurveto{\pgfqpoint{4.757535in}{0.641495in}}{\pgfqpoint{4.754263in}{0.633594in}}{\pgfqpoint{4.754263in}{0.625358in}}%
\pgfpathcurveto{\pgfqpoint{4.754263in}{0.617122in}}{\pgfqpoint{4.757535in}{0.609222in}}{\pgfqpoint{4.763359in}{0.603398in}}%
\pgfpathcurveto{\pgfqpoint{4.769183in}{0.597574in}}{\pgfqpoint{4.777083in}{0.594302in}}{\pgfqpoint{4.785319in}{0.594302in}}%
\pgfpathclose%
\pgfusepath{stroke,fill}%
\end{pgfscope}%
\begin{pgfscope}%
\pgfpathrectangle{\pgfqpoint{3.793912in}{0.557870in}}{\pgfqpoint{2.446088in}{1.484734in}}%
\pgfusepath{clip}%
\pgfsetbuttcap%
\pgfsetroundjoin%
\definecolor{currentfill}{rgb}{0.298039,0.447059,0.690196}%
\pgfsetfillcolor{currentfill}%
\pgfsetlinewidth{1.003750pt}%
\definecolor{currentstroke}{rgb}{0.298039,0.447059,0.690196}%
\pgfsetstrokecolor{currentstroke}%
\pgfsetdash{}{0pt}%
\pgfpathmoveto{\pgfqpoint{3.905098in}{1.930426in}}%
\pgfpathcurveto{\pgfqpoint{3.913334in}{1.930426in}}{\pgfqpoint{3.921234in}{1.933698in}}{\pgfqpoint{3.927058in}{1.939522in}}%
\pgfpathcurveto{\pgfqpoint{3.932882in}{1.945346in}}{\pgfqpoint{3.936155in}{1.953246in}}{\pgfqpoint{3.936155in}{1.961482in}}%
\pgfpathcurveto{\pgfqpoint{3.936155in}{1.969719in}}{\pgfqpoint{3.932882in}{1.977619in}}{\pgfqpoint{3.927058in}{1.983443in}}%
\pgfpathcurveto{\pgfqpoint{3.921234in}{1.989267in}}{\pgfqpoint{3.913334in}{1.992539in}}{\pgfqpoint{3.905098in}{1.992539in}}%
\pgfpathcurveto{\pgfqpoint{3.896862in}{1.992539in}}{\pgfqpoint{3.888962in}{1.989267in}}{\pgfqpoint{3.883138in}{1.983443in}}%
\pgfpathcurveto{\pgfqpoint{3.877314in}{1.977619in}}{\pgfqpoint{3.874042in}{1.969719in}}{\pgfqpoint{3.874042in}{1.961482in}}%
\pgfpathcurveto{\pgfqpoint{3.874042in}{1.953246in}}{\pgfqpoint{3.877314in}{1.945346in}}{\pgfqpoint{3.883138in}{1.939522in}}%
\pgfpathcurveto{\pgfqpoint{3.888962in}{1.933698in}}{\pgfqpoint{3.896862in}{1.930426in}}{\pgfqpoint{3.905098in}{1.930426in}}%
\pgfpathclose%
\pgfusepath{stroke,fill}%
\end{pgfscope}%
\begin{pgfscope}%
\pgfpathrectangle{\pgfqpoint{3.793912in}{0.557870in}}{\pgfqpoint{2.446088in}{1.484734in}}%
\pgfusepath{clip}%
\pgfsetbuttcap%
\pgfsetroundjoin%
\definecolor{currentfill}{rgb}{0.298039,0.447059,0.690196}%
\pgfsetfillcolor{currentfill}%
\pgfsetlinewidth{1.003750pt}%
\definecolor{currentstroke}{rgb}{0.298039,0.447059,0.690196}%
\pgfsetstrokecolor{currentstroke}%
\pgfsetdash{}{0pt}%
\pgfpathmoveto{\pgfqpoint{3.905098in}{1.930426in}}%
\pgfpathcurveto{\pgfqpoint{3.913334in}{1.930426in}}{\pgfqpoint{3.921234in}{1.933698in}}{\pgfqpoint{3.927058in}{1.939522in}}%
\pgfpathcurveto{\pgfqpoint{3.932882in}{1.945346in}}{\pgfqpoint{3.936155in}{1.953246in}}{\pgfqpoint{3.936155in}{1.961482in}}%
\pgfpathcurveto{\pgfqpoint{3.936155in}{1.969719in}}{\pgfqpoint{3.932882in}{1.977619in}}{\pgfqpoint{3.927058in}{1.983443in}}%
\pgfpathcurveto{\pgfqpoint{3.921234in}{1.989267in}}{\pgfqpoint{3.913334in}{1.992539in}}{\pgfqpoint{3.905098in}{1.992539in}}%
\pgfpathcurveto{\pgfqpoint{3.896862in}{1.992539in}}{\pgfqpoint{3.888962in}{1.989267in}}{\pgfqpoint{3.883138in}{1.983443in}}%
\pgfpathcurveto{\pgfqpoint{3.877314in}{1.977619in}}{\pgfqpoint{3.874042in}{1.969719in}}{\pgfqpoint{3.874042in}{1.961482in}}%
\pgfpathcurveto{\pgfqpoint{3.874042in}{1.953246in}}{\pgfqpoint{3.877314in}{1.945346in}}{\pgfqpoint{3.883138in}{1.939522in}}%
\pgfpathcurveto{\pgfqpoint{3.888962in}{1.933698in}}{\pgfqpoint{3.896862in}{1.930426in}}{\pgfqpoint{3.905098in}{1.930426in}}%
\pgfpathclose%
\pgfusepath{stroke,fill}%
\end{pgfscope}%
\begin{pgfscope}%
\pgfpathrectangle{\pgfqpoint{3.793912in}{0.557870in}}{\pgfqpoint{2.446088in}{1.484734in}}%
\pgfusepath{clip}%
\pgfsetbuttcap%
\pgfsetroundjoin%
\definecolor{currentfill}{rgb}{0.298039,0.447059,0.690196}%
\pgfsetfillcolor{currentfill}%
\pgfsetlinewidth{1.003750pt}%
\definecolor{currentstroke}{rgb}{0.298039,0.447059,0.690196}%
\pgfsetstrokecolor{currentstroke}%
\pgfsetdash{}{0pt}%
\pgfpathmoveto{\pgfqpoint{3.905098in}{1.930426in}}%
\pgfpathcurveto{\pgfqpoint{3.913334in}{1.930426in}}{\pgfqpoint{3.921234in}{1.933698in}}{\pgfqpoint{3.927058in}{1.939522in}}%
\pgfpathcurveto{\pgfqpoint{3.932882in}{1.945346in}}{\pgfqpoint{3.936155in}{1.953246in}}{\pgfqpoint{3.936155in}{1.961482in}}%
\pgfpathcurveto{\pgfqpoint{3.936155in}{1.969719in}}{\pgfqpoint{3.932882in}{1.977619in}}{\pgfqpoint{3.927058in}{1.983443in}}%
\pgfpathcurveto{\pgfqpoint{3.921234in}{1.989267in}}{\pgfqpoint{3.913334in}{1.992539in}}{\pgfqpoint{3.905098in}{1.992539in}}%
\pgfpathcurveto{\pgfqpoint{3.896862in}{1.992539in}}{\pgfqpoint{3.888962in}{1.989267in}}{\pgfqpoint{3.883138in}{1.983443in}}%
\pgfpathcurveto{\pgfqpoint{3.877314in}{1.977619in}}{\pgfqpoint{3.874042in}{1.969719in}}{\pgfqpoint{3.874042in}{1.961482in}}%
\pgfpathcurveto{\pgfqpoint{3.874042in}{1.953246in}}{\pgfqpoint{3.877314in}{1.945346in}}{\pgfqpoint{3.883138in}{1.939522in}}%
\pgfpathcurveto{\pgfqpoint{3.888962in}{1.933698in}}{\pgfqpoint{3.896862in}{1.930426in}}{\pgfqpoint{3.905098in}{1.930426in}}%
\pgfpathclose%
\pgfusepath{stroke,fill}%
\end{pgfscope}%
\begin{pgfscope}%
\pgfpathrectangle{\pgfqpoint{3.793912in}{0.557870in}}{\pgfqpoint{2.446088in}{1.484734in}}%
\pgfusepath{clip}%
\pgfsetbuttcap%
\pgfsetroundjoin%
\definecolor{currentfill}{rgb}{0.298039,0.447059,0.690196}%
\pgfsetfillcolor{currentfill}%
\pgfsetlinewidth{1.003750pt}%
\definecolor{currentstroke}{rgb}{0.298039,0.447059,0.690196}%
\pgfsetstrokecolor{currentstroke}%
\pgfsetdash{}{0pt}%
\pgfpathmoveto{\pgfqpoint{3.951425in}{0.594302in}}%
\pgfpathcurveto{\pgfqpoint{3.959662in}{0.594302in}}{\pgfqpoint{3.967562in}{0.597574in}}{\pgfqpoint{3.973386in}{0.603398in}}%
\pgfpathcurveto{\pgfqpoint{3.979210in}{0.609222in}}{\pgfqpoint{3.982482in}{0.617122in}}{\pgfqpoint{3.982482in}{0.625358in}}%
\pgfpathcurveto{\pgfqpoint{3.982482in}{0.633594in}}{\pgfqpoint{3.979210in}{0.641495in}}{\pgfqpoint{3.973386in}{0.647318in}}%
\pgfpathcurveto{\pgfqpoint{3.967562in}{0.653142in}}{\pgfqpoint{3.959662in}{0.656415in}}{\pgfqpoint{3.951425in}{0.656415in}}%
\pgfpathcurveto{\pgfqpoint{3.943189in}{0.656415in}}{\pgfqpoint{3.935289in}{0.653142in}}{\pgfqpoint{3.929465in}{0.647318in}}%
\pgfpathcurveto{\pgfqpoint{3.923641in}{0.641495in}}{\pgfqpoint{3.920369in}{0.633594in}}{\pgfqpoint{3.920369in}{0.625358in}}%
\pgfpathcurveto{\pgfqpoint{3.920369in}{0.617122in}}{\pgfqpoint{3.923641in}{0.609222in}}{\pgfqpoint{3.929465in}{0.603398in}}%
\pgfpathcurveto{\pgfqpoint{3.935289in}{0.597574in}}{\pgfqpoint{3.943189in}{0.594302in}}{\pgfqpoint{3.951425in}{0.594302in}}%
\pgfpathclose%
\pgfusepath{stroke,fill}%
\end{pgfscope}%
\begin{pgfscope}%
\pgfpathrectangle{\pgfqpoint{3.793912in}{0.557870in}}{\pgfqpoint{2.446088in}{1.484734in}}%
\pgfusepath{clip}%
\pgfsetbuttcap%
\pgfsetroundjoin%
\definecolor{currentfill}{rgb}{0.298039,0.447059,0.690196}%
\pgfsetfillcolor{currentfill}%
\pgfsetlinewidth{1.003750pt}%
\definecolor{currentstroke}{rgb}{0.298039,0.447059,0.690196}%
\pgfsetstrokecolor{currentstroke}%
\pgfsetdash{}{0pt}%
\pgfpathmoveto{\pgfqpoint{3.905098in}{1.930426in}}%
\pgfpathcurveto{\pgfqpoint{3.913334in}{1.930426in}}{\pgfqpoint{3.921234in}{1.933698in}}{\pgfqpoint{3.927058in}{1.939522in}}%
\pgfpathcurveto{\pgfqpoint{3.932882in}{1.945346in}}{\pgfqpoint{3.936155in}{1.953246in}}{\pgfqpoint{3.936155in}{1.961482in}}%
\pgfpathcurveto{\pgfqpoint{3.936155in}{1.969719in}}{\pgfqpoint{3.932882in}{1.977619in}}{\pgfqpoint{3.927058in}{1.983443in}}%
\pgfpathcurveto{\pgfqpoint{3.921234in}{1.989267in}}{\pgfqpoint{3.913334in}{1.992539in}}{\pgfqpoint{3.905098in}{1.992539in}}%
\pgfpathcurveto{\pgfqpoint{3.896862in}{1.992539in}}{\pgfqpoint{3.888962in}{1.989267in}}{\pgfqpoint{3.883138in}{1.983443in}}%
\pgfpathcurveto{\pgfqpoint{3.877314in}{1.977619in}}{\pgfqpoint{3.874042in}{1.969719in}}{\pgfqpoint{3.874042in}{1.961482in}}%
\pgfpathcurveto{\pgfqpoint{3.874042in}{1.953246in}}{\pgfqpoint{3.877314in}{1.945346in}}{\pgfqpoint{3.883138in}{1.939522in}}%
\pgfpathcurveto{\pgfqpoint{3.888962in}{1.933698in}}{\pgfqpoint{3.896862in}{1.930426in}}{\pgfqpoint{3.905098in}{1.930426in}}%
\pgfpathclose%
\pgfusepath{stroke,fill}%
\end{pgfscope}%
\begin{pgfscope}%
\pgfpathrectangle{\pgfqpoint{3.793912in}{0.557870in}}{\pgfqpoint{2.446088in}{1.484734in}}%
\pgfusepath{clip}%
\pgfsetbuttcap%
\pgfsetroundjoin%
\definecolor{currentfill}{rgb}{0.298039,0.447059,0.690196}%
\pgfsetfillcolor{currentfill}%
\pgfsetlinewidth{1.003750pt}%
\definecolor{currentstroke}{rgb}{0.298039,0.447059,0.690196}%
\pgfsetstrokecolor{currentstroke}%
\pgfsetdash{}{0pt}%
\pgfpathmoveto{\pgfqpoint{4.738992in}{0.594302in}}%
\pgfpathcurveto{\pgfqpoint{4.747228in}{0.594302in}}{\pgfqpoint{4.755128in}{0.597574in}}{\pgfqpoint{4.760952in}{0.603398in}}%
\pgfpathcurveto{\pgfqpoint{4.766776in}{0.609222in}}{\pgfqpoint{4.770048in}{0.617122in}}{\pgfqpoint{4.770048in}{0.625358in}}%
\pgfpathcurveto{\pgfqpoint{4.770048in}{0.633594in}}{\pgfqpoint{4.766776in}{0.641495in}}{\pgfqpoint{4.760952in}{0.647318in}}%
\pgfpathcurveto{\pgfqpoint{4.755128in}{0.653142in}}{\pgfqpoint{4.747228in}{0.656415in}}{\pgfqpoint{4.738992in}{0.656415in}}%
\pgfpathcurveto{\pgfqpoint{4.730755in}{0.656415in}}{\pgfqpoint{4.722855in}{0.653142in}}{\pgfqpoint{4.717031in}{0.647318in}}%
\pgfpathcurveto{\pgfqpoint{4.711207in}{0.641495in}}{\pgfqpoint{4.707935in}{0.633594in}}{\pgfqpoint{4.707935in}{0.625358in}}%
\pgfpathcurveto{\pgfqpoint{4.707935in}{0.617122in}}{\pgfqpoint{4.711207in}{0.609222in}}{\pgfqpoint{4.717031in}{0.603398in}}%
\pgfpathcurveto{\pgfqpoint{4.722855in}{0.597574in}}{\pgfqpoint{4.730755in}{0.594302in}}{\pgfqpoint{4.738992in}{0.594302in}}%
\pgfpathclose%
\pgfusepath{stroke,fill}%
\end{pgfscope}%
\begin{pgfscope}%
\pgfpathrectangle{\pgfqpoint{3.793912in}{0.557870in}}{\pgfqpoint{2.446088in}{1.484734in}}%
\pgfusepath{clip}%
\pgfsetbuttcap%
\pgfsetroundjoin%
\definecolor{currentfill}{rgb}{0.298039,0.447059,0.690196}%
\pgfsetfillcolor{currentfill}%
\pgfsetlinewidth{1.003750pt}%
\definecolor{currentstroke}{rgb}{0.298039,0.447059,0.690196}%
\pgfsetstrokecolor{currentstroke}%
\pgfsetdash{}{0pt}%
\pgfpathmoveto{\pgfqpoint{3.905098in}{1.930426in}}%
\pgfpathcurveto{\pgfqpoint{3.913334in}{1.930426in}}{\pgfqpoint{3.921234in}{1.933698in}}{\pgfqpoint{3.927058in}{1.939522in}}%
\pgfpathcurveto{\pgfqpoint{3.932882in}{1.945346in}}{\pgfqpoint{3.936155in}{1.953246in}}{\pgfqpoint{3.936155in}{1.961482in}}%
\pgfpathcurveto{\pgfqpoint{3.936155in}{1.969719in}}{\pgfqpoint{3.932882in}{1.977619in}}{\pgfqpoint{3.927058in}{1.983443in}}%
\pgfpathcurveto{\pgfqpoint{3.921234in}{1.989267in}}{\pgfqpoint{3.913334in}{1.992539in}}{\pgfqpoint{3.905098in}{1.992539in}}%
\pgfpathcurveto{\pgfqpoint{3.896862in}{1.992539in}}{\pgfqpoint{3.888962in}{1.989267in}}{\pgfqpoint{3.883138in}{1.983443in}}%
\pgfpathcurveto{\pgfqpoint{3.877314in}{1.977619in}}{\pgfqpoint{3.874042in}{1.969719in}}{\pgfqpoint{3.874042in}{1.961482in}}%
\pgfpathcurveto{\pgfqpoint{3.874042in}{1.953246in}}{\pgfqpoint{3.877314in}{1.945346in}}{\pgfqpoint{3.883138in}{1.939522in}}%
\pgfpathcurveto{\pgfqpoint{3.888962in}{1.933698in}}{\pgfqpoint{3.896862in}{1.930426in}}{\pgfqpoint{3.905098in}{1.930426in}}%
\pgfpathclose%
\pgfusepath{stroke,fill}%
\end{pgfscope}%
\begin{pgfscope}%
\pgfpathrectangle{\pgfqpoint{3.793912in}{0.557870in}}{\pgfqpoint{2.446088in}{1.484734in}}%
\pgfusepath{clip}%
\pgfsetbuttcap%
\pgfsetroundjoin%
\definecolor{currentfill}{rgb}{0.298039,0.447059,0.690196}%
\pgfsetfillcolor{currentfill}%
\pgfsetlinewidth{1.003750pt}%
\definecolor{currentstroke}{rgb}{0.298039,0.447059,0.690196}%
\pgfsetstrokecolor{currentstroke}%
\pgfsetdash{}{0pt}%
\pgfpathmoveto{\pgfqpoint{3.951425in}{0.594302in}}%
\pgfpathcurveto{\pgfqpoint{3.959662in}{0.594302in}}{\pgfqpoint{3.967562in}{0.597574in}}{\pgfqpoint{3.973386in}{0.603398in}}%
\pgfpathcurveto{\pgfqpoint{3.979210in}{0.609222in}}{\pgfqpoint{3.982482in}{0.617122in}}{\pgfqpoint{3.982482in}{0.625358in}}%
\pgfpathcurveto{\pgfqpoint{3.982482in}{0.633594in}}{\pgfqpoint{3.979210in}{0.641495in}}{\pgfqpoint{3.973386in}{0.647318in}}%
\pgfpathcurveto{\pgfqpoint{3.967562in}{0.653142in}}{\pgfqpoint{3.959662in}{0.656415in}}{\pgfqpoint{3.951425in}{0.656415in}}%
\pgfpathcurveto{\pgfqpoint{3.943189in}{0.656415in}}{\pgfqpoint{3.935289in}{0.653142in}}{\pgfqpoint{3.929465in}{0.647318in}}%
\pgfpathcurveto{\pgfqpoint{3.923641in}{0.641495in}}{\pgfqpoint{3.920369in}{0.633594in}}{\pgfqpoint{3.920369in}{0.625358in}}%
\pgfpathcurveto{\pgfqpoint{3.920369in}{0.617122in}}{\pgfqpoint{3.923641in}{0.609222in}}{\pgfqpoint{3.929465in}{0.603398in}}%
\pgfpathcurveto{\pgfqpoint{3.935289in}{0.597574in}}{\pgfqpoint{3.943189in}{0.594302in}}{\pgfqpoint{3.951425in}{0.594302in}}%
\pgfpathclose%
\pgfusepath{stroke,fill}%
\end{pgfscope}%
\begin{pgfscope}%
\pgfpathrectangle{\pgfqpoint{3.793912in}{0.557870in}}{\pgfqpoint{2.446088in}{1.484734in}}%
\pgfusepath{clip}%
\pgfsetbuttcap%
\pgfsetroundjoin%
\definecolor{currentfill}{rgb}{0.298039,0.447059,0.690196}%
\pgfsetfillcolor{currentfill}%
\pgfsetlinewidth{1.003750pt}%
\definecolor{currentstroke}{rgb}{0.298039,0.447059,0.690196}%
\pgfsetstrokecolor{currentstroke}%
\pgfsetdash{}{0pt}%
\pgfpathmoveto{\pgfqpoint{3.905098in}{1.916792in}}%
\pgfpathcurveto{\pgfqpoint{3.913334in}{1.916792in}}{\pgfqpoint{3.921234in}{1.920064in}}{\pgfqpoint{3.927058in}{1.925888in}}%
\pgfpathcurveto{\pgfqpoint{3.932882in}{1.931712in}}{\pgfqpoint{3.936155in}{1.939612in}}{\pgfqpoint{3.936155in}{1.947848in}}%
\pgfpathcurveto{\pgfqpoint{3.936155in}{1.956085in}}{\pgfqpoint{3.932882in}{1.963985in}}{\pgfqpoint{3.927058in}{1.969809in}}%
\pgfpathcurveto{\pgfqpoint{3.921234in}{1.975633in}}{\pgfqpoint{3.913334in}{1.978905in}}{\pgfqpoint{3.905098in}{1.978905in}}%
\pgfpathcurveto{\pgfqpoint{3.896862in}{1.978905in}}{\pgfqpoint{3.888962in}{1.975633in}}{\pgfqpoint{3.883138in}{1.969809in}}%
\pgfpathcurveto{\pgfqpoint{3.877314in}{1.963985in}}{\pgfqpoint{3.874042in}{1.956085in}}{\pgfqpoint{3.874042in}{1.947848in}}%
\pgfpathcurveto{\pgfqpoint{3.874042in}{1.939612in}}{\pgfqpoint{3.877314in}{1.931712in}}{\pgfqpoint{3.883138in}{1.925888in}}%
\pgfpathcurveto{\pgfqpoint{3.888962in}{1.920064in}}{\pgfqpoint{3.896862in}{1.916792in}}{\pgfqpoint{3.905098in}{1.916792in}}%
\pgfpathclose%
\pgfusepath{stroke,fill}%
\end{pgfscope}%
\begin{pgfscope}%
\pgfpathrectangle{\pgfqpoint{3.793912in}{0.557870in}}{\pgfqpoint{2.446088in}{1.484734in}}%
\pgfusepath{clip}%
\pgfsetbuttcap%
\pgfsetroundjoin%
\definecolor{currentfill}{rgb}{0.298039,0.447059,0.690196}%
\pgfsetfillcolor{currentfill}%
\pgfsetlinewidth{1.003750pt}%
\definecolor{currentstroke}{rgb}{0.298039,0.447059,0.690196}%
\pgfsetstrokecolor{currentstroke}%
\pgfsetdash{}{0pt}%
\pgfpathmoveto{\pgfqpoint{3.905098in}{1.930426in}}%
\pgfpathcurveto{\pgfqpoint{3.913334in}{1.930426in}}{\pgfqpoint{3.921234in}{1.933698in}}{\pgfqpoint{3.927058in}{1.939522in}}%
\pgfpathcurveto{\pgfqpoint{3.932882in}{1.945346in}}{\pgfqpoint{3.936155in}{1.953246in}}{\pgfqpoint{3.936155in}{1.961482in}}%
\pgfpathcurveto{\pgfqpoint{3.936155in}{1.969719in}}{\pgfqpoint{3.932882in}{1.977619in}}{\pgfqpoint{3.927058in}{1.983443in}}%
\pgfpathcurveto{\pgfqpoint{3.921234in}{1.989267in}}{\pgfqpoint{3.913334in}{1.992539in}}{\pgfqpoint{3.905098in}{1.992539in}}%
\pgfpathcurveto{\pgfqpoint{3.896862in}{1.992539in}}{\pgfqpoint{3.888962in}{1.989267in}}{\pgfqpoint{3.883138in}{1.983443in}}%
\pgfpathcurveto{\pgfqpoint{3.877314in}{1.977619in}}{\pgfqpoint{3.874042in}{1.969719in}}{\pgfqpoint{3.874042in}{1.961482in}}%
\pgfpathcurveto{\pgfqpoint{3.874042in}{1.953246in}}{\pgfqpoint{3.877314in}{1.945346in}}{\pgfqpoint{3.883138in}{1.939522in}}%
\pgfpathcurveto{\pgfqpoint{3.888962in}{1.933698in}}{\pgfqpoint{3.896862in}{1.930426in}}{\pgfqpoint{3.905098in}{1.930426in}}%
\pgfpathclose%
\pgfusepath{stroke,fill}%
\end{pgfscope}%
\begin{pgfscope}%
\pgfpathrectangle{\pgfqpoint{3.793912in}{0.557870in}}{\pgfqpoint{2.446088in}{1.484734in}}%
\pgfusepath{clip}%
\pgfsetbuttcap%
\pgfsetroundjoin%
\definecolor{currentfill}{rgb}{0.298039,0.447059,0.690196}%
\pgfsetfillcolor{currentfill}%
\pgfsetlinewidth{1.003750pt}%
\definecolor{currentstroke}{rgb}{0.298039,0.447059,0.690196}%
\pgfsetstrokecolor{currentstroke}%
\pgfsetdash{}{0pt}%
\pgfpathmoveto{\pgfqpoint{3.905098in}{1.603212in}}%
\pgfpathcurveto{\pgfqpoint{3.913334in}{1.603212in}}{\pgfqpoint{3.921234in}{1.606484in}}{\pgfqpoint{3.927058in}{1.612308in}}%
\pgfpathcurveto{\pgfqpoint{3.932882in}{1.618132in}}{\pgfqpoint{3.936155in}{1.626032in}}{\pgfqpoint{3.936155in}{1.634268in}}%
\pgfpathcurveto{\pgfqpoint{3.936155in}{1.642505in}}{\pgfqpoint{3.932882in}{1.650405in}}{\pgfqpoint{3.927058in}{1.656229in}}%
\pgfpathcurveto{\pgfqpoint{3.921234in}{1.662052in}}{\pgfqpoint{3.913334in}{1.665325in}}{\pgfqpoint{3.905098in}{1.665325in}}%
\pgfpathcurveto{\pgfqpoint{3.896862in}{1.665325in}}{\pgfqpoint{3.888962in}{1.662052in}}{\pgfqpoint{3.883138in}{1.656229in}}%
\pgfpathcurveto{\pgfqpoint{3.877314in}{1.650405in}}{\pgfqpoint{3.874042in}{1.642505in}}{\pgfqpoint{3.874042in}{1.634268in}}%
\pgfpathcurveto{\pgfqpoint{3.874042in}{1.626032in}}{\pgfqpoint{3.877314in}{1.618132in}}{\pgfqpoint{3.883138in}{1.612308in}}%
\pgfpathcurveto{\pgfqpoint{3.888962in}{1.606484in}}{\pgfqpoint{3.896862in}{1.603212in}}{\pgfqpoint{3.905098in}{1.603212in}}%
\pgfpathclose%
\pgfusepath{stroke,fill}%
\end{pgfscope}%
\begin{pgfscope}%
\pgfpathrectangle{\pgfqpoint{3.793912in}{0.557870in}}{\pgfqpoint{2.446088in}{1.484734in}}%
\pgfusepath{clip}%
\pgfsetbuttcap%
\pgfsetroundjoin%
\definecolor{currentfill}{rgb}{0.298039,0.447059,0.690196}%
\pgfsetfillcolor{currentfill}%
\pgfsetlinewidth{1.003750pt}%
\definecolor{currentstroke}{rgb}{0.298039,0.447059,0.690196}%
\pgfsetstrokecolor{currentstroke}%
\pgfsetdash{}{0pt}%
\pgfpathmoveto{\pgfqpoint{3.905098in}{1.930426in}}%
\pgfpathcurveto{\pgfqpoint{3.913334in}{1.930426in}}{\pgfqpoint{3.921234in}{1.933698in}}{\pgfqpoint{3.927058in}{1.939522in}}%
\pgfpathcurveto{\pgfqpoint{3.932882in}{1.945346in}}{\pgfqpoint{3.936155in}{1.953246in}}{\pgfqpoint{3.936155in}{1.961482in}}%
\pgfpathcurveto{\pgfqpoint{3.936155in}{1.969719in}}{\pgfqpoint{3.932882in}{1.977619in}}{\pgfqpoint{3.927058in}{1.983443in}}%
\pgfpathcurveto{\pgfqpoint{3.921234in}{1.989267in}}{\pgfqpoint{3.913334in}{1.992539in}}{\pgfqpoint{3.905098in}{1.992539in}}%
\pgfpathcurveto{\pgfqpoint{3.896862in}{1.992539in}}{\pgfqpoint{3.888962in}{1.989267in}}{\pgfqpoint{3.883138in}{1.983443in}}%
\pgfpathcurveto{\pgfqpoint{3.877314in}{1.977619in}}{\pgfqpoint{3.874042in}{1.969719in}}{\pgfqpoint{3.874042in}{1.961482in}}%
\pgfpathcurveto{\pgfqpoint{3.874042in}{1.953246in}}{\pgfqpoint{3.877314in}{1.945346in}}{\pgfqpoint{3.883138in}{1.939522in}}%
\pgfpathcurveto{\pgfqpoint{3.888962in}{1.933698in}}{\pgfqpoint{3.896862in}{1.930426in}}{\pgfqpoint{3.905098in}{1.930426in}}%
\pgfpathclose%
\pgfusepath{stroke,fill}%
\end{pgfscope}%
\begin{pgfscope}%
\pgfpathrectangle{\pgfqpoint{3.793912in}{0.557870in}}{\pgfqpoint{2.446088in}{1.484734in}}%
\pgfusepath{clip}%
\pgfsetbuttcap%
\pgfsetroundjoin%
\definecolor{currentfill}{rgb}{0.298039,0.447059,0.690196}%
\pgfsetfillcolor{currentfill}%
\pgfsetlinewidth{1.003750pt}%
\definecolor{currentstroke}{rgb}{0.298039,0.447059,0.690196}%
\pgfsetstrokecolor{currentstroke}%
\pgfsetdash{}{0pt}%
\pgfpathmoveto{\pgfqpoint{5.387575in}{0.594302in}}%
\pgfpathcurveto{\pgfqpoint{5.395812in}{0.594302in}}{\pgfqpoint{5.403712in}{0.597574in}}{\pgfqpoint{5.409536in}{0.603398in}}%
\pgfpathcurveto{\pgfqpoint{5.415360in}{0.609222in}}{\pgfqpoint{5.418632in}{0.617122in}}{\pgfqpoint{5.418632in}{0.625358in}}%
\pgfpathcurveto{\pgfqpoint{5.418632in}{0.633594in}}{\pgfqpoint{5.415360in}{0.641495in}}{\pgfqpoint{5.409536in}{0.647318in}}%
\pgfpathcurveto{\pgfqpoint{5.403712in}{0.653142in}}{\pgfqpoint{5.395812in}{0.656415in}}{\pgfqpoint{5.387575in}{0.656415in}}%
\pgfpathcurveto{\pgfqpoint{5.379339in}{0.656415in}}{\pgfqpoint{5.371439in}{0.653142in}}{\pgfqpoint{5.365615in}{0.647318in}}%
\pgfpathcurveto{\pgfqpoint{5.359791in}{0.641495in}}{\pgfqpoint{5.356519in}{0.633594in}}{\pgfqpoint{5.356519in}{0.625358in}}%
\pgfpathcurveto{\pgfqpoint{5.356519in}{0.617122in}}{\pgfqpoint{5.359791in}{0.609222in}}{\pgfqpoint{5.365615in}{0.603398in}}%
\pgfpathcurveto{\pgfqpoint{5.371439in}{0.597574in}}{\pgfqpoint{5.379339in}{0.594302in}}{\pgfqpoint{5.387575in}{0.594302in}}%
\pgfpathclose%
\pgfusepath{stroke,fill}%
\end{pgfscope}%
\begin{pgfscope}%
\pgfpathrectangle{\pgfqpoint{3.793912in}{0.557870in}}{\pgfqpoint{2.446088in}{1.484734in}}%
\pgfusepath{clip}%
\pgfsetbuttcap%
\pgfsetroundjoin%
\definecolor{currentfill}{rgb}{0.298039,0.447059,0.690196}%
\pgfsetfillcolor{currentfill}%
\pgfsetlinewidth{1.003750pt}%
\definecolor{currentstroke}{rgb}{0.298039,0.447059,0.690196}%
\pgfsetstrokecolor{currentstroke}%
\pgfsetdash{}{0pt}%
\pgfpathmoveto{\pgfqpoint{3.905098in}{1.930426in}}%
\pgfpathcurveto{\pgfqpoint{3.913334in}{1.930426in}}{\pgfqpoint{3.921234in}{1.933698in}}{\pgfqpoint{3.927058in}{1.939522in}}%
\pgfpathcurveto{\pgfqpoint{3.932882in}{1.945346in}}{\pgfqpoint{3.936155in}{1.953246in}}{\pgfqpoint{3.936155in}{1.961482in}}%
\pgfpathcurveto{\pgfqpoint{3.936155in}{1.969719in}}{\pgfqpoint{3.932882in}{1.977619in}}{\pgfqpoint{3.927058in}{1.983443in}}%
\pgfpathcurveto{\pgfqpoint{3.921234in}{1.989267in}}{\pgfqpoint{3.913334in}{1.992539in}}{\pgfqpoint{3.905098in}{1.992539in}}%
\pgfpathcurveto{\pgfqpoint{3.896862in}{1.992539in}}{\pgfqpoint{3.888962in}{1.989267in}}{\pgfqpoint{3.883138in}{1.983443in}}%
\pgfpathcurveto{\pgfqpoint{3.877314in}{1.977619in}}{\pgfqpoint{3.874042in}{1.969719in}}{\pgfqpoint{3.874042in}{1.961482in}}%
\pgfpathcurveto{\pgfqpoint{3.874042in}{1.953246in}}{\pgfqpoint{3.877314in}{1.945346in}}{\pgfqpoint{3.883138in}{1.939522in}}%
\pgfpathcurveto{\pgfqpoint{3.888962in}{1.933698in}}{\pgfqpoint{3.896862in}{1.930426in}}{\pgfqpoint{3.905098in}{1.930426in}}%
\pgfpathclose%
\pgfusepath{stroke,fill}%
\end{pgfscope}%
\begin{pgfscope}%
\pgfpathrectangle{\pgfqpoint{3.793912in}{0.557870in}}{\pgfqpoint{2.446088in}{1.484734in}}%
\pgfusepath{clip}%
\pgfsetbuttcap%
\pgfsetroundjoin%
\definecolor{currentfill}{rgb}{0.298039,0.447059,0.690196}%
\pgfsetfillcolor{currentfill}%
\pgfsetlinewidth{1.003750pt}%
\definecolor{currentstroke}{rgb}{0.298039,0.447059,0.690196}%
\pgfsetstrokecolor{currentstroke}%
\pgfsetdash{}{0pt}%
\pgfpathmoveto{\pgfqpoint{3.905098in}{1.930426in}}%
\pgfpathcurveto{\pgfqpoint{3.913334in}{1.930426in}}{\pgfqpoint{3.921234in}{1.933698in}}{\pgfqpoint{3.927058in}{1.939522in}}%
\pgfpathcurveto{\pgfqpoint{3.932882in}{1.945346in}}{\pgfqpoint{3.936155in}{1.953246in}}{\pgfqpoint{3.936155in}{1.961482in}}%
\pgfpathcurveto{\pgfqpoint{3.936155in}{1.969719in}}{\pgfqpoint{3.932882in}{1.977619in}}{\pgfqpoint{3.927058in}{1.983443in}}%
\pgfpathcurveto{\pgfqpoint{3.921234in}{1.989267in}}{\pgfqpoint{3.913334in}{1.992539in}}{\pgfqpoint{3.905098in}{1.992539in}}%
\pgfpathcurveto{\pgfqpoint{3.896862in}{1.992539in}}{\pgfqpoint{3.888962in}{1.989267in}}{\pgfqpoint{3.883138in}{1.983443in}}%
\pgfpathcurveto{\pgfqpoint{3.877314in}{1.977619in}}{\pgfqpoint{3.874042in}{1.969719in}}{\pgfqpoint{3.874042in}{1.961482in}}%
\pgfpathcurveto{\pgfqpoint{3.874042in}{1.953246in}}{\pgfqpoint{3.877314in}{1.945346in}}{\pgfqpoint{3.883138in}{1.939522in}}%
\pgfpathcurveto{\pgfqpoint{3.888962in}{1.933698in}}{\pgfqpoint{3.896862in}{1.930426in}}{\pgfqpoint{3.905098in}{1.930426in}}%
\pgfpathclose%
\pgfusepath{stroke,fill}%
\end{pgfscope}%
\begin{pgfscope}%
\pgfpathrectangle{\pgfqpoint{3.793912in}{0.557870in}}{\pgfqpoint{2.446088in}{1.484734in}}%
\pgfusepath{clip}%
\pgfsetbuttcap%
\pgfsetroundjoin%
\definecolor{currentfill}{rgb}{0.298039,0.447059,0.690196}%
\pgfsetfillcolor{currentfill}%
\pgfsetlinewidth{1.003750pt}%
\definecolor{currentstroke}{rgb}{0.298039,0.447059,0.690196}%
\pgfsetstrokecolor{currentstroke}%
\pgfsetdash{}{0pt}%
\pgfpathmoveto{\pgfqpoint{5.225430in}{0.594302in}}%
\pgfpathcurveto{\pgfqpoint{5.233666in}{0.594302in}}{\pgfqpoint{5.241566in}{0.597574in}}{\pgfqpoint{5.247390in}{0.603398in}}%
\pgfpathcurveto{\pgfqpoint{5.253214in}{0.609222in}}{\pgfqpoint{5.256486in}{0.617122in}}{\pgfqpoint{5.256486in}{0.625358in}}%
\pgfpathcurveto{\pgfqpoint{5.256486in}{0.633594in}}{\pgfqpoint{5.253214in}{0.641495in}}{\pgfqpoint{5.247390in}{0.647318in}}%
\pgfpathcurveto{\pgfqpoint{5.241566in}{0.653142in}}{\pgfqpoint{5.233666in}{0.656415in}}{\pgfqpoint{5.225430in}{0.656415in}}%
\pgfpathcurveto{\pgfqpoint{5.217193in}{0.656415in}}{\pgfqpoint{5.209293in}{0.653142in}}{\pgfqpoint{5.203469in}{0.647318in}}%
\pgfpathcurveto{\pgfqpoint{5.197645in}{0.641495in}}{\pgfqpoint{5.194373in}{0.633594in}}{\pgfqpoint{5.194373in}{0.625358in}}%
\pgfpathcurveto{\pgfqpoint{5.194373in}{0.617122in}}{\pgfqpoint{5.197645in}{0.609222in}}{\pgfqpoint{5.203469in}{0.603398in}}%
\pgfpathcurveto{\pgfqpoint{5.209293in}{0.597574in}}{\pgfqpoint{5.217193in}{0.594302in}}{\pgfqpoint{5.225430in}{0.594302in}}%
\pgfpathclose%
\pgfusepath{stroke,fill}%
\end{pgfscope}%
\begin{pgfscope}%
\pgfpathrectangle{\pgfqpoint{3.793912in}{0.557870in}}{\pgfqpoint{2.446088in}{1.484734in}}%
\pgfusepath{clip}%
\pgfsetbuttcap%
\pgfsetroundjoin%
\definecolor{currentfill}{rgb}{0.298039,0.447059,0.690196}%
\pgfsetfillcolor{currentfill}%
\pgfsetlinewidth{1.003750pt}%
\definecolor{currentstroke}{rgb}{0.298039,0.447059,0.690196}%
\pgfsetstrokecolor{currentstroke}%
\pgfsetdash{}{0pt}%
\pgfpathmoveto{\pgfqpoint{3.905098in}{1.385069in}}%
\pgfpathcurveto{\pgfqpoint{3.913334in}{1.385069in}}{\pgfqpoint{3.921234in}{1.388341in}}{\pgfqpoint{3.927058in}{1.394165in}}%
\pgfpathcurveto{\pgfqpoint{3.932882in}{1.399989in}}{\pgfqpoint{3.936155in}{1.407889in}}{\pgfqpoint{3.936155in}{1.416126in}}%
\pgfpathcurveto{\pgfqpoint{3.936155in}{1.424362in}}{\pgfqpoint{3.932882in}{1.432262in}}{\pgfqpoint{3.927058in}{1.438086in}}%
\pgfpathcurveto{\pgfqpoint{3.921234in}{1.443910in}}{\pgfqpoint{3.913334in}{1.447182in}}{\pgfqpoint{3.905098in}{1.447182in}}%
\pgfpathcurveto{\pgfqpoint{3.896862in}{1.447182in}}{\pgfqpoint{3.888962in}{1.443910in}}{\pgfqpoint{3.883138in}{1.438086in}}%
\pgfpathcurveto{\pgfqpoint{3.877314in}{1.432262in}}{\pgfqpoint{3.874042in}{1.424362in}}{\pgfqpoint{3.874042in}{1.416126in}}%
\pgfpathcurveto{\pgfqpoint{3.874042in}{1.407889in}}{\pgfqpoint{3.877314in}{1.399989in}}{\pgfqpoint{3.883138in}{1.394165in}}%
\pgfpathcurveto{\pgfqpoint{3.888962in}{1.388341in}}{\pgfqpoint{3.896862in}{1.385069in}}{\pgfqpoint{3.905098in}{1.385069in}}%
\pgfpathclose%
\pgfusepath{stroke,fill}%
\end{pgfscope}%
\begin{pgfscope}%
\pgfpathrectangle{\pgfqpoint{3.793912in}{0.557870in}}{\pgfqpoint{2.446088in}{1.484734in}}%
\pgfusepath{clip}%
\pgfsetbuttcap%
\pgfsetroundjoin%
\definecolor{currentfill}{rgb}{0.298039,0.447059,0.690196}%
\pgfsetfillcolor{currentfill}%
\pgfsetlinewidth{1.003750pt}%
\definecolor{currentstroke}{rgb}{0.298039,0.447059,0.690196}%
\pgfsetstrokecolor{currentstroke}%
\pgfsetdash{}{0pt}%
\pgfpathmoveto{\pgfqpoint{3.905098in}{1.344167in}}%
\pgfpathcurveto{\pgfqpoint{3.913334in}{1.344167in}}{\pgfqpoint{3.921234in}{1.347440in}}{\pgfqpoint{3.927058in}{1.353264in}}%
\pgfpathcurveto{\pgfqpoint{3.932882in}{1.359087in}}{\pgfqpoint{3.936155in}{1.366988in}}{\pgfqpoint{3.936155in}{1.375224in}}%
\pgfpathcurveto{\pgfqpoint{3.936155in}{1.383460in}}{\pgfqpoint{3.932882in}{1.391360in}}{\pgfqpoint{3.927058in}{1.397184in}}%
\pgfpathcurveto{\pgfqpoint{3.921234in}{1.403008in}}{\pgfqpoint{3.913334in}{1.406280in}}{\pgfqpoint{3.905098in}{1.406280in}}%
\pgfpathcurveto{\pgfqpoint{3.896862in}{1.406280in}}{\pgfqpoint{3.888962in}{1.403008in}}{\pgfqpoint{3.883138in}{1.397184in}}%
\pgfpathcurveto{\pgfqpoint{3.877314in}{1.391360in}}{\pgfqpoint{3.874042in}{1.383460in}}{\pgfqpoint{3.874042in}{1.375224in}}%
\pgfpathcurveto{\pgfqpoint{3.874042in}{1.366988in}}{\pgfqpoint{3.877314in}{1.359087in}}{\pgfqpoint{3.883138in}{1.353264in}}%
\pgfpathcurveto{\pgfqpoint{3.888962in}{1.347440in}}{\pgfqpoint{3.896862in}{1.344167in}}{\pgfqpoint{3.905098in}{1.344167in}}%
\pgfpathclose%
\pgfusepath{stroke,fill}%
\end{pgfscope}%
\begin{pgfscope}%
\pgfpathrectangle{\pgfqpoint{3.793912in}{0.557870in}}{\pgfqpoint{2.446088in}{1.484734in}}%
\pgfusepath{clip}%
\pgfsetbuttcap%
\pgfsetroundjoin%
\definecolor{currentfill}{rgb}{0.298039,0.447059,0.690196}%
\pgfsetfillcolor{currentfill}%
\pgfsetlinewidth{1.003750pt}%
\definecolor{currentstroke}{rgb}{0.298039,0.447059,0.690196}%
\pgfsetstrokecolor{currentstroke}%
\pgfsetdash{}{0pt}%
\pgfpathmoveto{\pgfqpoint{3.905098in}{1.289632in}}%
\pgfpathcurveto{\pgfqpoint{3.913334in}{1.289632in}}{\pgfqpoint{3.921234in}{1.292904in}}{\pgfqpoint{3.927058in}{1.298728in}}%
\pgfpathcurveto{\pgfqpoint{3.932882in}{1.304552in}}{\pgfqpoint{3.936155in}{1.312452in}}{\pgfqpoint{3.936155in}{1.320688in}}%
\pgfpathcurveto{\pgfqpoint{3.936155in}{1.328924in}}{\pgfqpoint{3.932882in}{1.336824in}}{\pgfqpoint{3.927058in}{1.342648in}}%
\pgfpathcurveto{\pgfqpoint{3.921234in}{1.348472in}}{\pgfqpoint{3.913334in}{1.351745in}}{\pgfqpoint{3.905098in}{1.351745in}}%
\pgfpathcurveto{\pgfqpoint{3.896862in}{1.351745in}}{\pgfqpoint{3.888962in}{1.348472in}}{\pgfqpoint{3.883138in}{1.342648in}}%
\pgfpathcurveto{\pgfqpoint{3.877314in}{1.336824in}}{\pgfqpoint{3.874042in}{1.328924in}}{\pgfqpoint{3.874042in}{1.320688in}}%
\pgfpathcurveto{\pgfqpoint{3.874042in}{1.312452in}}{\pgfqpoint{3.877314in}{1.304552in}}{\pgfqpoint{3.883138in}{1.298728in}}%
\pgfpathcurveto{\pgfqpoint{3.888962in}{1.292904in}}{\pgfqpoint{3.896862in}{1.289632in}}{\pgfqpoint{3.905098in}{1.289632in}}%
\pgfpathclose%
\pgfusepath{stroke,fill}%
\end{pgfscope}%
\begin{pgfscope}%
\pgfpathrectangle{\pgfqpoint{3.793912in}{0.557870in}}{\pgfqpoint{2.446088in}{1.484734in}}%
\pgfusepath{clip}%
\pgfsetbuttcap%
\pgfsetroundjoin%
\definecolor{currentfill}{rgb}{0.298039,0.447059,0.690196}%
\pgfsetfillcolor{currentfill}%
\pgfsetlinewidth{1.003750pt}%
\definecolor{currentstroke}{rgb}{0.298039,0.447059,0.690196}%
\pgfsetstrokecolor{currentstroke}%
\pgfsetdash{}{0pt}%
\pgfpathmoveto{\pgfqpoint{3.905098in}{0.948784in}}%
\pgfpathcurveto{\pgfqpoint{3.913334in}{0.948784in}}{\pgfqpoint{3.921234in}{0.952056in}}{\pgfqpoint{3.927058in}{0.957880in}}%
\pgfpathcurveto{\pgfqpoint{3.932882in}{0.963704in}}{\pgfqpoint{3.936155in}{0.971604in}}{\pgfqpoint{3.936155in}{0.979840in}}%
\pgfpathcurveto{\pgfqpoint{3.936155in}{0.988076in}}{\pgfqpoint{3.932882in}{0.995976in}}{\pgfqpoint{3.927058in}{1.001800in}}%
\pgfpathcurveto{\pgfqpoint{3.921234in}{1.007624in}}{\pgfqpoint{3.913334in}{1.010897in}}{\pgfqpoint{3.905098in}{1.010897in}}%
\pgfpathcurveto{\pgfqpoint{3.896862in}{1.010897in}}{\pgfqpoint{3.888962in}{1.007624in}}{\pgfqpoint{3.883138in}{1.001800in}}%
\pgfpathcurveto{\pgfqpoint{3.877314in}{0.995976in}}{\pgfqpoint{3.874042in}{0.988076in}}{\pgfqpoint{3.874042in}{0.979840in}}%
\pgfpathcurveto{\pgfqpoint{3.874042in}{0.971604in}}{\pgfqpoint{3.877314in}{0.963704in}}{\pgfqpoint{3.883138in}{0.957880in}}%
\pgfpathcurveto{\pgfqpoint{3.888962in}{0.952056in}}{\pgfqpoint{3.896862in}{0.948784in}}{\pgfqpoint{3.905098in}{0.948784in}}%
\pgfpathclose%
\pgfusepath{stroke,fill}%
\end{pgfscope}%
\begin{pgfscope}%
\pgfpathrectangle{\pgfqpoint{3.793912in}{0.557870in}}{\pgfqpoint{2.446088in}{1.484734in}}%
\pgfusepath{clip}%
\pgfsetbuttcap%
\pgfsetroundjoin%
\definecolor{currentfill}{rgb}{0.298039,0.447059,0.690196}%
\pgfsetfillcolor{currentfill}%
\pgfsetlinewidth{1.003750pt}%
\definecolor{currentstroke}{rgb}{0.298039,0.447059,0.690196}%
\pgfsetstrokecolor{currentstroke}%
\pgfsetdash{}{0pt}%
\pgfpathmoveto{\pgfqpoint{3.905098in}{0.989685in}}%
\pgfpathcurveto{\pgfqpoint{3.913334in}{0.989685in}}{\pgfqpoint{3.921234in}{0.992958in}}{\pgfqpoint{3.927058in}{0.998782in}}%
\pgfpathcurveto{\pgfqpoint{3.932882in}{1.004606in}}{\pgfqpoint{3.936155in}{1.012506in}}{\pgfqpoint{3.936155in}{1.020742in}}%
\pgfpathcurveto{\pgfqpoint{3.936155in}{1.028978in}}{\pgfqpoint{3.932882in}{1.036878in}}{\pgfqpoint{3.927058in}{1.042702in}}%
\pgfpathcurveto{\pgfqpoint{3.921234in}{1.048526in}}{\pgfqpoint{3.913334in}{1.051798in}}{\pgfqpoint{3.905098in}{1.051798in}}%
\pgfpathcurveto{\pgfqpoint{3.896862in}{1.051798in}}{\pgfqpoint{3.888962in}{1.048526in}}{\pgfqpoint{3.883138in}{1.042702in}}%
\pgfpathcurveto{\pgfqpoint{3.877314in}{1.036878in}}{\pgfqpoint{3.874042in}{1.028978in}}{\pgfqpoint{3.874042in}{1.020742in}}%
\pgfpathcurveto{\pgfqpoint{3.874042in}{1.012506in}}{\pgfqpoint{3.877314in}{1.004606in}}{\pgfqpoint{3.883138in}{0.998782in}}%
\pgfpathcurveto{\pgfqpoint{3.888962in}{0.992958in}}{\pgfqpoint{3.896862in}{0.989685in}}{\pgfqpoint{3.905098in}{0.989685in}}%
\pgfpathclose%
\pgfusepath{stroke,fill}%
\end{pgfscope}%
\begin{pgfscope}%
\pgfpathrectangle{\pgfqpoint{3.793912in}{0.557870in}}{\pgfqpoint{2.446088in}{1.484734in}}%
\pgfusepath{clip}%
\pgfsetbuttcap%
\pgfsetroundjoin%
\definecolor{currentfill}{rgb}{0.298039,0.447059,0.690196}%
\pgfsetfillcolor{currentfill}%
\pgfsetlinewidth{1.003750pt}%
\definecolor{currentstroke}{rgb}{0.298039,0.447059,0.690196}%
\pgfsetstrokecolor{currentstroke}%
\pgfsetdash{}{0pt}%
\pgfpathmoveto{\pgfqpoint{3.905098in}{1.930426in}}%
\pgfpathcurveto{\pgfqpoint{3.913334in}{1.930426in}}{\pgfqpoint{3.921234in}{1.933698in}}{\pgfqpoint{3.927058in}{1.939522in}}%
\pgfpathcurveto{\pgfqpoint{3.932882in}{1.945346in}}{\pgfqpoint{3.936155in}{1.953246in}}{\pgfqpoint{3.936155in}{1.961482in}}%
\pgfpathcurveto{\pgfqpoint{3.936155in}{1.969719in}}{\pgfqpoint{3.932882in}{1.977619in}}{\pgfqpoint{3.927058in}{1.983443in}}%
\pgfpathcurveto{\pgfqpoint{3.921234in}{1.989267in}}{\pgfqpoint{3.913334in}{1.992539in}}{\pgfqpoint{3.905098in}{1.992539in}}%
\pgfpathcurveto{\pgfqpoint{3.896862in}{1.992539in}}{\pgfqpoint{3.888962in}{1.989267in}}{\pgfqpoint{3.883138in}{1.983443in}}%
\pgfpathcurveto{\pgfqpoint{3.877314in}{1.977619in}}{\pgfqpoint{3.874042in}{1.969719in}}{\pgfqpoint{3.874042in}{1.961482in}}%
\pgfpathcurveto{\pgfqpoint{3.874042in}{1.953246in}}{\pgfqpoint{3.877314in}{1.945346in}}{\pgfqpoint{3.883138in}{1.939522in}}%
\pgfpathcurveto{\pgfqpoint{3.888962in}{1.933698in}}{\pgfqpoint{3.896862in}{1.930426in}}{\pgfqpoint{3.905098in}{1.930426in}}%
\pgfpathclose%
\pgfusepath{stroke,fill}%
\end{pgfscope}%
\begin{pgfscope}%
\pgfpathrectangle{\pgfqpoint{3.793912in}{0.557870in}}{\pgfqpoint{2.446088in}{1.484734in}}%
\pgfusepath{clip}%
\pgfsetbuttcap%
\pgfsetroundjoin%
\definecolor{currentfill}{rgb}{0.298039,0.447059,0.690196}%
\pgfsetfillcolor{currentfill}%
\pgfsetlinewidth{1.003750pt}%
\definecolor{currentstroke}{rgb}{0.298039,0.447059,0.690196}%
\pgfsetstrokecolor{currentstroke}%
\pgfsetdash{}{0pt}%
\pgfpathmoveto{\pgfqpoint{3.905098in}{0.907882in}}%
\pgfpathcurveto{\pgfqpoint{3.913334in}{0.907882in}}{\pgfqpoint{3.921234in}{0.911154in}}{\pgfqpoint{3.927058in}{0.916978in}}%
\pgfpathcurveto{\pgfqpoint{3.932882in}{0.922802in}}{\pgfqpoint{3.936155in}{0.930702in}}{\pgfqpoint{3.936155in}{0.938938in}}%
\pgfpathcurveto{\pgfqpoint{3.936155in}{0.947175in}}{\pgfqpoint{3.932882in}{0.955075in}}{\pgfqpoint{3.927058in}{0.960899in}}%
\pgfpathcurveto{\pgfqpoint{3.921234in}{0.966723in}}{\pgfqpoint{3.913334in}{0.969995in}}{\pgfqpoint{3.905098in}{0.969995in}}%
\pgfpathcurveto{\pgfqpoint{3.896862in}{0.969995in}}{\pgfqpoint{3.888962in}{0.966723in}}{\pgfqpoint{3.883138in}{0.960899in}}%
\pgfpathcurveto{\pgfqpoint{3.877314in}{0.955075in}}{\pgfqpoint{3.874042in}{0.947175in}}{\pgfqpoint{3.874042in}{0.938938in}}%
\pgfpathcurveto{\pgfqpoint{3.874042in}{0.930702in}}{\pgfqpoint{3.877314in}{0.922802in}}{\pgfqpoint{3.883138in}{0.916978in}}%
\pgfpathcurveto{\pgfqpoint{3.888962in}{0.911154in}}{\pgfqpoint{3.896862in}{0.907882in}}{\pgfqpoint{3.905098in}{0.907882in}}%
\pgfpathclose%
\pgfusepath{stroke,fill}%
\end{pgfscope}%
\begin{pgfscope}%
\pgfpathrectangle{\pgfqpoint{3.793912in}{0.557870in}}{\pgfqpoint{2.446088in}{1.484734in}}%
\pgfusepath{clip}%
\pgfsetbuttcap%
\pgfsetroundjoin%
\definecolor{currentfill}{rgb}{0.298039,0.447059,0.690196}%
\pgfsetfillcolor{currentfill}%
\pgfsetlinewidth{1.003750pt}%
\definecolor{currentstroke}{rgb}{0.298039,0.447059,0.690196}%
\pgfsetstrokecolor{currentstroke}%
\pgfsetdash{}{0pt}%
\pgfpathmoveto{\pgfqpoint{3.905098in}{1.930426in}}%
\pgfpathcurveto{\pgfqpoint{3.913334in}{1.930426in}}{\pgfqpoint{3.921234in}{1.933698in}}{\pgfqpoint{3.927058in}{1.939522in}}%
\pgfpathcurveto{\pgfqpoint{3.932882in}{1.945346in}}{\pgfqpoint{3.936155in}{1.953246in}}{\pgfqpoint{3.936155in}{1.961482in}}%
\pgfpathcurveto{\pgfqpoint{3.936155in}{1.969719in}}{\pgfqpoint{3.932882in}{1.977619in}}{\pgfqpoint{3.927058in}{1.983443in}}%
\pgfpathcurveto{\pgfqpoint{3.921234in}{1.989267in}}{\pgfqpoint{3.913334in}{1.992539in}}{\pgfqpoint{3.905098in}{1.992539in}}%
\pgfpathcurveto{\pgfqpoint{3.896862in}{1.992539in}}{\pgfqpoint{3.888962in}{1.989267in}}{\pgfqpoint{3.883138in}{1.983443in}}%
\pgfpathcurveto{\pgfqpoint{3.877314in}{1.977619in}}{\pgfqpoint{3.874042in}{1.969719in}}{\pgfqpoint{3.874042in}{1.961482in}}%
\pgfpathcurveto{\pgfqpoint{3.874042in}{1.953246in}}{\pgfqpoint{3.877314in}{1.945346in}}{\pgfqpoint{3.883138in}{1.939522in}}%
\pgfpathcurveto{\pgfqpoint{3.888962in}{1.933698in}}{\pgfqpoint{3.896862in}{1.930426in}}{\pgfqpoint{3.905098in}{1.930426in}}%
\pgfpathclose%
\pgfusepath{stroke,fill}%
\end{pgfscope}%
\begin{pgfscope}%
\pgfpathrectangle{\pgfqpoint{3.793912in}{0.557870in}}{\pgfqpoint{2.446088in}{1.484734in}}%
\pgfusepath{clip}%
\pgfsetbuttcap%
\pgfsetroundjoin%
\definecolor{currentfill}{rgb}{0.298039,0.447059,0.690196}%
\pgfsetfillcolor{currentfill}%
\pgfsetlinewidth{1.003750pt}%
\definecolor{currentstroke}{rgb}{0.298039,0.447059,0.690196}%
\pgfsetstrokecolor{currentstroke}%
\pgfsetdash{}{0pt}%
\pgfpathmoveto{\pgfqpoint{3.905098in}{0.935150in}}%
\pgfpathcurveto{\pgfqpoint{3.913334in}{0.935150in}}{\pgfqpoint{3.921234in}{0.938422in}}{\pgfqpoint{3.927058in}{0.944246in}}%
\pgfpathcurveto{\pgfqpoint{3.932882in}{0.950070in}}{\pgfqpoint{3.936155in}{0.957970in}}{\pgfqpoint{3.936155in}{0.966206in}}%
\pgfpathcurveto{\pgfqpoint{3.936155in}{0.974442in}}{\pgfqpoint{3.932882in}{0.982343in}}{\pgfqpoint{3.927058in}{0.988166in}}%
\pgfpathcurveto{\pgfqpoint{3.921234in}{0.993990in}}{\pgfqpoint{3.913334in}{0.997263in}}{\pgfqpoint{3.905098in}{0.997263in}}%
\pgfpathcurveto{\pgfqpoint{3.896862in}{0.997263in}}{\pgfqpoint{3.888962in}{0.993990in}}{\pgfqpoint{3.883138in}{0.988166in}}%
\pgfpathcurveto{\pgfqpoint{3.877314in}{0.982343in}}{\pgfqpoint{3.874042in}{0.974442in}}{\pgfqpoint{3.874042in}{0.966206in}}%
\pgfpathcurveto{\pgfqpoint{3.874042in}{0.957970in}}{\pgfqpoint{3.877314in}{0.950070in}}{\pgfqpoint{3.883138in}{0.944246in}}%
\pgfpathcurveto{\pgfqpoint{3.888962in}{0.938422in}}{\pgfqpoint{3.896862in}{0.935150in}}{\pgfqpoint{3.905098in}{0.935150in}}%
\pgfpathclose%
\pgfusepath{stroke,fill}%
\end{pgfscope}%
\begin{pgfscope}%
\pgfpathrectangle{\pgfqpoint{3.793912in}{0.557870in}}{\pgfqpoint{2.446088in}{1.484734in}}%
\pgfusepath{clip}%
\pgfsetbuttcap%
\pgfsetroundjoin%
\definecolor{currentfill}{rgb}{0.298039,0.447059,0.690196}%
\pgfsetfillcolor{currentfill}%
\pgfsetlinewidth{1.003750pt}%
\definecolor{currentstroke}{rgb}{0.298039,0.447059,0.690196}%
\pgfsetstrokecolor{currentstroke}%
\pgfsetdash{}{0pt}%
\pgfpathmoveto{\pgfqpoint{3.905098in}{1.930426in}}%
\pgfpathcurveto{\pgfqpoint{3.913334in}{1.930426in}}{\pgfqpoint{3.921234in}{1.933698in}}{\pgfqpoint{3.927058in}{1.939522in}}%
\pgfpathcurveto{\pgfqpoint{3.932882in}{1.945346in}}{\pgfqpoint{3.936155in}{1.953246in}}{\pgfqpoint{3.936155in}{1.961482in}}%
\pgfpathcurveto{\pgfqpoint{3.936155in}{1.969719in}}{\pgfqpoint{3.932882in}{1.977619in}}{\pgfqpoint{3.927058in}{1.983443in}}%
\pgfpathcurveto{\pgfqpoint{3.921234in}{1.989267in}}{\pgfqpoint{3.913334in}{1.992539in}}{\pgfqpoint{3.905098in}{1.992539in}}%
\pgfpathcurveto{\pgfqpoint{3.896862in}{1.992539in}}{\pgfqpoint{3.888962in}{1.989267in}}{\pgfqpoint{3.883138in}{1.983443in}}%
\pgfpathcurveto{\pgfqpoint{3.877314in}{1.977619in}}{\pgfqpoint{3.874042in}{1.969719in}}{\pgfqpoint{3.874042in}{1.961482in}}%
\pgfpathcurveto{\pgfqpoint{3.874042in}{1.953246in}}{\pgfqpoint{3.877314in}{1.945346in}}{\pgfqpoint{3.883138in}{1.939522in}}%
\pgfpathcurveto{\pgfqpoint{3.888962in}{1.933698in}}{\pgfqpoint{3.896862in}{1.930426in}}{\pgfqpoint{3.905098in}{1.930426in}}%
\pgfpathclose%
\pgfusepath{stroke,fill}%
\end{pgfscope}%
\begin{pgfscope}%
\pgfpathrectangle{\pgfqpoint{3.793912in}{0.557870in}}{\pgfqpoint{2.446088in}{1.484734in}}%
\pgfusepath{clip}%
\pgfsetbuttcap%
\pgfsetroundjoin%
\definecolor{currentfill}{rgb}{0.298039,0.447059,0.690196}%
\pgfsetfillcolor{currentfill}%
\pgfsetlinewidth{1.003750pt}%
\definecolor{currentstroke}{rgb}{0.298039,0.447059,0.690196}%
\pgfsetstrokecolor{currentstroke}%
\pgfsetdash{}{0pt}%
\pgfpathmoveto{\pgfqpoint{3.905098in}{1.930426in}}%
\pgfpathcurveto{\pgfqpoint{3.913334in}{1.930426in}}{\pgfqpoint{3.921234in}{1.933698in}}{\pgfqpoint{3.927058in}{1.939522in}}%
\pgfpathcurveto{\pgfqpoint{3.932882in}{1.945346in}}{\pgfqpoint{3.936155in}{1.953246in}}{\pgfqpoint{3.936155in}{1.961482in}}%
\pgfpathcurveto{\pgfqpoint{3.936155in}{1.969719in}}{\pgfqpoint{3.932882in}{1.977619in}}{\pgfqpoint{3.927058in}{1.983443in}}%
\pgfpathcurveto{\pgfqpoint{3.921234in}{1.989267in}}{\pgfqpoint{3.913334in}{1.992539in}}{\pgfqpoint{3.905098in}{1.992539in}}%
\pgfpathcurveto{\pgfqpoint{3.896862in}{1.992539in}}{\pgfqpoint{3.888962in}{1.989267in}}{\pgfqpoint{3.883138in}{1.983443in}}%
\pgfpathcurveto{\pgfqpoint{3.877314in}{1.977619in}}{\pgfqpoint{3.874042in}{1.969719in}}{\pgfqpoint{3.874042in}{1.961482in}}%
\pgfpathcurveto{\pgfqpoint{3.874042in}{1.953246in}}{\pgfqpoint{3.877314in}{1.945346in}}{\pgfqpoint{3.883138in}{1.939522in}}%
\pgfpathcurveto{\pgfqpoint{3.888962in}{1.933698in}}{\pgfqpoint{3.896862in}{1.930426in}}{\pgfqpoint{3.905098in}{1.930426in}}%
\pgfpathclose%
\pgfusepath{stroke,fill}%
\end{pgfscope}%
\begin{pgfscope}%
\pgfpathrectangle{\pgfqpoint{3.793912in}{0.557870in}}{\pgfqpoint{2.446088in}{1.484734in}}%
\pgfusepath{clip}%
\pgfsetbuttcap%
\pgfsetroundjoin%
\definecolor{currentfill}{rgb}{0.298039,0.447059,0.690196}%
\pgfsetfillcolor{currentfill}%
\pgfsetlinewidth{1.003750pt}%
\definecolor{currentstroke}{rgb}{0.298039,0.447059,0.690196}%
\pgfsetstrokecolor{currentstroke}%
\pgfsetdash{}{0pt}%
\pgfpathmoveto{\pgfqpoint{4.414700in}{0.594302in}}%
\pgfpathcurveto{\pgfqpoint{4.422936in}{0.594302in}}{\pgfqpoint{4.430836in}{0.597574in}}{\pgfqpoint{4.436660in}{0.603398in}}%
\pgfpathcurveto{\pgfqpoint{4.442484in}{0.609222in}}{\pgfqpoint{4.445756in}{0.617122in}}{\pgfqpoint{4.445756in}{0.625358in}}%
\pgfpathcurveto{\pgfqpoint{4.445756in}{0.633594in}}{\pgfqpoint{4.442484in}{0.641495in}}{\pgfqpoint{4.436660in}{0.647318in}}%
\pgfpathcurveto{\pgfqpoint{4.430836in}{0.653142in}}{\pgfqpoint{4.422936in}{0.656415in}}{\pgfqpoint{4.414700in}{0.656415in}}%
\pgfpathcurveto{\pgfqpoint{4.406463in}{0.656415in}}{\pgfqpoint{4.398563in}{0.653142in}}{\pgfqpoint{4.392739in}{0.647318in}}%
\pgfpathcurveto{\pgfqpoint{4.386915in}{0.641495in}}{\pgfqpoint{4.383643in}{0.633594in}}{\pgfqpoint{4.383643in}{0.625358in}}%
\pgfpathcurveto{\pgfqpoint{4.383643in}{0.617122in}}{\pgfqpoint{4.386915in}{0.609222in}}{\pgfqpoint{4.392739in}{0.603398in}}%
\pgfpathcurveto{\pgfqpoint{4.398563in}{0.597574in}}{\pgfqpoint{4.406463in}{0.594302in}}{\pgfqpoint{4.414700in}{0.594302in}}%
\pgfpathclose%
\pgfusepath{stroke,fill}%
\end{pgfscope}%
\begin{pgfscope}%
\pgfpathrectangle{\pgfqpoint{3.793912in}{0.557870in}}{\pgfqpoint{2.446088in}{1.484734in}}%
\pgfusepath{clip}%
\pgfsetbuttcap%
\pgfsetroundjoin%
\definecolor{currentfill}{rgb}{0.298039,0.447059,0.690196}%
\pgfsetfillcolor{currentfill}%
\pgfsetlinewidth{1.003750pt}%
\definecolor{currentstroke}{rgb}{0.298039,0.447059,0.690196}%
\pgfsetstrokecolor{currentstroke}%
\pgfsetdash{}{0pt}%
\pgfpathmoveto{\pgfqpoint{3.905098in}{1.930426in}}%
\pgfpathcurveto{\pgfqpoint{3.913334in}{1.930426in}}{\pgfqpoint{3.921234in}{1.933698in}}{\pgfqpoint{3.927058in}{1.939522in}}%
\pgfpathcurveto{\pgfqpoint{3.932882in}{1.945346in}}{\pgfqpoint{3.936155in}{1.953246in}}{\pgfqpoint{3.936155in}{1.961482in}}%
\pgfpathcurveto{\pgfqpoint{3.936155in}{1.969719in}}{\pgfqpoint{3.932882in}{1.977619in}}{\pgfqpoint{3.927058in}{1.983443in}}%
\pgfpathcurveto{\pgfqpoint{3.921234in}{1.989267in}}{\pgfqpoint{3.913334in}{1.992539in}}{\pgfqpoint{3.905098in}{1.992539in}}%
\pgfpathcurveto{\pgfqpoint{3.896862in}{1.992539in}}{\pgfqpoint{3.888962in}{1.989267in}}{\pgfqpoint{3.883138in}{1.983443in}}%
\pgfpathcurveto{\pgfqpoint{3.877314in}{1.977619in}}{\pgfqpoint{3.874042in}{1.969719in}}{\pgfqpoint{3.874042in}{1.961482in}}%
\pgfpathcurveto{\pgfqpoint{3.874042in}{1.953246in}}{\pgfqpoint{3.877314in}{1.945346in}}{\pgfqpoint{3.883138in}{1.939522in}}%
\pgfpathcurveto{\pgfqpoint{3.888962in}{1.933698in}}{\pgfqpoint{3.896862in}{1.930426in}}{\pgfqpoint{3.905098in}{1.930426in}}%
\pgfpathclose%
\pgfusepath{stroke,fill}%
\end{pgfscope}%
\begin{pgfscope}%
\pgfpathrectangle{\pgfqpoint{3.793912in}{0.557870in}}{\pgfqpoint{2.446088in}{1.484734in}}%
\pgfusepath{clip}%
\pgfsetbuttcap%
\pgfsetroundjoin%
\definecolor{currentfill}{rgb}{0.298039,0.447059,0.690196}%
\pgfsetfillcolor{currentfill}%
\pgfsetlinewidth{1.003750pt}%
\definecolor{currentstroke}{rgb}{0.298039,0.447059,0.690196}%
\pgfsetstrokecolor{currentstroke}%
\pgfsetdash{}{0pt}%
\pgfpathmoveto{\pgfqpoint{5.364412in}{0.594302in}}%
\pgfpathcurveto{\pgfqpoint{5.372648in}{0.594302in}}{\pgfqpoint{5.380548in}{0.597574in}}{\pgfqpoint{5.386372in}{0.603398in}}%
\pgfpathcurveto{\pgfqpoint{5.392196in}{0.609222in}}{\pgfqpoint{5.395468in}{0.617122in}}{\pgfqpoint{5.395468in}{0.625358in}}%
\pgfpathcurveto{\pgfqpoint{5.395468in}{0.633594in}}{\pgfqpoint{5.392196in}{0.641495in}}{\pgfqpoint{5.386372in}{0.647318in}}%
\pgfpathcurveto{\pgfqpoint{5.380548in}{0.653142in}}{\pgfqpoint{5.372648in}{0.656415in}}{\pgfqpoint{5.364412in}{0.656415in}}%
\pgfpathcurveto{\pgfqpoint{5.356175in}{0.656415in}}{\pgfqpoint{5.348275in}{0.653142in}}{\pgfqpoint{5.342451in}{0.647318in}}%
\pgfpathcurveto{\pgfqpoint{5.336628in}{0.641495in}}{\pgfqpoint{5.333355in}{0.633594in}}{\pgfqpoint{5.333355in}{0.625358in}}%
\pgfpathcurveto{\pgfqpoint{5.333355in}{0.617122in}}{\pgfqpoint{5.336628in}{0.609222in}}{\pgfqpoint{5.342451in}{0.603398in}}%
\pgfpathcurveto{\pgfqpoint{5.348275in}{0.597574in}}{\pgfqpoint{5.356175in}{0.594302in}}{\pgfqpoint{5.364412in}{0.594302in}}%
\pgfpathclose%
\pgfusepath{stroke,fill}%
\end{pgfscope}%
\begin{pgfscope}%
\pgfpathrectangle{\pgfqpoint{3.793912in}{0.557870in}}{\pgfqpoint{2.446088in}{1.484734in}}%
\pgfusepath{clip}%
\pgfsetbuttcap%
\pgfsetroundjoin%
\definecolor{currentfill}{rgb}{0.298039,0.447059,0.690196}%
\pgfsetfillcolor{currentfill}%
\pgfsetlinewidth{1.003750pt}%
\definecolor{currentstroke}{rgb}{0.298039,0.447059,0.690196}%
\pgfsetstrokecolor{currentstroke}%
\pgfsetdash{}{0pt}%
\pgfpathmoveto{\pgfqpoint{3.905098in}{0.621570in}}%
\pgfpathcurveto{\pgfqpoint{3.913334in}{0.621570in}}{\pgfqpoint{3.921234in}{0.624842in}}{\pgfqpoint{3.927058in}{0.630666in}}%
\pgfpathcurveto{\pgfqpoint{3.932882in}{0.636490in}}{\pgfqpoint{3.936155in}{0.644390in}}{\pgfqpoint{3.936155in}{0.652626in}}%
\pgfpathcurveto{\pgfqpoint{3.936155in}{0.660862in}}{\pgfqpoint{3.932882in}{0.668762in}}{\pgfqpoint{3.927058in}{0.674586in}}%
\pgfpathcurveto{\pgfqpoint{3.921234in}{0.680410in}}{\pgfqpoint{3.913334in}{0.683683in}}{\pgfqpoint{3.905098in}{0.683683in}}%
\pgfpathcurveto{\pgfqpoint{3.896862in}{0.683683in}}{\pgfqpoint{3.888962in}{0.680410in}}{\pgfqpoint{3.883138in}{0.674586in}}%
\pgfpathcurveto{\pgfqpoint{3.877314in}{0.668762in}}{\pgfqpoint{3.874042in}{0.660862in}}{\pgfqpoint{3.874042in}{0.652626in}}%
\pgfpathcurveto{\pgfqpoint{3.874042in}{0.644390in}}{\pgfqpoint{3.877314in}{0.636490in}}{\pgfqpoint{3.883138in}{0.630666in}}%
\pgfpathcurveto{\pgfqpoint{3.888962in}{0.624842in}}{\pgfqpoint{3.896862in}{0.621570in}}{\pgfqpoint{3.905098in}{0.621570in}}%
\pgfpathclose%
\pgfusepath{stroke,fill}%
\end{pgfscope}%
\begin{pgfscope}%
\pgfpathrectangle{\pgfqpoint{3.793912in}{0.557870in}}{\pgfqpoint{2.446088in}{1.484734in}}%
\pgfusepath{clip}%
\pgfsetbuttcap%
\pgfsetroundjoin%
\definecolor{currentfill}{rgb}{0.298039,0.447059,0.690196}%
\pgfsetfillcolor{currentfill}%
\pgfsetlinewidth{1.003750pt}%
\definecolor{currentstroke}{rgb}{0.298039,0.447059,0.690196}%
\pgfsetstrokecolor{currentstroke}%
\pgfsetdash{}{0pt}%
\pgfpathmoveto{\pgfqpoint{5.341248in}{0.594302in}}%
\pgfpathcurveto{\pgfqpoint{5.349484in}{0.594302in}}{\pgfqpoint{5.357384in}{0.597574in}}{\pgfqpoint{5.363208in}{0.603398in}}%
\pgfpathcurveto{\pgfqpoint{5.369032in}{0.609222in}}{\pgfqpoint{5.372305in}{0.617122in}}{\pgfqpoint{5.372305in}{0.625358in}}%
\pgfpathcurveto{\pgfqpoint{5.372305in}{0.633594in}}{\pgfqpoint{5.369032in}{0.641495in}}{\pgfqpoint{5.363208in}{0.647318in}}%
\pgfpathcurveto{\pgfqpoint{5.357384in}{0.653142in}}{\pgfqpoint{5.349484in}{0.656415in}}{\pgfqpoint{5.341248in}{0.656415in}}%
\pgfpathcurveto{\pgfqpoint{5.333012in}{0.656415in}}{\pgfqpoint{5.325112in}{0.653142in}}{\pgfqpoint{5.319288in}{0.647318in}}%
\pgfpathcurveto{\pgfqpoint{5.313464in}{0.641495in}}{\pgfqpoint{5.310192in}{0.633594in}}{\pgfqpoint{5.310192in}{0.625358in}}%
\pgfpathcurveto{\pgfqpoint{5.310192in}{0.617122in}}{\pgfqpoint{5.313464in}{0.609222in}}{\pgfqpoint{5.319288in}{0.603398in}}%
\pgfpathcurveto{\pgfqpoint{5.325112in}{0.597574in}}{\pgfqpoint{5.333012in}{0.594302in}}{\pgfqpoint{5.341248in}{0.594302in}}%
\pgfpathclose%
\pgfusepath{stroke,fill}%
\end{pgfscope}%
\begin{pgfscope}%
\pgfpathrectangle{\pgfqpoint{3.793912in}{0.557870in}}{\pgfqpoint{2.446088in}{1.484734in}}%
\pgfusepath{clip}%
\pgfsetbuttcap%
\pgfsetroundjoin%
\definecolor{currentfill}{rgb}{0.298039,0.447059,0.690196}%
\pgfsetfillcolor{currentfill}%
\pgfsetlinewidth{1.003750pt}%
\definecolor{currentstroke}{rgb}{0.298039,0.447059,0.690196}%
\pgfsetstrokecolor{currentstroke}%
\pgfsetdash{}{0pt}%
\pgfpathmoveto{\pgfqpoint{3.905098in}{1.930426in}}%
\pgfpathcurveto{\pgfqpoint{3.913334in}{1.930426in}}{\pgfqpoint{3.921234in}{1.933698in}}{\pgfqpoint{3.927058in}{1.939522in}}%
\pgfpathcurveto{\pgfqpoint{3.932882in}{1.945346in}}{\pgfqpoint{3.936155in}{1.953246in}}{\pgfqpoint{3.936155in}{1.961482in}}%
\pgfpathcurveto{\pgfqpoint{3.936155in}{1.969719in}}{\pgfqpoint{3.932882in}{1.977619in}}{\pgfqpoint{3.927058in}{1.983443in}}%
\pgfpathcurveto{\pgfqpoint{3.921234in}{1.989267in}}{\pgfqpoint{3.913334in}{1.992539in}}{\pgfqpoint{3.905098in}{1.992539in}}%
\pgfpathcurveto{\pgfqpoint{3.896862in}{1.992539in}}{\pgfqpoint{3.888962in}{1.989267in}}{\pgfqpoint{3.883138in}{1.983443in}}%
\pgfpathcurveto{\pgfqpoint{3.877314in}{1.977619in}}{\pgfqpoint{3.874042in}{1.969719in}}{\pgfqpoint{3.874042in}{1.961482in}}%
\pgfpathcurveto{\pgfqpoint{3.874042in}{1.953246in}}{\pgfqpoint{3.877314in}{1.945346in}}{\pgfqpoint{3.883138in}{1.939522in}}%
\pgfpathcurveto{\pgfqpoint{3.888962in}{1.933698in}}{\pgfqpoint{3.896862in}{1.930426in}}{\pgfqpoint{3.905098in}{1.930426in}}%
\pgfpathclose%
\pgfusepath{stroke,fill}%
\end{pgfscope}%
\begin{pgfscope}%
\pgfpathrectangle{\pgfqpoint{3.793912in}{0.557870in}}{\pgfqpoint{2.446088in}{1.484734in}}%
\pgfusepath{clip}%
\pgfsetbuttcap%
\pgfsetroundjoin%
\definecolor{currentfill}{rgb}{0.298039,0.447059,0.690196}%
\pgfsetfillcolor{currentfill}%
\pgfsetlinewidth{1.003750pt}%
\definecolor{currentstroke}{rgb}{0.298039,0.447059,0.690196}%
\pgfsetstrokecolor{currentstroke}%
\pgfsetdash{}{0pt}%
\pgfpathmoveto{\pgfqpoint{3.905098in}{1.930426in}}%
\pgfpathcurveto{\pgfqpoint{3.913334in}{1.930426in}}{\pgfqpoint{3.921234in}{1.933698in}}{\pgfqpoint{3.927058in}{1.939522in}}%
\pgfpathcurveto{\pgfqpoint{3.932882in}{1.945346in}}{\pgfqpoint{3.936155in}{1.953246in}}{\pgfqpoint{3.936155in}{1.961482in}}%
\pgfpathcurveto{\pgfqpoint{3.936155in}{1.969719in}}{\pgfqpoint{3.932882in}{1.977619in}}{\pgfqpoint{3.927058in}{1.983443in}}%
\pgfpathcurveto{\pgfqpoint{3.921234in}{1.989267in}}{\pgfqpoint{3.913334in}{1.992539in}}{\pgfqpoint{3.905098in}{1.992539in}}%
\pgfpathcurveto{\pgfqpoint{3.896862in}{1.992539in}}{\pgfqpoint{3.888962in}{1.989267in}}{\pgfqpoint{3.883138in}{1.983443in}}%
\pgfpathcurveto{\pgfqpoint{3.877314in}{1.977619in}}{\pgfqpoint{3.874042in}{1.969719in}}{\pgfqpoint{3.874042in}{1.961482in}}%
\pgfpathcurveto{\pgfqpoint{3.874042in}{1.953246in}}{\pgfqpoint{3.877314in}{1.945346in}}{\pgfqpoint{3.883138in}{1.939522in}}%
\pgfpathcurveto{\pgfqpoint{3.888962in}{1.933698in}}{\pgfqpoint{3.896862in}{1.930426in}}{\pgfqpoint{3.905098in}{1.930426in}}%
\pgfpathclose%
\pgfusepath{stroke,fill}%
\end{pgfscope}%
\begin{pgfscope}%
\pgfpathrectangle{\pgfqpoint{3.793912in}{0.557870in}}{\pgfqpoint{2.446088in}{1.484734in}}%
\pgfusepath{clip}%
\pgfsetbuttcap%
\pgfsetroundjoin%
\definecolor{currentfill}{rgb}{0.298039,0.447059,0.690196}%
\pgfsetfillcolor{currentfill}%
\pgfsetlinewidth{1.003750pt}%
\definecolor{currentstroke}{rgb}{0.298039,0.447059,0.690196}%
\pgfsetstrokecolor{currentstroke}%
\pgfsetdash{}{0pt}%
\pgfpathmoveto{\pgfqpoint{3.905098in}{1.153292in}}%
\pgfpathcurveto{\pgfqpoint{3.913334in}{1.153292in}}{\pgfqpoint{3.921234in}{1.156565in}}{\pgfqpoint{3.927058in}{1.162389in}}%
\pgfpathcurveto{\pgfqpoint{3.932882in}{1.168213in}}{\pgfqpoint{3.936155in}{1.176113in}}{\pgfqpoint{3.936155in}{1.184349in}}%
\pgfpathcurveto{\pgfqpoint{3.936155in}{1.192585in}}{\pgfqpoint{3.932882in}{1.200485in}}{\pgfqpoint{3.927058in}{1.206309in}}%
\pgfpathcurveto{\pgfqpoint{3.921234in}{1.212133in}}{\pgfqpoint{3.913334in}{1.215405in}}{\pgfqpoint{3.905098in}{1.215405in}}%
\pgfpathcurveto{\pgfqpoint{3.896862in}{1.215405in}}{\pgfqpoint{3.888962in}{1.212133in}}{\pgfqpoint{3.883138in}{1.206309in}}%
\pgfpathcurveto{\pgfqpoint{3.877314in}{1.200485in}}{\pgfqpoint{3.874042in}{1.192585in}}{\pgfqpoint{3.874042in}{1.184349in}}%
\pgfpathcurveto{\pgfqpoint{3.874042in}{1.176113in}}{\pgfqpoint{3.877314in}{1.168213in}}{\pgfqpoint{3.883138in}{1.162389in}}%
\pgfpathcurveto{\pgfqpoint{3.888962in}{1.156565in}}{\pgfqpoint{3.896862in}{1.153292in}}{\pgfqpoint{3.905098in}{1.153292in}}%
\pgfpathclose%
\pgfusepath{stroke,fill}%
\end{pgfscope}%
\begin{pgfscope}%
\pgfpathrectangle{\pgfqpoint{3.793912in}{0.557870in}}{\pgfqpoint{2.446088in}{1.484734in}}%
\pgfusepath{clip}%
\pgfsetbuttcap%
\pgfsetroundjoin%
\definecolor{currentfill}{rgb}{0.298039,0.447059,0.690196}%
\pgfsetfillcolor{currentfill}%
\pgfsetlinewidth{1.003750pt}%
\definecolor{currentstroke}{rgb}{0.298039,0.447059,0.690196}%
\pgfsetstrokecolor{currentstroke}%
\pgfsetdash{}{0pt}%
\pgfpathmoveto{\pgfqpoint{5.225430in}{0.594302in}}%
\pgfpathcurveto{\pgfqpoint{5.233666in}{0.594302in}}{\pgfqpoint{5.241566in}{0.597574in}}{\pgfqpoint{5.247390in}{0.603398in}}%
\pgfpathcurveto{\pgfqpoint{5.253214in}{0.609222in}}{\pgfqpoint{5.256486in}{0.617122in}}{\pgfqpoint{5.256486in}{0.625358in}}%
\pgfpathcurveto{\pgfqpoint{5.256486in}{0.633594in}}{\pgfqpoint{5.253214in}{0.641495in}}{\pgfqpoint{5.247390in}{0.647318in}}%
\pgfpathcurveto{\pgfqpoint{5.241566in}{0.653142in}}{\pgfqpoint{5.233666in}{0.656415in}}{\pgfqpoint{5.225430in}{0.656415in}}%
\pgfpathcurveto{\pgfqpoint{5.217193in}{0.656415in}}{\pgfqpoint{5.209293in}{0.653142in}}{\pgfqpoint{5.203469in}{0.647318in}}%
\pgfpathcurveto{\pgfqpoint{5.197645in}{0.641495in}}{\pgfqpoint{5.194373in}{0.633594in}}{\pgfqpoint{5.194373in}{0.625358in}}%
\pgfpathcurveto{\pgfqpoint{5.194373in}{0.617122in}}{\pgfqpoint{5.197645in}{0.609222in}}{\pgfqpoint{5.203469in}{0.603398in}}%
\pgfpathcurveto{\pgfqpoint{5.209293in}{0.597574in}}{\pgfqpoint{5.217193in}{0.594302in}}{\pgfqpoint{5.225430in}{0.594302in}}%
\pgfpathclose%
\pgfusepath{stroke,fill}%
\end{pgfscope}%
\begin{pgfscope}%
\pgfpathrectangle{\pgfqpoint{3.793912in}{0.557870in}}{\pgfqpoint{2.446088in}{1.484734in}}%
\pgfusepath{clip}%
\pgfsetbuttcap%
\pgfsetroundjoin%
\definecolor{currentfill}{rgb}{0.298039,0.447059,0.690196}%
\pgfsetfillcolor{currentfill}%
\pgfsetlinewidth{1.003750pt}%
\definecolor{currentstroke}{rgb}{0.298039,0.447059,0.690196}%
\pgfsetstrokecolor{currentstroke}%
\pgfsetdash{}{0pt}%
\pgfpathmoveto{\pgfqpoint{3.905098in}{1.330533in}}%
\pgfpathcurveto{\pgfqpoint{3.913334in}{1.330533in}}{\pgfqpoint{3.921234in}{1.333806in}}{\pgfqpoint{3.927058in}{1.339630in}}%
\pgfpathcurveto{\pgfqpoint{3.932882in}{1.345454in}}{\pgfqpoint{3.936155in}{1.353354in}}{\pgfqpoint{3.936155in}{1.361590in}}%
\pgfpathcurveto{\pgfqpoint{3.936155in}{1.369826in}}{\pgfqpoint{3.932882in}{1.377726in}}{\pgfqpoint{3.927058in}{1.383550in}}%
\pgfpathcurveto{\pgfqpoint{3.921234in}{1.389374in}}{\pgfqpoint{3.913334in}{1.392646in}}{\pgfqpoint{3.905098in}{1.392646in}}%
\pgfpathcurveto{\pgfqpoint{3.896862in}{1.392646in}}{\pgfqpoint{3.888962in}{1.389374in}}{\pgfqpoint{3.883138in}{1.383550in}}%
\pgfpathcurveto{\pgfqpoint{3.877314in}{1.377726in}}{\pgfqpoint{3.874042in}{1.369826in}}{\pgfqpoint{3.874042in}{1.361590in}}%
\pgfpathcurveto{\pgfqpoint{3.874042in}{1.353354in}}{\pgfqpoint{3.877314in}{1.345454in}}{\pgfqpoint{3.883138in}{1.339630in}}%
\pgfpathcurveto{\pgfqpoint{3.888962in}{1.333806in}}{\pgfqpoint{3.896862in}{1.330533in}}{\pgfqpoint{3.905098in}{1.330533in}}%
\pgfpathclose%
\pgfusepath{stroke,fill}%
\end{pgfscope}%
\begin{pgfscope}%
\pgfpathrectangle{\pgfqpoint{3.793912in}{0.557870in}}{\pgfqpoint{2.446088in}{1.484734in}}%
\pgfusepath{clip}%
\pgfsetbuttcap%
\pgfsetroundjoin%
\definecolor{currentfill}{rgb}{0.298039,0.447059,0.690196}%
\pgfsetfillcolor{currentfill}%
\pgfsetlinewidth{1.003750pt}%
\definecolor{currentstroke}{rgb}{0.298039,0.447059,0.690196}%
\pgfsetstrokecolor{currentstroke}%
\pgfsetdash{}{0pt}%
\pgfpathmoveto{\pgfqpoint{3.905098in}{1.385069in}}%
\pgfpathcurveto{\pgfqpoint{3.913334in}{1.385069in}}{\pgfqpoint{3.921234in}{1.388341in}}{\pgfqpoint{3.927058in}{1.394165in}}%
\pgfpathcurveto{\pgfqpoint{3.932882in}{1.399989in}}{\pgfqpoint{3.936155in}{1.407889in}}{\pgfqpoint{3.936155in}{1.416126in}}%
\pgfpathcurveto{\pgfqpoint{3.936155in}{1.424362in}}{\pgfqpoint{3.932882in}{1.432262in}}{\pgfqpoint{3.927058in}{1.438086in}}%
\pgfpathcurveto{\pgfqpoint{3.921234in}{1.443910in}}{\pgfqpoint{3.913334in}{1.447182in}}{\pgfqpoint{3.905098in}{1.447182in}}%
\pgfpathcurveto{\pgfqpoint{3.896862in}{1.447182in}}{\pgfqpoint{3.888962in}{1.443910in}}{\pgfqpoint{3.883138in}{1.438086in}}%
\pgfpathcurveto{\pgfqpoint{3.877314in}{1.432262in}}{\pgfqpoint{3.874042in}{1.424362in}}{\pgfqpoint{3.874042in}{1.416126in}}%
\pgfpathcurveto{\pgfqpoint{3.874042in}{1.407889in}}{\pgfqpoint{3.877314in}{1.399989in}}{\pgfqpoint{3.883138in}{1.394165in}}%
\pgfpathcurveto{\pgfqpoint{3.888962in}{1.388341in}}{\pgfqpoint{3.896862in}{1.385069in}}{\pgfqpoint{3.905098in}{1.385069in}}%
\pgfpathclose%
\pgfusepath{stroke,fill}%
\end{pgfscope}%
\begin{pgfscope}%
\pgfpathrectangle{\pgfqpoint{3.793912in}{0.557870in}}{\pgfqpoint{2.446088in}{1.484734in}}%
\pgfusepath{clip}%
\pgfsetbuttcap%
\pgfsetroundjoin%
\definecolor{currentfill}{rgb}{0.298039,0.447059,0.690196}%
\pgfsetfillcolor{currentfill}%
\pgfsetlinewidth{1.003750pt}%
\definecolor{currentstroke}{rgb}{0.298039,0.447059,0.690196}%
\pgfsetstrokecolor{currentstroke}%
\pgfsetdash{}{0pt}%
\pgfpathmoveto{\pgfqpoint{4.901138in}{0.594302in}}%
\pgfpathcurveto{\pgfqpoint{4.909374in}{0.594302in}}{\pgfqpoint{4.917274in}{0.597574in}}{\pgfqpoint{4.923098in}{0.603398in}}%
\pgfpathcurveto{\pgfqpoint{4.928922in}{0.609222in}}{\pgfqpoint{4.932194in}{0.617122in}}{\pgfqpoint{4.932194in}{0.625358in}}%
\pgfpathcurveto{\pgfqpoint{4.932194in}{0.633594in}}{\pgfqpoint{4.928922in}{0.641495in}}{\pgfqpoint{4.923098in}{0.647318in}}%
\pgfpathcurveto{\pgfqpoint{4.917274in}{0.653142in}}{\pgfqpoint{4.909374in}{0.656415in}}{\pgfqpoint{4.901138in}{0.656415in}}%
\pgfpathcurveto{\pgfqpoint{4.892901in}{0.656415in}}{\pgfqpoint{4.885001in}{0.653142in}}{\pgfqpoint{4.879177in}{0.647318in}}%
\pgfpathcurveto{\pgfqpoint{4.873353in}{0.641495in}}{\pgfqpoint{4.870081in}{0.633594in}}{\pgfqpoint{4.870081in}{0.625358in}}%
\pgfpathcurveto{\pgfqpoint{4.870081in}{0.617122in}}{\pgfqpoint{4.873353in}{0.609222in}}{\pgfqpoint{4.879177in}{0.603398in}}%
\pgfpathcurveto{\pgfqpoint{4.885001in}{0.597574in}}{\pgfqpoint{4.892901in}{0.594302in}}{\pgfqpoint{4.901138in}{0.594302in}}%
\pgfpathclose%
\pgfusepath{stroke,fill}%
\end{pgfscope}%
\begin{pgfscope}%
\pgfsetrectcap%
\pgfsetmiterjoin%
\pgfsetlinewidth{1.254687pt}%
\definecolor{currentstroke}{rgb}{1.000000,1.000000,1.000000}%
\pgfsetstrokecolor{currentstroke}%
\pgfsetdash{}{0pt}%
\pgfpathmoveto{\pgfqpoint{3.793912in}{0.557870in}}%
\pgfpathlineto{\pgfqpoint{3.793912in}{2.042604in}}%
\pgfusepath{stroke}%
\end{pgfscope}%
\begin{pgfscope}%
\pgfsetrectcap%
\pgfsetmiterjoin%
\pgfsetlinewidth{1.254687pt}%
\definecolor{currentstroke}{rgb}{1.000000,1.000000,1.000000}%
\pgfsetstrokecolor{currentstroke}%
\pgfsetdash{}{0pt}%
\pgfpathmoveto{\pgfqpoint{6.240000in}{0.557870in}}%
\pgfpathlineto{\pgfqpoint{6.240000in}{2.042604in}}%
\pgfusepath{stroke}%
\end{pgfscope}%
\begin{pgfscope}%
\pgfsetrectcap%
\pgfsetmiterjoin%
\pgfsetlinewidth{1.254687pt}%
\definecolor{currentstroke}{rgb}{1.000000,1.000000,1.000000}%
\pgfsetstrokecolor{currentstroke}%
\pgfsetdash{}{0pt}%
\pgfpathmoveto{\pgfqpoint{3.793912in}{0.557870in}}%
\pgfpathlineto{\pgfqpoint{6.240000in}{0.557870in}}%
\pgfusepath{stroke}%
\end{pgfscope}%
\begin{pgfscope}%
\pgfsetrectcap%
\pgfsetmiterjoin%
\pgfsetlinewidth{1.254687pt}%
\definecolor{currentstroke}{rgb}{1.000000,1.000000,1.000000}%
\pgfsetstrokecolor{currentstroke}%
\pgfsetdash{}{0pt}%
\pgfpathmoveto{\pgfqpoint{3.793912in}{2.042604in}}%
\pgfpathlineto{\pgfqpoint{6.240000in}{2.042604in}}%
\pgfusepath{stroke}%
\end{pgfscope}%
\begin{pgfscope}%
\definecolor{textcolor}{rgb}{0.150000,0.150000,0.150000}%
\pgfsetstrokecolor{textcolor}%
\pgfsetfillcolor{textcolor}%
\pgftext[x=5.016956in,y=2.125938in,,base]{\color{textcolor}\sffamily\fontsize{11.000000}{13.200000}\selectfont (b)}%
\end{pgfscope}%
\end{pgfpicture}%
\makeatother%
\endgroup%

    \caption{(a) Distribution plot of DOR of all TSC methods evaluated at two cluster centers when applied to classify heart failure.
             (b) Scatter plot of the same methods sensitivity-, and specificity-scores.}
    \label{fig:tsc_hf_dor_sens_spec_dist}
\end{figure}

Figure \ref{fig:tsc_hf_dor_sens_spec_dist}a shows that the DOR is rounded down to zero for all the two-cluster-center methods, meaning that the ratio of $TP \times TN$ is small compared to $FP \times FN$. 
From the scatterplot in figure \ref{fig:tsc_hf_dor_sens_spec_dist}b one can see that the distribution of sensitivity, and specificity are quite widespread.
Some sensitivity scores are as high as 1, and some specificity scores as high as high as $0.90$
However, from the scatterplot one can also ascertain that the methods that score above zero in sensitivity score zero in specificity, and vice versa. 
This observation explains why the DOR is zero for all the TSC methods evaluated at two cluster centers, 
and means that the TSC methods evaluated at two cluster centers do not perform well at identifying heart failure among patients. \bigskip

\begin{figure}[htb]
    \centering
    % \includegraphics[width=\textwidth]{results/tsc_hf_ari.png}
    %% Creator: Matplotlib, PGF backend
%%
%% To include the figure in your LaTeX document, write
%%   \input{<filename>.pgf}
%%
%% Make sure the required packages are loaded in your preamble
%%   \usepackage{pgf}
%%
%% Figures using additional raster images can only be included by \input if
%% they are in the same directory as the main LaTeX file. For loading figures
%% from other directories you can use the `import` package
%%   \usepackage{import}
%% and then include the figures with
%%   \import{<path to file>}{<filename>.pgf}
%%
%% Matplotlib used the following preamble
%%
\begingroup%
\makeatletter%
\begin{pgfpicture}%
\pgfpathrectangle{\pgfpointorigin}{\pgfqpoint{6.340000in}{2.340000in}}%
\pgfusepath{use as bounding box, clip}%
\begin{pgfscope}%
\pgfsetbuttcap%
\pgfsetmiterjoin%
\definecolor{currentfill}{rgb}{1.000000,1.000000,1.000000}%
\pgfsetfillcolor{currentfill}%
\pgfsetlinewidth{0.000000pt}%
\definecolor{currentstroke}{rgb}{1.000000,1.000000,1.000000}%
\pgfsetstrokecolor{currentstroke}%
\pgfsetdash{}{0pt}%
\pgfpathmoveto{\pgfqpoint{0.000000in}{-0.000000in}}%
\pgfpathlineto{\pgfqpoint{6.340000in}{-0.000000in}}%
\pgfpathlineto{\pgfqpoint{6.340000in}{2.340000in}}%
\pgfpathlineto{\pgfqpoint{0.000000in}{2.340000in}}%
\pgfpathclose%
\pgfusepath{fill}%
\end{pgfscope}%
\begin{pgfscope}%
\pgfsetbuttcap%
\pgfsetmiterjoin%
\definecolor{currentfill}{rgb}{0.917647,0.917647,0.949020}%
\pgfsetfillcolor{currentfill}%
\pgfsetlinewidth{0.000000pt}%
\definecolor{currentstroke}{rgb}{0.000000,0.000000,0.000000}%
\pgfsetstrokecolor{currentstroke}%
\pgfsetstrokeopacity{0.000000}%
\pgfsetdash{}{0pt}%
\pgfpathmoveto{\pgfqpoint{0.726852in}{0.557870in}}%
\pgfpathlineto{\pgfqpoint{6.240000in}{0.557870in}}%
\pgfpathlineto{\pgfqpoint{6.240000in}{2.240000in}}%
\pgfpathlineto{\pgfqpoint{0.726852in}{2.240000in}}%
\pgfpathclose%
\pgfusepath{fill}%
\end{pgfscope}%
\begin{pgfscope}%
\pgfpathrectangle{\pgfqpoint{0.726852in}{0.557870in}}{\pgfqpoint{5.513148in}{1.682130in}}%
\pgfusepath{clip}%
\pgfsetroundcap%
\pgfsetroundjoin%
\pgfsetlinewidth{1.003750pt}%
\definecolor{currentstroke}{rgb}{1.000000,1.000000,1.000000}%
\pgfsetstrokecolor{currentstroke}%
\pgfsetdash{}{0pt}%
\pgfpathmoveto{\pgfqpoint{1.096294in}{0.557870in}}%
\pgfpathlineto{\pgfqpoint{1.096294in}{2.240000in}}%
\pgfusepath{stroke}%
\end{pgfscope}%
\begin{pgfscope}%
\definecolor{textcolor}{rgb}{0.150000,0.150000,0.150000}%
\pgfsetstrokecolor{textcolor}%
\pgfsetfillcolor{textcolor}%
\pgftext[x=1.096294in,y=0.425926in,,top]{\color{textcolor}\sffamily\fontsize{11.000000}{13.200000}\selectfont \(\displaystyle 0.00\)}%
\end{pgfscope}%
\begin{pgfscope}%
\pgfpathrectangle{\pgfqpoint{0.726852in}{0.557870in}}{\pgfqpoint{5.513148in}{1.682130in}}%
\pgfusepath{clip}%
\pgfsetroundcap%
\pgfsetroundjoin%
\pgfsetlinewidth{1.003750pt}%
\definecolor{currentstroke}{rgb}{1.000000,1.000000,1.000000}%
\pgfsetstrokecolor{currentstroke}%
\pgfsetdash{}{0pt}%
\pgfpathmoveto{\pgfqpoint{2.059047in}{0.557870in}}%
\pgfpathlineto{\pgfqpoint{2.059047in}{2.240000in}}%
\pgfusepath{stroke}%
\end{pgfscope}%
\begin{pgfscope}%
\definecolor{textcolor}{rgb}{0.150000,0.150000,0.150000}%
\pgfsetstrokecolor{textcolor}%
\pgfsetfillcolor{textcolor}%
\pgftext[x=2.059047in,y=0.425926in,,top]{\color{textcolor}\sffamily\fontsize{11.000000}{13.200000}\selectfont \(\displaystyle 0.05\)}%
\end{pgfscope}%
\begin{pgfscope}%
\pgfpathrectangle{\pgfqpoint{0.726852in}{0.557870in}}{\pgfqpoint{5.513148in}{1.682130in}}%
\pgfusepath{clip}%
\pgfsetroundcap%
\pgfsetroundjoin%
\pgfsetlinewidth{1.003750pt}%
\definecolor{currentstroke}{rgb}{1.000000,1.000000,1.000000}%
\pgfsetstrokecolor{currentstroke}%
\pgfsetdash{}{0pt}%
\pgfpathmoveto{\pgfqpoint{3.021800in}{0.557870in}}%
\pgfpathlineto{\pgfqpoint{3.021800in}{2.240000in}}%
\pgfusepath{stroke}%
\end{pgfscope}%
\begin{pgfscope}%
\definecolor{textcolor}{rgb}{0.150000,0.150000,0.150000}%
\pgfsetstrokecolor{textcolor}%
\pgfsetfillcolor{textcolor}%
\pgftext[x=3.021800in,y=0.425926in,,top]{\color{textcolor}\sffamily\fontsize{11.000000}{13.200000}\selectfont \(\displaystyle 0.10\)}%
\end{pgfscope}%
\begin{pgfscope}%
\pgfpathrectangle{\pgfqpoint{0.726852in}{0.557870in}}{\pgfqpoint{5.513148in}{1.682130in}}%
\pgfusepath{clip}%
\pgfsetroundcap%
\pgfsetroundjoin%
\pgfsetlinewidth{1.003750pt}%
\definecolor{currentstroke}{rgb}{1.000000,1.000000,1.000000}%
\pgfsetstrokecolor{currentstroke}%
\pgfsetdash{}{0pt}%
\pgfpathmoveto{\pgfqpoint{3.984552in}{0.557870in}}%
\pgfpathlineto{\pgfqpoint{3.984552in}{2.240000in}}%
\pgfusepath{stroke}%
\end{pgfscope}%
\begin{pgfscope}%
\definecolor{textcolor}{rgb}{0.150000,0.150000,0.150000}%
\pgfsetstrokecolor{textcolor}%
\pgfsetfillcolor{textcolor}%
\pgftext[x=3.984552in,y=0.425926in,,top]{\color{textcolor}\sffamily\fontsize{11.000000}{13.200000}\selectfont \(\displaystyle 0.15\)}%
\end{pgfscope}%
\begin{pgfscope}%
\pgfpathrectangle{\pgfqpoint{0.726852in}{0.557870in}}{\pgfqpoint{5.513148in}{1.682130in}}%
\pgfusepath{clip}%
\pgfsetroundcap%
\pgfsetroundjoin%
\pgfsetlinewidth{1.003750pt}%
\definecolor{currentstroke}{rgb}{1.000000,1.000000,1.000000}%
\pgfsetstrokecolor{currentstroke}%
\pgfsetdash{}{0pt}%
\pgfpathmoveto{\pgfqpoint{4.947305in}{0.557870in}}%
\pgfpathlineto{\pgfqpoint{4.947305in}{2.240000in}}%
\pgfusepath{stroke}%
\end{pgfscope}%
\begin{pgfscope}%
\definecolor{textcolor}{rgb}{0.150000,0.150000,0.150000}%
\pgfsetstrokecolor{textcolor}%
\pgfsetfillcolor{textcolor}%
\pgftext[x=4.947305in,y=0.425926in,,top]{\color{textcolor}\sffamily\fontsize{11.000000}{13.200000}\selectfont \(\displaystyle 0.20\)}%
\end{pgfscope}%
\begin{pgfscope}%
\pgfpathrectangle{\pgfqpoint{0.726852in}{0.557870in}}{\pgfqpoint{5.513148in}{1.682130in}}%
\pgfusepath{clip}%
\pgfsetroundcap%
\pgfsetroundjoin%
\pgfsetlinewidth{1.003750pt}%
\definecolor{currentstroke}{rgb}{1.000000,1.000000,1.000000}%
\pgfsetstrokecolor{currentstroke}%
\pgfsetdash{}{0pt}%
\pgfpathmoveto{\pgfqpoint{5.910058in}{0.557870in}}%
\pgfpathlineto{\pgfqpoint{5.910058in}{2.240000in}}%
\pgfusepath{stroke}%
\end{pgfscope}%
\begin{pgfscope}%
\definecolor{textcolor}{rgb}{0.150000,0.150000,0.150000}%
\pgfsetstrokecolor{textcolor}%
\pgfsetfillcolor{textcolor}%
\pgftext[x=5.910058in,y=0.425926in,,top]{\color{textcolor}\sffamily\fontsize{11.000000}{13.200000}\selectfont \(\displaystyle 0.25\)}%
\end{pgfscope}%
\begin{pgfscope}%
\definecolor{textcolor}{rgb}{0.150000,0.150000,0.150000}%
\pgfsetstrokecolor{textcolor}%
\pgfsetfillcolor{textcolor}%
\pgftext[x=3.483426in,y=0.235185in,,top]{\color{textcolor}\sffamily\fontsize{11.000000}{13.200000}\selectfont ARI}%
\end{pgfscope}%
\begin{pgfscope}%
\pgfpathrectangle{\pgfqpoint{0.726852in}{0.557870in}}{\pgfqpoint{5.513148in}{1.682130in}}%
\pgfusepath{clip}%
\pgfsetroundcap%
\pgfsetroundjoin%
\pgfsetlinewidth{1.003750pt}%
\definecolor{currentstroke}{rgb}{1.000000,1.000000,1.000000}%
\pgfsetstrokecolor{currentstroke}%
\pgfsetdash{}{0pt}%
\pgfpathmoveto{\pgfqpoint{0.726852in}{0.557870in}}%
\pgfpathlineto{\pgfqpoint{6.240000in}{0.557870in}}%
\pgfusepath{stroke}%
\end{pgfscope}%
\begin{pgfscope}%
\definecolor{textcolor}{rgb}{0.150000,0.150000,0.150000}%
\pgfsetstrokecolor{textcolor}%
\pgfsetfillcolor{textcolor}%
\pgftext[x=0.518866in,y=0.505064in,left,base]{\color{textcolor}\sffamily\fontsize{11.000000}{13.200000}\selectfont \(\displaystyle 0\)}%
\end{pgfscope}%
\begin{pgfscope}%
\pgfpathrectangle{\pgfqpoint{0.726852in}{0.557870in}}{\pgfqpoint{5.513148in}{1.682130in}}%
\pgfusepath{clip}%
\pgfsetroundcap%
\pgfsetroundjoin%
\pgfsetlinewidth{1.003750pt}%
\definecolor{currentstroke}{rgb}{1.000000,1.000000,1.000000}%
\pgfsetstrokecolor{currentstroke}%
\pgfsetdash{}{0pt}%
\pgfpathmoveto{\pgfqpoint{0.726852in}{0.911051in}}%
\pgfpathlineto{\pgfqpoint{6.240000in}{0.911051in}}%
\pgfusepath{stroke}%
\end{pgfscope}%
\begin{pgfscope}%
\definecolor{textcolor}{rgb}{0.150000,0.150000,0.150000}%
\pgfsetstrokecolor{textcolor}%
\pgfsetfillcolor{textcolor}%
\pgftext[x=0.366782in,y=0.858244in,left,base]{\color{textcolor}\sffamily\fontsize{11.000000}{13.200000}\selectfont \(\displaystyle 250\)}%
\end{pgfscope}%
\begin{pgfscope}%
\pgfpathrectangle{\pgfqpoint{0.726852in}{0.557870in}}{\pgfqpoint{5.513148in}{1.682130in}}%
\pgfusepath{clip}%
\pgfsetroundcap%
\pgfsetroundjoin%
\pgfsetlinewidth{1.003750pt}%
\definecolor{currentstroke}{rgb}{1.000000,1.000000,1.000000}%
\pgfsetstrokecolor{currentstroke}%
\pgfsetdash{}{0pt}%
\pgfpathmoveto{\pgfqpoint{0.726852in}{1.264232in}}%
\pgfpathlineto{\pgfqpoint{6.240000in}{1.264232in}}%
\pgfusepath{stroke}%
\end{pgfscope}%
\begin{pgfscope}%
\definecolor{textcolor}{rgb}{0.150000,0.150000,0.150000}%
\pgfsetstrokecolor{textcolor}%
\pgfsetfillcolor{textcolor}%
\pgftext[x=0.366782in,y=1.211425in,left,base]{\color{textcolor}\sffamily\fontsize{11.000000}{13.200000}\selectfont \(\displaystyle 500\)}%
\end{pgfscope}%
\begin{pgfscope}%
\pgfpathrectangle{\pgfqpoint{0.726852in}{0.557870in}}{\pgfqpoint{5.513148in}{1.682130in}}%
\pgfusepath{clip}%
\pgfsetroundcap%
\pgfsetroundjoin%
\pgfsetlinewidth{1.003750pt}%
\definecolor{currentstroke}{rgb}{1.000000,1.000000,1.000000}%
\pgfsetstrokecolor{currentstroke}%
\pgfsetdash{}{0pt}%
\pgfpathmoveto{\pgfqpoint{0.726852in}{1.617413in}}%
\pgfpathlineto{\pgfqpoint{6.240000in}{1.617413in}}%
\pgfusepath{stroke}%
\end{pgfscope}%
\begin{pgfscope}%
\definecolor{textcolor}{rgb}{0.150000,0.150000,0.150000}%
\pgfsetstrokecolor{textcolor}%
\pgfsetfillcolor{textcolor}%
\pgftext[x=0.366782in,y=1.564606in,left,base]{\color{textcolor}\sffamily\fontsize{11.000000}{13.200000}\selectfont \(\displaystyle 750\)}%
\end{pgfscope}%
\begin{pgfscope}%
\pgfpathrectangle{\pgfqpoint{0.726852in}{0.557870in}}{\pgfqpoint{5.513148in}{1.682130in}}%
\pgfusepath{clip}%
\pgfsetroundcap%
\pgfsetroundjoin%
\pgfsetlinewidth{1.003750pt}%
\definecolor{currentstroke}{rgb}{1.000000,1.000000,1.000000}%
\pgfsetstrokecolor{currentstroke}%
\pgfsetdash{}{0pt}%
\pgfpathmoveto{\pgfqpoint{0.726852in}{1.970594in}}%
\pgfpathlineto{\pgfqpoint{6.240000in}{1.970594in}}%
\pgfusepath{stroke}%
\end{pgfscope}%
\begin{pgfscope}%
\definecolor{textcolor}{rgb}{0.150000,0.150000,0.150000}%
\pgfsetstrokecolor{textcolor}%
\pgfsetfillcolor{textcolor}%
\pgftext[x=0.290741in,y=1.917787in,left,base]{\color{textcolor}\sffamily\fontsize{11.000000}{13.200000}\selectfont \(\displaystyle 1000\)}%
\end{pgfscope}%
\begin{pgfscope}%
\definecolor{textcolor}{rgb}{0.150000,0.150000,0.150000}%
\pgfsetstrokecolor{textcolor}%
\pgfsetfillcolor{textcolor}%
\pgftext[x=0.235185in,y=1.398935in,,bottom,rotate=90.000000]{\color{textcolor}\sffamily\fontsize{11.000000}{13.200000}\selectfont Occurance}%
\end{pgfscope}%
\begin{pgfscope}%
\pgfpathrectangle{\pgfqpoint{0.726852in}{0.557870in}}{\pgfqpoint{5.513148in}{1.682130in}}%
\pgfusepath{clip}%
\pgfsetbuttcap%
\pgfsetmiterjoin%
\definecolor{currentfill}{rgb}{0.298039,0.447059,0.690196}%
\pgfsetfillcolor{currentfill}%
\pgfsetfillopacity{0.400000}%
\pgfsetlinewidth{1.003750pt}%
\definecolor{currentstroke}{rgb}{1.000000,1.000000,1.000000}%
\pgfsetstrokecolor{currentstroke}%
\pgfsetstrokeopacity{0.400000}%
\pgfsetdash{}{0pt}%
\pgfpathmoveto{\pgfqpoint{0.977450in}{0.557870in}}%
\pgfpathlineto{\pgfqpoint{1.177928in}{0.557870in}}%
\pgfpathlineto{\pgfqpoint{1.177928in}{2.159899in}}%
\pgfpathlineto{\pgfqpoint{0.977450in}{2.159899in}}%
\pgfpathclose%
\pgfusepath{stroke,fill}%
\end{pgfscope}%
\begin{pgfscope}%
\pgfpathrectangle{\pgfqpoint{0.726852in}{0.557870in}}{\pgfqpoint{5.513148in}{1.682130in}}%
\pgfusepath{clip}%
\pgfsetbuttcap%
\pgfsetmiterjoin%
\definecolor{currentfill}{rgb}{0.298039,0.447059,0.690196}%
\pgfsetfillcolor{currentfill}%
\pgfsetfillopacity{0.400000}%
\pgfsetlinewidth{1.003750pt}%
\definecolor{currentstroke}{rgb}{1.000000,1.000000,1.000000}%
\pgfsetstrokecolor{currentstroke}%
\pgfsetstrokeopacity{0.400000}%
\pgfsetdash{}{0pt}%
\pgfpathmoveto{\pgfqpoint{1.177928in}{0.557870in}}%
\pgfpathlineto{\pgfqpoint{1.378406in}{0.557870in}}%
\pgfpathlineto{\pgfqpoint{1.378406in}{0.864431in}}%
\pgfpathlineto{\pgfqpoint{1.177928in}{0.864431in}}%
\pgfpathclose%
\pgfusepath{stroke,fill}%
\end{pgfscope}%
\begin{pgfscope}%
\pgfpathrectangle{\pgfqpoint{0.726852in}{0.557870in}}{\pgfqpoint{5.513148in}{1.682130in}}%
\pgfusepath{clip}%
\pgfsetbuttcap%
\pgfsetmiterjoin%
\definecolor{currentfill}{rgb}{0.298039,0.447059,0.690196}%
\pgfsetfillcolor{currentfill}%
\pgfsetfillopacity{0.400000}%
\pgfsetlinewidth{1.003750pt}%
\definecolor{currentstroke}{rgb}{1.000000,1.000000,1.000000}%
\pgfsetstrokecolor{currentstroke}%
\pgfsetstrokeopacity{0.400000}%
\pgfsetdash{}{0pt}%
\pgfpathmoveto{\pgfqpoint{1.378406in}{0.557870in}}%
\pgfpathlineto{\pgfqpoint{1.578884in}{0.557870in}}%
\pgfpathlineto{\pgfqpoint{1.578884in}{0.752826in}}%
\pgfpathlineto{\pgfqpoint{1.378406in}{0.752826in}}%
\pgfpathclose%
\pgfusepath{stroke,fill}%
\end{pgfscope}%
\begin{pgfscope}%
\pgfpathrectangle{\pgfqpoint{0.726852in}{0.557870in}}{\pgfqpoint{5.513148in}{1.682130in}}%
\pgfusepath{clip}%
\pgfsetbuttcap%
\pgfsetmiterjoin%
\definecolor{currentfill}{rgb}{0.298039,0.447059,0.690196}%
\pgfsetfillcolor{currentfill}%
\pgfsetfillopacity{0.400000}%
\pgfsetlinewidth{1.003750pt}%
\definecolor{currentstroke}{rgb}{1.000000,1.000000,1.000000}%
\pgfsetstrokecolor{currentstroke}%
\pgfsetstrokeopacity{0.400000}%
\pgfsetdash{}{0pt}%
\pgfpathmoveto{\pgfqpoint{1.578884in}{0.557870in}}%
\pgfpathlineto{\pgfqpoint{1.779362in}{0.557870in}}%
\pgfpathlineto{\pgfqpoint{1.779362in}{0.660999in}}%
\pgfpathlineto{\pgfqpoint{1.578884in}{0.660999in}}%
\pgfpathclose%
\pgfusepath{stroke,fill}%
\end{pgfscope}%
\begin{pgfscope}%
\pgfpathrectangle{\pgfqpoint{0.726852in}{0.557870in}}{\pgfqpoint{5.513148in}{1.682130in}}%
\pgfusepath{clip}%
\pgfsetbuttcap%
\pgfsetmiterjoin%
\definecolor{currentfill}{rgb}{0.298039,0.447059,0.690196}%
\pgfsetfillcolor{currentfill}%
\pgfsetfillopacity{0.400000}%
\pgfsetlinewidth{1.003750pt}%
\definecolor{currentstroke}{rgb}{1.000000,1.000000,1.000000}%
\pgfsetstrokecolor{currentstroke}%
\pgfsetstrokeopacity{0.400000}%
\pgfsetdash{}{0pt}%
\pgfpathmoveto{\pgfqpoint{1.779362in}{0.557870in}}%
\pgfpathlineto{\pgfqpoint{1.979840in}{0.557870in}}%
\pgfpathlineto{\pgfqpoint{1.979840in}{0.659586in}}%
\pgfpathlineto{\pgfqpoint{1.779362in}{0.659586in}}%
\pgfpathclose%
\pgfusepath{stroke,fill}%
\end{pgfscope}%
\begin{pgfscope}%
\pgfpathrectangle{\pgfqpoint{0.726852in}{0.557870in}}{\pgfqpoint{5.513148in}{1.682130in}}%
\pgfusepath{clip}%
\pgfsetbuttcap%
\pgfsetmiterjoin%
\definecolor{currentfill}{rgb}{0.298039,0.447059,0.690196}%
\pgfsetfillcolor{currentfill}%
\pgfsetfillopacity{0.400000}%
\pgfsetlinewidth{1.003750pt}%
\definecolor{currentstroke}{rgb}{1.000000,1.000000,1.000000}%
\pgfsetstrokecolor{currentstroke}%
\pgfsetstrokeopacity{0.400000}%
\pgfsetdash{}{0pt}%
\pgfpathmoveto{\pgfqpoint{1.979840in}{0.557870in}}%
\pgfpathlineto{\pgfqpoint{2.180318in}{0.557870in}}%
\pgfpathlineto{\pgfqpoint{2.180318in}{0.734461in}}%
\pgfpathlineto{\pgfqpoint{1.979840in}{0.734461in}}%
\pgfpathclose%
\pgfusepath{stroke,fill}%
\end{pgfscope}%
\begin{pgfscope}%
\pgfpathrectangle{\pgfqpoint{0.726852in}{0.557870in}}{\pgfqpoint{5.513148in}{1.682130in}}%
\pgfusepath{clip}%
\pgfsetbuttcap%
\pgfsetmiterjoin%
\definecolor{currentfill}{rgb}{0.298039,0.447059,0.690196}%
\pgfsetfillcolor{currentfill}%
\pgfsetfillopacity{0.400000}%
\pgfsetlinewidth{1.003750pt}%
\definecolor{currentstroke}{rgb}{1.000000,1.000000,1.000000}%
\pgfsetstrokecolor{currentstroke}%
\pgfsetstrokeopacity{0.400000}%
\pgfsetdash{}{0pt}%
\pgfpathmoveto{\pgfqpoint{2.180318in}{0.557870in}}%
\pgfpathlineto{\pgfqpoint{2.380796in}{0.557870in}}%
\pgfpathlineto{\pgfqpoint{2.380796in}{0.711857in}}%
\pgfpathlineto{\pgfqpoint{2.180318in}{0.711857in}}%
\pgfpathclose%
\pgfusepath{stroke,fill}%
\end{pgfscope}%
\begin{pgfscope}%
\pgfpathrectangle{\pgfqpoint{0.726852in}{0.557870in}}{\pgfqpoint{5.513148in}{1.682130in}}%
\pgfusepath{clip}%
\pgfsetbuttcap%
\pgfsetmiterjoin%
\definecolor{currentfill}{rgb}{0.298039,0.447059,0.690196}%
\pgfsetfillcolor{currentfill}%
\pgfsetfillopacity{0.400000}%
\pgfsetlinewidth{1.003750pt}%
\definecolor{currentstroke}{rgb}{1.000000,1.000000,1.000000}%
\pgfsetstrokecolor{currentstroke}%
\pgfsetstrokeopacity{0.400000}%
\pgfsetdash{}{0pt}%
\pgfpathmoveto{\pgfqpoint{2.380796in}{0.557870in}}%
\pgfpathlineto{\pgfqpoint{2.581275in}{0.557870in}}%
\pgfpathlineto{\pgfqpoint{2.581275in}{0.699143in}}%
\pgfpathlineto{\pgfqpoint{2.380796in}{0.699143in}}%
\pgfpathclose%
\pgfusepath{stroke,fill}%
\end{pgfscope}%
\begin{pgfscope}%
\pgfpathrectangle{\pgfqpoint{0.726852in}{0.557870in}}{\pgfqpoint{5.513148in}{1.682130in}}%
\pgfusepath{clip}%
\pgfsetbuttcap%
\pgfsetmiterjoin%
\definecolor{currentfill}{rgb}{0.298039,0.447059,0.690196}%
\pgfsetfillcolor{currentfill}%
\pgfsetfillopacity{0.400000}%
\pgfsetlinewidth{1.003750pt}%
\definecolor{currentstroke}{rgb}{1.000000,1.000000,1.000000}%
\pgfsetstrokecolor{currentstroke}%
\pgfsetstrokeopacity{0.400000}%
\pgfsetdash{}{0pt}%
\pgfpathmoveto{\pgfqpoint{2.581275in}{0.557870in}}%
\pgfpathlineto{\pgfqpoint{2.781753in}{0.557870in}}%
\pgfpathlineto{\pgfqpoint{2.781753in}{0.683603in}}%
\pgfpathlineto{\pgfqpoint{2.581275in}{0.683603in}}%
\pgfpathclose%
\pgfusepath{stroke,fill}%
\end{pgfscope}%
\begin{pgfscope}%
\pgfpathrectangle{\pgfqpoint{0.726852in}{0.557870in}}{\pgfqpoint{5.513148in}{1.682130in}}%
\pgfusepath{clip}%
\pgfsetbuttcap%
\pgfsetmiterjoin%
\definecolor{currentfill}{rgb}{0.298039,0.447059,0.690196}%
\pgfsetfillcolor{currentfill}%
\pgfsetfillopacity{0.400000}%
\pgfsetlinewidth{1.003750pt}%
\definecolor{currentstroke}{rgb}{1.000000,1.000000,1.000000}%
\pgfsetstrokecolor{currentstroke}%
\pgfsetstrokeopacity{0.400000}%
\pgfsetdash{}{0pt}%
\pgfpathmoveto{\pgfqpoint{2.781753in}{0.557870in}}%
\pgfpathlineto{\pgfqpoint{2.982231in}{0.557870in}}%
\pgfpathlineto{\pgfqpoint{2.982231in}{0.683603in}}%
\pgfpathlineto{\pgfqpoint{2.781753in}{0.683603in}}%
\pgfpathclose%
\pgfusepath{stroke,fill}%
\end{pgfscope}%
\begin{pgfscope}%
\pgfpathrectangle{\pgfqpoint{0.726852in}{0.557870in}}{\pgfqpoint{5.513148in}{1.682130in}}%
\pgfusepath{clip}%
\pgfsetbuttcap%
\pgfsetmiterjoin%
\definecolor{currentfill}{rgb}{0.298039,0.447059,0.690196}%
\pgfsetfillcolor{currentfill}%
\pgfsetfillopacity{0.400000}%
\pgfsetlinewidth{1.003750pt}%
\definecolor{currentstroke}{rgb}{1.000000,1.000000,1.000000}%
\pgfsetstrokecolor{currentstroke}%
\pgfsetstrokeopacity{0.400000}%
\pgfsetdash{}{0pt}%
\pgfpathmoveto{\pgfqpoint{2.982231in}{0.557870in}}%
\pgfpathlineto{\pgfqpoint{3.182709in}{0.557870in}}%
\pgfpathlineto{\pgfqpoint{3.182709in}{0.687841in}}%
\pgfpathlineto{\pgfqpoint{2.982231in}{0.687841in}}%
\pgfpathclose%
\pgfusepath{stroke,fill}%
\end{pgfscope}%
\begin{pgfscope}%
\pgfpathrectangle{\pgfqpoint{0.726852in}{0.557870in}}{\pgfqpoint{5.513148in}{1.682130in}}%
\pgfusepath{clip}%
\pgfsetbuttcap%
\pgfsetmiterjoin%
\definecolor{currentfill}{rgb}{0.298039,0.447059,0.690196}%
\pgfsetfillcolor{currentfill}%
\pgfsetfillopacity{0.400000}%
\pgfsetlinewidth{1.003750pt}%
\definecolor{currentstroke}{rgb}{1.000000,1.000000,1.000000}%
\pgfsetstrokecolor{currentstroke}%
\pgfsetstrokeopacity{0.400000}%
\pgfsetdash{}{0pt}%
\pgfpathmoveto{\pgfqpoint{3.182709in}{0.557870in}}%
\pgfpathlineto{\pgfqpoint{3.383187in}{0.557870in}}%
\pgfpathlineto{\pgfqpoint{3.383187in}{0.686428in}}%
\pgfpathlineto{\pgfqpoint{3.182709in}{0.686428in}}%
\pgfpathclose%
\pgfusepath{stroke,fill}%
\end{pgfscope}%
\begin{pgfscope}%
\pgfpathrectangle{\pgfqpoint{0.726852in}{0.557870in}}{\pgfqpoint{5.513148in}{1.682130in}}%
\pgfusepath{clip}%
\pgfsetbuttcap%
\pgfsetmiterjoin%
\definecolor{currentfill}{rgb}{0.298039,0.447059,0.690196}%
\pgfsetfillcolor{currentfill}%
\pgfsetfillopacity{0.400000}%
\pgfsetlinewidth{1.003750pt}%
\definecolor{currentstroke}{rgb}{1.000000,1.000000,1.000000}%
\pgfsetstrokecolor{currentstroke}%
\pgfsetstrokeopacity{0.400000}%
\pgfsetdash{}{0pt}%
\pgfpathmoveto{\pgfqpoint{3.383187in}{0.557870in}}%
\pgfpathlineto{\pgfqpoint{3.583665in}{0.557870in}}%
\pgfpathlineto{\pgfqpoint{3.583665in}{0.642634in}}%
\pgfpathlineto{\pgfqpoint{3.383187in}{0.642634in}}%
\pgfpathclose%
\pgfusepath{stroke,fill}%
\end{pgfscope}%
\begin{pgfscope}%
\pgfpathrectangle{\pgfqpoint{0.726852in}{0.557870in}}{\pgfqpoint{5.513148in}{1.682130in}}%
\pgfusepath{clip}%
\pgfsetbuttcap%
\pgfsetmiterjoin%
\definecolor{currentfill}{rgb}{0.298039,0.447059,0.690196}%
\pgfsetfillcolor{currentfill}%
\pgfsetfillopacity{0.400000}%
\pgfsetlinewidth{1.003750pt}%
\definecolor{currentstroke}{rgb}{1.000000,1.000000,1.000000}%
\pgfsetstrokecolor{currentstroke}%
\pgfsetstrokeopacity{0.400000}%
\pgfsetdash{}{0pt}%
\pgfpathmoveto{\pgfqpoint{3.583665in}{0.557870in}}%
\pgfpathlineto{\pgfqpoint{3.784143in}{0.557870in}}%
\pgfpathlineto{\pgfqpoint{3.784143in}{0.600252in}}%
\pgfpathlineto{\pgfqpoint{3.583665in}{0.600252in}}%
\pgfpathclose%
\pgfusepath{stroke,fill}%
\end{pgfscope}%
\begin{pgfscope}%
\pgfpathrectangle{\pgfqpoint{0.726852in}{0.557870in}}{\pgfqpoint{5.513148in}{1.682130in}}%
\pgfusepath{clip}%
\pgfsetbuttcap%
\pgfsetmiterjoin%
\definecolor{currentfill}{rgb}{0.298039,0.447059,0.690196}%
\pgfsetfillcolor{currentfill}%
\pgfsetfillopacity{0.400000}%
\pgfsetlinewidth{1.003750pt}%
\definecolor{currentstroke}{rgb}{1.000000,1.000000,1.000000}%
\pgfsetstrokecolor{currentstroke}%
\pgfsetstrokeopacity{0.400000}%
\pgfsetdash{}{0pt}%
\pgfpathmoveto{\pgfqpoint{3.784143in}{0.557870in}}%
\pgfpathlineto{\pgfqpoint{3.984621in}{0.557870in}}%
\pgfpathlineto{\pgfqpoint{3.984621in}{0.591776in}}%
\pgfpathlineto{\pgfqpoint{3.784143in}{0.591776in}}%
\pgfpathclose%
\pgfusepath{stroke,fill}%
\end{pgfscope}%
\begin{pgfscope}%
\pgfpathrectangle{\pgfqpoint{0.726852in}{0.557870in}}{\pgfqpoint{5.513148in}{1.682130in}}%
\pgfusepath{clip}%
\pgfsetbuttcap%
\pgfsetmiterjoin%
\definecolor{currentfill}{rgb}{0.298039,0.447059,0.690196}%
\pgfsetfillcolor{currentfill}%
\pgfsetfillopacity{0.400000}%
\pgfsetlinewidth{1.003750pt}%
\definecolor{currentstroke}{rgb}{1.000000,1.000000,1.000000}%
\pgfsetstrokecolor{currentstroke}%
\pgfsetstrokeopacity{0.400000}%
\pgfsetdash{}{0pt}%
\pgfpathmoveto{\pgfqpoint{3.984621in}{0.557870in}}%
\pgfpathlineto{\pgfqpoint{4.185099in}{0.557870in}}%
\pgfpathlineto{\pgfqpoint{4.185099in}{0.588950in}}%
\pgfpathlineto{\pgfqpoint{3.984621in}{0.588950in}}%
\pgfpathclose%
\pgfusepath{stroke,fill}%
\end{pgfscope}%
\begin{pgfscope}%
\pgfpathrectangle{\pgfqpoint{0.726852in}{0.557870in}}{\pgfqpoint{5.513148in}{1.682130in}}%
\pgfusepath{clip}%
\pgfsetbuttcap%
\pgfsetmiterjoin%
\definecolor{currentfill}{rgb}{0.298039,0.447059,0.690196}%
\pgfsetfillcolor{currentfill}%
\pgfsetfillopacity{0.400000}%
\pgfsetlinewidth{1.003750pt}%
\definecolor{currentstroke}{rgb}{1.000000,1.000000,1.000000}%
\pgfsetstrokecolor{currentstroke}%
\pgfsetstrokeopacity{0.400000}%
\pgfsetdash{}{0pt}%
\pgfpathmoveto{\pgfqpoint{4.185099in}{0.557870in}}%
\pgfpathlineto{\pgfqpoint{4.385578in}{0.557870in}}%
\pgfpathlineto{\pgfqpoint{4.385578in}{0.591776in}}%
\pgfpathlineto{\pgfqpoint{4.185099in}{0.591776in}}%
\pgfpathclose%
\pgfusepath{stroke,fill}%
\end{pgfscope}%
\begin{pgfscope}%
\pgfpathrectangle{\pgfqpoint{0.726852in}{0.557870in}}{\pgfqpoint{5.513148in}{1.682130in}}%
\pgfusepath{clip}%
\pgfsetbuttcap%
\pgfsetmiterjoin%
\definecolor{currentfill}{rgb}{0.298039,0.447059,0.690196}%
\pgfsetfillcolor{currentfill}%
\pgfsetfillopacity{0.400000}%
\pgfsetlinewidth{1.003750pt}%
\definecolor{currentstroke}{rgb}{1.000000,1.000000,1.000000}%
\pgfsetstrokecolor{currentstroke}%
\pgfsetstrokeopacity{0.400000}%
\pgfsetdash{}{0pt}%
\pgfpathmoveto{\pgfqpoint{4.385578in}{0.557870in}}%
\pgfpathlineto{\pgfqpoint{4.586056in}{0.557870in}}%
\pgfpathlineto{\pgfqpoint{4.586056in}{0.584712in}}%
\pgfpathlineto{\pgfqpoint{4.385578in}{0.584712in}}%
\pgfpathclose%
\pgfusepath{stroke,fill}%
\end{pgfscope}%
\begin{pgfscope}%
\pgfpathrectangle{\pgfqpoint{0.726852in}{0.557870in}}{\pgfqpoint{5.513148in}{1.682130in}}%
\pgfusepath{clip}%
\pgfsetbuttcap%
\pgfsetmiterjoin%
\definecolor{currentfill}{rgb}{0.298039,0.447059,0.690196}%
\pgfsetfillcolor{currentfill}%
\pgfsetfillopacity{0.400000}%
\pgfsetlinewidth{1.003750pt}%
\definecolor{currentstroke}{rgb}{1.000000,1.000000,1.000000}%
\pgfsetstrokecolor{currentstroke}%
\pgfsetstrokeopacity{0.400000}%
\pgfsetdash{}{0pt}%
\pgfpathmoveto{\pgfqpoint{4.586056in}{0.557870in}}%
\pgfpathlineto{\pgfqpoint{4.786534in}{0.557870in}}%
\pgfpathlineto{\pgfqpoint{4.786534in}{0.588950in}}%
\pgfpathlineto{\pgfqpoint{4.586056in}{0.588950in}}%
\pgfpathclose%
\pgfusepath{stroke,fill}%
\end{pgfscope}%
\begin{pgfscope}%
\pgfpathrectangle{\pgfqpoint{0.726852in}{0.557870in}}{\pgfqpoint{5.513148in}{1.682130in}}%
\pgfusepath{clip}%
\pgfsetbuttcap%
\pgfsetmiterjoin%
\definecolor{currentfill}{rgb}{0.298039,0.447059,0.690196}%
\pgfsetfillcolor{currentfill}%
\pgfsetfillopacity{0.400000}%
\pgfsetlinewidth{1.003750pt}%
\definecolor{currentstroke}{rgb}{1.000000,1.000000,1.000000}%
\pgfsetstrokecolor{currentstroke}%
\pgfsetstrokeopacity{0.400000}%
\pgfsetdash{}{0pt}%
\pgfpathmoveto{\pgfqpoint{4.786534in}{0.557870in}}%
\pgfpathlineto{\pgfqpoint{4.987012in}{0.557870in}}%
\pgfpathlineto{\pgfqpoint{4.987012in}{0.566347in}}%
\pgfpathlineto{\pgfqpoint{4.786534in}{0.566347in}}%
\pgfpathclose%
\pgfusepath{stroke,fill}%
\end{pgfscope}%
\begin{pgfscope}%
\pgfpathrectangle{\pgfqpoint{0.726852in}{0.557870in}}{\pgfqpoint{5.513148in}{1.682130in}}%
\pgfusepath{clip}%
\pgfsetbuttcap%
\pgfsetmiterjoin%
\definecolor{currentfill}{rgb}{0.298039,0.447059,0.690196}%
\pgfsetfillcolor{currentfill}%
\pgfsetfillopacity{0.400000}%
\pgfsetlinewidth{1.003750pt}%
\definecolor{currentstroke}{rgb}{1.000000,1.000000,1.000000}%
\pgfsetstrokecolor{currentstroke}%
\pgfsetstrokeopacity{0.400000}%
\pgfsetdash{}{0pt}%
\pgfpathmoveto{\pgfqpoint{4.987012in}{0.557870in}}%
\pgfpathlineto{\pgfqpoint{5.187490in}{0.557870in}}%
\pgfpathlineto{\pgfqpoint{5.187490in}{0.577648in}}%
\pgfpathlineto{\pgfqpoint{4.987012in}{0.577648in}}%
\pgfpathclose%
\pgfusepath{stroke,fill}%
\end{pgfscope}%
\begin{pgfscope}%
\pgfpathrectangle{\pgfqpoint{0.726852in}{0.557870in}}{\pgfqpoint{5.513148in}{1.682130in}}%
\pgfusepath{clip}%
\pgfsetbuttcap%
\pgfsetmiterjoin%
\definecolor{currentfill}{rgb}{0.298039,0.447059,0.690196}%
\pgfsetfillcolor{currentfill}%
\pgfsetfillopacity{0.400000}%
\pgfsetlinewidth{1.003750pt}%
\definecolor{currentstroke}{rgb}{1.000000,1.000000,1.000000}%
\pgfsetstrokecolor{currentstroke}%
\pgfsetstrokeopacity{0.400000}%
\pgfsetdash{}{0pt}%
\pgfpathmoveto{\pgfqpoint{5.187490in}{0.557870in}}%
\pgfpathlineto{\pgfqpoint{5.387968in}{0.557870in}}%
\pgfpathlineto{\pgfqpoint{5.387968in}{0.573410in}}%
\pgfpathlineto{\pgfqpoint{5.187490in}{0.573410in}}%
\pgfpathclose%
\pgfusepath{stroke,fill}%
\end{pgfscope}%
\begin{pgfscope}%
\pgfpathrectangle{\pgfqpoint{0.726852in}{0.557870in}}{\pgfqpoint{5.513148in}{1.682130in}}%
\pgfusepath{clip}%
\pgfsetbuttcap%
\pgfsetmiterjoin%
\definecolor{currentfill}{rgb}{0.298039,0.447059,0.690196}%
\pgfsetfillcolor{currentfill}%
\pgfsetfillopacity{0.400000}%
\pgfsetlinewidth{1.003750pt}%
\definecolor{currentstroke}{rgb}{1.000000,1.000000,1.000000}%
\pgfsetstrokecolor{currentstroke}%
\pgfsetstrokeopacity{0.400000}%
\pgfsetdash{}{0pt}%
\pgfpathmoveto{\pgfqpoint{5.387968in}{0.557870in}}%
\pgfpathlineto{\pgfqpoint{5.588446in}{0.557870in}}%
\pgfpathlineto{\pgfqpoint{5.588446in}{0.566347in}}%
\pgfpathlineto{\pgfqpoint{5.387968in}{0.566347in}}%
\pgfpathclose%
\pgfusepath{stroke,fill}%
\end{pgfscope}%
\begin{pgfscope}%
\pgfpathrectangle{\pgfqpoint{0.726852in}{0.557870in}}{\pgfqpoint{5.513148in}{1.682130in}}%
\pgfusepath{clip}%
\pgfsetbuttcap%
\pgfsetmiterjoin%
\definecolor{currentfill}{rgb}{0.298039,0.447059,0.690196}%
\pgfsetfillcolor{currentfill}%
\pgfsetfillopacity{0.400000}%
\pgfsetlinewidth{1.003750pt}%
\definecolor{currentstroke}{rgb}{1.000000,1.000000,1.000000}%
\pgfsetstrokecolor{currentstroke}%
\pgfsetstrokeopacity{0.400000}%
\pgfsetdash{}{0pt}%
\pgfpathmoveto{\pgfqpoint{5.588446in}{0.557870in}}%
\pgfpathlineto{\pgfqpoint{5.788924in}{0.557870in}}%
\pgfpathlineto{\pgfqpoint{5.788924in}{0.562108in}}%
\pgfpathlineto{\pgfqpoint{5.588446in}{0.562108in}}%
\pgfpathclose%
\pgfusepath{stroke,fill}%
\end{pgfscope}%
\begin{pgfscope}%
\pgfpathrectangle{\pgfqpoint{0.726852in}{0.557870in}}{\pgfqpoint{5.513148in}{1.682130in}}%
\pgfusepath{clip}%
\pgfsetbuttcap%
\pgfsetmiterjoin%
\definecolor{currentfill}{rgb}{0.298039,0.447059,0.690196}%
\pgfsetfillcolor{currentfill}%
\pgfsetfillopacity{0.400000}%
\pgfsetlinewidth{1.003750pt}%
\definecolor{currentstroke}{rgb}{1.000000,1.000000,1.000000}%
\pgfsetstrokecolor{currentstroke}%
\pgfsetstrokeopacity{0.400000}%
\pgfsetdash{}{0pt}%
\pgfpathmoveto{\pgfqpoint{5.788924in}{0.557870in}}%
\pgfpathlineto{\pgfqpoint{5.989402in}{0.557870in}}%
\pgfpathlineto{\pgfqpoint{5.989402in}{0.566347in}}%
\pgfpathlineto{\pgfqpoint{5.788924in}{0.566347in}}%
\pgfpathclose%
\pgfusepath{stroke,fill}%
\end{pgfscope}%
\begin{pgfscope}%
\pgfsetrectcap%
\pgfsetmiterjoin%
\pgfsetlinewidth{1.254687pt}%
\definecolor{currentstroke}{rgb}{1.000000,1.000000,1.000000}%
\pgfsetstrokecolor{currentstroke}%
\pgfsetdash{}{0pt}%
\pgfpathmoveto{\pgfqpoint{0.726852in}{0.557870in}}%
\pgfpathlineto{\pgfqpoint{0.726852in}{2.240000in}}%
\pgfusepath{stroke}%
\end{pgfscope}%
\begin{pgfscope}%
\pgfsetrectcap%
\pgfsetmiterjoin%
\pgfsetlinewidth{1.254687pt}%
\definecolor{currentstroke}{rgb}{1.000000,1.000000,1.000000}%
\pgfsetstrokecolor{currentstroke}%
\pgfsetdash{}{0pt}%
\pgfpathmoveto{\pgfqpoint{6.240000in}{0.557870in}}%
\pgfpathlineto{\pgfqpoint{6.240000in}{2.240000in}}%
\pgfusepath{stroke}%
\end{pgfscope}%
\begin{pgfscope}%
\pgfsetrectcap%
\pgfsetmiterjoin%
\pgfsetlinewidth{1.254687pt}%
\definecolor{currentstroke}{rgb}{1.000000,1.000000,1.000000}%
\pgfsetstrokecolor{currentstroke}%
\pgfsetdash{}{0pt}%
\pgfpathmoveto{\pgfqpoint{0.726852in}{0.557870in}}%
\pgfpathlineto{\pgfqpoint{6.240000in}{0.557870in}}%
\pgfusepath{stroke}%
\end{pgfscope}%
\begin{pgfscope}%
\pgfsetrectcap%
\pgfsetmiterjoin%
\pgfsetlinewidth{1.254687pt}%
\definecolor{currentstroke}{rgb}{1.000000,1.000000,1.000000}%
\pgfsetstrokecolor{currentstroke}%
\pgfsetdash{}{0pt}%
\pgfpathmoveto{\pgfqpoint{0.726852in}{2.240000in}}%
\pgfpathlineto{\pgfqpoint{6.240000in}{2.240000in}}%
\pgfusepath{stroke}%
\end{pgfscope}%
\end{pgfpicture}%
\makeatother%
\endgroup%

    \caption{Distribution plot of ARI of all TSC methods evaluated at $\{2,9\}$ cluster centers when applied to classify heart failure.}
    \label{fig:tsc_hf_ari}
\end{figure}

\begin{table*}[htb]
    \centering
    \ra{1.3}
    \begin{tabular}{lr}
        \toprule
        Dataset-Method             &  ARI \\
        \midrule
        gls/2CH/regular/centroid/2 & 0.25 \\
        gls/2CH/scaled/centroid/2  & 0.25 \\
        gls/2CH/scaled/centroid/3  & 0.24 \\
        gls/2CH/regular/centroid/3 & 0.24 \\
        gls/2CH/scaled/average/2   & 0.24 \\
        \bottomrule
    \end{tabular}
    \caption{The five highest ARI scores attained when applying TSC for detecting heart failure.
             The \textbf{Dataset-Method} column indicates \textit{Dataset used}$/$\textit{View used}$/$\textit{Linkage criteria of method}$/$\textit{Number of cluster centers}.}
    \label{tab:tsc_hf_ari}
\end{table*}

As mentioned in section REFERENCE DOR, sensitivity and specificity are only well defined for clustering methods evaluated at two cluster centers,
so to determine whether the same clustering methods evaluated at a different number of cluster centers the ARI is used.
From figure \ref{fig:tsc_hf_ari} one can see that the majority of the ARIs of all TSC methods and patient heart failure diagnosis are located close to zero, 
indicating that these cluster assignments have little correlation with heart failure. 
There are a few ARIs that are $0.25$, but from inspection of table \ref{tab:tsc_hf_ari} one can see that they belong to cluster methods evaluated at two cluster centers, 
which have already been shown to have a low performance at predicting heart failure.
So overall, TSC does not perform well at detecting heart failure. \bigskip
\newpage

\subsection{Peak-value Clustering}

\begin{figure}[htb]
    \centering
    % \includegraphics[width=\textwidth]{results/pvc_hf_dor_sens_spec_dist.png}
    %% Creator: Matplotlib, PGF backend
%%
%% To include the figure in your LaTeX document, write
%%   \input{<filename>.pgf}
%%
%% Make sure the required packages are loaded in your preamble
%%   \usepackage{pgf}
%%
%% Figures using additional raster images can only be included by \input if
%% they are in the same directory as the main LaTeX file. For loading figures
%% from other directories you can use the `import` package
%%   \usepackage{import}
%% and then include the figures with
%%   \import{<path to file>}{<filename>.pgf}
%%
%% Matplotlib used the following preamble
%%
\begingroup%
\makeatletter%
\begin{pgfpicture}%
\pgfpathrectangle{\pgfpointorigin}{\pgfqpoint{6.360543in}{2.340000in}}%
\pgfusepath{use as bounding box, clip}%
\begin{pgfscope}%
\pgfsetbuttcap%
\pgfsetmiterjoin%
\definecolor{currentfill}{rgb}{1.000000,1.000000,1.000000}%
\pgfsetfillcolor{currentfill}%
\pgfsetlinewidth{0.000000pt}%
\definecolor{currentstroke}{rgb}{1.000000,1.000000,1.000000}%
\pgfsetstrokecolor{currentstroke}%
\pgfsetdash{}{0pt}%
\pgfpathmoveto{\pgfqpoint{0.000000in}{-0.000000in}}%
\pgfpathlineto{\pgfqpoint{6.360543in}{-0.000000in}}%
\pgfpathlineto{\pgfqpoint{6.360543in}{2.340000in}}%
\pgfpathlineto{\pgfqpoint{0.000000in}{2.340000in}}%
\pgfpathclose%
\pgfusepath{fill}%
\end{pgfscope}%
\begin{pgfscope}%
\pgfsetbuttcap%
\pgfsetmiterjoin%
\definecolor{currentfill}{rgb}{0.917647,0.917647,0.949020}%
\pgfsetfillcolor{currentfill}%
\pgfsetlinewidth{0.000000pt}%
\definecolor{currentstroke}{rgb}{0.000000,0.000000,0.000000}%
\pgfsetstrokecolor{currentstroke}%
\pgfsetstrokeopacity{0.000000}%
\pgfsetdash{}{0pt}%
\pgfpathmoveto{\pgfqpoint{0.498727in}{0.557870in}}%
\pgfpathlineto{\pgfqpoint{3.020856in}{0.557870in}}%
\pgfpathlineto{\pgfqpoint{3.020856in}{2.042604in}}%
\pgfpathlineto{\pgfqpoint{0.498727in}{2.042604in}}%
\pgfpathclose%
\pgfusepath{fill}%
\end{pgfscope}%
\begin{pgfscope}%
\pgfpathrectangle{\pgfqpoint{0.498727in}{0.557870in}}{\pgfqpoint{2.522130in}{1.484734in}}%
\pgfusepath{clip}%
\pgfsetroundcap%
\pgfsetroundjoin%
\pgfsetlinewidth{1.003750pt}%
\definecolor{currentstroke}{rgb}{1.000000,1.000000,1.000000}%
\pgfsetstrokecolor{currentstroke}%
\pgfsetdash{}{0pt}%
\pgfpathmoveto{\pgfqpoint{0.613369in}{0.557870in}}%
\pgfpathlineto{\pgfqpoint{0.613369in}{2.042604in}}%
\pgfusepath{stroke}%
\end{pgfscope}%
\begin{pgfscope}%
\definecolor{textcolor}{rgb}{0.150000,0.150000,0.150000}%
\pgfsetstrokecolor{textcolor}%
\pgfsetfillcolor{textcolor}%
\pgftext[x=0.613369in,y=0.425926in,,top]{\color{textcolor}\sffamily\fontsize{11.000000}{13.200000}\selectfont \(\displaystyle 0\)}%
\end{pgfscope}%
\begin{pgfscope}%
\pgfpathrectangle{\pgfqpoint{0.498727in}{0.557870in}}{\pgfqpoint{2.522130in}{1.484734in}}%
\pgfusepath{clip}%
\pgfsetroundcap%
\pgfsetroundjoin%
\pgfsetlinewidth{1.003750pt}%
\definecolor{currentstroke}{rgb}{1.000000,1.000000,1.000000}%
\pgfsetstrokecolor{currentstroke}%
\pgfsetdash{}{0pt}%
\pgfpathmoveto{\pgfqpoint{1.602509in}{0.557870in}}%
\pgfpathlineto{\pgfqpoint{1.602509in}{2.042604in}}%
\pgfusepath{stroke}%
\end{pgfscope}%
\begin{pgfscope}%
\definecolor{textcolor}{rgb}{0.150000,0.150000,0.150000}%
\pgfsetstrokecolor{textcolor}%
\pgfsetfillcolor{textcolor}%
\pgftext[x=1.602509in,y=0.425926in,,top]{\color{textcolor}\sffamily\fontsize{11.000000}{13.200000}\selectfont \(\displaystyle 5\)}%
\end{pgfscope}%
\begin{pgfscope}%
\pgfpathrectangle{\pgfqpoint{0.498727in}{0.557870in}}{\pgfqpoint{2.522130in}{1.484734in}}%
\pgfusepath{clip}%
\pgfsetroundcap%
\pgfsetroundjoin%
\pgfsetlinewidth{1.003750pt}%
\definecolor{currentstroke}{rgb}{1.000000,1.000000,1.000000}%
\pgfsetstrokecolor{currentstroke}%
\pgfsetdash{}{0pt}%
\pgfpathmoveto{\pgfqpoint{2.591650in}{0.557870in}}%
\pgfpathlineto{\pgfqpoint{2.591650in}{2.042604in}}%
\pgfusepath{stroke}%
\end{pgfscope}%
\begin{pgfscope}%
\definecolor{textcolor}{rgb}{0.150000,0.150000,0.150000}%
\pgfsetstrokecolor{textcolor}%
\pgfsetfillcolor{textcolor}%
\pgftext[x=2.591650in,y=0.425926in,,top]{\color{textcolor}\sffamily\fontsize{11.000000}{13.200000}\selectfont \(\displaystyle 10\)}%
\end{pgfscope}%
\begin{pgfscope}%
\definecolor{textcolor}{rgb}{0.150000,0.150000,0.150000}%
\pgfsetstrokecolor{textcolor}%
\pgfsetfillcolor{textcolor}%
\pgftext[x=1.759792in,y=0.235185in,,top]{\color{textcolor}\sffamily\fontsize{11.000000}{13.200000}\selectfont DOR}%
\end{pgfscope}%
\begin{pgfscope}%
\pgfpathrectangle{\pgfqpoint{0.498727in}{0.557870in}}{\pgfqpoint{2.522130in}{1.484734in}}%
\pgfusepath{clip}%
\pgfsetroundcap%
\pgfsetroundjoin%
\pgfsetlinewidth{1.003750pt}%
\definecolor{currentstroke}{rgb}{1.000000,1.000000,1.000000}%
\pgfsetstrokecolor{currentstroke}%
\pgfsetdash{}{0pt}%
\pgfpathmoveto{\pgfqpoint{0.498727in}{0.557870in}}%
\pgfpathlineto{\pgfqpoint{3.020856in}{0.557870in}}%
\pgfusepath{stroke}%
\end{pgfscope}%
\begin{pgfscope}%
\definecolor{textcolor}{rgb}{0.150000,0.150000,0.150000}%
\pgfsetstrokecolor{textcolor}%
\pgfsetfillcolor{textcolor}%
\pgftext[x=0.290741in,y=0.505064in,left,base]{\color{textcolor}\sffamily\fontsize{11.000000}{13.200000}\selectfont \(\displaystyle 0\)}%
\end{pgfscope}%
\begin{pgfscope}%
\pgfpathrectangle{\pgfqpoint{0.498727in}{0.557870in}}{\pgfqpoint{2.522130in}{1.484734in}}%
\pgfusepath{clip}%
\pgfsetroundcap%
\pgfsetroundjoin%
\pgfsetlinewidth{1.003750pt}%
\definecolor{currentstroke}{rgb}{1.000000,1.000000,1.000000}%
\pgfsetstrokecolor{currentstroke}%
\pgfsetdash{}{0pt}%
\pgfpathmoveto{\pgfqpoint{0.498727in}{0.961879in}}%
\pgfpathlineto{\pgfqpoint{3.020856in}{0.961879in}}%
\pgfusepath{stroke}%
\end{pgfscope}%
\begin{pgfscope}%
\definecolor{textcolor}{rgb}{0.150000,0.150000,0.150000}%
\pgfsetstrokecolor{textcolor}%
\pgfsetfillcolor{textcolor}%
\pgftext[x=0.290741in,y=0.909073in,left,base]{\color{textcolor}\sffamily\fontsize{11.000000}{13.200000}\selectfont \(\displaystyle 2\)}%
\end{pgfscope}%
\begin{pgfscope}%
\pgfpathrectangle{\pgfqpoint{0.498727in}{0.557870in}}{\pgfqpoint{2.522130in}{1.484734in}}%
\pgfusepath{clip}%
\pgfsetroundcap%
\pgfsetroundjoin%
\pgfsetlinewidth{1.003750pt}%
\definecolor{currentstroke}{rgb}{1.000000,1.000000,1.000000}%
\pgfsetstrokecolor{currentstroke}%
\pgfsetdash{}{0pt}%
\pgfpathmoveto{\pgfqpoint{0.498727in}{1.365889in}}%
\pgfpathlineto{\pgfqpoint{3.020856in}{1.365889in}}%
\pgfusepath{stroke}%
\end{pgfscope}%
\begin{pgfscope}%
\definecolor{textcolor}{rgb}{0.150000,0.150000,0.150000}%
\pgfsetstrokecolor{textcolor}%
\pgfsetfillcolor{textcolor}%
\pgftext[x=0.290741in,y=1.313082in,left,base]{\color{textcolor}\sffamily\fontsize{11.000000}{13.200000}\selectfont \(\displaystyle 4\)}%
\end{pgfscope}%
\begin{pgfscope}%
\pgfpathrectangle{\pgfqpoint{0.498727in}{0.557870in}}{\pgfqpoint{2.522130in}{1.484734in}}%
\pgfusepath{clip}%
\pgfsetroundcap%
\pgfsetroundjoin%
\pgfsetlinewidth{1.003750pt}%
\definecolor{currentstroke}{rgb}{1.000000,1.000000,1.000000}%
\pgfsetstrokecolor{currentstroke}%
\pgfsetdash{}{0pt}%
\pgfpathmoveto{\pgfqpoint{0.498727in}{1.769898in}}%
\pgfpathlineto{\pgfqpoint{3.020856in}{1.769898in}}%
\pgfusepath{stroke}%
\end{pgfscope}%
\begin{pgfscope}%
\definecolor{textcolor}{rgb}{0.150000,0.150000,0.150000}%
\pgfsetstrokecolor{textcolor}%
\pgfsetfillcolor{textcolor}%
\pgftext[x=0.290741in,y=1.717091in,left,base]{\color{textcolor}\sffamily\fontsize{11.000000}{13.200000}\selectfont \(\displaystyle 6\)}%
\end{pgfscope}%
\begin{pgfscope}%
\definecolor{textcolor}{rgb}{0.150000,0.150000,0.150000}%
\pgfsetstrokecolor{textcolor}%
\pgfsetfillcolor{textcolor}%
\pgftext[x=0.235185in,y=1.300237in,,bottom,rotate=90.000000]{\color{textcolor}\sffamily\fontsize{11.000000}{13.200000}\selectfont Occurance}%
\end{pgfscope}%
\begin{pgfscope}%
\pgfpathrectangle{\pgfqpoint{0.498727in}{0.557870in}}{\pgfqpoint{2.522130in}{1.484734in}}%
\pgfusepath{clip}%
\pgfsetbuttcap%
\pgfsetmiterjoin%
\definecolor{currentfill}{rgb}{0.298039,0.447059,0.690196}%
\pgfsetfillcolor{currentfill}%
\pgfsetfillopacity{0.400000}%
\pgfsetlinewidth{1.003750pt}%
\definecolor{currentstroke}{rgb}{1.000000,1.000000,1.000000}%
\pgfsetstrokecolor{currentstroke}%
\pgfsetstrokeopacity{0.400000}%
\pgfsetdash{}{0pt}%
\pgfpathmoveto{\pgfqpoint{0.613369in}{0.557870in}}%
\pgfpathlineto{\pgfqpoint{0.842654in}{0.557870in}}%
\pgfpathlineto{\pgfqpoint{0.842654in}{1.971903in}}%
\pgfpathlineto{\pgfqpoint{0.613369in}{1.971903in}}%
\pgfpathclose%
\pgfusepath{stroke,fill}%
\end{pgfscope}%
\begin{pgfscope}%
\pgfpathrectangle{\pgfqpoint{0.498727in}{0.557870in}}{\pgfqpoint{2.522130in}{1.484734in}}%
\pgfusepath{clip}%
\pgfsetbuttcap%
\pgfsetmiterjoin%
\definecolor{currentfill}{rgb}{0.298039,0.447059,0.690196}%
\pgfsetfillcolor{currentfill}%
\pgfsetfillopacity{0.400000}%
\pgfsetlinewidth{1.003750pt}%
\definecolor{currentstroke}{rgb}{1.000000,1.000000,1.000000}%
\pgfsetstrokecolor{currentstroke}%
\pgfsetstrokeopacity{0.400000}%
\pgfsetdash{}{0pt}%
\pgfpathmoveto{\pgfqpoint{0.842654in}{0.557870in}}%
\pgfpathlineto{\pgfqpoint{1.071938in}{0.557870in}}%
\pgfpathlineto{\pgfqpoint{1.071938in}{0.557870in}}%
\pgfpathlineto{\pgfqpoint{0.842654in}{0.557870in}}%
\pgfpathclose%
\pgfusepath{stroke,fill}%
\end{pgfscope}%
\begin{pgfscope}%
\pgfpathrectangle{\pgfqpoint{0.498727in}{0.557870in}}{\pgfqpoint{2.522130in}{1.484734in}}%
\pgfusepath{clip}%
\pgfsetbuttcap%
\pgfsetmiterjoin%
\definecolor{currentfill}{rgb}{0.298039,0.447059,0.690196}%
\pgfsetfillcolor{currentfill}%
\pgfsetfillopacity{0.400000}%
\pgfsetlinewidth{1.003750pt}%
\definecolor{currentstroke}{rgb}{1.000000,1.000000,1.000000}%
\pgfsetstrokecolor{currentstroke}%
\pgfsetstrokeopacity{0.400000}%
\pgfsetdash{}{0pt}%
\pgfpathmoveto{\pgfqpoint{1.071938in}{0.557870in}}%
\pgfpathlineto{\pgfqpoint{1.301223in}{0.557870in}}%
\pgfpathlineto{\pgfqpoint{1.301223in}{0.759875in}}%
\pgfpathlineto{\pgfqpoint{1.071938in}{0.759875in}}%
\pgfpathclose%
\pgfusepath{stroke,fill}%
\end{pgfscope}%
\begin{pgfscope}%
\pgfpathrectangle{\pgfqpoint{0.498727in}{0.557870in}}{\pgfqpoint{2.522130in}{1.484734in}}%
\pgfusepath{clip}%
\pgfsetbuttcap%
\pgfsetmiterjoin%
\definecolor{currentfill}{rgb}{0.298039,0.447059,0.690196}%
\pgfsetfillcolor{currentfill}%
\pgfsetfillopacity{0.400000}%
\pgfsetlinewidth{1.003750pt}%
\definecolor{currentstroke}{rgb}{1.000000,1.000000,1.000000}%
\pgfsetstrokecolor{currentstroke}%
\pgfsetstrokeopacity{0.400000}%
\pgfsetdash{}{0pt}%
\pgfpathmoveto{\pgfqpoint{1.301223in}{0.557870in}}%
\pgfpathlineto{\pgfqpoint{1.530507in}{0.557870in}}%
\pgfpathlineto{\pgfqpoint{1.530507in}{1.163884in}}%
\pgfpathlineto{\pgfqpoint{1.301223in}{1.163884in}}%
\pgfpathclose%
\pgfusepath{stroke,fill}%
\end{pgfscope}%
\begin{pgfscope}%
\pgfpathrectangle{\pgfqpoint{0.498727in}{0.557870in}}{\pgfqpoint{2.522130in}{1.484734in}}%
\pgfusepath{clip}%
\pgfsetbuttcap%
\pgfsetmiterjoin%
\definecolor{currentfill}{rgb}{0.298039,0.447059,0.690196}%
\pgfsetfillcolor{currentfill}%
\pgfsetfillopacity{0.400000}%
\pgfsetlinewidth{1.003750pt}%
\definecolor{currentstroke}{rgb}{1.000000,1.000000,1.000000}%
\pgfsetstrokecolor{currentstroke}%
\pgfsetstrokeopacity{0.400000}%
\pgfsetdash{}{0pt}%
\pgfpathmoveto{\pgfqpoint{1.530507in}{0.557870in}}%
\pgfpathlineto{\pgfqpoint{1.759792in}{0.557870in}}%
\pgfpathlineto{\pgfqpoint{1.759792in}{0.961879in}}%
\pgfpathlineto{\pgfqpoint{1.530507in}{0.961879in}}%
\pgfpathclose%
\pgfusepath{stroke,fill}%
\end{pgfscope}%
\begin{pgfscope}%
\pgfpathrectangle{\pgfqpoint{0.498727in}{0.557870in}}{\pgfqpoint{2.522130in}{1.484734in}}%
\pgfusepath{clip}%
\pgfsetbuttcap%
\pgfsetmiterjoin%
\definecolor{currentfill}{rgb}{0.298039,0.447059,0.690196}%
\pgfsetfillcolor{currentfill}%
\pgfsetfillopacity{0.400000}%
\pgfsetlinewidth{1.003750pt}%
\definecolor{currentstroke}{rgb}{1.000000,1.000000,1.000000}%
\pgfsetstrokecolor{currentstroke}%
\pgfsetstrokeopacity{0.400000}%
\pgfsetdash{}{0pt}%
\pgfpathmoveto{\pgfqpoint{1.759792in}{0.557870in}}%
\pgfpathlineto{\pgfqpoint{1.989076in}{0.557870in}}%
\pgfpathlineto{\pgfqpoint{1.989076in}{1.163884in}}%
\pgfpathlineto{\pgfqpoint{1.759792in}{1.163884in}}%
\pgfpathclose%
\pgfusepath{stroke,fill}%
\end{pgfscope}%
\begin{pgfscope}%
\pgfpathrectangle{\pgfqpoint{0.498727in}{0.557870in}}{\pgfqpoint{2.522130in}{1.484734in}}%
\pgfusepath{clip}%
\pgfsetbuttcap%
\pgfsetmiterjoin%
\definecolor{currentfill}{rgb}{0.298039,0.447059,0.690196}%
\pgfsetfillcolor{currentfill}%
\pgfsetfillopacity{0.400000}%
\pgfsetlinewidth{1.003750pt}%
\definecolor{currentstroke}{rgb}{1.000000,1.000000,1.000000}%
\pgfsetstrokecolor{currentstroke}%
\pgfsetstrokeopacity{0.400000}%
\pgfsetdash{}{0pt}%
\pgfpathmoveto{\pgfqpoint{1.989076in}{0.557870in}}%
\pgfpathlineto{\pgfqpoint{2.218361in}{0.557870in}}%
\pgfpathlineto{\pgfqpoint{2.218361in}{0.759875in}}%
\pgfpathlineto{\pgfqpoint{1.989076in}{0.759875in}}%
\pgfpathclose%
\pgfusepath{stroke,fill}%
\end{pgfscope}%
\begin{pgfscope}%
\pgfpathrectangle{\pgfqpoint{0.498727in}{0.557870in}}{\pgfqpoint{2.522130in}{1.484734in}}%
\pgfusepath{clip}%
\pgfsetbuttcap%
\pgfsetmiterjoin%
\definecolor{currentfill}{rgb}{0.298039,0.447059,0.690196}%
\pgfsetfillcolor{currentfill}%
\pgfsetfillopacity{0.400000}%
\pgfsetlinewidth{1.003750pt}%
\definecolor{currentstroke}{rgb}{1.000000,1.000000,1.000000}%
\pgfsetstrokecolor{currentstroke}%
\pgfsetstrokeopacity{0.400000}%
\pgfsetdash{}{0pt}%
\pgfpathmoveto{\pgfqpoint{2.218361in}{0.557870in}}%
\pgfpathlineto{\pgfqpoint{2.447645in}{0.557870in}}%
\pgfpathlineto{\pgfqpoint{2.447645in}{0.759875in}}%
\pgfpathlineto{\pgfqpoint{2.218361in}{0.759875in}}%
\pgfpathclose%
\pgfusepath{stroke,fill}%
\end{pgfscope}%
\begin{pgfscope}%
\pgfpathrectangle{\pgfqpoint{0.498727in}{0.557870in}}{\pgfqpoint{2.522130in}{1.484734in}}%
\pgfusepath{clip}%
\pgfsetbuttcap%
\pgfsetmiterjoin%
\definecolor{currentfill}{rgb}{0.298039,0.447059,0.690196}%
\pgfsetfillcolor{currentfill}%
\pgfsetfillopacity{0.400000}%
\pgfsetlinewidth{1.003750pt}%
\definecolor{currentstroke}{rgb}{1.000000,1.000000,1.000000}%
\pgfsetstrokecolor{currentstroke}%
\pgfsetstrokeopacity{0.400000}%
\pgfsetdash{}{0pt}%
\pgfpathmoveto{\pgfqpoint{2.447645in}{0.557870in}}%
\pgfpathlineto{\pgfqpoint{2.676930in}{0.557870in}}%
\pgfpathlineto{\pgfqpoint{2.676930in}{0.557870in}}%
\pgfpathlineto{\pgfqpoint{2.447645in}{0.557870in}}%
\pgfpathclose%
\pgfusepath{stroke,fill}%
\end{pgfscope}%
\begin{pgfscope}%
\pgfpathrectangle{\pgfqpoint{0.498727in}{0.557870in}}{\pgfqpoint{2.522130in}{1.484734in}}%
\pgfusepath{clip}%
\pgfsetbuttcap%
\pgfsetmiterjoin%
\definecolor{currentfill}{rgb}{0.298039,0.447059,0.690196}%
\pgfsetfillcolor{currentfill}%
\pgfsetfillopacity{0.400000}%
\pgfsetlinewidth{1.003750pt}%
\definecolor{currentstroke}{rgb}{1.000000,1.000000,1.000000}%
\pgfsetstrokecolor{currentstroke}%
\pgfsetstrokeopacity{0.400000}%
\pgfsetdash{}{0pt}%
\pgfpathmoveto{\pgfqpoint{2.676930in}{0.557870in}}%
\pgfpathlineto{\pgfqpoint{2.906214in}{0.557870in}}%
\pgfpathlineto{\pgfqpoint{2.906214in}{1.163884in}}%
\pgfpathlineto{\pgfqpoint{2.676930in}{1.163884in}}%
\pgfpathclose%
\pgfusepath{stroke,fill}%
\end{pgfscope}%
\begin{pgfscope}%
\pgfsetrectcap%
\pgfsetmiterjoin%
\pgfsetlinewidth{1.254687pt}%
\definecolor{currentstroke}{rgb}{1.000000,1.000000,1.000000}%
\pgfsetstrokecolor{currentstroke}%
\pgfsetdash{}{0pt}%
\pgfpathmoveto{\pgfqpoint{0.498727in}{0.557870in}}%
\pgfpathlineto{\pgfqpoint{0.498727in}{2.042604in}}%
\pgfusepath{stroke}%
\end{pgfscope}%
\begin{pgfscope}%
\pgfsetrectcap%
\pgfsetmiterjoin%
\pgfsetlinewidth{1.254687pt}%
\definecolor{currentstroke}{rgb}{1.000000,1.000000,1.000000}%
\pgfsetstrokecolor{currentstroke}%
\pgfsetdash{}{0pt}%
\pgfpathmoveto{\pgfqpoint{3.020856in}{0.557870in}}%
\pgfpathlineto{\pgfqpoint{3.020856in}{2.042604in}}%
\pgfusepath{stroke}%
\end{pgfscope}%
\begin{pgfscope}%
\pgfsetrectcap%
\pgfsetmiterjoin%
\pgfsetlinewidth{1.254687pt}%
\definecolor{currentstroke}{rgb}{1.000000,1.000000,1.000000}%
\pgfsetstrokecolor{currentstroke}%
\pgfsetdash{}{0pt}%
\pgfpathmoveto{\pgfqpoint{0.498727in}{0.557870in}}%
\pgfpathlineto{\pgfqpoint{3.020856in}{0.557870in}}%
\pgfusepath{stroke}%
\end{pgfscope}%
\begin{pgfscope}%
\pgfsetrectcap%
\pgfsetmiterjoin%
\pgfsetlinewidth{1.254687pt}%
\definecolor{currentstroke}{rgb}{1.000000,1.000000,1.000000}%
\pgfsetstrokecolor{currentstroke}%
\pgfsetdash{}{0pt}%
\pgfpathmoveto{\pgfqpoint{0.498727in}{2.042604in}}%
\pgfpathlineto{\pgfqpoint{3.020856in}{2.042604in}}%
\pgfusepath{stroke}%
\end{pgfscope}%
\begin{pgfscope}%
\definecolor{textcolor}{rgb}{0.150000,0.150000,0.150000}%
\pgfsetstrokecolor{textcolor}%
\pgfsetfillcolor{textcolor}%
\pgftext[x=1.759792in,y=2.125938in,,base]{\color{textcolor}\sffamily\fontsize{11.000000}{13.200000}\selectfont (a)}%
\end{pgfscope}%
\begin{pgfscope}%
\pgfsetbuttcap%
\pgfsetmiterjoin%
\definecolor{currentfill}{rgb}{0.917647,0.917647,0.949020}%
\pgfsetfillcolor{currentfill}%
\pgfsetlinewidth{0.000000pt}%
\definecolor{currentstroke}{rgb}{0.000000,0.000000,0.000000}%
\pgfsetstrokecolor{currentstroke}%
\pgfsetstrokeopacity{0.000000}%
\pgfsetdash{}{0pt}%
\pgfpathmoveto{\pgfqpoint{3.717870in}{0.557870in}}%
\pgfpathlineto{\pgfqpoint{6.240000in}{0.557870in}}%
\pgfpathlineto{\pgfqpoint{6.240000in}{2.042604in}}%
\pgfpathlineto{\pgfqpoint{3.717870in}{2.042604in}}%
\pgfpathclose%
\pgfusepath{fill}%
\end{pgfscope}%
\begin{pgfscope}%
\pgfpathrectangle{\pgfqpoint{3.717870in}{0.557870in}}{\pgfqpoint{2.522130in}{1.484734in}}%
\pgfusepath{clip}%
\pgfsetroundcap%
\pgfsetroundjoin%
\pgfsetlinewidth{1.003750pt}%
\definecolor{currentstroke}{rgb}{1.000000,1.000000,1.000000}%
\pgfsetstrokecolor{currentstroke}%
\pgfsetdash{}{0pt}%
\pgfpathmoveto{\pgfqpoint{3.832513in}{0.557870in}}%
\pgfpathlineto{\pgfqpoint{3.832513in}{2.042604in}}%
\pgfusepath{stroke}%
\end{pgfscope}%
\begin{pgfscope}%
\definecolor{textcolor}{rgb}{0.150000,0.150000,0.150000}%
\pgfsetstrokecolor{textcolor}%
\pgfsetfillcolor{textcolor}%
\pgftext[x=3.832513in,y=0.425926in,,top]{\color{textcolor}\sffamily\fontsize{11.000000}{13.200000}\selectfont \(\displaystyle 0.00\)}%
\end{pgfscope}%
\begin{pgfscope}%
\pgfpathrectangle{\pgfqpoint{3.717870in}{0.557870in}}{\pgfqpoint{2.522130in}{1.484734in}}%
\pgfusepath{clip}%
\pgfsetroundcap%
\pgfsetroundjoin%
\pgfsetlinewidth{1.003750pt}%
\definecolor{currentstroke}{rgb}{1.000000,1.000000,1.000000}%
\pgfsetstrokecolor{currentstroke}%
\pgfsetdash{}{0pt}%
\pgfpathmoveto{\pgfqpoint{4.405724in}{0.557870in}}%
\pgfpathlineto{\pgfqpoint{4.405724in}{2.042604in}}%
\pgfusepath{stroke}%
\end{pgfscope}%
\begin{pgfscope}%
\definecolor{textcolor}{rgb}{0.150000,0.150000,0.150000}%
\pgfsetstrokecolor{textcolor}%
\pgfsetfillcolor{textcolor}%
\pgftext[x=4.405724in,y=0.425926in,,top]{\color{textcolor}\sffamily\fontsize{11.000000}{13.200000}\selectfont \(\displaystyle 0.25\)}%
\end{pgfscope}%
\begin{pgfscope}%
\pgfpathrectangle{\pgfqpoint{3.717870in}{0.557870in}}{\pgfqpoint{2.522130in}{1.484734in}}%
\pgfusepath{clip}%
\pgfsetroundcap%
\pgfsetroundjoin%
\pgfsetlinewidth{1.003750pt}%
\definecolor{currentstroke}{rgb}{1.000000,1.000000,1.000000}%
\pgfsetstrokecolor{currentstroke}%
\pgfsetdash{}{0pt}%
\pgfpathmoveto{\pgfqpoint{4.978935in}{0.557870in}}%
\pgfpathlineto{\pgfqpoint{4.978935in}{2.042604in}}%
\pgfusepath{stroke}%
\end{pgfscope}%
\begin{pgfscope}%
\definecolor{textcolor}{rgb}{0.150000,0.150000,0.150000}%
\pgfsetstrokecolor{textcolor}%
\pgfsetfillcolor{textcolor}%
\pgftext[x=4.978935in,y=0.425926in,,top]{\color{textcolor}\sffamily\fontsize{11.000000}{13.200000}\selectfont \(\displaystyle 0.50\)}%
\end{pgfscope}%
\begin{pgfscope}%
\pgfpathrectangle{\pgfqpoint{3.717870in}{0.557870in}}{\pgfqpoint{2.522130in}{1.484734in}}%
\pgfusepath{clip}%
\pgfsetroundcap%
\pgfsetroundjoin%
\pgfsetlinewidth{1.003750pt}%
\definecolor{currentstroke}{rgb}{1.000000,1.000000,1.000000}%
\pgfsetstrokecolor{currentstroke}%
\pgfsetdash{}{0pt}%
\pgfpathmoveto{\pgfqpoint{5.552146in}{0.557870in}}%
\pgfpathlineto{\pgfqpoint{5.552146in}{2.042604in}}%
\pgfusepath{stroke}%
\end{pgfscope}%
\begin{pgfscope}%
\definecolor{textcolor}{rgb}{0.150000,0.150000,0.150000}%
\pgfsetstrokecolor{textcolor}%
\pgfsetfillcolor{textcolor}%
\pgftext[x=5.552146in,y=0.425926in,,top]{\color{textcolor}\sffamily\fontsize{11.000000}{13.200000}\selectfont \(\displaystyle 0.75\)}%
\end{pgfscope}%
\begin{pgfscope}%
\pgfpathrectangle{\pgfqpoint{3.717870in}{0.557870in}}{\pgfqpoint{2.522130in}{1.484734in}}%
\pgfusepath{clip}%
\pgfsetroundcap%
\pgfsetroundjoin%
\pgfsetlinewidth{1.003750pt}%
\definecolor{currentstroke}{rgb}{1.000000,1.000000,1.000000}%
\pgfsetstrokecolor{currentstroke}%
\pgfsetdash{}{0pt}%
\pgfpathmoveto{\pgfqpoint{6.125358in}{0.557870in}}%
\pgfpathlineto{\pgfqpoint{6.125358in}{2.042604in}}%
\pgfusepath{stroke}%
\end{pgfscope}%
\begin{pgfscope}%
\definecolor{textcolor}{rgb}{0.150000,0.150000,0.150000}%
\pgfsetstrokecolor{textcolor}%
\pgfsetfillcolor{textcolor}%
\pgftext[x=6.125358in,y=0.425926in,,top]{\color{textcolor}\sffamily\fontsize{11.000000}{13.200000}\selectfont \(\displaystyle 1.00\)}%
\end{pgfscope}%
\begin{pgfscope}%
\definecolor{textcolor}{rgb}{0.150000,0.150000,0.150000}%
\pgfsetstrokecolor{textcolor}%
\pgfsetfillcolor{textcolor}%
\pgftext[x=4.978935in,y=0.235185in,,top]{\color{textcolor}\sffamily\fontsize{11.000000}{13.200000}\selectfont Specificity}%
\end{pgfscope}%
\begin{pgfscope}%
\pgfpathrectangle{\pgfqpoint{3.717870in}{0.557870in}}{\pgfqpoint{2.522130in}{1.484734in}}%
\pgfusepath{clip}%
\pgfsetroundcap%
\pgfsetroundjoin%
\pgfsetlinewidth{1.003750pt}%
\definecolor{currentstroke}{rgb}{1.000000,1.000000,1.000000}%
\pgfsetstrokecolor{currentstroke}%
\pgfsetdash{}{0pt}%
\pgfpathmoveto{\pgfqpoint{3.717870in}{0.625358in}}%
\pgfpathlineto{\pgfqpoint{6.240000in}{0.625358in}}%
\pgfusepath{stroke}%
\end{pgfscope}%
\begin{pgfscope}%
\definecolor{textcolor}{rgb}{0.150000,0.150000,0.150000}%
\pgfsetstrokecolor{textcolor}%
\pgfsetfillcolor{textcolor}%
\pgftext[x=3.391597in,y=0.572552in,left,base]{\color{textcolor}\sffamily\fontsize{11.000000}{13.200000}\selectfont \(\displaystyle 0.0\)}%
\end{pgfscope}%
\begin{pgfscope}%
\pgfpathrectangle{\pgfqpoint{3.717870in}{0.557870in}}{\pgfqpoint{2.522130in}{1.484734in}}%
\pgfusepath{clip}%
\pgfsetroundcap%
\pgfsetroundjoin%
\pgfsetlinewidth{1.003750pt}%
\definecolor{currentstroke}{rgb}{1.000000,1.000000,1.000000}%
\pgfsetstrokecolor{currentstroke}%
\pgfsetdash{}{0pt}%
\pgfpathmoveto{\pgfqpoint{3.717870in}{1.300237in}}%
\pgfpathlineto{\pgfqpoint{6.240000in}{1.300237in}}%
\pgfusepath{stroke}%
\end{pgfscope}%
\begin{pgfscope}%
\definecolor{textcolor}{rgb}{0.150000,0.150000,0.150000}%
\pgfsetstrokecolor{textcolor}%
\pgfsetfillcolor{textcolor}%
\pgftext[x=3.391597in,y=1.247431in,left,base]{\color{textcolor}\sffamily\fontsize{11.000000}{13.200000}\selectfont \(\displaystyle 0.5\)}%
\end{pgfscope}%
\begin{pgfscope}%
\pgfpathrectangle{\pgfqpoint{3.717870in}{0.557870in}}{\pgfqpoint{2.522130in}{1.484734in}}%
\pgfusepath{clip}%
\pgfsetroundcap%
\pgfsetroundjoin%
\pgfsetlinewidth{1.003750pt}%
\definecolor{currentstroke}{rgb}{1.000000,1.000000,1.000000}%
\pgfsetstrokecolor{currentstroke}%
\pgfsetdash{}{0pt}%
\pgfpathmoveto{\pgfqpoint{3.717870in}{1.975116in}}%
\pgfpathlineto{\pgfqpoint{6.240000in}{1.975116in}}%
\pgfusepath{stroke}%
\end{pgfscope}%
\begin{pgfscope}%
\definecolor{textcolor}{rgb}{0.150000,0.150000,0.150000}%
\pgfsetstrokecolor{textcolor}%
\pgfsetfillcolor{textcolor}%
\pgftext[x=3.391597in,y=1.922310in,left,base]{\color{textcolor}\sffamily\fontsize{11.000000}{13.200000}\selectfont \(\displaystyle 1.0\)}%
\end{pgfscope}%
\begin{pgfscope}%
\definecolor{textcolor}{rgb}{0.150000,0.150000,0.150000}%
\pgfsetstrokecolor{textcolor}%
\pgfsetfillcolor{textcolor}%
\pgftext[x=3.336042in,y=1.300237in,,bottom,rotate=90.000000]{\color{textcolor}\sffamily\fontsize{11.000000}{13.200000}\selectfont Sensitivity}%
\end{pgfscope}%
\begin{pgfscope}%
\pgfpathrectangle{\pgfqpoint{3.717870in}{0.557870in}}{\pgfqpoint{2.522130in}{1.484734in}}%
\pgfusepath{clip}%
\pgfsetbuttcap%
\pgfsetroundjoin%
\definecolor{currentfill}{rgb}{0.298039,0.447059,0.690196}%
\pgfsetfillcolor{currentfill}%
\pgfsetlinewidth{1.003750pt}%
\definecolor{currentstroke}{rgb}{0.298039,0.447059,0.690196}%
\pgfsetstrokecolor{currentstroke}%
\pgfsetdash{}{0pt}%
\pgfpathmoveto{\pgfqpoint{6.102198in}{0.594302in}}%
\pgfpathcurveto{\pgfqpoint{6.110434in}{0.594302in}}{\pgfqpoint{6.118334in}{0.597574in}}{\pgfqpoint{6.124158in}{0.603398in}}%
\pgfpathcurveto{\pgfqpoint{6.129982in}{0.609222in}}{\pgfqpoint{6.133254in}{0.617122in}}{\pgfqpoint{6.133254in}{0.625358in}}%
\pgfpathcurveto{\pgfqpoint{6.133254in}{0.633594in}}{\pgfqpoint{6.129982in}{0.641495in}}{\pgfqpoint{6.124158in}{0.647318in}}%
\pgfpathcurveto{\pgfqpoint{6.118334in}{0.653142in}}{\pgfqpoint{6.110434in}{0.656415in}}{\pgfqpoint{6.102198in}{0.656415in}}%
\pgfpathcurveto{\pgfqpoint{6.093961in}{0.656415in}}{\pgfqpoint{6.086061in}{0.653142in}}{\pgfqpoint{6.080237in}{0.647318in}}%
\pgfpathcurveto{\pgfqpoint{6.074414in}{0.641495in}}{\pgfqpoint{6.071141in}{0.633594in}}{\pgfqpoint{6.071141in}{0.625358in}}%
\pgfpathcurveto{\pgfqpoint{6.071141in}{0.617122in}}{\pgfqpoint{6.074414in}{0.609222in}}{\pgfqpoint{6.080237in}{0.603398in}}%
\pgfpathcurveto{\pgfqpoint{6.086061in}{0.597574in}}{\pgfqpoint{6.093961in}{0.594302in}}{\pgfqpoint{6.102198in}{0.594302in}}%
\pgfpathclose%
\pgfusepath{stroke,fill}%
\end{pgfscope}%
\begin{pgfscope}%
\pgfpathrectangle{\pgfqpoint{3.717870in}{0.557870in}}{\pgfqpoint{2.522130in}{1.484734in}}%
\pgfusepath{clip}%
\pgfsetbuttcap%
\pgfsetroundjoin%
\definecolor{currentfill}{rgb}{0.298039,0.447059,0.690196}%
\pgfsetfillcolor{currentfill}%
\pgfsetlinewidth{1.003750pt}%
\definecolor{currentstroke}{rgb}{0.298039,0.447059,0.690196}%
\pgfsetstrokecolor{currentstroke}%
\pgfsetdash{}{0pt}%
\pgfpathmoveto{\pgfqpoint{5.731637in}{0.821629in}}%
\pgfpathcurveto{\pgfqpoint{5.739873in}{0.821629in}}{\pgfqpoint{5.747773in}{0.824902in}}{\pgfqpoint{5.753597in}{0.830726in}}%
\pgfpathcurveto{\pgfqpoint{5.759421in}{0.836550in}}{\pgfqpoint{5.762693in}{0.844450in}}{\pgfqpoint{5.762693in}{0.852686in}}%
\pgfpathcurveto{\pgfqpoint{5.762693in}{0.860922in}}{\pgfqpoint{5.759421in}{0.868822in}}{\pgfqpoint{5.753597in}{0.874646in}}%
\pgfpathcurveto{\pgfqpoint{5.747773in}{0.880470in}}{\pgfqpoint{5.739873in}{0.883742in}}{\pgfqpoint{5.731637in}{0.883742in}}%
\pgfpathcurveto{\pgfqpoint{5.723401in}{0.883742in}}{\pgfqpoint{5.715501in}{0.880470in}}{\pgfqpoint{5.709677in}{0.874646in}}%
\pgfpathcurveto{\pgfqpoint{5.703853in}{0.868822in}}{\pgfqpoint{5.700580in}{0.860922in}}{\pgfqpoint{5.700580in}{0.852686in}}%
\pgfpathcurveto{\pgfqpoint{5.700580in}{0.844450in}}{\pgfqpoint{5.703853in}{0.836550in}}{\pgfqpoint{5.709677in}{0.830726in}}%
\pgfpathcurveto{\pgfqpoint{5.715501in}{0.824902in}}{\pgfqpoint{5.723401in}{0.821629in}}{\pgfqpoint{5.731637in}{0.821629in}}%
\pgfpathclose%
\pgfusepath{stroke,fill}%
\end{pgfscope}%
\begin{pgfscope}%
\pgfpathrectangle{\pgfqpoint{3.717870in}{0.557870in}}{\pgfqpoint{2.522130in}{1.484734in}}%
\pgfusepath{clip}%
\pgfsetbuttcap%
\pgfsetroundjoin%
\definecolor{currentfill}{rgb}{0.298039,0.447059,0.690196}%
\pgfsetfillcolor{currentfill}%
\pgfsetlinewidth{1.003750pt}%
\definecolor{currentstroke}{rgb}{0.298039,0.447059,0.690196}%
\pgfsetstrokecolor{currentstroke}%
\pgfsetdash{}{0pt}%
\pgfpathmoveto{\pgfqpoint{6.102198in}{0.594302in}}%
\pgfpathcurveto{\pgfqpoint{6.110434in}{0.594302in}}{\pgfqpoint{6.118334in}{0.597574in}}{\pgfqpoint{6.124158in}{0.603398in}}%
\pgfpathcurveto{\pgfqpoint{6.129982in}{0.609222in}}{\pgfqpoint{6.133254in}{0.617122in}}{\pgfqpoint{6.133254in}{0.625358in}}%
\pgfpathcurveto{\pgfqpoint{6.133254in}{0.633594in}}{\pgfqpoint{6.129982in}{0.641495in}}{\pgfqpoint{6.124158in}{0.647318in}}%
\pgfpathcurveto{\pgfqpoint{6.118334in}{0.653142in}}{\pgfqpoint{6.110434in}{0.656415in}}{\pgfqpoint{6.102198in}{0.656415in}}%
\pgfpathcurveto{\pgfqpoint{6.093961in}{0.656415in}}{\pgfqpoint{6.086061in}{0.653142in}}{\pgfqpoint{6.080237in}{0.647318in}}%
\pgfpathcurveto{\pgfqpoint{6.074414in}{0.641495in}}{\pgfqpoint{6.071141in}{0.633594in}}{\pgfqpoint{6.071141in}{0.625358in}}%
\pgfpathcurveto{\pgfqpoint{6.071141in}{0.617122in}}{\pgfqpoint{6.074414in}{0.609222in}}{\pgfqpoint{6.080237in}{0.603398in}}%
\pgfpathcurveto{\pgfqpoint{6.086061in}{0.597574in}}{\pgfqpoint{6.093961in}{0.594302in}}{\pgfqpoint{6.102198in}{0.594302in}}%
\pgfpathclose%
\pgfusepath{stroke,fill}%
\end{pgfscope}%
\begin{pgfscope}%
\pgfpathrectangle{\pgfqpoint{3.717870in}{0.557870in}}{\pgfqpoint{2.522130in}{1.484734in}}%
\pgfusepath{clip}%
\pgfsetbuttcap%
\pgfsetroundjoin%
\definecolor{currentfill}{rgb}{0.298039,0.447059,0.690196}%
\pgfsetfillcolor{currentfill}%
\pgfsetlinewidth{1.003750pt}%
\definecolor{currentstroke}{rgb}{0.298039,0.447059,0.690196}%
\pgfsetstrokecolor{currentstroke}%
\pgfsetdash{}{0pt}%
\pgfpathmoveto{\pgfqpoint{5.523196in}{1.432572in}}%
\pgfpathcurveto{\pgfqpoint{5.531433in}{1.432572in}}{\pgfqpoint{5.539333in}{1.435845in}}{\pgfqpoint{5.545157in}{1.441669in}}%
\pgfpathcurveto{\pgfqpoint{5.550981in}{1.447493in}}{\pgfqpoint{5.554253in}{1.455393in}}{\pgfqpoint{5.554253in}{1.463629in}}%
\pgfpathcurveto{\pgfqpoint{5.554253in}{1.471865in}}{\pgfqpoint{5.550981in}{1.479765in}}{\pgfqpoint{5.545157in}{1.485589in}}%
\pgfpathcurveto{\pgfqpoint{5.539333in}{1.491413in}}{\pgfqpoint{5.531433in}{1.494685in}}{\pgfqpoint{5.523196in}{1.494685in}}%
\pgfpathcurveto{\pgfqpoint{5.514960in}{1.494685in}}{\pgfqpoint{5.507060in}{1.491413in}}{\pgfqpoint{5.501236in}{1.485589in}}%
\pgfpathcurveto{\pgfqpoint{5.495412in}{1.479765in}}{\pgfqpoint{5.492140in}{1.471865in}}{\pgfqpoint{5.492140in}{1.463629in}}%
\pgfpathcurveto{\pgfqpoint{5.492140in}{1.455393in}}{\pgfqpoint{5.495412in}{1.447493in}}{\pgfqpoint{5.501236in}{1.441669in}}%
\pgfpathcurveto{\pgfqpoint{5.507060in}{1.435845in}}{\pgfqpoint{5.514960in}{1.432572in}}{\pgfqpoint{5.523196in}{1.432572in}}%
\pgfpathclose%
\pgfusepath{stroke,fill}%
\end{pgfscope}%
\begin{pgfscope}%
\pgfpathrectangle{\pgfqpoint{3.717870in}{0.557870in}}{\pgfqpoint{2.522130in}{1.484734in}}%
\pgfusepath{clip}%
\pgfsetbuttcap%
\pgfsetroundjoin%
\definecolor{currentfill}{rgb}{0.298039,0.447059,0.690196}%
\pgfsetfillcolor{currentfill}%
\pgfsetlinewidth{1.003750pt}%
\definecolor{currentstroke}{rgb}{0.298039,0.447059,0.690196}%
\pgfsetstrokecolor{currentstroke}%
\pgfsetdash{}{0pt}%
\pgfpathmoveto{\pgfqpoint{5.488456in}{1.401374in}}%
\pgfpathcurveto{\pgfqpoint{5.496693in}{1.401374in}}{\pgfqpoint{5.504593in}{1.404646in}}{\pgfqpoint{5.510417in}{1.410470in}}%
\pgfpathcurveto{\pgfqpoint{5.516241in}{1.416294in}}{\pgfqpoint{5.519513in}{1.424194in}}{\pgfqpoint{5.519513in}{1.432430in}}%
\pgfpathcurveto{\pgfqpoint{5.519513in}{1.440666in}}{\pgfqpoint{5.516241in}{1.448566in}}{\pgfqpoint{5.510417in}{1.454390in}}%
\pgfpathcurveto{\pgfqpoint{5.504593in}{1.460214in}}{\pgfqpoint{5.496693in}{1.463487in}}{\pgfqpoint{5.488456in}{1.463487in}}%
\pgfpathcurveto{\pgfqpoint{5.480220in}{1.463487in}}{\pgfqpoint{5.472320in}{1.460214in}}{\pgfqpoint{5.466496in}{1.454390in}}%
\pgfpathcurveto{\pgfqpoint{5.460672in}{1.448566in}}{\pgfqpoint{5.457400in}{1.440666in}}{\pgfqpoint{5.457400in}{1.432430in}}%
\pgfpathcurveto{\pgfqpoint{5.457400in}{1.424194in}}{\pgfqpoint{5.460672in}{1.416294in}}{\pgfqpoint{5.466496in}{1.410470in}}%
\pgfpathcurveto{\pgfqpoint{5.472320in}{1.404646in}}{\pgfqpoint{5.480220in}{1.401374in}}{\pgfqpoint{5.488456in}{1.401374in}}%
\pgfpathclose%
\pgfusepath{stroke,fill}%
\end{pgfscope}%
\begin{pgfscope}%
\pgfpathrectangle{\pgfqpoint{3.717870in}{0.557870in}}{\pgfqpoint{2.522130in}{1.484734in}}%
\pgfusepath{clip}%
\pgfsetbuttcap%
\pgfsetroundjoin%
\definecolor{currentfill}{rgb}{0.298039,0.447059,0.690196}%
\pgfsetfillcolor{currentfill}%
\pgfsetlinewidth{1.003750pt}%
\definecolor{currentstroke}{rgb}{0.298039,0.447059,0.690196}%
\pgfsetstrokecolor{currentstroke}%
\pgfsetdash{}{0pt}%
\pgfpathmoveto{\pgfqpoint{5.641313in}{1.443119in}}%
\pgfpathcurveto{\pgfqpoint{5.649549in}{1.443119in}}{\pgfqpoint{5.657449in}{1.446391in}}{\pgfqpoint{5.663273in}{1.452215in}}%
\pgfpathcurveto{\pgfqpoint{5.669097in}{1.458039in}}{\pgfqpoint{5.672369in}{1.465939in}}{\pgfqpoint{5.672369in}{1.474175in}}%
\pgfpathcurveto{\pgfqpoint{5.672369in}{1.482411in}}{\pgfqpoint{5.669097in}{1.490311in}}{\pgfqpoint{5.663273in}{1.496135in}}%
\pgfpathcurveto{\pgfqpoint{5.657449in}{1.501959in}}{\pgfqpoint{5.649549in}{1.505232in}}{\pgfqpoint{5.641313in}{1.505232in}}%
\pgfpathcurveto{\pgfqpoint{5.633076in}{1.505232in}}{\pgfqpoint{5.625176in}{1.501959in}}{\pgfqpoint{5.619352in}{1.496135in}}%
\pgfpathcurveto{\pgfqpoint{5.613528in}{1.490311in}}{\pgfqpoint{5.610256in}{1.482411in}}{\pgfqpoint{5.610256in}{1.474175in}}%
\pgfpathcurveto{\pgfqpoint{5.610256in}{1.465939in}}{\pgfqpoint{5.613528in}{1.458039in}}{\pgfqpoint{5.619352in}{1.452215in}}%
\pgfpathcurveto{\pgfqpoint{5.625176in}{1.446391in}}{\pgfqpoint{5.633076in}{1.443119in}}{\pgfqpoint{5.641313in}{1.443119in}}%
\pgfpathclose%
\pgfusepath{stroke,fill}%
\end{pgfscope}%
\begin{pgfscope}%
\pgfpathrectangle{\pgfqpoint{3.717870in}{0.557870in}}{\pgfqpoint{2.522130in}{1.484734in}}%
\pgfusepath{clip}%
\pgfsetbuttcap%
\pgfsetroundjoin%
\definecolor{currentfill}{rgb}{0.298039,0.447059,0.690196}%
\pgfsetfillcolor{currentfill}%
\pgfsetlinewidth{1.003750pt}%
\definecolor{currentstroke}{rgb}{0.298039,0.447059,0.690196}%
\pgfsetstrokecolor{currentstroke}%
\pgfsetdash{}{0pt}%
\pgfpathmoveto{\pgfqpoint{5.615837in}{1.387458in}}%
\pgfpathcurveto{\pgfqpoint{5.624073in}{1.387458in}}{\pgfqpoint{5.631973in}{1.390731in}}{\pgfqpoint{5.637797in}{1.396555in}}%
\pgfpathcurveto{\pgfqpoint{5.643621in}{1.402379in}}{\pgfqpoint{5.646893in}{1.410279in}}{\pgfqpoint{5.646893in}{1.418515in}}%
\pgfpathcurveto{\pgfqpoint{5.646893in}{1.426751in}}{\pgfqpoint{5.643621in}{1.434651in}}{\pgfqpoint{5.637797in}{1.440475in}}%
\pgfpathcurveto{\pgfqpoint{5.631973in}{1.446299in}}{\pgfqpoint{5.624073in}{1.449571in}}{\pgfqpoint{5.615837in}{1.449571in}}%
\pgfpathcurveto{\pgfqpoint{5.607600in}{1.449571in}}{\pgfqpoint{5.599700in}{1.446299in}}{\pgfqpoint{5.593876in}{1.440475in}}%
\pgfpathcurveto{\pgfqpoint{5.588052in}{1.434651in}}{\pgfqpoint{5.584780in}{1.426751in}}{\pgfqpoint{5.584780in}{1.418515in}}%
\pgfpathcurveto{\pgfqpoint{5.584780in}{1.410279in}}{\pgfqpoint{5.588052in}{1.402379in}}{\pgfqpoint{5.593876in}{1.396555in}}%
\pgfpathcurveto{\pgfqpoint{5.599700in}{1.390731in}}{\pgfqpoint{5.607600in}{1.387458in}}{\pgfqpoint{5.615837in}{1.387458in}}%
\pgfpathclose%
\pgfusepath{stroke,fill}%
\end{pgfscope}%
\begin{pgfscope}%
\pgfpathrectangle{\pgfqpoint{3.717870in}{0.557870in}}{\pgfqpoint{2.522130in}{1.484734in}}%
\pgfusepath{clip}%
\pgfsetbuttcap%
\pgfsetroundjoin%
\definecolor{currentfill}{rgb}{0.298039,0.447059,0.690196}%
\pgfsetfillcolor{currentfill}%
\pgfsetlinewidth{1.003750pt}%
\definecolor{currentstroke}{rgb}{0.298039,0.447059,0.690196}%
\pgfsetstrokecolor{currentstroke}%
\pgfsetdash{}{0pt}%
\pgfpathmoveto{\pgfqpoint{6.099882in}{0.652356in}}%
\pgfpathcurveto{\pgfqpoint{6.108118in}{0.652356in}}{\pgfqpoint{6.116018in}{0.655628in}}{\pgfqpoint{6.121842in}{0.661452in}}%
\pgfpathcurveto{\pgfqpoint{6.127666in}{0.667276in}}{\pgfqpoint{6.130938in}{0.675176in}}{\pgfqpoint{6.130938in}{0.683412in}}%
\pgfpathcurveto{\pgfqpoint{6.130938in}{0.691649in}}{\pgfqpoint{6.127666in}{0.699549in}}{\pgfqpoint{6.121842in}{0.705373in}}%
\pgfpathcurveto{\pgfqpoint{6.116018in}{0.711196in}}{\pgfqpoint{6.108118in}{0.714469in}}{\pgfqpoint{6.099882in}{0.714469in}}%
\pgfpathcurveto{\pgfqpoint{6.091645in}{0.714469in}}{\pgfqpoint{6.083745in}{0.711196in}}{\pgfqpoint{6.077921in}{0.705373in}}%
\pgfpathcurveto{\pgfqpoint{6.072097in}{0.699549in}}{\pgfqpoint{6.068825in}{0.691649in}}{\pgfqpoint{6.068825in}{0.683412in}}%
\pgfpathcurveto{\pgfqpoint{6.068825in}{0.675176in}}{\pgfqpoint{6.072097in}{0.667276in}}{\pgfqpoint{6.077921in}{0.661452in}}%
\pgfpathcurveto{\pgfqpoint{6.083745in}{0.655628in}}{\pgfqpoint{6.091645in}{0.652356in}}{\pgfqpoint{6.099882in}{0.652356in}}%
\pgfpathclose%
\pgfusepath{stroke,fill}%
\end{pgfscope}%
\begin{pgfscope}%
\pgfpathrectangle{\pgfqpoint{3.717870in}{0.557870in}}{\pgfqpoint{2.522130in}{1.484734in}}%
\pgfusepath{clip}%
\pgfsetbuttcap%
\pgfsetroundjoin%
\definecolor{currentfill}{rgb}{0.298039,0.447059,0.690196}%
\pgfsetfillcolor{currentfill}%
\pgfsetlinewidth{1.003750pt}%
\definecolor{currentstroke}{rgb}{0.298039,0.447059,0.690196}%
\pgfsetstrokecolor{currentstroke}%
\pgfsetdash{}{0pt}%
\pgfpathmoveto{\pgfqpoint{3.908941in}{1.929546in}}%
\pgfpathcurveto{\pgfqpoint{3.917177in}{1.929546in}}{\pgfqpoint{3.925077in}{1.932819in}}{\pgfqpoint{3.930901in}{1.938642in}}%
\pgfpathcurveto{\pgfqpoint{3.936725in}{1.944466in}}{\pgfqpoint{3.939997in}{1.952366in}}{\pgfqpoint{3.939997in}{1.960603in}}%
\pgfpathcurveto{\pgfqpoint{3.939997in}{1.968839in}}{\pgfqpoint{3.936725in}{1.976739in}}{\pgfqpoint{3.930901in}{1.982563in}}%
\pgfpathcurveto{\pgfqpoint{3.925077in}{1.988387in}}{\pgfqpoint{3.917177in}{1.991659in}}{\pgfqpoint{3.908941in}{1.991659in}}%
\pgfpathcurveto{\pgfqpoint{3.900705in}{1.991659in}}{\pgfqpoint{3.892805in}{1.988387in}}{\pgfqpoint{3.886981in}{1.982563in}}%
\pgfpathcurveto{\pgfqpoint{3.881157in}{1.976739in}}{\pgfqpoint{3.877884in}{1.968839in}}{\pgfqpoint{3.877884in}{1.960603in}}%
\pgfpathcurveto{\pgfqpoint{3.877884in}{1.952366in}}{\pgfqpoint{3.881157in}{1.944466in}}{\pgfqpoint{3.886981in}{1.938642in}}%
\pgfpathcurveto{\pgfqpoint{3.892805in}{1.932819in}}{\pgfqpoint{3.900705in}{1.929546in}}{\pgfqpoint{3.908941in}{1.929546in}}%
\pgfpathclose%
\pgfusepath{stroke,fill}%
\end{pgfscope}%
\begin{pgfscope}%
\pgfpathrectangle{\pgfqpoint{3.717870in}{0.557870in}}{\pgfqpoint{2.522130in}{1.484734in}}%
\pgfusepath{clip}%
\pgfsetbuttcap%
\pgfsetroundjoin%
\definecolor{currentfill}{rgb}{0.298039,0.447059,0.690196}%
\pgfsetfillcolor{currentfill}%
\pgfsetlinewidth{1.003750pt}%
\definecolor{currentstroke}{rgb}{0.298039,0.447059,0.690196}%
\pgfsetstrokecolor{currentstroke}%
\pgfsetdash{}{0pt}%
\pgfpathmoveto{\pgfqpoint{5.615837in}{1.407059in}}%
\pgfpathcurveto{\pgfqpoint{5.624073in}{1.407059in}}{\pgfqpoint{5.631973in}{1.410332in}}{\pgfqpoint{5.637797in}{1.416155in}}%
\pgfpathcurveto{\pgfqpoint{5.643621in}{1.421979in}}{\pgfqpoint{5.646893in}{1.429879in}}{\pgfqpoint{5.646893in}{1.438116in}}%
\pgfpathcurveto{\pgfqpoint{5.646893in}{1.446352in}}{\pgfqpoint{5.643621in}{1.454252in}}{\pgfqpoint{5.637797in}{1.460076in}}%
\pgfpathcurveto{\pgfqpoint{5.631973in}{1.465900in}}{\pgfqpoint{5.624073in}{1.469172in}}{\pgfqpoint{5.615837in}{1.469172in}}%
\pgfpathcurveto{\pgfqpoint{5.607600in}{1.469172in}}{\pgfqpoint{5.599700in}{1.465900in}}{\pgfqpoint{5.593876in}{1.460076in}}%
\pgfpathcurveto{\pgfqpoint{5.588052in}{1.454252in}}{\pgfqpoint{5.584780in}{1.446352in}}{\pgfqpoint{5.584780in}{1.438116in}}%
\pgfpathcurveto{\pgfqpoint{5.584780in}{1.429879in}}{\pgfqpoint{5.588052in}{1.421979in}}{\pgfqpoint{5.593876in}{1.416155in}}%
\pgfpathcurveto{\pgfqpoint{5.599700in}{1.410332in}}{\pgfqpoint{5.607600in}{1.407059in}}{\pgfqpoint{5.615837in}{1.407059in}}%
\pgfpathclose%
\pgfusepath{stroke,fill}%
\end{pgfscope}%
\begin{pgfscope}%
\pgfpathrectangle{\pgfqpoint{3.717870in}{0.557870in}}{\pgfqpoint{2.522130in}{1.484734in}}%
\pgfusepath{clip}%
\pgfsetbuttcap%
\pgfsetroundjoin%
\definecolor{currentfill}{rgb}{0.298039,0.447059,0.690196}%
\pgfsetfillcolor{currentfill}%
\pgfsetlinewidth{1.003750pt}%
\definecolor{currentstroke}{rgb}{0.298039,0.447059,0.690196}%
\pgfsetstrokecolor{currentstroke}%
\pgfsetdash{}{0pt}%
\pgfpathmoveto{\pgfqpoint{5.476876in}{1.688316in}}%
\pgfpathcurveto{\pgfqpoint{5.485113in}{1.688316in}}{\pgfqpoint{5.493013in}{1.691588in}}{\pgfqpoint{5.498837in}{1.697412in}}%
\pgfpathcurveto{\pgfqpoint{5.504661in}{1.703236in}}{\pgfqpoint{5.507933in}{1.711136in}}{\pgfqpoint{5.507933in}{1.719373in}}%
\pgfpathcurveto{\pgfqpoint{5.507933in}{1.727609in}}{\pgfqpoint{5.504661in}{1.735509in}}{\pgfqpoint{5.498837in}{1.741333in}}%
\pgfpathcurveto{\pgfqpoint{5.493013in}{1.747157in}}{\pgfqpoint{5.485113in}{1.750429in}}{\pgfqpoint{5.476876in}{1.750429in}}%
\pgfpathcurveto{\pgfqpoint{5.468640in}{1.750429in}}{\pgfqpoint{5.460740in}{1.747157in}}{\pgfqpoint{5.454916in}{1.741333in}}%
\pgfpathcurveto{\pgfqpoint{5.449092in}{1.735509in}}{\pgfqpoint{5.445820in}{1.727609in}}{\pgfqpoint{5.445820in}{1.719373in}}%
\pgfpathcurveto{\pgfqpoint{5.445820in}{1.711136in}}{\pgfqpoint{5.449092in}{1.703236in}}{\pgfqpoint{5.454916in}{1.697412in}}%
\pgfpathcurveto{\pgfqpoint{5.460740in}{1.691588in}}{\pgfqpoint{5.468640in}{1.688316in}}{\pgfqpoint{5.476876in}{1.688316in}}%
\pgfpathclose%
\pgfusepath{stroke,fill}%
\end{pgfscope}%
\begin{pgfscope}%
\pgfpathrectangle{\pgfqpoint{3.717870in}{0.557870in}}{\pgfqpoint{2.522130in}{1.484734in}}%
\pgfusepath{clip}%
\pgfsetbuttcap%
\pgfsetroundjoin%
\definecolor{currentfill}{rgb}{0.298039,0.447059,0.690196}%
\pgfsetfillcolor{currentfill}%
\pgfsetlinewidth{1.003750pt}%
\definecolor{currentstroke}{rgb}{0.298039,0.447059,0.690196}%
\pgfsetstrokecolor{currentstroke}%
\pgfsetdash{}{0pt}%
\pgfpathmoveto{\pgfqpoint{5.314756in}{1.745148in}}%
\pgfpathcurveto{\pgfqpoint{5.322992in}{1.745148in}}{\pgfqpoint{5.330892in}{1.748420in}}{\pgfqpoint{5.336716in}{1.754244in}}%
\pgfpathcurveto{\pgfqpoint{5.342540in}{1.760068in}}{\pgfqpoint{5.345812in}{1.767968in}}{\pgfqpoint{5.345812in}{1.776205in}}%
\pgfpathcurveto{\pgfqpoint{5.345812in}{1.784441in}}{\pgfqpoint{5.342540in}{1.792341in}}{\pgfqpoint{5.336716in}{1.798165in}}%
\pgfpathcurveto{\pgfqpoint{5.330892in}{1.803989in}}{\pgfqpoint{5.322992in}{1.807261in}}{\pgfqpoint{5.314756in}{1.807261in}}%
\pgfpathcurveto{\pgfqpoint{5.306520in}{1.807261in}}{\pgfqpoint{5.298620in}{1.803989in}}{\pgfqpoint{5.292796in}{1.798165in}}%
\pgfpathcurveto{\pgfqpoint{5.286972in}{1.792341in}}{\pgfqpoint{5.283699in}{1.784441in}}{\pgfqpoint{5.283699in}{1.776205in}}%
\pgfpathcurveto{\pgfqpoint{5.283699in}{1.767968in}}{\pgfqpoint{5.286972in}{1.760068in}}{\pgfqpoint{5.292796in}{1.754244in}}%
\pgfpathcurveto{\pgfqpoint{5.298620in}{1.748420in}}{\pgfqpoint{5.306520in}{1.745148in}}{\pgfqpoint{5.314756in}{1.745148in}}%
\pgfpathclose%
\pgfusepath{stroke,fill}%
\end{pgfscope}%
\begin{pgfscope}%
\pgfpathrectangle{\pgfqpoint{3.717870in}{0.557870in}}{\pgfqpoint{2.522130in}{1.484734in}}%
\pgfusepath{clip}%
\pgfsetbuttcap%
\pgfsetroundjoin%
\definecolor{currentfill}{rgb}{0.298039,0.447059,0.690196}%
\pgfsetfillcolor{currentfill}%
\pgfsetlinewidth{1.003750pt}%
\definecolor{currentstroke}{rgb}{0.298039,0.447059,0.690196}%
\pgfsetstrokecolor{currentstroke}%
\pgfsetdash{}{0pt}%
\pgfpathmoveto{\pgfqpoint{5.268436in}{1.773564in}}%
\pgfpathcurveto{\pgfqpoint{5.276672in}{1.773564in}}{\pgfqpoint{5.284572in}{1.776836in}}{\pgfqpoint{5.290396in}{1.782660in}}%
\pgfpathcurveto{\pgfqpoint{5.296220in}{1.788484in}}{\pgfqpoint{5.299492in}{1.796384in}}{\pgfqpoint{5.299492in}{1.804621in}}%
\pgfpathcurveto{\pgfqpoint{5.299492in}{1.812857in}}{\pgfqpoint{5.296220in}{1.820757in}}{\pgfqpoint{5.290396in}{1.826581in}}%
\pgfpathcurveto{\pgfqpoint{5.284572in}{1.832405in}}{\pgfqpoint{5.276672in}{1.835677in}}{\pgfqpoint{5.268436in}{1.835677in}}%
\pgfpathcurveto{\pgfqpoint{5.260200in}{1.835677in}}{\pgfqpoint{5.252300in}{1.832405in}}{\pgfqpoint{5.246476in}{1.826581in}}%
\pgfpathcurveto{\pgfqpoint{5.240652in}{1.820757in}}{\pgfqpoint{5.237379in}{1.812857in}}{\pgfqpoint{5.237379in}{1.804621in}}%
\pgfpathcurveto{\pgfqpoint{5.237379in}{1.796384in}}{\pgfqpoint{5.240652in}{1.788484in}}{\pgfqpoint{5.246476in}{1.782660in}}%
\pgfpathcurveto{\pgfqpoint{5.252300in}{1.776836in}}{\pgfqpoint{5.260200in}{1.773564in}}{\pgfqpoint{5.268436in}{1.773564in}}%
\pgfpathclose%
\pgfusepath{stroke,fill}%
\end{pgfscope}%
\begin{pgfscope}%
\pgfpathrectangle{\pgfqpoint{3.717870in}{0.557870in}}{\pgfqpoint{2.522130in}{1.484734in}}%
\pgfusepath{clip}%
\pgfsetbuttcap%
\pgfsetroundjoin%
\definecolor{currentfill}{rgb}{0.298039,0.447059,0.690196}%
\pgfsetfillcolor{currentfill}%
\pgfsetlinewidth{1.003750pt}%
\definecolor{currentstroke}{rgb}{0.298039,0.447059,0.690196}%
\pgfsetstrokecolor{currentstroke}%
\pgfsetdash{}{0pt}%
\pgfpathmoveto{\pgfqpoint{3.832513in}{1.930145in}}%
\pgfpathcurveto{\pgfqpoint{3.840749in}{1.930145in}}{\pgfqpoint{3.848649in}{1.933417in}}{\pgfqpoint{3.854473in}{1.939241in}}%
\pgfpathcurveto{\pgfqpoint{3.860297in}{1.945065in}}{\pgfqpoint{3.863569in}{1.952965in}}{\pgfqpoint{3.863569in}{1.961201in}}%
\pgfpathcurveto{\pgfqpoint{3.863569in}{1.969438in}}{\pgfqpoint{3.860297in}{1.977338in}}{\pgfqpoint{3.854473in}{1.983161in}}%
\pgfpathcurveto{\pgfqpoint{3.848649in}{1.988985in}}{\pgfqpoint{3.840749in}{1.992258in}}{\pgfqpoint{3.832513in}{1.992258in}}%
\pgfpathcurveto{\pgfqpoint{3.824276in}{1.992258in}}{\pgfqpoint{3.816376in}{1.988985in}}{\pgfqpoint{3.810552in}{1.983161in}}%
\pgfpathcurveto{\pgfqpoint{3.804728in}{1.977338in}}{\pgfqpoint{3.801456in}{1.969438in}}{\pgfqpoint{3.801456in}{1.961201in}}%
\pgfpathcurveto{\pgfqpoint{3.801456in}{1.952965in}}{\pgfqpoint{3.804728in}{1.945065in}}{\pgfqpoint{3.810552in}{1.939241in}}%
\pgfpathcurveto{\pgfqpoint{3.816376in}{1.933417in}}{\pgfqpoint{3.824276in}{1.930145in}}{\pgfqpoint{3.832513in}{1.930145in}}%
\pgfpathclose%
\pgfusepath{stroke,fill}%
\end{pgfscope}%
\begin{pgfscope}%
\pgfpathrectangle{\pgfqpoint{3.717870in}{0.557870in}}{\pgfqpoint{2.522130in}{1.484734in}}%
\pgfusepath{clip}%
\pgfsetbuttcap%
\pgfsetroundjoin%
\definecolor{currentfill}{rgb}{0.298039,0.447059,0.690196}%
\pgfsetfillcolor{currentfill}%
\pgfsetlinewidth{1.003750pt}%
\definecolor{currentstroke}{rgb}{0.298039,0.447059,0.690196}%
\pgfsetstrokecolor{currentstroke}%
\pgfsetdash{}{0pt}%
\pgfpathmoveto{\pgfqpoint{5.208220in}{1.749249in}}%
\pgfpathcurveto{\pgfqpoint{5.216456in}{1.749249in}}{\pgfqpoint{5.224356in}{1.752522in}}{\pgfqpoint{5.230180in}{1.758346in}}%
\pgfpathcurveto{\pgfqpoint{5.236004in}{1.764169in}}{\pgfqpoint{5.239276in}{1.772070in}}{\pgfqpoint{5.239276in}{1.780306in}}%
\pgfpathcurveto{\pgfqpoint{5.239276in}{1.788542in}}{\pgfqpoint{5.236004in}{1.796442in}}{\pgfqpoint{5.230180in}{1.802266in}}%
\pgfpathcurveto{\pgfqpoint{5.224356in}{1.808090in}}{\pgfqpoint{5.216456in}{1.811362in}}{\pgfqpoint{5.208220in}{1.811362in}}%
\pgfpathcurveto{\pgfqpoint{5.199983in}{1.811362in}}{\pgfqpoint{5.192083in}{1.808090in}}{\pgfqpoint{5.186259in}{1.802266in}}%
\pgfpathcurveto{\pgfqpoint{5.180436in}{1.796442in}}{\pgfqpoint{5.177163in}{1.788542in}}{\pgfqpoint{5.177163in}{1.780306in}}%
\pgfpathcurveto{\pgfqpoint{5.177163in}{1.772070in}}{\pgfqpoint{5.180436in}{1.764169in}}{\pgfqpoint{5.186259in}{1.758346in}}%
\pgfpathcurveto{\pgfqpoint{5.192083in}{1.752522in}}{\pgfqpoint{5.199983in}{1.749249in}}{\pgfqpoint{5.208220in}{1.749249in}}%
\pgfpathclose%
\pgfusepath{stroke,fill}%
\end{pgfscope}%
\begin{pgfscope}%
\pgfpathrectangle{\pgfqpoint{3.717870in}{0.557870in}}{\pgfqpoint{2.522130in}{1.484734in}}%
\pgfusepath{clip}%
\pgfsetbuttcap%
\pgfsetroundjoin%
\definecolor{currentfill}{rgb}{0.298039,0.447059,0.690196}%
\pgfsetfillcolor{currentfill}%
\pgfsetlinewidth{1.003750pt}%
\definecolor{currentstroke}{rgb}{0.298039,0.447059,0.690196}%
\pgfsetstrokecolor{currentstroke}%
\pgfsetdash{}{0pt}%
\pgfpathmoveto{\pgfqpoint{3.832513in}{1.930145in}}%
\pgfpathcurveto{\pgfqpoint{3.840749in}{1.930145in}}{\pgfqpoint{3.848649in}{1.933417in}}{\pgfqpoint{3.854473in}{1.939241in}}%
\pgfpathcurveto{\pgfqpoint{3.860297in}{1.945065in}}{\pgfqpoint{3.863569in}{1.952965in}}{\pgfqpoint{3.863569in}{1.961201in}}%
\pgfpathcurveto{\pgfqpoint{3.863569in}{1.969438in}}{\pgfqpoint{3.860297in}{1.977338in}}{\pgfqpoint{3.854473in}{1.983161in}}%
\pgfpathcurveto{\pgfqpoint{3.848649in}{1.988985in}}{\pgfqpoint{3.840749in}{1.992258in}}{\pgfqpoint{3.832513in}{1.992258in}}%
\pgfpathcurveto{\pgfqpoint{3.824276in}{1.992258in}}{\pgfqpoint{3.816376in}{1.988985in}}{\pgfqpoint{3.810552in}{1.983161in}}%
\pgfpathcurveto{\pgfqpoint{3.804728in}{1.977338in}}{\pgfqpoint{3.801456in}{1.969438in}}{\pgfqpoint{3.801456in}{1.961201in}}%
\pgfpathcurveto{\pgfqpoint{3.801456in}{1.952965in}}{\pgfqpoint{3.804728in}{1.945065in}}{\pgfqpoint{3.810552in}{1.939241in}}%
\pgfpathcurveto{\pgfqpoint{3.816376in}{1.933417in}}{\pgfqpoint{3.824276in}{1.930145in}}{\pgfqpoint{3.832513in}{1.930145in}}%
\pgfpathclose%
\pgfusepath{stroke,fill}%
\end{pgfscope}%
\begin{pgfscope}%
\pgfpathrectangle{\pgfqpoint{3.717870in}{0.557870in}}{\pgfqpoint{2.522130in}{1.484734in}}%
\pgfusepath{clip}%
\pgfsetbuttcap%
\pgfsetroundjoin%
\definecolor{currentfill}{rgb}{0.298039,0.447059,0.690196}%
\pgfsetfillcolor{currentfill}%
\pgfsetlinewidth{1.003750pt}%
\definecolor{currentstroke}{rgb}{0.298039,0.447059,0.690196}%
\pgfsetstrokecolor{currentstroke}%
\pgfsetdash{}{0pt}%
\pgfpathmoveto{\pgfqpoint{5.692265in}{1.401374in}}%
\pgfpathcurveto{\pgfqpoint{5.700501in}{1.401374in}}{\pgfqpoint{5.708401in}{1.404646in}}{\pgfqpoint{5.714225in}{1.410470in}}%
\pgfpathcurveto{\pgfqpoint{5.720049in}{1.416294in}}{\pgfqpoint{5.723321in}{1.424194in}}{\pgfqpoint{5.723321in}{1.432430in}}%
\pgfpathcurveto{\pgfqpoint{5.723321in}{1.440666in}}{\pgfqpoint{5.720049in}{1.448566in}}{\pgfqpoint{5.714225in}{1.454390in}}%
\pgfpathcurveto{\pgfqpoint{5.708401in}{1.460214in}}{\pgfqpoint{5.700501in}{1.463487in}}{\pgfqpoint{5.692265in}{1.463487in}}%
\pgfpathcurveto{\pgfqpoint{5.684029in}{1.463487in}}{\pgfqpoint{5.676128in}{1.460214in}}{\pgfqpoint{5.670305in}{1.454390in}}%
\pgfpathcurveto{\pgfqpoint{5.664481in}{1.448566in}}{\pgfqpoint{5.661208in}{1.440666in}}{\pgfqpoint{5.661208in}{1.432430in}}%
\pgfpathcurveto{\pgfqpoint{5.661208in}{1.424194in}}{\pgfqpoint{5.664481in}{1.416294in}}{\pgfqpoint{5.670305in}{1.410470in}}%
\pgfpathcurveto{\pgfqpoint{5.676128in}{1.404646in}}{\pgfqpoint{5.684029in}{1.401374in}}{\pgfqpoint{5.692265in}{1.401374in}}%
\pgfpathclose%
\pgfusepath{stroke,fill}%
\end{pgfscope}%
\begin{pgfscope}%
\pgfpathrectangle{\pgfqpoint{3.717870in}{0.557870in}}{\pgfqpoint{2.522130in}{1.484734in}}%
\pgfusepath{clip}%
\pgfsetbuttcap%
\pgfsetroundjoin%
\definecolor{currentfill}{rgb}{0.298039,0.447059,0.690196}%
\pgfsetfillcolor{currentfill}%
\pgfsetlinewidth{1.003750pt}%
\definecolor{currentstroke}{rgb}{0.298039,0.447059,0.690196}%
\pgfsetstrokecolor{currentstroke}%
\pgfsetdash{}{0pt}%
\pgfpathmoveto{\pgfqpoint{3.832513in}{1.929546in}}%
\pgfpathcurveto{\pgfqpoint{3.840749in}{1.929546in}}{\pgfqpoint{3.848649in}{1.932819in}}{\pgfqpoint{3.854473in}{1.938642in}}%
\pgfpathcurveto{\pgfqpoint{3.860297in}{1.944466in}}{\pgfqpoint{3.863569in}{1.952366in}}{\pgfqpoint{3.863569in}{1.960603in}}%
\pgfpathcurveto{\pgfqpoint{3.863569in}{1.968839in}}{\pgfqpoint{3.860297in}{1.976739in}}{\pgfqpoint{3.854473in}{1.982563in}}%
\pgfpathcurveto{\pgfqpoint{3.848649in}{1.988387in}}{\pgfqpoint{3.840749in}{1.991659in}}{\pgfqpoint{3.832513in}{1.991659in}}%
\pgfpathcurveto{\pgfqpoint{3.824276in}{1.991659in}}{\pgfqpoint{3.816376in}{1.988387in}}{\pgfqpoint{3.810552in}{1.982563in}}%
\pgfpathcurveto{\pgfqpoint{3.804728in}{1.976739in}}{\pgfqpoint{3.801456in}{1.968839in}}{\pgfqpoint{3.801456in}{1.960603in}}%
\pgfpathcurveto{\pgfqpoint{3.801456in}{1.952366in}}{\pgfqpoint{3.804728in}{1.944466in}}{\pgfqpoint{3.810552in}{1.938642in}}%
\pgfpathcurveto{\pgfqpoint{3.816376in}{1.932819in}}{\pgfqpoint{3.824276in}{1.929546in}}{\pgfqpoint{3.832513in}{1.929546in}}%
\pgfpathclose%
\pgfusepath{stroke,fill}%
\end{pgfscope}%
\begin{pgfscope}%
\pgfpathrectangle{\pgfqpoint{3.717870in}{0.557870in}}{\pgfqpoint{2.522130in}{1.484734in}}%
\pgfusepath{clip}%
\pgfsetbuttcap%
\pgfsetroundjoin%
\definecolor{currentfill}{rgb}{0.298039,0.447059,0.690196}%
\pgfsetfillcolor{currentfill}%
\pgfsetlinewidth{1.003750pt}%
\definecolor{currentstroke}{rgb}{0.298039,0.447059,0.690196}%
\pgfsetstrokecolor{currentstroke}%
\pgfsetdash{}{0pt}%
\pgfpathmoveto{\pgfqpoint{5.386552in}{1.610249in}}%
\pgfpathcurveto{\pgfqpoint{5.394788in}{1.610249in}}{\pgfqpoint{5.402688in}{1.613521in}}{\pgfqpoint{5.408512in}{1.619345in}}%
\pgfpathcurveto{\pgfqpoint{5.414336in}{1.625169in}}{\pgfqpoint{5.417609in}{1.633069in}}{\pgfqpoint{5.417609in}{1.641305in}}%
\pgfpathcurveto{\pgfqpoint{5.417609in}{1.649541in}}{\pgfqpoint{5.414336in}{1.657441in}}{\pgfqpoint{5.408512in}{1.663265in}}%
\pgfpathcurveto{\pgfqpoint{5.402688in}{1.669089in}}{\pgfqpoint{5.394788in}{1.672362in}}{\pgfqpoint{5.386552in}{1.672362in}}%
\pgfpathcurveto{\pgfqpoint{5.378316in}{1.672362in}}{\pgfqpoint{5.370416in}{1.669089in}}{\pgfqpoint{5.364592in}{1.663265in}}%
\pgfpathcurveto{\pgfqpoint{5.358768in}{1.657441in}}{\pgfqpoint{5.355496in}{1.649541in}}{\pgfqpoint{5.355496in}{1.641305in}}%
\pgfpathcurveto{\pgfqpoint{5.355496in}{1.633069in}}{\pgfqpoint{5.358768in}{1.625169in}}{\pgfqpoint{5.364592in}{1.619345in}}%
\pgfpathcurveto{\pgfqpoint{5.370416in}{1.613521in}}{\pgfqpoint{5.378316in}{1.610249in}}{\pgfqpoint{5.386552in}{1.610249in}}%
\pgfpathclose%
\pgfusepath{stroke,fill}%
\end{pgfscope}%
\begin{pgfscope}%
\pgfpathrectangle{\pgfqpoint{3.717870in}{0.557870in}}{\pgfqpoint{2.522130in}{1.484734in}}%
\pgfusepath{clip}%
\pgfsetbuttcap%
\pgfsetroundjoin%
\definecolor{currentfill}{rgb}{0.298039,0.447059,0.690196}%
\pgfsetfillcolor{currentfill}%
\pgfsetlinewidth{1.003750pt}%
\definecolor{currentstroke}{rgb}{0.298039,0.447059,0.690196}%
\pgfsetstrokecolor{currentstroke}%
\pgfsetdash{}{0pt}%
\pgfpathmoveto{\pgfqpoint{3.832513in}{1.929546in}}%
\pgfpathcurveto{\pgfqpoint{3.840749in}{1.929546in}}{\pgfqpoint{3.848649in}{1.932819in}}{\pgfqpoint{3.854473in}{1.938642in}}%
\pgfpathcurveto{\pgfqpoint{3.860297in}{1.944466in}}{\pgfqpoint{3.863569in}{1.952366in}}{\pgfqpoint{3.863569in}{1.960603in}}%
\pgfpathcurveto{\pgfqpoint{3.863569in}{1.968839in}}{\pgfqpoint{3.860297in}{1.976739in}}{\pgfqpoint{3.854473in}{1.982563in}}%
\pgfpathcurveto{\pgfqpoint{3.848649in}{1.988387in}}{\pgfqpoint{3.840749in}{1.991659in}}{\pgfqpoint{3.832513in}{1.991659in}}%
\pgfpathcurveto{\pgfqpoint{3.824276in}{1.991659in}}{\pgfqpoint{3.816376in}{1.988387in}}{\pgfqpoint{3.810552in}{1.982563in}}%
\pgfpathcurveto{\pgfqpoint{3.804728in}{1.976739in}}{\pgfqpoint{3.801456in}{1.968839in}}{\pgfqpoint{3.801456in}{1.960603in}}%
\pgfpathcurveto{\pgfqpoint{3.801456in}{1.952366in}}{\pgfqpoint{3.804728in}{1.944466in}}{\pgfqpoint{3.810552in}{1.938642in}}%
\pgfpathcurveto{\pgfqpoint{3.816376in}{1.932819in}}{\pgfqpoint{3.824276in}{1.929546in}}{\pgfqpoint{3.832513in}{1.929546in}}%
\pgfpathclose%
\pgfusepath{stroke,fill}%
\end{pgfscope}%
\begin{pgfscope}%
\pgfpathrectangle{\pgfqpoint{3.717870in}{0.557870in}}{\pgfqpoint{2.522130in}{1.484734in}}%
\pgfusepath{clip}%
\pgfsetbuttcap%
\pgfsetroundjoin%
\definecolor{currentfill}{rgb}{0.298039,0.447059,0.690196}%
\pgfsetfillcolor{currentfill}%
\pgfsetlinewidth{1.003750pt}%
\definecolor{currentstroke}{rgb}{0.298039,0.447059,0.690196}%
\pgfsetstrokecolor{currentstroke}%
\pgfsetdash{}{0pt}%
\pgfpathmoveto{\pgfqpoint{5.208220in}{1.726357in}}%
\pgfpathcurveto{\pgfqpoint{5.216456in}{1.726357in}}{\pgfqpoint{5.224356in}{1.729629in}}{\pgfqpoint{5.230180in}{1.735453in}}%
\pgfpathcurveto{\pgfqpoint{5.236004in}{1.741277in}}{\pgfqpoint{5.239276in}{1.749177in}}{\pgfqpoint{5.239276in}{1.757413in}}%
\pgfpathcurveto{\pgfqpoint{5.239276in}{1.765650in}}{\pgfqpoint{5.236004in}{1.773550in}}{\pgfqpoint{5.230180in}{1.779374in}}%
\pgfpathcurveto{\pgfqpoint{5.224356in}{1.785198in}}{\pgfqpoint{5.216456in}{1.788470in}}{\pgfqpoint{5.208220in}{1.788470in}}%
\pgfpathcurveto{\pgfqpoint{5.199983in}{1.788470in}}{\pgfqpoint{5.192083in}{1.785198in}}{\pgfqpoint{5.186259in}{1.779374in}}%
\pgfpathcurveto{\pgfqpoint{5.180436in}{1.773550in}}{\pgfqpoint{5.177163in}{1.765650in}}{\pgfqpoint{5.177163in}{1.757413in}}%
\pgfpathcurveto{\pgfqpoint{5.177163in}{1.749177in}}{\pgfqpoint{5.180436in}{1.741277in}}{\pgfqpoint{5.186259in}{1.735453in}}%
\pgfpathcurveto{\pgfqpoint{5.192083in}{1.729629in}}{\pgfqpoint{5.199983in}{1.726357in}}{\pgfqpoint{5.208220in}{1.726357in}}%
\pgfpathclose%
\pgfusepath{stroke,fill}%
\end{pgfscope}%
\begin{pgfscope}%
\pgfsetrectcap%
\pgfsetmiterjoin%
\pgfsetlinewidth{1.254687pt}%
\definecolor{currentstroke}{rgb}{1.000000,1.000000,1.000000}%
\pgfsetstrokecolor{currentstroke}%
\pgfsetdash{}{0pt}%
\pgfpathmoveto{\pgfqpoint{3.717870in}{0.557870in}}%
\pgfpathlineto{\pgfqpoint{3.717870in}{2.042604in}}%
\pgfusepath{stroke}%
\end{pgfscope}%
\begin{pgfscope}%
\pgfsetrectcap%
\pgfsetmiterjoin%
\pgfsetlinewidth{1.254687pt}%
\definecolor{currentstroke}{rgb}{1.000000,1.000000,1.000000}%
\pgfsetstrokecolor{currentstroke}%
\pgfsetdash{}{0pt}%
\pgfpathmoveto{\pgfqpoint{6.240000in}{0.557870in}}%
\pgfpathlineto{\pgfqpoint{6.240000in}{2.042604in}}%
\pgfusepath{stroke}%
\end{pgfscope}%
\begin{pgfscope}%
\pgfsetrectcap%
\pgfsetmiterjoin%
\pgfsetlinewidth{1.254687pt}%
\definecolor{currentstroke}{rgb}{1.000000,1.000000,1.000000}%
\pgfsetstrokecolor{currentstroke}%
\pgfsetdash{}{0pt}%
\pgfpathmoveto{\pgfqpoint{3.717870in}{0.557870in}}%
\pgfpathlineto{\pgfqpoint{6.240000in}{0.557870in}}%
\pgfusepath{stroke}%
\end{pgfscope}%
\begin{pgfscope}%
\pgfsetrectcap%
\pgfsetmiterjoin%
\pgfsetlinewidth{1.254687pt}%
\definecolor{currentstroke}{rgb}{1.000000,1.000000,1.000000}%
\pgfsetstrokecolor{currentstroke}%
\pgfsetdash{}{0pt}%
\pgfpathmoveto{\pgfqpoint{3.717870in}{2.042604in}}%
\pgfpathlineto{\pgfqpoint{6.240000in}{2.042604in}}%
\pgfusepath{stroke}%
\end{pgfscope}%
\begin{pgfscope}%
\definecolor{textcolor}{rgb}{0.150000,0.150000,0.150000}%
\pgfsetstrokecolor{textcolor}%
\pgfsetfillcolor{textcolor}%
\pgftext[x=4.978935in,y=2.125938in,,base]{\color{textcolor}\sffamily\fontsize{11.000000}{13.200000}\selectfont (b)}%
\end{pgfscope}%
\end{pgfpicture}%
\makeatother%
\endgroup%

    \caption{(a) Distribution plot of DOR of all PVC methods evaluated at two cluster centers when applied to classify heart failure.
             (b) Scatter plot of the same methods sensitivity-, and specificity-scores.}
    \label{fig:pvc_hf_dor_sens_spec_dist}
\end{figure}

\begin{table*}[htb]
    \centering
    \ra{1.3}
    \begin{tabular}{p{3.5cm}p{1.5cm}p{1.65cm}p{1.65cm}p{1cm}}
        \toprule
        Dataset-Method    &  Accuracy &  Sensitivity &  Specificity &   DOR \\
        \midrule
        gls-EF/ward/2     &      0.75 &         0.87 &         0.63 & 11.59 \\
        gls-EF/complete/2 &      0.76 &         0.81 &         0.72 & 10.85 \\
        gls-EF/average/2  &      0.75 &         0.85 &         0.65 & 10.58 \\
        rls-EF/complete/2 &      0.73 &         0.86 &         0.60 &  8.89 \\
        gls-rls-EF/ward/2 &      0.72 &         0.84 &         0.60 &  7.80 \\
        \bottomrule
    \end{tabular}
    \caption{The accuracy, DOR, sensitivity and specicity scores of the five best performing two-cluster-center PVC methods in terms of DOR, at detecting heart failure.
             The \textbf{Dataset-Method} column indicates \textit{Dataset used}$/$\textit{Linkage criteria of method}$/$\textit{Number of cluster centers}.}
    \label{tab:pvc_hf_dor_sens_spec_dist}
\end{table*}

From figure \ref{fig:pvc_hf_dor_sens_spec_dist}a one can see that the DOR scores are substantially higher for PVC methods evaluated at two cluster centers to predict heart failure, 
than they are for the corresponding TSC methods. 
The scatterplot in figure \ref{fig:pvc_hf_dor_sens_spec_dist}b also shows that their exist multiple methods with both sensitivity and specificity above $0.6$ for the same PVC methods.
The exact metrics for the top performing PVC methods at predicting heart failure are given in table \ref{tab:pvc_hf_dor_sens_spec_dist}.
Common to the three highest performing PVC methods is that they all use the dataset that is a combination of peak systolic GLS values and EF values.
The highest DOR recorded is acheived when using the Ward linkage criteria, but it is not given that this is the ''best'' method.
The \textit{gls-EF/complete/2} method acheives a specificity score that is nine points higher at the cost of the sensitivity being six points lower, 
and it also has the highest overall accuracy of all the PVC methods by a small margin of one point. 
To get a better idea of how the different cluster methods perform at identifying heart failure, a scatterplot of the clusters is depicted in figure 
\ref{fig:scatter_gls_ef_hf_cluster_assignments}. \bigskip

\begin{figure}[htb]
    \centering
    \begin{subfigure}[b]{0.49\textwidth}
        \centering
        \includegraphics[width=0.99\textwidth]{results/scatter_gls_EF_hf.png}
        \caption{Heart failure.}
        \label{fig:scatter_gls_ef_hf}
    \end{subfigure}
    \begin{subfigure}[b]{0.49\textwidth}
        \centering
        \includegraphics[width=0.99\textwidth]{results/scatter_gls_EF_ward2.png}
        \caption{\textit{Ward/2} cluster assignments.}
        \label{fig:scatter_gls_ef_ward2}
    \end{subfigure}\\
    \begin{subfigure}[b]{0.49\textwidth}
        \centering
        \includegraphics[width=0.99\textwidth]{results/scatter_gls_EF_complete2.png}
        \caption{\textit{Complete/2} cluster assignments.}
        \label{fig:scatter_gls_ef_complete2}
    \end{subfigure}
    \begin{subfigure}[b]{0.49\textwidth}
        \centering
        \includegraphics[width=0.99\textwidth]{results/scatter_gls_EF_average2.png}
        \caption{\textit{Average/2} cluster assignments.}
        \label{fig:scatter_gls_ef_average2}
    \end{subfigure}
    \caption{Scatterplot of peak GLS values in each view. Colors in the of the different dots are given by heart failure diagnosis, and cluster assignments of 
             ward/2, complete/2 and average/2 methods. Numbers are not included on the axes because the point of the figure is to illustrate the separability 
             of clusters, and heart failure.}
             \label{fig:scatter_gls_ef_hf_cluster_assignments}
\end{figure}

In figure \ref{fig:scatter_gls_ef_hf_cluster_assignments} scatterplots patients are plotted with the dimensions: 4-chamber peak systolic GLS, 2-chamber peak systolic GLS and EF. 
The colors of the points correspond to wheather the patient has heart failure or not, and which cluster the points belong to.
The plots are actually a lower dimensional projection of the GLS-EF peak-value dataset. 
This particular projection was chosen as it was found to be the projection where heart failure patients were as separable as possible. 
From plots \ref{fig:scatter_gls_ef_hf_cluster_assignments}b-d one can see that the clusters are fairly separable, 
heart failure on the other hand is not as easy to separate in these dimensions as can be seen in plot \ref{fig:scatter_gls_ef_average2}. 
\textit{Ward/2} and \textit{Complete/2} can in some sense be considered as binary classifiers where values under a certain threshold are categorized as heart failure.
The \textit{ward/2} method has the highest threshold for what is considered heart failure, and \textit{complete/2} has the lowest, 
which explains their difference in sensitivity and specificity score. 
From figure \ref{fig:pvc_hf_ari} one can see that the many of the ARI of PVC methods for classifying heart failure are close to zero, but substantially more of the methods score above zero in ARI
than the TSC methods, as can be seen by a comparison of figure \ref{fig:tsc_hf_ari} and \ref{fig:pvc_hf_ari}. Table \ref{tab:pvc_hf_ari} shows that the three highest ARIs are attained by the same
three methods that acheived the highest DORs. This means that there are most likely no methods evaluated at a higher number of cluster centers that will outperform \textit{ward/2},
or \textit{complete/2} at classifying heart failure. In addition, the conclusion will be that \textit{complete/2} is the best performing PVC method when classifying heart failure, 
since it has the highest overall accuracy (76$\%$), highest ARI (0.27), and second highest DOR (10.85). 

\begin{figure}[htb]
    \centering
    % \includegraphics[width=\textwidth]{results/pvc_hf_ari.png}
    %% Creator: Matplotlib, PGF backend
%%
%% To include the figure in your LaTeX document, write
%%   \input{<filename>.pgf}
%%
%% Make sure the required packages are loaded in your preamble
%%   \usepackage{pgf}
%%
%% Figures using additional raster images can only be included by \input if
%% they are in the same directory as the main LaTeX file. For loading figures
%% from other directories you can use the `import` package
%%   \usepackage{import}
%% and then include the figures with
%%   \import{<path to file>}{<filename>.pgf}
%%
%% Matplotlib used the following preamble
%%
\begingroup%
\makeatletter%
\begin{pgfpicture}%
\pgfpathrectangle{\pgfpointorigin}{\pgfqpoint{6.340000in}{2.340000in}}%
\pgfusepath{use as bounding box, clip}%
\begin{pgfscope}%
\pgfsetbuttcap%
\pgfsetmiterjoin%
\definecolor{currentfill}{rgb}{1.000000,1.000000,1.000000}%
\pgfsetfillcolor{currentfill}%
\pgfsetlinewidth{0.000000pt}%
\definecolor{currentstroke}{rgb}{1.000000,1.000000,1.000000}%
\pgfsetstrokecolor{currentstroke}%
\pgfsetdash{}{0pt}%
\pgfpathmoveto{\pgfqpoint{0.000000in}{-0.000000in}}%
\pgfpathlineto{\pgfqpoint{6.340000in}{-0.000000in}}%
\pgfpathlineto{\pgfqpoint{6.340000in}{2.340000in}}%
\pgfpathlineto{\pgfqpoint{0.000000in}{2.340000in}}%
\pgfpathclose%
\pgfusepath{fill}%
\end{pgfscope}%
\begin{pgfscope}%
\pgfsetbuttcap%
\pgfsetmiterjoin%
\definecolor{currentfill}{rgb}{0.917647,0.917647,0.949020}%
\pgfsetfillcolor{currentfill}%
\pgfsetlinewidth{0.000000pt}%
\definecolor{currentstroke}{rgb}{0.000000,0.000000,0.000000}%
\pgfsetstrokecolor{currentstroke}%
\pgfsetstrokeopacity{0.000000}%
\pgfsetdash{}{0pt}%
\pgfpathmoveto{\pgfqpoint{0.574769in}{0.557870in}}%
\pgfpathlineto{\pgfqpoint{6.240000in}{0.557870in}}%
\pgfpathlineto{\pgfqpoint{6.240000in}{2.240000in}}%
\pgfpathlineto{\pgfqpoint{0.574769in}{2.240000in}}%
\pgfpathclose%
\pgfusepath{fill}%
\end{pgfscope}%
\begin{pgfscope}%
\pgfpathrectangle{\pgfqpoint{0.574769in}{0.557870in}}{\pgfqpoint{5.665231in}{1.682130in}}%
\pgfusepath{clip}%
\pgfsetroundcap%
\pgfsetroundjoin%
\pgfsetlinewidth{1.003750pt}%
\definecolor{currentstroke}{rgb}{1.000000,1.000000,1.000000}%
\pgfsetstrokecolor{currentstroke}%
\pgfsetdash{}{0pt}%
\pgfpathmoveto{\pgfqpoint{0.895162in}{0.557870in}}%
\pgfpathlineto{\pgfqpoint{0.895162in}{2.240000in}}%
\pgfusepath{stroke}%
\end{pgfscope}%
\begin{pgfscope}%
\definecolor{textcolor}{rgb}{0.150000,0.150000,0.150000}%
\pgfsetstrokecolor{textcolor}%
\pgfsetfillcolor{textcolor}%
\pgftext[x=0.895162in,y=0.425926in,,top]{\color{textcolor}\sffamily\fontsize{11.000000}{13.200000}\selectfont \(\displaystyle 0.00\)}%
\end{pgfscope}%
\begin{pgfscope}%
\pgfpathrectangle{\pgfqpoint{0.574769in}{0.557870in}}{\pgfqpoint{5.665231in}{1.682130in}}%
\pgfusepath{clip}%
\pgfsetroundcap%
\pgfsetroundjoin%
\pgfsetlinewidth{1.003750pt}%
\definecolor{currentstroke}{rgb}{1.000000,1.000000,1.000000}%
\pgfsetstrokecolor{currentstroke}%
\pgfsetdash{}{0pt}%
\pgfpathmoveto{\pgfqpoint{1.819167in}{0.557870in}}%
\pgfpathlineto{\pgfqpoint{1.819167in}{2.240000in}}%
\pgfusepath{stroke}%
\end{pgfscope}%
\begin{pgfscope}%
\definecolor{textcolor}{rgb}{0.150000,0.150000,0.150000}%
\pgfsetstrokecolor{textcolor}%
\pgfsetfillcolor{textcolor}%
\pgftext[x=1.819167in,y=0.425926in,,top]{\color{textcolor}\sffamily\fontsize{11.000000}{13.200000}\selectfont \(\displaystyle 0.05\)}%
\end{pgfscope}%
\begin{pgfscope}%
\pgfpathrectangle{\pgfqpoint{0.574769in}{0.557870in}}{\pgfqpoint{5.665231in}{1.682130in}}%
\pgfusepath{clip}%
\pgfsetroundcap%
\pgfsetroundjoin%
\pgfsetlinewidth{1.003750pt}%
\definecolor{currentstroke}{rgb}{1.000000,1.000000,1.000000}%
\pgfsetstrokecolor{currentstroke}%
\pgfsetdash{}{0pt}%
\pgfpathmoveto{\pgfqpoint{2.743172in}{0.557870in}}%
\pgfpathlineto{\pgfqpoint{2.743172in}{2.240000in}}%
\pgfusepath{stroke}%
\end{pgfscope}%
\begin{pgfscope}%
\definecolor{textcolor}{rgb}{0.150000,0.150000,0.150000}%
\pgfsetstrokecolor{textcolor}%
\pgfsetfillcolor{textcolor}%
\pgftext[x=2.743172in,y=0.425926in,,top]{\color{textcolor}\sffamily\fontsize{11.000000}{13.200000}\selectfont \(\displaystyle 0.10\)}%
\end{pgfscope}%
\begin{pgfscope}%
\pgfpathrectangle{\pgfqpoint{0.574769in}{0.557870in}}{\pgfqpoint{5.665231in}{1.682130in}}%
\pgfusepath{clip}%
\pgfsetroundcap%
\pgfsetroundjoin%
\pgfsetlinewidth{1.003750pt}%
\definecolor{currentstroke}{rgb}{1.000000,1.000000,1.000000}%
\pgfsetstrokecolor{currentstroke}%
\pgfsetdash{}{0pt}%
\pgfpathmoveto{\pgfqpoint{3.667178in}{0.557870in}}%
\pgfpathlineto{\pgfqpoint{3.667178in}{2.240000in}}%
\pgfusepath{stroke}%
\end{pgfscope}%
\begin{pgfscope}%
\definecolor{textcolor}{rgb}{0.150000,0.150000,0.150000}%
\pgfsetstrokecolor{textcolor}%
\pgfsetfillcolor{textcolor}%
\pgftext[x=3.667178in,y=0.425926in,,top]{\color{textcolor}\sffamily\fontsize{11.000000}{13.200000}\selectfont \(\displaystyle 0.15\)}%
\end{pgfscope}%
\begin{pgfscope}%
\pgfpathrectangle{\pgfqpoint{0.574769in}{0.557870in}}{\pgfqpoint{5.665231in}{1.682130in}}%
\pgfusepath{clip}%
\pgfsetroundcap%
\pgfsetroundjoin%
\pgfsetlinewidth{1.003750pt}%
\definecolor{currentstroke}{rgb}{1.000000,1.000000,1.000000}%
\pgfsetstrokecolor{currentstroke}%
\pgfsetdash{}{0pt}%
\pgfpathmoveto{\pgfqpoint{4.591183in}{0.557870in}}%
\pgfpathlineto{\pgfqpoint{4.591183in}{2.240000in}}%
\pgfusepath{stroke}%
\end{pgfscope}%
\begin{pgfscope}%
\definecolor{textcolor}{rgb}{0.150000,0.150000,0.150000}%
\pgfsetstrokecolor{textcolor}%
\pgfsetfillcolor{textcolor}%
\pgftext[x=4.591183in,y=0.425926in,,top]{\color{textcolor}\sffamily\fontsize{11.000000}{13.200000}\selectfont \(\displaystyle 0.20\)}%
\end{pgfscope}%
\begin{pgfscope}%
\pgfpathrectangle{\pgfqpoint{0.574769in}{0.557870in}}{\pgfqpoint{5.665231in}{1.682130in}}%
\pgfusepath{clip}%
\pgfsetroundcap%
\pgfsetroundjoin%
\pgfsetlinewidth{1.003750pt}%
\definecolor{currentstroke}{rgb}{1.000000,1.000000,1.000000}%
\pgfsetstrokecolor{currentstroke}%
\pgfsetdash{}{0pt}%
\pgfpathmoveto{\pgfqpoint{5.515188in}{0.557870in}}%
\pgfpathlineto{\pgfqpoint{5.515188in}{2.240000in}}%
\pgfusepath{stroke}%
\end{pgfscope}%
\begin{pgfscope}%
\definecolor{textcolor}{rgb}{0.150000,0.150000,0.150000}%
\pgfsetstrokecolor{textcolor}%
\pgfsetfillcolor{textcolor}%
\pgftext[x=5.515188in,y=0.425926in,,top]{\color{textcolor}\sffamily\fontsize{11.000000}{13.200000}\selectfont \(\displaystyle 0.25\)}%
\end{pgfscope}%
\begin{pgfscope}%
\definecolor{textcolor}{rgb}{0.150000,0.150000,0.150000}%
\pgfsetstrokecolor{textcolor}%
\pgfsetfillcolor{textcolor}%
\pgftext[x=3.407384in,y=0.235185in,,top]{\color{textcolor}\sffamily\fontsize{11.000000}{13.200000}\selectfont ARI}%
\end{pgfscope}%
\begin{pgfscope}%
\pgfpathrectangle{\pgfqpoint{0.574769in}{0.557870in}}{\pgfqpoint{5.665231in}{1.682130in}}%
\pgfusepath{clip}%
\pgfsetroundcap%
\pgfsetroundjoin%
\pgfsetlinewidth{1.003750pt}%
\definecolor{currentstroke}{rgb}{1.000000,1.000000,1.000000}%
\pgfsetstrokecolor{currentstroke}%
\pgfsetdash{}{0pt}%
\pgfpathmoveto{\pgfqpoint{0.574769in}{0.557870in}}%
\pgfpathlineto{\pgfqpoint{6.240000in}{0.557870in}}%
\pgfusepath{stroke}%
\end{pgfscope}%
\begin{pgfscope}%
\definecolor{textcolor}{rgb}{0.150000,0.150000,0.150000}%
\pgfsetstrokecolor{textcolor}%
\pgfsetfillcolor{textcolor}%
\pgftext[x=0.366783in,y=0.505064in,left,base]{\color{textcolor}\sffamily\fontsize{11.000000}{13.200000}\selectfont \(\displaystyle 0\)}%
\end{pgfscope}%
\begin{pgfscope}%
\pgfpathrectangle{\pgfqpoint{0.574769in}{0.557870in}}{\pgfqpoint{5.665231in}{1.682130in}}%
\pgfusepath{clip}%
\pgfsetroundcap%
\pgfsetroundjoin%
\pgfsetlinewidth{1.003750pt}%
\definecolor{currentstroke}{rgb}{1.000000,1.000000,1.000000}%
\pgfsetstrokecolor{currentstroke}%
\pgfsetdash{}{0pt}%
\pgfpathmoveto{\pgfqpoint{0.574769in}{0.963447in}}%
\pgfpathlineto{\pgfqpoint{6.240000in}{0.963447in}}%
\pgfusepath{stroke}%
\end{pgfscope}%
\begin{pgfscope}%
\definecolor{textcolor}{rgb}{0.150000,0.150000,0.150000}%
\pgfsetstrokecolor{textcolor}%
\pgfsetfillcolor{textcolor}%
\pgftext[x=0.290741in,y=0.910640in,left,base]{\color{textcolor}\sffamily\fontsize{11.000000}{13.200000}\selectfont \(\displaystyle 20\)}%
\end{pgfscope}%
\begin{pgfscope}%
\pgfpathrectangle{\pgfqpoint{0.574769in}{0.557870in}}{\pgfqpoint{5.665231in}{1.682130in}}%
\pgfusepath{clip}%
\pgfsetroundcap%
\pgfsetroundjoin%
\pgfsetlinewidth{1.003750pt}%
\definecolor{currentstroke}{rgb}{1.000000,1.000000,1.000000}%
\pgfsetstrokecolor{currentstroke}%
\pgfsetdash{}{0pt}%
\pgfpathmoveto{\pgfqpoint{0.574769in}{1.369024in}}%
\pgfpathlineto{\pgfqpoint{6.240000in}{1.369024in}}%
\pgfusepath{stroke}%
\end{pgfscope}%
\begin{pgfscope}%
\definecolor{textcolor}{rgb}{0.150000,0.150000,0.150000}%
\pgfsetstrokecolor{textcolor}%
\pgfsetfillcolor{textcolor}%
\pgftext[x=0.290741in,y=1.316217in,left,base]{\color{textcolor}\sffamily\fontsize{11.000000}{13.200000}\selectfont \(\displaystyle 40\)}%
\end{pgfscope}%
\begin{pgfscope}%
\pgfpathrectangle{\pgfqpoint{0.574769in}{0.557870in}}{\pgfqpoint{5.665231in}{1.682130in}}%
\pgfusepath{clip}%
\pgfsetroundcap%
\pgfsetroundjoin%
\pgfsetlinewidth{1.003750pt}%
\definecolor{currentstroke}{rgb}{1.000000,1.000000,1.000000}%
\pgfsetstrokecolor{currentstroke}%
\pgfsetdash{}{0pt}%
\pgfpathmoveto{\pgfqpoint{0.574769in}{1.774601in}}%
\pgfpathlineto{\pgfqpoint{6.240000in}{1.774601in}}%
\pgfusepath{stroke}%
\end{pgfscope}%
\begin{pgfscope}%
\definecolor{textcolor}{rgb}{0.150000,0.150000,0.150000}%
\pgfsetstrokecolor{textcolor}%
\pgfsetfillcolor{textcolor}%
\pgftext[x=0.290741in,y=1.721794in,left,base]{\color{textcolor}\sffamily\fontsize{11.000000}{13.200000}\selectfont \(\displaystyle 60\)}%
\end{pgfscope}%
\begin{pgfscope}%
\pgfpathrectangle{\pgfqpoint{0.574769in}{0.557870in}}{\pgfqpoint{5.665231in}{1.682130in}}%
\pgfusepath{clip}%
\pgfsetroundcap%
\pgfsetroundjoin%
\pgfsetlinewidth{1.003750pt}%
\definecolor{currentstroke}{rgb}{1.000000,1.000000,1.000000}%
\pgfsetstrokecolor{currentstroke}%
\pgfsetdash{}{0pt}%
\pgfpathmoveto{\pgfqpoint{0.574769in}{2.180177in}}%
\pgfpathlineto{\pgfqpoint{6.240000in}{2.180177in}}%
\pgfusepath{stroke}%
\end{pgfscope}%
\begin{pgfscope}%
\definecolor{textcolor}{rgb}{0.150000,0.150000,0.150000}%
\pgfsetstrokecolor{textcolor}%
\pgfsetfillcolor{textcolor}%
\pgftext[x=0.290741in,y=2.127371in,left,base]{\color{textcolor}\sffamily\fontsize{11.000000}{13.200000}\selectfont \(\displaystyle 80\)}%
\end{pgfscope}%
\begin{pgfscope}%
\definecolor{textcolor}{rgb}{0.150000,0.150000,0.150000}%
\pgfsetstrokecolor{textcolor}%
\pgfsetfillcolor{textcolor}%
\pgftext[x=0.235185in,y=1.398935in,,bottom,rotate=90.000000]{\color{textcolor}\sffamily\fontsize{11.000000}{13.200000}\selectfont Occurance}%
\end{pgfscope}%
\begin{pgfscope}%
\pgfpathrectangle{\pgfqpoint{0.574769in}{0.557870in}}{\pgfqpoint{5.665231in}{1.682130in}}%
\pgfusepath{clip}%
\pgfsetbuttcap%
\pgfsetmiterjoin%
\definecolor{currentfill}{rgb}{0.298039,0.447059,0.690196}%
\pgfsetfillcolor{currentfill}%
\pgfsetfillopacity{0.400000}%
\pgfsetlinewidth{1.003750pt}%
\definecolor{currentstroke}{rgb}{1.000000,1.000000,1.000000}%
\pgfsetstrokecolor{currentstroke}%
\pgfsetstrokeopacity{0.400000}%
\pgfsetdash{}{0pt}%
\pgfpathmoveto{\pgfqpoint{0.832279in}{0.557870in}}%
\pgfpathlineto{\pgfqpoint{1.568023in}{0.557870in}}%
\pgfpathlineto{\pgfqpoint{1.568023in}{2.159899in}}%
\pgfpathlineto{\pgfqpoint{0.832279in}{2.159899in}}%
\pgfpathclose%
\pgfusepath{stroke,fill}%
\end{pgfscope}%
\begin{pgfscope}%
\pgfpathrectangle{\pgfqpoint{0.574769in}{0.557870in}}{\pgfqpoint{5.665231in}{1.682130in}}%
\pgfusepath{clip}%
\pgfsetbuttcap%
\pgfsetmiterjoin%
\definecolor{currentfill}{rgb}{0.298039,0.447059,0.690196}%
\pgfsetfillcolor{currentfill}%
\pgfsetfillopacity{0.400000}%
\pgfsetlinewidth{1.003750pt}%
\definecolor{currentstroke}{rgb}{1.000000,1.000000,1.000000}%
\pgfsetstrokecolor{currentstroke}%
\pgfsetstrokeopacity{0.400000}%
\pgfsetdash{}{0pt}%
\pgfpathmoveto{\pgfqpoint{1.568023in}{0.557870in}}%
\pgfpathlineto{\pgfqpoint{2.303768in}{0.557870in}}%
\pgfpathlineto{\pgfqpoint{2.303768in}{0.943168in}}%
\pgfpathlineto{\pgfqpoint{1.568023in}{0.943168in}}%
\pgfpathclose%
\pgfusepath{stroke,fill}%
\end{pgfscope}%
\begin{pgfscope}%
\pgfpathrectangle{\pgfqpoint{0.574769in}{0.557870in}}{\pgfqpoint{5.665231in}{1.682130in}}%
\pgfusepath{clip}%
\pgfsetbuttcap%
\pgfsetmiterjoin%
\definecolor{currentfill}{rgb}{0.298039,0.447059,0.690196}%
\pgfsetfillcolor{currentfill}%
\pgfsetfillopacity{0.400000}%
\pgfsetlinewidth{1.003750pt}%
\definecolor{currentstroke}{rgb}{1.000000,1.000000,1.000000}%
\pgfsetstrokecolor{currentstroke}%
\pgfsetstrokeopacity{0.400000}%
\pgfsetdash{}{0pt}%
\pgfpathmoveto{\pgfqpoint{2.303768in}{0.557870in}}%
\pgfpathlineto{\pgfqpoint{3.039512in}{0.557870in}}%
\pgfpathlineto{\pgfqpoint{3.039512in}{1.551533in}}%
\pgfpathlineto{\pgfqpoint{2.303768in}{1.551533in}}%
\pgfpathclose%
\pgfusepath{stroke,fill}%
\end{pgfscope}%
\begin{pgfscope}%
\pgfpathrectangle{\pgfqpoint{0.574769in}{0.557870in}}{\pgfqpoint{5.665231in}{1.682130in}}%
\pgfusepath{clip}%
\pgfsetbuttcap%
\pgfsetmiterjoin%
\definecolor{currentfill}{rgb}{0.298039,0.447059,0.690196}%
\pgfsetfillcolor{currentfill}%
\pgfsetfillopacity{0.400000}%
\pgfsetlinewidth{1.003750pt}%
\definecolor{currentstroke}{rgb}{1.000000,1.000000,1.000000}%
\pgfsetstrokecolor{currentstroke}%
\pgfsetstrokeopacity{0.400000}%
\pgfsetdash{}{0pt}%
\pgfpathmoveto{\pgfqpoint{3.039512in}{0.557870in}}%
\pgfpathlineto{\pgfqpoint{3.775256in}{0.557870in}}%
\pgfpathlineto{\pgfqpoint{3.775256in}{1.328466in}}%
\pgfpathlineto{\pgfqpoint{3.039512in}{1.328466in}}%
\pgfpathclose%
\pgfusepath{stroke,fill}%
\end{pgfscope}%
\begin{pgfscope}%
\pgfpathrectangle{\pgfqpoint{0.574769in}{0.557870in}}{\pgfqpoint{5.665231in}{1.682130in}}%
\pgfusepath{clip}%
\pgfsetbuttcap%
\pgfsetmiterjoin%
\definecolor{currentfill}{rgb}{0.298039,0.447059,0.690196}%
\pgfsetfillcolor{currentfill}%
\pgfsetfillopacity{0.400000}%
\pgfsetlinewidth{1.003750pt}%
\definecolor{currentstroke}{rgb}{1.000000,1.000000,1.000000}%
\pgfsetstrokecolor{currentstroke}%
\pgfsetstrokeopacity{0.400000}%
\pgfsetdash{}{0pt}%
\pgfpathmoveto{\pgfqpoint{3.775256in}{0.557870in}}%
\pgfpathlineto{\pgfqpoint{4.511001in}{0.557870in}}%
\pgfpathlineto{\pgfqpoint{4.511001in}{1.024284in}}%
\pgfpathlineto{\pgfqpoint{3.775256in}{1.024284in}}%
\pgfpathclose%
\pgfusepath{stroke,fill}%
\end{pgfscope}%
\begin{pgfscope}%
\pgfpathrectangle{\pgfqpoint{0.574769in}{0.557870in}}{\pgfqpoint{5.665231in}{1.682130in}}%
\pgfusepath{clip}%
\pgfsetbuttcap%
\pgfsetmiterjoin%
\definecolor{currentfill}{rgb}{0.298039,0.447059,0.690196}%
\pgfsetfillcolor{currentfill}%
\pgfsetfillopacity{0.400000}%
\pgfsetlinewidth{1.003750pt}%
\definecolor{currentstroke}{rgb}{1.000000,1.000000,1.000000}%
\pgfsetstrokecolor{currentstroke}%
\pgfsetstrokeopacity{0.400000}%
\pgfsetdash{}{0pt}%
\pgfpathmoveto{\pgfqpoint{4.511001in}{0.557870in}}%
\pgfpathlineto{\pgfqpoint{5.246745in}{0.557870in}}%
\pgfpathlineto{\pgfqpoint{5.246745in}{0.659264in}}%
\pgfpathlineto{\pgfqpoint{4.511001in}{0.659264in}}%
\pgfpathclose%
\pgfusepath{stroke,fill}%
\end{pgfscope}%
\begin{pgfscope}%
\pgfpathrectangle{\pgfqpoint{0.574769in}{0.557870in}}{\pgfqpoint{5.665231in}{1.682130in}}%
\pgfusepath{clip}%
\pgfsetbuttcap%
\pgfsetmiterjoin%
\definecolor{currentfill}{rgb}{0.298039,0.447059,0.690196}%
\pgfsetfillcolor{currentfill}%
\pgfsetfillopacity{0.400000}%
\pgfsetlinewidth{1.003750pt}%
\definecolor{currentstroke}{rgb}{1.000000,1.000000,1.000000}%
\pgfsetstrokecolor{currentstroke}%
\pgfsetstrokeopacity{0.400000}%
\pgfsetdash{}{0pt}%
\pgfpathmoveto{\pgfqpoint{5.246745in}{0.557870in}}%
\pgfpathlineto{\pgfqpoint{5.982489in}{0.557870in}}%
\pgfpathlineto{\pgfqpoint{5.982489in}{0.618707in}}%
\pgfpathlineto{\pgfqpoint{5.246745in}{0.618707in}}%
\pgfpathclose%
\pgfusepath{stroke,fill}%
\end{pgfscope}%
\begin{pgfscope}%
\pgfsetrectcap%
\pgfsetmiterjoin%
\pgfsetlinewidth{1.254687pt}%
\definecolor{currentstroke}{rgb}{1.000000,1.000000,1.000000}%
\pgfsetstrokecolor{currentstroke}%
\pgfsetdash{}{0pt}%
\pgfpathmoveto{\pgfqpoint{0.574769in}{0.557870in}}%
\pgfpathlineto{\pgfqpoint{0.574769in}{2.240000in}}%
\pgfusepath{stroke}%
\end{pgfscope}%
\begin{pgfscope}%
\pgfsetrectcap%
\pgfsetmiterjoin%
\pgfsetlinewidth{1.254687pt}%
\definecolor{currentstroke}{rgb}{1.000000,1.000000,1.000000}%
\pgfsetstrokecolor{currentstroke}%
\pgfsetdash{}{0pt}%
\pgfpathmoveto{\pgfqpoint{6.240000in}{0.557870in}}%
\pgfpathlineto{\pgfqpoint{6.240000in}{2.240000in}}%
\pgfusepath{stroke}%
\end{pgfscope}%
\begin{pgfscope}%
\pgfsetrectcap%
\pgfsetmiterjoin%
\pgfsetlinewidth{1.254687pt}%
\definecolor{currentstroke}{rgb}{1.000000,1.000000,1.000000}%
\pgfsetstrokecolor{currentstroke}%
\pgfsetdash{}{0pt}%
\pgfpathmoveto{\pgfqpoint{0.574769in}{0.557870in}}%
\pgfpathlineto{\pgfqpoint{6.240000in}{0.557870in}}%
\pgfusepath{stroke}%
\end{pgfscope}%
\begin{pgfscope}%
\pgfsetrectcap%
\pgfsetmiterjoin%
\pgfsetlinewidth{1.254687pt}%
\definecolor{currentstroke}{rgb}{1.000000,1.000000,1.000000}%
\pgfsetstrokecolor{currentstroke}%
\pgfsetdash{}{0pt}%
\pgfpathmoveto{\pgfqpoint{0.574769in}{2.240000in}}%
\pgfpathlineto{\pgfqpoint{6.240000in}{2.240000in}}%
\pgfusepath{stroke}%
\end{pgfscope}%
\end{pgfpicture}%
\makeatother%
\endgroup%

    \caption{Distribution plot of ARI of all PVC methods evaluated at $\{2,9\}$ cluster centers when applied to classify heart failure.}
    \label{fig:pvc_hf_ari}
\end{figure}

\begin{table*}[htb]
    \centering
    \ra{1.3}
    \begin{tabular}{lr}
        \toprule
        Dataset-Method    &  ARI \\
        \midrule
        gls-EF/complete/2 & 0.27 \\
        gls-EF/ward/2     & 0.24 \\
        gls-EF/average/2  & 0.24 \\
        rls-EF/complete/2 & 0.21 \\
        gls-EF/complete/3 & 0.21 \\
        \bottomrule
    \end{tabular}
    \caption{The five highest ARI scores attained when applying PVC for detecting heart failure.
             The \textbf{Dataset-Method} column indicates \textit{Dataset used}$/$\textit{Linkage criteria of method}$/$\textit{Number of cluster centers}.}
    \label{tab:pvc_hf_ari}
\end{table*}

\newpage 

\subsection{Deep Neural Network}

From the distribution plot in figure \ref{fig:dl_hf_dor_sens_spec_dist}a one can see that the most frequent DOR by the NN models is zero when training them to predict heart failure.
In the scatterplot in figure \ref{fig:dl_hf_dor_sens_spec_dist}b one can see that sensitivity scores vary between $0.15$ and $0.65$, and the specificity scores vary between $0$ and $0.68$.
The highest DOR of $1.36$ is attained by using only the GLS curve from the 4-chamber view as input, as can be seen from table \ref{tab:dl_hf_dor_sens_spec_dist}.
In fact the five highest DORs attained by NN models trained to classify heart failure are acheived using only curves from a single view as input.
There does not seem to be a particular view that is favored, as 4-chamber view, 2-chamber view and apical-view are are all found in the NN variations in table \ref{tab:dl_hf_dor_sens_spec_dist}.
The overall accuracy of the model variations are also fairly low, $0.54$ being the highest accuracy acheived.
Since the heart failure dataset is fairly evenly distribution (recall figure \ref{fig:hf_ind_dist}) an accuracy of $0.54$ is not much better than what could be acheived 
by randomly guessing the label.

\begin{figure}[htb]
    \centering
    % \includegraphics[width=\textwidth]{results/dl_hf_dor_sens_spec_dist.png}
    %% Creator: Matplotlib, PGF backend
%%
%% To include the figure in your LaTeX document, write
%%   \input{<filename>.pgf}
%%
%% Make sure the required packages are loaded in your preamble
%%   \usepackage{pgf}
%%
%% Figures using additional raster images can only be included by \input if
%% they are in the same directory as the main LaTeX file. For loading figures
%% from other directories you can use the `import` package
%%   \usepackage{import}
%% and then include the figures with
%%   \import{<path to file>}{<filename>.pgf}
%%
%% Matplotlib used the following preamble
%%
\begingroup%
\makeatletter%
\begin{pgfpicture}%
\pgfpathrectangle{\pgfpointorigin}{\pgfqpoint{6.246672in}{2.340000in}}%
\pgfusepath{use as bounding box, clip}%
\begin{pgfscope}%
\pgfsetbuttcap%
\pgfsetmiterjoin%
\definecolor{currentfill}{rgb}{1.000000,1.000000,1.000000}%
\pgfsetfillcolor{currentfill}%
\pgfsetlinewidth{0.000000pt}%
\definecolor{currentstroke}{rgb}{1.000000,1.000000,1.000000}%
\pgfsetstrokecolor{currentstroke}%
\pgfsetdash{}{0pt}%
\pgfpathmoveto{\pgfqpoint{0.000000in}{-0.000000in}}%
\pgfpathlineto{\pgfqpoint{6.246672in}{-0.000000in}}%
\pgfpathlineto{\pgfqpoint{6.246672in}{2.340000in}}%
\pgfpathlineto{\pgfqpoint{0.000000in}{2.340000in}}%
\pgfpathclose%
\pgfusepath{fill}%
\end{pgfscope}%
\begin{pgfscope}%
\pgfsetbuttcap%
\pgfsetmiterjoin%
\definecolor{currentfill}{rgb}{0.917647,0.917647,0.949020}%
\pgfsetfillcolor{currentfill}%
\pgfsetlinewidth{0.000000pt}%
\definecolor{currentstroke}{rgb}{0.000000,0.000000,0.000000}%
\pgfsetstrokecolor{currentstroke}%
\pgfsetstrokeopacity{0.000000}%
\pgfsetdash{}{0pt}%
\pgfpathmoveto{\pgfqpoint{0.574769in}{0.557870in}}%
\pgfpathlineto{\pgfqpoint{2.999734in}{0.557870in}}%
\pgfpathlineto{\pgfqpoint{2.999734in}{2.042604in}}%
\pgfpathlineto{\pgfqpoint{0.574769in}{2.042604in}}%
\pgfpathclose%
\pgfusepath{fill}%
\end{pgfscope}%
\begin{pgfscope}%
\pgfpathrectangle{\pgfqpoint{0.574769in}{0.557870in}}{\pgfqpoint{2.424965in}{1.484734in}}%
\pgfusepath{clip}%
\pgfsetroundcap%
\pgfsetroundjoin%
\pgfsetlinewidth{1.003750pt}%
\definecolor{currentstroke}{rgb}{1.000000,1.000000,1.000000}%
\pgfsetstrokecolor{currentstroke}%
\pgfsetdash{}{0pt}%
\pgfpathmoveto{\pgfqpoint{0.684994in}{0.557870in}}%
\pgfpathlineto{\pgfqpoint{0.684994in}{2.042604in}}%
\pgfusepath{stroke}%
\end{pgfscope}%
\begin{pgfscope}%
\definecolor{textcolor}{rgb}{0.150000,0.150000,0.150000}%
\pgfsetstrokecolor{textcolor}%
\pgfsetfillcolor{textcolor}%
\pgftext[x=0.684994in,y=0.425926in,,top]{\color{textcolor}\sffamily\fontsize{11.000000}{13.200000}\selectfont \(\displaystyle 0.0\)}%
\end{pgfscope}%
\begin{pgfscope}%
\pgfpathrectangle{\pgfqpoint{0.574769in}{0.557870in}}{\pgfqpoint{2.424965in}{1.484734in}}%
\pgfusepath{clip}%
\pgfsetroundcap%
\pgfsetroundjoin%
\pgfsetlinewidth{1.003750pt}%
\definecolor{currentstroke}{rgb}{1.000000,1.000000,1.000000}%
\pgfsetstrokecolor{currentstroke}%
\pgfsetdash{}{0pt}%
\pgfpathmoveto{\pgfqpoint{1.496956in}{0.557870in}}%
\pgfpathlineto{\pgfqpoint{1.496956in}{2.042604in}}%
\pgfusepath{stroke}%
\end{pgfscope}%
\begin{pgfscope}%
\definecolor{textcolor}{rgb}{0.150000,0.150000,0.150000}%
\pgfsetstrokecolor{textcolor}%
\pgfsetfillcolor{textcolor}%
\pgftext[x=1.496956in,y=0.425926in,,top]{\color{textcolor}\sffamily\fontsize{11.000000}{13.200000}\selectfont \(\displaystyle 0.5\)}%
\end{pgfscope}%
\begin{pgfscope}%
\pgfpathrectangle{\pgfqpoint{0.574769in}{0.557870in}}{\pgfqpoint{2.424965in}{1.484734in}}%
\pgfusepath{clip}%
\pgfsetroundcap%
\pgfsetroundjoin%
\pgfsetlinewidth{1.003750pt}%
\definecolor{currentstroke}{rgb}{1.000000,1.000000,1.000000}%
\pgfsetstrokecolor{currentstroke}%
\pgfsetdash{}{0pt}%
\pgfpathmoveto{\pgfqpoint{2.308918in}{0.557870in}}%
\pgfpathlineto{\pgfqpoint{2.308918in}{2.042604in}}%
\pgfusepath{stroke}%
\end{pgfscope}%
\begin{pgfscope}%
\definecolor{textcolor}{rgb}{0.150000,0.150000,0.150000}%
\pgfsetstrokecolor{textcolor}%
\pgfsetfillcolor{textcolor}%
\pgftext[x=2.308918in,y=0.425926in,,top]{\color{textcolor}\sffamily\fontsize{11.000000}{13.200000}\selectfont \(\displaystyle 1.0\)}%
\end{pgfscope}%
\begin{pgfscope}%
\definecolor{textcolor}{rgb}{0.150000,0.150000,0.150000}%
\pgfsetstrokecolor{textcolor}%
\pgfsetfillcolor{textcolor}%
\pgftext[x=1.787251in,y=0.235185in,,top]{\color{textcolor}\sffamily\fontsize{11.000000}{13.200000}\selectfont DOR}%
\end{pgfscope}%
\begin{pgfscope}%
\pgfpathrectangle{\pgfqpoint{0.574769in}{0.557870in}}{\pgfqpoint{2.424965in}{1.484734in}}%
\pgfusepath{clip}%
\pgfsetroundcap%
\pgfsetroundjoin%
\pgfsetlinewidth{1.003750pt}%
\definecolor{currentstroke}{rgb}{1.000000,1.000000,1.000000}%
\pgfsetstrokecolor{currentstroke}%
\pgfsetdash{}{0pt}%
\pgfpathmoveto{\pgfqpoint{0.574769in}{0.557870in}}%
\pgfpathlineto{\pgfqpoint{2.999734in}{0.557870in}}%
\pgfusepath{stroke}%
\end{pgfscope}%
\begin{pgfscope}%
\definecolor{textcolor}{rgb}{0.150000,0.150000,0.150000}%
\pgfsetstrokecolor{textcolor}%
\pgfsetfillcolor{textcolor}%
\pgftext[x=0.366783in,y=0.505064in,left,base]{\color{textcolor}\sffamily\fontsize{11.000000}{13.200000}\selectfont \(\displaystyle 0\)}%
\end{pgfscope}%
\begin{pgfscope}%
\pgfpathrectangle{\pgfqpoint{0.574769in}{0.557870in}}{\pgfqpoint{2.424965in}{1.484734in}}%
\pgfusepath{clip}%
\pgfsetroundcap%
\pgfsetroundjoin%
\pgfsetlinewidth{1.003750pt}%
\definecolor{currentstroke}{rgb}{1.000000,1.000000,1.000000}%
\pgfsetstrokecolor{currentstroke}%
\pgfsetdash{}{0pt}%
\pgfpathmoveto{\pgfqpoint{0.574769in}{1.101729in}}%
\pgfpathlineto{\pgfqpoint{2.999734in}{1.101729in}}%
\pgfusepath{stroke}%
\end{pgfscope}%
\begin{pgfscope}%
\definecolor{textcolor}{rgb}{0.150000,0.150000,0.150000}%
\pgfsetstrokecolor{textcolor}%
\pgfsetfillcolor{textcolor}%
\pgftext[x=0.366783in,y=1.048922in,left,base]{\color{textcolor}\sffamily\fontsize{11.000000}{13.200000}\selectfont \(\displaystyle 5\)}%
\end{pgfscope}%
\begin{pgfscope}%
\pgfpathrectangle{\pgfqpoint{0.574769in}{0.557870in}}{\pgfqpoint{2.424965in}{1.484734in}}%
\pgfusepath{clip}%
\pgfsetroundcap%
\pgfsetroundjoin%
\pgfsetlinewidth{1.003750pt}%
\definecolor{currentstroke}{rgb}{1.000000,1.000000,1.000000}%
\pgfsetstrokecolor{currentstroke}%
\pgfsetdash{}{0pt}%
\pgfpathmoveto{\pgfqpoint{0.574769in}{1.645587in}}%
\pgfpathlineto{\pgfqpoint{2.999734in}{1.645587in}}%
\pgfusepath{stroke}%
\end{pgfscope}%
\begin{pgfscope}%
\definecolor{textcolor}{rgb}{0.150000,0.150000,0.150000}%
\pgfsetstrokecolor{textcolor}%
\pgfsetfillcolor{textcolor}%
\pgftext[x=0.290741in,y=1.592781in,left,base]{\color{textcolor}\sffamily\fontsize{11.000000}{13.200000}\selectfont \(\displaystyle 10\)}%
\end{pgfscope}%
\begin{pgfscope}%
\definecolor{textcolor}{rgb}{0.150000,0.150000,0.150000}%
\pgfsetstrokecolor{textcolor}%
\pgfsetfillcolor{textcolor}%
\pgftext[x=0.235185in,y=1.300237in,,bottom,rotate=90.000000]{\color{textcolor}\sffamily\fontsize{11.000000}{13.200000}\selectfont Occurance}%
\end{pgfscope}%
\begin{pgfscope}%
\pgfpathrectangle{\pgfqpoint{0.574769in}{0.557870in}}{\pgfqpoint{2.424965in}{1.484734in}}%
\pgfusepath{clip}%
\pgfsetbuttcap%
\pgfsetmiterjoin%
\definecolor{currentfill}{rgb}{0.298039,0.447059,0.690196}%
\pgfsetfillcolor{currentfill}%
\pgfsetfillopacity{0.400000}%
\pgfsetlinewidth{1.003750pt}%
\definecolor{currentstroke}{rgb}{1.000000,1.000000,1.000000}%
\pgfsetstrokecolor{currentstroke}%
\pgfsetstrokeopacity{0.400000}%
\pgfsetdash{}{0pt}%
\pgfpathmoveto{\pgfqpoint{0.684994in}{0.557870in}}%
\pgfpathlineto{\pgfqpoint{0.905446in}{0.557870in}}%
\pgfpathlineto{\pgfqpoint{0.905446in}{1.971903in}}%
\pgfpathlineto{\pgfqpoint{0.684994in}{1.971903in}}%
\pgfpathclose%
\pgfusepath{stroke,fill}%
\end{pgfscope}%
\begin{pgfscope}%
\pgfpathrectangle{\pgfqpoint{0.574769in}{0.557870in}}{\pgfqpoint{2.424965in}{1.484734in}}%
\pgfusepath{clip}%
\pgfsetbuttcap%
\pgfsetmiterjoin%
\definecolor{currentfill}{rgb}{0.298039,0.447059,0.690196}%
\pgfsetfillcolor{currentfill}%
\pgfsetfillopacity{0.400000}%
\pgfsetlinewidth{1.003750pt}%
\definecolor{currentstroke}{rgb}{1.000000,1.000000,1.000000}%
\pgfsetstrokecolor{currentstroke}%
\pgfsetstrokeopacity{0.400000}%
\pgfsetdash{}{0pt}%
\pgfpathmoveto{\pgfqpoint{0.905446in}{0.557870in}}%
\pgfpathlineto{\pgfqpoint{1.125897in}{0.557870in}}%
\pgfpathlineto{\pgfqpoint{1.125897in}{1.101729in}}%
\pgfpathlineto{\pgfqpoint{0.905446in}{1.101729in}}%
\pgfpathclose%
\pgfusepath{stroke,fill}%
\end{pgfscope}%
\begin{pgfscope}%
\pgfpathrectangle{\pgfqpoint{0.574769in}{0.557870in}}{\pgfqpoint{2.424965in}{1.484734in}}%
\pgfusepath{clip}%
\pgfsetbuttcap%
\pgfsetmiterjoin%
\definecolor{currentfill}{rgb}{0.298039,0.447059,0.690196}%
\pgfsetfillcolor{currentfill}%
\pgfsetfillopacity{0.400000}%
\pgfsetlinewidth{1.003750pt}%
\definecolor{currentstroke}{rgb}{1.000000,1.000000,1.000000}%
\pgfsetstrokecolor{currentstroke}%
\pgfsetstrokeopacity{0.400000}%
\pgfsetdash{}{0pt}%
\pgfpathmoveto{\pgfqpoint{1.125897in}{0.557870in}}%
\pgfpathlineto{\pgfqpoint{1.346348in}{0.557870in}}%
\pgfpathlineto{\pgfqpoint{1.346348in}{1.210501in}}%
\pgfpathlineto{\pgfqpoint{1.125897in}{1.210501in}}%
\pgfpathclose%
\pgfusepath{stroke,fill}%
\end{pgfscope}%
\begin{pgfscope}%
\pgfpathrectangle{\pgfqpoint{0.574769in}{0.557870in}}{\pgfqpoint{2.424965in}{1.484734in}}%
\pgfusepath{clip}%
\pgfsetbuttcap%
\pgfsetmiterjoin%
\definecolor{currentfill}{rgb}{0.298039,0.447059,0.690196}%
\pgfsetfillcolor{currentfill}%
\pgfsetfillopacity{0.400000}%
\pgfsetlinewidth{1.003750pt}%
\definecolor{currentstroke}{rgb}{1.000000,1.000000,1.000000}%
\pgfsetstrokecolor{currentstroke}%
\pgfsetstrokeopacity{0.400000}%
\pgfsetdash{}{0pt}%
\pgfpathmoveto{\pgfqpoint{1.346348in}{0.557870in}}%
\pgfpathlineto{\pgfqpoint{1.566800in}{0.557870in}}%
\pgfpathlineto{\pgfqpoint{1.566800in}{0.884185in}}%
\pgfpathlineto{\pgfqpoint{1.346348in}{0.884185in}}%
\pgfpathclose%
\pgfusepath{stroke,fill}%
\end{pgfscope}%
\begin{pgfscope}%
\pgfpathrectangle{\pgfqpoint{0.574769in}{0.557870in}}{\pgfqpoint{2.424965in}{1.484734in}}%
\pgfusepath{clip}%
\pgfsetbuttcap%
\pgfsetmiterjoin%
\definecolor{currentfill}{rgb}{0.298039,0.447059,0.690196}%
\pgfsetfillcolor{currentfill}%
\pgfsetfillopacity{0.400000}%
\pgfsetlinewidth{1.003750pt}%
\definecolor{currentstroke}{rgb}{1.000000,1.000000,1.000000}%
\pgfsetstrokecolor{currentstroke}%
\pgfsetstrokeopacity{0.400000}%
\pgfsetdash{}{0pt}%
\pgfpathmoveto{\pgfqpoint{1.566800in}{0.557870in}}%
\pgfpathlineto{\pgfqpoint{1.787251in}{0.557870in}}%
\pgfpathlineto{\pgfqpoint{1.787251in}{0.557870in}}%
\pgfpathlineto{\pgfqpoint{1.566800in}{0.557870in}}%
\pgfpathclose%
\pgfusepath{stroke,fill}%
\end{pgfscope}%
\begin{pgfscope}%
\pgfpathrectangle{\pgfqpoint{0.574769in}{0.557870in}}{\pgfqpoint{2.424965in}{1.484734in}}%
\pgfusepath{clip}%
\pgfsetbuttcap%
\pgfsetmiterjoin%
\definecolor{currentfill}{rgb}{0.298039,0.447059,0.690196}%
\pgfsetfillcolor{currentfill}%
\pgfsetfillopacity{0.400000}%
\pgfsetlinewidth{1.003750pt}%
\definecolor{currentstroke}{rgb}{1.000000,1.000000,1.000000}%
\pgfsetstrokecolor{currentstroke}%
\pgfsetstrokeopacity{0.400000}%
\pgfsetdash{}{0pt}%
\pgfpathmoveto{\pgfqpoint{1.787251in}{0.557870in}}%
\pgfpathlineto{\pgfqpoint{2.007703in}{0.557870in}}%
\pgfpathlineto{\pgfqpoint{2.007703in}{0.666642in}}%
\pgfpathlineto{\pgfqpoint{1.787251in}{0.666642in}}%
\pgfpathclose%
\pgfusepath{stroke,fill}%
\end{pgfscope}%
\begin{pgfscope}%
\pgfpathrectangle{\pgfqpoint{0.574769in}{0.557870in}}{\pgfqpoint{2.424965in}{1.484734in}}%
\pgfusepath{clip}%
\pgfsetbuttcap%
\pgfsetmiterjoin%
\definecolor{currentfill}{rgb}{0.298039,0.447059,0.690196}%
\pgfsetfillcolor{currentfill}%
\pgfsetfillopacity{0.400000}%
\pgfsetlinewidth{1.003750pt}%
\definecolor{currentstroke}{rgb}{1.000000,1.000000,1.000000}%
\pgfsetstrokecolor{currentstroke}%
\pgfsetstrokeopacity{0.400000}%
\pgfsetdash{}{0pt}%
\pgfpathmoveto{\pgfqpoint{2.007703in}{0.557870in}}%
\pgfpathlineto{\pgfqpoint{2.228154in}{0.557870in}}%
\pgfpathlineto{\pgfqpoint{2.228154in}{0.775414in}}%
\pgfpathlineto{\pgfqpoint{2.007703in}{0.775414in}}%
\pgfpathclose%
\pgfusepath{stroke,fill}%
\end{pgfscope}%
\begin{pgfscope}%
\pgfpathrectangle{\pgfqpoint{0.574769in}{0.557870in}}{\pgfqpoint{2.424965in}{1.484734in}}%
\pgfusepath{clip}%
\pgfsetbuttcap%
\pgfsetmiterjoin%
\definecolor{currentfill}{rgb}{0.298039,0.447059,0.690196}%
\pgfsetfillcolor{currentfill}%
\pgfsetfillopacity{0.400000}%
\pgfsetlinewidth{1.003750pt}%
\definecolor{currentstroke}{rgb}{1.000000,1.000000,1.000000}%
\pgfsetstrokecolor{currentstroke}%
\pgfsetstrokeopacity{0.400000}%
\pgfsetdash{}{0pt}%
\pgfpathmoveto{\pgfqpoint{2.228154in}{0.557870in}}%
\pgfpathlineto{\pgfqpoint{2.448605in}{0.557870in}}%
\pgfpathlineto{\pgfqpoint{2.448605in}{0.775414in}}%
\pgfpathlineto{\pgfqpoint{2.228154in}{0.775414in}}%
\pgfpathclose%
\pgfusepath{stroke,fill}%
\end{pgfscope}%
\begin{pgfscope}%
\pgfpathrectangle{\pgfqpoint{0.574769in}{0.557870in}}{\pgfqpoint{2.424965in}{1.484734in}}%
\pgfusepath{clip}%
\pgfsetbuttcap%
\pgfsetmiterjoin%
\definecolor{currentfill}{rgb}{0.298039,0.447059,0.690196}%
\pgfsetfillcolor{currentfill}%
\pgfsetfillopacity{0.400000}%
\pgfsetlinewidth{1.003750pt}%
\definecolor{currentstroke}{rgb}{1.000000,1.000000,1.000000}%
\pgfsetstrokecolor{currentstroke}%
\pgfsetstrokeopacity{0.400000}%
\pgfsetdash{}{0pt}%
\pgfpathmoveto{\pgfqpoint{2.448605in}{0.557870in}}%
\pgfpathlineto{\pgfqpoint{2.669057in}{0.557870in}}%
\pgfpathlineto{\pgfqpoint{2.669057in}{0.775414in}}%
\pgfpathlineto{\pgfqpoint{2.448605in}{0.775414in}}%
\pgfpathclose%
\pgfusepath{stroke,fill}%
\end{pgfscope}%
\begin{pgfscope}%
\pgfpathrectangle{\pgfqpoint{0.574769in}{0.557870in}}{\pgfqpoint{2.424965in}{1.484734in}}%
\pgfusepath{clip}%
\pgfsetbuttcap%
\pgfsetmiterjoin%
\definecolor{currentfill}{rgb}{0.298039,0.447059,0.690196}%
\pgfsetfillcolor{currentfill}%
\pgfsetfillopacity{0.400000}%
\pgfsetlinewidth{1.003750pt}%
\definecolor{currentstroke}{rgb}{1.000000,1.000000,1.000000}%
\pgfsetstrokecolor{currentstroke}%
\pgfsetstrokeopacity{0.400000}%
\pgfsetdash{}{0pt}%
\pgfpathmoveto{\pgfqpoint{2.669057in}{0.557870in}}%
\pgfpathlineto{\pgfqpoint{2.889508in}{0.557870in}}%
\pgfpathlineto{\pgfqpoint{2.889508in}{0.775414in}}%
\pgfpathlineto{\pgfqpoint{2.669057in}{0.775414in}}%
\pgfpathclose%
\pgfusepath{stroke,fill}%
\end{pgfscope}%
\begin{pgfscope}%
\pgfsetrectcap%
\pgfsetmiterjoin%
\pgfsetlinewidth{1.254687pt}%
\definecolor{currentstroke}{rgb}{1.000000,1.000000,1.000000}%
\pgfsetstrokecolor{currentstroke}%
\pgfsetdash{}{0pt}%
\pgfpathmoveto{\pgfqpoint{0.574769in}{0.557870in}}%
\pgfpathlineto{\pgfqpoint{0.574769in}{2.042604in}}%
\pgfusepath{stroke}%
\end{pgfscope}%
\begin{pgfscope}%
\pgfsetrectcap%
\pgfsetmiterjoin%
\pgfsetlinewidth{1.254687pt}%
\definecolor{currentstroke}{rgb}{1.000000,1.000000,1.000000}%
\pgfsetstrokecolor{currentstroke}%
\pgfsetdash{}{0pt}%
\pgfpathmoveto{\pgfqpoint{2.999734in}{0.557870in}}%
\pgfpathlineto{\pgfqpoint{2.999734in}{2.042604in}}%
\pgfusepath{stroke}%
\end{pgfscope}%
\begin{pgfscope}%
\pgfsetrectcap%
\pgfsetmiterjoin%
\pgfsetlinewidth{1.254687pt}%
\definecolor{currentstroke}{rgb}{1.000000,1.000000,1.000000}%
\pgfsetstrokecolor{currentstroke}%
\pgfsetdash{}{0pt}%
\pgfpathmoveto{\pgfqpoint{0.574769in}{0.557870in}}%
\pgfpathlineto{\pgfqpoint{2.999734in}{0.557870in}}%
\pgfusepath{stroke}%
\end{pgfscope}%
\begin{pgfscope}%
\pgfsetrectcap%
\pgfsetmiterjoin%
\pgfsetlinewidth{1.254687pt}%
\definecolor{currentstroke}{rgb}{1.000000,1.000000,1.000000}%
\pgfsetstrokecolor{currentstroke}%
\pgfsetdash{}{0pt}%
\pgfpathmoveto{\pgfqpoint{0.574769in}{2.042604in}}%
\pgfpathlineto{\pgfqpoint{2.999734in}{2.042604in}}%
\pgfusepath{stroke}%
\end{pgfscope}%
\begin{pgfscope}%
\definecolor{textcolor}{rgb}{0.150000,0.150000,0.150000}%
\pgfsetstrokecolor{textcolor}%
\pgfsetfillcolor{textcolor}%
\pgftext[x=1.787251in,y=2.125938in,,base]{\color{textcolor}\sffamily\fontsize{11.000000}{13.200000}\selectfont (a)}%
\end{pgfscope}%
\begin{pgfscope}%
\pgfsetbuttcap%
\pgfsetmiterjoin%
\definecolor{currentfill}{rgb}{0.917647,0.917647,0.949020}%
\pgfsetfillcolor{currentfill}%
\pgfsetlinewidth{0.000000pt}%
\definecolor{currentstroke}{rgb}{0.000000,0.000000,0.000000}%
\pgfsetstrokecolor{currentstroke}%
\pgfsetstrokeopacity{0.000000}%
\pgfsetdash{}{0pt}%
\pgfpathmoveto{\pgfqpoint{3.696748in}{0.557870in}}%
\pgfpathlineto{\pgfqpoint{6.121713in}{0.557870in}}%
\pgfpathlineto{\pgfqpoint{6.121713in}{2.042604in}}%
\pgfpathlineto{\pgfqpoint{3.696748in}{2.042604in}}%
\pgfpathclose%
\pgfusepath{fill}%
\end{pgfscope}%
\begin{pgfscope}%
\pgfpathrectangle{\pgfqpoint{3.696748in}{0.557870in}}{\pgfqpoint{2.424965in}{1.484734in}}%
\pgfusepath{clip}%
\pgfsetroundcap%
\pgfsetroundjoin%
\pgfsetlinewidth{1.003750pt}%
\definecolor{currentstroke}{rgb}{1.000000,1.000000,1.000000}%
\pgfsetstrokecolor{currentstroke}%
\pgfsetdash{}{0pt}%
\pgfpathmoveto{\pgfqpoint{3.806973in}{0.557870in}}%
\pgfpathlineto{\pgfqpoint{3.806973in}{2.042604in}}%
\pgfusepath{stroke}%
\end{pgfscope}%
\begin{pgfscope}%
\definecolor{textcolor}{rgb}{0.150000,0.150000,0.150000}%
\pgfsetstrokecolor{textcolor}%
\pgfsetfillcolor{textcolor}%
\pgftext[x=3.806973in,y=0.425926in,,top]{\color{textcolor}\sffamily\fontsize{11.000000}{13.200000}\selectfont \(\displaystyle 0.00\)}%
\end{pgfscope}%
\begin{pgfscope}%
\pgfpathrectangle{\pgfqpoint{3.696748in}{0.557870in}}{\pgfqpoint{2.424965in}{1.484734in}}%
\pgfusepath{clip}%
\pgfsetroundcap%
\pgfsetroundjoin%
\pgfsetlinewidth{1.003750pt}%
\definecolor{currentstroke}{rgb}{1.000000,1.000000,1.000000}%
\pgfsetstrokecolor{currentstroke}%
\pgfsetdash{}{0pt}%
\pgfpathmoveto{\pgfqpoint{4.358102in}{0.557870in}}%
\pgfpathlineto{\pgfqpoint{4.358102in}{2.042604in}}%
\pgfusepath{stroke}%
\end{pgfscope}%
\begin{pgfscope}%
\definecolor{textcolor}{rgb}{0.150000,0.150000,0.150000}%
\pgfsetstrokecolor{textcolor}%
\pgfsetfillcolor{textcolor}%
\pgftext[x=4.358102in,y=0.425926in,,top]{\color{textcolor}\sffamily\fontsize{11.000000}{13.200000}\selectfont \(\displaystyle 0.25\)}%
\end{pgfscope}%
\begin{pgfscope}%
\pgfpathrectangle{\pgfqpoint{3.696748in}{0.557870in}}{\pgfqpoint{2.424965in}{1.484734in}}%
\pgfusepath{clip}%
\pgfsetroundcap%
\pgfsetroundjoin%
\pgfsetlinewidth{1.003750pt}%
\definecolor{currentstroke}{rgb}{1.000000,1.000000,1.000000}%
\pgfsetstrokecolor{currentstroke}%
\pgfsetdash{}{0pt}%
\pgfpathmoveto{\pgfqpoint{4.909230in}{0.557870in}}%
\pgfpathlineto{\pgfqpoint{4.909230in}{2.042604in}}%
\pgfusepath{stroke}%
\end{pgfscope}%
\begin{pgfscope}%
\definecolor{textcolor}{rgb}{0.150000,0.150000,0.150000}%
\pgfsetstrokecolor{textcolor}%
\pgfsetfillcolor{textcolor}%
\pgftext[x=4.909230in,y=0.425926in,,top]{\color{textcolor}\sffamily\fontsize{11.000000}{13.200000}\selectfont \(\displaystyle 0.50\)}%
\end{pgfscope}%
\begin{pgfscope}%
\pgfpathrectangle{\pgfqpoint{3.696748in}{0.557870in}}{\pgfqpoint{2.424965in}{1.484734in}}%
\pgfusepath{clip}%
\pgfsetroundcap%
\pgfsetroundjoin%
\pgfsetlinewidth{1.003750pt}%
\definecolor{currentstroke}{rgb}{1.000000,1.000000,1.000000}%
\pgfsetstrokecolor{currentstroke}%
\pgfsetdash{}{0pt}%
\pgfpathmoveto{\pgfqpoint{5.460359in}{0.557870in}}%
\pgfpathlineto{\pgfqpoint{5.460359in}{2.042604in}}%
\pgfusepath{stroke}%
\end{pgfscope}%
\begin{pgfscope}%
\definecolor{textcolor}{rgb}{0.150000,0.150000,0.150000}%
\pgfsetstrokecolor{textcolor}%
\pgfsetfillcolor{textcolor}%
\pgftext[x=5.460359in,y=0.425926in,,top]{\color{textcolor}\sffamily\fontsize{11.000000}{13.200000}\selectfont \(\displaystyle 0.75\)}%
\end{pgfscope}%
\begin{pgfscope}%
\pgfpathrectangle{\pgfqpoint{3.696748in}{0.557870in}}{\pgfqpoint{2.424965in}{1.484734in}}%
\pgfusepath{clip}%
\pgfsetroundcap%
\pgfsetroundjoin%
\pgfsetlinewidth{1.003750pt}%
\definecolor{currentstroke}{rgb}{1.000000,1.000000,1.000000}%
\pgfsetstrokecolor{currentstroke}%
\pgfsetdash{}{0pt}%
\pgfpathmoveto{\pgfqpoint{6.011487in}{0.557870in}}%
\pgfpathlineto{\pgfqpoint{6.011487in}{2.042604in}}%
\pgfusepath{stroke}%
\end{pgfscope}%
\begin{pgfscope}%
\definecolor{textcolor}{rgb}{0.150000,0.150000,0.150000}%
\pgfsetstrokecolor{textcolor}%
\pgfsetfillcolor{textcolor}%
\pgftext[x=6.011487in,y=0.425926in,,top]{\color{textcolor}\sffamily\fontsize{11.000000}{13.200000}\selectfont \(\displaystyle 1.00\)}%
\end{pgfscope}%
\begin{pgfscope}%
\definecolor{textcolor}{rgb}{0.150000,0.150000,0.150000}%
\pgfsetstrokecolor{textcolor}%
\pgfsetfillcolor{textcolor}%
\pgftext[x=4.909230in,y=0.235185in,,top]{\color{textcolor}\sffamily\fontsize{11.000000}{13.200000}\selectfont Specificity}%
\end{pgfscope}%
\begin{pgfscope}%
\pgfpathrectangle{\pgfqpoint{3.696748in}{0.557870in}}{\pgfqpoint{2.424965in}{1.484734in}}%
\pgfusepath{clip}%
\pgfsetroundcap%
\pgfsetroundjoin%
\pgfsetlinewidth{1.003750pt}%
\definecolor{currentstroke}{rgb}{1.000000,1.000000,1.000000}%
\pgfsetstrokecolor{currentstroke}%
\pgfsetdash{}{0pt}%
\pgfpathmoveto{\pgfqpoint{3.696748in}{0.625358in}}%
\pgfpathlineto{\pgfqpoint{6.121713in}{0.625358in}}%
\pgfusepath{stroke}%
\end{pgfscope}%
\begin{pgfscope}%
\definecolor{textcolor}{rgb}{0.150000,0.150000,0.150000}%
\pgfsetstrokecolor{textcolor}%
\pgfsetfillcolor{textcolor}%
\pgftext[x=3.370474in,y=0.572552in,left,base]{\color{textcolor}\sffamily\fontsize{11.000000}{13.200000}\selectfont \(\displaystyle 0.0\)}%
\end{pgfscope}%
\begin{pgfscope}%
\pgfpathrectangle{\pgfqpoint{3.696748in}{0.557870in}}{\pgfqpoint{2.424965in}{1.484734in}}%
\pgfusepath{clip}%
\pgfsetroundcap%
\pgfsetroundjoin%
\pgfsetlinewidth{1.003750pt}%
\definecolor{currentstroke}{rgb}{1.000000,1.000000,1.000000}%
\pgfsetstrokecolor{currentstroke}%
\pgfsetdash{}{0pt}%
\pgfpathmoveto{\pgfqpoint{3.696748in}{1.300237in}}%
\pgfpathlineto{\pgfqpoint{6.121713in}{1.300237in}}%
\pgfusepath{stroke}%
\end{pgfscope}%
\begin{pgfscope}%
\definecolor{textcolor}{rgb}{0.150000,0.150000,0.150000}%
\pgfsetstrokecolor{textcolor}%
\pgfsetfillcolor{textcolor}%
\pgftext[x=3.370474in,y=1.247431in,left,base]{\color{textcolor}\sffamily\fontsize{11.000000}{13.200000}\selectfont \(\displaystyle 0.5\)}%
\end{pgfscope}%
\begin{pgfscope}%
\pgfpathrectangle{\pgfqpoint{3.696748in}{0.557870in}}{\pgfqpoint{2.424965in}{1.484734in}}%
\pgfusepath{clip}%
\pgfsetroundcap%
\pgfsetroundjoin%
\pgfsetlinewidth{1.003750pt}%
\definecolor{currentstroke}{rgb}{1.000000,1.000000,1.000000}%
\pgfsetstrokecolor{currentstroke}%
\pgfsetdash{}{0pt}%
\pgfpathmoveto{\pgfqpoint{3.696748in}{1.975116in}}%
\pgfpathlineto{\pgfqpoint{6.121713in}{1.975116in}}%
\pgfusepath{stroke}%
\end{pgfscope}%
\begin{pgfscope}%
\definecolor{textcolor}{rgb}{0.150000,0.150000,0.150000}%
\pgfsetstrokecolor{textcolor}%
\pgfsetfillcolor{textcolor}%
\pgftext[x=3.370474in,y=1.922310in,left,base]{\color{textcolor}\sffamily\fontsize{11.000000}{13.200000}\selectfont \(\displaystyle 1.0\)}%
\end{pgfscope}%
\begin{pgfscope}%
\definecolor{textcolor}{rgb}{0.150000,0.150000,0.150000}%
\pgfsetstrokecolor{textcolor}%
\pgfsetfillcolor{textcolor}%
\pgftext[x=3.314919in,y=1.300237in,,bottom,rotate=90.000000]{\color{textcolor}\sffamily\fontsize{11.000000}{13.200000}\selectfont Sensitivity}%
\end{pgfscope}%
\begin{pgfscope}%
\pgfpathrectangle{\pgfqpoint{3.696748in}{0.557870in}}{\pgfqpoint{2.424965in}{1.484734in}}%
\pgfusepath{clip}%
\pgfsetbuttcap%
\pgfsetroundjoin%
\definecolor{currentfill}{rgb}{0.298039,0.447059,0.690196}%
\pgfsetfillcolor{currentfill}%
\pgfsetlinewidth{1.003750pt}%
\definecolor{currentstroke}{rgb}{0.298039,0.447059,0.690196}%
\pgfsetstrokecolor{currentstroke}%
\pgfsetdash{}{0pt}%
\pgfpathmoveto{\pgfqpoint{5.151727in}{1.221462in}}%
\pgfpathcurveto{\pgfqpoint{5.159963in}{1.221462in}}{\pgfqpoint{5.167863in}{1.224734in}}{\pgfqpoint{5.173687in}{1.230558in}}%
\pgfpathcurveto{\pgfqpoint{5.179511in}{1.236382in}}{\pgfqpoint{5.182783in}{1.244282in}}{\pgfqpoint{5.182783in}{1.252519in}}%
\pgfpathcurveto{\pgfqpoint{5.182783in}{1.260755in}}{\pgfqpoint{5.179511in}{1.268655in}}{\pgfqpoint{5.173687in}{1.274479in}}%
\pgfpathcurveto{\pgfqpoint{5.167863in}{1.280303in}}{\pgfqpoint{5.159963in}{1.283575in}}{\pgfqpoint{5.151727in}{1.283575in}}%
\pgfpathcurveto{\pgfqpoint{5.143490in}{1.283575in}}{\pgfqpoint{5.135590in}{1.280303in}}{\pgfqpoint{5.129766in}{1.274479in}}%
\pgfpathcurveto{\pgfqpoint{5.123943in}{1.268655in}}{\pgfqpoint{5.120670in}{1.260755in}}{\pgfqpoint{5.120670in}{1.252519in}}%
\pgfpathcurveto{\pgfqpoint{5.120670in}{1.244282in}}{\pgfqpoint{5.123943in}{1.236382in}}{\pgfqpoint{5.129766in}{1.230558in}}%
\pgfpathcurveto{\pgfqpoint{5.135590in}{1.224734in}}{\pgfqpoint{5.143490in}{1.221462in}}{\pgfqpoint{5.151727in}{1.221462in}}%
\pgfpathclose%
\pgfusepath{stroke,fill}%
\end{pgfscope}%
\begin{pgfscope}%
\pgfpathrectangle{\pgfqpoint{3.696748in}{0.557870in}}{\pgfqpoint{2.424965in}{1.484734in}}%
\pgfusepath{clip}%
\pgfsetbuttcap%
\pgfsetroundjoin%
\definecolor{currentfill}{rgb}{0.298039,0.447059,0.690196}%
\pgfsetfillcolor{currentfill}%
\pgfsetlinewidth{1.003750pt}%
\definecolor{currentstroke}{rgb}{0.298039,0.447059,0.690196}%
\pgfsetstrokecolor{currentstroke}%
\pgfsetdash{}{0pt}%
\pgfpathmoveto{\pgfqpoint{5.085591in}{1.248730in}}%
\pgfpathcurveto{\pgfqpoint{5.093828in}{1.248730in}}{\pgfqpoint{5.101728in}{1.252002in}}{\pgfqpoint{5.107552in}{1.257826in}}%
\pgfpathcurveto{\pgfqpoint{5.113376in}{1.263650in}}{\pgfqpoint{5.116648in}{1.271550in}}{\pgfqpoint{5.116648in}{1.279786in}}%
\pgfpathcurveto{\pgfqpoint{5.116648in}{1.288023in}}{\pgfqpoint{5.113376in}{1.295923in}}{\pgfqpoint{5.107552in}{1.301747in}}%
\pgfpathcurveto{\pgfqpoint{5.101728in}{1.307571in}}{\pgfqpoint{5.093828in}{1.310843in}}{\pgfqpoint{5.085591in}{1.310843in}}%
\pgfpathcurveto{\pgfqpoint{5.077355in}{1.310843in}}{\pgfqpoint{5.069455in}{1.307571in}}{\pgfqpoint{5.063631in}{1.301747in}}%
\pgfpathcurveto{\pgfqpoint{5.057807in}{1.295923in}}{\pgfqpoint{5.054535in}{1.288023in}}{\pgfqpoint{5.054535in}{1.279786in}}%
\pgfpathcurveto{\pgfqpoint{5.054535in}{1.271550in}}{\pgfqpoint{5.057807in}{1.263650in}}{\pgfqpoint{5.063631in}{1.257826in}}%
\pgfpathcurveto{\pgfqpoint{5.069455in}{1.252002in}}{\pgfqpoint{5.077355in}{1.248730in}}{\pgfqpoint{5.085591in}{1.248730in}}%
\pgfpathclose%
\pgfusepath{stroke,fill}%
\end{pgfscope}%
\begin{pgfscope}%
\pgfpathrectangle{\pgfqpoint{3.696748in}{0.557870in}}{\pgfqpoint{2.424965in}{1.484734in}}%
\pgfusepath{clip}%
\pgfsetbuttcap%
\pgfsetroundjoin%
\definecolor{currentfill}{rgb}{0.298039,0.447059,0.690196}%
\pgfsetfillcolor{currentfill}%
\pgfsetlinewidth{1.003750pt}%
\definecolor{currentstroke}{rgb}{0.298039,0.447059,0.690196}%
\pgfsetstrokecolor{currentstroke}%
\pgfsetdash{}{0pt}%
\pgfpathmoveto{\pgfqpoint{5.306043in}{1.080215in}}%
\pgfpathcurveto{\pgfqpoint{5.314279in}{1.080215in}}{\pgfqpoint{5.322179in}{1.083487in}}{\pgfqpoint{5.328003in}{1.089311in}}%
\pgfpathcurveto{\pgfqpoint{5.333827in}{1.095135in}}{\pgfqpoint{5.337099in}{1.103035in}}{\pgfqpoint{5.337099in}{1.111271in}}%
\pgfpathcurveto{\pgfqpoint{5.337099in}{1.119507in}}{\pgfqpoint{5.333827in}{1.127407in}}{\pgfqpoint{5.328003in}{1.133231in}}%
\pgfpathcurveto{\pgfqpoint{5.322179in}{1.139055in}}{\pgfqpoint{5.314279in}{1.142328in}}{\pgfqpoint{5.306043in}{1.142328in}}%
\pgfpathcurveto{\pgfqpoint{5.297806in}{1.142328in}}{\pgfqpoint{5.289906in}{1.139055in}}{\pgfqpoint{5.284082in}{1.133231in}}%
\pgfpathcurveto{\pgfqpoint{5.278259in}{1.127407in}}{\pgfqpoint{5.274986in}{1.119507in}}{\pgfqpoint{5.274986in}{1.111271in}}%
\pgfpathcurveto{\pgfqpoint{5.274986in}{1.103035in}}{\pgfqpoint{5.278259in}{1.095135in}}{\pgfqpoint{5.284082in}{1.089311in}}%
\pgfpathcurveto{\pgfqpoint{5.289906in}{1.083487in}}{\pgfqpoint{5.297806in}{1.080215in}}{\pgfqpoint{5.306043in}{1.080215in}}%
\pgfpathclose%
\pgfusepath{stroke,fill}%
\end{pgfscope}%
\begin{pgfscope}%
\pgfpathrectangle{\pgfqpoint{3.696748in}{0.557870in}}{\pgfqpoint{2.424965in}{1.484734in}}%
\pgfusepath{clip}%
\pgfsetbuttcap%
\pgfsetroundjoin%
\definecolor{currentfill}{rgb}{0.298039,0.447059,0.690196}%
\pgfsetfillcolor{currentfill}%
\pgfsetlinewidth{1.003750pt}%
\definecolor{currentstroke}{rgb}{0.298039,0.447059,0.690196}%
\pgfsetstrokecolor{currentstroke}%
\pgfsetdash{}{0pt}%
\pgfpathmoveto{\pgfqpoint{4.688779in}{1.448230in}}%
\pgfpathcurveto{\pgfqpoint{4.697015in}{1.448230in}}{\pgfqpoint{4.704915in}{1.451503in}}{\pgfqpoint{4.710739in}{1.457327in}}%
\pgfpathcurveto{\pgfqpoint{4.716563in}{1.463150in}}{\pgfqpoint{4.719835in}{1.471050in}}{\pgfqpoint{4.719835in}{1.479287in}}%
\pgfpathcurveto{\pgfqpoint{4.719835in}{1.487523in}}{\pgfqpoint{4.716563in}{1.495423in}}{\pgfqpoint{4.710739in}{1.501247in}}%
\pgfpathcurveto{\pgfqpoint{4.704915in}{1.507071in}}{\pgfqpoint{4.697015in}{1.510343in}}{\pgfqpoint{4.688779in}{1.510343in}}%
\pgfpathcurveto{\pgfqpoint{4.680543in}{1.510343in}}{\pgfqpoint{4.672643in}{1.507071in}}{\pgfqpoint{4.666819in}{1.501247in}}%
\pgfpathcurveto{\pgfqpoint{4.660995in}{1.495423in}}{\pgfqpoint{4.657722in}{1.487523in}}{\pgfqpoint{4.657722in}{1.479287in}}%
\pgfpathcurveto{\pgfqpoint{4.657722in}{1.471050in}}{\pgfqpoint{4.660995in}{1.463150in}}{\pgfqpoint{4.666819in}{1.457327in}}%
\pgfpathcurveto{\pgfqpoint{4.672643in}{1.451503in}}{\pgfqpoint{4.680543in}{1.448230in}}{\pgfqpoint{4.688779in}{1.448230in}}%
\pgfpathclose%
\pgfusepath{stroke,fill}%
\end{pgfscope}%
\begin{pgfscope}%
\pgfpathrectangle{\pgfqpoint{3.696748in}{0.557870in}}{\pgfqpoint{2.424965in}{1.484734in}}%
\pgfusepath{clip}%
\pgfsetbuttcap%
\pgfsetroundjoin%
\definecolor{currentfill}{rgb}{0.298039,0.447059,0.690196}%
\pgfsetfillcolor{currentfill}%
\pgfsetlinewidth{1.003750pt}%
\definecolor{currentstroke}{rgb}{0.298039,0.447059,0.690196}%
\pgfsetstrokecolor{currentstroke}%
\pgfsetdash{}{0pt}%
\pgfpathmoveto{\pgfqpoint{4.693324in}{1.412337in}}%
\pgfpathcurveto{\pgfqpoint{4.701561in}{1.412337in}}{\pgfqpoint{4.709461in}{1.415609in}}{\pgfqpoint{4.715285in}{1.421433in}}%
\pgfpathcurveto{\pgfqpoint{4.721108in}{1.427257in}}{\pgfqpoint{4.724381in}{1.435157in}}{\pgfqpoint{4.724381in}{1.443393in}}%
\pgfpathcurveto{\pgfqpoint{4.724381in}{1.451630in}}{\pgfqpoint{4.721108in}{1.459530in}}{\pgfqpoint{4.715285in}{1.465354in}}%
\pgfpathcurveto{\pgfqpoint{4.709461in}{1.471178in}}{\pgfqpoint{4.701561in}{1.474450in}}{\pgfqpoint{4.693324in}{1.474450in}}%
\pgfpathcurveto{\pgfqpoint{4.685088in}{1.474450in}}{\pgfqpoint{4.677188in}{1.471178in}}{\pgfqpoint{4.671364in}{1.465354in}}%
\pgfpathcurveto{\pgfqpoint{4.665540in}{1.459530in}}{\pgfqpoint{4.662268in}{1.451630in}}{\pgfqpoint{4.662268in}{1.443393in}}%
\pgfpathcurveto{\pgfqpoint{4.662268in}{1.435157in}}{\pgfqpoint{4.665540in}{1.427257in}}{\pgfqpoint{4.671364in}{1.421433in}}%
\pgfpathcurveto{\pgfqpoint{4.677188in}{1.415609in}}{\pgfqpoint{4.685088in}{1.412337in}}{\pgfqpoint{4.693324in}{1.412337in}}%
\pgfpathclose%
\pgfusepath{stroke,fill}%
\end{pgfscope}%
\begin{pgfscope}%
\pgfpathrectangle{\pgfqpoint{3.696748in}{0.557870in}}{\pgfqpoint{2.424965in}{1.484734in}}%
\pgfusepath{clip}%
\pgfsetbuttcap%
\pgfsetroundjoin%
\definecolor{currentfill}{rgb}{0.298039,0.447059,0.690196}%
\pgfsetfillcolor{currentfill}%
\pgfsetlinewidth{1.003750pt}%
\definecolor{currentstroke}{rgb}{0.298039,0.447059,0.690196}%
\pgfsetstrokecolor{currentstroke}%
\pgfsetdash{}{0pt}%
\pgfpathmoveto{\pgfqpoint{4.953321in}{1.248730in}}%
\pgfpathcurveto{\pgfqpoint{4.961557in}{1.248730in}}{\pgfqpoint{4.969457in}{1.252002in}}{\pgfqpoint{4.975281in}{1.257826in}}%
\pgfpathcurveto{\pgfqpoint{4.981105in}{1.263650in}}{\pgfqpoint{4.984377in}{1.271550in}}{\pgfqpoint{4.984377in}{1.279786in}}%
\pgfpathcurveto{\pgfqpoint{4.984377in}{1.288023in}}{\pgfqpoint{4.981105in}{1.295923in}}{\pgfqpoint{4.975281in}{1.301747in}}%
\pgfpathcurveto{\pgfqpoint{4.969457in}{1.307571in}}{\pgfqpoint{4.961557in}{1.310843in}}{\pgfqpoint{4.953321in}{1.310843in}}%
\pgfpathcurveto{\pgfqpoint{4.945084in}{1.310843in}}{\pgfqpoint{4.937184in}{1.307571in}}{\pgfqpoint{4.931360in}{1.301747in}}%
\pgfpathcurveto{\pgfqpoint{4.925536in}{1.295923in}}{\pgfqpoint{4.922264in}{1.288023in}}{\pgfqpoint{4.922264in}{1.279786in}}%
\pgfpathcurveto{\pgfqpoint{4.922264in}{1.271550in}}{\pgfqpoint{4.925536in}{1.263650in}}{\pgfqpoint{4.931360in}{1.257826in}}%
\pgfpathcurveto{\pgfqpoint{4.937184in}{1.252002in}}{\pgfqpoint{4.945084in}{1.248730in}}{\pgfqpoint{4.953321in}{1.248730in}}%
\pgfpathclose%
\pgfusepath{stroke,fill}%
\end{pgfscope}%
\begin{pgfscope}%
\pgfpathrectangle{\pgfqpoint{3.696748in}{0.557870in}}{\pgfqpoint{2.424965in}{1.484734in}}%
\pgfusepath{clip}%
\pgfsetbuttcap%
\pgfsetroundjoin%
\definecolor{currentfill}{rgb}{0.298039,0.447059,0.690196}%
\pgfsetfillcolor{currentfill}%
\pgfsetlinewidth{1.003750pt}%
\definecolor{currentstroke}{rgb}{0.298039,0.447059,0.690196}%
\pgfsetstrokecolor{currentstroke}%
\pgfsetdash{}{0pt}%
\pgfpathmoveto{\pgfqpoint{4.909230in}{1.248730in}}%
\pgfpathcurveto{\pgfqpoint{4.917467in}{1.248730in}}{\pgfqpoint{4.925367in}{1.252002in}}{\pgfqpoint{4.931191in}{1.257826in}}%
\pgfpathcurveto{\pgfqpoint{4.937014in}{1.263650in}}{\pgfqpoint{4.940287in}{1.271550in}}{\pgfqpoint{4.940287in}{1.279786in}}%
\pgfpathcurveto{\pgfqpoint{4.940287in}{1.288023in}}{\pgfqpoint{4.937014in}{1.295923in}}{\pgfqpoint{4.931191in}{1.301747in}}%
\pgfpathcurveto{\pgfqpoint{4.925367in}{1.307571in}}{\pgfqpoint{4.917467in}{1.310843in}}{\pgfqpoint{4.909230in}{1.310843in}}%
\pgfpathcurveto{\pgfqpoint{4.900994in}{1.310843in}}{\pgfqpoint{4.893094in}{1.307571in}}{\pgfqpoint{4.887270in}{1.301747in}}%
\pgfpathcurveto{\pgfqpoint{4.881446in}{1.295923in}}{\pgfqpoint{4.878174in}{1.288023in}}{\pgfqpoint{4.878174in}{1.279786in}}%
\pgfpathcurveto{\pgfqpoint{4.878174in}{1.271550in}}{\pgfqpoint{4.881446in}{1.263650in}}{\pgfqpoint{4.887270in}{1.257826in}}%
\pgfpathcurveto{\pgfqpoint{4.893094in}{1.252002in}}{\pgfqpoint{4.900994in}{1.248730in}}{\pgfqpoint{4.909230in}{1.248730in}}%
\pgfpathclose%
\pgfusepath{stroke,fill}%
\end{pgfscope}%
\begin{pgfscope}%
\pgfpathrectangle{\pgfqpoint{3.696748in}{0.557870in}}{\pgfqpoint{2.424965in}{1.484734in}}%
\pgfusepath{clip}%
\pgfsetbuttcap%
\pgfsetroundjoin%
\definecolor{currentfill}{rgb}{0.298039,0.447059,0.690196}%
\pgfsetfillcolor{currentfill}%
\pgfsetlinewidth{1.003750pt}%
\definecolor{currentstroke}{rgb}{0.298039,0.447059,0.690196}%
\pgfsetstrokecolor{currentstroke}%
\pgfsetdash{}{0pt}%
\pgfpathmoveto{\pgfqpoint{4.693324in}{1.363664in}}%
\pgfpathcurveto{\pgfqpoint{4.701561in}{1.363664in}}{\pgfqpoint{4.709461in}{1.366936in}}{\pgfqpoint{4.715285in}{1.372760in}}%
\pgfpathcurveto{\pgfqpoint{4.721108in}{1.378584in}}{\pgfqpoint{4.724381in}{1.386484in}}{\pgfqpoint{4.724381in}{1.394720in}}%
\pgfpathcurveto{\pgfqpoint{4.724381in}{1.402957in}}{\pgfqpoint{4.721108in}{1.410857in}}{\pgfqpoint{4.715285in}{1.416681in}}%
\pgfpathcurveto{\pgfqpoint{4.709461in}{1.422504in}}{\pgfqpoint{4.701561in}{1.425777in}}{\pgfqpoint{4.693324in}{1.425777in}}%
\pgfpathcurveto{\pgfqpoint{4.685088in}{1.425777in}}{\pgfqpoint{4.677188in}{1.422504in}}{\pgfqpoint{4.671364in}{1.416681in}}%
\pgfpathcurveto{\pgfqpoint{4.665540in}{1.410857in}}{\pgfqpoint{4.662268in}{1.402957in}}{\pgfqpoint{4.662268in}{1.394720in}}%
\pgfpathcurveto{\pgfqpoint{4.662268in}{1.386484in}}{\pgfqpoint{4.665540in}{1.378584in}}{\pgfqpoint{4.671364in}{1.372760in}}%
\pgfpathcurveto{\pgfqpoint{4.677188in}{1.366936in}}{\pgfqpoint{4.685088in}{1.363664in}}{\pgfqpoint{4.693324in}{1.363664in}}%
\pgfpathclose%
\pgfusepath{stroke,fill}%
\end{pgfscope}%
\begin{pgfscope}%
\pgfpathrectangle{\pgfqpoint{3.696748in}{0.557870in}}{\pgfqpoint{2.424965in}{1.484734in}}%
\pgfusepath{clip}%
\pgfsetbuttcap%
\pgfsetroundjoin%
\definecolor{currentfill}{rgb}{0.298039,0.447059,0.690196}%
\pgfsetfillcolor{currentfill}%
\pgfsetlinewidth{1.003750pt}%
\definecolor{currentstroke}{rgb}{0.298039,0.447059,0.690196}%
\pgfsetstrokecolor{currentstroke}%
\pgfsetdash{}{0pt}%
\pgfpathmoveto{\pgfqpoint{4.556508in}{1.417654in}}%
\pgfpathcurveto{\pgfqpoint{4.564744in}{1.417654in}}{\pgfqpoint{4.572644in}{1.420926in}}{\pgfqpoint{4.578468in}{1.426750in}}%
\pgfpathcurveto{\pgfqpoint{4.584292in}{1.432574in}}{\pgfqpoint{4.587565in}{1.440474in}}{\pgfqpoint{4.587565in}{1.448711in}}%
\pgfpathcurveto{\pgfqpoint{4.587565in}{1.456947in}}{\pgfqpoint{4.584292in}{1.464847in}}{\pgfqpoint{4.578468in}{1.470671in}}%
\pgfpathcurveto{\pgfqpoint{4.572644in}{1.476495in}}{\pgfqpoint{4.564744in}{1.479767in}}{\pgfqpoint{4.556508in}{1.479767in}}%
\pgfpathcurveto{\pgfqpoint{4.548272in}{1.479767in}}{\pgfqpoint{4.540372in}{1.476495in}}{\pgfqpoint{4.534548in}{1.470671in}}%
\pgfpathcurveto{\pgfqpoint{4.528724in}{1.464847in}}{\pgfqpoint{4.525452in}{1.456947in}}{\pgfqpoint{4.525452in}{1.448711in}}%
\pgfpathcurveto{\pgfqpoint{4.525452in}{1.440474in}}{\pgfqpoint{4.528724in}{1.432574in}}{\pgfqpoint{4.534548in}{1.426750in}}%
\pgfpathcurveto{\pgfqpoint{4.540372in}{1.420926in}}{\pgfqpoint{4.548272in}{1.417654in}}{\pgfqpoint{4.556508in}{1.417654in}}%
\pgfpathclose%
\pgfusepath{stroke,fill}%
\end{pgfscope}%
\begin{pgfscope}%
\pgfpathrectangle{\pgfqpoint{3.696748in}{0.557870in}}{\pgfqpoint{2.424965in}{1.484734in}}%
\pgfusepath{clip}%
\pgfsetbuttcap%
\pgfsetroundjoin%
\definecolor{currentfill}{rgb}{0.298039,0.447059,0.690196}%
\pgfsetfillcolor{currentfill}%
\pgfsetlinewidth{1.003750pt}%
\definecolor{currentstroke}{rgb}{0.298039,0.447059,0.690196}%
\pgfsetstrokecolor{currentstroke}%
\pgfsetdash{}{0pt}%
\pgfpathmoveto{\pgfqpoint{4.490373in}{1.296176in}}%
\pgfpathcurveto{\pgfqpoint{4.498609in}{1.296176in}}{\pgfqpoint{4.506509in}{1.299448in}}{\pgfqpoint{4.512333in}{1.305272in}}%
\pgfpathcurveto{\pgfqpoint{4.518157in}{1.311096in}}{\pgfqpoint{4.521429in}{1.318996in}}{\pgfqpoint{4.521429in}{1.327232in}}%
\pgfpathcurveto{\pgfqpoint{4.521429in}{1.335469in}}{\pgfqpoint{4.518157in}{1.343369in}}{\pgfqpoint{4.512333in}{1.349193in}}%
\pgfpathcurveto{\pgfqpoint{4.506509in}{1.355017in}}{\pgfqpoint{4.498609in}{1.358289in}}{\pgfqpoint{4.490373in}{1.358289in}}%
\pgfpathcurveto{\pgfqpoint{4.482136in}{1.358289in}}{\pgfqpoint{4.474236in}{1.355017in}}{\pgfqpoint{4.468412in}{1.349193in}}%
\pgfpathcurveto{\pgfqpoint{4.462588in}{1.343369in}}{\pgfqpoint{4.459316in}{1.335469in}}{\pgfqpoint{4.459316in}{1.327232in}}%
\pgfpathcurveto{\pgfqpoint{4.459316in}{1.318996in}}{\pgfqpoint{4.462588in}{1.311096in}}{\pgfqpoint{4.468412in}{1.305272in}}%
\pgfpathcurveto{\pgfqpoint{4.474236in}{1.299448in}}{\pgfqpoint{4.482136in}{1.296176in}}{\pgfqpoint{4.490373in}{1.296176in}}%
\pgfpathclose%
\pgfusepath{stroke,fill}%
\end{pgfscope}%
\begin{pgfscope}%
\pgfpathrectangle{\pgfqpoint{3.696748in}{0.557870in}}{\pgfqpoint{2.424965in}{1.484734in}}%
\pgfusepath{clip}%
\pgfsetbuttcap%
\pgfsetroundjoin%
\definecolor{currentfill}{rgb}{0.298039,0.447059,0.690196}%
\pgfsetfillcolor{currentfill}%
\pgfsetlinewidth{1.003750pt}%
\definecolor{currentstroke}{rgb}{0.298039,0.447059,0.690196}%
\pgfsetstrokecolor{currentstroke}%
\pgfsetdash{}{0pt}%
\pgfpathmoveto{\pgfqpoint{4.843095in}{1.048812in}}%
\pgfpathcurveto{\pgfqpoint{4.851331in}{1.048812in}}{\pgfqpoint{4.859231in}{1.052084in}}{\pgfqpoint{4.865055in}{1.057908in}}%
\pgfpathcurveto{\pgfqpoint{4.870879in}{1.063732in}}{\pgfqpoint{4.874151in}{1.071632in}}{\pgfqpoint{4.874151in}{1.079869in}}%
\pgfpathcurveto{\pgfqpoint{4.874151in}{1.088105in}}{\pgfqpoint{4.870879in}{1.096005in}}{\pgfqpoint{4.865055in}{1.101829in}}%
\pgfpathcurveto{\pgfqpoint{4.859231in}{1.107653in}}{\pgfqpoint{4.851331in}{1.110925in}}{\pgfqpoint{4.843095in}{1.110925in}}%
\pgfpathcurveto{\pgfqpoint{4.834859in}{1.110925in}}{\pgfqpoint{4.826958in}{1.107653in}}{\pgfqpoint{4.821135in}{1.101829in}}%
\pgfpathcurveto{\pgfqpoint{4.815311in}{1.096005in}}{\pgfqpoint{4.812038in}{1.088105in}}{\pgfqpoint{4.812038in}{1.079869in}}%
\pgfpathcurveto{\pgfqpoint{4.812038in}{1.071632in}}{\pgfqpoint{4.815311in}{1.063732in}}{\pgfqpoint{4.821135in}{1.057908in}}%
\pgfpathcurveto{\pgfqpoint{4.826958in}{1.052084in}}{\pgfqpoint{4.834859in}{1.048812in}}{\pgfqpoint{4.843095in}{1.048812in}}%
\pgfpathclose%
\pgfusepath{stroke,fill}%
\end{pgfscope}%
\begin{pgfscope}%
\pgfpathrectangle{\pgfqpoint{3.696748in}{0.557870in}}{\pgfqpoint{2.424965in}{1.484734in}}%
\pgfusepath{clip}%
\pgfsetbuttcap%
\pgfsetroundjoin%
\definecolor{currentfill}{rgb}{0.298039,0.447059,0.690196}%
\pgfsetfillcolor{currentfill}%
\pgfsetlinewidth{1.003750pt}%
\definecolor{currentstroke}{rgb}{0.298039,0.447059,0.690196}%
\pgfsetstrokecolor{currentstroke}%
\pgfsetdash{}{0pt}%
\pgfpathmoveto{\pgfqpoint{4.420601in}{1.316899in}}%
\pgfpathcurveto{\pgfqpoint{4.428837in}{1.316899in}}{\pgfqpoint{4.436737in}{1.320172in}}{\pgfqpoint{4.442561in}{1.325996in}}%
\pgfpathcurveto{\pgfqpoint{4.448385in}{1.331820in}}{\pgfqpoint{4.451657in}{1.339720in}}{\pgfqpoint{4.451657in}{1.347956in}}%
\pgfpathcurveto{\pgfqpoint{4.451657in}{1.356192in}}{\pgfqpoint{4.448385in}{1.364092in}}{\pgfqpoint{4.442561in}{1.369916in}}%
\pgfpathcurveto{\pgfqpoint{4.436737in}{1.375740in}}{\pgfqpoint{4.428837in}{1.379012in}}{\pgfqpoint{4.420601in}{1.379012in}}%
\pgfpathcurveto{\pgfqpoint{4.412365in}{1.379012in}}{\pgfqpoint{4.404465in}{1.375740in}}{\pgfqpoint{4.398641in}{1.369916in}}%
\pgfpathcurveto{\pgfqpoint{4.392817in}{1.364092in}}{\pgfqpoint{4.389544in}{1.356192in}}{\pgfqpoint{4.389544in}{1.347956in}}%
\pgfpathcurveto{\pgfqpoint{4.389544in}{1.339720in}}{\pgfqpoint{4.392817in}{1.331820in}}{\pgfqpoint{4.398641in}{1.325996in}}%
\pgfpathcurveto{\pgfqpoint{4.404465in}{1.320172in}}{\pgfqpoint{4.412365in}{1.316899in}}{\pgfqpoint{4.420601in}{1.316899in}}%
\pgfpathclose%
\pgfusepath{stroke,fill}%
\end{pgfscope}%
\begin{pgfscope}%
\pgfpathrectangle{\pgfqpoint{3.696748in}{0.557870in}}{\pgfqpoint{2.424965in}{1.484734in}}%
\pgfusepath{clip}%
\pgfsetbuttcap%
\pgfsetroundjoin%
\definecolor{currentfill}{rgb}{0.298039,0.447059,0.690196}%
\pgfsetfillcolor{currentfill}%
\pgfsetlinewidth{1.003750pt}%
\definecolor{currentstroke}{rgb}{0.298039,0.447059,0.690196}%
\pgfsetstrokecolor{currentstroke}%
\pgfsetdash{}{0pt}%
\pgfpathmoveto{\pgfqpoint{4.829686in}{1.003319in}}%
\pgfpathcurveto{\pgfqpoint{4.837922in}{1.003319in}}{\pgfqpoint{4.845822in}{1.006592in}}{\pgfqpoint{4.851646in}{1.012416in}}%
\pgfpathcurveto{\pgfqpoint{4.857470in}{1.018239in}}{\pgfqpoint{4.860742in}{1.026140in}}{\pgfqpoint{4.860742in}{1.034376in}}%
\pgfpathcurveto{\pgfqpoint{4.860742in}{1.042612in}}{\pgfqpoint{4.857470in}{1.050512in}}{\pgfqpoint{4.851646in}{1.056336in}}%
\pgfpathcurveto{\pgfqpoint{4.845822in}{1.062160in}}{\pgfqpoint{4.837922in}{1.065432in}}{\pgfqpoint{4.829686in}{1.065432in}}%
\pgfpathcurveto{\pgfqpoint{4.821450in}{1.065432in}}{\pgfqpoint{4.813550in}{1.062160in}}{\pgfqpoint{4.807726in}{1.056336in}}%
\pgfpathcurveto{\pgfqpoint{4.801902in}{1.050512in}}{\pgfqpoint{4.798629in}{1.042612in}}{\pgfqpoint{4.798629in}{1.034376in}}%
\pgfpathcurveto{\pgfqpoint{4.798629in}{1.026140in}}{\pgfqpoint{4.801902in}{1.018239in}}{\pgfqpoint{4.807726in}{1.012416in}}%
\pgfpathcurveto{\pgfqpoint{4.813550in}{1.006592in}}{\pgfqpoint{4.821450in}{1.003319in}}{\pgfqpoint{4.829686in}{1.003319in}}%
\pgfpathclose%
\pgfusepath{stroke,fill}%
\end{pgfscope}%
\begin{pgfscope}%
\pgfpathrectangle{\pgfqpoint{3.696748in}{0.557870in}}{\pgfqpoint{2.424965in}{1.484734in}}%
\pgfusepath{clip}%
\pgfsetbuttcap%
\pgfsetroundjoin%
\definecolor{currentfill}{rgb}{0.298039,0.447059,0.690196}%
\pgfsetfillcolor{currentfill}%
\pgfsetlinewidth{1.003750pt}%
\definecolor{currentstroke}{rgb}{0.298039,0.447059,0.690196}%
\pgfsetstrokecolor{currentstroke}%
\pgfsetdash{}{0pt}%
\pgfpathmoveto{\pgfqpoint{4.625143in}{1.090131in}}%
\pgfpathcurveto{\pgfqpoint{4.633380in}{1.090131in}}{\pgfqpoint{4.641280in}{1.093403in}}{\pgfqpoint{4.647104in}{1.099227in}}%
\pgfpathcurveto{\pgfqpoint{4.652928in}{1.105051in}}{\pgfqpoint{4.656200in}{1.112951in}}{\pgfqpoint{4.656200in}{1.121188in}}%
\pgfpathcurveto{\pgfqpoint{4.656200in}{1.129424in}}{\pgfqpoint{4.652928in}{1.137324in}}{\pgfqpoint{4.647104in}{1.143148in}}%
\pgfpathcurveto{\pgfqpoint{4.641280in}{1.148972in}}{\pgfqpoint{4.633380in}{1.152244in}}{\pgfqpoint{4.625143in}{1.152244in}}%
\pgfpathcurveto{\pgfqpoint{4.616907in}{1.152244in}}{\pgfqpoint{4.609007in}{1.148972in}}{\pgfqpoint{4.603183in}{1.143148in}}%
\pgfpathcurveto{\pgfqpoint{4.597359in}{1.137324in}}{\pgfqpoint{4.594087in}{1.129424in}}{\pgfqpoint{4.594087in}{1.121188in}}%
\pgfpathcurveto{\pgfqpoint{4.594087in}{1.112951in}}{\pgfqpoint{4.597359in}{1.105051in}}{\pgfqpoint{4.603183in}{1.099227in}}%
\pgfpathcurveto{\pgfqpoint{4.609007in}{1.093403in}}{\pgfqpoint{4.616907in}{1.090131in}}{\pgfqpoint{4.625143in}{1.090131in}}%
\pgfpathclose%
\pgfusepath{stroke,fill}%
\end{pgfscope}%
\begin{pgfscope}%
\pgfpathrectangle{\pgfqpoint{3.696748in}{0.557870in}}{\pgfqpoint{2.424965in}{1.484734in}}%
\pgfusepath{clip}%
\pgfsetbuttcap%
\pgfsetroundjoin%
\definecolor{currentfill}{rgb}{0.298039,0.447059,0.690196}%
\pgfsetfillcolor{currentfill}%
\pgfsetlinewidth{1.003750pt}%
\definecolor{currentstroke}{rgb}{0.298039,0.447059,0.690196}%
\pgfsetstrokecolor{currentstroke}%
\pgfsetdash{}{0pt}%
\pgfpathmoveto{\pgfqpoint{4.336057in}{1.269181in}}%
\pgfpathcurveto{\pgfqpoint{4.344293in}{1.269181in}}{\pgfqpoint{4.352193in}{1.272453in}}{\pgfqpoint{4.358017in}{1.278277in}}%
\pgfpathcurveto{\pgfqpoint{4.363841in}{1.284101in}}{\pgfqpoint{4.367113in}{1.292001in}}{\pgfqpoint{4.367113in}{1.300237in}}%
\pgfpathcurveto{\pgfqpoint{4.367113in}{1.308474in}}{\pgfqpoint{4.363841in}{1.316374in}}{\pgfqpoint{4.358017in}{1.322197in}}%
\pgfpathcurveto{\pgfqpoint{4.352193in}{1.328021in}}{\pgfqpoint{4.344293in}{1.331294in}}{\pgfqpoint{4.336057in}{1.331294in}}%
\pgfpathcurveto{\pgfqpoint{4.327820in}{1.331294in}}{\pgfqpoint{4.319920in}{1.328021in}}{\pgfqpoint{4.314096in}{1.322197in}}%
\pgfpathcurveto{\pgfqpoint{4.308272in}{1.316374in}}{\pgfqpoint{4.305000in}{1.308474in}}{\pgfqpoint{4.305000in}{1.300237in}}%
\pgfpathcurveto{\pgfqpoint{4.305000in}{1.292001in}}{\pgfqpoint{4.308272in}{1.284101in}}{\pgfqpoint{4.314096in}{1.278277in}}%
\pgfpathcurveto{\pgfqpoint{4.319920in}{1.272453in}}{\pgfqpoint{4.327820in}{1.269181in}}{\pgfqpoint{4.336057in}{1.269181in}}%
\pgfpathclose%
\pgfusepath{stroke,fill}%
\end{pgfscope}%
\begin{pgfscope}%
\pgfpathrectangle{\pgfqpoint{3.696748in}{0.557870in}}{\pgfqpoint{2.424965in}{1.484734in}}%
\pgfusepath{clip}%
\pgfsetbuttcap%
\pgfsetroundjoin%
\definecolor{currentfill}{rgb}{0.298039,0.447059,0.690196}%
\pgfsetfillcolor{currentfill}%
\pgfsetlinewidth{1.003750pt}%
\definecolor{currentstroke}{rgb}{0.298039,0.447059,0.690196}%
\pgfsetstrokecolor{currentstroke}%
\pgfsetdash{}{0pt}%
\pgfpathmoveto{\pgfqpoint{4.579690in}{1.080215in}}%
\pgfpathcurveto{\pgfqpoint{4.587926in}{1.080215in}}{\pgfqpoint{4.595826in}{1.083487in}}{\pgfqpoint{4.601650in}{1.089311in}}%
\pgfpathcurveto{\pgfqpoint{4.607474in}{1.095135in}}{\pgfqpoint{4.610746in}{1.103035in}}{\pgfqpoint{4.610746in}{1.111271in}}%
\pgfpathcurveto{\pgfqpoint{4.610746in}{1.119507in}}{\pgfqpoint{4.607474in}{1.127407in}}{\pgfqpoint{4.601650in}{1.133231in}}%
\pgfpathcurveto{\pgfqpoint{4.595826in}{1.139055in}}{\pgfqpoint{4.587926in}{1.142328in}}{\pgfqpoint{4.579690in}{1.142328in}}%
\pgfpathcurveto{\pgfqpoint{4.571453in}{1.142328in}}{\pgfqpoint{4.563553in}{1.139055in}}{\pgfqpoint{4.557729in}{1.133231in}}%
\pgfpathcurveto{\pgfqpoint{4.551905in}{1.127407in}}{\pgfqpoint{4.548633in}{1.119507in}}{\pgfqpoint{4.548633in}{1.111271in}}%
\pgfpathcurveto{\pgfqpoint{4.548633in}{1.103035in}}{\pgfqpoint{4.551905in}{1.095135in}}{\pgfqpoint{4.557729in}{1.089311in}}%
\pgfpathcurveto{\pgfqpoint{4.563553in}{1.083487in}}{\pgfqpoint{4.571453in}{1.080215in}}{\pgfqpoint{4.579690in}{1.080215in}}%
\pgfpathclose%
\pgfusepath{stroke,fill}%
\end{pgfscope}%
\begin{pgfscope}%
\pgfpathrectangle{\pgfqpoint{3.696748in}{0.557870in}}{\pgfqpoint{2.424965in}{1.484734in}}%
\pgfusepath{clip}%
\pgfsetbuttcap%
\pgfsetroundjoin%
\definecolor{currentfill}{rgb}{0.298039,0.447059,0.690196}%
\pgfsetfillcolor{currentfill}%
\pgfsetlinewidth{1.003750pt}%
\definecolor{currentstroke}{rgb}{0.298039,0.447059,0.690196}%
\pgfsetstrokecolor{currentstroke}%
\pgfsetdash{}{0pt}%
\pgfpathmoveto{\pgfqpoint{4.170605in}{1.385069in}}%
\pgfpathcurveto{\pgfqpoint{4.178841in}{1.385069in}}{\pgfqpoint{4.186741in}{1.388341in}}{\pgfqpoint{4.192565in}{1.394165in}}%
\pgfpathcurveto{\pgfqpoint{4.198389in}{1.399989in}}{\pgfqpoint{4.201661in}{1.407889in}}{\pgfqpoint{4.201661in}{1.416126in}}%
\pgfpathcurveto{\pgfqpoint{4.201661in}{1.424362in}}{\pgfqpoint{4.198389in}{1.432262in}}{\pgfqpoint{4.192565in}{1.438086in}}%
\pgfpathcurveto{\pgfqpoint{4.186741in}{1.443910in}}{\pgfqpoint{4.178841in}{1.447182in}}{\pgfqpoint{4.170605in}{1.447182in}}%
\pgfpathcurveto{\pgfqpoint{4.162368in}{1.447182in}}{\pgfqpoint{4.154468in}{1.443910in}}{\pgfqpoint{4.148644in}{1.438086in}}%
\pgfpathcurveto{\pgfqpoint{4.142820in}{1.432262in}}{\pgfqpoint{4.139548in}{1.424362in}}{\pgfqpoint{4.139548in}{1.416126in}}%
\pgfpathcurveto{\pgfqpoint{4.139548in}{1.407889in}}{\pgfqpoint{4.142820in}{1.399989in}}{\pgfqpoint{4.148644in}{1.394165in}}%
\pgfpathcurveto{\pgfqpoint{4.154468in}{1.388341in}}{\pgfqpoint{4.162368in}{1.385069in}}{\pgfqpoint{4.170605in}{1.385069in}}%
\pgfpathclose%
\pgfusepath{stroke,fill}%
\end{pgfscope}%
\begin{pgfscope}%
\pgfpathrectangle{\pgfqpoint{3.696748in}{0.557870in}}{\pgfqpoint{2.424965in}{1.484734in}}%
\pgfusepath{clip}%
\pgfsetbuttcap%
\pgfsetroundjoin%
\definecolor{currentfill}{rgb}{0.298039,0.447059,0.690196}%
\pgfsetfillcolor{currentfill}%
\pgfsetlinewidth{1.003750pt}%
\definecolor{currentstroke}{rgb}{0.298039,0.447059,0.690196}%
\pgfsetstrokecolor{currentstroke}%
\pgfsetdash{}{0pt}%
\pgfpathmoveto{\pgfqpoint{4.997411in}{0.853346in}}%
\pgfpathcurveto{\pgfqpoint{5.005647in}{0.853346in}}{\pgfqpoint{5.013547in}{0.856618in}}{\pgfqpoint{5.019371in}{0.862442in}}%
\pgfpathcurveto{\pgfqpoint{5.025195in}{0.868266in}}{\pgfqpoint{5.028467in}{0.876166in}}{\pgfqpoint{5.028467in}{0.884403in}}%
\pgfpathcurveto{\pgfqpoint{5.028467in}{0.892639in}}{\pgfqpoint{5.025195in}{0.900539in}}{\pgfqpoint{5.019371in}{0.906363in}}%
\pgfpathcurveto{\pgfqpoint{5.013547in}{0.912187in}}{\pgfqpoint{5.005647in}{0.915459in}}{\pgfqpoint{4.997411in}{0.915459in}}%
\pgfpathcurveto{\pgfqpoint{4.989175in}{0.915459in}}{\pgfqpoint{4.981274in}{0.912187in}}{\pgfqpoint{4.975451in}{0.906363in}}%
\pgfpathcurveto{\pgfqpoint{4.969627in}{0.900539in}}{\pgfqpoint{4.966354in}{0.892639in}}{\pgfqpoint{4.966354in}{0.884403in}}%
\pgfpathcurveto{\pgfqpoint{4.966354in}{0.876166in}}{\pgfqpoint{4.969627in}{0.868266in}}{\pgfqpoint{4.975451in}{0.862442in}}%
\pgfpathcurveto{\pgfqpoint{4.981274in}{0.856618in}}{\pgfqpoint{4.989175in}{0.853346in}}{\pgfqpoint{4.997411in}{0.853346in}}%
\pgfpathclose%
\pgfusepath{stroke,fill}%
\end{pgfscope}%
\begin{pgfscope}%
\pgfpathrectangle{\pgfqpoint{3.696748in}{0.557870in}}{\pgfqpoint{2.424965in}{1.484734in}}%
\pgfusepath{clip}%
\pgfsetbuttcap%
\pgfsetroundjoin%
\definecolor{currentfill}{rgb}{0.298039,0.447059,0.690196}%
\pgfsetfillcolor{currentfill}%
\pgfsetlinewidth{1.003750pt}%
\definecolor{currentstroke}{rgb}{0.298039,0.447059,0.690196}%
\pgfsetstrokecolor{currentstroke}%
\pgfsetdash{}{0pt}%
\pgfpathmoveto{\pgfqpoint{4.034243in}{1.466873in}}%
\pgfpathcurveto{\pgfqpoint{4.042479in}{1.466873in}}{\pgfqpoint{4.050379in}{1.470145in}}{\pgfqpoint{4.056203in}{1.475969in}}%
\pgfpathcurveto{\pgfqpoint{4.062027in}{1.481793in}}{\pgfqpoint{4.065299in}{1.489693in}}{\pgfqpoint{4.065299in}{1.497929in}}%
\pgfpathcurveto{\pgfqpoint{4.065299in}{1.506165in}}{\pgfqpoint{4.062027in}{1.514065in}}{\pgfqpoint{4.056203in}{1.519889in}}%
\pgfpathcurveto{\pgfqpoint{4.050379in}{1.525713in}}{\pgfqpoint{4.042479in}{1.528986in}}{\pgfqpoint{4.034243in}{1.528986in}}%
\pgfpathcurveto{\pgfqpoint{4.026007in}{1.528986in}}{\pgfqpoint{4.018107in}{1.525713in}}{\pgfqpoint{4.012283in}{1.519889in}}%
\pgfpathcurveto{\pgfqpoint{4.006459in}{1.514065in}}{\pgfqpoint{4.003186in}{1.506165in}}{\pgfqpoint{4.003186in}{1.497929in}}%
\pgfpathcurveto{\pgfqpoint{4.003186in}{1.489693in}}{\pgfqpoint{4.006459in}{1.481793in}}{\pgfqpoint{4.012283in}{1.475969in}}%
\pgfpathcurveto{\pgfqpoint{4.018107in}{1.470145in}}{\pgfqpoint{4.026007in}{1.466873in}}{\pgfqpoint{4.034243in}{1.466873in}}%
\pgfpathclose%
\pgfusepath{stroke,fill}%
\end{pgfscope}%
\begin{pgfscope}%
\pgfpathrectangle{\pgfqpoint{3.696748in}{0.557870in}}{\pgfqpoint{2.424965in}{1.484734in}}%
\pgfusepath{clip}%
\pgfsetbuttcap%
\pgfsetroundjoin%
\definecolor{currentfill}{rgb}{0.298039,0.447059,0.690196}%
\pgfsetfillcolor{currentfill}%
\pgfsetlinewidth{1.003750pt}%
\definecolor{currentstroke}{rgb}{0.298039,0.447059,0.690196}%
\pgfsetstrokecolor{currentstroke}%
\pgfsetdash{}{0pt}%
\pgfpathmoveto{\pgfqpoint{4.291966in}{1.153292in}}%
\pgfpathcurveto{\pgfqpoint{4.300203in}{1.153292in}}{\pgfqpoint{4.308103in}{1.156565in}}{\pgfqpoint{4.313927in}{1.162389in}}%
\pgfpathcurveto{\pgfqpoint{4.319751in}{1.168213in}}{\pgfqpoint{4.323023in}{1.176113in}}{\pgfqpoint{4.323023in}{1.184349in}}%
\pgfpathcurveto{\pgfqpoint{4.323023in}{1.192585in}}{\pgfqpoint{4.319751in}{1.200485in}}{\pgfqpoint{4.313927in}{1.206309in}}%
\pgfpathcurveto{\pgfqpoint{4.308103in}{1.212133in}}{\pgfqpoint{4.300203in}{1.215405in}}{\pgfqpoint{4.291966in}{1.215405in}}%
\pgfpathcurveto{\pgfqpoint{4.283730in}{1.215405in}}{\pgfqpoint{4.275830in}{1.212133in}}{\pgfqpoint{4.270006in}{1.206309in}}%
\pgfpathcurveto{\pgfqpoint{4.264182in}{1.200485in}}{\pgfqpoint{4.260910in}{1.192585in}}{\pgfqpoint{4.260910in}{1.184349in}}%
\pgfpathcurveto{\pgfqpoint{4.260910in}{1.176113in}}{\pgfqpoint{4.264182in}{1.168213in}}{\pgfqpoint{4.270006in}{1.162389in}}%
\pgfpathcurveto{\pgfqpoint{4.275830in}{1.156565in}}{\pgfqpoint{4.283730in}{1.153292in}}{\pgfqpoint{4.291966in}{1.153292in}}%
\pgfpathclose%
\pgfusepath{stroke,fill}%
\end{pgfscope}%
\begin{pgfscope}%
\pgfpathrectangle{\pgfqpoint{3.696748in}{0.557870in}}{\pgfqpoint{2.424965in}{1.484734in}}%
\pgfusepath{clip}%
\pgfsetbuttcap%
\pgfsetroundjoin%
\definecolor{currentfill}{rgb}{0.298039,0.447059,0.690196}%
\pgfsetfillcolor{currentfill}%
\pgfsetlinewidth{1.003750pt}%
\definecolor{currentstroke}{rgb}{0.298039,0.447059,0.690196}%
\pgfsetstrokecolor{currentstroke}%
\pgfsetdash{}{0pt}%
\pgfpathmoveto{\pgfqpoint{4.556963in}{0.938628in}}%
\pgfpathcurveto{\pgfqpoint{4.565199in}{0.938628in}}{\pgfqpoint{4.573099in}{0.941900in}}{\pgfqpoint{4.578923in}{0.947724in}}%
\pgfpathcurveto{\pgfqpoint{4.584747in}{0.953548in}}{\pgfqpoint{4.588019in}{0.961448in}}{\pgfqpoint{4.588019in}{0.969684in}}%
\pgfpathcurveto{\pgfqpoint{4.588019in}{0.977921in}}{\pgfqpoint{4.584747in}{0.985821in}}{\pgfqpoint{4.578923in}{0.991644in}}%
\pgfpathcurveto{\pgfqpoint{4.573099in}{0.997468in}}{\pgfqpoint{4.565199in}{1.000741in}}{\pgfqpoint{4.556963in}{1.000741in}}%
\pgfpathcurveto{\pgfqpoint{4.548726in}{1.000741in}}{\pgfqpoint{4.540826in}{0.997468in}}{\pgfqpoint{4.535002in}{0.991644in}}%
\pgfpathcurveto{\pgfqpoint{4.529178in}{0.985821in}}{\pgfqpoint{4.525906in}{0.977921in}}{\pgfqpoint{4.525906in}{0.969684in}}%
\pgfpathcurveto{\pgfqpoint{4.525906in}{0.961448in}}{\pgfqpoint{4.529178in}{0.953548in}}{\pgfqpoint{4.535002in}{0.947724in}}%
\pgfpathcurveto{\pgfqpoint{4.540826in}{0.941900in}}{\pgfqpoint{4.548726in}{0.938628in}}{\pgfqpoint{4.556963in}{0.938628in}}%
\pgfpathclose%
\pgfusepath{stroke,fill}%
\end{pgfscope}%
\begin{pgfscope}%
\pgfpathrectangle{\pgfqpoint{3.696748in}{0.557870in}}{\pgfqpoint{2.424965in}{1.484734in}}%
\pgfusepath{clip}%
\pgfsetbuttcap%
\pgfsetroundjoin%
\definecolor{currentfill}{rgb}{0.298039,0.447059,0.690196}%
\pgfsetfillcolor{currentfill}%
\pgfsetlinewidth{1.003750pt}%
\definecolor{currentstroke}{rgb}{0.298039,0.447059,0.690196}%
\pgfsetstrokecolor{currentstroke}%
\pgfsetdash{}{0pt}%
\pgfpathmoveto{\pgfqpoint{4.291966in}{1.062585in}}%
\pgfpathcurveto{\pgfqpoint{4.300203in}{1.062585in}}{\pgfqpoint{4.308103in}{1.065857in}}{\pgfqpoint{4.313927in}{1.071681in}}%
\pgfpathcurveto{\pgfqpoint{4.319751in}{1.077505in}}{\pgfqpoint{4.323023in}{1.085405in}}{\pgfqpoint{4.323023in}{1.093642in}}%
\pgfpathcurveto{\pgfqpoint{4.323023in}{1.101878in}}{\pgfqpoint{4.319751in}{1.109778in}}{\pgfqpoint{4.313927in}{1.115602in}}%
\pgfpathcurveto{\pgfqpoint{4.308103in}{1.121426in}}{\pgfqpoint{4.300203in}{1.124698in}}{\pgfqpoint{4.291966in}{1.124698in}}%
\pgfpathcurveto{\pgfqpoint{4.283730in}{1.124698in}}{\pgfqpoint{4.275830in}{1.121426in}}{\pgfqpoint{4.270006in}{1.115602in}}%
\pgfpathcurveto{\pgfqpoint{4.264182in}{1.109778in}}{\pgfqpoint{4.260910in}{1.101878in}}{\pgfqpoint{4.260910in}{1.093642in}}%
\pgfpathcurveto{\pgfqpoint{4.260910in}{1.085405in}}{\pgfqpoint{4.264182in}{1.077505in}}{\pgfqpoint{4.270006in}{1.071681in}}%
\pgfpathcurveto{\pgfqpoint{4.275830in}{1.065857in}}{\pgfqpoint{4.283730in}{1.062585in}}{\pgfqpoint{4.291966in}{1.062585in}}%
\pgfpathclose%
\pgfusepath{stroke,fill}%
\end{pgfscope}%
\begin{pgfscope}%
\pgfpathrectangle{\pgfqpoint{3.696748in}{0.557870in}}{\pgfqpoint{2.424965in}{1.484734in}}%
\pgfusepath{clip}%
\pgfsetbuttcap%
\pgfsetroundjoin%
\definecolor{currentfill}{rgb}{0.298039,0.447059,0.690196}%
\pgfsetfillcolor{currentfill}%
\pgfsetlinewidth{1.003750pt}%
\definecolor{currentstroke}{rgb}{0.298039,0.447059,0.690196}%
\pgfsetstrokecolor{currentstroke}%
\pgfsetdash{}{0pt}%
\pgfpathmoveto{\pgfqpoint{4.443328in}{0.935150in}}%
\pgfpathcurveto{\pgfqpoint{4.451564in}{0.935150in}}{\pgfqpoint{4.459464in}{0.938422in}}{\pgfqpoint{4.465288in}{0.944246in}}%
\pgfpathcurveto{\pgfqpoint{4.471112in}{0.950070in}}{\pgfqpoint{4.474384in}{0.957970in}}{\pgfqpoint{4.474384in}{0.966206in}}%
\pgfpathcurveto{\pgfqpoint{4.474384in}{0.974442in}}{\pgfqpoint{4.471112in}{0.982343in}}{\pgfqpoint{4.465288in}{0.988166in}}%
\pgfpathcurveto{\pgfqpoint{4.459464in}{0.993990in}}{\pgfqpoint{4.451564in}{0.997263in}}{\pgfqpoint{4.443328in}{0.997263in}}%
\pgfpathcurveto{\pgfqpoint{4.435092in}{0.997263in}}{\pgfqpoint{4.427192in}{0.993990in}}{\pgfqpoint{4.421368in}{0.988166in}}%
\pgfpathcurveto{\pgfqpoint{4.415544in}{0.982343in}}{\pgfqpoint{4.412271in}{0.974442in}}{\pgfqpoint{4.412271in}{0.966206in}}%
\pgfpathcurveto{\pgfqpoint{4.412271in}{0.957970in}}{\pgfqpoint{4.415544in}{0.950070in}}{\pgfqpoint{4.421368in}{0.944246in}}%
\pgfpathcurveto{\pgfqpoint{4.427192in}{0.938422in}}{\pgfqpoint{4.435092in}{0.935150in}}{\pgfqpoint{4.443328in}{0.935150in}}%
\pgfpathclose%
\pgfusepath{stroke,fill}%
\end{pgfscope}%
\begin{pgfscope}%
\pgfpathrectangle{\pgfqpoint{3.696748in}{0.557870in}}{\pgfqpoint{2.424965in}{1.484734in}}%
\pgfusepath{clip}%
\pgfsetbuttcap%
\pgfsetroundjoin%
\definecolor{currentfill}{rgb}{0.298039,0.447059,0.690196}%
\pgfsetfillcolor{currentfill}%
\pgfsetlinewidth{1.003750pt}%
\definecolor{currentstroke}{rgb}{0.298039,0.447059,0.690196}%
\pgfsetstrokecolor{currentstroke}%
\pgfsetdash{}{0pt}%
\pgfpathmoveto{\pgfqpoint{4.011516in}{1.289632in}}%
\pgfpathcurveto{\pgfqpoint{4.019752in}{1.289632in}}{\pgfqpoint{4.027652in}{1.292904in}}{\pgfqpoint{4.033476in}{1.298728in}}%
\pgfpathcurveto{\pgfqpoint{4.039300in}{1.304552in}}{\pgfqpoint{4.042572in}{1.312452in}}{\pgfqpoint{4.042572in}{1.320688in}}%
\pgfpathcurveto{\pgfqpoint{4.042572in}{1.328924in}}{\pgfqpoint{4.039300in}{1.336824in}}{\pgfqpoint{4.033476in}{1.342648in}}%
\pgfpathcurveto{\pgfqpoint{4.027652in}{1.348472in}}{\pgfqpoint{4.019752in}{1.351745in}}{\pgfqpoint{4.011516in}{1.351745in}}%
\pgfpathcurveto{\pgfqpoint{4.003280in}{1.351745in}}{\pgfqpoint{3.995380in}{1.348472in}}{\pgfqpoint{3.989556in}{1.342648in}}%
\pgfpathcurveto{\pgfqpoint{3.983732in}{1.336824in}}{\pgfqpoint{3.980459in}{1.328924in}}{\pgfqpoint{3.980459in}{1.320688in}}%
\pgfpathcurveto{\pgfqpoint{3.980459in}{1.312452in}}{\pgfqpoint{3.983732in}{1.304552in}}{\pgfqpoint{3.989556in}{1.298728in}}%
\pgfpathcurveto{\pgfqpoint{3.995380in}{1.292904in}}{\pgfqpoint{4.003280in}{1.289632in}}{\pgfqpoint{4.011516in}{1.289632in}}%
\pgfpathclose%
\pgfusepath{stroke,fill}%
\end{pgfscope}%
\begin{pgfscope}%
\pgfpathrectangle{\pgfqpoint{3.696748in}{0.557870in}}{\pgfqpoint{2.424965in}{1.484734in}}%
\pgfusepath{clip}%
\pgfsetbuttcap%
\pgfsetroundjoin%
\definecolor{currentfill}{rgb}{0.298039,0.447059,0.690196}%
\pgfsetfillcolor{currentfill}%
\pgfsetlinewidth{1.003750pt}%
\definecolor{currentstroke}{rgb}{0.298039,0.447059,0.690196}%
\pgfsetstrokecolor{currentstroke}%
\pgfsetdash{}{0pt}%
\pgfpathmoveto{\pgfqpoint{3.988789in}{1.296727in}}%
\pgfpathcurveto{\pgfqpoint{3.997025in}{1.296727in}}{\pgfqpoint{4.004925in}{1.299999in}}{\pgfqpoint{4.010749in}{1.305823in}}%
\pgfpathcurveto{\pgfqpoint{4.016573in}{1.311647in}}{\pgfqpoint{4.019845in}{1.319547in}}{\pgfqpoint{4.019845in}{1.327783in}}%
\pgfpathcurveto{\pgfqpoint{4.019845in}{1.336020in}}{\pgfqpoint{4.016573in}{1.343920in}}{\pgfqpoint{4.010749in}{1.349744in}}%
\pgfpathcurveto{\pgfqpoint{4.004925in}{1.355567in}}{\pgfqpoint{3.997025in}{1.358840in}}{\pgfqpoint{3.988789in}{1.358840in}}%
\pgfpathcurveto{\pgfqpoint{3.980553in}{1.358840in}}{\pgfqpoint{3.972653in}{1.355567in}}{\pgfqpoint{3.966829in}{1.349744in}}%
\pgfpathcurveto{\pgfqpoint{3.961005in}{1.343920in}}{\pgfqpoint{3.957732in}{1.336020in}}{\pgfqpoint{3.957732in}{1.327783in}}%
\pgfpathcurveto{\pgfqpoint{3.957732in}{1.319547in}}{\pgfqpoint{3.961005in}{1.311647in}}{\pgfqpoint{3.966829in}{1.305823in}}%
\pgfpathcurveto{\pgfqpoint{3.972653in}{1.299999in}}{\pgfqpoint{3.980553in}{1.296727in}}{\pgfqpoint{3.988789in}{1.296727in}}%
\pgfpathclose%
\pgfusepath{stroke,fill}%
\end{pgfscope}%
\begin{pgfscope}%
\pgfpathrectangle{\pgfqpoint{3.696748in}{0.557870in}}{\pgfqpoint{2.424965in}{1.484734in}}%
\pgfusepath{clip}%
\pgfsetbuttcap%
\pgfsetroundjoin%
\definecolor{currentfill}{rgb}{0.298039,0.447059,0.690196}%
\pgfsetfillcolor{currentfill}%
\pgfsetlinewidth{1.003750pt}%
\definecolor{currentstroke}{rgb}{0.298039,0.447059,0.690196}%
\pgfsetstrokecolor{currentstroke}%
\pgfsetdash{}{0pt}%
\pgfpathmoveto{\pgfqpoint{4.137650in}{1.071489in}}%
\pgfpathcurveto{\pgfqpoint{4.145887in}{1.071489in}}{\pgfqpoint{4.153787in}{1.074761in}}{\pgfqpoint{4.159611in}{1.080585in}}%
\pgfpathcurveto{\pgfqpoint{4.165435in}{1.086409in}}{\pgfqpoint{4.168707in}{1.094309in}}{\pgfqpoint{4.168707in}{1.102545in}}%
\pgfpathcurveto{\pgfqpoint{4.168707in}{1.110782in}}{\pgfqpoint{4.165435in}{1.118682in}}{\pgfqpoint{4.159611in}{1.124506in}}%
\pgfpathcurveto{\pgfqpoint{4.153787in}{1.130330in}}{\pgfqpoint{4.145887in}{1.133602in}}{\pgfqpoint{4.137650in}{1.133602in}}%
\pgfpathcurveto{\pgfqpoint{4.129414in}{1.133602in}}{\pgfqpoint{4.121514in}{1.130330in}}{\pgfqpoint{4.115690in}{1.124506in}}%
\pgfpathcurveto{\pgfqpoint{4.109866in}{1.118682in}}{\pgfqpoint{4.106594in}{1.110782in}}{\pgfqpoint{4.106594in}{1.102545in}}%
\pgfpathcurveto{\pgfqpoint{4.106594in}{1.094309in}}{\pgfqpoint{4.109866in}{1.086409in}}{\pgfqpoint{4.115690in}{1.080585in}}%
\pgfpathcurveto{\pgfqpoint{4.121514in}{1.074761in}}{\pgfqpoint{4.129414in}{1.071489in}}{\pgfqpoint{4.137650in}{1.071489in}}%
\pgfpathclose%
\pgfusepath{stroke,fill}%
\end{pgfscope}%
\begin{pgfscope}%
\pgfpathrectangle{\pgfqpoint{3.696748in}{0.557870in}}{\pgfqpoint{2.424965in}{1.484734in}}%
\pgfusepath{clip}%
\pgfsetbuttcap%
\pgfsetroundjoin%
\definecolor{currentfill}{rgb}{0.298039,0.447059,0.690196}%
\pgfsetfillcolor{currentfill}%
\pgfsetlinewidth{1.003750pt}%
\definecolor{currentstroke}{rgb}{0.298039,0.447059,0.690196}%
\pgfsetstrokecolor{currentstroke}%
\pgfsetdash{}{0pt}%
\pgfpathmoveto{\pgfqpoint{4.336057in}{0.894248in}}%
\pgfpathcurveto{\pgfqpoint{4.344293in}{0.894248in}}{\pgfqpoint{4.352193in}{0.897520in}}{\pgfqpoint{4.358017in}{0.903344in}}%
\pgfpathcurveto{\pgfqpoint{4.363841in}{0.909168in}}{\pgfqpoint{4.367113in}{0.917068in}}{\pgfqpoint{4.367113in}{0.925304in}}%
\pgfpathcurveto{\pgfqpoint{4.367113in}{0.933541in}}{\pgfqpoint{4.363841in}{0.941441in}}{\pgfqpoint{4.358017in}{0.947265in}}%
\pgfpathcurveto{\pgfqpoint{4.352193in}{0.953089in}}{\pgfqpoint{4.344293in}{0.956361in}}{\pgfqpoint{4.336057in}{0.956361in}}%
\pgfpathcurveto{\pgfqpoint{4.327820in}{0.956361in}}{\pgfqpoint{4.319920in}{0.953089in}}{\pgfqpoint{4.314096in}{0.947265in}}%
\pgfpathcurveto{\pgfqpoint{4.308272in}{0.941441in}}{\pgfqpoint{4.305000in}{0.933541in}}{\pgfqpoint{4.305000in}{0.925304in}}%
\pgfpathcurveto{\pgfqpoint{4.305000in}{0.917068in}}{\pgfqpoint{4.308272in}{0.909168in}}{\pgfqpoint{4.314096in}{0.903344in}}%
\pgfpathcurveto{\pgfqpoint{4.319920in}{0.897520in}}{\pgfqpoint{4.327820in}{0.894248in}}{\pgfqpoint{4.336057in}{0.894248in}}%
\pgfpathclose%
\pgfusepath{stroke,fill}%
\end{pgfscope}%
\begin{pgfscope}%
\pgfpathrectangle{\pgfqpoint{3.696748in}{0.557870in}}{\pgfqpoint{2.424965in}{1.484734in}}%
\pgfusepath{clip}%
\pgfsetbuttcap%
\pgfsetroundjoin%
\definecolor{currentfill}{rgb}{0.298039,0.447059,0.690196}%
\pgfsetfillcolor{currentfill}%
\pgfsetlinewidth{1.003750pt}%
\definecolor{currentstroke}{rgb}{0.298039,0.447059,0.690196}%
\pgfsetstrokecolor{currentstroke}%
\pgfsetdash{}{0pt}%
\pgfpathmoveto{\pgfqpoint{4.093560in}{1.085123in}}%
\pgfpathcurveto{\pgfqpoint{4.101796in}{1.085123in}}{\pgfqpoint{4.109697in}{1.088395in}}{\pgfqpoint{4.115520in}{1.094219in}}%
\pgfpathcurveto{\pgfqpoint{4.121344in}{1.100043in}}{\pgfqpoint{4.124617in}{1.107943in}}{\pgfqpoint{4.124617in}{1.116179in}}%
\pgfpathcurveto{\pgfqpoint{4.124617in}{1.124416in}}{\pgfqpoint{4.121344in}{1.132316in}}{\pgfqpoint{4.115520in}{1.138140in}}%
\pgfpathcurveto{\pgfqpoint{4.109697in}{1.143963in}}{\pgfqpoint{4.101796in}{1.147236in}}{\pgfqpoint{4.093560in}{1.147236in}}%
\pgfpathcurveto{\pgfqpoint{4.085324in}{1.147236in}}{\pgfqpoint{4.077424in}{1.143963in}}{\pgfqpoint{4.071600in}{1.138140in}}%
\pgfpathcurveto{\pgfqpoint{4.065776in}{1.132316in}}{\pgfqpoint{4.062504in}{1.124416in}}{\pgfqpoint{4.062504in}{1.116179in}}%
\pgfpathcurveto{\pgfqpoint{4.062504in}{1.107943in}}{\pgfqpoint{4.065776in}{1.100043in}}{\pgfqpoint{4.071600in}{1.094219in}}%
\pgfpathcurveto{\pgfqpoint{4.077424in}{1.088395in}}{\pgfqpoint{4.085324in}{1.085123in}}{\pgfqpoint{4.093560in}{1.085123in}}%
\pgfpathclose%
\pgfusepath{stroke,fill}%
\end{pgfscope}%
\begin{pgfscope}%
\pgfpathrectangle{\pgfqpoint{3.696748in}{0.557870in}}{\pgfqpoint{2.424965in}{1.484734in}}%
\pgfusepath{clip}%
\pgfsetbuttcap%
\pgfsetroundjoin%
\definecolor{currentfill}{rgb}{0.298039,0.447059,0.690196}%
\pgfsetfillcolor{currentfill}%
\pgfsetlinewidth{1.003750pt}%
\definecolor{currentstroke}{rgb}{0.298039,0.447059,0.690196}%
\pgfsetstrokecolor{currentstroke}%
\pgfsetdash{}{0pt}%
\pgfpathmoveto{\pgfqpoint{4.159696in}{0.966174in}}%
\pgfpathcurveto{\pgfqpoint{4.167932in}{0.966174in}}{\pgfqpoint{4.175832in}{0.969446in}}{\pgfqpoint{4.181656in}{0.975270in}}%
\pgfpathcurveto{\pgfqpoint{4.187480in}{0.981094in}}{\pgfqpoint{4.190752in}{0.988994in}}{\pgfqpoint{4.190752in}{0.997230in}}%
\pgfpathcurveto{\pgfqpoint{4.190752in}{1.005467in}}{\pgfqpoint{4.187480in}{1.013367in}}{\pgfqpoint{4.181656in}{1.019191in}}%
\pgfpathcurveto{\pgfqpoint{4.175832in}{1.025014in}}{\pgfqpoint{4.167932in}{1.028287in}}{\pgfqpoint{4.159696in}{1.028287in}}%
\pgfpathcurveto{\pgfqpoint{4.151459in}{1.028287in}}{\pgfqpoint{4.143559in}{1.025014in}}{\pgfqpoint{4.137735in}{1.019191in}}%
\pgfpathcurveto{\pgfqpoint{4.131911in}{1.013367in}}{\pgfqpoint{4.128639in}{1.005467in}}{\pgfqpoint{4.128639in}{0.997230in}}%
\pgfpathcurveto{\pgfqpoint{4.128639in}{0.988994in}}{\pgfqpoint{4.131911in}{0.981094in}}{\pgfqpoint{4.137735in}{0.975270in}}%
\pgfpathcurveto{\pgfqpoint{4.143559in}{0.969446in}}{\pgfqpoint{4.151459in}{0.966174in}}{\pgfqpoint{4.159696in}{0.966174in}}%
\pgfpathclose%
\pgfusepath{stroke,fill}%
\end{pgfscope}%
\begin{pgfscope}%
\pgfpathrectangle{\pgfqpoint{3.696748in}{0.557870in}}{\pgfqpoint{2.424965in}{1.484734in}}%
\pgfusepath{clip}%
\pgfsetbuttcap%
\pgfsetroundjoin%
\definecolor{currentfill}{rgb}{0.298039,0.447059,0.690196}%
\pgfsetfillcolor{currentfill}%
\pgfsetlinewidth{1.003750pt}%
\definecolor{currentstroke}{rgb}{0.298039,0.447059,0.690196}%
\pgfsetstrokecolor{currentstroke}%
\pgfsetdash{}{0pt}%
\pgfpathmoveto{\pgfqpoint{4.466055in}{0.773351in}}%
\pgfpathcurveto{\pgfqpoint{4.474291in}{0.773351in}}{\pgfqpoint{4.482191in}{0.776624in}}{\pgfqpoint{4.488015in}{0.782447in}}%
\pgfpathcurveto{\pgfqpoint{4.493839in}{0.788271in}}{\pgfqpoint{4.497111in}{0.796171in}}{\pgfqpoint{4.497111in}{0.804408in}}%
\pgfpathcurveto{\pgfqpoint{4.497111in}{0.812644in}}{\pgfqpoint{4.493839in}{0.820544in}}{\pgfqpoint{4.488015in}{0.826368in}}%
\pgfpathcurveto{\pgfqpoint{4.482191in}{0.832192in}}{\pgfqpoint{4.474291in}{0.835464in}}{\pgfqpoint{4.466055in}{0.835464in}}%
\pgfpathcurveto{\pgfqpoint{4.457819in}{0.835464in}}{\pgfqpoint{4.449918in}{0.832192in}}{\pgfqpoint{4.444095in}{0.826368in}}%
\pgfpathcurveto{\pgfqpoint{4.438271in}{0.820544in}}{\pgfqpoint{4.434998in}{0.812644in}}{\pgfqpoint{4.434998in}{0.804408in}}%
\pgfpathcurveto{\pgfqpoint{4.434998in}{0.796171in}}{\pgfqpoint{4.438271in}{0.788271in}}{\pgfqpoint{4.444095in}{0.782447in}}%
\pgfpathcurveto{\pgfqpoint{4.449918in}{0.776624in}}{\pgfqpoint{4.457819in}{0.773351in}}{\pgfqpoint{4.466055in}{0.773351in}}%
\pgfpathclose%
\pgfusepath{stroke,fill}%
\end{pgfscope}%
\begin{pgfscope}%
\pgfpathrectangle{\pgfqpoint{3.696748in}{0.557870in}}{\pgfqpoint{2.424965in}{1.484734in}}%
\pgfusepath{clip}%
\pgfsetbuttcap%
\pgfsetroundjoin%
\definecolor{currentfill}{rgb}{0.298039,0.447059,0.690196}%
\pgfsetfillcolor{currentfill}%
\pgfsetlinewidth{1.003750pt}%
\definecolor{currentstroke}{rgb}{0.298039,0.447059,0.690196}%
\pgfsetstrokecolor{currentstroke}%
\pgfsetdash{}{0pt}%
\pgfpathmoveto{\pgfqpoint{4.466055in}{0.773351in}}%
\pgfpathcurveto{\pgfqpoint{4.474291in}{0.773351in}}{\pgfqpoint{4.482191in}{0.776624in}}{\pgfqpoint{4.488015in}{0.782447in}}%
\pgfpathcurveto{\pgfqpoint{4.493839in}{0.788271in}}{\pgfqpoint{4.497111in}{0.796171in}}{\pgfqpoint{4.497111in}{0.804408in}}%
\pgfpathcurveto{\pgfqpoint{4.497111in}{0.812644in}}{\pgfqpoint{4.493839in}{0.820544in}}{\pgfqpoint{4.488015in}{0.826368in}}%
\pgfpathcurveto{\pgfqpoint{4.482191in}{0.832192in}}{\pgfqpoint{4.474291in}{0.835464in}}{\pgfqpoint{4.466055in}{0.835464in}}%
\pgfpathcurveto{\pgfqpoint{4.457819in}{0.835464in}}{\pgfqpoint{4.449918in}{0.832192in}}{\pgfqpoint{4.444095in}{0.826368in}}%
\pgfpathcurveto{\pgfqpoint{4.438271in}{0.820544in}}{\pgfqpoint{4.434998in}{0.812644in}}{\pgfqpoint{4.434998in}{0.804408in}}%
\pgfpathcurveto{\pgfqpoint{4.434998in}{0.796171in}}{\pgfqpoint{4.438271in}{0.788271in}}{\pgfqpoint{4.444095in}{0.782447in}}%
\pgfpathcurveto{\pgfqpoint{4.449918in}{0.776624in}}{\pgfqpoint{4.457819in}{0.773351in}}{\pgfqpoint{4.466055in}{0.773351in}}%
\pgfpathclose%
\pgfusepath{stroke,fill}%
\end{pgfscope}%
\begin{pgfscope}%
\pgfpathrectangle{\pgfqpoint{3.696748in}{0.557870in}}{\pgfqpoint{2.424965in}{1.484734in}}%
\pgfusepath{clip}%
\pgfsetbuttcap%
\pgfsetroundjoin%
\definecolor{currentfill}{rgb}{0.298039,0.447059,0.690196}%
\pgfsetfillcolor{currentfill}%
\pgfsetlinewidth{1.003750pt}%
\definecolor{currentstroke}{rgb}{0.298039,0.447059,0.690196}%
\pgfsetstrokecolor{currentstroke}%
\pgfsetdash{}{0pt}%
\pgfpathmoveto{\pgfqpoint{3.943335in}{1.227862in}}%
\pgfpathcurveto{\pgfqpoint{3.951571in}{1.227862in}}{\pgfqpoint{3.959471in}{1.231134in}}{\pgfqpoint{3.965295in}{1.236958in}}%
\pgfpathcurveto{\pgfqpoint{3.971119in}{1.242782in}}{\pgfqpoint{3.974392in}{1.250682in}}{\pgfqpoint{3.974392in}{1.258918in}}%
\pgfpathcurveto{\pgfqpoint{3.974392in}{1.267154in}}{\pgfqpoint{3.971119in}{1.275054in}}{\pgfqpoint{3.965295in}{1.280878in}}%
\pgfpathcurveto{\pgfqpoint{3.959471in}{1.286702in}}{\pgfqpoint{3.951571in}{1.289975in}}{\pgfqpoint{3.943335in}{1.289975in}}%
\pgfpathcurveto{\pgfqpoint{3.935099in}{1.289975in}}{\pgfqpoint{3.927199in}{1.286702in}}{\pgfqpoint{3.921375in}{1.280878in}}%
\pgfpathcurveto{\pgfqpoint{3.915551in}{1.275054in}}{\pgfqpoint{3.912279in}{1.267154in}}{\pgfqpoint{3.912279in}{1.258918in}}%
\pgfpathcurveto{\pgfqpoint{3.912279in}{1.250682in}}{\pgfqpoint{3.915551in}{1.242782in}}{\pgfqpoint{3.921375in}{1.236958in}}%
\pgfpathcurveto{\pgfqpoint{3.927199in}{1.231134in}}{\pgfqpoint{3.935099in}{1.227862in}}{\pgfqpoint{3.943335in}{1.227862in}}%
\pgfpathclose%
\pgfusepath{stroke,fill}%
\end{pgfscope}%
\begin{pgfscope}%
\pgfpathrectangle{\pgfqpoint{3.696748in}{0.557870in}}{\pgfqpoint{2.424965in}{1.484734in}}%
\pgfusepath{clip}%
\pgfsetbuttcap%
\pgfsetroundjoin%
\definecolor{currentfill}{rgb}{0.298039,0.447059,0.690196}%
\pgfsetfillcolor{currentfill}%
\pgfsetlinewidth{1.003750pt}%
\definecolor{currentstroke}{rgb}{0.298039,0.447059,0.690196}%
\pgfsetstrokecolor{currentstroke}%
\pgfsetdash{}{0pt}%
\pgfpathmoveto{\pgfqpoint{4.102424in}{0.962418in}}%
\pgfpathcurveto{\pgfqpoint{4.110660in}{0.962418in}}{\pgfqpoint{4.118560in}{0.965690in}}{\pgfqpoint{4.124384in}{0.971514in}}%
\pgfpathcurveto{\pgfqpoint{4.130208in}{0.977338in}}{\pgfqpoint{4.133480in}{0.985238in}}{\pgfqpoint{4.133480in}{0.993474in}}%
\pgfpathcurveto{\pgfqpoint{4.133480in}{1.001710in}}{\pgfqpoint{4.130208in}{1.009610in}}{\pgfqpoint{4.124384in}{1.015434in}}%
\pgfpathcurveto{\pgfqpoint{4.118560in}{1.021258in}}{\pgfqpoint{4.110660in}{1.024531in}}{\pgfqpoint{4.102424in}{1.024531in}}%
\pgfpathcurveto{\pgfqpoint{4.094187in}{1.024531in}}{\pgfqpoint{4.086287in}{1.021258in}}{\pgfqpoint{4.080463in}{1.015434in}}%
\pgfpathcurveto{\pgfqpoint{4.074639in}{1.009610in}}{\pgfqpoint{4.071367in}{1.001710in}}{\pgfqpoint{4.071367in}{0.993474in}}%
\pgfpathcurveto{\pgfqpoint{4.071367in}{0.985238in}}{\pgfqpoint{4.074639in}{0.977338in}}{\pgfqpoint{4.080463in}{0.971514in}}%
\pgfpathcurveto{\pgfqpoint{4.086287in}{0.965690in}}{\pgfqpoint{4.094187in}{0.962418in}}{\pgfqpoint{4.102424in}{0.962418in}}%
\pgfpathclose%
\pgfusepath{stroke,fill}%
\end{pgfscope}%
\begin{pgfscope}%
\pgfpathrectangle{\pgfqpoint{3.696748in}{0.557870in}}{\pgfqpoint{2.424965in}{1.484734in}}%
\pgfusepath{clip}%
\pgfsetbuttcap%
\pgfsetroundjoin%
\definecolor{currentfill}{rgb}{0.298039,0.447059,0.690196}%
\pgfsetfillcolor{currentfill}%
\pgfsetlinewidth{1.003750pt}%
\definecolor{currentstroke}{rgb}{0.298039,0.447059,0.690196}%
\pgfsetstrokecolor{currentstroke}%
\pgfsetdash{}{0pt}%
\pgfpathmoveto{\pgfqpoint{4.011516in}{1.030587in}}%
\pgfpathcurveto{\pgfqpoint{4.019752in}{1.030587in}}{\pgfqpoint{4.027652in}{1.033859in}}{\pgfqpoint{4.033476in}{1.039683in}}%
\pgfpathcurveto{\pgfqpoint{4.039300in}{1.045507in}}{\pgfqpoint{4.042572in}{1.053407in}}{\pgfqpoint{4.042572in}{1.061644in}}%
\pgfpathcurveto{\pgfqpoint{4.042572in}{1.069880in}}{\pgfqpoint{4.039300in}{1.077780in}}{\pgfqpoint{4.033476in}{1.083604in}}%
\pgfpathcurveto{\pgfqpoint{4.027652in}{1.089428in}}{\pgfqpoint{4.019752in}{1.092700in}}{\pgfqpoint{4.011516in}{1.092700in}}%
\pgfpathcurveto{\pgfqpoint{4.003280in}{1.092700in}}{\pgfqpoint{3.995380in}{1.089428in}}{\pgfqpoint{3.989556in}{1.083604in}}%
\pgfpathcurveto{\pgfqpoint{3.983732in}{1.077780in}}{\pgfqpoint{3.980459in}{1.069880in}}{\pgfqpoint{3.980459in}{1.061644in}}%
\pgfpathcurveto{\pgfqpoint{3.980459in}{1.053407in}}{\pgfqpoint{3.983732in}{1.045507in}}{\pgfqpoint{3.989556in}{1.039683in}}%
\pgfpathcurveto{\pgfqpoint{3.995380in}{1.033859in}}{\pgfqpoint{4.003280in}{1.030587in}}{\pgfqpoint{4.011516in}{1.030587in}}%
\pgfpathclose%
\pgfusepath{stroke,fill}%
\end{pgfscope}%
\begin{pgfscope}%
\pgfpathrectangle{\pgfqpoint{3.696748in}{0.557870in}}{\pgfqpoint{2.424965in}{1.484734in}}%
\pgfusepath{clip}%
\pgfsetbuttcap%
\pgfsetroundjoin%
\definecolor{currentfill}{rgb}{0.298039,0.447059,0.690196}%
\pgfsetfillcolor{currentfill}%
\pgfsetlinewidth{1.003750pt}%
\definecolor{currentstroke}{rgb}{0.298039,0.447059,0.690196}%
\pgfsetstrokecolor{currentstroke}%
\pgfsetdash{}{0pt}%
\pgfpathmoveto{\pgfqpoint{3.873109in}{1.158996in}}%
\pgfpathcurveto{\pgfqpoint{3.881345in}{1.158996in}}{\pgfqpoint{3.889245in}{1.162269in}}{\pgfqpoint{3.895069in}{1.168093in}}%
\pgfpathcurveto{\pgfqpoint{3.900893in}{1.173917in}}{\pgfqpoint{3.904165in}{1.181817in}}{\pgfqpoint{3.904165in}{1.190053in}}%
\pgfpathcurveto{\pgfqpoint{3.904165in}{1.198289in}}{\pgfqpoint{3.900893in}{1.206189in}}{\pgfqpoint{3.895069in}{1.212013in}}%
\pgfpathcurveto{\pgfqpoint{3.889245in}{1.217837in}}{\pgfqpoint{3.881345in}{1.221109in}}{\pgfqpoint{3.873109in}{1.221109in}}%
\pgfpathcurveto{\pgfqpoint{3.864873in}{1.221109in}}{\pgfqpoint{3.856972in}{1.217837in}}{\pgfqpoint{3.851149in}{1.212013in}}%
\pgfpathcurveto{\pgfqpoint{3.845325in}{1.206189in}}{\pgfqpoint{3.842052in}{1.198289in}}{\pgfqpoint{3.842052in}{1.190053in}}%
\pgfpathcurveto{\pgfqpoint{3.842052in}{1.181817in}}{\pgfqpoint{3.845325in}{1.173917in}}{\pgfqpoint{3.851149in}{1.168093in}}%
\pgfpathcurveto{\pgfqpoint{3.856972in}{1.162269in}}{\pgfqpoint{3.864873in}{1.158996in}}{\pgfqpoint{3.873109in}{1.158996in}}%
\pgfpathclose%
\pgfusepath{stroke,fill}%
\end{pgfscope}%
\begin{pgfscope}%
\pgfpathrectangle{\pgfqpoint{3.696748in}{0.557870in}}{\pgfqpoint{2.424965in}{1.484734in}}%
\pgfusepath{clip}%
\pgfsetbuttcap%
\pgfsetroundjoin%
\definecolor{currentfill}{rgb}{0.298039,0.447059,0.690196}%
\pgfsetfillcolor{currentfill}%
\pgfsetlinewidth{1.003750pt}%
\definecolor{currentstroke}{rgb}{0.298039,0.447059,0.690196}%
\pgfsetstrokecolor{currentstroke}%
\pgfsetdash{}{0pt}%
\pgfpathmoveto{\pgfqpoint{3.806973in}{1.161200in}}%
\pgfpathcurveto{\pgfqpoint{3.815210in}{1.161200in}}{\pgfqpoint{3.823110in}{1.164472in}}{\pgfqpoint{3.828934in}{1.170296in}}%
\pgfpathcurveto{\pgfqpoint{3.834758in}{1.176120in}}{\pgfqpoint{3.838030in}{1.184020in}}{\pgfqpoint{3.838030in}{1.192257in}}%
\pgfpathcurveto{\pgfqpoint{3.838030in}{1.200493in}}{\pgfqpoint{3.834758in}{1.208393in}}{\pgfqpoint{3.828934in}{1.214217in}}%
\pgfpathcurveto{\pgfqpoint{3.823110in}{1.220041in}}{\pgfqpoint{3.815210in}{1.223313in}}{\pgfqpoint{3.806973in}{1.223313in}}%
\pgfpathcurveto{\pgfqpoint{3.798737in}{1.223313in}}{\pgfqpoint{3.790837in}{1.220041in}}{\pgfqpoint{3.785013in}{1.214217in}}%
\pgfpathcurveto{\pgfqpoint{3.779189in}{1.208393in}}{\pgfqpoint{3.775917in}{1.200493in}}{\pgfqpoint{3.775917in}{1.192257in}}%
\pgfpathcurveto{\pgfqpoint{3.775917in}{1.184020in}}{\pgfqpoint{3.779189in}{1.176120in}}{\pgfqpoint{3.785013in}{1.170296in}}%
\pgfpathcurveto{\pgfqpoint{3.790837in}{1.164472in}}{\pgfqpoint{3.798737in}{1.161200in}}{\pgfqpoint{3.806973in}{1.161200in}}%
\pgfpathclose%
\pgfusepath{stroke,fill}%
\end{pgfscope}%
\begin{pgfscope}%
\pgfsetrectcap%
\pgfsetmiterjoin%
\pgfsetlinewidth{1.254687pt}%
\definecolor{currentstroke}{rgb}{1.000000,1.000000,1.000000}%
\pgfsetstrokecolor{currentstroke}%
\pgfsetdash{}{0pt}%
\pgfpathmoveto{\pgfqpoint{3.696748in}{0.557870in}}%
\pgfpathlineto{\pgfqpoint{3.696748in}{2.042604in}}%
\pgfusepath{stroke}%
\end{pgfscope}%
\begin{pgfscope}%
\pgfsetrectcap%
\pgfsetmiterjoin%
\pgfsetlinewidth{1.254687pt}%
\definecolor{currentstroke}{rgb}{1.000000,1.000000,1.000000}%
\pgfsetstrokecolor{currentstroke}%
\pgfsetdash{}{0pt}%
\pgfpathmoveto{\pgfqpoint{6.121713in}{0.557870in}}%
\pgfpathlineto{\pgfqpoint{6.121713in}{2.042604in}}%
\pgfusepath{stroke}%
\end{pgfscope}%
\begin{pgfscope}%
\pgfsetrectcap%
\pgfsetmiterjoin%
\pgfsetlinewidth{1.254687pt}%
\definecolor{currentstroke}{rgb}{1.000000,1.000000,1.000000}%
\pgfsetstrokecolor{currentstroke}%
\pgfsetdash{}{0pt}%
\pgfpathmoveto{\pgfqpoint{3.696748in}{0.557870in}}%
\pgfpathlineto{\pgfqpoint{6.121713in}{0.557870in}}%
\pgfusepath{stroke}%
\end{pgfscope}%
\begin{pgfscope}%
\pgfsetrectcap%
\pgfsetmiterjoin%
\pgfsetlinewidth{1.254687pt}%
\definecolor{currentstroke}{rgb}{1.000000,1.000000,1.000000}%
\pgfsetstrokecolor{currentstroke}%
\pgfsetdash{}{0pt}%
\pgfpathmoveto{\pgfqpoint{3.696748in}{2.042604in}}%
\pgfpathlineto{\pgfqpoint{6.121713in}{2.042604in}}%
\pgfusepath{stroke}%
\end{pgfscope}%
\begin{pgfscope}%
\definecolor{textcolor}{rgb}{0.150000,0.150000,0.150000}%
\pgfsetstrokecolor{textcolor}%
\pgfsetfillcolor{textcolor}%
\pgftext[x=4.909230in,y=2.125938in,,base]{\color{textcolor}\sffamily\fontsize{11.000000}{13.200000}\selectfont (b)}%
\end{pgfscope}%
\end{pgfpicture}%
\makeatother%
\endgroup%

    \caption{(a) Distribution plot of DOR of all NN models evaluated at two cluster centers when trained to predict heart failure.
             (b) Scatter plot of the same models sensitivity-, and specificity-scores.}
    \label{fig:dl_hf_dor_sens_spec_dist}
\end{figure}

\begin{table}
    \centering
    \ra{1.3}
    \begin{tabular}{lrrrr}
        \toprule
        Dataset-Model         &  Accuracy &  Sensitivity &  Specificity &  DOR \\
        \midrule
        gls/4CH/upsampled     &      0.54 &         0.46 &         0.61 & 1.36 \\
        rls/APLAX/regular     &      0.53 &         0.48 &         0.58 & 1.30 \\
        rls/4CH/regular       &      0.52 &         0.36 &         0.68 & 1.20 \\
        gls/APLAX/downsampled &      0.52 &         0.63 &         0.40 & 1.15 \\
        gls/2CH/downsampled   &      0.51 &         0.61 &         0.40 & 1.03 \\
        \bottomrule
    \end{tabular}
    \caption{The accuracy, DOR, sensitivity and specicity scores of the five best performing variations of the NN in terms of DOR, at detecting heart failure.
             The \textbf{Dataset-Method} column indicates \textit{Dataset used}$/$\textit{View used}$/$\textit{Whether curve has been upsampled, downsampled or is regular}.}
    \label{tab:dl_hf_dor_sens_spec_dist}
\end{table}

\newpage

\subsection{Peak-value Classifiers}

\begin{figure}[htb]
    \centering
    % \includegraphics[width=\textwidth]{results/pvmlc_hf_dor_sens_spec_dist.png}
    %% Creator: Matplotlib, PGF backend
%%
%% To include the figure in your LaTeX document, write
%%   \input{<filename>.pgf}
%%
%% Make sure the required packages are loaded in your preamble
%%   \usepackage{pgf}
%%
%% Figures using additional raster images can only be included by \input if
%% they are in the same directory as the main LaTeX file. For loading figures
%% from other directories you can use the `import` package
%%   \usepackage{import}
%% and then include the figures with
%%   \import{<path to file>}{<filename>.pgf}
%%
%% Matplotlib used the following preamble
%%
\begingroup%
\makeatletter%
\begin{pgfpicture}%
\pgfpathrectangle{\pgfpointorigin}{\pgfqpoint{6.362271in}{2.340000in}}%
\pgfusepath{use as bounding box, clip}%
\begin{pgfscope}%
\pgfsetbuttcap%
\pgfsetmiterjoin%
\definecolor{currentfill}{rgb}{1.000000,1.000000,1.000000}%
\pgfsetfillcolor{currentfill}%
\pgfsetlinewidth{0.000000pt}%
\definecolor{currentstroke}{rgb}{1.000000,1.000000,1.000000}%
\pgfsetstrokecolor{currentstroke}%
\pgfsetdash{}{0pt}%
\pgfpathmoveto{\pgfqpoint{0.000000in}{-0.000000in}}%
\pgfpathlineto{\pgfqpoint{6.362271in}{-0.000000in}}%
\pgfpathlineto{\pgfqpoint{6.362271in}{2.340000in}}%
\pgfpathlineto{\pgfqpoint{0.000000in}{2.340000in}}%
\pgfpathclose%
\pgfusepath{fill}%
\end{pgfscope}%
\begin{pgfscope}%
\pgfsetbuttcap%
\pgfsetmiterjoin%
\definecolor{currentfill}{rgb}{0.917647,0.917647,0.949020}%
\pgfsetfillcolor{currentfill}%
\pgfsetlinewidth{0.000000pt}%
\definecolor{currentstroke}{rgb}{0.000000,0.000000,0.000000}%
\pgfsetstrokecolor{currentstroke}%
\pgfsetstrokeopacity{0.000000}%
\pgfsetdash{}{0pt}%
\pgfpathmoveto{\pgfqpoint{0.574769in}{0.557870in}}%
\pgfpathlineto{\pgfqpoint{3.058877in}{0.557870in}}%
\pgfpathlineto{\pgfqpoint{3.058877in}{2.042604in}}%
\pgfpathlineto{\pgfqpoint{0.574769in}{2.042604in}}%
\pgfpathclose%
\pgfusepath{fill}%
\end{pgfscope}%
\begin{pgfscope}%
\pgfpathrectangle{\pgfqpoint{0.574769in}{0.557870in}}{\pgfqpoint{2.484109in}{1.484734in}}%
\pgfusepath{clip}%
\pgfsetroundcap%
\pgfsetroundjoin%
\pgfsetlinewidth{1.003750pt}%
\definecolor{currentstroke}{rgb}{1.000000,1.000000,1.000000}%
\pgfsetstrokecolor{currentstroke}%
\pgfsetdash{}{0pt}%
\pgfpathmoveto{\pgfqpoint{0.706036in}{0.557870in}}%
\pgfpathlineto{\pgfqpoint{0.706036in}{2.042604in}}%
\pgfusepath{stroke}%
\end{pgfscope}%
\begin{pgfscope}%
\definecolor{textcolor}{rgb}{0.150000,0.150000,0.150000}%
\pgfsetstrokecolor{textcolor}%
\pgfsetfillcolor{textcolor}%
\pgftext[x=0.706036in,y=0.425926in,,top]{\color{textcolor}\sffamily\fontsize{11.000000}{13.200000}\selectfont \(\displaystyle 2\)}%
\end{pgfscope}%
\begin{pgfscope}%
\pgfpathrectangle{\pgfqpoint{0.574769in}{0.557870in}}{\pgfqpoint{2.484109in}{1.484734in}}%
\pgfusepath{clip}%
\pgfsetroundcap%
\pgfsetroundjoin%
\pgfsetlinewidth{1.003750pt}%
\definecolor{currentstroke}{rgb}{1.000000,1.000000,1.000000}%
\pgfsetstrokecolor{currentstroke}%
\pgfsetdash{}{0pt}%
\pgfpathmoveto{\pgfqpoint{1.311681in}{0.557870in}}%
\pgfpathlineto{\pgfqpoint{1.311681in}{2.042604in}}%
\pgfusepath{stroke}%
\end{pgfscope}%
\begin{pgfscope}%
\definecolor{textcolor}{rgb}{0.150000,0.150000,0.150000}%
\pgfsetstrokecolor{textcolor}%
\pgfsetfillcolor{textcolor}%
\pgftext[x=1.311681in,y=0.425926in,,top]{\color{textcolor}\sffamily\fontsize{11.000000}{13.200000}\selectfont \(\displaystyle 4\)}%
\end{pgfscope}%
\begin{pgfscope}%
\pgfpathrectangle{\pgfqpoint{0.574769in}{0.557870in}}{\pgfqpoint{2.484109in}{1.484734in}}%
\pgfusepath{clip}%
\pgfsetroundcap%
\pgfsetroundjoin%
\pgfsetlinewidth{1.003750pt}%
\definecolor{currentstroke}{rgb}{1.000000,1.000000,1.000000}%
\pgfsetstrokecolor{currentstroke}%
\pgfsetdash{}{0pt}%
\pgfpathmoveto{\pgfqpoint{1.917327in}{0.557870in}}%
\pgfpathlineto{\pgfqpoint{1.917327in}{2.042604in}}%
\pgfusepath{stroke}%
\end{pgfscope}%
\begin{pgfscope}%
\definecolor{textcolor}{rgb}{0.150000,0.150000,0.150000}%
\pgfsetstrokecolor{textcolor}%
\pgfsetfillcolor{textcolor}%
\pgftext[x=1.917327in,y=0.425926in,,top]{\color{textcolor}\sffamily\fontsize{11.000000}{13.200000}\selectfont \(\displaystyle 6\)}%
\end{pgfscope}%
\begin{pgfscope}%
\pgfpathrectangle{\pgfqpoint{0.574769in}{0.557870in}}{\pgfqpoint{2.484109in}{1.484734in}}%
\pgfusepath{clip}%
\pgfsetroundcap%
\pgfsetroundjoin%
\pgfsetlinewidth{1.003750pt}%
\definecolor{currentstroke}{rgb}{1.000000,1.000000,1.000000}%
\pgfsetstrokecolor{currentstroke}%
\pgfsetdash{}{0pt}%
\pgfpathmoveto{\pgfqpoint{2.522973in}{0.557870in}}%
\pgfpathlineto{\pgfqpoint{2.522973in}{2.042604in}}%
\pgfusepath{stroke}%
\end{pgfscope}%
\begin{pgfscope}%
\definecolor{textcolor}{rgb}{0.150000,0.150000,0.150000}%
\pgfsetstrokecolor{textcolor}%
\pgfsetfillcolor{textcolor}%
\pgftext[x=2.522973in,y=0.425926in,,top]{\color{textcolor}\sffamily\fontsize{11.000000}{13.200000}\selectfont \(\displaystyle 8\)}%
\end{pgfscope}%
\begin{pgfscope}%
\definecolor{textcolor}{rgb}{0.150000,0.150000,0.150000}%
\pgfsetstrokecolor{textcolor}%
\pgfsetfillcolor{textcolor}%
\pgftext[x=1.816823in,y=0.235185in,,top]{\color{textcolor}\sffamily\fontsize{11.000000}{13.200000}\selectfont DOR}%
\end{pgfscope}%
\begin{pgfscope}%
\pgfpathrectangle{\pgfqpoint{0.574769in}{0.557870in}}{\pgfqpoint{2.484109in}{1.484734in}}%
\pgfusepath{clip}%
\pgfsetroundcap%
\pgfsetroundjoin%
\pgfsetlinewidth{1.003750pt}%
\definecolor{currentstroke}{rgb}{1.000000,1.000000,1.000000}%
\pgfsetstrokecolor{currentstroke}%
\pgfsetdash{}{0pt}%
\pgfpathmoveto{\pgfqpoint{0.574769in}{0.557870in}}%
\pgfpathlineto{\pgfqpoint{3.058877in}{0.557870in}}%
\pgfusepath{stroke}%
\end{pgfscope}%
\begin{pgfscope}%
\definecolor{textcolor}{rgb}{0.150000,0.150000,0.150000}%
\pgfsetstrokecolor{textcolor}%
\pgfsetfillcolor{textcolor}%
\pgftext[x=0.366783in,y=0.505064in,left,base]{\color{textcolor}\sffamily\fontsize{11.000000}{13.200000}\selectfont \(\displaystyle 0\)}%
\end{pgfscope}%
\begin{pgfscope}%
\pgfpathrectangle{\pgfqpoint{0.574769in}{0.557870in}}{\pgfqpoint{2.484109in}{1.484734in}}%
\pgfusepath{clip}%
\pgfsetroundcap%
\pgfsetroundjoin%
\pgfsetlinewidth{1.003750pt}%
\definecolor{currentstroke}{rgb}{1.000000,1.000000,1.000000}%
\pgfsetstrokecolor{currentstroke}%
\pgfsetdash{}{0pt}%
\pgfpathmoveto{\pgfqpoint{0.574769in}{1.264886in}}%
\pgfpathlineto{\pgfqpoint{3.058877in}{1.264886in}}%
\pgfusepath{stroke}%
\end{pgfscope}%
\begin{pgfscope}%
\definecolor{textcolor}{rgb}{0.150000,0.150000,0.150000}%
\pgfsetstrokecolor{textcolor}%
\pgfsetfillcolor{textcolor}%
\pgftext[x=0.366783in,y=1.212080in,left,base]{\color{textcolor}\sffamily\fontsize{11.000000}{13.200000}\selectfont \(\displaystyle 5\)}%
\end{pgfscope}%
\begin{pgfscope}%
\pgfpathrectangle{\pgfqpoint{0.574769in}{0.557870in}}{\pgfqpoint{2.484109in}{1.484734in}}%
\pgfusepath{clip}%
\pgfsetroundcap%
\pgfsetroundjoin%
\pgfsetlinewidth{1.003750pt}%
\definecolor{currentstroke}{rgb}{1.000000,1.000000,1.000000}%
\pgfsetstrokecolor{currentstroke}%
\pgfsetdash{}{0pt}%
\pgfpathmoveto{\pgfqpoint{0.574769in}{1.971903in}}%
\pgfpathlineto{\pgfqpoint{3.058877in}{1.971903in}}%
\pgfusepath{stroke}%
\end{pgfscope}%
\begin{pgfscope}%
\definecolor{textcolor}{rgb}{0.150000,0.150000,0.150000}%
\pgfsetstrokecolor{textcolor}%
\pgfsetfillcolor{textcolor}%
\pgftext[x=0.290741in,y=1.919096in,left,base]{\color{textcolor}\sffamily\fontsize{11.000000}{13.200000}\selectfont \(\displaystyle 10\)}%
\end{pgfscope}%
\begin{pgfscope}%
\definecolor{textcolor}{rgb}{0.150000,0.150000,0.150000}%
\pgfsetstrokecolor{textcolor}%
\pgfsetfillcolor{textcolor}%
\pgftext[x=0.235185in,y=1.300237in,,bottom,rotate=90.000000]{\color{textcolor}\sffamily\fontsize{11.000000}{13.200000}\selectfont Occurance}%
\end{pgfscope}%
\begin{pgfscope}%
\pgfpathrectangle{\pgfqpoint{0.574769in}{0.557870in}}{\pgfqpoint{2.484109in}{1.484734in}}%
\pgfusepath{clip}%
\pgfsetbuttcap%
\pgfsetmiterjoin%
\definecolor{currentfill}{rgb}{0.298039,0.447059,0.690196}%
\pgfsetfillcolor{currentfill}%
\pgfsetfillopacity{0.400000}%
\pgfsetlinewidth{1.003750pt}%
\definecolor{currentstroke}{rgb}{1.000000,1.000000,1.000000}%
\pgfsetstrokecolor{currentstroke}%
\pgfsetstrokeopacity{0.400000}%
\pgfsetdash{}{0pt}%
\pgfpathmoveto{\pgfqpoint{0.687683in}{0.557870in}}%
\pgfpathlineto{\pgfqpoint{0.913511in}{0.557870in}}%
\pgfpathlineto{\pgfqpoint{0.913511in}{1.264886in}}%
\pgfpathlineto{\pgfqpoint{0.687683in}{1.264886in}}%
\pgfpathclose%
\pgfusepath{stroke,fill}%
\end{pgfscope}%
\begin{pgfscope}%
\pgfpathrectangle{\pgfqpoint{0.574769in}{0.557870in}}{\pgfqpoint{2.484109in}{1.484734in}}%
\pgfusepath{clip}%
\pgfsetbuttcap%
\pgfsetmiterjoin%
\definecolor{currentfill}{rgb}{0.298039,0.447059,0.690196}%
\pgfsetfillcolor{currentfill}%
\pgfsetfillopacity{0.400000}%
\pgfsetlinewidth{1.003750pt}%
\definecolor{currentstroke}{rgb}{1.000000,1.000000,1.000000}%
\pgfsetstrokecolor{currentstroke}%
\pgfsetstrokeopacity{0.400000}%
\pgfsetdash{}{0pt}%
\pgfpathmoveto{\pgfqpoint{0.913511in}{0.557870in}}%
\pgfpathlineto{\pgfqpoint{1.139339in}{0.557870in}}%
\pgfpathlineto{\pgfqpoint{1.139339in}{1.406290in}}%
\pgfpathlineto{\pgfqpoint{0.913511in}{1.406290in}}%
\pgfpathclose%
\pgfusepath{stroke,fill}%
\end{pgfscope}%
\begin{pgfscope}%
\pgfpathrectangle{\pgfqpoint{0.574769in}{0.557870in}}{\pgfqpoint{2.484109in}{1.484734in}}%
\pgfusepath{clip}%
\pgfsetbuttcap%
\pgfsetmiterjoin%
\definecolor{currentfill}{rgb}{0.298039,0.447059,0.690196}%
\pgfsetfillcolor{currentfill}%
\pgfsetfillopacity{0.400000}%
\pgfsetlinewidth{1.003750pt}%
\definecolor{currentstroke}{rgb}{1.000000,1.000000,1.000000}%
\pgfsetstrokecolor{currentstroke}%
\pgfsetstrokeopacity{0.400000}%
\pgfsetdash{}{0pt}%
\pgfpathmoveto{\pgfqpoint{1.139339in}{0.557870in}}%
\pgfpathlineto{\pgfqpoint{1.365167in}{0.557870in}}%
\pgfpathlineto{\pgfqpoint{1.365167in}{0.699274in}}%
\pgfpathlineto{\pgfqpoint{1.139339in}{0.699274in}}%
\pgfpathclose%
\pgfusepath{stroke,fill}%
\end{pgfscope}%
\begin{pgfscope}%
\pgfpathrectangle{\pgfqpoint{0.574769in}{0.557870in}}{\pgfqpoint{2.484109in}{1.484734in}}%
\pgfusepath{clip}%
\pgfsetbuttcap%
\pgfsetmiterjoin%
\definecolor{currentfill}{rgb}{0.298039,0.447059,0.690196}%
\pgfsetfillcolor{currentfill}%
\pgfsetfillopacity{0.400000}%
\pgfsetlinewidth{1.003750pt}%
\definecolor{currentstroke}{rgb}{1.000000,1.000000,1.000000}%
\pgfsetstrokecolor{currentstroke}%
\pgfsetstrokeopacity{0.400000}%
\pgfsetdash{}{0pt}%
\pgfpathmoveto{\pgfqpoint{1.365167in}{0.557870in}}%
\pgfpathlineto{\pgfqpoint{1.590995in}{0.557870in}}%
\pgfpathlineto{\pgfqpoint{1.590995in}{1.830499in}}%
\pgfpathlineto{\pgfqpoint{1.365167in}{1.830499in}}%
\pgfpathclose%
\pgfusepath{stroke,fill}%
\end{pgfscope}%
\begin{pgfscope}%
\pgfpathrectangle{\pgfqpoint{0.574769in}{0.557870in}}{\pgfqpoint{2.484109in}{1.484734in}}%
\pgfusepath{clip}%
\pgfsetbuttcap%
\pgfsetmiterjoin%
\definecolor{currentfill}{rgb}{0.298039,0.447059,0.690196}%
\pgfsetfillcolor{currentfill}%
\pgfsetfillopacity{0.400000}%
\pgfsetlinewidth{1.003750pt}%
\definecolor{currentstroke}{rgb}{1.000000,1.000000,1.000000}%
\pgfsetstrokecolor{currentstroke}%
\pgfsetstrokeopacity{0.400000}%
\pgfsetdash{}{0pt}%
\pgfpathmoveto{\pgfqpoint{1.590995in}{0.557870in}}%
\pgfpathlineto{\pgfqpoint{1.816823in}{0.557870in}}%
\pgfpathlineto{\pgfqpoint{1.816823in}{1.689096in}}%
\pgfpathlineto{\pgfqpoint{1.590995in}{1.689096in}}%
\pgfpathclose%
\pgfusepath{stroke,fill}%
\end{pgfscope}%
\begin{pgfscope}%
\pgfpathrectangle{\pgfqpoint{0.574769in}{0.557870in}}{\pgfqpoint{2.484109in}{1.484734in}}%
\pgfusepath{clip}%
\pgfsetbuttcap%
\pgfsetmiterjoin%
\definecolor{currentfill}{rgb}{0.298039,0.447059,0.690196}%
\pgfsetfillcolor{currentfill}%
\pgfsetfillopacity{0.400000}%
\pgfsetlinewidth{1.003750pt}%
\definecolor{currentstroke}{rgb}{1.000000,1.000000,1.000000}%
\pgfsetstrokecolor{currentstroke}%
\pgfsetstrokeopacity{0.400000}%
\pgfsetdash{}{0pt}%
\pgfpathmoveto{\pgfqpoint{1.816823in}{0.557870in}}%
\pgfpathlineto{\pgfqpoint{2.042651in}{0.557870in}}%
\pgfpathlineto{\pgfqpoint{2.042651in}{1.689096in}}%
\pgfpathlineto{\pgfqpoint{1.816823in}{1.689096in}}%
\pgfpathclose%
\pgfusepath{stroke,fill}%
\end{pgfscope}%
\begin{pgfscope}%
\pgfpathrectangle{\pgfqpoint{0.574769in}{0.557870in}}{\pgfqpoint{2.484109in}{1.484734in}}%
\pgfusepath{clip}%
\pgfsetbuttcap%
\pgfsetmiterjoin%
\definecolor{currentfill}{rgb}{0.298039,0.447059,0.690196}%
\pgfsetfillcolor{currentfill}%
\pgfsetfillopacity{0.400000}%
\pgfsetlinewidth{1.003750pt}%
\definecolor{currentstroke}{rgb}{1.000000,1.000000,1.000000}%
\pgfsetstrokecolor{currentstroke}%
\pgfsetstrokeopacity{0.400000}%
\pgfsetdash{}{0pt}%
\pgfpathmoveto{\pgfqpoint{2.042651in}{0.557870in}}%
\pgfpathlineto{\pgfqpoint{2.268479in}{0.557870in}}%
\pgfpathlineto{\pgfqpoint{2.268479in}{1.971903in}}%
\pgfpathlineto{\pgfqpoint{2.042651in}{1.971903in}}%
\pgfpathclose%
\pgfusepath{stroke,fill}%
\end{pgfscope}%
\begin{pgfscope}%
\pgfpathrectangle{\pgfqpoint{0.574769in}{0.557870in}}{\pgfqpoint{2.484109in}{1.484734in}}%
\pgfusepath{clip}%
\pgfsetbuttcap%
\pgfsetmiterjoin%
\definecolor{currentfill}{rgb}{0.298039,0.447059,0.690196}%
\pgfsetfillcolor{currentfill}%
\pgfsetfillopacity{0.400000}%
\pgfsetlinewidth{1.003750pt}%
\definecolor{currentstroke}{rgb}{1.000000,1.000000,1.000000}%
\pgfsetstrokecolor{currentstroke}%
\pgfsetstrokeopacity{0.400000}%
\pgfsetdash{}{0pt}%
\pgfpathmoveto{\pgfqpoint{2.268479in}{0.557870in}}%
\pgfpathlineto{\pgfqpoint{2.494307in}{0.557870in}}%
\pgfpathlineto{\pgfqpoint{2.494307in}{1.689096in}}%
\pgfpathlineto{\pgfqpoint{2.268479in}{1.689096in}}%
\pgfpathclose%
\pgfusepath{stroke,fill}%
\end{pgfscope}%
\begin{pgfscope}%
\pgfpathrectangle{\pgfqpoint{0.574769in}{0.557870in}}{\pgfqpoint{2.484109in}{1.484734in}}%
\pgfusepath{clip}%
\pgfsetbuttcap%
\pgfsetmiterjoin%
\definecolor{currentfill}{rgb}{0.298039,0.447059,0.690196}%
\pgfsetfillcolor{currentfill}%
\pgfsetfillopacity{0.400000}%
\pgfsetlinewidth{1.003750pt}%
\definecolor{currentstroke}{rgb}{1.000000,1.000000,1.000000}%
\pgfsetstrokecolor{currentstroke}%
\pgfsetstrokeopacity{0.400000}%
\pgfsetdash{}{0pt}%
\pgfpathmoveto{\pgfqpoint{2.494307in}{0.557870in}}%
\pgfpathlineto{\pgfqpoint{2.720135in}{0.557870in}}%
\pgfpathlineto{\pgfqpoint{2.720135in}{0.699274in}}%
\pgfpathlineto{\pgfqpoint{2.494307in}{0.699274in}}%
\pgfpathclose%
\pgfusepath{stroke,fill}%
\end{pgfscope}%
\begin{pgfscope}%
\pgfpathrectangle{\pgfqpoint{0.574769in}{0.557870in}}{\pgfqpoint{2.484109in}{1.484734in}}%
\pgfusepath{clip}%
\pgfsetbuttcap%
\pgfsetmiterjoin%
\definecolor{currentfill}{rgb}{0.298039,0.447059,0.690196}%
\pgfsetfillcolor{currentfill}%
\pgfsetfillopacity{0.400000}%
\pgfsetlinewidth{1.003750pt}%
\definecolor{currentstroke}{rgb}{1.000000,1.000000,1.000000}%
\pgfsetstrokecolor{currentstroke}%
\pgfsetstrokeopacity{0.400000}%
\pgfsetdash{}{0pt}%
\pgfpathmoveto{\pgfqpoint{2.720135in}{0.557870in}}%
\pgfpathlineto{\pgfqpoint{2.945963in}{0.557870in}}%
\pgfpathlineto{\pgfqpoint{2.945963in}{1.406290in}}%
\pgfpathlineto{\pgfqpoint{2.720135in}{1.406290in}}%
\pgfpathclose%
\pgfusepath{stroke,fill}%
\end{pgfscope}%
\begin{pgfscope}%
\pgfsetrectcap%
\pgfsetmiterjoin%
\pgfsetlinewidth{1.254687pt}%
\definecolor{currentstroke}{rgb}{1.000000,1.000000,1.000000}%
\pgfsetstrokecolor{currentstroke}%
\pgfsetdash{}{0pt}%
\pgfpathmoveto{\pgfqpoint{0.574769in}{0.557870in}}%
\pgfpathlineto{\pgfqpoint{0.574769in}{2.042604in}}%
\pgfusepath{stroke}%
\end{pgfscope}%
\begin{pgfscope}%
\pgfsetrectcap%
\pgfsetmiterjoin%
\pgfsetlinewidth{1.254687pt}%
\definecolor{currentstroke}{rgb}{1.000000,1.000000,1.000000}%
\pgfsetstrokecolor{currentstroke}%
\pgfsetdash{}{0pt}%
\pgfpathmoveto{\pgfqpoint{3.058877in}{0.557870in}}%
\pgfpathlineto{\pgfqpoint{3.058877in}{2.042604in}}%
\pgfusepath{stroke}%
\end{pgfscope}%
\begin{pgfscope}%
\pgfsetrectcap%
\pgfsetmiterjoin%
\pgfsetlinewidth{1.254687pt}%
\definecolor{currentstroke}{rgb}{1.000000,1.000000,1.000000}%
\pgfsetstrokecolor{currentstroke}%
\pgfsetdash{}{0pt}%
\pgfpathmoveto{\pgfqpoint{0.574769in}{0.557870in}}%
\pgfpathlineto{\pgfqpoint{3.058877in}{0.557870in}}%
\pgfusepath{stroke}%
\end{pgfscope}%
\begin{pgfscope}%
\pgfsetrectcap%
\pgfsetmiterjoin%
\pgfsetlinewidth{1.254687pt}%
\definecolor{currentstroke}{rgb}{1.000000,1.000000,1.000000}%
\pgfsetstrokecolor{currentstroke}%
\pgfsetdash{}{0pt}%
\pgfpathmoveto{\pgfqpoint{0.574769in}{2.042604in}}%
\pgfpathlineto{\pgfqpoint{3.058877in}{2.042604in}}%
\pgfusepath{stroke}%
\end{pgfscope}%
\begin{pgfscope}%
\definecolor{textcolor}{rgb}{0.150000,0.150000,0.150000}%
\pgfsetstrokecolor{textcolor}%
\pgfsetfillcolor{textcolor}%
\pgftext[x=1.816823in,y=2.125938in,,base]{\color{textcolor}\sffamily\fontsize{11.000000}{13.200000}\selectfont (a)}%
\end{pgfscope}%
\begin{pgfscope}%
\pgfsetbuttcap%
\pgfsetmiterjoin%
\definecolor{currentfill}{rgb}{0.917647,0.917647,0.949020}%
\pgfsetfillcolor{currentfill}%
\pgfsetlinewidth{0.000000pt}%
\definecolor{currentstroke}{rgb}{0.000000,0.000000,0.000000}%
\pgfsetstrokecolor{currentstroke}%
\pgfsetstrokeopacity{0.000000}%
\pgfsetdash{}{0pt}%
\pgfpathmoveto{\pgfqpoint{3.755891in}{0.557870in}}%
\pgfpathlineto{\pgfqpoint{6.240000in}{0.557870in}}%
\pgfpathlineto{\pgfqpoint{6.240000in}{2.042604in}}%
\pgfpathlineto{\pgfqpoint{3.755891in}{2.042604in}}%
\pgfpathclose%
\pgfusepath{fill}%
\end{pgfscope}%
\begin{pgfscope}%
\pgfpathrectangle{\pgfqpoint{3.755891in}{0.557870in}}{\pgfqpoint{2.484109in}{1.484734in}}%
\pgfusepath{clip}%
\pgfsetroundcap%
\pgfsetroundjoin%
\pgfsetlinewidth{1.003750pt}%
\definecolor{currentstroke}{rgb}{1.000000,1.000000,1.000000}%
\pgfsetstrokecolor{currentstroke}%
\pgfsetdash{}{0pt}%
\pgfpathmoveto{\pgfqpoint{3.868805in}{0.557870in}}%
\pgfpathlineto{\pgfqpoint{3.868805in}{2.042604in}}%
\pgfusepath{stroke}%
\end{pgfscope}%
\begin{pgfscope}%
\definecolor{textcolor}{rgb}{0.150000,0.150000,0.150000}%
\pgfsetstrokecolor{textcolor}%
\pgfsetfillcolor{textcolor}%
\pgftext[x=3.868805in,y=0.425926in,,top]{\color{textcolor}\sffamily\fontsize{11.000000}{13.200000}\selectfont \(\displaystyle 0.00\)}%
\end{pgfscope}%
\begin{pgfscope}%
\pgfpathrectangle{\pgfqpoint{3.755891in}{0.557870in}}{\pgfqpoint{2.484109in}{1.484734in}}%
\pgfusepath{clip}%
\pgfsetroundcap%
\pgfsetroundjoin%
\pgfsetlinewidth{1.003750pt}%
\definecolor{currentstroke}{rgb}{1.000000,1.000000,1.000000}%
\pgfsetstrokecolor{currentstroke}%
\pgfsetdash{}{0pt}%
\pgfpathmoveto{\pgfqpoint{4.433376in}{0.557870in}}%
\pgfpathlineto{\pgfqpoint{4.433376in}{2.042604in}}%
\pgfusepath{stroke}%
\end{pgfscope}%
\begin{pgfscope}%
\definecolor{textcolor}{rgb}{0.150000,0.150000,0.150000}%
\pgfsetstrokecolor{textcolor}%
\pgfsetfillcolor{textcolor}%
\pgftext[x=4.433376in,y=0.425926in,,top]{\color{textcolor}\sffamily\fontsize{11.000000}{13.200000}\selectfont \(\displaystyle 0.25\)}%
\end{pgfscope}%
\begin{pgfscope}%
\pgfpathrectangle{\pgfqpoint{3.755891in}{0.557870in}}{\pgfqpoint{2.484109in}{1.484734in}}%
\pgfusepath{clip}%
\pgfsetroundcap%
\pgfsetroundjoin%
\pgfsetlinewidth{1.003750pt}%
\definecolor{currentstroke}{rgb}{1.000000,1.000000,1.000000}%
\pgfsetstrokecolor{currentstroke}%
\pgfsetdash{}{0pt}%
\pgfpathmoveto{\pgfqpoint{4.997946in}{0.557870in}}%
\pgfpathlineto{\pgfqpoint{4.997946in}{2.042604in}}%
\pgfusepath{stroke}%
\end{pgfscope}%
\begin{pgfscope}%
\definecolor{textcolor}{rgb}{0.150000,0.150000,0.150000}%
\pgfsetstrokecolor{textcolor}%
\pgfsetfillcolor{textcolor}%
\pgftext[x=4.997946in,y=0.425926in,,top]{\color{textcolor}\sffamily\fontsize{11.000000}{13.200000}\selectfont \(\displaystyle 0.50\)}%
\end{pgfscope}%
\begin{pgfscope}%
\pgfpathrectangle{\pgfqpoint{3.755891in}{0.557870in}}{\pgfqpoint{2.484109in}{1.484734in}}%
\pgfusepath{clip}%
\pgfsetroundcap%
\pgfsetroundjoin%
\pgfsetlinewidth{1.003750pt}%
\definecolor{currentstroke}{rgb}{1.000000,1.000000,1.000000}%
\pgfsetstrokecolor{currentstroke}%
\pgfsetdash{}{0pt}%
\pgfpathmoveto{\pgfqpoint{5.562516in}{0.557870in}}%
\pgfpathlineto{\pgfqpoint{5.562516in}{2.042604in}}%
\pgfusepath{stroke}%
\end{pgfscope}%
\begin{pgfscope}%
\definecolor{textcolor}{rgb}{0.150000,0.150000,0.150000}%
\pgfsetstrokecolor{textcolor}%
\pgfsetfillcolor{textcolor}%
\pgftext[x=5.562516in,y=0.425926in,,top]{\color{textcolor}\sffamily\fontsize{11.000000}{13.200000}\selectfont \(\displaystyle 0.75\)}%
\end{pgfscope}%
\begin{pgfscope}%
\pgfpathrectangle{\pgfqpoint{3.755891in}{0.557870in}}{\pgfqpoint{2.484109in}{1.484734in}}%
\pgfusepath{clip}%
\pgfsetroundcap%
\pgfsetroundjoin%
\pgfsetlinewidth{1.003750pt}%
\definecolor{currentstroke}{rgb}{1.000000,1.000000,1.000000}%
\pgfsetstrokecolor{currentstroke}%
\pgfsetdash{}{0pt}%
\pgfpathmoveto{\pgfqpoint{6.127086in}{0.557870in}}%
\pgfpathlineto{\pgfqpoint{6.127086in}{2.042604in}}%
\pgfusepath{stroke}%
\end{pgfscope}%
\begin{pgfscope}%
\definecolor{textcolor}{rgb}{0.150000,0.150000,0.150000}%
\pgfsetstrokecolor{textcolor}%
\pgfsetfillcolor{textcolor}%
\pgftext[x=6.127086in,y=0.425926in,,top]{\color{textcolor}\sffamily\fontsize{11.000000}{13.200000}\selectfont \(\displaystyle 1.00\)}%
\end{pgfscope}%
\begin{pgfscope}%
\definecolor{textcolor}{rgb}{0.150000,0.150000,0.150000}%
\pgfsetstrokecolor{textcolor}%
\pgfsetfillcolor{textcolor}%
\pgftext[x=4.997946in,y=0.235185in,,top]{\color{textcolor}\sffamily\fontsize{11.000000}{13.200000}\selectfont Specificity}%
\end{pgfscope}%
\begin{pgfscope}%
\pgfpathrectangle{\pgfqpoint{3.755891in}{0.557870in}}{\pgfqpoint{2.484109in}{1.484734in}}%
\pgfusepath{clip}%
\pgfsetroundcap%
\pgfsetroundjoin%
\pgfsetlinewidth{1.003750pt}%
\definecolor{currentstroke}{rgb}{1.000000,1.000000,1.000000}%
\pgfsetstrokecolor{currentstroke}%
\pgfsetdash{}{0pt}%
\pgfpathmoveto{\pgfqpoint{3.755891in}{0.625358in}}%
\pgfpathlineto{\pgfqpoint{6.240000in}{0.625358in}}%
\pgfusepath{stroke}%
\end{pgfscope}%
\begin{pgfscope}%
\definecolor{textcolor}{rgb}{0.150000,0.150000,0.150000}%
\pgfsetstrokecolor{textcolor}%
\pgfsetfillcolor{textcolor}%
\pgftext[x=3.429618in,y=0.572552in,left,base]{\color{textcolor}\sffamily\fontsize{11.000000}{13.200000}\selectfont \(\displaystyle 0.0\)}%
\end{pgfscope}%
\begin{pgfscope}%
\pgfpathrectangle{\pgfqpoint{3.755891in}{0.557870in}}{\pgfqpoint{2.484109in}{1.484734in}}%
\pgfusepath{clip}%
\pgfsetroundcap%
\pgfsetroundjoin%
\pgfsetlinewidth{1.003750pt}%
\definecolor{currentstroke}{rgb}{1.000000,1.000000,1.000000}%
\pgfsetstrokecolor{currentstroke}%
\pgfsetdash{}{0pt}%
\pgfpathmoveto{\pgfqpoint{3.755891in}{1.300237in}}%
\pgfpathlineto{\pgfqpoint{6.240000in}{1.300237in}}%
\pgfusepath{stroke}%
\end{pgfscope}%
\begin{pgfscope}%
\definecolor{textcolor}{rgb}{0.150000,0.150000,0.150000}%
\pgfsetstrokecolor{textcolor}%
\pgfsetfillcolor{textcolor}%
\pgftext[x=3.429618in,y=1.247431in,left,base]{\color{textcolor}\sffamily\fontsize{11.000000}{13.200000}\selectfont \(\displaystyle 0.5\)}%
\end{pgfscope}%
\begin{pgfscope}%
\pgfpathrectangle{\pgfqpoint{3.755891in}{0.557870in}}{\pgfqpoint{2.484109in}{1.484734in}}%
\pgfusepath{clip}%
\pgfsetroundcap%
\pgfsetroundjoin%
\pgfsetlinewidth{1.003750pt}%
\definecolor{currentstroke}{rgb}{1.000000,1.000000,1.000000}%
\pgfsetstrokecolor{currentstroke}%
\pgfsetdash{}{0pt}%
\pgfpathmoveto{\pgfqpoint{3.755891in}{1.975116in}}%
\pgfpathlineto{\pgfqpoint{6.240000in}{1.975116in}}%
\pgfusepath{stroke}%
\end{pgfscope}%
\begin{pgfscope}%
\definecolor{textcolor}{rgb}{0.150000,0.150000,0.150000}%
\pgfsetstrokecolor{textcolor}%
\pgfsetfillcolor{textcolor}%
\pgftext[x=3.429618in,y=1.922310in,left,base]{\color{textcolor}\sffamily\fontsize{11.000000}{13.200000}\selectfont \(\displaystyle 1.0\)}%
\end{pgfscope}%
\begin{pgfscope}%
\definecolor{textcolor}{rgb}{0.150000,0.150000,0.150000}%
\pgfsetstrokecolor{textcolor}%
\pgfsetfillcolor{textcolor}%
\pgftext[x=3.374062in,y=1.300237in,,bottom,rotate=90.000000]{\color{textcolor}\sffamily\fontsize{11.000000}{13.200000}\selectfont Sensitivity}%
\end{pgfscope}%
\begin{pgfscope}%
\pgfpathrectangle{\pgfqpoint{3.755891in}{0.557870in}}{\pgfqpoint{2.484109in}{1.484734in}}%
\pgfusepath{clip}%
\pgfsetbuttcap%
\pgfsetroundjoin%
\definecolor{currentfill}{rgb}{0.298039,0.447059,0.690196}%
\pgfsetfillcolor{currentfill}%
\pgfsetlinewidth{1.003750pt}%
\definecolor{currentstroke}{rgb}{0.298039,0.447059,0.690196}%
\pgfsetstrokecolor{currentstroke}%
\pgfsetdash{}{0pt}%
\pgfpathmoveto{\pgfqpoint{5.511191in}{1.162621in}}%
\pgfpathcurveto{\pgfqpoint{5.519428in}{1.162621in}}{\pgfqpoint{5.527328in}{1.165893in}}{\pgfqpoint{5.533152in}{1.171717in}}%
\pgfpathcurveto{\pgfqpoint{5.538975in}{1.177541in}}{\pgfqpoint{5.542248in}{1.185441in}}{\pgfqpoint{5.542248in}{1.193677in}}%
\pgfpathcurveto{\pgfqpoint{5.542248in}{1.201914in}}{\pgfqpoint{5.538975in}{1.209814in}}{\pgfqpoint{5.533152in}{1.215638in}}%
\pgfpathcurveto{\pgfqpoint{5.527328in}{1.221462in}}{\pgfqpoint{5.519428in}{1.224734in}}{\pgfqpoint{5.511191in}{1.224734in}}%
\pgfpathcurveto{\pgfqpoint{5.502955in}{1.224734in}}{\pgfqpoint{5.495055in}{1.221462in}}{\pgfqpoint{5.489231in}{1.215638in}}%
\pgfpathcurveto{\pgfqpoint{5.483407in}{1.209814in}}{\pgfqpoint{5.480135in}{1.201914in}}{\pgfqpoint{5.480135in}{1.193677in}}%
\pgfpathcurveto{\pgfqpoint{5.480135in}{1.185441in}}{\pgfqpoint{5.483407in}{1.177541in}}{\pgfqpoint{5.489231in}{1.171717in}}%
\pgfpathcurveto{\pgfqpoint{5.495055in}{1.165893in}}{\pgfqpoint{5.502955in}{1.162621in}}{\pgfqpoint{5.511191in}{1.162621in}}%
\pgfpathclose%
\pgfusepath{stroke,fill}%
\end{pgfscope}%
\begin{pgfscope}%
\pgfpathrectangle{\pgfqpoint{3.755891in}{0.557870in}}{\pgfqpoint{2.484109in}{1.484734in}}%
\pgfusepath{clip}%
\pgfsetbuttcap%
\pgfsetroundjoin%
\definecolor{currentfill}{rgb}{0.298039,0.447059,0.690196}%
\pgfsetfillcolor{currentfill}%
\pgfsetlinewidth{1.003750pt}%
\definecolor{currentstroke}{rgb}{0.298039,0.447059,0.690196}%
\pgfsetstrokecolor{currentstroke}%
\pgfsetdash{}{0pt}%
\pgfpathmoveto{\pgfqpoint{5.534002in}{1.446780in}}%
\pgfpathcurveto{\pgfqpoint{5.542238in}{1.446780in}}{\pgfqpoint{5.550139in}{1.450053in}}{\pgfqpoint{5.555962in}{1.455877in}}%
\pgfpathcurveto{\pgfqpoint{5.561786in}{1.461701in}}{\pgfqpoint{5.565059in}{1.469601in}}{\pgfqpoint{5.565059in}{1.477837in}}%
\pgfpathcurveto{\pgfqpoint{5.565059in}{1.486073in}}{\pgfqpoint{5.561786in}{1.493973in}}{\pgfqpoint{5.555962in}{1.499797in}}%
\pgfpathcurveto{\pgfqpoint{5.550139in}{1.505621in}}{\pgfqpoint{5.542238in}{1.508893in}}{\pgfqpoint{5.534002in}{1.508893in}}%
\pgfpathcurveto{\pgfqpoint{5.525766in}{1.508893in}}{\pgfqpoint{5.517866in}{1.505621in}}{\pgfqpoint{5.512042in}{1.499797in}}%
\pgfpathcurveto{\pgfqpoint{5.506218in}{1.493973in}}{\pgfqpoint{5.502946in}{1.486073in}}{\pgfqpoint{5.502946in}{1.477837in}}%
\pgfpathcurveto{\pgfqpoint{5.502946in}{1.469601in}}{\pgfqpoint{5.506218in}{1.461701in}}{\pgfqpoint{5.512042in}{1.455877in}}%
\pgfpathcurveto{\pgfqpoint{5.517866in}{1.450053in}}{\pgfqpoint{5.525766in}{1.446780in}}{\pgfqpoint{5.534002in}{1.446780in}}%
\pgfpathclose%
\pgfusepath{stroke,fill}%
\end{pgfscope}%
\begin{pgfscope}%
\pgfpathrectangle{\pgfqpoint{3.755891in}{0.557870in}}{\pgfqpoint{2.484109in}{1.484734in}}%
\pgfusepath{clip}%
\pgfsetbuttcap%
\pgfsetroundjoin%
\definecolor{currentfill}{rgb}{0.298039,0.447059,0.690196}%
\pgfsetfillcolor{currentfill}%
\pgfsetlinewidth{1.003750pt}%
\definecolor{currentstroke}{rgb}{0.298039,0.447059,0.690196}%
\pgfsetstrokecolor{currentstroke}%
\pgfsetdash{}{0pt}%
\pgfpathmoveto{\pgfqpoint{5.419948in}{1.588860in}}%
\pgfpathcurveto{\pgfqpoint{5.428184in}{1.588860in}}{\pgfqpoint{5.436084in}{1.592133in}}{\pgfqpoint{5.441908in}{1.597957in}}%
\pgfpathcurveto{\pgfqpoint{5.447732in}{1.603780in}}{\pgfqpoint{5.451004in}{1.611680in}}{\pgfqpoint{5.451004in}{1.619917in}}%
\pgfpathcurveto{\pgfqpoint{5.451004in}{1.628153in}}{\pgfqpoint{5.447732in}{1.636053in}}{\pgfqpoint{5.441908in}{1.641877in}}%
\pgfpathcurveto{\pgfqpoint{5.436084in}{1.647701in}}{\pgfqpoint{5.428184in}{1.650973in}}{\pgfqpoint{5.419948in}{1.650973in}}%
\pgfpathcurveto{\pgfqpoint{5.411711in}{1.650973in}}{\pgfqpoint{5.403811in}{1.647701in}}{\pgfqpoint{5.397987in}{1.641877in}}%
\pgfpathcurveto{\pgfqpoint{5.392163in}{1.636053in}}{\pgfqpoint{5.388891in}{1.628153in}}{\pgfqpoint{5.388891in}{1.619917in}}%
\pgfpathcurveto{\pgfqpoint{5.388891in}{1.611680in}}{\pgfqpoint{5.392163in}{1.603780in}}{\pgfqpoint{5.397987in}{1.597957in}}%
\pgfpathcurveto{\pgfqpoint{5.403811in}{1.592133in}}{\pgfqpoint{5.411711in}{1.588860in}}{\pgfqpoint{5.419948in}{1.588860in}}%
\pgfpathclose%
\pgfusepath{stroke,fill}%
\end{pgfscope}%
\begin{pgfscope}%
\pgfpathrectangle{\pgfqpoint{3.755891in}{0.557870in}}{\pgfqpoint{2.484109in}{1.484734in}}%
\pgfusepath{clip}%
\pgfsetbuttcap%
\pgfsetroundjoin%
\definecolor{currentfill}{rgb}{0.298039,0.447059,0.690196}%
\pgfsetfillcolor{currentfill}%
\pgfsetlinewidth{1.003750pt}%
\definecolor{currentstroke}{rgb}{0.298039,0.447059,0.690196}%
\pgfsetstrokecolor{currentstroke}%
\pgfsetdash{}{0pt}%
\pgfpathmoveto{\pgfqpoint{5.602435in}{1.205245in}}%
\pgfpathcurveto{\pgfqpoint{5.610671in}{1.205245in}}{\pgfqpoint{5.618571in}{1.208517in}}{\pgfqpoint{5.624395in}{1.214341in}}%
\pgfpathcurveto{\pgfqpoint{5.630219in}{1.220165in}}{\pgfqpoint{5.633491in}{1.228065in}}{\pgfqpoint{5.633491in}{1.236301in}}%
\pgfpathcurveto{\pgfqpoint{5.633491in}{1.244538in}}{\pgfqpoint{5.630219in}{1.252438in}}{\pgfqpoint{5.624395in}{1.258262in}}%
\pgfpathcurveto{\pgfqpoint{5.618571in}{1.264086in}}{\pgfqpoint{5.610671in}{1.267358in}}{\pgfqpoint{5.602435in}{1.267358in}}%
\pgfpathcurveto{\pgfqpoint{5.594199in}{1.267358in}}{\pgfqpoint{5.586299in}{1.264086in}}{\pgfqpoint{5.580475in}{1.258262in}}%
\pgfpathcurveto{\pgfqpoint{5.574651in}{1.252438in}}{\pgfqpoint{5.571378in}{1.244538in}}{\pgfqpoint{5.571378in}{1.236301in}}%
\pgfpathcurveto{\pgfqpoint{5.571378in}{1.228065in}}{\pgfqpoint{5.574651in}{1.220165in}}{\pgfqpoint{5.580475in}{1.214341in}}%
\pgfpathcurveto{\pgfqpoint{5.586299in}{1.208517in}}{\pgfqpoint{5.594199in}{1.205245in}}{\pgfqpoint{5.602435in}{1.205245in}}%
\pgfpathclose%
\pgfusepath{stroke,fill}%
\end{pgfscope}%
\begin{pgfscope}%
\pgfpathrectangle{\pgfqpoint{3.755891in}{0.557870in}}{\pgfqpoint{2.484109in}{1.484734in}}%
\pgfusepath{clip}%
\pgfsetbuttcap%
\pgfsetroundjoin%
\definecolor{currentfill}{rgb}{0.298039,0.447059,0.690196}%
\pgfsetfillcolor{currentfill}%
\pgfsetlinewidth{1.003750pt}%
\definecolor{currentstroke}{rgb}{0.298039,0.447059,0.690196}%
\pgfsetstrokecolor{currentstroke}%
\pgfsetdash{}{0pt}%
\pgfpathmoveto{\pgfqpoint{5.442759in}{1.588860in}}%
\pgfpathcurveto{\pgfqpoint{5.450995in}{1.588860in}}{\pgfqpoint{5.458895in}{1.592133in}}{\pgfqpoint{5.464719in}{1.597957in}}%
\pgfpathcurveto{\pgfqpoint{5.470543in}{1.603780in}}{\pgfqpoint{5.473815in}{1.611680in}}{\pgfqpoint{5.473815in}{1.619917in}}%
\pgfpathcurveto{\pgfqpoint{5.473815in}{1.628153in}}{\pgfqpoint{5.470543in}{1.636053in}}{\pgfqpoint{5.464719in}{1.641877in}}%
\pgfpathcurveto{\pgfqpoint{5.458895in}{1.647701in}}{\pgfqpoint{5.450995in}{1.650973in}}{\pgfqpoint{5.442759in}{1.650973in}}%
\pgfpathcurveto{\pgfqpoint{5.434522in}{1.650973in}}{\pgfqpoint{5.426622in}{1.647701in}}{\pgfqpoint{5.420798in}{1.641877in}}%
\pgfpathcurveto{\pgfqpoint{5.414974in}{1.636053in}}{\pgfqpoint{5.411702in}{1.628153in}}{\pgfqpoint{5.411702in}{1.619917in}}%
\pgfpathcurveto{\pgfqpoint{5.411702in}{1.611680in}}{\pgfqpoint{5.414974in}{1.603780in}}{\pgfqpoint{5.420798in}{1.597957in}}%
\pgfpathcurveto{\pgfqpoint{5.426622in}{1.592133in}}{\pgfqpoint{5.434522in}{1.588860in}}{\pgfqpoint{5.442759in}{1.588860in}}%
\pgfpathclose%
\pgfusepath{stroke,fill}%
\end{pgfscope}%
\begin{pgfscope}%
\pgfpathrectangle{\pgfqpoint{3.755891in}{0.557870in}}{\pgfqpoint{2.484109in}{1.484734in}}%
\pgfusepath{clip}%
\pgfsetbuttcap%
\pgfsetroundjoin%
\definecolor{currentfill}{rgb}{0.298039,0.447059,0.690196}%
\pgfsetfillcolor{currentfill}%
\pgfsetlinewidth{1.003750pt}%
\definecolor{currentstroke}{rgb}{0.298039,0.447059,0.690196}%
\pgfsetstrokecolor{currentstroke}%
\pgfsetdash{}{0pt}%
\pgfpathmoveto{\pgfqpoint{5.305893in}{1.475196in}}%
\pgfpathcurveto{\pgfqpoint{5.314129in}{1.475196in}}{\pgfqpoint{5.322029in}{1.478469in}}{\pgfqpoint{5.327853in}{1.484293in}}%
\pgfpathcurveto{\pgfqpoint{5.333677in}{1.490117in}}{\pgfqpoint{5.336950in}{1.498017in}}{\pgfqpoint{5.336950in}{1.506253in}}%
\pgfpathcurveto{\pgfqpoint{5.336950in}{1.514489in}}{\pgfqpoint{5.333677in}{1.522389in}}{\pgfqpoint{5.327853in}{1.528213in}}%
\pgfpathcurveto{\pgfqpoint{5.322029in}{1.534037in}}{\pgfqpoint{5.314129in}{1.537309in}}{\pgfqpoint{5.305893in}{1.537309in}}%
\pgfpathcurveto{\pgfqpoint{5.297657in}{1.537309in}}{\pgfqpoint{5.289757in}{1.534037in}}{\pgfqpoint{5.283933in}{1.528213in}}%
\pgfpathcurveto{\pgfqpoint{5.278109in}{1.522389in}}{\pgfqpoint{5.274837in}{1.514489in}}{\pgfqpoint{5.274837in}{1.506253in}}%
\pgfpathcurveto{\pgfqpoint{5.274837in}{1.498017in}}{\pgfqpoint{5.278109in}{1.490117in}}{\pgfqpoint{5.283933in}{1.484293in}}%
\pgfpathcurveto{\pgfqpoint{5.289757in}{1.478469in}}{\pgfqpoint{5.297657in}{1.475196in}}{\pgfqpoint{5.305893in}{1.475196in}}%
\pgfpathclose%
\pgfusepath{stroke,fill}%
\end{pgfscope}%
\begin{pgfscope}%
\pgfpathrectangle{\pgfqpoint{3.755891in}{0.557870in}}{\pgfqpoint{2.484109in}{1.484734in}}%
\pgfusepath{clip}%
\pgfsetbuttcap%
\pgfsetroundjoin%
\definecolor{currentfill}{rgb}{0.298039,0.447059,0.690196}%
\pgfsetfillcolor{currentfill}%
\pgfsetlinewidth{1.003750pt}%
\definecolor{currentstroke}{rgb}{0.298039,0.447059,0.690196}%
\pgfsetstrokecolor{currentstroke}%
\pgfsetdash{}{0pt}%
\pgfpathmoveto{\pgfqpoint{5.397137in}{1.588860in}}%
\pgfpathcurveto{\pgfqpoint{5.405373in}{1.588860in}}{\pgfqpoint{5.413273in}{1.592133in}}{\pgfqpoint{5.419097in}{1.597957in}}%
\pgfpathcurveto{\pgfqpoint{5.424921in}{1.603780in}}{\pgfqpoint{5.428193in}{1.611680in}}{\pgfqpoint{5.428193in}{1.619917in}}%
\pgfpathcurveto{\pgfqpoint{5.428193in}{1.628153in}}{\pgfqpoint{5.424921in}{1.636053in}}{\pgfqpoint{5.419097in}{1.641877in}}%
\pgfpathcurveto{\pgfqpoint{5.413273in}{1.647701in}}{\pgfqpoint{5.405373in}{1.650973in}}{\pgfqpoint{5.397137in}{1.650973in}}%
\pgfpathcurveto{\pgfqpoint{5.388900in}{1.650973in}}{\pgfqpoint{5.381000in}{1.647701in}}{\pgfqpoint{5.375176in}{1.641877in}}%
\pgfpathcurveto{\pgfqpoint{5.369352in}{1.636053in}}{\pgfqpoint{5.366080in}{1.628153in}}{\pgfqpoint{5.366080in}{1.619917in}}%
\pgfpathcurveto{\pgfqpoint{5.366080in}{1.611680in}}{\pgfqpoint{5.369352in}{1.603780in}}{\pgfqpoint{5.375176in}{1.597957in}}%
\pgfpathcurveto{\pgfqpoint{5.381000in}{1.592133in}}{\pgfqpoint{5.388900in}{1.588860in}}{\pgfqpoint{5.397137in}{1.588860in}}%
\pgfpathclose%
\pgfusepath{stroke,fill}%
\end{pgfscope}%
\begin{pgfscope}%
\pgfpathrectangle{\pgfqpoint{3.755891in}{0.557870in}}{\pgfqpoint{2.484109in}{1.484734in}}%
\pgfusepath{clip}%
\pgfsetbuttcap%
\pgfsetroundjoin%
\definecolor{currentfill}{rgb}{0.298039,0.447059,0.690196}%
\pgfsetfillcolor{currentfill}%
\pgfsetlinewidth{1.003750pt}%
\definecolor{currentstroke}{rgb}{0.298039,0.447059,0.690196}%
\pgfsetstrokecolor{currentstroke}%
\pgfsetdash{}{0pt}%
\pgfpathmoveto{\pgfqpoint{5.237460in}{1.574652in}}%
\pgfpathcurveto{\pgfqpoint{5.245697in}{1.574652in}}{\pgfqpoint{5.253597in}{1.577925in}}{\pgfqpoint{5.259421in}{1.583749in}}%
\pgfpathcurveto{\pgfqpoint{5.265244in}{1.589572in}}{\pgfqpoint{5.268517in}{1.597473in}}{\pgfqpoint{5.268517in}{1.605709in}}%
\pgfpathcurveto{\pgfqpoint{5.268517in}{1.613945in}}{\pgfqpoint{5.265244in}{1.621845in}}{\pgfqpoint{5.259421in}{1.627669in}}%
\pgfpathcurveto{\pgfqpoint{5.253597in}{1.633493in}}{\pgfqpoint{5.245697in}{1.636765in}}{\pgfqpoint{5.237460in}{1.636765in}}%
\pgfpathcurveto{\pgfqpoint{5.229224in}{1.636765in}}{\pgfqpoint{5.221324in}{1.633493in}}{\pgfqpoint{5.215500in}{1.627669in}}%
\pgfpathcurveto{\pgfqpoint{5.209676in}{1.621845in}}{\pgfqpoint{5.206404in}{1.613945in}}{\pgfqpoint{5.206404in}{1.605709in}}%
\pgfpathcurveto{\pgfqpoint{5.206404in}{1.597473in}}{\pgfqpoint{5.209676in}{1.589572in}}{\pgfqpoint{5.215500in}{1.583749in}}%
\pgfpathcurveto{\pgfqpoint{5.221324in}{1.577925in}}{\pgfqpoint{5.229224in}{1.574652in}}{\pgfqpoint{5.237460in}{1.574652in}}%
\pgfpathclose%
\pgfusepath{stroke,fill}%
\end{pgfscope}%
\begin{pgfscope}%
\pgfpathrectangle{\pgfqpoint{3.755891in}{0.557870in}}{\pgfqpoint{2.484109in}{1.484734in}}%
\pgfusepath{clip}%
\pgfsetbuttcap%
\pgfsetroundjoin%
\definecolor{currentfill}{rgb}{0.298039,0.447059,0.690196}%
\pgfsetfillcolor{currentfill}%
\pgfsetlinewidth{1.003750pt}%
\definecolor{currentstroke}{rgb}{0.298039,0.447059,0.690196}%
\pgfsetstrokecolor{currentstroke}%
\pgfsetdash{}{0pt}%
\pgfpathmoveto{\pgfqpoint{5.397137in}{1.503612in}}%
\pgfpathcurveto{\pgfqpoint{5.405373in}{1.503612in}}{\pgfqpoint{5.413273in}{1.506885in}}{\pgfqpoint{5.419097in}{1.512709in}}%
\pgfpathcurveto{\pgfqpoint{5.424921in}{1.518533in}}{\pgfqpoint{5.428193in}{1.526433in}}{\pgfqpoint{5.428193in}{1.534669in}}%
\pgfpathcurveto{\pgfqpoint{5.428193in}{1.542905in}}{\pgfqpoint{5.424921in}{1.550805in}}{\pgfqpoint{5.419097in}{1.556629in}}%
\pgfpathcurveto{\pgfqpoint{5.413273in}{1.562453in}}{\pgfqpoint{5.405373in}{1.565725in}}{\pgfqpoint{5.397137in}{1.565725in}}%
\pgfpathcurveto{\pgfqpoint{5.388900in}{1.565725in}}{\pgfqpoint{5.381000in}{1.562453in}}{\pgfqpoint{5.375176in}{1.556629in}}%
\pgfpathcurveto{\pgfqpoint{5.369352in}{1.550805in}}{\pgfqpoint{5.366080in}{1.542905in}}{\pgfqpoint{5.366080in}{1.534669in}}%
\pgfpathcurveto{\pgfqpoint{5.366080in}{1.526433in}}{\pgfqpoint{5.369352in}{1.518533in}}{\pgfqpoint{5.375176in}{1.512709in}}%
\pgfpathcurveto{\pgfqpoint{5.381000in}{1.506885in}}{\pgfqpoint{5.388900in}{1.503612in}}{\pgfqpoint{5.397137in}{1.503612in}}%
\pgfpathclose%
\pgfusepath{stroke,fill}%
\end{pgfscope}%
\begin{pgfscope}%
\pgfpathrectangle{\pgfqpoint{3.755891in}{0.557870in}}{\pgfqpoint{2.484109in}{1.484734in}}%
\pgfusepath{clip}%
\pgfsetbuttcap%
\pgfsetroundjoin%
\definecolor{currentfill}{rgb}{0.298039,0.447059,0.690196}%
\pgfsetfillcolor{currentfill}%
\pgfsetlinewidth{1.003750pt}%
\definecolor{currentstroke}{rgb}{0.298039,0.447059,0.690196}%
\pgfsetstrokecolor{currentstroke}%
\pgfsetdash{}{0pt}%
\pgfpathmoveto{\pgfqpoint{5.419948in}{1.574652in}}%
\pgfpathcurveto{\pgfqpoint{5.428184in}{1.574652in}}{\pgfqpoint{5.436084in}{1.577925in}}{\pgfqpoint{5.441908in}{1.583749in}}%
\pgfpathcurveto{\pgfqpoint{5.447732in}{1.589572in}}{\pgfqpoint{5.451004in}{1.597473in}}{\pgfqpoint{5.451004in}{1.605709in}}%
\pgfpathcurveto{\pgfqpoint{5.451004in}{1.613945in}}{\pgfqpoint{5.447732in}{1.621845in}}{\pgfqpoint{5.441908in}{1.627669in}}%
\pgfpathcurveto{\pgfqpoint{5.436084in}{1.633493in}}{\pgfqpoint{5.428184in}{1.636765in}}{\pgfqpoint{5.419948in}{1.636765in}}%
\pgfpathcurveto{\pgfqpoint{5.411711in}{1.636765in}}{\pgfqpoint{5.403811in}{1.633493in}}{\pgfqpoint{5.397987in}{1.627669in}}%
\pgfpathcurveto{\pgfqpoint{5.392163in}{1.621845in}}{\pgfqpoint{5.388891in}{1.613945in}}{\pgfqpoint{5.388891in}{1.605709in}}%
\pgfpathcurveto{\pgfqpoint{5.388891in}{1.597473in}}{\pgfqpoint{5.392163in}{1.589572in}}{\pgfqpoint{5.397987in}{1.583749in}}%
\pgfpathcurveto{\pgfqpoint{5.403811in}{1.577925in}}{\pgfqpoint{5.411711in}{1.574652in}}{\pgfqpoint{5.419948in}{1.574652in}}%
\pgfpathclose%
\pgfusepath{stroke,fill}%
\end{pgfscope}%
\begin{pgfscope}%
\pgfpathrectangle{\pgfqpoint{3.755891in}{0.557870in}}{\pgfqpoint{2.484109in}{1.484734in}}%
\pgfusepath{clip}%
\pgfsetbuttcap%
\pgfsetroundjoin%
\definecolor{currentfill}{rgb}{0.298039,0.447059,0.690196}%
\pgfsetfillcolor{currentfill}%
\pgfsetlinewidth{1.003750pt}%
\definecolor{currentstroke}{rgb}{0.298039,0.447059,0.690196}%
\pgfsetstrokecolor{currentstroke}%
\pgfsetdash{}{0pt}%
\pgfpathmoveto{\pgfqpoint{5.442759in}{1.517820in}}%
\pgfpathcurveto{\pgfqpoint{5.450995in}{1.517820in}}{\pgfqpoint{5.458895in}{1.521093in}}{\pgfqpoint{5.464719in}{1.526917in}}%
\pgfpathcurveto{\pgfqpoint{5.470543in}{1.532741in}}{\pgfqpoint{5.473815in}{1.540641in}}{\pgfqpoint{5.473815in}{1.548877in}}%
\pgfpathcurveto{\pgfqpoint{5.473815in}{1.557113in}}{\pgfqpoint{5.470543in}{1.565013in}}{\pgfqpoint{5.464719in}{1.570837in}}%
\pgfpathcurveto{\pgfqpoint{5.458895in}{1.576661in}}{\pgfqpoint{5.450995in}{1.579933in}}{\pgfqpoint{5.442759in}{1.579933in}}%
\pgfpathcurveto{\pgfqpoint{5.434522in}{1.579933in}}{\pgfqpoint{5.426622in}{1.576661in}}{\pgfqpoint{5.420798in}{1.570837in}}%
\pgfpathcurveto{\pgfqpoint{5.414974in}{1.565013in}}{\pgfqpoint{5.411702in}{1.557113in}}{\pgfqpoint{5.411702in}{1.548877in}}%
\pgfpathcurveto{\pgfqpoint{5.411702in}{1.540641in}}{\pgfqpoint{5.414974in}{1.532741in}}{\pgfqpoint{5.420798in}{1.526917in}}%
\pgfpathcurveto{\pgfqpoint{5.426622in}{1.521093in}}{\pgfqpoint{5.434522in}{1.517820in}}{\pgfqpoint{5.442759in}{1.517820in}}%
\pgfpathclose%
\pgfusepath{stroke,fill}%
\end{pgfscope}%
\begin{pgfscope}%
\pgfpathrectangle{\pgfqpoint{3.755891in}{0.557870in}}{\pgfqpoint{2.484109in}{1.484734in}}%
\pgfusepath{clip}%
\pgfsetbuttcap%
\pgfsetroundjoin%
\definecolor{currentfill}{rgb}{0.298039,0.447059,0.690196}%
\pgfsetfillcolor{currentfill}%
\pgfsetlinewidth{1.003750pt}%
\definecolor{currentstroke}{rgb}{0.298039,0.447059,0.690196}%
\pgfsetstrokecolor{currentstroke}%
\pgfsetdash{}{0pt}%
\pgfpathmoveto{\pgfqpoint{5.474694in}{1.484864in}}%
\pgfpathcurveto{\pgfqpoint{5.482930in}{1.484864in}}{\pgfqpoint{5.490830in}{1.488136in}}{\pgfqpoint{5.496654in}{1.493960in}}%
\pgfpathcurveto{\pgfqpoint{5.502478in}{1.499784in}}{\pgfqpoint{5.505750in}{1.507684in}}{\pgfqpoint{5.505750in}{1.515920in}}%
\pgfpathcurveto{\pgfqpoint{5.505750in}{1.524156in}}{\pgfqpoint{5.502478in}{1.532057in}}{\pgfqpoint{5.496654in}{1.537880in}}%
\pgfpathcurveto{\pgfqpoint{5.490830in}{1.543704in}}{\pgfqpoint{5.482930in}{1.546977in}}{\pgfqpoint{5.474694in}{1.546977in}}%
\pgfpathcurveto{\pgfqpoint{5.466458in}{1.546977in}}{\pgfqpoint{5.458557in}{1.543704in}}{\pgfqpoint{5.452734in}{1.537880in}}%
\pgfpathcurveto{\pgfqpoint{5.446910in}{1.532057in}}{\pgfqpoint{5.443637in}{1.524156in}}{\pgfqpoint{5.443637in}{1.515920in}}%
\pgfpathcurveto{\pgfqpoint{5.443637in}{1.507684in}}{\pgfqpoint{5.446910in}{1.499784in}}{\pgfqpoint{5.452734in}{1.493960in}}%
\pgfpathcurveto{\pgfqpoint{5.458557in}{1.488136in}}{\pgfqpoint{5.466458in}{1.484864in}}{\pgfqpoint{5.474694in}{1.484864in}}%
\pgfpathclose%
\pgfusepath{stroke,fill}%
\end{pgfscope}%
\begin{pgfscope}%
\pgfpathrectangle{\pgfqpoint{3.755891in}{0.557870in}}{\pgfqpoint{2.484109in}{1.484734in}}%
\pgfusepath{clip}%
\pgfsetbuttcap%
\pgfsetroundjoin%
\definecolor{currentfill}{rgb}{0.298039,0.447059,0.690196}%
\pgfsetfillcolor{currentfill}%
\pgfsetlinewidth{1.003750pt}%
\definecolor{currentstroke}{rgb}{0.298039,0.447059,0.690196}%
\pgfsetstrokecolor{currentstroke}%
\pgfsetdash{}{0pt}%
\pgfpathmoveto{\pgfqpoint{5.248866in}{1.693589in}}%
\pgfpathcurveto{\pgfqpoint{5.257102in}{1.693589in}}{\pgfqpoint{5.265002in}{1.696861in}}{\pgfqpoint{5.270826in}{1.702685in}}%
\pgfpathcurveto{\pgfqpoint{5.276650in}{1.708509in}}{\pgfqpoint{5.279922in}{1.716409in}}{\pgfqpoint{5.279922in}{1.724646in}}%
\pgfpathcurveto{\pgfqpoint{5.279922in}{1.732882in}}{\pgfqpoint{5.276650in}{1.740782in}}{\pgfqpoint{5.270826in}{1.746606in}}%
\pgfpathcurveto{\pgfqpoint{5.265002in}{1.752430in}}{\pgfqpoint{5.257102in}{1.755702in}}{\pgfqpoint{5.248866in}{1.755702in}}%
\pgfpathcurveto{\pgfqpoint{5.240629in}{1.755702in}}{\pgfqpoint{5.232729in}{1.752430in}}{\pgfqpoint{5.226905in}{1.746606in}}%
\pgfpathcurveto{\pgfqpoint{5.221082in}{1.740782in}}{\pgfqpoint{5.217809in}{1.732882in}}{\pgfqpoint{5.217809in}{1.724646in}}%
\pgfpathcurveto{\pgfqpoint{5.217809in}{1.716409in}}{\pgfqpoint{5.221082in}{1.708509in}}{\pgfqpoint{5.226905in}{1.702685in}}%
\pgfpathcurveto{\pgfqpoint{5.232729in}{1.696861in}}{\pgfqpoint{5.240629in}{1.693589in}}{\pgfqpoint{5.248866in}{1.693589in}}%
\pgfpathclose%
\pgfusepath{stroke,fill}%
\end{pgfscope}%
\begin{pgfscope}%
\pgfpathrectangle{\pgfqpoint{3.755891in}{0.557870in}}{\pgfqpoint{2.484109in}{1.484734in}}%
\pgfusepath{clip}%
\pgfsetbuttcap%
\pgfsetroundjoin%
\definecolor{currentfill}{rgb}{0.298039,0.447059,0.690196}%
\pgfsetfillcolor{currentfill}%
\pgfsetlinewidth{1.003750pt}%
\definecolor{currentstroke}{rgb}{0.298039,0.447059,0.690196}%
\pgfsetstrokecolor{currentstroke}%
\pgfsetdash{}{0pt}%
\pgfpathmoveto{\pgfqpoint{5.374326in}{1.610099in}}%
\pgfpathcurveto{\pgfqpoint{5.382562in}{1.610099in}}{\pgfqpoint{5.390462in}{1.613371in}}{\pgfqpoint{5.396286in}{1.619195in}}%
\pgfpathcurveto{\pgfqpoint{5.402110in}{1.625019in}}{\pgfqpoint{5.405382in}{1.632919in}}{\pgfqpoint{5.405382in}{1.641156in}}%
\pgfpathcurveto{\pgfqpoint{5.405382in}{1.649392in}}{\pgfqpoint{5.402110in}{1.657292in}}{\pgfqpoint{5.396286in}{1.663116in}}%
\pgfpathcurveto{\pgfqpoint{5.390462in}{1.668940in}}{\pgfqpoint{5.382562in}{1.672212in}}{\pgfqpoint{5.374326in}{1.672212in}}%
\pgfpathcurveto{\pgfqpoint{5.366089in}{1.672212in}}{\pgfqpoint{5.358189in}{1.668940in}}{\pgfqpoint{5.352366in}{1.663116in}}%
\pgfpathcurveto{\pgfqpoint{5.346542in}{1.657292in}}{\pgfqpoint{5.343269in}{1.649392in}}{\pgfqpoint{5.343269in}{1.641156in}}%
\pgfpathcurveto{\pgfqpoint{5.343269in}{1.632919in}}{\pgfqpoint{5.346542in}{1.625019in}}{\pgfqpoint{5.352366in}{1.619195in}}%
\pgfpathcurveto{\pgfqpoint{5.358189in}{1.613371in}}{\pgfqpoint{5.366089in}{1.610099in}}{\pgfqpoint{5.374326in}{1.610099in}}%
\pgfpathclose%
\pgfusepath{stroke,fill}%
\end{pgfscope}%
\begin{pgfscope}%
\pgfpathrectangle{\pgfqpoint{3.755891in}{0.557870in}}{\pgfqpoint{2.484109in}{1.484734in}}%
\pgfusepath{clip}%
\pgfsetbuttcap%
\pgfsetroundjoin%
\definecolor{currentfill}{rgb}{0.298039,0.447059,0.690196}%
\pgfsetfillcolor{currentfill}%
\pgfsetlinewidth{1.003750pt}%
\definecolor{currentstroke}{rgb}{0.298039,0.447059,0.690196}%
\pgfsetstrokecolor{currentstroke}%
\pgfsetdash{}{0pt}%
\pgfpathmoveto{\pgfqpoint{5.173590in}{1.554439in}}%
\pgfpathcurveto{\pgfqpoint{5.181826in}{1.554439in}}{\pgfqpoint{5.189726in}{1.557711in}}{\pgfqpoint{5.195550in}{1.563535in}}%
\pgfpathcurveto{\pgfqpoint{5.201374in}{1.569359in}}{\pgfqpoint{5.204646in}{1.577259in}}{\pgfqpoint{5.204646in}{1.585495in}}%
\pgfpathcurveto{\pgfqpoint{5.204646in}{1.593732in}}{\pgfqpoint{5.201374in}{1.601632in}}{\pgfqpoint{5.195550in}{1.607456in}}%
\pgfpathcurveto{\pgfqpoint{5.189726in}{1.613280in}}{\pgfqpoint{5.181826in}{1.616552in}}{\pgfqpoint{5.173590in}{1.616552in}}%
\pgfpathcurveto{\pgfqpoint{5.165353in}{1.616552in}}{\pgfqpoint{5.157453in}{1.613280in}}{\pgfqpoint{5.151629in}{1.607456in}}%
\pgfpathcurveto{\pgfqpoint{5.145806in}{1.601632in}}{\pgfqpoint{5.142533in}{1.593732in}}{\pgfqpoint{5.142533in}{1.585495in}}%
\pgfpathcurveto{\pgfqpoint{5.142533in}{1.577259in}}{\pgfqpoint{5.145806in}{1.569359in}}{\pgfqpoint{5.151629in}{1.563535in}}%
\pgfpathcurveto{\pgfqpoint{5.157453in}{1.557711in}}{\pgfqpoint{5.165353in}{1.554439in}}{\pgfqpoint{5.173590in}{1.554439in}}%
\pgfpathclose%
\pgfusepath{stroke,fill}%
\end{pgfscope}%
\begin{pgfscope}%
\pgfpathrectangle{\pgfqpoint{3.755891in}{0.557870in}}{\pgfqpoint{2.484109in}{1.484734in}}%
\pgfusepath{clip}%
\pgfsetbuttcap%
\pgfsetroundjoin%
\definecolor{currentfill}{rgb}{0.298039,0.447059,0.690196}%
\pgfsetfillcolor{currentfill}%
\pgfsetlinewidth{1.003750pt}%
\definecolor{currentstroke}{rgb}{0.298039,0.447059,0.690196}%
\pgfsetstrokecolor{currentstroke}%
\pgfsetdash{}{0pt}%
\pgfpathmoveto{\pgfqpoint{5.273958in}{1.624014in}}%
\pgfpathcurveto{\pgfqpoint{5.282194in}{1.624014in}}{\pgfqpoint{5.290094in}{1.627286in}}{\pgfqpoint{5.295918in}{1.633110in}}%
\pgfpathcurveto{\pgfqpoint{5.301742in}{1.638934in}}{\pgfqpoint{5.305014in}{1.646834in}}{\pgfqpoint{5.305014in}{1.655071in}}%
\pgfpathcurveto{\pgfqpoint{5.305014in}{1.663307in}}{\pgfqpoint{5.301742in}{1.671207in}}{\pgfqpoint{5.295918in}{1.677031in}}%
\pgfpathcurveto{\pgfqpoint{5.290094in}{1.682855in}}{\pgfqpoint{5.282194in}{1.686127in}}{\pgfqpoint{5.273958in}{1.686127in}}%
\pgfpathcurveto{\pgfqpoint{5.265721in}{1.686127in}}{\pgfqpoint{5.257821in}{1.682855in}}{\pgfqpoint{5.251997in}{1.677031in}}%
\pgfpathcurveto{\pgfqpoint{5.246174in}{1.671207in}}{\pgfqpoint{5.242901in}{1.663307in}}{\pgfqpoint{5.242901in}{1.655071in}}%
\pgfpathcurveto{\pgfqpoint{5.242901in}{1.646834in}}{\pgfqpoint{5.246174in}{1.638934in}}{\pgfqpoint{5.251997in}{1.633110in}}%
\pgfpathcurveto{\pgfqpoint{5.257821in}{1.627286in}}{\pgfqpoint{5.265721in}{1.624014in}}{\pgfqpoint{5.273958in}{1.624014in}}%
\pgfpathclose%
\pgfusepath{stroke,fill}%
\end{pgfscope}%
\begin{pgfscope}%
\pgfpathrectangle{\pgfqpoint{3.755891in}{0.557870in}}{\pgfqpoint{2.484109in}{1.484734in}}%
\pgfusepath{clip}%
\pgfsetbuttcap%
\pgfsetroundjoin%
\definecolor{currentfill}{rgb}{0.298039,0.447059,0.690196}%
\pgfsetfillcolor{currentfill}%
\pgfsetlinewidth{1.003750pt}%
\definecolor{currentstroke}{rgb}{0.298039,0.447059,0.690196}%
\pgfsetstrokecolor{currentstroke}%
\pgfsetdash{}{0pt}%
\pgfpathmoveto{\pgfqpoint{5.374326in}{1.624014in}}%
\pgfpathcurveto{\pgfqpoint{5.382562in}{1.624014in}}{\pgfqpoint{5.390462in}{1.627286in}}{\pgfqpoint{5.396286in}{1.633110in}}%
\pgfpathcurveto{\pgfqpoint{5.402110in}{1.638934in}}{\pgfqpoint{5.405382in}{1.646834in}}{\pgfqpoint{5.405382in}{1.655071in}}%
\pgfpathcurveto{\pgfqpoint{5.405382in}{1.663307in}}{\pgfqpoint{5.402110in}{1.671207in}}{\pgfqpoint{5.396286in}{1.677031in}}%
\pgfpathcurveto{\pgfqpoint{5.390462in}{1.682855in}}{\pgfqpoint{5.382562in}{1.686127in}}{\pgfqpoint{5.374326in}{1.686127in}}%
\pgfpathcurveto{\pgfqpoint{5.366089in}{1.686127in}}{\pgfqpoint{5.358189in}{1.682855in}}{\pgfqpoint{5.352366in}{1.677031in}}%
\pgfpathcurveto{\pgfqpoint{5.346542in}{1.671207in}}{\pgfqpoint{5.343269in}{1.663307in}}{\pgfqpoint{5.343269in}{1.655071in}}%
\pgfpathcurveto{\pgfqpoint{5.343269in}{1.646834in}}{\pgfqpoint{5.346542in}{1.638934in}}{\pgfqpoint{5.352366in}{1.633110in}}%
\pgfpathcurveto{\pgfqpoint{5.358189in}{1.627286in}}{\pgfqpoint{5.366089in}{1.624014in}}{\pgfqpoint{5.374326in}{1.624014in}}%
\pgfpathclose%
\pgfusepath{stroke,fill}%
\end{pgfscope}%
\begin{pgfscope}%
\pgfpathrectangle{\pgfqpoint{3.755891in}{0.557870in}}{\pgfqpoint{2.484109in}{1.484734in}}%
\pgfusepath{clip}%
\pgfsetbuttcap%
\pgfsetroundjoin%
\definecolor{currentfill}{rgb}{0.298039,0.447059,0.690196}%
\pgfsetfillcolor{currentfill}%
\pgfsetlinewidth{1.003750pt}%
\definecolor{currentstroke}{rgb}{0.298039,0.447059,0.690196}%
\pgfsetstrokecolor{currentstroke}%
\pgfsetdash{}{0pt}%
\pgfpathmoveto{\pgfqpoint{5.324142in}{1.554439in}}%
\pgfpathcurveto{\pgfqpoint{5.332378in}{1.554439in}}{\pgfqpoint{5.340278in}{1.557711in}}{\pgfqpoint{5.346102in}{1.563535in}}%
\pgfpathcurveto{\pgfqpoint{5.351926in}{1.569359in}}{\pgfqpoint{5.355198in}{1.577259in}}{\pgfqpoint{5.355198in}{1.585495in}}%
\pgfpathcurveto{\pgfqpoint{5.355198in}{1.593732in}}{\pgfqpoint{5.351926in}{1.601632in}}{\pgfqpoint{5.346102in}{1.607456in}}%
\pgfpathcurveto{\pgfqpoint{5.340278in}{1.613280in}}{\pgfqpoint{5.332378in}{1.616552in}}{\pgfqpoint{5.324142in}{1.616552in}}%
\pgfpathcurveto{\pgfqpoint{5.315905in}{1.616552in}}{\pgfqpoint{5.308005in}{1.613280in}}{\pgfqpoint{5.302181in}{1.607456in}}%
\pgfpathcurveto{\pgfqpoint{5.296358in}{1.601632in}}{\pgfqpoint{5.293085in}{1.593732in}}{\pgfqpoint{5.293085in}{1.585495in}}%
\pgfpathcurveto{\pgfqpoint{5.293085in}{1.577259in}}{\pgfqpoint{5.296358in}{1.569359in}}{\pgfqpoint{5.302181in}{1.563535in}}%
\pgfpathcurveto{\pgfqpoint{5.308005in}{1.557711in}}{\pgfqpoint{5.315905in}{1.554439in}}{\pgfqpoint{5.324142in}{1.554439in}}%
\pgfpathclose%
\pgfusepath{stroke,fill}%
\end{pgfscope}%
\begin{pgfscope}%
\pgfpathrectangle{\pgfqpoint{3.755891in}{0.557870in}}{\pgfqpoint{2.484109in}{1.484734in}}%
\pgfusepath{clip}%
\pgfsetbuttcap%
\pgfsetroundjoin%
\definecolor{currentfill}{rgb}{0.298039,0.447059,0.690196}%
\pgfsetfillcolor{currentfill}%
\pgfsetlinewidth{1.003750pt}%
\definecolor{currentstroke}{rgb}{0.298039,0.447059,0.690196}%
\pgfsetstrokecolor{currentstroke}%
\pgfsetdash{}{0pt}%
\pgfpathmoveto{\pgfqpoint{5.299050in}{1.624014in}}%
\pgfpathcurveto{\pgfqpoint{5.307286in}{1.624014in}}{\pgfqpoint{5.315186in}{1.627286in}}{\pgfqpoint{5.321010in}{1.633110in}}%
\pgfpathcurveto{\pgfqpoint{5.326834in}{1.638934in}}{\pgfqpoint{5.330106in}{1.646834in}}{\pgfqpoint{5.330106in}{1.655071in}}%
\pgfpathcurveto{\pgfqpoint{5.330106in}{1.663307in}}{\pgfqpoint{5.326834in}{1.671207in}}{\pgfqpoint{5.321010in}{1.677031in}}%
\pgfpathcurveto{\pgfqpoint{5.315186in}{1.682855in}}{\pgfqpoint{5.307286in}{1.686127in}}{\pgfqpoint{5.299050in}{1.686127in}}%
\pgfpathcurveto{\pgfqpoint{5.290813in}{1.686127in}}{\pgfqpoint{5.282913in}{1.682855in}}{\pgfqpoint{5.277089in}{1.677031in}}%
\pgfpathcurveto{\pgfqpoint{5.271266in}{1.671207in}}{\pgfqpoint{5.267993in}{1.663307in}}{\pgfqpoint{5.267993in}{1.655071in}}%
\pgfpathcurveto{\pgfqpoint{5.267993in}{1.646834in}}{\pgfqpoint{5.271266in}{1.638934in}}{\pgfqpoint{5.277089in}{1.633110in}}%
\pgfpathcurveto{\pgfqpoint{5.282913in}{1.627286in}}{\pgfqpoint{5.290813in}{1.624014in}}{\pgfqpoint{5.299050in}{1.624014in}}%
\pgfpathclose%
\pgfusepath{stroke,fill}%
\end{pgfscope}%
\begin{pgfscope}%
\pgfpathrectangle{\pgfqpoint{3.755891in}{0.557870in}}{\pgfqpoint{2.484109in}{1.484734in}}%
\pgfusepath{clip}%
\pgfsetbuttcap%
\pgfsetroundjoin%
\definecolor{currentfill}{rgb}{0.298039,0.447059,0.690196}%
\pgfsetfillcolor{currentfill}%
\pgfsetlinewidth{1.003750pt}%
\definecolor{currentstroke}{rgb}{0.298039,0.447059,0.690196}%
\pgfsetstrokecolor{currentstroke}%
\pgfsetdash{}{0pt}%
\pgfpathmoveto{\pgfqpoint{5.474694in}{1.610099in}}%
\pgfpathcurveto{\pgfqpoint{5.482930in}{1.610099in}}{\pgfqpoint{5.490830in}{1.613371in}}{\pgfqpoint{5.496654in}{1.619195in}}%
\pgfpathcurveto{\pgfqpoint{5.502478in}{1.625019in}}{\pgfqpoint{5.505750in}{1.632919in}}{\pgfqpoint{5.505750in}{1.641156in}}%
\pgfpathcurveto{\pgfqpoint{5.505750in}{1.649392in}}{\pgfqpoint{5.502478in}{1.657292in}}{\pgfqpoint{5.496654in}{1.663116in}}%
\pgfpathcurveto{\pgfqpoint{5.490830in}{1.668940in}}{\pgfqpoint{5.482930in}{1.672212in}}{\pgfqpoint{5.474694in}{1.672212in}}%
\pgfpathcurveto{\pgfqpoint{5.466458in}{1.672212in}}{\pgfqpoint{5.458557in}{1.668940in}}{\pgfqpoint{5.452734in}{1.663116in}}%
\pgfpathcurveto{\pgfqpoint{5.446910in}{1.657292in}}{\pgfqpoint{5.443637in}{1.649392in}}{\pgfqpoint{5.443637in}{1.641156in}}%
\pgfpathcurveto{\pgfqpoint{5.443637in}{1.632919in}}{\pgfqpoint{5.446910in}{1.625019in}}{\pgfqpoint{5.452734in}{1.619195in}}%
\pgfpathcurveto{\pgfqpoint{5.458557in}{1.613371in}}{\pgfqpoint{5.466458in}{1.610099in}}{\pgfqpoint{5.474694in}{1.610099in}}%
\pgfpathclose%
\pgfusepath{stroke,fill}%
\end{pgfscope}%
\begin{pgfscope}%
\pgfpathrectangle{\pgfqpoint{3.755891in}{0.557870in}}{\pgfqpoint{2.484109in}{1.484734in}}%
\pgfusepath{clip}%
\pgfsetbuttcap%
\pgfsetroundjoin%
\definecolor{currentfill}{rgb}{0.298039,0.447059,0.690196}%
\pgfsetfillcolor{currentfill}%
\pgfsetlinewidth{1.003750pt}%
\definecolor{currentstroke}{rgb}{0.298039,0.447059,0.690196}%
\pgfsetstrokecolor{currentstroke}%
\pgfsetdash{}{0pt}%
\pgfpathmoveto{\pgfqpoint{5.349234in}{1.526609in}}%
\pgfpathcurveto{\pgfqpoint{5.357470in}{1.526609in}}{\pgfqpoint{5.365370in}{1.529881in}}{\pgfqpoint{5.371194in}{1.535705in}}%
\pgfpathcurveto{\pgfqpoint{5.377018in}{1.541529in}}{\pgfqpoint{5.380290in}{1.549429in}}{\pgfqpoint{5.380290in}{1.557665in}}%
\pgfpathcurveto{\pgfqpoint{5.380290in}{1.565902in}}{\pgfqpoint{5.377018in}{1.573802in}}{\pgfqpoint{5.371194in}{1.579626in}}%
\pgfpathcurveto{\pgfqpoint{5.365370in}{1.585450in}}{\pgfqpoint{5.357470in}{1.588722in}}{\pgfqpoint{5.349234in}{1.588722in}}%
\pgfpathcurveto{\pgfqpoint{5.340997in}{1.588722in}}{\pgfqpoint{5.333097in}{1.585450in}}{\pgfqpoint{5.327273in}{1.579626in}}%
\pgfpathcurveto{\pgfqpoint{5.321450in}{1.573802in}}{\pgfqpoint{5.318177in}{1.565902in}}{\pgfqpoint{5.318177in}{1.557665in}}%
\pgfpathcurveto{\pgfqpoint{5.318177in}{1.549429in}}{\pgfqpoint{5.321450in}{1.541529in}}{\pgfqpoint{5.327273in}{1.535705in}}%
\pgfpathcurveto{\pgfqpoint{5.333097in}{1.529881in}}{\pgfqpoint{5.340997in}{1.526609in}}{\pgfqpoint{5.349234in}{1.526609in}}%
\pgfpathclose%
\pgfusepath{stroke,fill}%
\end{pgfscope}%
\begin{pgfscope}%
\pgfpathrectangle{\pgfqpoint{3.755891in}{0.557870in}}{\pgfqpoint{2.484109in}{1.484734in}}%
\pgfusepath{clip}%
\pgfsetbuttcap%
\pgfsetroundjoin%
\definecolor{currentfill}{rgb}{0.298039,0.447059,0.690196}%
\pgfsetfillcolor{currentfill}%
\pgfsetlinewidth{1.003750pt}%
\definecolor{currentstroke}{rgb}{0.298039,0.447059,0.690196}%
\pgfsetstrokecolor{currentstroke}%
\pgfsetdash{}{0pt}%
\pgfpathmoveto{\pgfqpoint{5.399418in}{1.537681in}}%
\pgfpathcurveto{\pgfqpoint{5.407654in}{1.537681in}}{\pgfqpoint{5.415554in}{1.540953in}}{\pgfqpoint{5.421378in}{1.546777in}}%
\pgfpathcurveto{\pgfqpoint{5.427202in}{1.552601in}}{\pgfqpoint{5.430474in}{1.560501in}}{\pgfqpoint{5.430474in}{1.568737in}}%
\pgfpathcurveto{\pgfqpoint{5.430474in}{1.576974in}}{\pgfqpoint{5.427202in}{1.584874in}}{\pgfqpoint{5.421378in}{1.590698in}}%
\pgfpathcurveto{\pgfqpoint{5.415554in}{1.596522in}}{\pgfqpoint{5.407654in}{1.599794in}}{\pgfqpoint{5.399418in}{1.599794in}}%
\pgfpathcurveto{\pgfqpoint{5.391181in}{1.599794in}}{\pgfqpoint{5.383281in}{1.596522in}}{\pgfqpoint{5.377458in}{1.590698in}}%
\pgfpathcurveto{\pgfqpoint{5.371634in}{1.584874in}}{\pgfqpoint{5.368361in}{1.576974in}}{\pgfqpoint{5.368361in}{1.568737in}}%
\pgfpathcurveto{\pgfqpoint{5.368361in}{1.560501in}}{\pgfqpoint{5.371634in}{1.552601in}}{\pgfqpoint{5.377458in}{1.546777in}}%
\pgfpathcurveto{\pgfqpoint{5.383281in}{1.540953in}}{\pgfqpoint{5.391181in}{1.537681in}}{\pgfqpoint{5.399418in}{1.537681in}}%
\pgfpathclose%
\pgfusepath{stroke,fill}%
\end{pgfscope}%
\begin{pgfscope}%
\pgfpathrectangle{\pgfqpoint{3.755891in}{0.557870in}}{\pgfqpoint{2.484109in}{1.484734in}}%
\pgfusepath{clip}%
\pgfsetbuttcap%
\pgfsetroundjoin%
\definecolor{currentfill}{rgb}{0.298039,0.447059,0.690196}%
\pgfsetfillcolor{currentfill}%
\pgfsetlinewidth{1.003750pt}%
\definecolor{currentstroke}{rgb}{0.298039,0.447059,0.690196}%
\pgfsetstrokecolor{currentstroke}%
\pgfsetdash{}{0pt}%
\pgfpathmoveto{\pgfqpoint{5.248866in}{1.624762in}}%
\pgfpathcurveto{\pgfqpoint{5.257102in}{1.624762in}}{\pgfqpoint{5.265002in}{1.628034in}}{\pgfqpoint{5.270826in}{1.633858in}}%
\pgfpathcurveto{\pgfqpoint{5.276650in}{1.639682in}}{\pgfqpoint{5.279922in}{1.647582in}}{\pgfqpoint{5.279922in}{1.655819in}}%
\pgfpathcurveto{\pgfqpoint{5.279922in}{1.664055in}}{\pgfqpoint{5.276650in}{1.671955in}}{\pgfqpoint{5.270826in}{1.677779in}}%
\pgfpathcurveto{\pgfqpoint{5.265002in}{1.683603in}}{\pgfqpoint{5.257102in}{1.686875in}}{\pgfqpoint{5.248866in}{1.686875in}}%
\pgfpathcurveto{\pgfqpoint{5.240629in}{1.686875in}}{\pgfqpoint{5.232729in}{1.683603in}}{\pgfqpoint{5.226905in}{1.677779in}}%
\pgfpathcurveto{\pgfqpoint{5.221082in}{1.671955in}}{\pgfqpoint{5.217809in}{1.664055in}}{\pgfqpoint{5.217809in}{1.655819in}}%
\pgfpathcurveto{\pgfqpoint{5.217809in}{1.647582in}}{\pgfqpoint{5.221082in}{1.639682in}}{\pgfqpoint{5.226905in}{1.633858in}}%
\pgfpathcurveto{\pgfqpoint{5.232729in}{1.628034in}}{\pgfqpoint{5.240629in}{1.624762in}}{\pgfqpoint{5.248866in}{1.624762in}}%
\pgfpathclose%
\pgfusepath{stroke,fill}%
\end{pgfscope}%
\begin{pgfscope}%
\pgfpathrectangle{\pgfqpoint{3.755891in}{0.557870in}}{\pgfqpoint{2.484109in}{1.484734in}}%
\pgfusepath{clip}%
\pgfsetbuttcap%
\pgfsetroundjoin%
\definecolor{currentfill}{rgb}{0.298039,0.447059,0.690196}%
\pgfsetfillcolor{currentfill}%
\pgfsetlinewidth{1.003750pt}%
\definecolor{currentstroke}{rgb}{0.298039,0.447059,0.690196}%
\pgfsetstrokecolor{currentstroke}%
\pgfsetdash{}{0pt}%
\pgfpathmoveto{\pgfqpoint{5.424510in}{1.566708in}}%
\pgfpathcurveto{\pgfqpoint{5.432746in}{1.566708in}}{\pgfqpoint{5.440646in}{1.569980in}}{\pgfqpoint{5.446470in}{1.575804in}}%
\pgfpathcurveto{\pgfqpoint{5.452294in}{1.581628in}}{\pgfqpoint{5.455566in}{1.589528in}}{\pgfqpoint{5.455566in}{1.597765in}}%
\pgfpathcurveto{\pgfqpoint{5.455566in}{1.606001in}}{\pgfqpoint{5.452294in}{1.613901in}}{\pgfqpoint{5.446470in}{1.619725in}}%
\pgfpathcurveto{\pgfqpoint{5.440646in}{1.625549in}}{\pgfqpoint{5.432746in}{1.628821in}}{\pgfqpoint{5.424510in}{1.628821in}}%
\pgfpathcurveto{\pgfqpoint{5.416273in}{1.628821in}}{\pgfqpoint{5.408373in}{1.625549in}}{\pgfqpoint{5.402550in}{1.619725in}}%
\pgfpathcurveto{\pgfqpoint{5.396726in}{1.613901in}}{\pgfqpoint{5.393453in}{1.606001in}}{\pgfqpoint{5.393453in}{1.597765in}}%
\pgfpathcurveto{\pgfqpoint{5.393453in}{1.589528in}}{\pgfqpoint{5.396726in}{1.581628in}}{\pgfqpoint{5.402550in}{1.575804in}}%
\pgfpathcurveto{\pgfqpoint{5.408373in}{1.569980in}}{\pgfqpoint{5.416273in}{1.566708in}}{\pgfqpoint{5.424510in}{1.566708in}}%
\pgfpathclose%
\pgfusepath{stroke,fill}%
\end{pgfscope}%
\begin{pgfscope}%
\pgfpathrectangle{\pgfqpoint{3.755891in}{0.557870in}}{\pgfqpoint{2.484109in}{1.484734in}}%
\pgfusepath{clip}%
\pgfsetbuttcap%
\pgfsetroundjoin%
\definecolor{currentfill}{rgb}{0.298039,0.447059,0.690196}%
\pgfsetfillcolor{currentfill}%
\pgfsetlinewidth{1.003750pt}%
\definecolor{currentstroke}{rgb}{0.298039,0.447059,0.690196}%
\pgfsetstrokecolor{currentstroke}%
\pgfsetdash{}{0pt}%
\pgfpathmoveto{\pgfqpoint{5.098314in}{1.450600in}}%
\pgfpathcurveto{\pgfqpoint{5.106550in}{1.450600in}}{\pgfqpoint{5.114450in}{1.453872in}}{\pgfqpoint{5.120274in}{1.459696in}}%
\pgfpathcurveto{\pgfqpoint{5.126098in}{1.465520in}}{\pgfqpoint{5.129370in}{1.473420in}}{\pgfqpoint{5.129370in}{1.481656in}}%
\pgfpathcurveto{\pgfqpoint{5.129370in}{1.489893in}}{\pgfqpoint{5.126098in}{1.497793in}}{\pgfqpoint{5.120274in}{1.503617in}}%
\pgfpathcurveto{\pgfqpoint{5.114450in}{1.509441in}}{\pgfqpoint{5.106550in}{1.512713in}}{\pgfqpoint{5.098314in}{1.512713in}}%
\pgfpathcurveto{\pgfqpoint{5.090077in}{1.512713in}}{\pgfqpoint{5.082177in}{1.509441in}}{\pgfqpoint{5.076353in}{1.503617in}}%
\pgfpathcurveto{\pgfqpoint{5.070529in}{1.497793in}}{\pgfqpoint{5.067257in}{1.489893in}}{\pgfqpoint{5.067257in}{1.481656in}}%
\pgfpathcurveto{\pgfqpoint{5.067257in}{1.473420in}}{\pgfqpoint{5.070529in}{1.465520in}}{\pgfqpoint{5.076353in}{1.459696in}}%
\pgfpathcurveto{\pgfqpoint{5.082177in}{1.453872in}}{\pgfqpoint{5.090077in}{1.450600in}}{\pgfqpoint{5.098314in}{1.450600in}}%
\pgfpathclose%
\pgfusepath{stroke,fill}%
\end{pgfscope}%
\begin{pgfscope}%
\pgfpathrectangle{\pgfqpoint{3.755891in}{0.557870in}}{\pgfqpoint{2.484109in}{1.484734in}}%
\pgfusepath{clip}%
\pgfsetbuttcap%
\pgfsetroundjoin%
\definecolor{currentfill}{rgb}{0.298039,0.447059,0.690196}%
\pgfsetfillcolor{currentfill}%
\pgfsetlinewidth{1.003750pt}%
\definecolor{currentstroke}{rgb}{0.298039,0.447059,0.690196}%
\pgfsetstrokecolor{currentstroke}%
\pgfsetdash{}{0pt}%
\pgfpathmoveto{\pgfqpoint{5.424510in}{1.624762in}}%
\pgfpathcurveto{\pgfqpoint{5.432746in}{1.624762in}}{\pgfqpoint{5.440646in}{1.628034in}}{\pgfqpoint{5.446470in}{1.633858in}}%
\pgfpathcurveto{\pgfqpoint{5.452294in}{1.639682in}}{\pgfqpoint{5.455566in}{1.647582in}}{\pgfqpoint{5.455566in}{1.655819in}}%
\pgfpathcurveto{\pgfqpoint{5.455566in}{1.664055in}}{\pgfqpoint{5.452294in}{1.671955in}}{\pgfqpoint{5.446470in}{1.677779in}}%
\pgfpathcurveto{\pgfqpoint{5.440646in}{1.683603in}}{\pgfqpoint{5.432746in}{1.686875in}}{\pgfqpoint{5.424510in}{1.686875in}}%
\pgfpathcurveto{\pgfqpoint{5.416273in}{1.686875in}}{\pgfqpoint{5.408373in}{1.683603in}}{\pgfqpoint{5.402550in}{1.677779in}}%
\pgfpathcurveto{\pgfqpoint{5.396726in}{1.671955in}}{\pgfqpoint{5.393453in}{1.664055in}}{\pgfqpoint{5.393453in}{1.655819in}}%
\pgfpathcurveto{\pgfqpoint{5.393453in}{1.647582in}}{\pgfqpoint{5.396726in}{1.639682in}}{\pgfqpoint{5.402550in}{1.633858in}}%
\pgfpathcurveto{\pgfqpoint{5.408373in}{1.628034in}}{\pgfqpoint{5.416273in}{1.624762in}}{\pgfqpoint{5.424510in}{1.624762in}}%
\pgfpathclose%
\pgfusepath{stroke,fill}%
\end{pgfscope}%
\begin{pgfscope}%
\pgfpathrectangle{\pgfqpoint{3.755891in}{0.557870in}}{\pgfqpoint{2.484109in}{1.484734in}}%
\pgfusepath{clip}%
\pgfsetbuttcap%
\pgfsetroundjoin%
\definecolor{currentfill}{rgb}{0.298039,0.447059,0.690196}%
\pgfsetfillcolor{currentfill}%
\pgfsetlinewidth{1.003750pt}%
\definecolor{currentstroke}{rgb}{0.298039,0.447059,0.690196}%
\pgfsetstrokecolor{currentstroke}%
\pgfsetdash{}{0pt}%
\pgfpathmoveto{\pgfqpoint{5.399418in}{1.610249in}}%
\pgfpathcurveto{\pgfqpoint{5.407654in}{1.610249in}}{\pgfqpoint{5.415554in}{1.613521in}}{\pgfqpoint{5.421378in}{1.619345in}}%
\pgfpathcurveto{\pgfqpoint{5.427202in}{1.625169in}}{\pgfqpoint{5.430474in}{1.633069in}}{\pgfqpoint{5.430474in}{1.641305in}}%
\pgfpathcurveto{\pgfqpoint{5.430474in}{1.649541in}}{\pgfqpoint{5.427202in}{1.657441in}}{\pgfqpoint{5.421378in}{1.663265in}}%
\pgfpathcurveto{\pgfqpoint{5.415554in}{1.669089in}}{\pgfqpoint{5.407654in}{1.672362in}}{\pgfqpoint{5.399418in}{1.672362in}}%
\pgfpathcurveto{\pgfqpoint{5.391181in}{1.672362in}}{\pgfqpoint{5.383281in}{1.669089in}}{\pgfqpoint{5.377458in}{1.663265in}}%
\pgfpathcurveto{\pgfqpoint{5.371634in}{1.657441in}}{\pgfqpoint{5.368361in}{1.649541in}}{\pgfqpoint{5.368361in}{1.641305in}}%
\pgfpathcurveto{\pgfqpoint{5.368361in}{1.633069in}}{\pgfqpoint{5.371634in}{1.625169in}}{\pgfqpoint{5.377458in}{1.619345in}}%
\pgfpathcurveto{\pgfqpoint{5.383281in}{1.613521in}}{\pgfqpoint{5.391181in}{1.610249in}}{\pgfqpoint{5.399418in}{1.610249in}}%
\pgfpathclose%
\pgfusepath{stroke,fill}%
\end{pgfscope}%
\begin{pgfscope}%
\pgfpathrectangle{\pgfqpoint{3.755891in}{0.557870in}}{\pgfqpoint{2.484109in}{1.484734in}}%
\pgfusepath{clip}%
\pgfsetbuttcap%
\pgfsetroundjoin%
\definecolor{currentfill}{rgb}{0.298039,0.447059,0.690196}%
\pgfsetfillcolor{currentfill}%
\pgfsetlinewidth{1.003750pt}%
\definecolor{currentstroke}{rgb}{0.298039,0.447059,0.690196}%
\pgfsetstrokecolor{currentstroke}%
\pgfsetdash{}{0pt}%
\pgfpathmoveto{\pgfqpoint{5.349234in}{1.566708in}}%
\pgfpathcurveto{\pgfqpoint{5.357470in}{1.566708in}}{\pgfqpoint{5.365370in}{1.569980in}}{\pgfqpoint{5.371194in}{1.575804in}}%
\pgfpathcurveto{\pgfqpoint{5.377018in}{1.581628in}}{\pgfqpoint{5.380290in}{1.589528in}}{\pgfqpoint{5.380290in}{1.597765in}}%
\pgfpathcurveto{\pgfqpoint{5.380290in}{1.606001in}}{\pgfqpoint{5.377018in}{1.613901in}}{\pgfqpoint{5.371194in}{1.619725in}}%
\pgfpathcurveto{\pgfqpoint{5.365370in}{1.625549in}}{\pgfqpoint{5.357470in}{1.628821in}}{\pgfqpoint{5.349234in}{1.628821in}}%
\pgfpathcurveto{\pgfqpoint{5.340997in}{1.628821in}}{\pgfqpoint{5.333097in}{1.625549in}}{\pgfqpoint{5.327273in}{1.619725in}}%
\pgfpathcurveto{\pgfqpoint{5.321450in}{1.613901in}}{\pgfqpoint{5.318177in}{1.606001in}}{\pgfqpoint{5.318177in}{1.597765in}}%
\pgfpathcurveto{\pgfqpoint{5.318177in}{1.589528in}}{\pgfqpoint{5.321450in}{1.581628in}}{\pgfqpoint{5.327273in}{1.575804in}}%
\pgfpathcurveto{\pgfqpoint{5.333097in}{1.569980in}}{\pgfqpoint{5.340997in}{1.566708in}}{\pgfqpoint{5.349234in}{1.566708in}}%
\pgfpathclose%
\pgfusepath{stroke,fill}%
\end{pgfscope}%
\begin{pgfscope}%
\pgfpathrectangle{\pgfqpoint{3.755891in}{0.557870in}}{\pgfqpoint{2.484109in}{1.484734in}}%
\pgfusepath{clip}%
\pgfsetbuttcap%
\pgfsetroundjoin%
\definecolor{currentfill}{rgb}{0.298039,0.447059,0.690196}%
\pgfsetfillcolor{currentfill}%
\pgfsetlinewidth{1.003750pt}%
\definecolor{currentstroke}{rgb}{0.298039,0.447059,0.690196}%
\pgfsetstrokecolor{currentstroke}%
\pgfsetdash{}{0pt}%
\pgfpathmoveto{\pgfqpoint{5.349234in}{1.639276in}}%
\pgfpathcurveto{\pgfqpoint{5.357470in}{1.639276in}}{\pgfqpoint{5.365370in}{1.642548in}}{\pgfqpoint{5.371194in}{1.648372in}}%
\pgfpathcurveto{\pgfqpoint{5.377018in}{1.654196in}}{\pgfqpoint{5.380290in}{1.662096in}}{\pgfqpoint{5.380290in}{1.670332in}}%
\pgfpathcurveto{\pgfqpoint{5.380290in}{1.678568in}}{\pgfqpoint{5.377018in}{1.686469in}}{\pgfqpoint{5.371194in}{1.692292in}}%
\pgfpathcurveto{\pgfqpoint{5.365370in}{1.698116in}}{\pgfqpoint{5.357470in}{1.701389in}}{\pgfqpoint{5.349234in}{1.701389in}}%
\pgfpathcurveto{\pgfqpoint{5.340997in}{1.701389in}}{\pgfqpoint{5.333097in}{1.698116in}}{\pgfqpoint{5.327273in}{1.692292in}}%
\pgfpathcurveto{\pgfqpoint{5.321450in}{1.686469in}}{\pgfqpoint{5.318177in}{1.678568in}}{\pgfqpoint{5.318177in}{1.670332in}}%
\pgfpathcurveto{\pgfqpoint{5.318177in}{1.662096in}}{\pgfqpoint{5.321450in}{1.654196in}}{\pgfqpoint{5.327273in}{1.648372in}}%
\pgfpathcurveto{\pgfqpoint{5.333097in}{1.642548in}}{\pgfqpoint{5.340997in}{1.639276in}}{\pgfqpoint{5.349234in}{1.639276in}}%
\pgfpathclose%
\pgfusepath{stroke,fill}%
\end{pgfscope}%
\begin{pgfscope}%
\pgfpathrectangle{\pgfqpoint{3.755891in}{0.557870in}}{\pgfqpoint{2.484109in}{1.484734in}}%
\pgfusepath{clip}%
\pgfsetbuttcap%
\pgfsetroundjoin%
\definecolor{currentfill}{rgb}{0.298039,0.447059,0.690196}%
\pgfsetfillcolor{currentfill}%
\pgfsetlinewidth{1.003750pt}%
\definecolor{currentstroke}{rgb}{0.298039,0.447059,0.690196}%
\pgfsetstrokecolor{currentstroke}%
\pgfsetdash{}{0pt}%
\pgfpathmoveto{\pgfqpoint{5.449602in}{1.610249in}}%
\pgfpathcurveto{\pgfqpoint{5.457838in}{1.610249in}}{\pgfqpoint{5.465738in}{1.613521in}}{\pgfqpoint{5.471562in}{1.619345in}}%
\pgfpathcurveto{\pgfqpoint{5.477386in}{1.625169in}}{\pgfqpoint{5.480658in}{1.633069in}}{\pgfqpoint{5.480658in}{1.641305in}}%
\pgfpathcurveto{\pgfqpoint{5.480658in}{1.649541in}}{\pgfqpoint{5.477386in}{1.657441in}}{\pgfqpoint{5.471562in}{1.663265in}}%
\pgfpathcurveto{\pgfqpoint{5.465738in}{1.669089in}}{\pgfqpoint{5.457838in}{1.672362in}}{\pgfqpoint{5.449602in}{1.672362in}}%
\pgfpathcurveto{\pgfqpoint{5.441366in}{1.672362in}}{\pgfqpoint{5.433465in}{1.669089in}}{\pgfqpoint{5.427642in}{1.663265in}}%
\pgfpathcurveto{\pgfqpoint{5.421818in}{1.657441in}}{\pgfqpoint{5.418545in}{1.649541in}}{\pgfqpoint{5.418545in}{1.641305in}}%
\pgfpathcurveto{\pgfqpoint{5.418545in}{1.633069in}}{\pgfqpoint{5.421818in}{1.625169in}}{\pgfqpoint{5.427642in}{1.619345in}}%
\pgfpathcurveto{\pgfqpoint{5.433465in}{1.613521in}}{\pgfqpoint{5.441366in}{1.610249in}}{\pgfqpoint{5.449602in}{1.610249in}}%
\pgfpathclose%
\pgfusepath{stroke,fill}%
\end{pgfscope}%
\begin{pgfscope}%
\pgfpathrectangle{\pgfqpoint{3.755891in}{0.557870in}}{\pgfqpoint{2.484109in}{1.484734in}}%
\pgfusepath{clip}%
\pgfsetbuttcap%
\pgfsetroundjoin%
\definecolor{currentfill}{rgb}{0.298039,0.447059,0.690196}%
\pgfsetfillcolor{currentfill}%
\pgfsetlinewidth{1.003750pt}%
\definecolor{currentstroke}{rgb}{0.298039,0.447059,0.690196}%
\pgfsetstrokecolor{currentstroke}%
\pgfsetdash{}{0pt}%
\pgfpathmoveto{\pgfqpoint{5.349234in}{1.378032in}}%
\pgfpathcurveto{\pgfqpoint{5.357470in}{1.378032in}}{\pgfqpoint{5.365370in}{1.381304in}}{\pgfqpoint{5.371194in}{1.387128in}}%
\pgfpathcurveto{\pgfqpoint{5.377018in}{1.392952in}}{\pgfqpoint{5.380290in}{1.400852in}}{\pgfqpoint{5.380290in}{1.409089in}}%
\pgfpathcurveto{\pgfqpoint{5.380290in}{1.417325in}}{\pgfqpoint{5.377018in}{1.425225in}}{\pgfqpoint{5.371194in}{1.431049in}}%
\pgfpathcurveto{\pgfqpoint{5.365370in}{1.436873in}}{\pgfqpoint{5.357470in}{1.440145in}}{\pgfqpoint{5.349234in}{1.440145in}}%
\pgfpathcurveto{\pgfqpoint{5.340997in}{1.440145in}}{\pgfqpoint{5.333097in}{1.436873in}}{\pgfqpoint{5.327273in}{1.431049in}}%
\pgfpathcurveto{\pgfqpoint{5.321450in}{1.425225in}}{\pgfqpoint{5.318177in}{1.417325in}}{\pgfqpoint{5.318177in}{1.409089in}}%
\pgfpathcurveto{\pgfqpoint{5.318177in}{1.400852in}}{\pgfqpoint{5.321450in}{1.392952in}}{\pgfqpoint{5.327273in}{1.387128in}}%
\pgfpathcurveto{\pgfqpoint{5.333097in}{1.381304in}}{\pgfqpoint{5.340997in}{1.378032in}}{\pgfqpoint{5.349234in}{1.378032in}}%
\pgfpathclose%
\pgfusepath{stroke,fill}%
\end{pgfscope}%
\begin{pgfscope}%
\pgfpathrectangle{\pgfqpoint{3.755891in}{0.557870in}}{\pgfqpoint{2.484109in}{1.484734in}}%
\pgfusepath{clip}%
\pgfsetbuttcap%
\pgfsetroundjoin%
\definecolor{currentfill}{rgb}{0.298039,0.447059,0.690196}%
\pgfsetfillcolor{currentfill}%
\pgfsetlinewidth{1.003750pt}%
\definecolor{currentstroke}{rgb}{0.298039,0.447059,0.690196}%
\pgfsetstrokecolor{currentstroke}%
\pgfsetdash{}{0pt}%
\pgfpathmoveto{\pgfqpoint{5.556813in}{1.191037in}}%
\pgfpathcurveto{\pgfqpoint{5.565049in}{1.191037in}}{\pgfqpoint{5.572949in}{1.194309in}}{\pgfqpoint{5.578773in}{1.200133in}}%
\pgfpathcurveto{\pgfqpoint{5.584597in}{1.205957in}}{\pgfqpoint{5.587870in}{1.213857in}}{\pgfqpoint{5.587870in}{1.222093in}}%
\pgfpathcurveto{\pgfqpoint{5.587870in}{1.230330in}}{\pgfqpoint{5.584597in}{1.238230in}}{\pgfqpoint{5.578773in}{1.244054in}}%
\pgfpathcurveto{\pgfqpoint{5.572949in}{1.249878in}}{\pgfqpoint{5.565049in}{1.253150in}}{\pgfqpoint{5.556813in}{1.253150in}}%
\pgfpathcurveto{\pgfqpoint{5.548577in}{1.253150in}}{\pgfqpoint{5.540677in}{1.249878in}}{\pgfqpoint{5.534853in}{1.244054in}}%
\pgfpathcurveto{\pgfqpoint{5.529029in}{1.238230in}}{\pgfqpoint{5.525757in}{1.230330in}}{\pgfqpoint{5.525757in}{1.222093in}}%
\pgfpathcurveto{\pgfqpoint{5.525757in}{1.213857in}}{\pgfqpoint{5.529029in}{1.205957in}}{\pgfqpoint{5.534853in}{1.200133in}}%
\pgfpathcurveto{\pgfqpoint{5.540677in}{1.194309in}}{\pgfqpoint{5.548577in}{1.191037in}}{\pgfqpoint{5.556813in}{1.191037in}}%
\pgfpathclose%
\pgfusepath{stroke,fill}%
\end{pgfscope}%
\begin{pgfscope}%
\pgfpathrectangle{\pgfqpoint{3.755891in}{0.557870in}}{\pgfqpoint{2.484109in}{1.484734in}}%
\pgfusepath{clip}%
\pgfsetbuttcap%
\pgfsetroundjoin%
\definecolor{currentfill}{rgb}{0.298039,0.447059,0.690196}%
\pgfsetfillcolor{currentfill}%
\pgfsetlinewidth{1.003750pt}%
\definecolor{currentstroke}{rgb}{0.298039,0.447059,0.690196}%
\pgfsetstrokecolor{currentstroke}%
\pgfsetdash{}{0pt}%
\pgfpathmoveto{\pgfqpoint{5.534002in}{1.588860in}}%
\pgfpathcurveto{\pgfqpoint{5.542238in}{1.588860in}}{\pgfqpoint{5.550139in}{1.592133in}}{\pgfqpoint{5.555962in}{1.597957in}}%
\pgfpathcurveto{\pgfqpoint{5.561786in}{1.603780in}}{\pgfqpoint{5.565059in}{1.611680in}}{\pgfqpoint{5.565059in}{1.619917in}}%
\pgfpathcurveto{\pgfqpoint{5.565059in}{1.628153in}}{\pgfqpoint{5.561786in}{1.636053in}}{\pgfqpoint{5.555962in}{1.641877in}}%
\pgfpathcurveto{\pgfqpoint{5.550139in}{1.647701in}}{\pgfqpoint{5.542238in}{1.650973in}}{\pgfqpoint{5.534002in}{1.650973in}}%
\pgfpathcurveto{\pgfqpoint{5.525766in}{1.650973in}}{\pgfqpoint{5.517866in}{1.647701in}}{\pgfqpoint{5.512042in}{1.641877in}}%
\pgfpathcurveto{\pgfqpoint{5.506218in}{1.636053in}}{\pgfqpoint{5.502946in}{1.628153in}}{\pgfqpoint{5.502946in}{1.619917in}}%
\pgfpathcurveto{\pgfqpoint{5.502946in}{1.611680in}}{\pgfqpoint{5.506218in}{1.603780in}}{\pgfqpoint{5.512042in}{1.597957in}}%
\pgfpathcurveto{\pgfqpoint{5.517866in}{1.592133in}}{\pgfqpoint{5.525766in}{1.588860in}}{\pgfqpoint{5.534002in}{1.588860in}}%
\pgfpathclose%
\pgfusepath{stroke,fill}%
\end{pgfscope}%
\begin{pgfscope}%
\pgfpathrectangle{\pgfqpoint{3.755891in}{0.557870in}}{\pgfqpoint{2.484109in}{1.484734in}}%
\pgfusepath{clip}%
\pgfsetbuttcap%
\pgfsetroundjoin%
\definecolor{currentfill}{rgb}{0.298039,0.447059,0.690196}%
\pgfsetfillcolor{currentfill}%
\pgfsetlinewidth{1.003750pt}%
\definecolor{currentstroke}{rgb}{0.298039,0.447059,0.690196}%
\pgfsetstrokecolor{currentstroke}%
\pgfsetdash{}{0pt}%
\pgfpathmoveto{\pgfqpoint{5.556813in}{1.603068in}}%
\pgfpathcurveto{\pgfqpoint{5.565049in}{1.603068in}}{\pgfqpoint{5.572949in}{1.606341in}}{\pgfqpoint{5.578773in}{1.612164in}}%
\pgfpathcurveto{\pgfqpoint{5.584597in}{1.617988in}}{\pgfqpoint{5.587870in}{1.625888in}}{\pgfqpoint{5.587870in}{1.634125in}}%
\pgfpathcurveto{\pgfqpoint{5.587870in}{1.642361in}}{\pgfqpoint{5.584597in}{1.650261in}}{\pgfqpoint{5.578773in}{1.656085in}}%
\pgfpathcurveto{\pgfqpoint{5.572949in}{1.661909in}}{\pgfqpoint{5.565049in}{1.665181in}}{\pgfqpoint{5.556813in}{1.665181in}}%
\pgfpathcurveto{\pgfqpoint{5.548577in}{1.665181in}}{\pgfqpoint{5.540677in}{1.661909in}}{\pgfqpoint{5.534853in}{1.656085in}}%
\pgfpathcurveto{\pgfqpoint{5.529029in}{1.650261in}}{\pgfqpoint{5.525757in}{1.642361in}}{\pgfqpoint{5.525757in}{1.634125in}}%
\pgfpathcurveto{\pgfqpoint{5.525757in}{1.625888in}}{\pgfqpoint{5.529029in}{1.617988in}}{\pgfqpoint{5.534853in}{1.612164in}}%
\pgfpathcurveto{\pgfqpoint{5.540677in}{1.606341in}}{\pgfqpoint{5.548577in}{1.603068in}}{\pgfqpoint{5.556813in}{1.603068in}}%
\pgfpathclose%
\pgfusepath{stroke,fill}%
\end{pgfscope}%
\begin{pgfscope}%
\pgfpathrectangle{\pgfqpoint{3.755891in}{0.557870in}}{\pgfqpoint{2.484109in}{1.484734in}}%
\pgfusepath{clip}%
\pgfsetbuttcap%
\pgfsetroundjoin%
\definecolor{currentfill}{rgb}{0.298039,0.447059,0.690196}%
\pgfsetfillcolor{currentfill}%
\pgfsetlinewidth{1.003750pt}%
\definecolor{currentstroke}{rgb}{0.298039,0.447059,0.690196}%
\pgfsetstrokecolor{currentstroke}%
\pgfsetdash{}{0pt}%
\pgfpathmoveto{\pgfqpoint{6.035842in}{0.722174in}}%
\pgfpathcurveto{\pgfqpoint{6.044079in}{0.722174in}}{\pgfqpoint{6.051979in}{0.725446in}}{\pgfqpoint{6.057803in}{0.731270in}}%
\pgfpathcurveto{\pgfqpoint{6.063626in}{0.737094in}}{\pgfqpoint{6.066899in}{0.744994in}}{\pgfqpoint{6.066899in}{0.753230in}}%
\pgfpathcurveto{\pgfqpoint{6.066899in}{0.761466in}}{\pgfqpoint{6.063626in}{0.769366in}}{\pgfqpoint{6.057803in}{0.775190in}}%
\pgfpathcurveto{\pgfqpoint{6.051979in}{0.781014in}}{\pgfqpoint{6.044079in}{0.784286in}}{\pgfqpoint{6.035842in}{0.784286in}}%
\pgfpathcurveto{\pgfqpoint{6.027606in}{0.784286in}}{\pgfqpoint{6.019706in}{0.781014in}}{\pgfqpoint{6.013882in}{0.775190in}}%
\pgfpathcurveto{\pgfqpoint{6.008058in}{0.769366in}}{\pgfqpoint{6.004786in}{0.761466in}}{\pgfqpoint{6.004786in}{0.753230in}}%
\pgfpathcurveto{\pgfqpoint{6.004786in}{0.744994in}}{\pgfqpoint{6.008058in}{0.737094in}}{\pgfqpoint{6.013882in}{0.731270in}}%
\pgfpathcurveto{\pgfqpoint{6.019706in}{0.725446in}}{\pgfqpoint{6.027606in}{0.722174in}}{\pgfqpoint{6.035842in}{0.722174in}}%
\pgfpathclose%
\pgfusepath{stroke,fill}%
\end{pgfscope}%
\begin{pgfscope}%
\pgfpathrectangle{\pgfqpoint{3.755891in}{0.557870in}}{\pgfqpoint{2.484109in}{1.484734in}}%
\pgfusepath{clip}%
\pgfsetbuttcap%
\pgfsetroundjoin%
\definecolor{currentfill}{rgb}{0.298039,0.447059,0.690196}%
\pgfsetfillcolor{currentfill}%
\pgfsetlinewidth{1.003750pt}%
\definecolor{currentstroke}{rgb}{0.298039,0.447059,0.690196}%
\pgfsetstrokecolor{currentstroke}%
\pgfsetdash{}{0pt}%
\pgfpathmoveto{\pgfqpoint{5.511191in}{1.645692in}}%
\pgfpathcurveto{\pgfqpoint{5.519428in}{1.645692in}}{\pgfqpoint{5.527328in}{1.648964in}}{\pgfqpoint{5.533152in}{1.654788in}}%
\pgfpathcurveto{\pgfqpoint{5.538975in}{1.660612in}}{\pgfqpoint{5.542248in}{1.668512in}}{\pgfqpoint{5.542248in}{1.676749in}}%
\pgfpathcurveto{\pgfqpoint{5.542248in}{1.684985in}}{\pgfqpoint{5.538975in}{1.692885in}}{\pgfqpoint{5.533152in}{1.698709in}}%
\pgfpathcurveto{\pgfqpoint{5.527328in}{1.704533in}}{\pgfqpoint{5.519428in}{1.707805in}}{\pgfqpoint{5.511191in}{1.707805in}}%
\pgfpathcurveto{\pgfqpoint{5.502955in}{1.707805in}}{\pgfqpoint{5.495055in}{1.704533in}}{\pgfqpoint{5.489231in}{1.698709in}}%
\pgfpathcurveto{\pgfqpoint{5.483407in}{1.692885in}}{\pgfqpoint{5.480135in}{1.684985in}}{\pgfqpoint{5.480135in}{1.676749in}}%
\pgfpathcurveto{\pgfqpoint{5.480135in}{1.668512in}}{\pgfqpoint{5.483407in}{1.660612in}}{\pgfqpoint{5.489231in}{1.654788in}}%
\pgfpathcurveto{\pgfqpoint{5.495055in}{1.648964in}}{\pgfqpoint{5.502955in}{1.645692in}}{\pgfqpoint{5.511191in}{1.645692in}}%
\pgfpathclose%
\pgfusepath{stroke,fill}%
\end{pgfscope}%
\begin{pgfscope}%
\pgfpathrectangle{\pgfqpoint{3.755891in}{0.557870in}}{\pgfqpoint{2.484109in}{1.484734in}}%
\pgfusepath{clip}%
\pgfsetbuttcap%
\pgfsetroundjoin%
\definecolor{currentfill}{rgb}{0.298039,0.447059,0.690196}%
\pgfsetfillcolor{currentfill}%
\pgfsetlinewidth{1.003750pt}%
\definecolor{currentstroke}{rgb}{0.298039,0.447059,0.690196}%
\pgfsetstrokecolor{currentstroke}%
\pgfsetdash{}{0pt}%
\pgfpathmoveto{\pgfqpoint{5.488380in}{1.546236in}}%
\pgfpathcurveto{\pgfqpoint{5.496617in}{1.546236in}}{\pgfqpoint{5.504517in}{1.549509in}}{\pgfqpoint{5.510341in}{1.555333in}}%
\pgfpathcurveto{\pgfqpoint{5.516165in}{1.561156in}}{\pgfqpoint{5.519437in}{1.569057in}}{\pgfqpoint{5.519437in}{1.577293in}}%
\pgfpathcurveto{\pgfqpoint{5.519437in}{1.585529in}}{\pgfqpoint{5.516165in}{1.593429in}}{\pgfqpoint{5.510341in}{1.599253in}}%
\pgfpathcurveto{\pgfqpoint{5.504517in}{1.605077in}}{\pgfqpoint{5.496617in}{1.608349in}}{\pgfqpoint{5.488380in}{1.608349in}}%
\pgfpathcurveto{\pgfqpoint{5.480144in}{1.608349in}}{\pgfqpoint{5.472244in}{1.605077in}}{\pgfqpoint{5.466420in}{1.599253in}}%
\pgfpathcurveto{\pgfqpoint{5.460596in}{1.593429in}}{\pgfqpoint{5.457324in}{1.585529in}}{\pgfqpoint{5.457324in}{1.577293in}}%
\pgfpathcurveto{\pgfqpoint{5.457324in}{1.569057in}}{\pgfqpoint{5.460596in}{1.561156in}}{\pgfqpoint{5.466420in}{1.555333in}}%
\pgfpathcurveto{\pgfqpoint{5.472244in}{1.549509in}}{\pgfqpoint{5.480144in}{1.546236in}}{\pgfqpoint{5.488380in}{1.546236in}}%
\pgfpathclose%
\pgfusepath{stroke,fill}%
\end{pgfscope}%
\begin{pgfscope}%
\pgfpathrectangle{\pgfqpoint{3.755891in}{0.557870in}}{\pgfqpoint{2.484109in}{1.484734in}}%
\pgfusepath{clip}%
\pgfsetbuttcap%
\pgfsetroundjoin%
\definecolor{currentfill}{rgb}{0.298039,0.447059,0.690196}%
\pgfsetfillcolor{currentfill}%
\pgfsetlinewidth{1.003750pt}%
\definecolor{currentstroke}{rgb}{0.298039,0.447059,0.690196}%
\pgfsetstrokecolor{currentstroke}%
\pgfsetdash{}{0pt}%
\pgfpathmoveto{\pgfqpoint{5.511191in}{1.631484in}}%
\pgfpathcurveto{\pgfqpoint{5.519428in}{1.631484in}}{\pgfqpoint{5.527328in}{1.634757in}}{\pgfqpoint{5.533152in}{1.640580in}}%
\pgfpathcurveto{\pgfqpoint{5.538975in}{1.646404in}}{\pgfqpoint{5.542248in}{1.654304in}}{\pgfqpoint{5.542248in}{1.662541in}}%
\pgfpathcurveto{\pgfqpoint{5.542248in}{1.670777in}}{\pgfqpoint{5.538975in}{1.678677in}}{\pgfqpoint{5.533152in}{1.684501in}}%
\pgfpathcurveto{\pgfqpoint{5.527328in}{1.690325in}}{\pgfqpoint{5.519428in}{1.693597in}}{\pgfqpoint{5.511191in}{1.693597in}}%
\pgfpathcurveto{\pgfqpoint{5.502955in}{1.693597in}}{\pgfqpoint{5.495055in}{1.690325in}}{\pgfqpoint{5.489231in}{1.684501in}}%
\pgfpathcurveto{\pgfqpoint{5.483407in}{1.678677in}}{\pgfqpoint{5.480135in}{1.670777in}}{\pgfqpoint{5.480135in}{1.662541in}}%
\pgfpathcurveto{\pgfqpoint{5.480135in}{1.654304in}}{\pgfqpoint{5.483407in}{1.646404in}}{\pgfqpoint{5.489231in}{1.640580in}}%
\pgfpathcurveto{\pgfqpoint{5.495055in}{1.634757in}}{\pgfqpoint{5.502955in}{1.631484in}}{\pgfqpoint{5.511191in}{1.631484in}}%
\pgfpathclose%
\pgfusepath{stroke,fill}%
\end{pgfscope}%
\begin{pgfscope}%
\pgfpathrectangle{\pgfqpoint{3.755891in}{0.557870in}}{\pgfqpoint{2.484109in}{1.484734in}}%
\pgfusepath{clip}%
\pgfsetbuttcap%
\pgfsetroundjoin%
\definecolor{currentfill}{rgb}{0.298039,0.447059,0.690196}%
\pgfsetfillcolor{currentfill}%
\pgfsetlinewidth{1.003750pt}%
\definecolor{currentstroke}{rgb}{0.298039,0.447059,0.690196}%
\pgfsetstrokecolor{currentstroke}%
\pgfsetdash{}{0pt}%
\pgfpathmoveto{\pgfqpoint{5.419948in}{1.645692in}}%
\pgfpathcurveto{\pgfqpoint{5.428184in}{1.645692in}}{\pgfqpoint{5.436084in}{1.648964in}}{\pgfqpoint{5.441908in}{1.654788in}}%
\pgfpathcurveto{\pgfqpoint{5.447732in}{1.660612in}}{\pgfqpoint{5.451004in}{1.668512in}}{\pgfqpoint{5.451004in}{1.676749in}}%
\pgfpathcurveto{\pgfqpoint{5.451004in}{1.684985in}}{\pgfqpoint{5.447732in}{1.692885in}}{\pgfqpoint{5.441908in}{1.698709in}}%
\pgfpathcurveto{\pgfqpoint{5.436084in}{1.704533in}}{\pgfqpoint{5.428184in}{1.707805in}}{\pgfqpoint{5.419948in}{1.707805in}}%
\pgfpathcurveto{\pgfqpoint{5.411711in}{1.707805in}}{\pgfqpoint{5.403811in}{1.704533in}}{\pgfqpoint{5.397987in}{1.698709in}}%
\pgfpathcurveto{\pgfqpoint{5.392163in}{1.692885in}}{\pgfqpoint{5.388891in}{1.684985in}}{\pgfqpoint{5.388891in}{1.676749in}}%
\pgfpathcurveto{\pgfqpoint{5.388891in}{1.668512in}}{\pgfqpoint{5.392163in}{1.660612in}}{\pgfqpoint{5.397987in}{1.654788in}}%
\pgfpathcurveto{\pgfqpoint{5.403811in}{1.648964in}}{\pgfqpoint{5.411711in}{1.645692in}}{\pgfqpoint{5.419948in}{1.645692in}}%
\pgfpathclose%
\pgfusepath{stroke,fill}%
\end{pgfscope}%
\begin{pgfscope}%
\pgfpathrectangle{\pgfqpoint{3.755891in}{0.557870in}}{\pgfqpoint{2.484109in}{1.484734in}}%
\pgfusepath{clip}%
\pgfsetbuttcap%
\pgfsetroundjoin%
\definecolor{currentfill}{rgb}{0.298039,0.447059,0.690196}%
\pgfsetfillcolor{currentfill}%
\pgfsetlinewidth{1.003750pt}%
\definecolor{currentstroke}{rgb}{0.298039,0.447059,0.690196}%
\pgfsetstrokecolor{currentstroke}%
\pgfsetdash{}{0pt}%
\pgfpathmoveto{\pgfqpoint{5.511191in}{1.574652in}}%
\pgfpathcurveto{\pgfqpoint{5.519428in}{1.574652in}}{\pgfqpoint{5.527328in}{1.577925in}}{\pgfqpoint{5.533152in}{1.583749in}}%
\pgfpathcurveto{\pgfqpoint{5.538975in}{1.589572in}}{\pgfqpoint{5.542248in}{1.597473in}}{\pgfqpoint{5.542248in}{1.605709in}}%
\pgfpathcurveto{\pgfqpoint{5.542248in}{1.613945in}}{\pgfqpoint{5.538975in}{1.621845in}}{\pgfqpoint{5.533152in}{1.627669in}}%
\pgfpathcurveto{\pgfqpoint{5.527328in}{1.633493in}}{\pgfqpoint{5.519428in}{1.636765in}}{\pgfqpoint{5.511191in}{1.636765in}}%
\pgfpathcurveto{\pgfqpoint{5.502955in}{1.636765in}}{\pgfqpoint{5.495055in}{1.633493in}}{\pgfqpoint{5.489231in}{1.627669in}}%
\pgfpathcurveto{\pgfqpoint{5.483407in}{1.621845in}}{\pgfqpoint{5.480135in}{1.613945in}}{\pgfqpoint{5.480135in}{1.605709in}}%
\pgfpathcurveto{\pgfqpoint{5.480135in}{1.597473in}}{\pgfqpoint{5.483407in}{1.589572in}}{\pgfqpoint{5.489231in}{1.583749in}}%
\pgfpathcurveto{\pgfqpoint{5.495055in}{1.577925in}}{\pgfqpoint{5.502955in}{1.574652in}}{\pgfqpoint{5.511191in}{1.574652in}}%
\pgfpathclose%
\pgfusepath{stroke,fill}%
\end{pgfscope}%
\begin{pgfscope}%
\pgfpathrectangle{\pgfqpoint{3.755891in}{0.557870in}}{\pgfqpoint{2.484109in}{1.484734in}}%
\pgfusepath{clip}%
\pgfsetbuttcap%
\pgfsetroundjoin%
\definecolor{currentfill}{rgb}{0.298039,0.447059,0.690196}%
\pgfsetfillcolor{currentfill}%
\pgfsetlinewidth{1.003750pt}%
\definecolor{currentstroke}{rgb}{0.298039,0.447059,0.690196}%
\pgfsetstrokecolor{currentstroke}%
\pgfsetdash{}{0pt}%
\pgfpathmoveto{\pgfqpoint{5.534002in}{1.617276in}}%
\pgfpathcurveto{\pgfqpoint{5.542238in}{1.617276in}}{\pgfqpoint{5.550139in}{1.620549in}}{\pgfqpoint{5.555962in}{1.626372in}}%
\pgfpathcurveto{\pgfqpoint{5.561786in}{1.632196in}}{\pgfqpoint{5.565059in}{1.640096in}}{\pgfqpoint{5.565059in}{1.648333in}}%
\pgfpathcurveto{\pgfqpoint{5.565059in}{1.656569in}}{\pgfqpoint{5.561786in}{1.664469in}}{\pgfqpoint{5.555962in}{1.670293in}}%
\pgfpathcurveto{\pgfqpoint{5.550139in}{1.676117in}}{\pgfqpoint{5.542238in}{1.679389in}}{\pgfqpoint{5.534002in}{1.679389in}}%
\pgfpathcurveto{\pgfqpoint{5.525766in}{1.679389in}}{\pgfqpoint{5.517866in}{1.676117in}}{\pgfqpoint{5.512042in}{1.670293in}}%
\pgfpathcurveto{\pgfqpoint{5.506218in}{1.664469in}}{\pgfqpoint{5.502946in}{1.656569in}}{\pgfqpoint{5.502946in}{1.648333in}}%
\pgfpathcurveto{\pgfqpoint{5.502946in}{1.640096in}}{\pgfqpoint{5.506218in}{1.632196in}}{\pgfqpoint{5.512042in}{1.626372in}}%
\pgfpathcurveto{\pgfqpoint{5.517866in}{1.620549in}}{\pgfqpoint{5.525766in}{1.617276in}}{\pgfqpoint{5.534002in}{1.617276in}}%
\pgfpathclose%
\pgfusepath{stroke,fill}%
\end{pgfscope}%
\begin{pgfscope}%
\pgfpathrectangle{\pgfqpoint{3.755891in}{0.557870in}}{\pgfqpoint{2.484109in}{1.484734in}}%
\pgfusepath{clip}%
\pgfsetbuttcap%
\pgfsetroundjoin%
\definecolor{currentfill}{rgb}{0.298039,0.447059,0.690196}%
\pgfsetfillcolor{currentfill}%
\pgfsetlinewidth{1.003750pt}%
\definecolor{currentstroke}{rgb}{0.298039,0.447059,0.690196}%
\pgfsetstrokecolor{currentstroke}%
\pgfsetdash{}{0pt}%
\pgfpathmoveto{\pgfqpoint{5.556813in}{1.546236in}}%
\pgfpathcurveto{\pgfqpoint{5.565049in}{1.546236in}}{\pgfqpoint{5.572949in}{1.549509in}}{\pgfqpoint{5.578773in}{1.555333in}}%
\pgfpathcurveto{\pgfqpoint{5.584597in}{1.561156in}}{\pgfqpoint{5.587870in}{1.569057in}}{\pgfqpoint{5.587870in}{1.577293in}}%
\pgfpathcurveto{\pgfqpoint{5.587870in}{1.585529in}}{\pgfqpoint{5.584597in}{1.593429in}}{\pgfqpoint{5.578773in}{1.599253in}}%
\pgfpathcurveto{\pgfqpoint{5.572949in}{1.605077in}}{\pgfqpoint{5.565049in}{1.608349in}}{\pgfqpoint{5.556813in}{1.608349in}}%
\pgfpathcurveto{\pgfqpoint{5.548577in}{1.608349in}}{\pgfqpoint{5.540677in}{1.605077in}}{\pgfqpoint{5.534853in}{1.599253in}}%
\pgfpathcurveto{\pgfqpoint{5.529029in}{1.593429in}}{\pgfqpoint{5.525757in}{1.585529in}}{\pgfqpoint{5.525757in}{1.577293in}}%
\pgfpathcurveto{\pgfqpoint{5.525757in}{1.569057in}}{\pgfqpoint{5.529029in}{1.561156in}}{\pgfqpoint{5.534853in}{1.555333in}}%
\pgfpathcurveto{\pgfqpoint{5.540677in}{1.549509in}}{\pgfqpoint{5.548577in}{1.546236in}}{\pgfqpoint{5.556813in}{1.546236in}}%
\pgfpathclose%
\pgfusepath{stroke,fill}%
\end{pgfscope}%
\begin{pgfscope}%
\pgfpathrectangle{\pgfqpoint{3.755891in}{0.557870in}}{\pgfqpoint{2.484109in}{1.484734in}}%
\pgfusepath{clip}%
\pgfsetbuttcap%
\pgfsetroundjoin%
\definecolor{currentfill}{rgb}{0.298039,0.447059,0.690196}%
\pgfsetfillcolor{currentfill}%
\pgfsetlinewidth{1.003750pt}%
\definecolor{currentstroke}{rgb}{0.298039,0.447059,0.690196}%
\pgfsetstrokecolor{currentstroke}%
\pgfsetdash{}{0pt}%
\pgfpathmoveto{\pgfqpoint{5.549970in}{1.624014in}}%
\pgfpathcurveto{\pgfqpoint{5.558206in}{1.624014in}}{\pgfqpoint{5.566106in}{1.627286in}}{\pgfqpoint{5.571930in}{1.633110in}}%
\pgfpathcurveto{\pgfqpoint{5.577754in}{1.638934in}}{\pgfqpoint{5.581026in}{1.646834in}}{\pgfqpoint{5.581026in}{1.655071in}}%
\pgfpathcurveto{\pgfqpoint{5.581026in}{1.663307in}}{\pgfqpoint{5.577754in}{1.671207in}}{\pgfqpoint{5.571930in}{1.677031in}}%
\pgfpathcurveto{\pgfqpoint{5.566106in}{1.682855in}}{\pgfqpoint{5.558206in}{1.686127in}}{\pgfqpoint{5.549970in}{1.686127in}}%
\pgfpathcurveto{\pgfqpoint{5.541734in}{1.686127in}}{\pgfqpoint{5.533833in}{1.682855in}}{\pgfqpoint{5.528010in}{1.677031in}}%
\pgfpathcurveto{\pgfqpoint{5.522186in}{1.671207in}}{\pgfqpoint{5.518913in}{1.663307in}}{\pgfqpoint{5.518913in}{1.655071in}}%
\pgfpathcurveto{\pgfqpoint{5.518913in}{1.646834in}}{\pgfqpoint{5.522186in}{1.638934in}}{\pgfqpoint{5.528010in}{1.633110in}}%
\pgfpathcurveto{\pgfqpoint{5.533833in}{1.627286in}}{\pgfqpoint{5.541734in}{1.624014in}}{\pgfqpoint{5.549970in}{1.624014in}}%
\pgfpathclose%
\pgfusepath{stroke,fill}%
\end{pgfscope}%
\begin{pgfscope}%
\pgfpathrectangle{\pgfqpoint{3.755891in}{0.557870in}}{\pgfqpoint{2.484109in}{1.484734in}}%
\pgfusepath{clip}%
\pgfsetbuttcap%
\pgfsetroundjoin%
\definecolor{currentfill}{rgb}{0.298039,0.447059,0.690196}%
\pgfsetfillcolor{currentfill}%
\pgfsetlinewidth{1.003750pt}%
\definecolor{currentstroke}{rgb}{0.298039,0.447059,0.690196}%
\pgfsetstrokecolor{currentstroke}%
\pgfsetdash{}{0pt}%
\pgfpathmoveto{\pgfqpoint{5.299050in}{1.637929in}}%
\pgfpathcurveto{\pgfqpoint{5.307286in}{1.637929in}}{\pgfqpoint{5.315186in}{1.641201in}}{\pgfqpoint{5.321010in}{1.647025in}}%
\pgfpathcurveto{\pgfqpoint{5.326834in}{1.652849in}}{\pgfqpoint{5.330106in}{1.660749in}}{\pgfqpoint{5.330106in}{1.668986in}}%
\pgfpathcurveto{\pgfqpoint{5.330106in}{1.677222in}}{\pgfqpoint{5.326834in}{1.685122in}}{\pgfqpoint{5.321010in}{1.690946in}}%
\pgfpathcurveto{\pgfqpoint{5.315186in}{1.696770in}}{\pgfqpoint{5.307286in}{1.700042in}}{\pgfqpoint{5.299050in}{1.700042in}}%
\pgfpathcurveto{\pgfqpoint{5.290813in}{1.700042in}}{\pgfqpoint{5.282913in}{1.696770in}}{\pgfqpoint{5.277089in}{1.690946in}}%
\pgfpathcurveto{\pgfqpoint{5.271266in}{1.685122in}}{\pgfqpoint{5.267993in}{1.677222in}}{\pgfqpoint{5.267993in}{1.668986in}}%
\pgfpathcurveto{\pgfqpoint{5.267993in}{1.660749in}}{\pgfqpoint{5.271266in}{1.652849in}}{\pgfqpoint{5.277089in}{1.647025in}}%
\pgfpathcurveto{\pgfqpoint{5.282913in}{1.641201in}}{\pgfqpoint{5.290813in}{1.637929in}}{\pgfqpoint{5.299050in}{1.637929in}}%
\pgfpathclose%
\pgfusepath{stroke,fill}%
\end{pgfscope}%
\begin{pgfscope}%
\pgfpathrectangle{\pgfqpoint{3.755891in}{0.557870in}}{\pgfqpoint{2.484109in}{1.484734in}}%
\pgfusepath{clip}%
\pgfsetbuttcap%
\pgfsetroundjoin%
\definecolor{currentfill}{rgb}{0.298039,0.447059,0.690196}%
\pgfsetfillcolor{currentfill}%
\pgfsetlinewidth{1.003750pt}%
\definecolor{currentstroke}{rgb}{0.298039,0.447059,0.690196}%
\pgfsetstrokecolor{currentstroke}%
\pgfsetdash{}{0pt}%
\pgfpathmoveto{\pgfqpoint{5.549970in}{1.610099in}}%
\pgfpathcurveto{\pgfqpoint{5.558206in}{1.610099in}}{\pgfqpoint{5.566106in}{1.613371in}}{\pgfqpoint{5.571930in}{1.619195in}}%
\pgfpathcurveto{\pgfqpoint{5.577754in}{1.625019in}}{\pgfqpoint{5.581026in}{1.632919in}}{\pgfqpoint{5.581026in}{1.641156in}}%
\pgfpathcurveto{\pgfqpoint{5.581026in}{1.649392in}}{\pgfqpoint{5.577754in}{1.657292in}}{\pgfqpoint{5.571930in}{1.663116in}}%
\pgfpathcurveto{\pgfqpoint{5.566106in}{1.668940in}}{\pgfqpoint{5.558206in}{1.672212in}}{\pgfqpoint{5.549970in}{1.672212in}}%
\pgfpathcurveto{\pgfqpoint{5.541734in}{1.672212in}}{\pgfqpoint{5.533833in}{1.668940in}}{\pgfqpoint{5.528010in}{1.663116in}}%
\pgfpathcurveto{\pgfqpoint{5.522186in}{1.657292in}}{\pgfqpoint{5.518913in}{1.649392in}}{\pgfqpoint{5.518913in}{1.641156in}}%
\pgfpathcurveto{\pgfqpoint{5.518913in}{1.632919in}}{\pgfqpoint{5.522186in}{1.625019in}}{\pgfqpoint{5.528010in}{1.619195in}}%
\pgfpathcurveto{\pgfqpoint{5.533833in}{1.613371in}}{\pgfqpoint{5.541734in}{1.610099in}}{\pgfqpoint{5.549970in}{1.610099in}}%
\pgfpathclose%
\pgfusepath{stroke,fill}%
\end{pgfscope}%
\begin{pgfscope}%
\pgfpathrectangle{\pgfqpoint{3.755891in}{0.557870in}}{\pgfqpoint{2.484109in}{1.484734in}}%
\pgfusepath{clip}%
\pgfsetbuttcap%
\pgfsetroundjoin%
\definecolor{currentfill}{rgb}{0.298039,0.447059,0.690196}%
\pgfsetfillcolor{currentfill}%
\pgfsetlinewidth{1.003750pt}%
\definecolor{currentstroke}{rgb}{0.298039,0.447059,0.690196}%
\pgfsetstrokecolor{currentstroke}%
\pgfsetdash{}{0pt}%
\pgfpathmoveto{\pgfqpoint{5.173590in}{1.554439in}}%
\pgfpathcurveto{\pgfqpoint{5.181826in}{1.554439in}}{\pgfqpoint{5.189726in}{1.557711in}}{\pgfqpoint{5.195550in}{1.563535in}}%
\pgfpathcurveto{\pgfqpoint{5.201374in}{1.569359in}}{\pgfqpoint{5.204646in}{1.577259in}}{\pgfqpoint{5.204646in}{1.585495in}}%
\pgfpathcurveto{\pgfqpoint{5.204646in}{1.593732in}}{\pgfqpoint{5.201374in}{1.601632in}}{\pgfqpoint{5.195550in}{1.607456in}}%
\pgfpathcurveto{\pgfqpoint{5.189726in}{1.613280in}}{\pgfqpoint{5.181826in}{1.616552in}}{\pgfqpoint{5.173590in}{1.616552in}}%
\pgfpathcurveto{\pgfqpoint{5.165353in}{1.616552in}}{\pgfqpoint{5.157453in}{1.613280in}}{\pgfqpoint{5.151629in}{1.607456in}}%
\pgfpathcurveto{\pgfqpoint{5.145806in}{1.601632in}}{\pgfqpoint{5.142533in}{1.593732in}}{\pgfqpoint{5.142533in}{1.585495in}}%
\pgfpathcurveto{\pgfqpoint{5.142533in}{1.577259in}}{\pgfqpoint{5.145806in}{1.569359in}}{\pgfqpoint{5.151629in}{1.563535in}}%
\pgfpathcurveto{\pgfqpoint{5.157453in}{1.557711in}}{\pgfqpoint{5.165353in}{1.554439in}}{\pgfqpoint{5.173590in}{1.554439in}}%
\pgfpathclose%
\pgfusepath{stroke,fill}%
\end{pgfscope}%
\begin{pgfscope}%
\pgfpathrectangle{\pgfqpoint{3.755891in}{0.557870in}}{\pgfqpoint{2.484109in}{1.484734in}}%
\pgfusepath{clip}%
\pgfsetbuttcap%
\pgfsetroundjoin%
\definecolor{currentfill}{rgb}{0.298039,0.447059,0.690196}%
\pgfsetfillcolor{currentfill}%
\pgfsetlinewidth{1.003750pt}%
\definecolor{currentstroke}{rgb}{0.298039,0.447059,0.690196}%
\pgfsetstrokecolor{currentstroke}%
\pgfsetdash{}{0pt}%
\pgfpathmoveto{\pgfqpoint{5.424510in}{1.651844in}}%
\pgfpathcurveto{\pgfqpoint{5.432746in}{1.651844in}}{\pgfqpoint{5.440646in}{1.655116in}}{\pgfqpoint{5.446470in}{1.660940in}}%
\pgfpathcurveto{\pgfqpoint{5.452294in}{1.666764in}}{\pgfqpoint{5.455566in}{1.674664in}}{\pgfqpoint{5.455566in}{1.682901in}}%
\pgfpathcurveto{\pgfqpoint{5.455566in}{1.691137in}}{\pgfqpoint{5.452294in}{1.699037in}}{\pgfqpoint{5.446470in}{1.704861in}}%
\pgfpathcurveto{\pgfqpoint{5.440646in}{1.710685in}}{\pgfqpoint{5.432746in}{1.713957in}}{\pgfqpoint{5.424510in}{1.713957in}}%
\pgfpathcurveto{\pgfqpoint{5.416273in}{1.713957in}}{\pgfqpoint{5.408373in}{1.710685in}}{\pgfqpoint{5.402550in}{1.704861in}}%
\pgfpathcurveto{\pgfqpoint{5.396726in}{1.699037in}}{\pgfqpoint{5.393453in}{1.691137in}}{\pgfqpoint{5.393453in}{1.682901in}}%
\pgfpathcurveto{\pgfqpoint{5.393453in}{1.674664in}}{\pgfqpoint{5.396726in}{1.666764in}}{\pgfqpoint{5.402550in}{1.660940in}}%
\pgfpathcurveto{\pgfqpoint{5.408373in}{1.655116in}}{\pgfqpoint{5.416273in}{1.651844in}}{\pgfqpoint{5.424510in}{1.651844in}}%
\pgfpathclose%
\pgfusepath{stroke,fill}%
\end{pgfscope}%
\begin{pgfscope}%
\pgfpathrectangle{\pgfqpoint{3.755891in}{0.557870in}}{\pgfqpoint{2.484109in}{1.484734in}}%
\pgfusepath{clip}%
\pgfsetbuttcap%
\pgfsetroundjoin%
\definecolor{currentfill}{rgb}{0.298039,0.447059,0.690196}%
\pgfsetfillcolor{currentfill}%
\pgfsetlinewidth{1.003750pt}%
\definecolor{currentstroke}{rgb}{0.298039,0.447059,0.690196}%
\pgfsetstrokecolor{currentstroke}%
\pgfsetdash{}{0pt}%
\pgfpathmoveto{\pgfqpoint{5.449602in}{1.540524in}}%
\pgfpathcurveto{\pgfqpoint{5.457838in}{1.540524in}}{\pgfqpoint{5.465738in}{1.543796in}}{\pgfqpoint{5.471562in}{1.549620in}}%
\pgfpathcurveto{\pgfqpoint{5.477386in}{1.555444in}}{\pgfqpoint{5.480658in}{1.563344in}}{\pgfqpoint{5.480658in}{1.571580in}}%
\pgfpathcurveto{\pgfqpoint{5.480658in}{1.579817in}}{\pgfqpoint{5.477386in}{1.587717in}}{\pgfqpoint{5.471562in}{1.593541in}}%
\pgfpathcurveto{\pgfqpoint{5.465738in}{1.599365in}}{\pgfqpoint{5.457838in}{1.602637in}}{\pgfqpoint{5.449602in}{1.602637in}}%
\pgfpathcurveto{\pgfqpoint{5.441366in}{1.602637in}}{\pgfqpoint{5.433465in}{1.599365in}}{\pgfqpoint{5.427642in}{1.593541in}}%
\pgfpathcurveto{\pgfqpoint{5.421818in}{1.587717in}}{\pgfqpoint{5.418545in}{1.579817in}}{\pgfqpoint{5.418545in}{1.571580in}}%
\pgfpathcurveto{\pgfqpoint{5.418545in}{1.563344in}}{\pgfqpoint{5.421818in}{1.555444in}}{\pgfqpoint{5.427642in}{1.549620in}}%
\pgfpathcurveto{\pgfqpoint{5.433465in}{1.543796in}}{\pgfqpoint{5.441366in}{1.540524in}}{\pgfqpoint{5.449602in}{1.540524in}}%
\pgfpathclose%
\pgfusepath{stroke,fill}%
\end{pgfscope}%
\begin{pgfscope}%
\pgfpathrectangle{\pgfqpoint{3.755891in}{0.557870in}}{\pgfqpoint{2.484109in}{1.484734in}}%
\pgfusepath{clip}%
\pgfsetbuttcap%
\pgfsetroundjoin%
\definecolor{currentfill}{rgb}{0.298039,0.447059,0.690196}%
\pgfsetfillcolor{currentfill}%
\pgfsetlinewidth{1.003750pt}%
\definecolor{currentstroke}{rgb}{0.298039,0.447059,0.690196}%
\pgfsetstrokecolor{currentstroke}%
\pgfsetdash{}{0pt}%
\pgfpathmoveto{\pgfqpoint{5.374326in}{1.582269in}}%
\pgfpathcurveto{\pgfqpoint{5.382562in}{1.582269in}}{\pgfqpoint{5.390462in}{1.585541in}}{\pgfqpoint{5.396286in}{1.591365in}}%
\pgfpathcurveto{\pgfqpoint{5.402110in}{1.597189in}}{\pgfqpoint{5.405382in}{1.605089in}}{\pgfqpoint{5.405382in}{1.613325in}}%
\pgfpathcurveto{\pgfqpoint{5.405382in}{1.621562in}}{\pgfqpoint{5.402110in}{1.629462in}}{\pgfqpoint{5.396286in}{1.635286in}}%
\pgfpathcurveto{\pgfqpoint{5.390462in}{1.641110in}}{\pgfqpoint{5.382562in}{1.644382in}}{\pgfqpoint{5.374326in}{1.644382in}}%
\pgfpathcurveto{\pgfqpoint{5.366089in}{1.644382in}}{\pgfqpoint{5.358189in}{1.641110in}}{\pgfqpoint{5.352366in}{1.635286in}}%
\pgfpathcurveto{\pgfqpoint{5.346542in}{1.629462in}}{\pgfqpoint{5.343269in}{1.621562in}}{\pgfqpoint{5.343269in}{1.613325in}}%
\pgfpathcurveto{\pgfqpoint{5.343269in}{1.605089in}}{\pgfqpoint{5.346542in}{1.597189in}}{\pgfqpoint{5.352366in}{1.591365in}}%
\pgfpathcurveto{\pgfqpoint{5.358189in}{1.585541in}}{\pgfqpoint{5.366089in}{1.582269in}}{\pgfqpoint{5.374326in}{1.582269in}}%
\pgfpathclose%
\pgfusepath{stroke,fill}%
\end{pgfscope}%
\begin{pgfscope}%
\pgfpathrectangle{\pgfqpoint{3.755891in}{0.557870in}}{\pgfqpoint{2.484109in}{1.484734in}}%
\pgfusepath{clip}%
\pgfsetbuttcap%
\pgfsetroundjoin%
\definecolor{currentfill}{rgb}{0.298039,0.447059,0.690196}%
\pgfsetfillcolor{currentfill}%
\pgfsetlinewidth{1.003750pt}%
\definecolor{currentstroke}{rgb}{0.298039,0.447059,0.690196}%
\pgfsetstrokecolor{currentstroke}%
\pgfsetdash{}{0pt}%
\pgfpathmoveto{\pgfqpoint{5.374326in}{1.665759in}}%
\pgfpathcurveto{\pgfqpoint{5.382562in}{1.665759in}}{\pgfqpoint{5.390462in}{1.669031in}}{\pgfqpoint{5.396286in}{1.674855in}}%
\pgfpathcurveto{\pgfqpoint{5.402110in}{1.680679in}}{\pgfqpoint{5.405382in}{1.688579in}}{\pgfqpoint{5.405382in}{1.696816in}}%
\pgfpathcurveto{\pgfqpoint{5.405382in}{1.705052in}}{\pgfqpoint{5.402110in}{1.712952in}}{\pgfqpoint{5.396286in}{1.718776in}}%
\pgfpathcurveto{\pgfqpoint{5.390462in}{1.724600in}}{\pgfqpoint{5.382562in}{1.727872in}}{\pgfqpoint{5.374326in}{1.727872in}}%
\pgfpathcurveto{\pgfqpoint{5.366089in}{1.727872in}}{\pgfqpoint{5.358189in}{1.724600in}}{\pgfqpoint{5.352366in}{1.718776in}}%
\pgfpathcurveto{\pgfqpoint{5.346542in}{1.712952in}}{\pgfqpoint{5.343269in}{1.705052in}}{\pgfqpoint{5.343269in}{1.696816in}}%
\pgfpathcurveto{\pgfqpoint{5.343269in}{1.688579in}}{\pgfqpoint{5.346542in}{1.680679in}}{\pgfqpoint{5.352366in}{1.674855in}}%
\pgfpathcurveto{\pgfqpoint{5.358189in}{1.669031in}}{\pgfqpoint{5.366089in}{1.665759in}}{\pgfqpoint{5.374326in}{1.665759in}}%
\pgfpathclose%
\pgfusepath{stroke,fill}%
\end{pgfscope}%
\begin{pgfscope}%
\pgfpathrectangle{\pgfqpoint{3.755891in}{0.557870in}}{\pgfqpoint{2.484109in}{1.484734in}}%
\pgfusepath{clip}%
\pgfsetbuttcap%
\pgfsetroundjoin%
\definecolor{currentfill}{rgb}{0.298039,0.447059,0.690196}%
\pgfsetfillcolor{currentfill}%
\pgfsetlinewidth{1.003750pt}%
\definecolor{currentstroke}{rgb}{0.298039,0.447059,0.690196}%
\pgfsetstrokecolor{currentstroke}%
\pgfsetdash{}{0pt}%
\pgfpathmoveto{\pgfqpoint{5.499786in}{1.596184in}}%
\pgfpathcurveto{\pgfqpoint{5.508022in}{1.596184in}}{\pgfqpoint{5.515922in}{1.599456in}}{\pgfqpoint{5.521746in}{1.605280in}}%
\pgfpathcurveto{\pgfqpoint{5.527570in}{1.611104in}}{\pgfqpoint{5.530842in}{1.619004in}}{\pgfqpoint{5.530842in}{1.627240in}}%
\pgfpathcurveto{\pgfqpoint{5.530842in}{1.635477in}}{\pgfqpoint{5.527570in}{1.643377in}}{\pgfqpoint{5.521746in}{1.649201in}}%
\pgfpathcurveto{\pgfqpoint{5.515922in}{1.655025in}}{\pgfqpoint{5.508022in}{1.658297in}}{\pgfqpoint{5.499786in}{1.658297in}}%
\pgfpathcurveto{\pgfqpoint{5.491550in}{1.658297in}}{\pgfqpoint{5.483649in}{1.655025in}}{\pgfqpoint{5.477826in}{1.649201in}}%
\pgfpathcurveto{\pgfqpoint{5.472002in}{1.643377in}}{\pgfqpoint{5.468729in}{1.635477in}}{\pgfqpoint{5.468729in}{1.627240in}}%
\pgfpathcurveto{\pgfqpoint{5.468729in}{1.619004in}}{\pgfqpoint{5.472002in}{1.611104in}}{\pgfqpoint{5.477826in}{1.605280in}}%
\pgfpathcurveto{\pgfqpoint{5.483649in}{1.599456in}}{\pgfqpoint{5.491550in}{1.596184in}}{\pgfqpoint{5.499786in}{1.596184in}}%
\pgfpathclose%
\pgfusepath{stroke,fill}%
\end{pgfscope}%
\begin{pgfscope}%
\pgfpathrectangle{\pgfqpoint{3.755891in}{0.557870in}}{\pgfqpoint{2.484109in}{1.484734in}}%
\pgfusepath{clip}%
\pgfsetbuttcap%
\pgfsetroundjoin%
\definecolor{currentfill}{rgb}{0.298039,0.447059,0.690196}%
\pgfsetfillcolor{currentfill}%
\pgfsetlinewidth{1.003750pt}%
\definecolor{currentstroke}{rgb}{0.298039,0.447059,0.690196}%
\pgfsetstrokecolor{currentstroke}%
\pgfsetdash{}{0pt}%
\pgfpathmoveto{\pgfqpoint{5.349234in}{1.540524in}}%
\pgfpathcurveto{\pgfqpoint{5.357470in}{1.540524in}}{\pgfqpoint{5.365370in}{1.543796in}}{\pgfqpoint{5.371194in}{1.549620in}}%
\pgfpathcurveto{\pgfqpoint{5.377018in}{1.555444in}}{\pgfqpoint{5.380290in}{1.563344in}}{\pgfqpoint{5.380290in}{1.571580in}}%
\pgfpathcurveto{\pgfqpoint{5.380290in}{1.579817in}}{\pgfqpoint{5.377018in}{1.587717in}}{\pgfqpoint{5.371194in}{1.593541in}}%
\pgfpathcurveto{\pgfqpoint{5.365370in}{1.599365in}}{\pgfqpoint{5.357470in}{1.602637in}}{\pgfqpoint{5.349234in}{1.602637in}}%
\pgfpathcurveto{\pgfqpoint{5.340997in}{1.602637in}}{\pgfqpoint{5.333097in}{1.599365in}}{\pgfqpoint{5.327273in}{1.593541in}}%
\pgfpathcurveto{\pgfqpoint{5.321450in}{1.587717in}}{\pgfqpoint{5.318177in}{1.579817in}}{\pgfqpoint{5.318177in}{1.571580in}}%
\pgfpathcurveto{\pgfqpoint{5.318177in}{1.563344in}}{\pgfqpoint{5.321450in}{1.555444in}}{\pgfqpoint{5.327273in}{1.549620in}}%
\pgfpathcurveto{\pgfqpoint{5.333097in}{1.543796in}}{\pgfqpoint{5.340997in}{1.540524in}}{\pgfqpoint{5.349234in}{1.540524in}}%
\pgfpathclose%
\pgfusepath{stroke,fill}%
\end{pgfscope}%
\begin{pgfscope}%
\pgfpathrectangle{\pgfqpoint{3.755891in}{0.557870in}}{\pgfqpoint{2.484109in}{1.484734in}}%
\pgfusepath{clip}%
\pgfsetbuttcap%
\pgfsetroundjoin%
\definecolor{currentfill}{rgb}{0.298039,0.447059,0.690196}%
\pgfsetfillcolor{currentfill}%
\pgfsetlinewidth{1.003750pt}%
\definecolor{currentstroke}{rgb}{0.298039,0.447059,0.690196}%
\pgfsetstrokecolor{currentstroke}%
\pgfsetdash{}{0pt}%
\pgfpathmoveto{\pgfqpoint{5.549970in}{1.392546in}}%
\pgfpathcurveto{\pgfqpoint{5.558206in}{1.392546in}}{\pgfqpoint{5.566106in}{1.395818in}}{\pgfqpoint{5.571930in}{1.401642in}}%
\pgfpathcurveto{\pgfqpoint{5.577754in}{1.407466in}}{\pgfqpoint{5.581026in}{1.415366in}}{\pgfqpoint{5.581026in}{1.423602in}}%
\pgfpathcurveto{\pgfqpoint{5.581026in}{1.431838in}}{\pgfqpoint{5.577754in}{1.439739in}}{\pgfqpoint{5.571930in}{1.445562in}}%
\pgfpathcurveto{\pgfqpoint{5.566106in}{1.451386in}}{\pgfqpoint{5.558206in}{1.454659in}}{\pgfqpoint{5.549970in}{1.454659in}}%
\pgfpathcurveto{\pgfqpoint{5.541734in}{1.454659in}}{\pgfqpoint{5.533833in}{1.451386in}}{\pgfqpoint{5.528010in}{1.445562in}}%
\pgfpathcurveto{\pgfqpoint{5.522186in}{1.439739in}}{\pgfqpoint{5.518913in}{1.431838in}}{\pgfqpoint{5.518913in}{1.423602in}}%
\pgfpathcurveto{\pgfqpoint{5.518913in}{1.415366in}}{\pgfqpoint{5.522186in}{1.407466in}}{\pgfqpoint{5.528010in}{1.401642in}}%
\pgfpathcurveto{\pgfqpoint{5.533833in}{1.395818in}}{\pgfqpoint{5.541734in}{1.392546in}}{\pgfqpoint{5.549970in}{1.392546in}}%
\pgfpathclose%
\pgfusepath{stroke,fill}%
\end{pgfscope}%
\begin{pgfscope}%
\pgfpathrectangle{\pgfqpoint{3.755891in}{0.557870in}}{\pgfqpoint{2.484109in}{1.484734in}}%
\pgfusepath{clip}%
\pgfsetbuttcap%
\pgfsetroundjoin%
\definecolor{currentfill}{rgb}{0.298039,0.447059,0.690196}%
\pgfsetfillcolor{currentfill}%
\pgfsetlinewidth{1.003750pt}%
\definecolor{currentstroke}{rgb}{0.298039,0.447059,0.690196}%
\pgfsetstrokecolor{currentstroke}%
\pgfsetdash{}{0pt}%
\pgfpathmoveto{\pgfqpoint{5.349234in}{1.595735in}}%
\pgfpathcurveto{\pgfqpoint{5.357470in}{1.595735in}}{\pgfqpoint{5.365370in}{1.599007in}}{\pgfqpoint{5.371194in}{1.604831in}}%
\pgfpathcurveto{\pgfqpoint{5.377018in}{1.610655in}}{\pgfqpoint{5.380290in}{1.618555in}}{\pgfqpoint{5.380290in}{1.626792in}}%
\pgfpathcurveto{\pgfqpoint{5.380290in}{1.635028in}}{\pgfqpoint{5.377018in}{1.642928in}}{\pgfqpoint{5.371194in}{1.648752in}}%
\pgfpathcurveto{\pgfqpoint{5.365370in}{1.654576in}}{\pgfqpoint{5.357470in}{1.657848in}}{\pgfqpoint{5.349234in}{1.657848in}}%
\pgfpathcurveto{\pgfqpoint{5.340997in}{1.657848in}}{\pgfqpoint{5.333097in}{1.654576in}}{\pgfqpoint{5.327273in}{1.648752in}}%
\pgfpathcurveto{\pgfqpoint{5.321450in}{1.642928in}}{\pgfqpoint{5.318177in}{1.635028in}}{\pgfqpoint{5.318177in}{1.626792in}}%
\pgfpathcurveto{\pgfqpoint{5.318177in}{1.618555in}}{\pgfqpoint{5.321450in}{1.610655in}}{\pgfqpoint{5.327273in}{1.604831in}}%
\pgfpathcurveto{\pgfqpoint{5.333097in}{1.599007in}}{\pgfqpoint{5.340997in}{1.595735in}}{\pgfqpoint{5.349234in}{1.595735in}}%
\pgfpathclose%
\pgfusepath{stroke,fill}%
\end{pgfscope}%
\begin{pgfscope}%
\pgfpathrectangle{\pgfqpoint{3.755891in}{0.557870in}}{\pgfqpoint{2.484109in}{1.484734in}}%
\pgfusepath{clip}%
\pgfsetbuttcap%
\pgfsetroundjoin%
\definecolor{currentfill}{rgb}{0.298039,0.447059,0.690196}%
\pgfsetfillcolor{currentfill}%
\pgfsetlinewidth{1.003750pt}%
\definecolor{currentstroke}{rgb}{0.298039,0.447059,0.690196}%
\pgfsetstrokecolor{currentstroke}%
\pgfsetdash{}{0pt}%
\pgfpathmoveto{\pgfqpoint{5.524878in}{1.581222in}}%
\pgfpathcurveto{\pgfqpoint{5.533114in}{1.581222in}}{\pgfqpoint{5.541014in}{1.584494in}}{\pgfqpoint{5.546838in}{1.590318in}}%
\pgfpathcurveto{\pgfqpoint{5.552662in}{1.596142in}}{\pgfqpoint{5.555934in}{1.604042in}}{\pgfqpoint{5.555934in}{1.612278in}}%
\pgfpathcurveto{\pgfqpoint{5.555934in}{1.620514in}}{\pgfqpoint{5.552662in}{1.628414in}}{\pgfqpoint{5.546838in}{1.634238in}}%
\pgfpathcurveto{\pgfqpoint{5.541014in}{1.640062in}}{\pgfqpoint{5.533114in}{1.643335in}}{\pgfqpoint{5.524878in}{1.643335in}}%
\pgfpathcurveto{\pgfqpoint{5.516642in}{1.643335in}}{\pgfqpoint{5.508741in}{1.640062in}}{\pgfqpoint{5.502918in}{1.634238in}}%
\pgfpathcurveto{\pgfqpoint{5.497094in}{1.628414in}}{\pgfqpoint{5.493821in}{1.620514in}}{\pgfqpoint{5.493821in}{1.612278in}}%
\pgfpathcurveto{\pgfqpoint{5.493821in}{1.604042in}}{\pgfqpoint{5.497094in}{1.596142in}}{\pgfqpoint{5.502918in}{1.590318in}}%
\pgfpathcurveto{\pgfqpoint{5.508741in}{1.584494in}}{\pgfqpoint{5.516642in}{1.581222in}}{\pgfqpoint{5.524878in}{1.581222in}}%
\pgfpathclose%
\pgfusepath{stroke,fill}%
\end{pgfscope}%
\begin{pgfscope}%
\pgfpathrectangle{\pgfqpoint{3.755891in}{0.557870in}}{\pgfqpoint{2.484109in}{1.484734in}}%
\pgfusepath{clip}%
\pgfsetbuttcap%
\pgfsetroundjoin%
\definecolor{currentfill}{rgb}{0.298039,0.447059,0.690196}%
\pgfsetfillcolor{currentfill}%
\pgfsetlinewidth{1.003750pt}%
\definecolor{currentstroke}{rgb}{0.298039,0.447059,0.690196}%
\pgfsetstrokecolor{currentstroke}%
\pgfsetdash{}{0pt}%
\pgfpathmoveto{\pgfqpoint{5.223774in}{1.523167in}}%
\pgfpathcurveto{\pgfqpoint{5.232010in}{1.523167in}}{\pgfqpoint{5.239910in}{1.526440in}}{\pgfqpoint{5.245734in}{1.532264in}}%
\pgfpathcurveto{\pgfqpoint{5.251558in}{1.538088in}}{\pgfqpoint{5.254830in}{1.545988in}}{\pgfqpoint{5.254830in}{1.554224in}}%
\pgfpathcurveto{\pgfqpoint{5.254830in}{1.562460in}}{\pgfqpoint{5.251558in}{1.570360in}}{\pgfqpoint{5.245734in}{1.576184in}}%
\pgfpathcurveto{\pgfqpoint{5.239910in}{1.582008in}}{\pgfqpoint{5.232010in}{1.585280in}}{\pgfqpoint{5.223774in}{1.585280in}}%
\pgfpathcurveto{\pgfqpoint{5.215537in}{1.585280in}}{\pgfqpoint{5.207637in}{1.582008in}}{\pgfqpoint{5.201813in}{1.576184in}}%
\pgfpathcurveto{\pgfqpoint{5.195990in}{1.570360in}}{\pgfqpoint{5.192717in}{1.562460in}}{\pgfqpoint{5.192717in}{1.554224in}}%
\pgfpathcurveto{\pgfqpoint{5.192717in}{1.545988in}}{\pgfqpoint{5.195990in}{1.538088in}}{\pgfqpoint{5.201813in}{1.532264in}}%
\pgfpathcurveto{\pgfqpoint{5.207637in}{1.526440in}}{\pgfqpoint{5.215537in}{1.523167in}}{\pgfqpoint{5.223774in}{1.523167in}}%
\pgfpathclose%
\pgfusepath{stroke,fill}%
\end{pgfscope}%
\begin{pgfscope}%
\pgfpathrectangle{\pgfqpoint{3.755891in}{0.557870in}}{\pgfqpoint{2.484109in}{1.484734in}}%
\pgfusepath{clip}%
\pgfsetbuttcap%
\pgfsetroundjoin%
\definecolor{currentfill}{rgb}{0.298039,0.447059,0.690196}%
\pgfsetfillcolor{currentfill}%
\pgfsetlinewidth{1.003750pt}%
\definecolor{currentstroke}{rgb}{0.298039,0.447059,0.690196}%
\pgfsetstrokecolor{currentstroke}%
\pgfsetdash{}{0pt}%
\pgfpathmoveto{\pgfqpoint{5.399418in}{1.552195in}}%
\pgfpathcurveto{\pgfqpoint{5.407654in}{1.552195in}}{\pgfqpoint{5.415554in}{1.555467in}}{\pgfqpoint{5.421378in}{1.561291in}}%
\pgfpathcurveto{\pgfqpoint{5.427202in}{1.567115in}}{\pgfqpoint{5.430474in}{1.575015in}}{\pgfqpoint{5.430474in}{1.583251in}}%
\pgfpathcurveto{\pgfqpoint{5.430474in}{1.591487in}}{\pgfqpoint{5.427202in}{1.599387in}}{\pgfqpoint{5.421378in}{1.605211in}}%
\pgfpathcurveto{\pgfqpoint{5.415554in}{1.611035in}}{\pgfqpoint{5.407654in}{1.614308in}}{\pgfqpoint{5.399418in}{1.614308in}}%
\pgfpathcurveto{\pgfqpoint{5.391181in}{1.614308in}}{\pgfqpoint{5.383281in}{1.611035in}}{\pgfqpoint{5.377458in}{1.605211in}}%
\pgfpathcurveto{\pgfqpoint{5.371634in}{1.599387in}}{\pgfqpoint{5.368361in}{1.591487in}}{\pgfqpoint{5.368361in}{1.583251in}}%
\pgfpathcurveto{\pgfqpoint{5.368361in}{1.575015in}}{\pgfqpoint{5.371634in}{1.567115in}}{\pgfqpoint{5.377458in}{1.561291in}}%
\pgfpathcurveto{\pgfqpoint{5.383281in}{1.555467in}}{\pgfqpoint{5.391181in}{1.552195in}}{\pgfqpoint{5.399418in}{1.552195in}}%
\pgfpathclose%
\pgfusepath{stroke,fill}%
\end{pgfscope}%
\begin{pgfscope}%
\pgfpathrectangle{\pgfqpoint{3.755891in}{0.557870in}}{\pgfqpoint{2.484109in}{1.484734in}}%
\pgfusepath{clip}%
\pgfsetbuttcap%
\pgfsetroundjoin%
\definecolor{currentfill}{rgb}{0.298039,0.447059,0.690196}%
\pgfsetfillcolor{currentfill}%
\pgfsetlinewidth{1.003750pt}%
\definecolor{currentstroke}{rgb}{0.298039,0.447059,0.690196}%
\pgfsetstrokecolor{currentstroke}%
\pgfsetdash{}{0pt}%
\pgfpathmoveto{\pgfqpoint{5.474694in}{1.595735in}}%
\pgfpathcurveto{\pgfqpoint{5.482930in}{1.595735in}}{\pgfqpoint{5.490830in}{1.599007in}}{\pgfqpoint{5.496654in}{1.604831in}}%
\pgfpathcurveto{\pgfqpoint{5.502478in}{1.610655in}}{\pgfqpoint{5.505750in}{1.618555in}}{\pgfqpoint{5.505750in}{1.626792in}}%
\pgfpathcurveto{\pgfqpoint{5.505750in}{1.635028in}}{\pgfqpoint{5.502478in}{1.642928in}}{\pgfqpoint{5.496654in}{1.648752in}}%
\pgfpathcurveto{\pgfqpoint{5.490830in}{1.654576in}}{\pgfqpoint{5.482930in}{1.657848in}}{\pgfqpoint{5.474694in}{1.657848in}}%
\pgfpathcurveto{\pgfqpoint{5.466458in}{1.657848in}}{\pgfqpoint{5.458557in}{1.654576in}}{\pgfqpoint{5.452734in}{1.648752in}}%
\pgfpathcurveto{\pgfqpoint{5.446910in}{1.642928in}}{\pgfqpoint{5.443637in}{1.635028in}}{\pgfqpoint{5.443637in}{1.626792in}}%
\pgfpathcurveto{\pgfqpoint{5.443637in}{1.618555in}}{\pgfqpoint{5.446910in}{1.610655in}}{\pgfqpoint{5.452734in}{1.604831in}}%
\pgfpathcurveto{\pgfqpoint{5.458557in}{1.599007in}}{\pgfqpoint{5.466458in}{1.595735in}}{\pgfqpoint{5.474694in}{1.595735in}}%
\pgfpathclose%
\pgfusepath{stroke,fill}%
\end{pgfscope}%
\begin{pgfscope}%
\pgfpathrectangle{\pgfqpoint{3.755891in}{0.557870in}}{\pgfqpoint{2.484109in}{1.484734in}}%
\pgfusepath{clip}%
\pgfsetbuttcap%
\pgfsetroundjoin%
\definecolor{currentfill}{rgb}{0.298039,0.447059,0.690196}%
\pgfsetfillcolor{currentfill}%
\pgfsetlinewidth{1.003750pt}%
\definecolor{currentstroke}{rgb}{0.298039,0.447059,0.690196}%
\pgfsetstrokecolor{currentstroke}%
\pgfsetdash{}{0pt}%
\pgfpathmoveto{\pgfqpoint{5.424510in}{1.610249in}}%
\pgfpathcurveto{\pgfqpoint{5.432746in}{1.610249in}}{\pgfqpoint{5.440646in}{1.613521in}}{\pgfqpoint{5.446470in}{1.619345in}}%
\pgfpathcurveto{\pgfqpoint{5.452294in}{1.625169in}}{\pgfqpoint{5.455566in}{1.633069in}}{\pgfqpoint{5.455566in}{1.641305in}}%
\pgfpathcurveto{\pgfqpoint{5.455566in}{1.649541in}}{\pgfqpoint{5.452294in}{1.657441in}}{\pgfqpoint{5.446470in}{1.663265in}}%
\pgfpathcurveto{\pgfqpoint{5.440646in}{1.669089in}}{\pgfqpoint{5.432746in}{1.672362in}}{\pgfqpoint{5.424510in}{1.672362in}}%
\pgfpathcurveto{\pgfqpoint{5.416273in}{1.672362in}}{\pgfqpoint{5.408373in}{1.669089in}}{\pgfqpoint{5.402550in}{1.663265in}}%
\pgfpathcurveto{\pgfqpoint{5.396726in}{1.657441in}}{\pgfqpoint{5.393453in}{1.649541in}}{\pgfqpoint{5.393453in}{1.641305in}}%
\pgfpathcurveto{\pgfqpoint{5.393453in}{1.633069in}}{\pgfqpoint{5.396726in}{1.625169in}}{\pgfqpoint{5.402550in}{1.619345in}}%
\pgfpathcurveto{\pgfqpoint{5.408373in}{1.613521in}}{\pgfqpoint{5.416273in}{1.610249in}}{\pgfqpoint{5.424510in}{1.610249in}}%
\pgfpathclose%
\pgfusepath{stroke,fill}%
\end{pgfscope}%
\begin{pgfscope}%
\pgfpathrectangle{\pgfqpoint{3.755891in}{0.557870in}}{\pgfqpoint{2.484109in}{1.484734in}}%
\pgfusepath{clip}%
\pgfsetbuttcap%
\pgfsetroundjoin%
\definecolor{currentfill}{rgb}{0.298039,0.447059,0.690196}%
\pgfsetfillcolor{currentfill}%
\pgfsetlinewidth{1.003750pt}%
\definecolor{currentstroke}{rgb}{0.298039,0.447059,0.690196}%
\pgfsetstrokecolor{currentstroke}%
\pgfsetdash{}{0pt}%
\pgfpathmoveto{\pgfqpoint{5.399418in}{1.639276in}}%
\pgfpathcurveto{\pgfqpoint{5.407654in}{1.639276in}}{\pgfqpoint{5.415554in}{1.642548in}}{\pgfqpoint{5.421378in}{1.648372in}}%
\pgfpathcurveto{\pgfqpoint{5.427202in}{1.654196in}}{\pgfqpoint{5.430474in}{1.662096in}}{\pgfqpoint{5.430474in}{1.670332in}}%
\pgfpathcurveto{\pgfqpoint{5.430474in}{1.678568in}}{\pgfqpoint{5.427202in}{1.686469in}}{\pgfqpoint{5.421378in}{1.692292in}}%
\pgfpathcurveto{\pgfqpoint{5.415554in}{1.698116in}}{\pgfqpoint{5.407654in}{1.701389in}}{\pgfqpoint{5.399418in}{1.701389in}}%
\pgfpathcurveto{\pgfqpoint{5.391181in}{1.701389in}}{\pgfqpoint{5.383281in}{1.698116in}}{\pgfqpoint{5.377458in}{1.692292in}}%
\pgfpathcurveto{\pgfqpoint{5.371634in}{1.686469in}}{\pgfqpoint{5.368361in}{1.678568in}}{\pgfqpoint{5.368361in}{1.670332in}}%
\pgfpathcurveto{\pgfqpoint{5.368361in}{1.662096in}}{\pgfqpoint{5.371634in}{1.654196in}}{\pgfqpoint{5.377458in}{1.648372in}}%
\pgfpathcurveto{\pgfqpoint{5.383281in}{1.642548in}}{\pgfqpoint{5.391181in}{1.639276in}}{\pgfqpoint{5.399418in}{1.639276in}}%
\pgfpathclose%
\pgfusepath{stroke,fill}%
\end{pgfscope}%
\begin{pgfscope}%
\pgfpathrectangle{\pgfqpoint{3.755891in}{0.557870in}}{\pgfqpoint{2.484109in}{1.484734in}}%
\pgfusepath{clip}%
\pgfsetbuttcap%
\pgfsetroundjoin%
\definecolor{currentfill}{rgb}{0.298039,0.447059,0.690196}%
\pgfsetfillcolor{currentfill}%
\pgfsetlinewidth{1.003750pt}%
\definecolor{currentstroke}{rgb}{0.298039,0.447059,0.690196}%
\pgfsetstrokecolor{currentstroke}%
\pgfsetdash{}{0pt}%
\pgfpathmoveto{\pgfqpoint{5.449602in}{1.624762in}}%
\pgfpathcurveto{\pgfqpoint{5.457838in}{1.624762in}}{\pgfqpoint{5.465738in}{1.628034in}}{\pgfqpoint{5.471562in}{1.633858in}}%
\pgfpathcurveto{\pgfqpoint{5.477386in}{1.639682in}}{\pgfqpoint{5.480658in}{1.647582in}}{\pgfqpoint{5.480658in}{1.655819in}}%
\pgfpathcurveto{\pgfqpoint{5.480658in}{1.664055in}}{\pgfqpoint{5.477386in}{1.671955in}}{\pgfqpoint{5.471562in}{1.677779in}}%
\pgfpathcurveto{\pgfqpoint{5.465738in}{1.683603in}}{\pgfqpoint{5.457838in}{1.686875in}}{\pgfqpoint{5.449602in}{1.686875in}}%
\pgfpathcurveto{\pgfqpoint{5.441366in}{1.686875in}}{\pgfqpoint{5.433465in}{1.683603in}}{\pgfqpoint{5.427642in}{1.677779in}}%
\pgfpathcurveto{\pgfqpoint{5.421818in}{1.671955in}}{\pgfqpoint{5.418545in}{1.664055in}}{\pgfqpoint{5.418545in}{1.655819in}}%
\pgfpathcurveto{\pgfqpoint{5.418545in}{1.647582in}}{\pgfqpoint{5.421818in}{1.639682in}}{\pgfqpoint{5.427642in}{1.633858in}}%
\pgfpathcurveto{\pgfqpoint{5.433465in}{1.628034in}}{\pgfqpoint{5.441366in}{1.624762in}}{\pgfqpoint{5.449602in}{1.624762in}}%
\pgfpathclose%
\pgfusepath{stroke,fill}%
\end{pgfscope}%
\begin{pgfscope}%
\pgfpathrectangle{\pgfqpoint{3.755891in}{0.557870in}}{\pgfqpoint{2.484109in}{1.484734in}}%
\pgfusepath{clip}%
\pgfsetbuttcap%
\pgfsetroundjoin%
\definecolor{currentfill}{rgb}{0.298039,0.447059,0.690196}%
\pgfsetfillcolor{currentfill}%
\pgfsetlinewidth{1.003750pt}%
\definecolor{currentstroke}{rgb}{0.298039,0.447059,0.690196}%
\pgfsetstrokecolor{currentstroke}%
\pgfsetdash{}{0pt}%
\pgfpathmoveto{\pgfqpoint{5.399418in}{1.421573in}}%
\pgfpathcurveto{\pgfqpoint{5.407654in}{1.421573in}}{\pgfqpoint{5.415554in}{1.424845in}}{\pgfqpoint{5.421378in}{1.430669in}}%
\pgfpathcurveto{\pgfqpoint{5.427202in}{1.436493in}}{\pgfqpoint{5.430474in}{1.444393in}}{\pgfqpoint{5.430474in}{1.452629in}}%
\pgfpathcurveto{\pgfqpoint{5.430474in}{1.460866in}}{\pgfqpoint{5.427202in}{1.468766in}}{\pgfqpoint{5.421378in}{1.474590in}}%
\pgfpathcurveto{\pgfqpoint{5.415554in}{1.480413in}}{\pgfqpoint{5.407654in}{1.483686in}}{\pgfqpoint{5.399418in}{1.483686in}}%
\pgfpathcurveto{\pgfqpoint{5.391181in}{1.483686in}}{\pgfqpoint{5.383281in}{1.480413in}}{\pgfqpoint{5.377458in}{1.474590in}}%
\pgfpathcurveto{\pgfqpoint{5.371634in}{1.468766in}}{\pgfqpoint{5.368361in}{1.460866in}}{\pgfqpoint{5.368361in}{1.452629in}}%
\pgfpathcurveto{\pgfqpoint{5.368361in}{1.444393in}}{\pgfqpoint{5.371634in}{1.436493in}}{\pgfqpoint{5.377458in}{1.430669in}}%
\pgfpathcurveto{\pgfqpoint{5.383281in}{1.424845in}}{\pgfqpoint{5.391181in}{1.421573in}}{\pgfqpoint{5.399418in}{1.421573in}}%
\pgfpathclose%
\pgfusepath{stroke,fill}%
\end{pgfscope}%
\begin{pgfscope}%
\pgfsetrectcap%
\pgfsetmiterjoin%
\pgfsetlinewidth{1.254687pt}%
\definecolor{currentstroke}{rgb}{1.000000,1.000000,1.000000}%
\pgfsetstrokecolor{currentstroke}%
\pgfsetdash{}{0pt}%
\pgfpathmoveto{\pgfqpoint{3.755891in}{0.557870in}}%
\pgfpathlineto{\pgfqpoint{3.755891in}{2.042604in}}%
\pgfusepath{stroke}%
\end{pgfscope}%
\begin{pgfscope}%
\pgfsetrectcap%
\pgfsetmiterjoin%
\pgfsetlinewidth{1.254687pt}%
\definecolor{currentstroke}{rgb}{1.000000,1.000000,1.000000}%
\pgfsetstrokecolor{currentstroke}%
\pgfsetdash{}{0pt}%
\pgfpathmoveto{\pgfqpoint{6.240000in}{0.557870in}}%
\pgfpathlineto{\pgfqpoint{6.240000in}{2.042604in}}%
\pgfusepath{stroke}%
\end{pgfscope}%
\begin{pgfscope}%
\pgfsetrectcap%
\pgfsetmiterjoin%
\pgfsetlinewidth{1.254687pt}%
\definecolor{currentstroke}{rgb}{1.000000,1.000000,1.000000}%
\pgfsetstrokecolor{currentstroke}%
\pgfsetdash{}{0pt}%
\pgfpathmoveto{\pgfqpoint{3.755891in}{0.557870in}}%
\pgfpathlineto{\pgfqpoint{6.240000in}{0.557870in}}%
\pgfusepath{stroke}%
\end{pgfscope}%
\begin{pgfscope}%
\pgfsetrectcap%
\pgfsetmiterjoin%
\pgfsetlinewidth{1.254687pt}%
\definecolor{currentstroke}{rgb}{1.000000,1.000000,1.000000}%
\pgfsetstrokecolor{currentstroke}%
\pgfsetdash{}{0pt}%
\pgfpathmoveto{\pgfqpoint{3.755891in}{2.042604in}}%
\pgfpathlineto{\pgfqpoint{6.240000in}{2.042604in}}%
\pgfusepath{stroke}%
\end{pgfscope}%
\begin{pgfscope}%
\definecolor{textcolor}{rgb}{0.150000,0.150000,0.150000}%
\pgfsetstrokecolor{textcolor}%
\pgfsetfillcolor{textcolor}%
\pgftext[x=4.997946in,y=2.125938in,,base]{\color{textcolor}\sffamily\fontsize{11.000000}{13.200000}\selectfont (b)}%
\end{pgfscope}%
\end{pgfpicture}%
\makeatother%
\endgroup%

    \caption{Distribution of DOR, sensitivity and specificity for the different peak-value classifiers trained to predict heart failure.}
    \label{fig:pvmlc_hf_dor_sens_spec_dis}
\end{figure}

\subsection{Comparisons}

\newpage

\section{Case Study: Patient Diagnosis}

\subsection{Time-series Clustering}

\begin{figure}[htb]
    \centering
    % \includegraphics[width=\textwidth]{results/tsc_ind_dor_sens_spec_dist.png}
    %% Creator: Matplotlib, PGF backend
%%
%% To include the figure in your LaTeX document, write
%%   \input{<filename>.pgf}
%%
%% Make sure the required packages are loaded in your preamble
%%   \usepackage{pgf}
%%
%% Figures using additional raster images can only be included by \input if
%% they are in the same directory as the main LaTeX file. For loading figures
%% from other directories you can use the `import` package
%%   \usepackage{import}
%% and then include the figures with
%%   \import{<path to file>}{<filename>.pgf}
%%
%% Matplotlib used the following preamble
%%
\begingroup%
\makeatletter%
\begin{pgfpicture}%
\pgfpathrectangle{\pgfpointorigin}{\pgfqpoint{6.364000in}{2.340000in}}%
\pgfusepath{use as bounding box, clip}%
\begin{pgfscope}%
\pgfsetbuttcap%
\pgfsetmiterjoin%
\definecolor{currentfill}{rgb}{1.000000,1.000000,1.000000}%
\pgfsetfillcolor{currentfill}%
\pgfsetlinewidth{0.000000pt}%
\definecolor{currentstroke}{rgb}{1.000000,1.000000,1.000000}%
\pgfsetstrokecolor{currentstroke}%
\pgfsetdash{}{0pt}%
\pgfpathmoveto{\pgfqpoint{0.000000in}{-0.000000in}}%
\pgfpathlineto{\pgfqpoint{6.364000in}{-0.000000in}}%
\pgfpathlineto{\pgfqpoint{6.364000in}{2.340000in}}%
\pgfpathlineto{\pgfqpoint{0.000000in}{2.340000in}}%
\pgfpathclose%
\pgfusepath{fill}%
\end{pgfscope}%
\begin{pgfscope}%
\pgfsetbuttcap%
\pgfsetmiterjoin%
\definecolor{currentfill}{rgb}{0.917647,0.917647,0.949020}%
\pgfsetfillcolor{currentfill}%
\pgfsetlinewidth{0.000000pt}%
\definecolor{currentstroke}{rgb}{0.000000,0.000000,0.000000}%
\pgfsetstrokecolor{currentstroke}%
\pgfsetstrokeopacity{0.000000}%
\pgfsetdash{}{0pt}%
\pgfpathmoveto{\pgfqpoint{0.650810in}{0.557870in}}%
\pgfpathlineto{\pgfqpoint{3.096898in}{0.557870in}}%
\pgfpathlineto{\pgfqpoint{3.096898in}{2.042604in}}%
\pgfpathlineto{\pgfqpoint{0.650810in}{2.042604in}}%
\pgfpathclose%
\pgfusepath{fill}%
\end{pgfscope}%
\begin{pgfscope}%
\pgfpathrectangle{\pgfqpoint{0.650810in}{0.557870in}}{\pgfqpoint{2.446088in}{1.484734in}}%
\pgfusepath{clip}%
\pgfsetroundcap%
\pgfsetroundjoin%
\pgfsetlinewidth{1.003750pt}%
\definecolor{currentstroke}{rgb}{1.000000,1.000000,1.000000}%
\pgfsetstrokecolor{currentstroke}%
\pgfsetdash{}{0pt}%
\pgfpathmoveto{\pgfqpoint{0.761996in}{0.557870in}}%
\pgfpathlineto{\pgfqpoint{0.761996in}{2.042604in}}%
\pgfusepath{stroke}%
\end{pgfscope}%
\begin{pgfscope}%
\definecolor{textcolor}{rgb}{0.150000,0.150000,0.150000}%
\pgfsetstrokecolor{textcolor}%
\pgfsetfillcolor{textcolor}%
\pgftext[x=0.761996in,y=0.425926in,,top]{\color{textcolor}\sffamily\fontsize{11.000000}{13.200000}\selectfont \(\displaystyle -0.50\)}%
\end{pgfscope}%
\begin{pgfscope}%
\pgfpathrectangle{\pgfqpoint{0.650810in}{0.557870in}}{\pgfqpoint{2.446088in}{1.484734in}}%
\pgfusepath{clip}%
\pgfsetroundcap%
\pgfsetroundjoin%
\pgfsetlinewidth{1.003750pt}%
\definecolor{currentstroke}{rgb}{1.000000,1.000000,1.000000}%
\pgfsetstrokecolor{currentstroke}%
\pgfsetdash{}{0pt}%
\pgfpathmoveto{\pgfqpoint{1.317925in}{0.557870in}}%
\pgfpathlineto{\pgfqpoint{1.317925in}{2.042604in}}%
\pgfusepath{stroke}%
\end{pgfscope}%
\begin{pgfscope}%
\definecolor{textcolor}{rgb}{0.150000,0.150000,0.150000}%
\pgfsetstrokecolor{textcolor}%
\pgfsetfillcolor{textcolor}%
\pgftext[x=1.317925in,y=0.425926in,,top]{\color{textcolor}\sffamily\fontsize{11.000000}{13.200000}\selectfont \(\displaystyle -0.25\)}%
\end{pgfscope}%
\begin{pgfscope}%
\pgfpathrectangle{\pgfqpoint{0.650810in}{0.557870in}}{\pgfqpoint{2.446088in}{1.484734in}}%
\pgfusepath{clip}%
\pgfsetroundcap%
\pgfsetroundjoin%
\pgfsetlinewidth{1.003750pt}%
\definecolor{currentstroke}{rgb}{1.000000,1.000000,1.000000}%
\pgfsetstrokecolor{currentstroke}%
\pgfsetdash{}{0pt}%
\pgfpathmoveto{\pgfqpoint{1.873854in}{0.557870in}}%
\pgfpathlineto{\pgfqpoint{1.873854in}{2.042604in}}%
\pgfusepath{stroke}%
\end{pgfscope}%
\begin{pgfscope}%
\definecolor{textcolor}{rgb}{0.150000,0.150000,0.150000}%
\pgfsetstrokecolor{textcolor}%
\pgfsetfillcolor{textcolor}%
\pgftext[x=1.873854in,y=0.425926in,,top]{\color{textcolor}\sffamily\fontsize{11.000000}{13.200000}\selectfont \(\displaystyle 0.00\)}%
\end{pgfscope}%
\begin{pgfscope}%
\pgfpathrectangle{\pgfqpoint{0.650810in}{0.557870in}}{\pgfqpoint{2.446088in}{1.484734in}}%
\pgfusepath{clip}%
\pgfsetroundcap%
\pgfsetroundjoin%
\pgfsetlinewidth{1.003750pt}%
\definecolor{currentstroke}{rgb}{1.000000,1.000000,1.000000}%
\pgfsetstrokecolor{currentstroke}%
\pgfsetdash{}{0pt}%
\pgfpathmoveto{\pgfqpoint{2.429783in}{0.557870in}}%
\pgfpathlineto{\pgfqpoint{2.429783in}{2.042604in}}%
\pgfusepath{stroke}%
\end{pgfscope}%
\begin{pgfscope}%
\definecolor{textcolor}{rgb}{0.150000,0.150000,0.150000}%
\pgfsetstrokecolor{textcolor}%
\pgfsetfillcolor{textcolor}%
\pgftext[x=2.429783in,y=0.425926in,,top]{\color{textcolor}\sffamily\fontsize{11.000000}{13.200000}\selectfont \(\displaystyle 0.25\)}%
\end{pgfscope}%
\begin{pgfscope}%
\pgfpathrectangle{\pgfqpoint{0.650810in}{0.557870in}}{\pgfqpoint{2.446088in}{1.484734in}}%
\pgfusepath{clip}%
\pgfsetroundcap%
\pgfsetroundjoin%
\pgfsetlinewidth{1.003750pt}%
\definecolor{currentstroke}{rgb}{1.000000,1.000000,1.000000}%
\pgfsetstrokecolor{currentstroke}%
\pgfsetdash{}{0pt}%
\pgfpathmoveto{\pgfqpoint{2.985712in}{0.557870in}}%
\pgfpathlineto{\pgfqpoint{2.985712in}{2.042604in}}%
\pgfusepath{stroke}%
\end{pgfscope}%
\begin{pgfscope}%
\definecolor{textcolor}{rgb}{0.150000,0.150000,0.150000}%
\pgfsetstrokecolor{textcolor}%
\pgfsetfillcolor{textcolor}%
\pgftext[x=2.985712in,y=0.425926in,,top]{\color{textcolor}\sffamily\fontsize{11.000000}{13.200000}\selectfont \(\displaystyle 0.50\)}%
\end{pgfscope}%
\begin{pgfscope}%
\definecolor{textcolor}{rgb}{0.150000,0.150000,0.150000}%
\pgfsetstrokecolor{textcolor}%
\pgfsetfillcolor{textcolor}%
\pgftext[x=1.873854in,y=0.235185in,,top]{\color{textcolor}\sffamily\fontsize{11.000000}{13.200000}\selectfont DOR}%
\end{pgfscope}%
\begin{pgfscope}%
\pgfpathrectangle{\pgfqpoint{0.650810in}{0.557870in}}{\pgfqpoint{2.446088in}{1.484734in}}%
\pgfusepath{clip}%
\pgfsetroundcap%
\pgfsetroundjoin%
\pgfsetlinewidth{1.003750pt}%
\definecolor{currentstroke}{rgb}{1.000000,1.000000,1.000000}%
\pgfsetstrokecolor{currentstroke}%
\pgfsetdash{}{0pt}%
\pgfpathmoveto{\pgfqpoint{0.650810in}{0.557870in}}%
\pgfpathlineto{\pgfqpoint{3.096898in}{0.557870in}}%
\pgfusepath{stroke}%
\end{pgfscope}%
\begin{pgfscope}%
\definecolor{textcolor}{rgb}{0.150000,0.150000,0.150000}%
\pgfsetstrokecolor{textcolor}%
\pgfsetfillcolor{textcolor}%
\pgftext[x=0.442824in,y=0.505064in,left,base]{\color{textcolor}\sffamily\fontsize{11.000000}{13.200000}\selectfont \(\displaystyle 0\)}%
\end{pgfscope}%
\begin{pgfscope}%
\pgfpathrectangle{\pgfqpoint{0.650810in}{0.557870in}}{\pgfqpoint{2.446088in}{1.484734in}}%
\pgfusepath{clip}%
\pgfsetroundcap%
\pgfsetroundjoin%
\pgfsetlinewidth{1.003750pt}%
\definecolor{currentstroke}{rgb}{1.000000,1.000000,1.000000}%
\pgfsetstrokecolor{currentstroke}%
\pgfsetdash{}{0pt}%
\pgfpathmoveto{\pgfqpoint{0.650810in}{1.008199in}}%
\pgfpathlineto{\pgfqpoint{3.096898in}{1.008199in}}%
\pgfusepath{stroke}%
\end{pgfscope}%
\begin{pgfscope}%
\definecolor{textcolor}{rgb}{0.150000,0.150000,0.150000}%
\pgfsetstrokecolor{textcolor}%
\pgfsetfillcolor{textcolor}%
\pgftext[x=0.290741in,y=0.955392in,left,base]{\color{textcolor}\sffamily\fontsize{11.000000}{13.200000}\selectfont \(\displaystyle 100\)}%
\end{pgfscope}%
\begin{pgfscope}%
\pgfpathrectangle{\pgfqpoint{0.650810in}{0.557870in}}{\pgfqpoint{2.446088in}{1.484734in}}%
\pgfusepath{clip}%
\pgfsetroundcap%
\pgfsetroundjoin%
\pgfsetlinewidth{1.003750pt}%
\definecolor{currentstroke}{rgb}{1.000000,1.000000,1.000000}%
\pgfsetstrokecolor{currentstroke}%
\pgfsetdash{}{0pt}%
\pgfpathmoveto{\pgfqpoint{0.650810in}{1.458528in}}%
\pgfpathlineto{\pgfqpoint{3.096898in}{1.458528in}}%
\pgfusepath{stroke}%
\end{pgfscope}%
\begin{pgfscope}%
\definecolor{textcolor}{rgb}{0.150000,0.150000,0.150000}%
\pgfsetstrokecolor{textcolor}%
\pgfsetfillcolor{textcolor}%
\pgftext[x=0.290741in,y=1.405721in,left,base]{\color{textcolor}\sffamily\fontsize{11.000000}{13.200000}\selectfont \(\displaystyle 200\)}%
\end{pgfscope}%
\begin{pgfscope}%
\pgfpathrectangle{\pgfqpoint{0.650810in}{0.557870in}}{\pgfqpoint{2.446088in}{1.484734in}}%
\pgfusepath{clip}%
\pgfsetroundcap%
\pgfsetroundjoin%
\pgfsetlinewidth{1.003750pt}%
\definecolor{currentstroke}{rgb}{1.000000,1.000000,1.000000}%
\pgfsetstrokecolor{currentstroke}%
\pgfsetdash{}{0pt}%
\pgfpathmoveto{\pgfqpoint{0.650810in}{1.908857in}}%
\pgfpathlineto{\pgfqpoint{3.096898in}{1.908857in}}%
\pgfusepath{stroke}%
\end{pgfscope}%
\begin{pgfscope}%
\definecolor{textcolor}{rgb}{0.150000,0.150000,0.150000}%
\pgfsetstrokecolor{textcolor}%
\pgfsetfillcolor{textcolor}%
\pgftext[x=0.290741in,y=1.856050in,left,base]{\color{textcolor}\sffamily\fontsize{11.000000}{13.200000}\selectfont \(\displaystyle 300\)}%
\end{pgfscope}%
\begin{pgfscope}%
\definecolor{textcolor}{rgb}{0.150000,0.150000,0.150000}%
\pgfsetstrokecolor{textcolor}%
\pgfsetfillcolor{textcolor}%
\pgftext[x=0.235185in,y=1.300237in,,bottom,rotate=90.000000]{\color{textcolor}\sffamily\fontsize{11.000000}{13.200000}\selectfont Occurance}%
\end{pgfscope}%
\begin{pgfscope}%
\pgfpathrectangle{\pgfqpoint{0.650810in}{0.557870in}}{\pgfqpoint{2.446088in}{1.484734in}}%
\pgfusepath{clip}%
\pgfsetbuttcap%
\pgfsetmiterjoin%
\definecolor{currentfill}{rgb}{0.298039,0.447059,0.690196}%
\pgfsetfillcolor{currentfill}%
\pgfsetfillopacity{0.400000}%
\pgfsetlinewidth{1.003750pt}%
\definecolor{currentstroke}{rgb}{1.000000,1.000000,1.000000}%
\pgfsetstrokecolor{currentstroke}%
\pgfsetstrokeopacity{0.400000}%
\pgfsetdash{}{0pt}%
\pgfpathmoveto{\pgfqpoint{0.761996in}{0.557870in}}%
\pgfpathlineto{\pgfqpoint{0.984368in}{0.557870in}}%
\pgfpathlineto{\pgfqpoint{0.984368in}{0.557870in}}%
\pgfpathlineto{\pgfqpoint{0.761996in}{0.557870in}}%
\pgfpathclose%
\pgfusepath{stroke,fill}%
\end{pgfscope}%
\begin{pgfscope}%
\pgfpathrectangle{\pgfqpoint{0.650810in}{0.557870in}}{\pgfqpoint{2.446088in}{1.484734in}}%
\pgfusepath{clip}%
\pgfsetbuttcap%
\pgfsetmiterjoin%
\definecolor{currentfill}{rgb}{0.298039,0.447059,0.690196}%
\pgfsetfillcolor{currentfill}%
\pgfsetfillopacity{0.400000}%
\pgfsetlinewidth{1.003750pt}%
\definecolor{currentstroke}{rgb}{1.000000,1.000000,1.000000}%
\pgfsetstrokecolor{currentstroke}%
\pgfsetstrokeopacity{0.400000}%
\pgfsetdash{}{0pt}%
\pgfpathmoveto{\pgfqpoint{0.984368in}{0.557870in}}%
\pgfpathlineto{\pgfqpoint{1.206739in}{0.557870in}}%
\pgfpathlineto{\pgfqpoint{1.206739in}{0.557870in}}%
\pgfpathlineto{\pgfqpoint{0.984368in}{0.557870in}}%
\pgfpathclose%
\pgfusepath{stroke,fill}%
\end{pgfscope}%
\begin{pgfscope}%
\pgfpathrectangle{\pgfqpoint{0.650810in}{0.557870in}}{\pgfqpoint{2.446088in}{1.484734in}}%
\pgfusepath{clip}%
\pgfsetbuttcap%
\pgfsetmiterjoin%
\definecolor{currentfill}{rgb}{0.298039,0.447059,0.690196}%
\pgfsetfillcolor{currentfill}%
\pgfsetfillopacity{0.400000}%
\pgfsetlinewidth{1.003750pt}%
\definecolor{currentstroke}{rgb}{1.000000,1.000000,1.000000}%
\pgfsetstrokecolor{currentstroke}%
\pgfsetstrokeopacity{0.400000}%
\pgfsetdash{}{0pt}%
\pgfpathmoveto{\pgfqpoint{1.206739in}{0.557870in}}%
\pgfpathlineto{\pgfqpoint{1.429111in}{0.557870in}}%
\pgfpathlineto{\pgfqpoint{1.429111in}{0.557870in}}%
\pgfpathlineto{\pgfqpoint{1.206739in}{0.557870in}}%
\pgfpathclose%
\pgfusepath{stroke,fill}%
\end{pgfscope}%
\begin{pgfscope}%
\pgfpathrectangle{\pgfqpoint{0.650810in}{0.557870in}}{\pgfqpoint{2.446088in}{1.484734in}}%
\pgfusepath{clip}%
\pgfsetbuttcap%
\pgfsetmiterjoin%
\definecolor{currentfill}{rgb}{0.298039,0.447059,0.690196}%
\pgfsetfillcolor{currentfill}%
\pgfsetfillopacity{0.400000}%
\pgfsetlinewidth{1.003750pt}%
\definecolor{currentstroke}{rgb}{1.000000,1.000000,1.000000}%
\pgfsetstrokecolor{currentstroke}%
\pgfsetstrokeopacity{0.400000}%
\pgfsetdash{}{0pt}%
\pgfpathmoveto{\pgfqpoint{1.429111in}{0.557870in}}%
\pgfpathlineto{\pgfqpoint{1.651483in}{0.557870in}}%
\pgfpathlineto{\pgfqpoint{1.651483in}{0.557870in}}%
\pgfpathlineto{\pgfqpoint{1.429111in}{0.557870in}}%
\pgfpathclose%
\pgfusepath{stroke,fill}%
\end{pgfscope}%
\begin{pgfscope}%
\pgfpathrectangle{\pgfqpoint{0.650810in}{0.557870in}}{\pgfqpoint{2.446088in}{1.484734in}}%
\pgfusepath{clip}%
\pgfsetbuttcap%
\pgfsetmiterjoin%
\definecolor{currentfill}{rgb}{0.298039,0.447059,0.690196}%
\pgfsetfillcolor{currentfill}%
\pgfsetfillopacity{0.400000}%
\pgfsetlinewidth{1.003750pt}%
\definecolor{currentstroke}{rgb}{1.000000,1.000000,1.000000}%
\pgfsetstrokecolor{currentstroke}%
\pgfsetstrokeopacity{0.400000}%
\pgfsetdash{}{0pt}%
\pgfpathmoveto{\pgfqpoint{1.651483in}{0.557870in}}%
\pgfpathlineto{\pgfqpoint{1.873854in}{0.557870in}}%
\pgfpathlineto{\pgfqpoint{1.873854in}{0.557870in}}%
\pgfpathlineto{\pgfqpoint{1.651483in}{0.557870in}}%
\pgfpathclose%
\pgfusepath{stroke,fill}%
\end{pgfscope}%
\begin{pgfscope}%
\pgfpathrectangle{\pgfqpoint{0.650810in}{0.557870in}}{\pgfqpoint{2.446088in}{1.484734in}}%
\pgfusepath{clip}%
\pgfsetbuttcap%
\pgfsetmiterjoin%
\definecolor{currentfill}{rgb}{0.298039,0.447059,0.690196}%
\pgfsetfillcolor{currentfill}%
\pgfsetfillopacity{0.400000}%
\pgfsetlinewidth{1.003750pt}%
\definecolor{currentstroke}{rgb}{1.000000,1.000000,1.000000}%
\pgfsetstrokecolor{currentstroke}%
\pgfsetstrokeopacity{0.400000}%
\pgfsetdash{}{0pt}%
\pgfpathmoveto{\pgfqpoint{1.873854in}{0.557870in}}%
\pgfpathlineto{\pgfqpoint{2.096226in}{0.557870in}}%
\pgfpathlineto{\pgfqpoint{2.096226in}{1.971903in}}%
\pgfpathlineto{\pgfqpoint{1.873854in}{1.971903in}}%
\pgfpathclose%
\pgfusepath{stroke,fill}%
\end{pgfscope}%
\begin{pgfscope}%
\pgfpathrectangle{\pgfqpoint{0.650810in}{0.557870in}}{\pgfqpoint{2.446088in}{1.484734in}}%
\pgfusepath{clip}%
\pgfsetbuttcap%
\pgfsetmiterjoin%
\definecolor{currentfill}{rgb}{0.298039,0.447059,0.690196}%
\pgfsetfillcolor{currentfill}%
\pgfsetfillopacity{0.400000}%
\pgfsetlinewidth{1.003750pt}%
\definecolor{currentstroke}{rgb}{1.000000,1.000000,1.000000}%
\pgfsetstrokecolor{currentstroke}%
\pgfsetstrokeopacity{0.400000}%
\pgfsetdash{}{0pt}%
\pgfpathmoveto{\pgfqpoint{2.096226in}{0.557870in}}%
\pgfpathlineto{\pgfqpoint{2.318598in}{0.557870in}}%
\pgfpathlineto{\pgfqpoint{2.318598in}{0.557870in}}%
\pgfpathlineto{\pgfqpoint{2.096226in}{0.557870in}}%
\pgfpathclose%
\pgfusepath{stroke,fill}%
\end{pgfscope}%
\begin{pgfscope}%
\pgfpathrectangle{\pgfqpoint{0.650810in}{0.557870in}}{\pgfqpoint{2.446088in}{1.484734in}}%
\pgfusepath{clip}%
\pgfsetbuttcap%
\pgfsetmiterjoin%
\definecolor{currentfill}{rgb}{0.298039,0.447059,0.690196}%
\pgfsetfillcolor{currentfill}%
\pgfsetfillopacity{0.400000}%
\pgfsetlinewidth{1.003750pt}%
\definecolor{currentstroke}{rgb}{1.000000,1.000000,1.000000}%
\pgfsetstrokecolor{currentstroke}%
\pgfsetstrokeopacity{0.400000}%
\pgfsetdash{}{0pt}%
\pgfpathmoveto{\pgfqpoint{2.318598in}{0.557870in}}%
\pgfpathlineto{\pgfqpoint{2.540969in}{0.557870in}}%
\pgfpathlineto{\pgfqpoint{2.540969in}{0.557870in}}%
\pgfpathlineto{\pgfqpoint{2.318598in}{0.557870in}}%
\pgfpathclose%
\pgfusepath{stroke,fill}%
\end{pgfscope}%
\begin{pgfscope}%
\pgfpathrectangle{\pgfqpoint{0.650810in}{0.557870in}}{\pgfqpoint{2.446088in}{1.484734in}}%
\pgfusepath{clip}%
\pgfsetbuttcap%
\pgfsetmiterjoin%
\definecolor{currentfill}{rgb}{0.298039,0.447059,0.690196}%
\pgfsetfillcolor{currentfill}%
\pgfsetfillopacity{0.400000}%
\pgfsetlinewidth{1.003750pt}%
\definecolor{currentstroke}{rgb}{1.000000,1.000000,1.000000}%
\pgfsetstrokecolor{currentstroke}%
\pgfsetstrokeopacity{0.400000}%
\pgfsetdash{}{0pt}%
\pgfpathmoveto{\pgfqpoint{2.540969in}{0.557870in}}%
\pgfpathlineto{\pgfqpoint{2.763341in}{0.557870in}}%
\pgfpathlineto{\pgfqpoint{2.763341in}{0.557870in}}%
\pgfpathlineto{\pgfqpoint{2.540969in}{0.557870in}}%
\pgfpathclose%
\pgfusepath{stroke,fill}%
\end{pgfscope}%
\begin{pgfscope}%
\pgfpathrectangle{\pgfqpoint{0.650810in}{0.557870in}}{\pgfqpoint{2.446088in}{1.484734in}}%
\pgfusepath{clip}%
\pgfsetbuttcap%
\pgfsetmiterjoin%
\definecolor{currentfill}{rgb}{0.298039,0.447059,0.690196}%
\pgfsetfillcolor{currentfill}%
\pgfsetfillopacity{0.400000}%
\pgfsetlinewidth{1.003750pt}%
\definecolor{currentstroke}{rgb}{1.000000,1.000000,1.000000}%
\pgfsetstrokecolor{currentstroke}%
\pgfsetstrokeopacity{0.400000}%
\pgfsetdash{}{0pt}%
\pgfpathmoveto{\pgfqpoint{2.763341in}{0.557870in}}%
\pgfpathlineto{\pgfqpoint{2.985712in}{0.557870in}}%
\pgfpathlineto{\pgfqpoint{2.985712in}{0.557870in}}%
\pgfpathlineto{\pgfqpoint{2.763341in}{0.557870in}}%
\pgfpathclose%
\pgfusepath{stroke,fill}%
\end{pgfscope}%
\begin{pgfscope}%
\pgfsetrectcap%
\pgfsetmiterjoin%
\pgfsetlinewidth{1.254687pt}%
\definecolor{currentstroke}{rgb}{1.000000,1.000000,1.000000}%
\pgfsetstrokecolor{currentstroke}%
\pgfsetdash{}{0pt}%
\pgfpathmoveto{\pgfqpoint{0.650810in}{0.557870in}}%
\pgfpathlineto{\pgfqpoint{0.650810in}{2.042604in}}%
\pgfusepath{stroke}%
\end{pgfscope}%
\begin{pgfscope}%
\pgfsetrectcap%
\pgfsetmiterjoin%
\pgfsetlinewidth{1.254687pt}%
\definecolor{currentstroke}{rgb}{1.000000,1.000000,1.000000}%
\pgfsetstrokecolor{currentstroke}%
\pgfsetdash{}{0pt}%
\pgfpathmoveto{\pgfqpoint{3.096898in}{0.557870in}}%
\pgfpathlineto{\pgfqpoint{3.096898in}{2.042604in}}%
\pgfusepath{stroke}%
\end{pgfscope}%
\begin{pgfscope}%
\pgfsetrectcap%
\pgfsetmiterjoin%
\pgfsetlinewidth{1.254687pt}%
\definecolor{currentstroke}{rgb}{1.000000,1.000000,1.000000}%
\pgfsetstrokecolor{currentstroke}%
\pgfsetdash{}{0pt}%
\pgfpathmoveto{\pgfqpoint{0.650810in}{0.557870in}}%
\pgfpathlineto{\pgfqpoint{3.096898in}{0.557870in}}%
\pgfusepath{stroke}%
\end{pgfscope}%
\begin{pgfscope}%
\pgfsetrectcap%
\pgfsetmiterjoin%
\pgfsetlinewidth{1.254687pt}%
\definecolor{currentstroke}{rgb}{1.000000,1.000000,1.000000}%
\pgfsetstrokecolor{currentstroke}%
\pgfsetdash{}{0pt}%
\pgfpathmoveto{\pgfqpoint{0.650810in}{2.042604in}}%
\pgfpathlineto{\pgfqpoint{3.096898in}{2.042604in}}%
\pgfusepath{stroke}%
\end{pgfscope}%
\begin{pgfscope}%
\definecolor{textcolor}{rgb}{0.150000,0.150000,0.150000}%
\pgfsetstrokecolor{textcolor}%
\pgfsetfillcolor{textcolor}%
\pgftext[x=1.873854in,y=2.125938in,,base]{\color{textcolor}\sffamily\fontsize{11.000000}{13.200000}\selectfont (a)}%
\end{pgfscope}%
\begin{pgfscope}%
\pgfsetbuttcap%
\pgfsetmiterjoin%
\definecolor{currentfill}{rgb}{0.917647,0.917647,0.949020}%
\pgfsetfillcolor{currentfill}%
\pgfsetlinewidth{0.000000pt}%
\definecolor{currentstroke}{rgb}{0.000000,0.000000,0.000000}%
\pgfsetstrokecolor{currentstroke}%
\pgfsetstrokeopacity{0.000000}%
\pgfsetdash{}{0pt}%
\pgfpathmoveto{\pgfqpoint{3.793912in}{0.557870in}}%
\pgfpathlineto{\pgfqpoint{6.240000in}{0.557870in}}%
\pgfpathlineto{\pgfqpoint{6.240000in}{2.042604in}}%
\pgfpathlineto{\pgfqpoint{3.793912in}{2.042604in}}%
\pgfpathclose%
\pgfusepath{fill}%
\end{pgfscope}%
\begin{pgfscope}%
\pgfpathrectangle{\pgfqpoint{3.793912in}{0.557870in}}{\pgfqpoint{2.446088in}{1.484734in}}%
\pgfusepath{clip}%
\pgfsetroundcap%
\pgfsetroundjoin%
\pgfsetlinewidth{1.003750pt}%
\definecolor{currentstroke}{rgb}{1.000000,1.000000,1.000000}%
\pgfsetstrokecolor{currentstroke}%
\pgfsetdash{}{0pt}%
\pgfpathmoveto{\pgfqpoint{3.905098in}{0.557870in}}%
\pgfpathlineto{\pgfqpoint{3.905098in}{2.042604in}}%
\pgfusepath{stroke}%
\end{pgfscope}%
\begin{pgfscope}%
\definecolor{textcolor}{rgb}{0.150000,0.150000,0.150000}%
\pgfsetstrokecolor{textcolor}%
\pgfsetfillcolor{textcolor}%
\pgftext[x=3.905098in,y=0.425926in,,top]{\color{textcolor}\sffamily\fontsize{11.000000}{13.200000}\selectfont \(\displaystyle 0.00\)}%
\end{pgfscope}%
\begin{pgfscope}%
\pgfpathrectangle{\pgfqpoint{3.793912in}{0.557870in}}{\pgfqpoint{2.446088in}{1.484734in}}%
\pgfusepath{clip}%
\pgfsetroundcap%
\pgfsetroundjoin%
\pgfsetlinewidth{1.003750pt}%
\definecolor{currentstroke}{rgb}{1.000000,1.000000,1.000000}%
\pgfsetstrokecolor{currentstroke}%
\pgfsetdash{}{0pt}%
\pgfpathmoveto{\pgfqpoint{4.461027in}{0.557870in}}%
\pgfpathlineto{\pgfqpoint{4.461027in}{2.042604in}}%
\pgfusepath{stroke}%
\end{pgfscope}%
\begin{pgfscope}%
\definecolor{textcolor}{rgb}{0.150000,0.150000,0.150000}%
\pgfsetstrokecolor{textcolor}%
\pgfsetfillcolor{textcolor}%
\pgftext[x=4.461027in,y=0.425926in,,top]{\color{textcolor}\sffamily\fontsize{11.000000}{13.200000}\selectfont \(\displaystyle 0.25\)}%
\end{pgfscope}%
\begin{pgfscope}%
\pgfpathrectangle{\pgfqpoint{3.793912in}{0.557870in}}{\pgfqpoint{2.446088in}{1.484734in}}%
\pgfusepath{clip}%
\pgfsetroundcap%
\pgfsetroundjoin%
\pgfsetlinewidth{1.003750pt}%
\definecolor{currentstroke}{rgb}{1.000000,1.000000,1.000000}%
\pgfsetstrokecolor{currentstroke}%
\pgfsetdash{}{0pt}%
\pgfpathmoveto{\pgfqpoint{5.016956in}{0.557870in}}%
\pgfpathlineto{\pgfqpoint{5.016956in}{2.042604in}}%
\pgfusepath{stroke}%
\end{pgfscope}%
\begin{pgfscope}%
\definecolor{textcolor}{rgb}{0.150000,0.150000,0.150000}%
\pgfsetstrokecolor{textcolor}%
\pgfsetfillcolor{textcolor}%
\pgftext[x=5.016956in,y=0.425926in,,top]{\color{textcolor}\sffamily\fontsize{11.000000}{13.200000}\selectfont \(\displaystyle 0.50\)}%
\end{pgfscope}%
\begin{pgfscope}%
\pgfpathrectangle{\pgfqpoint{3.793912in}{0.557870in}}{\pgfqpoint{2.446088in}{1.484734in}}%
\pgfusepath{clip}%
\pgfsetroundcap%
\pgfsetroundjoin%
\pgfsetlinewidth{1.003750pt}%
\definecolor{currentstroke}{rgb}{1.000000,1.000000,1.000000}%
\pgfsetstrokecolor{currentstroke}%
\pgfsetdash{}{0pt}%
\pgfpathmoveto{\pgfqpoint{5.572885in}{0.557870in}}%
\pgfpathlineto{\pgfqpoint{5.572885in}{2.042604in}}%
\pgfusepath{stroke}%
\end{pgfscope}%
\begin{pgfscope}%
\definecolor{textcolor}{rgb}{0.150000,0.150000,0.150000}%
\pgfsetstrokecolor{textcolor}%
\pgfsetfillcolor{textcolor}%
\pgftext[x=5.572885in,y=0.425926in,,top]{\color{textcolor}\sffamily\fontsize{11.000000}{13.200000}\selectfont \(\displaystyle 0.75\)}%
\end{pgfscope}%
\begin{pgfscope}%
\pgfpathrectangle{\pgfqpoint{3.793912in}{0.557870in}}{\pgfqpoint{2.446088in}{1.484734in}}%
\pgfusepath{clip}%
\pgfsetroundcap%
\pgfsetroundjoin%
\pgfsetlinewidth{1.003750pt}%
\definecolor{currentstroke}{rgb}{1.000000,1.000000,1.000000}%
\pgfsetstrokecolor{currentstroke}%
\pgfsetdash{}{0pt}%
\pgfpathmoveto{\pgfqpoint{6.128814in}{0.557870in}}%
\pgfpathlineto{\pgfqpoint{6.128814in}{2.042604in}}%
\pgfusepath{stroke}%
\end{pgfscope}%
\begin{pgfscope}%
\definecolor{textcolor}{rgb}{0.150000,0.150000,0.150000}%
\pgfsetstrokecolor{textcolor}%
\pgfsetfillcolor{textcolor}%
\pgftext[x=6.128814in,y=0.425926in,,top]{\color{textcolor}\sffamily\fontsize{11.000000}{13.200000}\selectfont \(\displaystyle 1.00\)}%
\end{pgfscope}%
\begin{pgfscope}%
\definecolor{textcolor}{rgb}{0.150000,0.150000,0.150000}%
\pgfsetstrokecolor{textcolor}%
\pgfsetfillcolor{textcolor}%
\pgftext[x=5.016956in,y=0.235185in,,top]{\color{textcolor}\sffamily\fontsize{11.000000}{13.200000}\selectfont Specificity}%
\end{pgfscope}%
\begin{pgfscope}%
\pgfpathrectangle{\pgfqpoint{3.793912in}{0.557870in}}{\pgfqpoint{2.446088in}{1.484734in}}%
\pgfusepath{clip}%
\pgfsetroundcap%
\pgfsetroundjoin%
\pgfsetlinewidth{1.003750pt}%
\definecolor{currentstroke}{rgb}{1.000000,1.000000,1.000000}%
\pgfsetstrokecolor{currentstroke}%
\pgfsetdash{}{0pt}%
\pgfpathmoveto{\pgfqpoint{3.793912in}{0.625358in}}%
\pgfpathlineto{\pgfqpoint{6.240000in}{0.625358in}}%
\pgfusepath{stroke}%
\end{pgfscope}%
\begin{pgfscope}%
\definecolor{textcolor}{rgb}{0.150000,0.150000,0.150000}%
\pgfsetstrokecolor{textcolor}%
\pgfsetfillcolor{textcolor}%
\pgftext[x=3.467639in,y=0.572552in,left,base]{\color{textcolor}\sffamily\fontsize{11.000000}{13.200000}\selectfont \(\displaystyle 0.0\)}%
\end{pgfscope}%
\begin{pgfscope}%
\pgfpathrectangle{\pgfqpoint{3.793912in}{0.557870in}}{\pgfqpoint{2.446088in}{1.484734in}}%
\pgfusepath{clip}%
\pgfsetroundcap%
\pgfsetroundjoin%
\pgfsetlinewidth{1.003750pt}%
\definecolor{currentstroke}{rgb}{1.000000,1.000000,1.000000}%
\pgfsetstrokecolor{currentstroke}%
\pgfsetdash{}{0pt}%
\pgfpathmoveto{\pgfqpoint{3.793912in}{1.300237in}}%
\pgfpathlineto{\pgfqpoint{6.240000in}{1.300237in}}%
\pgfusepath{stroke}%
\end{pgfscope}%
\begin{pgfscope}%
\definecolor{textcolor}{rgb}{0.150000,0.150000,0.150000}%
\pgfsetstrokecolor{textcolor}%
\pgfsetfillcolor{textcolor}%
\pgftext[x=3.467639in,y=1.247431in,left,base]{\color{textcolor}\sffamily\fontsize{11.000000}{13.200000}\selectfont \(\displaystyle 0.5\)}%
\end{pgfscope}%
\begin{pgfscope}%
\pgfpathrectangle{\pgfqpoint{3.793912in}{0.557870in}}{\pgfqpoint{2.446088in}{1.484734in}}%
\pgfusepath{clip}%
\pgfsetroundcap%
\pgfsetroundjoin%
\pgfsetlinewidth{1.003750pt}%
\definecolor{currentstroke}{rgb}{1.000000,1.000000,1.000000}%
\pgfsetstrokecolor{currentstroke}%
\pgfsetdash{}{0pt}%
\pgfpathmoveto{\pgfqpoint{3.793912in}{1.975116in}}%
\pgfpathlineto{\pgfqpoint{6.240000in}{1.975116in}}%
\pgfusepath{stroke}%
\end{pgfscope}%
\begin{pgfscope}%
\definecolor{textcolor}{rgb}{0.150000,0.150000,0.150000}%
\pgfsetstrokecolor{textcolor}%
\pgfsetfillcolor{textcolor}%
\pgftext[x=3.467639in,y=1.922310in,left,base]{\color{textcolor}\sffamily\fontsize{11.000000}{13.200000}\selectfont \(\displaystyle 1.0\)}%
\end{pgfscope}%
\begin{pgfscope}%
\definecolor{textcolor}{rgb}{0.150000,0.150000,0.150000}%
\pgfsetstrokecolor{textcolor}%
\pgfsetfillcolor{textcolor}%
\pgftext[x=3.412083in,y=1.300237in,,bottom,rotate=90.000000]{\color{textcolor}\sffamily\fontsize{11.000000}{13.200000}\selectfont Sensitivity}%
\end{pgfscope}%
\begin{pgfscope}%
\pgfpathrectangle{\pgfqpoint{3.793912in}{0.557870in}}{\pgfqpoint{2.446088in}{1.484734in}}%
\pgfusepath{clip}%
\pgfsetbuttcap%
\pgfsetroundjoin%
\definecolor{currentfill}{rgb}{0.298039,0.447059,0.690196}%
\pgfsetfillcolor{currentfill}%
\pgfsetlinewidth{1.003750pt}%
\definecolor{currentstroke}{rgb}{0.298039,0.447059,0.690196}%
\pgfsetstrokecolor{currentstroke}%
\pgfsetdash{}{0pt}%
\pgfpathmoveto{\pgfqpoint{3.905098in}{1.936025in}}%
\pgfpathcurveto{\pgfqpoint{3.913334in}{1.936025in}}{\pgfqpoint{3.921234in}{1.939298in}}{\pgfqpoint{3.927058in}{1.945122in}}%
\pgfpathcurveto{\pgfqpoint{3.932882in}{1.950946in}}{\pgfqpoint{3.936155in}{1.958846in}}{\pgfqpoint{3.936155in}{1.967082in}}%
\pgfpathcurveto{\pgfqpoint{3.936155in}{1.975318in}}{\pgfqpoint{3.932882in}{1.983218in}}{\pgfqpoint{3.927058in}{1.989042in}}%
\pgfpathcurveto{\pgfqpoint{3.921234in}{1.994866in}}{\pgfqpoint{3.913334in}{1.998138in}}{\pgfqpoint{3.905098in}{1.998138in}}%
\pgfpathcurveto{\pgfqpoint{3.896862in}{1.998138in}}{\pgfqpoint{3.888962in}{1.994866in}}{\pgfqpoint{3.883138in}{1.989042in}}%
\pgfpathcurveto{\pgfqpoint{3.877314in}{1.983218in}}{\pgfqpoint{3.874042in}{1.975318in}}{\pgfqpoint{3.874042in}{1.967082in}}%
\pgfpathcurveto{\pgfqpoint{3.874042in}{1.958846in}}{\pgfqpoint{3.877314in}{1.950946in}}{\pgfqpoint{3.883138in}{1.945122in}}%
\pgfpathcurveto{\pgfqpoint{3.888962in}{1.939298in}}{\pgfqpoint{3.896862in}{1.936025in}}{\pgfqpoint{3.905098in}{1.936025in}}%
\pgfpathclose%
\pgfusepath{stroke,fill}%
\end{pgfscope}%
\begin{pgfscope}%
\pgfpathrectangle{\pgfqpoint{3.793912in}{0.557870in}}{\pgfqpoint{2.446088in}{1.484734in}}%
\pgfusepath{clip}%
\pgfsetbuttcap%
\pgfsetroundjoin%
\definecolor{currentfill}{rgb}{0.298039,0.447059,0.690196}%
\pgfsetfillcolor{currentfill}%
\pgfsetlinewidth{1.003750pt}%
\definecolor{currentstroke}{rgb}{0.298039,0.447059,0.690196}%
\pgfsetstrokecolor{currentstroke}%
\pgfsetdash{}{0pt}%
\pgfpathmoveto{\pgfqpoint{3.905098in}{1.261146in}}%
\pgfpathcurveto{\pgfqpoint{3.913334in}{1.261146in}}{\pgfqpoint{3.921234in}{1.264419in}}{\pgfqpoint{3.927058in}{1.270243in}}%
\pgfpathcurveto{\pgfqpoint{3.932882in}{1.276067in}}{\pgfqpoint{3.936155in}{1.283967in}}{\pgfqpoint{3.936155in}{1.292203in}}%
\pgfpathcurveto{\pgfqpoint{3.936155in}{1.300439in}}{\pgfqpoint{3.932882in}{1.308339in}}{\pgfqpoint{3.927058in}{1.314163in}}%
\pgfpathcurveto{\pgfqpoint{3.921234in}{1.319987in}}{\pgfqpoint{3.913334in}{1.323259in}}{\pgfqpoint{3.905098in}{1.323259in}}%
\pgfpathcurveto{\pgfqpoint{3.896862in}{1.323259in}}{\pgfqpoint{3.888962in}{1.319987in}}{\pgfqpoint{3.883138in}{1.314163in}}%
\pgfpathcurveto{\pgfqpoint{3.877314in}{1.308339in}}{\pgfqpoint{3.874042in}{1.300439in}}{\pgfqpoint{3.874042in}{1.292203in}}%
\pgfpathcurveto{\pgfqpoint{3.874042in}{1.283967in}}{\pgfqpoint{3.877314in}{1.276067in}}{\pgfqpoint{3.883138in}{1.270243in}}%
\pgfpathcurveto{\pgfqpoint{3.888962in}{1.264419in}}{\pgfqpoint{3.896862in}{1.261146in}}{\pgfqpoint{3.905098in}{1.261146in}}%
\pgfpathclose%
\pgfusepath{stroke,fill}%
\end{pgfscope}%
\begin{pgfscope}%
\pgfpathrectangle{\pgfqpoint{3.793912in}{0.557870in}}{\pgfqpoint{2.446088in}{1.484734in}}%
\pgfusepath{clip}%
\pgfsetbuttcap%
\pgfsetroundjoin%
\definecolor{currentfill}{rgb}{0.298039,0.447059,0.690196}%
\pgfsetfillcolor{currentfill}%
\pgfsetlinewidth{1.003750pt}%
\definecolor{currentstroke}{rgb}{0.298039,0.447059,0.690196}%
\pgfsetstrokecolor{currentstroke}%
\pgfsetdash{}{0pt}%
\pgfpathmoveto{\pgfqpoint{3.905098in}{1.261146in}}%
\pgfpathcurveto{\pgfqpoint{3.913334in}{1.261146in}}{\pgfqpoint{3.921234in}{1.264419in}}{\pgfqpoint{3.927058in}{1.270243in}}%
\pgfpathcurveto{\pgfqpoint{3.932882in}{1.276067in}}{\pgfqpoint{3.936155in}{1.283967in}}{\pgfqpoint{3.936155in}{1.292203in}}%
\pgfpathcurveto{\pgfqpoint{3.936155in}{1.300439in}}{\pgfqpoint{3.932882in}{1.308339in}}{\pgfqpoint{3.927058in}{1.314163in}}%
\pgfpathcurveto{\pgfqpoint{3.921234in}{1.319987in}}{\pgfqpoint{3.913334in}{1.323259in}}{\pgfqpoint{3.905098in}{1.323259in}}%
\pgfpathcurveto{\pgfqpoint{3.896862in}{1.323259in}}{\pgfqpoint{3.888962in}{1.319987in}}{\pgfqpoint{3.883138in}{1.314163in}}%
\pgfpathcurveto{\pgfqpoint{3.877314in}{1.308339in}}{\pgfqpoint{3.874042in}{1.300439in}}{\pgfqpoint{3.874042in}{1.292203in}}%
\pgfpathcurveto{\pgfqpoint{3.874042in}{1.283967in}}{\pgfqpoint{3.877314in}{1.276067in}}{\pgfqpoint{3.883138in}{1.270243in}}%
\pgfpathcurveto{\pgfqpoint{3.888962in}{1.264419in}}{\pgfqpoint{3.896862in}{1.261146in}}{\pgfqpoint{3.905098in}{1.261146in}}%
\pgfpathclose%
\pgfusepath{stroke,fill}%
\end{pgfscope}%
\begin{pgfscope}%
\pgfpathrectangle{\pgfqpoint{3.793912in}{0.557870in}}{\pgfqpoint{2.446088in}{1.484734in}}%
\pgfusepath{clip}%
\pgfsetbuttcap%
\pgfsetroundjoin%
\definecolor{currentfill}{rgb}{0.298039,0.447059,0.690196}%
\pgfsetfillcolor{currentfill}%
\pgfsetlinewidth{1.003750pt}%
\definecolor{currentstroke}{rgb}{0.298039,0.447059,0.690196}%
\pgfsetstrokecolor{currentstroke}%
\pgfsetdash{}{0pt}%
\pgfpathmoveto{\pgfqpoint{3.905098in}{1.325421in}}%
\pgfpathcurveto{\pgfqpoint{3.913334in}{1.325421in}}{\pgfqpoint{3.921234in}{1.328693in}}{\pgfqpoint{3.927058in}{1.334517in}}%
\pgfpathcurveto{\pgfqpoint{3.932882in}{1.340341in}}{\pgfqpoint{3.936155in}{1.348241in}}{\pgfqpoint{3.936155in}{1.356477in}}%
\pgfpathcurveto{\pgfqpoint{3.936155in}{1.364713in}}{\pgfqpoint{3.932882in}{1.372613in}}{\pgfqpoint{3.927058in}{1.378437in}}%
\pgfpathcurveto{\pgfqpoint{3.921234in}{1.384261in}}{\pgfqpoint{3.913334in}{1.387534in}}{\pgfqpoint{3.905098in}{1.387534in}}%
\pgfpathcurveto{\pgfqpoint{3.896862in}{1.387534in}}{\pgfqpoint{3.888962in}{1.384261in}}{\pgfqpoint{3.883138in}{1.378437in}}%
\pgfpathcurveto{\pgfqpoint{3.877314in}{1.372613in}}{\pgfqpoint{3.874042in}{1.364713in}}{\pgfqpoint{3.874042in}{1.356477in}}%
\pgfpathcurveto{\pgfqpoint{3.874042in}{1.348241in}}{\pgfqpoint{3.877314in}{1.340341in}}{\pgfqpoint{3.883138in}{1.334517in}}%
\pgfpathcurveto{\pgfqpoint{3.888962in}{1.328693in}}{\pgfqpoint{3.896862in}{1.325421in}}{\pgfqpoint{3.905098in}{1.325421in}}%
\pgfpathclose%
\pgfusepath{stroke,fill}%
\end{pgfscope}%
\begin{pgfscope}%
\pgfpathrectangle{\pgfqpoint{3.793912in}{0.557870in}}{\pgfqpoint{2.446088in}{1.484734in}}%
\pgfusepath{clip}%
\pgfsetbuttcap%
\pgfsetroundjoin%
\definecolor{currentfill}{rgb}{0.298039,0.447059,0.690196}%
\pgfsetfillcolor{currentfill}%
\pgfsetlinewidth{1.003750pt}%
\definecolor{currentstroke}{rgb}{0.298039,0.447059,0.690196}%
\pgfsetstrokecolor{currentstroke}%
\pgfsetdash{}{0pt}%
\pgfpathmoveto{\pgfqpoint{3.905098in}{1.124564in}}%
\pgfpathcurveto{\pgfqpoint{3.913334in}{1.124564in}}{\pgfqpoint{3.921234in}{1.127836in}}{\pgfqpoint{3.927058in}{1.133660in}}%
\pgfpathcurveto{\pgfqpoint{3.932882in}{1.139484in}}{\pgfqpoint{3.936155in}{1.147384in}}{\pgfqpoint{3.936155in}{1.155620in}}%
\pgfpathcurveto{\pgfqpoint{3.936155in}{1.163857in}}{\pgfqpoint{3.932882in}{1.171757in}}{\pgfqpoint{3.927058in}{1.177581in}}%
\pgfpathcurveto{\pgfqpoint{3.921234in}{1.183404in}}{\pgfqpoint{3.913334in}{1.186677in}}{\pgfqpoint{3.905098in}{1.186677in}}%
\pgfpathcurveto{\pgfqpoint{3.896862in}{1.186677in}}{\pgfqpoint{3.888962in}{1.183404in}}{\pgfqpoint{3.883138in}{1.177581in}}%
\pgfpathcurveto{\pgfqpoint{3.877314in}{1.171757in}}{\pgfqpoint{3.874042in}{1.163857in}}{\pgfqpoint{3.874042in}{1.155620in}}%
\pgfpathcurveto{\pgfqpoint{3.874042in}{1.147384in}}{\pgfqpoint{3.877314in}{1.139484in}}{\pgfqpoint{3.883138in}{1.133660in}}%
\pgfpathcurveto{\pgfqpoint{3.888962in}{1.127836in}}{\pgfqpoint{3.896862in}{1.124564in}}{\pgfqpoint{3.905098in}{1.124564in}}%
\pgfpathclose%
\pgfusepath{stroke,fill}%
\end{pgfscope}%
\begin{pgfscope}%
\pgfpathrectangle{\pgfqpoint{3.793912in}{0.557870in}}{\pgfqpoint{2.446088in}{1.484734in}}%
\pgfusepath{clip}%
\pgfsetbuttcap%
\pgfsetroundjoin%
\definecolor{currentfill}{rgb}{0.298039,0.447059,0.690196}%
\pgfsetfillcolor{currentfill}%
\pgfsetlinewidth{1.003750pt}%
\definecolor{currentstroke}{rgb}{0.298039,0.447059,0.690196}%
\pgfsetstrokecolor{currentstroke}%
\pgfsetdash{}{0pt}%
\pgfpathmoveto{\pgfqpoint{3.905098in}{0.779090in}}%
\pgfpathcurveto{\pgfqpoint{3.913334in}{0.779090in}}{\pgfqpoint{3.921234in}{0.782362in}}{\pgfqpoint{3.927058in}{0.788186in}}%
\pgfpathcurveto{\pgfqpoint{3.932882in}{0.794010in}}{\pgfqpoint{3.936155in}{0.801910in}}{\pgfqpoint{3.936155in}{0.810146in}}%
\pgfpathcurveto{\pgfqpoint{3.936155in}{0.818383in}}{\pgfqpoint{3.932882in}{0.826283in}}{\pgfqpoint{3.927058in}{0.832107in}}%
\pgfpathcurveto{\pgfqpoint{3.921234in}{0.837931in}}{\pgfqpoint{3.913334in}{0.841203in}}{\pgfqpoint{3.905098in}{0.841203in}}%
\pgfpathcurveto{\pgfqpoint{3.896862in}{0.841203in}}{\pgfqpoint{3.888962in}{0.837931in}}{\pgfqpoint{3.883138in}{0.832107in}}%
\pgfpathcurveto{\pgfqpoint{3.877314in}{0.826283in}}{\pgfqpoint{3.874042in}{0.818383in}}{\pgfqpoint{3.874042in}{0.810146in}}%
\pgfpathcurveto{\pgfqpoint{3.874042in}{0.801910in}}{\pgfqpoint{3.877314in}{0.794010in}}{\pgfqpoint{3.883138in}{0.788186in}}%
\pgfpathcurveto{\pgfqpoint{3.888962in}{0.782362in}}{\pgfqpoint{3.896862in}{0.779090in}}{\pgfqpoint{3.905098in}{0.779090in}}%
\pgfpathclose%
\pgfusepath{stroke,fill}%
\end{pgfscope}%
\begin{pgfscope}%
\pgfpathrectangle{\pgfqpoint{3.793912in}{0.557870in}}{\pgfqpoint{2.446088in}{1.484734in}}%
\pgfusepath{clip}%
\pgfsetbuttcap%
\pgfsetroundjoin%
\definecolor{currentfill}{rgb}{0.298039,0.447059,0.690196}%
\pgfsetfillcolor{currentfill}%
\pgfsetlinewidth{1.003750pt}%
\definecolor{currentstroke}{rgb}{0.298039,0.447059,0.690196}%
\pgfsetstrokecolor{currentstroke}%
\pgfsetdash{}{0pt}%
\pgfpathmoveto{\pgfqpoint{3.905098in}{0.851398in}}%
\pgfpathcurveto{\pgfqpoint{3.913334in}{0.851398in}}{\pgfqpoint{3.921234in}{0.854671in}}{\pgfqpoint{3.927058in}{0.860495in}}%
\pgfpathcurveto{\pgfqpoint{3.932882in}{0.866319in}}{\pgfqpoint{3.936155in}{0.874219in}}{\pgfqpoint{3.936155in}{0.882455in}}%
\pgfpathcurveto{\pgfqpoint{3.936155in}{0.890691in}}{\pgfqpoint{3.932882in}{0.898591in}}{\pgfqpoint{3.927058in}{0.904415in}}%
\pgfpathcurveto{\pgfqpoint{3.921234in}{0.910239in}}{\pgfqpoint{3.913334in}{0.913511in}}{\pgfqpoint{3.905098in}{0.913511in}}%
\pgfpathcurveto{\pgfqpoint{3.896862in}{0.913511in}}{\pgfqpoint{3.888962in}{0.910239in}}{\pgfqpoint{3.883138in}{0.904415in}}%
\pgfpathcurveto{\pgfqpoint{3.877314in}{0.898591in}}{\pgfqpoint{3.874042in}{0.890691in}}{\pgfqpoint{3.874042in}{0.882455in}}%
\pgfpathcurveto{\pgfqpoint{3.874042in}{0.874219in}}{\pgfqpoint{3.877314in}{0.866319in}}{\pgfqpoint{3.883138in}{0.860495in}}%
\pgfpathcurveto{\pgfqpoint{3.888962in}{0.854671in}}{\pgfqpoint{3.896862in}{0.851398in}}{\pgfqpoint{3.905098in}{0.851398in}}%
\pgfpathclose%
\pgfusepath{stroke,fill}%
\end{pgfscope}%
\begin{pgfscope}%
\pgfpathrectangle{\pgfqpoint{3.793912in}{0.557870in}}{\pgfqpoint{2.446088in}{1.484734in}}%
\pgfusepath{clip}%
\pgfsetbuttcap%
\pgfsetroundjoin%
\definecolor{currentfill}{rgb}{0.298039,0.447059,0.690196}%
\pgfsetfillcolor{currentfill}%
\pgfsetlinewidth{1.003750pt}%
\definecolor{currentstroke}{rgb}{0.298039,0.447059,0.690196}%
\pgfsetstrokecolor{currentstroke}%
\pgfsetdash{}{0pt}%
\pgfpathmoveto{\pgfqpoint{3.905098in}{1.936025in}}%
\pgfpathcurveto{\pgfqpoint{3.913334in}{1.936025in}}{\pgfqpoint{3.921234in}{1.939298in}}{\pgfqpoint{3.927058in}{1.945122in}}%
\pgfpathcurveto{\pgfqpoint{3.932882in}{1.950946in}}{\pgfqpoint{3.936155in}{1.958846in}}{\pgfqpoint{3.936155in}{1.967082in}}%
\pgfpathcurveto{\pgfqpoint{3.936155in}{1.975318in}}{\pgfqpoint{3.932882in}{1.983218in}}{\pgfqpoint{3.927058in}{1.989042in}}%
\pgfpathcurveto{\pgfqpoint{3.921234in}{1.994866in}}{\pgfqpoint{3.913334in}{1.998138in}}{\pgfqpoint{3.905098in}{1.998138in}}%
\pgfpathcurveto{\pgfqpoint{3.896862in}{1.998138in}}{\pgfqpoint{3.888962in}{1.994866in}}{\pgfqpoint{3.883138in}{1.989042in}}%
\pgfpathcurveto{\pgfqpoint{3.877314in}{1.983218in}}{\pgfqpoint{3.874042in}{1.975318in}}{\pgfqpoint{3.874042in}{1.967082in}}%
\pgfpathcurveto{\pgfqpoint{3.874042in}{1.958846in}}{\pgfqpoint{3.877314in}{1.950946in}}{\pgfqpoint{3.883138in}{1.945122in}}%
\pgfpathcurveto{\pgfqpoint{3.888962in}{1.939298in}}{\pgfqpoint{3.896862in}{1.936025in}}{\pgfqpoint{3.905098in}{1.936025in}}%
\pgfpathclose%
\pgfusepath{stroke,fill}%
\end{pgfscope}%
\begin{pgfscope}%
\pgfpathrectangle{\pgfqpoint{3.793912in}{0.557870in}}{\pgfqpoint{2.446088in}{1.484734in}}%
\pgfusepath{clip}%
\pgfsetbuttcap%
\pgfsetroundjoin%
\definecolor{currentfill}{rgb}{0.298039,0.447059,0.690196}%
\pgfsetfillcolor{currentfill}%
\pgfsetlinewidth{1.003750pt}%
\definecolor{currentstroke}{rgb}{0.298039,0.447059,0.690196}%
\pgfsetstrokecolor{currentstroke}%
\pgfsetdash{}{0pt}%
\pgfpathmoveto{\pgfqpoint{3.905098in}{1.759271in}}%
\pgfpathcurveto{\pgfqpoint{3.913334in}{1.759271in}}{\pgfqpoint{3.921234in}{1.762544in}}{\pgfqpoint{3.927058in}{1.768368in}}%
\pgfpathcurveto{\pgfqpoint{3.932882in}{1.774192in}}{\pgfqpoint{3.936155in}{1.782092in}}{\pgfqpoint{3.936155in}{1.790328in}}%
\pgfpathcurveto{\pgfqpoint{3.936155in}{1.798564in}}{\pgfqpoint{3.932882in}{1.806464in}}{\pgfqpoint{3.927058in}{1.812288in}}%
\pgfpathcurveto{\pgfqpoint{3.921234in}{1.818112in}}{\pgfqpoint{3.913334in}{1.821384in}}{\pgfqpoint{3.905098in}{1.821384in}}%
\pgfpathcurveto{\pgfqpoint{3.896862in}{1.821384in}}{\pgfqpoint{3.888962in}{1.818112in}}{\pgfqpoint{3.883138in}{1.812288in}}%
\pgfpathcurveto{\pgfqpoint{3.877314in}{1.806464in}}{\pgfqpoint{3.874042in}{1.798564in}}{\pgfqpoint{3.874042in}{1.790328in}}%
\pgfpathcurveto{\pgfqpoint{3.874042in}{1.782092in}}{\pgfqpoint{3.877314in}{1.774192in}}{\pgfqpoint{3.883138in}{1.768368in}}%
\pgfpathcurveto{\pgfqpoint{3.888962in}{1.762544in}}{\pgfqpoint{3.896862in}{1.759271in}}{\pgfqpoint{3.905098in}{1.759271in}}%
\pgfpathclose%
\pgfusepath{stroke,fill}%
\end{pgfscope}%
\begin{pgfscope}%
\pgfpathrectangle{\pgfqpoint{3.793912in}{0.557870in}}{\pgfqpoint{2.446088in}{1.484734in}}%
\pgfusepath{clip}%
\pgfsetbuttcap%
\pgfsetroundjoin%
\definecolor{currentfill}{rgb}{0.298039,0.447059,0.690196}%
\pgfsetfillcolor{currentfill}%
\pgfsetlinewidth{1.003750pt}%
\definecolor{currentstroke}{rgb}{0.298039,0.447059,0.690196}%
\pgfsetstrokecolor{currentstroke}%
\pgfsetdash{}{0pt}%
\pgfpathmoveto{\pgfqpoint{3.905098in}{1.936025in}}%
\pgfpathcurveto{\pgfqpoint{3.913334in}{1.936025in}}{\pgfqpoint{3.921234in}{1.939298in}}{\pgfqpoint{3.927058in}{1.945122in}}%
\pgfpathcurveto{\pgfqpoint{3.932882in}{1.950946in}}{\pgfqpoint{3.936155in}{1.958846in}}{\pgfqpoint{3.936155in}{1.967082in}}%
\pgfpathcurveto{\pgfqpoint{3.936155in}{1.975318in}}{\pgfqpoint{3.932882in}{1.983218in}}{\pgfqpoint{3.927058in}{1.989042in}}%
\pgfpathcurveto{\pgfqpoint{3.921234in}{1.994866in}}{\pgfqpoint{3.913334in}{1.998138in}}{\pgfqpoint{3.905098in}{1.998138in}}%
\pgfpathcurveto{\pgfqpoint{3.896862in}{1.998138in}}{\pgfqpoint{3.888962in}{1.994866in}}{\pgfqpoint{3.883138in}{1.989042in}}%
\pgfpathcurveto{\pgfqpoint{3.877314in}{1.983218in}}{\pgfqpoint{3.874042in}{1.975318in}}{\pgfqpoint{3.874042in}{1.967082in}}%
\pgfpathcurveto{\pgfqpoint{3.874042in}{1.958846in}}{\pgfqpoint{3.877314in}{1.950946in}}{\pgfqpoint{3.883138in}{1.945122in}}%
\pgfpathcurveto{\pgfqpoint{3.888962in}{1.939298in}}{\pgfqpoint{3.896862in}{1.936025in}}{\pgfqpoint{3.905098in}{1.936025in}}%
\pgfpathclose%
\pgfusepath{stroke,fill}%
\end{pgfscope}%
\begin{pgfscope}%
\pgfpathrectangle{\pgfqpoint{3.793912in}{0.557870in}}{\pgfqpoint{2.446088in}{1.484734in}}%
\pgfusepath{clip}%
\pgfsetbuttcap%
\pgfsetroundjoin%
\definecolor{currentfill}{rgb}{0.298039,0.447059,0.690196}%
\pgfsetfillcolor{currentfill}%
\pgfsetlinewidth{1.003750pt}%
\definecolor{currentstroke}{rgb}{0.298039,0.447059,0.690196}%
\pgfsetstrokecolor{currentstroke}%
\pgfsetdash{}{0pt}%
\pgfpathmoveto{\pgfqpoint{3.905098in}{1.510209in}}%
\pgfpathcurveto{\pgfqpoint{3.913334in}{1.510209in}}{\pgfqpoint{3.921234in}{1.513481in}}{\pgfqpoint{3.927058in}{1.519305in}}%
\pgfpathcurveto{\pgfqpoint{3.932882in}{1.525129in}}{\pgfqpoint{3.936155in}{1.533029in}}{\pgfqpoint{3.936155in}{1.541265in}}%
\pgfpathcurveto{\pgfqpoint{3.936155in}{1.549502in}}{\pgfqpoint{3.932882in}{1.557402in}}{\pgfqpoint{3.927058in}{1.563226in}}%
\pgfpathcurveto{\pgfqpoint{3.921234in}{1.569050in}}{\pgfqpoint{3.913334in}{1.572322in}}{\pgfqpoint{3.905098in}{1.572322in}}%
\pgfpathcurveto{\pgfqpoint{3.896862in}{1.572322in}}{\pgfqpoint{3.888962in}{1.569050in}}{\pgfqpoint{3.883138in}{1.563226in}}%
\pgfpathcurveto{\pgfqpoint{3.877314in}{1.557402in}}{\pgfqpoint{3.874042in}{1.549502in}}{\pgfqpoint{3.874042in}{1.541265in}}%
\pgfpathcurveto{\pgfqpoint{3.874042in}{1.533029in}}{\pgfqpoint{3.877314in}{1.525129in}}{\pgfqpoint{3.883138in}{1.519305in}}%
\pgfpathcurveto{\pgfqpoint{3.888962in}{1.513481in}}{\pgfqpoint{3.896862in}{1.510209in}}{\pgfqpoint{3.905098in}{1.510209in}}%
\pgfpathclose%
\pgfusepath{stroke,fill}%
\end{pgfscope}%
\begin{pgfscope}%
\pgfpathrectangle{\pgfqpoint{3.793912in}{0.557870in}}{\pgfqpoint{2.446088in}{1.484734in}}%
\pgfusepath{clip}%
\pgfsetbuttcap%
\pgfsetroundjoin%
\definecolor{currentfill}{rgb}{0.298039,0.447059,0.690196}%
\pgfsetfillcolor{currentfill}%
\pgfsetlinewidth{1.003750pt}%
\definecolor{currentstroke}{rgb}{0.298039,0.447059,0.690196}%
\pgfsetstrokecolor{currentstroke}%
\pgfsetdash{}{0pt}%
\pgfpathmoveto{\pgfqpoint{3.905098in}{1.936025in}}%
\pgfpathcurveto{\pgfqpoint{3.913334in}{1.936025in}}{\pgfqpoint{3.921234in}{1.939298in}}{\pgfqpoint{3.927058in}{1.945122in}}%
\pgfpathcurveto{\pgfqpoint{3.932882in}{1.950946in}}{\pgfqpoint{3.936155in}{1.958846in}}{\pgfqpoint{3.936155in}{1.967082in}}%
\pgfpathcurveto{\pgfqpoint{3.936155in}{1.975318in}}{\pgfqpoint{3.932882in}{1.983218in}}{\pgfqpoint{3.927058in}{1.989042in}}%
\pgfpathcurveto{\pgfqpoint{3.921234in}{1.994866in}}{\pgfqpoint{3.913334in}{1.998138in}}{\pgfqpoint{3.905098in}{1.998138in}}%
\pgfpathcurveto{\pgfqpoint{3.896862in}{1.998138in}}{\pgfqpoint{3.888962in}{1.994866in}}{\pgfqpoint{3.883138in}{1.989042in}}%
\pgfpathcurveto{\pgfqpoint{3.877314in}{1.983218in}}{\pgfqpoint{3.874042in}{1.975318in}}{\pgfqpoint{3.874042in}{1.967082in}}%
\pgfpathcurveto{\pgfqpoint{3.874042in}{1.958846in}}{\pgfqpoint{3.877314in}{1.950946in}}{\pgfqpoint{3.883138in}{1.945122in}}%
\pgfpathcurveto{\pgfqpoint{3.888962in}{1.939298in}}{\pgfqpoint{3.896862in}{1.936025in}}{\pgfqpoint{3.905098in}{1.936025in}}%
\pgfpathclose%
\pgfusepath{stroke,fill}%
\end{pgfscope}%
\begin{pgfscope}%
\pgfpathrectangle{\pgfqpoint{3.793912in}{0.557870in}}{\pgfqpoint{2.446088in}{1.484734in}}%
\pgfusepath{clip}%
\pgfsetbuttcap%
\pgfsetroundjoin%
\definecolor{currentfill}{rgb}{0.298039,0.447059,0.690196}%
\pgfsetfillcolor{currentfill}%
\pgfsetlinewidth{1.003750pt}%
\definecolor{currentstroke}{rgb}{0.298039,0.447059,0.690196}%
\pgfsetstrokecolor{currentstroke}%
\pgfsetdash{}{0pt}%
\pgfpathmoveto{\pgfqpoint{3.905098in}{1.936025in}}%
\pgfpathcurveto{\pgfqpoint{3.913334in}{1.936025in}}{\pgfqpoint{3.921234in}{1.939298in}}{\pgfqpoint{3.927058in}{1.945122in}}%
\pgfpathcurveto{\pgfqpoint{3.932882in}{1.950946in}}{\pgfqpoint{3.936155in}{1.958846in}}{\pgfqpoint{3.936155in}{1.967082in}}%
\pgfpathcurveto{\pgfqpoint{3.936155in}{1.975318in}}{\pgfqpoint{3.932882in}{1.983218in}}{\pgfqpoint{3.927058in}{1.989042in}}%
\pgfpathcurveto{\pgfqpoint{3.921234in}{1.994866in}}{\pgfqpoint{3.913334in}{1.998138in}}{\pgfqpoint{3.905098in}{1.998138in}}%
\pgfpathcurveto{\pgfqpoint{3.896862in}{1.998138in}}{\pgfqpoint{3.888962in}{1.994866in}}{\pgfqpoint{3.883138in}{1.989042in}}%
\pgfpathcurveto{\pgfqpoint{3.877314in}{1.983218in}}{\pgfqpoint{3.874042in}{1.975318in}}{\pgfqpoint{3.874042in}{1.967082in}}%
\pgfpathcurveto{\pgfqpoint{3.874042in}{1.958846in}}{\pgfqpoint{3.877314in}{1.950946in}}{\pgfqpoint{3.883138in}{1.945122in}}%
\pgfpathcurveto{\pgfqpoint{3.888962in}{1.939298in}}{\pgfqpoint{3.896862in}{1.936025in}}{\pgfqpoint{3.905098in}{1.936025in}}%
\pgfpathclose%
\pgfusepath{stroke,fill}%
\end{pgfscope}%
\begin{pgfscope}%
\pgfpathrectangle{\pgfqpoint{3.793912in}{0.557870in}}{\pgfqpoint{2.446088in}{1.484734in}}%
\pgfusepath{clip}%
\pgfsetbuttcap%
\pgfsetroundjoin%
\definecolor{currentfill}{rgb}{0.298039,0.447059,0.690196}%
\pgfsetfillcolor{currentfill}%
\pgfsetlinewidth{1.003750pt}%
\definecolor{currentstroke}{rgb}{0.298039,0.447059,0.690196}%
\pgfsetstrokecolor{currentstroke}%
\pgfsetdash{}{0pt}%
\pgfpathmoveto{\pgfqpoint{3.905098in}{1.936025in}}%
\pgfpathcurveto{\pgfqpoint{3.913334in}{1.936025in}}{\pgfqpoint{3.921234in}{1.939298in}}{\pgfqpoint{3.927058in}{1.945122in}}%
\pgfpathcurveto{\pgfqpoint{3.932882in}{1.950946in}}{\pgfqpoint{3.936155in}{1.958846in}}{\pgfqpoint{3.936155in}{1.967082in}}%
\pgfpathcurveto{\pgfqpoint{3.936155in}{1.975318in}}{\pgfqpoint{3.932882in}{1.983218in}}{\pgfqpoint{3.927058in}{1.989042in}}%
\pgfpathcurveto{\pgfqpoint{3.921234in}{1.994866in}}{\pgfqpoint{3.913334in}{1.998138in}}{\pgfqpoint{3.905098in}{1.998138in}}%
\pgfpathcurveto{\pgfqpoint{3.896862in}{1.998138in}}{\pgfqpoint{3.888962in}{1.994866in}}{\pgfqpoint{3.883138in}{1.989042in}}%
\pgfpathcurveto{\pgfqpoint{3.877314in}{1.983218in}}{\pgfqpoint{3.874042in}{1.975318in}}{\pgfqpoint{3.874042in}{1.967082in}}%
\pgfpathcurveto{\pgfqpoint{3.874042in}{1.958846in}}{\pgfqpoint{3.877314in}{1.950946in}}{\pgfqpoint{3.883138in}{1.945122in}}%
\pgfpathcurveto{\pgfqpoint{3.888962in}{1.939298in}}{\pgfqpoint{3.896862in}{1.936025in}}{\pgfqpoint{3.905098in}{1.936025in}}%
\pgfpathclose%
\pgfusepath{stroke,fill}%
\end{pgfscope}%
\begin{pgfscope}%
\pgfpathrectangle{\pgfqpoint{3.793912in}{0.557870in}}{\pgfqpoint{2.446088in}{1.484734in}}%
\pgfusepath{clip}%
\pgfsetbuttcap%
\pgfsetroundjoin%
\definecolor{currentfill}{rgb}{0.298039,0.447059,0.690196}%
\pgfsetfillcolor{currentfill}%
\pgfsetlinewidth{1.003750pt}%
\definecolor{currentstroke}{rgb}{0.298039,0.447059,0.690196}%
\pgfsetstrokecolor{currentstroke}%
\pgfsetdash{}{0pt}%
\pgfpathmoveto{\pgfqpoint{3.905098in}{1.936025in}}%
\pgfpathcurveto{\pgfqpoint{3.913334in}{1.936025in}}{\pgfqpoint{3.921234in}{1.939298in}}{\pgfqpoint{3.927058in}{1.945122in}}%
\pgfpathcurveto{\pgfqpoint{3.932882in}{1.950946in}}{\pgfqpoint{3.936155in}{1.958846in}}{\pgfqpoint{3.936155in}{1.967082in}}%
\pgfpathcurveto{\pgfqpoint{3.936155in}{1.975318in}}{\pgfqpoint{3.932882in}{1.983218in}}{\pgfqpoint{3.927058in}{1.989042in}}%
\pgfpathcurveto{\pgfqpoint{3.921234in}{1.994866in}}{\pgfqpoint{3.913334in}{1.998138in}}{\pgfqpoint{3.905098in}{1.998138in}}%
\pgfpathcurveto{\pgfqpoint{3.896862in}{1.998138in}}{\pgfqpoint{3.888962in}{1.994866in}}{\pgfqpoint{3.883138in}{1.989042in}}%
\pgfpathcurveto{\pgfqpoint{3.877314in}{1.983218in}}{\pgfqpoint{3.874042in}{1.975318in}}{\pgfqpoint{3.874042in}{1.967082in}}%
\pgfpathcurveto{\pgfqpoint{3.874042in}{1.958846in}}{\pgfqpoint{3.877314in}{1.950946in}}{\pgfqpoint{3.883138in}{1.945122in}}%
\pgfpathcurveto{\pgfqpoint{3.888962in}{1.939298in}}{\pgfqpoint{3.896862in}{1.936025in}}{\pgfqpoint{3.905098in}{1.936025in}}%
\pgfpathclose%
\pgfusepath{stroke,fill}%
\end{pgfscope}%
\begin{pgfscope}%
\pgfpathrectangle{\pgfqpoint{3.793912in}{0.557870in}}{\pgfqpoint{2.446088in}{1.484734in}}%
\pgfusepath{clip}%
\pgfsetbuttcap%
\pgfsetroundjoin%
\definecolor{currentfill}{rgb}{0.298039,0.447059,0.690196}%
\pgfsetfillcolor{currentfill}%
\pgfsetlinewidth{1.003750pt}%
\definecolor{currentstroke}{rgb}{0.298039,0.447059,0.690196}%
\pgfsetstrokecolor{currentstroke}%
\pgfsetdash{}{0pt}%
\pgfpathmoveto{\pgfqpoint{3.905098in}{1.036187in}}%
\pgfpathcurveto{\pgfqpoint{3.913334in}{1.036187in}}{\pgfqpoint{3.921234in}{1.039459in}}{\pgfqpoint{3.927058in}{1.045283in}}%
\pgfpathcurveto{\pgfqpoint{3.932882in}{1.051107in}}{\pgfqpoint{3.936155in}{1.059007in}}{\pgfqpoint{3.936155in}{1.067243in}}%
\pgfpathcurveto{\pgfqpoint{3.936155in}{1.075480in}}{\pgfqpoint{3.932882in}{1.083380in}}{\pgfqpoint{3.927058in}{1.089204in}}%
\pgfpathcurveto{\pgfqpoint{3.921234in}{1.095027in}}{\pgfqpoint{3.913334in}{1.098300in}}{\pgfqpoint{3.905098in}{1.098300in}}%
\pgfpathcurveto{\pgfqpoint{3.896862in}{1.098300in}}{\pgfqpoint{3.888962in}{1.095027in}}{\pgfqpoint{3.883138in}{1.089204in}}%
\pgfpathcurveto{\pgfqpoint{3.877314in}{1.083380in}}{\pgfqpoint{3.874042in}{1.075480in}}{\pgfqpoint{3.874042in}{1.067243in}}%
\pgfpathcurveto{\pgfqpoint{3.874042in}{1.059007in}}{\pgfqpoint{3.877314in}{1.051107in}}{\pgfqpoint{3.883138in}{1.045283in}}%
\pgfpathcurveto{\pgfqpoint{3.888962in}{1.039459in}}{\pgfqpoint{3.896862in}{1.036187in}}{\pgfqpoint{3.905098in}{1.036187in}}%
\pgfpathclose%
\pgfusepath{stroke,fill}%
\end{pgfscope}%
\begin{pgfscope}%
\pgfpathrectangle{\pgfqpoint{3.793912in}{0.557870in}}{\pgfqpoint{2.446088in}{1.484734in}}%
\pgfusepath{clip}%
\pgfsetbuttcap%
\pgfsetroundjoin%
\definecolor{currentfill}{rgb}{0.298039,0.447059,0.690196}%
\pgfsetfillcolor{currentfill}%
\pgfsetlinewidth{1.003750pt}%
\definecolor{currentstroke}{rgb}{0.298039,0.447059,0.690196}%
\pgfsetstrokecolor{currentstroke}%
\pgfsetdash{}{0pt}%
\pgfpathmoveto{\pgfqpoint{3.905098in}{1.936025in}}%
\pgfpathcurveto{\pgfqpoint{3.913334in}{1.936025in}}{\pgfqpoint{3.921234in}{1.939298in}}{\pgfqpoint{3.927058in}{1.945122in}}%
\pgfpathcurveto{\pgfqpoint{3.932882in}{1.950946in}}{\pgfqpoint{3.936155in}{1.958846in}}{\pgfqpoint{3.936155in}{1.967082in}}%
\pgfpathcurveto{\pgfqpoint{3.936155in}{1.975318in}}{\pgfqpoint{3.932882in}{1.983218in}}{\pgfqpoint{3.927058in}{1.989042in}}%
\pgfpathcurveto{\pgfqpoint{3.921234in}{1.994866in}}{\pgfqpoint{3.913334in}{1.998138in}}{\pgfqpoint{3.905098in}{1.998138in}}%
\pgfpathcurveto{\pgfqpoint{3.896862in}{1.998138in}}{\pgfqpoint{3.888962in}{1.994866in}}{\pgfqpoint{3.883138in}{1.989042in}}%
\pgfpathcurveto{\pgfqpoint{3.877314in}{1.983218in}}{\pgfqpoint{3.874042in}{1.975318in}}{\pgfqpoint{3.874042in}{1.967082in}}%
\pgfpathcurveto{\pgfqpoint{3.874042in}{1.958846in}}{\pgfqpoint{3.877314in}{1.950946in}}{\pgfqpoint{3.883138in}{1.945122in}}%
\pgfpathcurveto{\pgfqpoint{3.888962in}{1.939298in}}{\pgfqpoint{3.896862in}{1.936025in}}{\pgfqpoint{3.905098in}{1.936025in}}%
\pgfpathclose%
\pgfusepath{stroke,fill}%
\end{pgfscope}%
\begin{pgfscope}%
\pgfpathrectangle{\pgfqpoint{3.793912in}{0.557870in}}{\pgfqpoint{2.446088in}{1.484734in}}%
\pgfusepath{clip}%
\pgfsetbuttcap%
\pgfsetroundjoin%
\definecolor{currentfill}{rgb}{0.298039,0.447059,0.690196}%
\pgfsetfillcolor{currentfill}%
\pgfsetlinewidth{1.003750pt}%
\definecolor{currentstroke}{rgb}{0.298039,0.447059,0.690196}%
\pgfsetstrokecolor{currentstroke}%
\pgfsetdash{}{0pt}%
\pgfpathmoveto{\pgfqpoint{3.905098in}{1.293284in}}%
\pgfpathcurveto{\pgfqpoint{3.913334in}{1.293284in}}{\pgfqpoint{3.921234in}{1.296556in}}{\pgfqpoint{3.927058in}{1.302380in}}%
\pgfpathcurveto{\pgfqpoint{3.932882in}{1.308204in}}{\pgfqpoint{3.936155in}{1.316104in}}{\pgfqpoint{3.936155in}{1.324340in}}%
\pgfpathcurveto{\pgfqpoint{3.936155in}{1.332576in}}{\pgfqpoint{3.932882in}{1.340476in}}{\pgfqpoint{3.927058in}{1.346300in}}%
\pgfpathcurveto{\pgfqpoint{3.921234in}{1.352124in}}{\pgfqpoint{3.913334in}{1.355397in}}{\pgfqpoint{3.905098in}{1.355397in}}%
\pgfpathcurveto{\pgfqpoint{3.896862in}{1.355397in}}{\pgfqpoint{3.888962in}{1.352124in}}{\pgfqpoint{3.883138in}{1.346300in}}%
\pgfpathcurveto{\pgfqpoint{3.877314in}{1.340476in}}{\pgfqpoint{3.874042in}{1.332576in}}{\pgfqpoint{3.874042in}{1.324340in}}%
\pgfpathcurveto{\pgfqpoint{3.874042in}{1.316104in}}{\pgfqpoint{3.877314in}{1.308204in}}{\pgfqpoint{3.883138in}{1.302380in}}%
\pgfpathcurveto{\pgfqpoint{3.888962in}{1.296556in}}{\pgfqpoint{3.896862in}{1.293284in}}{\pgfqpoint{3.905098in}{1.293284in}}%
\pgfpathclose%
\pgfusepath{stroke,fill}%
\end{pgfscope}%
\begin{pgfscope}%
\pgfpathrectangle{\pgfqpoint{3.793912in}{0.557870in}}{\pgfqpoint{2.446088in}{1.484734in}}%
\pgfusepath{clip}%
\pgfsetbuttcap%
\pgfsetroundjoin%
\definecolor{currentfill}{rgb}{0.298039,0.447059,0.690196}%
\pgfsetfillcolor{currentfill}%
\pgfsetlinewidth{1.003750pt}%
\definecolor{currentstroke}{rgb}{0.298039,0.447059,0.690196}%
\pgfsetstrokecolor{currentstroke}%
\pgfsetdash{}{0pt}%
\pgfpathmoveto{\pgfqpoint{3.905098in}{1.936025in}}%
\pgfpathcurveto{\pgfqpoint{3.913334in}{1.936025in}}{\pgfqpoint{3.921234in}{1.939298in}}{\pgfqpoint{3.927058in}{1.945122in}}%
\pgfpathcurveto{\pgfqpoint{3.932882in}{1.950946in}}{\pgfqpoint{3.936155in}{1.958846in}}{\pgfqpoint{3.936155in}{1.967082in}}%
\pgfpathcurveto{\pgfqpoint{3.936155in}{1.975318in}}{\pgfqpoint{3.932882in}{1.983218in}}{\pgfqpoint{3.927058in}{1.989042in}}%
\pgfpathcurveto{\pgfqpoint{3.921234in}{1.994866in}}{\pgfqpoint{3.913334in}{1.998138in}}{\pgfqpoint{3.905098in}{1.998138in}}%
\pgfpathcurveto{\pgfqpoint{3.896862in}{1.998138in}}{\pgfqpoint{3.888962in}{1.994866in}}{\pgfqpoint{3.883138in}{1.989042in}}%
\pgfpathcurveto{\pgfqpoint{3.877314in}{1.983218in}}{\pgfqpoint{3.874042in}{1.975318in}}{\pgfqpoint{3.874042in}{1.967082in}}%
\pgfpathcurveto{\pgfqpoint{3.874042in}{1.958846in}}{\pgfqpoint{3.877314in}{1.950946in}}{\pgfqpoint{3.883138in}{1.945122in}}%
\pgfpathcurveto{\pgfqpoint{3.888962in}{1.939298in}}{\pgfqpoint{3.896862in}{1.936025in}}{\pgfqpoint{3.905098in}{1.936025in}}%
\pgfpathclose%
\pgfusepath{stroke,fill}%
\end{pgfscope}%
\begin{pgfscope}%
\pgfpathrectangle{\pgfqpoint{3.793912in}{0.557870in}}{\pgfqpoint{2.446088in}{1.484734in}}%
\pgfusepath{clip}%
\pgfsetbuttcap%
\pgfsetroundjoin%
\definecolor{currentfill}{rgb}{0.298039,0.447059,0.690196}%
\pgfsetfillcolor{currentfill}%
\pgfsetlinewidth{1.003750pt}%
\definecolor{currentstroke}{rgb}{0.298039,0.447059,0.690196}%
\pgfsetstrokecolor{currentstroke}%
\pgfsetdash{}{0pt}%
\pgfpathmoveto{\pgfqpoint{3.905098in}{1.936025in}}%
\pgfpathcurveto{\pgfqpoint{3.913334in}{1.936025in}}{\pgfqpoint{3.921234in}{1.939298in}}{\pgfqpoint{3.927058in}{1.945122in}}%
\pgfpathcurveto{\pgfqpoint{3.932882in}{1.950946in}}{\pgfqpoint{3.936155in}{1.958846in}}{\pgfqpoint{3.936155in}{1.967082in}}%
\pgfpathcurveto{\pgfqpoint{3.936155in}{1.975318in}}{\pgfqpoint{3.932882in}{1.983218in}}{\pgfqpoint{3.927058in}{1.989042in}}%
\pgfpathcurveto{\pgfqpoint{3.921234in}{1.994866in}}{\pgfqpoint{3.913334in}{1.998138in}}{\pgfqpoint{3.905098in}{1.998138in}}%
\pgfpathcurveto{\pgfqpoint{3.896862in}{1.998138in}}{\pgfqpoint{3.888962in}{1.994866in}}{\pgfqpoint{3.883138in}{1.989042in}}%
\pgfpathcurveto{\pgfqpoint{3.877314in}{1.983218in}}{\pgfqpoint{3.874042in}{1.975318in}}{\pgfqpoint{3.874042in}{1.967082in}}%
\pgfpathcurveto{\pgfqpoint{3.874042in}{1.958846in}}{\pgfqpoint{3.877314in}{1.950946in}}{\pgfqpoint{3.883138in}{1.945122in}}%
\pgfpathcurveto{\pgfqpoint{3.888962in}{1.939298in}}{\pgfqpoint{3.896862in}{1.936025in}}{\pgfqpoint{3.905098in}{1.936025in}}%
\pgfpathclose%
\pgfusepath{stroke,fill}%
\end{pgfscope}%
\begin{pgfscope}%
\pgfpathrectangle{\pgfqpoint{3.793912in}{0.557870in}}{\pgfqpoint{2.446088in}{1.484734in}}%
\pgfusepath{clip}%
\pgfsetbuttcap%
\pgfsetroundjoin%
\definecolor{currentfill}{rgb}{0.298039,0.447059,0.690196}%
\pgfsetfillcolor{currentfill}%
\pgfsetlinewidth{1.003750pt}%
\definecolor{currentstroke}{rgb}{0.298039,0.447059,0.690196}%
\pgfsetstrokecolor{currentstroke}%
\pgfsetdash{}{0pt}%
\pgfpathmoveto{\pgfqpoint{3.905098in}{1.936025in}}%
\pgfpathcurveto{\pgfqpoint{3.913334in}{1.936025in}}{\pgfqpoint{3.921234in}{1.939298in}}{\pgfqpoint{3.927058in}{1.945122in}}%
\pgfpathcurveto{\pgfqpoint{3.932882in}{1.950946in}}{\pgfqpoint{3.936155in}{1.958846in}}{\pgfqpoint{3.936155in}{1.967082in}}%
\pgfpathcurveto{\pgfqpoint{3.936155in}{1.975318in}}{\pgfqpoint{3.932882in}{1.983218in}}{\pgfqpoint{3.927058in}{1.989042in}}%
\pgfpathcurveto{\pgfqpoint{3.921234in}{1.994866in}}{\pgfqpoint{3.913334in}{1.998138in}}{\pgfqpoint{3.905098in}{1.998138in}}%
\pgfpathcurveto{\pgfqpoint{3.896862in}{1.998138in}}{\pgfqpoint{3.888962in}{1.994866in}}{\pgfqpoint{3.883138in}{1.989042in}}%
\pgfpathcurveto{\pgfqpoint{3.877314in}{1.983218in}}{\pgfqpoint{3.874042in}{1.975318in}}{\pgfqpoint{3.874042in}{1.967082in}}%
\pgfpathcurveto{\pgfqpoint{3.874042in}{1.958846in}}{\pgfqpoint{3.877314in}{1.950946in}}{\pgfqpoint{3.883138in}{1.945122in}}%
\pgfpathcurveto{\pgfqpoint{3.888962in}{1.939298in}}{\pgfqpoint{3.896862in}{1.936025in}}{\pgfqpoint{3.905098in}{1.936025in}}%
\pgfpathclose%
\pgfusepath{stroke,fill}%
\end{pgfscope}%
\begin{pgfscope}%
\pgfpathrectangle{\pgfqpoint{3.793912in}{0.557870in}}{\pgfqpoint{2.446088in}{1.484734in}}%
\pgfusepath{clip}%
\pgfsetbuttcap%
\pgfsetroundjoin%
\definecolor{currentfill}{rgb}{0.298039,0.447059,0.690196}%
\pgfsetfillcolor{currentfill}%
\pgfsetlinewidth{1.003750pt}%
\definecolor{currentstroke}{rgb}{0.298039,0.447059,0.690196}%
\pgfsetstrokecolor{currentstroke}%
\pgfsetdash{}{0pt}%
\pgfpathmoveto{\pgfqpoint{3.905098in}{1.936025in}}%
\pgfpathcurveto{\pgfqpoint{3.913334in}{1.936025in}}{\pgfqpoint{3.921234in}{1.939298in}}{\pgfqpoint{3.927058in}{1.945122in}}%
\pgfpathcurveto{\pgfqpoint{3.932882in}{1.950946in}}{\pgfqpoint{3.936155in}{1.958846in}}{\pgfqpoint{3.936155in}{1.967082in}}%
\pgfpathcurveto{\pgfqpoint{3.936155in}{1.975318in}}{\pgfqpoint{3.932882in}{1.983218in}}{\pgfqpoint{3.927058in}{1.989042in}}%
\pgfpathcurveto{\pgfqpoint{3.921234in}{1.994866in}}{\pgfqpoint{3.913334in}{1.998138in}}{\pgfqpoint{3.905098in}{1.998138in}}%
\pgfpathcurveto{\pgfqpoint{3.896862in}{1.998138in}}{\pgfqpoint{3.888962in}{1.994866in}}{\pgfqpoint{3.883138in}{1.989042in}}%
\pgfpathcurveto{\pgfqpoint{3.877314in}{1.983218in}}{\pgfqpoint{3.874042in}{1.975318in}}{\pgfqpoint{3.874042in}{1.967082in}}%
\pgfpathcurveto{\pgfqpoint{3.874042in}{1.958846in}}{\pgfqpoint{3.877314in}{1.950946in}}{\pgfqpoint{3.883138in}{1.945122in}}%
\pgfpathcurveto{\pgfqpoint{3.888962in}{1.939298in}}{\pgfqpoint{3.896862in}{1.936025in}}{\pgfqpoint{3.905098in}{1.936025in}}%
\pgfpathclose%
\pgfusepath{stroke,fill}%
\end{pgfscope}%
\begin{pgfscope}%
\pgfpathrectangle{\pgfqpoint{3.793912in}{0.557870in}}{\pgfqpoint{2.446088in}{1.484734in}}%
\pgfusepath{clip}%
\pgfsetbuttcap%
\pgfsetroundjoin%
\definecolor{currentfill}{rgb}{0.298039,0.447059,0.690196}%
\pgfsetfillcolor{currentfill}%
\pgfsetlinewidth{1.003750pt}%
\definecolor{currentstroke}{rgb}{0.298039,0.447059,0.690196}%
\pgfsetstrokecolor{currentstroke}%
\pgfsetdash{}{0pt}%
\pgfpathmoveto{\pgfqpoint{3.905098in}{1.220975in}}%
\pgfpathcurveto{\pgfqpoint{3.913334in}{1.220975in}}{\pgfqpoint{3.921234in}{1.224247in}}{\pgfqpoint{3.927058in}{1.230071in}}%
\pgfpathcurveto{\pgfqpoint{3.932882in}{1.235895in}}{\pgfqpoint{3.936155in}{1.243795in}}{\pgfqpoint{3.936155in}{1.252032in}}%
\pgfpathcurveto{\pgfqpoint{3.936155in}{1.260268in}}{\pgfqpoint{3.932882in}{1.268168in}}{\pgfqpoint{3.927058in}{1.273992in}}%
\pgfpathcurveto{\pgfqpoint{3.921234in}{1.279816in}}{\pgfqpoint{3.913334in}{1.283088in}}{\pgfqpoint{3.905098in}{1.283088in}}%
\pgfpathcurveto{\pgfqpoint{3.896862in}{1.283088in}}{\pgfqpoint{3.888962in}{1.279816in}}{\pgfqpoint{3.883138in}{1.273992in}}%
\pgfpathcurveto{\pgfqpoint{3.877314in}{1.268168in}}{\pgfqpoint{3.874042in}{1.260268in}}{\pgfqpoint{3.874042in}{1.252032in}}%
\pgfpathcurveto{\pgfqpoint{3.874042in}{1.243795in}}{\pgfqpoint{3.877314in}{1.235895in}}{\pgfqpoint{3.883138in}{1.230071in}}%
\pgfpathcurveto{\pgfqpoint{3.888962in}{1.224247in}}{\pgfqpoint{3.896862in}{1.220975in}}{\pgfqpoint{3.905098in}{1.220975in}}%
\pgfpathclose%
\pgfusepath{stroke,fill}%
\end{pgfscope}%
\begin{pgfscope}%
\pgfpathrectangle{\pgfqpoint{3.793912in}{0.557870in}}{\pgfqpoint{2.446088in}{1.484734in}}%
\pgfusepath{clip}%
\pgfsetbuttcap%
\pgfsetroundjoin%
\definecolor{currentfill}{rgb}{0.298039,0.447059,0.690196}%
\pgfsetfillcolor{currentfill}%
\pgfsetlinewidth{1.003750pt}%
\definecolor{currentstroke}{rgb}{0.298039,0.447059,0.690196}%
\pgfsetstrokecolor{currentstroke}%
\pgfsetdash{}{0pt}%
\pgfpathmoveto{\pgfqpoint{3.905098in}{1.405763in}}%
\pgfpathcurveto{\pgfqpoint{3.913334in}{1.405763in}}{\pgfqpoint{3.921234in}{1.409036in}}{\pgfqpoint{3.927058in}{1.414860in}}%
\pgfpathcurveto{\pgfqpoint{3.932882in}{1.420684in}}{\pgfqpoint{3.936155in}{1.428584in}}{\pgfqpoint{3.936155in}{1.436820in}}%
\pgfpathcurveto{\pgfqpoint{3.936155in}{1.445056in}}{\pgfqpoint{3.932882in}{1.452956in}}{\pgfqpoint{3.927058in}{1.458780in}}%
\pgfpathcurveto{\pgfqpoint{3.921234in}{1.464604in}}{\pgfqpoint{3.913334in}{1.467876in}}{\pgfqpoint{3.905098in}{1.467876in}}%
\pgfpathcurveto{\pgfqpoint{3.896862in}{1.467876in}}{\pgfqpoint{3.888962in}{1.464604in}}{\pgfqpoint{3.883138in}{1.458780in}}%
\pgfpathcurveto{\pgfqpoint{3.877314in}{1.452956in}}{\pgfqpoint{3.874042in}{1.445056in}}{\pgfqpoint{3.874042in}{1.436820in}}%
\pgfpathcurveto{\pgfqpoint{3.874042in}{1.428584in}}{\pgfqpoint{3.877314in}{1.420684in}}{\pgfqpoint{3.883138in}{1.414860in}}%
\pgfpathcurveto{\pgfqpoint{3.888962in}{1.409036in}}{\pgfqpoint{3.896862in}{1.405763in}}{\pgfqpoint{3.905098in}{1.405763in}}%
\pgfpathclose%
\pgfusepath{stroke,fill}%
\end{pgfscope}%
\begin{pgfscope}%
\pgfpathrectangle{\pgfqpoint{3.793912in}{0.557870in}}{\pgfqpoint{2.446088in}{1.484734in}}%
\pgfusepath{clip}%
\pgfsetbuttcap%
\pgfsetroundjoin%
\definecolor{currentfill}{rgb}{0.298039,0.447059,0.690196}%
\pgfsetfillcolor{currentfill}%
\pgfsetlinewidth{1.003750pt}%
\definecolor{currentstroke}{rgb}{0.298039,0.447059,0.690196}%
\pgfsetstrokecolor{currentstroke}%
\pgfsetdash{}{0pt}%
\pgfpathmoveto{\pgfqpoint{3.905098in}{1.325421in}}%
\pgfpathcurveto{\pgfqpoint{3.913334in}{1.325421in}}{\pgfqpoint{3.921234in}{1.328693in}}{\pgfqpoint{3.927058in}{1.334517in}}%
\pgfpathcurveto{\pgfqpoint{3.932882in}{1.340341in}}{\pgfqpoint{3.936155in}{1.348241in}}{\pgfqpoint{3.936155in}{1.356477in}}%
\pgfpathcurveto{\pgfqpoint{3.936155in}{1.364713in}}{\pgfqpoint{3.932882in}{1.372613in}}{\pgfqpoint{3.927058in}{1.378437in}}%
\pgfpathcurveto{\pgfqpoint{3.921234in}{1.384261in}}{\pgfqpoint{3.913334in}{1.387534in}}{\pgfqpoint{3.905098in}{1.387534in}}%
\pgfpathcurveto{\pgfqpoint{3.896862in}{1.387534in}}{\pgfqpoint{3.888962in}{1.384261in}}{\pgfqpoint{3.883138in}{1.378437in}}%
\pgfpathcurveto{\pgfqpoint{3.877314in}{1.372613in}}{\pgfqpoint{3.874042in}{1.364713in}}{\pgfqpoint{3.874042in}{1.356477in}}%
\pgfpathcurveto{\pgfqpoint{3.874042in}{1.348241in}}{\pgfqpoint{3.877314in}{1.340341in}}{\pgfqpoint{3.883138in}{1.334517in}}%
\pgfpathcurveto{\pgfqpoint{3.888962in}{1.328693in}}{\pgfqpoint{3.896862in}{1.325421in}}{\pgfqpoint{3.905098in}{1.325421in}}%
\pgfpathclose%
\pgfusepath{stroke,fill}%
\end{pgfscope}%
\begin{pgfscope}%
\pgfpathrectangle{\pgfqpoint{3.793912in}{0.557870in}}{\pgfqpoint{2.446088in}{1.484734in}}%
\pgfusepath{clip}%
\pgfsetbuttcap%
\pgfsetroundjoin%
\definecolor{currentfill}{rgb}{0.298039,0.447059,0.690196}%
\pgfsetfillcolor{currentfill}%
\pgfsetlinewidth{1.003750pt}%
\definecolor{currentstroke}{rgb}{0.298039,0.447059,0.690196}%
\pgfsetstrokecolor{currentstroke}%
\pgfsetdash{}{0pt}%
\pgfpathmoveto{\pgfqpoint{3.905098in}{1.261146in}}%
\pgfpathcurveto{\pgfqpoint{3.913334in}{1.261146in}}{\pgfqpoint{3.921234in}{1.264419in}}{\pgfqpoint{3.927058in}{1.270243in}}%
\pgfpathcurveto{\pgfqpoint{3.932882in}{1.276067in}}{\pgfqpoint{3.936155in}{1.283967in}}{\pgfqpoint{3.936155in}{1.292203in}}%
\pgfpathcurveto{\pgfqpoint{3.936155in}{1.300439in}}{\pgfqpoint{3.932882in}{1.308339in}}{\pgfqpoint{3.927058in}{1.314163in}}%
\pgfpathcurveto{\pgfqpoint{3.921234in}{1.319987in}}{\pgfqpoint{3.913334in}{1.323259in}}{\pgfqpoint{3.905098in}{1.323259in}}%
\pgfpathcurveto{\pgfqpoint{3.896862in}{1.323259in}}{\pgfqpoint{3.888962in}{1.319987in}}{\pgfqpoint{3.883138in}{1.314163in}}%
\pgfpathcurveto{\pgfqpoint{3.877314in}{1.308339in}}{\pgfqpoint{3.874042in}{1.300439in}}{\pgfqpoint{3.874042in}{1.292203in}}%
\pgfpathcurveto{\pgfqpoint{3.874042in}{1.283967in}}{\pgfqpoint{3.877314in}{1.276067in}}{\pgfqpoint{3.883138in}{1.270243in}}%
\pgfpathcurveto{\pgfqpoint{3.888962in}{1.264419in}}{\pgfqpoint{3.896862in}{1.261146in}}{\pgfqpoint{3.905098in}{1.261146in}}%
\pgfpathclose%
\pgfusepath{stroke,fill}%
\end{pgfscope}%
\begin{pgfscope}%
\pgfpathrectangle{\pgfqpoint{3.793912in}{0.557870in}}{\pgfqpoint{2.446088in}{1.484734in}}%
\pgfusepath{clip}%
\pgfsetbuttcap%
\pgfsetroundjoin%
\definecolor{currentfill}{rgb}{0.298039,0.447059,0.690196}%
\pgfsetfillcolor{currentfill}%
\pgfsetlinewidth{1.003750pt}%
\definecolor{currentstroke}{rgb}{0.298039,0.447059,0.690196}%
\pgfsetstrokecolor{currentstroke}%
\pgfsetdash{}{0pt}%
\pgfpathmoveto{\pgfqpoint{3.905098in}{1.341489in}}%
\pgfpathcurveto{\pgfqpoint{3.913334in}{1.341489in}}{\pgfqpoint{3.921234in}{1.344761in}}{\pgfqpoint{3.927058in}{1.350585in}}%
\pgfpathcurveto{\pgfqpoint{3.932882in}{1.356409in}}{\pgfqpoint{3.936155in}{1.364309in}}{\pgfqpoint{3.936155in}{1.372546in}}%
\pgfpathcurveto{\pgfqpoint{3.936155in}{1.380782in}}{\pgfqpoint{3.932882in}{1.388682in}}{\pgfqpoint{3.927058in}{1.394506in}}%
\pgfpathcurveto{\pgfqpoint{3.921234in}{1.400330in}}{\pgfqpoint{3.913334in}{1.403602in}}{\pgfqpoint{3.905098in}{1.403602in}}%
\pgfpathcurveto{\pgfqpoint{3.896862in}{1.403602in}}{\pgfqpoint{3.888962in}{1.400330in}}{\pgfqpoint{3.883138in}{1.394506in}}%
\pgfpathcurveto{\pgfqpoint{3.877314in}{1.388682in}}{\pgfqpoint{3.874042in}{1.380782in}}{\pgfqpoint{3.874042in}{1.372546in}}%
\pgfpathcurveto{\pgfqpoint{3.874042in}{1.364309in}}{\pgfqpoint{3.877314in}{1.356409in}}{\pgfqpoint{3.883138in}{1.350585in}}%
\pgfpathcurveto{\pgfqpoint{3.888962in}{1.344761in}}{\pgfqpoint{3.896862in}{1.341489in}}{\pgfqpoint{3.905098in}{1.341489in}}%
\pgfpathclose%
\pgfusepath{stroke,fill}%
\end{pgfscope}%
\begin{pgfscope}%
\pgfpathrectangle{\pgfqpoint{3.793912in}{0.557870in}}{\pgfqpoint{2.446088in}{1.484734in}}%
\pgfusepath{clip}%
\pgfsetbuttcap%
\pgfsetroundjoin%
\definecolor{currentfill}{rgb}{0.298039,0.447059,0.690196}%
\pgfsetfillcolor{currentfill}%
\pgfsetlinewidth{1.003750pt}%
\definecolor{currentstroke}{rgb}{0.298039,0.447059,0.690196}%
\pgfsetstrokecolor{currentstroke}%
\pgfsetdash{}{0pt}%
\pgfpathmoveto{\pgfqpoint{3.905098in}{0.706782in}}%
\pgfpathcurveto{\pgfqpoint{3.913334in}{0.706782in}}{\pgfqpoint{3.921234in}{0.710054in}}{\pgfqpoint{3.927058in}{0.715878in}}%
\pgfpathcurveto{\pgfqpoint{3.932882in}{0.721702in}}{\pgfqpoint{3.936155in}{0.729602in}}{\pgfqpoint{3.936155in}{0.737838in}}%
\pgfpathcurveto{\pgfqpoint{3.936155in}{0.746074in}}{\pgfqpoint{3.932882in}{0.753974in}}{\pgfqpoint{3.927058in}{0.759798in}}%
\pgfpathcurveto{\pgfqpoint{3.921234in}{0.765622in}}{\pgfqpoint{3.913334in}{0.768895in}}{\pgfqpoint{3.905098in}{0.768895in}}%
\pgfpathcurveto{\pgfqpoint{3.896862in}{0.768895in}}{\pgfqpoint{3.888962in}{0.765622in}}{\pgfqpoint{3.883138in}{0.759798in}}%
\pgfpathcurveto{\pgfqpoint{3.877314in}{0.753974in}}{\pgfqpoint{3.874042in}{0.746074in}}{\pgfqpoint{3.874042in}{0.737838in}}%
\pgfpathcurveto{\pgfqpoint{3.874042in}{0.729602in}}{\pgfqpoint{3.877314in}{0.721702in}}{\pgfqpoint{3.883138in}{0.715878in}}%
\pgfpathcurveto{\pgfqpoint{3.888962in}{0.710054in}}{\pgfqpoint{3.896862in}{0.706782in}}{\pgfqpoint{3.905098in}{0.706782in}}%
\pgfpathclose%
\pgfusepath{stroke,fill}%
\end{pgfscope}%
\begin{pgfscope}%
\pgfpathrectangle{\pgfqpoint{3.793912in}{0.557870in}}{\pgfqpoint{2.446088in}{1.484734in}}%
\pgfusepath{clip}%
\pgfsetbuttcap%
\pgfsetroundjoin%
\definecolor{currentfill}{rgb}{0.298039,0.447059,0.690196}%
\pgfsetfillcolor{currentfill}%
\pgfsetlinewidth{1.003750pt}%
\definecolor{currentstroke}{rgb}{0.298039,0.447059,0.690196}%
\pgfsetstrokecolor{currentstroke}%
\pgfsetdash{}{0pt}%
\pgfpathmoveto{\pgfqpoint{3.905098in}{1.429866in}}%
\pgfpathcurveto{\pgfqpoint{3.913334in}{1.429866in}}{\pgfqpoint{3.921234in}{1.433139in}}{\pgfqpoint{3.927058in}{1.438962in}}%
\pgfpathcurveto{\pgfqpoint{3.932882in}{1.444786in}}{\pgfqpoint{3.936155in}{1.452686in}}{\pgfqpoint{3.936155in}{1.460923in}}%
\pgfpathcurveto{\pgfqpoint{3.936155in}{1.469159in}}{\pgfqpoint{3.932882in}{1.477059in}}{\pgfqpoint{3.927058in}{1.482883in}}%
\pgfpathcurveto{\pgfqpoint{3.921234in}{1.488707in}}{\pgfqpoint{3.913334in}{1.491979in}}{\pgfqpoint{3.905098in}{1.491979in}}%
\pgfpathcurveto{\pgfqpoint{3.896862in}{1.491979in}}{\pgfqpoint{3.888962in}{1.488707in}}{\pgfqpoint{3.883138in}{1.482883in}}%
\pgfpathcurveto{\pgfqpoint{3.877314in}{1.477059in}}{\pgfqpoint{3.874042in}{1.469159in}}{\pgfqpoint{3.874042in}{1.460923in}}%
\pgfpathcurveto{\pgfqpoint{3.874042in}{1.452686in}}{\pgfqpoint{3.877314in}{1.444786in}}{\pgfqpoint{3.883138in}{1.438962in}}%
\pgfpathcurveto{\pgfqpoint{3.888962in}{1.433139in}}{\pgfqpoint{3.896862in}{1.429866in}}{\pgfqpoint{3.905098in}{1.429866in}}%
\pgfpathclose%
\pgfusepath{stroke,fill}%
\end{pgfscope}%
\begin{pgfscope}%
\pgfpathrectangle{\pgfqpoint{3.793912in}{0.557870in}}{\pgfqpoint{2.446088in}{1.484734in}}%
\pgfusepath{clip}%
\pgfsetbuttcap%
\pgfsetroundjoin%
\definecolor{currentfill}{rgb}{0.298039,0.447059,0.690196}%
\pgfsetfillcolor{currentfill}%
\pgfsetlinewidth{1.003750pt}%
\definecolor{currentstroke}{rgb}{0.298039,0.447059,0.690196}%
\pgfsetstrokecolor{currentstroke}%
\pgfsetdash{}{0pt}%
\pgfpathmoveto{\pgfqpoint{3.905098in}{1.044221in}}%
\pgfpathcurveto{\pgfqpoint{3.913334in}{1.044221in}}{\pgfqpoint{3.921234in}{1.047493in}}{\pgfqpoint{3.927058in}{1.053317in}}%
\pgfpathcurveto{\pgfqpoint{3.932882in}{1.059141in}}{\pgfqpoint{3.936155in}{1.067041in}}{\pgfqpoint{3.936155in}{1.075278in}}%
\pgfpathcurveto{\pgfqpoint{3.936155in}{1.083514in}}{\pgfqpoint{3.932882in}{1.091414in}}{\pgfqpoint{3.927058in}{1.097238in}}%
\pgfpathcurveto{\pgfqpoint{3.921234in}{1.103062in}}{\pgfqpoint{3.913334in}{1.106334in}}{\pgfqpoint{3.905098in}{1.106334in}}%
\pgfpathcurveto{\pgfqpoint{3.896862in}{1.106334in}}{\pgfqpoint{3.888962in}{1.103062in}}{\pgfqpoint{3.883138in}{1.097238in}}%
\pgfpathcurveto{\pgfqpoint{3.877314in}{1.091414in}}{\pgfqpoint{3.874042in}{1.083514in}}{\pgfqpoint{3.874042in}{1.075278in}}%
\pgfpathcurveto{\pgfqpoint{3.874042in}{1.067041in}}{\pgfqpoint{3.877314in}{1.059141in}}{\pgfqpoint{3.883138in}{1.053317in}}%
\pgfpathcurveto{\pgfqpoint{3.888962in}{1.047493in}}{\pgfqpoint{3.896862in}{1.044221in}}{\pgfqpoint{3.905098in}{1.044221in}}%
\pgfpathclose%
\pgfusepath{stroke,fill}%
\end{pgfscope}%
\begin{pgfscope}%
\pgfpathrectangle{\pgfqpoint{3.793912in}{0.557870in}}{\pgfqpoint{2.446088in}{1.484734in}}%
\pgfusepath{clip}%
\pgfsetbuttcap%
\pgfsetroundjoin%
\definecolor{currentfill}{rgb}{0.298039,0.447059,0.690196}%
\pgfsetfillcolor{currentfill}%
\pgfsetlinewidth{1.003750pt}%
\definecolor{currentstroke}{rgb}{0.298039,0.447059,0.690196}%
\pgfsetstrokecolor{currentstroke}%
\pgfsetdash{}{0pt}%
\pgfpathmoveto{\pgfqpoint{3.905098in}{1.453969in}}%
\pgfpathcurveto{\pgfqpoint{3.913334in}{1.453969in}}{\pgfqpoint{3.921234in}{1.457241in}}{\pgfqpoint{3.927058in}{1.463065in}}%
\pgfpathcurveto{\pgfqpoint{3.932882in}{1.468889in}}{\pgfqpoint{3.936155in}{1.476789in}}{\pgfqpoint{3.936155in}{1.485026in}}%
\pgfpathcurveto{\pgfqpoint{3.936155in}{1.493262in}}{\pgfqpoint{3.932882in}{1.501162in}}{\pgfqpoint{3.927058in}{1.506986in}}%
\pgfpathcurveto{\pgfqpoint{3.921234in}{1.512810in}}{\pgfqpoint{3.913334in}{1.516082in}}{\pgfqpoint{3.905098in}{1.516082in}}%
\pgfpathcurveto{\pgfqpoint{3.896862in}{1.516082in}}{\pgfqpoint{3.888962in}{1.512810in}}{\pgfqpoint{3.883138in}{1.506986in}}%
\pgfpathcurveto{\pgfqpoint{3.877314in}{1.501162in}}{\pgfqpoint{3.874042in}{1.493262in}}{\pgfqpoint{3.874042in}{1.485026in}}%
\pgfpathcurveto{\pgfqpoint{3.874042in}{1.476789in}}{\pgfqpoint{3.877314in}{1.468889in}}{\pgfqpoint{3.883138in}{1.463065in}}%
\pgfpathcurveto{\pgfqpoint{3.888962in}{1.457241in}}{\pgfqpoint{3.896862in}{1.453969in}}{\pgfqpoint{3.905098in}{1.453969in}}%
\pgfpathclose%
\pgfusepath{stroke,fill}%
\end{pgfscope}%
\begin{pgfscope}%
\pgfpathrectangle{\pgfqpoint{3.793912in}{0.557870in}}{\pgfqpoint{2.446088in}{1.484734in}}%
\pgfusepath{clip}%
\pgfsetbuttcap%
\pgfsetroundjoin%
\definecolor{currentfill}{rgb}{0.298039,0.447059,0.690196}%
\pgfsetfillcolor{currentfill}%
\pgfsetlinewidth{1.003750pt}%
\definecolor{currentstroke}{rgb}{0.298039,0.447059,0.690196}%
\pgfsetstrokecolor{currentstroke}%
\pgfsetdash{}{0pt}%
\pgfpathmoveto{\pgfqpoint{3.905098in}{0.819261in}}%
\pgfpathcurveto{\pgfqpoint{3.913334in}{0.819261in}}{\pgfqpoint{3.921234in}{0.822534in}}{\pgfqpoint{3.927058in}{0.828358in}}%
\pgfpathcurveto{\pgfqpoint{3.932882in}{0.834182in}}{\pgfqpoint{3.936155in}{0.842082in}}{\pgfqpoint{3.936155in}{0.850318in}}%
\pgfpathcurveto{\pgfqpoint{3.936155in}{0.858554in}}{\pgfqpoint{3.932882in}{0.866454in}}{\pgfqpoint{3.927058in}{0.872278in}}%
\pgfpathcurveto{\pgfqpoint{3.921234in}{0.878102in}}{\pgfqpoint{3.913334in}{0.881374in}}{\pgfqpoint{3.905098in}{0.881374in}}%
\pgfpathcurveto{\pgfqpoint{3.896862in}{0.881374in}}{\pgfqpoint{3.888962in}{0.878102in}}{\pgfqpoint{3.883138in}{0.872278in}}%
\pgfpathcurveto{\pgfqpoint{3.877314in}{0.866454in}}{\pgfqpoint{3.874042in}{0.858554in}}{\pgfqpoint{3.874042in}{0.850318in}}%
\pgfpathcurveto{\pgfqpoint{3.874042in}{0.842082in}}{\pgfqpoint{3.877314in}{0.834182in}}{\pgfqpoint{3.883138in}{0.828358in}}%
\pgfpathcurveto{\pgfqpoint{3.888962in}{0.822534in}}{\pgfqpoint{3.896862in}{0.819261in}}{\pgfqpoint{3.905098in}{0.819261in}}%
\pgfpathclose%
\pgfusepath{stroke,fill}%
\end{pgfscope}%
\begin{pgfscope}%
\pgfpathrectangle{\pgfqpoint{3.793912in}{0.557870in}}{\pgfqpoint{2.446088in}{1.484734in}}%
\pgfusepath{clip}%
\pgfsetbuttcap%
\pgfsetroundjoin%
\definecolor{currentfill}{rgb}{0.298039,0.447059,0.690196}%
\pgfsetfillcolor{currentfill}%
\pgfsetlinewidth{1.003750pt}%
\definecolor{currentstroke}{rgb}{0.298039,0.447059,0.690196}%
\pgfsetstrokecolor{currentstroke}%
\pgfsetdash{}{0pt}%
\pgfpathmoveto{\pgfqpoint{3.905098in}{0.819261in}}%
\pgfpathcurveto{\pgfqpoint{3.913334in}{0.819261in}}{\pgfqpoint{3.921234in}{0.822534in}}{\pgfqpoint{3.927058in}{0.828358in}}%
\pgfpathcurveto{\pgfqpoint{3.932882in}{0.834182in}}{\pgfqpoint{3.936155in}{0.842082in}}{\pgfqpoint{3.936155in}{0.850318in}}%
\pgfpathcurveto{\pgfqpoint{3.936155in}{0.858554in}}{\pgfqpoint{3.932882in}{0.866454in}}{\pgfqpoint{3.927058in}{0.872278in}}%
\pgfpathcurveto{\pgfqpoint{3.921234in}{0.878102in}}{\pgfqpoint{3.913334in}{0.881374in}}{\pgfqpoint{3.905098in}{0.881374in}}%
\pgfpathcurveto{\pgfqpoint{3.896862in}{0.881374in}}{\pgfqpoint{3.888962in}{0.878102in}}{\pgfqpoint{3.883138in}{0.872278in}}%
\pgfpathcurveto{\pgfqpoint{3.877314in}{0.866454in}}{\pgfqpoint{3.874042in}{0.858554in}}{\pgfqpoint{3.874042in}{0.850318in}}%
\pgfpathcurveto{\pgfqpoint{3.874042in}{0.842082in}}{\pgfqpoint{3.877314in}{0.834182in}}{\pgfqpoint{3.883138in}{0.828358in}}%
\pgfpathcurveto{\pgfqpoint{3.888962in}{0.822534in}}{\pgfqpoint{3.896862in}{0.819261in}}{\pgfqpoint{3.905098in}{0.819261in}}%
\pgfpathclose%
\pgfusepath{stroke,fill}%
\end{pgfscope}%
\begin{pgfscope}%
\pgfpathrectangle{\pgfqpoint{3.793912in}{0.557870in}}{\pgfqpoint{2.446088in}{1.484734in}}%
\pgfusepath{clip}%
\pgfsetbuttcap%
\pgfsetroundjoin%
\definecolor{currentfill}{rgb}{0.298039,0.447059,0.690196}%
\pgfsetfillcolor{currentfill}%
\pgfsetlinewidth{1.003750pt}%
\definecolor{currentstroke}{rgb}{0.298039,0.447059,0.690196}%
\pgfsetstrokecolor{currentstroke}%
\pgfsetdash{}{0pt}%
\pgfpathmoveto{\pgfqpoint{3.905098in}{1.936025in}}%
\pgfpathcurveto{\pgfqpoint{3.913334in}{1.936025in}}{\pgfqpoint{3.921234in}{1.939298in}}{\pgfqpoint{3.927058in}{1.945122in}}%
\pgfpathcurveto{\pgfqpoint{3.932882in}{1.950946in}}{\pgfqpoint{3.936155in}{1.958846in}}{\pgfqpoint{3.936155in}{1.967082in}}%
\pgfpathcurveto{\pgfqpoint{3.936155in}{1.975318in}}{\pgfqpoint{3.932882in}{1.983218in}}{\pgfqpoint{3.927058in}{1.989042in}}%
\pgfpathcurveto{\pgfqpoint{3.921234in}{1.994866in}}{\pgfqpoint{3.913334in}{1.998138in}}{\pgfqpoint{3.905098in}{1.998138in}}%
\pgfpathcurveto{\pgfqpoint{3.896862in}{1.998138in}}{\pgfqpoint{3.888962in}{1.994866in}}{\pgfqpoint{3.883138in}{1.989042in}}%
\pgfpathcurveto{\pgfqpoint{3.877314in}{1.983218in}}{\pgfqpoint{3.874042in}{1.975318in}}{\pgfqpoint{3.874042in}{1.967082in}}%
\pgfpathcurveto{\pgfqpoint{3.874042in}{1.958846in}}{\pgfqpoint{3.877314in}{1.950946in}}{\pgfqpoint{3.883138in}{1.945122in}}%
\pgfpathcurveto{\pgfqpoint{3.888962in}{1.939298in}}{\pgfqpoint{3.896862in}{1.936025in}}{\pgfqpoint{3.905098in}{1.936025in}}%
\pgfpathclose%
\pgfusepath{stroke,fill}%
\end{pgfscope}%
\begin{pgfscope}%
\pgfpathrectangle{\pgfqpoint{3.793912in}{0.557870in}}{\pgfqpoint{2.446088in}{1.484734in}}%
\pgfusepath{clip}%
\pgfsetbuttcap%
\pgfsetroundjoin%
\definecolor{currentfill}{rgb}{0.298039,0.447059,0.690196}%
\pgfsetfillcolor{currentfill}%
\pgfsetlinewidth{1.003750pt}%
\definecolor{currentstroke}{rgb}{0.298039,0.447059,0.690196}%
\pgfsetstrokecolor{currentstroke}%
\pgfsetdash{}{0pt}%
\pgfpathmoveto{\pgfqpoint{3.905098in}{1.936025in}}%
\pgfpathcurveto{\pgfqpoint{3.913334in}{1.936025in}}{\pgfqpoint{3.921234in}{1.939298in}}{\pgfqpoint{3.927058in}{1.945122in}}%
\pgfpathcurveto{\pgfqpoint{3.932882in}{1.950946in}}{\pgfqpoint{3.936155in}{1.958846in}}{\pgfqpoint{3.936155in}{1.967082in}}%
\pgfpathcurveto{\pgfqpoint{3.936155in}{1.975318in}}{\pgfqpoint{3.932882in}{1.983218in}}{\pgfqpoint{3.927058in}{1.989042in}}%
\pgfpathcurveto{\pgfqpoint{3.921234in}{1.994866in}}{\pgfqpoint{3.913334in}{1.998138in}}{\pgfqpoint{3.905098in}{1.998138in}}%
\pgfpathcurveto{\pgfqpoint{3.896862in}{1.998138in}}{\pgfqpoint{3.888962in}{1.994866in}}{\pgfqpoint{3.883138in}{1.989042in}}%
\pgfpathcurveto{\pgfqpoint{3.877314in}{1.983218in}}{\pgfqpoint{3.874042in}{1.975318in}}{\pgfqpoint{3.874042in}{1.967082in}}%
\pgfpathcurveto{\pgfqpoint{3.874042in}{1.958846in}}{\pgfqpoint{3.877314in}{1.950946in}}{\pgfqpoint{3.883138in}{1.945122in}}%
\pgfpathcurveto{\pgfqpoint{3.888962in}{1.939298in}}{\pgfqpoint{3.896862in}{1.936025in}}{\pgfqpoint{3.905098in}{1.936025in}}%
\pgfpathclose%
\pgfusepath{stroke,fill}%
\end{pgfscope}%
\begin{pgfscope}%
\pgfpathrectangle{\pgfqpoint{3.793912in}{0.557870in}}{\pgfqpoint{2.446088in}{1.484734in}}%
\pgfusepath{clip}%
\pgfsetbuttcap%
\pgfsetroundjoin%
\definecolor{currentfill}{rgb}{0.298039,0.447059,0.690196}%
\pgfsetfillcolor{currentfill}%
\pgfsetlinewidth{1.003750pt}%
\definecolor{currentstroke}{rgb}{0.298039,0.447059,0.690196}%
\pgfsetstrokecolor{currentstroke}%
\pgfsetdash{}{0pt}%
\pgfpathmoveto{\pgfqpoint{3.905098in}{1.936025in}}%
\pgfpathcurveto{\pgfqpoint{3.913334in}{1.936025in}}{\pgfqpoint{3.921234in}{1.939298in}}{\pgfqpoint{3.927058in}{1.945122in}}%
\pgfpathcurveto{\pgfqpoint{3.932882in}{1.950946in}}{\pgfqpoint{3.936155in}{1.958846in}}{\pgfqpoint{3.936155in}{1.967082in}}%
\pgfpathcurveto{\pgfqpoint{3.936155in}{1.975318in}}{\pgfqpoint{3.932882in}{1.983218in}}{\pgfqpoint{3.927058in}{1.989042in}}%
\pgfpathcurveto{\pgfqpoint{3.921234in}{1.994866in}}{\pgfqpoint{3.913334in}{1.998138in}}{\pgfqpoint{3.905098in}{1.998138in}}%
\pgfpathcurveto{\pgfqpoint{3.896862in}{1.998138in}}{\pgfqpoint{3.888962in}{1.994866in}}{\pgfqpoint{3.883138in}{1.989042in}}%
\pgfpathcurveto{\pgfqpoint{3.877314in}{1.983218in}}{\pgfqpoint{3.874042in}{1.975318in}}{\pgfqpoint{3.874042in}{1.967082in}}%
\pgfpathcurveto{\pgfqpoint{3.874042in}{1.958846in}}{\pgfqpoint{3.877314in}{1.950946in}}{\pgfqpoint{3.883138in}{1.945122in}}%
\pgfpathcurveto{\pgfqpoint{3.888962in}{1.939298in}}{\pgfqpoint{3.896862in}{1.936025in}}{\pgfqpoint{3.905098in}{1.936025in}}%
\pgfpathclose%
\pgfusepath{stroke,fill}%
\end{pgfscope}%
\begin{pgfscope}%
\pgfpathrectangle{\pgfqpoint{3.793912in}{0.557870in}}{\pgfqpoint{2.446088in}{1.484734in}}%
\pgfusepath{clip}%
\pgfsetbuttcap%
\pgfsetroundjoin%
\definecolor{currentfill}{rgb}{0.298039,0.447059,0.690196}%
\pgfsetfillcolor{currentfill}%
\pgfsetlinewidth{1.003750pt}%
\definecolor{currentstroke}{rgb}{0.298039,0.447059,0.690196}%
\pgfsetstrokecolor{currentstroke}%
\pgfsetdash{}{0pt}%
\pgfpathmoveto{\pgfqpoint{3.905098in}{1.486106in}}%
\pgfpathcurveto{\pgfqpoint{3.913334in}{1.486106in}}{\pgfqpoint{3.921234in}{1.489378in}}{\pgfqpoint{3.927058in}{1.495202in}}%
\pgfpathcurveto{\pgfqpoint{3.932882in}{1.501026in}}{\pgfqpoint{3.936155in}{1.508926in}}{\pgfqpoint{3.936155in}{1.517163in}}%
\pgfpathcurveto{\pgfqpoint{3.936155in}{1.525399in}}{\pgfqpoint{3.932882in}{1.533299in}}{\pgfqpoint{3.927058in}{1.539123in}}%
\pgfpathcurveto{\pgfqpoint{3.921234in}{1.544947in}}{\pgfqpoint{3.913334in}{1.548219in}}{\pgfqpoint{3.905098in}{1.548219in}}%
\pgfpathcurveto{\pgfqpoint{3.896862in}{1.548219in}}{\pgfqpoint{3.888962in}{1.544947in}}{\pgfqpoint{3.883138in}{1.539123in}}%
\pgfpathcurveto{\pgfqpoint{3.877314in}{1.533299in}}{\pgfqpoint{3.874042in}{1.525399in}}{\pgfqpoint{3.874042in}{1.517163in}}%
\pgfpathcurveto{\pgfqpoint{3.874042in}{1.508926in}}{\pgfqpoint{3.877314in}{1.501026in}}{\pgfqpoint{3.883138in}{1.495202in}}%
\pgfpathcurveto{\pgfqpoint{3.888962in}{1.489378in}}{\pgfqpoint{3.896862in}{1.486106in}}{\pgfqpoint{3.905098in}{1.486106in}}%
\pgfpathclose%
\pgfusepath{stroke,fill}%
\end{pgfscope}%
\begin{pgfscope}%
\pgfpathrectangle{\pgfqpoint{3.793912in}{0.557870in}}{\pgfqpoint{2.446088in}{1.484734in}}%
\pgfusepath{clip}%
\pgfsetbuttcap%
\pgfsetroundjoin%
\definecolor{currentfill}{rgb}{0.298039,0.447059,0.690196}%
\pgfsetfillcolor{currentfill}%
\pgfsetlinewidth{1.003750pt}%
\definecolor{currentstroke}{rgb}{0.298039,0.447059,0.690196}%
\pgfsetstrokecolor{currentstroke}%
\pgfsetdash{}{0pt}%
\pgfpathmoveto{\pgfqpoint{3.905098in}{1.936025in}}%
\pgfpathcurveto{\pgfqpoint{3.913334in}{1.936025in}}{\pgfqpoint{3.921234in}{1.939298in}}{\pgfqpoint{3.927058in}{1.945122in}}%
\pgfpathcurveto{\pgfqpoint{3.932882in}{1.950946in}}{\pgfqpoint{3.936155in}{1.958846in}}{\pgfqpoint{3.936155in}{1.967082in}}%
\pgfpathcurveto{\pgfqpoint{3.936155in}{1.975318in}}{\pgfqpoint{3.932882in}{1.983218in}}{\pgfqpoint{3.927058in}{1.989042in}}%
\pgfpathcurveto{\pgfqpoint{3.921234in}{1.994866in}}{\pgfqpoint{3.913334in}{1.998138in}}{\pgfqpoint{3.905098in}{1.998138in}}%
\pgfpathcurveto{\pgfqpoint{3.896862in}{1.998138in}}{\pgfqpoint{3.888962in}{1.994866in}}{\pgfqpoint{3.883138in}{1.989042in}}%
\pgfpathcurveto{\pgfqpoint{3.877314in}{1.983218in}}{\pgfqpoint{3.874042in}{1.975318in}}{\pgfqpoint{3.874042in}{1.967082in}}%
\pgfpathcurveto{\pgfqpoint{3.874042in}{1.958846in}}{\pgfqpoint{3.877314in}{1.950946in}}{\pgfqpoint{3.883138in}{1.945122in}}%
\pgfpathcurveto{\pgfqpoint{3.888962in}{1.939298in}}{\pgfqpoint{3.896862in}{1.936025in}}{\pgfqpoint{3.905098in}{1.936025in}}%
\pgfpathclose%
\pgfusepath{stroke,fill}%
\end{pgfscope}%
\begin{pgfscope}%
\pgfpathrectangle{\pgfqpoint{3.793912in}{0.557870in}}{\pgfqpoint{2.446088in}{1.484734in}}%
\pgfusepath{clip}%
\pgfsetbuttcap%
\pgfsetroundjoin%
\definecolor{currentfill}{rgb}{0.298039,0.447059,0.690196}%
\pgfsetfillcolor{currentfill}%
\pgfsetlinewidth{1.003750pt}%
\definecolor{currentstroke}{rgb}{0.298039,0.447059,0.690196}%
\pgfsetstrokecolor{currentstroke}%
\pgfsetdash{}{0pt}%
\pgfpathmoveto{\pgfqpoint{3.905098in}{1.936025in}}%
\pgfpathcurveto{\pgfqpoint{3.913334in}{1.936025in}}{\pgfqpoint{3.921234in}{1.939298in}}{\pgfqpoint{3.927058in}{1.945122in}}%
\pgfpathcurveto{\pgfqpoint{3.932882in}{1.950946in}}{\pgfqpoint{3.936155in}{1.958846in}}{\pgfqpoint{3.936155in}{1.967082in}}%
\pgfpathcurveto{\pgfqpoint{3.936155in}{1.975318in}}{\pgfqpoint{3.932882in}{1.983218in}}{\pgfqpoint{3.927058in}{1.989042in}}%
\pgfpathcurveto{\pgfqpoint{3.921234in}{1.994866in}}{\pgfqpoint{3.913334in}{1.998138in}}{\pgfqpoint{3.905098in}{1.998138in}}%
\pgfpathcurveto{\pgfqpoint{3.896862in}{1.998138in}}{\pgfqpoint{3.888962in}{1.994866in}}{\pgfqpoint{3.883138in}{1.989042in}}%
\pgfpathcurveto{\pgfqpoint{3.877314in}{1.983218in}}{\pgfqpoint{3.874042in}{1.975318in}}{\pgfqpoint{3.874042in}{1.967082in}}%
\pgfpathcurveto{\pgfqpoint{3.874042in}{1.958846in}}{\pgfqpoint{3.877314in}{1.950946in}}{\pgfqpoint{3.883138in}{1.945122in}}%
\pgfpathcurveto{\pgfqpoint{3.888962in}{1.939298in}}{\pgfqpoint{3.896862in}{1.936025in}}{\pgfqpoint{3.905098in}{1.936025in}}%
\pgfpathclose%
\pgfusepath{stroke,fill}%
\end{pgfscope}%
\begin{pgfscope}%
\pgfpathrectangle{\pgfqpoint{3.793912in}{0.557870in}}{\pgfqpoint{2.446088in}{1.484734in}}%
\pgfusepath{clip}%
\pgfsetbuttcap%
\pgfsetroundjoin%
\definecolor{currentfill}{rgb}{0.298039,0.447059,0.690196}%
\pgfsetfillcolor{currentfill}%
\pgfsetlinewidth{1.003750pt}%
\definecolor{currentstroke}{rgb}{0.298039,0.447059,0.690196}%
\pgfsetstrokecolor{currentstroke}%
\pgfsetdash{}{0pt}%
\pgfpathmoveto{\pgfqpoint{3.905098in}{1.936025in}}%
\pgfpathcurveto{\pgfqpoint{3.913334in}{1.936025in}}{\pgfqpoint{3.921234in}{1.939298in}}{\pgfqpoint{3.927058in}{1.945122in}}%
\pgfpathcurveto{\pgfqpoint{3.932882in}{1.950946in}}{\pgfqpoint{3.936155in}{1.958846in}}{\pgfqpoint{3.936155in}{1.967082in}}%
\pgfpathcurveto{\pgfqpoint{3.936155in}{1.975318in}}{\pgfqpoint{3.932882in}{1.983218in}}{\pgfqpoint{3.927058in}{1.989042in}}%
\pgfpathcurveto{\pgfqpoint{3.921234in}{1.994866in}}{\pgfqpoint{3.913334in}{1.998138in}}{\pgfqpoint{3.905098in}{1.998138in}}%
\pgfpathcurveto{\pgfqpoint{3.896862in}{1.998138in}}{\pgfqpoint{3.888962in}{1.994866in}}{\pgfqpoint{3.883138in}{1.989042in}}%
\pgfpathcurveto{\pgfqpoint{3.877314in}{1.983218in}}{\pgfqpoint{3.874042in}{1.975318in}}{\pgfqpoint{3.874042in}{1.967082in}}%
\pgfpathcurveto{\pgfqpoint{3.874042in}{1.958846in}}{\pgfqpoint{3.877314in}{1.950946in}}{\pgfqpoint{3.883138in}{1.945122in}}%
\pgfpathcurveto{\pgfqpoint{3.888962in}{1.939298in}}{\pgfqpoint{3.896862in}{1.936025in}}{\pgfqpoint{3.905098in}{1.936025in}}%
\pgfpathclose%
\pgfusepath{stroke,fill}%
\end{pgfscope}%
\begin{pgfscope}%
\pgfpathrectangle{\pgfqpoint{3.793912in}{0.557870in}}{\pgfqpoint{2.446088in}{1.484734in}}%
\pgfusepath{clip}%
\pgfsetbuttcap%
\pgfsetroundjoin%
\definecolor{currentfill}{rgb}{0.298039,0.447059,0.690196}%
\pgfsetfillcolor{currentfill}%
\pgfsetlinewidth{1.003750pt}%
\definecolor{currentstroke}{rgb}{0.298039,0.447059,0.690196}%
\pgfsetstrokecolor{currentstroke}%
\pgfsetdash{}{0pt}%
\pgfpathmoveto{\pgfqpoint{3.905098in}{1.936025in}}%
\pgfpathcurveto{\pgfqpoint{3.913334in}{1.936025in}}{\pgfqpoint{3.921234in}{1.939298in}}{\pgfqpoint{3.927058in}{1.945122in}}%
\pgfpathcurveto{\pgfqpoint{3.932882in}{1.950946in}}{\pgfqpoint{3.936155in}{1.958846in}}{\pgfqpoint{3.936155in}{1.967082in}}%
\pgfpathcurveto{\pgfqpoint{3.936155in}{1.975318in}}{\pgfqpoint{3.932882in}{1.983218in}}{\pgfqpoint{3.927058in}{1.989042in}}%
\pgfpathcurveto{\pgfqpoint{3.921234in}{1.994866in}}{\pgfqpoint{3.913334in}{1.998138in}}{\pgfqpoint{3.905098in}{1.998138in}}%
\pgfpathcurveto{\pgfqpoint{3.896862in}{1.998138in}}{\pgfqpoint{3.888962in}{1.994866in}}{\pgfqpoint{3.883138in}{1.989042in}}%
\pgfpathcurveto{\pgfqpoint{3.877314in}{1.983218in}}{\pgfqpoint{3.874042in}{1.975318in}}{\pgfqpoint{3.874042in}{1.967082in}}%
\pgfpathcurveto{\pgfqpoint{3.874042in}{1.958846in}}{\pgfqpoint{3.877314in}{1.950946in}}{\pgfqpoint{3.883138in}{1.945122in}}%
\pgfpathcurveto{\pgfqpoint{3.888962in}{1.939298in}}{\pgfqpoint{3.896862in}{1.936025in}}{\pgfqpoint{3.905098in}{1.936025in}}%
\pgfpathclose%
\pgfusepath{stroke,fill}%
\end{pgfscope}%
\begin{pgfscope}%
\pgfpathrectangle{\pgfqpoint{3.793912in}{0.557870in}}{\pgfqpoint{2.446088in}{1.484734in}}%
\pgfusepath{clip}%
\pgfsetbuttcap%
\pgfsetroundjoin%
\definecolor{currentfill}{rgb}{0.298039,0.447059,0.690196}%
\pgfsetfillcolor{currentfill}%
\pgfsetlinewidth{1.003750pt}%
\definecolor{currentstroke}{rgb}{0.298039,0.447059,0.690196}%
\pgfsetstrokecolor{currentstroke}%
\pgfsetdash{}{0pt}%
\pgfpathmoveto{\pgfqpoint{3.905098in}{1.156701in}}%
\pgfpathcurveto{\pgfqpoint{3.913334in}{1.156701in}}{\pgfqpoint{3.921234in}{1.159973in}}{\pgfqpoint{3.927058in}{1.165797in}}%
\pgfpathcurveto{\pgfqpoint{3.932882in}{1.171621in}}{\pgfqpoint{3.936155in}{1.179521in}}{\pgfqpoint{3.936155in}{1.187757in}}%
\pgfpathcurveto{\pgfqpoint{3.936155in}{1.195994in}}{\pgfqpoint{3.932882in}{1.203894in}}{\pgfqpoint{3.927058in}{1.209718in}}%
\pgfpathcurveto{\pgfqpoint{3.921234in}{1.215542in}}{\pgfqpoint{3.913334in}{1.218814in}}{\pgfqpoint{3.905098in}{1.218814in}}%
\pgfpathcurveto{\pgfqpoint{3.896862in}{1.218814in}}{\pgfqpoint{3.888962in}{1.215542in}}{\pgfqpoint{3.883138in}{1.209718in}}%
\pgfpathcurveto{\pgfqpoint{3.877314in}{1.203894in}}{\pgfqpoint{3.874042in}{1.195994in}}{\pgfqpoint{3.874042in}{1.187757in}}%
\pgfpathcurveto{\pgfqpoint{3.874042in}{1.179521in}}{\pgfqpoint{3.877314in}{1.171621in}}{\pgfqpoint{3.883138in}{1.165797in}}%
\pgfpathcurveto{\pgfqpoint{3.888962in}{1.159973in}}{\pgfqpoint{3.896862in}{1.156701in}}{\pgfqpoint{3.905098in}{1.156701in}}%
\pgfpathclose%
\pgfusepath{stroke,fill}%
\end{pgfscope}%
\begin{pgfscope}%
\pgfpathrectangle{\pgfqpoint{3.793912in}{0.557870in}}{\pgfqpoint{2.446088in}{1.484734in}}%
\pgfusepath{clip}%
\pgfsetbuttcap%
\pgfsetroundjoin%
\definecolor{currentfill}{rgb}{0.298039,0.447059,0.690196}%
\pgfsetfillcolor{currentfill}%
\pgfsetlinewidth{1.003750pt}%
\definecolor{currentstroke}{rgb}{0.298039,0.447059,0.690196}%
\pgfsetstrokecolor{currentstroke}%
\pgfsetdash{}{0pt}%
\pgfpathmoveto{\pgfqpoint{3.905098in}{1.936025in}}%
\pgfpathcurveto{\pgfqpoint{3.913334in}{1.936025in}}{\pgfqpoint{3.921234in}{1.939298in}}{\pgfqpoint{3.927058in}{1.945122in}}%
\pgfpathcurveto{\pgfqpoint{3.932882in}{1.950946in}}{\pgfqpoint{3.936155in}{1.958846in}}{\pgfqpoint{3.936155in}{1.967082in}}%
\pgfpathcurveto{\pgfqpoint{3.936155in}{1.975318in}}{\pgfqpoint{3.932882in}{1.983218in}}{\pgfqpoint{3.927058in}{1.989042in}}%
\pgfpathcurveto{\pgfqpoint{3.921234in}{1.994866in}}{\pgfqpoint{3.913334in}{1.998138in}}{\pgfqpoint{3.905098in}{1.998138in}}%
\pgfpathcurveto{\pgfqpoint{3.896862in}{1.998138in}}{\pgfqpoint{3.888962in}{1.994866in}}{\pgfqpoint{3.883138in}{1.989042in}}%
\pgfpathcurveto{\pgfqpoint{3.877314in}{1.983218in}}{\pgfqpoint{3.874042in}{1.975318in}}{\pgfqpoint{3.874042in}{1.967082in}}%
\pgfpathcurveto{\pgfqpoint{3.874042in}{1.958846in}}{\pgfqpoint{3.877314in}{1.950946in}}{\pgfqpoint{3.883138in}{1.945122in}}%
\pgfpathcurveto{\pgfqpoint{3.888962in}{1.939298in}}{\pgfqpoint{3.896862in}{1.936025in}}{\pgfqpoint{3.905098in}{1.936025in}}%
\pgfpathclose%
\pgfusepath{stroke,fill}%
\end{pgfscope}%
\begin{pgfscope}%
\pgfpathrectangle{\pgfqpoint{3.793912in}{0.557870in}}{\pgfqpoint{2.446088in}{1.484734in}}%
\pgfusepath{clip}%
\pgfsetbuttcap%
\pgfsetroundjoin%
\definecolor{currentfill}{rgb}{0.298039,0.447059,0.690196}%
\pgfsetfillcolor{currentfill}%
\pgfsetlinewidth{1.003750pt}%
\definecolor{currentstroke}{rgb}{0.298039,0.447059,0.690196}%
\pgfsetstrokecolor{currentstroke}%
\pgfsetdash{}{0pt}%
\pgfpathmoveto{\pgfqpoint{3.905098in}{1.004050in}}%
\pgfpathcurveto{\pgfqpoint{3.913334in}{1.004050in}}{\pgfqpoint{3.921234in}{1.007322in}}{\pgfqpoint{3.927058in}{1.013146in}}%
\pgfpathcurveto{\pgfqpoint{3.932882in}{1.018970in}}{\pgfqpoint{3.936155in}{1.026870in}}{\pgfqpoint{3.936155in}{1.035106in}}%
\pgfpathcurveto{\pgfqpoint{3.936155in}{1.043342in}}{\pgfqpoint{3.932882in}{1.051243in}}{\pgfqpoint{3.927058in}{1.057066in}}%
\pgfpathcurveto{\pgfqpoint{3.921234in}{1.062890in}}{\pgfqpoint{3.913334in}{1.066163in}}{\pgfqpoint{3.905098in}{1.066163in}}%
\pgfpathcurveto{\pgfqpoint{3.896862in}{1.066163in}}{\pgfqpoint{3.888962in}{1.062890in}}{\pgfqpoint{3.883138in}{1.057066in}}%
\pgfpathcurveto{\pgfqpoint{3.877314in}{1.051243in}}{\pgfqpoint{3.874042in}{1.043342in}}{\pgfqpoint{3.874042in}{1.035106in}}%
\pgfpathcurveto{\pgfqpoint{3.874042in}{1.026870in}}{\pgfqpoint{3.877314in}{1.018970in}}{\pgfqpoint{3.883138in}{1.013146in}}%
\pgfpathcurveto{\pgfqpoint{3.888962in}{1.007322in}}{\pgfqpoint{3.896862in}{1.004050in}}{\pgfqpoint{3.905098in}{1.004050in}}%
\pgfpathclose%
\pgfusepath{stroke,fill}%
\end{pgfscope}%
\begin{pgfscope}%
\pgfpathrectangle{\pgfqpoint{3.793912in}{0.557870in}}{\pgfqpoint{2.446088in}{1.484734in}}%
\pgfusepath{clip}%
\pgfsetbuttcap%
\pgfsetroundjoin%
\definecolor{currentfill}{rgb}{0.298039,0.447059,0.690196}%
\pgfsetfillcolor{currentfill}%
\pgfsetlinewidth{1.003750pt}%
\definecolor{currentstroke}{rgb}{0.298039,0.447059,0.690196}%
\pgfsetstrokecolor{currentstroke}%
\pgfsetdash{}{0pt}%
\pgfpathmoveto{\pgfqpoint{3.905098in}{1.936025in}}%
\pgfpathcurveto{\pgfqpoint{3.913334in}{1.936025in}}{\pgfqpoint{3.921234in}{1.939298in}}{\pgfqpoint{3.927058in}{1.945122in}}%
\pgfpathcurveto{\pgfqpoint{3.932882in}{1.950946in}}{\pgfqpoint{3.936155in}{1.958846in}}{\pgfqpoint{3.936155in}{1.967082in}}%
\pgfpathcurveto{\pgfqpoint{3.936155in}{1.975318in}}{\pgfqpoint{3.932882in}{1.983218in}}{\pgfqpoint{3.927058in}{1.989042in}}%
\pgfpathcurveto{\pgfqpoint{3.921234in}{1.994866in}}{\pgfqpoint{3.913334in}{1.998138in}}{\pgfqpoint{3.905098in}{1.998138in}}%
\pgfpathcurveto{\pgfqpoint{3.896862in}{1.998138in}}{\pgfqpoint{3.888962in}{1.994866in}}{\pgfqpoint{3.883138in}{1.989042in}}%
\pgfpathcurveto{\pgfqpoint{3.877314in}{1.983218in}}{\pgfqpoint{3.874042in}{1.975318in}}{\pgfqpoint{3.874042in}{1.967082in}}%
\pgfpathcurveto{\pgfqpoint{3.874042in}{1.958846in}}{\pgfqpoint{3.877314in}{1.950946in}}{\pgfqpoint{3.883138in}{1.945122in}}%
\pgfpathcurveto{\pgfqpoint{3.888962in}{1.939298in}}{\pgfqpoint{3.896862in}{1.936025in}}{\pgfqpoint{3.905098in}{1.936025in}}%
\pgfpathclose%
\pgfusepath{stroke,fill}%
\end{pgfscope}%
\begin{pgfscope}%
\pgfpathrectangle{\pgfqpoint{3.793912in}{0.557870in}}{\pgfqpoint{2.446088in}{1.484734in}}%
\pgfusepath{clip}%
\pgfsetbuttcap%
\pgfsetroundjoin%
\definecolor{currentfill}{rgb}{0.298039,0.447059,0.690196}%
\pgfsetfillcolor{currentfill}%
\pgfsetlinewidth{1.003750pt}%
\definecolor{currentstroke}{rgb}{0.298039,0.447059,0.690196}%
\pgfsetstrokecolor{currentstroke}%
\pgfsetdash{}{0pt}%
\pgfpathmoveto{\pgfqpoint{3.905098in}{1.936025in}}%
\pgfpathcurveto{\pgfqpoint{3.913334in}{1.936025in}}{\pgfqpoint{3.921234in}{1.939298in}}{\pgfqpoint{3.927058in}{1.945122in}}%
\pgfpathcurveto{\pgfqpoint{3.932882in}{1.950946in}}{\pgfqpoint{3.936155in}{1.958846in}}{\pgfqpoint{3.936155in}{1.967082in}}%
\pgfpathcurveto{\pgfqpoint{3.936155in}{1.975318in}}{\pgfqpoint{3.932882in}{1.983218in}}{\pgfqpoint{3.927058in}{1.989042in}}%
\pgfpathcurveto{\pgfqpoint{3.921234in}{1.994866in}}{\pgfqpoint{3.913334in}{1.998138in}}{\pgfqpoint{3.905098in}{1.998138in}}%
\pgfpathcurveto{\pgfqpoint{3.896862in}{1.998138in}}{\pgfqpoint{3.888962in}{1.994866in}}{\pgfqpoint{3.883138in}{1.989042in}}%
\pgfpathcurveto{\pgfqpoint{3.877314in}{1.983218in}}{\pgfqpoint{3.874042in}{1.975318in}}{\pgfqpoint{3.874042in}{1.967082in}}%
\pgfpathcurveto{\pgfqpoint{3.874042in}{1.958846in}}{\pgfqpoint{3.877314in}{1.950946in}}{\pgfqpoint{3.883138in}{1.945122in}}%
\pgfpathcurveto{\pgfqpoint{3.888962in}{1.939298in}}{\pgfqpoint{3.896862in}{1.936025in}}{\pgfqpoint{3.905098in}{1.936025in}}%
\pgfpathclose%
\pgfusepath{stroke,fill}%
\end{pgfscope}%
\begin{pgfscope}%
\pgfpathrectangle{\pgfqpoint{3.793912in}{0.557870in}}{\pgfqpoint{2.446088in}{1.484734in}}%
\pgfusepath{clip}%
\pgfsetbuttcap%
\pgfsetroundjoin%
\definecolor{currentfill}{rgb}{0.298039,0.447059,0.690196}%
\pgfsetfillcolor{currentfill}%
\pgfsetlinewidth{1.003750pt}%
\definecolor{currentstroke}{rgb}{0.298039,0.447059,0.690196}%
\pgfsetstrokecolor{currentstroke}%
\pgfsetdash{}{0pt}%
\pgfpathmoveto{\pgfqpoint{3.905098in}{1.936025in}}%
\pgfpathcurveto{\pgfqpoint{3.913334in}{1.936025in}}{\pgfqpoint{3.921234in}{1.939298in}}{\pgfqpoint{3.927058in}{1.945122in}}%
\pgfpathcurveto{\pgfqpoint{3.932882in}{1.950946in}}{\pgfqpoint{3.936155in}{1.958846in}}{\pgfqpoint{3.936155in}{1.967082in}}%
\pgfpathcurveto{\pgfqpoint{3.936155in}{1.975318in}}{\pgfqpoint{3.932882in}{1.983218in}}{\pgfqpoint{3.927058in}{1.989042in}}%
\pgfpathcurveto{\pgfqpoint{3.921234in}{1.994866in}}{\pgfqpoint{3.913334in}{1.998138in}}{\pgfqpoint{3.905098in}{1.998138in}}%
\pgfpathcurveto{\pgfqpoint{3.896862in}{1.998138in}}{\pgfqpoint{3.888962in}{1.994866in}}{\pgfqpoint{3.883138in}{1.989042in}}%
\pgfpathcurveto{\pgfqpoint{3.877314in}{1.983218in}}{\pgfqpoint{3.874042in}{1.975318in}}{\pgfqpoint{3.874042in}{1.967082in}}%
\pgfpathcurveto{\pgfqpoint{3.874042in}{1.958846in}}{\pgfqpoint{3.877314in}{1.950946in}}{\pgfqpoint{3.883138in}{1.945122in}}%
\pgfpathcurveto{\pgfqpoint{3.888962in}{1.939298in}}{\pgfqpoint{3.896862in}{1.936025in}}{\pgfqpoint{3.905098in}{1.936025in}}%
\pgfpathclose%
\pgfusepath{stroke,fill}%
\end{pgfscope}%
\begin{pgfscope}%
\pgfpathrectangle{\pgfqpoint{3.793912in}{0.557870in}}{\pgfqpoint{2.446088in}{1.484734in}}%
\pgfusepath{clip}%
\pgfsetbuttcap%
\pgfsetroundjoin%
\definecolor{currentfill}{rgb}{0.298039,0.447059,0.690196}%
\pgfsetfillcolor{currentfill}%
\pgfsetlinewidth{1.003750pt}%
\definecolor{currentstroke}{rgb}{0.298039,0.447059,0.690196}%
\pgfsetstrokecolor{currentstroke}%
\pgfsetdash{}{0pt}%
\pgfpathmoveto{\pgfqpoint{3.905098in}{1.429866in}}%
\pgfpathcurveto{\pgfqpoint{3.913334in}{1.429866in}}{\pgfqpoint{3.921234in}{1.433139in}}{\pgfqpoint{3.927058in}{1.438962in}}%
\pgfpathcurveto{\pgfqpoint{3.932882in}{1.444786in}}{\pgfqpoint{3.936155in}{1.452686in}}{\pgfqpoint{3.936155in}{1.460923in}}%
\pgfpathcurveto{\pgfqpoint{3.936155in}{1.469159in}}{\pgfqpoint{3.932882in}{1.477059in}}{\pgfqpoint{3.927058in}{1.482883in}}%
\pgfpathcurveto{\pgfqpoint{3.921234in}{1.488707in}}{\pgfqpoint{3.913334in}{1.491979in}}{\pgfqpoint{3.905098in}{1.491979in}}%
\pgfpathcurveto{\pgfqpoint{3.896862in}{1.491979in}}{\pgfqpoint{3.888962in}{1.488707in}}{\pgfqpoint{3.883138in}{1.482883in}}%
\pgfpathcurveto{\pgfqpoint{3.877314in}{1.477059in}}{\pgfqpoint{3.874042in}{1.469159in}}{\pgfqpoint{3.874042in}{1.460923in}}%
\pgfpathcurveto{\pgfqpoint{3.874042in}{1.452686in}}{\pgfqpoint{3.877314in}{1.444786in}}{\pgfqpoint{3.883138in}{1.438962in}}%
\pgfpathcurveto{\pgfqpoint{3.888962in}{1.433139in}}{\pgfqpoint{3.896862in}{1.429866in}}{\pgfqpoint{3.905098in}{1.429866in}}%
\pgfpathclose%
\pgfusepath{stroke,fill}%
\end{pgfscope}%
\begin{pgfscope}%
\pgfpathrectangle{\pgfqpoint{3.793912in}{0.557870in}}{\pgfqpoint{2.446088in}{1.484734in}}%
\pgfusepath{clip}%
\pgfsetbuttcap%
\pgfsetroundjoin%
\definecolor{currentfill}{rgb}{0.298039,0.447059,0.690196}%
\pgfsetfillcolor{currentfill}%
\pgfsetlinewidth{1.003750pt}%
\definecolor{currentstroke}{rgb}{0.298039,0.447059,0.690196}%
\pgfsetstrokecolor{currentstroke}%
\pgfsetdash{}{0pt}%
\pgfpathmoveto{\pgfqpoint{3.905098in}{1.044221in}}%
\pgfpathcurveto{\pgfqpoint{3.913334in}{1.044221in}}{\pgfqpoint{3.921234in}{1.047493in}}{\pgfqpoint{3.927058in}{1.053317in}}%
\pgfpathcurveto{\pgfqpoint{3.932882in}{1.059141in}}{\pgfqpoint{3.936155in}{1.067041in}}{\pgfqpoint{3.936155in}{1.075278in}}%
\pgfpathcurveto{\pgfqpoint{3.936155in}{1.083514in}}{\pgfqpoint{3.932882in}{1.091414in}}{\pgfqpoint{3.927058in}{1.097238in}}%
\pgfpathcurveto{\pgfqpoint{3.921234in}{1.103062in}}{\pgfqpoint{3.913334in}{1.106334in}}{\pgfqpoint{3.905098in}{1.106334in}}%
\pgfpathcurveto{\pgfqpoint{3.896862in}{1.106334in}}{\pgfqpoint{3.888962in}{1.103062in}}{\pgfqpoint{3.883138in}{1.097238in}}%
\pgfpathcurveto{\pgfqpoint{3.877314in}{1.091414in}}{\pgfqpoint{3.874042in}{1.083514in}}{\pgfqpoint{3.874042in}{1.075278in}}%
\pgfpathcurveto{\pgfqpoint{3.874042in}{1.067041in}}{\pgfqpoint{3.877314in}{1.059141in}}{\pgfqpoint{3.883138in}{1.053317in}}%
\pgfpathcurveto{\pgfqpoint{3.888962in}{1.047493in}}{\pgfqpoint{3.896862in}{1.044221in}}{\pgfqpoint{3.905098in}{1.044221in}}%
\pgfpathclose%
\pgfusepath{stroke,fill}%
\end{pgfscope}%
\begin{pgfscope}%
\pgfpathrectangle{\pgfqpoint{3.793912in}{0.557870in}}{\pgfqpoint{2.446088in}{1.484734in}}%
\pgfusepath{clip}%
\pgfsetbuttcap%
\pgfsetroundjoin%
\definecolor{currentfill}{rgb}{0.298039,0.447059,0.690196}%
\pgfsetfillcolor{currentfill}%
\pgfsetlinewidth{1.003750pt}%
\definecolor{currentstroke}{rgb}{0.298039,0.447059,0.690196}%
\pgfsetstrokecolor{currentstroke}%
\pgfsetdash{}{0pt}%
\pgfpathmoveto{\pgfqpoint{3.905098in}{1.453969in}}%
\pgfpathcurveto{\pgfqpoint{3.913334in}{1.453969in}}{\pgfqpoint{3.921234in}{1.457241in}}{\pgfqpoint{3.927058in}{1.463065in}}%
\pgfpathcurveto{\pgfqpoint{3.932882in}{1.468889in}}{\pgfqpoint{3.936155in}{1.476789in}}{\pgfqpoint{3.936155in}{1.485026in}}%
\pgfpathcurveto{\pgfqpoint{3.936155in}{1.493262in}}{\pgfqpoint{3.932882in}{1.501162in}}{\pgfqpoint{3.927058in}{1.506986in}}%
\pgfpathcurveto{\pgfqpoint{3.921234in}{1.512810in}}{\pgfqpoint{3.913334in}{1.516082in}}{\pgfqpoint{3.905098in}{1.516082in}}%
\pgfpathcurveto{\pgfqpoint{3.896862in}{1.516082in}}{\pgfqpoint{3.888962in}{1.512810in}}{\pgfqpoint{3.883138in}{1.506986in}}%
\pgfpathcurveto{\pgfqpoint{3.877314in}{1.501162in}}{\pgfqpoint{3.874042in}{1.493262in}}{\pgfqpoint{3.874042in}{1.485026in}}%
\pgfpathcurveto{\pgfqpoint{3.874042in}{1.476789in}}{\pgfqpoint{3.877314in}{1.468889in}}{\pgfqpoint{3.883138in}{1.463065in}}%
\pgfpathcurveto{\pgfqpoint{3.888962in}{1.457241in}}{\pgfqpoint{3.896862in}{1.453969in}}{\pgfqpoint{3.905098in}{1.453969in}}%
\pgfpathclose%
\pgfusepath{stroke,fill}%
\end{pgfscope}%
\begin{pgfscope}%
\pgfpathrectangle{\pgfqpoint{3.793912in}{0.557870in}}{\pgfqpoint{2.446088in}{1.484734in}}%
\pgfusepath{clip}%
\pgfsetbuttcap%
\pgfsetroundjoin%
\definecolor{currentfill}{rgb}{0.298039,0.447059,0.690196}%
\pgfsetfillcolor{currentfill}%
\pgfsetlinewidth{1.003750pt}%
\definecolor{currentstroke}{rgb}{0.298039,0.447059,0.690196}%
\pgfsetstrokecolor{currentstroke}%
\pgfsetdash{}{0pt}%
\pgfpathmoveto{\pgfqpoint{3.905098in}{0.819261in}}%
\pgfpathcurveto{\pgfqpoint{3.913334in}{0.819261in}}{\pgfqpoint{3.921234in}{0.822534in}}{\pgfqpoint{3.927058in}{0.828358in}}%
\pgfpathcurveto{\pgfqpoint{3.932882in}{0.834182in}}{\pgfqpoint{3.936155in}{0.842082in}}{\pgfqpoint{3.936155in}{0.850318in}}%
\pgfpathcurveto{\pgfqpoint{3.936155in}{0.858554in}}{\pgfqpoint{3.932882in}{0.866454in}}{\pgfqpoint{3.927058in}{0.872278in}}%
\pgfpathcurveto{\pgfqpoint{3.921234in}{0.878102in}}{\pgfqpoint{3.913334in}{0.881374in}}{\pgfqpoint{3.905098in}{0.881374in}}%
\pgfpathcurveto{\pgfqpoint{3.896862in}{0.881374in}}{\pgfqpoint{3.888962in}{0.878102in}}{\pgfqpoint{3.883138in}{0.872278in}}%
\pgfpathcurveto{\pgfqpoint{3.877314in}{0.866454in}}{\pgfqpoint{3.874042in}{0.858554in}}{\pgfqpoint{3.874042in}{0.850318in}}%
\pgfpathcurveto{\pgfqpoint{3.874042in}{0.842082in}}{\pgfqpoint{3.877314in}{0.834182in}}{\pgfqpoint{3.883138in}{0.828358in}}%
\pgfpathcurveto{\pgfqpoint{3.888962in}{0.822534in}}{\pgfqpoint{3.896862in}{0.819261in}}{\pgfqpoint{3.905098in}{0.819261in}}%
\pgfpathclose%
\pgfusepath{stroke,fill}%
\end{pgfscope}%
\begin{pgfscope}%
\pgfpathrectangle{\pgfqpoint{3.793912in}{0.557870in}}{\pgfqpoint{2.446088in}{1.484734in}}%
\pgfusepath{clip}%
\pgfsetbuttcap%
\pgfsetroundjoin%
\definecolor{currentfill}{rgb}{0.298039,0.447059,0.690196}%
\pgfsetfillcolor{currentfill}%
\pgfsetlinewidth{1.003750pt}%
\definecolor{currentstroke}{rgb}{0.298039,0.447059,0.690196}%
\pgfsetstrokecolor{currentstroke}%
\pgfsetdash{}{0pt}%
\pgfpathmoveto{\pgfqpoint{3.905098in}{0.819261in}}%
\pgfpathcurveto{\pgfqpoint{3.913334in}{0.819261in}}{\pgfqpoint{3.921234in}{0.822534in}}{\pgfqpoint{3.927058in}{0.828358in}}%
\pgfpathcurveto{\pgfqpoint{3.932882in}{0.834182in}}{\pgfqpoint{3.936155in}{0.842082in}}{\pgfqpoint{3.936155in}{0.850318in}}%
\pgfpathcurveto{\pgfqpoint{3.936155in}{0.858554in}}{\pgfqpoint{3.932882in}{0.866454in}}{\pgfqpoint{3.927058in}{0.872278in}}%
\pgfpathcurveto{\pgfqpoint{3.921234in}{0.878102in}}{\pgfqpoint{3.913334in}{0.881374in}}{\pgfqpoint{3.905098in}{0.881374in}}%
\pgfpathcurveto{\pgfqpoint{3.896862in}{0.881374in}}{\pgfqpoint{3.888962in}{0.878102in}}{\pgfqpoint{3.883138in}{0.872278in}}%
\pgfpathcurveto{\pgfqpoint{3.877314in}{0.866454in}}{\pgfqpoint{3.874042in}{0.858554in}}{\pgfqpoint{3.874042in}{0.850318in}}%
\pgfpathcurveto{\pgfqpoint{3.874042in}{0.842082in}}{\pgfqpoint{3.877314in}{0.834182in}}{\pgfqpoint{3.883138in}{0.828358in}}%
\pgfpathcurveto{\pgfqpoint{3.888962in}{0.822534in}}{\pgfqpoint{3.896862in}{0.819261in}}{\pgfqpoint{3.905098in}{0.819261in}}%
\pgfpathclose%
\pgfusepath{stroke,fill}%
\end{pgfscope}%
\begin{pgfscope}%
\pgfpathrectangle{\pgfqpoint{3.793912in}{0.557870in}}{\pgfqpoint{2.446088in}{1.484734in}}%
\pgfusepath{clip}%
\pgfsetbuttcap%
\pgfsetroundjoin%
\definecolor{currentfill}{rgb}{0.298039,0.447059,0.690196}%
\pgfsetfillcolor{currentfill}%
\pgfsetlinewidth{1.003750pt}%
\definecolor{currentstroke}{rgb}{0.298039,0.447059,0.690196}%
\pgfsetstrokecolor{currentstroke}%
\pgfsetdash{}{0pt}%
\pgfpathmoveto{\pgfqpoint{3.905098in}{1.124564in}}%
\pgfpathcurveto{\pgfqpoint{3.913334in}{1.124564in}}{\pgfqpoint{3.921234in}{1.127836in}}{\pgfqpoint{3.927058in}{1.133660in}}%
\pgfpathcurveto{\pgfqpoint{3.932882in}{1.139484in}}{\pgfqpoint{3.936155in}{1.147384in}}{\pgfqpoint{3.936155in}{1.155620in}}%
\pgfpathcurveto{\pgfqpoint{3.936155in}{1.163857in}}{\pgfqpoint{3.932882in}{1.171757in}}{\pgfqpoint{3.927058in}{1.177581in}}%
\pgfpathcurveto{\pgfqpoint{3.921234in}{1.183404in}}{\pgfqpoint{3.913334in}{1.186677in}}{\pgfqpoint{3.905098in}{1.186677in}}%
\pgfpathcurveto{\pgfqpoint{3.896862in}{1.186677in}}{\pgfqpoint{3.888962in}{1.183404in}}{\pgfqpoint{3.883138in}{1.177581in}}%
\pgfpathcurveto{\pgfqpoint{3.877314in}{1.171757in}}{\pgfqpoint{3.874042in}{1.163857in}}{\pgfqpoint{3.874042in}{1.155620in}}%
\pgfpathcurveto{\pgfqpoint{3.874042in}{1.147384in}}{\pgfqpoint{3.877314in}{1.139484in}}{\pgfqpoint{3.883138in}{1.133660in}}%
\pgfpathcurveto{\pgfqpoint{3.888962in}{1.127836in}}{\pgfqpoint{3.896862in}{1.124564in}}{\pgfqpoint{3.905098in}{1.124564in}}%
\pgfpathclose%
\pgfusepath{stroke,fill}%
\end{pgfscope}%
\begin{pgfscope}%
\pgfpathrectangle{\pgfqpoint{3.793912in}{0.557870in}}{\pgfqpoint{2.446088in}{1.484734in}}%
\pgfusepath{clip}%
\pgfsetbuttcap%
\pgfsetroundjoin%
\definecolor{currentfill}{rgb}{0.298039,0.447059,0.690196}%
\pgfsetfillcolor{currentfill}%
\pgfsetlinewidth{1.003750pt}%
\definecolor{currentstroke}{rgb}{0.298039,0.447059,0.690196}%
\pgfsetstrokecolor{currentstroke}%
\pgfsetdash{}{0pt}%
\pgfpathmoveto{\pgfqpoint{3.905098in}{1.526277in}}%
\pgfpathcurveto{\pgfqpoint{3.913334in}{1.526277in}}{\pgfqpoint{3.921234in}{1.529550in}}{\pgfqpoint{3.927058in}{1.535374in}}%
\pgfpathcurveto{\pgfqpoint{3.932882in}{1.541198in}}{\pgfqpoint{3.936155in}{1.549098in}}{\pgfqpoint{3.936155in}{1.557334in}}%
\pgfpathcurveto{\pgfqpoint{3.936155in}{1.565570in}}{\pgfqpoint{3.932882in}{1.573470in}}{\pgfqpoint{3.927058in}{1.579294in}}%
\pgfpathcurveto{\pgfqpoint{3.921234in}{1.585118in}}{\pgfqpoint{3.913334in}{1.588390in}}{\pgfqpoint{3.905098in}{1.588390in}}%
\pgfpathcurveto{\pgfqpoint{3.896862in}{1.588390in}}{\pgfqpoint{3.888962in}{1.585118in}}{\pgfqpoint{3.883138in}{1.579294in}}%
\pgfpathcurveto{\pgfqpoint{3.877314in}{1.573470in}}{\pgfqpoint{3.874042in}{1.565570in}}{\pgfqpoint{3.874042in}{1.557334in}}%
\pgfpathcurveto{\pgfqpoint{3.874042in}{1.549098in}}{\pgfqpoint{3.877314in}{1.541198in}}{\pgfqpoint{3.883138in}{1.535374in}}%
\pgfpathcurveto{\pgfqpoint{3.888962in}{1.529550in}}{\pgfqpoint{3.896862in}{1.526277in}}{\pgfqpoint{3.905098in}{1.526277in}}%
\pgfpathclose%
\pgfusepath{stroke,fill}%
\end{pgfscope}%
\begin{pgfscope}%
\pgfpathrectangle{\pgfqpoint{3.793912in}{0.557870in}}{\pgfqpoint{2.446088in}{1.484734in}}%
\pgfusepath{clip}%
\pgfsetbuttcap%
\pgfsetroundjoin%
\definecolor{currentfill}{rgb}{0.298039,0.447059,0.690196}%
\pgfsetfillcolor{currentfill}%
\pgfsetlinewidth{1.003750pt}%
\definecolor{currentstroke}{rgb}{0.298039,0.447059,0.690196}%
\pgfsetstrokecolor{currentstroke}%
\pgfsetdash{}{0pt}%
\pgfpathmoveto{\pgfqpoint{3.905098in}{1.044221in}}%
\pgfpathcurveto{\pgfqpoint{3.913334in}{1.044221in}}{\pgfqpoint{3.921234in}{1.047493in}}{\pgfqpoint{3.927058in}{1.053317in}}%
\pgfpathcurveto{\pgfqpoint{3.932882in}{1.059141in}}{\pgfqpoint{3.936155in}{1.067041in}}{\pgfqpoint{3.936155in}{1.075278in}}%
\pgfpathcurveto{\pgfqpoint{3.936155in}{1.083514in}}{\pgfqpoint{3.932882in}{1.091414in}}{\pgfqpoint{3.927058in}{1.097238in}}%
\pgfpathcurveto{\pgfqpoint{3.921234in}{1.103062in}}{\pgfqpoint{3.913334in}{1.106334in}}{\pgfqpoint{3.905098in}{1.106334in}}%
\pgfpathcurveto{\pgfqpoint{3.896862in}{1.106334in}}{\pgfqpoint{3.888962in}{1.103062in}}{\pgfqpoint{3.883138in}{1.097238in}}%
\pgfpathcurveto{\pgfqpoint{3.877314in}{1.091414in}}{\pgfqpoint{3.874042in}{1.083514in}}{\pgfqpoint{3.874042in}{1.075278in}}%
\pgfpathcurveto{\pgfqpoint{3.874042in}{1.067041in}}{\pgfqpoint{3.877314in}{1.059141in}}{\pgfqpoint{3.883138in}{1.053317in}}%
\pgfpathcurveto{\pgfqpoint{3.888962in}{1.047493in}}{\pgfqpoint{3.896862in}{1.044221in}}{\pgfqpoint{3.905098in}{1.044221in}}%
\pgfpathclose%
\pgfusepath{stroke,fill}%
\end{pgfscope}%
\begin{pgfscope}%
\pgfpathrectangle{\pgfqpoint{3.793912in}{0.557870in}}{\pgfqpoint{2.446088in}{1.484734in}}%
\pgfusepath{clip}%
\pgfsetbuttcap%
\pgfsetroundjoin%
\definecolor{currentfill}{rgb}{0.298039,0.447059,0.690196}%
\pgfsetfillcolor{currentfill}%
\pgfsetlinewidth{1.003750pt}%
\definecolor{currentstroke}{rgb}{0.298039,0.447059,0.690196}%
\pgfsetstrokecolor{currentstroke}%
\pgfsetdash{}{0pt}%
\pgfpathmoveto{\pgfqpoint{3.905098in}{1.550380in}}%
\pgfpathcurveto{\pgfqpoint{3.913334in}{1.550380in}}{\pgfqpoint{3.921234in}{1.553653in}}{\pgfqpoint{3.927058in}{1.559477in}}%
\pgfpathcurveto{\pgfqpoint{3.932882in}{1.565300in}}{\pgfqpoint{3.936155in}{1.573201in}}{\pgfqpoint{3.936155in}{1.581437in}}%
\pgfpathcurveto{\pgfqpoint{3.936155in}{1.589673in}}{\pgfqpoint{3.932882in}{1.597573in}}{\pgfqpoint{3.927058in}{1.603397in}}%
\pgfpathcurveto{\pgfqpoint{3.921234in}{1.609221in}}{\pgfqpoint{3.913334in}{1.612493in}}{\pgfqpoint{3.905098in}{1.612493in}}%
\pgfpathcurveto{\pgfqpoint{3.896862in}{1.612493in}}{\pgfqpoint{3.888962in}{1.609221in}}{\pgfqpoint{3.883138in}{1.603397in}}%
\pgfpathcurveto{\pgfqpoint{3.877314in}{1.597573in}}{\pgfqpoint{3.874042in}{1.589673in}}{\pgfqpoint{3.874042in}{1.581437in}}%
\pgfpathcurveto{\pgfqpoint{3.874042in}{1.573201in}}{\pgfqpoint{3.877314in}{1.565300in}}{\pgfqpoint{3.883138in}{1.559477in}}%
\pgfpathcurveto{\pgfqpoint{3.888962in}{1.553653in}}{\pgfqpoint{3.896862in}{1.550380in}}{\pgfqpoint{3.905098in}{1.550380in}}%
\pgfpathclose%
\pgfusepath{stroke,fill}%
\end{pgfscope}%
\begin{pgfscope}%
\pgfpathrectangle{\pgfqpoint{3.793912in}{0.557870in}}{\pgfqpoint{2.446088in}{1.484734in}}%
\pgfusepath{clip}%
\pgfsetbuttcap%
\pgfsetroundjoin%
\definecolor{currentfill}{rgb}{0.298039,0.447059,0.690196}%
\pgfsetfillcolor{currentfill}%
\pgfsetlinewidth{1.003750pt}%
\definecolor{currentstroke}{rgb}{0.298039,0.447059,0.690196}%
\pgfsetstrokecolor{currentstroke}%
\pgfsetdash{}{0pt}%
\pgfpathmoveto{\pgfqpoint{3.905098in}{1.220975in}}%
\pgfpathcurveto{\pgfqpoint{3.913334in}{1.220975in}}{\pgfqpoint{3.921234in}{1.224247in}}{\pgfqpoint{3.927058in}{1.230071in}}%
\pgfpathcurveto{\pgfqpoint{3.932882in}{1.235895in}}{\pgfqpoint{3.936155in}{1.243795in}}{\pgfqpoint{3.936155in}{1.252032in}}%
\pgfpathcurveto{\pgfqpoint{3.936155in}{1.260268in}}{\pgfqpoint{3.932882in}{1.268168in}}{\pgfqpoint{3.927058in}{1.273992in}}%
\pgfpathcurveto{\pgfqpoint{3.921234in}{1.279816in}}{\pgfqpoint{3.913334in}{1.283088in}}{\pgfqpoint{3.905098in}{1.283088in}}%
\pgfpathcurveto{\pgfqpoint{3.896862in}{1.283088in}}{\pgfqpoint{3.888962in}{1.279816in}}{\pgfqpoint{3.883138in}{1.273992in}}%
\pgfpathcurveto{\pgfqpoint{3.877314in}{1.268168in}}{\pgfqpoint{3.874042in}{1.260268in}}{\pgfqpoint{3.874042in}{1.252032in}}%
\pgfpathcurveto{\pgfqpoint{3.874042in}{1.243795in}}{\pgfqpoint{3.877314in}{1.235895in}}{\pgfqpoint{3.883138in}{1.230071in}}%
\pgfpathcurveto{\pgfqpoint{3.888962in}{1.224247in}}{\pgfqpoint{3.896862in}{1.220975in}}{\pgfqpoint{3.905098in}{1.220975in}}%
\pgfpathclose%
\pgfusepath{stroke,fill}%
\end{pgfscope}%
\begin{pgfscope}%
\pgfpathrectangle{\pgfqpoint{3.793912in}{0.557870in}}{\pgfqpoint{2.446088in}{1.484734in}}%
\pgfusepath{clip}%
\pgfsetbuttcap%
\pgfsetroundjoin%
\definecolor{currentfill}{rgb}{0.298039,0.447059,0.690196}%
\pgfsetfillcolor{currentfill}%
\pgfsetlinewidth{1.003750pt}%
\definecolor{currentstroke}{rgb}{0.298039,0.447059,0.690196}%
\pgfsetstrokecolor{currentstroke}%
\pgfsetdash{}{0pt}%
\pgfpathmoveto{\pgfqpoint{3.905098in}{1.220975in}}%
\pgfpathcurveto{\pgfqpoint{3.913334in}{1.220975in}}{\pgfqpoint{3.921234in}{1.224247in}}{\pgfqpoint{3.927058in}{1.230071in}}%
\pgfpathcurveto{\pgfqpoint{3.932882in}{1.235895in}}{\pgfqpoint{3.936155in}{1.243795in}}{\pgfqpoint{3.936155in}{1.252032in}}%
\pgfpathcurveto{\pgfqpoint{3.936155in}{1.260268in}}{\pgfqpoint{3.932882in}{1.268168in}}{\pgfqpoint{3.927058in}{1.273992in}}%
\pgfpathcurveto{\pgfqpoint{3.921234in}{1.279816in}}{\pgfqpoint{3.913334in}{1.283088in}}{\pgfqpoint{3.905098in}{1.283088in}}%
\pgfpathcurveto{\pgfqpoint{3.896862in}{1.283088in}}{\pgfqpoint{3.888962in}{1.279816in}}{\pgfqpoint{3.883138in}{1.273992in}}%
\pgfpathcurveto{\pgfqpoint{3.877314in}{1.268168in}}{\pgfqpoint{3.874042in}{1.260268in}}{\pgfqpoint{3.874042in}{1.252032in}}%
\pgfpathcurveto{\pgfqpoint{3.874042in}{1.243795in}}{\pgfqpoint{3.877314in}{1.235895in}}{\pgfqpoint{3.883138in}{1.230071in}}%
\pgfpathcurveto{\pgfqpoint{3.888962in}{1.224247in}}{\pgfqpoint{3.896862in}{1.220975in}}{\pgfqpoint{3.905098in}{1.220975in}}%
\pgfpathclose%
\pgfusepath{stroke,fill}%
\end{pgfscope}%
\begin{pgfscope}%
\pgfpathrectangle{\pgfqpoint{3.793912in}{0.557870in}}{\pgfqpoint{2.446088in}{1.484734in}}%
\pgfusepath{clip}%
\pgfsetbuttcap%
\pgfsetroundjoin%
\definecolor{currentfill}{rgb}{0.298039,0.447059,0.690196}%
\pgfsetfillcolor{currentfill}%
\pgfsetlinewidth{1.003750pt}%
\definecolor{currentstroke}{rgb}{0.298039,0.447059,0.690196}%
\pgfsetstrokecolor{currentstroke}%
\pgfsetdash{}{0pt}%
\pgfpathmoveto{\pgfqpoint{3.905098in}{1.936025in}}%
\pgfpathcurveto{\pgfqpoint{3.913334in}{1.936025in}}{\pgfqpoint{3.921234in}{1.939298in}}{\pgfqpoint{3.927058in}{1.945122in}}%
\pgfpathcurveto{\pgfqpoint{3.932882in}{1.950946in}}{\pgfqpoint{3.936155in}{1.958846in}}{\pgfqpoint{3.936155in}{1.967082in}}%
\pgfpathcurveto{\pgfqpoint{3.936155in}{1.975318in}}{\pgfqpoint{3.932882in}{1.983218in}}{\pgfqpoint{3.927058in}{1.989042in}}%
\pgfpathcurveto{\pgfqpoint{3.921234in}{1.994866in}}{\pgfqpoint{3.913334in}{1.998138in}}{\pgfqpoint{3.905098in}{1.998138in}}%
\pgfpathcurveto{\pgfqpoint{3.896862in}{1.998138in}}{\pgfqpoint{3.888962in}{1.994866in}}{\pgfqpoint{3.883138in}{1.989042in}}%
\pgfpathcurveto{\pgfqpoint{3.877314in}{1.983218in}}{\pgfqpoint{3.874042in}{1.975318in}}{\pgfqpoint{3.874042in}{1.967082in}}%
\pgfpathcurveto{\pgfqpoint{3.874042in}{1.958846in}}{\pgfqpoint{3.877314in}{1.950946in}}{\pgfqpoint{3.883138in}{1.945122in}}%
\pgfpathcurveto{\pgfqpoint{3.888962in}{1.939298in}}{\pgfqpoint{3.896862in}{1.936025in}}{\pgfqpoint{3.905098in}{1.936025in}}%
\pgfpathclose%
\pgfusepath{stroke,fill}%
\end{pgfscope}%
\begin{pgfscope}%
\pgfpathrectangle{\pgfqpoint{3.793912in}{0.557870in}}{\pgfqpoint{2.446088in}{1.484734in}}%
\pgfusepath{clip}%
\pgfsetbuttcap%
\pgfsetroundjoin%
\definecolor{currentfill}{rgb}{0.298039,0.447059,0.690196}%
\pgfsetfillcolor{currentfill}%
\pgfsetlinewidth{1.003750pt}%
\definecolor{currentstroke}{rgb}{0.298039,0.447059,0.690196}%
\pgfsetstrokecolor{currentstroke}%
\pgfsetdash{}{0pt}%
\pgfpathmoveto{\pgfqpoint{3.905098in}{1.936025in}}%
\pgfpathcurveto{\pgfqpoint{3.913334in}{1.936025in}}{\pgfqpoint{3.921234in}{1.939298in}}{\pgfqpoint{3.927058in}{1.945122in}}%
\pgfpathcurveto{\pgfqpoint{3.932882in}{1.950946in}}{\pgfqpoint{3.936155in}{1.958846in}}{\pgfqpoint{3.936155in}{1.967082in}}%
\pgfpathcurveto{\pgfqpoint{3.936155in}{1.975318in}}{\pgfqpoint{3.932882in}{1.983218in}}{\pgfqpoint{3.927058in}{1.989042in}}%
\pgfpathcurveto{\pgfqpoint{3.921234in}{1.994866in}}{\pgfqpoint{3.913334in}{1.998138in}}{\pgfqpoint{3.905098in}{1.998138in}}%
\pgfpathcurveto{\pgfqpoint{3.896862in}{1.998138in}}{\pgfqpoint{3.888962in}{1.994866in}}{\pgfqpoint{3.883138in}{1.989042in}}%
\pgfpathcurveto{\pgfqpoint{3.877314in}{1.983218in}}{\pgfqpoint{3.874042in}{1.975318in}}{\pgfqpoint{3.874042in}{1.967082in}}%
\pgfpathcurveto{\pgfqpoint{3.874042in}{1.958846in}}{\pgfqpoint{3.877314in}{1.950946in}}{\pgfqpoint{3.883138in}{1.945122in}}%
\pgfpathcurveto{\pgfqpoint{3.888962in}{1.939298in}}{\pgfqpoint{3.896862in}{1.936025in}}{\pgfqpoint{3.905098in}{1.936025in}}%
\pgfpathclose%
\pgfusepath{stroke,fill}%
\end{pgfscope}%
\begin{pgfscope}%
\pgfpathrectangle{\pgfqpoint{3.793912in}{0.557870in}}{\pgfqpoint{2.446088in}{1.484734in}}%
\pgfusepath{clip}%
\pgfsetbuttcap%
\pgfsetroundjoin%
\definecolor{currentfill}{rgb}{0.298039,0.447059,0.690196}%
\pgfsetfillcolor{currentfill}%
\pgfsetlinewidth{1.003750pt}%
\definecolor{currentstroke}{rgb}{0.298039,0.447059,0.690196}%
\pgfsetstrokecolor{currentstroke}%
\pgfsetdash{}{0pt}%
\pgfpathmoveto{\pgfqpoint{3.905098in}{1.936025in}}%
\pgfpathcurveto{\pgfqpoint{3.913334in}{1.936025in}}{\pgfqpoint{3.921234in}{1.939298in}}{\pgfqpoint{3.927058in}{1.945122in}}%
\pgfpathcurveto{\pgfqpoint{3.932882in}{1.950946in}}{\pgfqpoint{3.936155in}{1.958846in}}{\pgfqpoint{3.936155in}{1.967082in}}%
\pgfpathcurveto{\pgfqpoint{3.936155in}{1.975318in}}{\pgfqpoint{3.932882in}{1.983218in}}{\pgfqpoint{3.927058in}{1.989042in}}%
\pgfpathcurveto{\pgfqpoint{3.921234in}{1.994866in}}{\pgfqpoint{3.913334in}{1.998138in}}{\pgfqpoint{3.905098in}{1.998138in}}%
\pgfpathcurveto{\pgfqpoint{3.896862in}{1.998138in}}{\pgfqpoint{3.888962in}{1.994866in}}{\pgfqpoint{3.883138in}{1.989042in}}%
\pgfpathcurveto{\pgfqpoint{3.877314in}{1.983218in}}{\pgfqpoint{3.874042in}{1.975318in}}{\pgfqpoint{3.874042in}{1.967082in}}%
\pgfpathcurveto{\pgfqpoint{3.874042in}{1.958846in}}{\pgfqpoint{3.877314in}{1.950946in}}{\pgfqpoint{3.883138in}{1.945122in}}%
\pgfpathcurveto{\pgfqpoint{3.888962in}{1.939298in}}{\pgfqpoint{3.896862in}{1.936025in}}{\pgfqpoint{3.905098in}{1.936025in}}%
\pgfpathclose%
\pgfusepath{stroke,fill}%
\end{pgfscope}%
\begin{pgfscope}%
\pgfpathrectangle{\pgfqpoint{3.793912in}{0.557870in}}{\pgfqpoint{2.446088in}{1.484734in}}%
\pgfusepath{clip}%
\pgfsetbuttcap%
\pgfsetroundjoin%
\definecolor{currentfill}{rgb}{0.298039,0.447059,0.690196}%
\pgfsetfillcolor{currentfill}%
\pgfsetlinewidth{1.003750pt}%
\definecolor{currentstroke}{rgb}{0.298039,0.447059,0.690196}%
\pgfsetstrokecolor{currentstroke}%
\pgfsetdash{}{0pt}%
\pgfpathmoveto{\pgfqpoint{3.905098in}{1.156701in}}%
\pgfpathcurveto{\pgfqpoint{3.913334in}{1.156701in}}{\pgfqpoint{3.921234in}{1.159973in}}{\pgfqpoint{3.927058in}{1.165797in}}%
\pgfpathcurveto{\pgfqpoint{3.932882in}{1.171621in}}{\pgfqpoint{3.936155in}{1.179521in}}{\pgfqpoint{3.936155in}{1.187757in}}%
\pgfpathcurveto{\pgfqpoint{3.936155in}{1.195994in}}{\pgfqpoint{3.932882in}{1.203894in}}{\pgfqpoint{3.927058in}{1.209718in}}%
\pgfpathcurveto{\pgfqpoint{3.921234in}{1.215542in}}{\pgfqpoint{3.913334in}{1.218814in}}{\pgfqpoint{3.905098in}{1.218814in}}%
\pgfpathcurveto{\pgfqpoint{3.896862in}{1.218814in}}{\pgfqpoint{3.888962in}{1.215542in}}{\pgfqpoint{3.883138in}{1.209718in}}%
\pgfpathcurveto{\pgfqpoint{3.877314in}{1.203894in}}{\pgfqpoint{3.874042in}{1.195994in}}{\pgfqpoint{3.874042in}{1.187757in}}%
\pgfpathcurveto{\pgfqpoint{3.874042in}{1.179521in}}{\pgfqpoint{3.877314in}{1.171621in}}{\pgfqpoint{3.883138in}{1.165797in}}%
\pgfpathcurveto{\pgfqpoint{3.888962in}{1.159973in}}{\pgfqpoint{3.896862in}{1.156701in}}{\pgfqpoint{3.905098in}{1.156701in}}%
\pgfpathclose%
\pgfusepath{stroke,fill}%
\end{pgfscope}%
\begin{pgfscope}%
\pgfpathrectangle{\pgfqpoint{3.793912in}{0.557870in}}{\pgfqpoint{2.446088in}{1.484734in}}%
\pgfusepath{clip}%
\pgfsetbuttcap%
\pgfsetroundjoin%
\definecolor{currentfill}{rgb}{0.298039,0.447059,0.690196}%
\pgfsetfillcolor{currentfill}%
\pgfsetlinewidth{1.003750pt}%
\definecolor{currentstroke}{rgb}{0.298039,0.447059,0.690196}%
\pgfsetstrokecolor{currentstroke}%
\pgfsetdash{}{0pt}%
\pgfpathmoveto{\pgfqpoint{3.905098in}{1.936025in}}%
\pgfpathcurveto{\pgfqpoint{3.913334in}{1.936025in}}{\pgfqpoint{3.921234in}{1.939298in}}{\pgfqpoint{3.927058in}{1.945122in}}%
\pgfpathcurveto{\pgfqpoint{3.932882in}{1.950946in}}{\pgfqpoint{3.936155in}{1.958846in}}{\pgfqpoint{3.936155in}{1.967082in}}%
\pgfpathcurveto{\pgfqpoint{3.936155in}{1.975318in}}{\pgfqpoint{3.932882in}{1.983218in}}{\pgfqpoint{3.927058in}{1.989042in}}%
\pgfpathcurveto{\pgfqpoint{3.921234in}{1.994866in}}{\pgfqpoint{3.913334in}{1.998138in}}{\pgfqpoint{3.905098in}{1.998138in}}%
\pgfpathcurveto{\pgfqpoint{3.896862in}{1.998138in}}{\pgfqpoint{3.888962in}{1.994866in}}{\pgfqpoint{3.883138in}{1.989042in}}%
\pgfpathcurveto{\pgfqpoint{3.877314in}{1.983218in}}{\pgfqpoint{3.874042in}{1.975318in}}{\pgfqpoint{3.874042in}{1.967082in}}%
\pgfpathcurveto{\pgfqpoint{3.874042in}{1.958846in}}{\pgfqpoint{3.877314in}{1.950946in}}{\pgfqpoint{3.883138in}{1.945122in}}%
\pgfpathcurveto{\pgfqpoint{3.888962in}{1.939298in}}{\pgfqpoint{3.896862in}{1.936025in}}{\pgfqpoint{3.905098in}{1.936025in}}%
\pgfpathclose%
\pgfusepath{stroke,fill}%
\end{pgfscope}%
\begin{pgfscope}%
\pgfpathrectangle{\pgfqpoint{3.793912in}{0.557870in}}{\pgfqpoint{2.446088in}{1.484734in}}%
\pgfusepath{clip}%
\pgfsetbuttcap%
\pgfsetroundjoin%
\definecolor{currentfill}{rgb}{0.298039,0.447059,0.690196}%
\pgfsetfillcolor{currentfill}%
\pgfsetlinewidth{1.003750pt}%
\definecolor{currentstroke}{rgb}{0.298039,0.447059,0.690196}%
\pgfsetstrokecolor{currentstroke}%
\pgfsetdash{}{0pt}%
\pgfpathmoveto{\pgfqpoint{3.905098in}{1.936025in}}%
\pgfpathcurveto{\pgfqpoint{3.913334in}{1.936025in}}{\pgfqpoint{3.921234in}{1.939298in}}{\pgfqpoint{3.927058in}{1.945122in}}%
\pgfpathcurveto{\pgfqpoint{3.932882in}{1.950946in}}{\pgfqpoint{3.936155in}{1.958846in}}{\pgfqpoint{3.936155in}{1.967082in}}%
\pgfpathcurveto{\pgfqpoint{3.936155in}{1.975318in}}{\pgfqpoint{3.932882in}{1.983218in}}{\pgfqpoint{3.927058in}{1.989042in}}%
\pgfpathcurveto{\pgfqpoint{3.921234in}{1.994866in}}{\pgfqpoint{3.913334in}{1.998138in}}{\pgfqpoint{3.905098in}{1.998138in}}%
\pgfpathcurveto{\pgfqpoint{3.896862in}{1.998138in}}{\pgfqpoint{3.888962in}{1.994866in}}{\pgfqpoint{3.883138in}{1.989042in}}%
\pgfpathcurveto{\pgfqpoint{3.877314in}{1.983218in}}{\pgfqpoint{3.874042in}{1.975318in}}{\pgfqpoint{3.874042in}{1.967082in}}%
\pgfpathcurveto{\pgfqpoint{3.874042in}{1.958846in}}{\pgfqpoint{3.877314in}{1.950946in}}{\pgfqpoint{3.883138in}{1.945122in}}%
\pgfpathcurveto{\pgfqpoint{3.888962in}{1.939298in}}{\pgfqpoint{3.896862in}{1.936025in}}{\pgfqpoint{3.905098in}{1.936025in}}%
\pgfpathclose%
\pgfusepath{stroke,fill}%
\end{pgfscope}%
\begin{pgfscope}%
\pgfpathrectangle{\pgfqpoint{3.793912in}{0.557870in}}{\pgfqpoint{2.446088in}{1.484734in}}%
\pgfusepath{clip}%
\pgfsetbuttcap%
\pgfsetroundjoin%
\definecolor{currentfill}{rgb}{0.298039,0.447059,0.690196}%
\pgfsetfillcolor{currentfill}%
\pgfsetlinewidth{1.003750pt}%
\definecolor{currentstroke}{rgb}{0.298039,0.447059,0.690196}%
\pgfsetstrokecolor{currentstroke}%
\pgfsetdash{}{0pt}%
\pgfpathmoveto{\pgfqpoint{3.905098in}{1.936025in}}%
\pgfpathcurveto{\pgfqpoint{3.913334in}{1.936025in}}{\pgfqpoint{3.921234in}{1.939298in}}{\pgfqpoint{3.927058in}{1.945122in}}%
\pgfpathcurveto{\pgfqpoint{3.932882in}{1.950946in}}{\pgfqpoint{3.936155in}{1.958846in}}{\pgfqpoint{3.936155in}{1.967082in}}%
\pgfpathcurveto{\pgfqpoint{3.936155in}{1.975318in}}{\pgfqpoint{3.932882in}{1.983218in}}{\pgfqpoint{3.927058in}{1.989042in}}%
\pgfpathcurveto{\pgfqpoint{3.921234in}{1.994866in}}{\pgfqpoint{3.913334in}{1.998138in}}{\pgfqpoint{3.905098in}{1.998138in}}%
\pgfpathcurveto{\pgfqpoint{3.896862in}{1.998138in}}{\pgfqpoint{3.888962in}{1.994866in}}{\pgfqpoint{3.883138in}{1.989042in}}%
\pgfpathcurveto{\pgfqpoint{3.877314in}{1.983218in}}{\pgfqpoint{3.874042in}{1.975318in}}{\pgfqpoint{3.874042in}{1.967082in}}%
\pgfpathcurveto{\pgfqpoint{3.874042in}{1.958846in}}{\pgfqpoint{3.877314in}{1.950946in}}{\pgfqpoint{3.883138in}{1.945122in}}%
\pgfpathcurveto{\pgfqpoint{3.888962in}{1.939298in}}{\pgfqpoint{3.896862in}{1.936025in}}{\pgfqpoint{3.905098in}{1.936025in}}%
\pgfpathclose%
\pgfusepath{stroke,fill}%
\end{pgfscope}%
\begin{pgfscope}%
\pgfpathrectangle{\pgfqpoint{3.793912in}{0.557870in}}{\pgfqpoint{2.446088in}{1.484734in}}%
\pgfusepath{clip}%
\pgfsetbuttcap%
\pgfsetroundjoin%
\definecolor{currentfill}{rgb}{0.298039,0.447059,0.690196}%
\pgfsetfillcolor{currentfill}%
\pgfsetlinewidth{1.003750pt}%
\definecolor{currentstroke}{rgb}{0.298039,0.447059,0.690196}%
\pgfsetstrokecolor{currentstroke}%
\pgfsetdash{}{0pt}%
\pgfpathmoveto{\pgfqpoint{3.905098in}{1.936025in}}%
\pgfpathcurveto{\pgfqpoint{3.913334in}{1.936025in}}{\pgfqpoint{3.921234in}{1.939298in}}{\pgfqpoint{3.927058in}{1.945122in}}%
\pgfpathcurveto{\pgfqpoint{3.932882in}{1.950946in}}{\pgfqpoint{3.936155in}{1.958846in}}{\pgfqpoint{3.936155in}{1.967082in}}%
\pgfpathcurveto{\pgfqpoint{3.936155in}{1.975318in}}{\pgfqpoint{3.932882in}{1.983218in}}{\pgfqpoint{3.927058in}{1.989042in}}%
\pgfpathcurveto{\pgfqpoint{3.921234in}{1.994866in}}{\pgfqpoint{3.913334in}{1.998138in}}{\pgfqpoint{3.905098in}{1.998138in}}%
\pgfpathcurveto{\pgfqpoint{3.896862in}{1.998138in}}{\pgfqpoint{3.888962in}{1.994866in}}{\pgfqpoint{3.883138in}{1.989042in}}%
\pgfpathcurveto{\pgfqpoint{3.877314in}{1.983218in}}{\pgfqpoint{3.874042in}{1.975318in}}{\pgfqpoint{3.874042in}{1.967082in}}%
\pgfpathcurveto{\pgfqpoint{3.874042in}{1.958846in}}{\pgfqpoint{3.877314in}{1.950946in}}{\pgfqpoint{3.883138in}{1.945122in}}%
\pgfpathcurveto{\pgfqpoint{3.888962in}{1.939298in}}{\pgfqpoint{3.896862in}{1.936025in}}{\pgfqpoint{3.905098in}{1.936025in}}%
\pgfpathclose%
\pgfusepath{stroke,fill}%
\end{pgfscope}%
\begin{pgfscope}%
\pgfpathrectangle{\pgfqpoint{3.793912in}{0.557870in}}{\pgfqpoint{2.446088in}{1.484734in}}%
\pgfusepath{clip}%
\pgfsetbuttcap%
\pgfsetroundjoin%
\definecolor{currentfill}{rgb}{0.298039,0.447059,0.690196}%
\pgfsetfillcolor{currentfill}%
\pgfsetlinewidth{1.003750pt}%
\definecolor{currentstroke}{rgb}{0.298039,0.447059,0.690196}%
\pgfsetstrokecolor{currentstroke}%
\pgfsetdash{}{0pt}%
\pgfpathmoveto{\pgfqpoint{3.905098in}{1.936025in}}%
\pgfpathcurveto{\pgfqpoint{3.913334in}{1.936025in}}{\pgfqpoint{3.921234in}{1.939298in}}{\pgfqpoint{3.927058in}{1.945122in}}%
\pgfpathcurveto{\pgfqpoint{3.932882in}{1.950946in}}{\pgfqpoint{3.936155in}{1.958846in}}{\pgfqpoint{3.936155in}{1.967082in}}%
\pgfpathcurveto{\pgfqpoint{3.936155in}{1.975318in}}{\pgfqpoint{3.932882in}{1.983218in}}{\pgfqpoint{3.927058in}{1.989042in}}%
\pgfpathcurveto{\pgfqpoint{3.921234in}{1.994866in}}{\pgfqpoint{3.913334in}{1.998138in}}{\pgfqpoint{3.905098in}{1.998138in}}%
\pgfpathcurveto{\pgfqpoint{3.896862in}{1.998138in}}{\pgfqpoint{3.888962in}{1.994866in}}{\pgfqpoint{3.883138in}{1.989042in}}%
\pgfpathcurveto{\pgfqpoint{3.877314in}{1.983218in}}{\pgfqpoint{3.874042in}{1.975318in}}{\pgfqpoint{3.874042in}{1.967082in}}%
\pgfpathcurveto{\pgfqpoint{3.874042in}{1.958846in}}{\pgfqpoint{3.877314in}{1.950946in}}{\pgfqpoint{3.883138in}{1.945122in}}%
\pgfpathcurveto{\pgfqpoint{3.888962in}{1.939298in}}{\pgfqpoint{3.896862in}{1.936025in}}{\pgfqpoint{3.905098in}{1.936025in}}%
\pgfpathclose%
\pgfusepath{stroke,fill}%
\end{pgfscope}%
\begin{pgfscope}%
\pgfpathrectangle{\pgfqpoint{3.793912in}{0.557870in}}{\pgfqpoint{2.446088in}{1.484734in}}%
\pgfusepath{clip}%
\pgfsetbuttcap%
\pgfsetroundjoin%
\definecolor{currentfill}{rgb}{0.298039,0.447059,0.690196}%
\pgfsetfillcolor{currentfill}%
\pgfsetlinewidth{1.003750pt}%
\definecolor{currentstroke}{rgb}{0.298039,0.447059,0.690196}%
\pgfsetstrokecolor{currentstroke}%
\pgfsetdash{}{0pt}%
\pgfpathmoveto{\pgfqpoint{3.905098in}{1.936025in}}%
\pgfpathcurveto{\pgfqpoint{3.913334in}{1.936025in}}{\pgfqpoint{3.921234in}{1.939298in}}{\pgfqpoint{3.927058in}{1.945122in}}%
\pgfpathcurveto{\pgfqpoint{3.932882in}{1.950946in}}{\pgfqpoint{3.936155in}{1.958846in}}{\pgfqpoint{3.936155in}{1.967082in}}%
\pgfpathcurveto{\pgfqpoint{3.936155in}{1.975318in}}{\pgfqpoint{3.932882in}{1.983218in}}{\pgfqpoint{3.927058in}{1.989042in}}%
\pgfpathcurveto{\pgfqpoint{3.921234in}{1.994866in}}{\pgfqpoint{3.913334in}{1.998138in}}{\pgfqpoint{3.905098in}{1.998138in}}%
\pgfpathcurveto{\pgfqpoint{3.896862in}{1.998138in}}{\pgfqpoint{3.888962in}{1.994866in}}{\pgfqpoint{3.883138in}{1.989042in}}%
\pgfpathcurveto{\pgfqpoint{3.877314in}{1.983218in}}{\pgfqpoint{3.874042in}{1.975318in}}{\pgfqpoint{3.874042in}{1.967082in}}%
\pgfpathcurveto{\pgfqpoint{3.874042in}{1.958846in}}{\pgfqpoint{3.877314in}{1.950946in}}{\pgfqpoint{3.883138in}{1.945122in}}%
\pgfpathcurveto{\pgfqpoint{3.888962in}{1.939298in}}{\pgfqpoint{3.896862in}{1.936025in}}{\pgfqpoint{3.905098in}{1.936025in}}%
\pgfpathclose%
\pgfusepath{stroke,fill}%
\end{pgfscope}%
\begin{pgfscope}%
\pgfpathrectangle{\pgfqpoint{3.793912in}{0.557870in}}{\pgfqpoint{2.446088in}{1.484734in}}%
\pgfusepath{clip}%
\pgfsetbuttcap%
\pgfsetroundjoin%
\definecolor{currentfill}{rgb}{0.298039,0.447059,0.690196}%
\pgfsetfillcolor{currentfill}%
\pgfsetlinewidth{1.003750pt}%
\definecolor{currentstroke}{rgb}{0.298039,0.447059,0.690196}%
\pgfsetstrokecolor{currentstroke}%
\pgfsetdash{}{0pt}%
\pgfpathmoveto{\pgfqpoint{3.905098in}{1.092427in}}%
\pgfpathcurveto{\pgfqpoint{3.913334in}{1.092427in}}{\pgfqpoint{3.921234in}{1.095699in}}{\pgfqpoint{3.927058in}{1.101523in}}%
\pgfpathcurveto{\pgfqpoint{3.932882in}{1.107347in}}{\pgfqpoint{3.936155in}{1.115247in}}{\pgfqpoint{3.936155in}{1.123483in}}%
\pgfpathcurveto{\pgfqpoint{3.936155in}{1.131719in}}{\pgfqpoint{3.932882in}{1.139620in}}{\pgfqpoint{3.927058in}{1.145443in}}%
\pgfpathcurveto{\pgfqpoint{3.921234in}{1.151267in}}{\pgfqpoint{3.913334in}{1.154540in}}{\pgfqpoint{3.905098in}{1.154540in}}%
\pgfpathcurveto{\pgfqpoint{3.896862in}{1.154540in}}{\pgfqpoint{3.888962in}{1.151267in}}{\pgfqpoint{3.883138in}{1.145443in}}%
\pgfpathcurveto{\pgfqpoint{3.877314in}{1.139620in}}{\pgfqpoint{3.874042in}{1.131719in}}{\pgfqpoint{3.874042in}{1.123483in}}%
\pgfpathcurveto{\pgfqpoint{3.874042in}{1.115247in}}{\pgfqpoint{3.877314in}{1.107347in}}{\pgfqpoint{3.883138in}{1.101523in}}%
\pgfpathcurveto{\pgfqpoint{3.888962in}{1.095699in}}{\pgfqpoint{3.896862in}{1.092427in}}{\pgfqpoint{3.905098in}{1.092427in}}%
\pgfpathclose%
\pgfusepath{stroke,fill}%
\end{pgfscope}%
\begin{pgfscope}%
\pgfpathrectangle{\pgfqpoint{3.793912in}{0.557870in}}{\pgfqpoint{2.446088in}{1.484734in}}%
\pgfusepath{clip}%
\pgfsetbuttcap%
\pgfsetroundjoin%
\definecolor{currentfill}{rgb}{0.298039,0.447059,0.690196}%
\pgfsetfillcolor{currentfill}%
\pgfsetlinewidth{1.003750pt}%
\definecolor{currentstroke}{rgb}{0.298039,0.447059,0.690196}%
\pgfsetstrokecolor{currentstroke}%
\pgfsetdash{}{0pt}%
\pgfpathmoveto{\pgfqpoint{3.905098in}{1.936025in}}%
\pgfpathcurveto{\pgfqpoint{3.913334in}{1.936025in}}{\pgfqpoint{3.921234in}{1.939298in}}{\pgfqpoint{3.927058in}{1.945122in}}%
\pgfpathcurveto{\pgfqpoint{3.932882in}{1.950946in}}{\pgfqpoint{3.936155in}{1.958846in}}{\pgfqpoint{3.936155in}{1.967082in}}%
\pgfpathcurveto{\pgfqpoint{3.936155in}{1.975318in}}{\pgfqpoint{3.932882in}{1.983218in}}{\pgfqpoint{3.927058in}{1.989042in}}%
\pgfpathcurveto{\pgfqpoint{3.921234in}{1.994866in}}{\pgfqpoint{3.913334in}{1.998138in}}{\pgfqpoint{3.905098in}{1.998138in}}%
\pgfpathcurveto{\pgfqpoint{3.896862in}{1.998138in}}{\pgfqpoint{3.888962in}{1.994866in}}{\pgfqpoint{3.883138in}{1.989042in}}%
\pgfpathcurveto{\pgfqpoint{3.877314in}{1.983218in}}{\pgfqpoint{3.874042in}{1.975318in}}{\pgfqpoint{3.874042in}{1.967082in}}%
\pgfpathcurveto{\pgfqpoint{3.874042in}{1.958846in}}{\pgfqpoint{3.877314in}{1.950946in}}{\pgfqpoint{3.883138in}{1.945122in}}%
\pgfpathcurveto{\pgfqpoint{3.888962in}{1.939298in}}{\pgfqpoint{3.896862in}{1.936025in}}{\pgfqpoint{3.905098in}{1.936025in}}%
\pgfpathclose%
\pgfusepath{stroke,fill}%
\end{pgfscope}%
\begin{pgfscope}%
\pgfpathrectangle{\pgfqpoint{3.793912in}{0.557870in}}{\pgfqpoint{2.446088in}{1.484734in}}%
\pgfusepath{clip}%
\pgfsetbuttcap%
\pgfsetroundjoin%
\definecolor{currentfill}{rgb}{0.298039,0.447059,0.690196}%
\pgfsetfillcolor{currentfill}%
\pgfsetlinewidth{1.003750pt}%
\definecolor{currentstroke}{rgb}{0.298039,0.447059,0.690196}%
\pgfsetstrokecolor{currentstroke}%
\pgfsetdash{}{0pt}%
\pgfpathmoveto{\pgfqpoint{3.905098in}{1.936025in}}%
\pgfpathcurveto{\pgfqpoint{3.913334in}{1.936025in}}{\pgfqpoint{3.921234in}{1.939298in}}{\pgfqpoint{3.927058in}{1.945122in}}%
\pgfpathcurveto{\pgfqpoint{3.932882in}{1.950946in}}{\pgfqpoint{3.936155in}{1.958846in}}{\pgfqpoint{3.936155in}{1.967082in}}%
\pgfpathcurveto{\pgfqpoint{3.936155in}{1.975318in}}{\pgfqpoint{3.932882in}{1.983218in}}{\pgfqpoint{3.927058in}{1.989042in}}%
\pgfpathcurveto{\pgfqpoint{3.921234in}{1.994866in}}{\pgfqpoint{3.913334in}{1.998138in}}{\pgfqpoint{3.905098in}{1.998138in}}%
\pgfpathcurveto{\pgfqpoint{3.896862in}{1.998138in}}{\pgfqpoint{3.888962in}{1.994866in}}{\pgfqpoint{3.883138in}{1.989042in}}%
\pgfpathcurveto{\pgfqpoint{3.877314in}{1.983218in}}{\pgfqpoint{3.874042in}{1.975318in}}{\pgfqpoint{3.874042in}{1.967082in}}%
\pgfpathcurveto{\pgfqpoint{3.874042in}{1.958846in}}{\pgfqpoint{3.877314in}{1.950946in}}{\pgfqpoint{3.883138in}{1.945122in}}%
\pgfpathcurveto{\pgfqpoint{3.888962in}{1.939298in}}{\pgfqpoint{3.896862in}{1.936025in}}{\pgfqpoint{3.905098in}{1.936025in}}%
\pgfpathclose%
\pgfusepath{stroke,fill}%
\end{pgfscope}%
\begin{pgfscope}%
\pgfpathrectangle{\pgfqpoint{3.793912in}{0.557870in}}{\pgfqpoint{2.446088in}{1.484734in}}%
\pgfusepath{clip}%
\pgfsetbuttcap%
\pgfsetroundjoin%
\definecolor{currentfill}{rgb}{0.298039,0.447059,0.690196}%
\pgfsetfillcolor{currentfill}%
\pgfsetlinewidth{1.003750pt}%
\definecolor{currentstroke}{rgb}{0.298039,0.447059,0.690196}%
\pgfsetstrokecolor{currentstroke}%
\pgfsetdash{}{0pt}%
\pgfpathmoveto{\pgfqpoint{3.905098in}{1.936025in}}%
\pgfpathcurveto{\pgfqpoint{3.913334in}{1.936025in}}{\pgfqpoint{3.921234in}{1.939298in}}{\pgfqpoint{3.927058in}{1.945122in}}%
\pgfpathcurveto{\pgfqpoint{3.932882in}{1.950946in}}{\pgfqpoint{3.936155in}{1.958846in}}{\pgfqpoint{3.936155in}{1.967082in}}%
\pgfpathcurveto{\pgfqpoint{3.936155in}{1.975318in}}{\pgfqpoint{3.932882in}{1.983218in}}{\pgfqpoint{3.927058in}{1.989042in}}%
\pgfpathcurveto{\pgfqpoint{3.921234in}{1.994866in}}{\pgfqpoint{3.913334in}{1.998138in}}{\pgfqpoint{3.905098in}{1.998138in}}%
\pgfpathcurveto{\pgfqpoint{3.896862in}{1.998138in}}{\pgfqpoint{3.888962in}{1.994866in}}{\pgfqpoint{3.883138in}{1.989042in}}%
\pgfpathcurveto{\pgfqpoint{3.877314in}{1.983218in}}{\pgfqpoint{3.874042in}{1.975318in}}{\pgfqpoint{3.874042in}{1.967082in}}%
\pgfpathcurveto{\pgfqpoint{3.874042in}{1.958846in}}{\pgfqpoint{3.877314in}{1.950946in}}{\pgfqpoint{3.883138in}{1.945122in}}%
\pgfpathcurveto{\pgfqpoint{3.888962in}{1.939298in}}{\pgfqpoint{3.896862in}{1.936025in}}{\pgfqpoint{3.905098in}{1.936025in}}%
\pgfpathclose%
\pgfusepath{stroke,fill}%
\end{pgfscope}%
\begin{pgfscope}%
\pgfpathrectangle{\pgfqpoint{3.793912in}{0.557870in}}{\pgfqpoint{2.446088in}{1.484734in}}%
\pgfusepath{clip}%
\pgfsetbuttcap%
\pgfsetroundjoin%
\definecolor{currentfill}{rgb}{0.298039,0.447059,0.690196}%
\pgfsetfillcolor{currentfill}%
\pgfsetlinewidth{1.003750pt}%
\definecolor{currentstroke}{rgb}{0.298039,0.447059,0.690196}%
\pgfsetstrokecolor{currentstroke}%
\pgfsetdash{}{0pt}%
\pgfpathmoveto{\pgfqpoint{3.905098in}{1.124564in}}%
\pgfpathcurveto{\pgfqpoint{3.913334in}{1.124564in}}{\pgfqpoint{3.921234in}{1.127836in}}{\pgfqpoint{3.927058in}{1.133660in}}%
\pgfpathcurveto{\pgfqpoint{3.932882in}{1.139484in}}{\pgfqpoint{3.936155in}{1.147384in}}{\pgfqpoint{3.936155in}{1.155620in}}%
\pgfpathcurveto{\pgfqpoint{3.936155in}{1.163857in}}{\pgfqpoint{3.932882in}{1.171757in}}{\pgfqpoint{3.927058in}{1.177581in}}%
\pgfpathcurveto{\pgfqpoint{3.921234in}{1.183404in}}{\pgfqpoint{3.913334in}{1.186677in}}{\pgfqpoint{3.905098in}{1.186677in}}%
\pgfpathcurveto{\pgfqpoint{3.896862in}{1.186677in}}{\pgfqpoint{3.888962in}{1.183404in}}{\pgfqpoint{3.883138in}{1.177581in}}%
\pgfpathcurveto{\pgfqpoint{3.877314in}{1.171757in}}{\pgfqpoint{3.874042in}{1.163857in}}{\pgfqpoint{3.874042in}{1.155620in}}%
\pgfpathcurveto{\pgfqpoint{3.874042in}{1.147384in}}{\pgfqpoint{3.877314in}{1.139484in}}{\pgfqpoint{3.883138in}{1.133660in}}%
\pgfpathcurveto{\pgfqpoint{3.888962in}{1.127836in}}{\pgfqpoint{3.896862in}{1.124564in}}{\pgfqpoint{3.905098in}{1.124564in}}%
\pgfpathclose%
\pgfusepath{stroke,fill}%
\end{pgfscope}%
\begin{pgfscope}%
\pgfpathrectangle{\pgfqpoint{3.793912in}{0.557870in}}{\pgfqpoint{2.446088in}{1.484734in}}%
\pgfusepath{clip}%
\pgfsetbuttcap%
\pgfsetroundjoin%
\definecolor{currentfill}{rgb}{0.298039,0.447059,0.690196}%
\pgfsetfillcolor{currentfill}%
\pgfsetlinewidth{1.003750pt}%
\definecolor{currentstroke}{rgb}{0.298039,0.447059,0.690196}%
\pgfsetstrokecolor{currentstroke}%
\pgfsetdash{}{0pt}%
\pgfpathmoveto{\pgfqpoint{3.905098in}{1.526277in}}%
\pgfpathcurveto{\pgfqpoint{3.913334in}{1.526277in}}{\pgfqpoint{3.921234in}{1.529550in}}{\pgfqpoint{3.927058in}{1.535374in}}%
\pgfpathcurveto{\pgfqpoint{3.932882in}{1.541198in}}{\pgfqpoint{3.936155in}{1.549098in}}{\pgfqpoint{3.936155in}{1.557334in}}%
\pgfpathcurveto{\pgfqpoint{3.936155in}{1.565570in}}{\pgfqpoint{3.932882in}{1.573470in}}{\pgfqpoint{3.927058in}{1.579294in}}%
\pgfpathcurveto{\pgfqpoint{3.921234in}{1.585118in}}{\pgfqpoint{3.913334in}{1.588390in}}{\pgfqpoint{3.905098in}{1.588390in}}%
\pgfpathcurveto{\pgfqpoint{3.896862in}{1.588390in}}{\pgfqpoint{3.888962in}{1.585118in}}{\pgfqpoint{3.883138in}{1.579294in}}%
\pgfpathcurveto{\pgfqpoint{3.877314in}{1.573470in}}{\pgfqpoint{3.874042in}{1.565570in}}{\pgfqpoint{3.874042in}{1.557334in}}%
\pgfpathcurveto{\pgfqpoint{3.874042in}{1.549098in}}{\pgfqpoint{3.877314in}{1.541198in}}{\pgfqpoint{3.883138in}{1.535374in}}%
\pgfpathcurveto{\pgfqpoint{3.888962in}{1.529550in}}{\pgfqpoint{3.896862in}{1.526277in}}{\pgfqpoint{3.905098in}{1.526277in}}%
\pgfpathclose%
\pgfusepath{stroke,fill}%
\end{pgfscope}%
\begin{pgfscope}%
\pgfpathrectangle{\pgfqpoint{3.793912in}{0.557870in}}{\pgfqpoint{2.446088in}{1.484734in}}%
\pgfusepath{clip}%
\pgfsetbuttcap%
\pgfsetroundjoin%
\definecolor{currentfill}{rgb}{0.298039,0.447059,0.690196}%
\pgfsetfillcolor{currentfill}%
\pgfsetlinewidth{1.003750pt}%
\definecolor{currentstroke}{rgb}{0.298039,0.447059,0.690196}%
\pgfsetstrokecolor{currentstroke}%
\pgfsetdash{}{0pt}%
\pgfpathmoveto{\pgfqpoint{3.905098in}{1.044221in}}%
\pgfpathcurveto{\pgfqpoint{3.913334in}{1.044221in}}{\pgfqpoint{3.921234in}{1.047493in}}{\pgfqpoint{3.927058in}{1.053317in}}%
\pgfpathcurveto{\pgfqpoint{3.932882in}{1.059141in}}{\pgfqpoint{3.936155in}{1.067041in}}{\pgfqpoint{3.936155in}{1.075278in}}%
\pgfpathcurveto{\pgfqpoint{3.936155in}{1.083514in}}{\pgfqpoint{3.932882in}{1.091414in}}{\pgfqpoint{3.927058in}{1.097238in}}%
\pgfpathcurveto{\pgfqpoint{3.921234in}{1.103062in}}{\pgfqpoint{3.913334in}{1.106334in}}{\pgfqpoint{3.905098in}{1.106334in}}%
\pgfpathcurveto{\pgfqpoint{3.896862in}{1.106334in}}{\pgfqpoint{3.888962in}{1.103062in}}{\pgfqpoint{3.883138in}{1.097238in}}%
\pgfpathcurveto{\pgfqpoint{3.877314in}{1.091414in}}{\pgfqpoint{3.874042in}{1.083514in}}{\pgfqpoint{3.874042in}{1.075278in}}%
\pgfpathcurveto{\pgfqpoint{3.874042in}{1.067041in}}{\pgfqpoint{3.877314in}{1.059141in}}{\pgfqpoint{3.883138in}{1.053317in}}%
\pgfpathcurveto{\pgfqpoint{3.888962in}{1.047493in}}{\pgfqpoint{3.896862in}{1.044221in}}{\pgfqpoint{3.905098in}{1.044221in}}%
\pgfpathclose%
\pgfusepath{stroke,fill}%
\end{pgfscope}%
\begin{pgfscope}%
\pgfpathrectangle{\pgfqpoint{3.793912in}{0.557870in}}{\pgfqpoint{2.446088in}{1.484734in}}%
\pgfusepath{clip}%
\pgfsetbuttcap%
\pgfsetroundjoin%
\definecolor{currentfill}{rgb}{0.298039,0.447059,0.690196}%
\pgfsetfillcolor{currentfill}%
\pgfsetlinewidth{1.003750pt}%
\definecolor{currentstroke}{rgb}{0.298039,0.447059,0.690196}%
\pgfsetstrokecolor{currentstroke}%
\pgfsetdash{}{0pt}%
\pgfpathmoveto{\pgfqpoint{3.905098in}{1.550380in}}%
\pgfpathcurveto{\pgfqpoint{3.913334in}{1.550380in}}{\pgfqpoint{3.921234in}{1.553653in}}{\pgfqpoint{3.927058in}{1.559477in}}%
\pgfpathcurveto{\pgfqpoint{3.932882in}{1.565300in}}{\pgfqpoint{3.936155in}{1.573201in}}{\pgfqpoint{3.936155in}{1.581437in}}%
\pgfpathcurveto{\pgfqpoint{3.936155in}{1.589673in}}{\pgfqpoint{3.932882in}{1.597573in}}{\pgfqpoint{3.927058in}{1.603397in}}%
\pgfpathcurveto{\pgfqpoint{3.921234in}{1.609221in}}{\pgfqpoint{3.913334in}{1.612493in}}{\pgfqpoint{3.905098in}{1.612493in}}%
\pgfpathcurveto{\pgfqpoint{3.896862in}{1.612493in}}{\pgfqpoint{3.888962in}{1.609221in}}{\pgfqpoint{3.883138in}{1.603397in}}%
\pgfpathcurveto{\pgfqpoint{3.877314in}{1.597573in}}{\pgfqpoint{3.874042in}{1.589673in}}{\pgfqpoint{3.874042in}{1.581437in}}%
\pgfpathcurveto{\pgfqpoint{3.874042in}{1.573201in}}{\pgfqpoint{3.877314in}{1.565300in}}{\pgfqpoint{3.883138in}{1.559477in}}%
\pgfpathcurveto{\pgfqpoint{3.888962in}{1.553653in}}{\pgfqpoint{3.896862in}{1.550380in}}{\pgfqpoint{3.905098in}{1.550380in}}%
\pgfpathclose%
\pgfusepath{stroke,fill}%
\end{pgfscope}%
\begin{pgfscope}%
\pgfpathrectangle{\pgfqpoint{3.793912in}{0.557870in}}{\pgfqpoint{2.446088in}{1.484734in}}%
\pgfusepath{clip}%
\pgfsetbuttcap%
\pgfsetroundjoin%
\definecolor{currentfill}{rgb}{0.298039,0.447059,0.690196}%
\pgfsetfillcolor{currentfill}%
\pgfsetlinewidth{1.003750pt}%
\definecolor{currentstroke}{rgb}{0.298039,0.447059,0.690196}%
\pgfsetstrokecolor{currentstroke}%
\pgfsetdash{}{0pt}%
\pgfpathmoveto{\pgfqpoint{3.905098in}{1.220975in}}%
\pgfpathcurveto{\pgfqpoint{3.913334in}{1.220975in}}{\pgfqpoint{3.921234in}{1.224247in}}{\pgfqpoint{3.927058in}{1.230071in}}%
\pgfpathcurveto{\pgfqpoint{3.932882in}{1.235895in}}{\pgfqpoint{3.936155in}{1.243795in}}{\pgfqpoint{3.936155in}{1.252032in}}%
\pgfpathcurveto{\pgfqpoint{3.936155in}{1.260268in}}{\pgfqpoint{3.932882in}{1.268168in}}{\pgfqpoint{3.927058in}{1.273992in}}%
\pgfpathcurveto{\pgfqpoint{3.921234in}{1.279816in}}{\pgfqpoint{3.913334in}{1.283088in}}{\pgfqpoint{3.905098in}{1.283088in}}%
\pgfpathcurveto{\pgfqpoint{3.896862in}{1.283088in}}{\pgfqpoint{3.888962in}{1.279816in}}{\pgfqpoint{3.883138in}{1.273992in}}%
\pgfpathcurveto{\pgfqpoint{3.877314in}{1.268168in}}{\pgfqpoint{3.874042in}{1.260268in}}{\pgfqpoint{3.874042in}{1.252032in}}%
\pgfpathcurveto{\pgfqpoint{3.874042in}{1.243795in}}{\pgfqpoint{3.877314in}{1.235895in}}{\pgfqpoint{3.883138in}{1.230071in}}%
\pgfpathcurveto{\pgfqpoint{3.888962in}{1.224247in}}{\pgfqpoint{3.896862in}{1.220975in}}{\pgfqpoint{3.905098in}{1.220975in}}%
\pgfpathclose%
\pgfusepath{stroke,fill}%
\end{pgfscope}%
\begin{pgfscope}%
\pgfpathrectangle{\pgfqpoint{3.793912in}{0.557870in}}{\pgfqpoint{2.446088in}{1.484734in}}%
\pgfusepath{clip}%
\pgfsetbuttcap%
\pgfsetroundjoin%
\definecolor{currentfill}{rgb}{0.298039,0.447059,0.690196}%
\pgfsetfillcolor{currentfill}%
\pgfsetlinewidth{1.003750pt}%
\definecolor{currentstroke}{rgb}{0.298039,0.447059,0.690196}%
\pgfsetstrokecolor{currentstroke}%
\pgfsetdash{}{0pt}%
\pgfpathmoveto{\pgfqpoint{3.905098in}{1.220975in}}%
\pgfpathcurveto{\pgfqpoint{3.913334in}{1.220975in}}{\pgfqpoint{3.921234in}{1.224247in}}{\pgfqpoint{3.927058in}{1.230071in}}%
\pgfpathcurveto{\pgfqpoint{3.932882in}{1.235895in}}{\pgfqpoint{3.936155in}{1.243795in}}{\pgfqpoint{3.936155in}{1.252032in}}%
\pgfpathcurveto{\pgfqpoint{3.936155in}{1.260268in}}{\pgfqpoint{3.932882in}{1.268168in}}{\pgfqpoint{3.927058in}{1.273992in}}%
\pgfpathcurveto{\pgfqpoint{3.921234in}{1.279816in}}{\pgfqpoint{3.913334in}{1.283088in}}{\pgfqpoint{3.905098in}{1.283088in}}%
\pgfpathcurveto{\pgfqpoint{3.896862in}{1.283088in}}{\pgfqpoint{3.888962in}{1.279816in}}{\pgfqpoint{3.883138in}{1.273992in}}%
\pgfpathcurveto{\pgfqpoint{3.877314in}{1.268168in}}{\pgfqpoint{3.874042in}{1.260268in}}{\pgfqpoint{3.874042in}{1.252032in}}%
\pgfpathcurveto{\pgfqpoint{3.874042in}{1.243795in}}{\pgfqpoint{3.877314in}{1.235895in}}{\pgfqpoint{3.883138in}{1.230071in}}%
\pgfpathcurveto{\pgfqpoint{3.888962in}{1.224247in}}{\pgfqpoint{3.896862in}{1.220975in}}{\pgfqpoint{3.905098in}{1.220975in}}%
\pgfpathclose%
\pgfusepath{stroke,fill}%
\end{pgfscope}%
\begin{pgfscope}%
\pgfpathrectangle{\pgfqpoint{3.793912in}{0.557870in}}{\pgfqpoint{2.446088in}{1.484734in}}%
\pgfusepath{clip}%
\pgfsetbuttcap%
\pgfsetroundjoin%
\definecolor{currentfill}{rgb}{0.298039,0.447059,0.690196}%
\pgfsetfillcolor{currentfill}%
\pgfsetlinewidth{1.003750pt}%
\definecolor{currentstroke}{rgb}{0.298039,0.447059,0.690196}%
\pgfsetstrokecolor{currentstroke}%
\pgfsetdash{}{0pt}%
\pgfpathmoveto{\pgfqpoint{3.905098in}{1.936025in}}%
\pgfpathcurveto{\pgfqpoint{3.913334in}{1.936025in}}{\pgfqpoint{3.921234in}{1.939298in}}{\pgfqpoint{3.927058in}{1.945122in}}%
\pgfpathcurveto{\pgfqpoint{3.932882in}{1.950946in}}{\pgfqpoint{3.936155in}{1.958846in}}{\pgfqpoint{3.936155in}{1.967082in}}%
\pgfpathcurveto{\pgfqpoint{3.936155in}{1.975318in}}{\pgfqpoint{3.932882in}{1.983218in}}{\pgfqpoint{3.927058in}{1.989042in}}%
\pgfpathcurveto{\pgfqpoint{3.921234in}{1.994866in}}{\pgfqpoint{3.913334in}{1.998138in}}{\pgfqpoint{3.905098in}{1.998138in}}%
\pgfpathcurveto{\pgfqpoint{3.896862in}{1.998138in}}{\pgfqpoint{3.888962in}{1.994866in}}{\pgfqpoint{3.883138in}{1.989042in}}%
\pgfpathcurveto{\pgfqpoint{3.877314in}{1.983218in}}{\pgfqpoint{3.874042in}{1.975318in}}{\pgfqpoint{3.874042in}{1.967082in}}%
\pgfpathcurveto{\pgfqpoint{3.874042in}{1.958846in}}{\pgfqpoint{3.877314in}{1.950946in}}{\pgfqpoint{3.883138in}{1.945122in}}%
\pgfpathcurveto{\pgfqpoint{3.888962in}{1.939298in}}{\pgfqpoint{3.896862in}{1.936025in}}{\pgfqpoint{3.905098in}{1.936025in}}%
\pgfpathclose%
\pgfusepath{stroke,fill}%
\end{pgfscope}%
\begin{pgfscope}%
\pgfpathrectangle{\pgfqpoint{3.793912in}{0.557870in}}{\pgfqpoint{2.446088in}{1.484734in}}%
\pgfusepath{clip}%
\pgfsetbuttcap%
\pgfsetroundjoin%
\definecolor{currentfill}{rgb}{0.298039,0.447059,0.690196}%
\pgfsetfillcolor{currentfill}%
\pgfsetlinewidth{1.003750pt}%
\definecolor{currentstroke}{rgb}{0.298039,0.447059,0.690196}%
\pgfsetstrokecolor{currentstroke}%
\pgfsetdash{}{0pt}%
\pgfpathmoveto{\pgfqpoint{3.905098in}{1.076358in}}%
\pgfpathcurveto{\pgfqpoint{3.913334in}{1.076358in}}{\pgfqpoint{3.921234in}{1.079630in}}{\pgfqpoint{3.927058in}{1.085454in}}%
\pgfpathcurveto{\pgfqpoint{3.932882in}{1.091278in}}{\pgfqpoint{3.936155in}{1.099178in}}{\pgfqpoint{3.936155in}{1.107415in}}%
\pgfpathcurveto{\pgfqpoint{3.936155in}{1.115651in}}{\pgfqpoint{3.932882in}{1.123551in}}{\pgfqpoint{3.927058in}{1.129375in}}%
\pgfpathcurveto{\pgfqpoint{3.921234in}{1.135199in}}{\pgfqpoint{3.913334in}{1.138471in}}{\pgfqpoint{3.905098in}{1.138471in}}%
\pgfpathcurveto{\pgfqpoint{3.896862in}{1.138471in}}{\pgfqpoint{3.888962in}{1.135199in}}{\pgfqpoint{3.883138in}{1.129375in}}%
\pgfpathcurveto{\pgfqpoint{3.877314in}{1.123551in}}{\pgfqpoint{3.874042in}{1.115651in}}{\pgfqpoint{3.874042in}{1.107415in}}%
\pgfpathcurveto{\pgfqpoint{3.874042in}{1.099178in}}{\pgfqpoint{3.877314in}{1.091278in}}{\pgfqpoint{3.883138in}{1.085454in}}%
\pgfpathcurveto{\pgfqpoint{3.888962in}{1.079630in}}{\pgfqpoint{3.896862in}{1.076358in}}{\pgfqpoint{3.905098in}{1.076358in}}%
\pgfpathclose%
\pgfusepath{stroke,fill}%
\end{pgfscope}%
\begin{pgfscope}%
\pgfpathrectangle{\pgfqpoint{3.793912in}{0.557870in}}{\pgfqpoint{2.446088in}{1.484734in}}%
\pgfusepath{clip}%
\pgfsetbuttcap%
\pgfsetroundjoin%
\definecolor{currentfill}{rgb}{0.298039,0.447059,0.690196}%
\pgfsetfillcolor{currentfill}%
\pgfsetlinewidth{1.003750pt}%
\definecolor{currentstroke}{rgb}{0.298039,0.447059,0.690196}%
\pgfsetstrokecolor{currentstroke}%
\pgfsetdash{}{0pt}%
\pgfpathmoveto{\pgfqpoint{3.905098in}{1.285249in}}%
\pgfpathcurveto{\pgfqpoint{3.913334in}{1.285249in}}{\pgfqpoint{3.921234in}{1.288522in}}{\pgfqpoint{3.927058in}{1.294346in}}%
\pgfpathcurveto{\pgfqpoint{3.932882in}{1.300169in}}{\pgfqpoint{3.936155in}{1.308069in}}{\pgfqpoint{3.936155in}{1.316306in}}%
\pgfpathcurveto{\pgfqpoint{3.936155in}{1.324542in}}{\pgfqpoint{3.932882in}{1.332442in}}{\pgfqpoint{3.927058in}{1.338266in}}%
\pgfpathcurveto{\pgfqpoint{3.921234in}{1.344090in}}{\pgfqpoint{3.913334in}{1.347362in}}{\pgfqpoint{3.905098in}{1.347362in}}%
\pgfpathcurveto{\pgfqpoint{3.896862in}{1.347362in}}{\pgfqpoint{3.888962in}{1.344090in}}{\pgfqpoint{3.883138in}{1.338266in}}%
\pgfpathcurveto{\pgfqpoint{3.877314in}{1.332442in}}{\pgfqpoint{3.874042in}{1.324542in}}{\pgfqpoint{3.874042in}{1.316306in}}%
\pgfpathcurveto{\pgfqpoint{3.874042in}{1.308069in}}{\pgfqpoint{3.877314in}{1.300169in}}{\pgfqpoint{3.883138in}{1.294346in}}%
\pgfpathcurveto{\pgfqpoint{3.888962in}{1.288522in}}{\pgfqpoint{3.896862in}{1.285249in}}{\pgfqpoint{3.905098in}{1.285249in}}%
\pgfpathclose%
\pgfusepath{stroke,fill}%
\end{pgfscope}%
\begin{pgfscope}%
\pgfpathrectangle{\pgfqpoint{3.793912in}{0.557870in}}{\pgfqpoint{2.446088in}{1.484734in}}%
\pgfusepath{clip}%
\pgfsetbuttcap%
\pgfsetroundjoin%
\definecolor{currentfill}{rgb}{0.298039,0.447059,0.690196}%
\pgfsetfillcolor{currentfill}%
\pgfsetlinewidth{1.003750pt}%
\definecolor{currentstroke}{rgb}{0.298039,0.447059,0.690196}%
\pgfsetstrokecolor{currentstroke}%
\pgfsetdash{}{0pt}%
\pgfpathmoveto{\pgfqpoint{3.905098in}{1.309352in}}%
\pgfpathcurveto{\pgfqpoint{3.913334in}{1.309352in}}{\pgfqpoint{3.921234in}{1.312624in}}{\pgfqpoint{3.927058in}{1.318448in}}%
\pgfpathcurveto{\pgfqpoint{3.932882in}{1.324272in}}{\pgfqpoint{3.936155in}{1.332172in}}{\pgfqpoint{3.936155in}{1.340409in}}%
\pgfpathcurveto{\pgfqpoint{3.936155in}{1.348645in}}{\pgfqpoint{3.932882in}{1.356545in}}{\pgfqpoint{3.927058in}{1.362369in}}%
\pgfpathcurveto{\pgfqpoint{3.921234in}{1.368193in}}{\pgfqpoint{3.913334in}{1.371465in}}{\pgfqpoint{3.905098in}{1.371465in}}%
\pgfpathcurveto{\pgfqpoint{3.896862in}{1.371465in}}{\pgfqpoint{3.888962in}{1.368193in}}{\pgfqpoint{3.883138in}{1.362369in}}%
\pgfpathcurveto{\pgfqpoint{3.877314in}{1.356545in}}{\pgfqpoint{3.874042in}{1.348645in}}{\pgfqpoint{3.874042in}{1.340409in}}%
\pgfpathcurveto{\pgfqpoint{3.874042in}{1.332172in}}{\pgfqpoint{3.877314in}{1.324272in}}{\pgfqpoint{3.883138in}{1.318448in}}%
\pgfpathcurveto{\pgfqpoint{3.888962in}{1.312624in}}{\pgfqpoint{3.896862in}{1.309352in}}{\pgfqpoint{3.905098in}{1.309352in}}%
\pgfpathclose%
\pgfusepath{stroke,fill}%
\end{pgfscope}%
\begin{pgfscope}%
\pgfpathrectangle{\pgfqpoint{3.793912in}{0.557870in}}{\pgfqpoint{2.446088in}{1.484734in}}%
\pgfusepath{clip}%
\pgfsetbuttcap%
\pgfsetroundjoin%
\definecolor{currentfill}{rgb}{0.298039,0.447059,0.690196}%
\pgfsetfillcolor{currentfill}%
\pgfsetlinewidth{1.003750pt}%
\definecolor{currentstroke}{rgb}{0.298039,0.447059,0.690196}%
\pgfsetstrokecolor{currentstroke}%
\pgfsetdash{}{0pt}%
\pgfpathmoveto{\pgfqpoint{3.905098in}{1.317386in}}%
\pgfpathcurveto{\pgfqpoint{3.913334in}{1.317386in}}{\pgfqpoint{3.921234in}{1.320659in}}{\pgfqpoint{3.927058in}{1.326483in}}%
\pgfpathcurveto{\pgfqpoint{3.932882in}{1.332307in}}{\pgfqpoint{3.936155in}{1.340207in}}{\pgfqpoint{3.936155in}{1.348443in}}%
\pgfpathcurveto{\pgfqpoint{3.936155in}{1.356679in}}{\pgfqpoint{3.932882in}{1.364579in}}{\pgfqpoint{3.927058in}{1.370403in}}%
\pgfpathcurveto{\pgfqpoint{3.921234in}{1.376227in}}{\pgfqpoint{3.913334in}{1.379499in}}{\pgfqpoint{3.905098in}{1.379499in}}%
\pgfpathcurveto{\pgfqpoint{3.896862in}{1.379499in}}{\pgfqpoint{3.888962in}{1.376227in}}{\pgfqpoint{3.883138in}{1.370403in}}%
\pgfpathcurveto{\pgfqpoint{3.877314in}{1.364579in}}{\pgfqpoint{3.874042in}{1.356679in}}{\pgfqpoint{3.874042in}{1.348443in}}%
\pgfpathcurveto{\pgfqpoint{3.874042in}{1.340207in}}{\pgfqpoint{3.877314in}{1.332307in}}{\pgfqpoint{3.883138in}{1.326483in}}%
\pgfpathcurveto{\pgfqpoint{3.888962in}{1.320659in}}{\pgfqpoint{3.896862in}{1.317386in}}{\pgfqpoint{3.905098in}{1.317386in}}%
\pgfpathclose%
\pgfusepath{stroke,fill}%
\end{pgfscope}%
\begin{pgfscope}%
\pgfpathrectangle{\pgfqpoint{3.793912in}{0.557870in}}{\pgfqpoint{2.446088in}{1.484734in}}%
\pgfusepath{clip}%
\pgfsetbuttcap%
\pgfsetroundjoin%
\definecolor{currentfill}{rgb}{0.298039,0.447059,0.690196}%
\pgfsetfillcolor{currentfill}%
\pgfsetlinewidth{1.003750pt}%
\definecolor{currentstroke}{rgb}{0.298039,0.447059,0.690196}%
\pgfsetstrokecolor{currentstroke}%
\pgfsetdash{}{0pt}%
\pgfpathmoveto{\pgfqpoint{3.905098in}{1.229009in}}%
\pgfpathcurveto{\pgfqpoint{3.913334in}{1.229009in}}{\pgfqpoint{3.921234in}{1.232282in}}{\pgfqpoint{3.927058in}{1.238106in}}%
\pgfpathcurveto{\pgfqpoint{3.932882in}{1.243930in}}{\pgfqpoint{3.936155in}{1.251830in}}{\pgfqpoint{3.936155in}{1.260066in}}%
\pgfpathcurveto{\pgfqpoint{3.936155in}{1.268302in}}{\pgfqpoint{3.932882in}{1.276202in}}{\pgfqpoint{3.927058in}{1.282026in}}%
\pgfpathcurveto{\pgfqpoint{3.921234in}{1.287850in}}{\pgfqpoint{3.913334in}{1.291122in}}{\pgfqpoint{3.905098in}{1.291122in}}%
\pgfpathcurveto{\pgfqpoint{3.896862in}{1.291122in}}{\pgfqpoint{3.888962in}{1.287850in}}{\pgfqpoint{3.883138in}{1.282026in}}%
\pgfpathcurveto{\pgfqpoint{3.877314in}{1.276202in}}{\pgfqpoint{3.874042in}{1.268302in}}{\pgfqpoint{3.874042in}{1.260066in}}%
\pgfpathcurveto{\pgfqpoint{3.874042in}{1.251830in}}{\pgfqpoint{3.877314in}{1.243930in}}{\pgfqpoint{3.883138in}{1.238106in}}%
\pgfpathcurveto{\pgfqpoint{3.888962in}{1.232282in}}{\pgfqpoint{3.896862in}{1.229009in}}{\pgfqpoint{3.905098in}{1.229009in}}%
\pgfpathclose%
\pgfusepath{stroke,fill}%
\end{pgfscope}%
\begin{pgfscope}%
\pgfpathrectangle{\pgfqpoint{3.793912in}{0.557870in}}{\pgfqpoint{2.446088in}{1.484734in}}%
\pgfusepath{clip}%
\pgfsetbuttcap%
\pgfsetroundjoin%
\definecolor{currentfill}{rgb}{0.298039,0.447059,0.690196}%
\pgfsetfillcolor{currentfill}%
\pgfsetlinewidth{1.003750pt}%
\definecolor{currentstroke}{rgb}{0.298039,0.447059,0.690196}%
\pgfsetstrokecolor{currentstroke}%
\pgfsetdash{}{0pt}%
\pgfpathmoveto{\pgfqpoint{3.905098in}{0.955844in}}%
\pgfpathcurveto{\pgfqpoint{3.913334in}{0.955844in}}{\pgfqpoint{3.921234in}{0.959116in}}{\pgfqpoint{3.927058in}{0.964940in}}%
\pgfpathcurveto{\pgfqpoint{3.932882in}{0.970764in}}{\pgfqpoint{3.936155in}{0.978664in}}{\pgfqpoint{3.936155in}{0.986901in}}%
\pgfpathcurveto{\pgfqpoint{3.936155in}{0.995137in}}{\pgfqpoint{3.932882in}{1.003037in}}{\pgfqpoint{3.927058in}{1.008861in}}%
\pgfpathcurveto{\pgfqpoint{3.921234in}{1.014685in}}{\pgfqpoint{3.913334in}{1.017957in}}{\pgfqpoint{3.905098in}{1.017957in}}%
\pgfpathcurveto{\pgfqpoint{3.896862in}{1.017957in}}{\pgfqpoint{3.888962in}{1.014685in}}{\pgfqpoint{3.883138in}{1.008861in}}%
\pgfpathcurveto{\pgfqpoint{3.877314in}{1.003037in}}{\pgfqpoint{3.874042in}{0.995137in}}{\pgfqpoint{3.874042in}{0.986901in}}%
\pgfpathcurveto{\pgfqpoint{3.874042in}{0.978664in}}{\pgfqpoint{3.877314in}{0.970764in}}{\pgfqpoint{3.883138in}{0.964940in}}%
\pgfpathcurveto{\pgfqpoint{3.888962in}{0.959116in}}{\pgfqpoint{3.896862in}{0.955844in}}{\pgfqpoint{3.905098in}{0.955844in}}%
\pgfpathclose%
\pgfusepath{stroke,fill}%
\end{pgfscope}%
\begin{pgfscope}%
\pgfpathrectangle{\pgfqpoint{3.793912in}{0.557870in}}{\pgfqpoint{2.446088in}{1.484734in}}%
\pgfusepath{clip}%
\pgfsetbuttcap%
\pgfsetroundjoin%
\definecolor{currentfill}{rgb}{0.298039,0.447059,0.690196}%
\pgfsetfillcolor{currentfill}%
\pgfsetlinewidth{1.003750pt}%
\definecolor{currentstroke}{rgb}{0.298039,0.447059,0.690196}%
\pgfsetstrokecolor{currentstroke}%
\pgfsetdash{}{0pt}%
\pgfpathmoveto{\pgfqpoint{3.905098in}{1.936025in}}%
\pgfpathcurveto{\pgfqpoint{3.913334in}{1.936025in}}{\pgfqpoint{3.921234in}{1.939298in}}{\pgfqpoint{3.927058in}{1.945122in}}%
\pgfpathcurveto{\pgfqpoint{3.932882in}{1.950946in}}{\pgfqpoint{3.936155in}{1.958846in}}{\pgfqpoint{3.936155in}{1.967082in}}%
\pgfpathcurveto{\pgfqpoint{3.936155in}{1.975318in}}{\pgfqpoint{3.932882in}{1.983218in}}{\pgfqpoint{3.927058in}{1.989042in}}%
\pgfpathcurveto{\pgfqpoint{3.921234in}{1.994866in}}{\pgfqpoint{3.913334in}{1.998138in}}{\pgfqpoint{3.905098in}{1.998138in}}%
\pgfpathcurveto{\pgfqpoint{3.896862in}{1.998138in}}{\pgfqpoint{3.888962in}{1.994866in}}{\pgfqpoint{3.883138in}{1.989042in}}%
\pgfpathcurveto{\pgfqpoint{3.877314in}{1.983218in}}{\pgfqpoint{3.874042in}{1.975318in}}{\pgfqpoint{3.874042in}{1.967082in}}%
\pgfpathcurveto{\pgfqpoint{3.874042in}{1.958846in}}{\pgfqpoint{3.877314in}{1.950946in}}{\pgfqpoint{3.883138in}{1.945122in}}%
\pgfpathcurveto{\pgfqpoint{3.888962in}{1.939298in}}{\pgfqpoint{3.896862in}{1.936025in}}{\pgfqpoint{3.905098in}{1.936025in}}%
\pgfpathclose%
\pgfusepath{stroke,fill}%
\end{pgfscope}%
\begin{pgfscope}%
\pgfpathrectangle{\pgfqpoint{3.793912in}{0.557870in}}{\pgfqpoint{2.446088in}{1.484734in}}%
\pgfusepath{clip}%
\pgfsetbuttcap%
\pgfsetroundjoin%
\definecolor{currentfill}{rgb}{0.298039,0.447059,0.690196}%
\pgfsetfillcolor{currentfill}%
\pgfsetlinewidth{1.003750pt}%
\definecolor{currentstroke}{rgb}{0.298039,0.447059,0.690196}%
\pgfsetstrokecolor{currentstroke}%
\pgfsetdash{}{0pt}%
\pgfpathmoveto{\pgfqpoint{3.905098in}{0.843364in}}%
\pgfpathcurveto{\pgfqpoint{3.913334in}{0.843364in}}{\pgfqpoint{3.921234in}{0.846636in}}{\pgfqpoint{3.927058in}{0.852460in}}%
\pgfpathcurveto{\pgfqpoint{3.932882in}{0.858284in}}{\pgfqpoint{3.936155in}{0.866184in}}{\pgfqpoint{3.936155in}{0.874421in}}%
\pgfpathcurveto{\pgfqpoint{3.936155in}{0.882657in}}{\pgfqpoint{3.932882in}{0.890557in}}{\pgfqpoint{3.927058in}{0.896381in}}%
\pgfpathcurveto{\pgfqpoint{3.921234in}{0.902205in}}{\pgfqpoint{3.913334in}{0.905477in}}{\pgfqpoint{3.905098in}{0.905477in}}%
\pgfpathcurveto{\pgfqpoint{3.896862in}{0.905477in}}{\pgfqpoint{3.888962in}{0.902205in}}{\pgfqpoint{3.883138in}{0.896381in}}%
\pgfpathcurveto{\pgfqpoint{3.877314in}{0.890557in}}{\pgfqpoint{3.874042in}{0.882657in}}{\pgfqpoint{3.874042in}{0.874421in}}%
\pgfpathcurveto{\pgfqpoint{3.874042in}{0.866184in}}{\pgfqpoint{3.877314in}{0.858284in}}{\pgfqpoint{3.883138in}{0.852460in}}%
\pgfpathcurveto{\pgfqpoint{3.888962in}{0.846636in}}{\pgfqpoint{3.896862in}{0.843364in}}{\pgfqpoint{3.905098in}{0.843364in}}%
\pgfpathclose%
\pgfusepath{stroke,fill}%
\end{pgfscope}%
\begin{pgfscope}%
\pgfpathrectangle{\pgfqpoint{3.793912in}{0.557870in}}{\pgfqpoint{2.446088in}{1.484734in}}%
\pgfusepath{clip}%
\pgfsetbuttcap%
\pgfsetroundjoin%
\definecolor{currentfill}{rgb}{0.298039,0.447059,0.690196}%
\pgfsetfillcolor{currentfill}%
\pgfsetlinewidth{1.003750pt}%
\definecolor{currentstroke}{rgb}{0.298039,0.447059,0.690196}%
\pgfsetstrokecolor{currentstroke}%
\pgfsetdash{}{0pt}%
\pgfpathmoveto{\pgfqpoint{3.905098in}{1.936025in}}%
\pgfpathcurveto{\pgfqpoint{3.913334in}{1.936025in}}{\pgfqpoint{3.921234in}{1.939298in}}{\pgfqpoint{3.927058in}{1.945122in}}%
\pgfpathcurveto{\pgfqpoint{3.932882in}{1.950946in}}{\pgfqpoint{3.936155in}{1.958846in}}{\pgfqpoint{3.936155in}{1.967082in}}%
\pgfpathcurveto{\pgfqpoint{3.936155in}{1.975318in}}{\pgfqpoint{3.932882in}{1.983218in}}{\pgfqpoint{3.927058in}{1.989042in}}%
\pgfpathcurveto{\pgfqpoint{3.921234in}{1.994866in}}{\pgfqpoint{3.913334in}{1.998138in}}{\pgfqpoint{3.905098in}{1.998138in}}%
\pgfpathcurveto{\pgfqpoint{3.896862in}{1.998138in}}{\pgfqpoint{3.888962in}{1.994866in}}{\pgfqpoint{3.883138in}{1.989042in}}%
\pgfpathcurveto{\pgfqpoint{3.877314in}{1.983218in}}{\pgfqpoint{3.874042in}{1.975318in}}{\pgfqpoint{3.874042in}{1.967082in}}%
\pgfpathcurveto{\pgfqpoint{3.874042in}{1.958846in}}{\pgfqpoint{3.877314in}{1.950946in}}{\pgfqpoint{3.883138in}{1.945122in}}%
\pgfpathcurveto{\pgfqpoint{3.888962in}{1.939298in}}{\pgfqpoint{3.896862in}{1.936025in}}{\pgfqpoint{3.905098in}{1.936025in}}%
\pgfpathclose%
\pgfusepath{stroke,fill}%
\end{pgfscope}%
\begin{pgfscope}%
\pgfpathrectangle{\pgfqpoint{3.793912in}{0.557870in}}{\pgfqpoint{2.446088in}{1.484734in}}%
\pgfusepath{clip}%
\pgfsetbuttcap%
\pgfsetroundjoin%
\definecolor{currentfill}{rgb}{0.298039,0.447059,0.690196}%
\pgfsetfillcolor{currentfill}%
\pgfsetlinewidth{1.003750pt}%
\definecolor{currentstroke}{rgb}{0.298039,0.447059,0.690196}%
\pgfsetstrokecolor{currentstroke}%
\pgfsetdash{}{0pt}%
\pgfpathmoveto{\pgfqpoint{3.905098in}{0.867467in}}%
\pgfpathcurveto{\pgfqpoint{3.913334in}{0.867467in}}{\pgfqpoint{3.921234in}{0.870739in}}{\pgfqpoint{3.927058in}{0.876563in}}%
\pgfpathcurveto{\pgfqpoint{3.932882in}{0.882387in}}{\pgfqpoint{3.936155in}{0.890287in}}{\pgfqpoint{3.936155in}{0.898524in}}%
\pgfpathcurveto{\pgfqpoint{3.936155in}{0.906760in}}{\pgfqpoint{3.932882in}{0.914660in}}{\pgfqpoint{3.927058in}{0.920484in}}%
\pgfpathcurveto{\pgfqpoint{3.921234in}{0.926308in}}{\pgfqpoint{3.913334in}{0.929580in}}{\pgfqpoint{3.905098in}{0.929580in}}%
\pgfpathcurveto{\pgfqpoint{3.896862in}{0.929580in}}{\pgfqpoint{3.888962in}{0.926308in}}{\pgfqpoint{3.883138in}{0.920484in}}%
\pgfpathcurveto{\pgfqpoint{3.877314in}{0.914660in}}{\pgfqpoint{3.874042in}{0.906760in}}{\pgfqpoint{3.874042in}{0.898524in}}%
\pgfpathcurveto{\pgfqpoint{3.874042in}{0.890287in}}{\pgfqpoint{3.877314in}{0.882387in}}{\pgfqpoint{3.883138in}{0.876563in}}%
\pgfpathcurveto{\pgfqpoint{3.888962in}{0.870739in}}{\pgfqpoint{3.896862in}{0.867467in}}{\pgfqpoint{3.905098in}{0.867467in}}%
\pgfpathclose%
\pgfusepath{stroke,fill}%
\end{pgfscope}%
\begin{pgfscope}%
\pgfpathrectangle{\pgfqpoint{3.793912in}{0.557870in}}{\pgfqpoint{2.446088in}{1.484734in}}%
\pgfusepath{clip}%
\pgfsetbuttcap%
\pgfsetroundjoin%
\definecolor{currentfill}{rgb}{0.298039,0.447059,0.690196}%
\pgfsetfillcolor{currentfill}%
\pgfsetlinewidth{1.003750pt}%
\definecolor{currentstroke}{rgb}{0.298039,0.447059,0.690196}%
\pgfsetstrokecolor{currentstroke}%
\pgfsetdash{}{0pt}%
\pgfpathmoveto{\pgfqpoint{3.905098in}{1.936025in}}%
\pgfpathcurveto{\pgfqpoint{3.913334in}{1.936025in}}{\pgfqpoint{3.921234in}{1.939298in}}{\pgfqpoint{3.927058in}{1.945122in}}%
\pgfpathcurveto{\pgfqpoint{3.932882in}{1.950946in}}{\pgfqpoint{3.936155in}{1.958846in}}{\pgfqpoint{3.936155in}{1.967082in}}%
\pgfpathcurveto{\pgfqpoint{3.936155in}{1.975318in}}{\pgfqpoint{3.932882in}{1.983218in}}{\pgfqpoint{3.927058in}{1.989042in}}%
\pgfpathcurveto{\pgfqpoint{3.921234in}{1.994866in}}{\pgfqpoint{3.913334in}{1.998138in}}{\pgfqpoint{3.905098in}{1.998138in}}%
\pgfpathcurveto{\pgfqpoint{3.896862in}{1.998138in}}{\pgfqpoint{3.888962in}{1.994866in}}{\pgfqpoint{3.883138in}{1.989042in}}%
\pgfpathcurveto{\pgfqpoint{3.877314in}{1.983218in}}{\pgfqpoint{3.874042in}{1.975318in}}{\pgfqpoint{3.874042in}{1.967082in}}%
\pgfpathcurveto{\pgfqpoint{3.874042in}{1.958846in}}{\pgfqpoint{3.877314in}{1.950946in}}{\pgfqpoint{3.883138in}{1.945122in}}%
\pgfpathcurveto{\pgfqpoint{3.888962in}{1.939298in}}{\pgfqpoint{3.896862in}{1.936025in}}{\pgfqpoint{3.905098in}{1.936025in}}%
\pgfpathclose%
\pgfusepath{stroke,fill}%
\end{pgfscope}%
\begin{pgfscope}%
\pgfpathrectangle{\pgfqpoint{3.793912in}{0.557870in}}{\pgfqpoint{2.446088in}{1.484734in}}%
\pgfusepath{clip}%
\pgfsetbuttcap%
\pgfsetroundjoin%
\definecolor{currentfill}{rgb}{0.298039,0.447059,0.690196}%
\pgfsetfillcolor{currentfill}%
\pgfsetlinewidth{1.003750pt}%
\definecolor{currentstroke}{rgb}{0.298039,0.447059,0.690196}%
\pgfsetstrokecolor{currentstroke}%
\pgfsetdash{}{0pt}%
\pgfpathmoveto{\pgfqpoint{3.905098in}{1.936025in}}%
\pgfpathcurveto{\pgfqpoint{3.913334in}{1.936025in}}{\pgfqpoint{3.921234in}{1.939298in}}{\pgfqpoint{3.927058in}{1.945122in}}%
\pgfpathcurveto{\pgfqpoint{3.932882in}{1.950946in}}{\pgfqpoint{3.936155in}{1.958846in}}{\pgfqpoint{3.936155in}{1.967082in}}%
\pgfpathcurveto{\pgfqpoint{3.936155in}{1.975318in}}{\pgfqpoint{3.932882in}{1.983218in}}{\pgfqpoint{3.927058in}{1.989042in}}%
\pgfpathcurveto{\pgfqpoint{3.921234in}{1.994866in}}{\pgfqpoint{3.913334in}{1.998138in}}{\pgfqpoint{3.905098in}{1.998138in}}%
\pgfpathcurveto{\pgfqpoint{3.896862in}{1.998138in}}{\pgfqpoint{3.888962in}{1.994866in}}{\pgfqpoint{3.883138in}{1.989042in}}%
\pgfpathcurveto{\pgfqpoint{3.877314in}{1.983218in}}{\pgfqpoint{3.874042in}{1.975318in}}{\pgfqpoint{3.874042in}{1.967082in}}%
\pgfpathcurveto{\pgfqpoint{3.874042in}{1.958846in}}{\pgfqpoint{3.877314in}{1.950946in}}{\pgfqpoint{3.883138in}{1.945122in}}%
\pgfpathcurveto{\pgfqpoint{3.888962in}{1.939298in}}{\pgfqpoint{3.896862in}{1.936025in}}{\pgfqpoint{3.905098in}{1.936025in}}%
\pgfpathclose%
\pgfusepath{stroke,fill}%
\end{pgfscope}%
\begin{pgfscope}%
\pgfpathrectangle{\pgfqpoint{3.793912in}{0.557870in}}{\pgfqpoint{2.446088in}{1.484734in}}%
\pgfusepath{clip}%
\pgfsetbuttcap%
\pgfsetroundjoin%
\definecolor{currentfill}{rgb}{0.298039,0.447059,0.690196}%
\pgfsetfillcolor{currentfill}%
\pgfsetlinewidth{1.003750pt}%
\definecolor{currentstroke}{rgb}{0.298039,0.447059,0.690196}%
\pgfsetstrokecolor{currentstroke}%
\pgfsetdash{}{0pt}%
\pgfpathmoveto{\pgfqpoint{3.905098in}{1.936025in}}%
\pgfpathcurveto{\pgfqpoint{3.913334in}{1.936025in}}{\pgfqpoint{3.921234in}{1.939298in}}{\pgfqpoint{3.927058in}{1.945122in}}%
\pgfpathcurveto{\pgfqpoint{3.932882in}{1.950946in}}{\pgfqpoint{3.936155in}{1.958846in}}{\pgfqpoint{3.936155in}{1.967082in}}%
\pgfpathcurveto{\pgfqpoint{3.936155in}{1.975318in}}{\pgfqpoint{3.932882in}{1.983218in}}{\pgfqpoint{3.927058in}{1.989042in}}%
\pgfpathcurveto{\pgfqpoint{3.921234in}{1.994866in}}{\pgfqpoint{3.913334in}{1.998138in}}{\pgfqpoint{3.905098in}{1.998138in}}%
\pgfpathcurveto{\pgfqpoint{3.896862in}{1.998138in}}{\pgfqpoint{3.888962in}{1.994866in}}{\pgfqpoint{3.883138in}{1.989042in}}%
\pgfpathcurveto{\pgfqpoint{3.877314in}{1.983218in}}{\pgfqpoint{3.874042in}{1.975318in}}{\pgfqpoint{3.874042in}{1.967082in}}%
\pgfpathcurveto{\pgfqpoint{3.874042in}{1.958846in}}{\pgfqpoint{3.877314in}{1.950946in}}{\pgfqpoint{3.883138in}{1.945122in}}%
\pgfpathcurveto{\pgfqpoint{3.888962in}{1.939298in}}{\pgfqpoint{3.896862in}{1.936025in}}{\pgfqpoint{3.905098in}{1.936025in}}%
\pgfpathclose%
\pgfusepath{stroke,fill}%
\end{pgfscope}%
\begin{pgfscope}%
\pgfpathrectangle{\pgfqpoint{3.793912in}{0.557870in}}{\pgfqpoint{2.446088in}{1.484734in}}%
\pgfusepath{clip}%
\pgfsetbuttcap%
\pgfsetroundjoin%
\definecolor{currentfill}{rgb}{0.298039,0.447059,0.690196}%
\pgfsetfillcolor{currentfill}%
\pgfsetlinewidth{1.003750pt}%
\definecolor{currentstroke}{rgb}{0.298039,0.447059,0.690196}%
\pgfsetstrokecolor{currentstroke}%
\pgfsetdash{}{0pt}%
\pgfpathmoveto{\pgfqpoint{3.905098in}{1.936025in}}%
\pgfpathcurveto{\pgfqpoint{3.913334in}{1.936025in}}{\pgfqpoint{3.921234in}{1.939298in}}{\pgfqpoint{3.927058in}{1.945122in}}%
\pgfpathcurveto{\pgfqpoint{3.932882in}{1.950946in}}{\pgfqpoint{3.936155in}{1.958846in}}{\pgfqpoint{3.936155in}{1.967082in}}%
\pgfpathcurveto{\pgfqpoint{3.936155in}{1.975318in}}{\pgfqpoint{3.932882in}{1.983218in}}{\pgfqpoint{3.927058in}{1.989042in}}%
\pgfpathcurveto{\pgfqpoint{3.921234in}{1.994866in}}{\pgfqpoint{3.913334in}{1.998138in}}{\pgfqpoint{3.905098in}{1.998138in}}%
\pgfpathcurveto{\pgfqpoint{3.896862in}{1.998138in}}{\pgfqpoint{3.888962in}{1.994866in}}{\pgfqpoint{3.883138in}{1.989042in}}%
\pgfpathcurveto{\pgfqpoint{3.877314in}{1.983218in}}{\pgfqpoint{3.874042in}{1.975318in}}{\pgfqpoint{3.874042in}{1.967082in}}%
\pgfpathcurveto{\pgfqpoint{3.874042in}{1.958846in}}{\pgfqpoint{3.877314in}{1.950946in}}{\pgfqpoint{3.883138in}{1.945122in}}%
\pgfpathcurveto{\pgfqpoint{3.888962in}{1.939298in}}{\pgfqpoint{3.896862in}{1.936025in}}{\pgfqpoint{3.905098in}{1.936025in}}%
\pgfpathclose%
\pgfusepath{stroke,fill}%
\end{pgfscope}%
\begin{pgfscope}%
\pgfpathrectangle{\pgfqpoint{3.793912in}{0.557870in}}{\pgfqpoint{2.446088in}{1.484734in}}%
\pgfusepath{clip}%
\pgfsetbuttcap%
\pgfsetroundjoin%
\definecolor{currentfill}{rgb}{0.298039,0.447059,0.690196}%
\pgfsetfillcolor{currentfill}%
\pgfsetlinewidth{1.003750pt}%
\definecolor{currentstroke}{rgb}{0.298039,0.447059,0.690196}%
\pgfsetstrokecolor{currentstroke}%
\pgfsetdash{}{0pt}%
\pgfpathmoveto{\pgfqpoint{3.905098in}{1.470038in}}%
\pgfpathcurveto{\pgfqpoint{3.913334in}{1.470038in}}{\pgfqpoint{3.921234in}{1.473310in}}{\pgfqpoint{3.927058in}{1.479134in}}%
\pgfpathcurveto{\pgfqpoint{3.932882in}{1.484958in}}{\pgfqpoint{3.936155in}{1.492858in}}{\pgfqpoint{3.936155in}{1.501094in}}%
\pgfpathcurveto{\pgfqpoint{3.936155in}{1.509330in}}{\pgfqpoint{3.932882in}{1.517230in}}{\pgfqpoint{3.927058in}{1.523054in}}%
\pgfpathcurveto{\pgfqpoint{3.921234in}{1.528878in}}{\pgfqpoint{3.913334in}{1.532151in}}{\pgfqpoint{3.905098in}{1.532151in}}%
\pgfpathcurveto{\pgfqpoint{3.896862in}{1.532151in}}{\pgfqpoint{3.888962in}{1.528878in}}{\pgfqpoint{3.883138in}{1.523054in}}%
\pgfpathcurveto{\pgfqpoint{3.877314in}{1.517230in}}{\pgfqpoint{3.874042in}{1.509330in}}{\pgfqpoint{3.874042in}{1.501094in}}%
\pgfpathcurveto{\pgfqpoint{3.874042in}{1.492858in}}{\pgfqpoint{3.877314in}{1.484958in}}{\pgfqpoint{3.883138in}{1.479134in}}%
\pgfpathcurveto{\pgfqpoint{3.888962in}{1.473310in}}{\pgfqpoint{3.896862in}{1.470038in}}{\pgfqpoint{3.905098in}{1.470038in}}%
\pgfpathclose%
\pgfusepath{stroke,fill}%
\end{pgfscope}%
\begin{pgfscope}%
\pgfpathrectangle{\pgfqpoint{3.793912in}{0.557870in}}{\pgfqpoint{2.446088in}{1.484734in}}%
\pgfusepath{clip}%
\pgfsetbuttcap%
\pgfsetroundjoin%
\definecolor{currentfill}{rgb}{0.298039,0.447059,0.690196}%
\pgfsetfillcolor{currentfill}%
\pgfsetlinewidth{1.003750pt}%
\definecolor{currentstroke}{rgb}{0.298039,0.447059,0.690196}%
\pgfsetstrokecolor{currentstroke}%
\pgfsetdash{}{0pt}%
\pgfpathmoveto{\pgfqpoint{3.905098in}{1.936025in}}%
\pgfpathcurveto{\pgfqpoint{3.913334in}{1.936025in}}{\pgfqpoint{3.921234in}{1.939298in}}{\pgfqpoint{3.927058in}{1.945122in}}%
\pgfpathcurveto{\pgfqpoint{3.932882in}{1.950946in}}{\pgfqpoint{3.936155in}{1.958846in}}{\pgfqpoint{3.936155in}{1.967082in}}%
\pgfpathcurveto{\pgfqpoint{3.936155in}{1.975318in}}{\pgfqpoint{3.932882in}{1.983218in}}{\pgfqpoint{3.927058in}{1.989042in}}%
\pgfpathcurveto{\pgfqpoint{3.921234in}{1.994866in}}{\pgfqpoint{3.913334in}{1.998138in}}{\pgfqpoint{3.905098in}{1.998138in}}%
\pgfpathcurveto{\pgfqpoint{3.896862in}{1.998138in}}{\pgfqpoint{3.888962in}{1.994866in}}{\pgfqpoint{3.883138in}{1.989042in}}%
\pgfpathcurveto{\pgfqpoint{3.877314in}{1.983218in}}{\pgfqpoint{3.874042in}{1.975318in}}{\pgfqpoint{3.874042in}{1.967082in}}%
\pgfpathcurveto{\pgfqpoint{3.874042in}{1.958846in}}{\pgfqpoint{3.877314in}{1.950946in}}{\pgfqpoint{3.883138in}{1.945122in}}%
\pgfpathcurveto{\pgfqpoint{3.888962in}{1.939298in}}{\pgfqpoint{3.896862in}{1.936025in}}{\pgfqpoint{3.905098in}{1.936025in}}%
\pgfpathclose%
\pgfusepath{stroke,fill}%
\end{pgfscope}%
\begin{pgfscope}%
\pgfpathrectangle{\pgfqpoint{3.793912in}{0.557870in}}{\pgfqpoint{2.446088in}{1.484734in}}%
\pgfusepath{clip}%
\pgfsetbuttcap%
\pgfsetroundjoin%
\definecolor{currentfill}{rgb}{0.298039,0.447059,0.690196}%
\pgfsetfillcolor{currentfill}%
\pgfsetlinewidth{1.003750pt}%
\definecolor{currentstroke}{rgb}{0.298039,0.447059,0.690196}%
\pgfsetstrokecolor{currentstroke}%
\pgfsetdash{}{0pt}%
\pgfpathmoveto{\pgfqpoint{3.905098in}{1.502175in}}%
\pgfpathcurveto{\pgfqpoint{3.913334in}{1.502175in}}{\pgfqpoint{3.921234in}{1.505447in}}{\pgfqpoint{3.927058in}{1.511271in}}%
\pgfpathcurveto{\pgfqpoint{3.932882in}{1.517095in}}{\pgfqpoint{3.936155in}{1.524995in}}{\pgfqpoint{3.936155in}{1.533231in}}%
\pgfpathcurveto{\pgfqpoint{3.936155in}{1.541467in}}{\pgfqpoint{3.932882in}{1.549368in}}{\pgfqpoint{3.927058in}{1.555191in}}%
\pgfpathcurveto{\pgfqpoint{3.921234in}{1.561015in}}{\pgfqpoint{3.913334in}{1.564288in}}{\pgfqpoint{3.905098in}{1.564288in}}%
\pgfpathcurveto{\pgfqpoint{3.896862in}{1.564288in}}{\pgfqpoint{3.888962in}{1.561015in}}{\pgfqpoint{3.883138in}{1.555191in}}%
\pgfpathcurveto{\pgfqpoint{3.877314in}{1.549368in}}{\pgfqpoint{3.874042in}{1.541467in}}{\pgfqpoint{3.874042in}{1.533231in}}%
\pgfpathcurveto{\pgfqpoint{3.874042in}{1.524995in}}{\pgfqpoint{3.877314in}{1.517095in}}{\pgfqpoint{3.883138in}{1.511271in}}%
\pgfpathcurveto{\pgfqpoint{3.888962in}{1.505447in}}{\pgfqpoint{3.896862in}{1.502175in}}{\pgfqpoint{3.905098in}{1.502175in}}%
\pgfpathclose%
\pgfusepath{stroke,fill}%
\end{pgfscope}%
\begin{pgfscope}%
\pgfpathrectangle{\pgfqpoint{3.793912in}{0.557870in}}{\pgfqpoint{2.446088in}{1.484734in}}%
\pgfusepath{clip}%
\pgfsetbuttcap%
\pgfsetroundjoin%
\definecolor{currentfill}{rgb}{0.298039,0.447059,0.690196}%
\pgfsetfillcolor{currentfill}%
\pgfsetlinewidth{1.003750pt}%
\definecolor{currentstroke}{rgb}{0.298039,0.447059,0.690196}%
\pgfsetstrokecolor{currentstroke}%
\pgfsetdash{}{0pt}%
\pgfpathmoveto{\pgfqpoint{3.905098in}{1.936025in}}%
\pgfpathcurveto{\pgfqpoint{3.913334in}{1.936025in}}{\pgfqpoint{3.921234in}{1.939298in}}{\pgfqpoint{3.927058in}{1.945122in}}%
\pgfpathcurveto{\pgfqpoint{3.932882in}{1.950946in}}{\pgfqpoint{3.936155in}{1.958846in}}{\pgfqpoint{3.936155in}{1.967082in}}%
\pgfpathcurveto{\pgfqpoint{3.936155in}{1.975318in}}{\pgfqpoint{3.932882in}{1.983218in}}{\pgfqpoint{3.927058in}{1.989042in}}%
\pgfpathcurveto{\pgfqpoint{3.921234in}{1.994866in}}{\pgfqpoint{3.913334in}{1.998138in}}{\pgfqpoint{3.905098in}{1.998138in}}%
\pgfpathcurveto{\pgfqpoint{3.896862in}{1.998138in}}{\pgfqpoint{3.888962in}{1.994866in}}{\pgfqpoint{3.883138in}{1.989042in}}%
\pgfpathcurveto{\pgfqpoint{3.877314in}{1.983218in}}{\pgfqpoint{3.874042in}{1.975318in}}{\pgfqpoint{3.874042in}{1.967082in}}%
\pgfpathcurveto{\pgfqpoint{3.874042in}{1.958846in}}{\pgfqpoint{3.877314in}{1.950946in}}{\pgfqpoint{3.883138in}{1.945122in}}%
\pgfpathcurveto{\pgfqpoint{3.888962in}{1.939298in}}{\pgfqpoint{3.896862in}{1.936025in}}{\pgfqpoint{3.905098in}{1.936025in}}%
\pgfpathclose%
\pgfusepath{stroke,fill}%
\end{pgfscope}%
\begin{pgfscope}%
\pgfpathrectangle{\pgfqpoint{3.793912in}{0.557870in}}{\pgfqpoint{2.446088in}{1.484734in}}%
\pgfusepath{clip}%
\pgfsetbuttcap%
\pgfsetroundjoin%
\definecolor{currentfill}{rgb}{0.298039,0.447059,0.690196}%
\pgfsetfillcolor{currentfill}%
\pgfsetlinewidth{1.003750pt}%
\definecolor{currentstroke}{rgb}{0.298039,0.447059,0.690196}%
\pgfsetstrokecolor{currentstroke}%
\pgfsetdash{}{0pt}%
\pgfpathmoveto{\pgfqpoint{3.905098in}{1.936025in}}%
\pgfpathcurveto{\pgfqpoint{3.913334in}{1.936025in}}{\pgfqpoint{3.921234in}{1.939298in}}{\pgfqpoint{3.927058in}{1.945122in}}%
\pgfpathcurveto{\pgfqpoint{3.932882in}{1.950946in}}{\pgfqpoint{3.936155in}{1.958846in}}{\pgfqpoint{3.936155in}{1.967082in}}%
\pgfpathcurveto{\pgfqpoint{3.936155in}{1.975318in}}{\pgfqpoint{3.932882in}{1.983218in}}{\pgfqpoint{3.927058in}{1.989042in}}%
\pgfpathcurveto{\pgfqpoint{3.921234in}{1.994866in}}{\pgfqpoint{3.913334in}{1.998138in}}{\pgfqpoint{3.905098in}{1.998138in}}%
\pgfpathcurveto{\pgfqpoint{3.896862in}{1.998138in}}{\pgfqpoint{3.888962in}{1.994866in}}{\pgfqpoint{3.883138in}{1.989042in}}%
\pgfpathcurveto{\pgfqpoint{3.877314in}{1.983218in}}{\pgfqpoint{3.874042in}{1.975318in}}{\pgfqpoint{3.874042in}{1.967082in}}%
\pgfpathcurveto{\pgfqpoint{3.874042in}{1.958846in}}{\pgfqpoint{3.877314in}{1.950946in}}{\pgfqpoint{3.883138in}{1.945122in}}%
\pgfpathcurveto{\pgfqpoint{3.888962in}{1.939298in}}{\pgfqpoint{3.896862in}{1.936025in}}{\pgfqpoint{3.905098in}{1.936025in}}%
\pgfpathclose%
\pgfusepath{stroke,fill}%
\end{pgfscope}%
\begin{pgfscope}%
\pgfpathrectangle{\pgfqpoint{3.793912in}{0.557870in}}{\pgfqpoint{2.446088in}{1.484734in}}%
\pgfusepath{clip}%
\pgfsetbuttcap%
\pgfsetroundjoin%
\definecolor{currentfill}{rgb}{0.298039,0.447059,0.690196}%
\pgfsetfillcolor{currentfill}%
\pgfsetlinewidth{1.003750pt}%
\definecolor{currentstroke}{rgb}{0.298039,0.447059,0.690196}%
\pgfsetstrokecolor{currentstroke}%
\pgfsetdash{}{0pt}%
\pgfpathmoveto{\pgfqpoint{3.905098in}{1.936025in}}%
\pgfpathcurveto{\pgfqpoint{3.913334in}{1.936025in}}{\pgfqpoint{3.921234in}{1.939298in}}{\pgfqpoint{3.927058in}{1.945122in}}%
\pgfpathcurveto{\pgfqpoint{3.932882in}{1.950946in}}{\pgfqpoint{3.936155in}{1.958846in}}{\pgfqpoint{3.936155in}{1.967082in}}%
\pgfpathcurveto{\pgfqpoint{3.936155in}{1.975318in}}{\pgfqpoint{3.932882in}{1.983218in}}{\pgfqpoint{3.927058in}{1.989042in}}%
\pgfpathcurveto{\pgfqpoint{3.921234in}{1.994866in}}{\pgfqpoint{3.913334in}{1.998138in}}{\pgfqpoint{3.905098in}{1.998138in}}%
\pgfpathcurveto{\pgfqpoint{3.896862in}{1.998138in}}{\pgfqpoint{3.888962in}{1.994866in}}{\pgfqpoint{3.883138in}{1.989042in}}%
\pgfpathcurveto{\pgfqpoint{3.877314in}{1.983218in}}{\pgfqpoint{3.874042in}{1.975318in}}{\pgfqpoint{3.874042in}{1.967082in}}%
\pgfpathcurveto{\pgfqpoint{3.874042in}{1.958846in}}{\pgfqpoint{3.877314in}{1.950946in}}{\pgfqpoint{3.883138in}{1.945122in}}%
\pgfpathcurveto{\pgfqpoint{3.888962in}{1.939298in}}{\pgfqpoint{3.896862in}{1.936025in}}{\pgfqpoint{3.905098in}{1.936025in}}%
\pgfpathclose%
\pgfusepath{stroke,fill}%
\end{pgfscope}%
\begin{pgfscope}%
\pgfpathrectangle{\pgfqpoint{3.793912in}{0.557870in}}{\pgfqpoint{2.446088in}{1.484734in}}%
\pgfusepath{clip}%
\pgfsetbuttcap%
\pgfsetroundjoin%
\definecolor{currentfill}{rgb}{0.298039,0.447059,0.690196}%
\pgfsetfillcolor{currentfill}%
\pgfsetlinewidth{1.003750pt}%
\definecolor{currentstroke}{rgb}{0.298039,0.447059,0.690196}%
\pgfsetstrokecolor{currentstroke}%
\pgfsetdash{}{0pt}%
\pgfpathmoveto{\pgfqpoint{3.905098in}{1.936025in}}%
\pgfpathcurveto{\pgfqpoint{3.913334in}{1.936025in}}{\pgfqpoint{3.921234in}{1.939298in}}{\pgfqpoint{3.927058in}{1.945122in}}%
\pgfpathcurveto{\pgfqpoint{3.932882in}{1.950946in}}{\pgfqpoint{3.936155in}{1.958846in}}{\pgfqpoint{3.936155in}{1.967082in}}%
\pgfpathcurveto{\pgfqpoint{3.936155in}{1.975318in}}{\pgfqpoint{3.932882in}{1.983218in}}{\pgfqpoint{3.927058in}{1.989042in}}%
\pgfpathcurveto{\pgfqpoint{3.921234in}{1.994866in}}{\pgfqpoint{3.913334in}{1.998138in}}{\pgfqpoint{3.905098in}{1.998138in}}%
\pgfpathcurveto{\pgfqpoint{3.896862in}{1.998138in}}{\pgfqpoint{3.888962in}{1.994866in}}{\pgfqpoint{3.883138in}{1.989042in}}%
\pgfpathcurveto{\pgfqpoint{3.877314in}{1.983218in}}{\pgfqpoint{3.874042in}{1.975318in}}{\pgfqpoint{3.874042in}{1.967082in}}%
\pgfpathcurveto{\pgfqpoint{3.874042in}{1.958846in}}{\pgfqpoint{3.877314in}{1.950946in}}{\pgfqpoint{3.883138in}{1.945122in}}%
\pgfpathcurveto{\pgfqpoint{3.888962in}{1.939298in}}{\pgfqpoint{3.896862in}{1.936025in}}{\pgfqpoint{3.905098in}{1.936025in}}%
\pgfpathclose%
\pgfusepath{stroke,fill}%
\end{pgfscope}%
\begin{pgfscope}%
\pgfpathrectangle{\pgfqpoint{3.793912in}{0.557870in}}{\pgfqpoint{2.446088in}{1.484734in}}%
\pgfusepath{clip}%
\pgfsetbuttcap%
\pgfsetroundjoin%
\definecolor{currentfill}{rgb}{0.298039,0.447059,0.690196}%
\pgfsetfillcolor{currentfill}%
\pgfsetlinewidth{1.003750pt}%
\definecolor{currentstroke}{rgb}{0.298039,0.447059,0.690196}%
\pgfsetstrokecolor{currentstroke}%
\pgfsetdash{}{0pt}%
\pgfpathmoveto{\pgfqpoint{3.905098in}{1.076358in}}%
\pgfpathcurveto{\pgfqpoint{3.913334in}{1.076358in}}{\pgfqpoint{3.921234in}{1.079630in}}{\pgfqpoint{3.927058in}{1.085454in}}%
\pgfpathcurveto{\pgfqpoint{3.932882in}{1.091278in}}{\pgfqpoint{3.936155in}{1.099178in}}{\pgfqpoint{3.936155in}{1.107415in}}%
\pgfpathcurveto{\pgfqpoint{3.936155in}{1.115651in}}{\pgfqpoint{3.932882in}{1.123551in}}{\pgfqpoint{3.927058in}{1.129375in}}%
\pgfpathcurveto{\pgfqpoint{3.921234in}{1.135199in}}{\pgfqpoint{3.913334in}{1.138471in}}{\pgfqpoint{3.905098in}{1.138471in}}%
\pgfpathcurveto{\pgfqpoint{3.896862in}{1.138471in}}{\pgfqpoint{3.888962in}{1.135199in}}{\pgfqpoint{3.883138in}{1.129375in}}%
\pgfpathcurveto{\pgfqpoint{3.877314in}{1.123551in}}{\pgfqpoint{3.874042in}{1.115651in}}{\pgfqpoint{3.874042in}{1.107415in}}%
\pgfpathcurveto{\pgfqpoint{3.874042in}{1.099178in}}{\pgfqpoint{3.877314in}{1.091278in}}{\pgfqpoint{3.883138in}{1.085454in}}%
\pgfpathcurveto{\pgfqpoint{3.888962in}{1.079630in}}{\pgfqpoint{3.896862in}{1.076358in}}{\pgfqpoint{3.905098in}{1.076358in}}%
\pgfpathclose%
\pgfusepath{stroke,fill}%
\end{pgfscope}%
\begin{pgfscope}%
\pgfpathrectangle{\pgfqpoint{3.793912in}{0.557870in}}{\pgfqpoint{2.446088in}{1.484734in}}%
\pgfusepath{clip}%
\pgfsetbuttcap%
\pgfsetroundjoin%
\definecolor{currentfill}{rgb}{0.298039,0.447059,0.690196}%
\pgfsetfillcolor{currentfill}%
\pgfsetlinewidth{1.003750pt}%
\definecolor{currentstroke}{rgb}{0.298039,0.447059,0.690196}%
\pgfsetstrokecolor{currentstroke}%
\pgfsetdash{}{0pt}%
\pgfpathmoveto{\pgfqpoint{3.905098in}{1.285249in}}%
\pgfpathcurveto{\pgfqpoint{3.913334in}{1.285249in}}{\pgfqpoint{3.921234in}{1.288522in}}{\pgfqpoint{3.927058in}{1.294346in}}%
\pgfpathcurveto{\pgfqpoint{3.932882in}{1.300169in}}{\pgfqpoint{3.936155in}{1.308069in}}{\pgfqpoint{3.936155in}{1.316306in}}%
\pgfpathcurveto{\pgfqpoint{3.936155in}{1.324542in}}{\pgfqpoint{3.932882in}{1.332442in}}{\pgfqpoint{3.927058in}{1.338266in}}%
\pgfpathcurveto{\pgfqpoint{3.921234in}{1.344090in}}{\pgfqpoint{3.913334in}{1.347362in}}{\pgfqpoint{3.905098in}{1.347362in}}%
\pgfpathcurveto{\pgfqpoint{3.896862in}{1.347362in}}{\pgfqpoint{3.888962in}{1.344090in}}{\pgfqpoint{3.883138in}{1.338266in}}%
\pgfpathcurveto{\pgfqpoint{3.877314in}{1.332442in}}{\pgfqpoint{3.874042in}{1.324542in}}{\pgfqpoint{3.874042in}{1.316306in}}%
\pgfpathcurveto{\pgfqpoint{3.874042in}{1.308069in}}{\pgfqpoint{3.877314in}{1.300169in}}{\pgfqpoint{3.883138in}{1.294346in}}%
\pgfpathcurveto{\pgfqpoint{3.888962in}{1.288522in}}{\pgfqpoint{3.896862in}{1.285249in}}{\pgfqpoint{3.905098in}{1.285249in}}%
\pgfpathclose%
\pgfusepath{stroke,fill}%
\end{pgfscope}%
\begin{pgfscope}%
\pgfpathrectangle{\pgfqpoint{3.793912in}{0.557870in}}{\pgfqpoint{2.446088in}{1.484734in}}%
\pgfusepath{clip}%
\pgfsetbuttcap%
\pgfsetroundjoin%
\definecolor{currentfill}{rgb}{0.298039,0.447059,0.690196}%
\pgfsetfillcolor{currentfill}%
\pgfsetlinewidth{1.003750pt}%
\definecolor{currentstroke}{rgb}{0.298039,0.447059,0.690196}%
\pgfsetstrokecolor{currentstroke}%
\pgfsetdash{}{0pt}%
\pgfpathmoveto{\pgfqpoint{3.905098in}{1.309352in}}%
\pgfpathcurveto{\pgfqpoint{3.913334in}{1.309352in}}{\pgfqpoint{3.921234in}{1.312624in}}{\pgfqpoint{3.927058in}{1.318448in}}%
\pgfpathcurveto{\pgfqpoint{3.932882in}{1.324272in}}{\pgfqpoint{3.936155in}{1.332172in}}{\pgfqpoint{3.936155in}{1.340409in}}%
\pgfpathcurveto{\pgfqpoint{3.936155in}{1.348645in}}{\pgfqpoint{3.932882in}{1.356545in}}{\pgfqpoint{3.927058in}{1.362369in}}%
\pgfpathcurveto{\pgfqpoint{3.921234in}{1.368193in}}{\pgfqpoint{3.913334in}{1.371465in}}{\pgfqpoint{3.905098in}{1.371465in}}%
\pgfpathcurveto{\pgfqpoint{3.896862in}{1.371465in}}{\pgfqpoint{3.888962in}{1.368193in}}{\pgfqpoint{3.883138in}{1.362369in}}%
\pgfpathcurveto{\pgfqpoint{3.877314in}{1.356545in}}{\pgfqpoint{3.874042in}{1.348645in}}{\pgfqpoint{3.874042in}{1.340409in}}%
\pgfpathcurveto{\pgfqpoint{3.874042in}{1.332172in}}{\pgfqpoint{3.877314in}{1.324272in}}{\pgfqpoint{3.883138in}{1.318448in}}%
\pgfpathcurveto{\pgfqpoint{3.888962in}{1.312624in}}{\pgfqpoint{3.896862in}{1.309352in}}{\pgfqpoint{3.905098in}{1.309352in}}%
\pgfpathclose%
\pgfusepath{stroke,fill}%
\end{pgfscope}%
\begin{pgfscope}%
\pgfpathrectangle{\pgfqpoint{3.793912in}{0.557870in}}{\pgfqpoint{2.446088in}{1.484734in}}%
\pgfusepath{clip}%
\pgfsetbuttcap%
\pgfsetroundjoin%
\definecolor{currentfill}{rgb}{0.298039,0.447059,0.690196}%
\pgfsetfillcolor{currentfill}%
\pgfsetlinewidth{1.003750pt}%
\definecolor{currentstroke}{rgb}{0.298039,0.447059,0.690196}%
\pgfsetstrokecolor{currentstroke}%
\pgfsetdash{}{0pt}%
\pgfpathmoveto{\pgfqpoint{3.905098in}{1.317386in}}%
\pgfpathcurveto{\pgfqpoint{3.913334in}{1.317386in}}{\pgfqpoint{3.921234in}{1.320659in}}{\pgfqpoint{3.927058in}{1.326483in}}%
\pgfpathcurveto{\pgfqpoint{3.932882in}{1.332307in}}{\pgfqpoint{3.936155in}{1.340207in}}{\pgfqpoint{3.936155in}{1.348443in}}%
\pgfpathcurveto{\pgfqpoint{3.936155in}{1.356679in}}{\pgfqpoint{3.932882in}{1.364579in}}{\pgfqpoint{3.927058in}{1.370403in}}%
\pgfpathcurveto{\pgfqpoint{3.921234in}{1.376227in}}{\pgfqpoint{3.913334in}{1.379499in}}{\pgfqpoint{3.905098in}{1.379499in}}%
\pgfpathcurveto{\pgfqpoint{3.896862in}{1.379499in}}{\pgfqpoint{3.888962in}{1.376227in}}{\pgfqpoint{3.883138in}{1.370403in}}%
\pgfpathcurveto{\pgfqpoint{3.877314in}{1.364579in}}{\pgfqpoint{3.874042in}{1.356679in}}{\pgfqpoint{3.874042in}{1.348443in}}%
\pgfpathcurveto{\pgfqpoint{3.874042in}{1.340207in}}{\pgfqpoint{3.877314in}{1.332307in}}{\pgfqpoint{3.883138in}{1.326483in}}%
\pgfpathcurveto{\pgfqpoint{3.888962in}{1.320659in}}{\pgfqpoint{3.896862in}{1.317386in}}{\pgfqpoint{3.905098in}{1.317386in}}%
\pgfpathclose%
\pgfusepath{stroke,fill}%
\end{pgfscope}%
\begin{pgfscope}%
\pgfpathrectangle{\pgfqpoint{3.793912in}{0.557870in}}{\pgfqpoint{2.446088in}{1.484734in}}%
\pgfusepath{clip}%
\pgfsetbuttcap%
\pgfsetroundjoin%
\definecolor{currentfill}{rgb}{0.298039,0.447059,0.690196}%
\pgfsetfillcolor{currentfill}%
\pgfsetlinewidth{1.003750pt}%
\definecolor{currentstroke}{rgb}{0.298039,0.447059,0.690196}%
\pgfsetstrokecolor{currentstroke}%
\pgfsetdash{}{0pt}%
\pgfpathmoveto{\pgfqpoint{3.905098in}{1.229009in}}%
\pgfpathcurveto{\pgfqpoint{3.913334in}{1.229009in}}{\pgfqpoint{3.921234in}{1.232282in}}{\pgfqpoint{3.927058in}{1.238106in}}%
\pgfpathcurveto{\pgfqpoint{3.932882in}{1.243930in}}{\pgfqpoint{3.936155in}{1.251830in}}{\pgfqpoint{3.936155in}{1.260066in}}%
\pgfpathcurveto{\pgfqpoint{3.936155in}{1.268302in}}{\pgfqpoint{3.932882in}{1.276202in}}{\pgfqpoint{3.927058in}{1.282026in}}%
\pgfpathcurveto{\pgfqpoint{3.921234in}{1.287850in}}{\pgfqpoint{3.913334in}{1.291122in}}{\pgfqpoint{3.905098in}{1.291122in}}%
\pgfpathcurveto{\pgfqpoint{3.896862in}{1.291122in}}{\pgfqpoint{3.888962in}{1.287850in}}{\pgfqpoint{3.883138in}{1.282026in}}%
\pgfpathcurveto{\pgfqpoint{3.877314in}{1.276202in}}{\pgfqpoint{3.874042in}{1.268302in}}{\pgfqpoint{3.874042in}{1.260066in}}%
\pgfpathcurveto{\pgfqpoint{3.874042in}{1.251830in}}{\pgfqpoint{3.877314in}{1.243930in}}{\pgfqpoint{3.883138in}{1.238106in}}%
\pgfpathcurveto{\pgfqpoint{3.888962in}{1.232282in}}{\pgfqpoint{3.896862in}{1.229009in}}{\pgfqpoint{3.905098in}{1.229009in}}%
\pgfpathclose%
\pgfusepath{stroke,fill}%
\end{pgfscope}%
\begin{pgfscope}%
\pgfpathrectangle{\pgfqpoint{3.793912in}{0.557870in}}{\pgfqpoint{2.446088in}{1.484734in}}%
\pgfusepath{clip}%
\pgfsetbuttcap%
\pgfsetroundjoin%
\definecolor{currentfill}{rgb}{0.298039,0.447059,0.690196}%
\pgfsetfillcolor{currentfill}%
\pgfsetlinewidth{1.003750pt}%
\definecolor{currentstroke}{rgb}{0.298039,0.447059,0.690196}%
\pgfsetstrokecolor{currentstroke}%
\pgfsetdash{}{0pt}%
\pgfpathmoveto{\pgfqpoint{3.905098in}{0.955844in}}%
\pgfpathcurveto{\pgfqpoint{3.913334in}{0.955844in}}{\pgfqpoint{3.921234in}{0.959116in}}{\pgfqpoint{3.927058in}{0.964940in}}%
\pgfpathcurveto{\pgfqpoint{3.932882in}{0.970764in}}{\pgfqpoint{3.936155in}{0.978664in}}{\pgfqpoint{3.936155in}{0.986901in}}%
\pgfpathcurveto{\pgfqpoint{3.936155in}{0.995137in}}{\pgfqpoint{3.932882in}{1.003037in}}{\pgfqpoint{3.927058in}{1.008861in}}%
\pgfpathcurveto{\pgfqpoint{3.921234in}{1.014685in}}{\pgfqpoint{3.913334in}{1.017957in}}{\pgfqpoint{3.905098in}{1.017957in}}%
\pgfpathcurveto{\pgfqpoint{3.896862in}{1.017957in}}{\pgfqpoint{3.888962in}{1.014685in}}{\pgfqpoint{3.883138in}{1.008861in}}%
\pgfpathcurveto{\pgfqpoint{3.877314in}{1.003037in}}{\pgfqpoint{3.874042in}{0.995137in}}{\pgfqpoint{3.874042in}{0.986901in}}%
\pgfpathcurveto{\pgfqpoint{3.874042in}{0.978664in}}{\pgfqpoint{3.877314in}{0.970764in}}{\pgfqpoint{3.883138in}{0.964940in}}%
\pgfpathcurveto{\pgfqpoint{3.888962in}{0.959116in}}{\pgfqpoint{3.896862in}{0.955844in}}{\pgfqpoint{3.905098in}{0.955844in}}%
\pgfpathclose%
\pgfusepath{stroke,fill}%
\end{pgfscope}%
\begin{pgfscope}%
\pgfpathrectangle{\pgfqpoint{3.793912in}{0.557870in}}{\pgfqpoint{2.446088in}{1.484734in}}%
\pgfusepath{clip}%
\pgfsetbuttcap%
\pgfsetroundjoin%
\definecolor{currentfill}{rgb}{0.298039,0.447059,0.690196}%
\pgfsetfillcolor{currentfill}%
\pgfsetlinewidth{1.003750pt}%
\definecolor{currentstroke}{rgb}{0.298039,0.447059,0.690196}%
\pgfsetstrokecolor{currentstroke}%
\pgfsetdash{}{0pt}%
\pgfpathmoveto{\pgfqpoint{3.905098in}{1.936025in}}%
\pgfpathcurveto{\pgfqpoint{3.913334in}{1.936025in}}{\pgfqpoint{3.921234in}{1.939298in}}{\pgfqpoint{3.927058in}{1.945122in}}%
\pgfpathcurveto{\pgfqpoint{3.932882in}{1.950946in}}{\pgfqpoint{3.936155in}{1.958846in}}{\pgfqpoint{3.936155in}{1.967082in}}%
\pgfpathcurveto{\pgfqpoint{3.936155in}{1.975318in}}{\pgfqpoint{3.932882in}{1.983218in}}{\pgfqpoint{3.927058in}{1.989042in}}%
\pgfpathcurveto{\pgfqpoint{3.921234in}{1.994866in}}{\pgfqpoint{3.913334in}{1.998138in}}{\pgfqpoint{3.905098in}{1.998138in}}%
\pgfpathcurveto{\pgfqpoint{3.896862in}{1.998138in}}{\pgfqpoint{3.888962in}{1.994866in}}{\pgfqpoint{3.883138in}{1.989042in}}%
\pgfpathcurveto{\pgfqpoint{3.877314in}{1.983218in}}{\pgfqpoint{3.874042in}{1.975318in}}{\pgfqpoint{3.874042in}{1.967082in}}%
\pgfpathcurveto{\pgfqpoint{3.874042in}{1.958846in}}{\pgfqpoint{3.877314in}{1.950946in}}{\pgfqpoint{3.883138in}{1.945122in}}%
\pgfpathcurveto{\pgfqpoint{3.888962in}{1.939298in}}{\pgfqpoint{3.896862in}{1.936025in}}{\pgfqpoint{3.905098in}{1.936025in}}%
\pgfpathclose%
\pgfusepath{stroke,fill}%
\end{pgfscope}%
\begin{pgfscope}%
\pgfpathrectangle{\pgfqpoint{3.793912in}{0.557870in}}{\pgfqpoint{2.446088in}{1.484734in}}%
\pgfusepath{clip}%
\pgfsetbuttcap%
\pgfsetroundjoin%
\definecolor{currentfill}{rgb}{0.298039,0.447059,0.690196}%
\pgfsetfillcolor{currentfill}%
\pgfsetlinewidth{1.003750pt}%
\definecolor{currentstroke}{rgb}{0.298039,0.447059,0.690196}%
\pgfsetstrokecolor{currentstroke}%
\pgfsetdash{}{0pt}%
\pgfpathmoveto{\pgfqpoint{3.905098in}{0.899604in}}%
\pgfpathcurveto{\pgfqpoint{3.913334in}{0.899604in}}{\pgfqpoint{3.921234in}{0.902876in}}{\pgfqpoint{3.927058in}{0.908700in}}%
\pgfpathcurveto{\pgfqpoint{3.932882in}{0.914524in}}{\pgfqpoint{3.936155in}{0.922424in}}{\pgfqpoint{3.936155in}{0.930661in}}%
\pgfpathcurveto{\pgfqpoint{3.936155in}{0.938897in}}{\pgfqpoint{3.932882in}{0.946797in}}{\pgfqpoint{3.927058in}{0.952621in}}%
\pgfpathcurveto{\pgfqpoint{3.921234in}{0.958445in}}{\pgfqpoint{3.913334in}{0.961717in}}{\pgfqpoint{3.905098in}{0.961717in}}%
\pgfpathcurveto{\pgfqpoint{3.896862in}{0.961717in}}{\pgfqpoint{3.888962in}{0.958445in}}{\pgfqpoint{3.883138in}{0.952621in}}%
\pgfpathcurveto{\pgfqpoint{3.877314in}{0.946797in}}{\pgfqpoint{3.874042in}{0.938897in}}{\pgfqpoint{3.874042in}{0.930661in}}%
\pgfpathcurveto{\pgfqpoint{3.874042in}{0.922424in}}{\pgfqpoint{3.877314in}{0.914524in}}{\pgfqpoint{3.883138in}{0.908700in}}%
\pgfpathcurveto{\pgfqpoint{3.888962in}{0.902876in}}{\pgfqpoint{3.896862in}{0.899604in}}{\pgfqpoint{3.905098in}{0.899604in}}%
\pgfpathclose%
\pgfusepath{stroke,fill}%
\end{pgfscope}%
\begin{pgfscope}%
\pgfpathrectangle{\pgfqpoint{3.793912in}{0.557870in}}{\pgfqpoint{2.446088in}{1.484734in}}%
\pgfusepath{clip}%
\pgfsetbuttcap%
\pgfsetroundjoin%
\definecolor{currentfill}{rgb}{0.298039,0.447059,0.690196}%
\pgfsetfillcolor{currentfill}%
\pgfsetlinewidth{1.003750pt}%
\definecolor{currentstroke}{rgb}{0.298039,0.447059,0.690196}%
\pgfsetstrokecolor{currentstroke}%
\pgfsetdash{}{0pt}%
\pgfpathmoveto{\pgfqpoint{3.905098in}{1.172769in}}%
\pgfpathcurveto{\pgfqpoint{3.913334in}{1.172769in}}{\pgfqpoint{3.921234in}{1.176042in}}{\pgfqpoint{3.927058in}{1.181866in}}%
\pgfpathcurveto{\pgfqpoint{3.932882in}{1.187690in}}{\pgfqpoint{3.936155in}{1.195590in}}{\pgfqpoint{3.936155in}{1.203826in}}%
\pgfpathcurveto{\pgfqpoint{3.936155in}{1.212062in}}{\pgfqpoint{3.932882in}{1.219962in}}{\pgfqpoint{3.927058in}{1.225786in}}%
\pgfpathcurveto{\pgfqpoint{3.921234in}{1.231610in}}{\pgfqpoint{3.913334in}{1.234882in}}{\pgfqpoint{3.905098in}{1.234882in}}%
\pgfpathcurveto{\pgfqpoint{3.896862in}{1.234882in}}{\pgfqpoint{3.888962in}{1.231610in}}{\pgfqpoint{3.883138in}{1.225786in}}%
\pgfpathcurveto{\pgfqpoint{3.877314in}{1.219962in}}{\pgfqpoint{3.874042in}{1.212062in}}{\pgfqpoint{3.874042in}{1.203826in}}%
\pgfpathcurveto{\pgfqpoint{3.874042in}{1.195590in}}{\pgfqpoint{3.877314in}{1.187690in}}{\pgfqpoint{3.883138in}{1.181866in}}%
\pgfpathcurveto{\pgfqpoint{3.888962in}{1.176042in}}{\pgfqpoint{3.896862in}{1.172769in}}{\pgfqpoint{3.905098in}{1.172769in}}%
\pgfpathclose%
\pgfusepath{stroke,fill}%
\end{pgfscope}%
\begin{pgfscope}%
\pgfpathrectangle{\pgfqpoint{3.793912in}{0.557870in}}{\pgfqpoint{2.446088in}{1.484734in}}%
\pgfusepath{clip}%
\pgfsetbuttcap%
\pgfsetroundjoin%
\definecolor{currentfill}{rgb}{0.298039,0.447059,0.690196}%
\pgfsetfillcolor{currentfill}%
\pgfsetlinewidth{1.003750pt}%
\definecolor{currentstroke}{rgb}{0.298039,0.447059,0.690196}%
\pgfsetstrokecolor{currentstroke}%
\pgfsetdash{}{0pt}%
\pgfpathmoveto{\pgfqpoint{3.905098in}{1.188838in}}%
\pgfpathcurveto{\pgfqpoint{3.913334in}{1.188838in}}{\pgfqpoint{3.921234in}{1.192110in}}{\pgfqpoint{3.927058in}{1.197934in}}%
\pgfpathcurveto{\pgfqpoint{3.932882in}{1.203758in}}{\pgfqpoint{3.936155in}{1.211658in}}{\pgfqpoint{3.936155in}{1.219894in}}%
\pgfpathcurveto{\pgfqpoint{3.936155in}{1.228131in}}{\pgfqpoint{3.932882in}{1.236031in}}{\pgfqpoint{3.927058in}{1.241855in}}%
\pgfpathcurveto{\pgfqpoint{3.921234in}{1.247679in}}{\pgfqpoint{3.913334in}{1.250951in}}{\pgfqpoint{3.905098in}{1.250951in}}%
\pgfpathcurveto{\pgfqpoint{3.896862in}{1.250951in}}{\pgfqpoint{3.888962in}{1.247679in}}{\pgfqpoint{3.883138in}{1.241855in}}%
\pgfpathcurveto{\pgfqpoint{3.877314in}{1.236031in}}{\pgfqpoint{3.874042in}{1.228131in}}{\pgfqpoint{3.874042in}{1.219894in}}%
\pgfpathcurveto{\pgfqpoint{3.874042in}{1.211658in}}{\pgfqpoint{3.877314in}{1.203758in}}{\pgfqpoint{3.883138in}{1.197934in}}%
\pgfpathcurveto{\pgfqpoint{3.888962in}{1.192110in}}{\pgfqpoint{3.896862in}{1.188838in}}{\pgfqpoint{3.905098in}{1.188838in}}%
\pgfpathclose%
\pgfusepath{stroke,fill}%
\end{pgfscope}%
\begin{pgfscope}%
\pgfpathrectangle{\pgfqpoint{3.793912in}{0.557870in}}{\pgfqpoint{2.446088in}{1.484734in}}%
\pgfusepath{clip}%
\pgfsetbuttcap%
\pgfsetroundjoin%
\definecolor{currentfill}{rgb}{0.298039,0.447059,0.690196}%
\pgfsetfillcolor{currentfill}%
\pgfsetlinewidth{1.003750pt}%
\definecolor{currentstroke}{rgb}{0.298039,0.447059,0.690196}%
\pgfsetstrokecolor{currentstroke}%
\pgfsetdash{}{0pt}%
\pgfpathmoveto{\pgfqpoint{3.905098in}{0.666610in}}%
\pgfpathcurveto{\pgfqpoint{3.913334in}{0.666610in}}{\pgfqpoint{3.921234in}{0.669882in}}{\pgfqpoint{3.927058in}{0.675706in}}%
\pgfpathcurveto{\pgfqpoint{3.932882in}{0.681530in}}{\pgfqpoint{3.936155in}{0.689430in}}{\pgfqpoint{3.936155in}{0.697667in}}%
\pgfpathcurveto{\pgfqpoint{3.936155in}{0.705903in}}{\pgfqpoint{3.932882in}{0.713803in}}{\pgfqpoint{3.927058in}{0.719627in}}%
\pgfpathcurveto{\pgfqpoint{3.921234in}{0.725451in}}{\pgfqpoint{3.913334in}{0.728723in}}{\pgfqpoint{3.905098in}{0.728723in}}%
\pgfpathcurveto{\pgfqpoint{3.896862in}{0.728723in}}{\pgfqpoint{3.888962in}{0.725451in}}{\pgfqpoint{3.883138in}{0.719627in}}%
\pgfpathcurveto{\pgfqpoint{3.877314in}{0.713803in}}{\pgfqpoint{3.874042in}{0.705903in}}{\pgfqpoint{3.874042in}{0.697667in}}%
\pgfpathcurveto{\pgfqpoint{3.874042in}{0.689430in}}{\pgfqpoint{3.877314in}{0.681530in}}{\pgfqpoint{3.883138in}{0.675706in}}%
\pgfpathcurveto{\pgfqpoint{3.888962in}{0.669882in}}{\pgfqpoint{3.896862in}{0.666610in}}{\pgfqpoint{3.905098in}{0.666610in}}%
\pgfpathclose%
\pgfusepath{stroke,fill}%
\end{pgfscope}%
\begin{pgfscope}%
\pgfpathrectangle{\pgfqpoint{3.793912in}{0.557870in}}{\pgfqpoint{2.446088in}{1.484734in}}%
\pgfusepath{clip}%
\pgfsetbuttcap%
\pgfsetroundjoin%
\definecolor{currentfill}{rgb}{0.298039,0.447059,0.690196}%
\pgfsetfillcolor{currentfill}%
\pgfsetlinewidth{1.003750pt}%
\definecolor{currentstroke}{rgb}{0.298039,0.447059,0.690196}%
\pgfsetstrokecolor{currentstroke}%
\pgfsetdash{}{0pt}%
\pgfpathmoveto{\pgfqpoint{3.905098in}{1.936025in}}%
\pgfpathcurveto{\pgfqpoint{3.913334in}{1.936025in}}{\pgfqpoint{3.921234in}{1.939298in}}{\pgfqpoint{3.927058in}{1.945122in}}%
\pgfpathcurveto{\pgfqpoint{3.932882in}{1.950946in}}{\pgfqpoint{3.936155in}{1.958846in}}{\pgfqpoint{3.936155in}{1.967082in}}%
\pgfpathcurveto{\pgfqpoint{3.936155in}{1.975318in}}{\pgfqpoint{3.932882in}{1.983218in}}{\pgfqpoint{3.927058in}{1.989042in}}%
\pgfpathcurveto{\pgfqpoint{3.921234in}{1.994866in}}{\pgfqpoint{3.913334in}{1.998138in}}{\pgfqpoint{3.905098in}{1.998138in}}%
\pgfpathcurveto{\pgfqpoint{3.896862in}{1.998138in}}{\pgfqpoint{3.888962in}{1.994866in}}{\pgfqpoint{3.883138in}{1.989042in}}%
\pgfpathcurveto{\pgfqpoint{3.877314in}{1.983218in}}{\pgfqpoint{3.874042in}{1.975318in}}{\pgfqpoint{3.874042in}{1.967082in}}%
\pgfpathcurveto{\pgfqpoint{3.874042in}{1.958846in}}{\pgfqpoint{3.877314in}{1.950946in}}{\pgfqpoint{3.883138in}{1.945122in}}%
\pgfpathcurveto{\pgfqpoint{3.888962in}{1.939298in}}{\pgfqpoint{3.896862in}{1.936025in}}{\pgfqpoint{3.905098in}{1.936025in}}%
\pgfpathclose%
\pgfusepath{stroke,fill}%
\end{pgfscope}%
\begin{pgfscope}%
\pgfpathrectangle{\pgfqpoint{3.793912in}{0.557870in}}{\pgfqpoint{2.446088in}{1.484734in}}%
\pgfusepath{clip}%
\pgfsetbuttcap%
\pgfsetroundjoin%
\definecolor{currentfill}{rgb}{0.298039,0.447059,0.690196}%
\pgfsetfillcolor{currentfill}%
\pgfsetlinewidth{1.003750pt}%
\definecolor{currentstroke}{rgb}{0.298039,0.447059,0.690196}%
\pgfsetstrokecolor{currentstroke}%
\pgfsetdash{}{0pt}%
\pgfpathmoveto{\pgfqpoint{3.905098in}{1.936025in}}%
\pgfpathcurveto{\pgfqpoint{3.913334in}{1.936025in}}{\pgfqpoint{3.921234in}{1.939298in}}{\pgfqpoint{3.927058in}{1.945122in}}%
\pgfpathcurveto{\pgfqpoint{3.932882in}{1.950946in}}{\pgfqpoint{3.936155in}{1.958846in}}{\pgfqpoint{3.936155in}{1.967082in}}%
\pgfpathcurveto{\pgfqpoint{3.936155in}{1.975318in}}{\pgfqpoint{3.932882in}{1.983218in}}{\pgfqpoint{3.927058in}{1.989042in}}%
\pgfpathcurveto{\pgfqpoint{3.921234in}{1.994866in}}{\pgfqpoint{3.913334in}{1.998138in}}{\pgfqpoint{3.905098in}{1.998138in}}%
\pgfpathcurveto{\pgfqpoint{3.896862in}{1.998138in}}{\pgfqpoint{3.888962in}{1.994866in}}{\pgfqpoint{3.883138in}{1.989042in}}%
\pgfpathcurveto{\pgfqpoint{3.877314in}{1.983218in}}{\pgfqpoint{3.874042in}{1.975318in}}{\pgfqpoint{3.874042in}{1.967082in}}%
\pgfpathcurveto{\pgfqpoint{3.874042in}{1.958846in}}{\pgfqpoint{3.877314in}{1.950946in}}{\pgfqpoint{3.883138in}{1.945122in}}%
\pgfpathcurveto{\pgfqpoint{3.888962in}{1.939298in}}{\pgfqpoint{3.896862in}{1.936025in}}{\pgfqpoint{3.905098in}{1.936025in}}%
\pgfpathclose%
\pgfusepath{stroke,fill}%
\end{pgfscope}%
\begin{pgfscope}%
\pgfpathrectangle{\pgfqpoint{3.793912in}{0.557870in}}{\pgfqpoint{2.446088in}{1.484734in}}%
\pgfusepath{clip}%
\pgfsetbuttcap%
\pgfsetroundjoin%
\definecolor{currentfill}{rgb}{0.298039,0.447059,0.690196}%
\pgfsetfillcolor{currentfill}%
\pgfsetlinewidth{1.003750pt}%
\definecolor{currentstroke}{rgb}{0.298039,0.447059,0.690196}%
\pgfsetstrokecolor{currentstroke}%
\pgfsetdash{}{0pt}%
\pgfpathmoveto{\pgfqpoint{3.905098in}{1.566449in}}%
\pgfpathcurveto{\pgfqpoint{3.913334in}{1.566449in}}{\pgfqpoint{3.921234in}{1.569721in}}{\pgfqpoint{3.927058in}{1.575545in}}%
\pgfpathcurveto{\pgfqpoint{3.932882in}{1.581369in}}{\pgfqpoint{3.936155in}{1.589269in}}{\pgfqpoint{3.936155in}{1.597505in}}%
\pgfpathcurveto{\pgfqpoint{3.936155in}{1.605742in}}{\pgfqpoint{3.932882in}{1.613642in}}{\pgfqpoint{3.927058in}{1.619466in}}%
\pgfpathcurveto{\pgfqpoint{3.921234in}{1.625290in}}{\pgfqpoint{3.913334in}{1.628562in}}{\pgfqpoint{3.905098in}{1.628562in}}%
\pgfpathcurveto{\pgfqpoint{3.896862in}{1.628562in}}{\pgfqpoint{3.888962in}{1.625290in}}{\pgfqpoint{3.883138in}{1.619466in}}%
\pgfpathcurveto{\pgfqpoint{3.877314in}{1.613642in}}{\pgfqpoint{3.874042in}{1.605742in}}{\pgfqpoint{3.874042in}{1.597505in}}%
\pgfpathcurveto{\pgfqpoint{3.874042in}{1.589269in}}{\pgfqpoint{3.877314in}{1.581369in}}{\pgfqpoint{3.883138in}{1.575545in}}%
\pgfpathcurveto{\pgfqpoint{3.888962in}{1.569721in}}{\pgfqpoint{3.896862in}{1.566449in}}{\pgfqpoint{3.905098in}{1.566449in}}%
\pgfpathclose%
\pgfusepath{stroke,fill}%
\end{pgfscope}%
\begin{pgfscope}%
\pgfpathrectangle{\pgfqpoint{3.793912in}{0.557870in}}{\pgfqpoint{2.446088in}{1.484734in}}%
\pgfusepath{clip}%
\pgfsetbuttcap%
\pgfsetroundjoin%
\definecolor{currentfill}{rgb}{0.298039,0.447059,0.690196}%
\pgfsetfillcolor{currentfill}%
\pgfsetlinewidth{1.003750pt}%
\definecolor{currentstroke}{rgb}{0.298039,0.447059,0.690196}%
\pgfsetstrokecolor{currentstroke}%
\pgfsetdash{}{0pt}%
\pgfpathmoveto{\pgfqpoint{3.905098in}{1.936025in}}%
\pgfpathcurveto{\pgfqpoint{3.913334in}{1.936025in}}{\pgfqpoint{3.921234in}{1.939298in}}{\pgfqpoint{3.927058in}{1.945122in}}%
\pgfpathcurveto{\pgfqpoint{3.932882in}{1.950946in}}{\pgfqpoint{3.936155in}{1.958846in}}{\pgfqpoint{3.936155in}{1.967082in}}%
\pgfpathcurveto{\pgfqpoint{3.936155in}{1.975318in}}{\pgfqpoint{3.932882in}{1.983218in}}{\pgfqpoint{3.927058in}{1.989042in}}%
\pgfpathcurveto{\pgfqpoint{3.921234in}{1.994866in}}{\pgfqpoint{3.913334in}{1.998138in}}{\pgfqpoint{3.905098in}{1.998138in}}%
\pgfpathcurveto{\pgfqpoint{3.896862in}{1.998138in}}{\pgfqpoint{3.888962in}{1.994866in}}{\pgfqpoint{3.883138in}{1.989042in}}%
\pgfpathcurveto{\pgfqpoint{3.877314in}{1.983218in}}{\pgfqpoint{3.874042in}{1.975318in}}{\pgfqpoint{3.874042in}{1.967082in}}%
\pgfpathcurveto{\pgfqpoint{3.874042in}{1.958846in}}{\pgfqpoint{3.877314in}{1.950946in}}{\pgfqpoint{3.883138in}{1.945122in}}%
\pgfpathcurveto{\pgfqpoint{3.888962in}{1.939298in}}{\pgfqpoint{3.896862in}{1.936025in}}{\pgfqpoint{3.905098in}{1.936025in}}%
\pgfpathclose%
\pgfusepath{stroke,fill}%
\end{pgfscope}%
\begin{pgfscope}%
\pgfpathrectangle{\pgfqpoint{3.793912in}{0.557870in}}{\pgfqpoint{2.446088in}{1.484734in}}%
\pgfusepath{clip}%
\pgfsetbuttcap%
\pgfsetroundjoin%
\definecolor{currentfill}{rgb}{0.298039,0.447059,0.690196}%
\pgfsetfillcolor{currentfill}%
\pgfsetlinewidth{1.003750pt}%
\definecolor{currentstroke}{rgb}{0.298039,0.447059,0.690196}%
\pgfsetstrokecolor{currentstroke}%
\pgfsetdash{}{0pt}%
\pgfpathmoveto{\pgfqpoint{3.905098in}{1.261146in}}%
\pgfpathcurveto{\pgfqpoint{3.913334in}{1.261146in}}{\pgfqpoint{3.921234in}{1.264419in}}{\pgfqpoint{3.927058in}{1.270243in}}%
\pgfpathcurveto{\pgfqpoint{3.932882in}{1.276067in}}{\pgfqpoint{3.936155in}{1.283967in}}{\pgfqpoint{3.936155in}{1.292203in}}%
\pgfpathcurveto{\pgfqpoint{3.936155in}{1.300439in}}{\pgfqpoint{3.932882in}{1.308339in}}{\pgfqpoint{3.927058in}{1.314163in}}%
\pgfpathcurveto{\pgfqpoint{3.921234in}{1.319987in}}{\pgfqpoint{3.913334in}{1.323259in}}{\pgfqpoint{3.905098in}{1.323259in}}%
\pgfpathcurveto{\pgfqpoint{3.896862in}{1.323259in}}{\pgfqpoint{3.888962in}{1.319987in}}{\pgfqpoint{3.883138in}{1.314163in}}%
\pgfpathcurveto{\pgfqpoint{3.877314in}{1.308339in}}{\pgfqpoint{3.874042in}{1.300439in}}{\pgfqpoint{3.874042in}{1.292203in}}%
\pgfpathcurveto{\pgfqpoint{3.874042in}{1.283967in}}{\pgfqpoint{3.877314in}{1.276067in}}{\pgfqpoint{3.883138in}{1.270243in}}%
\pgfpathcurveto{\pgfqpoint{3.888962in}{1.264419in}}{\pgfqpoint{3.896862in}{1.261146in}}{\pgfqpoint{3.905098in}{1.261146in}}%
\pgfpathclose%
\pgfusepath{stroke,fill}%
\end{pgfscope}%
\begin{pgfscope}%
\pgfpathrectangle{\pgfqpoint{3.793912in}{0.557870in}}{\pgfqpoint{2.446088in}{1.484734in}}%
\pgfusepath{clip}%
\pgfsetbuttcap%
\pgfsetroundjoin%
\definecolor{currentfill}{rgb}{0.298039,0.447059,0.690196}%
\pgfsetfillcolor{currentfill}%
\pgfsetlinewidth{1.003750pt}%
\definecolor{currentstroke}{rgb}{0.298039,0.447059,0.690196}%
\pgfsetstrokecolor{currentstroke}%
\pgfsetdash{}{0pt}%
\pgfpathmoveto{\pgfqpoint{3.905098in}{1.936025in}}%
\pgfpathcurveto{\pgfqpoint{3.913334in}{1.936025in}}{\pgfqpoint{3.921234in}{1.939298in}}{\pgfqpoint{3.927058in}{1.945122in}}%
\pgfpathcurveto{\pgfqpoint{3.932882in}{1.950946in}}{\pgfqpoint{3.936155in}{1.958846in}}{\pgfqpoint{3.936155in}{1.967082in}}%
\pgfpathcurveto{\pgfqpoint{3.936155in}{1.975318in}}{\pgfqpoint{3.932882in}{1.983218in}}{\pgfqpoint{3.927058in}{1.989042in}}%
\pgfpathcurveto{\pgfqpoint{3.921234in}{1.994866in}}{\pgfqpoint{3.913334in}{1.998138in}}{\pgfqpoint{3.905098in}{1.998138in}}%
\pgfpathcurveto{\pgfqpoint{3.896862in}{1.998138in}}{\pgfqpoint{3.888962in}{1.994866in}}{\pgfqpoint{3.883138in}{1.989042in}}%
\pgfpathcurveto{\pgfqpoint{3.877314in}{1.983218in}}{\pgfqpoint{3.874042in}{1.975318in}}{\pgfqpoint{3.874042in}{1.967082in}}%
\pgfpathcurveto{\pgfqpoint{3.874042in}{1.958846in}}{\pgfqpoint{3.877314in}{1.950946in}}{\pgfqpoint{3.883138in}{1.945122in}}%
\pgfpathcurveto{\pgfqpoint{3.888962in}{1.939298in}}{\pgfqpoint{3.896862in}{1.936025in}}{\pgfqpoint{3.905098in}{1.936025in}}%
\pgfpathclose%
\pgfusepath{stroke,fill}%
\end{pgfscope}%
\begin{pgfscope}%
\pgfpathrectangle{\pgfqpoint{3.793912in}{0.557870in}}{\pgfqpoint{2.446088in}{1.484734in}}%
\pgfusepath{clip}%
\pgfsetbuttcap%
\pgfsetroundjoin%
\definecolor{currentfill}{rgb}{0.298039,0.447059,0.690196}%
\pgfsetfillcolor{currentfill}%
\pgfsetlinewidth{1.003750pt}%
\definecolor{currentstroke}{rgb}{0.298039,0.447059,0.690196}%
\pgfsetstrokecolor{currentstroke}%
\pgfsetdash{}{0pt}%
\pgfpathmoveto{\pgfqpoint{3.905098in}{1.936025in}}%
\pgfpathcurveto{\pgfqpoint{3.913334in}{1.936025in}}{\pgfqpoint{3.921234in}{1.939298in}}{\pgfqpoint{3.927058in}{1.945122in}}%
\pgfpathcurveto{\pgfqpoint{3.932882in}{1.950946in}}{\pgfqpoint{3.936155in}{1.958846in}}{\pgfqpoint{3.936155in}{1.967082in}}%
\pgfpathcurveto{\pgfqpoint{3.936155in}{1.975318in}}{\pgfqpoint{3.932882in}{1.983218in}}{\pgfqpoint{3.927058in}{1.989042in}}%
\pgfpathcurveto{\pgfqpoint{3.921234in}{1.994866in}}{\pgfqpoint{3.913334in}{1.998138in}}{\pgfqpoint{3.905098in}{1.998138in}}%
\pgfpathcurveto{\pgfqpoint{3.896862in}{1.998138in}}{\pgfqpoint{3.888962in}{1.994866in}}{\pgfqpoint{3.883138in}{1.989042in}}%
\pgfpathcurveto{\pgfqpoint{3.877314in}{1.983218in}}{\pgfqpoint{3.874042in}{1.975318in}}{\pgfqpoint{3.874042in}{1.967082in}}%
\pgfpathcurveto{\pgfqpoint{3.874042in}{1.958846in}}{\pgfqpoint{3.877314in}{1.950946in}}{\pgfqpoint{3.883138in}{1.945122in}}%
\pgfpathcurveto{\pgfqpoint{3.888962in}{1.939298in}}{\pgfqpoint{3.896862in}{1.936025in}}{\pgfqpoint{3.905098in}{1.936025in}}%
\pgfpathclose%
\pgfusepath{stroke,fill}%
\end{pgfscope}%
\begin{pgfscope}%
\pgfpathrectangle{\pgfqpoint{3.793912in}{0.557870in}}{\pgfqpoint{2.446088in}{1.484734in}}%
\pgfusepath{clip}%
\pgfsetbuttcap%
\pgfsetroundjoin%
\definecolor{currentfill}{rgb}{0.298039,0.447059,0.690196}%
\pgfsetfillcolor{currentfill}%
\pgfsetlinewidth{1.003750pt}%
\definecolor{currentstroke}{rgb}{0.298039,0.447059,0.690196}%
\pgfsetstrokecolor{currentstroke}%
\pgfsetdash{}{0pt}%
\pgfpathmoveto{\pgfqpoint{4.222772in}{0.594302in}}%
\pgfpathcurveto{\pgfqpoint{4.231008in}{0.594302in}}{\pgfqpoint{4.238908in}{0.597574in}}{\pgfqpoint{4.244732in}{0.603398in}}%
\pgfpathcurveto{\pgfqpoint{4.250556in}{0.609222in}}{\pgfqpoint{4.253828in}{0.617122in}}{\pgfqpoint{4.253828in}{0.625358in}}%
\pgfpathcurveto{\pgfqpoint{4.253828in}{0.633594in}}{\pgfqpoint{4.250556in}{0.641495in}}{\pgfqpoint{4.244732in}{0.647318in}}%
\pgfpathcurveto{\pgfqpoint{4.238908in}{0.653142in}}{\pgfqpoint{4.231008in}{0.656415in}}{\pgfqpoint{4.222772in}{0.656415in}}%
\pgfpathcurveto{\pgfqpoint{4.214535in}{0.656415in}}{\pgfqpoint{4.206635in}{0.653142in}}{\pgfqpoint{4.200812in}{0.647318in}}%
\pgfpathcurveto{\pgfqpoint{4.194988in}{0.641495in}}{\pgfqpoint{4.191715in}{0.633594in}}{\pgfqpoint{4.191715in}{0.625358in}}%
\pgfpathcurveto{\pgfqpoint{4.191715in}{0.617122in}}{\pgfqpoint{4.194988in}{0.609222in}}{\pgfqpoint{4.200812in}{0.603398in}}%
\pgfpathcurveto{\pgfqpoint{4.206635in}{0.597574in}}{\pgfqpoint{4.214535in}{0.594302in}}{\pgfqpoint{4.222772in}{0.594302in}}%
\pgfpathclose%
\pgfusepath{stroke,fill}%
\end{pgfscope}%
\begin{pgfscope}%
\pgfpathrectangle{\pgfqpoint{3.793912in}{0.557870in}}{\pgfqpoint{2.446088in}{1.484734in}}%
\pgfusepath{clip}%
\pgfsetbuttcap%
\pgfsetroundjoin%
\definecolor{currentfill}{rgb}{0.298039,0.447059,0.690196}%
\pgfsetfillcolor{currentfill}%
\pgfsetlinewidth{1.003750pt}%
\definecolor{currentstroke}{rgb}{0.298039,0.447059,0.690196}%
\pgfsetstrokecolor{currentstroke}%
\pgfsetdash{}{0pt}%
\pgfpathmoveto{\pgfqpoint{3.905098in}{1.936025in}}%
\pgfpathcurveto{\pgfqpoint{3.913334in}{1.936025in}}{\pgfqpoint{3.921234in}{1.939298in}}{\pgfqpoint{3.927058in}{1.945122in}}%
\pgfpathcurveto{\pgfqpoint{3.932882in}{1.950946in}}{\pgfqpoint{3.936155in}{1.958846in}}{\pgfqpoint{3.936155in}{1.967082in}}%
\pgfpathcurveto{\pgfqpoint{3.936155in}{1.975318in}}{\pgfqpoint{3.932882in}{1.983218in}}{\pgfqpoint{3.927058in}{1.989042in}}%
\pgfpathcurveto{\pgfqpoint{3.921234in}{1.994866in}}{\pgfqpoint{3.913334in}{1.998138in}}{\pgfqpoint{3.905098in}{1.998138in}}%
\pgfpathcurveto{\pgfqpoint{3.896862in}{1.998138in}}{\pgfqpoint{3.888962in}{1.994866in}}{\pgfqpoint{3.883138in}{1.989042in}}%
\pgfpathcurveto{\pgfqpoint{3.877314in}{1.983218in}}{\pgfqpoint{3.874042in}{1.975318in}}{\pgfqpoint{3.874042in}{1.967082in}}%
\pgfpathcurveto{\pgfqpoint{3.874042in}{1.958846in}}{\pgfqpoint{3.877314in}{1.950946in}}{\pgfqpoint{3.883138in}{1.945122in}}%
\pgfpathcurveto{\pgfqpoint{3.888962in}{1.939298in}}{\pgfqpoint{3.896862in}{1.936025in}}{\pgfqpoint{3.905098in}{1.936025in}}%
\pgfpathclose%
\pgfusepath{stroke,fill}%
\end{pgfscope}%
\begin{pgfscope}%
\pgfpathrectangle{\pgfqpoint{3.793912in}{0.557870in}}{\pgfqpoint{2.446088in}{1.484734in}}%
\pgfusepath{clip}%
\pgfsetbuttcap%
\pgfsetroundjoin%
\definecolor{currentfill}{rgb}{0.298039,0.447059,0.690196}%
\pgfsetfillcolor{currentfill}%
\pgfsetlinewidth{1.003750pt}%
\definecolor{currentstroke}{rgb}{0.298039,0.447059,0.690196}%
\pgfsetstrokecolor{currentstroke}%
\pgfsetdash{}{0pt}%
\pgfpathmoveto{\pgfqpoint{3.905098in}{1.341489in}}%
\pgfpathcurveto{\pgfqpoint{3.913334in}{1.341489in}}{\pgfqpoint{3.921234in}{1.344761in}}{\pgfqpoint{3.927058in}{1.350585in}}%
\pgfpathcurveto{\pgfqpoint{3.932882in}{1.356409in}}{\pgfqpoint{3.936155in}{1.364309in}}{\pgfqpoint{3.936155in}{1.372546in}}%
\pgfpathcurveto{\pgfqpoint{3.936155in}{1.380782in}}{\pgfqpoint{3.932882in}{1.388682in}}{\pgfqpoint{3.927058in}{1.394506in}}%
\pgfpathcurveto{\pgfqpoint{3.921234in}{1.400330in}}{\pgfqpoint{3.913334in}{1.403602in}}{\pgfqpoint{3.905098in}{1.403602in}}%
\pgfpathcurveto{\pgfqpoint{3.896862in}{1.403602in}}{\pgfqpoint{3.888962in}{1.400330in}}{\pgfqpoint{3.883138in}{1.394506in}}%
\pgfpathcurveto{\pgfqpoint{3.877314in}{1.388682in}}{\pgfqpoint{3.874042in}{1.380782in}}{\pgfqpoint{3.874042in}{1.372546in}}%
\pgfpathcurveto{\pgfqpoint{3.874042in}{1.364309in}}{\pgfqpoint{3.877314in}{1.356409in}}{\pgfqpoint{3.883138in}{1.350585in}}%
\pgfpathcurveto{\pgfqpoint{3.888962in}{1.344761in}}{\pgfqpoint{3.896862in}{1.341489in}}{\pgfqpoint{3.905098in}{1.341489in}}%
\pgfpathclose%
\pgfusepath{stroke,fill}%
\end{pgfscope}%
\begin{pgfscope}%
\pgfpathrectangle{\pgfqpoint{3.793912in}{0.557870in}}{\pgfqpoint{2.446088in}{1.484734in}}%
\pgfusepath{clip}%
\pgfsetbuttcap%
\pgfsetroundjoin%
\definecolor{currentfill}{rgb}{0.298039,0.447059,0.690196}%
\pgfsetfillcolor{currentfill}%
\pgfsetlinewidth{1.003750pt}%
\definecolor{currentstroke}{rgb}{0.298039,0.447059,0.690196}%
\pgfsetstrokecolor{currentstroke}%
\pgfsetdash{}{0pt}%
\pgfpathmoveto{\pgfqpoint{3.905098in}{0.610370in}}%
\pgfpathcurveto{\pgfqpoint{3.913334in}{0.610370in}}{\pgfqpoint{3.921234in}{0.613643in}}{\pgfqpoint{3.927058in}{0.619466in}}%
\pgfpathcurveto{\pgfqpoint{3.932882in}{0.625290in}}{\pgfqpoint{3.936155in}{0.633190in}}{\pgfqpoint{3.936155in}{0.641427in}}%
\pgfpathcurveto{\pgfqpoint{3.936155in}{0.649663in}}{\pgfqpoint{3.932882in}{0.657563in}}{\pgfqpoint{3.927058in}{0.663387in}}%
\pgfpathcurveto{\pgfqpoint{3.921234in}{0.669211in}}{\pgfqpoint{3.913334in}{0.672483in}}{\pgfqpoint{3.905098in}{0.672483in}}%
\pgfpathcurveto{\pgfqpoint{3.896862in}{0.672483in}}{\pgfqpoint{3.888962in}{0.669211in}}{\pgfqpoint{3.883138in}{0.663387in}}%
\pgfpathcurveto{\pgfqpoint{3.877314in}{0.657563in}}{\pgfqpoint{3.874042in}{0.649663in}}{\pgfqpoint{3.874042in}{0.641427in}}%
\pgfpathcurveto{\pgfqpoint{3.874042in}{0.633190in}}{\pgfqpoint{3.877314in}{0.625290in}}{\pgfqpoint{3.883138in}{0.619466in}}%
\pgfpathcurveto{\pgfqpoint{3.888962in}{0.613643in}}{\pgfqpoint{3.896862in}{0.610370in}}{\pgfqpoint{3.905098in}{0.610370in}}%
\pgfpathclose%
\pgfusepath{stroke,fill}%
\end{pgfscope}%
\begin{pgfscope}%
\pgfpathrectangle{\pgfqpoint{3.793912in}{0.557870in}}{\pgfqpoint{2.446088in}{1.484734in}}%
\pgfusepath{clip}%
\pgfsetbuttcap%
\pgfsetroundjoin%
\definecolor{currentfill}{rgb}{0.298039,0.447059,0.690196}%
\pgfsetfillcolor{currentfill}%
\pgfsetlinewidth{1.003750pt}%
\definecolor{currentstroke}{rgb}{0.298039,0.447059,0.690196}%
\pgfsetstrokecolor{currentstroke}%
\pgfsetdash{}{0pt}%
\pgfpathmoveto{\pgfqpoint{3.905098in}{1.357558in}}%
\pgfpathcurveto{\pgfqpoint{3.913334in}{1.357558in}}{\pgfqpoint{3.921234in}{1.360830in}}{\pgfqpoint{3.927058in}{1.366654in}}%
\pgfpathcurveto{\pgfqpoint{3.932882in}{1.372478in}}{\pgfqpoint{3.936155in}{1.380378in}}{\pgfqpoint{3.936155in}{1.388614in}}%
\pgfpathcurveto{\pgfqpoint{3.936155in}{1.396851in}}{\pgfqpoint{3.932882in}{1.404751in}}{\pgfqpoint{3.927058in}{1.410575in}}%
\pgfpathcurveto{\pgfqpoint{3.921234in}{1.416398in}}{\pgfqpoint{3.913334in}{1.419671in}}{\pgfqpoint{3.905098in}{1.419671in}}%
\pgfpathcurveto{\pgfqpoint{3.896862in}{1.419671in}}{\pgfqpoint{3.888962in}{1.416398in}}{\pgfqpoint{3.883138in}{1.410575in}}%
\pgfpathcurveto{\pgfqpoint{3.877314in}{1.404751in}}{\pgfqpoint{3.874042in}{1.396851in}}{\pgfqpoint{3.874042in}{1.388614in}}%
\pgfpathcurveto{\pgfqpoint{3.874042in}{1.380378in}}{\pgfqpoint{3.877314in}{1.372478in}}{\pgfqpoint{3.883138in}{1.366654in}}%
\pgfpathcurveto{\pgfqpoint{3.888962in}{1.360830in}}{\pgfqpoint{3.896862in}{1.357558in}}{\pgfqpoint{3.905098in}{1.357558in}}%
\pgfpathclose%
\pgfusepath{stroke,fill}%
\end{pgfscope}%
\begin{pgfscope}%
\pgfpathrectangle{\pgfqpoint{3.793912in}{0.557870in}}{\pgfqpoint{2.446088in}{1.484734in}}%
\pgfusepath{clip}%
\pgfsetbuttcap%
\pgfsetroundjoin%
\definecolor{currentfill}{rgb}{0.298039,0.447059,0.690196}%
\pgfsetfillcolor{currentfill}%
\pgfsetlinewidth{1.003750pt}%
\definecolor{currentstroke}{rgb}{0.298039,0.447059,0.690196}%
\pgfsetstrokecolor{currentstroke}%
\pgfsetdash{}{0pt}%
\pgfpathmoveto{\pgfqpoint{3.905098in}{1.936025in}}%
\pgfpathcurveto{\pgfqpoint{3.913334in}{1.936025in}}{\pgfqpoint{3.921234in}{1.939298in}}{\pgfqpoint{3.927058in}{1.945122in}}%
\pgfpathcurveto{\pgfqpoint{3.932882in}{1.950946in}}{\pgfqpoint{3.936155in}{1.958846in}}{\pgfqpoint{3.936155in}{1.967082in}}%
\pgfpathcurveto{\pgfqpoint{3.936155in}{1.975318in}}{\pgfqpoint{3.932882in}{1.983218in}}{\pgfqpoint{3.927058in}{1.989042in}}%
\pgfpathcurveto{\pgfqpoint{3.921234in}{1.994866in}}{\pgfqpoint{3.913334in}{1.998138in}}{\pgfqpoint{3.905098in}{1.998138in}}%
\pgfpathcurveto{\pgfqpoint{3.896862in}{1.998138in}}{\pgfqpoint{3.888962in}{1.994866in}}{\pgfqpoint{3.883138in}{1.989042in}}%
\pgfpathcurveto{\pgfqpoint{3.877314in}{1.983218in}}{\pgfqpoint{3.874042in}{1.975318in}}{\pgfqpoint{3.874042in}{1.967082in}}%
\pgfpathcurveto{\pgfqpoint{3.874042in}{1.958846in}}{\pgfqpoint{3.877314in}{1.950946in}}{\pgfqpoint{3.883138in}{1.945122in}}%
\pgfpathcurveto{\pgfqpoint{3.888962in}{1.939298in}}{\pgfqpoint{3.896862in}{1.936025in}}{\pgfqpoint{3.905098in}{1.936025in}}%
\pgfpathclose%
\pgfusepath{stroke,fill}%
\end{pgfscope}%
\begin{pgfscope}%
\pgfpathrectangle{\pgfqpoint{3.793912in}{0.557870in}}{\pgfqpoint{2.446088in}{1.484734in}}%
\pgfusepath{clip}%
\pgfsetbuttcap%
\pgfsetroundjoin%
\definecolor{currentfill}{rgb}{0.298039,0.447059,0.690196}%
\pgfsetfillcolor{currentfill}%
\pgfsetlinewidth{1.003750pt}%
\definecolor{currentstroke}{rgb}{0.298039,0.447059,0.690196}%
\pgfsetstrokecolor{currentstroke}%
\pgfsetdash{}{0pt}%
\pgfpathmoveto{\pgfqpoint{3.905098in}{1.936025in}}%
\pgfpathcurveto{\pgfqpoint{3.913334in}{1.936025in}}{\pgfqpoint{3.921234in}{1.939298in}}{\pgfqpoint{3.927058in}{1.945122in}}%
\pgfpathcurveto{\pgfqpoint{3.932882in}{1.950946in}}{\pgfqpoint{3.936155in}{1.958846in}}{\pgfqpoint{3.936155in}{1.967082in}}%
\pgfpathcurveto{\pgfqpoint{3.936155in}{1.975318in}}{\pgfqpoint{3.932882in}{1.983218in}}{\pgfqpoint{3.927058in}{1.989042in}}%
\pgfpathcurveto{\pgfqpoint{3.921234in}{1.994866in}}{\pgfqpoint{3.913334in}{1.998138in}}{\pgfqpoint{3.905098in}{1.998138in}}%
\pgfpathcurveto{\pgfqpoint{3.896862in}{1.998138in}}{\pgfqpoint{3.888962in}{1.994866in}}{\pgfqpoint{3.883138in}{1.989042in}}%
\pgfpathcurveto{\pgfqpoint{3.877314in}{1.983218in}}{\pgfqpoint{3.874042in}{1.975318in}}{\pgfqpoint{3.874042in}{1.967082in}}%
\pgfpathcurveto{\pgfqpoint{3.874042in}{1.958846in}}{\pgfqpoint{3.877314in}{1.950946in}}{\pgfqpoint{3.883138in}{1.945122in}}%
\pgfpathcurveto{\pgfqpoint{3.888962in}{1.939298in}}{\pgfqpoint{3.896862in}{1.936025in}}{\pgfqpoint{3.905098in}{1.936025in}}%
\pgfpathclose%
\pgfusepath{stroke,fill}%
\end{pgfscope}%
\begin{pgfscope}%
\pgfpathrectangle{\pgfqpoint{3.793912in}{0.557870in}}{\pgfqpoint{2.446088in}{1.484734in}}%
\pgfusepath{clip}%
\pgfsetbuttcap%
\pgfsetroundjoin%
\definecolor{currentfill}{rgb}{0.298039,0.447059,0.690196}%
\pgfsetfillcolor{currentfill}%
\pgfsetlinewidth{1.003750pt}%
\definecolor{currentstroke}{rgb}{0.298039,0.447059,0.690196}%
\pgfsetstrokecolor{currentstroke}%
\pgfsetdash{}{0pt}%
\pgfpathmoveto{\pgfqpoint{3.905098in}{0.610370in}}%
\pgfpathcurveto{\pgfqpoint{3.913334in}{0.610370in}}{\pgfqpoint{3.921234in}{0.613643in}}{\pgfqpoint{3.927058in}{0.619466in}}%
\pgfpathcurveto{\pgfqpoint{3.932882in}{0.625290in}}{\pgfqpoint{3.936155in}{0.633190in}}{\pgfqpoint{3.936155in}{0.641427in}}%
\pgfpathcurveto{\pgfqpoint{3.936155in}{0.649663in}}{\pgfqpoint{3.932882in}{0.657563in}}{\pgfqpoint{3.927058in}{0.663387in}}%
\pgfpathcurveto{\pgfqpoint{3.921234in}{0.669211in}}{\pgfqpoint{3.913334in}{0.672483in}}{\pgfqpoint{3.905098in}{0.672483in}}%
\pgfpathcurveto{\pgfqpoint{3.896862in}{0.672483in}}{\pgfqpoint{3.888962in}{0.669211in}}{\pgfqpoint{3.883138in}{0.663387in}}%
\pgfpathcurveto{\pgfqpoint{3.877314in}{0.657563in}}{\pgfqpoint{3.874042in}{0.649663in}}{\pgfqpoint{3.874042in}{0.641427in}}%
\pgfpathcurveto{\pgfqpoint{3.874042in}{0.633190in}}{\pgfqpoint{3.877314in}{0.625290in}}{\pgfqpoint{3.883138in}{0.619466in}}%
\pgfpathcurveto{\pgfqpoint{3.888962in}{0.613643in}}{\pgfqpoint{3.896862in}{0.610370in}}{\pgfqpoint{3.905098in}{0.610370in}}%
\pgfpathclose%
\pgfusepath{stroke,fill}%
\end{pgfscope}%
\begin{pgfscope}%
\pgfpathrectangle{\pgfqpoint{3.793912in}{0.557870in}}{\pgfqpoint{2.446088in}{1.484734in}}%
\pgfusepath{clip}%
\pgfsetbuttcap%
\pgfsetroundjoin%
\definecolor{currentfill}{rgb}{0.298039,0.447059,0.690196}%
\pgfsetfillcolor{currentfill}%
\pgfsetlinewidth{1.003750pt}%
\definecolor{currentstroke}{rgb}{0.298039,0.447059,0.690196}%
\pgfsetstrokecolor{currentstroke}%
\pgfsetdash{}{0pt}%
\pgfpathmoveto{\pgfqpoint{3.905098in}{1.936025in}}%
\pgfpathcurveto{\pgfqpoint{3.913334in}{1.936025in}}{\pgfqpoint{3.921234in}{1.939298in}}{\pgfqpoint{3.927058in}{1.945122in}}%
\pgfpathcurveto{\pgfqpoint{3.932882in}{1.950946in}}{\pgfqpoint{3.936155in}{1.958846in}}{\pgfqpoint{3.936155in}{1.967082in}}%
\pgfpathcurveto{\pgfqpoint{3.936155in}{1.975318in}}{\pgfqpoint{3.932882in}{1.983218in}}{\pgfqpoint{3.927058in}{1.989042in}}%
\pgfpathcurveto{\pgfqpoint{3.921234in}{1.994866in}}{\pgfqpoint{3.913334in}{1.998138in}}{\pgfqpoint{3.905098in}{1.998138in}}%
\pgfpathcurveto{\pgfqpoint{3.896862in}{1.998138in}}{\pgfqpoint{3.888962in}{1.994866in}}{\pgfqpoint{3.883138in}{1.989042in}}%
\pgfpathcurveto{\pgfqpoint{3.877314in}{1.983218in}}{\pgfqpoint{3.874042in}{1.975318in}}{\pgfqpoint{3.874042in}{1.967082in}}%
\pgfpathcurveto{\pgfqpoint{3.874042in}{1.958846in}}{\pgfqpoint{3.877314in}{1.950946in}}{\pgfqpoint{3.883138in}{1.945122in}}%
\pgfpathcurveto{\pgfqpoint{3.888962in}{1.939298in}}{\pgfqpoint{3.896862in}{1.936025in}}{\pgfqpoint{3.905098in}{1.936025in}}%
\pgfpathclose%
\pgfusepath{stroke,fill}%
\end{pgfscope}%
\begin{pgfscope}%
\pgfpathrectangle{\pgfqpoint{3.793912in}{0.557870in}}{\pgfqpoint{2.446088in}{1.484734in}}%
\pgfusepath{clip}%
\pgfsetbuttcap%
\pgfsetroundjoin%
\definecolor{currentfill}{rgb}{0.298039,0.447059,0.690196}%
\pgfsetfillcolor{currentfill}%
\pgfsetlinewidth{1.003750pt}%
\definecolor{currentstroke}{rgb}{0.298039,0.447059,0.690196}%
\pgfsetstrokecolor{currentstroke}%
\pgfsetdash{}{0pt}%
\pgfpathmoveto{\pgfqpoint{3.905098in}{1.229009in}}%
\pgfpathcurveto{\pgfqpoint{3.913334in}{1.229009in}}{\pgfqpoint{3.921234in}{1.232282in}}{\pgfqpoint{3.927058in}{1.238106in}}%
\pgfpathcurveto{\pgfqpoint{3.932882in}{1.243930in}}{\pgfqpoint{3.936155in}{1.251830in}}{\pgfqpoint{3.936155in}{1.260066in}}%
\pgfpathcurveto{\pgfqpoint{3.936155in}{1.268302in}}{\pgfqpoint{3.932882in}{1.276202in}}{\pgfqpoint{3.927058in}{1.282026in}}%
\pgfpathcurveto{\pgfqpoint{3.921234in}{1.287850in}}{\pgfqpoint{3.913334in}{1.291122in}}{\pgfqpoint{3.905098in}{1.291122in}}%
\pgfpathcurveto{\pgfqpoint{3.896862in}{1.291122in}}{\pgfqpoint{3.888962in}{1.287850in}}{\pgfqpoint{3.883138in}{1.282026in}}%
\pgfpathcurveto{\pgfqpoint{3.877314in}{1.276202in}}{\pgfqpoint{3.874042in}{1.268302in}}{\pgfqpoint{3.874042in}{1.260066in}}%
\pgfpathcurveto{\pgfqpoint{3.874042in}{1.251830in}}{\pgfqpoint{3.877314in}{1.243930in}}{\pgfqpoint{3.883138in}{1.238106in}}%
\pgfpathcurveto{\pgfqpoint{3.888962in}{1.232282in}}{\pgfqpoint{3.896862in}{1.229009in}}{\pgfqpoint{3.905098in}{1.229009in}}%
\pgfpathclose%
\pgfusepath{stroke,fill}%
\end{pgfscope}%
\begin{pgfscope}%
\pgfpathrectangle{\pgfqpoint{3.793912in}{0.557870in}}{\pgfqpoint{2.446088in}{1.484734in}}%
\pgfusepath{clip}%
\pgfsetbuttcap%
\pgfsetroundjoin%
\definecolor{currentfill}{rgb}{0.298039,0.447059,0.690196}%
\pgfsetfillcolor{currentfill}%
\pgfsetlinewidth{1.003750pt}%
\definecolor{currentstroke}{rgb}{0.298039,0.447059,0.690196}%
\pgfsetstrokecolor{currentstroke}%
\pgfsetdash{}{0pt}%
\pgfpathmoveto{\pgfqpoint{3.905098in}{1.172769in}}%
\pgfpathcurveto{\pgfqpoint{3.913334in}{1.172769in}}{\pgfqpoint{3.921234in}{1.176042in}}{\pgfqpoint{3.927058in}{1.181866in}}%
\pgfpathcurveto{\pgfqpoint{3.932882in}{1.187690in}}{\pgfqpoint{3.936155in}{1.195590in}}{\pgfqpoint{3.936155in}{1.203826in}}%
\pgfpathcurveto{\pgfqpoint{3.936155in}{1.212062in}}{\pgfqpoint{3.932882in}{1.219962in}}{\pgfqpoint{3.927058in}{1.225786in}}%
\pgfpathcurveto{\pgfqpoint{3.921234in}{1.231610in}}{\pgfqpoint{3.913334in}{1.234882in}}{\pgfqpoint{3.905098in}{1.234882in}}%
\pgfpathcurveto{\pgfqpoint{3.896862in}{1.234882in}}{\pgfqpoint{3.888962in}{1.231610in}}{\pgfqpoint{3.883138in}{1.225786in}}%
\pgfpathcurveto{\pgfqpoint{3.877314in}{1.219962in}}{\pgfqpoint{3.874042in}{1.212062in}}{\pgfqpoint{3.874042in}{1.203826in}}%
\pgfpathcurveto{\pgfqpoint{3.874042in}{1.195590in}}{\pgfqpoint{3.877314in}{1.187690in}}{\pgfqpoint{3.883138in}{1.181866in}}%
\pgfpathcurveto{\pgfqpoint{3.888962in}{1.176042in}}{\pgfqpoint{3.896862in}{1.172769in}}{\pgfqpoint{3.905098in}{1.172769in}}%
\pgfpathclose%
\pgfusepath{stroke,fill}%
\end{pgfscope}%
\begin{pgfscope}%
\pgfpathrectangle{\pgfqpoint{3.793912in}{0.557870in}}{\pgfqpoint{2.446088in}{1.484734in}}%
\pgfusepath{clip}%
\pgfsetbuttcap%
\pgfsetroundjoin%
\definecolor{currentfill}{rgb}{0.298039,0.447059,0.690196}%
\pgfsetfillcolor{currentfill}%
\pgfsetlinewidth{1.003750pt}%
\definecolor{currentstroke}{rgb}{0.298039,0.447059,0.690196}%
\pgfsetstrokecolor{currentstroke}%
\pgfsetdash{}{0pt}%
\pgfpathmoveto{\pgfqpoint{3.905098in}{1.220975in}}%
\pgfpathcurveto{\pgfqpoint{3.913334in}{1.220975in}}{\pgfqpoint{3.921234in}{1.224247in}}{\pgfqpoint{3.927058in}{1.230071in}}%
\pgfpathcurveto{\pgfqpoint{3.932882in}{1.235895in}}{\pgfqpoint{3.936155in}{1.243795in}}{\pgfqpoint{3.936155in}{1.252032in}}%
\pgfpathcurveto{\pgfqpoint{3.936155in}{1.260268in}}{\pgfqpoint{3.932882in}{1.268168in}}{\pgfqpoint{3.927058in}{1.273992in}}%
\pgfpathcurveto{\pgfqpoint{3.921234in}{1.279816in}}{\pgfqpoint{3.913334in}{1.283088in}}{\pgfqpoint{3.905098in}{1.283088in}}%
\pgfpathcurveto{\pgfqpoint{3.896862in}{1.283088in}}{\pgfqpoint{3.888962in}{1.279816in}}{\pgfqpoint{3.883138in}{1.273992in}}%
\pgfpathcurveto{\pgfqpoint{3.877314in}{1.268168in}}{\pgfqpoint{3.874042in}{1.260268in}}{\pgfqpoint{3.874042in}{1.252032in}}%
\pgfpathcurveto{\pgfqpoint{3.874042in}{1.243795in}}{\pgfqpoint{3.877314in}{1.235895in}}{\pgfqpoint{3.883138in}{1.230071in}}%
\pgfpathcurveto{\pgfqpoint{3.888962in}{1.224247in}}{\pgfqpoint{3.896862in}{1.220975in}}{\pgfqpoint{3.905098in}{1.220975in}}%
\pgfpathclose%
\pgfusepath{stroke,fill}%
\end{pgfscope}%
\begin{pgfscope}%
\pgfpathrectangle{\pgfqpoint{3.793912in}{0.557870in}}{\pgfqpoint{2.446088in}{1.484734in}}%
\pgfusepath{clip}%
\pgfsetbuttcap%
\pgfsetroundjoin%
\definecolor{currentfill}{rgb}{0.298039,0.447059,0.690196}%
\pgfsetfillcolor{currentfill}%
\pgfsetlinewidth{1.003750pt}%
\definecolor{currentstroke}{rgb}{0.298039,0.447059,0.690196}%
\pgfsetstrokecolor{currentstroke}%
\pgfsetdash{}{0pt}%
\pgfpathmoveto{\pgfqpoint{3.905098in}{0.618405in}}%
\pgfpathcurveto{\pgfqpoint{3.913334in}{0.618405in}}{\pgfqpoint{3.921234in}{0.621677in}}{\pgfqpoint{3.927058in}{0.627501in}}%
\pgfpathcurveto{\pgfqpoint{3.932882in}{0.633325in}}{\pgfqpoint{3.936155in}{0.641225in}}{\pgfqpoint{3.936155in}{0.649461in}}%
\pgfpathcurveto{\pgfqpoint{3.936155in}{0.657697in}}{\pgfqpoint{3.932882in}{0.665597in}}{\pgfqpoint{3.927058in}{0.671421in}}%
\pgfpathcurveto{\pgfqpoint{3.921234in}{0.677245in}}{\pgfqpoint{3.913334in}{0.680518in}}{\pgfqpoint{3.905098in}{0.680518in}}%
\pgfpathcurveto{\pgfqpoint{3.896862in}{0.680518in}}{\pgfqpoint{3.888962in}{0.677245in}}{\pgfqpoint{3.883138in}{0.671421in}}%
\pgfpathcurveto{\pgfqpoint{3.877314in}{0.665597in}}{\pgfqpoint{3.874042in}{0.657697in}}{\pgfqpoint{3.874042in}{0.649461in}}%
\pgfpathcurveto{\pgfqpoint{3.874042in}{0.641225in}}{\pgfqpoint{3.877314in}{0.633325in}}{\pgfqpoint{3.883138in}{0.627501in}}%
\pgfpathcurveto{\pgfqpoint{3.888962in}{0.621677in}}{\pgfqpoint{3.896862in}{0.618405in}}{\pgfqpoint{3.905098in}{0.618405in}}%
\pgfpathclose%
\pgfusepath{stroke,fill}%
\end{pgfscope}%
\begin{pgfscope}%
\pgfpathrectangle{\pgfqpoint{3.793912in}{0.557870in}}{\pgfqpoint{2.446088in}{1.484734in}}%
\pgfusepath{clip}%
\pgfsetbuttcap%
\pgfsetroundjoin%
\definecolor{currentfill}{rgb}{0.298039,0.447059,0.690196}%
\pgfsetfillcolor{currentfill}%
\pgfsetlinewidth{1.003750pt}%
\definecolor{currentstroke}{rgb}{0.298039,0.447059,0.690196}%
\pgfsetstrokecolor{currentstroke}%
\pgfsetdash{}{0pt}%
\pgfpathmoveto{\pgfqpoint{3.905098in}{1.212941in}}%
\pgfpathcurveto{\pgfqpoint{3.913334in}{1.212941in}}{\pgfqpoint{3.921234in}{1.216213in}}{\pgfqpoint{3.927058in}{1.222037in}}%
\pgfpathcurveto{\pgfqpoint{3.932882in}{1.227861in}}{\pgfqpoint{3.936155in}{1.235761in}}{\pgfqpoint{3.936155in}{1.243997in}}%
\pgfpathcurveto{\pgfqpoint{3.936155in}{1.252234in}}{\pgfqpoint{3.932882in}{1.260134in}}{\pgfqpoint{3.927058in}{1.265958in}}%
\pgfpathcurveto{\pgfqpoint{3.921234in}{1.271781in}}{\pgfqpoint{3.913334in}{1.275054in}}{\pgfqpoint{3.905098in}{1.275054in}}%
\pgfpathcurveto{\pgfqpoint{3.896862in}{1.275054in}}{\pgfqpoint{3.888962in}{1.271781in}}{\pgfqpoint{3.883138in}{1.265958in}}%
\pgfpathcurveto{\pgfqpoint{3.877314in}{1.260134in}}{\pgfqpoint{3.874042in}{1.252234in}}{\pgfqpoint{3.874042in}{1.243997in}}%
\pgfpathcurveto{\pgfqpoint{3.874042in}{1.235761in}}{\pgfqpoint{3.877314in}{1.227861in}}{\pgfqpoint{3.883138in}{1.222037in}}%
\pgfpathcurveto{\pgfqpoint{3.888962in}{1.216213in}}{\pgfqpoint{3.896862in}{1.212941in}}{\pgfqpoint{3.905098in}{1.212941in}}%
\pgfpathclose%
\pgfusepath{stroke,fill}%
\end{pgfscope}%
\begin{pgfscope}%
\pgfpathrectangle{\pgfqpoint{3.793912in}{0.557870in}}{\pgfqpoint{2.446088in}{1.484734in}}%
\pgfusepath{clip}%
\pgfsetbuttcap%
\pgfsetroundjoin%
\definecolor{currentfill}{rgb}{0.298039,0.447059,0.690196}%
\pgfsetfillcolor{currentfill}%
\pgfsetlinewidth{1.003750pt}%
\definecolor{currentstroke}{rgb}{0.298039,0.447059,0.690196}%
\pgfsetstrokecolor{currentstroke}%
\pgfsetdash{}{0pt}%
\pgfpathmoveto{\pgfqpoint{3.905098in}{1.936025in}}%
\pgfpathcurveto{\pgfqpoint{3.913334in}{1.936025in}}{\pgfqpoint{3.921234in}{1.939298in}}{\pgfqpoint{3.927058in}{1.945122in}}%
\pgfpathcurveto{\pgfqpoint{3.932882in}{1.950946in}}{\pgfqpoint{3.936155in}{1.958846in}}{\pgfqpoint{3.936155in}{1.967082in}}%
\pgfpathcurveto{\pgfqpoint{3.936155in}{1.975318in}}{\pgfqpoint{3.932882in}{1.983218in}}{\pgfqpoint{3.927058in}{1.989042in}}%
\pgfpathcurveto{\pgfqpoint{3.921234in}{1.994866in}}{\pgfqpoint{3.913334in}{1.998138in}}{\pgfqpoint{3.905098in}{1.998138in}}%
\pgfpathcurveto{\pgfqpoint{3.896862in}{1.998138in}}{\pgfqpoint{3.888962in}{1.994866in}}{\pgfqpoint{3.883138in}{1.989042in}}%
\pgfpathcurveto{\pgfqpoint{3.877314in}{1.983218in}}{\pgfqpoint{3.874042in}{1.975318in}}{\pgfqpoint{3.874042in}{1.967082in}}%
\pgfpathcurveto{\pgfqpoint{3.874042in}{1.958846in}}{\pgfqpoint{3.877314in}{1.950946in}}{\pgfqpoint{3.883138in}{1.945122in}}%
\pgfpathcurveto{\pgfqpoint{3.888962in}{1.939298in}}{\pgfqpoint{3.896862in}{1.936025in}}{\pgfqpoint{3.905098in}{1.936025in}}%
\pgfpathclose%
\pgfusepath{stroke,fill}%
\end{pgfscope}%
\begin{pgfscope}%
\pgfpathrectangle{\pgfqpoint{3.793912in}{0.557870in}}{\pgfqpoint{2.446088in}{1.484734in}}%
\pgfusepath{clip}%
\pgfsetbuttcap%
\pgfsetroundjoin%
\definecolor{currentfill}{rgb}{0.298039,0.447059,0.690196}%
\pgfsetfillcolor{currentfill}%
\pgfsetlinewidth{1.003750pt}%
\definecolor{currentstroke}{rgb}{0.298039,0.447059,0.690196}%
\pgfsetstrokecolor{currentstroke}%
\pgfsetdash{}{0pt}%
\pgfpathmoveto{\pgfqpoint{3.905098in}{1.309352in}}%
\pgfpathcurveto{\pgfqpoint{3.913334in}{1.309352in}}{\pgfqpoint{3.921234in}{1.312624in}}{\pgfqpoint{3.927058in}{1.318448in}}%
\pgfpathcurveto{\pgfqpoint{3.932882in}{1.324272in}}{\pgfqpoint{3.936155in}{1.332172in}}{\pgfqpoint{3.936155in}{1.340409in}}%
\pgfpathcurveto{\pgfqpoint{3.936155in}{1.348645in}}{\pgfqpoint{3.932882in}{1.356545in}}{\pgfqpoint{3.927058in}{1.362369in}}%
\pgfpathcurveto{\pgfqpoint{3.921234in}{1.368193in}}{\pgfqpoint{3.913334in}{1.371465in}}{\pgfqpoint{3.905098in}{1.371465in}}%
\pgfpathcurveto{\pgfqpoint{3.896862in}{1.371465in}}{\pgfqpoint{3.888962in}{1.368193in}}{\pgfqpoint{3.883138in}{1.362369in}}%
\pgfpathcurveto{\pgfqpoint{3.877314in}{1.356545in}}{\pgfqpoint{3.874042in}{1.348645in}}{\pgfqpoint{3.874042in}{1.340409in}}%
\pgfpathcurveto{\pgfqpoint{3.874042in}{1.332172in}}{\pgfqpoint{3.877314in}{1.324272in}}{\pgfqpoint{3.883138in}{1.318448in}}%
\pgfpathcurveto{\pgfqpoint{3.888962in}{1.312624in}}{\pgfqpoint{3.896862in}{1.309352in}}{\pgfqpoint{3.905098in}{1.309352in}}%
\pgfpathclose%
\pgfusepath{stroke,fill}%
\end{pgfscope}%
\begin{pgfscope}%
\pgfpathrectangle{\pgfqpoint{3.793912in}{0.557870in}}{\pgfqpoint{2.446088in}{1.484734in}}%
\pgfusepath{clip}%
\pgfsetbuttcap%
\pgfsetroundjoin%
\definecolor{currentfill}{rgb}{0.298039,0.447059,0.690196}%
\pgfsetfillcolor{currentfill}%
\pgfsetlinewidth{1.003750pt}%
\definecolor{currentstroke}{rgb}{0.298039,0.447059,0.690196}%
\pgfsetstrokecolor{currentstroke}%
\pgfsetdash{}{0pt}%
\pgfpathmoveto{\pgfqpoint{3.905098in}{1.936025in}}%
\pgfpathcurveto{\pgfqpoint{3.913334in}{1.936025in}}{\pgfqpoint{3.921234in}{1.939298in}}{\pgfqpoint{3.927058in}{1.945122in}}%
\pgfpathcurveto{\pgfqpoint{3.932882in}{1.950946in}}{\pgfqpoint{3.936155in}{1.958846in}}{\pgfqpoint{3.936155in}{1.967082in}}%
\pgfpathcurveto{\pgfqpoint{3.936155in}{1.975318in}}{\pgfqpoint{3.932882in}{1.983218in}}{\pgfqpoint{3.927058in}{1.989042in}}%
\pgfpathcurveto{\pgfqpoint{3.921234in}{1.994866in}}{\pgfqpoint{3.913334in}{1.998138in}}{\pgfqpoint{3.905098in}{1.998138in}}%
\pgfpathcurveto{\pgfqpoint{3.896862in}{1.998138in}}{\pgfqpoint{3.888962in}{1.994866in}}{\pgfqpoint{3.883138in}{1.989042in}}%
\pgfpathcurveto{\pgfqpoint{3.877314in}{1.983218in}}{\pgfqpoint{3.874042in}{1.975318in}}{\pgfqpoint{3.874042in}{1.967082in}}%
\pgfpathcurveto{\pgfqpoint{3.874042in}{1.958846in}}{\pgfqpoint{3.877314in}{1.950946in}}{\pgfqpoint{3.883138in}{1.945122in}}%
\pgfpathcurveto{\pgfqpoint{3.888962in}{1.939298in}}{\pgfqpoint{3.896862in}{1.936025in}}{\pgfqpoint{3.905098in}{1.936025in}}%
\pgfpathclose%
\pgfusepath{stroke,fill}%
\end{pgfscope}%
\begin{pgfscope}%
\pgfpathrectangle{\pgfqpoint{3.793912in}{0.557870in}}{\pgfqpoint{2.446088in}{1.484734in}}%
\pgfusepath{clip}%
\pgfsetbuttcap%
\pgfsetroundjoin%
\definecolor{currentfill}{rgb}{0.298039,0.447059,0.690196}%
\pgfsetfillcolor{currentfill}%
\pgfsetlinewidth{1.003750pt}%
\definecolor{currentstroke}{rgb}{0.298039,0.447059,0.690196}%
\pgfsetstrokecolor{currentstroke}%
\pgfsetdash{}{0pt}%
\pgfpathmoveto{\pgfqpoint{3.905098in}{1.172769in}}%
\pgfpathcurveto{\pgfqpoint{3.913334in}{1.172769in}}{\pgfqpoint{3.921234in}{1.176042in}}{\pgfqpoint{3.927058in}{1.181866in}}%
\pgfpathcurveto{\pgfqpoint{3.932882in}{1.187690in}}{\pgfqpoint{3.936155in}{1.195590in}}{\pgfqpoint{3.936155in}{1.203826in}}%
\pgfpathcurveto{\pgfqpoint{3.936155in}{1.212062in}}{\pgfqpoint{3.932882in}{1.219962in}}{\pgfqpoint{3.927058in}{1.225786in}}%
\pgfpathcurveto{\pgfqpoint{3.921234in}{1.231610in}}{\pgfqpoint{3.913334in}{1.234882in}}{\pgfqpoint{3.905098in}{1.234882in}}%
\pgfpathcurveto{\pgfqpoint{3.896862in}{1.234882in}}{\pgfqpoint{3.888962in}{1.231610in}}{\pgfqpoint{3.883138in}{1.225786in}}%
\pgfpathcurveto{\pgfqpoint{3.877314in}{1.219962in}}{\pgfqpoint{3.874042in}{1.212062in}}{\pgfqpoint{3.874042in}{1.203826in}}%
\pgfpathcurveto{\pgfqpoint{3.874042in}{1.195590in}}{\pgfqpoint{3.877314in}{1.187690in}}{\pgfqpoint{3.883138in}{1.181866in}}%
\pgfpathcurveto{\pgfqpoint{3.888962in}{1.176042in}}{\pgfqpoint{3.896862in}{1.172769in}}{\pgfqpoint{3.905098in}{1.172769in}}%
\pgfpathclose%
\pgfusepath{stroke,fill}%
\end{pgfscope}%
\begin{pgfscope}%
\pgfpathrectangle{\pgfqpoint{3.793912in}{0.557870in}}{\pgfqpoint{2.446088in}{1.484734in}}%
\pgfusepath{clip}%
\pgfsetbuttcap%
\pgfsetroundjoin%
\definecolor{currentfill}{rgb}{0.298039,0.447059,0.690196}%
\pgfsetfillcolor{currentfill}%
\pgfsetlinewidth{1.003750pt}%
\definecolor{currentstroke}{rgb}{0.298039,0.447059,0.690196}%
\pgfsetstrokecolor{currentstroke}%
\pgfsetdash{}{0pt}%
\pgfpathmoveto{\pgfqpoint{3.905098in}{1.936025in}}%
\pgfpathcurveto{\pgfqpoint{3.913334in}{1.936025in}}{\pgfqpoint{3.921234in}{1.939298in}}{\pgfqpoint{3.927058in}{1.945122in}}%
\pgfpathcurveto{\pgfqpoint{3.932882in}{1.950946in}}{\pgfqpoint{3.936155in}{1.958846in}}{\pgfqpoint{3.936155in}{1.967082in}}%
\pgfpathcurveto{\pgfqpoint{3.936155in}{1.975318in}}{\pgfqpoint{3.932882in}{1.983218in}}{\pgfqpoint{3.927058in}{1.989042in}}%
\pgfpathcurveto{\pgfqpoint{3.921234in}{1.994866in}}{\pgfqpoint{3.913334in}{1.998138in}}{\pgfqpoint{3.905098in}{1.998138in}}%
\pgfpathcurveto{\pgfqpoint{3.896862in}{1.998138in}}{\pgfqpoint{3.888962in}{1.994866in}}{\pgfqpoint{3.883138in}{1.989042in}}%
\pgfpathcurveto{\pgfqpoint{3.877314in}{1.983218in}}{\pgfqpoint{3.874042in}{1.975318in}}{\pgfqpoint{3.874042in}{1.967082in}}%
\pgfpathcurveto{\pgfqpoint{3.874042in}{1.958846in}}{\pgfqpoint{3.877314in}{1.950946in}}{\pgfqpoint{3.883138in}{1.945122in}}%
\pgfpathcurveto{\pgfqpoint{3.888962in}{1.939298in}}{\pgfqpoint{3.896862in}{1.936025in}}{\pgfqpoint{3.905098in}{1.936025in}}%
\pgfpathclose%
\pgfusepath{stroke,fill}%
\end{pgfscope}%
\begin{pgfscope}%
\pgfpathrectangle{\pgfqpoint{3.793912in}{0.557870in}}{\pgfqpoint{2.446088in}{1.484734in}}%
\pgfusepath{clip}%
\pgfsetbuttcap%
\pgfsetroundjoin%
\definecolor{currentfill}{rgb}{0.298039,0.447059,0.690196}%
\pgfsetfillcolor{currentfill}%
\pgfsetlinewidth{1.003750pt}%
\definecolor{currentstroke}{rgb}{0.298039,0.447059,0.690196}%
\pgfsetstrokecolor{currentstroke}%
\pgfsetdash{}{0pt}%
\pgfpathmoveto{\pgfqpoint{4.381609in}{0.594302in}}%
\pgfpathcurveto{\pgfqpoint{4.389845in}{0.594302in}}{\pgfqpoint{4.397745in}{0.597574in}}{\pgfqpoint{4.403569in}{0.603398in}}%
\pgfpathcurveto{\pgfqpoint{4.409393in}{0.609222in}}{\pgfqpoint{4.412665in}{0.617122in}}{\pgfqpoint{4.412665in}{0.625358in}}%
\pgfpathcurveto{\pgfqpoint{4.412665in}{0.633594in}}{\pgfqpoint{4.409393in}{0.641495in}}{\pgfqpoint{4.403569in}{0.647318in}}%
\pgfpathcurveto{\pgfqpoint{4.397745in}{0.653142in}}{\pgfqpoint{4.389845in}{0.656415in}}{\pgfqpoint{4.381609in}{0.656415in}}%
\pgfpathcurveto{\pgfqpoint{4.373372in}{0.656415in}}{\pgfqpoint{4.365472in}{0.653142in}}{\pgfqpoint{4.359648in}{0.647318in}}%
\pgfpathcurveto{\pgfqpoint{4.353824in}{0.641495in}}{\pgfqpoint{4.350552in}{0.633594in}}{\pgfqpoint{4.350552in}{0.625358in}}%
\pgfpathcurveto{\pgfqpoint{4.350552in}{0.617122in}}{\pgfqpoint{4.353824in}{0.609222in}}{\pgfqpoint{4.359648in}{0.603398in}}%
\pgfpathcurveto{\pgfqpoint{4.365472in}{0.597574in}}{\pgfqpoint{4.373372in}{0.594302in}}{\pgfqpoint{4.381609in}{0.594302in}}%
\pgfpathclose%
\pgfusepath{stroke,fill}%
\end{pgfscope}%
\begin{pgfscope}%
\pgfpathrectangle{\pgfqpoint{3.793912in}{0.557870in}}{\pgfqpoint{2.446088in}{1.484734in}}%
\pgfusepath{clip}%
\pgfsetbuttcap%
\pgfsetroundjoin%
\definecolor{currentfill}{rgb}{0.298039,0.447059,0.690196}%
\pgfsetfillcolor{currentfill}%
\pgfsetlinewidth{1.003750pt}%
\definecolor{currentstroke}{rgb}{0.298039,0.447059,0.690196}%
\pgfsetstrokecolor{currentstroke}%
\pgfsetdash{}{0pt}%
\pgfpathmoveto{\pgfqpoint{3.905098in}{1.936025in}}%
\pgfpathcurveto{\pgfqpoint{3.913334in}{1.936025in}}{\pgfqpoint{3.921234in}{1.939298in}}{\pgfqpoint{3.927058in}{1.945122in}}%
\pgfpathcurveto{\pgfqpoint{3.932882in}{1.950946in}}{\pgfqpoint{3.936155in}{1.958846in}}{\pgfqpoint{3.936155in}{1.967082in}}%
\pgfpathcurveto{\pgfqpoint{3.936155in}{1.975318in}}{\pgfqpoint{3.932882in}{1.983218in}}{\pgfqpoint{3.927058in}{1.989042in}}%
\pgfpathcurveto{\pgfqpoint{3.921234in}{1.994866in}}{\pgfqpoint{3.913334in}{1.998138in}}{\pgfqpoint{3.905098in}{1.998138in}}%
\pgfpathcurveto{\pgfqpoint{3.896862in}{1.998138in}}{\pgfqpoint{3.888962in}{1.994866in}}{\pgfqpoint{3.883138in}{1.989042in}}%
\pgfpathcurveto{\pgfqpoint{3.877314in}{1.983218in}}{\pgfqpoint{3.874042in}{1.975318in}}{\pgfqpoint{3.874042in}{1.967082in}}%
\pgfpathcurveto{\pgfqpoint{3.874042in}{1.958846in}}{\pgfqpoint{3.877314in}{1.950946in}}{\pgfqpoint{3.883138in}{1.945122in}}%
\pgfpathcurveto{\pgfqpoint{3.888962in}{1.939298in}}{\pgfqpoint{3.896862in}{1.936025in}}{\pgfqpoint{3.905098in}{1.936025in}}%
\pgfpathclose%
\pgfusepath{stroke,fill}%
\end{pgfscope}%
\begin{pgfscope}%
\pgfpathrectangle{\pgfqpoint{3.793912in}{0.557870in}}{\pgfqpoint{2.446088in}{1.484734in}}%
\pgfusepath{clip}%
\pgfsetbuttcap%
\pgfsetroundjoin%
\definecolor{currentfill}{rgb}{0.298039,0.447059,0.690196}%
\pgfsetfillcolor{currentfill}%
\pgfsetlinewidth{1.003750pt}%
\definecolor{currentstroke}{rgb}{0.298039,0.447059,0.690196}%
\pgfsetstrokecolor{currentstroke}%
\pgfsetdash{}{0pt}%
\pgfpathmoveto{\pgfqpoint{3.905098in}{1.638757in}}%
\pgfpathcurveto{\pgfqpoint{3.913334in}{1.638757in}}{\pgfqpoint{3.921234in}{1.642030in}}{\pgfqpoint{3.927058in}{1.647854in}}%
\pgfpathcurveto{\pgfqpoint{3.932882in}{1.653678in}}{\pgfqpoint{3.936155in}{1.661578in}}{\pgfqpoint{3.936155in}{1.669814in}}%
\pgfpathcurveto{\pgfqpoint{3.936155in}{1.678050in}}{\pgfqpoint{3.932882in}{1.685950in}}{\pgfqpoint{3.927058in}{1.691774in}}%
\pgfpathcurveto{\pgfqpoint{3.921234in}{1.697598in}}{\pgfqpoint{3.913334in}{1.700870in}}{\pgfqpoint{3.905098in}{1.700870in}}%
\pgfpathcurveto{\pgfqpoint{3.896862in}{1.700870in}}{\pgfqpoint{3.888962in}{1.697598in}}{\pgfqpoint{3.883138in}{1.691774in}}%
\pgfpathcurveto{\pgfqpoint{3.877314in}{1.685950in}}{\pgfqpoint{3.874042in}{1.678050in}}{\pgfqpoint{3.874042in}{1.669814in}}%
\pgfpathcurveto{\pgfqpoint{3.874042in}{1.661578in}}{\pgfqpoint{3.877314in}{1.653678in}}{\pgfqpoint{3.883138in}{1.647854in}}%
\pgfpathcurveto{\pgfqpoint{3.888962in}{1.642030in}}{\pgfqpoint{3.896862in}{1.638757in}}{\pgfqpoint{3.905098in}{1.638757in}}%
\pgfpathclose%
\pgfusepath{stroke,fill}%
\end{pgfscope}%
\begin{pgfscope}%
\pgfpathrectangle{\pgfqpoint{3.793912in}{0.557870in}}{\pgfqpoint{2.446088in}{1.484734in}}%
\pgfusepath{clip}%
\pgfsetbuttcap%
\pgfsetroundjoin%
\definecolor{currentfill}{rgb}{0.298039,0.447059,0.690196}%
\pgfsetfillcolor{currentfill}%
\pgfsetlinewidth{1.003750pt}%
\definecolor{currentstroke}{rgb}{0.298039,0.447059,0.690196}%
\pgfsetstrokecolor{currentstroke}%
\pgfsetdash{}{0pt}%
\pgfpathmoveto{\pgfqpoint{3.905098in}{1.936025in}}%
\pgfpathcurveto{\pgfqpoint{3.913334in}{1.936025in}}{\pgfqpoint{3.921234in}{1.939298in}}{\pgfqpoint{3.927058in}{1.945122in}}%
\pgfpathcurveto{\pgfqpoint{3.932882in}{1.950946in}}{\pgfqpoint{3.936155in}{1.958846in}}{\pgfqpoint{3.936155in}{1.967082in}}%
\pgfpathcurveto{\pgfqpoint{3.936155in}{1.975318in}}{\pgfqpoint{3.932882in}{1.983218in}}{\pgfqpoint{3.927058in}{1.989042in}}%
\pgfpathcurveto{\pgfqpoint{3.921234in}{1.994866in}}{\pgfqpoint{3.913334in}{1.998138in}}{\pgfqpoint{3.905098in}{1.998138in}}%
\pgfpathcurveto{\pgfqpoint{3.896862in}{1.998138in}}{\pgfqpoint{3.888962in}{1.994866in}}{\pgfqpoint{3.883138in}{1.989042in}}%
\pgfpathcurveto{\pgfqpoint{3.877314in}{1.983218in}}{\pgfqpoint{3.874042in}{1.975318in}}{\pgfqpoint{3.874042in}{1.967082in}}%
\pgfpathcurveto{\pgfqpoint{3.874042in}{1.958846in}}{\pgfqpoint{3.877314in}{1.950946in}}{\pgfqpoint{3.883138in}{1.945122in}}%
\pgfpathcurveto{\pgfqpoint{3.888962in}{1.939298in}}{\pgfqpoint{3.896862in}{1.936025in}}{\pgfqpoint{3.905098in}{1.936025in}}%
\pgfpathclose%
\pgfusepath{stroke,fill}%
\end{pgfscope}%
\begin{pgfscope}%
\pgfpathrectangle{\pgfqpoint{3.793912in}{0.557870in}}{\pgfqpoint{2.446088in}{1.484734in}}%
\pgfusepath{clip}%
\pgfsetbuttcap%
\pgfsetroundjoin%
\definecolor{currentfill}{rgb}{0.298039,0.447059,0.690196}%
\pgfsetfillcolor{currentfill}%
\pgfsetlinewidth{1.003750pt}%
\definecolor{currentstroke}{rgb}{0.298039,0.447059,0.690196}%
\pgfsetstrokecolor{currentstroke}%
\pgfsetdash{}{0pt}%
\pgfpathmoveto{\pgfqpoint{3.905098in}{0.843364in}}%
\pgfpathcurveto{\pgfqpoint{3.913334in}{0.843364in}}{\pgfqpoint{3.921234in}{0.846636in}}{\pgfqpoint{3.927058in}{0.852460in}}%
\pgfpathcurveto{\pgfqpoint{3.932882in}{0.858284in}}{\pgfqpoint{3.936155in}{0.866184in}}{\pgfqpoint{3.936155in}{0.874421in}}%
\pgfpathcurveto{\pgfqpoint{3.936155in}{0.882657in}}{\pgfqpoint{3.932882in}{0.890557in}}{\pgfqpoint{3.927058in}{0.896381in}}%
\pgfpathcurveto{\pgfqpoint{3.921234in}{0.902205in}}{\pgfqpoint{3.913334in}{0.905477in}}{\pgfqpoint{3.905098in}{0.905477in}}%
\pgfpathcurveto{\pgfqpoint{3.896862in}{0.905477in}}{\pgfqpoint{3.888962in}{0.902205in}}{\pgfqpoint{3.883138in}{0.896381in}}%
\pgfpathcurveto{\pgfqpoint{3.877314in}{0.890557in}}{\pgfqpoint{3.874042in}{0.882657in}}{\pgfqpoint{3.874042in}{0.874421in}}%
\pgfpathcurveto{\pgfqpoint{3.874042in}{0.866184in}}{\pgfqpoint{3.877314in}{0.858284in}}{\pgfqpoint{3.883138in}{0.852460in}}%
\pgfpathcurveto{\pgfqpoint{3.888962in}{0.846636in}}{\pgfqpoint{3.896862in}{0.843364in}}{\pgfqpoint{3.905098in}{0.843364in}}%
\pgfpathclose%
\pgfusepath{stroke,fill}%
\end{pgfscope}%
\begin{pgfscope}%
\pgfpathrectangle{\pgfqpoint{3.793912in}{0.557870in}}{\pgfqpoint{2.446088in}{1.484734in}}%
\pgfusepath{clip}%
\pgfsetbuttcap%
\pgfsetroundjoin%
\definecolor{currentfill}{rgb}{0.298039,0.447059,0.690196}%
\pgfsetfillcolor{currentfill}%
\pgfsetlinewidth{1.003750pt}%
\definecolor{currentstroke}{rgb}{0.298039,0.447059,0.690196}%
\pgfsetstrokecolor{currentstroke}%
\pgfsetdash{}{0pt}%
\pgfpathmoveto{\pgfqpoint{3.905098in}{1.936025in}}%
\pgfpathcurveto{\pgfqpoint{3.913334in}{1.936025in}}{\pgfqpoint{3.921234in}{1.939298in}}{\pgfqpoint{3.927058in}{1.945122in}}%
\pgfpathcurveto{\pgfqpoint{3.932882in}{1.950946in}}{\pgfqpoint{3.936155in}{1.958846in}}{\pgfqpoint{3.936155in}{1.967082in}}%
\pgfpathcurveto{\pgfqpoint{3.936155in}{1.975318in}}{\pgfqpoint{3.932882in}{1.983218in}}{\pgfqpoint{3.927058in}{1.989042in}}%
\pgfpathcurveto{\pgfqpoint{3.921234in}{1.994866in}}{\pgfqpoint{3.913334in}{1.998138in}}{\pgfqpoint{3.905098in}{1.998138in}}%
\pgfpathcurveto{\pgfqpoint{3.896862in}{1.998138in}}{\pgfqpoint{3.888962in}{1.994866in}}{\pgfqpoint{3.883138in}{1.989042in}}%
\pgfpathcurveto{\pgfqpoint{3.877314in}{1.983218in}}{\pgfqpoint{3.874042in}{1.975318in}}{\pgfqpoint{3.874042in}{1.967082in}}%
\pgfpathcurveto{\pgfqpoint{3.874042in}{1.958846in}}{\pgfqpoint{3.877314in}{1.950946in}}{\pgfqpoint{3.883138in}{1.945122in}}%
\pgfpathcurveto{\pgfqpoint{3.888962in}{1.939298in}}{\pgfqpoint{3.896862in}{1.936025in}}{\pgfqpoint{3.905098in}{1.936025in}}%
\pgfpathclose%
\pgfusepath{stroke,fill}%
\end{pgfscope}%
\begin{pgfscope}%
\pgfpathrectangle{\pgfqpoint{3.793912in}{0.557870in}}{\pgfqpoint{2.446088in}{1.484734in}}%
\pgfusepath{clip}%
\pgfsetbuttcap%
\pgfsetroundjoin%
\definecolor{currentfill}{rgb}{0.298039,0.447059,0.690196}%
\pgfsetfillcolor{currentfill}%
\pgfsetlinewidth{1.003750pt}%
\definecolor{currentstroke}{rgb}{0.298039,0.447059,0.690196}%
\pgfsetstrokecolor{currentstroke}%
\pgfsetdash{}{0pt}%
\pgfpathmoveto{\pgfqpoint{3.905098in}{1.936025in}}%
\pgfpathcurveto{\pgfqpoint{3.913334in}{1.936025in}}{\pgfqpoint{3.921234in}{1.939298in}}{\pgfqpoint{3.927058in}{1.945122in}}%
\pgfpathcurveto{\pgfqpoint{3.932882in}{1.950946in}}{\pgfqpoint{3.936155in}{1.958846in}}{\pgfqpoint{3.936155in}{1.967082in}}%
\pgfpathcurveto{\pgfqpoint{3.936155in}{1.975318in}}{\pgfqpoint{3.932882in}{1.983218in}}{\pgfqpoint{3.927058in}{1.989042in}}%
\pgfpathcurveto{\pgfqpoint{3.921234in}{1.994866in}}{\pgfqpoint{3.913334in}{1.998138in}}{\pgfqpoint{3.905098in}{1.998138in}}%
\pgfpathcurveto{\pgfqpoint{3.896862in}{1.998138in}}{\pgfqpoint{3.888962in}{1.994866in}}{\pgfqpoint{3.883138in}{1.989042in}}%
\pgfpathcurveto{\pgfqpoint{3.877314in}{1.983218in}}{\pgfqpoint{3.874042in}{1.975318in}}{\pgfqpoint{3.874042in}{1.967082in}}%
\pgfpathcurveto{\pgfqpoint{3.874042in}{1.958846in}}{\pgfqpoint{3.877314in}{1.950946in}}{\pgfqpoint{3.883138in}{1.945122in}}%
\pgfpathcurveto{\pgfqpoint{3.888962in}{1.939298in}}{\pgfqpoint{3.896862in}{1.936025in}}{\pgfqpoint{3.905098in}{1.936025in}}%
\pgfpathclose%
\pgfusepath{stroke,fill}%
\end{pgfscope}%
\begin{pgfscope}%
\pgfpathrectangle{\pgfqpoint{3.793912in}{0.557870in}}{\pgfqpoint{2.446088in}{1.484734in}}%
\pgfusepath{clip}%
\pgfsetbuttcap%
\pgfsetroundjoin%
\definecolor{currentfill}{rgb}{0.298039,0.447059,0.690196}%
\pgfsetfillcolor{currentfill}%
\pgfsetlinewidth{1.003750pt}%
\definecolor{currentstroke}{rgb}{0.298039,0.447059,0.690196}%
\pgfsetstrokecolor{currentstroke}%
\pgfsetdash{}{0pt}%
\pgfpathmoveto{\pgfqpoint{3.905098in}{1.936025in}}%
\pgfpathcurveto{\pgfqpoint{3.913334in}{1.936025in}}{\pgfqpoint{3.921234in}{1.939298in}}{\pgfqpoint{3.927058in}{1.945122in}}%
\pgfpathcurveto{\pgfqpoint{3.932882in}{1.950946in}}{\pgfqpoint{3.936155in}{1.958846in}}{\pgfqpoint{3.936155in}{1.967082in}}%
\pgfpathcurveto{\pgfqpoint{3.936155in}{1.975318in}}{\pgfqpoint{3.932882in}{1.983218in}}{\pgfqpoint{3.927058in}{1.989042in}}%
\pgfpathcurveto{\pgfqpoint{3.921234in}{1.994866in}}{\pgfqpoint{3.913334in}{1.998138in}}{\pgfqpoint{3.905098in}{1.998138in}}%
\pgfpathcurveto{\pgfqpoint{3.896862in}{1.998138in}}{\pgfqpoint{3.888962in}{1.994866in}}{\pgfqpoint{3.883138in}{1.989042in}}%
\pgfpathcurveto{\pgfqpoint{3.877314in}{1.983218in}}{\pgfqpoint{3.874042in}{1.975318in}}{\pgfqpoint{3.874042in}{1.967082in}}%
\pgfpathcurveto{\pgfqpoint{3.874042in}{1.958846in}}{\pgfqpoint{3.877314in}{1.950946in}}{\pgfqpoint{3.883138in}{1.945122in}}%
\pgfpathcurveto{\pgfqpoint{3.888962in}{1.939298in}}{\pgfqpoint{3.896862in}{1.936025in}}{\pgfqpoint{3.905098in}{1.936025in}}%
\pgfpathclose%
\pgfusepath{stroke,fill}%
\end{pgfscope}%
\begin{pgfscope}%
\pgfpathrectangle{\pgfqpoint{3.793912in}{0.557870in}}{\pgfqpoint{2.446088in}{1.484734in}}%
\pgfusepath{clip}%
\pgfsetbuttcap%
\pgfsetroundjoin%
\definecolor{currentfill}{rgb}{0.298039,0.447059,0.690196}%
\pgfsetfillcolor{currentfill}%
\pgfsetlinewidth{1.003750pt}%
\definecolor{currentstroke}{rgb}{0.298039,0.447059,0.690196}%
\pgfsetstrokecolor{currentstroke}%
\pgfsetdash{}{0pt}%
\pgfpathmoveto{\pgfqpoint{3.905098in}{1.936025in}}%
\pgfpathcurveto{\pgfqpoint{3.913334in}{1.936025in}}{\pgfqpoint{3.921234in}{1.939298in}}{\pgfqpoint{3.927058in}{1.945122in}}%
\pgfpathcurveto{\pgfqpoint{3.932882in}{1.950946in}}{\pgfqpoint{3.936155in}{1.958846in}}{\pgfqpoint{3.936155in}{1.967082in}}%
\pgfpathcurveto{\pgfqpoint{3.936155in}{1.975318in}}{\pgfqpoint{3.932882in}{1.983218in}}{\pgfqpoint{3.927058in}{1.989042in}}%
\pgfpathcurveto{\pgfqpoint{3.921234in}{1.994866in}}{\pgfqpoint{3.913334in}{1.998138in}}{\pgfqpoint{3.905098in}{1.998138in}}%
\pgfpathcurveto{\pgfqpoint{3.896862in}{1.998138in}}{\pgfqpoint{3.888962in}{1.994866in}}{\pgfqpoint{3.883138in}{1.989042in}}%
\pgfpathcurveto{\pgfqpoint{3.877314in}{1.983218in}}{\pgfqpoint{3.874042in}{1.975318in}}{\pgfqpoint{3.874042in}{1.967082in}}%
\pgfpathcurveto{\pgfqpoint{3.874042in}{1.958846in}}{\pgfqpoint{3.877314in}{1.950946in}}{\pgfqpoint{3.883138in}{1.945122in}}%
\pgfpathcurveto{\pgfqpoint{3.888962in}{1.939298in}}{\pgfqpoint{3.896862in}{1.936025in}}{\pgfqpoint{3.905098in}{1.936025in}}%
\pgfpathclose%
\pgfusepath{stroke,fill}%
\end{pgfscope}%
\begin{pgfscope}%
\pgfpathrectangle{\pgfqpoint{3.793912in}{0.557870in}}{\pgfqpoint{2.446088in}{1.484734in}}%
\pgfusepath{clip}%
\pgfsetbuttcap%
\pgfsetroundjoin%
\definecolor{currentfill}{rgb}{0.298039,0.447059,0.690196}%
\pgfsetfillcolor{currentfill}%
\pgfsetlinewidth{1.003750pt}%
\definecolor{currentstroke}{rgb}{0.298039,0.447059,0.690196}%
\pgfsetstrokecolor{currentstroke}%
\pgfsetdash{}{0pt}%
\pgfpathmoveto{\pgfqpoint{3.905098in}{1.936025in}}%
\pgfpathcurveto{\pgfqpoint{3.913334in}{1.936025in}}{\pgfqpoint{3.921234in}{1.939298in}}{\pgfqpoint{3.927058in}{1.945122in}}%
\pgfpathcurveto{\pgfqpoint{3.932882in}{1.950946in}}{\pgfqpoint{3.936155in}{1.958846in}}{\pgfqpoint{3.936155in}{1.967082in}}%
\pgfpathcurveto{\pgfqpoint{3.936155in}{1.975318in}}{\pgfqpoint{3.932882in}{1.983218in}}{\pgfqpoint{3.927058in}{1.989042in}}%
\pgfpathcurveto{\pgfqpoint{3.921234in}{1.994866in}}{\pgfqpoint{3.913334in}{1.998138in}}{\pgfqpoint{3.905098in}{1.998138in}}%
\pgfpathcurveto{\pgfqpoint{3.896862in}{1.998138in}}{\pgfqpoint{3.888962in}{1.994866in}}{\pgfqpoint{3.883138in}{1.989042in}}%
\pgfpathcurveto{\pgfqpoint{3.877314in}{1.983218in}}{\pgfqpoint{3.874042in}{1.975318in}}{\pgfqpoint{3.874042in}{1.967082in}}%
\pgfpathcurveto{\pgfqpoint{3.874042in}{1.958846in}}{\pgfqpoint{3.877314in}{1.950946in}}{\pgfqpoint{3.883138in}{1.945122in}}%
\pgfpathcurveto{\pgfqpoint{3.888962in}{1.939298in}}{\pgfqpoint{3.896862in}{1.936025in}}{\pgfqpoint{3.905098in}{1.936025in}}%
\pgfpathclose%
\pgfusepath{stroke,fill}%
\end{pgfscope}%
\begin{pgfscope}%
\pgfpathrectangle{\pgfqpoint{3.793912in}{0.557870in}}{\pgfqpoint{2.446088in}{1.484734in}}%
\pgfusepath{clip}%
\pgfsetbuttcap%
\pgfsetroundjoin%
\definecolor{currentfill}{rgb}{0.298039,0.447059,0.690196}%
\pgfsetfillcolor{currentfill}%
\pgfsetlinewidth{1.003750pt}%
\definecolor{currentstroke}{rgb}{0.298039,0.447059,0.690196}%
\pgfsetstrokecolor{currentstroke}%
\pgfsetdash{}{0pt}%
\pgfpathmoveto{\pgfqpoint{3.905098in}{1.936025in}}%
\pgfpathcurveto{\pgfqpoint{3.913334in}{1.936025in}}{\pgfqpoint{3.921234in}{1.939298in}}{\pgfqpoint{3.927058in}{1.945122in}}%
\pgfpathcurveto{\pgfqpoint{3.932882in}{1.950946in}}{\pgfqpoint{3.936155in}{1.958846in}}{\pgfqpoint{3.936155in}{1.967082in}}%
\pgfpathcurveto{\pgfqpoint{3.936155in}{1.975318in}}{\pgfqpoint{3.932882in}{1.983218in}}{\pgfqpoint{3.927058in}{1.989042in}}%
\pgfpathcurveto{\pgfqpoint{3.921234in}{1.994866in}}{\pgfqpoint{3.913334in}{1.998138in}}{\pgfqpoint{3.905098in}{1.998138in}}%
\pgfpathcurveto{\pgfqpoint{3.896862in}{1.998138in}}{\pgfqpoint{3.888962in}{1.994866in}}{\pgfqpoint{3.883138in}{1.989042in}}%
\pgfpathcurveto{\pgfqpoint{3.877314in}{1.983218in}}{\pgfqpoint{3.874042in}{1.975318in}}{\pgfqpoint{3.874042in}{1.967082in}}%
\pgfpathcurveto{\pgfqpoint{3.874042in}{1.958846in}}{\pgfqpoint{3.877314in}{1.950946in}}{\pgfqpoint{3.883138in}{1.945122in}}%
\pgfpathcurveto{\pgfqpoint{3.888962in}{1.939298in}}{\pgfqpoint{3.896862in}{1.936025in}}{\pgfqpoint{3.905098in}{1.936025in}}%
\pgfpathclose%
\pgfusepath{stroke,fill}%
\end{pgfscope}%
\begin{pgfscope}%
\pgfpathrectangle{\pgfqpoint{3.793912in}{0.557870in}}{\pgfqpoint{2.446088in}{1.484734in}}%
\pgfusepath{clip}%
\pgfsetbuttcap%
\pgfsetroundjoin%
\definecolor{currentfill}{rgb}{0.298039,0.447059,0.690196}%
\pgfsetfillcolor{currentfill}%
\pgfsetlinewidth{1.003750pt}%
\definecolor{currentstroke}{rgb}{0.298039,0.447059,0.690196}%
\pgfsetstrokecolor{currentstroke}%
\pgfsetdash{}{0pt}%
\pgfpathmoveto{\pgfqpoint{3.905098in}{1.245078in}}%
\pgfpathcurveto{\pgfqpoint{3.913334in}{1.245078in}}{\pgfqpoint{3.921234in}{1.248350in}}{\pgfqpoint{3.927058in}{1.254174in}}%
\pgfpathcurveto{\pgfqpoint{3.932882in}{1.259998in}}{\pgfqpoint{3.936155in}{1.267898in}}{\pgfqpoint{3.936155in}{1.276134in}}%
\pgfpathcurveto{\pgfqpoint{3.936155in}{1.284371in}}{\pgfqpoint{3.932882in}{1.292271in}}{\pgfqpoint{3.927058in}{1.298095in}}%
\pgfpathcurveto{\pgfqpoint{3.921234in}{1.303919in}}{\pgfqpoint{3.913334in}{1.307191in}}{\pgfqpoint{3.905098in}{1.307191in}}%
\pgfpathcurveto{\pgfqpoint{3.896862in}{1.307191in}}{\pgfqpoint{3.888962in}{1.303919in}}{\pgfqpoint{3.883138in}{1.298095in}}%
\pgfpathcurveto{\pgfqpoint{3.877314in}{1.292271in}}{\pgfqpoint{3.874042in}{1.284371in}}{\pgfqpoint{3.874042in}{1.276134in}}%
\pgfpathcurveto{\pgfqpoint{3.874042in}{1.267898in}}{\pgfqpoint{3.877314in}{1.259998in}}{\pgfqpoint{3.883138in}{1.254174in}}%
\pgfpathcurveto{\pgfqpoint{3.888962in}{1.248350in}}{\pgfqpoint{3.896862in}{1.245078in}}{\pgfqpoint{3.905098in}{1.245078in}}%
\pgfpathclose%
\pgfusepath{stroke,fill}%
\end{pgfscope}%
\begin{pgfscope}%
\pgfpathrectangle{\pgfqpoint{3.793912in}{0.557870in}}{\pgfqpoint{2.446088in}{1.484734in}}%
\pgfusepath{clip}%
\pgfsetbuttcap%
\pgfsetroundjoin%
\definecolor{currentfill}{rgb}{0.298039,0.447059,0.690196}%
\pgfsetfillcolor{currentfill}%
\pgfsetlinewidth{1.003750pt}%
\definecolor{currentstroke}{rgb}{0.298039,0.447059,0.690196}%
\pgfsetstrokecolor{currentstroke}%
\pgfsetdash{}{0pt}%
\pgfpathmoveto{\pgfqpoint{3.905098in}{1.936025in}}%
\pgfpathcurveto{\pgfqpoint{3.913334in}{1.936025in}}{\pgfqpoint{3.921234in}{1.939298in}}{\pgfqpoint{3.927058in}{1.945122in}}%
\pgfpathcurveto{\pgfqpoint{3.932882in}{1.950946in}}{\pgfqpoint{3.936155in}{1.958846in}}{\pgfqpoint{3.936155in}{1.967082in}}%
\pgfpathcurveto{\pgfqpoint{3.936155in}{1.975318in}}{\pgfqpoint{3.932882in}{1.983218in}}{\pgfqpoint{3.927058in}{1.989042in}}%
\pgfpathcurveto{\pgfqpoint{3.921234in}{1.994866in}}{\pgfqpoint{3.913334in}{1.998138in}}{\pgfqpoint{3.905098in}{1.998138in}}%
\pgfpathcurveto{\pgfqpoint{3.896862in}{1.998138in}}{\pgfqpoint{3.888962in}{1.994866in}}{\pgfqpoint{3.883138in}{1.989042in}}%
\pgfpathcurveto{\pgfqpoint{3.877314in}{1.983218in}}{\pgfqpoint{3.874042in}{1.975318in}}{\pgfqpoint{3.874042in}{1.967082in}}%
\pgfpathcurveto{\pgfqpoint{3.874042in}{1.958846in}}{\pgfqpoint{3.877314in}{1.950946in}}{\pgfqpoint{3.883138in}{1.945122in}}%
\pgfpathcurveto{\pgfqpoint{3.888962in}{1.939298in}}{\pgfqpoint{3.896862in}{1.936025in}}{\pgfqpoint{3.905098in}{1.936025in}}%
\pgfpathclose%
\pgfusepath{stroke,fill}%
\end{pgfscope}%
\begin{pgfscope}%
\pgfpathrectangle{\pgfqpoint{3.793912in}{0.557870in}}{\pgfqpoint{2.446088in}{1.484734in}}%
\pgfusepath{clip}%
\pgfsetbuttcap%
\pgfsetroundjoin%
\definecolor{currentfill}{rgb}{0.298039,0.447059,0.690196}%
\pgfsetfillcolor{currentfill}%
\pgfsetlinewidth{1.003750pt}%
\definecolor{currentstroke}{rgb}{0.298039,0.447059,0.690196}%
\pgfsetstrokecolor{currentstroke}%
\pgfsetdash{}{0pt}%
\pgfpathmoveto{\pgfqpoint{3.905098in}{1.936025in}}%
\pgfpathcurveto{\pgfqpoint{3.913334in}{1.936025in}}{\pgfqpoint{3.921234in}{1.939298in}}{\pgfqpoint{3.927058in}{1.945122in}}%
\pgfpathcurveto{\pgfqpoint{3.932882in}{1.950946in}}{\pgfqpoint{3.936155in}{1.958846in}}{\pgfqpoint{3.936155in}{1.967082in}}%
\pgfpathcurveto{\pgfqpoint{3.936155in}{1.975318in}}{\pgfqpoint{3.932882in}{1.983218in}}{\pgfqpoint{3.927058in}{1.989042in}}%
\pgfpathcurveto{\pgfqpoint{3.921234in}{1.994866in}}{\pgfqpoint{3.913334in}{1.998138in}}{\pgfqpoint{3.905098in}{1.998138in}}%
\pgfpathcurveto{\pgfqpoint{3.896862in}{1.998138in}}{\pgfqpoint{3.888962in}{1.994866in}}{\pgfqpoint{3.883138in}{1.989042in}}%
\pgfpathcurveto{\pgfqpoint{3.877314in}{1.983218in}}{\pgfqpoint{3.874042in}{1.975318in}}{\pgfqpoint{3.874042in}{1.967082in}}%
\pgfpathcurveto{\pgfqpoint{3.874042in}{1.958846in}}{\pgfqpoint{3.877314in}{1.950946in}}{\pgfqpoint{3.883138in}{1.945122in}}%
\pgfpathcurveto{\pgfqpoint{3.888962in}{1.939298in}}{\pgfqpoint{3.896862in}{1.936025in}}{\pgfqpoint{3.905098in}{1.936025in}}%
\pgfpathclose%
\pgfusepath{stroke,fill}%
\end{pgfscope}%
\begin{pgfscope}%
\pgfpathrectangle{\pgfqpoint{3.793912in}{0.557870in}}{\pgfqpoint{2.446088in}{1.484734in}}%
\pgfusepath{clip}%
\pgfsetbuttcap%
\pgfsetroundjoin%
\definecolor{currentfill}{rgb}{0.298039,0.447059,0.690196}%
\pgfsetfillcolor{currentfill}%
\pgfsetlinewidth{1.003750pt}%
\definecolor{currentstroke}{rgb}{0.298039,0.447059,0.690196}%
\pgfsetstrokecolor{currentstroke}%
\pgfsetdash{}{0pt}%
\pgfpathmoveto{\pgfqpoint{3.905098in}{1.936025in}}%
\pgfpathcurveto{\pgfqpoint{3.913334in}{1.936025in}}{\pgfqpoint{3.921234in}{1.939298in}}{\pgfqpoint{3.927058in}{1.945122in}}%
\pgfpathcurveto{\pgfqpoint{3.932882in}{1.950946in}}{\pgfqpoint{3.936155in}{1.958846in}}{\pgfqpoint{3.936155in}{1.967082in}}%
\pgfpathcurveto{\pgfqpoint{3.936155in}{1.975318in}}{\pgfqpoint{3.932882in}{1.983218in}}{\pgfqpoint{3.927058in}{1.989042in}}%
\pgfpathcurveto{\pgfqpoint{3.921234in}{1.994866in}}{\pgfqpoint{3.913334in}{1.998138in}}{\pgfqpoint{3.905098in}{1.998138in}}%
\pgfpathcurveto{\pgfqpoint{3.896862in}{1.998138in}}{\pgfqpoint{3.888962in}{1.994866in}}{\pgfqpoint{3.883138in}{1.989042in}}%
\pgfpathcurveto{\pgfqpoint{3.877314in}{1.983218in}}{\pgfqpoint{3.874042in}{1.975318in}}{\pgfqpoint{3.874042in}{1.967082in}}%
\pgfpathcurveto{\pgfqpoint{3.874042in}{1.958846in}}{\pgfqpoint{3.877314in}{1.950946in}}{\pgfqpoint{3.883138in}{1.945122in}}%
\pgfpathcurveto{\pgfqpoint{3.888962in}{1.939298in}}{\pgfqpoint{3.896862in}{1.936025in}}{\pgfqpoint{3.905098in}{1.936025in}}%
\pgfpathclose%
\pgfusepath{stroke,fill}%
\end{pgfscope}%
\begin{pgfscope}%
\pgfpathrectangle{\pgfqpoint{3.793912in}{0.557870in}}{\pgfqpoint{2.446088in}{1.484734in}}%
\pgfusepath{clip}%
\pgfsetbuttcap%
\pgfsetroundjoin%
\definecolor{currentfill}{rgb}{0.298039,0.447059,0.690196}%
\pgfsetfillcolor{currentfill}%
\pgfsetlinewidth{1.003750pt}%
\definecolor{currentstroke}{rgb}{0.298039,0.447059,0.690196}%
\pgfsetstrokecolor{currentstroke}%
\pgfsetdash{}{0pt}%
\pgfpathmoveto{\pgfqpoint{3.905098in}{1.936025in}}%
\pgfpathcurveto{\pgfqpoint{3.913334in}{1.936025in}}{\pgfqpoint{3.921234in}{1.939298in}}{\pgfqpoint{3.927058in}{1.945122in}}%
\pgfpathcurveto{\pgfqpoint{3.932882in}{1.950946in}}{\pgfqpoint{3.936155in}{1.958846in}}{\pgfqpoint{3.936155in}{1.967082in}}%
\pgfpathcurveto{\pgfqpoint{3.936155in}{1.975318in}}{\pgfqpoint{3.932882in}{1.983218in}}{\pgfqpoint{3.927058in}{1.989042in}}%
\pgfpathcurveto{\pgfqpoint{3.921234in}{1.994866in}}{\pgfqpoint{3.913334in}{1.998138in}}{\pgfqpoint{3.905098in}{1.998138in}}%
\pgfpathcurveto{\pgfqpoint{3.896862in}{1.998138in}}{\pgfqpoint{3.888962in}{1.994866in}}{\pgfqpoint{3.883138in}{1.989042in}}%
\pgfpathcurveto{\pgfqpoint{3.877314in}{1.983218in}}{\pgfqpoint{3.874042in}{1.975318in}}{\pgfqpoint{3.874042in}{1.967082in}}%
\pgfpathcurveto{\pgfqpoint{3.874042in}{1.958846in}}{\pgfqpoint{3.877314in}{1.950946in}}{\pgfqpoint{3.883138in}{1.945122in}}%
\pgfpathcurveto{\pgfqpoint{3.888962in}{1.939298in}}{\pgfqpoint{3.896862in}{1.936025in}}{\pgfqpoint{3.905098in}{1.936025in}}%
\pgfpathclose%
\pgfusepath{stroke,fill}%
\end{pgfscope}%
\begin{pgfscope}%
\pgfpathrectangle{\pgfqpoint{3.793912in}{0.557870in}}{\pgfqpoint{2.446088in}{1.484734in}}%
\pgfusepath{clip}%
\pgfsetbuttcap%
\pgfsetroundjoin%
\definecolor{currentfill}{rgb}{0.298039,0.447059,0.690196}%
\pgfsetfillcolor{currentfill}%
\pgfsetlinewidth{1.003750pt}%
\definecolor{currentstroke}{rgb}{0.298039,0.447059,0.690196}%
\pgfsetstrokecolor{currentstroke}%
\pgfsetdash{}{0pt}%
\pgfpathmoveto{\pgfqpoint{3.905098in}{1.245078in}}%
\pgfpathcurveto{\pgfqpoint{3.913334in}{1.245078in}}{\pgfqpoint{3.921234in}{1.248350in}}{\pgfqpoint{3.927058in}{1.254174in}}%
\pgfpathcurveto{\pgfqpoint{3.932882in}{1.259998in}}{\pgfqpoint{3.936155in}{1.267898in}}{\pgfqpoint{3.936155in}{1.276134in}}%
\pgfpathcurveto{\pgfqpoint{3.936155in}{1.284371in}}{\pgfqpoint{3.932882in}{1.292271in}}{\pgfqpoint{3.927058in}{1.298095in}}%
\pgfpathcurveto{\pgfqpoint{3.921234in}{1.303919in}}{\pgfqpoint{3.913334in}{1.307191in}}{\pgfqpoint{3.905098in}{1.307191in}}%
\pgfpathcurveto{\pgfqpoint{3.896862in}{1.307191in}}{\pgfqpoint{3.888962in}{1.303919in}}{\pgfqpoint{3.883138in}{1.298095in}}%
\pgfpathcurveto{\pgfqpoint{3.877314in}{1.292271in}}{\pgfqpoint{3.874042in}{1.284371in}}{\pgfqpoint{3.874042in}{1.276134in}}%
\pgfpathcurveto{\pgfqpoint{3.874042in}{1.267898in}}{\pgfqpoint{3.877314in}{1.259998in}}{\pgfqpoint{3.883138in}{1.254174in}}%
\pgfpathcurveto{\pgfqpoint{3.888962in}{1.248350in}}{\pgfqpoint{3.896862in}{1.245078in}}{\pgfqpoint{3.905098in}{1.245078in}}%
\pgfpathclose%
\pgfusepath{stroke,fill}%
\end{pgfscope}%
\begin{pgfscope}%
\pgfpathrectangle{\pgfqpoint{3.793912in}{0.557870in}}{\pgfqpoint{2.446088in}{1.484734in}}%
\pgfusepath{clip}%
\pgfsetbuttcap%
\pgfsetroundjoin%
\definecolor{currentfill}{rgb}{0.298039,0.447059,0.690196}%
\pgfsetfillcolor{currentfill}%
\pgfsetlinewidth{1.003750pt}%
\definecolor{currentstroke}{rgb}{0.298039,0.447059,0.690196}%
\pgfsetstrokecolor{currentstroke}%
\pgfsetdash{}{0pt}%
\pgfpathmoveto{\pgfqpoint{3.905098in}{1.196872in}}%
\pgfpathcurveto{\pgfqpoint{3.913334in}{1.196872in}}{\pgfqpoint{3.921234in}{1.200145in}}{\pgfqpoint{3.927058in}{1.205968in}}%
\pgfpathcurveto{\pgfqpoint{3.932882in}{1.211792in}}{\pgfqpoint{3.936155in}{1.219692in}}{\pgfqpoint{3.936155in}{1.227929in}}%
\pgfpathcurveto{\pgfqpoint{3.936155in}{1.236165in}}{\pgfqpoint{3.932882in}{1.244065in}}{\pgfqpoint{3.927058in}{1.249889in}}%
\pgfpathcurveto{\pgfqpoint{3.921234in}{1.255713in}}{\pgfqpoint{3.913334in}{1.258985in}}{\pgfqpoint{3.905098in}{1.258985in}}%
\pgfpathcurveto{\pgfqpoint{3.896862in}{1.258985in}}{\pgfqpoint{3.888962in}{1.255713in}}{\pgfqpoint{3.883138in}{1.249889in}}%
\pgfpathcurveto{\pgfqpoint{3.877314in}{1.244065in}}{\pgfqpoint{3.874042in}{1.236165in}}{\pgfqpoint{3.874042in}{1.227929in}}%
\pgfpathcurveto{\pgfqpoint{3.874042in}{1.219692in}}{\pgfqpoint{3.877314in}{1.211792in}}{\pgfqpoint{3.883138in}{1.205968in}}%
\pgfpathcurveto{\pgfqpoint{3.888962in}{1.200145in}}{\pgfqpoint{3.896862in}{1.196872in}}{\pgfqpoint{3.905098in}{1.196872in}}%
\pgfpathclose%
\pgfusepath{stroke,fill}%
\end{pgfscope}%
\begin{pgfscope}%
\pgfpathrectangle{\pgfqpoint{3.793912in}{0.557870in}}{\pgfqpoint{2.446088in}{1.484734in}}%
\pgfusepath{clip}%
\pgfsetbuttcap%
\pgfsetroundjoin%
\definecolor{currentfill}{rgb}{0.298039,0.447059,0.690196}%
\pgfsetfillcolor{currentfill}%
\pgfsetlinewidth{1.003750pt}%
\definecolor{currentstroke}{rgb}{0.298039,0.447059,0.690196}%
\pgfsetstrokecolor{currentstroke}%
\pgfsetdash{}{0pt}%
\pgfpathmoveto{\pgfqpoint{3.905098in}{1.156701in}}%
\pgfpathcurveto{\pgfqpoint{3.913334in}{1.156701in}}{\pgfqpoint{3.921234in}{1.159973in}}{\pgfqpoint{3.927058in}{1.165797in}}%
\pgfpathcurveto{\pgfqpoint{3.932882in}{1.171621in}}{\pgfqpoint{3.936155in}{1.179521in}}{\pgfqpoint{3.936155in}{1.187757in}}%
\pgfpathcurveto{\pgfqpoint{3.936155in}{1.195994in}}{\pgfqpoint{3.932882in}{1.203894in}}{\pgfqpoint{3.927058in}{1.209718in}}%
\pgfpathcurveto{\pgfqpoint{3.921234in}{1.215542in}}{\pgfqpoint{3.913334in}{1.218814in}}{\pgfqpoint{3.905098in}{1.218814in}}%
\pgfpathcurveto{\pgfqpoint{3.896862in}{1.218814in}}{\pgfqpoint{3.888962in}{1.215542in}}{\pgfqpoint{3.883138in}{1.209718in}}%
\pgfpathcurveto{\pgfqpoint{3.877314in}{1.203894in}}{\pgfqpoint{3.874042in}{1.195994in}}{\pgfqpoint{3.874042in}{1.187757in}}%
\pgfpathcurveto{\pgfqpoint{3.874042in}{1.179521in}}{\pgfqpoint{3.877314in}{1.171621in}}{\pgfqpoint{3.883138in}{1.165797in}}%
\pgfpathcurveto{\pgfqpoint{3.888962in}{1.159973in}}{\pgfqpoint{3.896862in}{1.156701in}}{\pgfqpoint{3.905098in}{1.156701in}}%
\pgfpathclose%
\pgfusepath{stroke,fill}%
\end{pgfscope}%
\begin{pgfscope}%
\pgfpathrectangle{\pgfqpoint{3.793912in}{0.557870in}}{\pgfqpoint{2.446088in}{1.484734in}}%
\pgfusepath{clip}%
\pgfsetbuttcap%
\pgfsetroundjoin%
\definecolor{currentfill}{rgb}{0.298039,0.447059,0.690196}%
\pgfsetfillcolor{currentfill}%
\pgfsetlinewidth{1.003750pt}%
\definecolor{currentstroke}{rgb}{0.298039,0.447059,0.690196}%
\pgfsetstrokecolor{currentstroke}%
\pgfsetdash{}{0pt}%
\pgfpathmoveto{\pgfqpoint{3.905098in}{1.936025in}}%
\pgfpathcurveto{\pgfqpoint{3.913334in}{1.936025in}}{\pgfqpoint{3.921234in}{1.939298in}}{\pgfqpoint{3.927058in}{1.945122in}}%
\pgfpathcurveto{\pgfqpoint{3.932882in}{1.950946in}}{\pgfqpoint{3.936155in}{1.958846in}}{\pgfqpoint{3.936155in}{1.967082in}}%
\pgfpathcurveto{\pgfqpoint{3.936155in}{1.975318in}}{\pgfqpoint{3.932882in}{1.983218in}}{\pgfqpoint{3.927058in}{1.989042in}}%
\pgfpathcurveto{\pgfqpoint{3.921234in}{1.994866in}}{\pgfqpoint{3.913334in}{1.998138in}}{\pgfqpoint{3.905098in}{1.998138in}}%
\pgfpathcurveto{\pgfqpoint{3.896862in}{1.998138in}}{\pgfqpoint{3.888962in}{1.994866in}}{\pgfqpoint{3.883138in}{1.989042in}}%
\pgfpathcurveto{\pgfqpoint{3.877314in}{1.983218in}}{\pgfqpoint{3.874042in}{1.975318in}}{\pgfqpoint{3.874042in}{1.967082in}}%
\pgfpathcurveto{\pgfqpoint{3.874042in}{1.958846in}}{\pgfqpoint{3.877314in}{1.950946in}}{\pgfqpoint{3.883138in}{1.945122in}}%
\pgfpathcurveto{\pgfqpoint{3.888962in}{1.939298in}}{\pgfqpoint{3.896862in}{1.936025in}}{\pgfqpoint{3.905098in}{1.936025in}}%
\pgfpathclose%
\pgfusepath{stroke,fill}%
\end{pgfscope}%
\begin{pgfscope}%
\pgfpathrectangle{\pgfqpoint{3.793912in}{0.557870in}}{\pgfqpoint{2.446088in}{1.484734in}}%
\pgfusepath{clip}%
\pgfsetbuttcap%
\pgfsetroundjoin%
\definecolor{currentfill}{rgb}{0.298039,0.447059,0.690196}%
\pgfsetfillcolor{currentfill}%
\pgfsetlinewidth{1.003750pt}%
\definecolor{currentstroke}{rgb}{0.298039,0.447059,0.690196}%
\pgfsetstrokecolor{currentstroke}%
\pgfsetdash{}{0pt}%
\pgfpathmoveto{\pgfqpoint{3.905098in}{1.936025in}}%
\pgfpathcurveto{\pgfqpoint{3.913334in}{1.936025in}}{\pgfqpoint{3.921234in}{1.939298in}}{\pgfqpoint{3.927058in}{1.945122in}}%
\pgfpathcurveto{\pgfqpoint{3.932882in}{1.950946in}}{\pgfqpoint{3.936155in}{1.958846in}}{\pgfqpoint{3.936155in}{1.967082in}}%
\pgfpathcurveto{\pgfqpoint{3.936155in}{1.975318in}}{\pgfqpoint{3.932882in}{1.983218in}}{\pgfqpoint{3.927058in}{1.989042in}}%
\pgfpathcurveto{\pgfqpoint{3.921234in}{1.994866in}}{\pgfqpoint{3.913334in}{1.998138in}}{\pgfqpoint{3.905098in}{1.998138in}}%
\pgfpathcurveto{\pgfqpoint{3.896862in}{1.998138in}}{\pgfqpoint{3.888962in}{1.994866in}}{\pgfqpoint{3.883138in}{1.989042in}}%
\pgfpathcurveto{\pgfqpoint{3.877314in}{1.983218in}}{\pgfqpoint{3.874042in}{1.975318in}}{\pgfqpoint{3.874042in}{1.967082in}}%
\pgfpathcurveto{\pgfqpoint{3.874042in}{1.958846in}}{\pgfqpoint{3.877314in}{1.950946in}}{\pgfqpoint{3.883138in}{1.945122in}}%
\pgfpathcurveto{\pgfqpoint{3.888962in}{1.939298in}}{\pgfqpoint{3.896862in}{1.936025in}}{\pgfqpoint{3.905098in}{1.936025in}}%
\pgfpathclose%
\pgfusepath{stroke,fill}%
\end{pgfscope}%
\begin{pgfscope}%
\pgfpathrectangle{\pgfqpoint{3.793912in}{0.557870in}}{\pgfqpoint{2.446088in}{1.484734in}}%
\pgfusepath{clip}%
\pgfsetbuttcap%
\pgfsetroundjoin%
\definecolor{currentfill}{rgb}{0.298039,0.447059,0.690196}%
\pgfsetfillcolor{currentfill}%
\pgfsetlinewidth{1.003750pt}%
\definecolor{currentstroke}{rgb}{0.298039,0.447059,0.690196}%
\pgfsetstrokecolor{currentstroke}%
\pgfsetdash{}{0pt}%
\pgfpathmoveto{\pgfqpoint{3.905098in}{1.333455in}}%
\pgfpathcurveto{\pgfqpoint{3.913334in}{1.333455in}}{\pgfqpoint{3.921234in}{1.336727in}}{\pgfqpoint{3.927058in}{1.342551in}}%
\pgfpathcurveto{\pgfqpoint{3.932882in}{1.348375in}}{\pgfqpoint{3.936155in}{1.356275in}}{\pgfqpoint{3.936155in}{1.364511in}}%
\pgfpathcurveto{\pgfqpoint{3.936155in}{1.372748in}}{\pgfqpoint{3.932882in}{1.380648in}}{\pgfqpoint{3.927058in}{1.386472in}}%
\pgfpathcurveto{\pgfqpoint{3.921234in}{1.392296in}}{\pgfqpoint{3.913334in}{1.395568in}}{\pgfqpoint{3.905098in}{1.395568in}}%
\pgfpathcurveto{\pgfqpoint{3.896862in}{1.395568in}}{\pgfqpoint{3.888962in}{1.392296in}}{\pgfqpoint{3.883138in}{1.386472in}}%
\pgfpathcurveto{\pgfqpoint{3.877314in}{1.380648in}}{\pgfqpoint{3.874042in}{1.372748in}}{\pgfqpoint{3.874042in}{1.364511in}}%
\pgfpathcurveto{\pgfqpoint{3.874042in}{1.356275in}}{\pgfqpoint{3.877314in}{1.348375in}}{\pgfqpoint{3.883138in}{1.342551in}}%
\pgfpathcurveto{\pgfqpoint{3.888962in}{1.336727in}}{\pgfqpoint{3.896862in}{1.333455in}}{\pgfqpoint{3.905098in}{1.333455in}}%
\pgfpathclose%
\pgfusepath{stroke,fill}%
\end{pgfscope}%
\begin{pgfscope}%
\pgfpathrectangle{\pgfqpoint{3.793912in}{0.557870in}}{\pgfqpoint{2.446088in}{1.484734in}}%
\pgfusepath{clip}%
\pgfsetbuttcap%
\pgfsetroundjoin%
\definecolor{currentfill}{rgb}{0.298039,0.447059,0.690196}%
\pgfsetfillcolor{currentfill}%
\pgfsetlinewidth{1.003750pt}%
\definecolor{currentstroke}{rgb}{0.298039,0.447059,0.690196}%
\pgfsetstrokecolor{currentstroke}%
\pgfsetdash{}{0pt}%
\pgfpathmoveto{\pgfqpoint{3.905098in}{1.936025in}}%
\pgfpathcurveto{\pgfqpoint{3.913334in}{1.936025in}}{\pgfqpoint{3.921234in}{1.939298in}}{\pgfqpoint{3.927058in}{1.945122in}}%
\pgfpathcurveto{\pgfqpoint{3.932882in}{1.950946in}}{\pgfqpoint{3.936155in}{1.958846in}}{\pgfqpoint{3.936155in}{1.967082in}}%
\pgfpathcurveto{\pgfqpoint{3.936155in}{1.975318in}}{\pgfqpoint{3.932882in}{1.983218in}}{\pgfqpoint{3.927058in}{1.989042in}}%
\pgfpathcurveto{\pgfqpoint{3.921234in}{1.994866in}}{\pgfqpoint{3.913334in}{1.998138in}}{\pgfqpoint{3.905098in}{1.998138in}}%
\pgfpathcurveto{\pgfqpoint{3.896862in}{1.998138in}}{\pgfqpoint{3.888962in}{1.994866in}}{\pgfqpoint{3.883138in}{1.989042in}}%
\pgfpathcurveto{\pgfqpoint{3.877314in}{1.983218in}}{\pgfqpoint{3.874042in}{1.975318in}}{\pgfqpoint{3.874042in}{1.967082in}}%
\pgfpathcurveto{\pgfqpoint{3.874042in}{1.958846in}}{\pgfqpoint{3.877314in}{1.950946in}}{\pgfqpoint{3.883138in}{1.945122in}}%
\pgfpathcurveto{\pgfqpoint{3.888962in}{1.939298in}}{\pgfqpoint{3.896862in}{1.936025in}}{\pgfqpoint{3.905098in}{1.936025in}}%
\pgfpathclose%
\pgfusepath{stroke,fill}%
\end{pgfscope}%
\begin{pgfscope}%
\pgfpathrectangle{\pgfqpoint{3.793912in}{0.557870in}}{\pgfqpoint{2.446088in}{1.484734in}}%
\pgfusepath{clip}%
\pgfsetbuttcap%
\pgfsetroundjoin%
\definecolor{currentfill}{rgb}{0.298039,0.447059,0.690196}%
\pgfsetfillcolor{currentfill}%
\pgfsetlinewidth{1.003750pt}%
\definecolor{currentstroke}{rgb}{0.298039,0.447059,0.690196}%
\pgfsetstrokecolor{currentstroke}%
\pgfsetdash{}{0pt}%
\pgfpathmoveto{\pgfqpoint{3.905098in}{1.084392in}}%
\pgfpathcurveto{\pgfqpoint{3.913334in}{1.084392in}}{\pgfqpoint{3.921234in}{1.087665in}}{\pgfqpoint{3.927058in}{1.093489in}}%
\pgfpathcurveto{\pgfqpoint{3.932882in}{1.099313in}}{\pgfqpoint{3.936155in}{1.107213in}}{\pgfqpoint{3.936155in}{1.115449in}}%
\pgfpathcurveto{\pgfqpoint{3.936155in}{1.123685in}}{\pgfqpoint{3.932882in}{1.131585in}}{\pgfqpoint{3.927058in}{1.137409in}}%
\pgfpathcurveto{\pgfqpoint{3.921234in}{1.143233in}}{\pgfqpoint{3.913334in}{1.146505in}}{\pgfqpoint{3.905098in}{1.146505in}}%
\pgfpathcurveto{\pgfqpoint{3.896862in}{1.146505in}}{\pgfqpoint{3.888962in}{1.143233in}}{\pgfqpoint{3.883138in}{1.137409in}}%
\pgfpathcurveto{\pgfqpoint{3.877314in}{1.131585in}}{\pgfqpoint{3.874042in}{1.123685in}}{\pgfqpoint{3.874042in}{1.115449in}}%
\pgfpathcurveto{\pgfqpoint{3.874042in}{1.107213in}}{\pgfqpoint{3.877314in}{1.099313in}}{\pgfqpoint{3.883138in}{1.093489in}}%
\pgfpathcurveto{\pgfqpoint{3.888962in}{1.087665in}}{\pgfqpoint{3.896862in}{1.084392in}}{\pgfqpoint{3.905098in}{1.084392in}}%
\pgfpathclose%
\pgfusepath{stroke,fill}%
\end{pgfscope}%
\begin{pgfscope}%
\pgfpathrectangle{\pgfqpoint{3.793912in}{0.557870in}}{\pgfqpoint{2.446088in}{1.484734in}}%
\pgfusepath{clip}%
\pgfsetbuttcap%
\pgfsetroundjoin%
\definecolor{currentfill}{rgb}{0.298039,0.447059,0.690196}%
\pgfsetfillcolor{currentfill}%
\pgfsetlinewidth{1.003750pt}%
\definecolor{currentstroke}{rgb}{0.298039,0.447059,0.690196}%
\pgfsetstrokecolor{currentstroke}%
\pgfsetdash{}{0pt}%
\pgfpathmoveto{\pgfqpoint{3.905098in}{1.936025in}}%
\pgfpathcurveto{\pgfqpoint{3.913334in}{1.936025in}}{\pgfqpoint{3.921234in}{1.939298in}}{\pgfqpoint{3.927058in}{1.945122in}}%
\pgfpathcurveto{\pgfqpoint{3.932882in}{1.950946in}}{\pgfqpoint{3.936155in}{1.958846in}}{\pgfqpoint{3.936155in}{1.967082in}}%
\pgfpathcurveto{\pgfqpoint{3.936155in}{1.975318in}}{\pgfqpoint{3.932882in}{1.983218in}}{\pgfqpoint{3.927058in}{1.989042in}}%
\pgfpathcurveto{\pgfqpoint{3.921234in}{1.994866in}}{\pgfqpoint{3.913334in}{1.998138in}}{\pgfqpoint{3.905098in}{1.998138in}}%
\pgfpathcurveto{\pgfqpoint{3.896862in}{1.998138in}}{\pgfqpoint{3.888962in}{1.994866in}}{\pgfqpoint{3.883138in}{1.989042in}}%
\pgfpathcurveto{\pgfqpoint{3.877314in}{1.983218in}}{\pgfqpoint{3.874042in}{1.975318in}}{\pgfqpoint{3.874042in}{1.967082in}}%
\pgfpathcurveto{\pgfqpoint{3.874042in}{1.958846in}}{\pgfqpoint{3.877314in}{1.950946in}}{\pgfqpoint{3.883138in}{1.945122in}}%
\pgfpathcurveto{\pgfqpoint{3.888962in}{1.939298in}}{\pgfqpoint{3.896862in}{1.936025in}}{\pgfqpoint{3.905098in}{1.936025in}}%
\pgfpathclose%
\pgfusepath{stroke,fill}%
\end{pgfscope}%
\begin{pgfscope}%
\pgfpathrectangle{\pgfqpoint{3.793912in}{0.557870in}}{\pgfqpoint{2.446088in}{1.484734in}}%
\pgfusepath{clip}%
\pgfsetbuttcap%
\pgfsetroundjoin%
\definecolor{currentfill}{rgb}{0.298039,0.447059,0.690196}%
\pgfsetfillcolor{currentfill}%
\pgfsetlinewidth{1.003750pt}%
\definecolor{currentstroke}{rgb}{0.298039,0.447059,0.690196}%
\pgfsetstrokecolor{currentstroke}%
\pgfsetdash{}{0pt}%
\pgfpathmoveto{\pgfqpoint{3.905098in}{1.936025in}}%
\pgfpathcurveto{\pgfqpoint{3.913334in}{1.936025in}}{\pgfqpoint{3.921234in}{1.939298in}}{\pgfqpoint{3.927058in}{1.945122in}}%
\pgfpathcurveto{\pgfqpoint{3.932882in}{1.950946in}}{\pgfqpoint{3.936155in}{1.958846in}}{\pgfqpoint{3.936155in}{1.967082in}}%
\pgfpathcurveto{\pgfqpoint{3.936155in}{1.975318in}}{\pgfqpoint{3.932882in}{1.983218in}}{\pgfqpoint{3.927058in}{1.989042in}}%
\pgfpathcurveto{\pgfqpoint{3.921234in}{1.994866in}}{\pgfqpoint{3.913334in}{1.998138in}}{\pgfqpoint{3.905098in}{1.998138in}}%
\pgfpathcurveto{\pgfqpoint{3.896862in}{1.998138in}}{\pgfqpoint{3.888962in}{1.994866in}}{\pgfqpoint{3.883138in}{1.989042in}}%
\pgfpathcurveto{\pgfqpoint{3.877314in}{1.983218in}}{\pgfqpoint{3.874042in}{1.975318in}}{\pgfqpoint{3.874042in}{1.967082in}}%
\pgfpathcurveto{\pgfqpoint{3.874042in}{1.958846in}}{\pgfqpoint{3.877314in}{1.950946in}}{\pgfqpoint{3.883138in}{1.945122in}}%
\pgfpathcurveto{\pgfqpoint{3.888962in}{1.939298in}}{\pgfqpoint{3.896862in}{1.936025in}}{\pgfqpoint{3.905098in}{1.936025in}}%
\pgfpathclose%
\pgfusepath{stroke,fill}%
\end{pgfscope}%
\begin{pgfscope}%
\pgfpathrectangle{\pgfqpoint{3.793912in}{0.557870in}}{\pgfqpoint{2.446088in}{1.484734in}}%
\pgfusepath{clip}%
\pgfsetbuttcap%
\pgfsetroundjoin%
\definecolor{currentfill}{rgb}{0.298039,0.447059,0.690196}%
\pgfsetfillcolor{currentfill}%
\pgfsetlinewidth{1.003750pt}%
\definecolor{currentstroke}{rgb}{0.298039,0.447059,0.690196}%
\pgfsetstrokecolor{currentstroke}%
\pgfsetdash{}{0pt}%
\pgfpathmoveto{\pgfqpoint{3.905098in}{1.936025in}}%
\pgfpathcurveto{\pgfqpoint{3.913334in}{1.936025in}}{\pgfqpoint{3.921234in}{1.939298in}}{\pgfqpoint{3.927058in}{1.945122in}}%
\pgfpathcurveto{\pgfqpoint{3.932882in}{1.950946in}}{\pgfqpoint{3.936155in}{1.958846in}}{\pgfqpoint{3.936155in}{1.967082in}}%
\pgfpathcurveto{\pgfqpoint{3.936155in}{1.975318in}}{\pgfqpoint{3.932882in}{1.983218in}}{\pgfqpoint{3.927058in}{1.989042in}}%
\pgfpathcurveto{\pgfqpoint{3.921234in}{1.994866in}}{\pgfqpoint{3.913334in}{1.998138in}}{\pgfqpoint{3.905098in}{1.998138in}}%
\pgfpathcurveto{\pgfqpoint{3.896862in}{1.998138in}}{\pgfqpoint{3.888962in}{1.994866in}}{\pgfqpoint{3.883138in}{1.989042in}}%
\pgfpathcurveto{\pgfqpoint{3.877314in}{1.983218in}}{\pgfqpoint{3.874042in}{1.975318in}}{\pgfqpoint{3.874042in}{1.967082in}}%
\pgfpathcurveto{\pgfqpoint{3.874042in}{1.958846in}}{\pgfqpoint{3.877314in}{1.950946in}}{\pgfqpoint{3.883138in}{1.945122in}}%
\pgfpathcurveto{\pgfqpoint{3.888962in}{1.939298in}}{\pgfqpoint{3.896862in}{1.936025in}}{\pgfqpoint{3.905098in}{1.936025in}}%
\pgfpathclose%
\pgfusepath{stroke,fill}%
\end{pgfscope}%
\begin{pgfscope}%
\pgfpathrectangle{\pgfqpoint{3.793912in}{0.557870in}}{\pgfqpoint{2.446088in}{1.484734in}}%
\pgfusepath{clip}%
\pgfsetbuttcap%
\pgfsetroundjoin%
\definecolor{currentfill}{rgb}{0.298039,0.447059,0.690196}%
\pgfsetfillcolor{currentfill}%
\pgfsetlinewidth{1.003750pt}%
\definecolor{currentstroke}{rgb}{0.298039,0.447059,0.690196}%
\pgfsetstrokecolor{currentstroke}%
\pgfsetdash{}{0pt}%
\pgfpathmoveto{\pgfqpoint{3.905098in}{0.883536in}}%
\pgfpathcurveto{\pgfqpoint{3.913334in}{0.883536in}}{\pgfqpoint{3.921234in}{0.886808in}}{\pgfqpoint{3.927058in}{0.892632in}}%
\pgfpathcurveto{\pgfqpoint{3.932882in}{0.898456in}}{\pgfqpoint{3.936155in}{0.906356in}}{\pgfqpoint{3.936155in}{0.914592in}}%
\pgfpathcurveto{\pgfqpoint{3.936155in}{0.922828in}}{\pgfqpoint{3.932882in}{0.930728in}}{\pgfqpoint{3.927058in}{0.936552in}}%
\pgfpathcurveto{\pgfqpoint{3.921234in}{0.942376in}}{\pgfqpoint{3.913334in}{0.945649in}}{\pgfqpoint{3.905098in}{0.945649in}}%
\pgfpathcurveto{\pgfqpoint{3.896862in}{0.945649in}}{\pgfqpoint{3.888962in}{0.942376in}}{\pgfqpoint{3.883138in}{0.936552in}}%
\pgfpathcurveto{\pgfqpoint{3.877314in}{0.930728in}}{\pgfqpoint{3.874042in}{0.922828in}}{\pgfqpoint{3.874042in}{0.914592in}}%
\pgfpathcurveto{\pgfqpoint{3.874042in}{0.906356in}}{\pgfqpoint{3.877314in}{0.898456in}}{\pgfqpoint{3.883138in}{0.892632in}}%
\pgfpathcurveto{\pgfqpoint{3.888962in}{0.886808in}}{\pgfqpoint{3.896862in}{0.883536in}}{\pgfqpoint{3.905098in}{0.883536in}}%
\pgfpathclose%
\pgfusepath{stroke,fill}%
\end{pgfscope}%
\begin{pgfscope}%
\pgfpathrectangle{\pgfqpoint{3.793912in}{0.557870in}}{\pgfqpoint{2.446088in}{1.484734in}}%
\pgfusepath{clip}%
\pgfsetbuttcap%
\pgfsetroundjoin%
\definecolor{currentfill}{rgb}{0.298039,0.447059,0.690196}%
\pgfsetfillcolor{currentfill}%
\pgfsetlinewidth{1.003750pt}%
\definecolor{currentstroke}{rgb}{0.298039,0.447059,0.690196}%
\pgfsetstrokecolor{currentstroke}%
\pgfsetdash{}{0pt}%
\pgfpathmoveto{\pgfqpoint{3.905098in}{1.936025in}}%
\pgfpathcurveto{\pgfqpoint{3.913334in}{1.936025in}}{\pgfqpoint{3.921234in}{1.939298in}}{\pgfqpoint{3.927058in}{1.945122in}}%
\pgfpathcurveto{\pgfqpoint{3.932882in}{1.950946in}}{\pgfqpoint{3.936155in}{1.958846in}}{\pgfqpoint{3.936155in}{1.967082in}}%
\pgfpathcurveto{\pgfqpoint{3.936155in}{1.975318in}}{\pgfqpoint{3.932882in}{1.983218in}}{\pgfqpoint{3.927058in}{1.989042in}}%
\pgfpathcurveto{\pgfqpoint{3.921234in}{1.994866in}}{\pgfqpoint{3.913334in}{1.998138in}}{\pgfqpoint{3.905098in}{1.998138in}}%
\pgfpathcurveto{\pgfqpoint{3.896862in}{1.998138in}}{\pgfqpoint{3.888962in}{1.994866in}}{\pgfqpoint{3.883138in}{1.989042in}}%
\pgfpathcurveto{\pgfqpoint{3.877314in}{1.983218in}}{\pgfqpoint{3.874042in}{1.975318in}}{\pgfqpoint{3.874042in}{1.967082in}}%
\pgfpathcurveto{\pgfqpoint{3.874042in}{1.958846in}}{\pgfqpoint{3.877314in}{1.950946in}}{\pgfqpoint{3.883138in}{1.945122in}}%
\pgfpathcurveto{\pgfqpoint{3.888962in}{1.939298in}}{\pgfqpoint{3.896862in}{1.936025in}}{\pgfqpoint{3.905098in}{1.936025in}}%
\pgfpathclose%
\pgfusepath{stroke,fill}%
\end{pgfscope}%
\begin{pgfscope}%
\pgfpathrectangle{\pgfqpoint{3.793912in}{0.557870in}}{\pgfqpoint{2.446088in}{1.484734in}}%
\pgfusepath{clip}%
\pgfsetbuttcap%
\pgfsetroundjoin%
\definecolor{currentfill}{rgb}{0.298039,0.447059,0.690196}%
\pgfsetfillcolor{currentfill}%
\pgfsetlinewidth{1.003750pt}%
\definecolor{currentstroke}{rgb}{0.298039,0.447059,0.690196}%
\pgfsetstrokecolor{currentstroke}%
\pgfsetdash{}{0pt}%
\pgfpathmoveto{\pgfqpoint{3.905098in}{0.931741in}}%
\pgfpathcurveto{\pgfqpoint{3.913334in}{0.931741in}}{\pgfqpoint{3.921234in}{0.935014in}}{\pgfqpoint{3.927058in}{0.940837in}}%
\pgfpathcurveto{\pgfqpoint{3.932882in}{0.946661in}}{\pgfqpoint{3.936155in}{0.954561in}}{\pgfqpoint{3.936155in}{0.962798in}}%
\pgfpathcurveto{\pgfqpoint{3.936155in}{0.971034in}}{\pgfqpoint{3.932882in}{0.978934in}}{\pgfqpoint{3.927058in}{0.984758in}}%
\pgfpathcurveto{\pgfqpoint{3.921234in}{0.990582in}}{\pgfqpoint{3.913334in}{0.993854in}}{\pgfqpoint{3.905098in}{0.993854in}}%
\pgfpathcurveto{\pgfqpoint{3.896862in}{0.993854in}}{\pgfqpoint{3.888962in}{0.990582in}}{\pgfqpoint{3.883138in}{0.984758in}}%
\pgfpathcurveto{\pgfqpoint{3.877314in}{0.978934in}}{\pgfqpoint{3.874042in}{0.971034in}}{\pgfqpoint{3.874042in}{0.962798in}}%
\pgfpathcurveto{\pgfqpoint{3.874042in}{0.954561in}}{\pgfqpoint{3.877314in}{0.946661in}}{\pgfqpoint{3.883138in}{0.940837in}}%
\pgfpathcurveto{\pgfqpoint{3.888962in}{0.935014in}}{\pgfqpoint{3.896862in}{0.931741in}}{\pgfqpoint{3.905098in}{0.931741in}}%
\pgfpathclose%
\pgfusepath{stroke,fill}%
\end{pgfscope}%
\begin{pgfscope}%
\pgfpathrectangle{\pgfqpoint{3.793912in}{0.557870in}}{\pgfqpoint{2.446088in}{1.484734in}}%
\pgfusepath{clip}%
\pgfsetbuttcap%
\pgfsetroundjoin%
\definecolor{currentfill}{rgb}{0.298039,0.447059,0.690196}%
\pgfsetfillcolor{currentfill}%
\pgfsetlinewidth{1.003750pt}%
\definecolor{currentstroke}{rgb}{0.298039,0.447059,0.690196}%
\pgfsetstrokecolor{currentstroke}%
\pgfsetdash{}{0pt}%
\pgfpathmoveto{\pgfqpoint{3.905098in}{1.936025in}}%
\pgfpathcurveto{\pgfqpoint{3.913334in}{1.936025in}}{\pgfqpoint{3.921234in}{1.939298in}}{\pgfqpoint{3.927058in}{1.945122in}}%
\pgfpathcurveto{\pgfqpoint{3.932882in}{1.950946in}}{\pgfqpoint{3.936155in}{1.958846in}}{\pgfqpoint{3.936155in}{1.967082in}}%
\pgfpathcurveto{\pgfqpoint{3.936155in}{1.975318in}}{\pgfqpoint{3.932882in}{1.983218in}}{\pgfqpoint{3.927058in}{1.989042in}}%
\pgfpathcurveto{\pgfqpoint{3.921234in}{1.994866in}}{\pgfqpoint{3.913334in}{1.998138in}}{\pgfqpoint{3.905098in}{1.998138in}}%
\pgfpathcurveto{\pgfqpoint{3.896862in}{1.998138in}}{\pgfqpoint{3.888962in}{1.994866in}}{\pgfqpoint{3.883138in}{1.989042in}}%
\pgfpathcurveto{\pgfqpoint{3.877314in}{1.983218in}}{\pgfqpoint{3.874042in}{1.975318in}}{\pgfqpoint{3.874042in}{1.967082in}}%
\pgfpathcurveto{\pgfqpoint{3.874042in}{1.958846in}}{\pgfqpoint{3.877314in}{1.950946in}}{\pgfqpoint{3.883138in}{1.945122in}}%
\pgfpathcurveto{\pgfqpoint{3.888962in}{1.939298in}}{\pgfqpoint{3.896862in}{1.936025in}}{\pgfqpoint{3.905098in}{1.936025in}}%
\pgfpathclose%
\pgfusepath{stroke,fill}%
\end{pgfscope}%
\begin{pgfscope}%
\pgfpathrectangle{\pgfqpoint{3.793912in}{0.557870in}}{\pgfqpoint{2.446088in}{1.484734in}}%
\pgfusepath{clip}%
\pgfsetbuttcap%
\pgfsetroundjoin%
\definecolor{currentfill}{rgb}{0.298039,0.447059,0.690196}%
\pgfsetfillcolor{currentfill}%
\pgfsetlinewidth{1.003750pt}%
\definecolor{currentstroke}{rgb}{0.298039,0.447059,0.690196}%
\pgfsetstrokecolor{currentstroke}%
\pgfsetdash{}{0pt}%
\pgfpathmoveto{\pgfqpoint{3.905098in}{1.630723in}}%
\pgfpathcurveto{\pgfqpoint{3.913334in}{1.630723in}}{\pgfqpoint{3.921234in}{1.633995in}}{\pgfqpoint{3.927058in}{1.639819in}}%
\pgfpathcurveto{\pgfqpoint{3.932882in}{1.645643in}}{\pgfqpoint{3.936155in}{1.653543in}}{\pgfqpoint{3.936155in}{1.661780in}}%
\pgfpathcurveto{\pgfqpoint{3.936155in}{1.670016in}}{\pgfqpoint{3.932882in}{1.677916in}}{\pgfqpoint{3.927058in}{1.683740in}}%
\pgfpathcurveto{\pgfqpoint{3.921234in}{1.689564in}}{\pgfqpoint{3.913334in}{1.692836in}}{\pgfqpoint{3.905098in}{1.692836in}}%
\pgfpathcurveto{\pgfqpoint{3.896862in}{1.692836in}}{\pgfqpoint{3.888962in}{1.689564in}}{\pgfqpoint{3.883138in}{1.683740in}}%
\pgfpathcurveto{\pgfqpoint{3.877314in}{1.677916in}}{\pgfqpoint{3.874042in}{1.670016in}}{\pgfqpoint{3.874042in}{1.661780in}}%
\pgfpathcurveto{\pgfqpoint{3.874042in}{1.653543in}}{\pgfqpoint{3.877314in}{1.645643in}}{\pgfqpoint{3.883138in}{1.639819in}}%
\pgfpathcurveto{\pgfqpoint{3.888962in}{1.633995in}}{\pgfqpoint{3.896862in}{1.630723in}}{\pgfqpoint{3.905098in}{1.630723in}}%
\pgfpathclose%
\pgfusepath{stroke,fill}%
\end{pgfscope}%
\begin{pgfscope}%
\pgfpathrectangle{\pgfqpoint{3.793912in}{0.557870in}}{\pgfqpoint{2.446088in}{1.484734in}}%
\pgfusepath{clip}%
\pgfsetbuttcap%
\pgfsetroundjoin%
\definecolor{currentfill}{rgb}{0.298039,0.447059,0.690196}%
\pgfsetfillcolor{currentfill}%
\pgfsetlinewidth{1.003750pt}%
\definecolor{currentstroke}{rgb}{0.298039,0.447059,0.690196}%
\pgfsetstrokecolor{currentstroke}%
\pgfsetdash{}{0pt}%
\pgfpathmoveto{\pgfqpoint{3.905098in}{1.936025in}}%
\pgfpathcurveto{\pgfqpoint{3.913334in}{1.936025in}}{\pgfqpoint{3.921234in}{1.939298in}}{\pgfqpoint{3.927058in}{1.945122in}}%
\pgfpathcurveto{\pgfqpoint{3.932882in}{1.950946in}}{\pgfqpoint{3.936155in}{1.958846in}}{\pgfqpoint{3.936155in}{1.967082in}}%
\pgfpathcurveto{\pgfqpoint{3.936155in}{1.975318in}}{\pgfqpoint{3.932882in}{1.983218in}}{\pgfqpoint{3.927058in}{1.989042in}}%
\pgfpathcurveto{\pgfqpoint{3.921234in}{1.994866in}}{\pgfqpoint{3.913334in}{1.998138in}}{\pgfqpoint{3.905098in}{1.998138in}}%
\pgfpathcurveto{\pgfqpoint{3.896862in}{1.998138in}}{\pgfqpoint{3.888962in}{1.994866in}}{\pgfqpoint{3.883138in}{1.989042in}}%
\pgfpathcurveto{\pgfqpoint{3.877314in}{1.983218in}}{\pgfqpoint{3.874042in}{1.975318in}}{\pgfqpoint{3.874042in}{1.967082in}}%
\pgfpathcurveto{\pgfqpoint{3.874042in}{1.958846in}}{\pgfqpoint{3.877314in}{1.950946in}}{\pgfqpoint{3.883138in}{1.945122in}}%
\pgfpathcurveto{\pgfqpoint{3.888962in}{1.939298in}}{\pgfqpoint{3.896862in}{1.936025in}}{\pgfqpoint{3.905098in}{1.936025in}}%
\pgfpathclose%
\pgfusepath{stroke,fill}%
\end{pgfscope}%
\begin{pgfscope}%
\pgfpathrectangle{\pgfqpoint{3.793912in}{0.557870in}}{\pgfqpoint{2.446088in}{1.484734in}}%
\pgfusepath{clip}%
\pgfsetbuttcap%
\pgfsetroundjoin%
\definecolor{currentfill}{rgb}{0.298039,0.447059,0.690196}%
\pgfsetfillcolor{currentfill}%
\pgfsetlinewidth{1.003750pt}%
\definecolor{currentstroke}{rgb}{0.298039,0.447059,0.690196}%
\pgfsetstrokecolor{currentstroke}%
\pgfsetdash{}{0pt}%
\pgfpathmoveto{\pgfqpoint{3.905098in}{0.618405in}}%
\pgfpathcurveto{\pgfqpoint{3.913334in}{0.618405in}}{\pgfqpoint{3.921234in}{0.621677in}}{\pgfqpoint{3.927058in}{0.627501in}}%
\pgfpathcurveto{\pgfqpoint{3.932882in}{0.633325in}}{\pgfqpoint{3.936155in}{0.641225in}}{\pgfqpoint{3.936155in}{0.649461in}}%
\pgfpathcurveto{\pgfqpoint{3.936155in}{0.657697in}}{\pgfqpoint{3.932882in}{0.665597in}}{\pgfqpoint{3.927058in}{0.671421in}}%
\pgfpathcurveto{\pgfqpoint{3.921234in}{0.677245in}}{\pgfqpoint{3.913334in}{0.680518in}}{\pgfqpoint{3.905098in}{0.680518in}}%
\pgfpathcurveto{\pgfqpoint{3.896862in}{0.680518in}}{\pgfqpoint{3.888962in}{0.677245in}}{\pgfqpoint{3.883138in}{0.671421in}}%
\pgfpathcurveto{\pgfqpoint{3.877314in}{0.665597in}}{\pgfqpoint{3.874042in}{0.657697in}}{\pgfqpoint{3.874042in}{0.649461in}}%
\pgfpathcurveto{\pgfqpoint{3.874042in}{0.641225in}}{\pgfqpoint{3.877314in}{0.633325in}}{\pgfqpoint{3.883138in}{0.627501in}}%
\pgfpathcurveto{\pgfqpoint{3.888962in}{0.621677in}}{\pgfqpoint{3.896862in}{0.618405in}}{\pgfqpoint{3.905098in}{0.618405in}}%
\pgfpathclose%
\pgfusepath{stroke,fill}%
\end{pgfscope}%
\begin{pgfscope}%
\pgfpathrectangle{\pgfqpoint{3.793912in}{0.557870in}}{\pgfqpoint{2.446088in}{1.484734in}}%
\pgfusepath{clip}%
\pgfsetbuttcap%
\pgfsetroundjoin%
\definecolor{currentfill}{rgb}{0.298039,0.447059,0.690196}%
\pgfsetfillcolor{currentfill}%
\pgfsetlinewidth{1.003750pt}%
\definecolor{currentstroke}{rgb}{0.298039,0.447059,0.690196}%
\pgfsetstrokecolor{currentstroke}%
\pgfsetdash{}{0pt}%
\pgfpathmoveto{\pgfqpoint{3.905098in}{1.936025in}}%
\pgfpathcurveto{\pgfqpoint{3.913334in}{1.936025in}}{\pgfqpoint{3.921234in}{1.939298in}}{\pgfqpoint{3.927058in}{1.945122in}}%
\pgfpathcurveto{\pgfqpoint{3.932882in}{1.950946in}}{\pgfqpoint{3.936155in}{1.958846in}}{\pgfqpoint{3.936155in}{1.967082in}}%
\pgfpathcurveto{\pgfqpoint{3.936155in}{1.975318in}}{\pgfqpoint{3.932882in}{1.983218in}}{\pgfqpoint{3.927058in}{1.989042in}}%
\pgfpathcurveto{\pgfqpoint{3.921234in}{1.994866in}}{\pgfqpoint{3.913334in}{1.998138in}}{\pgfqpoint{3.905098in}{1.998138in}}%
\pgfpathcurveto{\pgfqpoint{3.896862in}{1.998138in}}{\pgfqpoint{3.888962in}{1.994866in}}{\pgfqpoint{3.883138in}{1.989042in}}%
\pgfpathcurveto{\pgfqpoint{3.877314in}{1.983218in}}{\pgfqpoint{3.874042in}{1.975318in}}{\pgfqpoint{3.874042in}{1.967082in}}%
\pgfpathcurveto{\pgfqpoint{3.874042in}{1.958846in}}{\pgfqpoint{3.877314in}{1.950946in}}{\pgfqpoint{3.883138in}{1.945122in}}%
\pgfpathcurveto{\pgfqpoint{3.888962in}{1.939298in}}{\pgfqpoint{3.896862in}{1.936025in}}{\pgfqpoint{3.905098in}{1.936025in}}%
\pgfpathclose%
\pgfusepath{stroke,fill}%
\end{pgfscope}%
\begin{pgfscope}%
\pgfpathrectangle{\pgfqpoint{3.793912in}{0.557870in}}{\pgfqpoint{2.446088in}{1.484734in}}%
\pgfusepath{clip}%
\pgfsetbuttcap%
\pgfsetroundjoin%
\definecolor{currentfill}{rgb}{0.298039,0.447059,0.690196}%
\pgfsetfillcolor{currentfill}%
\pgfsetlinewidth{1.003750pt}%
\definecolor{currentstroke}{rgb}{0.298039,0.447059,0.690196}%
\pgfsetstrokecolor{currentstroke}%
\pgfsetdash{}{0pt}%
\pgfpathmoveto{\pgfqpoint{3.905098in}{1.333455in}}%
\pgfpathcurveto{\pgfqpoint{3.913334in}{1.333455in}}{\pgfqpoint{3.921234in}{1.336727in}}{\pgfqpoint{3.927058in}{1.342551in}}%
\pgfpathcurveto{\pgfqpoint{3.932882in}{1.348375in}}{\pgfqpoint{3.936155in}{1.356275in}}{\pgfqpoint{3.936155in}{1.364511in}}%
\pgfpathcurveto{\pgfqpoint{3.936155in}{1.372748in}}{\pgfqpoint{3.932882in}{1.380648in}}{\pgfqpoint{3.927058in}{1.386472in}}%
\pgfpathcurveto{\pgfqpoint{3.921234in}{1.392296in}}{\pgfqpoint{3.913334in}{1.395568in}}{\pgfqpoint{3.905098in}{1.395568in}}%
\pgfpathcurveto{\pgfqpoint{3.896862in}{1.395568in}}{\pgfqpoint{3.888962in}{1.392296in}}{\pgfqpoint{3.883138in}{1.386472in}}%
\pgfpathcurveto{\pgfqpoint{3.877314in}{1.380648in}}{\pgfqpoint{3.874042in}{1.372748in}}{\pgfqpoint{3.874042in}{1.364511in}}%
\pgfpathcurveto{\pgfqpoint{3.874042in}{1.356275in}}{\pgfqpoint{3.877314in}{1.348375in}}{\pgfqpoint{3.883138in}{1.342551in}}%
\pgfpathcurveto{\pgfqpoint{3.888962in}{1.336727in}}{\pgfqpoint{3.896862in}{1.333455in}}{\pgfqpoint{3.905098in}{1.333455in}}%
\pgfpathclose%
\pgfusepath{stroke,fill}%
\end{pgfscope}%
\begin{pgfscope}%
\pgfpathrectangle{\pgfqpoint{3.793912in}{0.557870in}}{\pgfqpoint{2.446088in}{1.484734in}}%
\pgfusepath{clip}%
\pgfsetbuttcap%
\pgfsetroundjoin%
\definecolor{currentfill}{rgb}{0.298039,0.447059,0.690196}%
\pgfsetfillcolor{currentfill}%
\pgfsetlinewidth{1.003750pt}%
\definecolor{currentstroke}{rgb}{0.298039,0.447059,0.690196}%
\pgfsetstrokecolor{currentstroke}%
\pgfsetdash{}{0pt}%
\pgfpathmoveto{\pgfqpoint{3.905098in}{1.936025in}}%
\pgfpathcurveto{\pgfqpoint{3.913334in}{1.936025in}}{\pgfqpoint{3.921234in}{1.939298in}}{\pgfqpoint{3.927058in}{1.945122in}}%
\pgfpathcurveto{\pgfqpoint{3.932882in}{1.950946in}}{\pgfqpoint{3.936155in}{1.958846in}}{\pgfqpoint{3.936155in}{1.967082in}}%
\pgfpathcurveto{\pgfqpoint{3.936155in}{1.975318in}}{\pgfqpoint{3.932882in}{1.983218in}}{\pgfqpoint{3.927058in}{1.989042in}}%
\pgfpathcurveto{\pgfqpoint{3.921234in}{1.994866in}}{\pgfqpoint{3.913334in}{1.998138in}}{\pgfqpoint{3.905098in}{1.998138in}}%
\pgfpathcurveto{\pgfqpoint{3.896862in}{1.998138in}}{\pgfqpoint{3.888962in}{1.994866in}}{\pgfqpoint{3.883138in}{1.989042in}}%
\pgfpathcurveto{\pgfqpoint{3.877314in}{1.983218in}}{\pgfqpoint{3.874042in}{1.975318in}}{\pgfqpoint{3.874042in}{1.967082in}}%
\pgfpathcurveto{\pgfqpoint{3.874042in}{1.958846in}}{\pgfqpoint{3.877314in}{1.950946in}}{\pgfqpoint{3.883138in}{1.945122in}}%
\pgfpathcurveto{\pgfqpoint{3.888962in}{1.939298in}}{\pgfqpoint{3.896862in}{1.936025in}}{\pgfqpoint{3.905098in}{1.936025in}}%
\pgfpathclose%
\pgfusepath{stroke,fill}%
\end{pgfscope}%
\begin{pgfscope}%
\pgfpathrectangle{\pgfqpoint{3.793912in}{0.557870in}}{\pgfqpoint{2.446088in}{1.484734in}}%
\pgfusepath{clip}%
\pgfsetbuttcap%
\pgfsetroundjoin%
\definecolor{currentfill}{rgb}{0.298039,0.447059,0.690196}%
\pgfsetfillcolor{currentfill}%
\pgfsetlinewidth{1.003750pt}%
\definecolor{currentstroke}{rgb}{0.298039,0.447059,0.690196}%
\pgfsetstrokecolor{currentstroke}%
\pgfsetdash{}{0pt}%
\pgfpathmoveto{\pgfqpoint{4.063935in}{0.594302in}}%
\pgfpathcurveto{\pgfqpoint{4.072171in}{0.594302in}}{\pgfqpoint{4.080071in}{0.597574in}}{\pgfqpoint{4.085895in}{0.603398in}}%
\pgfpathcurveto{\pgfqpoint{4.091719in}{0.609222in}}{\pgfqpoint{4.094991in}{0.617122in}}{\pgfqpoint{4.094991in}{0.625358in}}%
\pgfpathcurveto{\pgfqpoint{4.094991in}{0.633594in}}{\pgfqpoint{4.091719in}{0.641495in}}{\pgfqpoint{4.085895in}{0.647318in}}%
\pgfpathcurveto{\pgfqpoint{4.080071in}{0.653142in}}{\pgfqpoint{4.072171in}{0.656415in}}{\pgfqpoint{4.063935in}{0.656415in}}%
\pgfpathcurveto{\pgfqpoint{4.055699in}{0.656415in}}{\pgfqpoint{4.047799in}{0.653142in}}{\pgfqpoint{4.041975in}{0.647318in}}%
\pgfpathcurveto{\pgfqpoint{4.036151in}{0.641495in}}{\pgfqpoint{4.032878in}{0.633594in}}{\pgfqpoint{4.032878in}{0.625358in}}%
\pgfpathcurveto{\pgfqpoint{4.032878in}{0.617122in}}{\pgfqpoint{4.036151in}{0.609222in}}{\pgfqpoint{4.041975in}{0.603398in}}%
\pgfpathcurveto{\pgfqpoint{4.047799in}{0.597574in}}{\pgfqpoint{4.055699in}{0.594302in}}{\pgfqpoint{4.063935in}{0.594302in}}%
\pgfpathclose%
\pgfusepath{stroke,fill}%
\end{pgfscope}%
\begin{pgfscope}%
\pgfpathrectangle{\pgfqpoint{3.793912in}{0.557870in}}{\pgfqpoint{2.446088in}{1.484734in}}%
\pgfusepath{clip}%
\pgfsetbuttcap%
\pgfsetroundjoin%
\definecolor{currentfill}{rgb}{0.298039,0.447059,0.690196}%
\pgfsetfillcolor{currentfill}%
\pgfsetlinewidth{1.003750pt}%
\definecolor{currentstroke}{rgb}{0.298039,0.447059,0.690196}%
\pgfsetstrokecolor{currentstroke}%
\pgfsetdash{}{0pt}%
\pgfpathmoveto{\pgfqpoint{4.063935in}{0.594302in}}%
\pgfpathcurveto{\pgfqpoint{4.072171in}{0.594302in}}{\pgfqpoint{4.080071in}{0.597574in}}{\pgfqpoint{4.085895in}{0.603398in}}%
\pgfpathcurveto{\pgfqpoint{4.091719in}{0.609222in}}{\pgfqpoint{4.094991in}{0.617122in}}{\pgfqpoint{4.094991in}{0.625358in}}%
\pgfpathcurveto{\pgfqpoint{4.094991in}{0.633594in}}{\pgfqpoint{4.091719in}{0.641495in}}{\pgfqpoint{4.085895in}{0.647318in}}%
\pgfpathcurveto{\pgfqpoint{4.080071in}{0.653142in}}{\pgfqpoint{4.072171in}{0.656415in}}{\pgfqpoint{4.063935in}{0.656415in}}%
\pgfpathcurveto{\pgfqpoint{4.055699in}{0.656415in}}{\pgfqpoint{4.047799in}{0.653142in}}{\pgfqpoint{4.041975in}{0.647318in}}%
\pgfpathcurveto{\pgfqpoint{4.036151in}{0.641495in}}{\pgfqpoint{4.032878in}{0.633594in}}{\pgfqpoint{4.032878in}{0.625358in}}%
\pgfpathcurveto{\pgfqpoint{4.032878in}{0.617122in}}{\pgfqpoint{4.036151in}{0.609222in}}{\pgfqpoint{4.041975in}{0.603398in}}%
\pgfpathcurveto{\pgfqpoint{4.047799in}{0.597574in}}{\pgfqpoint{4.055699in}{0.594302in}}{\pgfqpoint{4.063935in}{0.594302in}}%
\pgfpathclose%
\pgfusepath{stroke,fill}%
\end{pgfscope}%
\begin{pgfscope}%
\pgfpathrectangle{\pgfqpoint{3.793912in}{0.557870in}}{\pgfqpoint{2.446088in}{1.484734in}}%
\pgfusepath{clip}%
\pgfsetbuttcap%
\pgfsetroundjoin%
\definecolor{currentfill}{rgb}{0.298039,0.447059,0.690196}%
\pgfsetfillcolor{currentfill}%
\pgfsetlinewidth{1.003750pt}%
\definecolor{currentstroke}{rgb}{0.298039,0.447059,0.690196}%
\pgfsetstrokecolor{currentstroke}%
\pgfsetdash{}{0pt}%
\pgfpathmoveto{\pgfqpoint{3.905098in}{1.936025in}}%
\pgfpathcurveto{\pgfqpoint{3.913334in}{1.936025in}}{\pgfqpoint{3.921234in}{1.939298in}}{\pgfqpoint{3.927058in}{1.945122in}}%
\pgfpathcurveto{\pgfqpoint{3.932882in}{1.950946in}}{\pgfqpoint{3.936155in}{1.958846in}}{\pgfqpoint{3.936155in}{1.967082in}}%
\pgfpathcurveto{\pgfqpoint{3.936155in}{1.975318in}}{\pgfqpoint{3.932882in}{1.983218in}}{\pgfqpoint{3.927058in}{1.989042in}}%
\pgfpathcurveto{\pgfqpoint{3.921234in}{1.994866in}}{\pgfqpoint{3.913334in}{1.998138in}}{\pgfqpoint{3.905098in}{1.998138in}}%
\pgfpathcurveto{\pgfqpoint{3.896862in}{1.998138in}}{\pgfqpoint{3.888962in}{1.994866in}}{\pgfqpoint{3.883138in}{1.989042in}}%
\pgfpathcurveto{\pgfqpoint{3.877314in}{1.983218in}}{\pgfqpoint{3.874042in}{1.975318in}}{\pgfqpoint{3.874042in}{1.967082in}}%
\pgfpathcurveto{\pgfqpoint{3.874042in}{1.958846in}}{\pgfqpoint{3.877314in}{1.950946in}}{\pgfqpoint{3.883138in}{1.945122in}}%
\pgfpathcurveto{\pgfqpoint{3.888962in}{1.939298in}}{\pgfqpoint{3.896862in}{1.936025in}}{\pgfqpoint{3.905098in}{1.936025in}}%
\pgfpathclose%
\pgfusepath{stroke,fill}%
\end{pgfscope}%
\begin{pgfscope}%
\pgfpathrectangle{\pgfqpoint{3.793912in}{0.557870in}}{\pgfqpoint{2.446088in}{1.484734in}}%
\pgfusepath{clip}%
\pgfsetbuttcap%
\pgfsetroundjoin%
\definecolor{currentfill}{rgb}{0.298039,0.447059,0.690196}%
\pgfsetfillcolor{currentfill}%
\pgfsetlinewidth{1.003750pt}%
\definecolor{currentstroke}{rgb}{0.298039,0.447059,0.690196}%
\pgfsetstrokecolor{currentstroke}%
\pgfsetdash{}{0pt}%
\pgfpathmoveto{\pgfqpoint{3.905098in}{1.269181in}}%
\pgfpathcurveto{\pgfqpoint{3.913334in}{1.269181in}}{\pgfqpoint{3.921234in}{1.272453in}}{\pgfqpoint{3.927058in}{1.278277in}}%
\pgfpathcurveto{\pgfqpoint{3.932882in}{1.284101in}}{\pgfqpoint{3.936155in}{1.292001in}}{\pgfqpoint{3.936155in}{1.300237in}}%
\pgfpathcurveto{\pgfqpoint{3.936155in}{1.308474in}}{\pgfqpoint{3.932882in}{1.316374in}}{\pgfqpoint{3.927058in}{1.322197in}}%
\pgfpathcurveto{\pgfqpoint{3.921234in}{1.328021in}}{\pgfqpoint{3.913334in}{1.331294in}}{\pgfqpoint{3.905098in}{1.331294in}}%
\pgfpathcurveto{\pgfqpoint{3.896862in}{1.331294in}}{\pgfqpoint{3.888962in}{1.328021in}}{\pgfqpoint{3.883138in}{1.322197in}}%
\pgfpathcurveto{\pgfqpoint{3.877314in}{1.316374in}}{\pgfqpoint{3.874042in}{1.308474in}}{\pgfqpoint{3.874042in}{1.300237in}}%
\pgfpathcurveto{\pgfqpoint{3.874042in}{1.292001in}}{\pgfqpoint{3.877314in}{1.284101in}}{\pgfqpoint{3.883138in}{1.278277in}}%
\pgfpathcurveto{\pgfqpoint{3.888962in}{1.272453in}}{\pgfqpoint{3.896862in}{1.269181in}}{\pgfqpoint{3.905098in}{1.269181in}}%
\pgfpathclose%
\pgfusepath{stroke,fill}%
\end{pgfscope}%
\begin{pgfscope}%
\pgfpathrectangle{\pgfqpoint{3.793912in}{0.557870in}}{\pgfqpoint{2.446088in}{1.484734in}}%
\pgfusepath{clip}%
\pgfsetbuttcap%
\pgfsetroundjoin%
\definecolor{currentfill}{rgb}{0.298039,0.447059,0.690196}%
\pgfsetfillcolor{currentfill}%
\pgfsetlinewidth{1.003750pt}%
\definecolor{currentstroke}{rgb}{0.298039,0.447059,0.690196}%
\pgfsetstrokecolor{currentstroke}%
\pgfsetdash{}{0pt}%
\pgfpathmoveto{\pgfqpoint{3.905098in}{1.936025in}}%
\pgfpathcurveto{\pgfqpoint{3.913334in}{1.936025in}}{\pgfqpoint{3.921234in}{1.939298in}}{\pgfqpoint{3.927058in}{1.945122in}}%
\pgfpathcurveto{\pgfqpoint{3.932882in}{1.950946in}}{\pgfqpoint{3.936155in}{1.958846in}}{\pgfqpoint{3.936155in}{1.967082in}}%
\pgfpathcurveto{\pgfqpoint{3.936155in}{1.975318in}}{\pgfqpoint{3.932882in}{1.983218in}}{\pgfqpoint{3.927058in}{1.989042in}}%
\pgfpathcurveto{\pgfqpoint{3.921234in}{1.994866in}}{\pgfqpoint{3.913334in}{1.998138in}}{\pgfqpoint{3.905098in}{1.998138in}}%
\pgfpathcurveto{\pgfqpoint{3.896862in}{1.998138in}}{\pgfqpoint{3.888962in}{1.994866in}}{\pgfqpoint{3.883138in}{1.989042in}}%
\pgfpathcurveto{\pgfqpoint{3.877314in}{1.983218in}}{\pgfqpoint{3.874042in}{1.975318in}}{\pgfqpoint{3.874042in}{1.967082in}}%
\pgfpathcurveto{\pgfqpoint{3.874042in}{1.958846in}}{\pgfqpoint{3.877314in}{1.950946in}}{\pgfqpoint{3.883138in}{1.945122in}}%
\pgfpathcurveto{\pgfqpoint{3.888962in}{1.939298in}}{\pgfqpoint{3.896862in}{1.936025in}}{\pgfqpoint{3.905098in}{1.936025in}}%
\pgfpathclose%
\pgfusepath{stroke,fill}%
\end{pgfscope}%
\begin{pgfscope}%
\pgfpathrectangle{\pgfqpoint{3.793912in}{0.557870in}}{\pgfqpoint{2.446088in}{1.484734in}}%
\pgfusepath{clip}%
\pgfsetbuttcap%
\pgfsetroundjoin%
\definecolor{currentfill}{rgb}{0.298039,0.447059,0.690196}%
\pgfsetfillcolor{currentfill}%
\pgfsetlinewidth{1.003750pt}%
\definecolor{currentstroke}{rgb}{0.298039,0.447059,0.690196}%
\pgfsetstrokecolor{currentstroke}%
\pgfsetdash{}{0pt}%
\pgfpathmoveto{\pgfqpoint{3.905098in}{1.092427in}}%
\pgfpathcurveto{\pgfqpoint{3.913334in}{1.092427in}}{\pgfqpoint{3.921234in}{1.095699in}}{\pgfqpoint{3.927058in}{1.101523in}}%
\pgfpathcurveto{\pgfqpoint{3.932882in}{1.107347in}}{\pgfqpoint{3.936155in}{1.115247in}}{\pgfqpoint{3.936155in}{1.123483in}}%
\pgfpathcurveto{\pgfqpoint{3.936155in}{1.131719in}}{\pgfqpoint{3.932882in}{1.139620in}}{\pgfqpoint{3.927058in}{1.145443in}}%
\pgfpathcurveto{\pgfqpoint{3.921234in}{1.151267in}}{\pgfqpoint{3.913334in}{1.154540in}}{\pgfqpoint{3.905098in}{1.154540in}}%
\pgfpathcurveto{\pgfqpoint{3.896862in}{1.154540in}}{\pgfqpoint{3.888962in}{1.151267in}}{\pgfqpoint{3.883138in}{1.145443in}}%
\pgfpathcurveto{\pgfqpoint{3.877314in}{1.139620in}}{\pgfqpoint{3.874042in}{1.131719in}}{\pgfqpoint{3.874042in}{1.123483in}}%
\pgfpathcurveto{\pgfqpoint{3.874042in}{1.115247in}}{\pgfqpoint{3.877314in}{1.107347in}}{\pgfqpoint{3.883138in}{1.101523in}}%
\pgfpathcurveto{\pgfqpoint{3.888962in}{1.095699in}}{\pgfqpoint{3.896862in}{1.092427in}}{\pgfqpoint{3.905098in}{1.092427in}}%
\pgfpathclose%
\pgfusepath{stroke,fill}%
\end{pgfscope}%
\begin{pgfscope}%
\pgfpathrectangle{\pgfqpoint{3.793912in}{0.557870in}}{\pgfqpoint{2.446088in}{1.484734in}}%
\pgfusepath{clip}%
\pgfsetbuttcap%
\pgfsetroundjoin%
\definecolor{currentfill}{rgb}{0.298039,0.447059,0.690196}%
\pgfsetfillcolor{currentfill}%
\pgfsetlinewidth{1.003750pt}%
\definecolor{currentstroke}{rgb}{0.298039,0.447059,0.690196}%
\pgfsetstrokecolor{currentstroke}%
\pgfsetdash{}{0pt}%
\pgfpathmoveto{\pgfqpoint{3.905098in}{1.936025in}}%
\pgfpathcurveto{\pgfqpoint{3.913334in}{1.936025in}}{\pgfqpoint{3.921234in}{1.939298in}}{\pgfqpoint{3.927058in}{1.945122in}}%
\pgfpathcurveto{\pgfqpoint{3.932882in}{1.950946in}}{\pgfqpoint{3.936155in}{1.958846in}}{\pgfqpoint{3.936155in}{1.967082in}}%
\pgfpathcurveto{\pgfqpoint{3.936155in}{1.975318in}}{\pgfqpoint{3.932882in}{1.983218in}}{\pgfqpoint{3.927058in}{1.989042in}}%
\pgfpathcurveto{\pgfqpoint{3.921234in}{1.994866in}}{\pgfqpoint{3.913334in}{1.998138in}}{\pgfqpoint{3.905098in}{1.998138in}}%
\pgfpathcurveto{\pgfqpoint{3.896862in}{1.998138in}}{\pgfqpoint{3.888962in}{1.994866in}}{\pgfqpoint{3.883138in}{1.989042in}}%
\pgfpathcurveto{\pgfqpoint{3.877314in}{1.983218in}}{\pgfqpoint{3.874042in}{1.975318in}}{\pgfqpoint{3.874042in}{1.967082in}}%
\pgfpathcurveto{\pgfqpoint{3.874042in}{1.958846in}}{\pgfqpoint{3.877314in}{1.950946in}}{\pgfqpoint{3.883138in}{1.945122in}}%
\pgfpathcurveto{\pgfqpoint{3.888962in}{1.939298in}}{\pgfqpoint{3.896862in}{1.936025in}}{\pgfqpoint{3.905098in}{1.936025in}}%
\pgfpathclose%
\pgfusepath{stroke,fill}%
\end{pgfscope}%
\begin{pgfscope}%
\pgfpathrectangle{\pgfqpoint{3.793912in}{0.557870in}}{\pgfqpoint{2.446088in}{1.484734in}}%
\pgfusepath{clip}%
\pgfsetbuttcap%
\pgfsetroundjoin%
\definecolor{currentfill}{rgb}{0.298039,0.447059,0.690196}%
\pgfsetfillcolor{currentfill}%
\pgfsetlinewidth{1.003750pt}%
\definecolor{currentstroke}{rgb}{0.298039,0.447059,0.690196}%
\pgfsetstrokecolor{currentstroke}%
\pgfsetdash{}{0pt}%
\pgfpathmoveto{\pgfqpoint{3.905098in}{1.936025in}}%
\pgfpathcurveto{\pgfqpoint{3.913334in}{1.936025in}}{\pgfqpoint{3.921234in}{1.939298in}}{\pgfqpoint{3.927058in}{1.945122in}}%
\pgfpathcurveto{\pgfqpoint{3.932882in}{1.950946in}}{\pgfqpoint{3.936155in}{1.958846in}}{\pgfqpoint{3.936155in}{1.967082in}}%
\pgfpathcurveto{\pgfqpoint{3.936155in}{1.975318in}}{\pgfqpoint{3.932882in}{1.983218in}}{\pgfqpoint{3.927058in}{1.989042in}}%
\pgfpathcurveto{\pgfqpoint{3.921234in}{1.994866in}}{\pgfqpoint{3.913334in}{1.998138in}}{\pgfqpoint{3.905098in}{1.998138in}}%
\pgfpathcurveto{\pgfqpoint{3.896862in}{1.998138in}}{\pgfqpoint{3.888962in}{1.994866in}}{\pgfqpoint{3.883138in}{1.989042in}}%
\pgfpathcurveto{\pgfqpoint{3.877314in}{1.983218in}}{\pgfqpoint{3.874042in}{1.975318in}}{\pgfqpoint{3.874042in}{1.967082in}}%
\pgfpathcurveto{\pgfqpoint{3.874042in}{1.958846in}}{\pgfqpoint{3.877314in}{1.950946in}}{\pgfqpoint{3.883138in}{1.945122in}}%
\pgfpathcurveto{\pgfqpoint{3.888962in}{1.939298in}}{\pgfqpoint{3.896862in}{1.936025in}}{\pgfqpoint{3.905098in}{1.936025in}}%
\pgfpathclose%
\pgfusepath{stroke,fill}%
\end{pgfscope}%
\begin{pgfscope}%
\pgfpathrectangle{\pgfqpoint{3.793912in}{0.557870in}}{\pgfqpoint{2.446088in}{1.484734in}}%
\pgfusepath{clip}%
\pgfsetbuttcap%
\pgfsetroundjoin%
\definecolor{currentfill}{rgb}{0.298039,0.447059,0.690196}%
\pgfsetfillcolor{currentfill}%
\pgfsetlinewidth{1.003750pt}%
\definecolor{currentstroke}{rgb}{0.298039,0.447059,0.690196}%
\pgfsetstrokecolor{currentstroke}%
\pgfsetdash{}{0pt}%
\pgfpathmoveto{\pgfqpoint{3.905098in}{1.936025in}}%
\pgfpathcurveto{\pgfqpoint{3.913334in}{1.936025in}}{\pgfqpoint{3.921234in}{1.939298in}}{\pgfqpoint{3.927058in}{1.945122in}}%
\pgfpathcurveto{\pgfqpoint{3.932882in}{1.950946in}}{\pgfqpoint{3.936155in}{1.958846in}}{\pgfqpoint{3.936155in}{1.967082in}}%
\pgfpathcurveto{\pgfqpoint{3.936155in}{1.975318in}}{\pgfqpoint{3.932882in}{1.983218in}}{\pgfqpoint{3.927058in}{1.989042in}}%
\pgfpathcurveto{\pgfqpoint{3.921234in}{1.994866in}}{\pgfqpoint{3.913334in}{1.998138in}}{\pgfqpoint{3.905098in}{1.998138in}}%
\pgfpathcurveto{\pgfqpoint{3.896862in}{1.998138in}}{\pgfqpoint{3.888962in}{1.994866in}}{\pgfqpoint{3.883138in}{1.989042in}}%
\pgfpathcurveto{\pgfqpoint{3.877314in}{1.983218in}}{\pgfqpoint{3.874042in}{1.975318in}}{\pgfqpoint{3.874042in}{1.967082in}}%
\pgfpathcurveto{\pgfqpoint{3.874042in}{1.958846in}}{\pgfqpoint{3.877314in}{1.950946in}}{\pgfqpoint{3.883138in}{1.945122in}}%
\pgfpathcurveto{\pgfqpoint{3.888962in}{1.939298in}}{\pgfqpoint{3.896862in}{1.936025in}}{\pgfqpoint{3.905098in}{1.936025in}}%
\pgfpathclose%
\pgfusepath{stroke,fill}%
\end{pgfscope}%
\begin{pgfscope}%
\pgfpathrectangle{\pgfqpoint{3.793912in}{0.557870in}}{\pgfqpoint{2.446088in}{1.484734in}}%
\pgfusepath{clip}%
\pgfsetbuttcap%
\pgfsetroundjoin%
\definecolor{currentfill}{rgb}{0.298039,0.447059,0.690196}%
\pgfsetfillcolor{currentfill}%
\pgfsetlinewidth{1.003750pt}%
\definecolor{currentstroke}{rgb}{0.298039,0.447059,0.690196}%
\pgfsetstrokecolor{currentstroke}%
\pgfsetdash{}{0pt}%
\pgfpathmoveto{\pgfqpoint{3.905098in}{1.936025in}}%
\pgfpathcurveto{\pgfqpoint{3.913334in}{1.936025in}}{\pgfqpoint{3.921234in}{1.939298in}}{\pgfqpoint{3.927058in}{1.945122in}}%
\pgfpathcurveto{\pgfqpoint{3.932882in}{1.950946in}}{\pgfqpoint{3.936155in}{1.958846in}}{\pgfqpoint{3.936155in}{1.967082in}}%
\pgfpathcurveto{\pgfqpoint{3.936155in}{1.975318in}}{\pgfqpoint{3.932882in}{1.983218in}}{\pgfqpoint{3.927058in}{1.989042in}}%
\pgfpathcurveto{\pgfqpoint{3.921234in}{1.994866in}}{\pgfqpoint{3.913334in}{1.998138in}}{\pgfqpoint{3.905098in}{1.998138in}}%
\pgfpathcurveto{\pgfqpoint{3.896862in}{1.998138in}}{\pgfqpoint{3.888962in}{1.994866in}}{\pgfqpoint{3.883138in}{1.989042in}}%
\pgfpathcurveto{\pgfqpoint{3.877314in}{1.983218in}}{\pgfqpoint{3.874042in}{1.975318in}}{\pgfqpoint{3.874042in}{1.967082in}}%
\pgfpathcurveto{\pgfqpoint{3.874042in}{1.958846in}}{\pgfqpoint{3.877314in}{1.950946in}}{\pgfqpoint{3.883138in}{1.945122in}}%
\pgfpathcurveto{\pgfqpoint{3.888962in}{1.939298in}}{\pgfqpoint{3.896862in}{1.936025in}}{\pgfqpoint{3.905098in}{1.936025in}}%
\pgfpathclose%
\pgfusepath{stroke,fill}%
\end{pgfscope}%
\begin{pgfscope}%
\pgfpathrectangle{\pgfqpoint{3.793912in}{0.557870in}}{\pgfqpoint{2.446088in}{1.484734in}}%
\pgfusepath{clip}%
\pgfsetbuttcap%
\pgfsetroundjoin%
\definecolor{currentfill}{rgb}{0.298039,0.447059,0.690196}%
\pgfsetfillcolor{currentfill}%
\pgfsetlinewidth{1.003750pt}%
\definecolor{currentstroke}{rgb}{0.298039,0.447059,0.690196}%
\pgfsetstrokecolor{currentstroke}%
\pgfsetdash{}{0pt}%
\pgfpathmoveto{\pgfqpoint{3.905098in}{1.180804in}}%
\pgfpathcurveto{\pgfqpoint{3.913334in}{1.180804in}}{\pgfqpoint{3.921234in}{1.184076in}}{\pgfqpoint{3.927058in}{1.189900in}}%
\pgfpathcurveto{\pgfqpoint{3.932882in}{1.195724in}}{\pgfqpoint{3.936155in}{1.203624in}}{\pgfqpoint{3.936155in}{1.211860in}}%
\pgfpathcurveto{\pgfqpoint{3.936155in}{1.220096in}}{\pgfqpoint{3.932882in}{1.227997in}}{\pgfqpoint{3.927058in}{1.233820in}}%
\pgfpathcurveto{\pgfqpoint{3.921234in}{1.239644in}}{\pgfqpoint{3.913334in}{1.242917in}}{\pgfqpoint{3.905098in}{1.242917in}}%
\pgfpathcurveto{\pgfqpoint{3.896862in}{1.242917in}}{\pgfqpoint{3.888962in}{1.239644in}}{\pgfqpoint{3.883138in}{1.233820in}}%
\pgfpathcurveto{\pgfqpoint{3.877314in}{1.227997in}}{\pgfqpoint{3.874042in}{1.220096in}}{\pgfqpoint{3.874042in}{1.211860in}}%
\pgfpathcurveto{\pgfqpoint{3.874042in}{1.203624in}}{\pgfqpoint{3.877314in}{1.195724in}}{\pgfqpoint{3.883138in}{1.189900in}}%
\pgfpathcurveto{\pgfqpoint{3.888962in}{1.184076in}}{\pgfqpoint{3.896862in}{1.180804in}}{\pgfqpoint{3.905098in}{1.180804in}}%
\pgfpathclose%
\pgfusepath{stroke,fill}%
\end{pgfscope}%
\begin{pgfscope}%
\pgfpathrectangle{\pgfqpoint{3.793912in}{0.557870in}}{\pgfqpoint{2.446088in}{1.484734in}}%
\pgfusepath{clip}%
\pgfsetbuttcap%
\pgfsetroundjoin%
\definecolor{currentfill}{rgb}{0.298039,0.447059,0.690196}%
\pgfsetfillcolor{currentfill}%
\pgfsetlinewidth{1.003750pt}%
\definecolor{currentstroke}{rgb}{0.298039,0.447059,0.690196}%
\pgfsetstrokecolor{currentstroke}%
\pgfsetdash{}{0pt}%
\pgfpathmoveto{\pgfqpoint{3.905098in}{1.936025in}}%
\pgfpathcurveto{\pgfqpoint{3.913334in}{1.936025in}}{\pgfqpoint{3.921234in}{1.939298in}}{\pgfqpoint{3.927058in}{1.945122in}}%
\pgfpathcurveto{\pgfqpoint{3.932882in}{1.950946in}}{\pgfqpoint{3.936155in}{1.958846in}}{\pgfqpoint{3.936155in}{1.967082in}}%
\pgfpathcurveto{\pgfqpoint{3.936155in}{1.975318in}}{\pgfqpoint{3.932882in}{1.983218in}}{\pgfqpoint{3.927058in}{1.989042in}}%
\pgfpathcurveto{\pgfqpoint{3.921234in}{1.994866in}}{\pgfqpoint{3.913334in}{1.998138in}}{\pgfqpoint{3.905098in}{1.998138in}}%
\pgfpathcurveto{\pgfqpoint{3.896862in}{1.998138in}}{\pgfqpoint{3.888962in}{1.994866in}}{\pgfqpoint{3.883138in}{1.989042in}}%
\pgfpathcurveto{\pgfqpoint{3.877314in}{1.983218in}}{\pgfqpoint{3.874042in}{1.975318in}}{\pgfqpoint{3.874042in}{1.967082in}}%
\pgfpathcurveto{\pgfqpoint{3.874042in}{1.958846in}}{\pgfqpoint{3.877314in}{1.950946in}}{\pgfqpoint{3.883138in}{1.945122in}}%
\pgfpathcurveto{\pgfqpoint{3.888962in}{1.939298in}}{\pgfqpoint{3.896862in}{1.936025in}}{\pgfqpoint{3.905098in}{1.936025in}}%
\pgfpathclose%
\pgfusepath{stroke,fill}%
\end{pgfscope}%
\begin{pgfscope}%
\pgfpathrectangle{\pgfqpoint{3.793912in}{0.557870in}}{\pgfqpoint{2.446088in}{1.484734in}}%
\pgfusepath{clip}%
\pgfsetbuttcap%
\pgfsetroundjoin%
\definecolor{currentfill}{rgb}{0.298039,0.447059,0.690196}%
\pgfsetfillcolor{currentfill}%
\pgfsetlinewidth{1.003750pt}%
\definecolor{currentstroke}{rgb}{0.298039,0.447059,0.690196}%
\pgfsetstrokecolor{currentstroke}%
\pgfsetdash{}{0pt}%
\pgfpathmoveto{\pgfqpoint{3.905098in}{1.220975in}}%
\pgfpathcurveto{\pgfqpoint{3.913334in}{1.220975in}}{\pgfqpoint{3.921234in}{1.224247in}}{\pgfqpoint{3.927058in}{1.230071in}}%
\pgfpathcurveto{\pgfqpoint{3.932882in}{1.235895in}}{\pgfqpoint{3.936155in}{1.243795in}}{\pgfqpoint{3.936155in}{1.252032in}}%
\pgfpathcurveto{\pgfqpoint{3.936155in}{1.260268in}}{\pgfqpoint{3.932882in}{1.268168in}}{\pgfqpoint{3.927058in}{1.273992in}}%
\pgfpathcurveto{\pgfqpoint{3.921234in}{1.279816in}}{\pgfqpoint{3.913334in}{1.283088in}}{\pgfqpoint{3.905098in}{1.283088in}}%
\pgfpathcurveto{\pgfqpoint{3.896862in}{1.283088in}}{\pgfqpoint{3.888962in}{1.279816in}}{\pgfqpoint{3.883138in}{1.273992in}}%
\pgfpathcurveto{\pgfqpoint{3.877314in}{1.268168in}}{\pgfqpoint{3.874042in}{1.260268in}}{\pgfqpoint{3.874042in}{1.252032in}}%
\pgfpathcurveto{\pgfqpoint{3.874042in}{1.243795in}}{\pgfqpoint{3.877314in}{1.235895in}}{\pgfqpoint{3.883138in}{1.230071in}}%
\pgfpathcurveto{\pgfqpoint{3.888962in}{1.224247in}}{\pgfqpoint{3.896862in}{1.220975in}}{\pgfqpoint{3.905098in}{1.220975in}}%
\pgfpathclose%
\pgfusepath{stroke,fill}%
\end{pgfscope}%
\begin{pgfscope}%
\pgfpathrectangle{\pgfqpoint{3.793912in}{0.557870in}}{\pgfqpoint{2.446088in}{1.484734in}}%
\pgfusepath{clip}%
\pgfsetbuttcap%
\pgfsetroundjoin%
\definecolor{currentfill}{rgb}{0.298039,0.447059,0.690196}%
\pgfsetfillcolor{currentfill}%
\pgfsetlinewidth{1.003750pt}%
\definecolor{currentstroke}{rgb}{0.298039,0.447059,0.690196}%
\pgfsetstrokecolor{currentstroke}%
\pgfsetdash{}{0pt}%
\pgfpathmoveto{\pgfqpoint{3.905098in}{1.936025in}}%
\pgfpathcurveto{\pgfqpoint{3.913334in}{1.936025in}}{\pgfqpoint{3.921234in}{1.939298in}}{\pgfqpoint{3.927058in}{1.945122in}}%
\pgfpathcurveto{\pgfqpoint{3.932882in}{1.950946in}}{\pgfqpoint{3.936155in}{1.958846in}}{\pgfqpoint{3.936155in}{1.967082in}}%
\pgfpathcurveto{\pgfqpoint{3.936155in}{1.975318in}}{\pgfqpoint{3.932882in}{1.983218in}}{\pgfqpoint{3.927058in}{1.989042in}}%
\pgfpathcurveto{\pgfqpoint{3.921234in}{1.994866in}}{\pgfqpoint{3.913334in}{1.998138in}}{\pgfqpoint{3.905098in}{1.998138in}}%
\pgfpathcurveto{\pgfqpoint{3.896862in}{1.998138in}}{\pgfqpoint{3.888962in}{1.994866in}}{\pgfqpoint{3.883138in}{1.989042in}}%
\pgfpathcurveto{\pgfqpoint{3.877314in}{1.983218in}}{\pgfqpoint{3.874042in}{1.975318in}}{\pgfqpoint{3.874042in}{1.967082in}}%
\pgfpathcurveto{\pgfqpoint{3.874042in}{1.958846in}}{\pgfqpoint{3.877314in}{1.950946in}}{\pgfqpoint{3.883138in}{1.945122in}}%
\pgfpathcurveto{\pgfqpoint{3.888962in}{1.939298in}}{\pgfqpoint{3.896862in}{1.936025in}}{\pgfqpoint{3.905098in}{1.936025in}}%
\pgfpathclose%
\pgfusepath{stroke,fill}%
\end{pgfscope}%
\begin{pgfscope}%
\pgfpathrectangle{\pgfqpoint{3.793912in}{0.557870in}}{\pgfqpoint{2.446088in}{1.484734in}}%
\pgfusepath{clip}%
\pgfsetbuttcap%
\pgfsetroundjoin%
\definecolor{currentfill}{rgb}{0.298039,0.447059,0.690196}%
\pgfsetfillcolor{currentfill}%
\pgfsetlinewidth{1.003750pt}%
\definecolor{currentstroke}{rgb}{0.298039,0.447059,0.690196}%
\pgfsetstrokecolor{currentstroke}%
\pgfsetdash{}{0pt}%
\pgfpathmoveto{\pgfqpoint{3.905098in}{1.470038in}}%
\pgfpathcurveto{\pgfqpoint{3.913334in}{1.470038in}}{\pgfqpoint{3.921234in}{1.473310in}}{\pgfqpoint{3.927058in}{1.479134in}}%
\pgfpathcurveto{\pgfqpoint{3.932882in}{1.484958in}}{\pgfqpoint{3.936155in}{1.492858in}}{\pgfqpoint{3.936155in}{1.501094in}}%
\pgfpathcurveto{\pgfqpoint{3.936155in}{1.509330in}}{\pgfqpoint{3.932882in}{1.517230in}}{\pgfqpoint{3.927058in}{1.523054in}}%
\pgfpathcurveto{\pgfqpoint{3.921234in}{1.528878in}}{\pgfqpoint{3.913334in}{1.532151in}}{\pgfqpoint{3.905098in}{1.532151in}}%
\pgfpathcurveto{\pgfqpoint{3.896862in}{1.532151in}}{\pgfqpoint{3.888962in}{1.528878in}}{\pgfqpoint{3.883138in}{1.523054in}}%
\pgfpathcurveto{\pgfqpoint{3.877314in}{1.517230in}}{\pgfqpoint{3.874042in}{1.509330in}}{\pgfqpoint{3.874042in}{1.501094in}}%
\pgfpathcurveto{\pgfqpoint{3.874042in}{1.492858in}}{\pgfqpoint{3.877314in}{1.484958in}}{\pgfqpoint{3.883138in}{1.479134in}}%
\pgfpathcurveto{\pgfqpoint{3.888962in}{1.473310in}}{\pgfqpoint{3.896862in}{1.470038in}}{\pgfqpoint{3.905098in}{1.470038in}}%
\pgfpathclose%
\pgfusepath{stroke,fill}%
\end{pgfscope}%
\begin{pgfscope}%
\pgfpathrectangle{\pgfqpoint{3.793912in}{0.557870in}}{\pgfqpoint{2.446088in}{1.484734in}}%
\pgfusepath{clip}%
\pgfsetbuttcap%
\pgfsetroundjoin%
\definecolor{currentfill}{rgb}{0.298039,0.447059,0.690196}%
\pgfsetfillcolor{currentfill}%
\pgfsetlinewidth{1.003750pt}%
\definecolor{currentstroke}{rgb}{0.298039,0.447059,0.690196}%
\pgfsetstrokecolor{currentstroke}%
\pgfsetdash{}{0pt}%
\pgfpathmoveto{\pgfqpoint{3.905098in}{1.936025in}}%
\pgfpathcurveto{\pgfqpoint{3.913334in}{1.936025in}}{\pgfqpoint{3.921234in}{1.939298in}}{\pgfqpoint{3.927058in}{1.945122in}}%
\pgfpathcurveto{\pgfqpoint{3.932882in}{1.950946in}}{\pgfqpoint{3.936155in}{1.958846in}}{\pgfqpoint{3.936155in}{1.967082in}}%
\pgfpathcurveto{\pgfqpoint{3.936155in}{1.975318in}}{\pgfqpoint{3.932882in}{1.983218in}}{\pgfqpoint{3.927058in}{1.989042in}}%
\pgfpathcurveto{\pgfqpoint{3.921234in}{1.994866in}}{\pgfqpoint{3.913334in}{1.998138in}}{\pgfqpoint{3.905098in}{1.998138in}}%
\pgfpathcurveto{\pgfqpoint{3.896862in}{1.998138in}}{\pgfqpoint{3.888962in}{1.994866in}}{\pgfqpoint{3.883138in}{1.989042in}}%
\pgfpathcurveto{\pgfqpoint{3.877314in}{1.983218in}}{\pgfqpoint{3.874042in}{1.975318in}}{\pgfqpoint{3.874042in}{1.967082in}}%
\pgfpathcurveto{\pgfqpoint{3.874042in}{1.958846in}}{\pgfqpoint{3.877314in}{1.950946in}}{\pgfqpoint{3.883138in}{1.945122in}}%
\pgfpathcurveto{\pgfqpoint{3.888962in}{1.939298in}}{\pgfqpoint{3.896862in}{1.936025in}}{\pgfqpoint{3.905098in}{1.936025in}}%
\pgfpathclose%
\pgfusepath{stroke,fill}%
\end{pgfscope}%
\begin{pgfscope}%
\pgfpathrectangle{\pgfqpoint{3.793912in}{0.557870in}}{\pgfqpoint{2.446088in}{1.484734in}}%
\pgfusepath{clip}%
\pgfsetbuttcap%
\pgfsetroundjoin%
\definecolor{currentfill}{rgb}{0.298039,0.447059,0.690196}%
\pgfsetfillcolor{currentfill}%
\pgfsetlinewidth{1.003750pt}%
\definecolor{currentstroke}{rgb}{0.298039,0.447059,0.690196}%
\pgfsetstrokecolor{currentstroke}%
\pgfsetdash{}{0pt}%
\pgfpathmoveto{\pgfqpoint{3.905098in}{1.020118in}}%
\pgfpathcurveto{\pgfqpoint{3.913334in}{1.020118in}}{\pgfqpoint{3.921234in}{1.023391in}}{\pgfqpoint{3.927058in}{1.029214in}}%
\pgfpathcurveto{\pgfqpoint{3.932882in}{1.035038in}}{\pgfqpoint{3.936155in}{1.042938in}}{\pgfqpoint{3.936155in}{1.051175in}}%
\pgfpathcurveto{\pgfqpoint{3.936155in}{1.059411in}}{\pgfqpoint{3.932882in}{1.067311in}}{\pgfqpoint{3.927058in}{1.073135in}}%
\pgfpathcurveto{\pgfqpoint{3.921234in}{1.078959in}}{\pgfqpoint{3.913334in}{1.082231in}}{\pgfqpoint{3.905098in}{1.082231in}}%
\pgfpathcurveto{\pgfqpoint{3.896862in}{1.082231in}}{\pgfqpoint{3.888962in}{1.078959in}}{\pgfqpoint{3.883138in}{1.073135in}}%
\pgfpathcurveto{\pgfqpoint{3.877314in}{1.067311in}}{\pgfqpoint{3.874042in}{1.059411in}}{\pgfqpoint{3.874042in}{1.051175in}}%
\pgfpathcurveto{\pgfqpoint{3.874042in}{1.042938in}}{\pgfqpoint{3.877314in}{1.035038in}}{\pgfqpoint{3.883138in}{1.029214in}}%
\pgfpathcurveto{\pgfqpoint{3.888962in}{1.023391in}}{\pgfqpoint{3.896862in}{1.020118in}}{\pgfqpoint{3.905098in}{1.020118in}}%
\pgfpathclose%
\pgfusepath{stroke,fill}%
\end{pgfscope}%
\begin{pgfscope}%
\pgfpathrectangle{\pgfqpoint{3.793912in}{0.557870in}}{\pgfqpoint{2.446088in}{1.484734in}}%
\pgfusepath{clip}%
\pgfsetbuttcap%
\pgfsetroundjoin%
\definecolor{currentfill}{rgb}{0.298039,0.447059,0.690196}%
\pgfsetfillcolor{currentfill}%
\pgfsetlinewidth{1.003750pt}%
\definecolor{currentstroke}{rgb}{0.298039,0.447059,0.690196}%
\pgfsetstrokecolor{currentstroke}%
\pgfsetdash{}{0pt}%
\pgfpathmoveto{\pgfqpoint{3.905098in}{0.610370in}}%
\pgfpathcurveto{\pgfqpoint{3.913334in}{0.610370in}}{\pgfqpoint{3.921234in}{0.613643in}}{\pgfqpoint{3.927058in}{0.619466in}}%
\pgfpathcurveto{\pgfqpoint{3.932882in}{0.625290in}}{\pgfqpoint{3.936155in}{0.633190in}}{\pgfqpoint{3.936155in}{0.641427in}}%
\pgfpathcurveto{\pgfqpoint{3.936155in}{0.649663in}}{\pgfqpoint{3.932882in}{0.657563in}}{\pgfqpoint{3.927058in}{0.663387in}}%
\pgfpathcurveto{\pgfqpoint{3.921234in}{0.669211in}}{\pgfqpoint{3.913334in}{0.672483in}}{\pgfqpoint{3.905098in}{0.672483in}}%
\pgfpathcurveto{\pgfqpoint{3.896862in}{0.672483in}}{\pgfqpoint{3.888962in}{0.669211in}}{\pgfqpoint{3.883138in}{0.663387in}}%
\pgfpathcurveto{\pgfqpoint{3.877314in}{0.657563in}}{\pgfqpoint{3.874042in}{0.649663in}}{\pgfqpoint{3.874042in}{0.641427in}}%
\pgfpathcurveto{\pgfqpoint{3.874042in}{0.633190in}}{\pgfqpoint{3.877314in}{0.625290in}}{\pgfqpoint{3.883138in}{0.619466in}}%
\pgfpathcurveto{\pgfqpoint{3.888962in}{0.613643in}}{\pgfqpoint{3.896862in}{0.610370in}}{\pgfqpoint{3.905098in}{0.610370in}}%
\pgfpathclose%
\pgfusepath{stroke,fill}%
\end{pgfscope}%
\begin{pgfscope}%
\pgfpathrectangle{\pgfqpoint{3.793912in}{0.557870in}}{\pgfqpoint{2.446088in}{1.484734in}}%
\pgfusepath{clip}%
\pgfsetbuttcap%
\pgfsetroundjoin%
\definecolor{currentfill}{rgb}{0.298039,0.447059,0.690196}%
\pgfsetfillcolor{currentfill}%
\pgfsetlinewidth{1.003750pt}%
\definecolor{currentstroke}{rgb}{0.298039,0.447059,0.690196}%
\pgfsetstrokecolor{currentstroke}%
\pgfsetdash{}{0pt}%
\pgfpathmoveto{\pgfqpoint{3.905098in}{0.883536in}}%
\pgfpathcurveto{\pgfqpoint{3.913334in}{0.883536in}}{\pgfqpoint{3.921234in}{0.886808in}}{\pgfqpoint{3.927058in}{0.892632in}}%
\pgfpathcurveto{\pgfqpoint{3.932882in}{0.898456in}}{\pgfqpoint{3.936155in}{0.906356in}}{\pgfqpoint{3.936155in}{0.914592in}}%
\pgfpathcurveto{\pgfqpoint{3.936155in}{0.922828in}}{\pgfqpoint{3.932882in}{0.930728in}}{\pgfqpoint{3.927058in}{0.936552in}}%
\pgfpathcurveto{\pgfqpoint{3.921234in}{0.942376in}}{\pgfqpoint{3.913334in}{0.945649in}}{\pgfqpoint{3.905098in}{0.945649in}}%
\pgfpathcurveto{\pgfqpoint{3.896862in}{0.945649in}}{\pgfqpoint{3.888962in}{0.942376in}}{\pgfqpoint{3.883138in}{0.936552in}}%
\pgfpathcurveto{\pgfqpoint{3.877314in}{0.930728in}}{\pgfqpoint{3.874042in}{0.922828in}}{\pgfqpoint{3.874042in}{0.914592in}}%
\pgfpathcurveto{\pgfqpoint{3.874042in}{0.906356in}}{\pgfqpoint{3.877314in}{0.898456in}}{\pgfqpoint{3.883138in}{0.892632in}}%
\pgfpathcurveto{\pgfqpoint{3.888962in}{0.886808in}}{\pgfqpoint{3.896862in}{0.883536in}}{\pgfqpoint{3.905098in}{0.883536in}}%
\pgfpathclose%
\pgfusepath{stroke,fill}%
\end{pgfscope}%
\begin{pgfscope}%
\pgfpathrectangle{\pgfqpoint{3.793912in}{0.557870in}}{\pgfqpoint{2.446088in}{1.484734in}}%
\pgfusepath{clip}%
\pgfsetbuttcap%
\pgfsetroundjoin%
\definecolor{currentfill}{rgb}{0.298039,0.447059,0.690196}%
\pgfsetfillcolor{currentfill}%
\pgfsetlinewidth{1.003750pt}%
\definecolor{currentstroke}{rgb}{0.298039,0.447059,0.690196}%
\pgfsetstrokecolor{currentstroke}%
\pgfsetdash{}{0pt}%
\pgfpathmoveto{\pgfqpoint{3.905098in}{1.936025in}}%
\pgfpathcurveto{\pgfqpoint{3.913334in}{1.936025in}}{\pgfqpoint{3.921234in}{1.939298in}}{\pgfqpoint{3.927058in}{1.945122in}}%
\pgfpathcurveto{\pgfqpoint{3.932882in}{1.950946in}}{\pgfqpoint{3.936155in}{1.958846in}}{\pgfqpoint{3.936155in}{1.967082in}}%
\pgfpathcurveto{\pgfqpoint{3.936155in}{1.975318in}}{\pgfqpoint{3.932882in}{1.983218in}}{\pgfqpoint{3.927058in}{1.989042in}}%
\pgfpathcurveto{\pgfqpoint{3.921234in}{1.994866in}}{\pgfqpoint{3.913334in}{1.998138in}}{\pgfqpoint{3.905098in}{1.998138in}}%
\pgfpathcurveto{\pgfqpoint{3.896862in}{1.998138in}}{\pgfqpoint{3.888962in}{1.994866in}}{\pgfqpoint{3.883138in}{1.989042in}}%
\pgfpathcurveto{\pgfqpoint{3.877314in}{1.983218in}}{\pgfqpoint{3.874042in}{1.975318in}}{\pgfqpoint{3.874042in}{1.967082in}}%
\pgfpathcurveto{\pgfqpoint{3.874042in}{1.958846in}}{\pgfqpoint{3.877314in}{1.950946in}}{\pgfqpoint{3.883138in}{1.945122in}}%
\pgfpathcurveto{\pgfqpoint{3.888962in}{1.939298in}}{\pgfqpoint{3.896862in}{1.936025in}}{\pgfqpoint{3.905098in}{1.936025in}}%
\pgfpathclose%
\pgfusepath{stroke,fill}%
\end{pgfscope}%
\begin{pgfscope}%
\pgfpathrectangle{\pgfqpoint{3.793912in}{0.557870in}}{\pgfqpoint{2.446088in}{1.484734in}}%
\pgfusepath{clip}%
\pgfsetbuttcap%
\pgfsetroundjoin%
\definecolor{currentfill}{rgb}{0.298039,0.447059,0.690196}%
\pgfsetfillcolor{currentfill}%
\pgfsetlinewidth{1.003750pt}%
\definecolor{currentstroke}{rgb}{0.298039,0.447059,0.690196}%
\pgfsetstrokecolor{currentstroke}%
\pgfsetdash{}{0pt}%
\pgfpathmoveto{\pgfqpoint{3.905098in}{0.634473in}}%
\pgfpathcurveto{\pgfqpoint{3.913334in}{0.634473in}}{\pgfqpoint{3.921234in}{0.637745in}}{\pgfqpoint{3.927058in}{0.643569in}}%
\pgfpathcurveto{\pgfqpoint{3.932882in}{0.649393in}}{\pgfqpoint{3.936155in}{0.657293in}}{\pgfqpoint{3.936155in}{0.665530in}}%
\pgfpathcurveto{\pgfqpoint{3.936155in}{0.673766in}}{\pgfqpoint{3.932882in}{0.681666in}}{\pgfqpoint{3.927058in}{0.687490in}}%
\pgfpathcurveto{\pgfqpoint{3.921234in}{0.693314in}}{\pgfqpoint{3.913334in}{0.696586in}}{\pgfqpoint{3.905098in}{0.696586in}}%
\pgfpathcurveto{\pgfqpoint{3.896862in}{0.696586in}}{\pgfqpoint{3.888962in}{0.693314in}}{\pgfqpoint{3.883138in}{0.687490in}}%
\pgfpathcurveto{\pgfqpoint{3.877314in}{0.681666in}}{\pgfqpoint{3.874042in}{0.673766in}}{\pgfqpoint{3.874042in}{0.665530in}}%
\pgfpathcurveto{\pgfqpoint{3.874042in}{0.657293in}}{\pgfqpoint{3.877314in}{0.649393in}}{\pgfqpoint{3.883138in}{0.643569in}}%
\pgfpathcurveto{\pgfqpoint{3.888962in}{0.637745in}}{\pgfqpoint{3.896862in}{0.634473in}}{\pgfqpoint{3.905098in}{0.634473in}}%
\pgfpathclose%
\pgfusepath{stroke,fill}%
\end{pgfscope}%
\begin{pgfscope}%
\pgfpathrectangle{\pgfqpoint{3.793912in}{0.557870in}}{\pgfqpoint{2.446088in}{1.484734in}}%
\pgfusepath{clip}%
\pgfsetbuttcap%
\pgfsetroundjoin%
\definecolor{currentfill}{rgb}{0.298039,0.447059,0.690196}%
\pgfsetfillcolor{currentfill}%
\pgfsetlinewidth{1.003750pt}%
\definecolor{currentstroke}{rgb}{0.298039,0.447059,0.690196}%
\pgfsetstrokecolor{currentstroke}%
\pgfsetdash{}{0pt}%
\pgfpathmoveto{\pgfqpoint{3.905098in}{1.936025in}}%
\pgfpathcurveto{\pgfqpoint{3.913334in}{1.936025in}}{\pgfqpoint{3.921234in}{1.939298in}}{\pgfqpoint{3.927058in}{1.945122in}}%
\pgfpathcurveto{\pgfqpoint{3.932882in}{1.950946in}}{\pgfqpoint{3.936155in}{1.958846in}}{\pgfqpoint{3.936155in}{1.967082in}}%
\pgfpathcurveto{\pgfqpoint{3.936155in}{1.975318in}}{\pgfqpoint{3.932882in}{1.983218in}}{\pgfqpoint{3.927058in}{1.989042in}}%
\pgfpathcurveto{\pgfqpoint{3.921234in}{1.994866in}}{\pgfqpoint{3.913334in}{1.998138in}}{\pgfqpoint{3.905098in}{1.998138in}}%
\pgfpathcurveto{\pgfqpoint{3.896862in}{1.998138in}}{\pgfqpoint{3.888962in}{1.994866in}}{\pgfqpoint{3.883138in}{1.989042in}}%
\pgfpathcurveto{\pgfqpoint{3.877314in}{1.983218in}}{\pgfqpoint{3.874042in}{1.975318in}}{\pgfqpoint{3.874042in}{1.967082in}}%
\pgfpathcurveto{\pgfqpoint{3.874042in}{1.958846in}}{\pgfqpoint{3.877314in}{1.950946in}}{\pgfqpoint{3.883138in}{1.945122in}}%
\pgfpathcurveto{\pgfqpoint{3.888962in}{1.939298in}}{\pgfqpoint{3.896862in}{1.936025in}}{\pgfqpoint{3.905098in}{1.936025in}}%
\pgfpathclose%
\pgfusepath{stroke,fill}%
\end{pgfscope}%
\begin{pgfscope}%
\pgfpathrectangle{\pgfqpoint{3.793912in}{0.557870in}}{\pgfqpoint{2.446088in}{1.484734in}}%
\pgfusepath{clip}%
\pgfsetbuttcap%
\pgfsetroundjoin%
\definecolor{currentfill}{rgb}{0.298039,0.447059,0.690196}%
\pgfsetfillcolor{currentfill}%
\pgfsetlinewidth{1.003750pt}%
\definecolor{currentstroke}{rgb}{0.298039,0.447059,0.690196}%
\pgfsetstrokecolor{currentstroke}%
\pgfsetdash{}{0pt}%
\pgfpathmoveto{\pgfqpoint{3.905098in}{1.936025in}}%
\pgfpathcurveto{\pgfqpoint{3.913334in}{1.936025in}}{\pgfqpoint{3.921234in}{1.939298in}}{\pgfqpoint{3.927058in}{1.945122in}}%
\pgfpathcurveto{\pgfqpoint{3.932882in}{1.950946in}}{\pgfqpoint{3.936155in}{1.958846in}}{\pgfqpoint{3.936155in}{1.967082in}}%
\pgfpathcurveto{\pgfqpoint{3.936155in}{1.975318in}}{\pgfqpoint{3.932882in}{1.983218in}}{\pgfqpoint{3.927058in}{1.989042in}}%
\pgfpathcurveto{\pgfqpoint{3.921234in}{1.994866in}}{\pgfqpoint{3.913334in}{1.998138in}}{\pgfqpoint{3.905098in}{1.998138in}}%
\pgfpathcurveto{\pgfqpoint{3.896862in}{1.998138in}}{\pgfqpoint{3.888962in}{1.994866in}}{\pgfqpoint{3.883138in}{1.989042in}}%
\pgfpathcurveto{\pgfqpoint{3.877314in}{1.983218in}}{\pgfqpoint{3.874042in}{1.975318in}}{\pgfqpoint{3.874042in}{1.967082in}}%
\pgfpathcurveto{\pgfqpoint{3.874042in}{1.958846in}}{\pgfqpoint{3.877314in}{1.950946in}}{\pgfqpoint{3.883138in}{1.945122in}}%
\pgfpathcurveto{\pgfqpoint{3.888962in}{1.939298in}}{\pgfqpoint{3.896862in}{1.936025in}}{\pgfqpoint{3.905098in}{1.936025in}}%
\pgfpathclose%
\pgfusepath{stroke,fill}%
\end{pgfscope}%
\begin{pgfscope}%
\pgfpathrectangle{\pgfqpoint{3.793912in}{0.557870in}}{\pgfqpoint{2.446088in}{1.484734in}}%
\pgfusepath{clip}%
\pgfsetbuttcap%
\pgfsetroundjoin%
\definecolor{currentfill}{rgb}{0.298039,0.447059,0.690196}%
\pgfsetfillcolor{currentfill}%
\pgfsetlinewidth{1.003750pt}%
\definecolor{currentstroke}{rgb}{0.298039,0.447059,0.690196}%
\pgfsetstrokecolor{currentstroke}%
\pgfsetdash{}{0pt}%
\pgfpathmoveto{\pgfqpoint{3.905098in}{1.381661in}}%
\pgfpathcurveto{\pgfqpoint{3.913334in}{1.381661in}}{\pgfqpoint{3.921234in}{1.384933in}}{\pgfqpoint{3.927058in}{1.390757in}}%
\pgfpathcurveto{\pgfqpoint{3.932882in}{1.396581in}}{\pgfqpoint{3.936155in}{1.404481in}}{\pgfqpoint{3.936155in}{1.412717in}}%
\pgfpathcurveto{\pgfqpoint{3.936155in}{1.420953in}}{\pgfqpoint{3.932882in}{1.428853in}}{\pgfqpoint{3.927058in}{1.434677in}}%
\pgfpathcurveto{\pgfqpoint{3.921234in}{1.440501in}}{\pgfqpoint{3.913334in}{1.443774in}}{\pgfqpoint{3.905098in}{1.443774in}}%
\pgfpathcurveto{\pgfqpoint{3.896862in}{1.443774in}}{\pgfqpoint{3.888962in}{1.440501in}}{\pgfqpoint{3.883138in}{1.434677in}}%
\pgfpathcurveto{\pgfqpoint{3.877314in}{1.428853in}}{\pgfqpoint{3.874042in}{1.420953in}}{\pgfqpoint{3.874042in}{1.412717in}}%
\pgfpathcurveto{\pgfqpoint{3.874042in}{1.404481in}}{\pgfqpoint{3.877314in}{1.396581in}}{\pgfqpoint{3.883138in}{1.390757in}}%
\pgfpathcurveto{\pgfqpoint{3.888962in}{1.384933in}}{\pgfqpoint{3.896862in}{1.381661in}}{\pgfqpoint{3.905098in}{1.381661in}}%
\pgfpathclose%
\pgfusepath{stroke,fill}%
\end{pgfscope}%
\begin{pgfscope}%
\pgfpathrectangle{\pgfqpoint{3.793912in}{0.557870in}}{\pgfqpoint{2.446088in}{1.484734in}}%
\pgfusepath{clip}%
\pgfsetbuttcap%
\pgfsetroundjoin%
\definecolor{currentfill}{rgb}{0.298039,0.447059,0.690196}%
\pgfsetfillcolor{currentfill}%
\pgfsetlinewidth{1.003750pt}%
\definecolor{currentstroke}{rgb}{0.298039,0.447059,0.690196}%
\pgfsetstrokecolor{currentstroke}%
\pgfsetdash{}{0pt}%
\pgfpathmoveto{\pgfqpoint{3.905098in}{0.610370in}}%
\pgfpathcurveto{\pgfqpoint{3.913334in}{0.610370in}}{\pgfqpoint{3.921234in}{0.613643in}}{\pgfqpoint{3.927058in}{0.619466in}}%
\pgfpathcurveto{\pgfqpoint{3.932882in}{0.625290in}}{\pgfqpoint{3.936155in}{0.633190in}}{\pgfqpoint{3.936155in}{0.641427in}}%
\pgfpathcurveto{\pgfqpoint{3.936155in}{0.649663in}}{\pgfqpoint{3.932882in}{0.657563in}}{\pgfqpoint{3.927058in}{0.663387in}}%
\pgfpathcurveto{\pgfqpoint{3.921234in}{0.669211in}}{\pgfqpoint{3.913334in}{0.672483in}}{\pgfqpoint{3.905098in}{0.672483in}}%
\pgfpathcurveto{\pgfqpoint{3.896862in}{0.672483in}}{\pgfqpoint{3.888962in}{0.669211in}}{\pgfqpoint{3.883138in}{0.663387in}}%
\pgfpathcurveto{\pgfqpoint{3.877314in}{0.657563in}}{\pgfqpoint{3.874042in}{0.649663in}}{\pgfqpoint{3.874042in}{0.641427in}}%
\pgfpathcurveto{\pgfqpoint{3.874042in}{0.633190in}}{\pgfqpoint{3.877314in}{0.625290in}}{\pgfqpoint{3.883138in}{0.619466in}}%
\pgfpathcurveto{\pgfqpoint{3.888962in}{0.613643in}}{\pgfqpoint{3.896862in}{0.610370in}}{\pgfqpoint{3.905098in}{0.610370in}}%
\pgfpathclose%
\pgfusepath{stroke,fill}%
\end{pgfscope}%
\begin{pgfscope}%
\pgfpathrectangle{\pgfqpoint{3.793912in}{0.557870in}}{\pgfqpoint{2.446088in}{1.484734in}}%
\pgfusepath{clip}%
\pgfsetbuttcap%
\pgfsetroundjoin%
\definecolor{currentfill}{rgb}{0.298039,0.447059,0.690196}%
\pgfsetfillcolor{currentfill}%
\pgfsetlinewidth{1.003750pt}%
\definecolor{currentstroke}{rgb}{0.298039,0.447059,0.690196}%
\pgfsetstrokecolor{currentstroke}%
\pgfsetdash{}{0pt}%
\pgfpathmoveto{\pgfqpoint{3.905098in}{0.947810in}}%
\pgfpathcurveto{\pgfqpoint{3.913334in}{0.947810in}}{\pgfqpoint{3.921234in}{0.951082in}}{\pgfqpoint{3.927058in}{0.956906in}}%
\pgfpathcurveto{\pgfqpoint{3.932882in}{0.962730in}}{\pgfqpoint{3.936155in}{0.970630in}}{\pgfqpoint{3.936155in}{0.978866in}}%
\pgfpathcurveto{\pgfqpoint{3.936155in}{0.987103in}}{\pgfqpoint{3.932882in}{0.995003in}}{\pgfqpoint{3.927058in}{1.000827in}}%
\pgfpathcurveto{\pgfqpoint{3.921234in}{1.006650in}}{\pgfqpoint{3.913334in}{1.009923in}}{\pgfqpoint{3.905098in}{1.009923in}}%
\pgfpathcurveto{\pgfqpoint{3.896862in}{1.009923in}}{\pgfqpoint{3.888962in}{1.006650in}}{\pgfqpoint{3.883138in}{1.000827in}}%
\pgfpathcurveto{\pgfqpoint{3.877314in}{0.995003in}}{\pgfqpoint{3.874042in}{0.987103in}}{\pgfqpoint{3.874042in}{0.978866in}}%
\pgfpathcurveto{\pgfqpoint{3.874042in}{0.970630in}}{\pgfqpoint{3.877314in}{0.962730in}}{\pgfqpoint{3.883138in}{0.956906in}}%
\pgfpathcurveto{\pgfqpoint{3.888962in}{0.951082in}}{\pgfqpoint{3.896862in}{0.947810in}}{\pgfqpoint{3.905098in}{0.947810in}}%
\pgfpathclose%
\pgfusepath{stroke,fill}%
\end{pgfscope}%
\begin{pgfscope}%
\pgfpathrectangle{\pgfqpoint{3.793912in}{0.557870in}}{\pgfqpoint{2.446088in}{1.484734in}}%
\pgfusepath{clip}%
\pgfsetbuttcap%
\pgfsetroundjoin%
\definecolor{currentfill}{rgb}{0.298039,0.447059,0.690196}%
\pgfsetfillcolor{currentfill}%
\pgfsetlinewidth{1.003750pt}%
\definecolor{currentstroke}{rgb}{0.298039,0.447059,0.690196}%
\pgfsetstrokecolor{currentstroke}%
\pgfsetdash{}{0pt}%
\pgfpathmoveto{\pgfqpoint{3.905098in}{1.936025in}}%
\pgfpathcurveto{\pgfqpoint{3.913334in}{1.936025in}}{\pgfqpoint{3.921234in}{1.939298in}}{\pgfqpoint{3.927058in}{1.945122in}}%
\pgfpathcurveto{\pgfqpoint{3.932882in}{1.950946in}}{\pgfqpoint{3.936155in}{1.958846in}}{\pgfqpoint{3.936155in}{1.967082in}}%
\pgfpathcurveto{\pgfqpoint{3.936155in}{1.975318in}}{\pgfqpoint{3.932882in}{1.983218in}}{\pgfqpoint{3.927058in}{1.989042in}}%
\pgfpathcurveto{\pgfqpoint{3.921234in}{1.994866in}}{\pgfqpoint{3.913334in}{1.998138in}}{\pgfqpoint{3.905098in}{1.998138in}}%
\pgfpathcurveto{\pgfqpoint{3.896862in}{1.998138in}}{\pgfqpoint{3.888962in}{1.994866in}}{\pgfqpoint{3.883138in}{1.989042in}}%
\pgfpathcurveto{\pgfqpoint{3.877314in}{1.983218in}}{\pgfqpoint{3.874042in}{1.975318in}}{\pgfqpoint{3.874042in}{1.967082in}}%
\pgfpathcurveto{\pgfqpoint{3.874042in}{1.958846in}}{\pgfqpoint{3.877314in}{1.950946in}}{\pgfqpoint{3.883138in}{1.945122in}}%
\pgfpathcurveto{\pgfqpoint{3.888962in}{1.939298in}}{\pgfqpoint{3.896862in}{1.936025in}}{\pgfqpoint{3.905098in}{1.936025in}}%
\pgfpathclose%
\pgfusepath{stroke,fill}%
\end{pgfscope}%
\begin{pgfscope}%
\pgfpathrectangle{\pgfqpoint{3.793912in}{0.557870in}}{\pgfqpoint{2.446088in}{1.484734in}}%
\pgfusepath{clip}%
\pgfsetbuttcap%
\pgfsetroundjoin%
\definecolor{currentfill}{rgb}{0.298039,0.447059,0.690196}%
\pgfsetfillcolor{currentfill}%
\pgfsetlinewidth{1.003750pt}%
\definecolor{currentstroke}{rgb}{0.298039,0.447059,0.690196}%
\pgfsetstrokecolor{currentstroke}%
\pgfsetdash{}{0pt}%
\pgfpathmoveto{\pgfqpoint{3.905098in}{1.936025in}}%
\pgfpathcurveto{\pgfqpoint{3.913334in}{1.936025in}}{\pgfqpoint{3.921234in}{1.939298in}}{\pgfqpoint{3.927058in}{1.945122in}}%
\pgfpathcurveto{\pgfqpoint{3.932882in}{1.950946in}}{\pgfqpoint{3.936155in}{1.958846in}}{\pgfqpoint{3.936155in}{1.967082in}}%
\pgfpathcurveto{\pgfqpoint{3.936155in}{1.975318in}}{\pgfqpoint{3.932882in}{1.983218in}}{\pgfqpoint{3.927058in}{1.989042in}}%
\pgfpathcurveto{\pgfqpoint{3.921234in}{1.994866in}}{\pgfqpoint{3.913334in}{1.998138in}}{\pgfqpoint{3.905098in}{1.998138in}}%
\pgfpathcurveto{\pgfqpoint{3.896862in}{1.998138in}}{\pgfqpoint{3.888962in}{1.994866in}}{\pgfqpoint{3.883138in}{1.989042in}}%
\pgfpathcurveto{\pgfqpoint{3.877314in}{1.983218in}}{\pgfqpoint{3.874042in}{1.975318in}}{\pgfqpoint{3.874042in}{1.967082in}}%
\pgfpathcurveto{\pgfqpoint{3.874042in}{1.958846in}}{\pgfqpoint{3.877314in}{1.950946in}}{\pgfqpoint{3.883138in}{1.945122in}}%
\pgfpathcurveto{\pgfqpoint{3.888962in}{1.939298in}}{\pgfqpoint{3.896862in}{1.936025in}}{\pgfqpoint{3.905098in}{1.936025in}}%
\pgfpathclose%
\pgfusepath{stroke,fill}%
\end{pgfscope}%
\begin{pgfscope}%
\pgfpathrectangle{\pgfqpoint{3.793912in}{0.557870in}}{\pgfqpoint{2.446088in}{1.484734in}}%
\pgfusepath{clip}%
\pgfsetbuttcap%
\pgfsetroundjoin%
\definecolor{currentfill}{rgb}{0.298039,0.447059,0.690196}%
\pgfsetfillcolor{currentfill}%
\pgfsetlinewidth{1.003750pt}%
\definecolor{currentstroke}{rgb}{0.298039,0.447059,0.690196}%
\pgfsetstrokecolor{currentstroke}%
\pgfsetdash{}{0pt}%
\pgfpathmoveto{\pgfqpoint{3.905098in}{1.936025in}}%
\pgfpathcurveto{\pgfqpoint{3.913334in}{1.936025in}}{\pgfqpoint{3.921234in}{1.939298in}}{\pgfqpoint{3.927058in}{1.945122in}}%
\pgfpathcurveto{\pgfqpoint{3.932882in}{1.950946in}}{\pgfqpoint{3.936155in}{1.958846in}}{\pgfqpoint{3.936155in}{1.967082in}}%
\pgfpathcurveto{\pgfqpoint{3.936155in}{1.975318in}}{\pgfqpoint{3.932882in}{1.983218in}}{\pgfqpoint{3.927058in}{1.989042in}}%
\pgfpathcurveto{\pgfqpoint{3.921234in}{1.994866in}}{\pgfqpoint{3.913334in}{1.998138in}}{\pgfqpoint{3.905098in}{1.998138in}}%
\pgfpathcurveto{\pgfqpoint{3.896862in}{1.998138in}}{\pgfqpoint{3.888962in}{1.994866in}}{\pgfqpoint{3.883138in}{1.989042in}}%
\pgfpathcurveto{\pgfqpoint{3.877314in}{1.983218in}}{\pgfqpoint{3.874042in}{1.975318in}}{\pgfqpoint{3.874042in}{1.967082in}}%
\pgfpathcurveto{\pgfqpoint{3.874042in}{1.958846in}}{\pgfqpoint{3.877314in}{1.950946in}}{\pgfqpoint{3.883138in}{1.945122in}}%
\pgfpathcurveto{\pgfqpoint{3.888962in}{1.939298in}}{\pgfqpoint{3.896862in}{1.936025in}}{\pgfqpoint{3.905098in}{1.936025in}}%
\pgfpathclose%
\pgfusepath{stroke,fill}%
\end{pgfscope}%
\begin{pgfscope}%
\pgfpathrectangle{\pgfqpoint{3.793912in}{0.557870in}}{\pgfqpoint{2.446088in}{1.484734in}}%
\pgfusepath{clip}%
\pgfsetbuttcap%
\pgfsetroundjoin%
\definecolor{currentfill}{rgb}{0.298039,0.447059,0.690196}%
\pgfsetfillcolor{currentfill}%
\pgfsetlinewidth{1.003750pt}%
\definecolor{currentstroke}{rgb}{0.298039,0.447059,0.690196}%
\pgfsetstrokecolor{currentstroke}%
\pgfsetdash{}{0pt}%
\pgfpathmoveto{\pgfqpoint{3.905098in}{1.936025in}}%
\pgfpathcurveto{\pgfqpoint{3.913334in}{1.936025in}}{\pgfqpoint{3.921234in}{1.939298in}}{\pgfqpoint{3.927058in}{1.945122in}}%
\pgfpathcurveto{\pgfqpoint{3.932882in}{1.950946in}}{\pgfqpoint{3.936155in}{1.958846in}}{\pgfqpoint{3.936155in}{1.967082in}}%
\pgfpathcurveto{\pgfqpoint{3.936155in}{1.975318in}}{\pgfqpoint{3.932882in}{1.983218in}}{\pgfqpoint{3.927058in}{1.989042in}}%
\pgfpathcurveto{\pgfqpoint{3.921234in}{1.994866in}}{\pgfqpoint{3.913334in}{1.998138in}}{\pgfqpoint{3.905098in}{1.998138in}}%
\pgfpathcurveto{\pgfqpoint{3.896862in}{1.998138in}}{\pgfqpoint{3.888962in}{1.994866in}}{\pgfqpoint{3.883138in}{1.989042in}}%
\pgfpathcurveto{\pgfqpoint{3.877314in}{1.983218in}}{\pgfqpoint{3.874042in}{1.975318in}}{\pgfqpoint{3.874042in}{1.967082in}}%
\pgfpathcurveto{\pgfqpoint{3.874042in}{1.958846in}}{\pgfqpoint{3.877314in}{1.950946in}}{\pgfqpoint{3.883138in}{1.945122in}}%
\pgfpathcurveto{\pgfqpoint{3.888962in}{1.939298in}}{\pgfqpoint{3.896862in}{1.936025in}}{\pgfqpoint{3.905098in}{1.936025in}}%
\pgfpathclose%
\pgfusepath{stroke,fill}%
\end{pgfscope}%
\begin{pgfscope}%
\pgfpathrectangle{\pgfqpoint{3.793912in}{0.557870in}}{\pgfqpoint{2.446088in}{1.484734in}}%
\pgfusepath{clip}%
\pgfsetbuttcap%
\pgfsetroundjoin%
\definecolor{currentfill}{rgb}{0.298039,0.447059,0.690196}%
\pgfsetfillcolor{currentfill}%
\pgfsetlinewidth{1.003750pt}%
\definecolor{currentstroke}{rgb}{0.298039,0.447059,0.690196}%
\pgfsetstrokecolor{currentstroke}%
\pgfsetdash{}{0pt}%
\pgfpathmoveto{\pgfqpoint{3.905098in}{1.229009in}}%
\pgfpathcurveto{\pgfqpoint{3.913334in}{1.229009in}}{\pgfqpoint{3.921234in}{1.232282in}}{\pgfqpoint{3.927058in}{1.238106in}}%
\pgfpathcurveto{\pgfqpoint{3.932882in}{1.243930in}}{\pgfqpoint{3.936155in}{1.251830in}}{\pgfqpoint{3.936155in}{1.260066in}}%
\pgfpathcurveto{\pgfqpoint{3.936155in}{1.268302in}}{\pgfqpoint{3.932882in}{1.276202in}}{\pgfqpoint{3.927058in}{1.282026in}}%
\pgfpathcurveto{\pgfqpoint{3.921234in}{1.287850in}}{\pgfqpoint{3.913334in}{1.291122in}}{\pgfqpoint{3.905098in}{1.291122in}}%
\pgfpathcurveto{\pgfqpoint{3.896862in}{1.291122in}}{\pgfqpoint{3.888962in}{1.287850in}}{\pgfqpoint{3.883138in}{1.282026in}}%
\pgfpathcurveto{\pgfqpoint{3.877314in}{1.276202in}}{\pgfqpoint{3.874042in}{1.268302in}}{\pgfqpoint{3.874042in}{1.260066in}}%
\pgfpathcurveto{\pgfqpoint{3.874042in}{1.251830in}}{\pgfqpoint{3.877314in}{1.243930in}}{\pgfqpoint{3.883138in}{1.238106in}}%
\pgfpathcurveto{\pgfqpoint{3.888962in}{1.232282in}}{\pgfqpoint{3.896862in}{1.229009in}}{\pgfqpoint{3.905098in}{1.229009in}}%
\pgfpathclose%
\pgfusepath{stroke,fill}%
\end{pgfscope}%
\begin{pgfscope}%
\pgfpathrectangle{\pgfqpoint{3.793912in}{0.557870in}}{\pgfqpoint{2.446088in}{1.484734in}}%
\pgfusepath{clip}%
\pgfsetbuttcap%
\pgfsetroundjoin%
\definecolor{currentfill}{rgb}{0.298039,0.447059,0.690196}%
\pgfsetfillcolor{currentfill}%
\pgfsetlinewidth{1.003750pt}%
\definecolor{currentstroke}{rgb}{0.298039,0.447059,0.690196}%
\pgfsetstrokecolor{currentstroke}%
\pgfsetdash{}{0pt}%
\pgfpathmoveto{\pgfqpoint{3.905098in}{1.534312in}}%
\pgfpathcurveto{\pgfqpoint{3.913334in}{1.534312in}}{\pgfqpoint{3.921234in}{1.537584in}}{\pgfqpoint{3.927058in}{1.543408in}}%
\pgfpathcurveto{\pgfqpoint{3.932882in}{1.549232in}}{\pgfqpoint{3.936155in}{1.557132in}}{\pgfqpoint{3.936155in}{1.565368in}}%
\pgfpathcurveto{\pgfqpoint{3.936155in}{1.573605in}}{\pgfqpoint{3.932882in}{1.581505in}}{\pgfqpoint{3.927058in}{1.587329in}}%
\pgfpathcurveto{\pgfqpoint{3.921234in}{1.593152in}}{\pgfqpoint{3.913334in}{1.596425in}}{\pgfqpoint{3.905098in}{1.596425in}}%
\pgfpathcurveto{\pgfqpoint{3.896862in}{1.596425in}}{\pgfqpoint{3.888962in}{1.593152in}}{\pgfqpoint{3.883138in}{1.587329in}}%
\pgfpathcurveto{\pgfqpoint{3.877314in}{1.581505in}}{\pgfqpoint{3.874042in}{1.573605in}}{\pgfqpoint{3.874042in}{1.565368in}}%
\pgfpathcurveto{\pgfqpoint{3.874042in}{1.557132in}}{\pgfqpoint{3.877314in}{1.549232in}}{\pgfqpoint{3.883138in}{1.543408in}}%
\pgfpathcurveto{\pgfqpoint{3.888962in}{1.537584in}}{\pgfqpoint{3.896862in}{1.534312in}}{\pgfqpoint{3.905098in}{1.534312in}}%
\pgfpathclose%
\pgfusepath{stroke,fill}%
\end{pgfscope}%
\begin{pgfscope}%
\pgfpathrectangle{\pgfqpoint{3.793912in}{0.557870in}}{\pgfqpoint{2.446088in}{1.484734in}}%
\pgfusepath{clip}%
\pgfsetbuttcap%
\pgfsetroundjoin%
\definecolor{currentfill}{rgb}{0.298039,0.447059,0.690196}%
\pgfsetfillcolor{currentfill}%
\pgfsetlinewidth{1.003750pt}%
\definecolor{currentstroke}{rgb}{0.298039,0.447059,0.690196}%
\pgfsetstrokecolor{currentstroke}%
\pgfsetdash{}{0pt}%
\pgfpathmoveto{\pgfqpoint{3.905098in}{1.229009in}}%
\pgfpathcurveto{\pgfqpoint{3.913334in}{1.229009in}}{\pgfqpoint{3.921234in}{1.232282in}}{\pgfqpoint{3.927058in}{1.238106in}}%
\pgfpathcurveto{\pgfqpoint{3.932882in}{1.243930in}}{\pgfqpoint{3.936155in}{1.251830in}}{\pgfqpoint{3.936155in}{1.260066in}}%
\pgfpathcurveto{\pgfqpoint{3.936155in}{1.268302in}}{\pgfqpoint{3.932882in}{1.276202in}}{\pgfqpoint{3.927058in}{1.282026in}}%
\pgfpathcurveto{\pgfqpoint{3.921234in}{1.287850in}}{\pgfqpoint{3.913334in}{1.291122in}}{\pgfqpoint{3.905098in}{1.291122in}}%
\pgfpathcurveto{\pgfqpoint{3.896862in}{1.291122in}}{\pgfqpoint{3.888962in}{1.287850in}}{\pgfqpoint{3.883138in}{1.282026in}}%
\pgfpathcurveto{\pgfqpoint{3.877314in}{1.276202in}}{\pgfqpoint{3.874042in}{1.268302in}}{\pgfqpoint{3.874042in}{1.260066in}}%
\pgfpathcurveto{\pgfqpoint{3.874042in}{1.251830in}}{\pgfqpoint{3.877314in}{1.243930in}}{\pgfqpoint{3.883138in}{1.238106in}}%
\pgfpathcurveto{\pgfqpoint{3.888962in}{1.232282in}}{\pgfqpoint{3.896862in}{1.229009in}}{\pgfqpoint{3.905098in}{1.229009in}}%
\pgfpathclose%
\pgfusepath{stroke,fill}%
\end{pgfscope}%
\begin{pgfscope}%
\pgfpathrectangle{\pgfqpoint{3.793912in}{0.557870in}}{\pgfqpoint{2.446088in}{1.484734in}}%
\pgfusepath{clip}%
\pgfsetbuttcap%
\pgfsetroundjoin%
\definecolor{currentfill}{rgb}{0.298039,0.447059,0.690196}%
\pgfsetfillcolor{currentfill}%
\pgfsetlinewidth{1.003750pt}%
\definecolor{currentstroke}{rgb}{0.298039,0.447059,0.690196}%
\pgfsetstrokecolor{currentstroke}%
\pgfsetdash{}{0pt}%
\pgfpathmoveto{\pgfqpoint{3.905098in}{1.936025in}}%
\pgfpathcurveto{\pgfqpoint{3.913334in}{1.936025in}}{\pgfqpoint{3.921234in}{1.939298in}}{\pgfqpoint{3.927058in}{1.945122in}}%
\pgfpathcurveto{\pgfqpoint{3.932882in}{1.950946in}}{\pgfqpoint{3.936155in}{1.958846in}}{\pgfqpoint{3.936155in}{1.967082in}}%
\pgfpathcurveto{\pgfqpoint{3.936155in}{1.975318in}}{\pgfqpoint{3.932882in}{1.983218in}}{\pgfqpoint{3.927058in}{1.989042in}}%
\pgfpathcurveto{\pgfqpoint{3.921234in}{1.994866in}}{\pgfqpoint{3.913334in}{1.998138in}}{\pgfqpoint{3.905098in}{1.998138in}}%
\pgfpathcurveto{\pgfqpoint{3.896862in}{1.998138in}}{\pgfqpoint{3.888962in}{1.994866in}}{\pgfqpoint{3.883138in}{1.989042in}}%
\pgfpathcurveto{\pgfqpoint{3.877314in}{1.983218in}}{\pgfqpoint{3.874042in}{1.975318in}}{\pgfqpoint{3.874042in}{1.967082in}}%
\pgfpathcurveto{\pgfqpoint{3.874042in}{1.958846in}}{\pgfqpoint{3.877314in}{1.950946in}}{\pgfqpoint{3.883138in}{1.945122in}}%
\pgfpathcurveto{\pgfqpoint{3.888962in}{1.939298in}}{\pgfqpoint{3.896862in}{1.936025in}}{\pgfqpoint{3.905098in}{1.936025in}}%
\pgfpathclose%
\pgfusepath{stroke,fill}%
\end{pgfscope}%
\begin{pgfscope}%
\pgfpathrectangle{\pgfqpoint{3.793912in}{0.557870in}}{\pgfqpoint{2.446088in}{1.484734in}}%
\pgfusepath{clip}%
\pgfsetbuttcap%
\pgfsetroundjoin%
\definecolor{currentfill}{rgb}{0.298039,0.447059,0.690196}%
\pgfsetfillcolor{currentfill}%
\pgfsetlinewidth{1.003750pt}%
\definecolor{currentstroke}{rgb}{0.298039,0.447059,0.690196}%
\pgfsetstrokecolor{currentstroke}%
\pgfsetdash{}{0pt}%
\pgfpathmoveto{\pgfqpoint{3.905098in}{0.859433in}}%
\pgfpathcurveto{\pgfqpoint{3.913334in}{0.859433in}}{\pgfqpoint{3.921234in}{0.862705in}}{\pgfqpoint{3.927058in}{0.868529in}}%
\pgfpathcurveto{\pgfqpoint{3.932882in}{0.874353in}}{\pgfqpoint{3.936155in}{0.882253in}}{\pgfqpoint{3.936155in}{0.890489in}}%
\pgfpathcurveto{\pgfqpoint{3.936155in}{0.898726in}}{\pgfqpoint{3.932882in}{0.906626in}}{\pgfqpoint{3.927058in}{0.912449in}}%
\pgfpathcurveto{\pgfqpoint{3.921234in}{0.918273in}}{\pgfqpoint{3.913334in}{0.921546in}}{\pgfqpoint{3.905098in}{0.921546in}}%
\pgfpathcurveto{\pgfqpoint{3.896862in}{0.921546in}}{\pgfqpoint{3.888962in}{0.918273in}}{\pgfqpoint{3.883138in}{0.912449in}}%
\pgfpathcurveto{\pgfqpoint{3.877314in}{0.906626in}}{\pgfqpoint{3.874042in}{0.898726in}}{\pgfqpoint{3.874042in}{0.890489in}}%
\pgfpathcurveto{\pgfqpoint{3.874042in}{0.882253in}}{\pgfqpoint{3.877314in}{0.874353in}}{\pgfqpoint{3.883138in}{0.868529in}}%
\pgfpathcurveto{\pgfqpoint{3.888962in}{0.862705in}}{\pgfqpoint{3.896862in}{0.859433in}}{\pgfqpoint{3.905098in}{0.859433in}}%
\pgfpathclose%
\pgfusepath{stroke,fill}%
\end{pgfscope}%
\begin{pgfscope}%
\pgfpathrectangle{\pgfqpoint{3.793912in}{0.557870in}}{\pgfqpoint{2.446088in}{1.484734in}}%
\pgfusepath{clip}%
\pgfsetbuttcap%
\pgfsetroundjoin%
\definecolor{currentfill}{rgb}{0.298039,0.447059,0.690196}%
\pgfsetfillcolor{currentfill}%
\pgfsetlinewidth{1.003750pt}%
\definecolor{currentstroke}{rgb}{0.298039,0.447059,0.690196}%
\pgfsetstrokecolor{currentstroke}%
\pgfsetdash{}{0pt}%
\pgfpathmoveto{\pgfqpoint{3.905098in}{1.936025in}}%
\pgfpathcurveto{\pgfqpoint{3.913334in}{1.936025in}}{\pgfqpoint{3.921234in}{1.939298in}}{\pgfqpoint{3.927058in}{1.945122in}}%
\pgfpathcurveto{\pgfqpoint{3.932882in}{1.950946in}}{\pgfqpoint{3.936155in}{1.958846in}}{\pgfqpoint{3.936155in}{1.967082in}}%
\pgfpathcurveto{\pgfqpoint{3.936155in}{1.975318in}}{\pgfqpoint{3.932882in}{1.983218in}}{\pgfqpoint{3.927058in}{1.989042in}}%
\pgfpathcurveto{\pgfqpoint{3.921234in}{1.994866in}}{\pgfqpoint{3.913334in}{1.998138in}}{\pgfqpoint{3.905098in}{1.998138in}}%
\pgfpathcurveto{\pgfqpoint{3.896862in}{1.998138in}}{\pgfqpoint{3.888962in}{1.994866in}}{\pgfqpoint{3.883138in}{1.989042in}}%
\pgfpathcurveto{\pgfqpoint{3.877314in}{1.983218in}}{\pgfqpoint{3.874042in}{1.975318in}}{\pgfqpoint{3.874042in}{1.967082in}}%
\pgfpathcurveto{\pgfqpoint{3.874042in}{1.958846in}}{\pgfqpoint{3.877314in}{1.950946in}}{\pgfqpoint{3.883138in}{1.945122in}}%
\pgfpathcurveto{\pgfqpoint{3.888962in}{1.939298in}}{\pgfqpoint{3.896862in}{1.936025in}}{\pgfqpoint{3.905098in}{1.936025in}}%
\pgfpathclose%
\pgfusepath{stroke,fill}%
\end{pgfscope}%
\begin{pgfscope}%
\pgfpathrectangle{\pgfqpoint{3.793912in}{0.557870in}}{\pgfqpoint{2.446088in}{1.484734in}}%
\pgfusepath{clip}%
\pgfsetbuttcap%
\pgfsetroundjoin%
\definecolor{currentfill}{rgb}{0.298039,0.447059,0.690196}%
\pgfsetfillcolor{currentfill}%
\pgfsetlinewidth{1.003750pt}%
\definecolor{currentstroke}{rgb}{0.298039,0.447059,0.690196}%
\pgfsetstrokecolor{currentstroke}%
\pgfsetdash{}{0pt}%
\pgfpathmoveto{\pgfqpoint{3.905098in}{1.237044in}}%
\pgfpathcurveto{\pgfqpoint{3.913334in}{1.237044in}}{\pgfqpoint{3.921234in}{1.240316in}}{\pgfqpoint{3.927058in}{1.246140in}}%
\pgfpathcurveto{\pgfqpoint{3.932882in}{1.251964in}}{\pgfqpoint{3.936155in}{1.259864in}}{\pgfqpoint{3.936155in}{1.268100in}}%
\pgfpathcurveto{\pgfqpoint{3.936155in}{1.276336in}}{\pgfqpoint{3.932882in}{1.284236in}}{\pgfqpoint{3.927058in}{1.290060in}}%
\pgfpathcurveto{\pgfqpoint{3.921234in}{1.295884in}}{\pgfqpoint{3.913334in}{1.299157in}}{\pgfqpoint{3.905098in}{1.299157in}}%
\pgfpathcurveto{\pgfqpoint{3.896862in}{1.299157in}}{\pgfqpoint{3.888962in}{1.295884in}}{\pgfqpoint{3.883138in}{1.290060in}}%
\pgfpathcurveto{\pgfqpoint{3.877314in}{1.284236in}}{\pgfqpoint{3.874042in}{1.276336in}}{\pgfqpoint{3.874042in}{1.268100in}}%
\pgfpathcurveto{\pgfqpoint{3.874042in}{1.259864in}}{\pgfqpoint{3.877314in}{1.251964in}}{\pgfqpoint{3.883138in}{1.246140in}}%
\pgfpathcurveto{\pgfqpoint{3.888962in}{1.240316in}}{\pgfqpoint{3.896862in}{1.237044in}}{\pgfqpoint{3.905098in}{1.237044in}}%
\pgfpathclose%
\pgfusepath{stroke,fill}%
\end{pgfscope}%
\begin{pgfscope}%
\pgfpathrectangle{\pgfqpoint{3.793912in}{0.557870in}}{\pgfqpoint{2.446088in}{1.484734in}}%
\pgfusepath{clip}%
\pgfsetbuttcap%
\pgfsetroundjoin%
\definecolor{currentfill}{rgb}{0.298039,0.447059,0.690196}%
\pgfsetfillcolor{currentfill}%
\pgfsetlinewidth{1.003750pt}%
\definecolor{currentstroke}{rgb}{0.298039,0.447059,0.690196}%
\pgfsetstrokecolor{currentstroke}%
\pgfsetdash{}{0pt}%
\pgfpathmoveto{\pgfqpoint{3.905098in}{1.220975in}}%
\pgfpathcurveto{\pgfqpoint{3.913334in}{1.220975in}}{\pgfqpoint{3.921234in}{1.224247in}}{\pgfqpoint{3.927058in}{1.230071in}}%
\pgfpathcurveto{\pgfqpoint{3.932882in}{1.235895in}}{\pgfqpoint{3.936155in}{1.243795in}}{\pgfqpoint{3.936155in}{1.252032in}}%
\pgfpathcurveto{\pgfqpoint{3.936155in}{1.260268in}}{\pgfqpoint{3.932882in}{1.268168in}}{\pgfqpoint{3.927058in}{1.273992in}}%
\pgfpathcurveto{\pgfqpoint{3.921234in}{1.279816in}}{\pgfqpoint{3.913334in}{1.283088in}}{\pgfqpoint{3.905098in}{1.283088in}}%
\pgfpathcurveto{\pgfqpoint{3.896862in}{1.283088in}}{\pgfqpoint{3.888962in}{1.279816in}}{\pgfqpoint{3.883138in}{1.273992in}}%
\pgfpathcurveto{\pgfqpoint{3.877314in}{1.268168in}}{\pgfqpoint{3.874042in}{1.260268in}}{\pgfqpoint{3.874042in}{1.252032in}}%
\pgfpathcurveto{\pgfqpoint{3.874042in}{1.243795in}}{\pgfqpoint{3.877314in}{1.235895in}}{\pgfqpoint{3.883138in}{1.230071in}}%
\pgfpathcurveto{\pgfqpoint{3.888962in}{1.224247in}}{\pgfqpoint{3.896862in}{1.220975in}}{\pgfqpoint{3.905098in}{1.220975in}}%
\pgfpathclose%
\pgfusepath{stroke,fill}%
\end{pgfscope}%
\begin{pgfscope}%
\pgfpathrectangle{\pgfqpoint{3.793912in}{0.557870in}}{\pgfqpoint{2.446088in}{1.484734in}}%
\pgfusepath{clip}%
\pgfsetbuttcap%
\pgfsetroundjoin%
\definecolor{currentfill}{rgb}{0.298039,0.447059,0.690196}%
\pgfsetfillcolor{currentfill}%
\pgfsetlinewidth{1.003750pt}%
\definecolor{currentstroke}{rgb}{0.298039,0.447059,0.690196}%
\pgfsetstrokecolor{currentstroke}%
\pgfsetdash{}{0pt}%
\pgfpathmoveto{\pgfqpoint{3.905098in}{1.253112in}}%
\pgfpathcurveto{\pgfqpoint{3.913334in}{1.253112in}}{\pgfqpoint{3.921234in}{1.256384in}}{\pgfqpoint{3.927058in}{1.262208in}}%
\pgfpathcurveto{\pgfqpoint{3.932882in}{1.268032in}}{\pgfqpoint{3.936155in}{1.275932in}}{\pgfqpoint{3.936155in}{1.284169in}}%
\pgfpathcurveto{\pgfqpoint{3.936155in}{1.292405in}}{\pgfqpoint{3.932882in}{1.300305in}}{\pgfqpoint{3.927058in}{1.306129in}}%
\pgfpathcurveto{\pgfqpoint{3.921234in}{1.311953in}}{\pgfqpoint{3.913334in}{1.315225in}}{\pgfqpoint{3.905098in}{1.315225in}}%
\pgfpathcurveto{\pgfqpoint{3.896862in}{1.315225in}}{\pgfqpoint{3.888962in}{1.311953in}}{\pgfqpoint{3.883138in}{1.306129in}}%
\pgfpathcurveto{\pgfqpoint{3.877314in}{1.300305in}}{\pgfqpoint{3.874042in}{1.292405in}}{\pgfqpoint{3.874042in}{1.284169in}}%
\pgfpathcurveto{\pgfqpoint{3.874042in}{1.275932in}}{\pgfqpoint{3.877314in}{1.268032in}}{\pgfqpoint{3.883138in}{1.262208in}}%
\pgfpathcurveto{\pgfqpoint{3.888962in}{1.256384in}}{\pgfqpoint{3.896862in}{1.253112in}}{\pgfqpoint{3.905098in}{1.253112in}}%
\pgfpathclose%
\pgfusepath{stroke,fill}%
\end{pgfscope}%
\begin{pgfscope}%
\pgfpathrectangle{\pgfqpoint{3.793912in}{0.557870in}}{\pgfqpoint{2.446088in}{1.484734in}}%
\pgfusepath{clip}%
\pgfsetbuttcap%
\pgfsetroundjoin%
\definecolor{currentfill}{rgb}{0.298039,0.447059,0.690196}%
\pgfsetfillcolor{currentfill}%
\pgfsetlinewidth{1.003750pt}%
\definecolor{currentstroke}{rgb}{0.298039,0.447059,0.690196}%
\pgfsetstrokecolor{currentstroke}%
\pgfsetdash{}{0pt}%
\pgfpathmoveto{\pgfqpoint{3.905098in}{1.269181in}}%
\pgfpathcurveto{\pgfqpoint{3.913334in}{1.269181in}}{\pgfqpoint{3.921234in}{1.272453in}}{\pgfqpoint{3.927058in}{1.278277in}}%
\pgfpathcurveto{\pgfqpoint{3.932882in}{1.284101in}}{\pgfqpoint{3.936155in}{1.292001in}}{\pgfqpoint{3.936155in}{1.300237in}}%
\pgfpathcurveto{\pgfqpoint{3.936155in}{1.308474in}}{\pgfqpoint{3.932882in}{1.316374in}}{\pgfqpoint{3.927058in}{1.322197in}}%
\pgfpathcurveto{\pgfqpoint{3.921234in}{1.328021in}}{\pgfqpoint{3.913334in}{1.331294in}}{\pgfqpoint{3.905098in}{1.331294in}}%
\pgfpathcurveto{\pgfqpoint{3.896862in}{1.331294in}}{\pgfqpoint{3.888962in}{1.328021in}}{\pgfqpoint{3.883138in}{1.322197in}}%
\pgfpathcurveto{\pgfqpoint{3.877314in}{1.316374in}}{\pgfqpoint{3.874042in}{1.308474in}}{\pgfqpoint{3.874042in}{1.300237in}}%
\pgfpathcurveto{\pgfqpoint{3.874042in}{1.292001in}}{\pgfqpoint{3.877314in}{1.284101in}}{\pgfqpoint{3.883138in}{1.278277in}}%
\pgfpathcurveto{\pgfqpoint{3.888962in}{1.272453in}}{\pgfqpoint{3.896862in}{1.269181in}}{\pgfqpoint{3.905098in}{1.269181in}}%
\pgfpathclose%
\pgfusepath{stroke,fill}%
\end{pgfscope}%
\begin{pgfscope}%
\pgfpathrectangle{\pgfqpoint{3.793912in}{0.557870in}}{\pgfqpoint{2.446088in}{1.484734in}}%
\pgfusepath{clip}%
\pgfsetbuttcap%
\pgfsetroundjoin%
\definecolor{currentfill}{rgb}{0.298039,0.447059,0.690196}%
\pgfsetfillcolor{currentfill}%
\pgfsetlinewidth{1.003750pt}%
\definecolor{currentstroke}{rgb}{0.298039,0.447059,0.690196}%
\pgfsetstrokecolor{currentstroke}%
\pgfsetdash{}{0pt}%
\pgfpathmoveto{\pgfqpoint{3.905098in}{1.936025in}}%
\pgfpathcurveto{\pgfqpoint{3.913334in}{1.936025in}}{\pgfqpoint{3.921234in}{1.939298in}}{\pgfqpoint{3.927058in}{1.945122in}}%
\pgfpathcurveto{\pgfqpoint{3.932882in}{1.950946in}}{\pgfqpoint{3.936155in}{1.958846in}}{\pgfqpoint{3.936155in}{1.967082in}}%
\pgfpathcurveto{\pgfqpoint{3.936155in}{1.975318in}}{\pgfqpoint{3.932882in}{1.983218in}}{\pgfqpoint{3.927058in}{1.989042in}}%
\pgfpathcurveto{\pgfqpoint{3.921234in}{1.994866in}}{\pgfqpoint{3.913334in}{1.998138in}}{\pgfqpoint{3.905098in}{1.998138in}}%
\pgfpathcurveto{\pgfqpoint{3.896862in}{1.998138in}}{\pgfqpoint{3.888962in}{1.994866in}}{\pgfqpoint{3.883138in}{1.989042in}}%
\pgfpathcurveto{\pgfqpoint{3.877314in}{1.983218in}}{\pgfqpoint{3.874042in}{1.975318in}}{\pgfqpoint{3.874042in}{1.967082in}}%
\pgfpathcurveto{\pgfqpoint{3.874042in}{1.958846in}}{\pgfqpoint{3.877314in}{1.950946in}}{\pgfqpoint{3.883138in}{1.945122in}}%
\pgfpathcurveto{\pgfqpoint{3.888962in}{1.939298in}}{\pgfqpoint{3.896862in}{1.936025in}}{\pgfqpoint{3.905098in}{1.936025in}}%
\pgfpathclose%
\pgfusepath{stroke,fill}%
\end{pgfscope}%
\begin{pgfscope}%
\pgfpathrectangle{\pgfqpoint{3.793912in}{0.557870in}}{\pgfqpoint{2.446088in}{1.484734in}}%
\pgfusepath{clip}%
\pgfsetbuttcap%
\pgfsetroundjoin%
\definecolor{currentfill}{rgb}{0.298039,0.447059,0.690196}%
\pgfsetfillcolor{currentfill}%
\pgfsetlinewidth{1.003750pt}%
\definecolor{currentstroke}{rgb}{0.298039,0.447059,0.690196}%
\pgfsetstrokecolor{currentstroke}%
\pgfsetdash{}{0pt}%
\pgfpathmoveto{\pgfqpoint{3.905098in}{1.084392in}}%
\pgfpathcurveto{\pgfqpoint{3.913334in}{1.084392in}}{\pgfqpoint{3.921234in}{1.087665in}}{\pgfqpoint{3.927058in}{1.093489in}}%
\pgfpathcurveto{\pgfqpoint{3.932882in}{1.099313in}}{\pgfqpoint{3.936155in}{1.107213in}}{\pgfqpoint{3.936155in}{1.115449in}}%
\pgfpathcurveto{\pgfqpoint{3.936155in}{1.123685in}}{\pgfqpoint{3.932882in}{1.131585in}}{\pgfqpoint{3.927058in}{1.137409in}}%
\pgfpathcurveto{\pgfqpoint{3.921234in}{1.143233in}}{\pgfqpoint{3.913334in}{1.146505in}}{\pgfqpoint{3.905098in}{1.146505in}}%
\pgfpathcurveto{\pgfqpoint{3.896862in}{1.146505in}}{\pgfqpoint{3.888962in}{1.143233in}}{\pgfqpoint{3.883138in}{1.137409in}}%
\pgfpathcurveto{\pgfqpoint{3.877314in}{1.131585in}}{\pgfqpoint{3.874042in}{1.123685in}}{\pgfqpoint{3.874042in}{1.115449in}}%
\pgfpathcurveto{\pgfqpoint{3.874042in}{1.107213in}}{\pgfqpoint{3.877314in}{1.099313in}}{\pgfqpoint{3.883138in}{1.093489in}}%
\pgfpathcurveto{\pgfqpoint{3.888962in}{1.087665in}}{\pgfqpoint{3.896862in}{1.084392in}}{\pgfqpoint{3.905098in}{1.084392in}}%
\pgfpathclose%
\pgfusepath{stroke,fill}%
\end{pgfscope}%
\begin{pgfscope}%
\pgfpathrectangle{\pgfqpoint{3.793912in}{0.557870in}}{\pgfqpoint{2.446088in}{1.484734in}}%
\pgfusepath{clip}%
\pgfsetbuttcap%
\pgfsetroundjoin%
\definecolor{currentfill}{rgb}{0.298039,0.447059,0.690196}%
\pgfsetfillcolor{currentfill}%
\pgfsetlinewidth{1.003750pt}%
\definecolor{currentstroke}{rgb}{0.298039,0.447059,0.690196}%
\pgfsetstrokecolor{currentstroke}%
\pgfsetdash{}{0pt}%
\pgfpathmoveto{\pgfqpoint{3.905098in}{1.936025in}}%
\pgfpathcurveto{\pgfqpoint{3.913334in}{1.936025in}}{\pgfqpoint{3.921234in}{1.939298in}}{\pgfqpoint{3.927058in}{1.945122in}}%
\pgfpathcurveto{\pgfqpoint{3.932882in}{1.950946in}}{\pgfqpoint{3.936155in}{1.958846in}}{\pgfqpoint{3.936155in}{1.967082in}}%
\pgfpathcurveto{\pgfqpoint{3.936155in}{1.975318in}}{\pgfqpoint{3.932882in}{1.983218in}}{\pgfqpoint{3.927058in}{1.989042in}}%
\pgfpathcurveto{\pgfqpoint{3.921234in}{1.994866in}}{\pgfqpoint{3.913334in}{1.998138in}}{\pgfqpoint{3.905098in}{1.998138in}}%
\pgfpathcurveto{\pgfqpoint{3.896862in}{1.998138in}}{\pgfqpoint{3.888962in}{1.994866in}}{\pgfqpoint{3.883138in}{1.989042in}}%
\pgfpathcurveto{\pgfqpoint{3.877314in}{1.983218in}}{\pgfqpoint{3.874042in}{1.975318in}}{\pgfqpoint{3.874042in}{1.967082in}}%
\pgfpathcurveto{\pgfqpoint{3.874042in}{1.958846in}}{\pgfqpoint{3.877314in}{1.950946in}}{\pgfqpoint{3.883138in}{1.945122in}}%
\pgfpathcurveto{\pgfqpoint{3.888962in}{1.939298in}}{\pgfqpoint{3.896862in}{1.936025in}}{\pgfqpoint{3.905098in}{1.936025in}}%
\pgfpathclose%
\pgfusepath{stroke,fill}%
\end{pgfscope}%
\begin{pgfscope}%
\pgfpathrectangle{\pgfqpoint{3.793912in}{0.557870in}}{\pgfqpoint{2.446088in}{1.484734in}}%
\pgfusepath{clip}%
\pgfsetbuttcap%
\pgfsetroundjoin%
\definecolor{currentfill}{rgb}{0.298039,0.447059,0.690196}%
\pgfsetfillcolor{currentfill}%
\pgfsetlinewidth{1.003750pt}%
\definecolor{currentstroke}{rgb}{0.298039,0.447059,0.690196}%
\pgfsetstrokecolor{currentstroke}%
\pgfsetdash{}{0pt}%
\pgfpathmoveto{\pgfqpoint{3.905098in}{0.795159in}}%
\pgfpathcurveto{\pgfqpoint{3.913334in}{0.795159in}}{\pgfqpoint{3.921234in}{0.798431in}}{\pgfqpoint{3.927058in}{0.804255in}}%
\pgfpathcurveto{\pgfqpoint{3.932882in}{0.810079in}}{\pgfqpoint{3.936155in}{0.817979in}}{\pgfqpoint{3.936155in}{0.826215in}}%
\pgfpathcurveto{\pgfqpoint{3.936155in}{0.834451in}}{\pgfqpoint{3.932882in}{0.842351in}}{\pgfqpoint{3.927058in}{0.848175in}}%
\pgfpathcurveto{\pgfqpoint{3.921234in}{0.853999in}}{\pgfqpoint{3.913334in}{0.857272in}}{\pgfqpoint{3.905098in}{0.857272in}}%
\pgfpathcurveto{\pgfqpoint{3.896862in}{0.857272in}}{\pgfqpoint{3.888962in}{0.853999in}}{\pgfqpoint{3.883138in}{0.848175in}}%
\pgfpathcurveto{\pgfqpoint{3.877314in}{0.842351in}}{\pgfqpoint{3.874042in}{0.834451in}}{\pgfqpoint{3.874042in}{0.826215in}}%
\pgfpathcurveto{\pgfqpoint{3.874042in}{0.817979in}}{\pgfqpoint{3.877314in}{0.810079in}}{\pgfqpoint{3.883138in}{0.804255in}}%
\pgfpathcurveto{\pgfqpoint{3.888962in}{0.798431in}}{\pgfqpoint{3.896862in}{0.795159in}}{\pgfqpoint{3.905098in}{0.795159in}}%
\pgfpathclose%
\pgfusepath{stroke,fill}%
\end{pgfscope}%
\begin{pgfscope}%
\pgfpathrectangle{\pgfqpoint{3.793912in}{0.557870in}}{\pgfqpoint{2.446088in}{1.484734in}}%
\pgfusepath{clip}%
\pgfsetbuttcap%
\pgfsetroundjoin%
\definecolor{currentfill}{rgb}{0.298039,0.447059,0.690196}%
\pgfsetfillcolor{currentfill}%
\pgfsetlinewidth{1.003750pt}%
\definecolor{currentstroke}{rgb}{0.298039,0.447059,0.690196}%
\pgfsetstrokecolor{currentstroke}%
\pgfsetdash{}{0pt}%
\pgfpathmoveto{\pgfqpoint{3.905098in}{0.610370in}}%
\pgfpathcurveto{\pgfqpoint{3.913334in}{0.610370in}}{\pgfqpoint{3.921234in}{0.613643in}}{\pgfqpoint{3.927058in}{0.619466in}}%
\pgfpathcurveto{\pgfqpoint{3.932882in}{0.625290in}}{\pgfqpoint{3.936155in}{0.633190in}}{\pgfqpoint{3.936155in}{0.641427in}}%
\pgfpathcurveto{\pgfqpoint{3.936155in}{0.649663in}}{\pgfqpoint{3.932882in}{0.657563in}}{\pgfqpoint{3.927058in}{0.663387in}}%
\pgfpathcurveto{\pgfqpoint{3.921234in}{0.669211in}}{\pgfqpoint{3.913334in}{0.672483in}}{\pgfqpoint{3.905098in}{0.672483in}}%
\pgfpathcurveto{\pgfqpoint{3.896862in}{0.672483in}}{\pgfqpoint{3.888962in}{0.669211in}}{\pgfqpoint{3.883138in}{0.663387in}}%
\pgfpathcurveto{\pgfqpoint{3.877314in}{0.657563in}}{\pgfqpoint{3.874042in}{0.649663in}}{\pgfqpoint{3.874042in}{0.641427in}}%
\pgfpathcurveto{\pgfqpoint{3.874042in}{0.633190in}}{\pgfqpoint{3.877314in}{0.625290in}}{\pgfqpoint{3.883138in}{0.619466in}}%
\pgfpathcurveto{\pgfqpoint{3.888962in}{0.613643in}}{\pgfqpoint{3.896862in}{0.610370in}}{\pgfqpoint{3.905098in}{0.610370in}}%
\pgfpathclose%
\pgfusepath{stroke,fill}%
\end{pgfscope}%
\begin{pgfscope}%
\pgfpathrectangle{\pgfqpoint{3.793912in}{0.557870in}}{\pgfqpoint{2.446088in}{1.484734in}}%
\pgfusepath{clip}%
\pgfsetbuttcap%
\pgfsetroundjoin%
\definecolor{currentfill}{rgb}{0.298039,0.447059,0.690196}%
\pgfsetfillcolor{currentfill}%
\pgfsetlinewidth{1.003750pt}%
\definecolor{currentstroke}{rgb}{0.298039,0.447059,0.690196}%
\pgfsetstrokecolor{currentstroke}%
\pgfsetdash{}{0pt}%
\pgfpathmoveto{\pgfqpoint{3.905098in}{0.907638in}}%
\pgfpathcurveto{\pgfqpoint{3.913334in}{0.907638in}}{\pgfqpoint{3.921234in}{0.910911in}}{\pgfqpoint{3.927058in}{0.916735in}}%
\pgfpathcurveto{\pgfqpoint{3.932882in}{0.922559in}}{\pgfqpoint{3.936155in}{0.930459in}}{\pgfqpoint{3.936155in}{0.938695in}}%
\pgfpathcurveto{\pgfqpoint{3.936155in}{0.946931in}}{\pgfqpoint{3.932882in}{0.954831in}}{\pgfqpoint{3.927058in}{0.960655in}}%
\pgfpathcurveto{\pgfqpoint{3.921234in}{0.966479in}}{\pgfqpoint{3.913334in}{0.969751in}}{\pgfqpoint{3.905098in}{0.969751in}}%
\pgfpathcurveto{\pgfqpoint{3.896862in}{0.969751in}}{\pgfqpoint{3.888962in}{0.966479in}}{\pgfqpoint{3.883138in}{0.960655in}}%
\pgfpathcurveto{\pgfqpoint{3.877314in}{0.954831in}}{\pgfqpoint{3.874042in}{0.946931in}}{\pgfqpoint{3.874042in}{0.938695in}}%
\pgfpathcurveto{\pgfqpoint{3.874042in}{0.930459in}}{\pgfqpoint{3.877314in}{0.922559in}}{\pgfqpoint{3.883138in}{0.916735in}}%
\pgfpathcurveto{\pgfqpoint{3.888962in}{0.910911in}}{\pgfqpoint{3.896862in}{0.907638in}}{\pgfqpoint{3.905098in}{0.907638in}}%
\pgfpathclose%
\pgfusepath{stroke,fill}%
\end{pgfscope}%
\begin{pgfscope}%
\pgfpathrectangle{\pgfqpoint{3.793912in}{0.557870in}}{\pgfqpoint{2.446088in}{1.484734in}}%
\pgfusepath{clip}%
\pgfsetbuttcap%
\pgfsetroundjoin%
\definecolor{currentfill}{rgb}{0.298039,0.447059,0.690196}%
\pgfsetfillcolor{currentfill}%
\pgfsetlinewidth{1.003750pt}%
\definecolor{currentstroke}{rgb}{0.298039,0.447059,0.690196}%
\pgfsetstrokecolor{currentstroke}%
\pgfsetdash{}{0pt}%
\pgfpathmoveto{\pgfqpoint{3.905098in}{1.936025in}}%
\pgfpathcurveto{\pgfqpoint{3.913334in}{1.936025in}}{\pgfqpoint{3.921234in}{1.939298in}}{\pgfqpoint{3.927058in}{1.945122in}}%
\pgfpathcurveto{\pgfqpoint{3.932882in}{1.950946in}}{\pgfqpoint{3.936155in}{1.958846in}}{\pgfqpoint{3.936155in}{1.967082in}}%
\pgfpathcurveto{\pgfqpoint{3.936155in}{1.975318in}}{\pgfqpoint{3.932882in}{1.983218in}}{\pgfqpoint{3.927058in}{1.989042in}}%
\pgfpathcurveto{\pgfqpoint{3.921234in}{1.994866in}}{\pgfqpoint{3.913334in}{1.998138in}}{\pgfqpoint{3.905098in}{1.998138in}}%
\pgfpathcurveto{\pgfqpoint{3.896862in}{1.998138in}}{\pgfqpoint{3.888962in}{1.994866in}}{\pgfqpoint{3.883138in}{1.989042in}}%
\pgfpathcurveto{\pgfqpoint{3.877314in}{1.983218in}}{\pgfqpoint{3.874042in}{1.975318in}}{\pgfqpoint{3.874042in}{1.967082in}}%
\pgfpathcurveto{\pgfqpoint{3.874042in}{1.958846in}}{\pgfqpoint{3.877314in}{1.950946in}}{\pgfqpoint{3.883138in}{1.945122in}}%
\pgfpathcurveto{\pgfqpoint{3.888962in}{1.939298in}}{\pgfqpoint{3.896862in}{1.936025in}}{\pgfqpoint{3.905098in}{1.936025in}}%
\pgfpathclose%
\pgfusepath{stroke,fill}%
\end{pgfscope}%
\begin{pgfscope}%
\pgfpathrectangle{\pgfqpoint{3.793912in}{0.557870in}}{\pgfqpoint{2.446088in}{1.484734in}}%
\pgfusepath{clip}%
\pgfsetbuttcap%
\pgfsetroundjoin%
\definecolor{currentfill}{rgb}{0.298039,0.447059,0.690196}%
\pgfsetfillcolor{currentfill}%
\pgfsetlinewidth{1.003750pt}%
\definecolor{currentstroke}{rgb}{0.298039,0.447059,0.690196}%
\pgfsetstrokecolor{currentstroke}%
\pgfsetdash{}{0pt}%
\pgfpathmoveto{\pgfqpoint{3.905098in}{1.936025in}}%
\pgfpathcurveto{\pgfqpoint{3.913334in}{1.936025in}}{\pgfqpoint{3.921234in}{1.939298in}}{\pgfqpoint{3.927058in}{1.945122in}}%
\pgfpathcurveto{\pgfqpoint{3.932882in}{1.950946in}}{\pgfqpoint{3.936155in}{1.958846in}}{\pgfqpoint{3.936155in}{1.967082in}}%
\pgfpathcurveto{\pgfqpoint{3.936155in}{1.975318in}}{\pgfqpoint{3.932882in}{1.983218in}}{\pgfqpoint{3.927058in}{1.989042in}}%
\pgfpathcurveto{\pgfqpoint{3.921234in}{1.994866in}}{\pgfqpoint{3.913334in}{1.998138in}}{\pgfqpoint{3.905098in}{1.998138in}}%
\pgfpathcurveto{\pgfqpoint{3.896862in}{1.998138in}}{\pgfqpoint{3.888962in}{1.994866in}}{\pgfqpoint{3.883138in}{1.989042in}}%
\pgfpathcurveto{\pgfqpoint{3.877314in}{1.983218in}}{\pgfqpoint{3.874042in}{1.975318in}}{\pgfqpoint{3.874042in}{1.967082in}}%
\pgfpathcurveto{\pgfqpoint{3.874042in}{1.958846in}}{\pgfqpoint{3.877314in}{1.950946in}}{\pgfqpoint{3.883138in}{1.945122in}}%
\pgfpathcurveto{\pgfqpoint{3.888962in}{1.939298in}}{\pgfqpoint{3.896862in}{1.936025in}}{\pgfqpoint{3.905098in}{1.936025in}}%
\pgfpathclose%
\pgfusepath{stroke,fill}%
\end{pgfscope}%
\begin{pgfscope}%
\pgfpathrectangle{\pgfqpoint{3.793912in}{0.557870in}}{\pgfqpoint{2.446088in}{1.484734in}}%
\pgfusepath{clip}%
\pgfsetbuttcap%
\pgfsetroundjoin%
\definecolor{currentfill}{rgb}{0.298039,0.447059,0.690196}%
\pgfsetfillcolor{currentfill}%
\pgfsetlinewidth{1.003750pt}%
\definecolor{currentstroke}{rgb}{0.298039,0.447059,0.690196}%
\pgfsetstrokecolor{currentstroke}%
\pgfsetdash{}{0pt}%
\pgfpathmoveto{\pgfqpoint{3.905098in}{1.936025in}}%
\pgfpathcurveto{\pgfqpoint{3.913334in}{1.936025in}}{\pgfqpoint{3.921234in}{1.939298in}}{\pgfqpoint{3.927058in}{1.945122in}}%
\pgfpathcurveto{\pgfqpoint{3.932882in}{1.950946in}}{\pgfqpoint{3.936155in}{1.958846in}}{\pgfqpoint{3.936155in}{1.967082in}}%
\pgfpathcurveto{\pgfqpoint{3.936155in}{1.975318in}}{\pgfqpoint{3.932882in}{1.983218in}}{\pgfqpoint{3.927058in}{1.989042in}}%
\pgfpathcurveto{\pgfqpoint{3.921234in}{1.994866in}}{\pgfqpoint{3.913334in}{1.998138in}}{\pgfqpoint{3.905098in}{1.998138in}}%
\pgfpathcurveto{\pgfqpoint{3.896862in}{1.998138in}}{\pgfqpoint{3.888962in}{1.994866in}}{\pgfqpoint{3.883138in}{1.989042in}}%
\pgfpathcurveto{\pgfqpoint{3.877314in}{1.983218in}}{\pgfqpoint{3.874042in}{1.975318in}}{\pgfqpoint{3.874042in}{1.967082in}}%
\pgfpathcurveto{\pgfqpoint{3.874042in}{1.958846in}}{\pgfqpoint{3.877314in}{1.950946in}}{\pgfqpoint{3.883138in}{1.945122in}}%
\pgfpathcurveto{\pgfqpoint{3.888962in}{1.939298in}}{\pgfqpoint{3.896862in}{1.936025in}}{\pgfqpoint{3.905098in}{1.936025in}}%
\pgfpathclose%
\pgfusepath{stroke,fill}%
\end{pgfscope}%
\begin{pgfscope}%
\pgfpathrectangle{\pgfqpoint{3.793912in}{0.557870in}}{\pgfqpoint{2.446088in}{1.484734in}}%
\pgfusepath{clip}%
\pgfsetbuttcap%
\pgfsetroundjoin%
\definecolor{currentfill}{rgb}{0.298039,0.447059,0.690196}%
\pgfsetfillcolor{currentfill}%
\pgfsetlinewidth{1.003750pt}%
\definecolor{currentstroke}{rgb}{0.298039,0.447059,0.690196}%
\pgfsetstrokecolor{currentstroke}%
\pgfsetdash{}{0pt}%
\pgfpathmoveto{\pgfqpoint{3.905098in}{1.936025in}}%
\pgfpathcurveto{\pgfqpoint{3.913334in}{1.936025in}}{\pgfqpoint{3.921234in}{1.939298in}}{\pgfqpoint{3.927058in}{1.945122in}}%
\pgfpathcurveto{\pgfqpoint{3.932882in}{1.950946in}}{\pgfqpoint{3.936155in}{1.958846in}}{\pgfqpoint{3.936155in}{1.967082in}}%
\pgfpathcurveto{\pgfqpoint{3.936155in}{1.975318in}}{\pgfqpoint{3.932882in}{1.983218in}}{\pgfqpoint{3.927058in}{1.989042in}}%
\pgfpathcurveto{\pgfqpoint{3.921234in}{1.994866in}}{\pgfqpoint{3.913334in}{1.998138in}}{\pgfqpoint{3.905098in}{1.998138in}}%
\pgfpathcurveto{\pgfqpoint{3.896862in}{1.998138in}}{\pgfqpoint{3.888962in}{1.994866in}}{\pgfqpoint{3.883138in}{1.989042in}}%
\pgfpathcurveto{\pgfqpoint{3.877314in}{1.983218in}}{\pgfqpoint{3.874042in}{1.975318in}}{\pgfqpoint{3.874042in}{1.967082in}}%
\pgfpathcurveto{\pgfqpoint{3.874042in}{1.958846in}}{\pgfqpoint{3.877314in}{1.950946in}}{\pgfqpoint{3.883138in}{1.945122in}}%
\pgfpathcurveto{\pgfqpoint{3.888962in}{1.939298in}}{\pgfqpoint{3.896862in}{1.936025in}}{\pgfqpoint{3.905098in}{1.936025in}}%
\pgfpathclose%
\pgfusepath{stroke,fill}%
\end{pgfscope}%
\begin{pgfscope}%
\pgfpathrectangle{\pgfqpoint{3.793912in}{0.557870in}}{\pgfqpoint{2.446088in}{1.484734in}}%
\pgfusepath{clip}%
\pgfsetbuttcap%
\pgfsetroundjoin%
\definecolor{currentfill}{rgb}{0.298039,0.447059,0.690196}%
\pgfsetfillcolor{currentfill}%
\pgfsetlinewidth{1.003750pt}%
\definecolor{currentstroke}{rgb}{0.298039,0.447059,0.690196}%
\pgfsetstrokecolor{currentstroke}%
\pgfsetdash{}{0pt}%
\pgfpathmoveto{\pgfqpoint{3.905098in}{1.325421in}}%
\pgfpathcurveto{\pgfqpoint{3.913334in}{1.325421in}}{\pgfqpoint{3.921234in}{1.328693in}}{\pgfqpoint{3.927058in}{1.334517in}}%
\pgfpathcurveto{\pgfqpoint{3.932882in}{1.340341in}}{\pgfqpoint{3.936155in}{1.348241in}}{\pgfqpoint{3.936155in}{1.356477in}}%
\pgfpathcurveto{\pgfqpoint{3.936155in}{1.364713in}}{\pgfqpoint{3.932882in}{1.372613in}}{\pgfqpoint{3.927058in}{1.378437in}}%
\pgfpathcurveto{\pgfqpoint{3.921234in}{1.384261in}}{\pgfqpoint{3.913334in}{1.387534in}}{\pgfqpoint{3.905098in}{1.387534in}}%
\pgfpathcurveto{\pgfqpoint{3.896862in}{1.387534in}}{\pgfqpoint{3.888962in}{1.384261in}}{\pgfqpoint{3.883138in}{1.378437in}}%
\pgfpathcurveto{\pgfqpoint{3.877314in}{1.372613in}}{\pgfqpoint{3.874042in}{1.364713in}}{\pgfqpoint{3.874042in}{1.356477in}}%
\pgfpathcurveto{\pgfqpoint{3.874042in}{1.348241in}}{\pgfqpoint{3.877314in}{1.340341in}}{\pgfqpoint{3.883138in}{1.334517in}}%
\pgfpathcurveto{\pgfqpoint{3.888962in}{1.328693in}}{\pgfqpoint{3.896862in}{1.325421in}}{\pgfqpoint{3.905098in}{1.325421in}}%
\pgfpathclose%
\pgfusepath{stroke,fill}%
\end{pgfscope}%
\begin{pgfscope}%
\pgfpathrectangle{\pgfqpoint{3.793912in}{0.557870in}}{\pgfqpoint{2.446088in}{1.484734in}}%
\pgfusepath{clip}%
\pgfsetbuttcap%
\pgfsetroundjoin%
\definecolor{currentfill}{rgb}{0.298039,0.447059,0.690196}%
\pgfsetfillcolor{currentfill}%
\pgfsetlinewidth{1.003750pt}%
\definecolor{currentstroke}{rgb}{0.298039,0.447059,0.690196}%
\pgfsetstrokecolor{currentstroke}%
\pgfsetdash{}{0pt}%
\pgfpathmoveto{\pgfqpoint{3.905098in}{0.610370in}}%
\pgfpathcurveto{\pgfqpoint{3.913334in}{0.610370in}}{\pgfqpoint{3.921234in}{0.613643in}}{\pgfqpoint{3.927058in}{0.619466in}}%
\pgfpathcurveto{\pgfqpoint{3.932882in}{0.625290in}}{\pgfqpoint{3.936155in}{0.633190in}}{\pgfqpoint{3.936155in}{0.641427in}}%
\pgfpathcurveto{\pgfqpoint{3.936155in}{0.649663in}}{\pgfqpoint{3.932882in}{0.657563in}}{\pgfqpoint{3.927058in}{0.663387in}}%
\pgfpathcurveto{\pgfqpoint{3.921234in}{0.669211in}}{\pgfqpoint{3.913334in}{0.672483in}}{\pgfqpoint{3.905098in}{0.672483in}}%
\pgfpathcurveto{\pgfqpoint{3.896862in}{0.672483in}}{\pgfqpoint{3.888962in}{0.669211in}}{\pgfqpoint{3.883138in}{0.663387in}}%
\pgfpathcurveto{\pgfqpoint{3.877314in}{0.657563in}}{\pgfqpoint{3.874042in}{0.649663in}}{\pgfqpoint{3.874042in}{0.641427in}}%
\pgfpathcurveto{\pgfqpoint{3.874042in}{0.633190in}}{\pgfqpoint{3.877314in}{0.625290in}}{\pgfqpoint{3.883138in}{0.619466in}}%
\pgfpathcurveto{\pgfqpoint{3.888962in}{0.613643in}}{\pgfqpoint{3.896862in}{0.610370in}}{\pgfqpoint{3.905098in}{0.610370in}}%
\pgfpathclose%
\pgfusepath{stroke,fill}%
\end{pgfscope}%
\begin{pgfscope}%
\pgfpathrectangle{\pgfqpoint{3.793912in}{0.557870in}}{\pgfqpoint{2.446088in}{1.484734in}}%
\pgfusepath{clip}%
\pgfsetbuttcap%
\pgfsetroundjoin%
\definecolor{currentfill}{rgb}{0.298039,0.447059,0.690196}%
\pgfsetfillcolor{currentfill}%
\pgfsetlinewidth{1.003750pt}%
\definecolor{currentstroke}{rgb}{0.298039,0.447059,0.690196}%
\pgfsetstrokecolor{currentstroke}%
\pgfsetdash{}{0pt}%
\pgfpathmoveto{\pgfqpoint{3.905098in}{1.341489in}}%
\pgfpathcurveto{\pgfqpoint{3.913334in}{1.341489in}}{\pgfqpoint{3.921234in}{1.344761in}}{\pgfqpoint{3.927058in}{1.350585in}}%
\pgfpathcurveto{\pgfqpoint{3.932882in}{1.356409in}}{\pgfqpoint{3.936155in}{1.364309in}}{\pgfqpoint{3.936155in}{1.372546in}}%
\pgfpathcurveto{\pgfqpoint{3.936155in}{1.380782in}}{\pgfqpoint{3.932882in}{1.388682in}}{\pgfqpoint{3.927058in}{1.394506in}}%
\pgfpathcurveto{\pgfqpoint{3.921234in}{1.400330in}}{\pgfqpoint{3.913334in}{1.403602in}}{\pgfqpoint{3.905098in}{1.403602in}}%
\pgfpathcurveto{\pgfqpoint{3.896862in}{1.403602in}}{\pgfqpoint{3.888962in}{1.400330in}}{\pgfqpoint{3.883138in}{1.394506in}}%
\pgfpathcurveto{\pgfqpoint{3.877314in}{1.388682in}}{\pgfqpoint{3.874042in}{1.380782in}}{\pgfqpoint{3.874042in}{1.372546in}}%
\pgfpathcurveto{\pgfqpoint{3.874042in}{1.364309in}}{\pgfqpoint{3.877314in}{1.356409in}}{\pgfqpoint{3.883138in}{1.350585in}}%
\pgfpathcurveto{\pgfqpoint{3.888962in}{1.344761in}}{\pgfqpoint{3.896862in}{1.341489in}}{\pgfqpoint{3.905098in}{1.341489in}}%
\pgfpathclose%
\pgfusepath{stroke,fill}%
\end{pgfscope}%
\begin{pgfscope}%
\pgfpathrectangle{\pgfqpoint{3.793912in}{0.557870in}}{\pgfqpoint{2.446088in}{1.484734in}}%
\pgfusepath{clip}%
\pgfsetbuttcap%
\pgfsetroundjoin%
\definecolor{currentfill}{rgb}{0.298039,0.447059,0.690196}%
\pgfsetfillcolor{currentfill}%
\pgfsetlinewidth{1.003750pt}%
\definecolor{currentstroke}{rgb}{0.298039,0.447059,0.690196}%
\pgfsetstrokecolor{currentstroke}%
\pgfsetdash{}{0pt}%
\pgfpathmoveto{\pgfqpoint{3.905098in}{1.936025in}}%
\pgfpathcurveto{\pgfqpoint{3.913334in}{1.936025in}}{\pgfqpoint{3.921234in}{1.939298in}}{\pgfqpoint{3.927058in}{1.945122in}}%
\pgfpathcurveto{\pgfqpoint{3.932882in}{1.950946in}}{\pgfqpoint{3.936155in}{1.958846in}}{\pgfqpoint{3.936155in}{1.967082in}}%
\pgfpathcurveto{\pgfqpoint{3.936155in}{1.975318in}}{\pgfqpoint{3.932882in}{1.983218in}}{\pgfqpoint{3.927058in}{1.989042in}}%
\pgfpathcurveto{\pgfqpoint{3.921234in}{1.994866in}}{\pgfqpoint{3.913334in}{1.998138in}}{\pgfqpoint{3.905098in}{1.998138in}}%
\pgfpathcurveto{\pgfqpoint{3.896862in}{1.998138in}}{\pgfqpoint{3.888962in}{1.994866in}}{\pgfqpoint{3.883138in}{1.989042in}}%
\pgfpathcurveto{\pgfqpoint{3.877314in}{1.983218in}}{\pgfqpoint{3.874042in}{1.975318in}}{\pgfqpoint{3.874042in}{1.967082in}}%
\pgfpathcurveto{\pgfqpoint{3.874042in}{1.958846in}}{\pgfqpoint{3.877314in}{1.950946in}}{\pgfqpoint{3.883138in}{1.945122in}}%
\pgfpathcurveto{\pgfqpoint{3.888962in}{1.939298in}}{\pgfqpoint{3.896862in}{1.936025in}}{\pgfqpoint{3.905098in}{1.936025in}}%
\pgfpathclose%
\pgfusepath{stroke,fill}%
\end{pgfscope}%
\begin{pgfscope}%
\pgfpathrectangle{\pgfqpoint{3.793912in}{0.557870in}}{\pgfqpoint{2.446088in}{1.484734in}}%
\pgfusepath{clip}%
\pgfsetbuttcap%
\pgfsetroundjoin%
\definecolor{currentfill}{rgb}{0.298039,0.447059,0.690196}%
\pgfsetfillcolor{currentfill}%
\pgfsetlinewidth{1.003750pt}%
\definecolor{currentstroke}{rgb}{0.298039,0.447059,0.690196}%
\pgfsetstrokecolor{currentstroke}%
\pgfsetdash{}{0pt}%
\pgfpathmoveto{\pgfqpoint{3.905098in}{0.610370in}}%
\pgfpathcurveto{\pgfqpoint{3.913334in}{0.610370in}}{\pgfqpoint{3.921234in}{0.613643in}}{\pgfqpoint{3.927058in}{0.619466in}}%
\pgfpathcurveto{\pgfqpoint{3.932882in}{0.625290in}}{\pgfqpoint{3.936155in}{0.633190in}}{\pgfqpoint{3.936155in}{0.641427in}}%
\pgfpathcurveto{\pgfqpoint{3.936155in}{0.649663in}}{\pgfqpoint{3.932882in}{0.657563in}}{\pgfqpoint{3.927058in}{0.663387in}}%
\pgfpathcurveto{\pgfqpoint{3.921234in}{0.669211in}}{\pgfqpoint{3.913334in}{0.672483in}}{\pgfqpoint{3.905098in}{0.672483in}}%
\pgfpathcurveto{\pgfqpoint{3.896862in}{0.672483in}}{\pgfqpoint{3.888962in}{0.669211in}}{\pgfqpoint{3.883138in}{0.663387in}}%
\pgfpathcurveto{\pgfqpoint{3.877314in}{0.657563in}}{\pgfqpoint{3.874042in}{0.649663in}}{\pgfqpoint{3.874042in}{0.641427in}}%
\pgfpathcurveto{\pgfqpoint{3.874042in}{0.633190in}}{\pgfqpoint{3.877314in}{0.625290in}}{\pgfqpoint{3.883138in}{0.619466in}}%
\pgfpathcurveto{\pgfqpoint{3.888962in}{0.613643in}}{\pgfqpoint{3.896862in}{0.610370in}}{\pgfqpoint{3.905098in}{0.610370in}}%
\pgfpathclose%
\pgfusepath{stroke,fill}%
\end{pgfscope}%
\begin{pgfscope}%
\pgfpathrectangle{\pgfqpoint{3.793912in}{0.557870in}}{\pgfqpoint{2.446088in}{1.484734in}}%
\pgfusepath{clip}%
\pgfsetbuttcap%
\pgfsetroundjoin%
\definecolor{currentfill}{rgb}{0.298039,0.447059,0.690196}%
\pgfsetfillcolor{currentfill}%
\pgfsetlinewidth{1.003750pt}%
\definecolor{currentstroke}{rgb}{0.298039,0.447059,0.690196}%
\pgfsetstrokecolor{currentstroke}%
\pgfsetdash{}{0pt}%
\pgfpathmoveto{\pgfqpoint{3.905098in}{1.936025in}}%
\pgfpathcurveto{\pgfqpoint{3.913334in}{1.936025in}}{\pgfqpoint{3.921234in}{1.939298in}}{\pgfqpoint{3.927058in}{1.945122in}}%
\pgfpathcurveto{\pgfqpoint{3.932882in}{1.950946in}}{\pgfqpoint{3.936155in}{1.958846in}}{\pgfqpoint{3.936155in}{1.967082in}}%
\pgfpathcurveto{\pgfqpoint{3.936155in}{1.975318in}}{\pgfqpoint{3.932882in}{1.983218in}}{\pgfqpoint{3.927058in}{1.989042in}}%
\pgfpathcurveto{\pgfqpoint{3.921234in}{1.994866in}}{\pgfqpoint{3.913334in}{1.998138in}}{\pgfqpoint{3.905098in}{1.998138in}}%
\pgfpathcurveto{\pgfqpoint{3.896862in}{1.998138in}}{\pgfqpoint{3.888962in}{1.994866in}}{\pgfqpoint{3.883138in}{1.989042in}}%
\pgfpathcurveto{\pgfqpoint{3.877314in}{1.983218in}}{\pgfqpoint{3.874042in}{1.975318in}}{\pgfqpoint{3.874042in}{1.967082in}}%
\pgfpathcurveto{\pgfqpoint{3.874042in}{1.958846in}}{\pgfqpoint{3.877314in}{1.950946in}}{\pgfqpoint{3.883138in}{1.945122in}}%
\pgfpathcurveto{\pgfqpoint{3.888962in}{1.939298in}}{\pgfqpoint{3.896862in}{1.936025in}}{\pgfqpoint{3.905098in}{1.936025in}}%
\pgfpathclose%
\pgfusepath{stroke,fill}%
\end{pgfscope}%
\begin{pgfscope}%
\pgfpathrectangle{\pgfqpoint{3.793912in}{0.557870in}}{\pgfqpoint{2.446088in}{1.484734in}}%
\pgfusepath{clip}%
\pgfsetbuttcap%
\pgfsetroundjoin%
\definecolor{currentfill}{rgb}{0.298039,0.447059,0.690196}%
\pgfsetfillcolor{currentfill}%
\pgfsetlinewidth{1.003750pt}%
\definecolor{currentstroke}{rgb}{0.298039,0.447059,0.690196}%
\pgfsetstrokecolor{currentstroke}%
\pgfsetdash{}{0pt}%
\pgfpathmoveto{\pgfqpoint{3.905098in}{1.172769in}}%
\pgfpathcurveto{\pgfqpoint{3.913334in}{1.172769in}}{\pgfqpoint{3.921234in}{1.176042in}}{\pgfqpoint{3.927058in}{1.181866in}}%
\pgfpathcurveto{\pgfqpoint{3.932882in}{1.187690in}}{\pgfqpoint{3.936155in}{1.195590in}}{\pgfqpoint{3.936155in}{1.203826in}}%
\pgfpathcurveto{\pgfqpoint{3.936155in}{1.212062in}}{\pgfqpoint{3.932882in}{1.219962in}}{\pgfqpoint{3.927058in}{1.225786in}}%
\pgfpathcurveto{\pgfqpoint{3.921234in}{1.231610in}}{\pgfqpoint{3.913334in}{1.234882in}}{\pgfqpoint{3.905098in}{1.234882in}}%
\pgfpathcurveto{\pgfqpoint{3.896862in}{1.234882in}}{\pgfqpoint{3.888962in}{1.231610in}}{\pgfqpoint{3.883138in}{1.225786in}}%
\pgfpathcurveto{\pgfqpoint{3.877314in}{1.219962in}}{\pgfqpoint{3.874042in}{1.212062in}}{\pgfqpoint{3.874042in}{1.203826in}}%
\pgfpathcurveto{\pgfqpoint{3.874042in}{1.195590in}}{\pgfqpoint{3.877314in}{1.187690in}}{\pgfqpoint{3.883138in}{1.181866in}}%
\pgfpathcurveto{\pgfqpoint{3.888962in}{1.176042in}}{\pgfqpoint{3.896862in}{1.172769in}}{\pgfqpoint{3.905098in}{1.172769in}}%
\pgfpathclose%
\pgfusepath{stroke,fill}%
\end{pgfscope}%
\begin{pgfscope}%
\pgfpathrectangle{\pgfqpoint{3.793912in}{0.557870in}}{\pgfqpoint{2.446088in}{1.484734in}}%
\pgfusepath{clip}%
\pgfsetbuttcap%
\pgfsetroundjoin%
\definecolor{currentfill}{rgb}{0.298039,0.447059,0.690196}%
\pgfsetfillcolor{currentfill}%
\pgfsetlinewidth{1.003750pt}%
\definecolor{currentstroke}{rgb}{0.298039,0.447059,0.690196}%
\pgfsetstrokecolor{currentstroke}%
\pgfsetdash{}{0pt}%
\pgfpathmoveto{\pgfqpoint{3.905098in}{1.245078in}}%
\pgfpathcurveto{\pgfqpoint{3.913334in}{1.245078in}}{\pgfqpoint{3.921234in}{1.248350in}}{\pgfqpoint{3.927058in}{1.254174in}}%
\pgfpathcurveto{\pgfqpoint{3.932882in}{1.259998in}}{\pgfqpoint{3.936155in}{1.267898in}}{\pgfqpoint{3.936155in}{1.276134in}}%
\pgfpathcurveto{\pgfqpoint{3.936155in}{1.284371in}}{\pgfqpoint{3.932882in}{1.292271in}}{\pgfqpoint{3.927058in}{1.298095in}}%
\pgfpathcurveto{\pgfqpoint{3.921234in}{1.303919in}}{\pgfqpoint{3.913334in}{1.307191in}}{\pgfqpoint{3.905098in}{1.307191in}}%
\pgfpathcurveto{\pgfqpoint{3.896862in}{1.307191in}}{\pgfqpoint{3.888962in}{1.303919in}}{\pgfqpoint{3.883138in}{1.298095in}}%
\pgfpathcurveto{\pgfqpoint{3.877314in}{1.292271in}}{\pgfqpoint{3.874042in}{1.284371in}}{\pgfqpoint{3.874042in}{1.276134in}}%
\pgfpathcurveto{\pgfqpoint{3.874042in}{1.267898in}}{\pgfqpoint{3.877314in}{1.259998in}}{\pgfqpoint{3.883138in}{1.254174in}}%
\pgfpathcurveto{\pgfqpoint{3.888962in}{1.248350in}}{\pgfqpoint{3.896862in}{1.245078in}}{\pgfqpoint{3.905098in}{1.245078in}}%
\pgfpathclose%
\pgfusepath{stroke,fill}%
\end{pgfscope}%
\begin{pgfscope}%
\pgfpathrectangle{\pgfqpoint{3.793912in}{0.557870in}}{\pgfqpoint{2.446088in}{1.484734in}}%
\pgfusepath{clip}%
\pgfsetbuttcap%
\pgfsetroundjoin%
\definecolor{currentfill}{rgb}{0.298039,0.447059,0.690196}%
\pgfsetfillcolor{currentfill}%
\pgfsetlinewidth{1.003750pt}%
\definecolor{currentstroke}{rgb}{0.298039,0.447059,0.690196}%
\pgfsetstrokecolor{currentstroke}%
\pgfsetdash{}{0pt}%
\pgfpathmoveto{\pgfqpoint{3.905098in}{1.188838in}}%
\pgfpathcurveto{\pgfqpoint{3.913334in}{1.188838in}}{\pgfqpoint{3.921234in}{1.192110in}}{\pgfqpoint{3.927058in}{1.197934in}}%
\pgfpathcurveto{\pgfqpoint{3.932882in}{1.203758in}}{\pgfqpoint{3.936155in}{1.211658in}}{\pgfqpoint{3.936155in}{1.219894in}}%
\pgfpathcurveto{\pgfqpoint{3.936155in}{1.228131in}}{\pgfqpoint{3.932882in}{1.236031in}}{\pgfqpoint{3.927058in}{1.241855in}}%
\pgfpathcurveto{\pgfqpoint{3.921234in}{1.247679in}}{\pgfqpoint{3.913334in}{1.250951in}}{\pgfqpoint{3.905098in}{1.250951in}}%
\pgfpathcurveto{\pgfqpoint{3.896862in}{1.250951in}}{\pgfqpoint{3.888962in}{1.247679in}}{\pgfqpoint{3.883138in}{1.241855in}}%
\pgfpathcurveto{\pgfqpoint{3.877314in}{1.236031in}}{\pgfqpoint{3.874042in}{1.228131in}}{\pgfqpoint{3.874042in}{1.219894in}}%
\pgfpathcurveto{\pgfqpoint{3.874042in}{1.211658in}}{\pgfqpoint{3.877314in}{1.203758in}}{\pgfqpoint{3.883138in}{1.197934in}}%
\pgfpathcurveto{\pgfqpoint{3.888962in}{1.192110in}}{\pgfqpoint{3.896862in}{1.188838in}}{\pgfqpoint{3.905098in}{1.188838in}}%
\pgfpathclose%
\pgfusepath{stroke,fill}%
\end{pgfscope}%
\begin{pgfscope}%
\pgfpathrectangle{\pgfqpoint{3.793912in}{0.557870in}}{\pgfqpoint{2.446088in}{1.484734in}}%
\pgfusepath{clip}%
\pgfsetbuttcap%
\pgfsetroundjoin%
\definecolor{currentfill}{rgb}{0.298039,0.447059,0.690196}%
\pgfsetfillcolor{currentfill}%
\pgfsetlinewidth{1.003750pt}%
\definecolor{currentstroke}{rgb}{0.298039,0.447059,0.690196}%
\pgfsetstrokecolor{currentstroke}%
\pgfsetdash{}{0pt}%
\pgfpathmoveto{\pgfqpoint{3.905098in}{1.261146in}}%
\pgfpathcurveto{\pgfqpoint{3.913334in}{1.261146in}}{\pgfqpoint{3.921234in}{1.264419in}}{\pgfqpoint{3.927058in}{1.270243in}}%
\pgfpathcurveto{\pgfqpoint{3.932882in}{1.276067in}}{\pgfqpoint{3.936155in}{1.283967in}}{\pgfqpoint{3.936155in}{1.292203in}}%
\pgfpathcurveto{\pgfqpoint{3.936155in}{1.300439in}}{\pgfqpoint{3.932882in}{1.308339in}}{\pgfqpoint{3.927058in}{1.314163in}}%
\pgfpathcurveto{\pgfqpoint{3.921234in}{1.319987in}}{\pgfqpoint{3.913334in}{1.323259in}}{\pgfqpoint{3.905098in}{1.323259in}}%
\pgfpathcurveto{\pgfqpoint{3.896862in}{1.323259in}}{\pgfqpoint{3.888962in}{1.319987in}}{\pgfqpoint{3.883138in}{1.314163in}}%
\pgfpathcurveto{\pgfqpoint{3.877314in}{1.308339in}}{\pgfqpoint{3.874042in}{1.300439in}}{\pgfqpoint{3.874042in}{1.292203in}}%
\pgfpathcurveto{\pgfqpoint{3.874042in}{1.283967in}}{\pgfqpoint{3.877314in}{1.276067in}}{\pgfqpoint{3.883138in}{1.270243in}}%
\pgfpathcurveto{\pgfqpoint{3.888962in}{1.264419in}}{\pgfqpoint{3.896862in}{1.261146in}}{\pgfqpoint{3.905098in}{1.261146in}}%
\pgfpathclose%
\pgfusepath{stroke,fill}%
\end{pgfscope}%
\begin{pgfscope}%
\pgfpathrectangle{\pgfqpoint{3.793912in}{0.557870in}}{\pgfqpoint{2.446088in}{1.484734in}}%
\pgfusepath{clip}%
\pgfsetbuttcap%
\pgfsetroundjoin%
\definecolor{currentfill}{rgb}{0.298039,0.447059,0.690196}%
\pgfsetfillcolor{currentfill}%
\pgfsetlinewidth{1.003750pt}%
\definecolor{currentstroke}{rgb}{0.298039,0.447059,0.690196}%
\pgfsetstrokecolor{currentstroke}%
\pgfsetdash{}{0pt}%
\pgfpathmoveto{\pgfqpoint{3.905098in}{1.936025in}}%
\pgfpathcurveto{\pgfqpoint{3.913334in}{1.936025in}}{\pgfqpoint{3.921234in}{1.939298in}}{\pgfqpoint{3.927058in}{1.945122in}}%
\pgfpathcurveto{\pgfqpoint{3.932882in}{1.950946in}}{\pgfqpoint{3.936155in}{1.958846in}}{\pgfqpoint{3.936155in}{1.967082in}}%
\pgfpathcurveto{\pgfqpoint{3.936155in}{1.975318in}}{\pgfqpoint{3.932882in}{1.983218in}}{\pgfqpoint{3.927058in}{1.989042in}}%
\pgfpathcurveto{\pgfqpoint{3.921234in}{1.994866in}}{\pgfqpoint{3.913334in}{1.998138in}}{\pgfqpoint{3.905098in}{1.998138in}}%
\pgfpathcurveto{\pgfqpoint{3.896862in}{1.998138in}}{\pgfqpoint{3.888962in}{1.994866in}}{\pgfqpoint{3.883138in}{1.989042in}}%
\pgfpathcurveto{\pgfqpoint{3.877314in}{1.983218in}}{\pgfqpoint{3.874042in}{1.975318in}}{\pgfqpoint{3.874042in}{1.967082in}}%
\pgfpathcurveto{\pgfqpoint{3.874042in}{1.958846in}}{\pgfqpoint{3.877314in}{1.950946in}}{\pgfqpoint{3.883138in}{1.945122in}}%
\pgfpathcurveto{\pgfqpoint{3.888962in}{1.939298in}}{\pgfqpoint{3.896862in}{1.936025in}}{\pgfqpoint{3.905098in}{1.936025in}}%
\pgfpathclose%
\pgfusepath{stroke,fill}%
\end{pgfscope}%
\begin{pgfscope}%
\pgfpathrectangle{\pgfqpoint{3.793912in}{0.557870in}}{\pgfqpoint{2.446088in}{1.484734in}}%
\pgfusepath{clip}%
\pgfsetbuttcap%
\pgfsetroundjoin%
\definecolor{currentfill}{rgb}{0.298039,0.447059,0.690196}%
\pgfsetfillcolor{currentfill}%
\pgfsetlinewidth{1.003750pt}%
\definecolor{currentstroke}{rgb}{0.298039,0.447059,0.690196}%
\pgfsetstrokecolor{currentstroke}%
\pgfsetdash{}{0pt}%
\pgfpathmoveto{\pgfqpoint{3.905098in}{1.237044in}}%
\pgfpathcurveto{\pgfqpoint{3.913334in}{1.237044in}}{\pgfqpoint{3.921234in}{1.240316in}}{\pgfqpoint{3.927058in}{1.246140in}}%
\pgfpathcurveto{\pgfqpoint{3.932882in}{1.251964in}}{\pgfqpoint{3.936155in}{1.259864in}}{\pgfqpoint{3.936155in}{1.268100in}}%
\pgfpathcurveto{\pgfqpoint{3.936155in}{1.276336in}}{\pgfqpoint{3.932882in}{1.284236in}}{\pgfqpoint{3.927058in}{1.290060in}}%
\pgfpathcurveto{\pgfqpoint{3.921234in}{1.295884in}}{\pgfqpoint{3.913334in}{1.299157in}}{\pgfqpoint{3.905098in}{1.299157in}}%
\pgfpathcurveto{\pgfqpoint{3.896862in}{1.299157in}}{\pgfqpoint{3.888962in}{1.295884in}}{\pgfqpoint{3.883138in}{1.290060in}}%
\pgfpathcurveto{\pgfqpoint{3.877314in}{1.284236in}}{\pgfqpoint{3.874042in}{1.276336in}}{\pgfqpoint{3.874042in}{1.268100in}}%
\pgfpathcurveto{\pgfqpoint{3.874042in}{1.259864in}}{\pgfqpoint{3.877314in}{1.251964in}}{\pgfqpoint{3.883138in}{1.246140in}}%
\pgfpathcurveto{\pgfqpoint{3.888962in}{1.240316in}}{\pgfqpoint{3.896862in}{1.237044in}}{\pgfqpoint{3.905098in}{1.237044in}}%
\pgfpathclose%
\pgfusepath{stroke,fill}%
\end{pgfscope}%
\begin{pgfscope}%
\pgfpathrectangle{\pgfqpoint{3.793912in}{0.557870in}}{\pgfqpoint{2.446088in}{1.484734in}}%
\pgfusepath{clip}%
\pgfsetbuttcap%
\pgfsetroundjoin%
\definecolor{currentfill}{rgb}{0.298039,0.447059,0.690196}%
\pgfsetfillcolor{currentfill}%
\pgfsetlinewidth{1.003750pt}%
\definecolor{currentstroke}{rgb}{0.298039,0.447059,0.690196}%
\pgfsetstrokecolor{currentstroke}%
\pgfsetdash{}{0pt}%
\pgfpathmoveto{\pgfqpoint{3.905098in}{1.936025in}}%
\pgfpathcurveto{\pgfqpoint{3.913334in}{1.936025in}}{\pgfqpoint{3.921234in}{1.939298in}}{\pgfqpoint{3.927058in}{1.945122in}}%
\pgfpathcurveto{\pgfqpoint{3.932882in}{1.950946in}}{\pgfqpoint{3.936155in}{1.958846in}}{\pgfqpoint{3.936155in}{1.967082in}}%
\pgfpathcurveto{\pgfqpoint{3.936155in}{1.975318in}}{\pgfqpoint{3.932882in}{1.983218in}}{\pgfqpoint{3.927058in}{1.989042in}}%
\pgfpathcurveto{\pgfqpoint{3.921234in}{1.994866in}}{\pgfqpoint{3.913334in}{1.998138in}}{\pgfqpoint{3.905098in}{1.998138in}}%
\pgfpathcurveto{\pgfqpoint{3.896862in}{1.998138in}}{\pgfqpoint{3.888962in}{1.994866in}}{\pgfqpoint{3.883138in}{1.989042in}}%
\pgfpathcurveto{\pgfqpoint{3.877314in}{1.983218in}}{\pgfqpoint{3.874042in}{1.975318in}}{\pgfqpoint{3.874042in}{1.967082in}}%
\pgfpathcurveto{\pgfqpoint{3.874042in}{1.958846in}}{\pgfqpoint{3.877314in}{1.950946in}}{\pgfqpoint{3.883138in}{1.945122in}}%
\pgfpathcurveto{\pgfqpoint{3.888962in}{1.939298in}}{\pgfqpoint{3.896862in}{1.936025in}}{\pgfqpoint{3.905098in}{1.936025in}}%
\pgfpathclose%
\pgfusepath{stroke,fill}%
\end{pgfscope}%
\begin{pgfscope}%
\pgfpathrectangle{\pgfqpoint{3.793912in}{0.557870in}}{\pgfqpoint{2.446088in}{1.484734in}}%
\pgfusepath{clip}%
\pgfsetbuttcap%
\pgfsetroundjoin%
\definecolor{currentfill}{rgb}{0.298039,0.447059,0.690196}%
\pgfsetfillcolor{currentfill}%
\pgfsetlinewidth{1.003750pt}%
\definecolor{currentstroke}{rgb}{0.298039,0.447059,0.690196}%
\pgfsetstrokecolor{currentstroke}%
\pgfsetdash{}{0pt}%
\pgfpathmoveto{\pgfqpoint{3.905098in}{0.779090in}}%
\pgfpathcurveto{\pgfqpoint{3.913334in}{0.779090in}}{\pgfqpoint{3.921234in}{0.782362in}}{\pgfqpoint{3.927058in}{0.788186in}}%
\pgfpathcurveto{\pgfqpoint{3.932882in}{0.794010in}}{\pgfqpoint{3.936155in}{0.801910in}}{\pgfqpoint{3.936155in}{0.810146in}}%
\pgfpathcurveto{\pgfqpoint{3.936155in}{0.818383in}}{\pgfqpoint{3.932882in}{0.826283in}}{\pgfqpoint{3.927058in}{0.832107in}}%
\pgfpathcurveto{\pgfqpoint{3.921234in}{0.837931in}}{\pgfqpoint{3.913334in}{0.841203in}}{\pgfqpoint{3.905098in}{0.841203in}}%
\pgfpathcurveto{\pgfqpoint{3.896862in}{0.841203in}}{\pgfqpoint{3.888962in}{0.837931in}}{\pgfqpoint{3.883138in}{0.832107in}}%
\pgfpathcurveto{\pgfqpoint{3.877314in}{0.826283in}}{\pgfqpoint{3.874042in}{0.818383in}}{\pgfqpoint{3.874042in}{0.810146in}}%
\pgfpathcurveto{\pgfqpoint{3.874042in}{0.801910in}}{\pgfqpoint{3.877314in}{0.794010in}}{\pgfqpoint{3.883138in}{0.788186in}}%
\pgfpathcurveto{\pgfqpoint{3.888962in}{0.782362in}}{\pgfqpoint{3.896862in}{0.779090in}}{\pgfqpoint{3.905098in}{0.779090in}}%
\pgfpathclose%
\pgfusepath{stroke,fill}%
\end{pgfscope}%
\begin{pgfscope}%
\pgfpathrectangle{\pgfqpoint{3.793912in}{0.557870in}}{\pgfqpoint{2.446088in}{1.484734in}}%
\pgfusepath{clip}%
\pgfsetbuttcap%
\pgfsetroundjoin%
\definecolor{currentfill}{rgb}{0.298039,0.447059,0.690196}%
\pgfsetfillcolor{currentfill}%
\pgfsetlinewidth{1.003750pt}%
\definecolor{currentstroke}{rgb}{0.298039,0.447059,0.690196}%
\pgfsetstrokecolor{currentstroke}%
\pgfsetdash{}{0pt}%
\pgfpathmoveto{\pgfqpoint{3.905098in}{1.180804in}}%
\pgfpathcurveto{\pgfqpoint{3.913334in}{1.180804in}}{\pgfqpoint{3.921234in}{1.184076in}}{\pgfqpoint{3.927058in}{1.189900in}}%
\pgfpathcurveto{\pgfqpoint{3.932882in}{1.195724in}}{\pgfqpoint{3.936155in}{1.203624in}}{\pgfqpoint{3.936155in}{1.211860in}}%
\pgfpathcurveto{\pgfqpoint{3.936155in}{1.220096in}}{\pgfqpoint{3.932882in}{1.227997in}}{\pgfqpoint{3.927058in}{1.233820in}}%
\pgfpathcurveto{\pgfqpoint{3.921234in}{1.239644in}}{\pgfqpoint{3.913334in}{1.242917in}}{\pgfqpoint{3.905098in}{1.242917in}}%
\pgfpathcurveto{\pgfqpoint{3.896862in}{1.242917in}}{\pgfqpoint{3.888962in}{1.239644in}}{\pgfqpoint{3.883138in}{1.233820in}}%
\pgfpathcurveto{\pgfqpoint{3.877314in}{1.227997in}}{\pgfqpoint{3.874042in}{1.220096in}}{\pgfqpoint{3.874042in}{1.211860in}}%
\pgfpathcurveto{\pgfqpoint{3.874042in}{1.203624in}}{\pgfqpoint{3.877314in}{1.195724in}}{\pgfqpoint{3.883138in}{1.189900in}}%
\pgfpathcurveto{\pgfqpoint{3.888962in}{1.184076in}}{\pgfqpoint{3.896862in}{1.180804in}}{\pgfqpoint{3.905098in}{1.180804in}}%
\pgfpathclose%
\pgfusepath{stroke,fill}%
\end{pgfscope}%
\begin{pgfscope}%
\pgfpathrectangle{\pgfqpoint{3.793912in}{0.557870in}}{\pgfqpoint{2.446088in}{1.484734in}}%
\pgfusepath{clip}%
\pgfsetbuttcap%
\pgfsetroundjoin%
\definecolor{currentfill}{rgb}{0.298039,0.447059,0.690196}%
\pgfsetfillcolor{currentfill}%
\pgfsetlinewidth{1.003750pt}%
\definecolor{currentstroke}{rgb}{0.298039,0.447059,0.690196}%
\pgfsetstrokecolor{currentstroke}%
\pgfsetdash{}{0pt}%
\pgfpathmoveto{\pgfqpoint{3.905098in}{1.180804in}}%
\pgfpathcurveto{\pgfqpoint{3.913334in}{1.180804in}}{\pgfqpoint{3.921234in}{1.184076in}}{\pgfqpoint{3.927058in}{1.189900in}}%
\pgfpathcurveto{\pgfqpoint{3.932882in}{1.195724in}}{\pgfqpoint{3.936155in}{1.203624in}}{\pgfqpoint{3.936155in}{1.211860in}}%
\pgfpathcurveto{\pgfqpoint{3.936155in}{1.220096in}}{\pgfqpoint{3.932882in}{1.227997in}}{\pgfqpoint{3.927058in}{1.233820in}}%
\pgfpathcurveto{\pgfqpoint{3.921234in}{1.239644in}}{\pgfqpoint{3.913334in}{1.242917in}}{\pgfqpoint{3.905098in}{1.242917in}}%
\pgfpathcurveto{\pgfqpoint{3.896862in}{1.242917in}}{\pgfqpoint{3.888962in}{1.239644in}}{\pgfqpoint{3.883138in}{1.233820in}}%
\pgfpathcurveto{\pgfqpoint{3.877314in}{1.227997in}}{\pgfqpoint{3.874042in}{1.220096in}}{\pgfqpoint{3.874042in}{1.211860in}}%
\pgfpathcurveto{\pgfqpoint{3.874042in}{1.203624in}}{\pgfqpoint{3.877314in}{1.195724in}}{\pgfqpoint{3.883138in}{1.189900in}}%
\pgfpathcurveto{\pgfqpoint{3.888962in}{1.184076in}}{\pgfqpoint{3.896862in}{1.180804in}}{\pgfqpoint{3.905098in}{1.180804in}}%
\pgfpathclose%
\pgfusepath{stroke,fill}%
\end{pgfscope}%
\begin{pgfscope}%
\pgfpathrectangle{\pgfqpoint{3.793912in}{0.557870in}}{\pgfqpoint{2.446088in}{1.484734in}}%
\pgfusepath{clip}%
\pgfsetbuttcap%
\pgfsetroundjoin%
\definecolor{currentfill}{rgb}{0.298039,0.447059,0.690196}%
\pgfsetfillcolor{currentfill}%
\pgfsetlinewidth{1.003750pt}%
\definecolor{currentstroke}{rgb}{0.298039,0.447059,0.690196}%
\pgfsetstrokecolor{currentstroke}%
\pgfsetdash{}{0pt}%
\pgfpathmoveto{\pgfqpoint{4.302190in}{0.594302in}}%
\pgfpathcurveto{\pgfqpoint{4.310426in}{0.594302in}}{\pgfqpoint{4.318327in}{0.597574in}}{\pgfqpoint{4.324150in}{0.603398in}}%
\pgfpathcurveto{\pgfqpoint{4.329974in}{0.609222in}}{\pgfqpoint{4.333247in}{0.617122in}}{\pgfqpoint{4.333247in}{0.625358in}}%
\pgfpathcurveto{\pgfqpoint{4.333247in}{0.633594in}}{\pgfqpoint{4.329974in}{0.641495in}}{\pgfqpoint{4.324150in}{0.647318in}}%
\pgfpathcurveto{\pgfqpoint{4.318327in}{0.653142in}}{\pgfqpoint{4.310426in}{0.656415in}}{\pgfqpoint{4.302190in}{0.656415in}}%
\pgfpathcurveto{\pgfqpoint{4.293954in}{0.656415in}}{\pgfqpoint{4.286054in}{0.653142in}}{\pgfqpoint{4.280230in}{0.647318in}}%
\pgfpathcurveto{\pgfqpoint{4.274406in}{0.641495in}}{\pgfqpoint{4.271134in}{0.633594in}}{\pgfqpoint{4.271134in}{0.625358in}}%
\pgfpathcurveto{\pgfqpoint{4.271134in}{0.617122in}}{\pgfqpoint{4.274406in}{0.609222in}}{\pgfqpoint{4.280230in}{0.603398in}}%
\pgfpathcurveto{\pgfqpoint{4.286054in}{0.597574in}}{\pgfqpoint{4.293954in}{0.594302in}}{\pgfqpoint{4.302190in}{0.594302in}}%
\pgfpathclose%
\pgfusepath{stroke,fill}%
\end{pgfscope}%
\begin{pgfscope}%
\pgfpathrectangle{\pgfqpoint{3.793912in}{0.557870in}}{\pgfqpoint{2.446088in}{1.484734in}}%
\pgfusepath{clip}%
\pgfsetbuttcap%
\pgfsetroundjoin%
\definecolor{currentfill}{rgb}{0.298039,0.447059,0.690196}%
\pgfsetfillcolor{currentfill}%
\pgfsetlinewidth{1.003750pt}%
\definecolor{currentstroke}{rgb}{0.298039,0.447059,0.690196}%
\pgfsetstrokecolor{currentstroke}%
\pgfsetdash{}{0pt}%
\pgfpathmoveto{\pgfqpoint{3.905098in}{1.936025in}}%
\pgfpathcurveto{\pgfqpoint{3.913334in}{1.936025in}}{\pgfqpoint{3.921234in}{1.939298in}}{\pgfqpoint{3.927058in}{1.945122in}}%
\pgfpathcurveto{\pgfqpoint{3.932882in}{1.950946in}}{\pgfqpoint{3.936155in}{1.958846in}}{\pgfqpoint{3.936155in}{1.967082in}}%
\pgfpathcurveto{\pgfqpoint{3.936155in}{1.975318in}}{\pgfqpoint{3.932882in}{1.983218in}}{\pgfqpoint{3.927058in}{1.989042in}}%
\pgfpathcurveto{\pgfqpoint{3.921234in}{1.994866in}}{\pgfqpoint{3.913334in}{1.998138in}}{\pgfqpoint{3.905098in}{1.998138in}}%
\pgfpathcurveto{\pgfqpoint{3.896862in}{1.998138in}}{\pgfqpoint{3.888962in}{1.994866in}}{\pgfqpoint{3.883138in}{1.989042in}}%
\pgfpathcurveto{\pgfqpoint{3.877314in}{1.983218in}}{\pgfqpoint{3.874042in}{1.975318in}}{\pgfqpoint{3.874042in}{1.967082in}}%
\pgfpathcurveto{\pgfqpoint{3.874042in}{1.958846in}}{\pgfqpoint{3.877314in}{1.950946in}}{\pgfqpoint{3.883138in}{1.945122in}}%
\pgfpathcurveto{\pgfqpoint{3.888962in}{1.939298in}}{\pgfqpoint{3.896862in}{1.936025in}}{\pgfqpoint{3.905098in}{1.936025in}}%
\pgfpathclose%
\pgfusepath{stroke,fill}%
\end{pgfscope}%
\begin{pgfscope}%
\pgfpathrectangle{\pgfqpoint{3.793912in}{0.557870in}}{\pgfqpoint{2.446088in}{1.484734in}}%
\pgfusepath{clip}%
\pgfsetbuttcap%
\pgfsetroundjoin%
\definecolor{currentfill}{rgb}{0.298039,0.447059,0.690196}%
\pgfsetfillcolor{currentfill}%
\pgfsetlinewidth{1.003750pt}%
\definecolor{currentstroke}{rgb}{0.298039,0.447059,0.690196}%
\pgfsetstrokecolor{currentstroke}%
\pgfsetdash{}{0pt}%
\pgfpathmoveto{\pgfqpoint{3.905098in}{1.526277in}}%
\pgfpathcurveto{\pgfqpoint{3.913334in}{1.526277in}}{\pgfqpoint{3.921234in}{1.529550in}}{\pgfqpoint{3.927058in}{1.535374in}}%
\pgfpathcurveto{\pgfqpoint{3.932882in}{1.541198in}}{\pgfqpoint{3.936155in}{1.549098in}}{\pgfqpoint{3.936155in}{1.557334in}}%
\pgfpathcurveto{\pgfqpoint{3.936155in}{1.565570in}}{\pgfqpoint{3.932882in}{1.573470in}}{\pgfqpoint{3.927058in}{1.579294in}}%
\pgfpathcurveto{\pgfqpoint{3.921234in}{1.585118in}}{\pgfqpoint{3.913334in}{1.588390in}}{\pgfqpoint{3.905098in}{1.588390in}}%
\pgfpathcurveto{\pgfqpoint{3.896862in}{1.588390in}}{\pgfqpoint{3.888962in}{1.585118in}}{\pgfqpoint{3.883138in}{1.579294in}}%
\pgfpathcurveto{\pgfqpoint{3.877314in}{1.573470in}}{\pgfqpoint{3.874042in}{1.565570in}}{\pgfqpoint{3.874042in}{1.557334in}}%
\pgfpathcurveto{\pgfqpoint{3.874042in}{1.549098in}}{\pgfqpoint{3.877314in}{1.541198in}}{\pgfqpoint{3.883138in}{1.535374in}}%
\pgfpathcurveto{\pgfqpoint{3.888962in}{1.529550in}}{\pgfqpoint{3.896862in}{1.526277in}}{\pgfqpoint{3.905098in}{1.526277in}}%
\pgfpathclose%
\pgfusepath{stroke,fill}%
\end{pgfscope}%
\begin{pgfscope}%
\pgfpathrectangle{\pgfqpoint{3.793912in}{0.557870in}}{\pgfqpoint{2.446088in}{1.484734in}}%
\pgfusepath{clip}%
\pgfsetbuttcap%
\pgfsetroundjoin%
\definecolor{currentfill}{rgb}{0.298039,0.447059,0.690196}%
\pgfsetfillcolor{currentfill}%
\pgfsetlinewidth{1.003750pt}%
\definecolor{currentstroke}{rgb}{0.298039,0.447059,0.690196}%
\pgfsetstrokecolor{currentstroke}%
\pgfsetdash{}{0pt}%
\pgfpathmoveto{\pgfqpoint{3.905098in}{1.936025in}}%
\pgfpathcurveto{\pgfqpoint{3.913334in}{1.936025in}}{\pgfqpoint{3.921234in}{1.939298in}}{\pgfqpoint{3.927058in}{1.945122in}}%
\pgfpathcurveto{\pgfqpoint{3.932882in}{1.950946in}}{\pgfqpoint{3.936155in}{1.958846in}}{\pgfqpoint{3.936155in}{1.967082in}}%
\pgfpathcurveto{\pgfqpoint{3.936155in}{1.975318in}}{\pgfqpoint{3.932882in}{1.983218in}}{\pgfqpoint{3.927058in}{1.989042in}}%
\pgfpathcurveto{\pgfqpoint{3.921234in}{1.994866in}}{\pgfqpoint{3.913334in}{1.998138in}}{\pgfqpoint{3.905098in}{1.998138in}}%
\pgfpathcurveto{\pgfqpoint{3.896862in}{1.998138in}}{\pgfqpoint{3.888962in}{1.994866in}}{\pgfqpoint{3.883138in}{1.989042in}}%
\pgfpathcurveto{\pgfqpoint{3.877314in}{1.983218in}}{\pgfqpoint{3.874042in}{1.975318in}}{\pgfqpoint{3.874042in}{1.967082in}}%
\pgfpathcurveto{\pgfqpoint{3.874042in}{1.958846in}}{\pgfqpoint{3.877314in}{1.950946in}}{\pgfqpoint{3.883138in}{1.945122in}}%
\pgfpathcurveto{\pgfqpoint{3.888962in}{1.939298in}}{\pgfqpoint{3.896862in}{1.936025in}}{\pgfqpoint{3.905098in}{1.936025in}}%
\pgfpathclose%
\pgfusepath{stroke,fill}%
\end{pgfscope}%
\begin{pgfscope}%
\pgfpathrectangle{\pgfqpoint{3.793912in}{0.557870in}}{\pgfqpoint{2.446088in}{1.484734in}}%
\pgfusepath{clip}%
\pgfsetbuttcap%
\pgfsetroundjoin%
\definecolor{currentfill}{rgb}{0.298039,0.447059,0.690196}%
\pgfsetfillcolor{currentfill}%
\pgfsetlinewidth{1.003750pt}%
\definecolor{currentstroke}{rgb}{0.298039,0.447059,0.690196}%
\pgfsetstrokecolor{currentstroke}%
\pgfsetdash{}{0pt}%
\pgfpathmoveto{\pgfqpoint{3.905098in}{1.357558in}}%
\pgfpathcurveto{\pgfqpoint{3.913334in}{1.357558in}}{\pgfqpoint{3.921234in}{1.360830in}}{\pgfqpoint{3.927058in}{1.366654in}}%
\pgfpathcurveto{\pgfqpoint{3.932882in}{1.372478in}}{\pgfqpoint{3.936155in}{1.380378in}}{\pgfqpoint{3.936155in}{1.388614in}}%
\pgfpathcurveto{\pgfqpoint{3.936155in}{1.396851in}}{\pgfqpoint{3.932882in}{1.404751in}}{\pgfqpoint{3.927058in}{1.410575in}}%
\pgfpathcurveto{\pgfqpoint{3.921234in}{1.416398in}}{\pgfqpoint{3.913334in}{1.419671in}}{\pgfqpoint{3.905098in}{1.419671in}}%
\pgfpathcurveto{\pgfqpoint{3.896862in}{1.419671in}}{\pgfqpoint{3.888962in}{1.416398in}}{\pgfqpoint{3.883138in}{1.410575in}}%
\pgfpathcurveto{\pgfqpoint{3.877314in}{1.404751in}}{\pgfqpoint{3.874042in}{1.396851in}}{\pgfqpoint{3.874042in}{1.388614in}}%
\pgfpathcurveto{\pgfqpoint{3.874042in}{1.380378in}}{\pgfqpoint{3.877314in}{1.372478in}}{\pgfqpoint{3.883138in}{1.366654in}}%
\pgfpathcurveto{\pgfqpoint{3.888962in}{1.360830in}}{\pgfqpoint{3.896862in}{1.357558in}}{\pgfqpoint{3.905098in}{1.357558in}}%
\pgfpathclose%
\pgfusepath{stroke,fill}%
\end{pgfscope}%
\begin{pgfscope}%
\pgfpathrectangle{\pgfqpoint{3.793912in}{0.557870in}}{\pgfqpoint{2.446088in}{1.484734in}}%
\pgfusepath{clip}%
\pgfsetbuttcap%
\pgfsetroundjoin%
\definecolor{currentfill}{rgb}{0.298039,0.447059,0.690196}%
\pgfsetfillcolor{currentfill}%
\pgfsetlinewidth{1.003750pt}%
\definecolor{currentstroke}{rgb}{0.298039,0.447059,0.690196}%
\pgfsetstrokecolor{currentstroke}%
\pgfsetdash{}{0pt}%
\pgfpathmoveto{\pgfqpoint{3.905098in}{1.936025in}}%
\pgfpathcurveto{\pgfqpoint{3.913334in}{1.936025in}}{\pgfqpoint{3.921234in}{1.939298in}}{\pgfqpoint{3.927058in}{1.945122in}}%
\pgfpathcurveto{\pgfqpoint{3.932882in}{1.950946in}}{\pgfqpoint{3.936155in}{1.958846in}}{\pgfqpoint{3.936155in}{1.967082in}}%
\pgfpathcurveto{\pgfqpoint{3.936155in}{1.975318in}}{\pgfqpoint{3.932882in}{1.983218in}}{\pgfqpoint{3.927058in}{1.989042in}}%
\pgfpathcurveto{\pgfqpoint{3.921234in}{1.994866in}}{\pgfqpoint{3.913334in}{1.998138in}}{\pgfqpoint{3.905098in}{1.998138in}}%
\pgfpathcurveto{\pgfqpoint{3.896862in}{1.998138in}}{\pgfqpoint{3.888962in}{1.994866in}}{\pgfqpoint{3.883138in}{1.989042in}}%
\pgfpathcurveto{\pgfqpoint{3.877314in}{1.983218in}}{\pgfqpoint{3.874042in}{1.975318in}}{\pgfqpoint{3.874042in}{1.967082in}}%
\pgfpathcurveto{\pgfqpoint{3.874042in}{1.958846in}}{\pgfqpoint{3.877314in}{1.950946in}}{\pgfqpoint{3.883138in}{1.945122in}}%
\pgfpathcurveto{\pgfqpoint{3.888962in}{1.939298in}}{\pgfqpoint{3.896862in}{1.936025in}}{\pgfqpoint{3.905098in}{1.936025in}}%
\pgfpathclose%
\pgfusepath{stroke,fill}%
\end{pgfscope}%
\begin{pgfscope}%
\pgfpathrectangle{\pgfqpoint{3.793912in}{0.557870in}}{\pgfqpoint{2.446088in}{1.484734in}}%
\pgfusepath{clip}%
\pgfsetbuttcap%
\pgfsetroundjoin%
\definecolor{currentfill}{rgb}{0.298039,0.447059,0.690196}%
\pgfsetfillcolor{currentfill}%
\pgfsetlinewidth{1.003750pt}%
\definecolor{currentstroke}{rgb}{0.298039,0.447059,0.690196}%
\pgfsetstrokecolor{currentstroke}%
\pgfsetdash{}{0pt}%
\pgfpathmoveto{\pgfqpoint{3.905098in}{1.936025in}}%
\pgfpathcurveto{\pgfqpoint{3.913334in}{1.936025in}}{\pgfqpoint{3.921234in}{1.939298in}}{\pgfqpoint{3.927058in}{1.945122in}}%
\pgfpathcurveto{\pgfqpoint{3.932882in}{1.950946in}}{\pgfqpoint{3.936155in}{1.958846in}}{\pgfqpoint{3.936155in}{1.967082in}}%
\pgfpathcurveto{\pgfqpoint{3.936155in}{1.975318in}}{\pgfqpoint{3.932882in}{1.983218in}}{\pgfqpoint{3.927058in}{1.989042in}}%
\pgfpathcurveto{\pgfqpoint{3.921234in}{1.994866in}}{\pgfqpoint{3.913334in}{1.998138in}}{\pgfqpoint{3.905098in}{1.998138in}}%
\pgfpathcurveto{\pgfqpoint{3.896862in}{1.998138in}}{\pgfqpoint{3.888962in}{1.994866in}}{\pgfqpoint{3.883138in}{1.989042in}}%
\pgfpathcurveto{\pgfqpoint{3.877314in}{1.983218in}}{\pgfqpoint{3.874042in}{1.975318in}}{\pgfqpoint{3.874042in}{1.967082in}}%
\pgfpathcurveto{\pgfqpoint{3.874042in}{1.958846in}}{\pgfqpoint{3.877314in}{1.950946in}}{\pgfqpoint{3.883138in}{1.945122in}}%
\pgfpathcurveto{\pgfqpoint{3.888962in}{1.939298in}}{\pgfqpoint{3.896862in}{1.936025in}}{\pgfqpoint{3.905098in}{1.936025in}}%
\pgfpathclose%
\pgfusepath{stroke,fill}%
\end{pgfscope}%
\begin{pgfscope}%
\pgfpathrectangle{\pgfqpoint{3.793912in}{0.557870in}}{\pgfqpoint{2.446088in}{1.484734in}}%
\pgfusepath{clip}%
\pgfsetbuttcap%
\pgfsetroundjoin%
\definecolor{currentfill}{rgb}{0.298039,0.447059,0.690196}%
\pgfsetfillcolor{currentfill}%
\pgfsetlinewidth{1.003750pt}%
\definecolor{currentstroke}{rgb}{0.298039,0.447059,0.690196}%
\pgfsetstrokecolor{currentstroke}%
\pgfsetdash{}{0pt}%
\pgfpathmoveto{\pgfqpoint{3.905098in}{1.936025in}}%
\pgfpathcurveto{\pgfqpoint{3.913334in}{1.936025in}}{\pgfqpoint{3.921234in}{1.939298in}}{\pgfqpoint{3.927058in}{1.945122in}}%
\pgfpathcurveto{\pgfqpoint{3.932882in}{1.950946in}}{\pgfqpoint{3.936155in}{1.958846in}}{\pgfqpoint{3.936155in}{1.967082in}}%
\pgfpathcurveto{\pgfqpoint{3.936155in}{1.975318in}}{\pgfqpoint{3.932882in}{1.983218in}}{\pgfqpoint{3.927058in}{1.989042in}}%
\pgfpathcurveto{\pgfqpoint{3.921234in}{1.994866in}}{\pgfqpoint{3.913334in}{1.998138in}}{\pgfqpoint{3.905098in}{1.998138in}}%
\pgfpathcurveto{\pgfqpoint{3.896862in}{1.998138in}}{\pgfqpoint{3.888962in}{1.994866in}}{\pgfqpoint{3.883138in}{1.989042in}}%
\pgfpathcurveto{\pgfqpoint{3.877314in}{1.983218in}}{\pgfqpoint{3.874042in}{1.975318in}}{\pgfqpoint{3.874042in}{1.967082in}}%
\pgfpathcurveto{\pgfqpoint{3.874042in}{1.958846in}}{\pgfqpoint{3.877314in}{1.950946in}}{\pgfqpoint{3.883138in}{1.945122in}}%
\pgfpathcurveto{\pgfqpoint{3.888962in}{1.939298in}}{\pgfqpoint{3.896862in}{1.936025in}}{\pgfqpoint{3.905098in}{1.936025in}}%
\pgfpathclose%
\pgfusepath{stroke,fill}%
\end{pgfscope}%
\begin{pgfscope}%
\pgfpathrectangle{\pgfqpoint{3.793912in}{0.557870in}}{\pgfqpoint{2.446088in}{1.484734in}}%
\pgfusepath{clip}%
\pgfsetbuttcap%
\pgfsetroundjoin%
\definecolor{currentfill}{rgb}{0.298039,0.447059,0.690196}%
\pgfsetfillcolor{currentfill}%
\pgfsetlinewidth{1.003750pt}%
\definecolor{currentstroke}{rgb}{0.298039,0.447059,0.690196}%
\pgfsetstrokecolor{currentstroke}%
\pgfsetdash{}{0pt}%
\pgfpathmoveto{\pgfqpoint{3.905098in}{1.936025in}}%
\pgfpathcurveto{\pgfqpoint{3.913334in}{1.936025in}}{\pgfqpoint{3.921234in}{1.939298in}}{\pgfqpoint{3.927058in}{1.945122in}}%
\pgfpathcurveto{\pgfqpoint{3.932882in}{1.950946in}}{\pgfqpoint{3.936155in}{1.958846in}}{\pgfqpoint{3.936155in}{1.967082in}}%
\pgfpathcurveto{\pgfqpoint{3.936155in}{1.975318in}}{\pgfqpoint{3.932882in}{1.983218in}}{\pgfqpoint{3.927058in}{1.989042in}}%
\pgfpathcurveto{\pgfqpoint{3.921234in}{1.994866in}}{\pgfqpoint{3.913334in}{1.998138in}}{\pgfqpoint{3.905098in}{1.998138in}}%
\pgfpathcurveto{\pgfqpoint{3.896862in}{1.998138in}}{\pgfqpoint{3.888962in}{1.994866in}}{\pgfqpoint{3.883138in}{1.989042in}}%
\pgfpathcurveto{\pgfqpoint{3.877314in}{1.983218in}}{\pgfqpoint{3.874042in}{1.975318in}}{\pgfqpoint{3.874042in}{1.967082in}}%
\pgfpathcurveto{\pgfqpoint{3.874042in}{1.958846in}}{\pgfqpoint{3.877314in}{1.950946in}}{\pgfqpoint{3.883138in}{1.945122in}}%
\pgfpathcurveto{\pgfqpoint{3.888962in}{1.939298in}}{\pgfqpoint{3.896862in}{1.936025in}}{\pgfqpoint{3.905098in}{1.936025in}}%
\pgfpathclose%
\pgfusepath{stroke,fill}%
\end{pgfscope}%
\begin{pgfscope}%
\pgfpathrectangle{\pgfqpoint{3.793912in}{0.557870in}}{\pgfqpoint{2.446088in}{1.484734in}}%
\pgfusepath{clip}%
\pgfsetbuttcap%
\pgfsetroundjoin%
\definecolor{currentfill}{rgb}{0.298039,0.447059,0.690196}%
\pgfsetfillcolor{currentfill}%
\pgfsetlinewidth{1.003750pt}%
\definecolor{currentstroke}{rgb}{0.298039,0.447059,0.690196}%
\pgfsetstrokecolor{currentstroke}%
\pgfsetdash{}{0pt}%
\pgfpathmoveto{\pgfqpoint{3.905098in}{1.936025in}}%
\pgfpathcurveto{\pgfqpoint{3.913334in}{1.936025in}}{\pgfqpoint{3.921234in}{1.939298in}}{\pgfqpoint{3.927058in}{1.945122in}}%
\pgfpathcurveto{\pgfqpoint{3.932882in}{1.950946in}}{\pgfqpoint{3.936155in}{1.958846in}}{\pgfqpoint{3.936155in}{1.967082in}}%
\pgfpathcurveto{\pgfqpoint{3.936155in}{1.975318in}}{\pgfqpoint{3.932882in}{1.983218in}}{\pgfqpoint{3.927058in}{1.989042in}}%
\pgfpathcurveto{\pgfqpoint{3.921234in}{1.994866in}}{\pgfqpoint{3.913334in}{1.998138in}}{\pgfqpoint{3.905098in}{1.998138in}}%
\pgfpathcurveto{\pgfqpoint{3.896862in}{1.998138in}}{\pgfqpoint{3.888962in}{1.994866in}}{\pgfqpoint{3.883138in}{1.989042in}}%
\pgfpathcurveto{\pgfqpoint{3.877314in}{1.983218in}}{\pgfqpoint{3.874042in}{1.975318in}}{\pgfqpoint{3.874042in}{1.967082in}}%
\pgfpathcurveto{\pgfqpoint{3.874042in}{1.958846in}}{\pgfqpoint{3.877314in}{1.950946in}}{\pgfqpoint{3.883138in}{1.945122in}}%
\pgfpathcurveto{\pgfqpoint{3.888962in}{1.939298in}}{\pgfqpoint{3.896862in}{1.936025in}}{\pgfqpoint{3.905098in}{1.936025in}}%
\pgfpathclose%
\pgfusepath{stroke,fill}%
\end{pgfscope}%
\begin{pgfscope}%
\pgfpathrectangle{\pgfqpoint{3.793912in}{0.557870in}}{\pgfqpoint{2.446088in}{1.484734in}}%
\pgfusepath{clip}%
\pgfsetbuttcap%
\pgfsetroundjoin%
\definecolor{currentfill}{rgb}{0.298039,0.447059,0.690196}%
\pgfsetfillcolor{currentfill}%
\pgfsetlinewidth{1.003750pt}%
\definecolor{currentstroke}{rgb}{0.298039,0.447059,0.690196}%
\pgfsetstrokecolor{currentstroke}%
\pgfsetdash{}{0pt}%
\pgfpathmoveto{\pgfqpoint{3.905098in}{1.936025in}}%
\pgfpathcurveto{\pgfqpoint{3.913334in}{1.936025in}}{\pgfqpoint{3.921234in}{1.939298in}}{\pgfqpoint{3.927058in}{1.945122in}}%
\pgfpathcurveto{\pgfqpoint{3.932882in}{1.950946in}}{\pgfqpoint{3.936155in}{1.958846in}}{\pgfqpoint{3.936155in}{1.967082in}}%
\pgfpathcurveto{\pgfqpoint{3.936155in}{1.975318in}}{\pgfqpoint{3.932882in}{1.983218in}}{\pgfqpoint{3.927058in}{1.989042in}}%
\pgfpathcurveto{\pgfqpoint{3.921234in}{1.994866in}}{\pgfqpoint{3.913334in}{1.998138in}}{\pgfqpoint{3.905098in}{1.998138in}}%
\pgfpathcurveto{\pgfqpoint{3.896862in}{1.998138in}}{\pgfqpoint{3.888962in}{1.994866in}}{\pgfqpoint{3.883138in}{1.989042in}}%
\pgfpathcurveto{\pgfqpoint{3.877314in}{1.983218in}}{\pgfqpoint{3.874042in}{1.975318in}}{\pgfqpoint{3.874042in}{1.967082in}}%
\pgfpathcurveto{\pgfqpoint{3.874042in}{1.958846in}}{\pgfqpoint{3.877314in}{1.950946in}}{\pgfqpoint{3.883138in}{1.945122in}}%
\pgfpathcurveto{\pgfqpoint{3.888962in}{1.939298in}}{\pgfqpoint{3.896862in}{1.936025in}}{\pgfqpoint{3.905098in}{1.936025in}}%
\pgfpathclose%
\pgfusepath{stroke,fill}%
\end{pgfscope}%
\begin{pgfscope}%
\pgfpathrectangle{\pgfqpoint{3.793912in}{0.557870in}}{\pgfqpoint{2.446088in}{1.484734in}}%
\pgfusepath{clip}%
\pgfsetbuttcap%
\pgfsetroundjoin%
\definecolor{currentfill}{rgb}{0.298039,0.447059,0.690196}%
\pgfsetfillcolor{currentfill}%
\pgfsetlinewidth{1.003750pt}%
\definecolor{currentstroke}{rgb}{0.298039,0.447059,0.690196}%
\pgfsetstrokecolor{currentstroke}%
\pgfsetdash{}{0pt}%
\pgfpathmoveto{\pgfqpoint{3.905098in}{1.261146in}}%
\pgfpathcurveto{\pgfqpoint{3.913334in}{1.261146in}}{\pgfqpoint{3.921234in}{1.264419in}}{\pgfqpoint{3.927058in}{1.270243in}}%
\pgfpathcurveto{\pgfqpoint{3.932882in}{1.276067in}}{\pgfqpoint{3.936155in}{1.283967in}}{\pgfqpoint{3.936155in}{1.292203in}}%
\pgfpathcurveto{\pgfqpoint{3.936155in}{1.300439in}}{\pgfqpoint{3.932882in}{1.308339in}}{\pgfqpoint{3.927058in}{1.314163in}}%
\pgfpathcurveto{\pgfqpoint{3.921234in}{1.319987in}}{\pgfqpoint{3.913334in}{1.323259in}}{\pgfqpoint{3.905098in}{1.323259in}}%
\pgfpathcurveto{\pgfqpoint{3.896862in}{1.323259in}}{\pgfqpoint{3.888962in}{1.319987in}}{\pgfqpoint{3.883138in}{1.314163in}}%
\pgfpathcurveto{\pgfqpoint{3.877314in}{1.308339in}}{\pgfqpoint{3.874042in}{1.300439in}}{\pgfqpoint{3.874042in}{1.292203in}}%
\pgfpathcurveto{\pgfqpoint{3.874042in}{1.283967in}}{\pgfqpoint{3.877314in}{1.276067in}}{\pgfqpoint{3.883138in}{1.270243in}}%
\pgfpathcurveto{\pgfqpoint{3.888962in}{1.264419in}}{\pgfqpoint{3.896862in}{1.261146in}}{\pgfqpoint{3.905098in}{1.261146in}}%
\pgfpathclose%
\pgfusepath{stroke,fill}%
\end{pgfscope}%
\begin{pgfscope}%
\pgfpathrectangle{\pgfqpoint{3.793912in}{0.557870in}}{\pgfqpoint{2.446088in}{1.484734in}}%
\pgfusepath{clip}%
\pgfsetbuttcap%
\pgfsetroundjoin%
\definecolor{currentfill}{rgb}{0.298039,0.447059,0.690196}%
\pgfsetfillcolor{currentfill}%
\pgfsetlinewidth{1.003750pt}%
\definecolor{currentstroke}{rgb}{0.298039,0.447059,0.690196}%
\pgfsetstrokecolor{currentstroke}%
\pgfsetdash{}{0pt}%
\pgfpathmoveto{\pgfqpoint{3.905098in}{1.936025in}}%
\pgfpathcurveto{\pgfqpoint{3.913334in}{1.936025in}}{\pgfqpoint{3.921234in}{1.939298in}}{\pgfqpoint{3.927058in}{1.945122in}}%
\pgfpathcurveto{\pgfqpoint{3.932882in}{1.950946in}}{\pgfqpoint{3.936155in}{1.958846in}}{\pgfqpoint{3.936155in}{1.967082in}}%
\pgfpathcurveto{\pgfqpoint{3.936155in}{1.975318in}}{\pgfqpoint{3.932882in}{1.983218in}}{\pgfqpoint{3.927058in}{1.989042in}}%
\pgfpathcurveto{\pgfqpoint{3.921234in}{1.994866in}}{\pgfqpoint{3.913334in}{1.998138in}}{\pgfqpoint{3.905098in}{1.998138in}}%
\pgfpathcurveto{\pgfqpoint{3.896862in}{1.998138in}}{\pgfqpoint{3.888962in}{1.994866in}}{\pgfqpoint{3.883138in}{1.989042in}}%
\pgfpathcurveto{\pgfqpoint{3.877314in}{1.983218in}}{\pgfqpoint{3.874042in}{1.975318in}}{\pgfqpoint{3.874042in}{1.967082in}}%
\pgfpathcurveto{\pgfqpoint{3.874042in}{1.958846in}}{\pgfqpoint{3.877314in}{1.950946in}}{\pgfqpoint{3.883138in}{1.945122in}}%
\pgfpathcurveto{\pgfqpoint{3.888962in}{1.939298in}}{\pgfqpoint{3.896862in}{1.936025in}}{\pgfqpoint{3.905098in}{1.936025in}}%
\pgfpathclose%
\pgfusepath{stroke,fill}%
\end{pgfscope}%
\begin{pgfscope}%
\pgfpathrectangle{\pgfqpoint{3.793912in}{0.557870in}}{\pgfqpoint{2.446088in}{1.484734in}}%
\pgfusepath{clip}%
\pgfsetbuttcap%
\pgfsetroundjoin%
\definecolor{currentfill}{rgb}{0.298039,0.447059,0.690196}%
\pgfsetfillcolor{currentfill}%
\pgfsetlinewidth{1.003750pt}%
\definecolor{currentstroke}{rgb}{0.298039,0.447059,0.690196}%
\pgfsetstrokecolor{currentstroke}%
\pgfsetdash{}{0pt}%
\pgfpathmoveto{\pgfqpoint{3.905098in}{1.936025in}}%
\pgfpathcurveto{\pgfqpoint{3.913334in}{1.936025in}}{\pgfqpoint{3.921234in}{1.939298in}}{\pgfqpoint{3.927058in}{1.945122in}}%
\pgfpathcurveto{\pgfqpoint{3.932882in}{1.950946in}}{\pgfqpoint{3.936155in}{1.958846in}}{\pgfqpoint{3.936155in}{1.967082in}}%
\pgfpathcurveto{\pgfqpoint{3.936155in}{1.975318in}}{\pgfqpoint{3.932882in}{1.983218in}}{\pgfqpoint{3.927058in}{1.989042in}}%
\pgfpathcurveto{\pgfqpoint{3.921234in}{1.994866in}}{\pgfqpoint{3.913334in}{1.998138in}}{\pgfqpoint{3.905098in}{1.998138in}}%
\pgfpathcurveto{\pgfqpoint{3.896862in}{1.998138in}}{\pgfqpoint{3.888962in}{1.994866in}}{\pgfqpoint{3.883138in}{1.989042in}}%
\pgfpathcurveto{\pgfqpoint{3.877314in}{1.983218in}}{\pgfqpoint{3.874042in}{1.975318in}}{\pgfqpoint{3.874042in}{1.967082in}}%
\pgfpathcurveto{\pgfqpoint{3.874042in}{1.958846in}}{\pgfqpoint{3.877314in}{1.950946in}}{\pgfqpoint{3.883138in}{1.945122in}}%
\pgfpathcurveto{\pgfqpoint{3.888962in}{1.939298in}}{\pgfqpoint{3.896862in}{1.936025in}}{\pgfqpoint{3.905098in}{1.936025in}}%
\pgfpathclose%
\pgfusepath{stroke,fill}%
\end{pgfscope}%
\begin{pgfscope}%
\pgfpathrectangle{\pgfqpoint{3.793912in}{0.557870in}}{\pgfqpoint{2.446088in}{1.484734in}}%
\pgfusepath{clip}%
\pgfsetbuttcap%
\pgfsetroundjoin%
\definecolor{currentfill}{rgb}{0.298039,0.447059,0.690196}%
\pgfsetfillcolor{currentfill}%
\pgfsetlinewidth{1.003750pt}%
\definecolor{currentstroke}{rgb}{0.298039,0.447059,0.690196}%
\pgfsetstrokecolor{currentstroke}%
\pgfsetdash{}{0pt}%
\pgfpathmoveto{\pgfqpoint{3.905098in}{1.936025in}}%
\pgfpathcurveto{\pgfqpoint{3.913334in}{1.936025in}}{\pgfqpoint{3.921234in}{1.939298in}}{\pgfqpoint{3.927058in}{1.945122in}}%
\pgfpathcurveto{\pgfqpoint{3.932882in}{1.950946in}}{\pgfqpoint{3.936155in}{1.958846in}}{\pgfqpoint{3.936155in}{1.967082in}}%
\pgfpathcurveto{\pgfqpoint{3.936155in}{1.975318in}}{\pgfqpoint{3.932882in}{1.983218in}}{\pgfqpoint{3.927058in}{1.989042in}}%
\pgfpathcurveto{\pgfqpoint{3.921234in}{1.994866in}}{\pgfqpoint{3.913334in}{1.998138in}}{\pgfqpoint{3.905098in}{1.998138in}}%
\pgfpathcurveto{\pgfqpoint{3.896862in}{1.998138in}}{\pgfqpoint{3.888962in}{1.994866in}}{\pgfqpoint{3.883138in}{1.989042in}}%
\pgfpathcurveto{\pgfqpoint{3.877314in}{1.983218in}}{\pgfqpoint{3.874042in}{1.975318in}}{\pgfqpoint{3.874042in}{1.967082in}}%
\pgfpathcurveto{\pgfqpoint{3.874042in}{1.958846in}}{\pgfqpoint{3.877314in}{1.950946in}}{\pgfqpoint{3.883138in}{1.945122in}}%
\pgfpathcurveto{\pgfqpoint{3.888962in}{1.939298in}}{\pgfqpoint{3.896862in}{1.936025in}}{\pgfqpoint{3.905098in}{1.936025in}}%
\pgfpathclose%
\pgfusepath{stroke,fill}%
\end{pgfscope}%
\begin{pgfscope}%
\pgfpathrectangle{\pgfqpoint{3.793912in}{0.557870in}}{\pgfqpoint{2.446088in}{1.484734in}}%
\pgfusepath{clip}%
\pgfsetbuttcap%
\pgfsetroundjoin%
\definecolor{currentfill}{rgb}{0.298039,0.447059,0.690196}%
\pgfsetfillcolor{currentfill}%
\pgfsetlinewidth{1.003750pt}%
\definecolor{currentstroke}{rgb}{0.298039,0.447059,0.690196}%
\pgfsetstrokecolor{currentstroke}%
\pgfsetdash{}{0pt}%
\pgfpathmoveto{\pgfqpoint{3.905098in}{1.936025in}}%
\pgfpathcurveto{\pgfqpoint{3.913334in}{1.936025in}}{\pgfqpoint{3.921234in}{1.939298in}}{\pgfqpoint{3.927058in}{1.945122in}}%
\pgfpathcurveto{\pgfqpoint{3.932882in}{1.950946in}}{\pgfqpoint{3.936155in}{1.958846in}}{\pgfqpoint{3.936155in}{1.967082in}}%
\pgfpathcurveto{\pgfqpoint{3.936155in}{1.975318in}}{\pgfqpoint{3.932882in}{1.983218in}}{\pgfqpoint{3.927058in}{1.989042in}}%
\pgfpathcurveto{\pgfqpoint{3.921234in}{1.994866in}}{\pgfqpoint{3.913334in}{1.998138in}}{\pgfqpoint{3.905098in}{1.998138in}}%
\pgfpathcurveto{\pgfqpoint{3.896862in}{1.998138in}}{\pgfqpoint{3.888962in}{1.994866in}}{\pgfqpoint{3.883138in}{1.989042in}}%
\pgfpathcurveto{\pgfqpoint{3.877314in}{1.983218in}}{\pgfqpoint{3.874042in}{1.975318in}}{\pgfqpoint{3.874042in}{1.967082in}}%
\pgfpathcurveto{\pgfqpoint{3.874042in}{1.958846in}}{\pgfqpoint{3.877314in}{1.950946in}}{\pgfqpoint{3.883138in}{1.945122in}}%
\pgfpathcurveto{\pgfqpoint{3.888962in}{1.939298in}}{\pgfqpoint{3.896862in}{1.936025in}}{\pgfqpoint{3.905098in}{1.936025in}}%
\pgfpathclose%
\pgfusepath{stroke,fill}%
\end{pgfscope}%
\begin{pgfscope}%
\pgfpathrectangle{\pgfqpoint{3.793912in}{0.557870in}}{\pgfqpoint{2.446088in}{1.484734in}}%
\pgfusepath{clip}%
\pgfsetbuttcap%
\pgfsetroundjoin%
\definecolor{currentfill}{rgb}{0.298039,0.447059,0.690196}%
\pgfsetfillcolor{currentfill}%
\pgfsetlinewidth{1.003750pt}%
\definecolor{currentstroke}{rgb}{0.298039,0.447059,0.690196}%
\pgfsetstrokecolor{currentstroke}%
\pgfsetdash{}{0pt}%
\pgfpathmoveto{\pgfqpoint{3.905098in}{1.172769in}}%
\pgfpathcurveto{\pgfqpoint{3.913334in}{1.172769in}}{\pgfqpoint{3.921234in}{1.176042in}}{\pgfqpoint{3.927058in}{1.181866in}}%
\pgfpathcurveto{\pgfqpoint{3.932882in}{1.187690in}}{\pgfqpoint{3.936155in}{1.195590in}}{\pgfqpoint{3.936155in}{1.203826in}}%
\pgfpathcurveto{\pgfqpoint{3.936155in}{1.212062in}}{\pgfqpoint{3.932882in}{1.219962in}}{\pgfqpoint{3.927058in}{1.225786in}}%
\pgfpathcurveto{\pgfqpoint{3.921234in}{1.231610in}}{\pgfqpoint{3.913334in}{1.234882in}}{\pgfqpoint{3.905098in}{1.234882in}}%
\pgfpathcurveto{\pgfqpoint{3.896862in}{1.234882in}}{\pgfqpoint{3.888962in}{1.231610in}}{\pgfqpoint{3.883138in}{1.225786in}}%
\pgfpathcurveto{\pgfqpoint{3.877314in}{1.219962in}}{\pgfqpoint{3.874042in}{1.212062in}}{\pgfqpoint{3.874042in}{1.203826in}}%
\pgfpathcurveto{\pgfqpoint{3.874042in}{1.195590in}}{\pgfqpoint{3.877314in}{1.187690in}}{\pgfqpoint{3.883138in}{1.181866in}}%
\pgfpathcurveto{\pgfqpoint{3.888962in}{1.176042in}}{\pgfqpoint{3.896862in}{1.172769in}}{\pgfqpoint{3.905098in}{1.172769in}}%
\pgfpathclose%
\pgfusepath{stroke,fill}%
\end{pgfscope}%
\begin{pgfscope}%
\pgfpathrectangle{\pgfqpoint{3.793912in}{0.557870in}}{\pgfqpoint{2.446088in}{1.484734in}}%
\pgfusepath{clip}%
\pgfsetbuttcap%
\pgfsetroundjoin%
\definecolor{currentfill}{rgb}{0.298039,0.447059,0.690196}%
\pgfsetfillcolor{currentfill}%
\pgfsetlinewidth{1.003750pt}%
\definecolor{currentstroke}{rgb}{0.298039,0.447059,0.690196}%
\pgfsetstrokecolor{currentstroke}%
\pgfsetdash{}{0pt}%
\pgfpathmoveto{\pgfqpoint{3.905098in}{1.301318in}}%
\pgfpathcurveto{\pgfqpoint{3.913334in}{1.301318in}}{\pgfqpoint{3.921234in}{1.304590in}}{\pgfqpoint{3.927058in}{1.310414in}}%
\pgfpathcurveto{\pgfqpoint{3.932882in}{1.316238in}}{\pgfqpoint{3.936155in}{1.324138in}}{\pgfqpoint{3.936155in}{1.332374in}}%
\pgfpathcurveto{\pgfqpoint{3.936155in}{1.340611in}}{\pgfqpoint{3.932882in}{1.348511in}}{\pgfqpoint{3.927058in}{1.354335in}}%
\pgfpathcurveto{\pgfqpoint{3.921234in}{1.360159in}}{\pgfqpoint{3.913334in}{1.363431in}}{\pgfqpoint{3.905098in}{1.363431in}}%
\pgfpathcurveto{\pgfqpoint{3.896862in}{1.363431in}}{\pgfqpoint{3.888962in}{1.360159in}}{\pgfqpoint{3.883138in}{1.354335in}}%
\pgfpathcurveto{\pgfqpoint{3.877314in}{1.348511in}}{\pgfqpoint{3.874042in}{1.340611in}}{\pgfqpoint{3.874042in}{1.332374in}}%
\pgfpathcurveto{\pgfqpoint{3.874042in}{1.324138in}}{\pgfqpoint{3.877314in}{1.316238in}}{\pgfqpoint{3.883138in}{1.310414in}}%
\pgfpathcurveto{\pgfqpoint{3.888962in}{1.304590in}}{\pgfqpoint{3.896862in}{1.301318in}}{\pgfqpoint{3.905098in}{1.301318in}}%
\pgfpathclose%
\pgfusepath{stroke,fill}%
\end{pgfscope}%
\begin{pgfscope}%
\pgfpathrectangle{\pgfqpoint{3.793912in}{0.557870in}}{\pgfqpoint{2.446088in}{1.484734in}}%
\pgfusepath{clip}%
\pgfsetbuttcap%
\pgfsetroundjoin%
\definecolor{currentfill}{rgb}{0.298039,0.447059,0.690196}%
\pgfsetfillcolor{currentfill}%
\pgfsetlinewidth{1.003750pt}%
\definecolor{currentstroke}{rgb}{0.298039,0.447059,0.690196}%
\pgfsetstrokecolor{currentstroke}%
\pgfsetdash{}{0pt}%
\pgfpathmoveto{\pgfqpoint{3.905098in}{1.196872in}}%
\pgfpathcurveto{\pgfqpoint{3.913334in}{1.196872in}}{\pgfqpoint{3.921234in}{1.200145in}}{\pgfqpoint{3.927058in}{1.205968in}}%
\pgfpathcurveto{\pgfqpoint{3.932882in}{1.211792in}}{\pgfqpoint{3.936155in}{1.219692in}}{\pgfqpoint{3.936155in}{1.227929in}}%
\pgfpathcurveto{\pgfqpoint{3.936155in}{1.236165in}}{\pgfqpoint{3.932882in}{1.244065in}}{\pgfqpoint{3.927058in}{1.249889in}}%
\pgfpathcurveto{\pgfqpoint{3.921234in}{1.255713in}}{\pgfqpoint{3.913334in}{1.258985in}}{\pgfqpoint{3.905098in}{1.258985in}}%
\pgfpathcurveto{\pgfqpoint{3.896862in}{1.258985in}}{\pgfqpoint{3.888962in}{1.255713in}}{\pgfqpoint{3.883138in}{1.249889in}}%
\pgfpathcurveto{\pgfqpoint{3.877314in}{1.244065in}}{\pgfqpoint{3.874042in}{1.236165in}}{\pgfqpoint{3.874042in}{1.227929in}}%
\pgfpathcurveto{\pgfqpoint{3.874042in}{1.219692in}}{\pgfqpoint{3.877314in}{1.211792in}}{\pgfqpoint{3.883138in}{1.205968in}}%
\pgfpathcurveto{\pgfqpoint{3.888962in}{1.200145in}}{\pgfqpoint{3.896862in}{1.196872in}}{\pgfqpoint{3.905098in}{1.196872in}}%
\pgfpathclose%
\pgfusepath{stroke,fill}%
\end{pgfscope}%
\begin{pgfscope}%
\pgfpathrectangle{\pgfqpoint{3.793912in}{0.557870in}}{\pgfqpoint{2.446088in}{1.484734in}}%
\pgfusepath{clip}%
\pgfsetbuttcap%
\pgfsetroundjoin%
\definecolor{currentfill}{rgb}{0.298039,0.447059,0.690196}%
\pgfsetfillcolor{currentfill}%
\pgfsetlinewidth{1.003750pt}%
\definecolor{currentstroke}{rgb}{0.298039,0.447059,0.690196}%
\pgfsetstrokecolor{currentstroke}%
\pgfsetdash{}{0pt}%
\pgfpathmoveto{\pgfqpoint{3.905098in}{1.919957in}}%
\pgfpathcurveto{\pgfqpoint{3.913334in}{1.919957in}}{\pgfqpoint{3.921234in}{1.923229in}}{\pgfqpoint{3.927058in}{1.929053in}}%
\pgfpathcurveto{\pgfqpoint{3.932882in}{1.934877in}}{\pgfqpoint{3.936155in}{1.942777in}}{\pgfqpoint{3.936155in}{1.951013in}}%
\pgfpathcurveto{\pgfqpoint{3.936155in}{1.959250in}}{\pgfqpoint{3.932882in}{1.967150in}}{\pgfqpoint{3.927058in}{1.972974in}}%
\pgfpathcurveto{\pgfqpoint{3.921234in}{1.978798in}}{\pgfqpoint{3.913334in}{1.982070in}}{\pgfqpoint{3.905098in}{1.982070in}}%
\pgfpathcurveto{\pgfqpoint{3.896862in}{1.982070in}}{\pgfqpoint{3.888962in}{1.978798in}}{\pgfqpoint{3.883138in}{1.972974in}}%
\pgfpathcurveto{\pgfqpoint{3.877314in}{1.967150in}}{\pgfqpoint{3.874042in}{1.959250in}}{\pgfqpoint{3.874042in}{1.951013in}}%
\pgfpathcurveto{\pgfqpoint{3.874042in}{1.942777in}}{\pgfqpoint{3.877314in}{1.934877in}}{\pgfqpoint{3.883138in}{1.929053in}}%
\pgfpathcurveto{\pgfqpoint{3.888962in}{1.923229in}}{\pgfqpoint{3.896862in}{1.919957in}}{\pgfqpoint{3.905098in}{1.919957in}}%
\pgfpathclose%
\pgfusepath{stroke,fill}%
\end{pgfscope}%
\begin{pgfscope}%
\pgfpathrectangle{\pgfqpoint{3.793912in}{0.557870in}}{\pgfqpoint{2.446088in}{1.484734in}}%
\pgfusepath{clip}%
\pgfsetbuttcap%
\pgfsetroundjoin%
\definecolor{currentfill}{rgb}{0.298039,0.447059,0.690196}%
\pgfsetfillcolor{currentfill}%
\pgfsetlinewidth{1.003750pt}%
\definecolor{currentstroke}{rgb}{0.298039,0.447059,0.690196}%
\pgfsetstrokecolor{currentstroke}%
\pgfsetdash{}{0pt}%
\pgfpathmoveto{\pgfqpoint{3.905098in}{0.899604in}}%
\pgfpathcurveto{\pgfqpoint{3.913334in}{0.899604in}}{\pgfqpoint{3.921234in}{0.902876in}}{\pgfqpoint{3.927058in}{0.908700in}}%
\pgfpathcurveto{\pgfqpoint{3.932882in}{0.914524in}}{\pgfqpoint{3.936155in}{0.922424in}}{\pgfqpoint{3.936155in}{0.930661in}}%
\pgfpathcurveto{\pgfqpoint{3.936155in}{0.938897in}}{\pgfqpoint{3.932882in}{0.946797in}}{\pgfqpoint{3.927058in}{0.952621in}}%
\pgfpathcurveto{\pgfqpoint{3.921234in}{0.958445in}}{\pgfqpoint{3.913334in}{0.961717in}}{\pgfqpoint{3.905098in}{0.961717in}}%
\pgfpathcurveto{\pgfqpoint{3.896862in}{0.961717in}}{\pgfqpoint{3.888962in}{0.958445in}}{\pgfqpoint{3.883138in}{0.952621in}}%
\pgfpathcurveto{\pgfqpoint{3.877314in}{0.946797in}}{\pgfqpoint{3.874042in}{0.938897in}}{\pgfqpoint{3.874042in}{0.930661in}}%
\pgfpathcurveto{\pgfqpoint{3.874042in}{0.922424in}}{\pgfqpoint{3.877314in}{0.914524in}}{\pgfqpoint{3.883138in}{0.908700in}}%
\pgfpathcurveto{\pgfqpoint{3.888962in}{0.902876in}}{\pgfqpoint{3.896862in}{0.899604in}}{\pgfqpoint{3.905098in}{0.899604in}}%
\pgfpathclose%
\pgfusepath{stroke,fill}%
\end{pgfscope}%
\begin{pgfscope}%
\pgfpathrectangle{\pgfqpoint{3.793912in}{0.557870in}}{\pgfqpoint{2.446088in}{1.484734in}}%
\pgfusepath{clip}%
\pgfsetbuttcap%
\pgfsetroundjoin%
\definecolor{currentfill}{rgb}{0.298039,0.447059,0.690196}%
\pgfsetfillcolor{currentfill}%
\pgfsetlinewidth{1.003750pt}%
\definecolor{currentstroke}{rgb}{0.298039,0.447059,0.690196}%
\pgfsetstrokecolor{currentstroke}%
\pgfsetdash{}{0pt}%
\pgfpathmoveto{\pgfqpoint{3.905098in}{1.936025in}}%
\pgfpathcurveto{\pgfqpoint{3.913334in}{1.936025in}}{\pgfqpoint{3.921234in}{1.939298in}}{\pgfqpoint{3.927058in}{1.945122in}}%
\pgfpathcurveto{\pgfqpoint{3.932882in}{1.950946in}}{\pgfqpoint{3.936155in}{1.958846in}}{\pgfqpoint{3.936155in}{1.967082in}}%
\pgfpathcurveto{\pgfqpoint{3.936155in}{1.975318in}}{\pgfqpoint{3.932882in}{1.983218in}}{\pgfqpoint{3.927058in}{1.989042in}}%
\pgfpathcurveto{\pgfqpoint{3.921234in}{1.994866in}}{\pgfqpoint{3.913334in}{1.998138in}}{\pgfqpoint{3.905098in}{1.998138in}}%
\pgfpathcurveto{\pgfqpoint{3.896862in}{1.998138in}}{\pgfqpoint{3.888962in}{1.994866in}}{\pgfqpoint{3.883138in}{1.989042in}}%
\pgfpathcurveto{\pgfqpoint{3.877314in}{1.983218in}}{\pgfqpoint{3.874042in}{1.975318in}}{\pgfqpoint{3.874042in}{1.967082in}}%
\pgfpathcurveto{\pgfqpoint{3.874042in}{1.958846in}}{\pgfqpoint{3.877314in}{1.950946in}}{\pgfqpoint{3.883138in}{1.945122in}}%
\pgfpathcurveto{\pgfqpoint{3.888962in}{1.939298in}}{\pgfqpoint{3.896862in}{1.936025in}}{\pgfqpoint{3.905098in}{1.936025in}}%
\pgfpathclose%
\pgfusepath{stroke,fill}%
\end{pgfscope}%
\begin{pgfscope}%
\pgfpathrectangle{\pgfqpoint{3.793912in}{0.557870in}}{\pgfqpoint{2.446088in}{1.484734in}}%
\pgfusepath{clip}%
\pgfsetbuttcap%
\pgfsetroundjoin%
\definecolor{currentfill}{rgb}{0.298039,0.447059,0.690196}%
\pgfsetfillcolor{currentfill}%
\pgfsetlinewidth{1.003750pt}%
\definecolor{currentstroke}{rgb}{0.298039,0.447059,0.690196}%
\pgfsetstrokecolor{currentstroke}%
\pgfsetdash{}{0pt}%
\pgfpathmoveto{\pgfqpoint{3.905098in}{1.140632in}}%
\pgfpathcurveto{\pgfqpoint{3.913334in}{1.140632in}}{\pgfqpoint{3.921234in}{1.143905in}}{\pgfqpoint{3.927058in}{1.149729in}}%
\pgfpathcurveto{\pgfqpoint{3.932882in}{1.155553in}}{\pgfqpoint{3.936155in}{1.163453in}}{\pgfqpoint{3.936155in}{1.171689in}}%
\pgfpathcurveto{\pgfqpoint{3.936155in}{1.179925in}}{\pgfqpoint{3.932882in}{1.187825in}}{\pgfqpoint{3.927058in}{1.193649in}}%
\pgfpathcurveto{\pgfqpoint{3.921234in}{1.199473in}}{\pgfqpoint{3.913334in}{1.202745in}}{\pgfqpoint{3.905098in}{1.202745in}}%
\pgfpathcurveto{\pgfqpoint{3.896862in}{1.202745in}}{\pgfqpoint{3.888962in}{1.199473in}}{\pgfqpoint{3.883138in}{1.193649in}}%
\pgfpathcurveto{\pgfqpoint{3.877314in}{1.187825in}}{\pgfqpoint{3.874042in}{1.179925in}}{\pgfqpoint{3.874042in}{1.171689in}}%
\pgfpathcurveto{\pgfqpoint{3.874042in}{1.163453in}}{\pgfqpoint{3.877314in}{1.155553in}}{\pgfqpoint{3.883138in}{1.149729in}}%
\pgfpathcurveto{\pgfqpoint{3.888962in}{1.143905in}}{\pgfqpoint{3.896862in}{1.140632in}}{\pgfqpoint{3.905098in}{1.140632in}}%
\pgfpathclose%
\pgfusepath{stroke,fill}%
\end{pgfscope}%
\begin{pgfscope}%
\pgfpathrectangle{\pgfqpoint{3.793912in}{0.557870in}}{\pgfqpoint{2.446088in}{1.484734in}}%
\pgfusepath{clip}%
\pgfsetbuttcap%
\pgfsetroundjoin%
\definecolor{currentfill}{rgb}{0.298039,0.447059,0.690196}%
\pgfsetfillcolor{currentfill}%
\pgfsetlinewidth{1.003750pt}%
\definecolor{currentstroke}{rgb}{0.298039,0.447059,0.690196}%
\pgfsetstrokecolor{currentstroke}%
\pgfsetdash{}{0pt}%
\pgfpathmoveto{\pgfqpoint{3.905098in}{1.936025in}}%
\pgfpathcurveto{\pgfqpoint{3.913334in}{1.936025in}}{\pgfqpoint{3.921234in}{1.939298in}}{\pgfqpoint{3.927058in}{1.945122in}}%
\pgfpathcurveto{\pgfqpoint{3.932882in}{1.950946in}}{\pgfqpoint{3.936155in}{1.958846in}}{\pgfqpoint{3.936155in}{1.967082in}}%
\pgfpathcurveto{\pgfqpoint{3.936155in}{1.975318in}}{\pgfqpoint{3.932882in}{1.983218in}}{\pgfqpoint{3.927058in}{1.989042in}}%
\pgfpathcurveto{\pgfqpoint{3.921234in}{1.994866in}}{\pgfqpoint{3.913334in}{1.998138in}}{\pgfqpoint{3.905098in}{1.998138in}}%
\pgfpathcurveto{\pgfqpoint{3.896862in}{1.998138in}}{\pgfqpoint{3.888962in}{1.994866in}}{\pgfqpoint{3.883138in}{1.989042in}}%
\pgfpathcurveto{\pgfqpoint{3.877314in}{1.983218in}}{\pgfqpoint{3.874042in}{1.975318in}}{\pgfqpoint{3.874042in}{1.967082in}}%
\pgfpathcurveto{\pgfqpoint{3.874042in}{1.958846in}}{\pgfqpoint{3.877314in}{1.950946in}}{\pgfqpoint{3.883138in}{1.945122in}}%
\pgfpathcurveto{\pgfqpoint{3.888962in}{1.939298in}}{\pgfqpoint{3.896862in}{1.936025in}}{\pgfqpoint{3.905098in}{1.936025in}}%
\pgfpathclose%
\pgfusepath{stroke,fill}%
\end{pgfscope}%
\begin{pgfscope}%
\pgfpathrectangle{\pgfqpoint{3.793912in}{0.557870in}}{\pgfqpoint{2.446088in}{1.484734in}}%
\pgfusepath{clip}%
\pgfsetbuttcap%
\pgfsetroundjoin%
\definecolor{currentfill}{rgb}{0.298039,0.447059,0.690196}%
\pgfsetfillcolor{currentfill}%
\pgfsetlinewidth{1.003750pt}%
\definecolor{currentstroke}{rgb}{0.298039,0.447059,0.690196}%
\pgfsetstrokecolor{currentstroke}%
\pgfsetdash{}{0pt}%
\pgfpathmoveto{\pgfqpoint{3.905098in}{1.052255in}}%
\pgfpathcurveto{\pgfqpoint{3.913334in}{1.052255in}}{\pgfqpoint{3.921234in}{1.055528in}}{\pgfqpoint{3.927058in}{1.061352in}}%
\pgfpathcurveto{\pgfqpoint{3.932882in}{1.067175in}}{\pgfqpoint{3.936155in}{1.075076in}}{\pgfqpoint{3.936155in}{1.083312in}}%
\pgfpathcurveto{\pgfqpoint{3.936155in}{1.091548in}}{\pgfqpoint{3.932882in}{1.099448in}}{\pgfqpoint{3.927058in}{1.105272in}}%
\pgfpathcurveto{\pgfqpoint{3.921234in}{1.111096in}}{\pgfqpoint{3.913334in}{1.114368in}}{\pgfqpoint{3.905098in}{1.114368in}}%
\pgfpathcurveto{\pgfqpoint{3.896862in}{1.114368in}}{\pgfqpoint{3.888962in}{1.111096in}}{\pgfqpoint{3.883138in}{1.105272in}}%
\pgfpathcurveto{\pgfqpoint{3.877314in}{1.099448in}}{\pgfqpoint{3.874042in}{1.091548in}}{\pgfqpoint{3.874042in}{1.083312in}}%
\pgfpathcurveto{\pgfqpoint{3.874042in}{1.075076in}}{\pgfqpoint{3.877314in}{1.067175in}}{\pgfqpoint{3.883138in}{1.061352in}}%
\pgfpathcurveto{\pgfqpoint{3.888962in}{1.055528in}}{\pgfqpoint{3.896862in}{1.052255in}}{\pgfqpoint{3.905098in}{1.052255in}}%
\pgfpathclose%
\pgfusepath{stroke,fill}%
\end{pgfscope}%
\begin{pgfscope}%
\pgfpathrectangle{\pgfqpoint{3.793912in}{0.557870in}}{\pgfqpoint{2.446088in}{1.484734in}}%
\pgfusepath{clip}%
\pgfsetbuttcap%
\pgfsetroundjoin%
\definecolor{currentfill}{rgb}{0.298039,0.447059,0.690196}%
\pgfsetfillcolor{currentfill}%
\pgfsetlinewidth{1.003750pt}%
\definecolor{currentstroke}{rgb}{0.298039,0.447059,0.690196}%
\pgfsetstrokecolor{currentstroke}%
\pgfsetdash{}{0pt}%
\pgfpathmoveto{\pgfqpoint{3.905098in}{1.936025in}}%
\pgfpathcurveto{\pgfqpoint{3.913334in}{1.936025in}}{\pgfqpoint{3.921234in}{1.939298in}}{\pgfqpoint{3.927058in}{1.945122in}}%
\pgfpathcurveto{\pgfqpoint{3.932882in}{1.950946in}}{\pgfqpoint{3.936155in}{1.958846in}}{\pgfqpoint{3.936155in}{1.967082in}}%
\pgfpathcurveto{\pgfqpoint{3.936155in}{1.975318in}}{\pgfqpoint{3.932882in}{1.983218in}}{\pgfqpoint{3.927058in}{1.989042in}}%
\pgfpathcurveto{\pgfqpoint{3.921234in}{1.994866in}}{\pgfqpoint{3.913334in}{1.998138in}}{\pgfqpoint{3.905098in}{1.998138in}}%
\pgfpathcurveto{\pgfqpoint{3.896862in}{1.998138in}}{\pgfqpoint{3.888962in}{1.994866in}}{\pgfqpoint{3.883138in}{1.989042in}}%
\pgfpathcurveto{\pgfqpoint{3.877314in}{1.983218in}}{\pgfqpoint{3.874042in}{1.975318in}}{\pgfqpoint{3.874042in}{1.967082in}}%
\pgfpathcurveto{\pgfqpoint{3.874042in}{1.958846in}}{\pgfqpoint{3.877314in}{1.950946in}}{\pgfqpoint{3.883138in}{1.945122in}}%
\pgfpathcurveto{\pgfqpoint{3.888962in}{1.939298in}}{\pgfqpoint{3.896862in}{1.936025in}}{\pgfqpoint{3.905098in}{1.936025in}}%
\pgfpathclose%
\pgfusepath{stroke,fill}%
\end{pgfscope}%
\begin{pgfscope}%
\pgfpathrectangle{\pgfqpoint{3.793912in}{0.557870in}}{\pgfqpoint{2.446088in}{1.484734in}}%
\pgfusepath{clip}%
\pgfsetbuttcap%
\pgfsetroundjoin%
\definecolor{currentfill}{rgb}{0.298039,0.447059,0.690196}%
\pgfsetfillcolor{currentfill}%
\pgfsetlinewidth{1.003750pt}%
\definecolor{currentstroke}{rgb}{0.298039,0.447059,0.690196}%
\pgfsetstrokecolor{currentstroke}%
\pgfsetdash{}{0pt}%
\pgfpathmoveto{\pgfqpoint{3.905098in}{1.936025in}}%
\pgfpathcurveto{\pgfqpoint{3.913334in}{1.936025in}}{\pgfqpoint{3.921234in}{1.939298in}}{\pgfqpoint{3.927058in}{1.945122in}}%
\pgfpathcurveto{\pgfqpoint{3.932882in}{1.950946in}}{\pgfqpoint{3.936155in}{1.958846in}}{\pgfqpoint{3.936155in}{1.967082in}}%
\pgfpathcurveto{\pgfqpoint{3.936155in}{1.975318in}}{\pgfqpoint{3.932882in}{1.983218in}}{\pgfqpoint{3.927058in}{1.989042in}}%
\pgfpathcurveto{\pgfqpoint{3.921234in}{1.994866in}}{\pgfqpoint{3.913334in}{1.998138in}}{\pgfqpoint{3.905098in}{1.998138in}}%
\pgfpathcurveto{\pgfqpoint{3.896862in}{1.998138in}}{\pgfqpoint{3.888962in}{1.994866in}}{\pgfqpoint{3.883138in}{1.989042in}}%
\pgfpathcurveto{\pgfqpoint{3.877314in}{1.983218in}}{\pgfqpoint{3.874042in}{1.975318in}}{\pgfqpoint{3.874042in}{1.967082in}}%
\pgfpathcurveto{\pgfqpoint{3.874042in}{1.958846in}}{\pgfqpoint{3.877314in}{1.950946in}}{\pgfqpoint{3.883138in}{1.945122in}}%
\pgfpathcurveto{\pgfqpoint{3.888962in}{1.939298in}}{\pgfqpoint{3.896862in}{1.936025in}}{\pgfqpoint{3.905098in}{1.936025in}}%
\pgfpathclose%
\pgfusepath{stroke,fill}%
\end{pgfscope}%
\begin{pgfscope}%
\pgfpathrectangle{\pgfqpoint{3.793912in}{0.557870in}}{\pgfqpoint{2.446088in}{1.484734in}}%
\pgfusepath{clip}%
\pgfsetbuttcap%
\pgfsetroundjoin%
\definecolor{currentfill}{rgb}{0.298039,0.447059,0.690196}%
\pgfsetfillcolor{currentfill}%
\pgfsetlinewidth{1.003750pt}%
\definecolor{currentstroke}{rgb}{0.298039,0.447059,0.690196}%
\pgfsetstrokecolor{currentstroke}%
\pgfsetdash{}{0pt}%
\pgfpathmoveto{\pgfqpoint{3.905098in}{1.936025in}}%
\pgfpathcurveto{\pgfqpoint{3.913334in}{1.936025in}}{\pgfqpoint{3.921234in}{1.939298in}}{\pgfqpoint{3.927058in}{1.945122in}}%
\pgfpathcurveto{\pgfqpoint{3.932882in}{1.950946in}}{\pgfqpoint{3.936155in}{1.958846in}}{\pgfqpoint{3.936155in}{1.967082in}}%
\pgfpathcurveto{\pgfqpoint{3.936155in}{1.975318in}}{\pgfqpoint{3.932882in}{1.983218in}}{\pgfqpoint{3.927058in}{1.989042in}}%
\pgfpathcurveto{\pgfqpoint{3.921234in}{1.994866in}}{\pgfqpoint{3.913334in}{1.998138in}}{\pgfqpoint{3.905098in}{1.998138in}}%
\pgfpathcurveto{\pgfqpoint{3.896862in}{1.998138in}}{\pgfqpoint{3.888962in}{1.994866in}}{\pgfqpoint{3.883138in}{1.989042in}}%
\pgfpathcurveto{\pgfqpoint{3.877314in}{1.983218in}}{\pgfqpoint{3.874042in}{1.975318in}}{\pgfqpoint{3.874042in}{1.967082in}}%
\pgfpathcurveto{\pgfqpoint{3.874042in}{1.958846in}}{\pgfqpoint{3.877314in}{1.950946in}}{\pgfqpoint{3.883138in}{1.945122in}}%
\pgfpathcurveto{\pgfqpoint{3.888962in}{1.939298in}}{\pgfqpoint{3.896862in}{1.936025in}}{\pgfqpoint{3.905098in}{1.936025in}}%
\pgfpathclose%
\pgfusepath{stroke,fill}%
\end{pgfscope}%
\begin{pgfscope}%
\pgfpathrectangle{\pgfqpoint{3.793912in}{0.557870in}}{\pgfqpoint{2.446088in}{1.484734in}}%
\pgfusepath{clip}%
\pgfsetbuttcap%
\pgfsetroundjoin%
\definecolor{currentfill}{rgb}{0.298039,0.447059,0.690196}%
\pgfsetfillcolor{currentfill}%
\pgfsetlinewidth{1.003750pt}%
\definecolor{currentstroke}{rgb}{0.298039,0.447059,0.690196}%
\pgfsetstrokecolor{currentstroke}%
\pgfsetdash{}{0pt}%
\pgfpathmoveto{\pgfqpoint{3.905098in}{1.936025in}}%
\pgfpathcurveto{\pgfqpoint{3.913334in}{1.936025in}}{\pgfqpoint{3.921234in}{1.939298in}}{\pgfqpoint{3.927058in}{1.945122in}}%
\pgfpathcurveto{\pgfqpoint{3.932882in}{1.950946in}}{\pgfqpoint{3.936155in}{1.958846in}}{\pgfqpoint{3.936155in}{1.967082in}}%
\pgfpathcurveto{\pgfqpoint{3.936155in}{1.975318in}}{\pgfqpoint{3.932882in}{1.983218in}}{\pgfqpoint{3.927058in}{1.989042in}}%
\pgfpathcurveto{\pgfqpoint{3.921234in}{1.994866in}}{\pgfqpoint{3.913334in}{1.998138in}}{\pgfqpoint{3.905098in}{1.998138in}}%
\pgfpathcurveto{\pgfqpoint{3.896862in}{1.998138in}}{\pgfqpoint{3.888962in}{1.994866in}}{\pgfqpoint{3.883138in}{1.989042in}}%
\pgfpathcurveto{\pgfqpoint{3.877314in}{1.983218in}}{\pgfqpoint{3.874042in}{1.975318in}}{\pgfqpoint{3.874042in}{1.967082in}}%
\pgfpathcurveto{\pgfqpoint{3.874042in}{1.958846in}}{\pgfqpoint{3.877314in}{1.950946in}}{\pgfqpoint{3.883138in}{1.945122in}}%
\pgfpathcurveto{\pgfqpoint{3.888962in}{1.939298in}}{\pgfqpoint{3.896862in}{1.936025in}}{\pgfqpoint{3.905098in}{1.936025in}}%
\pgfpathclose%
\pgfusepath{stroke,fill}%
\end{pgfscope}%
\begin{pgfscope}%
\pgfpathrectangle{\pgfqpoint{3.793912in}{0.557870in}}{\pgfqpoint{2.446088in}{1.484734in}}%
\pgfusepath{clip}%
\pgfsetbuttcap%
\pgfsetroundjoin%
\definecolor{currentfill}{rgb}{0.298039,0.447059,0.690196}%
\pgfsetfillcolor{currentfill}%
\pgfsetlinewidth{1.003750pt}%
\definecolor{currentstroke}{rgb}{0.298039,0.447059,0.690196}%
\pgfsetstrokecolor{currentstroke}%
\pgfsetdash{}{0pt}%
\pgfpathmoveto{\pgfqpoint{3.905098in}{0.979947in}}%
\pgfpathcurveto{\pgfqpoint{3.913334in}{0.979947in}}{\pgfqpoint{3.921234in}{0.983219in}}{\pgfqpoint{3.927058in}{0.989043in}}%
\pgfpathcurveto{\pgfqpoint{3.932882in}{0.994867in}}{\pgfqpoint{3.936155in}{1.002767in}}{\pgfqpoint{3.936155in}{1.011003in}}%
\pgfpathcurveto{\pgfqpoint{3.936155in}{1.019240in}}{\pgfqpoint{3.932882in}{1.027140in}}{\pgfqpoint{3.927058in}{1.032964in}}%
\pgfpathcurveto{\pgfqpoint{3.921234in}{1.038788in}}{\pgfqpoint{3.913334in}{1.042060in}}{\pgfqpoint{3.905098in}{1.042060in}}%
\pgfpathcurveto{\pgfqpoint{3.896862in}{1.042060in}}{\pgfqpoint{3.888962in}{1.038788in}}{\pgfqpoint{3.883138in}{1.032964in}}%
\pgfpathcurveto{\pgfqpoint{3.877314in}{1.027140in}}{\pgfqpoint{3.874042in}{1.019240in}}{\pgfqpoint{3.874042in}{1.011003in}}%
\pgfpathcurveto{\pgfqpoint{3.874042in}{1.002767in}}{\pgfqpoint{3.877314in}{0.994867in}}{\pgfqpoint{3.883138in}{0.989043in}}%
\pgfpathcurveto{\pgfqpoint{3.888962in}{0.983219in}}{\pgfqpoint{3.896862in}{0.979947in}}{\pgfqpoint{3.905098in}{0.979947in}}%
\pgfpathclose%
\pgfusepath{stroke,fill}%
\end{pgfscope}%
\begin{pgfscope}%
\pgfpathrectangle{\pgfqpoint{3.793912in}{0.557870in}}{\pgfqpoint{2.446088in}{1.484734in}}%
\pgfusepath{clip}%
\pgfsetbuttcap%
\pgfsetroundjoin%
\definecolor{currentfill}{rgb}{0.298039,0.447059,0.690196}%
\pgfsetfillcolor{currentfill}%
\pgfsetlinewidth{1.003750pt}%
\definecolor{currentstroke}{rgb}{0.298039,0.447059,0.690196}%
\pgfsetstrokecolor{currentstroke}%
\pgfsetdash{}{0pt}%
\pgfpathmoveto{\pgfqpoint{3.905098in}{1.936025in}}%
\pgfpathcurveto{\pgfqpoint{3.913334in}{1.936025in}}{\pgfqpoint{3.921234in}{1.939298in}}{\pgfqpoint{3.927058in}{1.945122in}}%
\pgfpathcurveto{\pgfqpoint{3.932882in}{1.950946in}}{\pgfqpoint{3.936155in}{1.958846in}}{\pgfqpoint{3.936155in}{1.967082in}}%
\pgfpathcurveto{\pgfqpoint{3.936155in}{1.975318in}}{\pgfqpoint{3.932882in}{1.983218in}}{\pgfqpoint{3.927058in}{1.989042in}}%
\pgfpathcurveto{\pgfqpoint{3.921234in}{1.994866in}}{\pgfqpoint{3.913334in}{1.998138in}}{\pgfqpoint{3.905098in}{1.998138in}}%
\pgfpathcurveto{\pgfqpoint{3.896862in}{1.998138in}}{\pgfqpoint{3.888962in}{1.994866in}}{\pgfqpoint{3.883138in}{1.989042in}}%
\pgfpathcurveto{\pgfqpoint{3.877314in}{1.983218in}}{\pgfqpoint{3.874042in}{1.975318in}}{\pgfqpoint{3.874042in}{1.967082in}}%
\pgfpathcurveto{\pgfqpoint{3.874042in}{1.958846in}}{\pgfqpoint{3.877314in}{1.950946in}}{\pgfqpoint{3.883138in}{1.945122in}}%
\pgfpathcurveto{\pgfqpoint{3.888962in}{1.939298in}}{\pgfqpoint{3.896862in}{1.936025in}}{\pgfqpoint{3.905098in}{1.936025in}}%
\pgfpathclose%
\pgfusepath{stroke,fill}%
\end{pgfscope}%
\begin{pgfscope}%
\pgfpathrectangle{\pgfqpoint{3.793912in}{0.557870in}}{\pgfqpoint{2.446088in}{1.484734in}}%
\pgfusepath{clip}%
\pgfsetbuttcap%
\pgfsetroundjoin%
\definecolor{currentfill}{rgb}{0.298039,0.447059,0.690196}%
\pgfsetfillcolor{currentfill}%
\pgfsetlinewidth{1.003750pt}%
\definecolor{currentstroke}{rgb}{0.298039,0.447059,0.690196}%
\pgfsetstrokecolor{currentstroke}%
\pgfsetdash{}{0pt}%
\pgfpathmoveto{\pgfqpoint{3.905098in}{1.052255in}}%
\pgfpathcurveto{\pgfqpoint{3.913334in}{1.052255in}}{\pgfqpoint{3.921234in}{1.055528in}}{\pgfqpoint{3.927058in}{1.061352in}}%
\pgfpathcurveto{\pgfqpoint{3.932882in}{1.067175in}}{\pgfqpoint{3.936155in}{1.075076in}}{\pgfqpoint{3.936155in}{1.083312in}}%
\pgfpathcurveto{\pgfqpoint{3.936155in}{1.091548in}}{\pgfqpoint{3.932882in}{1.099448in}}{\pgfqpoint{3.927058in}{1.105272in}}%
\pgfpathcurveto{\pgfqpoint{3.921234in}{1.111096in}}{\pgfqpoint{3.913334in}{1.114368in}}{\pgfqpoint{3.905098in}{1.114368in}}%
\pgfpathcurveto{\pgfqpoint{3.896862in}{1.114368in}}{\pgfqpoint{3.888962in}{1.111096in}}{\pgfqpoint{3.883138in}{1.105272in}}%
\pgfpathcurveto{\pgfqpoint{3.877314in}{1.099448in}}{\pgfqpoint{3.874042in}{1.091548in}}{\pgfqpoint{3.874042in}{1.083312in}}%
\pgfpathcurveto{\pgfqpoint{3.874042in}{1.075076in}}{\pgfqpoint{3.877314in}{1.067175in}}{\pgfqpoint{3.883138in}{1.061352in}}%
\pgfpathcurveto{\pgfqpoint{3.888962in}{1.055528in}}{\pgfqpoint{3.896862in}{1.052255in}}{\pgfqpoint{3.905098in}{1.052255in}}%
\pgfpathclose%
\pgfusepath{stroke,fill}%
\end{pgfscope}%
\begin{pgfscope}%
\pgfpathrectangle{\pgfqpoint{3.793912in}{0.557870in}}{\pgfqpoint{2.446088in}{1.484734in}}%
\pgfusepath{clip}%
\pgfsetbuttcap%
\pgfsetroundjoin%
\definecolor{currentfill}{rgb}{0.298039,0.447059,0.690196}%
\pgfsetfillcolor{currentfill}%
\pgfsetlinewidth{1.003750pt}%
\definecolor{currentstroke}{rgb}{0.298039,0.447059,0.690196}%
\pgfsetstrokecolor{currentstroke}%
\pgfsetdash{}{0pt}%
\pgfpathmoveto{\pgfqpoint{3.905098in}{1.936025in}}%
\pgfpathcurveto{\pgfqpoint{3.913334in}{1.936025in}}{\pgfqpoint{3.921234in}{1.939298in}}{\pgfqpoint{3.927058in}{1.945122in}}%
\pgfpathcurveto{\pgfqpoint{3.932882in}{1.950946in}}{\pgfqpoint{3.936155in}{1.958846in}}{\pgfqpoint{3.936155in}{1.967082in}}%
\pgfpathcurveto{\pgfqpoint{3.936155in}{1.975318in}}{\pgfqpoint{3.932882in}{1.983218in}}{\pgfqpoint{3.927058in}{1.989042in}}%
\pgfpathcurveto{\pgfqpoint{3.921234in}{1.994866in}}{\pgfqpoint{3.913334in}{1.998138in}}{\pgfqpoint{3.905098in}{1.998138in}}%
\pgfpathcurveto{\pgfqpoint{3.896862in}{1.998138in}}{\pgfqpoint{3.888962in}{1.994866in}}{\pgfqpoint{3.883138in}{1.989042in}}%
\pgfpathcurveto{\pgfqpoint{3.877314in}{1.983218in}}{\pgfqpoint{3.874042in}{1.975318in}}{\pgfqpoint{3.874042in}{1.967082in}}%
\pgfpathcurveto{\pgfqpoint{3.874042in}{1.958846in}}{\pgfqpoint{3.877314in}{1.950946in}}{\pgfqpoint{3.883138in}{1.945122in}}%
\pgfpathcurveto{\pgfqpoint{3.888962in}{1.939298in}}{\pgfqpoint{3.896862in}{1.936025in}}{\pgfqpoint{3.905098in}{1.936025in}}%
\pgfpathclose%
\pgfusepath{stroke,fill}%
\end{pgfscope}%
\begin{pgfscope}%
\pgfpathrectangle{\pgfqpoint{3.793912in}{0.557870in}}{\pgfqpoint{2.446088in}{1.484734in}}%
\pgfusepath{clip}%
\pgfsetbuttcap%
\pgfsetroundjoin%
\definecolor{currentfill}{rgb}{0.298039,0.447059,0.690196}%
\pgfsetfillcolor{currentfill}%
\pgfsetlinewidth{1.003750pt}%
\definecolor{currentstroke}{rgb}{0.298039,0.447059,0.690196}%
\pgfsetstrokecolor{currentstroke}%
\pgfsetdash{}{0pt}%
\pgfpathmoveto{\pgfqpoint{3.905098in}{1.936025in}}%
\pgfpathcurveto{\pgfqpoint{3.913334in}{1.936025in}}{\pgfqpoint{3.921234in}{1.939298in}}{\pgfqpoint{3.927058in}{1.945122in}}%
\pgfpathcurveto{\pgfqpoint{3.932882in}{1.950946in}}{\pgfqpoint{3.936155in}{1.958846in}}{\pgfqpoint{3.936155in}{1.967082in}}%
\pgfpathcurveto{\pgfqpoint{3.936155in}{1.975318in}}{\pgfqpoint{3.932882in}{1.983218in}}{\pgfqpoint{3.927058in}{1.989042in}}%
\pgfpathcurveto{\pgfqpoint{3.921234in}{1.994866in}}{\pgfqpoint{3.913334in}{1.998138in}}{\pgfqpoint{3.905098in}{1.998138in}}%
\pgfpathcurveto{\pgfqpoint{3.896862in}{1.998138in}}{\pgfqpoint{3.888962in}{1.994866in}}{\pgfqpoint{3.883138in}{1.989042in}}%
\pgfpathcurveto{\pgfqpoint{3.877314in}{1.983218in}}{\pgfqpoint{3.874042in}{1.975318in}}{\pgfqpoint{3.874042in}{1.967082in}}%
\pgfpathcurveto{\pgfqpoint{3.874042in}{1.958846in}}{\pgfqpoint{3.877314in}{1.950946in}}{\pgfqpoint{3.883138in}{1.945122in}}%
\pgfpathcurveto{\pgfqpoint{3.888962in}{1.939298in}}{\pgfqpoint{3.896862in}{1.936025in}}{\pgfqpoint{3.905098in}{1.936025in}}%
\pgfpathclose%
\pgfusepath{stroke,fill}%
\end{pgfscope}%
\begin{pgfscope}%
\pgfpathrectangle{\pgfqpoint{3.793912in}{0.557870in}}{\pgfqpoint{2.446088in}{1.484734in}}%
\pgfusepath{clip}%
\pgfsetbuttcap%
\pgfsetroundjoin%
\definecolor{currentfill}{rgb}{0.298039,0.447059,0.690196}%
\pgfsetfillcolor{currentfill}%
\pgfsetlinewidth{1.003750pt}%
\definecolor{currentstroke}{rgb}{0.298039,0.447059,0.690196}%
\pgfsetstrokecolor{currentstroke}%
\pgfsetdash{}{0pt}%
\pgfpathmoveto{\pgfqpoint{3.905098in}{1.936025in}}%
\pgfpathcurveto{\pgfqpoint{3.913334in}{1.936025in}}{\pgfqpoint{3.921234in}{1.939298in}}{\pgfqpoint{3.927058in}{1.945122in}}%
\pgfpathcurveto{\pgfqpoint{3.932882in}{1.950946in}}{\pgfqpoint{3.936155in}{1.958846in}}{\pgfqpoint{3.936155in}{1.967082in}}%
\pgfpathcurveto{\pgfqpoint{3.936155in}{1.975318in}}{\pgfqpoint{3.932882in}{1.983218in}}{\pgfqpoint{3.927058in}{1.989042in}}%
\pgfpathcurveto{\pgfqpoint{3.921234in}{1.994866in}}{\pgfqpoint{3.913334in}{1.998138in}}{\pgfqpoint{3.905098in}{1.998138in}}%
\pgfpathcurveto{\pgfqpoint{3.896862in}{1.998138in}}{\pgfqpoint{3.888962in}{1.994866in}}{\pgfqpoint{3.883138in}{1.989042in}}%
\pgfpathcurveto{\pgfqpoint{3.877314in}{1.983218in}}{\pgfqpoint{3.874042in}{1.975318in}}{\pgfqpoint{3.874042in}{1.967082in}}%
\pgfpathcurveto{\pgfqpoint{3.874042in}{1.958846in}}{\pgfqpoint{3.877314in}{1.950946in}}{\pgfqpoint{3.883138in}{1.945122in}}%
\pgfpathcurveto{\pgfqpoint{3.888962in}{1.939298in}}{\pgfqpoint{3.896862in}{1.936025in}}{\pgfqpoint{3.905098in}{1.936025in}}%
\pgfpathclose%
\pgfusepath{stroke,fill}%
\end{pgfscope}%
\begin{pgfscope}%
\pgfpathrectangle{\pgfqpoint{3.793912in}{0.557870in}}{\pgfqpoint{2.446088in}{1.484734in}}%
\pgfusepath{clip}%
\pgfsetbuttcap%
\pgfsetroundjoin%
\definecolor{currentfill}{rgb}{0.298039,0.447059,0.690196}%
\pgfsetfillcolor{currentfill}%
\pgfsetlinewidth{1.003750pt}%
\definecolor{currentstroke}{rgb}{0.298039,0.447059,0.690196}%
\pgfsetstrokecolor{currentstroke}%
\pgfsetdash{}{0pt}%
\pgfpathmoveto{\pgfqpoint{4.063935in}{0.594302in}}%
\pgfpathcurveto{\pgfqpoint{4.072171in}{0.594302in}}{\pgfqpoint{4.080071in}{0.597574in}}{\pgfqpoint{4.085895in}{0.603398in}}%
\pgfpathcurveto{\pgfqpoint{4.091719in}{0.609222in}}{\pgfqpoint{4.094991in}{0.617122in}}{\pgfqpoint{4.094991in}{0.625358in}}%
\pgfpathcurveto{\pgfqpoint{4.094991in}{0.633594in}}{\pgfqpoint{4.091719in}{0.641495in}}{\pgfqpoint{4.085895in}{0.647318in}}%
\pgfpathcurveto{\pgfqpoint{4.080071in}{0.653142in}}{\pgfqpoint{4.072171in}{0.656415in}}{\pgfqpoint{4.063935in}{0.656415in}}%
\pgfpathcurveto{\pgfqpoint{4.055699in}{0.656415in}}{\pgfqpoint{4.047799in}{0.653142in}}{\pgfqpoint{4.041975in}{0.647318in}}%
\pgfpathcurveto{\pgfqpoint{4.036151in}{0.641495in}}{\pgfqpoint{4.032878in}{0.633594in}}{\pgfqpoint{4.032878in}{0.625358in}}%
\pgfpathcurveto{\pgfqpoint{4.032878in}{0.617122in}}{\pgfqpoint{4.036151in}{0.609222in}}{\pgfqpoint{4.041975in}{0.603398in}}%
\pgfpathcurveto{\pgfqpoint{4.047799in}{0.597574in}}{\pgfqpoint{4.055699in}{0.594302in}}{\pgfqpoint{4.063935in}{0.594302in}}%
\pgfpathclose%
\pgfusepath{stroke,fill}%
\end{pgfscope}%
\begin{pgfscope}%
\pgfpathrectangle{\pgfqpoint{3.793912in}{0.557870in}}{\pgfqpoint{2.446088in}{1.484734in}}%
\pgfusepath{clip}%
\pgfsetbuttcap%
\pgfsetroundjoin%
\definecolor{currentfill}{rgb}{0.298039,0.447059,0.690196}%
\pgfsetfillcolor{currentfill}%
\pgfsetlinewidth{1.003750pt}%
\definecolor{currentstroke}{rgb}{0.298039,0.447059,0.690196}%
\pgfsetstrokecolor{currentstroke}%
\pgfsetdash{}{0pt}%
\pgfpathmoveto{\pgfqpoint{3.905098in}{1.936025in}}%
\pgfpathcurveto{\pgfqpoint{3.913334in}{1.936025in}}{\pgfqpoint{3.921234in}{1.939298in}}{\pgfqpoint{3.927058in}{1.945122in}}%
\pgfpathcurveto{\pgfqpoint{3.932882in}{1.950946in}}{\pgfqpoint{3.936155in}{1.958846in}}{\pgfqpoint{3.936155in}{1.967082in}}%
\pgfpathcurveto{\pgfqpoint{3.936155in}{1.975318in}}{\pgfqpoint{3.932882in}{1.983218in}}{\pgfqpoint{3.927058in}{1.989042in}}%
\pgfpathcurveto{\pgfqpoint{3.921234in}{1.994866in}}{\pgfqpoint{3.913334in}{1.998138in}}{\pgfqpoint{3.905098in}{1.998138in}}%
\pgfpathcurveto{\pgfqpoint{3.896862in}{1.998138in}}{\pgfqpoint{3.888962in}{1.994866in}}{\pgfqpoint{3.883138in}{1.989042in}}%
\pgfpathcurveto{\pgfqpoint{3.877314in}{1.983218in}}{\pgfqpoint{3.874042in}{1.975318in}}{\pgfqpoint{3.874042in}{1.967082in}}%
\pgfpathcurveto{\pgfqpoint{3.874042in}{1.958846in}}{\pgfqpoint{3.877314in}{1.950946in}}{\pgfqpoint{3.883138in}{1.945122in}}%
\pgfpathcurveto{\pgfqpoint{3.888962in}{1.939298in}}{\pgfqpoint{3.896862in}{1.936025in}}{\pgfqpoint{3.905098in}{1.936025in}}%
\pgfpathclose%
\pgfusepath{stroke,fill}%
\end{pgfscope}%
\begin{pgfscope}%
\pgfpathrectangle{\pgfqpoint{3.793912in}{0.557870in}}{\pgfqpoint{2.446088in}{1.484734in}}%
\pgfusepath{clip}%
\pgfsetbuttcap%
\pgfsetroundjoin%
\definecolor{currentfill}{rgb}{0.298039,0.447059,0.690196}%
\pgfsetfillcolor{currentfill}%
\pgfsetlinewidth{1.003750pt}%
\definecolor{currentstroke}{rgb}{0.298039,0.447059,0.690196}%
\pgfsetstrokecolor{currentstroke}%
\pgfsetdash{}{0pt}%
\pgfpathmoveto{\pgfqpoint{3.905098in}{0.971913in}}%
\pgfpathcurveto{\pgfqpoint{3.913334in}{0.971913in}}{\pgfqpoint{3.921234in}{0.975185in}}{\pgfqpoint{3.927058in}{0.981009in}}%
\pgfpathcurveto{\pgfqpoint{3.932882in}{0.986833in}}{\pgfqpoint{3.936155in}{0.994733in}}{\pgfqpoint{3.936155in}{1.002969in}}%
\pgfpathcurveto{\pgfqpoint{3.936155in}{1.011205in}}{\pgfqpoint{3.932882in}{1.019105in}}{\pgfqpoint{3.927058in}{1.024929in}}%
\pgfpathcurveto{\pgfqpoint{3.921234in}{1.030753in}}{\pgfqpoint{3.913334in}{1.034026in}}{\pgfqpoint{3.905098in}{1.034026in}}%
\pgfpathcurveto{\pgfqpoint{3.896862in}{1.034026in}}{\pgfqpoint{3.888962in}{1.030753in}}{\pgfqpoint{3.883138in}{1.024929in}}%
\pgfpathcurveto{\pgfqpoint{3.877314in}{1.019105in}}{\pgfqpoint{3.874042in}{1.011205in}}{\pgfqpoint{3.874042in}{1.002969in}}%
\pgfpathcurveto{\pgfqpoint{3.874042in}{0.994733in}}{\pgfqpoint{3.877314in}{0.986833in}}{\pgfqpoint{3.883138in}{0.981009in}}%
\pgfpathcurveto{\pgfqpoint{3.888962in}{0.975185in}}{\pgfqpoint{3.896862in}{0.971913in}}{\pgfqpoint{3.905098in}{0.971913in}}%
\pgfpathclose%
\pgfusepath{stroke,fill}%
\end{pgfscope}%
\begin{pgfscope}%
\pgfpathrectangle{\pgfqpoint{3.793912in}{0.557870in}}{\pgfqpoint{2.446088in}{1.484734in}}%
\pgfusepath{clip}%
\pgfsetbuttcap%
\pgfsetroundjoin%
\definecolor{currentfill}{rgb}{0.298039,0.447059,0.690196}%
\pgfsetfillcolor{currentfill}%
\pgfsetlinewidth{1.003750pt}%
\definecolor{currentstroke}{rgb}{0.298039,0.447059,0.690196}%
\pgfsetstrokecolor{currentstroke}%
\pgfsetdash{}{0pt}%
\pgfpathmoveto{\pgfqpoint{3.905098in}{1.936025in}}%
\pgfpathcurveto{\pgfqpoint{3.913334in}{1.936025in}}{\pgfqpoint{3.921234in}{1.939298in}}{\pgfqpoint{3.927058in}{1.945122in}}%
\pgfpathcurveto{\pgfqpoint{3.932882in}{1.950946in}}{\pgfqpoint{3.936155in}{1.958846in}}{\pgfqpoint{3.936155in}{1.967082in}}%
\pgfpathcurveto{\pgfqpoint{3.936155in}{1.975318in}}{\pgfqpoint{3.932882in}{1.983218in}}{\pgfqpoint{3.927058in}{1.989042in}}%
\pgfpathcurveto{\pgfqpoint{3.921234in}{1.994866in}}{\pgfqpoint{3.913334in}{1.998138in}}{\pgfqpoint{3.905098in}{1.998138in}}%
\pgfpathcurveto{\pgfqpoint{3.896862in}{1.998138in}}{\pgfqpoint{3.888962in}{1.994866in}}{\pgfqpoint{3.883138in}{1.989042in}}%
\pgfpathcurveto{\pgfqpoint{3.877314in}{1.983218in}}{\pgfqpoint{3.874042in}{1.975318in}}{\pgfqpoint{3.874042in}{1.967082in}}%
\pgfpathcurveto{\pgfqpoint{3.874042in}{1.958846in}}{\pgfqpoint{3.877314in}{1.950946in}}{\pgfqpoint{3.883138in}{1.945122in}}%
\pgfpathcurveto{\pgfqpoint{3.888962in}{1.939298in}}{\pgfqpoint{3.896862in}{1.936025in}}{\pgfqpoint{3.905098in}{1.936025in}}%
\pgfpathclose%
\pgfusepath{stroke,fill}%
\end{pgfscope}%
\begin{pgfscope}%
\pgfpathrectangle{\pgfqpoint{3.793912in}{0.557870in}}{\pgfqpoint{2.446088in}{1.484734in}}%
\pgfusepath{clip}%
\pgfsetbuttcap%
\pgfsetroundjoin%
\definecolor{currentfill}{rgb}{0.298039,0.447059,0.690196}%
\pgfsetfillcolor{currentfill}%
\pgfsetlinewidth{1.003750pt}%
\definecolor{currentstroke}{rgb}{0.298039,0.447059,0.690196}%
\pgfsetstrokecolor{currentstroke}%
\pgfsetdash{}{0pt}%
\pgfpathmoveto{\pgfqpoint{4.063935in}{0.594302in}}%
\pgfpathcurveto{\pgfqpoint{4.072171in}{0.594302in}}{\pgfqpoint{4.080071in}{0.597574in}}{\pgfqpoint{4.085895in}{0.603398in}}%
\pgfpathcurveto{\pgfqpoint{4.091719in}{0.609222in}}{\pgfqpoint{4.094991in}{0.617122in}}{\pgfqpoint{4.094991in}{0.625358in}}%
\pgfpathcurveto{\pgfqpoint{4.094991in}{0.633594in}}{\pgfqpoint{4.091719in}{0.641495in}}{\pgfqpoint{4.085895in}{0.647318in}}%
\pgfpathcurveto{\pgfqpoint{4.080071in}{0.653142in}}{\pgfqpoint{4.072171in}{0.656415in}}{\pgfqpoint{4.063935in}{0.656415in}}%
\pgfpathcurveto{\pgfqpoint{4.055699in}{0.656415in}}{\pgfqpoint{4.047799in}{0.653142in}}{\pgfqpoint{4.041975in}{0.647318in}}%
\pgfpathcurveto{\pgfqpoint{4.036151in}{0.641495in}}{\pgfqpoint{4.032878in}{0.633594in}}{\pgfqpoint{4.032878in}{0.625358in}}%
\pgfpathcurveto{\pgfqpoint{4.032878in}{0.617122in}}{\pgfqpoint{4.036151in}{0.609222in}}{\pgfqpoint{4.041975in}{0.603398in}}%
\pgfpathcurveto{\pgfqpoint{4.047799in}{0.597574in}}{\pgfqpoint{4.055699in}{0.594302in}}{\pgfqpoint{4.063935in}{0.594302in}}%
\pgfpathclose%
\pgfusepath{stroke,fill}%
\end{pgfscope}%
\begin{pgfscope}%
\pgfpathrectangle{\pgfqpoint{3.793912in}{0.557870in}}{\pgfqpoint{2.446088in}{1.484734in}}%
\pgfusepath{clip}%
\pgfsetbuttcap%
\pgfsetroundjoin%
\definecolor{currentfill}{rgb}{0.298039,0.447059,0.690196}%
\pgfsetfillcolor{currentfill}%
\pgfsetlinewidth{1.003750pt}%
\definecolor{currentstroke}{rgb}{0.298039,0.447059,0.690196}%
\pgfsetstrokecolor{currentstroke}%
\pgfsetdash{}{0pt}%
\pgfpathmoveto{\pgfqpoint{3.905098in}{1.927991in}}%
\pgfpathcurveto{\pgfqpoint{3.913334in}{1.927991in}}{\pgfqpoint{3.921234in}{1.931264in}}{\pgfqpoint{3.927058in}{1.937087in}}%
\pgfpathcurveto{\pgfqpoint{3.932882in}{1.942911in}}{\pgfqpoint{3.936155in}{1.950811in}}{\pgfqpoint{3.936155in}{1.959048in}}%
\pgfpathcurveto{\pgfqpoint{3.936155in}{1.967284in}}{\pgfqpoint{3.932882in}{1.975184in}}{\pgfqpoint{3.927058in}{1.981008in}}%
\pgfpathcurveto{\pgfqpoint{3.921234in}{1.986832in}}{\pgfqpoint{3.913334in}{1.990104in}}{\pgfqpoint{3.905098in}{1.990104in}}%
\pgfpathcurveto{\pgfqpoint{3.896862in}{1.990104in}}{\pgfqpoint{3.888962in}{1.986832in}}{\pgfqpoint{3.883138in}{1.981008in}}%
\pgfpathcurveto{\pgfqpoint{3.877314in}{1.975184in}}{\pgfqpoint{3.874042in}{1.967284in}}{\pgfqpoint{3.874042in}{1.959048in}}%
\pgfpathcurveto{\pgfqpoint{3.874042in}{1.950811in}}{\pgfqpoint{3.877314in}{1.942911in}}{\pgfqpoint{3.883138in}{1.937087in}}%
\pgfpathcurveto{\pgfqpoint{3.888962in}{1.931264in}}{\pgfqpoint{3.896862in}{1.927991in}}{\pgfqpoint{3.905098in}{1.927991in}}%
\pgfpathclose%
\pgfusepath{stroke,fill}%
\end{pgfscope}%
\begin{pgfscope}%
\pgfpathrectangle{\pgfqpoint{3.793912in}{0.557870in}}{\pgfqpoint{2.446088in}{1.484734in}}%
\pgfusepath{clip}%
\pgfsetbuttcap%
\pgfsetroundjoin%
\definecolor{currentfill}{rgb}{0.298039,0.447059,0.690196}%
\pgfsetfillcolor{currentfill}%
\pgfsetlinewidth{1.003750pt}%
\definecolor{currentstroke}{rgb}{0.298039,0.447059,0.690196}%
\pgfsetstrokecolor{currentstroke}%
\pgfsetdash{}{0pt}%
\pgfpathmoveto{\pgfqpoint{3.905098in}{1.936025in}}%
\pgfpathcurveto{\pgfqpoint{3.913334in}{1.936025in}}{\pgfqpoint{3.921234in}{1.939298in}}{\pgfqpoint{3.927058in}{1.945122in}}%
\pgfpathcurveto{\pgfqpoint{3.932882in}{1.950946in}}{\pgfqpoint{3.936155in}{1.958846in}}{\pgfqpoint{3.936155in}{1.967082in}}%
\pgfpathcurveto{\pgfqpoint{3.936155in}{1.975318in}}{\pgfqpoint{3.932882in}{1.983218in}}{\pgfqpoint{3.927058in}{1.989042in}}%
\pgfpathcurveto{\pgfqpoint{3.921234in}{1.994866in}}{\pgfqpoint{3.913334in}{1.998138in}}{\pgfqpoint{3.905098in}{1.998138in}}%
\pgfpathcurveto{\pgfqpoint{3.896862in}{1.998138in}}{\pgfqpoint{3.888962in}{1.994866in}}{\pgfqpoint{3.883138in}{1.989042in}}%
\pgfpathcurveto{\pgfqpoint{3.877314in}{1.983218in}}{\pgfqpoint{3.874042in}{1.975318in}}{\pgfqpoint{3.874042in}{1.967082in}}%
\pgfpathcurveto{\pgfqpoint{3.874042in}{1.958846in}}{\pgfqpoint{3.877314in}{1.950946in}}{\pgfqpoint{3.883138in}{1.945122in}}%
\pgfpathcurveto{\pgfqpoint{3.888962in}{1.939298in}}{\pgfqpoint{3.896862in}{1.936025in}}{\pgfqpoint{3.905098in}{1.936025in}}%
\pgfpathclose%
\pgfusepath{stroke,fill}%
\end{pgfscope}%
\begin{pgfscope}%
\pgfpathrectangle{\pgfqpoint{3.793912in}{0.557870in}}{\pgfqpoint{2.446088in}{1.484734in}}%
\pgfusepath{clip}%
\pgfsetbuttcap%
\pgfsetroundjoin%
\definecolor{currentfill}{rgb}{0.298039,0.447059,0.690196}%
\pgfsetfillcolor{currentfill}%
\pgfsetlinewidth{1.003750pt}%
\definecolor{currentstroke}{rgb}{0.298039,0.447059,0.690196}%
\pgfsetstrokecolor{currentstroke}%
\pgfsetdash{}{0pt}%
\pgfpathmoveto{\pgfqpoint{3.905098in}{1.389695in}}%
\pgfpathcurveto{\pgfqpoint{3.913334in}{1.389695in}}{\pgfqpoint{3.921234in}{1.392967in}}{\pgfqpoint{3.927058in}{1.398791in}}%
\pgfpathcurveto{\pgfqpoint{3.932882in}{1.404615in}}{\pgfqpoint{3.936155in}{1.412515in}}{\pgfqpoint{3.936155in}{1.420751in}}%
\pgfpathcurveto{\pgfqpoint{3.936155in}{1.428988in}}{\pgfqpoint{3.932882in}{1.436888in}}{\pgfqpoint{3.927058in}{1.442712in}}%
\pgfpathcurveto{\pgfqpoint{3.921234in}{1.448536in}}{\pgfqpoint{3.913334in}{1.451808in}}{\pgfqpoint{3.905098in}{1.451808in}}%
\pgfpathcurveto{\pgfqpoint{3.896862in}{1.451808in}}{\pgfqpoint{3.888962in}{1.448536in}}{\pgfqpoint{3.883138in}{1.442712in}}%
\pgfpathcurveto{\pgfqpoint{3.877314in}{1.436888in}}{\pgfqpoint{3.874042in}{1.428988in}}{\pgfqpoint{3.874042in}{1.420751in}}%
\pgfpathcurveto{\pgfqpoint{3.874042in}{1.412515in}}{\pgfqpoint{3.877314in}{1.404615in}}{\pgfqpoint{3.883138in}{1.398791in}}%
\pgfpathcurveto{\pgfqpoint{3.888962in}{1.392967in}}{\pgfqpoint{3.896862in}{1.389695in}}{\pgfqpoint{3.905098in}{1.389695in}}%
\pgfpathclose%
\pgfusepath{stroke,fill}%
\end{pgfscope}%
\begin{pgfscope}%
\pgfpathrectangle{\pgfqpoint{3.793912in}{0.557870in}}{\pgfqpoint{2.446088in}{1.484734in}}%
\pgfusepath{clip}%
\pgfsetbuttcap%
\pgfsetroundjoin%
\definecolor{currentfill}{rgb}{0.298039,0.447059,0.690196}%
\pgfsetfillcolor{currentfill}%
\pgfsetlinewidth{1.003750pt}%
\definecolor{currentstroke}{rgb}{0.298039,0.447059,0.690196}%
\pgfsetstrokecolor{currentstroke}%
\pgfsetdash{}{0pt}%
\pgfpathmoveto{\pgfqpoint{3.905098in}{1.936025in}}%
\pgfpathcurveto{\pgfqpoint{3.913334in}{1.936025in}}{\pgfqpoint{3.921234in}{1.939298in}}{\pgfqpoint{3.927058in}{1.945122in}}%
\pgfpathcurveto{\pgfqpoint{3.932882in}{1.950946in}}{\pgfqpoint{3.936155in}{1.958846in}}{\pgfqpoint{3.936155in}{1.967082in}}%
\pgfpathcurveto{\pgfqpoint{3.936155in}{1.975318in}}{\pgfqpoint{3.932882in}{1.983218in}}{\pgfqpoint{3.927058in}{1.989042in}}%
\pgfpathcurveto{\pgfqpoint{3.921234in}{1.994866in}}{\pgfqpoint{3.913334in}{1.998138in}}{\pgfqpoint{3.905098in}{1.998138in}}%
\pgfpathcurveto{\pgfqpoint{3.896862in}{1.998138in}}{\pgfqpoint{3.888962in}{1.994866in}}{\pgfqpoint{3.883138in}{1.989042in}}%
\pgfpathcurveto{\pgfqpoint{3.877314in}{1.983218in}}{\pgfqpoint{3.874042in}{1.975318in}}{\pgfqpoint{3.874042in}{1.967082in}}%
\pgfpathcurveto{\pgfqpoint{3.874042in}{1.958846in}}{\pgfqpoint{3.877314in}{1.950946in}}{\pgfqpoint{3.883138in}{1.945122in}}%
\pgfpathcurveto{\pgfqpoint{3.888962in}{1.939298in}}{\pgfqpoint{3.896862in}{1.936025in}}{\pgfqpoint{3.905098in}{1.936025in}}%
\pgfpathclose%
\pgfusepath{stroke,fill}%
\end{pgfscope}%
\begin{pgfscope}%
\pgfpathrectangle{\pgfqpoint{3.793912in}{0.557870in}}{\pgfqpoint{2.446088in}{1.484734in}}%
\pgfusepath{clip}%
\pgfsetbuttcap%
\pgfsetroundjoin%
\definecolor{currentfill}{rgb}{0.298039,0.447059,0.690196}%
\pgfsetfillcolor{currentfill}%
\pgfsetlinewidth{1.003750pt}%
\definecolor{currentstroke}{rgb}{0.298039,0.447059,0.690196}%
\pgfsetstrokecolor{currentstroke}%
\pgfsetdash{}{0pt}%
\pgfpathmoveto{\pgfqpoint{3.905098in}{1.052255in}}%
\pgfpathcurveto{\pgfqpoint{3.913334in}{1.052255in}}{\pgfqpoint{3.921234in}{1.055528in}}{\pgfqpoint{3.927058in}{1.061352in}}%
\pgfpathcurveto{\pgfqpoint{3.932882in}{1.067175in}}{\pgfqpoint{3.936155in}{1.075076in}}{\pgfqpoint{3.936155in}{1.083312in}}%
\pgfpathcurveto{\pgfqpoint{3.936155in}{1.091548in}}{\pgfqpoint{3.932882in}{1.099448in}}{\pgfqpoint{3.927058in}{1.105272in}}%
\pgfpathcurveto{\pgfqpoint{3.921234in}{1.111096in}}{\pgfqpoint{3.913334in}{1.114368in}}{\pgfqpoint{3.905098in}{1.114368in}}%
\pgfpathcurveto{\pgfqpoint{3.896862in}{1.114368in}}{\pgfqpoint{3.888962in}{1.111096in}}{\pgfqpoint{3.883138in}{1.105272in}}%
\pgfpathcurveto{\pgfqpoint{3.877314in}{1.099448in}}{\pgfqpoint{3.874042in}{1.091548in}}{\pgfqpoint{3.874042in}{1.083312in}}%
\pgfpathcurveto{\pgfqpoint{3.874042in}{1.075076in}}{\pgfqpoint{3.877314in}{1.067175in}}{\pgfqpoint{3.883138in}{1.061352in}}%
\pgfpathcurveto{\pgfqpoint{3.888962in}{1.055528in}}{\pgfqpoint{3.896862in}{1.052255in}}{\pgfqpoint{3.905098in}{1.052255in}}%
\pgfpathclose%
\pgfusepath{stroke,fill}%
\end{pgfscope}%
\begin{pgfscope}%
\pgfpathrectangle{\pgfqpoint{3.793912in}{0.557870in}}{\pgfqpoint{2.446088in}{1.484734in}}%
\pgfusepath{clip}%
\pgfsetbuttcap%
\pgfsetroundjoin%
\definecolor{currentfill}{rgb}{0.298039,0.447059,0.690196}%
\pgfsetfillcolor{currentfill}%
\pgfsetlinewidth{1.003750pt}%
\definecolor{currentstroke}{rgb}{0.298039,0.447059,0.690196}%
\pgfsetstrokecolor{currentstroke}%
\pgfsetdash{}{0pt}%
\pgfpathmoveto{\pgfqpoint{3.905098in}{1.936025in}}%
\pgfpathcurveto{\pgfqpoint{3.913334in}{1.936025in}}{\pgfqpoint{3.921234in}{1.939298in}}{\pgfqpoint{3.927058in}{1.945122in}}%
\pgfpathcurveto{\pgfqpoint{3.932882in}{1.950946in}}{\pgfqpoint{3.936155in}{1.958846in}}{\pgfqpoint{3.936155in}{1.967082in}}%
\pgfpathcurveto{\pgfqpoint{3.936155in}{1.975318in}}{\pgfqpoint{3.932882in}{1.983218in}}{\pgfqpoint{3.927058in}{1.989042in}}%
\pgfpathcurveto{\pgfqpoint{3.921234in}{1.994866in}}{\pgfqpoint{3.913334in}{1.998138in}}{\pgfqpoint{3.905098in}{1.998138in}}%
\pgfpathcurveto{\pgfqpoint{3.896862in}{1.998138in}}{\pgfqpoint{3.888962in}{1.994866in}}{\pgfqpoint{3.883138in}{1.989042in}}%
\pgfpathcurveto{\pgfqpoint{3.877314in}{1.983218in}}{\pgfqpoint{3.874042in}{1.975318in}}{\pgfqpoint{3.874042in}{1.967082in}}%
\pgfpathcurveto{\pgfqpoint{3.874042in}{1.958846in}}{\pgfqpoint{3.877314in}{1.950946in}}{\pgfqpoint{3.883138in}{1.945122in}}%
\pgfpathcurveto{\pgfqpoint{3.888962in}{1.939298in}}{\pgfqpoint{3.896862in}{1.936025in}}{\pgfqpoint{3.905098in}{1.936025in}}%
\pgfpathclose%
\pgfusepath{stroke,fill}%
\end{pgfscope}%
\begin{pgfscope}%
\pgfpathrectangle{\pgfqpoint{3.793912in}{0.557870in}}{\pgfqpoint{2.446088in}{1.484734in}}%
\pgfusepath{clip}%
\pgfsetbuttcap%
\pgfsetroundjoin%
\definecolor{currentfill}{rgb}{0.298039,0.447059,0.690196}%
\pgfsetfillcolor{currentfill}%
\pgfsetlinewidth{1.003750pt}%
\definecolor{currentstroke}{rgb}{0.298039,0.447059,0.690196}%
\pgfsetstrokecolor{currentstroke}%
\pgfsetdash{}{0pt}%
\pgfpathmoveto{\pgfqpoint{3.905098in}{1.936025in}}%
\pgfpathcurveto{\pgfqpoint{3.913334in}{1.936025in}}{\pgfqpoint{3.921234in}{1.939298in}}{\pgfqpoint{3.927058in}{1.945122in}}%
\pgfpathcurveto{\pgfqpoint{3.932882in}{1.950946in}}{\pgfqpoint{3.936155in}{1.958846in}}{\pgfqpoint{3.936155in}{1.967082in}}%
\pgfpathcurveto{\pgfqpoint{3.936155in}{1.975318in}}{\pgfqpoint{3.932882in}{1.983218in}}{\pgfqpoint{3.927058in}{1.989042in}}%
\pgfpathcurveto{\pgfqpoint{3.921234in}{1.994866in}}{\pgfqpoint{3.913334in}{1.998138in}}{\pgfqpoint{3.905098in}{1.998138in}}%
\pgfpathcurveto{\pgfqpoint{3.896862in}{1.998138in}}{\pgfqpoint{3.888962in}{1.994866in}}{\pgfqpoint{3.883138in}{1.989042in}}%
\pgfpathcurveto{\pgfqpoint{3.877314in}{1.983218in}}{\pgfqpoint{3.874042in}{1.975318in}}{\pgfqpoint{3.874042in}{1.967082in}}%
\pgfpathcurveto{\pgfqpoint{3.874042in}{1.958846in}}{\pgfqpoint{3.877314in}{1.950946in}}{\pgfqpoint{3.883138in}{1.945122in}}%
\pgfpathcurveto{\pgfqpoint{3.888962in}{1.939298in}}{\pgfqpoint{3.896862in}{1.936025in}}{\pgfqpoint{3.905098in}{1.936025in}}%
\pgfpathclose%
\pgfusepath{stroke,fill}%
\end{pgfscope}%
\begin{pgfscope}%
\pgfpathrectangle{\pgfqpoint{3.793912in}{0.557870in}}{\pgfqpoint{2.446088in}{1.484734in}}%
\pgfusepath{clip}%
\pgfsetbuttcap%
\pgfsetroundjoin%
\definecolor{currentfill}{rgb}{0.298039,0.447059,0.690196}%
\pgfsetfillcolor{currentfill}%
\pgfsetlinewidth{1.003750pt}%
\definecolor{currentstroke}{rgb}{0.298039,0.447059,0.690196}%
\pgfsetstrokecolor{currentstroke}%
\pgfsetdash{}{0pt}%
\pgfpathmoveto{\pgfqpoint{3.905098in}{1.936025in}}%
\pgfpathcurveto{\pgfqpoint{3.913334in}{1.936025in}}{\pgfqpoint{3.921234in}{1.939298in}}{\pgfqpoint{3.927058in}{1.945122in}}%
\pgfpathcurveto{\pgfqpoint{3.932882in}{1.950946in}}{\pgfqpoint{3.936155in}{1.958846in}}{\pgfqpoint{3.936155in}{1.967082in}}%
\pgfpathcurveto{\pgfqpoint{3.936155in}{1.975318in}}{\pgfqpoint{3.932882in}{1.983218in}}{\pgfqpoint{3.927058in}{1.989042in}}%
\pgfpathcurveto{\pgfqpoint{3.921234in}{1.994866in}}{\pgfqpoint{3.913334in}{1.998138in}}{\pgfqpoint{3.905098in}{1.998138in}}%
\pgfpathcurveto{\pgfqpoint{3.896862in}{1.998138in}}{\pgfqpoint{3.888962in}{1.994866in}}{\pgfqpoint{3.883138in}{1.989042in}}%
\pgfpathcurveto{\pgfqpoint{3.877314in}{1.983218in}}{\pgfqpoint{3.874042in}{1.975318in}}{\pgfqpoint{3.874042in}{1.967082in}}%
\pgfpathcurveto{\pgfqpoint{3.874042in}{1.958846in}}{\pgfqpoint{3.877314in}{1.950946in}}{\pgfqpoint{3.883138in}{1.945122in}}%
\pgfpathcurveto{\pgfqpoint{3.888962in}{1.939298in}}{\pgfqpoint{3.896862in}{1.936025in}}{\pgfqpoint{3.905098in}{1.936025in}}%
\pgfpathclose%
\pgfusepath{stroke,fill}%
\end{pgfscope}%
\begin{pgfscope}%
\pgfpathrectangle{\pgfqpoint{3.793912in}{0.557870in}}{\pgfqpoint{2.446088in}{1.484734in}}%
\pgfusepath{clip}%
\pgfsetbuttcap%
\pgfsetroundjoin%
\definecolor{currentfill}{rgb}{0.298039,0.447059,0.690196}%
\pgfsetfillcolor{currentfill}%
\pgfsetlinewidth{1.003750pt}%
\definecolor{currentstroke}{rgb}{0.298039,0.447059,0.690196}%
\pgfsetstrokecolor{currentstroke}%
\pgfsetdash{}{0pt}%
\pgfpathmoveto{\pgfqpoint{3.905098in}{1.084392in}}%
\pgfpathcurveto{\pgfqpoint{3.913334in}{1.084392in}}{\pgfqpoint{3.921234in}{1.087665in}}{\pgfqpoint{3.927058in}{1.093489in}}%
\pgfpathcurveto{\pgfqpoint{3.932882in}{1.099313in}}{\pgfqpoint{3.936155in}{1.107213in}}{\pgfqpoint{3.936155in}{1.115449in}}%
\pgfpathcurveto{\pgfqpoint{3.936155in}{1.123685in}}{\pgfqpoint{3.932882in}{1.131585in}}{\pgfqpoint{3.927058in}{1.137409in}}%
\pgfpathcurveto{\pgfqpoint{3.921234in}{1.143233in}}{\pgfqpoint{3.913334in}{1.146505in}}{\pgfqpoint{3.905098in}{1.146505in}}%
\pgfpathcurveto{\pgfqpoint{3.896862in}{1.146505in}}{\pgfqpoint{3.888962in}{1.143233in}}{\pgfqpoint{3.883138in}{1.137409in}}%
\pgfpathcurveto{\pgfqpoint{3.877314in}{1.131585in}}{\pgfqpoint{3.874042in}{1.123685in}}{\pgfqpoint{3.874042in}{1.115449in}}%
\pgfpathcurveto{\pgfqpoint{3.874042in}{1.107213in}}{\pgfqpoint{3.877314in}{1.099313in}}{\pgfqpoint{3.883138in}{1.093489in}}%
\pgfpathcurveto{\pgfqpoint{3.888962in}{1.087665in}}{\pgfqpoint{3.896862in}{1.084392in}}{\pgfqpoint{3.905098in}{1.084392in}}%
\pgfpathclose%
\pgfusepath{stroke,fill}%
\end{pgfscope}%
\begin{pgfscope}%
\pgfpathrectangle{\pgfqpoint{3.793912in}{0.557870in}}{\pgfqpoint{2.446088in}{1.484734in}}%
\pgfusepath{clip}%
\pgfsetbuttcap%
\pgfsetroundjoin%
\definecolor{currentfill}{rgb}{0.298039,0.447059,0.690196}%
\pgfsetfillcolor{currentfill}%
\pgfsetlinewidth{1.003750pt}%
\definecolor{currentstroke}{rgb}{0.298039,0.447059,0.690196}%
\pgfsetstrokecolor{currentstroke}%
\pgfsetdash{}{0pt}%
\pgfpathmoveto{\pgfqpoint{3.905098in}{1.220975in}}%
\pgfpathcurveto{\pgfqpoint{3.913334in}{1.220975in}}{\pgfqpoint{3.921234in}{1.224247in}}{\pgfqpoint{3.927058in}{1.230071in}}%
\pgfpathcurveto{\pgfqpoint{3.932882in}{1.235895in}}{\pgfqpoint{3.936155in}{1.243795in}}{\pgfqpoint{3.936155in}{1.252032in}}%
\pgfpathcurveto{\pgfqpoint{3.936155in}{1.260268in}}{\pgfqpoint{3.932882in}{1.268168in}}{\pgfqpoint{3.927058in}{1.273992in}}%
\pgfpathcurveto{\pgfqpoint{3.921234in}{1.279816in}}{\pgfqpoint{3.913334in}{1.283088in}}{\pgfqpoint{3.905098in}{1.283088in}}%
\pgfpathcurveto{\pgfqpoint{3.896862in}{1.283088in}}{\pgfqpoint{3.888962in}{1.279816in}}{\pgfqpoint{3.883138in}{1.273992in}}%
\pgfpathcurveto{\pgfqpoint{3.877314in}{1.268168in}}{\pgfqpoint{3.874042in}{1.260268in}}{\pgfqpoint{3.874042in}{1.252032in}}%
\pgfpathcurveto{\pgfqpoint{3.874042in}{1.243795in}}{\pgfqpoint{3.877314in}{1.235895in}}{\pgfqpoint{3.883138in}{1.230071in}}%
\pgfpathcurveto{\pgfqpoint{3.888962in}{1.224247in}}{\pgfqpoint{3.896862in}{1.220975in}}{\pgfqpoint{3.905098in}{1.220975in}}%
\pgfpathclose%
\pgfusepath{stroke,fill}%
\end{pgfscope}%
\begin{pgfscope}%
\pgfpathrectangle{\pgfqpoint{3.793912in}{0.557870in}}{\pgfqpoint{2.446088in}{1.484734in}}%
\pgfusepath{clip}%
\pgfsetbuttcap%
\pgfsetroundjoin%
\definecolor{currentfill}{rgb}{0.298039,0.447059,0.690196}%
\pgfsetfillcolor{currentfill}%
\pgfsetlinewidth{1.003750pt}%
\definecolor{currentstroke}{rgb}{0.298039,0.447059,0.690196}%
\pgfsetstrokecolor{currentstroke}%
\pgfsetdash{}{0pt}%
\pgfpathmoveto{\pgfqpoint{3.905098in}{1.188838in}}%
\pgfpathcurveto{\pgfqpoint{3.913334in}{1.188838in}}{\pgfqpoint{3.921234in}{1.192110in}}{\pgfqpoint{3.927058in}{1.197934in}}%
\pgfpathcurveto{\pgfqpoint{3.932882in}{1.203758in}}{\pgfqpoint{3.936155in}{1.211658in}}{\pgfqpoint{3.936155in}{1.219894in}}%
\pgfpathcurveto{\pgfqpoint{3.936155in}{1.228131in}}{\pgfqpoint{3.932882in}{1.236031in}}{\pgfqpoint{3.927058in}{1.241855in}}%
\pgfpathcurveto{\pgfqpoint{3.921234in}{1.247679in}}{\pgfqpoint{3.913334in}{1.250951in}}{\pgfqpoint{3.905098in}{1.250951in}}%
\pgfpathcurveto{\pgfqpoint{3.896862in}{1.250951in}}{\pgfqpoint{3.888962in}{1.247679in}}{\pgfqpoint{3.883138in}{1.241855in}}%
\pgfpathcurveto{\pgfqpoint{3.877314in}{1.236031in}}{\pgfqpoint{3.874042in}{1.228131in}}{\pgfqpoint{3.874042in}{1.219894in}}%
\pgfpathcurveto{\pgfqpoint{3.874042in}{1.211658in}}{\pgfqpoint{3.877314in}{1.203758in}}{\pgfqpoint{3.883138in}{1.197934in}}%
\pgfpathcurveto{\pgfqpoint{3.888962in}{1.192110in}}{\pgfqpoint{3.896862in}{1.188838in}}{\pgfqpoint{3.905098in}{1.188838in}}%
\pgfpathclose%
\pgfusepath{stroke,fill}%
\end{pgfscope}%
\begin{pgfscope}%
\pgfpathrectangle{\pgfqpoint{3.793912in}{0.557870in}}{\pgfqpoint{2.446088in}{1.484734in}}%
\pgfusepath{clip}%
\pgfsetbuttcap%
\pgfsetroundjoin%
\definecolor{currentfill}{rgb}{0.298039,0.447059,0.690196}%
\pgfsetfillcolor{currentfill}%
\pgfsetlinewidth{1.003750pt}%
\definecolor{currentstroke}{rgb}{0.298039,0.447059,0.690196}%
\pgfsetstrokecolor{currentstroke}%
\pgfsetdash{}{0pt}%
\pgfpathmoveto{\pgfqpoint{3.905098in}{1.140632in}}%
\pgfpathcurveto{\pgfqpoint{3.913334in}{1.140632in}}{\pgfqpoint{3.921234in}{1.143905in}}{\pgfqpoint{3.927058in}{1.149729in}}%
\pgfpathcurveto{\pgfqpoint{3.932882in}{1.155553in}}{\pgfqpoint{3.936155in}{1.163453in}}{\pgfqpoint{3.936155in}{1.171689in}}%
\pgfpathcurveto{\pgfqpoint{3.936155in}{1.179925in}}{\pgfqpoint{3.932882in}{1.187825in}}{\pgfqpoint{3.927058in}{1.193649in}}%
\pgfpathcurveto{\pgfqpoint{3.921234in}{1.199473in}}{\pgfqpoint{3.913334in}{1.202745in}}{\pgfqpoint{3.905098in}{1.202745in}}%
\pgfpathcurveto{\pgfqpoint{3.896862in}{1.202745in}}{\pgfqpoint{3.888962in}{1.199473in}}{\pgfqpoint{3.883138in}{1.193649in}}%
\pgfpathcurveto{\pgfqpoint{3.877314in}{1.187825in}}{\pgfqpoint{3.874042in}{1.179925in}}{\pgfqpoint{3.874042in}{1.171689in}}%
\pgfpathcurveto{\pgfqpoint{3.874042in}{1.163453in}}{\pgfqpoint{3.877314in}{1.155553in}}{\pgfqpoint{3.883138in}{1.149729in}}%
\pgfpathcurveto{\pgfqpoint{3.888962in}{1.143905in}}{\pgfqpoint{3.896862in}{1.140632in}}{\pgfqpoint{3.905098in}{1.140632in}}%
\pgfpathclose%
\pgfusepath{stroke,fill}%
\end{pgfscope}%
\begin{pgfscope}%
\pgfpathrectangle{\pgfqpoint{3.793912in}{0.557870in}}{\pgfqpoint{2.446088in}{1.484734in}}%
\pgfusepath{clip}%
\pgfsetbuttcap%
\pgfsetroundjoin%
\definecolor{currentfill}{rgb}{0.298039,0.447059,0.690196}%
\pgfsetfillcolor{currentfill}%
\pgfsetlinewidth{1.003750pt}%
\definecolor{currentstroke}{rgb}{0.298039,0.447059,0.690196}%
\pgfsetstrokecolor{currentstroke}%
\pgfsetdash{}{0pt}%
\pgfpathmoveto{\pgfqpoint{3.905098in}{0.827296in}}%
\pgfpathcurveto{\pgfqpoint{3.913334in}{0.827296in}}{\pgfqpoint{3.921234in}{0.830568in}}{\pgfqpoint{3.927058in}{0.836392in}}%
\pgfpathcurveto{\pgfqpoint{3.932882in}{0.842216in}}{\pgfqpoint{3.936155in}{0.850116in}}{\pgfqpoint{3.936155in}{0.858352in}}%
\pgfpathcurveto{\pgfqpoint{3.936155in}{0.866588in}}{\pgfqpoint{3.932882in}{0.874488in}}{\pgfqpoint{3.927058in}{0.880312in}}%
\pgfpathcurveto{\pgfqpoint{3.921234in}{0.886136in}}{\pgfqpoint{3.913334in}{0.889409in}}{\pgfqpoint{3.905098in}{0.889409in}}%
\pgfpathcurveto{\pgfqpoint{3.896862in}{0.889409in}}{\pgfqpoint{3.888962in}{0.886136in}}{\pgfqpoint{3.883138in}{0.880312in}}%
\pgfpathcurveto{\pgfqpoint{3.877314in}{0.874488in}}{\pgfqpoint{3.874042in}{0.866588in}}{\pgfqpoint{3.874042in}{0.858352in}}%
\pgfpathcurveto{\pgfqpoint{3.874042in}{0.850116in}}{\pgfqpoint{3.877314in}{0.842216in}}{\pgfqpoint{3.883138in}{0.836392in}}%
\pgfpathcurveto{\pgfqpoint{3.888962in}{0.830568in}}{\pgfqpoint{3.896862in}{0.827296in}}{\pgfqpoint{3.905098in}{0.827296in}}%
\pgfpathclose%
\pgfusepath{stroke,fill}%
\end{pgfscope}%
\begin{pgfscope}%
\pgfpathrectangle{\pgfqpoint{3.793912in}{0.557870in}}{\pgfqpoint{2.446088in}{1.484734in}}%
\pgfusepath{clip}%
\pgfsetbuttcap%
\pgfsetroundjoin%
\definecolor{currentfill}{rgb}{0.298039,0.447059,0.690196}%
\pgfsetfillcolor{currentfill}%
\pgfsetlinewidth{1.003750pt}%
\definecolor{currentstroke}{rgb}{0.298039,0.447059,0.690196}%
\pgfsetstrokecolor{currentstroke}%
\pgfsetdash{}{0pt}%
\pgfpathmoveto{\pgfqpoint{3.905098in}{0.907638in}}%
\pgfpathcurveto{\pgfqpoint{3.913334in}{0.907638in}}{\pgfqpoint{3.921234in}{0.910911in}}{\pgfqpoint{3.927058in}{0.916735in}}%
\pgfpathcurveto{\pgfqpoint{3.932882in}{0.922559in}}{\pgfqpoint{3.936155in}{0.930459in}}{\pgfqpoint{3.936155in}{0.938695in}}%
\pgfpathcurveto{\pgfqpoint{3.936155in}{0.946931in}}{\pgfqpoint{3.932882in}{0.954831in}}{\pgfqpoint{3.927058in}{0.960655in}}%
\pgfpathcurveto{\pgfqpoint{3.921234in}{0.966479in}}{\pgfqpoint{3.913334in}{0.969751in}}{\pgfqpoint{3.905098in}{0.969751in}}%
\pgfpathcurveto{\pgfqpoint{3.896862in}{0.969751in}}{\pgfqpoint{3.888962in}{0.966479in}}{\pgfqpoint{3.883138in}{0.960655in}}%
\pgfpathcurveto{\pgfqpoint{3.877314in}{0.954831in}}{\pgfqpoint{3.874042in}{0.946931in}}{\pgfqpoint{3.874042in}{0.938695in}}%
\pgfpathcurveto{\pgfqpoint{3.874042in}{0.930459in}}{\pgfqpoint{3.877314in}{0.922559in}}{\pgfqpoint{3.883138in}{0.916735in}}%
\pgfpathcurveto{\pgfqpoint{3.888962in}{0.910911in}}{\pgfqpoint{3.896862in}{0.907638in}}{\pgfqpoint{3.905098in}{0.907638in}}%
\pgfpathclose%
\pgfusepath{stroke,fill}%
\end{pgfscope}%
\begin{pgfscope}%
\pgfpathrectangle{\pgfqpoint{3.793912in}{0.557870in}}{\pgfqpoint{2.446088in}{1.484734in}}%
\pgfusepath{clip}%
\pgfsetbuttcap%
\pgfsetroundjoin%
\definecolor{currentfill}{rgb}{0.298039,0.447059,0.690196}%
\pgfsetfillcolor{currentfill}%
\pgfsetlinewidth{1.003750pt}%
\definecolor{currentstroke}{rgb}{0.298039,0.447059,0.690196}%
\pgfsetstrokecolor{currentstroke}%
\pgfsetdash{}{0pt}%
\pgfpathmoveto{\pgfqpoint{3.905098in}{1.936025in}}%
\pgfpathcurveto{\pgfqpoint{3.913334in}{1.936025in}}{\pgfqpoint{3.921234in}{1.939298in}}{\pgfqpoint{3.927058in}{1.945122in}}%
\pgfpathcurveto{\pgfqpoint{3.932882in}{1.950946in}}{\pgfqpoint{3.936155in}{1.958846in}}{\pgfqpoint{3.936155in}{1.967082in}}%
\pgfpathcurveto{\pgfqpoint{3.936155in}{1.975318in}}{\pgfqpoint{3.932882in}{1.983218in}}{\pgfqpoint{3.927058in}{1.989042in}}%
\pgfpathcurveto{\pgfqpoint{3.921234in}{1.994866in}}{\pgfqpoint{3.913334in}{1.998138in}}{\pgfqpoint{3.905098in}{1.998138in}}%
\pgfpathcurveto{\pgfqpoint{3.896862in}{1.998138in}}{\pgfqpoint{3.888962in}{1.994866in}}{\pgfqpoint{3.883138in}{1.989042in}}%
\pgfpathcurveto{\pgfqpoint{3.877314in}{1.983218in}}{\pgfqpoint{3.874042in}{1.975318in}}{\pgfqpoint{3.874042in}{1.967082in}}%
\pgfpathcurveto{\pgfqpoint{3.874042in}{1.958846in}}{\pgfqpoint{3.877314in}{1.950946in}}{\pgfqpoint{3.883138in}{1.945122in}}%
\pgfpathcurveto{\pgfqpoint{3.888962in}{1.939298in}}{\pgfqpoint{3.896862in}{1.936025in}}{\pgfqpoint{3.905098in}{1.936025in}}%
\pgfpathclose%
\pgfusepath{stroke,fill}%
\end{pgfscope}%
\begin{pgfscope}%
\pgfpathrectangle{\pgfqpoint{3.793912in}{0.557870in}}{\pgfqpoint{2.446088in}{1.484734in}}%
\pgfusepath{clip}%
\pgfsetbuttcap%
\pgfsetroundjoin%
\definecolor{currentfill}{rgb}{0.298039,0.447059,0.690196}%
\pgfsetfillcolor{currentfill}%
\pgfsetlinewidth{1.003750pt}%
\definecolor{currentstroke}{rgb}{0.298039,0.447059,0.690196}%
\pgfsetstrokecolor{currentstroke}%
\pgfsetdash{}{0pt}%
\pgfpathmoveto{\pgfqpoint{3.905098in}{0.859433in}}%
\pgfpathcurveto{\pgfqpoint{3.913334in}{0.859433in}}{\pgfqpoint{3.921234in}{0.862705in}}{\pgfqpoint{3.927058in}{0.868529in}}%
\pgfpathcurveto{\pgfqpoint{3.932882in}{0.874353in}}{\pgfqpoint{3.936155in}{0.882253in}}{\pgfqpoint{3.936155in}{0.890489in}}%
\pgfpathcurveto{\pgfqpoint{3.936155in}{0.898726in}}{\pgfqpoint{3.932882in}{0.906626in}}{\pgfqpoint{3.927058in}{0.912449in}}%
\pgfpathcurveto{\pgfqpoint{3.921234in}{0.918273in}}{\pgfqpoint{3.913334in}{0.921546in}}{\pgfqpoint{3.905098in}{0.921546in}}%
\pgfpathcurveto{\pgfqpoint{3.896862in}{0.921546in}}{\pgfqpoint{3.888962in}{0.918273in}}{\pgfqpoint{3.883138in}{0.912449in}}%
\pgfpathcurveto{\pgfqpoint{3.877314in}{0.906626in}}{\pgfqpoint{3.874042in}{0.898726in}}{\pgfqpoint{3.874042in}{0.890489in}}%
\pgfpathcurveto{\pgfqpoint{3.874042in}{0.882253in}}{\pgfqpoint{3.877314in}{0.874353in}}{\pgfqpoint{3.883138in}{0.868529in}}%
\pgfpathcurveto{\pgfqpoint{3.888962in}{0.862705in}}{\pgfqpoint{3.896862in}{0.859433in}}{\pgfqpoint{3.905098in}{0.859433in}}%
\pgfpathclose%
\pgfusepath{stroke,fill}%
\end{pgfscope}%
\begin{pgfscope}%
\pgfpathrectangle{\pgfqpoint{3.793912in}{0.557870in}}{\pgfqpoint{2.446088in}{1.484734in}}%
\pgfusepath{clip}%
\pgfsetbuttcap%
\pgfsetroundjoin%
\definecolor{currentfill}{rgb}{0.298039,0.447059,0.690196}%
\pgfsetfillcolor{currentfill}%
\pgfsetlinewidth{1.003750pt}%
\definecolor{currentstroke}{rgb}{0.298039,0.447059,0.690196}%
\pgfsetstrokecolor{currentstroke}%
\pgfsetdash{}{0pt}%
\pgfpathmoveto{\pgfqpoint{3.905098in}{1.936025in}}%
\pgfpathcurveto{\pgfqpoint{3.913334in}{1.936025in}}{\pgfqpoint{3.921234in}{1.939298in}}{\pgfqpoint{3.927058in}{1.945122in}}%
\pgfpathcurveto{\pgfqpoint{3.932882in}{1.950946in}}{\pgfqpoint{3.936155in}{1.958846in}}{\pgfqpoint{3.936155in}{1.967082in}}%
\pgfpathcurveto{\pgfqpoint{3.936155in}{1.975318in}}{\pgfqpoint{3.932882in}{1.983218in}}{\pgfqpoint{3.927058in}{1.989042in}}%
\pgfpathcurveto{\pgfqpoint{3.921234in}{1.994866in}}{\pgfqpoint{3.913334in}{1.998138in}}{\pgfqpoint{3.905098in}{1.998138in}}%
\pgfpathcurveto{\pgfqpoint{3.896862in}{1.998138in}}{\pgfqpoint{3.888962in}{1.994866in}}{\pgfqpoint{3.883138in}{1.989042in}}%
\pgfpathcurveto{\pgfqpoint{3.877314in}{1.983218in}}{\pgfqpoint{3.874042in}{1.975318in}}{\pgfqpoint{3.874042in}{1.967082in}}%
\pgfpathcurveto{\pgfqpoint{3.874042in}{1.958846in}}{\pgfqpoint{3.877314in}{1.950946in}}{\pgfqpoint{3.883138in}{1.945122in}}%
\pgfpathcurveto{\pgfqpoint{3.888962in}{1.939298in}}{\pgfqpoint{3.896862in}{1.936025in}}{\pgfqpoint{3.905098in}{1.936025in}}%
\pgfpathclose%
\pgfusepath{stroke,fill}%
\end{pgfscope}%
\begin{pgfscope}%
\pgfpathrectangle{\pgfqpoint{3.793912in}{0.557870in}}{\pgfqpoint{2.446088in}{1.484734in}}%
\pgfusepath{clip}%
\pgfsetbuttcap%
\pgfsetroundjoin%
\definecolor{currentfill}{rgb}{0.298039,0.447059,0.690196}%
\pgfsetfillcolor{currentfill}%
\pgfsetlinewidth{1.003750pt}%
\definecolor{currentstroke}{rgb}{0.298039,0.447059,0.690196}%
\pgfsetstrokecolor{currentstroke}%
\pgfsetdash{}{0pt}%
\pgfpathmoveto{\pgfqpoint{3.905098in}{0.883536in}}%
\pgfpathcurveto{\pgfqpoint{3.913334in}{0.883536in}}{\pgfqpoint{3.921234in}{0.886808in}}{\pgfqpoint{3.927058in}{0.892632in}}%
\pgfpathcurveto{\pgfqpoint{3.932882in}{0.898456in}}{\pgfqpoint{3.936155in}{0.906356in}}{\pgfqpoint{3.936155in}{0.914592in}}%
\pgfpathcurveto{\pgfqpoint{3.936155in}{0.922828in}}{\pgfqpoint{3.932882in}{0.930728in}}{\pgfqpoint{3.927058in}{0.936552in}}%
\pgfpathcurveto{\pgfqpoint{3.921234in}{0.942376in}}{\pgfqpoint{3.913334in}{0.945649in}}{\pgfqpoint{3.905098in}{0.945649in}}%
\pgfpathcurveto{\pgfqpoint{3.896862in}{0.945649in}}{\pgfqpoint{3.888962in}{0.942376in}}{\pgfqpoint{3.883138in}{0.936552in}}%
\pgfpathcurveto{\pgfqpoint{3.877314in}{0.930728in}}{\pgfqpoint{3.874042in}{0.922828in}}{\pgfqpoint{3.874042in}{0.914592in}}%
\pgfpathcurveto{\pgfqpoint{3.874042in}{0.906356in}}{\pgfqpoint{3.877314in}{0.898456in}}{\pgfqpoint{3.883138in}{0.892632in}}%
\pgfpathcurveto{\pgfqpoint{3.888962in}{0.886808in}}{\pgfqpoint{3.896862in}{0.883536in}}{\pgfqpoint{3.905098in}{0.883536in}}%
\pgfpathclose%
\pgfusepath{stroke,fill}%
\end{pgfscope}%
\begin{pgfscope}%
\pgfpathrectangle{\pgfqpoint{3.793912in}{0.557870in}}{\pgfqpoint{2.446088in}{1.484734in}}%
\pgfusepath{clip}%
\pgfsetbuttcap%
\pgfsetroundjoin%
\definecolor{currentfill}{rgb}{0.298039,0.447059,0.690196}%
\pgfsetfillcolor{currentfill}%
\pgfsetlinewidth{1.003750pt}%
\definecolor{currentstroke}{rgb}{0.298039,0.447059,0.690196}%
\pgfsetstrokecolor{currentstroke}%
\pgfsetdash{}{0pt}%
\pgfpathmoveto{\pgfqpoint{3.905098in}{1.936025in}}%
\pgfpathcurveto{\pgfqpoint{3.913334in}{1.936025in}}{\pgfqpoint{3.921234in}{1.939298in}}{\pgfqpoint{3.927058in}{1.945122in}}%
\pgfpathcurveto{\pgfqpoint{3.932882in}{1.950946in}}{\pgfqpoint{3.936155in}{1.958846in}}{\pgfqpoint{3.936155in}{1.967082in}}%
\pgfpathcurveto{\pgfqpoint{3.936155in}{1.975318in}}{\pgfqpoint{3.932882in}{1.983218in}}{\pgfqpoint{3.927058in}{1.989042in}}%
\pgfpathcurveto{\pgfqpoint{3.921234in}{1.994866in}}{\pgfqpoint{3.913334in}{1.998138in}}{\pgfqpoint{3.905098in}{1.998138in}}%
\pgfpathcurveto{\pgfqpoint{3.896862in}{1.998138in}}{\pgfqpoint{3.888962in}{1.994866in}}{\pgfqpoint{3.883138in}{1.989042in}}%
\pgfpathcurveto{\pgfqpoint{3.877314in}{1.983218in}}{\pgfqpoint{3.874042in}{1.975318in}}{\pgfqpoint{3.874042in}{1.967082in}}%
\pgfpathcurveto{\pgfqpoint{3.874042in}{1.958846in}}{\pgfqpoint{3.877314in}{1.950946in}}{\pgfqpoint{3.883138in}{1.945122in}}%
\pgfpathcurveto{\pgfqpoint{3.888962in}{1.939298in}}{\pgfqpoint{3.896862in}{1.936025in}}{\pgfqpoint{3.905098in}{1.936025in}}%
\pgfpathclose%
\pgfusepath{stroke,fill}%
\end{pgfscope}%
\begin{pgfscope}%
\pgfpathrectangle{\pgfqpoint{3.793912in}{0.557870in}}{\pgfqpoint{2.446088in}{1.484734in}}%
\pgfusepath{clip}%
\pgfsetbuttcap%
\pgfsetroundjoin%
\definecolor{currentfill}{rgb}{0.298039,0.447059,0.690196}%
\pgfsetfillcolor{currentfill}%
\pgfsetlinewidth{1.003750pt}%
\definecolor{currentstroke}{rgb}{0.298039,0.447059,0.690196}%
\pgfsetstrokecolor{currentstroke}%
\pgfsetdash{}{0pt}%
\pgfpathmoveto{\pgfqpoint{3.905098in}{1.936025in}}%
\pgfpathcurveto{\pgfqpoint{3.913334in}{1.936025in}}{\pgfqpoint{3.921234in}{1.939298in}}{\pgfqpoint{3.927058in}{1.945122in}}%
\pgfpathcurveto{\pgfqpoint{3.932882in}{1.950946in}}{\pgfqpoint{3.936155in}{1.958846in}}{\pgfqpoint{3.936155in}{1.967082in}}%
\pgfpathcurveto{\pgfqpoint{3.936155in}{1.975318in}}{\pgfqpoint{3.932882in}{1.983218in}}{\pgfqpoint{3.927058in}{1.989042in}}%
\pgfpathcurveto{\pgfqpoint{3.921234in}{1.994866in}}{\pgfqpoint{3.913334in}{1.998138in}}{\pgfqpoint{3.905098in}{1.998138in}}%
\pgfpathcurveto{\pgfqpoint{3.896862in}{1.998138in}}{\pgfqpoint{3.888962in}{1.994866in}}{\pgfqpoint{3.883138in}{1.989042in}}%
\pgfpathcurveto{\pgfqpoint{3.877314in}{1.983218in}}{\pgfqpoint{3.874042in}{1.975318in}}{\pgfqpoint{3.874042in}{1.967082in}}%
\pgfpathcurveto{\pgfqpoint{3.874042in}{1.958846in}}{\pgfqpoint{3.877314in}{1.950946in}}{\pgfqpoint{3.883138in}{1.945122in}}%
\pgfpathcurveto{\pgfqpoint{3.888962in}{1.939298in}}{\pgfqpoint{3.896862in}{1.936025in}}{\pgfqpoint{3.905098in}{1.936025in}}%
\pgfpathclose%
\pgfusepath{stroke,fill}%
\end{pgfscope}%
\begin{pgfscope}%
\pgfpathrectangle{\pgfqpoint{3.793912in}{0.557870in}}{\pgfqpoint{2.446088in}{1.484734in}}%
\pgfusepath{clip}%
\pgfsetbuttcap%
\pgfsetroundjoin%
\definecolor{currentfill}{rgb}{0.298039,0.447059,0.690196}%
\pgfsetfillcolor{currentfill}%
\pgfsetlinewidth{1.003750pt}%
\definecolor{currentstroke}{rgb}{0.298039,0.447059,0.690196}%
\pgfsetstrokecolor{currentstroke}%
\pgfsetdash{}{0pt}%
\pgfpathmoveto{\pgfqpoint{3.905098in}{0.843364in}}%
\pgfpathcurveto{\pgfqpoint{3.913334in}{0.843364in}}{\pgfqpoint{3.921234in}{0.846636in}}{\pgfqpoint{3.927058in}{0.852460in}}%
\pgfpathcurveto{\pgfqpoint{3.932882in}{0.858284in}}{\pgfqpoint{3.936155in}{0.866184in}}{\pgfqpoint{3.936155in}{0.874421in}}%
\pgfpathcurveto{\pgfqpoint{3.936155in}{0.882657in}}{\pgfqpoint{3.932882in}{0.890557in}}{\pgfqpoint{3.927058in}{0.896381in}}%
\pgfpathcurveto{\pgfqpoint{3.921234in}{0.902205in}}{\pgfqpoint{3.913334in}{0.905477in}}{\pgfqpoint{3.905098in}{0.905477in}}%
\pgfpathcurveto{\pgfqpoint{3.896862in}{0.905477in}}{\pgfqpoint{3.888962in}{0.902205in}}{\pgfqpoint{3.883138in}{0.896381in}}%
\pgfpathcurveto{\pgfqpoint{3.877314in}{0.890557in}}{\pgfqpoint{3.874042in}{0.882657in}}{\pgfqpoint{3.874042in}{0.874421in}}%
\pgfpathcurveto{\pgfqpoint{3.874042in}{0.866184in}}{\pgfqpoint{3.877314in}{0.858284in}}{\pgfqpoint{3.883138in}{0.852460in}}%
\pgfpathcurveto{\pgfqpoint{3.888962in}{0.846636in}}{\pgfqpoint{3.896862in}{0.843364in}}{\pgfqpoint{3.905098in}{0.843364in}}%
\pgfpathclose%
\pgfusepath{stroke,fill}%
\end{pgfscope}%
\begin{pgfscope}%
\pgfpathrectangle{\pgfqpoint{3.793912in}{0.557870in}}{\pgfqpoint{2.446088in}{1.484734in}}%
\pgfusepath{clip}%
\pgfsetbuttcap%
\pgfsetroundjoin%
\definecolor{currentfill}{rgb}{0.298039,0.447059,0.690196}%
\pgfsetfillcolor{currentfill}%
\pgfsetlinewidth{1.003750pt}%
\definecolor{currentstroke}{rgb}{0.298039,0.447059,0.690196}%
\pgfsetstrokecolor{currentstroke}%
\pgfsetdash{}{0pt}%
\pgfpathmoveto{\pgfqpoint{3.905098in}{1.936025in}}%
\pgfpathcurveto{\pgfqpoint{3.913334in}{1.936025in}}{\pgfqpoint{3.921234in}{1.939298in}}{\pgfqpoint{3.927058in}{1.945122in}}%
\pgfpathcurveto{\pgfqpoint{3.932882in}{1.950946in}}{\pgfqpoint{3.936155in}{1.958846in}}{\pgfqpoint{3.936155in}{1.967082in}}%
\pgfpathcurveto{\pgfqpoint{3.936155in}{1.975318in}}{\pgfqpoint{3.932882in}{1.983218in}}{\pgfqpoint{3.927058in}{1.989042in}}%
\pgfpathcurveto{\pgfqpoint{3.921234in}{1.994866in}}{\pgfqpoint{3.913334in}{1.998138in}}{\pgfqpoint{3.905098in}{1.998138in}}%
\pgfpathcurveto{\pgfqpoint{3.896862in}{1.998138in}}{\pgfqpoint{3.888962in}{1.994866in}}{\pgfqpoint{3.883138in}{1.989042in}}%
\pgfpathcurveto{\pgfqpoint{3.877314in}{1.983218in}}{\pgfqpoint{3.874042in}{1.975318in}}{\pgfqpoint{3.874042in}{1.967082in}}%
\pgfpathcurveto{\pgfqpoint{3.874042in}{1.958846in}}{\pgfqpoint{3.877314in}{1.950946in}}{\pgfqpoint{3.883138in}{1.945122in}}%
\pgfpathcurveto{\pgfqpoint{3.888962in}{1.939298in}}{\pgfqpoint{3.896862in}{1.936025in}}{\pgfqpoint{3.905098in}{1.936025in}}%
\pgfpathclose%
\pgfusepath{stroke,fill}%
\end{pgfscope}%
\begin{pgfscope}%
\pgfpathrectangle{\pgfqpoint{3.793912in}{0.557870in}}{\pgfqpoint{2.446088in}{1.484734in}}%
\pgfusepath{clip}%
\pgfsetbuttcap%
\pgfsetroundjoin%
\definecolor{currentfill}{rgb}{0.298039,0.447059,0.690196}%
\pgfsetfillcolor{currentfill}%
\pgfsetlinewidth{1.003750pt}%
\definecolor{currentstroke}{rgb}{0.298039,0.447059,0.690196}%
\pgfsetstrokecolor{currentstroke}%
\pgfsetdash{}{0pt}%
\pgfpathmoveto{\pgfqpoint{3.905098in}{1.397729in}}%
\pgfpathcurveto{\pgfqpoint{3.913334in}{1.397729in}}{\pgfqpoint{3.921234in}{1.401001in}}{\pgfqpoint{3.927058in}{1.406825in}}%
\pgfpathcurveto{\pgfqpoint{3.932882in}{1.412649in}}{\pgfqpoint{3.936155in}{1.420549in}}{\pgfqpoint{3.936155in}{1.428786in}}%
\pgfpathcurveto{\pgfqpoint{3.936155in}{1.437022in}}{\pgfqpoint{3.932882in}{1.444922in}}{\pgfqpoint{3.927058in}{1.450746in}}%
\pgfpathcurveto{\pgfqpoint{3.921234in}{1.456570in}}{\pgfqpoint{3.913334in}{1.459842in}}{\pgfqpoint{3.905098in}{1.459842in}}%
\pgfpathcurveto{\pgfqpoint{3.896862in}{1.459842in}}{\pgfqpoint{3.888962in}{1.456570in}}{\pgfqpoint{3.883138in}{1.450746in}}%
\pgfpathcurveto{\pgfqpoint{3.877314in}{1.444922in}}{\pgfqpoint{3.874042in}{1.437022in}}{\pgfqpoint{3.874042in}{1.428786in}}%
\pgfpathcurveto{\pgfqpoint{3.874042in}{1.420549in}}{\pgfqpoint{3.877314in}{1.412649in}}{\pgfqpoint{3.883138in}{1.406825in}}%
\pgfpathcurveto{\pgfqpoint{3.888962in}{1.401001in}}{\pgfqpoint{3.896862in}{1.397729in}}{\pgfqpoint{3.905098in}{1.397729in}}%
\pgfpathclose%
\pgfusepath{stroke,fill}%
\end{pgfscope}%
\begin{pgfscope}%
\pgfpathrectangle{\pgfqpoint{3.793912in}{0.557870in}}{\pgfqpoint{2.446088in}{1.484734in}}%
\pgfusepath{clip}%
\pgfsetbuttcap%
\pgfsetroundjoin%
\definecolor{currentfill}{rgb}{0.298039,0.447059,0.690196}%
\pgfsetfillcolor{currentfill}%
\pgfsetlinewidth{1.003750pt}%
\definecolor{currentstroke}{rgb}{0.298039,0.447059,0.690196}%
\pgfsetstrokecolor{currentstroke}%
\pgfsetdash{}{0pt}%
\pgfpathmoveto{\pgfqpoint{3.905098in}{0.610370in}}%
\pgfpathcurveto{\pgfqpoint{3.913334in}{0.610370in}}{\pgfqpoint{3.921234in}{0.613643in}}{\pgfqpoint{3.927058in}{0.619466in}}%
\pgfpathcurveto{\pgfqpoint{3.932882in}{0.625290in}}{\pgfqpoint{3.936155in}{0.633190in}}{\pgfqpoint{3.936155in}{0.641427in}}%
\pgfpathcurveto{\pgfqpoint{3.936155in}{0.649663in}}{\pgfqpoint{3.932882in}{0.657563in}}{\pgfqpoint{3.927058in}{0.663387in}}%
\pgfpathcurveto{\pgfqpoint{3.921234in}{0.669211in}}{\pgfqpoint{3.913334in}{0.672483in}}{\pgfqpoint{3.905098in}{0.672483in}}%
\pgfpathcurveto{\pgfqpoint{3.896862in}{0.672483in}}{\pgfqpoint{3.888962in}{0.669211in}}{\pgfqpoint{3.883138in}{0.663387in}}%
\pgfpathcurveto{\pgfqpoint{3.877314in}{0.657563in}}{\pgfqpoint{3.874042in}{0.649663in}}{\pgfqpoint{3.874042in}{0.641427in}}%
\pgfpathcurveto{\pgfqpoint{3.874042in}{0.633190in}}{\pgfqpoint{3.877314in}{0.625290in}}{\pgfqpoint{3.883138in}{0.619466in}}%
\pgfpathcurveto{\pgfqpoint{3.888962in}{0.613643in}}{\pgfqpoint{3.896862in}{0.610370in}}{\pgfqpoint{3.905098in}{0.610370in}}%
\pgfpathclose%
\pgfusepath{stroke,fill}%
\end{pgfscope}%
\begin{pgfscope}%
\pgfpathrectangle{\pgfqpoint{3.793912in}{0.557870in}}{\pgfqpoint{2.446088in}{1.484734in}}%
\pgfusepath{clip}%
\pgfsetbuttcap%
\pgfsetroundjoin%
\definecolor{currentfill}{rgb}{0.298039,0.447059,0.690196}%
\pgfsetfillcolor{currentfill}%
\pgfsetlinewidth{1.003750pt}%
\definecolor{currentstroke}{rgb}{0.298039,0.447059,0.690196}%
\pgfsetstrokecolor{currentstroke}%
\pgfsetdash{}{0pt}%
\pgfpathmoveto{\pgfqpoint{3.905098in}{1.421832in}}%
\pgfpathcurveto{\pgfqpoint{3.913334in}{1.421832in}}{\pgfqpoint{3.921234in}{1.425104in}}{\pgfqpoint{3.927058in}{1.430928in}}%
\pgfpathcurveto{\pgfqpoint{3.932882in}{1.436752in}}{\pgfqpoint{3.936155in}{1.444652in}}{\pgfqpoint{3.936155in}{1.452888in}}%
\pgfpathcurveto{\pgfqpoint{3.936155in}{1.461125in}}{\pgfqpoint{3.932882in}{1.469025in}}{\pgfqpoint{3.927058in}{1.474849in}}%
\pgfpathcurveto{\pgfqpoint{3.921234in}{1.480673in}}{\pgfqpoint{3.913334in}{1.483945in}}{\pgfqpoint{3.905098in}{1.483945in}}%
\pgfpathcurveto{\pgfqpoint{3.896862in}{1.483945in}}{\pgfqpoint{3.888962in}{1.480673in}}{\pgfqpoint{3.883138in}{1.474849in}}%
\pgfpathcurveto{\pgfqpoint{3.877314in}{1.469025in}}{\pgfqpoint{3.874042in}{1.461125in}}{\pgfqpoint{3.874042in}{1.452888in}}%
\pgfpathcurveto{\pgfqpoint{3.874042in}{1.444652in}}{\pgfqpoint{3.877314in}{1.436752in}}{\pgfqpoint{3.883138in}{1.430928in}}%
\pgfpathcurveto{\pgfqpoint{3.888962in}{1.425104in}}{\pgfqpoint{3.896862in}{1.421832in}}{\pgfqpoint{3.905098in}{1.421832in}}%
\pgfpathclose%
\pgfusepath{stroke,fill}%
\end{pgfscope}%
\begin{pgfscope}%
\pgfpathrectangle{\pgfqpoint{3.793912in}{0.557870in}}{\pgfqpoint{2.446088in}{1.484734in}}%
\pgfusepath{clip}%
\pgfsetbuttcap%
\pgfsetroundjoin%
\definecolor{currentfill}{rgb}{0.298039,0.447059,0.690196}%
\pgfsetfillcolor{currentfill}%
\pgfsetlinewidth{1.003750pt}%
\definecolor{currentstroke}{rgb}{0.298039,0.447059,0.690196}%
\pgfsetstrokecolor{currentstroke}%
\pgfsetdash{}{0pt}%
\pgfpathmoveto{\pgfqpoint{3.905098in}{1.936025in}}%
\pgfpathcurveto{\pgfqpoint{3.913334in}{1.936025in}}{\pgfqpoint{3.921234in}{1.939298in}}{\pgfqpoint{3.927058in}{1.945122in}}%
\pgfpathcurveto{\pgfqpoint{3.932882in}{1.950946in}}{\pgfqpoint{3.936155in}{1.958846in}}{\pgfqpoint{3.936155in}{1.967082in}}%
\pgfpathcurveto{\pgfqpoint{3.936155in}{1.975318in}}{\pgfqpoint{3.932882in}{1.983218in}}{\pgfqpoint{3.927058in}{1.989042in}}%
\pgfpathcurveto{\pgfqpoint{3.921234in}{1.994866in}}{\pgfqpoint{3.913334in}{1.998138in}}{\pgfqpoint{3.905098in}{1.998138in}}%
\pgfpathcurveto{\pgfqpoint{3.896862in}{1.998138in}}{\pgfqpoint{3.888962in}{1.994866in}}{\pgfqpoint{3.883138in}{1.989042in}}%
\pgfpathcurveto{\pgfqpoint{3.877314in}{1.983218in}}{\pgfqpoint{3.874042in}{1.975318in}}{\pgfqpoint{3.874042in}{1.967082in}}%
\pgfpathcurveto{\pgfqpoint{3.874042in}{1.958846in}}{\pgfqpoint{3.877314in}{1.950946in}}{\pgfqpoint{3.883138in}{1.945122in}}%
\pgfpathcurveto{\pgfqpoint{3.888962in}{1.939298in}}{\pgfqpoint{3.896862in}{1.936025in}}{\pgfqpoint{3.905098in}{1.936025in}}%
\pgfpathclose%
\pgfusepath{stroke,fill}%
\end{pgfscope}%
\begin{pgfscope}%
\pgfpathrectangle{\pgfqpoint{3.793912in}{0.557870in}}{\pgfqpoint{2.446088in}{1.484734in}}%
\pgfusepath{clip}%
\pgfsetbuttcap%
\pgfsetroundjoin%
\definecolor{currentfill}{rgb}{0.298039,0.447059,0.690196}%
\pgfsetfillcolor{currentfill}%
\pgfsetlinewidth{1.003750pt}%
\definecolor{currentstroke}{rgb}{0.298039,0.447059,0.690196}%
\pgfsetstrokecolor{currentstroke}%
\pgfsetdash{}{0pt}%
\pgfpathmoveto{\pgfqpoint{3.905098in}{1.936025in}}%
\pgfpathcurveto{\pgfqpoint{3.913334in}{1.936025in}}{\pgfqpoint{3.921234in}{1.939298in}}{\pgfqpoint{3.927058in}{1.945122in}}%
\pgfpathcurveto{\pgfqpoint{3.932882in}{1.950946in}}{\pgfqpoint{3.936155in}{1.958846in}}{\pgfqpoint{3.936155in}{1.967082in}}%
\pgfpathcurveto{\pgfqpoint{3.936155in}{1.975318in}}{\pgfqpoint{3.932882in}{1.983218in}}{\pgfqpoint{3.927058in}{1.989042in}}%
\pgfpathcurveto{\pgfqpoint{3.921234in}{1.994866in}}{\pgfqpoint{3.913334in}{1.998138in}}{\pgfqpoint{3.905098in}{1.998138in}}%
\pgfpathcurveto{\pgfqpoint{3.896862in}{1.998138in}}{\pgfqpoint{3.888962in}{1.994866in}}{\pgfqpoint{3.883138in}{1.989042in}}%
\pgfpathcurveto{\pgfqpoint{3.877314in}{1.983218in}}{\pgfqpoint{3.874042in}{1.975318in}}{\pgfqpoint{3.874042in}{1.967082in}}%
\pgfpathcurveto{\pgfqpoint{3.874042in}{1.958846in}}{\pgfqpoint{3.877314in}{1.950946in}}{\pgfqpoint{3.883138in}{1.945122in}}%
\pgfpathcurveto{\pgfqpoint{3.888962in}{1.939298in}}{\pgfqpoint{3.896862in}{1.936025in}}{\pgfqpoint{3.905098in}{1.936025in}}%
\pgfpathclose%
\pgfusepath{stroke,fill}%
\end{pgfscope}%
\begin{pgfscope}%
\pgfpathrectangle{\pgfqpoint{3.793912in}{0.557870in}}{\pgfqpoint{2.446088in}{1.484734in}}%
\pgfusepath{clip}%
\pgfsetbuttcap%
\pgfsetroundjoin%
\definecolor{currentfill}{rgb}{0.298039,0.447059,0.690196}%
\pgfsetfillcolor{currentfill}%
\pgfsetlinewidth{1.003750pt}%
\definecolor{currentstroke}{rgb}{0.298039,0.447059,0.690196}%
\pgfsetstrokecolor{currentstroke}%
\pgfsetdash{}{0pt}%
\pgfpathmoveto{\pgfqpoint{3.905098in}{1.936025in}}%
\pgfpathcurveto{\pgfqpoint{3.913334in}{1.936025in}}{\pgfqpoint{3.921234in}{1.939298in}}{\pgfqpoint{3.927058in}{1.945122in}}%
\pgfpathcurveto{\pgfqpoint{3.932882in}{1.950946in}}{\pgfqpoint{3.936155in}{1.958846in}}{\pgfqpoint{3.936155in}{1.967082in}}%
\pgfpathcurveto{\pgfqpoint{3.936155in}{1.975318in}}{\pgfqpoint{3.932882in}{1.983218in}}{\pgfqpoint{3.927058in}{1.989042in}}%
\pgfpathcurveto{\pgfqpoint{3.921234in}{1.994866in}}{\pgfqpoint{3.913334in}{1.998138in}}{\pgfqpoint{3.905098in}{1.998138in}}%
\pgfpathcurveto{\pgfqpoint{3.896862in}{1.998138in}}{\pgfqpoint{3.888962in}{1.994866in}}{\pgfqpoint{3.883138in}{1.989042in}}%
\pgfpathcurveto{\pgfqpoint{3.877314in}{1.983218in}}{\pgfqpoint{3.874042in}{1.975318in}}{\pgfqpoint{3.874042in}{1.967082in}}%
\pgfpathcurveto{\pgfqpoint{3.874042in}{1.958846in}}{\pgfqpoint{3.877314in}{1.950946in}}{\pgfqpoint{3.883138in}{1.945122in}}%
\pgfpathcurveto{\pgfqpoint{3.888962in}{1.939298in}}{\pgfqpoint{3.896862in}{1.936025in}}{\pgfqpoint{3.905098in}{1.936025in}}%
\pgfpathclose%
\pgfusepath{stroke,fill}%
\end{pgfscope}%
\begin{pgfscope}%
\pgfpathrectangle{\pgfqpoint{3.793912in}{0.557870in}}{\pgfqpoint{2.446088in}{1.484734in}}%
\pgfusepath{clip}%
\pgfsetbuttcap%
\pgfsetroundjoin%
\definecolor{currentfill}{rgb}{0.298039,0.447059,0.690196}%
\pgfsetfillcolor{currentfill}%
\pgfsetlinewidth{1.003750pt}%
\definecolor{currentstroke}{rgb}{0.298039,0.447059,0.690196}%
\pgfsetstrokecolor{currentstroke}%
\pgfsetdash{}{0pt}%
\pgfpathmoveto{\pgfqpoint{3.905098in}{1.004050in}}%
\pgfpathcurveto{\pgfqpoint{3.913334in}{1.004050in}}{\pgfqpoint{3.921234in}{1.007322in}}{\pgfqpoint{3.927058in}{1.013146in}}%
\pgfpathcurveto{\pgfqpoint{3.932882in}{1.018970in}}{\pgfqpoint{3.936155in}{1.026870in}}{\pgfqpoint{3.936155in}{1.035106in}}%
\pgfpathcurveto{\pgfqpoint{3.936155in}{1.043342in}}{\pgfqpoint{3.932882in}{1.051243in}}{\pgfqpoint{3.927058in}{1.057066in}}%
\pgfpathcurveto{\pgfqpoint{3.921234in}{1.062890in}}{\pgfqpoint{3.913334in}{1.066163in}}{\pgfqpoint{3.905098in}{1.066163in}}%
\pgfpathcurveto{\pgfqpoint{3.896862in}{1.066163in}}{\pgfqpoint{3.888962in}{1.062890in}}{\pgfqpoint{3.883138in}{1.057066in}}%
\pgfpathcurveto{\pgfqpoint{3.877314in}{1.051243in}}{\pgfqpoint{3.874042in}{1.043342in}}{\pgfqpoint{3.874042in}{1.035106in}}%
\pgfpathcurveto{\pgfqpoint{3.874042in}{1.026870in}}{\pgfqpoint{3.877314in}{1.018970in}}{\pgfqpoint{3.883138in}{1.013146in}}%
\pgfpathcurveto{\pgfqpoint{3.888962in}{1.007322in}}{\pgfqpoint{3.896862in}{1.004050in}}{\pgfqpoint{3.905098in}{1.004050in}}%
\pgfpathclose%
\pgfusepath{stroke,fill}%
\end{pgfscope}%
\begin{pgfscope}%
\pgfpathrectangle{\pgfqpoint{3.793912in}{0.557870in}}{\pgfqpoint{2.446088in}{1.484734in}}%
\pgfusepath{clip}%
\pgfsetbuttcap%
\pgfsetroundjoin%
\definecolor{currentfill}{rgb}{0.298039,0.447059,0.690196}%
\pgfsetfillcolor{currentfill}%
\pgfsetlinewidth{1.003750pt}%
\definecolor{currentstroke}{rgb}{0.298039,0.447059,0.690196}%
\pgfsetstrokecolor{currentstroke}%
\pgfsetdash{}{0pt}%
\pgfpathmoveto{\pgfqpoint{3.905098in}{0.996015in}}%
\pgfpathcurveto{\pgfqpoint{3.913334in}{0.996015in}}{\pgfqpoint{3.921234in}{0.999288in}}{\pgfqpoint{3.927058in}{1.005112in}}%
\pgfpathcurveto{\pgfqpoint{3.932882in}{1.010936in}}{\pgfqpoint{3.936155in}{1.018836in}}{\pgfqpoint{3.936155in}{1.027072in}}%
\pgfpathcurveto{\pgfqpoint{3.936155in}{1.035308in}}{\pgfqpoint{3.932882in}{1.043208in}}{\pgfqpoint{3.927058in}{1.049032in}}%
\pgfpathcurveto{\pgfqpoint{3.921234in}{1.054856in}}{\pgfqpoint{3.913334in}{1.058128in}}{\pgfqpoint{3.905098in}{1.058128in}}%
\pgfpathcurveto{\pgfqpoint{3.896862in}{1.058128in}}{\pgfqpoint{3.888962in}{1.054856in}}{\pgfqpoint{3.883138in}{1.049032in}}%
\pgfpathcurveto{\pgfqpoint{3.877314in}{1.043208in}}{\pgfqpoint{3.874042in}{1.035308in}}{\pgfqpoint{3.874042in}{1.027072in}}%
\pgfpathcurveto{\pgfqpoint{3.874042in}{1.018836in}}{\pgfqpoint{3.877314in}{1.010936in}}{\pgfqpoint{3.883138in}{1.005112in}}%
\pgfpathcurveto{\pgfqpoint{3.888962in}{0.999288in}}{\pgfqpoint{3.896862in}{0.996015in}}{\pgfqpoint{3.905098in}{0.996015in}}%
\pgfpathclose%
\pgfusepath{stroke,fill}%
\end{pgfscope}%
\begin{pgfscope}%
\pgfpathrectangle{\pgfqpoint{3.793912in}{0.557870in}}{\pgfqpoint{2.446088in}{1.484734in}}%
\pgfusepath{clip}%
\pgfsetbuttcap%
\pgfsetroundjoin%
\definecolor{currentfill}{rgb}{0.298039,0.447059,0.690196}%
\pgfsetfillcolor{currentfill}%
\pgfsetlinewidth{1.003750pt}%
\definecolor{currentstroke}{rgb}{0.298039,0.447059,0.690196}%
\pgfsetstrokecolor{currentstroke}%
\pgfsetdash{}{0pt}%
\pgfpathmoveto{\pgfqpoint{3.905098in}{1.172769in}}%
\pgfpathcurveto{\pgfqpoint{3.913334in}{1.172769in}}{\pgfqpoint{3.921234in}{1.176042in}}{\pgfqpoint{3.927058in}{1.181866in}}%
\pgfpathcurveto{\pgfqpoint{3.932882in}{1.187690in}}{\pgfqpoint{3.936155in}{1.195590in}}{\pgfqpoint{3.936155in}{1.203826in}}%
\pgfpathcurveto{\pgfqpoint{3.936155in}{1.212062in}}{\pgfqpoint{3.932882in}{1.219962in}}{\pgfqpoint{3.927058in}{1.225786in}}%
\pgfpathcurveto{\pgfqpoint{3.921234in}{1.231610in}}{\pgfqpoint{3.913334in}{1.234882in}}{\pgfqpoint{3.905098in}{1.234882in}}%
\pgfpathcurveto{\pgfqpoint{3.896862in}{1.234882in}}{\pgfqpoint{3.888962in}{1.231610in}}{\pgfqpoint{3.883138in}{1.225786in}}%
\pgfpathcurveto{\pgfqpoint{3.877314in}{1.219962in}}{\pgfqpoint{3.874042in}{1.212062in}}{\pgfqpoint{3.874042in}{1.203826in}}%
\pgfpathcurveto{\pgfqpoint{3.874042in}{1.195590in}}{\pgfqpoint{3.877314in}{1.187690in}}{\pgfqpoint{3.883138in}{1.181866in}}%
\pgfpathcurveto{\pgfqpoint{3.888962in}{1.176042in}}{\pgfqpoint{3.896862in}{1.172769in}}{\pgfqpoint{3.905098in}{1.172769in}}%
\pgfpathclose%
\pgfusepath{stroke,fill}%
\end{pgfscope}%
\begin{pgfscope}%
\pgfpathrectangle{\pgfqpoint{3.793912in}{0.557870in}}{\pgfqpoint{2.446088in}{1.484734in}}%
\pgfusepath{clip}%
\pgfsetbuttcap%
\pgfsetroundjoin%
\definecolor{currentfill}{rgb}{0.298039,0.447059,0.690196}%
\pgfsetfillcolor{currentfill}%
\pgfsetlinewidth{1.003750pt}%
\definecolor{currentstroke}{rgb}{0.298039,0.447059,0.690196}%
\pgfsetstrokecolor{currentstroke}%
\pgfsetdash{}{0pt}%
\pgfpathmoveto{\pgfqpoint{3.905098in}{1.212941in}}%
\pgfpathcurveto{\pgfqpoint{3.913334in}{1.212941in}}{\pgfqpoint{3.921234in}{1.216213in}}{\pgfqpoint{3.927058in}{1.222037in}}%
\pgfpathcurveto{\pgfqpoint{3.932882in}{1.227861in}}{\pgfqpoint{3.936155in}{1.235761in}}{\pgfqpoint{3.936155in}{1.243997in}}%
\pgfpathcurveto{\pgfqpoint{3.936155in}{1.252234in}}{\pgfqpoint{3.932882in}{1.260134in}}{\pgfqpoint{3.927058in}{1.265958in}}%
\pgfpathcurveto{\pgfqpoint{3.921234in}{1.271781in}}{\pgfqpoint{3.913334in}{1.275054in}}{\pgfqpoint{3.905098in}{1.275054in}}%
\pgfpathcurveto{\pgfqpoint{3.896862in}{1.275054in}}{\pgfqpoint{3.888962in}{1.271781in}}{\pgfqpoint{3.883138in}{1.265958in}}%
\pgfpathcurveto{\pgfqpoint{3.877314in}{1.260134in}}{\pgfqpoint{3.874042in}{1.252234in}}{\pgfqpoint{3.874042in}{1.243997in}}%
\pgfpathcurveto{\pgfqpoint{3.874042in}{1.235761in}}{\pgfqpoint{3.877314in}{1.227861in}}{\pgfqpoint{3.883138in}{1.222037in}}%
\pgfpathcurveto{\pgfqpoint{3.888962in}{1.216213in}}{\pgfqpoint{3.896862in}{1.212941in}}{\pgfqpoint{3.905098in}{1.212941in}}%
\pgfpathclose%
\pgfusepath{stroke,fill}%
\end{pgfscope}%
\begin{pgfscope}%
\pgfpathrectangle{\pgfqpoint{3.793912in}{0.557870in}}{\pgfqpoint{2.446088in}{1.484734in}}%
\pgfusepath{clip}%
\pgfsetbuttcap%
\pgfsetroundjoin%
\definecolor{currentfill}{rgb}{0.298039,0.447059,0.690196}%
\pgfsetfillcolor{currentfill}%
\pgfsetlinewidth{1.003750pt}%
\definecolor{currentstroke}{rgb}{0.298039,0.447059,0.690196}%
\pgfsetstrokecolor{currentstroke}%
\pgfsetdash{}{0pt}%
\pgfpathmoveto{\pgfqpoint{3.905098in}{1.936025in}}%
\pgfpathcurveto{\pgfqpoint{3.913334in}{1.936025in}}{\pgfqpoint{3.921234in}{1.939298in}}{\pgfqpoint{3.927058in}{1.945122in}}%
\pgfpathcurveto{\pgfqpoint{3.932882in}{1.950946in}}{\pgfqpoint{3.936155in}{1.958846in}}{\pgfqpoint{3.936155in}{1.967082in}}%
\pgfpathcurveto{\pgfqpoint{3.936155in}{1.975318in}}{\pgfqpoint{3.932882in}{1.983218in}}{\pgfqpoint{3.927058in}{1.989042in}}%
\pgfpathcurveto{\pgfqpoint{3.921234in}{1.994866in}}{\pgfqpoint{3.913334in}{1.998138in}}{\pgfqpoint{3.905098in}{1.998138in}}%
\pgfpathcurveto{\pgfqpoint{3.896862in}{1.998138in}}{\pgfqpoint{3.888962in}{1.994866in}}{\pgfqpoint{3.883138in}{1.989042in}}%
\pgfpathcurveto{\pgfqpoint{3.877314in}{1.983218in}}{\pgfqpoint{3.874042in}{1.975318in}}{\pgfqpoint{3.874042in}{1.967082in}}%
\pgfpathcurveto{\pgfqpoint{3.874042in}{1.958846in}}{\pgfqpoint{3.877314in}{1.950946in}}{\pgfqpoint{3.883138in}{1.945122in}}%
\pgfpathcurveto{\pgfqpoint{3.888962in}{1.939298in}}{\pgfqpoint{3.896862in}{1.936025in}}{\pgfqpoint{3.905098in}{1.936025in}}%
\pgfpathclose%
\pgfusepath{stroke,fill}%
\end{pgfscope}%
\begin{pgfscope}%
\pgfpathrectangle{\pgfqpoint{3.793912in}{0.557870in}}{\pgfqpoint{2.446088in}{1.484734in}}%
\pgfusepath{clip}%
\pgfsetbuttcap%
\pgfsetroundjoin%
\definecolor{currentfill}{rgb}{0.298039,0.447059,0.690196}%
\pgfsetfillcolor{currentfill}%
\pgfsetlinewidth{1.003750pt}%
\definecolor{currentstroke}{rgb}{0.298039,0.447059,0.690196}%
\pgfsetstrokecolor{currentstroke}%
\pgfsetdash{}{0pt}%
\pgfpathmoveto{\pgfqpoint{3.905098in}{0.843364in}}%
\pgfpathcurveto{\pgfqpoint{3.913334in}{0.843364in}}{\pgfqpoint{3.921234in}{0.846636in}}{\pgfqpoint{3.927058in}{0.852460in}}%
\pgfpathcurveto{\pgfqpoint{3.932882in}{0.858284in}}{\pgfqpoint{3.936155in}{0.866184in}}{\pgfqpoint{3.936155in}{0.874421in}}%
\pgfpathcurveto{\pgfqpoint{3.936155in}{0.882657in}}{\pgfqpoint{3.932882in}{0.890557in}}{\pgfqpoint{3.927058in}{0.896381in}}%
\pgfpathcurveto{\pgfqpoint{3.921234in}{0.902205in}}{\pgfqpoint{3.913334in}{0.905477in}}{\pgfqpoint{3.905098in}{0.905477in}}%
\pgfpathcurveto{\pgfqpoint{3.896862in}{0.905477in}}{\pgfqpoint{3.888962in}{0.902205in}}{\pgfqpoint{3.883138in}{0.896381in}}%
\pgfpathcurveto{\pgfqpoint{3.877314in}{0.890557in}}{\pgfqpoint{3.874042in}{0.882657in}}{\pgfqpoint{3.874042in}{0.874421in}}%
\pgfpathcurveto{\pgfqpoint{3.874042in}{0.866184in}}{\pgfqpoint{3.877314in}{0.858284in}}{\pgfqpoint{3.883138in}{0.852460in}}%
\pgfpathcurveto{\pgfqpoint{3.888962in}{0.846636in}}{\pgfqpoint{3.896862in}{0.843364in}}{\pgfqpoint{3.905098in}{0.843364in}}%
\pgfpathclose%
\pgfusepath{stroke,fill}%
\end{pgfscope}%
\begin{pgfscope}%
\pgfsetrectcap%
\pgfsetmiterjoin%
\pgfsetlinewidth{1.254687pt}%
\definecolor{currentstroke}{rgb}{1.000000,1.000000,1.000000}%
\pgfsetstrokecolor{currentstroke}%
\pgfsetdash{}{0pt}%
\pgfpathmoveto{\pgfqpoint{3.793912in}{0.557870in}}%
\pgfpathlineto{\pgfqpoint{3.793912in}{2.042604in}}%
\pgfusepath{stroke}%
\end{pgfscope}%
\begin{pgfscope}%
\pgfsetrectcap%
\pgfsetmiterjoin%
\pgfsetlinewidth{1.254687pt}%
\definecolor{currentstroke}{rgb}{1.000000,1.000000,1.000000}%
\pgfsetstrokecolor{currentstroke}%
\pgfsetdash{}{0pt}%
\pgfpathmoveto{\pgfqpoint{6.240000in}{0.557870in}}%
\pgfpathlineto{\pgfqpoint{6.240000in}{2.042604in}}%
\pgfusepath{stroke}%
\end{pgfscope}%
\begin{pgfscope}%
\pgfsetrectcap%
\pgfsetmiterjoin%
\pgfsetlinewidth{1.254687pt}%
\definecolor{currentstroke}{rgb}{1.000000,1.000000,1.000000}%
\pgfsetstrokecolor{currentstroke}%
\pgfsetdash{}{0pt}%
\pgfpathmoveto{\pgfqpoint{3.793912in}{0.557870in}}%
\pgfpathlineto{\pgfqpoint{6.240000in}{0.557870in}}%
\pgfusepath{stroke}%
\end{pgfscope}%
\begin{pgfscope}%
\pgfsetrectcap%
\pgfsetmiterjoin%
\pgfsetlinewidth{1.254687pt}%
\definecolor{currentstroke}{rgb}{1.000000,1.000000,1.000000}%
\pgfsetstrokecolor{currentstroke}%
\pgfsetdash{}{0pt}%
\pgfpathmoveto{\pgfqpoint{3.793912in}{2.042604in}}%
\pgfpathlineto{\pgfqpoint{6.240000in}{2.042604in}}%
\pgfusepath{stroke}%
\end{pgfscope}%
\begin{pgfscope}%
\definecolor{textcolor}{rgb}{0.150000,0.150000,0.150000}%
\pgfsetstrokecolor{textcolor}%
\pgfsetfillcolor{textcolor}%
\pgftext[x=5.016956in,y=2.125938in,,base]{\color{textcolor}\sffamily\fontsize{11.000000}{13.200000}\selectfont (b)}%
\end{pgfscope}%
\end{pgfpicture}%
\makeatother%
\endgroup%

    \caption{Distribution of DOR, sensitivity and specificity for the different TSC methods when classifying patient diagnosis.}
    \label{fig:tsc_ind_dor_sens_spec_dist}
\end{figure}

\begin{figure}[htb]
    \centering
    % \includegraphics[width=\textwidth]{results/tsc-ind-ari.png}
    %% Creator: Matplotlib, PGF backend
%%
%% To include the figure in your LaTeX document, write
%%   \input{<filename>.pgf}
%%
%% Make sure the required packages are loaded in your preamble
%%   \usepackage{pgf}
%%
%% Figures using additional raster images can only be included by \input if
%% they are in the same directory as the main LaTeX file. For loading figures
%% from other directories you can use the `import` package
%%   \usepackage{import}
%% and then include the figures with
%%   \import{<path to file>}{<filename>.pgf}
%%
%% Matplotlib used the following preamble
%%
\begingroup%
\makeatletter%
\begin{pgfpicture}%
\pgfpathrectangle{\pgfpointorigin}{\pgfqpoint{6.340000in}{2.340000in}}%
\pgfusepath{use as bounding box, clip}%
\begin{pgfscope}%
\pgfsetbuttcap%
\pgfsetmiterjoin%
\definecolor{currentfill}{rgb}{1.000000,1.000000,1.000000}%
\pgfsetfillcolor{currentfill}%
\pgfsetlinewidth{0.000000pt}%
\definecolor{currentstroke}{rgb}{1.000000,1.000000,1.000000}%
\pgfsetstrokecolor{currentstroke}%
\pgfsetdash{}{0pt}%
\pgfpathmoveto{\pgfqpoint{0.000000in}{-0.000000in}}%
\pgfpathlineto{\pgfqpoint{6.340000in}{-0.000000in}}%
\pgfpathlineto{\pgfqpoint{6.340000in}{2.340000in}}%
\pgfpathlineto{\pgfqpoint{0.000000in}{2.340000in}}%
\pgfpathclose%
\pgfusepath{fill}%
\end{pgfscope}%
\begin{pgfscope}%
\pgfsetbuttcap%
\pgfsetmiterjoin%
\definecolor{currentfill}{rgb}{0.917647,0.917647,0.949020}%
\pgfsetfillcolor{currentfill}%
\pgfsetlinewidth{0.000000pt}%
\definecolor{currentstroke}{rgb}{0.000000,0.000000,0.000000}%
\pgfsetstrokecolor{currentstroke}%
\pgfsetstrokeopacity{0.000000}%
\pgfsetdash{}{0pt}%
\pgfpathmoveto{\pgfqpoint{0.650810in}{0.557870in}}%
\pgfpathlineto{\pgfqpoint{6.240000in}{0.557870in}}%
\pgfpathlineto{\pgfqpoint{6.240000in}{2.240000in}}%
\pgfpathlineto{\pgfqpoint{0.650810in}{2.240000in}}%
\pgfpathclose%
\pgfusepath{fill}%
\end{pgfscope}%
\begin{pgfscope}%
\pgfpathrectangle{\pgfqpoint{0.650810in}{0.557870in}}{\pgfqpoint{5.589190in}{1.682130in}}%
\pgfusepath{clip}%
\pgfsetroundcap%
\pgfsetroundjoin%
\pgfsetlinewidth{1.003750pt}%
\definecolor{currentstroke}{rgb}{1.000000,1.000000,1.000000}%
\pgfsetstrokecolor{currentstroke}%
\pgfsetdash{}{0pt}%
\pgfpathmoveto{\pgfqpoint{1.025210in}{0.557870in}}%
\pgfpathlineto{\pgfqpoint{1.025210in}{2.240000in}}%
\pgfusepath{stroke}%
\end{pgfscope}%
\begin{pgfscope}%
\definecolor{textcolor}{rgb}{0.150000,0.150000,0.150000}%
\pgfsetstrokecolor{textcolor}%
\pgfsetfillcolor{textcolor}%
\pgftext[x=1.025210in,y=0.425926in,,top]{\color{textcolor}\sffamily\fontsize{11.000000}{13.200000}\selectfont \(\displaystyle -0.1\)}%
\end{pgfscope}%
\begin{pgfscope}%
\pgfpathrectangle{\pgfqpoint{0.650810in}{0.557870in}}{\pgfqpoint{5.589190in}{1.682130in}}%
\pgfusepath{clip}%
\pgfsetroundcap%
\pgfsetroundjoin%
\pgfsetlinewidth{1.003750pt}%
\definecolor{currentstroke}{rgb}{1.000000,1.000000,1.000000}%
\pgfsetstrokecolor{currentstroke}%
\pgfsetdash{}{0pt}%
\pgfpathmoveto{\pgfqpoint{2.135005in}{0.557870in}}%
\pgfpathlineto{\pgfqpoint{2.135005in}{2.240000in}}%
\pgfusepath{stroke}%
\end{pgfscope}%
\begin{pgfscope}%
\definecolor{textcolor}{rgb}{0.150000,0.150000,0.150000}%
\pgfsetstrokecolor{textcolor}%
\pgfsetfillcolor{textcolor}%
\pgftext[x=2.135005in,y=0.425926in,,top]{\color{textcolor}\sffamily\fontsize{11.000000}{13.200000}\selectfont \(\displaystyle 0.0\)}%
\end{pgfscope}%
\begin{pgfscope}%
\pgfpathrectangle{\pgfqpoint{0.650810in}{0.557870in}}{\pgfqpoint{5.589190in}{1.682130in}}%
\pgfusepath{clip}%
\pgfsetroundcap%
\pgfsetroundjoin%
\pgfsetlinewidth{1.003750pt}%
\definecolor{currentstroke}{rgb}{1.000000,1.000000,1.000000}%
\pgfsetstrokecolor{currentstroke}%
\pgfsetdash{}{0pt}%
\pgfpathmoveto{\pgfqpoint{3.244800in}{0.557870in}}%
\pgfpathlineto{\pgfqpoint{3.244800in}{2.240000in}}%
\pgfusepath{stroke}%
\end{pgfscope}%
\begin{pgfscope}%
\definecolor{textcolor}{rgb}{0.150000,0.150000,0.150000}%
\pgfsetstrokecolor{textcolor}%
\pgfsetfillcolor{textcolor}%
\pgftext[x=3.244800in,y=0.425926in,,top]{\color{textcolor}\sffamily\fontsize{11.000000}{13.200000}\selectfont \(\displaystyle 0.1\)}%
\end{pgfscope}%
\begin{pgfscope}%
\pgfpathrectangle{\pgfqpoint{0.650810in}{0.557870in}}{\pgfqpoint{5.589190in}{1.682130in}}%
\pgfusepath{clip}%
\pgfsetroundcap%
\pgfsetroundjoin%
\pgfsetlinewidth{1.003750pt}%
\definecolor{currentstroke}{rgb}{1.000000,1.000000,1.000000}%
\pgfsetstrokecolor{currentstroke}%
\pgfsetdash{}{0pt}%
\pgfpathmoveto{\pgfqpoint{4.354596in}{0.557870in}}%
\pgfpathlineto{\pgfqpoint{4.354596in}{2.240000in}}%
\pgfusepath{stroke}%
\end{pgfscope}%
\begin{pgfscope}%
\definecolor{textcolor}{rgb}{0.150000,0.150000,0.150000}%
\pgfsetstrokecolor{textcolor}%
\pgfsetfillcolor{textcolor}%
\pgftext[x=4.354596in,y=0.425926in,,top]{\color{textcolor}\sffamily\fontsize{11.000000}{13.200000}\selectfont \(\displaystyle 0.2\)}%
\end{pgfscope}%
\begin{pgfscope}%
\pgfpathrectangle{\pgfqpoint{0.650810in}{0.557870in}}{\pgfqpoint{5.589190in}{1.682130in}}%
\pgfusepath{clip}%
\pgfsetroundcap%
\pgfsetroundjoin%
\pgfsetlinewidth{1.003750pt}%
\definecolor{currentstroke}{rgb}{1.000000,1.000000,1.000000}%
\pgfsetstrokecolor{currentstroke}%
\pgfsetdash{}{0pt}%
\pgfpathmoveto{\pgfqpoint{5.464391in}{0.557870in}}%
\pgfpathlineto{\pgfqpoint{5.464391in}{2.240000in}}%
\pgfusepath{stroke}%
\end{pgfscope}%
\begin{pgfscope}%
\definecolor{textcolor}{rgb}{0.150000,0.150000,0.150000}%
\pgfsetstrokecolor{textcolor}%
\pgfsetfillcolor{textcolor}%
\pgftext[x=5.464391in,y=0.425926in,,top]{\color{textcolor}\sffamily\fontsize{11.000000}{13.200000}\selectfont \(\displaystyle 0.3\)}%
\end{pgfscope}%
\begin{pgfscope}%
\definecolor{textcolor}{rgb}{0.150000,0.150000,0.150000}%
\pgfsetstrokecolor{textcolor}%
\pgfsetfillcolor{textcolor}%
\pgftext[x=3.445405in,y=0.235185in,,top]{\color{textcolor}\sffamily\fontsize{11.000000}{13.200000}\selectfont ARI}%
\end{pgfscope}%
\begin{pgfscope}%
\pgfpathrectangle{\pgfqpoint{0.650810in}{0.557870in}}{\pgfqpoint{5.589190in}{1.682130in}}%
\pgfusepath{clip}%
\pgfsetroundcap%
\pgfsetroundjoin%
\pgfsetlinewidth{1.003750pt}%
\definecolor{currentstroke}{rgb}{1.000000,1.000000,1.000000}%
\pgfsetstrokecolor{currentstroke}%
\pgfsetdash{}{0pt}%
\pgfpathmoveto{\pgfqpoint{0.650810in}{0.557870in}}%
\pgfpathlineto{\pgfqpoint{6.240000in}{0.557870in}}%
\pgfusepath{stroke}%
\end{pgfscope}%
\begin{pgfscope}%
\definecolor{textcolor}{rgb}{0.150000,0.150000,0.150000}%
\pgfsetstrokecolor{textcolor}%
\pgfsetfillcolor{textcolor}%
\pgftext[x=0.442824in,y=0.505064in,left,base]{\color{textcolor}\sffamily\fontsize{11.000000}{13.200000}\selectfont \(\displaystyle 0\)}%
\end{pgfscope}%
\begin{pgfscope}%
\pgfpathrectangle{\pgfqpoint{0.650810in}{0.557870in}}{\pgfqpoint{5.589190in}{1.682130in}}%
\pgfusepath{clip}%
\pgfsetroundcap%
\pgfsetroundjoin%
\pgfsetlinewidth{1.003750pt}%
\definecolor{currentstroke}{rgb}{1.000000,1.000000,1.000000}%
\pgfsetstrokecolor{currentstroke}%
\pgfsetdash{}{0pt}%
\pgfpathmoveto{\pgfqpoint{0.650810in}{1.006618in}}%
\pgfpathlineto{\pgfqpoint{6.240000in}{1.006618in}}%
\pgfusepath{stroke}%
\end{pgfscope}%
\begin{pgfscope}%
\definecolor{textcolor}{rgb}{0.150000,0.150000,0.150000}%
\pgfsetstrokecolor{textcolor}%
\pgfsetfillcolor{textcolor}%
\pgftext[x=0.290741in,y=0.953811in,left,base]{\color{textcolor}\sffamily\fontsize{11.000000}{13.200000}\selectfont \(\displaystyle 100\)}%
\end{pgfscope}%
\begin{pgfscope}%
\pgfpathrectangle{\pgfqpoint{0.650810in}{0.557870in}}{\pgfqpoint{5.589190in}{1.682130in}}%
\pgfusepath{clip}%
\pgfsetroundcap%
\pgfsetroundjoin%
\pgfsetlinewidth{1.003750pt}%
\definecolor{currentstroke}{rgb}{1.000000,1.000000,1.000000}%
\pgfsetstrokecolor{currentstroke}%
\pgfsetdash{}{0pt}%
\pgfpathmoveto{\pgfqpoint{0.650810in}{1.455365in}}%
\pgfpathlineto{\pgfqpoint{6.240000in}{1.455365in}}%
\pgfusepath{stroke}%
\end{pgfscope}%
\begin{pgfscope}%
\definecolor{textcolor}{rgb}{0.150000,0.150000,0.150000}%
\pgfsetstrokecolor{textcolor}%
\pgfsetfillcolor{textcolor}%
\pgftext[x=0.290741in,y=1.402558in,left,base]{\color{textcolor}\sffamily\fontsize{11.000000}{13.200000}\selectfont \(\displaystyle 200\)}%
\end{pgfscope}%
\begin{pgfscope}%
\pgfpathrectangle{\pgfqpoint{0.650810in}{0.557870in}}{\pgfqpoint{5.589190in}{1.682130in}}%
\pgfusepath{clip}%
\pgfsetroundcap%
\pgfsetroundjoin%
\pgfsetlinewidth{1.003750pt}%
\definecolor{currentstroke}{rgb}{1.000000,1.000000,1.000000}%
\pgfsetstrokecolor{currentstroke}%
\pgfsetdash{}{0pt}%
\pgfpathmoveto{\pgfqpoint{0.650810in}{1.904113in}}%
\pgfpathlineto{\pgfqpoint{6.240000in}{1.904113in}}%
\pgfusepath{stroke}%
\end{pgfscope}%
\begin{pgfscope}%
\definecolor{textcolor}{rgb}{0.150000,0.150000,0.150000}%
\pgfsetstrokecolor{textcolor}%
\pgfsetfillcolor{textcolor}%
\pgftext[x=0.290741in,y=1.851306in,left,base]{\color{textcolor}\sffamily\fontsize{11.000000}{13.200000}\selectfont \(\displaystyle 300\)}%
\end{pgfscope}%
\begin{pgfscope}%
\definecolor{textcolor}{rgb}{0.150000,0.150000,0.150000}%
\pgfsetstrokecolor{textcolor}%
\pgfsetfillcolor{textcolor}%
\pgftext[x=0.235185in,y=1.398935in,,bottom,rotate=90.000000]{\color{textcolor}\sffamily\fontsize{11.000000}{13.200000}\selectfont Occurance}%
\end{pgfscope}%
\begin{pgfscope}%
\pgfpathrectangle{\pgfqpoint{0.650810in}{0.557870in}}{\pgfqpoint{5.589190in}{1.682130in}}%
\pgfusepath{clip}%
\pgfsetbuttcap%
\pgfsetmiterjoin%
\definecolor{currentfill}{rgb}{0.298039,0.447059,0.690196}%
\pgfsetfillcolor{currentfill}%
\pgfsetfillopacity{0.400000}%
\pgfsetlinewidth{1.003750pt}%
\definecolor{currentstroke}{rgb}{1.000000,1.000000,1.000000}%
\pgfsetstrokecolor{currentstroke}%
\pgfsetstrokeopacity{0.400000}%
\pgfsetdash{}{0pt}%
\pgfpathmoveto{\pgfqpoint{0.904864in}{0.557870in}}%
\pgfpathlineto{\pgfqpoint{1.050038in}{0.557870in}}%
\pgfpathlineto{\pgfqpoint{1.050038in}{0.580308in}}%
\pgfpathlineto{\pgfqpoint{0.904864in}{0.580308in}}%
\pgfpathclose%
\pgfusepath{stroke,fill}%
\end{pgfscope}%
\begin{pgfscope}%
\pgfpathrectangle{\pgfqpoint{0.650810in}{0.557870in}}{\pgfqpoint{5.589190in}{1.682130in}}%
\pgfusepath{clip}%
\pgfsetbuttcap%
\pgfsetmiterjoin%
\definecolor{currentfill}{rgb}{0.298039,0.447059,0.690196}%
\pgfsetfillcolor{currentfill}%
\pgfsetfillopacity{0.400000}%
\pgfsetlinewidth{1.003750pt}%
\definecolor{currentstroke}{rgb}{1.000000,1.000000,1.000000}%
\pgfsetstrokecolor{currentstroke}%
\pgfsetstrokeopacity{0.400000}%
\pgfsetdash{}{0pt}%
\pgfpathmoveto{\pgfqpoint{1.050038in}{0.557870in}}%
\pgfpathlineto{\pgfqpoint{1.195212in}{0.557870in}}%
\pgfpathlineto{\pgfqpoint{1.195212in}{0.602745in}}%
\pgfpathlineto{\pgfqpoint{1.050038in}{0.602745in}}%
\pgfpathclose%
\pgfusepath{stroke,fill}%
\end{pgfscope}%
\begin{pgfscope}%
\pgfpathrectangle{\pgfqpoint{0.650810in}{0.557870in}}{\pgfqpoint{5.589190in}{1.682130in}}%
\pgfusepath{clip}%
\pgfsetbuttcap%
\pgfsetmiterjoin%
\definecolor{currentfill}{rgb}{0.298039,0.447059,0.690196}%
\pgfsetfillcolor{currentfill}%
\pgfsetfillopacity{0.400000}%
\pgfsetlinewidth{1.003750pt}%
\definecolor{currentstroke}{rgb}{1.000000,1.000000,1.000000}%
\pgfsetstrokecolor{currentstroke}%
\pgfsetstrokeopacity{0.400000}%
\pgfsetdash{}{0pt}%
\pgfpathmoveto{\pgfqpoint{1.195212in}{0.557870in}}%
\pgfpathlineto{\pgfqpoint{1.340386in}{0.557870in}}%
\pgfpathlineto{\pgfqpoint{1.340386in}{0.670057in}}%
\pgfpathlineto{\pgfqpoint{1.195212in}{0.670057in}}%
\pgfpathclose%
\pgfusepath{stroke,fill}%
\end{pgfscope}%
\begin{pgfscope}%
\pgfpathrectangle{\pgfqpoint{0.650810in}{0.557870in}}{\pgfqpoint{5.589190in}{1.682130in}}%
\pgfusepath{clip}%
\pgfsetbuttcap%
\pgfsetmiterjoin%
\definecolor{currentfill}{rgb}{0.298039,0.447059,0.690196}%
\pgfsetfillcolor{currentfill}%
\pgfsetfillopacity{0.400000}%
\pgfsetlinewidth{1.003750pt}%
\definecolor{currentstroke}{rgb}{1.000000,1.000000,1.000000}%
\pgfsetstrokecolor{currentstroke}%
\pgfsetstrokeopacity{0.400000}%
\pgfsetdash{}{0pt}%
\pgfpathmoveto{\pgfqpoint{1.340386in}{0.557870in}}%
\pgfpathlineto{\pgfqpoint{1.485560in}{0.557870in}}%
\pgfpathlineto{\pgfqpoint{1.485560in}{0.692495in}}%
\pgfpathlineto{\pgfqpoint{1.340386in}{0.692495in}}%
\pgfpathclose%
\pgfusepath{stroke,fill}%
\end{pgfscope}%
\begin{pgfscope}%
\pgfpathrectangle{\pgfqpoint{0.650810in}{0.557870in}}{\pgfqpoint{5.589190in}{1.682130in}}%
\pgfusepath{clip}%
\pgfsetbuttcap%
\pgfsetmiterjoin%
\definecolor{currentfill}{rgb}{0.298039,0.447059,0.690196}%
\pgfsetfillcolor{currentfill}%
\pgfsetfillopacity{0.400000}%
\pgfsetlinewidth{1.003750pt}%
\definecolor{currentstroke}{rgb}{1.000000,1.000000,1.000000}%
\pgfsetstrokecolor{currentstroke}%
\pgfsetstrokeopacity{0.400000}%
\pgfsetdash{}{0pt}%
\pgfpathmoveto{\pgfqpoint{1.485560in}{0.557870in}}%
\pgfpathlineto{\pgfqpoint{1.630733in}{0.557870in}}%
\pgfpathlineto{\pgfqpoint{1.630733in}{0.782244in}}%
\pgfpathlineto{\pgfqpoint{1.485560in}{0.782244in}}%
\pgfpathclose%
\pgfusepath{stroke,fill}%
\end{pgfscope}%
\begin{pgfscope}%
\pgfpathrectangle{\pgfqpoint{0.650810in}{0.557870in}}{\pgfqpoint{5.589190in}{1.682130in}}%
\pgfusepath{clip}%
\pgfsetbuttcap%
\pgfsetmiterjoin%
\definecolor{currentfill}{rgb}{0.298039,0.447059,0.690196}%
\pgfsetfillcolor{currentfill}%
\pgfsetfillopacity{0.400000}%
\pgfsetlinewidth{1.003750pt}%
\definecolor{currentstroke}{rgb}{1.000000,1.000000,1.000000}%
\pgfsetstrokecolor{currentstroke}%
\pgfsetstrokeopacity{0.400000}%
\pgfsetdash{}{0pt}%
\pgfpathmoveto{\pgfqpoint{1.630733in}{0.557870in}}%
\pgfpathlineto{\pgfqpoint{1.775907in}{0.557870in}}%
\pgfpathlineto{\pgfqpoint{1.775907in}{0.889943in}}%
\pgfpathlineto{\pgfqpoint{1.630733in}{0.889943in}}%
\pgfpathclose%
\pgfusepath{stroke,fill}%
\end{pgfscope}%
\begin{pgfscope}%
\pgfpathrectangle{\pgfqpoint{0.650810in}{0.557870in}}{\pgfqpoint{5.589190in}{1.682130in}}%
\pgfusepath{clip}%
\pgfsetbuttcap%
\pgfsetmiterjoin%
\definecolor{currentfill}{rgb}{0.298039,0.447059,0.690196}%
\pgfsetfillcolor{currentfill}%
\pgfsetfillopacity{0.400000}%
\pgfsetlinewidth{1.003750pt}%
\definecolor{currentstroke}{rgb}{1.000000,1.000000,1.000000}%
\pgfsetstrokecolor{currentstroke}%
\pgfsetstrokeopacity{0.400000}%
\pgfsetdash{}{0pt}%
\pgfpathmoveto{\pgfqpoint{1.775907in}{0.557870in}}%
\pgfpathlineto{\pgfqpoint{1.921081in}{0.557870in}}%
\pgfpathlineto{\pgfqpoint{1.921081in}{1.181629in}}%
\pgfpathlineto{\pgfqpoint{1.775907in}{1.181629in}}%
\pgfpathclose%
\pgfusepath{stroke,fill}%
\end{pgfscope}%
\begin{pgfscope}%
\pgfpathrectangle{\pgfqpoint{0.650810in}{0.557870in}}{\pgfqpoint{5.589190in}{1.682130in}}%
\pgfusepath{clip}%
\pgfsetbuttcap%
\pgfsetmiterjoin%
\definecolor{currentfill}{rgb}{0.298039,0.447059,0.690196}%
\pgfsetfillcolor{currentfill}%
\pgfsetfillopacity{0.400000}%
\pgfsetlinewidth{1.003750pt}%
\definecolor{currentstroke}{rgb}{1.000000,1.000000,1.000000}%
\pgfsetstrokecolor{currentstroke}%
\pgfsetstrokeopacity{0.400000}%
\pgfsetdash{}{0pt}%
\pgfpathmoveto{\pgfqpoint{1.921081in}{0.557870in}}%
\pgfpathlineto{\pgfqpoint{2.066255in}{0.557870in}}%
\pgfpathlineto{\pgfqpoint{2.066255in}{2.159899in}}%
\pgfpathlineto{\pgfqpoint{1.921081in}{2.159899in}}%
\pgfpathclose%
\pgfusepath{stroke,fill}%
\end{pgfscope}%
\begin{pgfscope}%
\pgfpathrectangle{\pgfqpoint{0.650810in}{0.557870in}}{\pgfqpoint{5.589190in}{1.682130in}}%
\pgfusepath{clip}%
\pgfsetbuttcap%
\pgfsetmiterjoin%
\definecolor{currentfill}{rgb}{0.298039,0.447059,0.690196}%
\pgfsetfillcolor{currentfill}%
\pgfsetfillopacity{0.400000}%
\pgfsetlinewidth{1.003750pt}%
\definecolor{currentstroke}{rgb}{1.000000,1.000000,1.000000}%
\pgfsetstrokecolor{currentstroke}%
\pgfsetstrokeopacity{0.400000}%
\pgfsetdash{}{0pt}%
\pgfpathmoveto{\pgfqpoint{2.066255in}{0.557870in}}%
\pgfpathlineto{\pgfqpoint{2.211428in}{0.557870in}}%
\pgfpathlineto{\pgfqpoint{2.211428in}{1.500240in}}%
\pgfpathlineto{\pgfqpoint{2.066255in}{1.500240in}}%
\pgfpathclose%
\pgfusepath{stroke,fill}%
\end{pgfscope}%
\begin{pgfscope}%
\pgfpathrectangle{\pgfqpoint{0.650810in}{0.557870in}}{\pgfqpoint{5.589190in}{1.682130in}}%
\pgfusepath{clip}%
\pgfsetbuttcap%
\pgfsetmiterjoin%
\definecolor{currentfill}{rgb}{0.298039,0.447059,0.690196}%
\pgfsetfillcolor{currentfill}%
\pgfsetfillopacity{0.400000}%
\pgfsetlinewidth{1.003750pt}%
\definecolor{currentstroke}{rgb}{1.000000,1.000000,1.000000}%
\pgfsetstrokecolor{currentstroke}%
\pgfsetstrokeopacity{0.400000}%
\pgfsetdash{}{0pt}%
\pgfpathmoveto{\pgfqpoint{2.211428in}{0.557870in}}%
\pgfpathlineto{\pgfqpoint{2.356602in}{0.557870in}}%
\pgfpathlineto{\pgfqpoint{2.356602in}{1.540627in}}%
\pgfpathlineto{\pgfqpoint{2.211428in}{1.540627in}}%
\pgfpathclose%
\pgfusepath{stroke,fill}%
\end{pgfscope}%
\begin{pgfscope}%
\pgfpathrectangle{\pgfqpoint{0.650810in}{0.557870in}}{\pgfqpoint{5.589190in}{1.682130in}}%
\pgfusepath{clip}%
\pgfsetbuttcap%
\pgfsetmiterjoin%
\definecolor{currentfill}{rgb}{0.298039,0.447059,0.690196}%
\pgfsetfillcolor{currentfill}%
\pgfsetfillopacity{0.400000}%
\pgfsetlinewidth{1.003750pt}%
\definecolor{currentstroke}{rgb}{1.000000,1.000000,1.000000}%
\pgfsetstrokecolor{currentstroke}%
\pgfsetstrokeopacity{0.400000}%
\pgfsetdash{}{0pt}%
\pgfpathmoveto{\pgfqpoint{2.356602in}{0.557870in}}%
\pgfpathlineto{\pgfqpoint{2.501776in}{0.557870in}}%
\pgfpathlineto{\pgfqpoint{2.501776in}{1.576527in}}%
\pgfpathlineto{\pgfqpoint{2.356602in}{1.576527in}}%
\pgfpathclose%
\pgfusepath{stroke,fill}%
\end{pgfscope}%
\begin{pgfscope}%
\pgfpathrectangle{\pgfqpoint{0.650810in}{0.557870in}}{\pgfqpoint{5.589190in}{1.682130in}}%
\pgfusepath{clip}%
\pgfsetbuttcap%
\pgfsetmiterjoin%
\definecolor{currentfill}{rgb}{0.298039,0.447059,0.690196}%
\pgfsetfillcolor{currentfill}%
\pgfsetfillopacity{0.400000}%
\pgfsetlinewidth{1.003750pt}%
\definecolor{currentstroke}{rgb}{1.000000,1.000000,1.000000}%
\pgfsetstrokecolor{currentstroke}%
\pgfsetstrokeopacity{0.400000}%
\pgfsetdash{}{0pt}%
\pgfpathmoveto{\pgfqpoint{2.501776in}{0.557870in}}%
\pgfpathlineto{\pgfqpoint{2.646950in}{0.557870in}}%
\pgfpathlineto{\pgfqpoint{2.646950in}{1.500240in}}%
\pgfpathlineto{\pgfqpoint{2.501776in}{1.500240in}}%
\pgfpathclose%
\pgfusepath{stroke,fill}%
\end{pgfscope}%
\begin{pgfscope}%
\pgfpathrectangle{\pgfqpoint{0.650810in}{0.557870in}}{\pgfqpoint{5.589190in}{1.682130in}}%
\pgfusepath{clip}%
\pgfsetbuttcap%
\pgfsetmiterjoin%
\definecolor{currentfill}{rgb}{0.298039,0.447059,0.690196}%
\pgfsetfillcolor{currentfill}%
\pgfsetfillopacity{0.400000}%
\pgfsetlinewidth{1.003750pt}%
\definecolor{currentstroke}{rgb}{1.000000,1.000000,1.000000}%
\pgfsetstrokecolor{currentstroke}%
\pgfsetstrokeopacity{0.400000}%
\pgfsetdash{}{0pt}%
\pgfpathmoveto{\pgfqpoint{2.646950in}{0.557870in}}%
\pgfpathlineto{\pgfqpoint{2.792123in}{0.557870in}}%
\pgfpathlineto{\pgfqpoint{2.792123in}{1.338691in}}%
\pgfpathlineto{\pgfqpoint{2.646950in}{1.338691in}}%
\pgfpathclose%
\pgfusepath{stroke,fill}%
\end{pgfscope}%
\begin{pgfscope}%
\pgfpathrectangle{\pgfqpoint{0.650810in}{0.557870in}}{\pgfqpoint{5.589190in}{1.682130in}}%
\pgfusepath{clip}%
\pgfsetbuttcap%
\pgfsetmiterjoin%
\definecolor{currentfill}{rgb}{0.298039,0.447059,0.690196}%
\pgfsetfillcolor{currentfill}%
\pgfsetfillopacity{0.400000}%
\pgfsetlinewidth{1.003750pt}%
\definecolor{currentstroke}{rgb}{1.000000,1.000000,1.000000}%
\pgfsetstrokecolor{currentstroke}%
\pgfsetstrokeopacity{0.400000}%
\pgfsetdash{}{0pt}%
\pgfpathmoveto{\pgfqpoint{2.792123in}{0.557870in}}%
\pgfpathlineto{\pgfqpoint{2.937297in}{0.557870in}}%
\pgfpathlineto{\pgfqpoint{2.937297in}{0.984180in}}%
\pgfpathlineto{\pgfqpoint{2.792123in}{0.984180in}}%
\pgfpathclose%
\pgfusepath{stroke,fill}%
\end{pgfscope}%
\begin{pgfscope}%
\pgfpathrectangle{\pgfqpoint{0.650810in}{0.557870in}}{\pgfqpoint{5.589190in}{1.682130in}}%
\pgfusepath{clip}%
\pgfsetbuttcap%
\pgfsetmiterjoin%
\definecolor{currentfill}{rgb}{0.298039,0.447059,0.690196}%
\pgfsetfillcolor{currentfill}%
\pgfsetfillopacity{0.400000}%
\pgfsetlinewidth{1.003750pt}%
\definecolor{currentstroke}{rgb}{1.000000,1.000000,1.000000}%
\pgfsetstrokecolor{currentstroke}%
\pgfsetstrokeopacity{0.400000}%
\pgfsetdash{}{0pt}%
\pgfpathmoveto{\pgfqpoint{2.937297in}{0.557870in}}%
\pgfpathlineto{\pgfqpoint{3.082471in}{0.557870in}}%
\pgfpathlineto{\pgfqpoint{3.082471in}{1.132267in}}%
\pgfpathlineto{\pgfqpoint{2.937297in}{1.132267in}}%
\pgfpathclose%
\pgfusepath{stroke,fill}%
\end{pgfscope}%
\begin{pgfscope}%
\pgfpathrectangle{\pgfqpoint{0.650810in}{0.557870in}}{\pgfqpoint{5.589190in}{1.682130in}}%
\pgfusepath{clip}%
\pgfsetbuttcap%
\pgfsetmiterjoin%
\definecolor{currentfill}{rgb}{0.298039,0.447059,0.690196}%
\pgfsetfillcolor{currentfill}%
\pgfsetfillopacity{0.400000}%
\pgfsetlinewidth{1.003750pt}%
\definecolor{currentstroke}{rgb}{1.000000,1.000000,1.000000}%
\pgfsetstrokecolor{currentstroke}%
\pgfsetstrokeopacity{0.400000}%
\pgfsetdash{}{0pt}%
\pgfpathmoveto{\pgfqpoint{3.082471in}{0.557870in}}%
\pgfpathlineto{\pgfqpoint{3.227645in}{0.557870in}}%
\pgfpathlineto{\pgfqpoint{3.227645in}{0.975205in}}%
\pgfpathlineto{\pgfqpoint{3.082471in}{0.975205in}}%
\pgfpathclose%
\pgfusepath{stroke,fill}%
\end{pgfscope}%
\begin{pgfscope}%
\pgfpathrectangle{\pgfqpoint{0.650810in}{0.557870in}}{\pgfqpoint{5.589190in}{1.682130in}}%
\pgfusepath{clip}%
\pgfsetbuttcap%
\pgfsetmiterjoin%
\definecolor{currentfill}{rgb}{0.298039,0.447059,0.690196}%
\pgfsetfillcolor{currentfill}%
\pgfsetfillopacity{0.400000}%
\pgfsetlinewidth{1.003750pt}%
\definecolor{currentstroke}{rgb}{1.000000,1.000000,1.000000}%
\pgfsetstrokecolor{currentstroke}%
\pgfsetstrokeopacity{0.400000}%
\pgfsetdash{}{0pt}%
\pgfpathmoveto{\pgfqpoint{3.227645in}{0.557870in}}%
\pgfpathlineto{\pgfqpoint{3.372818in}{0.557870in}}%
\pgfpathlineto{\pgfqpoint{3.372818in}{0.948281in}}%
\pgfpathlineto{\pgfqpoint{3.227645in}{0.948281in}}%
\pgfpathclose%
\pgfusepath{stroke,fill}%
\end{pgfscope}%
\begin{pgfscope}%
\pgfpathrectangle{\pgfqpoint{0.650810in}{0.557870in}}{\pgfqpoint{5.589190in}{1.682130in}}%
\pgfusepath{clip}%
\pgfsetbuttcap%
\pgfsetmiterjoin%
\definecolor{currentfill}{rgb}{0.298039,0.447059,0.690196}%
\pgfsetfillcolor{currentfill}%
\pgfsetfillopacity{0.400000}%
\pgfsetlinewidth{1.003750pt}%
\definecolor{currentstroke}{rgb}{1.000000,1.000000,1.000000}%
\pgfsetstrokecolor{currentstroke}%
\pgfsetstrokeopacity{0.400000}%
\pgfsetdash{}{0pt}%
\pgfpathmoveto{\pgfqpoint{3.372818in}{0.557870in}}%
\pgfpathlineto{\pgfqpoint{3.517992in}{0.557870in}}%
\pgfpathlineto{\pgfqpoint{3.517992in}{1.073930in}}%
\pgfpathlineto{\pgfqpoint{3.372818in}{1.073930in}}%
\pgfpathclose%
\pgfusepath{stroke,fill}%
\end{pgfscope}%
\begin{pgfscope}%
\pgfpathrectangle{\pgfqpoint{0.650810in}{0.557870in}}{\pgfqpoint{5.589190in}{1.682130in}}%
\pgfusepath{clip}%
\pgfsetbuttcap%
\pgfsetmiterjoin%
\definecolor{currentfill}{rgb}{0.298039,0.447059,0.690196}%
\pgfsetfillcolor{currentfill}%
\pgfsetfillopacity{0.400000}%
\pgfsetlinewidth{1.003750pt}%
\definecolor{currentstroke}{rgb}{1.000000,1.000000,1.000000}%
\pgfsetstrokecolor{currentstroke}%
\pgfsetstrokeopacity{0.400000}%
\pgfsetdash{}{0pt}%
\pgfpathmoveto{\pgfqpoint{3.517992in}{0.557870in}}%
\pgfpathlineto{\pgfqpoint{3.663166in}{0.557870in}}%
\pgfpathlineto{\pgfqpoint{3.663166in}{0.849556in}}%
\pgfpathlineto{\pgfqpoint{3.517992in}{0.849556in}}%
\pgfpathclose%
\pgfusepath{stroke,fill}%
\end{pgfscope}%
\begin{pgfscope}%
\pgfpathrectangle{\pgfqpoint{0.650810in}{0.557870in}}{\pgfqpoint{5.589190in}{1.682130in}}%
\pgfusepath{clip}%
\pgfsetbuttcap%
\pgfsetmiterjoin%
\definecolor{currentfill}{rgb}{0.298039,0.447059,0.690196}%
\pgfsetfillcolor{currentfill}%
\pgfsetfillopacity{0.400000}%
\pgfsetlinewidth{1.003750pt}%
\definecolor{currentstroke}{rgb}{1.000000,1.000000,1.000000}%
\pgfsetstrokecolor{currentstroke}%
\pgfsetstrokeopacity{0.400000}%
\pgfsetdash{}{0pt}%
\pgfpathmoveto{\pgfqpoint{3.663166in}{0.557870in}}%
\pgfpathlineto{\pgfqpoint{3.808340in}{0.557870in}}%
\pgfpathlineto{\pgfqpoint{3.808340in}{0.737369in}}%
\pgfpathlineto{\pgfqpoint{3.663166in}{0.737369in}}%
\pgfpathclose%
\pgfusepath{stroke,fill}%
\end{pgfscope}%
\begin{pgfscope}%
\pgfpathrectangle{\pgfqpoint{0.650810in}{0.557870in}}{\pgfqpoint{5.589190in}{1.682130in}}%
\pgfusepath{clip}%
\pgfsetbuttcap%
\pgfsetmiterjoin%
\definecolor{currentfill}{rgb}{0.298039,0.447059,0.690196}%
\pgfsetfillcolor{currentfill}%
\pgfsetfillopacity{0.400000}%
\pgfsetlinewidth{1.003750pt}%
\definecolor{currentstroke}{rgb}{1.000000,1.000000,1.000000}%
\pgfsetstrokecolor{currentstroke}%
\pgfsetstrokeopacity{0.400000}%
\pgfsetdash{}{0pt}%
\pgfpathmoveto{\pgfqpoint{3.808340in}{0.557870in}}%
\pgfpathlineto{\pgfqpoint{3.953513in}{0.557870in}}%
\pgfpathlineto{\pgfqpoint{3.953513in}{0.746344in}}%
\pgfpathlineto{\pgfqpoint{3.808340in}{0.746344in}}%
\pgfpathclose%
\pgfusepath{stroke,fill}%
\end{pgfscope}%
\begin{pgfscope}%
\pgfpathrectangle{\pgfqpoint{0.650810in}{0.557870in}}{\pgfqpoint{5.589190in}{1.682130in}}%
\pgfusepath{clip}%
\pgfsetbuttcap%
\pgfsetmiterjoin%
\definecolor{currentfill}{rgb}{0.298039,0.447059,0.690196}%
\pgfsetfillcolor{currentfill}%
\pgfsetfillopacity{0.400000}%
\pgfsetlinewidth{1.003750pt}%
\definecolor{currentstroke}{rgb}{1.000000,1.000000,1.000000}%
\pgfsetstrokecolor{currentstroke}%
\pgfsetstrokeopacity{0.400000}%
\pgfsetdash{}{0pt}%
\pgfpathmoveto{\pgfqpoint{3.953513in}{0.557870in}}%
\pgfpathlineto{\pgfqpoint{4.098687in}{0.557870in}}%
\pgfpathlineto{\pgfqpoint{4.098687in}{0.773269in}}%
\pgfpathlineto{\pgfqpoint{3.953513in}{0.773269in}}%
\pgfpathclose%
\pgfusepath{stroke,fill}%
\end{pgfscope}%
\begin{pgfscope}%
\pgfpathrectangle{\pgfqpoint{0.650810in}{0.557870in}}{\pgfqpoint{5.589190in}{1.682130in}}%
\pgfusepath{clip}%
\pgfsetbuttcap%
\pgfsetmiterjoin%
\definecolor{currentfill}{rgb}{0.298039,0.447059,0.690196}%
\pgfsetfillcolor{currentfill}%
\pgfsetfillopacity{0.400000}%
\pgfsetlinewidth{1.003750pt}%
\definecolor{currentstroke}{rgb}{1.000000,1.000000,1.000000}%
\pgfsetstrokecolor{currentstroke}%
\pgfsetstrokeopacity{0.400000}%
\pgfsetdash{}{0pt}%
\pgfpathmoveto{\pgfqpoint{4.098687in}{0.557870in}}%
\pgfpathlineto{\pgfqpoint{4.243861in}{0.557870in}}%
\pgfpathlineto{\pgfqpoint{4.243861in}{0.705957in}}%
\pgfpathlineto{\pgfqpoint{4.098687in}{0.705957in}}%
\pgfpathclose%
\pgfusepath{stroke,fill}%
\end{pgfscope}%
\begin{pgfscope}%
\pgfpathrectangle{\pgfqpoint{0.650810in}{0.557870in}}{\pgfqpoint{5.589190in}{1.682130in}}%
\pgfusepath{clip}%
\pgfsetbuttcap%
\pgfsetmiterjoin%
\definecolor{currentfill}{rgb}{0.298039,0.447059,0.690196}%
\pgfsetfillcolor{currentfill}%
\pgfsetfillopacity{0.400000}%
\pgfsetlinewidth{1.003750pt}%
\definecolor{currentstroke}{rgb}{1.000000,1.000000,1.000000}%
\pgfsetstrokecolor{currentstroke}%
\pgfsetstrokeopacity{0.400000}%
\pgfsetdash{}{0pt}%
\pgfpathmoveto{\pgfqpoint{4.243861in}{0.557870in}}%
\pgfpathlineto{\pgfqpoint{4.389035in}{0.557870in}}%
\pgfpathlineto{\pgfqpoint{4.389035in}{0.638645in}}%
\pgfpathlineto{\pgfqpoint{4.243861in}{0.638645in}}%
\pgfpathclose%
\pgfusepath{stroke,fill}%
\end{pgfscope}%
\begin{pgfscope}%
\pgfpathrectangle{\pgfqpoint{0.650810in}{0.557870in}}{\pgfqpoint{5.589190in}{1.682130in}}%
\pgfusepath{clip}%
\pgfsetbuttcap%
\pgfsetmiterjoin%
\definecolor{currentfill}{rgb}{0.298039,0.447059,0.690196}%
\pgfsetfillcolor{currentfill}%
\pgfsetfillopacity{0.400000}%
\pgfsetlinewidth{1.003750pt}%
\definecolor{currentstroke}{rgb}{1.000000,1.000000,1.000000}%
\pgfsetstrokecolor{currentstroke}%
\pgfsetstrokeopacity{0.400000}%
\pgfsetdash{}{0pt}%
\pgfpathmoveto{\pgfqpoint{4.389035in}{0.557870in}}%
\pgfpathlineto{\pgfqpoint{4.534208in}{0.557870in}}%
\pgfpathlineto{\pgfqpoint{4.534208in}{0.674545in}}%
\pgfpathlineto{\pgfqpoint{4.389035in}{0.674545in}}%
\pgfpathclose%
\pgfusepath{stroke,fill}%
\end{pgfscope}%
\begin{pgfscope}%
\pgfpathrectangle{\pgfqpoint{0.650810in}{0.557870in}}{\pgfqpoint{5.589190in}{1.682130in}}%
\pgfusepath{clip}%
\pgfsetbuttcap%
\pgfsetmiterjoin%
\definecolor{currentfill}{rgb}{0.298039,0.447059,0.690196}%
\pgfsetfillcolor{currentfill}%
\pgfsetfillopacity{0.400000}%
\pgfsetlinewidth{1.003750pt}%
\definecolor{currentstroke}{rgb}{1.000000,1.000000,1.000000}%
\pgfsetstrokecolor{currentstroke}%
\pgfsetstrokeopacity{0.400000}%
\pgfsetdash{}{0pt}%
\pgfpathmoveto{\pgfqpoint{4.534208in}{0.557870in}}%
\pgfpathlineto{\pgfqpoint{4.679382in}{0.557870in}}%
\pgfpathlineto{\pgfqpoint{4.679382in}{0.620695in}}%
\pgfpathlineto{\pgfqpoint{4.534208in}{0.620695in}}%
\pgfpathclose%
\pgfusepath{stroke,fill}%
\end{pgfscope}%
\begin{pgfscope}%
\pgfpathrectangle{\pgfqpoint{0.650810in}{0.557870in}}{\pgfqpoint{5.589190in}{1.682130in}}%
\pgfusepath{clip}%
\pgfsetbuttcap%
\pgfsetmiterjoin%
\definecolor{currentfill}{rgb}{0.298039,0.447059,0.690196}%
\pgfsetfillcolor{currentfill}%
\pgfsetfillopacity{0.400000}%
\pgfsetlinewidth{1.003750pt}%
\definecolor{currentstroke}{rgb}{1.000000,1.000000,1.000000}%
\pgfsetstrokecolor{currentstroke}%
\pgfsetstrokeopacity{0.400000}%
\pgfsetdash{}{0pt}%
\pgfpathmoveto{\pgfqpoint{4.679382in}{0.557870in}}%
\pgfpathlineto{\pgfqpoint{4.824556in}{0.557870in}}%
\pgfpathlineto{\pgfqpoint{4.824556in}{0.620695in}}%
\pgfpathlineto{\pgfqpoint{4.679382in}{0.620695in}}%
\pgfpathclose%
\pgfusepath{stroke,fill}%
\end{pgfscope}%
\begin{pgfscope}%
\pgfpathrectangle{\pgfqpoint{0.650810in}{0.557870in}}{\pgfqpoint{5.589190in}{1.682130in}}%
\pgfusepath{clip}%
\pgfsetbuttcap%
\pgfsetmiterjoin%
\definecolor{currentfill}{rgb}{0.298039,0.447059,0.690196}%
\pgfsetfillcolor{currentfill}%
\pgfsetfillopacity{0.400000}%
\pgfsetlinewidth{1.003750pt}%
\definecolor{currentstroke}{rgb}{1.000000,1.000000,1.000000}%
\pgfsetstrokecolor{currentstroke}%
\pgfsetstrokeopacity{0.400000}%
\pgfsetdash{}{0pt}%
\pgfpathmoveto{\pgfqpoint{4.824556in}{0.557870in}}%
\pgfpathlineto{\pgfqpoint{4.969730in}{0.557870in}}%
\pgfpathlineto{\pgfqpoint{4.969730in}{0.575820in}}%
\pgfpathlineto{\pgfqpoint{4.824556in}{0.575820in}}%
\pgfpathclose%
\pgfusepath{stroke,fill}%
\end{pgfscope}%
\begin{pgfscope}%
\pgfpathrectangle{\pgfqpoint{0.650810in}{0.557870in}}{\pgfqpoint{5.589190in}{1.682130in}}%
\pgfusepath{clip}%
\pgfsetbuttcap%
\pgfsetmiterjoin%
\definecolor{currentfill}{rgb}{0.298039,0.447059,0.690196}%
\pgfsetfillcolor{currentfill}%
\pgfsetfillopacity{0.400000}%
\pgfsetlinewidth{1.003750pt}%
\definecolor{currentstroke}{rgb}{1.000000,1.000000,1.000000}%
\pgfsetstrokecolor{currentstroke}%
\pgfsetstrokeopacity{0.400000}%
\pgfsetdash{}{0pt}%
\pgfpathmoveto{\pgfqpoint{4.969730in}{0.557870in}}%
\pgfpathlineto{\pgfqpoint{5.114903in}{0.557870in}}%
\pgfpathlineto{\pgfqpoint{5.114903in}{0.575820in}}%
\pgfpathlineto{\pgfqpoint{4.969730in}{0.575820in}}%
\pgfpathclose%
\pgfusepath{stroke,fill}%
\end{pgfscope}%
\begin{pgfscope}%
\pgfpathrectangle{\pgfqpoint{0.650810in}{0.557870in}}{\pgfqpoint{5.589190in}{1.682130in}}%
\pgfusepath{clip}%
\pgfsetbuttcap%
\pgfsetmiterjoin%
\definecolor{currentfill}{rgb}{0.298039,0.447059,0.690196}%
\pgfsetfillcolor{currentfill}%
\pgfsetfillopacity{0.400000}%
\pgfsetlinewidth{1.003750pt}%
\definecolor{currentstroke}{rgb}{1.000000,1.000000,1.000000}%
\pgfsetstrokecolor{currentstroke}%
\pgfsetstrokeopacity{0.400000}%
\pgfsetdash{}{0pt}%
\pgfpathmoveto{\pgfqpoint{5.114903in}{0.557870in}}%
\pgfpathlineto{\pgfqpoint{5.260077in}{0.557870in}}%
\pgfpathlineto{\pgfqpoint{5.260077in}{0.593770in}}%
\pgfpathlineto{\pgfqpoint{5.114903in}{0.593770in}}%
\pgfpathclose%
\pgfusepath{stroke,fill}%
\end{pgfscope}%
\begin{pgfscope}%
\pgfpathrectangle{\pgfqpoint{0.650810in}{0.557870in}}{\pgfqpoint{5.589190in}{1.682130in}}%
\pgfusepath{clip}%
\pgfsetbuttcap%
\pgfsetmiterjoin%
\definecolor{currentfill}{rgb}{0.298039,0.447059,0.690196}%
\pgfsetfillcolor{currentfill}%
\pgfsetfillopacity{0.400000}%
\pgfsetlinewidth{1.003750pt}%
\definecolor{currentstroke}{rgb}{1.000000,1.000000,1.000000}%
\pgfsetstrokecolor{currentstroke}%
\pgfsetstrokeopacity{0.400000}%
\pgfsetdash{}{0pt}%
\pgfpathmoveto{\pgfqpoint{5.260077in}{0.557870in}}%
\pgfpathlineto{\pgfqpoint{5.405251in}{0.557870in}}%
\pgfpathlineto{\pgfqpoint{5.405251in}{0.571333in}}%
\pgfpathlineto{\pgfqpoint{5.260077in}{0.571333in}}%
\pgfpathclose%
\pgfusepath{stroke,fill}%
\end{pgfscope}%
\begin{pgfscope}%
\pgfpathrectangle{\pgfqpoint{0.650810in}{0.557870in}}{\pgfqpoint{5.589190in}{1.682130in}}%
\pgfusepath{clip}%
\pgfsetbuttcap%
\pgfsetmiterjoin%
\definecolor{currentfill}{rgb}{0.298039,0.447059,0.690196}%
\pgfsetfillcolor{currentfill}%
\pgfsetfillopacity{0.400000}%
\pgfsetlinewidth{1.003750pt}%
\definecolor{currentstroke}{rgb}{1.000000,1.000000,1.000000}%
\pgfsetstrokecolor{currentstroke}%
\pgfsetstrokeopacity{0.400000}%
\pgfsetdash{}{0pt}%
\pgfpathmoveto{\pgfqpoint{5.405251in}{0.557870in}}%
\pgfpathlineto{\pgfqpoint{5.550425in}{0.557870in}}%
\pgfpathlineto{\pgfqpoint{5.550425in}{0.557870in}}%
\pgfpathlineto{\pgfqpoint{5.405251in}{0.557870in}}%
\pgfpathclose%
\pgfusepath{stroke,fill}%
\end{pgfscope}%
\begin{pgfscope}%
\pgfpathrectangle{\pgfqpoint{0.650810in}{0.557870in}}{\pgfqpoint{5.589190in}{1.682130in}}%
\pgfusepath{clip}%
\pgfsetbuttcap%
\pgfsetmiterjoin%
\definecolor{currentfill}{rgb}{0.298039,0.447059,0.690196}%
\pgfsetfillcolor{currentfill}%
\pgfsetfillopacity{0.400000}%
\pgfsetlinewidth{1.003750pt}%
\definecolor{currentstroke}{rgb}{1.000000,1.000000,1.000000}%
\pgfsetstrokecolor{currentstroke}%
\pgfsetstrokeopacity{0.400000}%
\pgfsetdash{}{0pt}%
\pgfpathmoveto{\pgfqpoint{5.550425in}{0.557870in}}%
\pgfpathlineto{\pgfqpoint{5.695598in}{0.557870in}}%
\pgfpathlineto{\pgfqpoint{5.695598in}{0.571333in}}%
\pgfpathlineto{\pgfqpoint{5.550425in}{0.571333in}}%
\pgfpathclose%
\pgfusepath{stroke,fill}%
\end{pgfscope}%
\begin{pgfscope}%
\pgfpathrectangle{\pgfqpoint{0.650810in}{0.557870in}}{\pgfqpoint{5.589190in}{1.682130in}}%
\pgfusepath{clip}%
\pgfsetbuttcap%
\pgfsetmiterjoin%
\definecolor{currentfill}{rgb}{0.298039,0.447059,0.690196}%
\pgfsetfillcolor{currentfill}%
\pgfsetfillopacity{0.400000}%
\pgfsetlinewidth{1.003750pt}%
\definecolor{currentstroke}{rgb}{1.000000,1.000000,1.000000}%
\pgfsetstrokecolor{currentstroke}%
\pgfsetstrokeopacity{0.400000}%
\pgfsetdash{}{0pt}%
\pgfpathmoveto{\pgfqpoint{5.695598in}{0.557870in}}%
\pgfpathlineto{\pgfqpoint{5.840772in}{0.557870in}}%
\pgfpathlineto{\pgfqpoint{5.840772in}{0.571333in}}%
\pgfpathlineto{\pgfqpoint{5.695598in}{0.571333in}}%
\pgfpathclose%
\pgfusepath{stroke,fill}%
\end{pgfscope}%
\begin{pgfscope}%
\pgfpathrectangle{\pgfqpoint{0.650810in}{0.557870in}}{\pgfqpoint{5.589190in}{1.682130in}}%
\pgfusepath{clip}%
\pgfsetbuttcap%
\pgfsetmiterjoin%
\definecolor{currentfill}{rgb}{0.298039,0.447059,0.690196}%
\pgfsetfillcolor{currentfill}%
\pgfsetfillopacity{0.400000}%
\pgfsetlinewidth{1.003750pt}%
\definecolor{currentstroke}{rgb}{1.000000,1.000000,1.000000}%
\pgfsetstrokecolor{currentstroke}%
\pgfsetstrokeopacity{0.400000}%
\pgfsetdash{}{0pt}%
\pgfpathmoveto{\pgfqpoint{5.840772in}{0.557870in}}%
\pgfpathlineto{\pgfqpoint{5.985946in}{0.557870in}}%
\pgfpathlineto{\pgfqpoint{5.985946in}{0.571333in}}%
\pgfpathlineto{\pgfqpoint{5.840772in}{0.571333in}}%
\pgfpathclose%
\pgfusepath{stroke,fill}%
\end{pgfscope}%
\begin{pgfscope}%
\pgfsetrectcap%
\pgfsetmiterjoin%
\pgfsetlinewidth{1.254687pt}%
\definecolor{currentstroke}{rgb}{1.000000,1.000000,1.000000}%
\pgfsetstrokecolor{currentstroke}%
\pgfsetdash{}{0pt}%
\pgfpathmoveto{\pgfqpoint{0.650810in}{0.557870in}}%
\pgfpathlineto{\pgfqpoint{0.650810in}{2.240000in}}%
\pgfusepath{stroke}%
\end{pgfscope}%
\begin{pgfscope}%
\pgfsetrectcap%
\pgfsetmiterjoin%
\pgfsetlinewidth{1.254687pt}%
\definecolor{currentstroke}{rgb}{1.000000,1.000000,1.000000}%
\pgfsetstrokecolor{currentstroke}%
\pgfsetdash{}{0pt}%
\pgfpathmoveto{\pgfqpoint{6.240000in}{0.557870in}}%
\pgfpathlineto{\pgfqpoint{6.240000in}{2.240000in}}%
\pgfusepath{stroke}%
\end{pgfscope}%
\begin{pgfscope}%
\pgfsetrectcap%
\pgfsetmiterjoin%
\pgfsetlinewidth{1.254687pt}%
\definecolor{currentstroke}{rgb}{1.000000,1.000000,1.000000}%
\pgfsetstrokecolor{currentstroke}%
\pgfsetdash{}{0pt}%
\pgfpathmoveto{\pgfqpoint{0.650810in}{0.557870in}}%
\pgfpathlineto{\pgfqpoint{6.240000in}{0.557870in}}%
\pgfusepath{stroke}%
\end{pgfscope}%
\begin{pgfscope}%
\pgfsetrectcap%
\pgfsetmiterjoin%
\pgfsetlinewidth{1.254687pt}%
\definecolor{currentstroke}{rgb}{1.000000,1.000000,1.000000}%
\pgfsetstrokecolor{currentstroke}%
\pgfsetdash{}{0pt}%
\pgfpathmoveto{\pgfqpoint{0.650810in}{2.240000in}}%
\pgfpathlineto{\pgfqpoint{6.240000in}{2.240000in}}%
\pgfusepath{stroke}%
\end{pgfscope}%
\end{pgfpicture}%
\makeatother%
\endgroup%

    \caption{ARI distribution of TSC methods when classifying patient diagnoses.}
    \label{fig:tsc_ind_ari}
\end{figure}

\newpage

\subsection{Peak-value Clustering}

\begin{figure}[htb]
    \centering
    % \includegraphics[width=\textwidth]{results/pvc_ind_dor_sens_spec_dist.png}
    %% Creator: Matplotlib, PGF backend
%%
%% To include the figure in your LaTeX document, write
%%   \input{<filename>.pgf}
%%
%% Make sure the required packages are loaded in your preamble
%%   \usepackage{pgf}
%%
%% Figures using additional raster images can only be included by \input if
%% they are in the same directory as the main LaTeX file. For loading figures
%% from other directories you can use the `import` package
%%   \usepackage{import}
%% and then include the figures with
%%   \import{<path to file>}{<filename>.pgf}
%%
%% Matplotlib used the following preamble
%%
\begingroup%
\makeatletter%
\begin{pgfpicture}%
\pgfpathrectangle{\pgfpointorigin}{\pgfqpoint{6.478830in}{2.340000in}}%
\pgfusepath{use as bounding box, clip}%
\begin{pgfscope}%
\pgfsetbuttcap%
\pgfsetmiterjoin%
\definecolor{currentfill}{rgb}{1.000000,1.000000,1.000000}%
\pgfsetfillcolor{currentfill}%
\pgfsetlinewidth{0.000000pt}%
\definecolor{currentstroke}{rgb}{1.000000,1.000000,1.000000}%
\pgfsetstrokecolor{currentstroke}%
\pgfsetdash{}{0pt}%
\pgfpathmoveto{\pgfqpoint{0.000000in}{-0.000000in}}%
\pgfpathlineto{\pgfqpoint{6.478830in}{-0.000000in}}%
\pgfpathlineto{\pgfqpoint{6.478830in}{2.340000in}}%
\pgfpathlineto{\pgfqpoint{0.000000in}{2.340000in}}%
\pgfpathclose%
\pgfusepath{fill}%
\end{pgfscope}%
\begin{pgfscope}%
\pgfsetbuttcap%
\pgfsetmiterjoin%
\definecolor{currentfill}{rgb}{0.917647,0.917647,0.949020}%
\pgfsetfillcolor{currentfill}%
\pgfsetlinewidth{0.000000pt}%
\definecolor{currentstroke}{rgb}{0.000000,0.000000,0.000000}%
\pgfsetstrokecolor{currentstroke}%
\pgfsetstrokeopacity{0.000000}%
\pgfsetdash{}{0pt}%
\pgfpathmoveto{\pgfqpoint{0.617014in}{0.557870in}}%
\pgfpathlineto{\pgfqpoint{3.139144in}{0.557870in}}%
\pgfpathlineto{\pgfqpoint{3.139144in}{2.042604in}}%
\pgfpathlineto{\pgfqpoint{0.617014in}{2.042604in}}%
\pgfpathclose%
\pgfusepath{fill}%
\end{pgfscope}%
\begin{pgfscope}%
\pgfpathrectangle{\pgfqpoint{0.617014in}{0.557870in}}{\pgfqpoint{2.522130in}{1.484734in}}%
\pgfusepath{clip}%
\pgfsetroundcap%
\pgfsetroundjoin%
\pgfsetlinewidth{1.003750pt}%
\definecolor{currentstroke}{rgb}{1.000000,1.000000,1.000000}%
\pgfsetstrokecolor{currentstroke}%
\pgfsetdash{}{0pt}%
\pgfpathmoveto{\pgfqpoint{0.731656in}{0.557870in}}%
\pgfpathlineto{\pgfqpoint{0.731656in}{2.042604in}}%
\pgfusepath{stroke}%
\end{pgfscope}%
\begin{pgfscope}%
\definecolor{textcolor}{rgb}{0.150000,0.150000,0.150000}%
\pgfsetstrokecolor{textcolor}%
\pgfsetfillcolor{textcolor}%
\pgftext[x=0.731656in,y=0.425926in,,top]{\color{textcolor}\sffamily\fontsize{11.000000}{13.200000}\selectfont \(\displaystyle 0\)}%
\end{pgfscope}%
\begin{pgfscope}%
\pgfpathrectangle{\pgfqpoint{0.617014in}{0.557870in}}{\pgfqpoint{2.522130in}{1.484734in}}%
\pgfusepath{clip}%
\pgfsetroundcap%
\pgfsetroundjoin%
\pgfsetlinewidth{1.003750pt}%
\definecolor{currentstroke}{rgb}{1.000000,1.000000,1.000000}%
\pgfsetstrokecolor{currentstroke}%
\pgfsetdash{}{0pt}%
\pgfpathmoveto{\pgfqpoint{1.348085in}{0.557870in}}%
\pgfpathlineto{\pgfqpoint{1.348085in}{2.042604in}}%
\pgfusepath{stroke}%
\end{pgfscope}%
\begin{pgfscope}%
\definecolor{textcolor}{rgb}{0.150000,0.150000,0.150000}%
\pgfsetstrokecolor{textcolor}%
\pgfsetfillcolor{textcolor}%
\pgftext[x=1.348085in,y=0.425926in,,top]{\color{textcolor}\sffamily\fontsize{11.000000}{13.200000}\selectfont \(\displaystyle 10\)}%
\end{pgfscope}%
\begin{pgfscope}%
\pgfpathrectangle{\pgfqpoint{0.617014in}{0.557870in}}{\pgfqpoint{2.522130in}{1.484734in}}%
\pgfusepath{clip}%
\pgfsetroundcap%
\pgfsetroundjoin%
\pgfsetlinewidth{1.003750pt}%
\definecolor{currentstroke}{rgb}{1.000000,1.000000,1.000000}%
\pgfsetstrokecolor{currentstroke}%
\pgfsetdash{}{0pt}%
\pgfpathmoveto{\pgfqpoint{1.964513in}{0.557870in}}%
\pgfpathlineto{\pgfqpoint{1.964513in}{2.042604in}}%
\pgfusepath{stroke}%
\end{pgfscope}%
\begin{pgfscope}%
\definecolor{textcolor}{rgb}{0.150000,0.150000,0.150000}%
\pgfsetstrokecolor{textcolor}%
\pgfsetfillcolor{textcolor}%
\pgftext[x=1.964513in,y=0.425926in,,top]{\color{textcolor}\sffamily\fontsize{11.000000}{13.200000}\selectfont \(\displaystyle 20\)}%
\end{pgfscope}%
\begin{pgfscope}%
\pgfpathrectangle{\pgfqpoint{0.617014in}{0.557870in}}{\pgfqpoint{2.522130in}{1.484734in}}%
\pgfusepath{clip}%
\pgfsetroundcap%
\pgfsetroundjoin%
\pgfsetlinewidth{1.003750pt}%
\definecolor{currentstroke}{rgb}{1.000000,1.000000,1.000000}%
\pgfsetstrokecolor{currentstroke}%
\pgfsetdash{}{0pt}%
\pgfpathmoveto{\pgfqpoint{2.580941in}{0.557870in}}%
\pgfpathlineto{\pgfqpoint{2.580941in}{2.042604in}}%
\pgfusepath{stroke}%
\end{pgfscope}%
\begin{pgfscope}%
\definecolor{textcolor}{rgb}{0.150000,0.150000,0.150000}%
\pgfsetstrokecolor{textcolor}%
\pgfsetfillcolor{textcolor}%
\pgftext[x=2.580941in,y=0.425926in,,top]{\color{textcolor}\sffamily\fontsize{11.000000}{13.200000}\selectfont \(\displaystyle 30\)}%
\end{pgfscope}%
\begin{pgfscope}%
\definecolor{textcolor}{rgb}{0.150000,0.150000,0.150000}%
\pgfsetstrokecolor{textcolor}%
\pgfsetfillcolor{textcolor}%
\pgftext[x=1.878079in,y=0.235185in,,top]{\color{textcolor}\sffamily\fontsize{11.000000}{13.200000}\selectfont DOR}%
\end{pgfscope}%
\begin{pgfscope}%
\pgfpathrectangle{\pgfqpoint{0.617014in}{0.557870in}}{\pgfqpoint{2.522130in}{1.484734in}}%
\pgfusepath{clip}%
\pgfsetroundcap%
\pgfsetroundjoin%
\pgfsetlinewidth{1.003750pt}%
\definecolor{currentstroke}{rgb}{1.000000,1.000000,1.000000}%
\pgfsetstrokecolor{currentstroke}%
\pgfsetdash{}{0pt}%
\pgfpathmoveto{\pgfqpoint{0.617014in}{0.557870in}}%
\pgfpathlineto{\pgfqpoint{3.139144in}{0.557870in}}%
\pgfusepath{stroke}%
\end{pgfscope}%
\begin{pgfscope}%
\definecolor{textcolor}{rgb}{0.150000,0.150000,0.150000}%
\pgfsetstrokecolor{textcolor}%
\pgfsetfillcolor{textcolor}%
\pgftext[x=0.290741in,y=0.505064in,left,base]{\color{textcolor}\sffamily\fontsize{11.000000}{13.200000}\selectfont \(\displaystyle 0.0\)}%
\end{pgfscope}%
\begin{pgfscope}%
\pgfpathrectangle{\pgfqpoint{0.617014in}{0.557870in}}{\pgfqpoint{2.522130in}{1.484734in}}%
\pgfusepath{clip}%
\pgfsetroundcap%
\pgfsetroundjoin%
\pgfsetlinewidth{1.003750pt}%
\definecolor{currentstroke}{rgb}{1.000000,1.000000,1.000000}%
\pgfsetstrokecolor{currentstroke}%
\pgfsetdash{}{0pt}%
\pgfpathmoveto{\pgfqpoint{0.617014in}{0.999755in}}%
\pgfpathlineto{\pgfqpoint{3.139144in}{0.999755in}}%
\pgfusepath{stroke}%
\end{pgfscope}%
\begin{pgfscope}%
\definecolor{textcolor}{rgb}{0.150000,0.150000,0.150000}%
\pgfsetstrokecolor{textcolor}%
\pgfsetfillcolor{textcolor}%
\pgftext[x=0.290741in,y=0.946949in,left,base]{\color{textcolor}\sffamily\fontsize{11.000000}{13.200000}\selectfont \(\displaystyle 2.5\)}%
\end{pgfscope}%
\begin{pgfscope}%
\pgfpathrectangle{\pgfqpoint{0.617014in}{0.557870in}}{\pgfqpoint{2.522130in}{1.484734in}}%
\pgfusepath{clip}%
\pgfsetroundcap%
\pgfsetroundjoin%
\pgfsetlinewidth{1.003750pt}%
\definecolor{currentstroke}{rgb}{1.000000,1.000000,1.000000}%
\pgfsetstrokecolor{currentstroke}%
\pgfsetdash{}{0pt}%
\pgfpathmoveto{\pgfqpoint{0.617014in}{1.441640in}}%
\pgfpathlineto{\pgfqpoint{3.139144in}{1.441640in}}%
\pgfusepath{stroke}%
\end{pgfscope}%
\begin{pgfscope}%
\definecolor{textcolor}{rgb}{0.150000,0.150000,0.150000}%
\pgfsetstrokecolor{textcolor}%
\pgfsetfillcolor{textcolor}%
\pgftext[x=0.290741in,y=1.388834in,left,base]{\color{textcolor}\sffamily\fontsize{11.000000}{13.200000}\selectfont \(\displaystyle 5.0\)}%
\end{pgfscope}%
\begin{pgfscope}%
\pgfpathrectangle{\pgfqpoint{0.617014in}{0.557870in}}{\pgfqpoint{2.522130in}{1.484734in}}%
\pgfusepath{clip}%
\pgfsetroundcap%
\pgfsetroundjoin%
\pgfsetlinewidth{1.003750pt}%
\definecolor{currentstroke}{rgb}{1.000000,1.000000,1.000000}%
\pgfsetstrokecolor{currentstroke}%
\pgfsetdash{}{0pt}%
\pgfpathmoveto{\pgfqpoint{0.617014in}{1.883526in}}%
\pgfpathlineto{\pgfqpoint{3.139144in}{1.883526in}}%
\pgfusepath{stroke}%
\end{pgfscope}%
\begin{pgfscope}%
\definecolor{textcolor}{rgb}{0.150000,0.150000,0.150000}%
\pgfsetstrokecolor{textcolor}%
\pgfsetfillcolor{textcolor}%
\pgftext[x=0.290741in,y=1.830719in,left,base]{\color{textcolor}\sffamily\fontsize{11.000000}{13.200000}\selectfont \(\displaystyle 7.5\)}%
\end{pgfscope}%
\begin{pgfscope}%
\definecolor{textcolor}{rgb}{0.150000,0.150000,0.150000}%
\pgfsetstrokecolor{textcolor}%
\pgfsetfillcolor{textcolor}%
\pgftext[x=0.235185in,y=1.300237in,,bottom,rotate=90.000000]{\color{textcolor}\sffamily\fontsize{11.000000}{13.200000}\selectfont Occurance}%
\end{pgfscope}%
\begin{pgfscope}%
\pgfpathrectangle{\pgfqpoint{0.617014in}{0.557870in}}{\pgfqpoint{2.522130in}{1.484734in}}%
\pgfusepath{clip}%
\pgfsetbuttcap%
\pgfsetmiterjoin%
\definecolor{currentfill}{rgb}{0.298039,0.447059,0.690196}%
\pgfsetfillcolor{currentfill}%
\pgfsetfillopacity{0.400000}%
\pgfsetlinewidth{1.003750pt}%
\definecolor{currentstroke}{rgb}{1.000000,1.000000,1.000000}%
\pgfsetstrokecolor{currentstroke}%
\pgfsetstrokeopacity{0.400000}%
\pgfsetdash{}{0pt}%
\pgfpathmoveto{\pgfqpoint{0.731656in}{0.557870in}}%
\pgfpathlineto{\pgfqpoint{0.960941in}{0.557870in}}%
\pgfpathlineto{\pgfqpoint{0.960941in}{1.971903in}}%
\pgfpathlineto{\pgfqpoint{0.731656in}{1.971903in}}%
\pgfpathclose%
\pgfusepath{stroke,fill}%
\end{pgfscope}%
\begin{pgfscope}%
\pgfpathrectangle{\pgfqpoint{0.617014in}{0.557870in}}{\pgfqpoint{2.522130in}{1.484734in}}%
\pgfusepath{clip}%
\pgfsetbuttcap%
\pgfsetmiterjoin%
\definecolor{currentfill}{rgb}{0.298039,0.447059,0.690196}%
\pgfsetfillcolor{currentfill}%
\pgfsetfillopacity{0.400000}%
\pgfsetlinewidth{1.003750pt}%
\definecolor{currentstroke}{rgb}{1.000000,1.000000,1.000000}%
\pgfsetstrokecolor{currentstroke}%
\pgfsetstrokeopacity{0.400000}%
\pgfsetdash{}{0pt}%
\pgfpathmoveto{\pgfqpoint{0.960941in}{0.557870in}}%
\pgfpathlineto{\pgfqpoint{1.190225in}{0.557870in}}%
\pgfpathlineto{\pgfqpoint{1.190225in}{0.734624in}}%
\pgfpathlineto{\pgfqpoint{0.960941in}{0.734624in}}%
\pgfpathclose%
\pgfusepath{stroke,fill}%
\end{pgfscope}%
\begin{pgfscope}%
\pgfpathrectangle{\pgfqpoint{0.617014in}{0.557870in}}{\pgfqpoint{2.522130in}{1.484734in}}%
\pgfusepath{clip}%
\pgfsetbuttcap%
\pgfsetmiterjoin%
\definecolor{currentfill}{rgb}{0.298039,0.447059,0.690196}%
\pgfsetfillcolor{currentfill}%
\pgfsetfillopacity{0.400000}%
\pgfsetlinewidth{1.003750pt}%
\definecolor{currentstroke}{rgb}{1.000000,1.000000,1.000000}%
\pgfsetstrokecolor{currentstroke}%
\pgfsetstrokeopacity{0.400000}%
\pgfsetdash{}{0pt}%
\pgfpathmoveto{\pgfqpoint{1.190225in}{0.557870in}}%
\pgfpathlineto{\pgfqpoint{1.419510in}{0.557870in}}%
\pgfpathlineto{\pgfqpoint{1.419510in}{0.911378in}}%
\pgfpathlineto{\pgfqpoint{1.190225in}{0.911378in}}%
\pgfpathclose%
\pgfusepath{stroke,fill}%
\end{pgfscope}%
\begin{pgfscope}%
\pgfpathrectangle{\pgfqpoint{0.617014in}{0.557870in}}{\pgfqpoint{2.522130in}{1.484734in}}%
\pgfusepath{clip}%
\pgfsetbuttcap%
\pgfsetmiterjoin%
\definecolor{currentfill}{rgb}{0.298039,0.447059,0.690196}%
\pgfsetfillcolor{currentfill}%
\pgfsetfillopacity{0.400000}%
\pgfsetlinewidth{1.003750pt}%
\definecolor{currentstroke}{rgb}{1.000000,1.000000,1.000000}%
\pgfsetstrokecolor{currentstroke}%
\pgfsetstrokeopacity{0.400000}%
\pgfsetdash{}{0pt}%
\pgfpathmoveto{\pgfqpoint{1.419510in}{0.557870in}}%
\pgfpathlineto{\pgfqpoint{1.648794in}{0.557870in}}%
\pgfpathlineto{\pgfqpoint{1.648794in}{1.264886in}}%
\pgfpathlineto{\pgfqpoint{1.419510in}{1.264886in}}%
\pgfpathclose%
\pgfusepath{stroke,fill}%
\end{pgfscope}%
\begin{pgfscope}%
\pgfpathrectangle{\pgfqpoint{0.617014in}{0.557870in}}{\pgfqpoint{2.522130in}{1.484734in}}%
\pgfusepath{clip}%
\pgfsetbuttcap%
\pgfsetmiterjoin%
\definecolor{currentfill}{rgb}{0.298039,0.447059,0.690196}%
\pgfsetfillcolor{currentfill}%
\pgfsetfillopacity{0.400000}%
\pgfsetlinewidth{1.003750pt}%
\definecolor{currentstroke}{rgb}{1.000000,1.000000,1.000000}%
\pgfsetstrokecolor{currentstroke}%
\pgfsetstrokeopacity{0.400000}%
\pgfsetdash{}{0pt}%
\pgfpathmoveto{\pgfqpoint{1.648794in}{0.557870in}}%
\pgfpathlineto{\pgfqpoint{1.878079in}{0.557870in}}%
\pgfpathlineto{\pgfqpoint{1.878079in}{0.557870in}}%
\pgfpathlineto{\pgfqpoint{1.648794in}{0.557870in}}%
\pgfpathclose%
\pgfusepath{stroke,fill}%
\end{pgfscope}%
\begin{pgfscope}%
\pgfpathrectangle{\pgfqpoint{0.617014in}{0.557870in}}{\pgfqpoint{2.522130in}{1.484734in}}%
\pgfusepath{clip}%
\pgfsetbuttcap%
\pgfsetmiterjoin%
\definecolor{currentfill}{rgb}{0.298039,0.447059,0.690196}%
\pgfsetfillcolor{currentfill}%
\pgfsetfillopacity{0.400000}%
\pgfsetlinewidth{1.003750pt}%
\definecolor{currentstroke}{rgb}{1.000000,1.000000,1.000000}%
\pgfsetstrokecolor{currentstroke}%
\pgfsetstrokeopacity{0.400000}%
\pgfsetdash{}{0pt}%
\pgfpathmoveto{\pgfqpoint{1.878079in}{0.557870in}}%
\pgfpathlineto{\pgfqpoint{2.107363in}{0.557870in}}%
\pgfpathlineto{\pgfqpoint{2.107363in}{0.734624in}}%
\pgfpathlineto{\pgfqpoint{1.878079in}{0.734624in}}%
\pgfpathclose%
\pgfusepath{stroke,fill}%
\end{pgfscope}%
\begin{pgfscope}%
\pgfpathrectangle{\pgfqpoint{0.617014in}{0.557870in}}{\pgfqpoint{2.522130in}{1.484734in}}%
\pgfusepath{clip}%
\pgfsetbuttcap%
\pgfsetmiterjoin%
\definecolor{currentfill}{rgb}{0.298039,0.447059,0.690196}%
\pgfsetfillcolor{currentfill}%
\pgfsetfillopacity{0.400000}%
\pgfsetlinewidth{1.003750pt}%
\definecolor{currentstroke}{rgb}{1.000000,1.000000,1.000000}%
\pgfsetstrokecolor{currentstroke}%
\pgfsetstrokeopacity{0.400000}%
\pgfsetdash{}{0pt}%
\pgfpathmoveto{\pgfqpoint{2.107363in}{0.557870in}}%
\pgfpathlineto{\pgfqpoint{2.336648in}{0.557870in}}%
\pgfpathlineto{\pgfqpoint{2.336648in}{0.734624in}}%
\pgfpathlineto{\pgfqpoint{2.107363in}{0.734624in}}%
\pgfpathclose%
\pgfusepath{stroke,fill}%
\end{pgfscope}%
\begin{pgfscope}%
\pgfpathrectangle{\pgfqpoint{0.617014in}{0.557870in}}{\pgfqpoint{2.522130in}{1.484734in}}%
\pgfusepath{clip}%
\pgfsetbuttcap%
\pgfsetmiterjoin%
\definecolor{currentfill}{rgb}{0.298039,0.447059,0.690196}%
\pgfsetfillcolor{currentfill}%
\pgfsetfillopacity{0.400000}%
\pgfsetlinewidth{1.003750pt}%
\definecolor{currentstroke}{rgb}{1.000000,1.000000,1.000000}%
\pgfsetstrokecolor{currentstroke}%
\pgfsetstrokeopacity{0.400000}%
\pgfsetdash{}{0pt}%
\pgfpathmoveto{\pgfqpoint{2.336648in}{0.557870in}}%
\pgfpathlineto{\pgfqpoint{2.565932in}{0.557870in}}%
\pgfpathlineto{\pgfqpoint{2.565932in}{0.557870in}}%
\pgfpathlineto{\pgfqpoint{2.336648in}{0.557870in}}%
\pgfpathclose%
\pgfusepath{stroke,fill}%
\end{pgfscope}%
\begin{pgfscope}%
\pgfpathrectangle{\pgfqpoint{0.617014in}{0.557870in}}{\pgfqpoint{2.522130in}{1.484734in}}%
\pgfusepath{clip}%
\pgfsetbuttcap%
\pgfsetmiterjoin%
\definecolor{currentfill}{rgb}{0.298039,0.447059,0.690196}%
\pgfsetfillcolor{currentfill}%
\pgfsetfillopacity{0.400000}%
\pgfsetlinewidth{1.003750pt}%
\definecolor{currentstroke}{rgb}{1.000000,1.000000,1.000000}%
\pgfsetstrokecolor{currentstroke}%
\pgfsetstrokeopacity{0.400000}%
\pgfsetdash{}{0pt}%
\pgfpathmoveto{\pgfqpoint{2.565932in}{0.557870in}}%
\pgfpathlineto{\pgfqpoint{2.795217in}{0.557870in}}%
\pgfpathlineto{\pgfqpoint{2.795217in}{0.911378in}}%
\pgfpathlineto{\pgfqpoint{2.565932in}{0.911378in}}%
\pgfpathclose%
\pgfusepath{stroke,fill}%
\end{pgfscope}%
\begin{pgfscope}%
\pgfpathrectangle{\pgfqpoint{0.617014in}{0.557870in}}{\pgfqpoint{2.522130in}{1.484734in}}%
\pgfusepath{clip}%
\pgfsetbuttcap%
\pgfsetmiterjoin%
\definecolor{currentfill}{rgb}{0.298039,0.447059,0.690196}%
\pgfsetfillcolor{currentfill}%
\pgfsetfillopacity{0.400000}%
\pgfsetlinewidth{1.003750pt}%
\definecolor{currentstroke}{rgb}{1.000000,1.000000,1.000000}%
\pgfsetstrokecolor{currentstroke}%
\pgfsetstrokeopacity{0.400000}%
\pgfsetdash{}{0pt}%
\pgfpathmoveto{\pgfqpoint{2.795217in}{0.557870in}}%
\pgfpathlineto{\pgfqpoint{3.024501in}{0.557870in}}%
\pgfpathlineto{\pgfqpoint{3.024501in}{0.911378in}}%
\pgfpathlineto{\pgfqpoint{2.795217in}{0.911378in}}%
\pgfpathclose%
\pgfusepath{stroke,fill}%
\end{pgfscope}%
\begin{pgfscope}%
\pgfsetrectcap%
\pgfsetmiterjoin%
\pgfsetlinewidth{1.254687pt}%
\definecolor{currentstroke}{rgb}{1.000000,1.000000,1.000000}%
\pgfsetstrokecolor{currentstroke}%
\pgfsetdash{}{0pt}%
\pgfpathmoveto{\pgfqpoint{0.617014in}{0.557870in}}%
\pgfpathlineto{\pgfqpoint{0.617014in}{2.042604in}}%
\pgfusepath{stroke}%
\end{pgfscope}%
\begin{pgfscope}%
\pgfsetrectcap%
\pgfsetmiterjoin%
\pgfsetlinewidth{1.254687pt}%
\definecolor{currentstroke}{rgb}{1.000000,1.000000,1.000000}%
\pgfsetstrokecolor{currentstroke}%
\pgfsetdash{}{0pt}%
\pgfpathmoveto{\pgfqpoint{3.139144in}{0.557870in}}%
\pgfpathlineto{\pgfqpoint{3.139144in}{2.042604in}}%
\pgfusepath{stroke}%
\end{pgfscope}%
\begin{pgfscope}%
\pgfsetrectcap%
\pgfsetmiterjoin%
\pgfsetlinewidth{1.254687pt}%
\definecolor{currentstroke}{rgb}{1.000000,1.000000,1.000000}%
\pgfsetstrokecolor{currentstroke}%
\pgfsetdash{}{0pt}%
\pgfpathmoveto{\pgfqpoint{0.617014in}{0.557870in}}%
\pgfpathlineto{\pgfqpoint{3.139144in}{0.557870in}}%
\pgfusepath{stroke}%
\end{pgfscope}%
\begin{pgfscope}%
\pgfsetrectcap%
\pgfsetmiterjoin%
\pgfsetlinewidth{1.254687pt}%
\definecolor{currentstroke}{rgb}{1.000000,1.000000,1.000000}%
\pgfsetstrokecolor{currentstroke}%
\pgfsetdash{}{0pt}%
\pgfpathmoveto{\pgfqpoint{0.617014in}{2.042604in}}%
\pgfpathlineto{\pgfqpoint{3.139144in}{2.042604in}}%
\pgfusepath{stroke}%
\end{pgfscope}%
\begin{pgfscope}%
\definecolor{textcolor}{rgb}{0.150000,0.150000,0.150000}%
\pgfsetstrokecolor{textcolor}%
\pgfsetfillcolor{textcolor}%
\pgftext[x=1.878079in,y=2.125938in,,base]{\color{textcolor}\sffamily\fontsize{11.000000}{13.200000}\selectfont (a)}%
\end{pgfscope}%
\begin{pgfscope}%
\pgfsetbuttcap%
\pgfsetmiterjoin%
\definecolor{currentfill}{rgb}{0.917647,0.917647,0.949020}%
\pgfsetfillcolor{currentfill}%
\pgfsetlinewidth{0.000000pt}%
\definecolor{currentstroke}{rgb}{0.000000,0.000000,0.000000}%
\pgfsetstrokecolor{currentstroke}%
\pgfsetstrokeopacity{0.000000}%
\pgfsetdash{}{0pt}%
\pgfpathmoveto{\pgfqpoint{3.836158in}{0.557870in}}%
\pgfpathlineto{\pgfqpoint{6.358287in}{0.557870in}}%
\pgfpathlineto{\pgfqpoint{6.358287in}{2.042604in}}%
\pgfpathlineto{\pgfqpoint{3.836158in}{2.042604in}}%
\pgfpathclose%
\pgfusepath{fill}%
\end{pgfscope}%
\begin{pgfscope}%
\pgfpathrectangle{\pgfqpoint{3.836158in}{0.557870in}}{\pgfqpoint{2.522130in}{1.484734in}}%
\pgfusepath{clip}%
\pgfsetroundcap%
\pgfsetroundjoin%
\pgfsetlinewidth{1.003750pt}%
\definecolor{currentstroke}{rgb}{1.000000,1.000000,1.000000}%
\pgfsetstrokecolor{currentstroke}%
\pgfsetdash{}{0pt}%
\pgfpathmoveto{\pgfqpoint{3.950800in}{0.557870in}}%
\pgfpathlineto{\pgfqpoint{3.950800in}{2.042604in}}%
\pgfusepath{stroke}%
\end{pgfscope}%
\begin{pgfscope}%
\definecolor{textcolor}{rgb}{0.150000,0.150000,0.150000}%
\pgfsetstrokecolor{textcolor}%
\pgfsetfillcolor{textcolor}%
\pgftext[x=3.950800in,y=0.425926in,,top]{\color{textcolor}\sffamily\fontsize{11.000000}{13.200000}\selectfont \(\displaystyle 0.00\)}%
\end{pgfscope}%
\begin{pgfscope}%
\pgfpathrectangle{\pgfqpoint{3.836158in}{0.557870in}}{\pgfqpoint{2.522130in}{1.484734in}}%
\pgfusepath{clip}%
\pgfsetroundcap%
\pgfsetroundjoin%
\pgfsetlinewidth{1.003750pt}%
\definecolor{currentstroke}{rgb}{1.000000,1.000000,1.000000}%
\pgfsetstrokecolor{currentstroke}%
\pgfsetdash{}{0pt}%
\pgfpathmoveto{\pgfqpoint{4.524011in}{0.557870in}}%
\pgfpathlineto{\pgfqpoint{4.524011in}{2.042604in}}%
\pgfusepath{stroke}%
\end{pgfscope}%
\begin{pgfscope}%
\definecolor{textcolor}{rgb}{0.150000,0.150000,0.150000}%
\pgfsetstrokecolor{textcolor}%
\pgfsetfillcolor{textcolor}%
\pgftext[x=4.524011in,y=0.425926in,,top]{\color{textcolor}\sffamily\fontsize{11.000000}{13.200000}\selectfont \(\displaystyle 0.25\)}%
\end{pgfscope}%
\begin{pgfscope}%
\pgfpathrectangle{\pgfqpoint{3.836158in}{0.557870in}}{\pgfqpoint{2.522130in}{1.484734in}}%
\pgfusepath{clip}%
\pgfsetroundcap%
\pgfsetroundjoin%
\pgfsetlinewidth{1.003750pt}%
\definecolor{currentstroke}{rgb}{1.000000,1.000000,1.000000}%
\pgfsetstrokecolor{currentstroke}%
\pgfsetdash{}{0pt}%
\pgfpathmoveto{\pgfqpoint{5.097222in}{0.557870in}}%
\pgfpathlineto{\pgfqpoint{5.097222in}{2.042604in}}%
\pgfusepath{stroke}%
\end{pgfscope}%
\begin{pgfscope}%
\definecolor{textcolor}{rgb}{0.150000,0.150000,0.150000}%
\pgfsetstrokecolor{textcolor}%
\pgfsetfillcolor{textcolor}%
\pgftext[x=5.097222in,y=0.425926in,,top]{\color{textcolor}\sffamily\fontsize{11.000000}{13.200000}\selectfont \(\displaystyle 0.50\)}%
\end{pgfscope}%
\begin{pgfscope}%
\pgfpathrectangle{\pgfqpoint{3.836158in}{0.557870in}}{\pgfqpoint{2.522130in}{1.484734in}}%
\pgfusepath{clip}%
\pgfsetroundcap%
\pgfsetroundjoin%
\pgfsetlinewidth{1.003750pt}%
\definecolor{currentstroke}{rgb}{1.000000,1.000000,1.000000}%
\pgfsetstrokecolor{currentstroke}%
\pgfsetdash{}{0pt}%
\pgfpathmoveto{\pgfqpoint{5.670434in}{0.557870in}}%
\pgfpathlineto{\pgfqpoint{5.670434in}{2.042604in}}%
\pgfusepath{stroke}%
\end{pgfscope}%
\begin{pgfscope}%
\definecolor{textcolor}{rgb}{0.150000,0.150000,0.150000}%
\pgfsetstrokecolor{textcolor}%
\pgfsetfillcolor{textcolor}%
\pgftext[x=5.670434in,y=0.425926in,,top]{\color{textcolor}\sffamily\fontsize{11.000000}{13.200000}\selectfont \(\displaystyle 0.75\)}%
\end{pgfscope}%
\begin{pgfscope}%
\pgfpathrectangle{\pgfqpoint{3.836158in}{0.557870in}}{\pgfqpoint{2.522130in}{1.484734in}}%
\pgfusepath{clip}%
\pgfsetroundcap%
\pgfsetroundjoin%
\pgfsetlinewidth{1.003750pt}%
\definecolor{currentstroke}{rgb}{1.000000,1.000000,1.000000}%
\pgfsetstrokecolor{currentstroke}%
\pgfsetdash{}{0pt}%
\pgfpathmoveto{\pgfqpoint{6.243645in}{0.557870in}}%
\pgfpathlineto{\pgfqpoint{6.243645in}{2.042604in}}%
\pgfusepath{stroke}%
\end{pgfscope}%
\begin{pgfscope}%
\definecolor{textcolor}{rgb}{0.150000,0.150000,0.150000}%
\pgfsetstrokecolor{textcolor}%
\pgfsetfillcolor{textcolor}%
\pgftext[x=6.243645in,y=0.425926in,,top]{\color{textcolor}\sffamily\fontsize{11.000000}{13.200000}\selectfont \(\displaystyle 1.00\)}%
\end{pgfscope}%
\begin{pgfscope}%
\definecolor{textcolor}{rgb}{0.150000,0.150000,0.150000}%
\pgfsetstrokecolor{textcolor}%
\pgfsetfillcolor{textcolor}%
\pgftext[x=5.097222in,y=0.235185in,,top]{\color{textcolor}\sffamily\fontsize{11.000000}{13.200000}\selectfont Specificity}%
\end{pgfscope}%
\begin{pgfscope}%
\pgfpathrectangle{\pgfqpoint{3.836158in}{0.557870in}}{\pgfqpoint{2.522130in}{1.484734in}}%
\pgfusepath{clip}%
\pgfsetroundcap%
\pgfsetroundjoin%
\pgfsetlinewidth{1.003750pt}%
\definecolor{currentstroke}{rgb}{1.000000,1.000000,1.000000}%
\pgfsetstrokecolor{currentstroke}%
\pgfsetdash{}{0pt}%
\pgfpathmoveto{\pgfqpoint{3.836158in}{0.625358in}}%
\pgfpathlineto{\pgfqpoint{6.358287in}{0.625358in}}%
\pgfusepath{stroke}%
\end{pgfscope}%
\begin{pgfscope}%
\definecolor{textcolor}{rgb}{0.150000,0.150000,0.150000}%
\pgfsetstrokecolor{textcolor}%
\pgfsetfillcolor{textcolor}%
\pgftext[x=3.509884in,y=0.572552in,left,base]{\color{textcolor}\sffamily\fontsize{11.000000}{13.200000}\selectfont \(\displaystyle 0.0\)}%
\end{pgfscope}%
\begin{pgfscope}%
\pgfpathrectangle{\pgfqpoint{3.836158in}{0.557870in}}{\pgfqpoint{2.522130in}{1.484734in}}%
\pgfusepath{clip}%
\pgfsetroundcap%
\pgfsetroundjoin%
\pgfsetlinewidth{1.003750pt}%
\definecolor{currentstroke}{rgb}{1.000000,1.000000,1.000000}%
\pgfsetstrokecolor{currentstroke}%
\pgfsetdash{}{0pt}%
\pgfpathmoveto{\pgfqpoint{3.836158in}{1.300237in}}%
\pgfpathlineto{\pgfqpoint{6.358287in}{1.300237in}}%
\pgfusepath{stroke}%
\end{pgfscope}%
\begin{pgfscope}%
\definecolor{textcolor}{rgb}{0.150000,0.150000,0.150000}%
\pgfsetstrokecolor{textcolor}%
\pgfsetfillcolor{textcolor}%
\pgftext[x=3.509884in,y=1.247431in,left,base]{\color{textcolor}\sffamily\fontsize{11.000000}{13.200000}\selectfont \(\displaystyle 0.5\)}%
\end{pgfscope}%
\begin{pgfscope}%
\pgfpathrectangle{\pgfqpoint{3.836158in}{0.557870in}}{\pgfqpoint{2.522130in}{1.484734in}}%
\pgfusepath{clip}%
\pgfsetroundcap%
\pgfsetroundjoin%
\pgfsetlinewidth{1.003750pt}%
\definecolor{currentstroke}{rgb}{1.000000,1.000000,1.000000}%
\pgfsetstrokecolor{currentstroke}%
\pgfsetdash{}{0pt}%
\pgfpathmoveto{\pgfqpoint{3.836158in}{1.975116in}}%
\pgfpathlineto{\pgfqpoint{6.358287in}{1.975116in}}%
\pgfusepath{stroke}%
\end{pgfscope}%
\begin{pgfscope}%
\definecolor{textcolor}{rgb}{0.150000,0.150000,0.150000}%
\pgfsetstrokecolor{textcolor}%
\pgfsetfillcolor{textcolor}%
\pgftext[x=3.509884in,y=1.922310in,left,base]{\color{textcolor}\sffamily\fontsize{11.000000}{13.200000}\selectfont \(\displaystyle 1.0\)}%
\end{pgfscope}%
\begin{pgfscope}%
\definecolor{textcolor}{rgb}{0.150000,0.150000,0.150000}%
\pgfsetstrokecolor{textcolor}%
\pgfsetfillcolor{textcolor}%
\pgftext[x=3.454329in,y=1.300237in,,bottom,rotate=90.000000]{\color{textcolor}\sffamily\fontsize{11.000000}{13.200000}\selectfont Sensitivity}%
\end{pgfscope}%
\begin{pgfscope}%
\pgfpathrectangle{\pgfqpoint{3.836158in}{0.557870in}}{\pgfqpoint{2.522130in}{1.484734in}}%
\pgfusepath{clip}%
\pgfsetbuttcap%
\pgfsetroundjoin%
\definecolor{currentfill}{rgb}{0.298039,0.447059,0.690196}%
\pgfsetfillcolor{currentfill}%
\pgfsetlinewidth{1.003750pt}%
\definecolor{currentstroke}{rgb}{0.298039,0.447059,0.690196}%
\pgfsetstrokecolor{currentstroke}%
\pgfsetdash{}{0pt}%
\pgfpathmoveto{\pgfqpoint{4.468539in}{1.730074in}}%
\pgfpathcurveto{\pgfqpoint{4.476775in}{1.730074in}}{\pgfqpoint{4.484675in}{1.733346in}}{\pgfqpoint{4.490499in}{1.739170in}}%
\pgfpathcurveto{\pgfqpoint{4.496323in}{1.744994in}}{\pgfqpoint{4.499596in}{1.752894in}}{\pgfqpoint{4.499596in}{1.761130in}}%
\pgfpathcurveto{\pgfqpoint{4.499596in}{1.769367in}}{\pgfqpoint{4.496323in}{1.777267in}}{\pgfqpoint{4.490499in}{1.783090in}}%
\pgfpathcurveto{\pgfqpoint{4.484675in}{1.788914in}}{\pgfqpoint{4.476775in}{1.792187in}}{\pgfqpoint{4.468539in}{1.792187in}}%
\pgfpathcurveto{\pgfqpoint{4.460303in}{1.792187in}}{\pgfqpoint{4.452403in}{1.788914in}}{\pgfqpoint{4.446579in}{1.783090in}}%
\pgfpathcurveto{\pgfqpoint{4.440755in}{1.777267in}}{\pgfqpoint{4.437483in}{1.769367in}}{\pgfqpoint{4.437483in}{1.761130in}}%
\pgfpathcurveto{\pgfqpoint{4.437483in}{1.752894in}}{\pgfqpoint{4.440755in}{1.744994in}}{\pgfqpoint{4.446579in}{1.739170in}}%
\pgfpathcurveto{\pgfqpoint{4.452403in}{1.733346in}}{\pgfqpoint{4.460303in}{1.730074in}}{\pgfqpoint{4.468539in}{1.730074in}}%
\pgfpathclose%
\pgfusepath{stroke,fill}%
\end{pgfscope}%
\begin{pgfscope}%
\pgfpathrectangle{\pgfqpoint{3.836158in}{0.557870in}}{\pgfqpoint{2.522130in}{1.484734in}}%
\pgfusepath{clip}%
\pgfsetbuttcap%
\pgfsetroundjoin%
\definecolor{currentfill}{rgb}{0.298039,0.447059,0.690196}%
\pgfsetfillcolor{currentfill}%
\pgfsetlinewidth{1.003750pt}%
\definecolor{currentstroke}{rgb}{0.298039,0.447059,0.690196}%
\pgfsetstrokecolor{currentstroke}%
\pgfsetdash{}{0pt}%
\pgfpathmoveto{\pgfqpoint{6.021757in}{1.269181in}}%
\pgfpathcurveto{\pgfqpoint{6.029993in}{1.269181in}}{\pgfqpoint{6.037893in}{1.272453in}}{\pgfqpoint{6.043717in}{1.278277in}}%
\pgfpathcurveto{\pgfqpoint{6.049541in}{1.284101in}}{\pgfqpoint{6.052813in}{1.292001in}}{\pgfqpoint{6.052813in}{1.300237in}}%
\pgfpathcurveto{\pgfqpoint{6.052813in}{1.308474in}}{\pgfqpoint{6.049541in}{1.316374in}}{\pgfqpoint{6.043717in}{1.322197in}}%
\pgfpathcurveto{\pgfqpoint{6.037893in}{1.328021in}}{\pgfqpoint{6.029993in}{1.331294in}}{\pgfqpoint{6.021757in}{1.331294in}}%
\pgfpathcurveto{\pgfqpoint{6.013520in}{1.331294in}}{\pgfqpoint{6.005620in}{1.328021in}}{\pgfqpoint{5.999796in}{1.322197in}}%
\pgfpathcurveto{\pgfqpoint{5.993973in}{1.316374in}}{\pgfqpoint{5.990700in}{1.308474in}}{\pgfqpoint{5.990700in}{1.300237in}}%
\pgfpathcurveto{\pgfqpoint{5.990700in}{1.292001in}}{\pgfqpoint{5.993973in}{1.284101in}}{\pgfqpoint{5.999796in}{1.278277in}}%
\pgfpathcurveto{\pgfqpoint{6.005620in}{1.272453in}}{\pgfqpoint{6.013520in}{1.269181in}}{\pgfqpoint{6.021757in}{1.269181in}}%
\pgfpathclose%
\pgfusepath{stroke,fill}%
\end{pgfscope}%
\begin{pgfscope}%
\pgfpathrectangle{\pgfqpoint{3.836158in}{0.557870in}}{\pgfqpoint{2.522130in}{1.484734in}}%
\pgfusepath{clip}%
\pgfsetbuttcap%
\pgfsetroundjoin%
\definecolor{currentfill}{rgb}{0.298039,0.447059,0.690196}%
\pgfsetfillcolor{currentfill}%
\pgfsetlinewidth{1.003750pt}%
\definecolor{currentstroke}{rgb}{0.298039,0.447059,0.690196}%
\pgfsetstrokecolor{currentstroke}%
\pgfsetdash{}{0pt}%
\pgfpathmoveto{\pgfqpoint{3.950800in}{1.935624in}}%
\pgfpathcurveto{\pgfqpoint{3.959036in}{1.935624in}}{\pgfqpoint{3.966936in}{1.938896in}}{\pgfqpoint{3.972760in}{1.944720in}}%
\pgfpathcurveto{\pgfqpoint{3.978584in}{1.950544in}}{\pgfqpoint{3.981856in}{1.958444in}}{\pgfqpoint{3.981856in}{1.966680in}}%
\pgfpathcurveto{\pgfqpoint{3.981856in}{1.974917in}}{\pgfqpoint{3.978584in}{1.982817in}}{\pgfqpoint{3.972760in}{1.988641in}}%
\pgfpathcurveto{\pgfqpoint{3.966936in}{1.994464in}}{\pgfqpoint{3.959036in}{1.997737in}}{\pgfqpoint{3.950800in}{1.997737in}}%
\pgfpathcurveto{\pgfqpoint{3.942564in}{1.997737in}}{\pgfqpoint{3.934664in}{1.994464in}}{\pgfqpoint{3.928840in}{1.988641in}}%
\pgfpathcurveto{\pgfqpoint{3.923016in}{1.982817in}}{\pgfqpoint{3.919743in}{1.974917in}}{\pgfqpoint{3.919743in}{1.966680in}}%
\pgfpathcurveto{\pgfqpoint{3.919743in}{1.958444in}}{\pgfqpoint{3.923016in}{1.950544in}}{\pgfqpoint{3.928840in}{1.944720in}}%
\pgfpathcurveto{\pgfqpoint{3.934664in}{1.938896in}}{\pgfqpoint{3.942564in}{1.935624in}}{\pgfqpoint{3.950800in}{1.935624in}}%
\pgfpathclose%
\pgfusepath{stroke,fill}%
\end{pgfscope}%
\begin{pgfscope}%
\pgfpathrectangle{\pgfqpoint{3.836158in}{0.557870in}}{\pgfqpoint{2.522130in}{1.484734in}}%
\pgfusepath{clip}%
\pgfsetbuttcap%
\pgfsetroundjoin%
\definecolor{currentfill}{rgb}{0.298039,0.447059,0.690196}%
\pgfsetfillcolor{currentfill}%
\pgfsetlinewidth{1.003750pt}%
\definecolor{currentstroke}{rgb}{0.298039,0.447059,0.690196}%
\pgfsetstrokecolor{currentstroke}%
\pgfsetdash{}{0pt}%
\pgfpathmoveto{\pgfqpoint{6.079870in}{1.277617in}}%
\pgfpathcurveto{\pgfqpoint{6.088107in}{1.277617in}}{\pgfqpoint{6.096007in}{1.280889in}}{\pgfqpoint{6.101831in}{1.286713in}}%
\pgfpathcurveto{\pgfqpoint{6.107655in}{1.292537in}}{\pgfqpoint{6.110927in}{1.300437in}}{\pgfqpoint{6.110927in}{1.308673in}}%
\pgfpathcurveto{\pgfqpoint{6.110927in}{1.316909in}}{\pgfqpoint{6.107655in}{1.324810in}}{\pgfqpoint{6.101831in}{1.330633in}}%
\pgfpathcurveto{\pgfqpoint{6.096007in}{1.336457in}}{\pgfqpoint{6.088107in}{1.339730in}}{\pgfqpoint{6.079870in}{1.339730in}}%
\pgfpathcurveto{\pgfqpoint{6.071634in}{1.339730in}}{\pgfqpoint{6.063734in}{1.336457in}}{\pgfqpoint{6.057910in}{1.330633in}}%
\pgfpathcurveto{\pgfqpoint{6.052086in}{1.324810in}}{\pgfqpoint{6.048814in}{1.316909in}}{\pgfqpoint{6.048814in}{1.308673in}}%
\pgfpathcurveto{\pgfqpoint{6.048814in}{1.300437in}}{\pgfqpoint{6.052086in}{1.292537in}}{\pgfqpoint{6.057910in}{1.286713in}}%
\pgfpathcurveto{\pgfqpoint{6.063734in}{1.280889in}}{\pgfqpoint{6.071634in}{1.277617in}}{\pgfqpoint{6.079870in}{1.277617in}}%
\pgfpathclose%
\pgfusepath{stroke,fill}%
\end{pgfscope}%
\begin{pgfscope}%
\pgfpathrectangle{\pgfqpoint{3.836158in}{0.557870in}}{\pgfqpoint{2.522130in}{1.484734in}}%
\pgfusepath{clip}%
\pgfsetbuttcap%
\pgfsetroundjoin%
\definecolor{currentfill}{rgb}{0.298039,0.447059,0.690196}%
\pgfsetfillcolor{currentfill}%
\pgfsetlinewidth{1.003750pt}%
\definecolor{currentstroke}{rgb}{0.298039,0.447059,0.690196}%
\pgfsetstrokecolor{currentstroke}%
\pgfsetdash{}{0pt}%
\pgfpathmoveto{\pgfqpoint{6.079870in}{1.252309in}}%
\pgfpathcurveto{\pgfqpoint{6.088107in}{1.252309in}}{\pgfqpoint{6.096007in}{1.255581in}}{\pgfqpoint{6.101831in}{1.261405in}}%
\pgfpathcurveto{\pgfqpoint{6.107655in}{1.267229in}}{\pgfqpoint{6.110927in}{1.275129in}}{\pgfqpoint{6.110927in}{1.283365in}}%
\pgfpathcurveto{\pgfqpoint{6.110927in}{1.291602in}}{\pgfqpoint{6.107655in}{1.299502in}}{\pgfqpoint{6.101831in}{1.305326in}}%
\pgfpathcurveto{\pgfqpoint{6.096007in}{1.311149in}}{\pgfqpoint{6.088107in}{1.314422in}}{\pgfqpoint{6.079870in}{1.314422in}}%
\pgfpathcurveto{\pgfqpoint{6.071634in}{1.314422in}}{\pgfqpoint{6.063734in}{1.311149in}}{\pgfqpoint{6.057910in}{1.305326in}}%
\pgfpathcurveto{\pgfqpoint{6.052086in}{1.299502in}}{\pgfqpoint{6.048814in}{1.291602in}}{\pgfqpoint{6.048814in}{1.283365in}}%
\pgfpathcurveto{\pgfqpoint{6.048814in}{1.275129in}}{\pgfqpoint{6.052086in}{1.267229in}}{\pgfqpoint{6.057910in}{1.261405in}}%
\pgfpathcurveto{\pgfqpoint{6.063734in}{1.255581in}}{\pgfqpoint{6.071634in}{1.252309in}}{\pgfqpoint{6.079870in}{1.252309in}}%
\pgfpathclose%
\pgfusepath{stroke,fill}%
\end{pgfscope}%
\begin{pgfscope}%
\pgfpathrectangle{\pgfqpoint{3.836158in}{0.557870in}}{\pgfqpoint{2.522130in}{1.484734in}}%
\pgfusepath{clip}%
\pgfsetbuttcap%
\pgfsetroundjoin%
\definecolor{currentfill}{rgb}{0.298039,0.447059,0.690196}%
\pgfsetfillcolor{currentfill}%
\pgfsetlinewidth{1.003750pt}%
\definecolor{currentstroke}{rgb}{0.298039,0.447059,0.690196}%
\pgfsetstrokecolor{currentstroke}%
\pgfsetdash{}{0pt}%
\pgfpathmoveto{\pgfqpoint{6.079870in}{1.227001in}}%
\pgfpathcurveto{\pgfqpoint{6.088107in}{1.227001in}}{\pgfqpoint{6.096007in}{1.230273in}}{\pgfqpoint{6.101831in}{1.236097in}}%
\pgfpathcurveto{\pgfqpoint{6.107655in}{1.241921in}}{\pgfqpoint{6.110927in}{1.249821in}}{\pgfqpoint{6.110927in}{1.258057in}}%
\pgfpathcurveto{\pgfqpoint{6.110927in}{1.266294in}}{\pgfqpoint{6.107655in}{1.274194in}}{\pgfqpoint{6.101831in}{1.280018in}}%
\pgfpathcurveto{\pgfqpoint{6.096007in}{1.285841in}}{\pgfqpoint{6.088107in}{1.289114in}}{\pgfqpoint{6.079870in}{1.289114in}}%
\pgfpathcurveto{\pgfqpoint{6.071634in}{1.289114in}}{\pgfqpoint{6.063734in}{1.285841in}}{\pgfqpoint{6.057910in}{1.280018in}}%
\pgfpathcurveto{\pgfqpoint{6.052086in}{1.274194in}}{\pgfqpoint{6.048814in}{1.266294in}}{\pgfqpoint{6.048814in}{1.258057in}}%
\pgfpathcurveto{\pgfqpoint{6.048814in}{1.249821in}}{\pgfqpoint{6.052086in}{1.241921in}}{\pgfqpoint{6.057910in}{1.236097in}}%
\pgfpathcurveto{\pgfqpoint{6.063734in}{1.230273in}}{\pgfqpoint{6.071634in}{1.227001in}}{\pgfqpoint{6.079870in}{1.227001in}}%
\pgfpathclose%
\pgfusepath{stroke,fill}%
\end{pgfscope}%
\begin{pgfscope}%
\pgfpathrectangle{\pgfqpoint{3.836158in}{0.557870in}}{\pgfqpoint{2.522130in}{1.484734in}}%
\pgfusepath{clip}%
\pgfsetbuttcap%
\pgfsetroundjoin%
\definecolor{currentfill}{rgb}{0.298039,0.447059,0.690196}%
\pgfsetfillcolor{currentfill}%
\pgfsetlinewidth{1.003750pt}%
\definecolor{currentstroke}{rgb}{0.298039,0.447059,0.690196}%
\pgfsetstrokecolor{currentstroke}%
\pgfsetdash{}{0pt}%
\pgfpathmoveto{\pgfqpoint{3.950800in}{1.900798in}}%
\pgfpathcurveto{\pgfqpoint{3.959036in}{1.900798in}}{\pgfqpoint{3.966936in}{1.904071in}}{\pgfqpoint{3.972760in}{1.909895in}}%
\pgfpathcurveto{\pgfqpoint{3.978584in}{1.915718in}}{\pgfqpoint{3.981856in}{1.923619in}}{\pgfqpoint{3.981856in}{1.931855in}}%
\pgfpathcurveto{\pgfqpoint{3.981856in}{1.940091in}}{\pgfqpoint{3.978584in}{1.947991in}}{\pgfqpoint{3.972760in}{1.953815in}}%
\pgfpathcurveto{\pgfqpoint{3.966936in}{1.959639in}}{\pgfqpoint{3.959036in}{1.962911in}}{\pgfqpoint{3.950800in}{1.962911in}}%
\pgfpathcurveto{\pgfqpoint{3.942564in}{1.962911in}}{\pgfqpoint{3.934664in}{1.959639in}}{\pgfqpoint{3.928840in}{1.953815in}}%
\pgfpathcurveto{\pgfqpoint{3.923016in}{1.947991in}}{\pgfqpoint{3.919743in}{1.940091in}}{\pgfqpoint{3.919743in}{1.931855in}}%
\pgfpathcurveto{\pgfqpoint{3.919743in}{1.923619in}}{\pgfqpoint{3.923016in}{1.915718in}}{\pgfqpoint{3.928840in}{1.909895in}}%
\pgfpathcurveto{\pgfqpoint{3.934664in}{1.904071in}}{\pgfqpoint{3.942564in}{1.900798in}}{\pgfqpoint{3.950800in}{1.900798in}}%
\pgfpathclose%
\pgfusepath{stroke,fill}%
\end{pgfscope}%
\begin{pgfscope}%
\pgfpathrectangle{\pgfqpoint{3.836158in}{0.557870in}}{\pgfqpoint{2.522130in}{1.484734in}}%
\pgfusepath{clip}%
\pgfsetbuttcap%
\pgfsetroundjoin%
\definecolor{currentfill}{rgb}{0.298039,0.447059,0.690196}%
\pgfsetfillcolor{currentfill}%
\pgfsetlinewidth{1.003750pt}%
\definecolor{currentstroke}{rgb}{0.298039,0.447059,0.690196}%
\pgfsetstrokecolor{currentstroke}%
\pgfsetdash{}{0pt}%
\pgfpathmoveto{\pgfqpoint{4.114575in}{1.926755in}}%
\pgfpathcurveto{\pgfqpoint{4.122811in}{1.926755in}}{\pgfqpoint{4.130711in}{1.930027in}}{\pgfqpoint{4.136535in}{1.935851in}}%
\pgfpathcurveto{\pgfqpoint{4.142359in}{1.941675in}}{\pgfqpoint{4.145631in}{1.949575in}}{\pgfqpoint{4.145631in}{1.957812in}}%
\pgfpathcurveto{\pgfqpoint{4.145631in}{1.966048in}}{\pgfqpoint{4.142359in}{1.973948in}}{\pgfqpoint{4.136535in}{1.979772in}}%
\pgfpathcurveto{\pgfqpoint{4.130711in}{1.985596in}}{\pgfqpoint{4.122811in}{1.988868in}}{\pgfqpoint{4.114575in}{1.988868in}}%
\pgfpathcurveto{\pgfqpoint{4.106338in}{1.988868in}}{\pgfqpoint{4.098438in}{1.985596in}}{\pgfqpoint{4.092614in}{1.979772in}}%
\pgfpathcurveto{\pgfqpoint{4.086790in}{1.973948in}}{\pgfqpoint{4.083518in}{1.966048in}}{\pgfqpoint{4.083518in}{1.957812in}}%
\pgfpathcurveto{\pgfqpoint{4.083518in}{1.949575in}}{\pgfqpoint{4.086790in}{1.941675in}}{\pgfqpoint{4.092614in}{1.935851in}}%
\pgfpathcurveto{\pgfqpoint{4.098438in}{1.930027in}}{\pgfqpoint{4.106338in}{1.926755in}}{\pgfqpoint{4.114575in}{1.926755in}}%
\pgfpathclose%
\pgfusepath{stroke,fill}%
\end{pgfscope}%
\begin{pgfscope}%
\pgfpathrectangle{\pgfqpoint{3.836158in}{0.557870in}}{\pgfqpoint{2.522130in}{1.484734in}}%
\pgfusepath{clip}%
\pgfsetbuttcap%
\pgfsetroundjoin%
\definecolor{currentfill}{rgb}{0.298039,0.447059,0.690196}%
\pgfsetfillcolor{currentfill}%
\pgfsetlinewidth{1.003750pt}%
\definecolor{currentstroke}{rgb}{0.298039,0.447059,0.690196}%
\pgfsetstrokecolor{currentstroke}%
\pgfsetdash{}{0pt}%
\pgfpathmoveto{\pgfqpoint{6.079870in}{1.234572in}}%
\pgfpathcurveto{\pgfqpoint{6.088107in}{1.234572in}}{\pgfqpoint{6.096007in}{1.237844in}}{\pgfqpoint{6.101831in}{1.243668in}}%
\pgfpathcurveto{\pgfqpoint{6.107655in}{1.249492in}}{\pgfqpoint{6.110927in}{1.257392in}}{\pgfqpoint{6.110927in}{1.265628in}}%
\pgfpathcurveto{\pgfqpoint{6.110927in}{1.273864in}}{\pgfqpoint{6.107655in}{1.281764in}}{\pgfqpoint{6.101831in}{1.287588in}}%
\pgfpathcurveto{\pgfqpoint{6.096007in}{1.293412in}}{\pgfqpoint{6.088107in}{1.296685in}}{\pgfqpoint{6.079870in}{1.296685in}}%
\pgfpathcurveto{\pgfqpoint{6.071634in}{1.296685in}}{\pgfqpoint{6.063734in}{1.293412in}}{\pgfqpoint{6.057910in}{1.287588in}}%
\pgfpathcurveto{\pgfqpoint{6.052086in}{1.281764in}}{\pgfqpoint{6.048814in}{1.273864in}}{\pgfqpoint{6.048814in}{1.265628in}}%
\pgfpathcurveto{\pgfqpoint{6.048814in}{1.257392in}}{\pgfqpoint{6.052086in}{1.249492in}}{\pgfqpoint{6.057910in}{1.243668in}}%
\pgfpathcurveto{\pgfqpoint{6.063734in}{1.237844in}}{\pgfqpoint{6.071634in}{1.234572in}}{\pgfqpoint{6.079870in}{1.234572in}}%
\pgfpathclose%
\pgfusepath{stroke,fill}%
\end{pgfscope}%
\begin{pgfscope}%
\pgfpathrectangle{\pgfqpoint{3.836158in}{0.557870in}}{\pgfqpoint{2.522130in}{1.484734in}}%
\pgfusepath{clip}%
\pgfsetbuttcap%
\pgfsetroundjoin%
\definecolor{currentfill}{rgb}{0.298039,0.447059,0.690196}%
\pgfsetfillcolor{currentfill}%
\pgfsetlinewidth{1.003750pt}%
\definecolor{currentstroke}{rgb}{0.298039,0.447059,0.690196}%
\pgfsetstrokecolor{currentstroke}%
\pgfsetdash{}{0pt}%
\pgfpathmoveto{\pgfqpoint{3.950800in}{1.935830in}}%
\pgfpathcurveto{\pgfqpoint{3.959036in}{1.935830in}}{\pgfqpoint{3.966936in}{1.939102in}}{\pgfqpoint{3.972760in}{1.944926in}}%
\pgfpathcurveto{\pgfqpoint{3.978584in}{1.950750in}}{\pgfqpoint{3.981856in}{1.958650in}}{\pgfqpoint{3.981856in}{1.966886in}}%
\pgfpathcurveto{\pgfqpoint{3.981856in}{1.975122in}}{\pgfqpoint{3.978584in}{1.983022in}}{\pgfqpoint{3.972760in}{1.988846in}}%
\pgfpathcurveto{\pgfqpoint{3.966936in}{1.994670in}}{\pgfqpoint{3.959036in}{1.997943in}}{\pgfqpoint{3.950800in}{1.997943in}}%
\pgfpathcurveto{\pgfqpoint{3.942564in}{1.997943in}}{\pgfqpoint{3.934664in}{1.994670in}}{\pgfqpoint{3.928840in}{1.988846in}}%
\pgfpathcurveto{\pgfqpoint{3.923016in}{1.983022in}}{\pgfqpoint{3.919743in}{1.975122in}}{\pgfqpoint{3.919743in}{1.966886in}}%
\pgfpathcurveto{\pgfqpoint{3.919743in}{1.958650in}}{\pgfqpoint{3.923016in}{1.950750in}}{\pgfqpoint{3.928840in}{1.944926in}}%
\pgfpathcurveto{\pgfqpoint{3.934664in}{1.939102in}}{\pgfqpoint{3.942564in}{1.935830in}}{\pgfqpoint{3.950800in}{1.935830in}}%
\pgfpathclose%
\pgfusepath{stroke,fill}%
\end{pgfscope}%
\begin{pgfscope}%
\pgfpathrectangle{\pgfqpoint{3.836158in}{0.557870in}}{\pgfqpoint{2.522130in}{1.484734in}}%
\pgfusepath{clip}%
\pgfsetbuttcap%
\pgfsetroundjoin%
\definecolor{currentfill}{rgb}{0.298039,0.447059,0.690196}%
\pgfsetfillcolor{currentfill}%
\pgfsetlinewidth{1.003750pt}%
\definecolor{currentstroke}{rgb}{0.298039,0.447059,0.690196}%
\pgfsetstrokecolor{currentstroke}%
\pgfsetdash{}{0pt}%
\pgfpathmoveto{\pgfqpoint{6.095719in}{1.442016in}}%
\pgfpathcurveto{\pgfqpoint{6.103956in}{1.442016in}}{\pgfqpoint{6.111856in}{1.445288in}}{\pgfqpoint{6.117680in}{1.451112in}}%
\pgfpathcurveto{\pgfqpoint{6.123504in}{1.456936in}}{\pgfqpoint{6.126776in}{1.464836in}}{\pgfqpoint{6.126776in}{1.473072in}}%
\pgfpathcurveto{\pgfqpoint{6.126776in}{1.481308in}}{\pgfqpoint{6.123504in}{1.489208in}}{\pgfqpoint{6.117680in}{1.495032in}}%
\pgfpathcurveto{\pgfqpoint{6.111856in}{1.500856in}}{\pgfqpoint{6.103956in}{1.504129in}}{\pgfqpoint{6.095719in}{1.504129in}}%
\pgfpathcurveto{\pgfqpoint{6.087483in}{1.504129in}}{\pgfqpoint{6.079583in}{1.500856in}}{\pgfqpoint{6.073759in}{1.495032in}}%
\pgfpathcurveto{\pgfqpoint{6.067935in}{1.489208in}}{\pgfqpoint{6.064663in}{1.481308in}}{\pgfqpoint{6.064663in}{1.473072in}}%
\pgfpathcurveto{\pgfqpoint{6.064663in}{1.464836in}}{\pgfqpoint{6.067935in}{1.456936in}}{\pgfqpoint{6.073759in}{1.451112in}}%
\pgfpathcurveto{\pgfqpoint{6.079583in}{1.445288in}}{\pgfqpoint{6.087483in}{1.442016in}}{\pgfqpoint{6.095719in}{1.442016in}}%
\pgfpathclose%
\pgfusepath{stroke,fill}%
\end{pgfscope}%
\begin{pgfscope}%
\pgfpathrectangle{\pgfqpoint{3.836158in}{0.557870in}}{\pgfqpoint{2.522130in}{1.484734in}}%
\pgfusepath{clip}%
\pgfsetbuttcap%
\pgfsetroundjoin%
\definecolor{currentfill}{rgb}{0.298039,0.447059,0.690196}%
\pgfsetfillcolor{currentfill}%
\pgfsetlinewidth{1.003750pt}%
\definecolor{currentstroke}{rgb}{0.298039,0.447059,0.690196}%
\pgfsetstrokecolor{currentstroke}%
\pgfsetdash{}{0pt}%
\pgfpathmoveto{\pgfqpoint{6.095719in}{1.532548in}}%
\pgfpathcurveto{\pgfqpoint{6.103956in}{1.532548in}}{\pgfqpoint{6.111856in}{1.535820in}}{\pgfqpoint{6.117680in}{1.541644in}}%
\pgfpathcurveto{\pgfqpoint{6.123504in}{1.547468in}}{\pgfqpoint{6.126776in}{1.555368in}}{\pgfqpoint{6.126776in}{1.563605in}}%
\pgfpathcurveto{\pgfqpoint{6.126776in}{1.571841in}}{\pgfqpoint{6.123504in}{1.579741in}}{\pgfqpoint{6.117680in}{1.585565in}}%
\pgfpathcurveto{\pgfqpoint{6.111856in}{1.591389in}}{\pgfqpoint{6.103956in}{1.594661in}}{\pgfqpoint{6.095719in}{1.594661in}}%
\pgfpathcurveto{\pgfqpoint{6.087483in}{1.594661in}}{\pgfqpoint{6.079583in}{1.591389in}}{\pgfqpoint{6.073759in}{1.585565in}}%
\pgfpathcurveto{\pgfqpoint{6.067935in}{1.579741in}}{\pgfqpoint{6.064663in}{1.571841in}}{\pgfqpoint{6.064663in}{1.563605in}}%
\pgfpathcurveto{\pgfqpoint{6.064663in}{1.555368in}}{\pgfqpoint{6.067935in}{1.547468in}}{\pgfqpoint{6.073759in}{1.541644in}}%
\pgfpathcurveto{\pgfqpoint{6.079583in}{1.535820in}}{\pgfqpoint{6.087483in}{1.532548in}}{\pgfqpoint{6.095719in}{1.532548in}}%
\pgfpathclose%
\pgfusepath{stroke,fill}%
\end{pgfscope}%
\begin{pgfscope}%
\pgfpathrectangle{\pgfqpoint{3.836158in}{0.557870in}}{\pgfqpoint{2.522130in}{1.484734in}}%
\pgfusepath{clip}%
\pgfsetbuttcap%
\pgfsetroundjoin%
\definecolor{currentfill}{rgb}{0.298039,0.447059,0.690196}%
\pgfsetfillcolor{currentfill}%
\pgfsetlinewidth{1.003750pt}%
\definecolor{currentstroke}{rgb}{0.298039,0.447059,0.690196}%
\pgfsetstrokecolor{currentstroke}%
\pgfsetdash{}{0pt}%
\pgfpathmoveto{\pgfqpoint{6.095719in}{1.565469in}}%
\pgfpathcurveto{\pgfqpoint{6.103956in}{1.565469in}}{\pgfqpoint{6.111856in}{1.568741in}}{\pgfqpoint{6.117680in}{1.574565in}}%
\pgfpathcurveto{\pgfqpoint{6.123504in}{1.580389in}}{\pgfqpoint{6.126776in}{1.588289in}}{\pgfqpoint{6.126776in}{1.596526in}}%
\pgfpathcurveto{\pgfqpoint{6.126776in}{1.604762in}}{\pgfqpoint{6.123504in}{1.612662in}}{\pgfqpoint{6.117680in}{1.618486in}}%
\pgfpathcurveto{\pgfqpoint{6.111856in}{1.624310in}}{\pgfqpoint{6.103956in}{1.627582in}}{\pgfqpoint{6.095719in}{1.627582in}}%
\pgfpathcurveto{\pgfqpoint{6.087483in}{1.627582in}}{\pgfqpoint{6.079583in}{1.624310in}}{\pgfqpoint{6.073759in}{1.618486in}}%
\pgfpathcurveto{\pgfqpoint{6.067935in}{1.612662in}}{\pgfqpoint{6.064663in}{1.604762in}}{\pgfqpoint{6.064663in}{1.596526in}}%
\pgfpathcurveto{\pgfqpoint{6.064663in}{1.588289in}}{\pgfqpoint{6.067935in}{1.580389in}}{\pgfqpoint{6.073759in}{1.574565in}}%
\pgfpathcurveto{\pgfqpoint{6.079583in}{1.568741in}}{\pgfqpoint{6.087483in}{1.565469in}}{\pgfqpoint{6.095719in}{1.565469in}}%
\pgfpathclose%
\pgfusepath{stroke,fill}%
\end{pgfscope}%
\begin{pgfscope}%
\pgfpathrectangle{\pgfqpoint{3.836158in}{0.557870in}}{\pgfqpoint{2.522130in}{1.484734in}}%
\pgfusepath{clip}%
\pgfsetbuttcap%
\pgfsetroundjoin%
\definecolor{currentfill}{rgb}{0.298039,0.447059,0.690196}%
\pgfsetfillcolor{currentfill}%
\pgfsetlinewidth{1.003750pt}%
\definecolor{currentstroke}{rgb}{0.298039,0.447059,0.690196}%
\pgfsetstrokecolor{currentstroke}%
\pgfsetdash{}{0pt}%
\pgfpathmoveto{\pgfqpoint{3.950800in}{1.935624in}}%
\pgfpathcurveto{\pgfqpoint{3.959036in}{1.935624in}}{\pgfqpoint{3.966936in}{1.938896in}}{\pgfqpoint{3.972760in}{1.944720in}}%
\pgfpathcurveto{\pgfqpoint{3.978584in}{1.950544in}}{\pgfqpoint{3.981856in}{1.958444in}}{\pgfqpoint{3.981856in}{1.966680in}}%
\pgfpathcurveto{\pgfqpoint{3.981856in}{1.974917in}}{\pgfqpoint{3.978584in}{1.982817in}}{\pgfqpoint{3.972760in}{1.988641in}}%
\pgfpathcurveto{\pgfqpoint{3.966936in}{1.994464in}}{\pgfqpoint{3.959036in}{1.997737in}}{\pgfqpoint{3.950800in}{1.997737in}}%
\pgfpathcurveto{\pgfqpoint{3.942564in}{1.997737in}}{\pgfqpoint{3.934664in}{1.994464in}}{\pgfqpoint{3.928840in}{1.988641in}}%
\pgfpathcurveto{\pgfqpoint{3.923016in}{1.982817in}}{\pgfqpoint{3.919743in}{1.974917in}}{\pgfqpoint{3.919743in}{1.966680in}}%
\pgfpathcurveto{\pgfqpoint{3.919743in}{1.958444in}}{\pgfqpoint{3.923016in}{1.950544in}}{\pgfqpoint{3.928840in}{1.944720in}}%
\pgfpathcurveto{\pgfqpoint{3.934664in}{1.938896in}}{\pgfqpoint{3.942564in}{1.935624in}}{\pgfqpoint{3.950800in}{1.935624in}}%
\pgfpathclose%
\pgfusepath{stroke,fill}%
\end{pgfscope}%
\begin{pgfscope}%
\pgfpathrectangle{\pgfqpoint{3.836158in}{0.557870in}}{\pgfqpoint{2.522130in}{1.484734in}}%
\pgfusepath{clip}%
\pgfsetbuttcap%
\pgfsetroundjoin%
\definecolor{currentfill}{rgb}{0.298039,0.447059,0.690196}%
\pgfsetfillcolor{currentfill}%
\pgfsetlinewidth{1.003750pt}%
\definecolor{currentstroke}{rgb}{0.298039,0.447059,0.690196}%
\pgfsetstrokecolor{currentstroke}%
\pgfsetdash{}{0pt}%
\pgfpathmoveto{\pgfqpoint{6.079870in}{1.581312in}}%
\pgfpathcurveto{\pgfqpoint{6.088107in}{1.581312in}}{\pgfqpoint{6.096007in}{1.584585in}}{\pgfqpoint{6.101831in}{1.590409in}}%
\pgfpathcurveto{\pgfqpoint{6.107655in}{1.596232in}}{\pgfqpoint{6.110927in}{1.604132in}}{\pgfqpoint{6.110927in}{1.612369in}}%
\pgfpathcurveto{\pgfqpoint{6.110927in}{1.620605in}}{\pgfqpoint{6.107655in}{1.628505in}}{\pgfqpoint{6.101831in}{1.634329in}}%
\pgfpathcurveto{\pgfqpoint{6.096007in}{1.640153in}}{\pgfqpoint{6.088107in}{1.643425in}}{\pgfqpoint{6.079870in}{1.643425in}}%
\pgfpathcurveto{\pgfqpoint{6.071634in}{1.643425in}}{\pgfqpoint{6.063734in}{1.640153in}}{\pgfqpoint{6.057910in}{1.634329in}}%
\pgfpathcurveto{\pgfqpoint{6.052086in}{1.628505in}}{\pgfqpoint{6.048814in}{1.620605in}}{\pgfqpoint{6.048814in}{1.612369in}}%
\pgfpathcurveto{\pgfqpoint{6.048814in}{1.604132in}}{\pgfqpoint{6.052086in}{1.596232in}}{\pgfqpoint{6.057910in}{1.590409in}}%
\pgfpathcurveto{\pgfqpoint{6.063734in}{1.584585in}}{\pgfqpoint{6.071634in}{1.581312in}}{\pgfqpoint{6.079870in}{1.581312in}}%
\pgfpathclose%
\pgfusepath{stroke,fill}%
\end{pgfscope}%
\begin{pgfscope}%
\pgfpathrectangle{\pgfqpoint{3.836158in}{0.557870in}}{\pgfqpoint{2.522130in}{1.484734in}}%
\pgfusepath{clip}%
\pgfsetbuttcap%
\pgfsetroundjoin%
\definecolor{currentfill}{rgb}{0.298039,0.447059,0.690196}%
\pgfsetfillcolor{currentfill}%
\pgfsetlinewidth{1.003750pt}%
\definecolor{currentstroke}{rgb}{0.298039,0.447059,0.690196}%
\pgfsetstrokecolor{currentstroke}%
\pgfsetdash{}{0pt}%
\pgfpathmoveto{\pgfqpoint{3.950800in}{1.935624in}}%
\pgfpathcurveto{\pgfqpoint{3.959036in}{1.935624in}}{\pgfqpoint{3.966936in}{1.938896in}}{\pgfqpoint{3.972760in}{1.944720in}}%
\pgfpathcurveto{\pgfqpoint{3.978584in}{1.950544in}}{\pgfqpoint{3.981856in}{1.958444in}}{\pgfqpoint{3.981856in}{1.966680in}}%
\pgfpathcurveto{\pgfqpoint{3.981856in}{1.974917in}}{\pgfqpoint{3.978584in}{1.982817in}}{\pgfqpoint{3.972760in}{1.988641in}}%
\pgfpathcurveto{\pgfqpoint{3.966936in}{1.994464in}}{\pgfqpoint{3.959036in}{1.997737in}}{\pgfqpoint{3.950800in}{1.997737in}}%
\pgfpathcurveto{\pgfqpoint{3.942564in}{1.997737in}}{\pgfqpoint{3.934664in}{1.994464in}}{\pgfqpoint{3.928840in}{1.988641in}}%
\pgfpathcurveto{\pgfqpoint{3.923016in}{1.982817in}}{\pgfqpoint{3.919743in}{1.974917in}}{\pgfqpoint{3.919743in}{1.966680in}}%
\pgfpathcurveto{\pgfqpoint{3.919743in}{1.958444in}}{\pgfqpoint{3.923016in}{1.950544in}}{\pgfqpoint{3.928840in}{1.944720in}}%
\pgfpathcurveto{\pgfqpoint{3.934664in}{1.938896in}}{\pgfqpoint{3.942564in}{1.935624in}}{\pgfqpoint{3.950800in}{1.935624in}}%
\pgfpathclose%
\pgfusepath{stroke,fill}%
\end{pgfscope}%
\begin{pgfscope}%
\pgfpathrectangle{\pgfqpoint{3.836158in}{0.557870in}}{\pgfqpoint{2.522130in}{1.484734in}}%
\pgfusepath{clip}%
\pgfsetbuttcap%
\pgfsetroundjoin%
\definecolor{currentfill}{rgb}{0.298039,0.447059,0.690196}%
\pgfsetfillcolor{currentfill}%
\pgfsetlinewidth{1.003750pt}%
\definecolor{currentstroke}{rgb}{0.298039,0.447059,0.690196}%
\pgfsetstrokecolor{currentstroke}%
\pgfsetdash{}{0pt}%
\pgfpathmoveto{\pgfqpoint{6.079870in}{1.210129in}}%
\pgfpathcurveto{\pgfqpoint{6.088107in}{1.210129in}}{\pgfqpoint{6.096007in}{1.213401in}}{\pgfqpoint{6.101831in}{1.219225in}}%
\pgfpathcurveto{\pgfqpoint{6.107655in}{1.225049in}}{\pgfqpoint{6.110927in}{1.232949in}}{\pgfqpoint{6.110927in}{1.241185in}}%
\pgfpathcurveto{\pgfqpoint{6.110927in}{1.249422in}}{\pgfqpoint{6.107655in}{1.257322in}}{\pgfqpoint{6.101831in}{1.263146in}}%
\pgfpathcurveto{\pgfqpoint{6.096007in}{1.268969in}}{\pgfqpoint{6.088107in}{1.272242in}}{\pgfqpoint{6.079870in}{1.272242in}}%
\pgfpathcurveto{\pgfqpoint{6.071634in}{1.272242in}}{\pgfqpoint{6.063734in}{1.268969in}}{\pgfqpoint{6.057910in}{1.263146in}}%
\pgfpathcurveto{\pgfqpoint{6.052086in}{1.257322in}}{\pgfqpoint{6.048814in}{1.249422in}}{\pgfqpoint{6.048814in}{1.241185in}}%
\pgfpathcurveto{\pgfqpoint{6.048814in}{1.232949in}}{\pgfqpoint{6.052086in}{1.225049in}}{\pgfqpoint{6.057910in}{1.219225in}}%
\pgfpathcurveto{\pgfqpoint{6.063734in}{1.213401in}}{\pgfqpoint{6.071634in}{1.210129in}}{\pgfqpoint{6.079870in}{1.210129in}}%
\pgfpathclose%
\pgfusepath{stroke,fill}%
\end{pgfscope}%
\begin{pgfscope}%
\pgfpathrectangle{\pgfqpoint{3.836158in}{0.557870in}}{\pgfqpoint{2.522130in}{1.484734in}}%
\pgfusepath{clip}%
\pgfsetbuttcap%
\pgfsetroundjoin%
\definecolor{currentfill}{rgb}{0.298039,0.447059,0.690196}%
\pgfsetfillcolor{currentfill}%
\pgfsetlinewidth{1.003750pt}%
\definecolor{currentstroke}{rgb}{0.298039,0.447059,0.690196}%
\pgfsetstrokecolor{currentstroke}%
\pgfsetdash{}{0pt}%
\pgfpathmoveto{\pgfqpoint{3.950800in}{1.935407in}}%
\pgfpathcurveto{\pgfqpoint{3.959036in}{1.935407in}}{\pgfqpoint{3.966936in}{1.938680in}}{\pgfqpoint{3.972760in}{1.944504in}}%
\pgfpathcurveto{\pgfqpoint{3.978584in}{1.950328in}}{\pgfqpoint{3.981856in}{1.958228in}}{\pgfqpoint{3.981856in}{1.966464in}}%
\pgfpathcurveto{\pgfqpoint{3.981856in}{1.974700in}}{\pgfqpoint{3.978584in}{1.982600in}}{\pgfqpoint{3.972760in}{1.988424in}}%
\pgfpathcurveto{\pgfqpoint{3.966936in}{1.994248in}}{\pgfqpoint{3.959036in}{1.997520in}}{\pgfqpoint{3.950800in}{1.997520in}}%
\pgfpathcurveto{\pgfqpoint{3.942564in}{1.997520in}}{\pgfqpoint{3.934664in}{1.994248in}}{\pgfqpoint{3.928840in}{1.988424in}}%
\pgfpathcurveto{\pgfqpoint{3.923016in}{1.982600in}}{\pgfqpoint{3.919743in}{1.974700in}}{\pgfqpoint{3.919743in}{1.966464in}}%
\pgfpathcurveto{\pgfqpoint{3.919743in}{1.958228in}}{\pgfqpoint{3.923016in}{1.950328in}}{\pgfqpoint{3.928840in}{1.944504in}}%
\pgfpathcurveto{\pgfqpoint{3.934664in}{1.938680in}}{\pgfqpoint{3.942564in}{1.935407in}}{\pgfqpoint{3.950800in}{1.935407in}}%
\pgfpathclose%
\pgfusepath{stroke,fill}%
\end{pgfscope}%
\begin{pgfscope}%
\pgfpathrectangle{\pgfqpoint{3.836158in}{0.557870in}}{\pgfqpoint{2.522130in}{1.484734in}}%
\pgfusepath{clip}%
\pgfsetbuttcap%
\pgfsetroundjoin%
\definecolor{currentfill}{rgb}{0.298039,0.447059,0.690196}%
\pgfsetfillcolor{currentfill}%
\pgfsetlinewidth{1.003750pt}%
\definecolor{currentstroke}{rgb}{0.298039,0.447059,0.690196}%
\pgfsetstrokecolor{currentstroke}%
\pgfsetdash{}{0pt}%
\pgfpathmoveto{\pgfqpoint{6.079870in}{1.433574in}}%
\pgfpathcurveto{\pgfqpoint{6.088107in}{1.433574in}}{\pgfqpoint{6.096007in}{1.436847in}}{\pgfqpoint{6.101831in}{1.442671in}}%
\pgfpathcurveto{\pgfqpoint{6.107655in}{1.448495in}}{\pgfqpoint{6.110927in}{1.456395in}}{\pgfqpoint{6.110927in}{1.464631in}}%
\pgfpathcurveto{\pgfqpoint{6.110927in}{1.472867in}}{\pgfqpoint{6.107655in}{1.480767in}}{\pgfqpoint{6.101831in}{1.486591in}}%
\pgfpathcurveto{\pgfqpoint{6.096007in}{1.492415in}}{\pgfqpoint{6.088107in}{1.495687in}}{\pgfqpoint{6.079870in}{1.495687in}}%
\pgfpathcurveto{\pgfqpoint{6.071634in}{1.495687in}}{\pgfqpoint{6.063734in}{1.492415in}}{\pgfqpoint{6.057910in}{1.486591in}}%
\pgfpathcurveto{\pgfqpoint{6.052086in}{1.480767in}}{\pgfqpoint{6.048814in}{1.472867in}}{\pgfqpoint{6.048814in}{1.464631in}}%
\pgfpathcurveto{\pgfqpoint{6.048814in}{1.456395in}}{\pgfqpoint{6.052086in}{1.448495in}}{\pgfqpoint{6.057910in}{1.442671in}}%
\pgfpathcurveto{\pgfqpoint{6.063734in}{1.436847in}}{\pgfqpoint{6.071634in}{1.433574in}}{\pgfqpoint{6.079870in}{1.433574in}}%
\pgfpathclose%
\pgfusepath{stroke,fill}%
\end{pgfscope}%
\begin{pgfscope}%
\pgfpathrectangle{\pgfqpoint{3.836158in}{0.557870in}}{\pgfqpoint{2.522130in}{1.484734in}}%
\pgfusepath{clip}%
\pgfsetbuttcap%
\pgfsetroundjoin%
\definecolor{currentfill}{rgb}{0.298039,0.447059,0.690196}%
\pgfsetfillcolor{currentfill}%
\pgfsetlinewidth{1.003750pt}%
\definecolor{currentstroke}{rgb}{0.298039,0.447059,0.690196}%
\pgfsetstrokecolor{currentstroke}%
\pgfsetdash{}{0pt}%
\pgfpathmoveto{\pgfqpoint{3.950800in}{1.935407in}}%
\pgfpathcurveto{\pgfqpoint{3.959036in}{1.935407in}}{\pgfqpoint{3.966936in}{1.938680in}}{\pgfqpoint{3.972760in}{1.944504in}}%
\pgfpathcurveto{\pgfqpoint{3.978584in}{1.950328in}}{\pgfqpoint{3.981856in}{1.958228in}}{\pgfqpoint{3.981856in}{1.966464in}}%
\pgfpathcurveto{\pgfqpoint{3.981856in}{1.974700in}}{\pgfqpoint{3.978584in}{1.982600in}}{\pgfqpoint{3.972760in}{1.988424in}}%
\pgfpathcurveto{\pgfqpoint{3.966936in}{1.994248in}}{\pgfqpoint{3.959036in}{1.997520in}}{\pgfqpoint{3.950800in}{1.997520in}}%
\pgfpathcurveto{\pgfqpoint{3.942564in}{1.997520in}}{\pgfqpoint{3.934664in}{1.994248in}}{\pgfqpoint{3.928840in}{1.988424in}}%
\pgfpathcurveto{\pgfqpoint{3.923016in}{1.982600in}}{\pgfqpoint{3.919743in}{1.974700in}}{\pgfqpoint{3.919743in}{1.966464in}}%
\pgfpathcurveto{\pgfqpoint{3.919743in}{1.958228in}}{\pgfqpoint{3.923016in}{1.950328in}}{\pgfqpoint{3.928840in}{1.944504in}}%
\pgfpathcurveto{\pgfqpoint{3.934664in}{1.938680in}}{\pgfqpoint{3.942564in}{1.935407in}}{\pgfqpoint{3.950800in}{1.935407in}}%
\pgfpathclose%
\pgfusepath{stroke,fill}%
\end{pgfscope}%
\begin{pgfscope}%
\pgfpathrectangle{\pgfqpoint{3.836158in}{0.557870in}}{\pgfqpoint{2.522130in}{1.484734in}}%
\pgfusepath{clip}%
\pgfsetbuttcap%
\pgfsetroundjoin%
\definecolor{currentfill}{rgb}{0.298039,0.447059,0.690196}%
\pgfsetfillcolor{currentfill}%
\pgfsetlinewidth{1.003750pt}%
\definecolor{currentstroke}{rgb}{0.298039,0.447059,0.690196}%
\pgfsetstrokecolor{currentstroke}%
\pgfsetdash{}{0pt}%
\pgfpathmoveto{\pgfqpoint{6.079870in}{1.563359in}}%
\pgfpathcurveto{\pgfqpoint{6.088107in}{1.563359in}}{\pgfqpoint{6.096007in}{1.566631in}}{\pgfqpoint{6.101831in}{1.572455in}}%
\pgfpathcurveto{\pgfqpoint{6.107655in}{1.578279in}}{\pgfqpoint{6.110927in}{1.586179in}}{\pgfqpoint{6.110927in}{1.594415in}}%
\pgfpathcurveto{\pgfqpoint{6.110927in}{1.602652in}}{\pgfqpoint{6.107655in}{1.610552in}}{\pgfqpoint{6.101831in}{1.616376in}}%
\pgfpathcurveto{\pgfqpoint{6.096007in}{1.622199in}}{\pgfqpoint{6.088107in}{1.625472in}}{\pgfqpoint{6.079870in}{1.625472in}}%
\pgfpathcurveto{\pgfqpoint{6.071634in}{1.625472in}}{\pgfqpoint{6.063734in}{1.622199in}}{\pgfqpoint{6.057910in}{1.616376in}}%
\pgfpathcurveto{\pgfqpoint{6.052086in}{1.610552in}}{\pgfqpoint{6.048814in}{1.602652in}}{\pgfqpoint{6.048814in}{1.594415in}}%
\pgfpathcurveto{\pgfqpoint{6.048814in}{1.586179in}}{\pgfqpoint{6.052086in}{1.578279in}}{\pgfqpoint{6.057910in}{1.572455in}}%
\pgfpathcurveto{\pgfqpoint{6.063734in}{1.566631in}}{\pgfqpoint{6.071634in}{1.563359in}}{\pgfqpoint{6.079870in}{1.563359in}}%
\pgfpathclose%
\pgfusepath{stroke,fill}%
\end{pgfscope}%
\begin{pgfscope}%
\pgfsetrectcap%
\pgfsetmiterjoin%
\pgfsetlinewidth{1.254687pt}%
\definecolor{currentstroke}{rgb}{1.000000,1.000000,1.000000}%
\pgfsetstrokecolor{currentstroke}%
\pgfsetdash{}{0pt}%
\pgfpathmoveto{\pgfqpoint{3.836158in}{0.557870in}}%
\pgfpathlineto{\pgfqpoint{3.836158in}{2.042604in}}%
\pgfusepath{stroke}%
\end{pgfscope}%
\begin{pgfscope}%
\pgfsetrectcap%
\pgfsetmiterjoin%
\pgfsetlinewidth{1.254687pt}%
\definecolor{currentstroke}{rgb}{1.000000,1.000000,1.000000}%
\pgfsetstrokecolor{currentstroke}%
\pgfsetdash{}{0pt}%
\pgfpathmoveto{\pgfqpoint{6.358287in}{0.557870in}}%
\pgfpathlineto{\pgfqpoint{6.358287in}{2.042604in}}%
\pgfusepath{stroke}%
\end{pgfscope}%
\begin{pgfscope}%
\pgfsetrectcap%
\pgfsetmiterjoin%
\pgfsetlinewidth{1.254687pt}%
\definecolor{currentstroke}{rgb}{1.000000,1.000000,1.000000}%
\pgfsetstrokecolor{currentstroke}%
\pgfsetdash{}{0pt}%
\pgfpathmoveto{\pgfqpoint{3.836158in}{0.557870in}}%
\pgfpathlineto{\pgfqpoint{6.358287in}{0.557870in}}%
\pgfusepath{stroke}%
\end{pgfscope}%
\begin{pgfscope}%
\pgfsetrectcap%
\pgfsetmiterjoin%
\pgfsetlinewidth{1.254687pt}%
\definecolor{currentstroke}{rgb}{1.000000,1.000000,1.000000}%
\pgfsetstrokecolor{currentstroke}%
\pgfsetdash{}{0pt}%
\pgfpathmoveto{\pgfqpoint{3.836158in}{2.042604in}}%
\pgfpathlineto{\pgfqpoint{6.358287in}{2.042604in}}%
\pgfusepath{stroke}%
\end{pgfscope}%
\begin{pgfscope}%
\definecolor{textcolor}{rgb}{0.150000,0.150000,0.150000}%
\pgfsetstrokecolor{textcolor}%
\pgfsetfillcolor{textcolor}%
\pgftext[x=5.097222in,y=2.125938in,,base]{\color{textcolor}\sffamily\fontsize{11.000000}{13.200000}\selectfont (b)}%
\end{pgfscope}%
\end{pgfpicture}%
\makeatother%
\endgroup%

    \caption{Distribution of DOR, sensitivity and specificity for the different PVC methods when classifying patient diagnosis.}
    \label{fig:pvc_ind_dor_sens_spec_dist}
\end{figure}

\begin{figure}[htb]
    \centering
    % \includegraphics[width=\textwidth]{results/pvc-ind-ari.png}
    %% Creator: Matplotlib, PGF backend
%%
%% To include the figure in your LaTeX document, write
%%   \input{<filename>.pgf}
%%
%% Make sure the required packages are loaded in your preamble
%%   \usepackage{pgf}
%%
%% Figures using additional raster images can only be included by \input if
%% they are in the same directory as the main LaTeX file. For loading figures
%% from other directories you can use the `import` package
%%   \usepackage{import}
%% and then include the figures with
%%   \import{<path to file>}{<filename>.pgf}
%%
%% Matplotlib used the following preamble
%%
\begingroup%
\makeatletter%
\begin{pgfpicture}%
\pgfpathrectangle{\pgfpointorigin}{\pgfqpoint{6.340000in}{2.340000in}}%
\pgfusepath{use as bounding box, clip}%
\begin{pgfscope}%
\pgfsetbuttcap%
\pgfsetmiterjoin%
\definecolor{currentfill}{rgb}{1.000000,1.000000,1.000000}%
\pgfsetfillcolor{currentfill}%
\pgfsetlinewidth{0.000000pt}%
\definecolor{currentstroke}{rgb}{1.000000,1.000000,1.000000}%
\pgfsetstrokecolor{currentstroke}%
\pgfsetdash{}{0pt}%
\pgfpathmoveto{\pgfqpoint{0.000000in}{-0.000000in}}%
\pgfpathlineto{\pgfqpoint{6.340000in}{-0.000000in}}%
\pgfpathlineto{\pgfqpoint{6.340000in}{2.340000in}}%
\pgfpathlineto{\pgfqpoint{0.000000in}{2.340000in}}%
\pgfpathclose%
\pgfusepath{fill}%
\end{pgfscope}%
\begin{pgfscope}%
\pgfsetbuttcap%
\pgfsetmiterjoin%
\definecolor{currentfill}{rgb}{0.917647,0.917647,0.949020}%
\pgfsetfillcolor{currentfill}%
\pgfsetlinewidth{0.000000pt}%
\definecolor{currentstroke}{rgb}{0.000000,0.000000,0.000000}%
\pgfsetstrokecolor{currentstroke}%
\pgfsetstrokeopacity{0.000000}%
\pgfsetdash{}{0pt}%
\pgfpathmoveto{\pgfqpoint{0.574769in}{0.557870in}}%
\pgfpathlineto{\pgfqpoint{6.240000in}{0.557870in}}%
\pgfpathlineto{\pgfqpoint{6.240000in}{2.240000in}}%
\pgfpathlineto{\pgfqpoint{0.574769in}{2.240000in}}%
\pgfpathclose%
\pgfusepath{fill}%
\end{pgfscope}%
\begin{pgfscope}%
\pgfpathrectangle{\pgfqpoint{0.574769in}{0.557870in}}{\pgfqpoint{5.665231in}{1.682130in}}%
\pgfusepath{clip}%
\pgfsetroundcap%
\pgfsetroundjoin%
\pgfsetlinewidth{1.003750pt}%
\definecolor{currentstroke}{rgb}{1.000000,1.000000,1.000000}%
\pgfsetstrokecolor{currentstroke}%
\pgfsetdash{}{0pt}%
\pgfpathmoveto{\pgfqpoint{1.053663in}{0.557870in}}%
\pgfpathlineto{\pgfqpoint{1.053663in}{2.240000in}}%
\pgfusepath{stroke}%
\end{pgfscope}%
\begin{pgfscope}%
\definecolor{textcolor}{rgb}{0.150000,0.150000,0.150000}%
\pgfsetstrokecolor{textcolor}%
\pgfsetfillcolor{textcolor}%
\pgftext[x=1.053663in,y=0.425926in,,top]{\color{textcolor}\sffamily\fontsize{11.000000}{13.200000}\selectfont \(\displaystyle -0.05\)}%
\end{pgfscope}%
\begin{pgfscope}%
\pgfpathrectangle{\pgfqpoint{0.574769in}{0.557870in}}{\pgfqpoint{5.665231in}{1.682130in}}%
\pgfusepath{clip}%
\pgfsetroundcap%
\pgfsetroundjoin%
\pgfsetlinewidth{1.003750pt}%
\definecolor{currentstroke}{rgb}{1.000000,1.000000,1.000000}%
\pgfsetstrokecolor{currentstroke}%
\pgfsetdash{}{0pt}%
\pgfpathmoveto{\pgfqpoint{1.815006in}{0.557870in}}%
\pgfpathlineto{\pgfqpoint{1.815006in}{2.240000in}}%
\pgfusepath{stroke}%
\end{pgfscope}%
\begin{pgfscope}%
\definecolor{textcolor}{rgb}{0.150000,0.150000,0.150000}%
\pgfsetstrokecolor{textcolor}%
\pgfsetfillcolor{textcolor}%
\pgftext[x=1.815006in,y=0.425926in,,top]{\color{textcolor}\sffamily\fontsize{11.000000}{13.200000}\selectfont \(\displaystyle 0.00\)}%
\end{pgfscope}%
\begin{pgfscope}%
\pgfpathrectangle{\pgfqpoint{0.574769in}{0.557870in}}{\pgfqpoint{5.665231in}{1.682130in}}%
\pgfusepath{clip}%
\pgfsetroundcap%
\pgfsetroundjoin%
\pgfsetlinewidth{1.003750pt}%
\definecolor{currentstroke}{rgb}{1.000000,1.000000,1.000000}%
\pgfsetstrokecolor{currentstroke}%
\pgfsetdash{}{0pt}%
\pgfpathmoveto{\pgfqpoint{2.576349in}{0.557870in}}%
\pgfpathlineto{\pgfqpoint{2.576349in}{2.240000in}}%
\pgfusepath{stroke}%
\end{pgfscope}%
\begin{pgfscope}%
\definecolor{textcolor}{rgb}{0.150000,0.150000,0.150000}%
\pgfsetstrokecolor{textcolor}%
\pgfsetfillcolor{textcolor}%
\pgftext[x=2.576349in,y=0.425926in,,top]{\color{textcolor}\sffamily\fontsize{11.000000}{13.200000}\selectfont \(\displaystyle 0.05\)}%
\end{pgfscope}%
\begin{pgfscope}%
\pgfpathrectangle{\pgfqpoint{0.574769in}{0.557870in}}{\pgfqpoint{5.665231in}{1.682130in}}%
\pgfusepath{clip}%
\pgfsetroundcap%
\pgfsetroundjoin%
\pgfsetlinewidth{1.003750pt}%
\definecolor{currentstroke}{rgb}{1.000000,1.000000,1.000000}%
\pgfsetstrokecolor{currentstroke}%
\pgfsetdash{}{0pt}%
\pgfpathmoveto{\pgfqpoint{3.337692in}{0.557870in}}%
\pgfpathlineto{\pgfqpoint{3.337692in}{2.240000in}}%
\pgfusepath{stroke}%
\end{pgfscope}%
\begin{pgfscope}%
\definecolor{textcolor}{rgb}{0.150000,0.150000,0.150000}%
\pgfsetstrokecolor{textcolor}%
\pgfsetfillcolor{textcolor}%
\pgftext[x=3.337692in,y=0.425926in,,top]{\color{textcolor}\sffamily\fontsize{11.000000}{13.200000}\selectfont \(\displaystyle 0.10\)}%
\end{pgfscope}%
\begin{pgfscope}%
\pgfpathrectangle{\pgfqpoint{0.574769in}{0.557870in}}{\pgfqpoint{5.665231in}{1.682130in}}%
\pgfusepath{clip}%
\pgfsetroundcap%
\pgfsetroundjoin%
\pgfsetlinewidth{1.003750pt}%
\definecolor{currentstroke}{rgb}{1.000000,1.000000,1.000000}%
\pgfsetstrokecolor{currentstroke}%
\pgfsetdash{}{0pt}%
\pgfpathmoveto{\pgfqpoint{4.099035in}{0.557870in}}%
\pgfpathlineto{\pgfqpoint{4.099035in}{2.240000in}}%
\pgfusepath{stroke}%
\end{pgfscope}%
\begin{pgfscope}%
\definecolor{textcolor}{rgb}{0.150000,0.150000,0.150000}%
\pgfsetstrokecolor{textcolor}%
\pgfsetfillcolor{textcolor}%
\pgftext[x=4.099035in,y=0.425926in,,top]{\color{textcolor}\sffamily\fontsize{11.000000}{13.200000}\selectfont \(\displaystyle 0.15\)}%
\end{pgfscope}%
\begin{pgfscope}%
\pgfpathrectangle{\pgfqpoint{0.574769in}{0.557870in}}{\pgfqpoint{5.665231in}{1.682130in}}%
\pgfusepath{clip}%
\pgfsetroundcap%
\pgfsetroundjoin%
\pgfsetlinewidth{1.003750pt}%
\definecolor{currentstroke}{rgb}{1.000000,1.000000,1.000000}%
\pgfsetstrokecolor{currentstroke}%
\pgfsetdash{}{0pt}%
\pgfpathmoveto{\pgfqpoint{4.860378in}{0.557870in}}%
\pgfpathlineto{\pgfqpoint{4.860378in}{2.240000in}}%
\pgfusepath{stroke}%
\end{pgfscope}%
\begin{pgfscope}%
\definecolor{textcolor}{rgb}{0.150000,0.150000,0.150000}%
\pgfsetstrokecolor{textcolor}%
\pgfsetfillcolor{textcolor}%
\pgftext[x=4.860378in,y=0.425926in,,top]{\color{textcolor}\sffamily\fontsize{11.000000}{13.200000}\selectfont \(\displaystyle 0.20\)}%
\end{pgfscope}%
\begin{pgfscope}%
\pgfpathrectangle{\pgfqpoint{0.574769in}{0.557870in}}{\pgfqpoint{5.665231in}{1.682130in}}%
\pgfusepath{clip}%
\pgfsetroundcap%
\pgfsetroundjoin%
\pgfsetlinewidth{1.003750pt}%
\definecolor{currentstroke}{rgb}{1.000000,1.000000,1.000000}%
\pgfsetstrokecolor{currentstroke}%
\pgfsetdash{}{0pt}%
\pgfpathmoveto{\pgfqpoint{5.621721in}{0.557870in}}%
\pgfpathlineto{\pgfqpoint{5.621721in}{2.240000in}}%
\pgfusepath{stroke}%
\end{pgfscope}%
\begin{pgfscope}%
\definecolor{textcolor}{rgb}{0.150000,0.150000,0.150000}%
\pgfsetstrokecolor{textcolor}%
\pgfsetfillcolor{textcolor}%
\pgftext[x=5.621721in,y=0.425926in,,top]{\color{textcolor}\sffamily\fontsize{11.000000}{13.200000}\selectfont \(\displaystyle 0.25\)}%
\end{pgfscope}%
\begin{pgfscope}%
\definecolor{textcolor}{rgb}{0.150000,0.150000,0.150000}%
\pgfsetstrokecolor{textcolor}%
\pgfsetfillcolor{textcolor}%
\pgftext[x=3.407384in,y=0.235185in,,top]{\color{textcolor}\sffamily\fontsize{11.000000}{13.200000}\selectfont ARI}%
\end{pgfscope}%
\begin{pgfscope}%
\pgfpathrectangle{\pgfqpoint{0.574769in}{0.557870in}}{\pgfqpoint{5.665231in}{1.682130in}}%
\pgfusepath{clip}%
\pgfsetroundcap%
\pgfsetroundjoin%
\pgfsetlinewidth{1.003750pt}%
\definecolor{currentstroke}{rgb}{1.000000,1.000000,1.000000}%
\pgfsetstrokecolor{currentstroke}%
\pgfsetdash{}{0pt}%
\pgfpathmoveto{\pgfqpoint{0.574769in}{0.557870in}}%
\pgfpathlineto{\pgfqpoint{6.240000in}{0.557870in}}%
\pgfusepath{stroke}%
\end{pgfscope}%
\begin{pgfscope}%
\definecolor{textcolor}{rgb}{0.150000,0.150000,0.150000}%
\pgfsetstrokecolor{textcolor}%
\pgfsetfillcolor{textcolor}%
\pgftext[x=0.366783in,y=0.505064in,left,base]{\color{textcolor}\sffamily\fontsize{11.000000}{13.200000}\selectfont \(\displaystyle 0\)}%
\end{pgfscope}%
\begin{pgfscope}%
\pgfpathrectangle{\pgfqpoint{0.574769in}{0.557870in}}{\pgfqpoint{5.665231in}{1.682130in}}%
\pgfusepath{clip}%
\pgfsetroundcap%
\pgfsetroundjoin%
\pgfsetlinewidth{1.003750pt}%
\definecolor{currentstroke}{rgb}{1.000000,1.000000,1.000000}%
\pgfsetstrokecolor{currentstroke}%
\pgfsetdash{}{0pt}%
\pgfpathmoveto{\pgfqpoint{0.574769in}{0.898727in}}%
\pgfpathlineto{\pgfqpoint{6.240000in}{0.898727in}}%
\pgfusepath{stroke}%
\end{pgfscope}%
\begin{pgfscope}%
\definecolor{textcolor}{rgb}{0.150000,0.150000,0.150000}%
\pgfsetstrokecolor{textcolor}%
\pgfsetfillcolor{textcolor}%
\pgftext[x=0.290741in,y=0.845921in,left,base]{\color{textcolor}\sffamily\fontsize{11.000000}{13.200000}\selectfont \(\displaystyle 10\)}%
\end{pgfscope}%
\begin{pgfscope}%
\pgfpathrectangle{\pgfqpoint{0.574769in}{0.557870in}}{\pgfqpoint{5.665231in}{1.682130in}}%
\pgfusepath{clip}%
\pgfsetroundcap%
\pgfsetroundjoin%
\pgfsetlinewidth{1.003750pt}%
\definecolor{currentstroke}{rgb}{1.000000,1.000000,1.000000}%
\pgfsetstrokecolor{currentstroke}%
\pgfsetdash{}{0pt}%
\pgfpathmoveto{\pgfqpoint{0.574769in}{1.239584in}}%
\pgfpathlineto{\pgfqpoint{6.240000in}{1.239584in}}%
\pgfusepath{stroke}%
\end{pgfscope}%
\begin{pgfscope}%
\definecolor{textcolor}{rgb}{0.150000,0.150000,0.150000}%
\pgfsetstrokecolor{textcolor}%
\pgfsetfillcolor{textcolor}%
\pgftext[x=0.290741in,y=1.186778in,left,base]{\color{textcolor}\sffamily\fontsize{11.000000}{13.200000}\selectfont \(\displaystyle 20\)}%
\end{pgfscope}%
\begin{pgfscope}%
\pgfpathrectangle{\pgfqpoint{0.574769in}{0.557870in}}{\pgfqpoint{5.665231in}{1.682130in}}%
\pgfusepath{clip}%
\pgfsetroundcap%
\pgfsetroundjoin%
\pgfsetlinewidth{1.003750pt}%
\definecolor{currentstroke}{rgb}{1.000000,1.000000,1.000000}%
\pgfsetstrokecolor{currentstroke}%
\pgfsetdash{}{0pt}%
\pgfpathmoveto{\pgfqpoint{0.574769in}{1.580442in}}%
\pgfpathlineto{\pgfqpoint{6.240000in}{1.580442in}}%
\pgfusepath{stroke}%
\end{pgfscope}%
\begin{pgfscope}%
\definecolor{textcolor}{rgb}{0.150000,0.150000,0.150000}%
\pgfsetstrokecolor{textcolor}%
\pgfsetfillcolor{textcolor}%
\pgftext[x=0.290741in,y=1.527635in,left,base]{\color{textcolor}\sffamily\fontsize{11.000000}{13.200000}\selectfont \(\displaystyle 30\)}%
\end{pgfscope}%
\begin{pgfscope}%
\pgfpathrectangle{\pgfqpoint{0.574769in}{0.557870in}}{\pgfqpoint{5.665231in}{1.682130in}}%
\pgfusepath{clip}%
\pgfsetroundcap%
\pgfsetroundjoin%
\pgfsetlinewidth{1.003750pt}%
\definecolor{currentstroke}{rgb}{1.000000,1.000000,1.000000}%
\pgfsetstrokecolor{currentstroke}%
\pgfsetdash{}{0pt}%
\pgfpathmoveto{\pgfqpoint{0.574769in}{1.921299in}}%
\pgfpathlineto{\pgfqpoint{6.240000in}{1.921299in}}%
\pgfusepath{stroke}%
\end{pgfscope}%
\begin{pgfscope}%
\definecolor{textcolor}{rgb}{0.150000,0.150000,0.150000}%
\pgfsetstrokecolor{textcolor}%
\pgfsetfillcolor{textcolor}%
\pgftext[x=0.290741in,y=1.868492in,left,base]{\color{textcolor}\sffamily\fontsize{11.000000}{13.200000}\selectfont \(\displaystyle 40\)}%
\end{pgfscope}%
\begin{pgfscope}%
\definecolor{textcolor}{rgb}{0.150000,0.150000,0.150000}%
\pgfsetstrokecolor{textcolor}%
\pgfsetfillcolor{textcolor}%
\pgftext[x=0.235185in,y=1.398935in,,bottom,rotate=90.000000]{\color{textcolor}\sffamily\fontsize{11.000000}{13.200000}\selectfont Occurance}%
\end{pgfscope}%
\begin{pgfscope}%
\pgfpathrectangle{\pgfqpoint{0.574769in}{0.557870in}}{\pgfqpoint{5.665231in}{1.682130in}}%
\pgfusepath{clip}%
\pgfsetbuttcap%
\pgfsetmiterjoin%
\definecolor{currentfill}{rgb}{0.298039,0.447059,0.690196}%
\pgfsetfillcolor{currentfill}%
\pgfsetfillopacity{0.400000}%
\pgfsetlinewidth{1.003750pt}%
\definecolor{currentstroke}{rgb}{1.000000,1.000000,1.000000}%
\pgfsetstrokecolor{currentstroke}%
\pgfsetstrokeopacity{0.400000}%
\pgfsetdash{}{0pt}%
\pgfpathmoveto{\pgfqpoint{0.832279in}{0.557870in}}%
\pgfpathlineto{\pgfqpoint{1.228449in}{0.557870in}}%
\pgfpathlineto{\pgfqpoint{1.228449in}{1.171413in}}%
\pgfpathlineto{\pgfqpoint{0.832279in}{1.171413in}}%
\pgfpathclose%
\pgfusepath{stroke,fill}%
\end{pgfscope}%
\begin{pgfscope}%
\pgfpathrectangle{\pgfqpoint{0.574769in}{0.557870in}}{\pgfqpoint{5.665231in}{1.682130in}}%
\pgfusepath{clip}%
\pgfsetbuttcap%
\pgfsetmiterjoin%
\definecolor{currentfill}{rgb}{0.298039,0.447059,0.690196}%
\pgfsetfillcolor{currentfill}%
\pgfsetfillopacity{0.400000}%
\pgfsetlinewidth{1.003750pt}%
\definecolor{currentstroke}{rgb}{1.000000,1.000000,1.000000}%
\pgfsetstrokecolor{currentstroke}%
\pgfsetstrokeopacity{0.400000}%
\pgfsetdash{}{0pt}%
\pgfpathmoveto{\pgfqpoint{1.228449in}{0.557870in}}%
\pgfpathlineto{\pgfqpoint{1.624619in}{0.557870in}}%
\pgfpathlineto{\pgfqpoint{1.624619in}{1.239584in}}%
\pgfpathlineto{\pgfqpoint{1.228449in}{1.239584in}}%
\pgfpathclose%
\pgfusepath{stroke,fill}%
\end{pgfscope}%
\begin{pgfscope}%
\pgfpathrectangle{\pgfqpoint{0.574769in}{0.557870in}}{\pgfqpoint{5.665231in}{1.682130in}}%
\pgfusepath{clip}%
\pgfsetbuttcap%
\pgfsetmiterjoin%
\definecolor{currentfill}{rgb}{0.298039,0.447059,0.690196}%
\pgfsetfillcolor{currentfill}%
\pgfsetfillopacity{0.400000}%
\pgfsetlinewidth{1.003750pt}%
\definecolor{currentstroke}{rgb}{1.000000,1.000000,1.000000}%
\pgfsetstrokecolor{currentstroke}%
\pgfsetstrokeopacity{0.400000}%
\pgfsetdash{}{0pt}%
\pgfpathmoveto{\pgfqpoint{1.624619in}{0.557870in}}%
\pgfpathlineto{\pgfqpoint{2.020789in}{0.557870in}}%
\pgfpathlineto{\pgfqpoint{2.020789in}{1.580442in}}%
\pgfpathlineto{\pgfqpoint{1.624619in}{1.580442in}}%
\pgfpathclose%
\pgfusepath{stroke,fill}%
\end{pgfscope}%
\begin{pgfscope}%
\pgfpathrectangle{\pgfqpoint{0.574769in}{0.557870in}}{\pgfqpoint{5.665231in}{1.682130in}}%
\pgfusepath{clip}%
\pgfsetbuttcap%
\pgfsetmiterjoin%
\definecolor{currentfill}{rgb}{0.298039,0.447059,0.690196}%
\pgfsetfillcolor{currentfill}%
\pgfsetfillopacity{0.400000}%
\pgfsetlinewidth{1.003750pt}%
\definecolor{currentstroke}{rgb}{1.000000,1.000000,1.000000}%
\pgfsetstrokecolor{currentstroke}%
\pgfsetstrokeopacity{0.400000}%
\pgfsetdash{}{0pt}%
\pgfpathmoveto{\pgfqpoint{2.020789in}{0.557870in}}%
\pgfpathlineto{\pgfqpoint{2.416959in}{0.557870in}}%
\pgfpathlineto{\pgfqpoint{2.416959in}{2.159899in}}%
\pgfpathlineto{\pgfqpoint{2.020789in}{2.159899in}}%
\pgfpathclose%
\pgfusepath{stroke,fill}%
\end{pgfscope}%
\begin{pgfscope}%
\pgfpathrectangle{\pgfqpoint{0.574769in}{0.557870in}}{\pgfqpoint{5.665231in}{1.682130in}}%
\pgfusepath{clip}%
\pgfsetbuttcap%
\pgfsetmiterjoin%
\definecolor{currentfill}{rgb}{0.298039,0.447059,0.690196}%
\pgfsetfillcolor{currentfill}%
\pgfsetfillopacity{0.400000}%
\pgfsetlinewidth{1.003750pt}%
\definecolor{currentstroke}{rgb}{1.000000,1.000000,1.000000}%
\pgfsetstrokecolor{currentstroke}%
\pgfsetstrokeopacity{0.400000}%
\pgfsetdash{}{0pt}%
\pgfpathmoveto{\pgfqpoint{2.416959in}{0.557870in}}%
\pgfpathlineto{\pgfqpoint{2.813129in}{0.557870in}}%
\pgfpathlineto{\pgfqpoint{2.813129in}{1.648613in}}%
\pgfpathlineto{\pgfqpoint{2.416959in}{1.648613in}}%
\pgfpathclose%
\pgfusepath{stroke,fill}%
\end{pgfscope}%
\begin{pgfscope}%
\pgfpathrectangle{\pgfqpoint{0.574769in}{0.557870in}}{\pgfqpoint{5.665231in}{1.682130in}}%
\pgfusepath{clip}%
\pgfsetbuttcap%
\pgfsetmiterjoin%
\definecolor{currentfill}{rgb}{0.298039,0.447059,0.690196}%
\pgfsetfillcolor{currentfill}%
\pgfsetfillopacity{0.400000}%
\pgfsetlinewidth{1.003750pt}%
\definecolor{currentstroke}{rgb}{1.000000,1.000000,1.000000}%
\pgfsetstrokecolor{currentstroke}%
\pgfsetstrokeopacity{0.400000}%
\pgfsetdash{}{0pt}%
\pgfpathmoveto{\pgfqpoint{2.813129in}{0.557870in}}%
\pgfpathlineto{\pgfqpoint{3.209299in}{0.557870in}}%
\pgfpathlineto{\pgfqpoint{3.209299in}{1.444099in}}%
\pgfpathlineto{\pgfqpoint{2.813129in}{1.444099in}}%
\pgfpathclose%
\pgfusepath{stroke,fill}%
\end{pgfscope}%
\begin{pgfscope}%
\pgfpathrectangle{\pgfqpoint{0.574769in}{0.557870in}}{\pgfqpoint{5.665231in}{1.682130in}}%
\pgfusepath{clip}%
\pgfsetbuttcap%
\pgfsetmiterjoin%
\definecolor{currentfill}{rgb}{0.298039,0.447059,0.690196}%
\pgfsetfillcolor{currentfill}%
\pgfsetfillopacity{0.400000}%
\pgfsetlinewidth{1.003750pt}%
\definecolor{currentstroke}{rgb}{1.000000,1.000000,1.000000}%
\pgfsetstrokecolor{currentstroke}%
\pgfsetstrokeopacity{0.400000}%
\pgfsetdash{}{0pt}%
\pgfpathmoveto{\pgfqpoint{3.209299in}{0.557870in}}%
\pgfpathlineto{\pgfqpoint{3.605469in}{0.557870in}}%
\pgfpathlineto{\pgfqpoint{3.605469in}{0.966899in}}%
\pgfpathlineto{\pgfqpoint{3.209299in}{0.966899in}}%
\pgfpathclose%
\pgfusepath{stroke,fill}%
\end{pgfscope}%
\begin{pgfscope}%
\pgfpathrectangle{\pgfqpoint{0.574769in}{0.557870in}}{\pgfqpoint{5.665231in}{1.682130in}}%
\pgfusepath{clip}%
\pgfsetbuttcap%
\pgfsetmiterjoin%
\definecolor{currentfill}{rgb}{0.298039,0.447059,0.690196}%
\pgfsetfillcolor{currentfill}%
\pgfsetfillopacity{0.400000}%
\pgfsetlinewidth{1.003750pt}%
\definecolor{currentstroke}{rgb}{1.000000,1.000000,1.000000}%
\pgfsetstrokecolor{currentstroke}%
\pgfsetstrokeopacity{0.400000}%
\pgfsetdash{}{0pt}%
\pgfpathmoveto{\pgfqpoint{3.605469in}{0.557870in}}%
\pgfpathlineto{\pgfqpoint{4.001639in}{0.557870in}}%
\pgfpathlineto{\pgfqpoint{4.001639in}{0.898727in}}%
\pgfpathlineto{\pgfqpoint{3.605469in}{0.898727in}}%
\pgfpathclose%
\pgfusepath{stroke,fill}%
\end{pgfscope}%
\begin{pgfscope}%
\pgfpathrectangle{\pgfqpoint{0.574769in}{0.557870in}}{\pgfqpoint{5.665231in}{1.682130in}}%
\pgfusepath{clip}%
\pgfsetbuttcap%
\pgfsetmiterjoin%
\definecolor{currentfill}{rgb}{0.298039,0.447059,0.690196}%
\pgfsetfillcolor{currentfill}%
\pgfsetfillopacity{0.400000}%
\pgfsetlinewidth{1.003750pt}%
\definecolor{currentstroke}{rgb}{1.000000,1.000000,1.000000}%
\pgfsetstrokecolor{currentstroke}%
\pgfsetstrokeopacity{0.400000}%
\pgfsetdash{}{0pt}%
\pgfpathmoveto{\pgfqpoint{4.001639in}{0.557870in}}%
\pgfpathlineto{\pgfqpoint{4.397809in}{0.557870in}}%
\pgfpathlineto{\pgfqpoint{4.397809in}{0.796470in}}%
\pgfpathlineto{\pgfqpoint{4.001639in}{0.796470in}}%
\pgfpathclose%
\pgfusepath{stroke,fill}%
\end{pgfscope}%
\begin{pgfscope}%
\pgfpathrectangle{\pgfqpoint{0.574769in}{0.557870in}}{\pgfqpoint{5.665231in}{1.682130in}}%
\pgfusepath{clip}%
\pgfsetbuttcap%
\pgfsetmiterjoin%
\definecolor{currentfill}{rgb}{0.298039,0.447059,0.690196}%
\pgfsetfillcolor{currentfill}%
\pgfsetfillopacity{0.400000}%
\pgfsetlinewidth{1.003750pt}%
\definecolor{currentstroke}{rgb}{1.000000,1.000000,1.000000}%
\pgfsetstrokecolor{currentstroke}%
\pgfsetstrokeopacity{0.400000}%
\pgfsetdash{}{0pt}%
\pgfpathmoveto{\pgfqpoint{4.397809in}{0.557870in}}%
\pgfpathlineto{\pgfqpoint{4.793979in}{0.557870in}}%
\pgfpathlineto{\pgfqpoint{4.793979in}{0.796470in}}%
\pgfpathlineto{\pgfqpoint{4.397809in}{0.796470in}}%
\pgfpathclose%
\pgfusepath{stroke,fill}%
\end{pgfscope}%
\begin{pgfscope}%
\pgfpathrectangle{\pgfqpoint{0.574769in}{0.557870in}}{\pgfqpoint{5.665231in}{1.682130in}}%
\pgfusepath{clip}%
\pgfsetbuttcap%
\pgfsetmiterjoin%
\definecolor{currentfill}{rgb}{0.298039,0.447059,0.690196}%
\pgfsetfillcolor{currentfill}%
\pgfsetfillopacity{0.400000}%
\pgfsetlinewidth{1.003750pt}%
\definecolor{currentstroke}{rgb}{1.000000,1.000000,1.000000}%
\pgfsetstrokecolor{currentstroke}%
\pgfsetstrokeopacity{0.400000}%
\pgfsetdash{}{0pt}%
\pgfpathmoveto{\pgfqpoint{4.793979in}{0.557870in}}%
\pgfpathlineto{\pgfqpoint{5.190149in}{0.557870in}}%
\pgfpathlineto{\pgfqpoint{5.190149in}{0.591956in}}%
\pgfpathlineto{\pgfqpoint{4.793979in}{0.591956in}}%
\pgfpathclose%
\pgfusepath{stroke,fill}%
\end{pgfscope}%
\begin{pgfscope}%
\pgfpathrectangle{\pgfqpoint{0.574769in}{0.557870in}}{\pgfqpoint{5.665231in}{1.682130in}}%
\pgfusepath{clip}%
\pgfsetbuttcap%
\pgfsetmiterjoin%
\definecolor{currentfill}{rgb}{0.298039,0.447059,0.690196}%
\pgfsetfillcolor{currentfill}%
\pgfsetfillopacity{0.400000}%
\pgfsetlinewidth{1.003750pt}%
\definecolor{currentstroke}{rgb}{1.000000,1.000000,1.000000}%
\pgfsetstrokecolor{currentstroke}%
\pgfsetstrokeopacity{0.400000}%
\pgfsetdash{}{0pt}%
\pgfpathmoveto{\pgfqpoint{5.190149in}{0.557870in}}%
\pgfpathlineto{\pgfqpoint{5.586319in}{0.557870in}}%
\pgfpathlineto{\pgfqpoint{5.586319in}{0.660127in}}%
\pgfpathlineto{\pgfqpoint{5.190149in}{0.660127in}}%
\pgfpathclose%
\pgfusepath{stroke,fill}%
\end{pgfscope}%
\begin{pgfscope}%
\pgfpathrectangle{\pgfqpoint{0.574769in}{0.557870in}}{\pgfqpoint{5.665231in}{1.682130in}}%
\pgfusepath{clip}%
\pgfsetbuttcap%
\pgfsetmiterjoin%
\definecolor{currentfill}{rgb}{0.298039,0.447059,0.690196}%
\pgfsetfillcolor{currentfill}%
\pgfsetfillopacity{0.400000}%
\pgfsetlinewidth{1.003750pt}%
\definecolor{currentstroke}{rgb}{1.000000,1.000000,1.000000}%
\pgfsetstrokecolor{currentstroke}%
\pgfsetstrokeopacity{0.400000}%
\pgfsetdash{}{0pt}%
\pgfpathmoveto{\pgfqpoint{5.586319in}{0.557870in}}%
\pgfpathlineto{\pgfqpoint{5.982489in}{0.557870in}}%
\pgfpathlineto{\pgfqpoint{5.982489in}{0.660127in}}%
\pgfpathlineto{\pgfqpoint{5.586319in}{0.660127in}}%
\pgfpathclose%
\pgfusepath{stroke,fill}%
\end{pgfscope}%
\begin{pgfscope}%
\pgfsetrectcap%
\pgfsetmiterjoin%
\pgfsetlinewidth{1.254687pt}%
\definecolor{currentstroke}{rgb}{1.000000,1.000000,1.000000}%
\pgfsetstrokecolor{currentstroke}%
\pgfsetdash{}{0pt}%
\pgfpathmoveto{\pgfqpoint{0.574769in}{0.557870in}}%
\pgfpathlineto{\pgfqpoint{0.574769in}{2.240000in}}%
\pgfusepath{stroke}%
\end{pgfscope}%
\begin{pgfscope}%
\pgfsetrectcap%
\pgfsetmiterjoin%
\pgfsetlinewidth{1.254687pt}%
\definecolor{currentstroke}{rgb}{1.000000,1.000000,1.000000}%
\pgfsetstrokecolor{currentstroke}%
\pgfsetdash{}{0pt}%
\pgfpathmoveto{\pgfqpoint{6.240000in}{0.557870in}}%
\pgfpathlineto{\pgfqpoint{6.240000in}{2.240000in}}%
\pgfusepath{stroke}%
\end{pgfscope}%
\begin{pgfscope}%
\pgfsetrectcap%
\pgfsetmiterjoin%
\pgfsetlinewidth{1.254687pt}%
\definecolor{currentstroke}{rgb}{1.000000,1.000000,1.000000}%
\pgfsetstrokecolor{currentstroke}%
\pgfsetdash{}{0pt}%
\pgfpathmoveto{\pgfqpoint{0.574769in}{0.557870in}}%
\pgfpathlineto{\pgfqpoint{6.240000in}{0.557870in}}%
\pgfusepath{stroke}%
\end{pgfscope}%
\begin{pgfscope}%
\pgfsetrectcap%
\pgfsetmiterjoin%
\pgfsetlinewidth{1.254687pt}%
\definecolor{currentstroke}{rgb}{1.000000,1.000000,1.000000}%
\pgfsetstrokecolor{currentstroke}%
\pgfsetdash{}{0pt}%
\pgfpathmoveto{\pgfqpoint{0.574769in}{2.240000in}}%
\pgfpathlineto{\pgfqpoint{6.240000in}{2.240000in}}%
\pgfusepath{stroke}%
\end{pgfscope}%
\end{pgfpicture}%
\makeatother%
\endgroup%

    \caption{ARI distribution of PVC methods when classifying patient diagnoses.}
    \label{fig:pvc_ind_ari}
\end{figure}

\newpage

\subsection{Deep Neural Network}

\begin{figure}[htb]
    \centering
    % \includegraphics[width=\textwidth]{results/dl_ind_dor_sens_spec_dist.png}
    %% Creator: Matplotlib, PGF backend
%%
%% To include the figure in your LaTeX document, write
%%   \input{<filename>.pgf}
%%
%% Make sure the required packages are loaded in your preamble
%%   \usepackage{pgf}
%%
%% Figures using additional raster images can only be included by \input if
%% they are in the same directory as the main LaTeX file. For loading figures
%% from other directories you can use the `import` package
%%   \usepackage{import}
%% and then include the figures with
%%   \import{<path to file>}{<filename>.pgf}
%%
%% Matplotlib used the following preamble
%%
\begingroup%
\makeatletter%
\begin{pgfpicture}%
\pgfpathrectangle{\pgfpointorigin}{\pgfqpoint{6.363231in}{2.340000in}}%
\pgfusepath{use as bounding box, clip}%
\begin{pgfscope}%
\pgfsetbuttcap%
\pgfsetmiterjoin%
\definecolor{currentfill}{rgb}{1.000000,1.000000,1.000000}%
\pgfsetfillcolor{currentfill}%
\pgfsetlinewidth{0.000000pt}%
\definecolor{currentstroke}{rgb}{1.000000,1.000000,1.000000}%
\pgfsetstrokecolor{currentstroke}%
\pgfsetdash{}{0pt}%
\pgfpathmoveto{\pgfqpoint{0.000000in}{-0.000000in}}%
\pgfpathlineto{\pgfqpoint{6.363231in}{-0.000000in}}%
\pgfpathlineto{\pgfqpoint{6.363231in}{2.340000in}}%
\pgfpathlineto{\pgfqpoint{0.000000in}{2.340000in}}%
\pgfpathclose%
\pgfusepath{fill}%
\end{pgfscope}%
\begin{pgfscope}%
\pgfsetbuttcap%
\pgfsetmiterjoin%
\definecolor{currentfill}{rgb}{0.917647,0.917647,0.949020}%
\pgfsetfillcolor{currentfill}%
\pgfsetlinewidth{0.000000pt}%
\definecolor{currentstroke}{rgb}{0.000000,0.000000,0.000000}%
\pgfsetstrokecolor{currentstroke}%
\pgfsetstrokeopacity{0.000000}%
\pgfsetdash{}{0pt}%
\pgfpathmoveto{\pgfqpoint{0.617014in}{0.557870in}}%
\pgfpathlineto{\pgfqpoint{3.080000in}{0.557870in}}%
\pgfpathlineto{\pgfqpoint{3.080000in}{2.042604in}}%
\pgfpathlineto{\pgfqpoint{0.617014in}{2.042604in}}%
\pgfpathclose%
\pgfusepath{fill}%
\end{pgfscope}%
\begin{pgfscope}%
\pgfpathrectangle{\pgfqpoint{0.617014in}{0.557870in}}{\pgfqpoint{2.462986in}{1.484734in}}%
\pgfusepath{clip}%
\pgfsetroundcap%
\pgfsetroundjoin%
\pgfsetlinewidth{1.003750pt}%
\definecolor{currentstroke}{rgb}{1.000000,1.000000,1.000000}%
\pgfsetstrokecolor{currentstroke}%
\pgfsetdash{}{0pt}%
\pgfpathmoveto{\pgfqpoint{0.728968in}{0.557870in}}%
\pgfpathlineto{\pgfqpoint{0.728968in}{2.042604in}}%
\pgfusepath{stroke}%
\end{pgfscope}%
\begin{pgfscope}%
\definecolor{textcolor}{rgb}{0.150000,0.150000,0.150000}%
\pgfsetstrokecolor{textcolor}%
\pgfsetfillcolor{textcolor}%
\pgftext[x=0.728968in,y=0.425926in,,top]{\color{textcolor}\sffamily\fontsize{11.000000}{13.200000}\selectfont \(\displaystyle -0.50\)}%
\end{pgfscope}%
\begin{pgfscope}%
\pgfpathrectangle{\pgfqpoint{0.617014in}{0.557870in}}{\pgfqpoint{2.462986in}{1.484734in}}%
\pgfusepath{clip}%
\pgfsetroundcap%
\pgfsetroundjoin%
\pgfsetlinewidth{1.003750pt}%
\definecolor{currentstroke}{rgb}{1.000000,1.000000,1.000000}%
\pgfsetstrokecolor{currentstroke}%
\pgfsetdash{}{0pt}%
\pgfpathmoveto{\pgfqpoint{1.288737in}{0.557870in}}%
\pgfpathlineto{\pgfqpoint{1.288737in}{2.042604in}}%
\pgfusepath{stroke}%
\end{pgfscope}%
\begin{pgfscope}%
\definecolor{textcolor}{rgb}{0.150000,0.150000,0.150000}%
\pgfsetstrokecolor{textcolor}%
\pgfsetfillcolor{textcolor}%
\pgftext[x=1.288737in,y=0.425926in,,top]{\color{textcolor}\sffamily\fontsize{11.000000}{13.200000}\selectfont \(\displaystyle -0.25\)}%
\end{pgfscope}%
\begin{pgfscope}%
\pgfpathrectangle{\pgfqpoint{0.617014in}{0.557870in}}{\pgfqpoint{2.462986in}{1.484734in}}%
\pgfusepath{clip}%
\pgfsetroundcap%
\pgfsetroundjoin%
\pgfsetlinewidth{1.003750pt}%
\definecolor{currentstroke}{rgb}{1.000000,1.000000,1.000000}%
\pgfsetstrokecolor{currentstroke}%
\pgfsetdash{}{0pt}%
\pgfpathmoveto{\pgfqpoint{1.848507in}{0.557870in}}%
\pgfpathlineto{\pgfqpoint{1.848507in}{2.042604in}}%
\pgfusepath{stroke}%
\end{pgfscope}%
\begin{pgfscope}%
\definecolor{textcolor}{rgb}{0.150000,0.150000,0.150000}%
\pgfsetstrokecolor{textcolor}%
\pgfsetfillcolor{textcolor}%
\pgftext[x=1.848507in,y=0.425926in,,top]{\color{textcolor}\sffamily\fontsize{11.000000}{13.200000}\selectfont \(\displaystyle 0.00\)}%
\end{pgfscope}%
\begin{pgfscope}%
\pgfpathrectangle{\pgfqpoint{0.617014in}{0.557870in}}{\pgfqpoint{2.462986in}{1.484734in}}%
\pgfusepath{clip}%
\pgfsetroundcap%
\pgfsetroundjoin%
\pgfsetlinewidth{1.003750pt}%
\definecolor{currentstroke}{rgb}{1.000000,1.000000,1.000000}%
\pgfsetstrokecolor{currentstroke}%
\pgfsetdash{}{0pt}%
\pgfpathmoveto{\pgfqpoint{2.408277in}{0.557870in}}%
\pgfpathlineto{\pgfqpoint{2.408277in}{2.042604in}}%
\pgfusepath{stroke}%
\end{pgfscope}%
\begin{pgfscope}%
\definecolor{textcolor}{rgb}{0.150000,0.150000,0.150000}%
\pgfsetstrokecolor{textcolor}%
\pgfsetfillcolor{textcolor}%
\pgftext[x=2.408277in,y=0.425926in,,top]{\color{textcolor}\sffamily\fontsize{11.000000}{13.200000}\selectfont \(\displaystyle 0.25\)}%
\end{pgfscope}%
\begin{pgfscope}%
\pgfpathrectangle{\pgfqpoint{0.617014in}{0.557870in}}{\pgfqpoint{2.462986in}{1.484734in}}%
\pgfusepath{clip}%
\pgfsetroundcap%
\pgfsetroundjoin%
\pgfsetlinewidth{1.003750pt}%
\definecolor{currentstroke}{rgb}{1.000000,1.000000,1.000000}%
\pgfsetstrokecolor{currentstroke}%
\pgfsetdash{}{0pt}%
\pgfpathmoveto{\pgfqpoint{2.968046in}{0.557870in}}%
\pgfpathlineto{\pgfqpoint{2.968046in}{2.042604in}}%
\pgfusepath{stroke}%
\end{pgfscope}%
\begin{pgfscope}%
\definecolor{textcolor}{rgb}{0.150000,0.150000,0.150000}%
\pgfsetstrokecolor{textcolor}%
\pgfsetfillcolor{textcolor}%
\pgftext[x=2.968046in,y=0.425926in,,top]{\color{textcolor}\sffamily\fontsize{11.000000}{13.200000}\selectfont \(\displaystyle 0.50\)}%
\end{pgfscope}%
\begin{pgfscope}%
\definecolor{textcolor}{rgb}{0.150000,0.150000,0.150000}%
\pgfsetstrokecolor{textcolor}%
\pgfsetfillcolor{textcolor}%
\pgftext[x=1.848507in,y=0.235185in,,top]{\color{textcolor}\sffamily\fontsize{11.000000}{13.200000}\selectfont DOR}%
\end{pgfscope}%
\begin{pgfscope}%
\pgfpathrectangle{\pgfqpoint{0.617014in}{0.557870in}}{\pgfqpoint{2.462986in}{1.484734in}}%
\pgfusepath{clip}%
\pgfsetroundcap%
\pgfsetroundjoin%
\pgfsetlinewidth{1.003750pt}%
\definecolor{currentstroke}{rgb}{1.000000,1.000000,1.000000}%
\pgfsetstrokecolor{currentstroke}%
\pgfsetdash{}{0pt}%
\pgfpathmoveto{\pgfqpoint{0.617014in}{0.557870in}}%
\pgfpathlineto{\pgfqpoint{3.080000in}{0.557870in}}%
\pgfusepath{stroke}%
\end{pgfscope}%
\begin{pgfscope}%
\definecolor{textcolor}{rgb}{0.150000,0.150000,0.150000}%
\pgfsetstrokecolor{textcolor}%
\pgfsetfillcolor{textcolor}%
\pgftext[x=0.290741in,y=0.505064in,left,base]{\color{textcolor}\sffamily\fontsize{11.000000}{13.200000}\selectfont \(\displaystyle 0.0\)}%
\end{pgfscope}%
\begin{pgfscope}%
\pgfpathrectangle{\pgfqpoint{0.617014in}{0.557870in}}{\pgfqpoint{2.462986in}{1.484734in}}%
\pgfusepath{clip}%
\pgfsetroundcap%
\pgfsetroundjoin%
\pgfsetlinewidth{1.003750pt}%
\definecolor{currentstroke}{rgb}{1.000000,1.000000,1.000000}%
\pgfsetstrokecolor{currentstroke}%
\pgfsetdash{}{0pt}%
\pgfpathmoveto{\pgfqpoint{0.617014in}{1.264886in}}%
\pgfpathlineto{\pgfqpoint{3.080000in}{1.264886in}}%
\pgfusepath{stroke}%
\end{pgfscope}%
\begin{pgfscope}%
\definecolor{textcolor}{rgb}{0.150000,0.150000,0.150000}%
\pgfsetstrokecolor{textcolor}%
\pgfsetfillcolor{textcolor}%
\pgftext[x=0.290741in,y=1.212080in,left,base]{\color{textcolor}\sffamily\fontsize{11.000000}{13.200000}\selectfont \(\displaystyle 0.5\)}%
\end{pgfscope}%
\begin{pgfscope}%
\pgfpathrectangle{\pgfqpoint{0.617014in}{0.557870in}}{\pgfqpoint{2.462986in}{1.484734in}}%
\pgfusepath{clip}%
\pgfsetroundcap%
\pgfsetroundjoin%
\pgfsetlinewidth{1.003750pt}%
\definecolor{currentstroke}{rgb}{1.000000,1.000000,1.000000}%
\pgfsetstrokecolor{currentstroke}%
\pgfsetdash{}{0pt}%
\pgfpathmoveto{\pgfqpoint{0.617014in}{1.971903in}}%
\pgfpathlineto{\pgfqpoint{3.080000in}{1.971903in}}%
\pgfusepath{stroke}%
\end{pgfscope}%
\begin{pgfscope}%
\definecolor{textcolor}{rgb}{0.150000,0.150000,0.150000}%
\pgfsetstrokecolor{textcolor}%
\pgfsetfillcolor{textcolor}%
\pgftext[x=0.290741in,y=1.919096in,left,base]{\color{textcolor}\sffamily\fontsize{11.000000}{13.200000}\selectfont \(\displaystyle 1.0\)}%
\end{pgfscope}%
\begin{pgfscope}%
\definecolor{textcolor}{rgb}{0.150000,0.150000,0.150000}%
\pgfsetstrokecolor{textcolor}%
\pgfsetfillcolor{textcolor}%
\pgftext[x=0.235185in,y=1.300237in,,bottom,rotate=90.000000]{\color{textcolor}\sffamily\fontsize{11.000000}{13.200000}\selectfont Occurance}%
\end{pgfscope}%
\begin{pgfscope}%
\pgfpathrectangle{\pgfqpoint{0.617014in}{0.557870in}}{\pgfqpoint{2.462986in}{1.484734in}}%
\pgfusepath{clip}%
\pgfsetbuttcap%
\pgfsetmiterjoin%
\definecolor{currentfill}{rgb}{0.298039,0.447059,0.690196}%
\pgfsetfillcolor{currentfill}%
\pgfsetfillopacity{0.400000}%
\pgfsetlinewidth{1.003750pt}%
\definecolor{currentstroke}{rgb}{1.000000,1.000000,1.000000}%
\pgfsetstrokecolor{currentstroke}%
\pgfsetstrokeopacity{0.400000}%
\pgfsetdash{}{0pt}%
\pgfpathmoveto{\pgfqpoint{0.728968in}{0.557870in}}%
\pgfpathlineto{\pgfqpoint{0.952876in}{0.557870in}}%
\pgfpathlineto{\pgfqpoint{0.952876in}{0.557870in}}%
\pgfpathlineto{\pgfqpoint{0.728968in}{0.557870in}}%
\pgfpathclose%
\pgfusepath{stroke,fill}%
\end{pgfscope}%
\begin{pgfscope}%
\pgfpathrectangle{\pgfqpoint{0.617014in}{0.557870in}}{\pgfqpoint{2.462986in}{1.484734in}}%
\pgfusepath{clip}%
\pgfsetbuttcap%
\pgfsetmiterjoin%
\definecolor{currentfill}{rgb}{0.298039,0.447059,0.690196}%
\pgfsetfillcolor{currentfill}%
\pgfsetfillopacity{0.400000}%
\pgfsetlinewidth{1.003750pt}%
\definecolor{currentstroke}{rgb}{1.000000,1.000000,1.000000}%
\pgfsetstrokecolor{currentstroke}%
\pgfsetstrokeopacity{0.400000}%
\pgfsetdash{}{0pt}%
\pgfpathmoveto{\pgfqpoint{0.952876in}{0.557870in}}%
\pgfpathlineto{\pgfqpoint{1.176784in}{0.557870in}}%
\pgfpathlineto{\pgfqpoint{1.176784in}{0.557870in}}%
\pgfpathlineto{\pgfqpoint{0.952876in}{0.557870in}}%
\pgfpathclose%
\pgfusepath{stroke,fill}%
\end{pgfscope}%
\begin{pgfscope}%
\pgfpathrectangle{\pgfqpoint{0.617014in}{0.557870in}}{\pgfqpoint{2.462986in}{1.484734in}}%
\pgfusepath{clip}%
\pgfsetbuttcap%
\pgfsetmiterjoin%
\definecolor{currentfill}{rgb}{0.298039,0.447059,0.690196}%
\pgfsetfillcolor{currentfill}%
\pgfsetfillopacity{0.400000}%
\pgfsetlinewidth{1.003750pt}%
\definecolor{currentstroke}{rgb}{1.000000,1.000000,1.000000}%
\pgfsetstrokecolor{currentstroke}%
\pgfsetstrokeopacity{0.400000}%
\pgfsetdash{}{0pt}%
\pgfpathmoveto{\pgfqpoint{1.176784in}{0.557870in}}%
\pgfpathlineto{\pgfqpoint{1.400691in}{0.557870in}}%
\pgfpathlineto{\pgfqpoint{1.400691in}{0.557870in}}%
\pgfpathlineto{\pgfqpoint{1.176784in}{0.557870in}}%
\pgfpathclose%
\pgfusepath{stroke,fill}%
\end{pgfscope}%
\begin{pgfscope}%
\pgfpathrectangle{\pgfqpoint{0.617014in}{0.557870in}}{\pgfqpoint{2.462986in}{1.484734in}}%
\pgfusepath{clip}%
\pgfsetbuttcap%
\pgfsetmiterjoin%
\definecolor{currentfill}{rgb}{0.298039,0.447059,0.690196}%
\pgfsetfillcolor{currentfill}%
\pgfsetfillopacity{0.400000}%
\pgfsetlinewidth{1.003750pt}%
\definecolor{currentstroke}{rgb}{1.000000,1.000000,1.000000}%
\pgfsetstrokecolor{currentstroke}%
\pgfsetstrokeopacity{0.400000}%
\pgfsetdash{}{0pt}%
\pgfpathmoveto{\pgfqpoint{1.400691in}{0.557870in}}%
\pgfpathlineto{\pgfqpoint{1.624599in}{0.557870in}}%
\pgfpathlineto{\pgfqpoint{1.624599in}{0.557870in}}%
\pgfpathlineto{\pgfqpoint{1.400691in}{0.557870in}}%
\pgfpathclose%
\pgfusepath{stroke,fill}%
\end{pgfscope}%
\begin{pgfscope}%
\pgfpathrectangle{\pgfqpoint{0.617014in}{0.557870in}}{\pgfqpoint{2.462986in}{1.484734in}}%
\pgfusepath{clip}%
\pgfsetbuttcap%
\pgfsetmiterjoin%
\definecolor{currentfill}{rgb}{0.298039,0.447059,0.690196}%
\pgfsetfillcolor{currentfill}%
\pgfsetfillopacity{0.400000}%
\pgfsetlinewidth{1.003750pt}%
\definecolor{currentstroke}{rgb}{1.000000,1.000000,1.000000}%
\pgfsetstrokecolor{currentstroke}%
\pgfsetstrokeopacity{0.400000}%
\pgfsetdash{}{0pt}%
\pgfpathmoveto{\pgfqpoint{1.624599in}{0.557870in}}%
\pgfpathlineto{\pgfqpoint{1.848507in}{0.557870in}}%
\pgfpathlineto{\pgfqpoint{1.848507in}{0.557870in}}%
\pgfpathlineto{\pgfqpoint{1.624599in}{0.557870in}}%
\pgfpathclose%
\pgfusepath{stroke,fill}%
\end{pgfscope}%
\begin{pgfscope}%
\pgfpathrectangle{\pgfqpoint{0.617014in}{0.557870in}}{\pgfqpoint{2.462986in}{1.484734in}}%
\pgfusepath{clip}%
\pgfsetbuttcap%
\pgfsetmiterjoin%
\definecolor{currentfill}{rgb}{0.298039,0.447059,0.690196}%
\pgfsetfillcolor{currentfill}%
\pgfsetfillopacity{0.400000}%
\pgfsetlinewidth{1.003750pt}%
\definecolor{currentstroke}{rgb}{1.000000,1.000000,1.000000}%
\pgfsetstrokecolor{currentstroke}%
\pgfsetstrokeopacity{0.400000}%
\pgfsetdash{}{0pt}%
\pgfpathmoveto{\pgfqpoint{1.848507in}{0.557870in}}%
\pgfpathlineto{\pgfqpoint{2.072415in}{0.557870in}}%
\pgfpathlineto{\pgfqpoint{2.072415in}{1.971903in}}%
\pgfpathlineto{\pgfqpoint{1.848507in}{1.971903in}}%
\pgfpathclose%
\pgfusepath{stroke,fill}%
\end{pgfscope}%
\begin{pgfscope}%
\pgfpathrectangle{\pgfqpoint{0.617014in}{0.557870in}}{\pgfqpoint{2.462986in}{1.484734in}}%
\pgfusepath{clip}%
\pgfsetbuttcap%
\pgfsetmiterjoin%
\definecolor{currentfill}{rgb}{0.298039,0.447059,0.690196}%
\pgfsetfillcolor{currentfill}%
\pgfsetfillopacity{0.400000}%
\pgfsetlinewidth{1.003750pt}%
\definecolor{currentstroke}{rgb}{1.000000,1.000000,1.000000}%
\pgfsetstrokecolor{currentstroke}%
\pgfsetstrokeopacity{0.400000}%
\pgfsetdash{}{0pt}%
\pgfpathmoveto{\pgfqpoint{2.072415in}{0.557870in}}%
\pgfpathlineto{\pgfqpoint{2.296323in}{0.557870in}}%
\pgfpathlineto{\pgfqpoint{2.296323in}{0.557870in}}%
\pgfpathlineto{\pgfqpoint{2.072415in}{0.557870in}}%
\pgfpathclose%
\pgfusepath{stroke,fill}%
\end{pgfscope}%
\begin{pgfscope}%
\pgfpathrectangle{\pgfqpoint{0.617014in}{0.557870in}}{\pgfqpoint{2.462986in}{1.484734in}}%
\pgfusepath{clip}%
\pgfsetbuttcap%
\pgfsetmiterjoin%
\definecolor{currentfill}{rgb}{0.298039,0.447059,0.690196}%
\pgfsetfillcolor{currentfill}%
\pgfsetfillopacity{0.400000}%
\pgfsetlinewidth{1.003750pt}%
\definecolor{currentstroke}{rgb}{1.000000,1.000000,1.000000}%
\pgfsetstrokecolor{currentstroke}%
\pgfsetstrokeopacity{0.400000}%
\pgfsetdash{}{0pt}%
\pgfpathmoveto{\pgfqpoint{2.296323in}{0.557870in}}%
\pgfpathlineto{\pgfqpoint{2.520230in}{0.557870in}}%
\pgfpathlineto{\pgfqpoint{2.520230in}{0.557870in}}%
\pgfpathlineto{\pgfqpoint{2.296323in}{0.557870in}}%
\pgfpathclose%
\pgfusepath{stroke,fill}%
\end{pgfscope}%
\begin{pgfscope}%
\pgfpathrectangle{\pgfqpoint{0.617014in}{0.557870in}}{\pgfqpoint{2.462986in}{1.484734in}}%
\pgfusepath{clip}%
\pgfsetbuttcap%
\pgfsetmiterjoin%
\definecolor{currentfill}{rgb}{0.298039,0.447059,0.690196}%
\pgfsetfillcolor{currentfill}%
\pgfsetfillopacity{0.400000}%
\pgfsetlinewidth{1.003750pt}%
\definecolor{currentstroke}{rgb}{1.000000,1.000000,1.000000}%
\pgfsetstrokecolor{currentstroke}%
\pgfsetstrokeopacity{0.400000}%
\pgfsetdash{}{0pt}%
\pgfpathmoveto{\pgfqpoint{2.520230in}{0.557870in}}%
\pgfpathlineto{\pgfqpoint{2.744138in}{0.557870in}}%
\pgfpathlineto{\pgfqpoint{2.744138in}{0.557870in}}%
\pgfpathlineto{\pgfqpoint{2.520230in}{0.557870in}}%
\pgfpathclose%
\pgfusepath{stroke,fill}%
\end{pgfscope}%
\begin{pgfscope}%
\pgfpathrectangle{\pgfqpoint{0.617014in}{0.557870in}}{\pgfqpoint{2.462986in}{1.484734in}}%
\pgfusepath{clip}%
\pgfsetbuttcap%
\pgfsetmiterjoin%
\definecolor{currentfill}{rgb}{0.298039,0.447059,0.690196}%
\pgfsetfillcolor{currentfill}%
\pgfsetfillopacity{0.400000}%
\pgfsetlinewidth{1.003750pt}%
\definecolor{currentstroke}{rgb}{1.000000,1.000000,1.000000}%
\pgfsetstrokecolor{currentstroke}%
\pgfsetstrokeopacity{0.400000}%
\pgfsetdash{}{0pt}%
\pgfpathmoveto{\pgfqpoint{2.744138in}{0.557870in}}%
\pgfpathlineto{\pgfqpoint{2.968046in}{0.557870in}}%
\pgfpathlineto{\pgfqpoint{2.968046in}{0.557870in}}%
\pgfpathlineto{\pgfqpoint{2.744138in}{0.557870in}}%
\pgfpathclose%
\pgfusepath{stroke,fill}%
\end{pgfscope}%
\begin{pgfscope}%
\pgfsetrectcap%
\pgfsetmiterjoin%
\pgfsetlinewidth{1.254687pt}%
\definecolor{currentstroke}{rgb}{1.000000,1.000000,1.000000}%
\pgfsetstrokecolor{currentstroke}%
\pgfsetdash{}{0pt}%
\pgfpathmoveto{\pgfqpoint{0.617014in}{0.557870in}}%
\pgfpathlineto{\pgfqpoint{0.617014in}{2.042604in}}%
\pgfusepath{stroke}%
\end{pgfscope}%
\begin{pgfscope}%
\pgfsetrectcap%
\pgfsetmiterjoin%
\pgfsetlinewidth{1.254687pt}%
\definecolor{currentstroke}{rgb}{1.000000,1.000000,1.000000}%
\pgfsetstrokecolor{currentstroke}%
\pgfsetdash{}{0pt}%
\pgfpathmoveto{\pgfqpoint{3.080000in}{0.557870in}}%
\pgfpathlineto{\pgfqpoint{3.080000in}{2.042604in}}%
\pgfusepath{stroke}%
\end{pgfscope}%
\begin{pgfscope}%
\pgfsetrectcap%
\pgfsetmiterjoin%
\pgfsetlinewidth{1.254687pt}%
\definecolor{currentstroke}{rgb}{1.000000,1.000000,1.000000}%
\pgfsetstrokecolor{currentstroke}%
\pgfsetdash{}{0pt}%
\pgfpathmoveto{\pgfqpoint{0.617014in}{0.557870in}}%
\pgfpathlineto{\pgfqpoint{3.080000in}{0.557870in}}%
\pgfusepath{stroke}%
\end{pgfscope}%
\begin{pgfscope}%
\pgfsetrectcap%
\pgfsetmiterjoin%
\pgfsetlinewidth{1.254687pt}%
\definecolor{currentstroke}{rgb}{1.000000,1.000000,1.000000}%
\pgfsetstrokecolor{currentstroke}%
\pgfsetdash{}{0pt}%
\pgfpathmoveto{\pgfqpoint{0.617014in}{2.042604in}}%
\pgfpathlineto{\pgfqpoint{3.080000in}{2.042604in}}%
\pgfusepath{stroke}%
\end{pgfscope}%
\begin{pgfscope}%
\definecolor{textcolor}{rgb}{0.150000,0.150000,0.150000}%
\pgfsetstrokecolor{textcolor}%
\pgfsetfillcolor{textcolor}%
\pgftext[x=1.848507in,y=2.125938in,,base]{\color{textcolor}\sffamily\fontsize{11.000000}{13.200000}\selectfont (a)}%
\end{pgfscope}%
\begin{pgfscope}%
\pgfsetbuttcap%
\pgfsetmiterjoin%
\definecolor{currentfill}{rgb}{0.917647,0.917647,0.949020}%
\pgfsetfillcolor{currentfill}%
\pgfsetlinewidth{0.000000pt}%
\definecolor{currentstroke}{rgb}{0.000000,0.000000,0.000000}%
\pgfsetstrokecolor{currentstroke}%
\pgfsetstrokeopacity{0.000000}%
\pgfsetdash{}{0pt}%
\pgfpathmoveto{\pgfqpoint{3.777014in}{0.557870in}}%
\pgfpathlineto{\pgfqpoint{6.240000in}{0.557870in}}%
\pgfpathlineto{\pgfqpoint{6.240000in}{2.042604in}}%
\pgfpathlineto{\pgfqpoint{3.777014in}{2.042604in}}%
\pgfpathclose%
\pgfusepath{fill}%
\end{pgfscope}%
\begin{pgfscope}%
\pgfpathrectangle{\pgfqpoint{3.777014in}{0.557870in}}{\pgfqpoint{2.462986in}{1.484734in}}%
\pgfusepath{clip}%
\pgfsetroundcap%
\pgfsetroundjoin%
\pgfsetlinewidth{1.003750pt}%
\definecolor{currentstroke}{rgb}{1.000000,1.000000,1.000000}%
\pgfsetstrokecolor{currentstroke}%
\pgfsetdash{}{0pt}%
\pgfpathmoveto{\pgfqpoint{3.888968in}{0.557870in}}%
\pgfpathlineto{\pgfqpoint{3.888968in}{2.042604in}}%
\pgfusepath{stroke}%
\end{pgfscope}%
\begin{pgfscope}%
\definecolor{textcolor}{rgb}{0.150000,0.150000,0.150000}%
\pgfsetstrokecolor{textcolor}%
\pgfsetfillcolor{textcolor}%
\pgftext[x=3.888968in,y=0.425926in,,top]{\color{textcolor}\sffamily\fontsize{11.000000}{13.200000}\selectfont \(\displaystyle 0.00\)}%
\end{pgfscope}%
\begin{pgfscope}%
\pgfpathrectangle{\pgfqpoint{3.777014in}{0.557870in}}{\pgfqpoint{2.462986in}{1.484734in}}%
\pgfusepath{clip}%
\pgfsetroundcap%
\pgfsetroundjoin%
\pgfsetlinewidth{1.003750pt}%
\definecolor{currentstroke}{rgb}{1.000000,1.000000,1.000000}%
\pgfsetstrokecolor{currentstroke}%
\pgfsetdash{}{0pt}%
\pgfpathmoveto{\pgfqpoint{4.448737in}{0.557870in}}%
\pgfpathlineto{\pgfqpoint{4.448737in}{2.042604in}}%
\pgfusepath{stroke}%
\end{pgfscope}%
\begin{pgfscope}%
\definecolor{textcolor}{rgb}{0.150000,0.150000,0.150000}%
\pgfsetstrokecolor{textcolor}%
\pgfsetfillcolor{textcolor}%
\pgftext[x=4.448737in,y=0.425926in,,top]{\color{textcolor}\sffamily\fontsize{11.000000}{13.200000}\selectfont \(\displaystyle 0.25\)}%
\end{pgfscope}%
\begin{pgfscope}%
\pgfpathrectangle{\pgfqpoint{3.777014in}{0.557870in}}{\pgfqpoint{2.462986in}{1.484734in}}%
\pgfusepath{clip}%
\pgfsetroundcap%
\pgfsetroundjoin%
\pgfsetlinewidth{1.003750pt}%
\definecolor{currentstroke}{rgb}{1.000000,1.000000,1.000000}%
\pgfsetstrokecolor{currentstroke}%
\pgfsetdash{}{0pt}%
\pgfpathmoveto{\pgfqpoint{5.008507in}{0.557870in}}%
\pgfpathlineto{\pgfqpoint{5.008507in}{2.042604in}}%
\pgfusepath{stroke}%
\end{pgfscope}%
\begin{pgfscope}%
\definecolor{textcolor}{rgb}{0.150000,0.150000,0.150000}%
\pgfsetstrokecolor{textcolor}%
\pgfsetfillcolor{textcolor}%
\pgftext[x=5.008507in,y=0.425926in,,top]{\color{textcolor}\sffamily\fontsize{11.000000}{13.200000}\selectfont \(\displaystyle 0.50\)}%
\end{pgfscope}%
\begin{pgfscope}%
\pgfpathrectangle{\pgfqpoint{3.777014in}{0.557870in}}{\pgfqpoint{2.462986in}{1.484734in}}%
\pgfusepath{clip}%
\pgfsetroundcap%
\pgfsetroundjoin%
\pgfsetlinewidth{1.003750pt}%
\definecolor{currentstroke}{rgb}{1.000000,1.000000,1.000000}%
\pgfsetstrokecolor{currentstroke}%
\pgfsetdash{}{0pt}%
\pgfpathmoveto{\pgfqpoint{5.568277in}{0.557870in}}%
\pgfpathlineto{\pgfqpoint{5.568277in}{2.042604in}}%
\pgfusepath{stroke}%
\end{pgfscope}%
\begin{pgfscope}%
\definecolor{textcolor}{rgb}{0.150000,0.150000,0.150000}%
\pgfsetstrokecolor{textcolor}%
\pgfsetfillcolor{textcolor}%
\pgftext[x=5.568277in,y=0.425926in,,top]{\color{textcolor}\sffamily\fontsize{11.000000}{13.200000}\selectfont \(\displaystyle 0.75\)}%
\end{pgfscope}%
\begin{pgfscope}%
\pgfpathrectangle{\pgfqpoint{3.777014in}{0.557870in}}{\pgfqpoint{2.462986in}{1.484734in}}%
\pgfusepath{clip}%
\pgfsetroundcap%
\pgfsetroundjoin%
\pgfsetlinewidth{1.003750pt}%
\definecolor{currentstroke}{rgb}{1.000000,1.000000,1.000000}%
\pgfsetstrokecolor{currentstroke}%
\pgfsetdash{}{0pt}%
\pgfpathmoveto{\pgfqpoint{6.128046in}{0.557870in}}%
\pgfpathlineto{\pgfqpoint{6.128046in}{2.042604in}}%
\pgfusepath{stroke}%
\end{pgfscope}%
\begin{pgfscope}%
\definecolor{textcolor}{rgb}{0.150000,0.150000,0.150000}%
\pgfsetstrokecolor{textcolor}%
\pgfsetfillcolor{textcolor}%
\pgftext[x=6.128046in,y=0.425926in,,top]{\color{textcolor}\sffamily\fontsize{11.000000}{13.200000}\selectfont \(\displaystyle 1.00\)}%
\end{pgfscope}%
\begin{pgfscope}%
\definecolor{textcolor}{rgb}{0.150000,0.150000,0.150000}%
\pgfsetstrokecolor{textcolor}%
\pgfsetfillcolor{textcolor}%
\pgftext[x=5.008507in,y=0.235185in,,top]{\color{textcolor}\sffamily\fontsize{11.000000}{13.200000}\selectfont Specificity}%
\end{pgfscope}%
\begin{pgfscope}%
\pgfpathrectangle{\pgfqpoint{3.777014in}{0.557870in}}{\pgfqpoint{2.462986in}{1.484734in}}%
\pgfusepath{clip}%
\pgfsetroundcap%
\pgfsetroundjoin%
\pgfsetlinewidth{1.003750pt}%
\definecolor{currentstroke}{rgb}{1.000000,1.000000,1.000000}%
\pgfsetstrokecolor{currentstroke}%
\pgfsetdash{}{0pt}%
\pgfpathmoveto{\pgfqpoint{3.777014in}{0.625358in}}%
\pgfpathlineto{\pgfqpoint{6.240000in}{0.625358in}}%
\pgfusepath{stroke}%
\end{pgfscope}%
\begin{pgfscope}%
\definecolor{textcolor}{rgb}{0.150000,0.150000,0.150000}%
\pgfsetstrokecolor{textcolor}%
\pgfsetfillcolor{textcolor}%
\pgftext[x=3.450741in,y=0.572552in,left,base]{\color{textcolor}\sffamily\fontsize{11.000000}{13.200000}\selectfont \(\displaystyle 0.0\)}%
\end{pgfscope}%
\begin{pgfscope}%
\pgfpathrectangle{\pgfqpoint{3.777014in}{0.557870in}}{\pgfqpoint{2.462986in}{1.484734in}}%
\pgfusepath{clip}%
\pgfsetroundcap%
\pgfsetroundjoin%
\pgfsetlinewidth{1.003750pt}%
\definecolor{currentstroke}{rgb}{1.000000,1.000000,1.000000}%
\pgfsetstrokecolor{currentstroke}%
\pgfsetdash{}{0pt}%
\pgfpathmoveto{\pgfqpoint{3.777014in}{1.300237in}}%
\pgfpathlineto{\pgfqpoint{6.240000in}{1.300237in}}%
\pgfusepath{stroke}%
\end{pgfscope}%
\begin{pgfscope}%
\definecolor{textcolor}{rgb}{0.150000,0.150000,0.150000}%
\pgfsetstrokecolor{textcolor}%
\pgfsetfillcolor{textcolor}%
\pgftext[x=3.450741in,y=1.247431in,left,base]{\color{textcolor}\sffamily\fontsize{11.000000}{13.200000}\selectfont \(\displaystyle 0.5\)}%
\end{pgfscope}%
\begin{pgfscope}%
\pgfpathrectangle{\pgfqpoint{3.777014in}{0.557870in}}{\pgfqpoint{2.462986in}{1.484734in}}%
\pgfusepath{clip}%
\pgfsetroundcap%
\pgfsetroundjoin%
\pgfsetlinewidth{1.003750pt}%
\definecolor{currentstroke}{rgb}{1.000000,1.000000,1.000000}%
\pgfsetstrokecolor{currentstroke}%
\pgfsetdash{}{0pt}%
\pgfpathmoveto{\pgfqpoint{3.777014in}{1.975116in}}%
\pgfpathlineto{\pgfqpoint{6.240000in}{1.975116in}}%
\pgfusepath{stroke}%
\end{pgfscope}%
\begin{pgfscope}%
\definecolor{textcolor}{rgb}{0.150000,0.150000,0.150000}%
\pgfsetstrokecolor{textcolor}%
\pgfsetfillcolor{textcolor}%
\pgftext[x=3.450741in,y=1.922310in,left,base]{\color{textcolor}\sffamily\fontsize{11.000000}{13.200000}\selectfont \(\displaystyle 1.0\)}%
\end{pgfscope}%
\begin{pgfscope}%
\definecolor{textcolor}{rgb}{0.150000,0.150000,0.150000}%
\pgfsetstrokecolor{textcolor}%
\pgfsetfillcolor{textcolor}%
\pgftext[x=3.395185in,y=1.300237in,,bottom,rotate=90.000000]{\color{textcolor}\sffamily\fontsize{11.000000}{13.200000}\selectfont Sensitivity}%
\end{pgfscope}%
\begin{pgfscope}%
\pgfpathrectangle{\pgfqpoint{3.777014in}{0.557870in}}{\pgfqpoint{2.462986in}{1.484734in}}%
\pgfusepath{clip}%
\pgfsetbuttcap%
\pgfsetroundjoin%
\definecolor{currentfill}{rgb}{0.298039,0.447059,0.690196}%
\pgfsetfillcolor{currentfill}%
\pgfsetlinewidth{1.003750pt}%
\definecolor{currentstroke}{rgb}{0.298039,0.447059,0.690196}%
\pgfsetstrokecolor{currentstroke}%
\pgfsetdash{}{0pt}%
\pgfpathmoveto{\pgfqpoint{3.888968in}{1.935977in}}%
\pgfpathcurveto{\pgfqpoint{3.897204in}{1.935977in}}{\pgfqpoint{3.905104in}{1.939250in}}{\pgfqpoint{3.910928in}{1.945074in}}%
\pgfpathcurveto{\pgfqpoint{3.916752in}{1.950898in}}{\pgfqpoint{3.920024in}{1.958798in}}{\pgfqpoint{3.920024in}{1.967034in}}%
\pgfpathcurveto{\pgfqpoint{3.920024in}{1.975270in}}{\pgfqpoint{3.916752in}{1.983170in}}{\pgfqpoint{3.910928in}{1.988994in}}%
\pgfpathcurveto{\pgfqpoint{3.905104in}{1.994818in}}{\pgfqpoint{3.897204in}{1.998090in}}{\pgfqpoint{3.888968in}{1.998090in}}%
\pgfpathcurveto{\pgfqpoint{3.880732in}{1.998090in}}{\pgfqpoint{3.872832in}{1.994818in}}{\pgfqpoint{3.867008in}{1.988994in}}%
\pgfpathcurveto{\pgfqpoint{3.861184in}{1.983170in}}{\pgfqpoint{3.857911in}{1.975270in}}{\pgfqpoint{3.857911in}{1.967034in}}%
\pgfpathcurveto{\pgfqpoint{3.857911in}{1.958798in}}{\pgfqpoint{3.861184in}{1.950898in}}{\pgfqpoint{3.867008in}{1.945074in}}%
\pgfpathcurveto{\pgfqpoint{3.872832in}{1.939250in}}{\pgfqpoint{3.880732in}{1.935977in}}{\pgfqpoint{3.888968in}{1.935977in}}%
\pgfpathclose%
\pgfusepath{stroke,fill}%
\end{pgfscope}%
\begin{pgfscope}%
\pgfsetrectcap%
\pgfsetmiterjoin%
\pgfsetlinewidth{1.254687pt}%
\definecolor{currentstroke}{rgb}{1.000000,1.000000,1.000000}%
\pgfsetstrokecolor{currentstroke}%
\pgfsetdash{}{0pt}%
\pgfpathmoveto{\pgfqpoint{3.777014in}{0.557870in}}%
\pgfpathlineto{\pgfqpoint{3.777014in}{2.042604in}}%
\pgfusepath{stroke}%
\end{pgfscope}%
\begin{pgfscope}%
\pgfsetrectcap%
\pgfsetmiterjoin%
\pgfsetlinewidth{1.254687pt}%
\definecolor{currentstroke}{rgb}{1.000000,1.000000,1.000000}%
\pgfsetstrokecolor{currentstroke}%
\pgfsetdash{}{0pt}%
\pgfpathmoveto{\pgfqpoint{6.240000in}{0.557870in}}%
\pgfpathlineto{\pgfqpoint{6.240000in}{2.042604in}}%
\pgfusepath{stroke}%
\end{pgfscope}%
\begin{pgfscope}%
\pgfsetrectcap%
\pgfsetmiterjoin%
\pgfsetlinewidth{1.254687pt}%
\definecolor{currentstroke}{rgb}{1.000000,1.000000,1.000000}%
\pgfsetstrokecolor{currentstroke}%
\pgfsetdash{}{0pt}%
\pgfpathmoveto{\pgfqpoint{3.777014in}{0.557870in}}%
\pgfpathlineto{\pgfqpoint{6.240000in}{0.557870in}}%
\pgfusepath{stroke}%
\end{pgfscope}%
\begin{pgfscope}%
\pgfsetrectcap%
\pgfsetmiterjoin%
\pgfsetlinewidth{1.254687pt}%
\definecolor{currentstroke}{rgb}{1.000000,1.000000,1.000000}%
\pgfsetstrokecolor{currentstroke}%
\pgfsetdash{}{0pt}%
\pgfpathmoveto{\pgfqpoint{3.777014in}{2.042604in}}%
\pgfpathlineto{\pgfqpoint{6.240000in}{2.042604in}}%
\pgfusepath{stroke}%
\end{pgfscope}%
\begin{pgfscope}%
\definecolor{textcolor}{rgb}{0.150000,0.150000,0.150000}%
\pgfsetstrokecolor{textcolor}%
\pgfsetfillcolor{textcolor}%
\pgftext[x=5.008507in,y=2.125938in,,base]{\color{textcolor}\sffamily\fontsize{11.000000}{13.200000}\selectfont (b)}%
\end{pgfscope}%
\end{pgfpicture}%
\makeatother%
\endgroup%

    \caption{Distribution of DOR, sensitivity and specificity for the NN-variations trained to predict patient diagnosis.}
    \label{fig:dl_ind_dor_sens_spec_dist}
\end{figure}

\newpage

\subsection{Peak-value Classifiers}

\begin{figure}[htb]
    \centering
    % \includegraphics[width=\textwidth]{results/pvmlc_ind_dor_sens_spec_dist.png}
    %% Creator: Matplotlib, PGF backend
%%
%% To include the figure in your LaTeX document, write
%%   \input{<filename>.pgf}
%%
%% Make sure the required packages are loaded in your preamble
%%   \usepackage{pgf}
%%
%% Figures using additional raster images can only be included by \input if
%% they are in the same directory as the main LaTeX file. For loading figures
%% from other directories you can use the `import` package
%%   \usepackage{import}
%% and then include the figures with
%%   \import{<path to file>}{<filename>.pgf}
%%
%% Matplotlib used the following preamble
%%
\begingroup%
\makeatletter%
\begin{pgfpicture}%
\pgfpathrectangle{\pgfpointorigin}{\pgfqpoint{6.362271in}{2.340000in}}%
\pgfusepath{use as bounding box, clip}%
\begin{pgfscope}%
\pgfsetbuttcap%
\pgfsetmiterjoin%
\definecolor{currentfill}{rgb}{1.000000,1.000000,1.000000}%
\pgfsetfillcolor{currentfill}%
\pgfsetlinewidth{0.000000pt}%
\definecolor{currentstroke}{rgb}{1.000000,1.000000,1.000000}%
\pgfsetstrokecolor{currentstroke}%
\pgfsetdash{}{0pt}%
\pgfpathmoveto{\pgfqpoint{0.000000in}{-0.000000in}}%
\pgfpathlineto{\pgfqpoint{6.362271in}{-0.000000in}}%
\pgfpathlineto{\pgfqpoint{6.362271in}{2.340000in}}%
\pgfpathlineto{\pgfqpoint{0.000000in}{2.340000in}}%
\pgfpathclose%
\pgfusepath{fill}%
\end{pgfscope}%
\begin{pgfscope}%
\pgfsetbuttcap%
\pgfsetmiterjoin%
\definecolor{currentfill}{rgb}{0.917647,0.917647,0.949020}%
\pgfsetfillcolor{currentfill}%
\pgfsetlinewidth{0.000000pt}%
\definecolor{currentstroke}{rgb}{0.000000,0.000000,0.000000}%
\pgfsetstrokecolor{currentstroke}%
\pgfsetstrokeopacity{0.000000}%
\pgfsetdash{}{0pt}%
\pgfpathmoveto{\pgfqpoint{0.574769in}{0.557870in}}%
\pgfpathlineto{\pgfqpoint{3.058877in}{0.557870in}}%
\pgfpathlineto{\pgfqpoint{3.058877in}{2.042604in}}%
\pgfpathlineto{\pgfqpoint{0.574769in}{2.042604in}}%
\pgfpathclose%
\pgfusepath{fill}%
\end{pgfscope}%
\begin{pgfscope}%
\pgfpathrectangle{\pgfqpoint{0.574769in}{0.557870in}}{\pgfqpoint{2.484109in}{1.484734in}}%
\pgfusepath{clip}%
\pgfsetroundcap%
\pgfsetroundjoin%
\pgfsetlinewidth{1.003750pt}%
\definecolor{currentstroke}{rgb}{1.000000,1.000000,1.000000}%
\pgfsetstrokecolor{currentstroke}%
\pgfsetdash{}{0pt}%
\pgfpathmoveto{\pgfqpoint{0.626035in}{0.557870in}}%
\pgfpathlineto{\pgfqpoint{0.626035in}{2.042604in}}%
\pgfusepath{stroke}%
\end{pgfscope}%
\begin{pgfscope}%
\definecolor{textcolor}{rgb}{0.150000,0.150000,0.150000}%
\pgfsetstrokecolor{textcolor}%
\pgfsetfillcolor{textcolor}%
\pgftext[x=0.626035in,y=0.425926in,,top]{\color{textcolor}\sffamily\fontsize{11.000000}{13.200000}\selectfont \(\displaystyle 0\)}%
\end{pgfscope}%
\begin{pgfscope}%
\pgfpathrectangle{\pgfqpoint{0.574769in}{0.557870in}}{\pgfqpoint{2.484109in}{1.484734in}}%
\pgfusepath{clip}%
\pgfsetroundcap%
\pgfsetroundjoin%
\pgfsetlinewidth{1.003750pt}%
\definecolor{currentstroke}{rgb}{1.000000,1.000000,1.000000}%
\pgfsetstrokecolor{currentstroke}%
\pgfsetdash{}{0pt}%
\pgfpathmoveto{\pgfqpoint{1.464058in}{0.557870in}}%
\pgfpathlineto{\pgfqpoint{1.464058in}{2.042604in}}%
\pgfusepath{stroke}%
\end{pgfscope}%
\begin{pgfscope}%
\definecolor{textcolor}{rgb}{0.150000,0.150000,0.150000}%
\pgfsetstrokecolor{textcolor}%
\pgfsetfillcolor{textcolor}%
\pgftext[x=1.464058in,y=0.425926in,,top]{\color{textcolor}\sffamily\fontsize{11.000000}{13.200000}\selectfont \(\displaystyle 50\)}%
\end{pgfscope}%
\begin{pgfscope}%
\pgfpathrectangle{\pgfqpoint{0.574769in}{0.557870in}}{\pgfqpoint{2.484109in}{1.484734in}}%
\pgfusepath{clip}%
\pgfsetroundcap%
\pgfsetroundjoin%
\pgfsetlinewidth{1.003750pt}%
\definecolor{currentstroke}{rgb}{1.000000,1.000000,1.000000}%
\pgfsetstrokecolor{currentstroke}%
\pgfsetdash{}{0pt}%
\pgfpathmoveto{\pgfqpoint{2.302082in}{0.557870in}}%
\pgfpathlineto{\pgfqpoint{2.302082in}{2.042604in}}%
\pgfusepath{stroke}%
\end{pgfscope}%
\begin{pgfscope}%
\definecolor{textcolor}{rgb}{0.150000,0.150000,0.150000}%
\pgfsetstrokecolor{textcolor}%
\pgfsetfillcolor{textcolor}%
\pgftext[x=2.302082in,y=0.425926in,,top]{\color{textcolor}\sffamily\fontsize{11.000000}{13.200000}\selectfont \(\displaystyle 100\)}%
\end{pgfscope}%
\begin{pgfscope}%
\definecolor{textcolor}{rgb}{0.150000,0.150000,0.150000}%
\pgfsetstrokecolor{textcolor}%
\pgfsetfillcolor{textcolor}%
\pgftext[x=1.816823in,y=0.235185in,,top]{\color{textcolor}\sffamily\fontsize{11.000000}{13.200000}\selectfont DOR}%
\end{pgfscope}%
\begin{pgfscope}%
\pgfpathrectangle{\pgfqpoint{0.574769in}{0.557870in}}{\pgfqpoint{2.484109in}{1.484734in}}%
\pgfusepath{clip}%
\pgfsetroundcap%
\pgfsetroundjoin%
\pgfsetlinewidth{1.003750pt}%
\definecolor{currentstroke}{rgb}{1.000000,1.000000,1.000000}%
\pgfsetstrokecolor{currentstroke}%
\pgfsetdash{}{0pt}%
\pgfpathmoveto{\pgfqpoint{0.574769in}{0.557870in}}%
\pgfpathlineto{\pgfqpoint{3.058877in}{0.557870in}}%
\pgfusepath{stroke}%
\end{pgfscope}%
\begin{pgfscope}%
\definecolor{textcolor}{rgb}{0.150000,0.150000,0.150000}%
\pgfsetstrokecolor{textcolor}%
\pgfsetfillcolor{textcolor}%
\pgftext[x=0.366783in,y=0.505064in,left,base]{\color{textcolor}\sffamily\fontsize{11.000000}{13.200000}\selectfont \(\displaystyle 0\)}%
\end{pgfscope}%
\begin{pgfscope}%
\pgfpathrectangle{\pgfqpoint{0.574769in}{0.557870in}}{\pgfqpoint{2.484109in}{1.484734in}}%
\pgfusepath{clip}%
\pgfsetroundcap%
\pgfsetroundjoin%
\pgfsetlinewidth{1.003750pt}%
\definecolor{currentstroke}{rgb}{1.000000,1.000000,1.000000}%
\pgfsetstrokecolor{currentstroke}%
\pgfsetdash{}{0pt}%
\pgfpathmoveto{\pgfqpoint{0.574769in}{1.200612in}}%
\pgfpathlineto{\pgfqpoint{3.058877in}{1.200612in}}%
\pgfusepath{stroke}%
\end{pgfscope}%
\begin{pgfscope}%
\definecolor{textcolor}{rgb}{0.150000,0.150000,0.150000}%
\pgfsetstrokecolor{textcolor}%
\pgfsetfillcolor{textcolor}%
\pgftext[x=0.290741in,y=1.147806in,left,base]{\color{textcolor}\sffamily\fontsize{11.000000}{13.200000}\selectfont \(\displaystyle 10\)}%
\end{pgfscope}%
\begin{pgfscope}%
\pgfpathrectangle{\pgfqpoint{0.574769in}{0.557870in}}{\pgfqpoint{2.484109in}{1.484734in}}%
\pgfusepath{clip}%
\pgfsetroundcap%
\pgfsetroundjoin%
\pgfsetlinewidth{1.003750pt}%
\definecolor{currentstroke}{rgb}{1.000000,1.000000,1.000000}%
\pgfsetstrokecolor{currentstroke}%
\pgfsetdash{}{0pt}%
\pgfpathmoveto{\pgfqpoint{0.574769in}{1.843354in}}%
\pgfpathlineto{\pgfqpoint{3.058877in}{1.843354in}}%
\pgfusepath{stroke}%
\end{pgfscope}%
\begin{pgfscope}%
\definecolor{textcolor}{rgb}{0.150000,0.150000,0.150000}%
\pgfsetstrokecolor{textcolor}%
\pgfsetfillcolor{textcolor}%
\pgftext[x=0.290741in,y=1.790547in,left,base]{\color{textcolor}\sffamily\fontsize{11.000000}{13.200000}\selectfont \(\displaystyle 20\)}%
\end{pgfscope}%
\begin{pgfscope}%
\definecolor{textcolor}{rgb}{0.150000,0.150000,0.150000}%
\pgfsetstrokecolor{textcolor}%
\pgfsetfillcolor{textcolor}%
\pgftext[x=0.235185in,y=1.300237in,,bottom,rotate=90.000000]{\color{textcolor}\sffamily\fontsize{11.000000}{13.200000}\selectfont Occurance}%
\end{pgfscope}%
\begin{pgfscope}%
\pgfpathrectangle{\pgfqpoint{0.574769in}{0.557870in}}{\pgfqpoint{2.484109in}{1.484734in}}%
\pgfusepath{clip}%
\pgfsetbuttcap%
\pgfsetmiterjoin%
\definecolor{currentfill}{rgb}{0.298039,0.447059,0.690196}%
\pgfsetfillcolor{currentfill}%
\pgfsetfillopacity{0.400000}%
\pgfsetlinewidth{1.003750pt}%
\definecolor{currentstroke}{rgb}{1.000000,1.000000,1.000000}%
\pgfsetstrokecolor{currentstroke}%
\pgfsetstrokeopacity{0.400000}%
\pgfsetdash{}{0pt}%
\pgfpathmoveto{\pgfqpoint{0.687683in}{0.557870in}}%
\pgfpathlineto{\pgfqpoint{0.913511in}{0.557870in}}%
\pgfpathlineto{\pgfqpoint{0.913511in}{1.971903in}}%
\pgfpathlineto{\pgfqpoint{0.687683in}{1.971903in}}%
\pgfpathclose%
\pgfusepath{stroke,fill}%
\end{pgfscope}%
\begin{pgfscope}%
\pgfpathrectangle{\pgfqpoint{0.574769in}{0.557870in}}{\pgfqpoint{2.484109in}{1.484734in}}%
\pgfusepath{clip}%
\pgfsetbuttcap%
\pgfsetmiterjoin%
\definecolor{currentfill}{rgb}{0.298039,0.447059,0.690196}%
\pgfsetfillcolor{currentfill}%
\pgfsetfillopacity{0.400000}%
\pgfsetlinewidth{1.003750pt}%
\definecolor{currentstroke}{rgb}{1.000000,1.000000,1.000000}%
\pgfsetstrokecolor{currentstroke}%
\pgfsetstrokeopacity{0.400000}%
\pgfsetdash{}{0pt}%
\pgfpathmoveto{\pgfqpoint{0.913511in}{0.557870in}}%
\pgfpathlineto{\pgfqpoint{1.139339in}{0.557870in}}%
\pgfpathlineto{\pgfqpoint{1.139339in}{1.521983in}}%
\pgfpathlineto{\pgfqpoint{0.913511in}{1.521983in}}%
\pgfpathclose%
\pgfusepath{stroke,fill}%
\end{pgfscope}%
\begin{pgfscope}%
\pgfpathrectangle{\pgfqpoint{0.574769in}{0.557870in}}{\pgfqpoint{2.484109in}{1.484734in}}%
\pgfusepath{clip}%
\pgfsetbuttcap%
\pgfsetmiterjoin%
\definecolor{currentfill}{rgb}{0.298039,0.447059,0.690196}%
\pgfsetfillcolor{currentfill}%
\pgfsetfillopacity{0.400000}%
\pgfsetlinewidth{1.003750pt}%
\definecolor{currentstroke}{rgb}{1.000000,1.000000,1.000000}%
\pgfsetstrokecolor{currentstroke}%
\pgfsetstrokeopacity{0.400000}%
\pgfsetdash{}{0pt}%
\pgfpathmoveto{\pgfqpoint{1.139339in}{0.557870in}}%
\pgfpathlineto{\pgfqpoint{1.365167in}{0.557870in}}%
\pgfpathlineto{\pgfqpoint{1.365167in}{0.750693in}}%
\pgfpathlineto{\pgfqpoint{1.139339in}{0.750693in}}%
\pgfpathclose%
\pgfusepath{stroke,fill}%
\end{pgfscope}%
\begin{pgfscope}%
\pgfpathrectangle{\pgfqpoint{0.574769in}{0.557870in}}{\pgfqpoint{2.484109in}{1.484734in}}%
\pgfusepath{clip}%
\pgfsetbuttcap%
\pgfsetmiterjoin%
\definecolor{currentfill}{rgb}{0.298039,0.447059,0.690196}%
\pgfsetfillcolor{currentfill}%
\pgfsetfillopacity{0.400000}%
\pgfsetlinewidth{1.003750pt}%
\definecolor{currentstroke}{rgb}{1.000000,1.000000,1.000000}%
\pgfsetstrokecolor{currentstroke}%
\pgfsetstrokeopacity{0.400000}%
\pgfsetdash{}{0pt}%
\pgfpathmoveto{\pgfqpoint{1.365167in}{0.557870in}}%
\pgfpathlineto{\pgfqpoint{1.590995in}{0.557870in}}%
\pgfpathlineto{\pgfqpoint{1.590995in}{1.007790in}}%
\pgfpathlineto{\pgfqpoint{1.365167in}{1.007790in}}%
\pgfpathclose%
\pgfusepath{stroke,fill}%
\end{pgfscope}%
\begin{pgfscope}%
\pgfpathrectangle{\pgfqpoint{0.574769in}{0.557870in}}{\pgfqpoint{2.484109in}{1.484734in}}%
\pgfusepath{clip}%
\pgfsetbuttcap%
\pgfsetmiterjoin%
\definecolor{currentfill}{rgb}{0.298039,0.447059,0.690196}%
\pgfsetfillcolor{currentfill}%
\pgfsetfillopacity{0.400000}%
\pgfsetlinewidth{1.003750pt}%
\definecolor{currentstroke}{rgb}{1.000000,1.000000,1.000000}%
\pgfsetstrokecolor{currentstroke}%
\pgfsetstrokeopacity{0.400000}%
\pgfsetdash{}{0pt}%
\pgfpathmoveto{\pgfqpoint{1.590995in}{0.557870in}}%
\pgfpathlineto{\pgfqpoint{1.816823in}{0.557870in}}%
\pgfpathlineto{\pgfqpoint{1.816823in}{0.879241in}}%
\pgfpathlineto{\pgfqpoint{1.590995in}{0.879241in}}%
\pgfpathclose%
\pgfusepath{stroke,fill}%
\end{pgfscope}%
\begin{pgfscope}%
\pgfpathrectangle{\pgfqpoint{0.574769in}{0.557870in}}{\pgfqpoint{2.484109in}{1.484734in}}%
\pgfusepath{clip}%
\pgfsetbuttcap%
\pgfsetmiterjoin%
\definecolor{currentfill}{rgb}{0.298039,0.447059,0.690196}%
\pgfsetfillcolor{currentfill}%
\pgfsetfillopacity{0.400000}%
\pgfsetlinewidth{1.003750pt}%
\definecolor{currentstroke}{rgb}{1.000000,1.000000,1.000000}%
\pgfsetstrokecolor{currentstroke}%
\pgfsetstrokeopacity{0.400000}%
\pgfsetdash{}{0pt}%
\pgfpathmoveto{\pgfqpoint{1.816823in}{0.557870in}}%
\pgfpathlineto{\pgfqpoint{2.042651in}{0.557870in}}%
\pgfpathlineto{\pgfqpoint{2.042651in}{0.814967in}}%
\pgfpathlineto{\pgfqpoint{1.816823in}{0.814967in}}%
\pgfpathclose%
\pgfusepath{stroke,fill}%
\end{pgfscope}%
\begin{pgfscope}%
\pgfpathrectangle{\pgfqpoint{0.574769in}{0.557870in}}{\pgfqpoint{2.484109in}{1.484734in}}%
\pgfusepath{clip}%
\pgfsetbuttcap%
\pgfsetmiterjoin%
\definecolor{currentfill}{rgb}{0.298039,0.447059,0.690196}%
\pgfsetfillcolor{currentfill}%
\pgfsetfillopacity{0.400000}%
\pgfsetlinewidth{1.003750pt}%
\definecolor{currentstroke}{rgb}{1.000000,1.000000,1.000000}%
\pgfsetstrokecolor{currentstroke}%
\pgfsetstrokeopacity{0.400000}%
\pgfsetdash{}{0pt}%
\pgfpathmoveto{\pgfqpoint{2.042651in}{0.557870in}}%
\pgfpathlineto{\pgfqpoint{2.268479in}{0.557870in}}%
\pgfpathlineto{\pgfqpoint{2.268479in}{0.622144in}}%
\pgfpathlineto{\pgfqpoint{2.042651in}{0.622144in}}%
\pgfpathclose%
\pgfusepath{stroke,fill}%
\end{pgfscope}%
\begin{pgfscope}%
\pgfpathrectangle{\pgfqpoint{0.574769in}{0.557870in}}{\pgfqpoint{2.484109in}{1.484734in}}%
\pgfusepath{clip}%
\pgfsetbuttcap%
\pgfsetmiterjoin%
\definecolor{currentfill}{rgb}{0.298039,0.447059,0.690196}%
\pgfsetfillcolor{currentfill}%
\pgfsetfillopacity{0.400000}%
\pgfsetlinewidth{1.003750pt}%
\definecolor{currentstroke}{rgb}{1.000000,1.000000,1.000000}%
\pgfsetstrokecolor{currentstroke}%
\pgfsetstrokeopacity{0.400000}%
\pgfsetdash{}{0pt}%
\pgfpathmoveto{\pgfqpoint{2.268479in}{0.557870in}}%
\pgfpathlineto{\pgfqpoint{2.494307in}{0.557870in}}%
\pgfpathlineto{\pgfqpoint{2.494307in}{0.557870in}}%
\pgfpathlineto{\pgfqpoint{2.268479in}{0.557870in}}%
\pgfpathclose%
\pgfusepath{stroke,fill}%
\end{pgfscope}%
\begin{pgfscope}%
\pgfpathrectangle{\pgfqpoint{0.574769in}{0.557870in}}{\pgfqpoint{2.484109in}{1.484734in}}%
\pgfusepath{clip}%
\pgfsetbuttcap%
\pgfsetmiterjoin%
\definecolor{currentfill}{rgb}{0.298039,0.447059,0.690196}%
\pgfsetfillcolor{currentfill}%
\pgfsetfillopacity{0.400000}%
\pgfsetlinewidth{1.003750pt}%
\definecolor{currentstroke}{rgb}{1.000000,1.000000,1.000000}%
\pgfsetstrokecolor{currentstroke}%
\pgfsetstrokeopacity{0.400000}%
\pgfsetdash{}{0pt}%
\pgfpathmoveto{\pgfqpoint{2.494307in}{0.557870in}}%
\pgfpathlineto{\pgfqpoint{2.720135in}{0.557870in}}%
\pgfpathlineto{\pgfqpoint{2.720135in}{0.557870in}}%
\pgfpathlineto{\pgfqpoint{2.494307in}{0.557870in}}%
\pgfpathclose%
\pgfusepath{stroke,fill}%
\end{pgfscope}%
\begin{pgfscope}%
\pgfpathrectangle{\pgfqpoint{0.574769in}{0.557870in}}{\pgfqpoint{2.484109in}{1.484734in}}%
\pgfusepath{clip}%
\pgfsetbuttcap%
\pgfsetmiterjoin%
\definecolor{currentfill}{rgb}{0.298039,0.447059,0.690196}%
\pgfsetfillcolor{currentfill}%
\pgfsetfillopacity{0.400000}%
\pgfsetlinewidth{1.003750pt}%
\definecolor{currentstroke}{rgb}{1.000000,1.000000,1.000000}%
\pgfsetstrokecolor{currentstroke}%
\pgfsetstrokeopacity{0.400000}%
\pgfsetdash{}{0pt}%
\pgfpathmoveto{\pgfqpoint{2.720135in}{0.557870in}}%
\pgfpathlineto{\pgfqpoint{2.945963in}{0.557870in}}%
\pgfpathlineto{\pgfqpoint{2.945963in}{0.622144in}}%
\pgfpathlineto{\pgfqpoint{2.720135in}{0.622144in}}%
\pgfpathclose%
\pgfusepath{stroke,fill}%
\end{pgfscope}%
\begin{pgfscope}%
\pgfsetrectcap%
\pgfsetmiterjoin%
\pgfsetlinewidth{1.254687pt}%
\definecolor{currentstroke}{rgb}{1.000000,1.000000,1.000000}%
\pgfsetstrokecolor{currentstroke}%
\pgfsetdash{}{0pt}%
\pgfpathmoveto{\pgfqpoint{0.574769in}{0.557870in}}%
\pgfpathlineto{\pgfqpoint{0.574769in}{2.042604in}}%
\pgfusepath{stroke}%
\end{pgfscope}%
\begin{pgfscope}%
\pgfsetrectcap%
\pgfsetmiterjoin%
\pgfsetlinewidth{1.254687pt}%
\definecolor{currentstroke}{rgb}{1.000000,1.000000,1.000000}%
\pgfsetstrokecolor{currentstroke}%
\pgfsetdash{}{0pt}%
\pgfpathmoveto{\pgfqpoint{3.058877in}{0.557870in}}%
\pgfpathlineto{\pgfqpoint{3.058877in}{2.042604in}}%
\pgfusepath{stroke}%
\end{pgfscope}%
\begin{pgfscope}%
\pgfsetrectcap%
\pgfsetmiterjoin%
\pgfsetlinewidth{1.254687pt}%
\definecolor{currentstroke}{rgb}{1.000000,1.000000,1.000000}%
\pgfsetstrokecolor{currentstroke}%
\pgfsetdash{}{0pt}%
\pgfpathmoveto{\pgfqpoint{0.574769in}{0.557870in}}%
\pgfpathlineto{\pgfqpoint{3.058877in}{0.557870in}}%
\pgfusepath{stroke}%
\end{pgfscope}%
\begin{pgfscope}%
\pgfsetrectcap%
\pgfsetmiterjoin%
\pgfsetlinewidth{1.254687pt}%
\definecolor{currentstroke}{rgb}{1.000000,1.000000,1.000000}%
\pgfsetstrokecolor{currentstroke}%
\pgfsetdash{}{0pt}%
\pgfpathmoveto{\pgfqpoint{0.574769in}{2.042604in}}%
\pgfpathlineto{\pgfqpoint{3.058877in}{2.042604in}}%
\pgfusepath{stroke}%
\end{pgfscope}%
\begin{pgfscope}%
\definecolor{textcolor}{rgb}{0.150000,0.150000,0.150000}%
\pgfsetstrokecolor{textcolor}%
\pgfsetfillcolor{textcolor}%
\pgftext[x=1.816823in,y=2.125938in,,base]{\color{textcolor}\sffamily\fontsize{11.000000}{13.200000}\selectfont (a)}%
\end{pgfscope}%
\begin{pgfscope}%
\pgfsetbuttcap%
\pgfsetmiterjoin%
\definecolor{currentfill}{rgb}{0.917647,0.917647,0.949020}%
\pgfsetfillcolor{currentfill}%
\pgfsetlinewidth{0.000000pt}%
\definecolor{currentstroke}{rgb}{0.000000,0.000000,0.000000}%
\pgfsetstrokecolor{currentstroke}%
\pgfsetstrokeopacity{0.000000}%
\pgfsetdash{}{0pt}%
\pgfpathmoveto{\pgfqpoint{3.755891in}{0.557870in}}%
\pgfpathlineto{\pgfqpoint{6.240000in}{0.557870in}}%
\pgfpathlineto{\pgfqpoint{6.240000in}{2.042604in}}%
\pgfpathlineto{\pgfqpoint{3.755891in}{2.042604in}}%
\pgfpathclose%
\pgfusepath{fill}%
\end{pgfscope}%
\begin{pgfscope}%
\pgfpathrectangle{\pgfqpoint{3.755891in}{0.557870in}}{\pgfqpoint{2.484109in}{1.484734in}}%
\pgfusepath{clip}%
\pgfsetroundcap%
\pgfsetroundjoin%
\pgfsetlinewidth{1.003750pt}%
\definecolor{currentstroke}{rgb}{1.000000,1.000000,1.000000}%
\pgfsetstrokecolor{currentstroke}%
\pgfsetdash{}{0pt}%
\pgfpathmoveto{\pgfqpoint{3.868805in}{0.557870in}}%
\pgfpathlineto{\pgfqpoint{3.868805in}{2.042604in}}%
\pgfusepath{stroke}%
\end{pgfscope}%
\begin{pgfscope}%
\definecolor{textcolor}{rgb}{0.150000,0.150000,0.150000}%
\pgfsetstrokecolor{textcolor}%
\pgfsetfillcolor{textcolor}%
\pgftext[x=3.868805in,y=0.425926in,,top]{\color{textcolor}\sffamily\fontsize{11.000000}{13.200000}\selectfont \(\displaystyle 0.00\)}%
\end{pgfscope}%
\begin{pgfscope}%
\pgfpathrectangle{\pgfqpoint{3.755891in}{0.557870in}}{\pgfqpoint{2.484109in}{1.484734in}}%
\pgfusepath{clip}%
\pgfsetroundcap%
\pgfsetroundjoin%
\pgfsetlinewidth{1.003750pt}%
\definecolor{currentstroke}{rgb}{1.000000,1.000000,1.000000}%
\pgfsetstrokecolor{currentstroke}%
\pgfsetdash{}{0pt}%
\pgfpathmoveto{\pgfqpoint{4.433376in}{0.557870in}}%
\pgfpathlineto{\pgfqpoint{4.433376in}{2.042604in}}%
\pgfusepath{stroke}%
\end{pgfscope}%
\begin{pgfscope}%
\definecolor{textcolor}{rgb}{0.150000,0.150000,0.150000}%
\pgfsetstrokecolor{textcolor}%
\pgfsetfillcolor{textcolor}%
\pgftext[x=4.433376in,y=0.425926in,,top]{\color{textcolor}\sffamily\fontsize{11.000000}{13.200000}\selectfont \(\displaystyle 0.25\)}%
\end{pgfscope}%
\begin{pgfscope}%
\pgfpathrectangle{\pgfqpoint{3.755891in}{0.557870in}}{\pgfqpoint{2.484109in}{1.484734in}}%
\pgfusepath{clip}%
\pgfsetroundcap%
\pgfsetroundjoin%
\pgfsetlinewidth{1.003750pt}%
\definecolor{currentstroke}{rgb}{1.000000,1.000000,1.000000}%
\pgfsetstrokecolor{currentstroke}%
\pgfsetdash{}{0pt}%
\pgfpathmoveto{\pgfqpoint{4.997946in}{0.557870in}}%
\pgfpathlineto{\pgfqpoint{4.997946in}{2.042604in}}%
\pgfusepath{stroke}%
\end{pgfscope}%
\begin{pgfscope}%
\definecolor{textcolor}{rgb}{0.150000,0.150000,0.150000}%
\pgfsetstrokecolor{textcolor}%
\pgfsetfillcolor{textcolor}%
\pgftext[x=4.997946in,y=0.425926in,,top]{\color{textcolor}\sffamily\fontsize{11.000000}{13.200000}\selectfont \(\displaystyle 0.50\)}%
\end{pgfscope}%
\begin{pgfscope}%
\pgfpathrectangle{\pgfqpoint{3.755891in}{0.557870in}}{\pgfqpoint{2.484109in}{1.484734in}}%
\pgfusepath{clip}%
\pgfsetroundcap%
\pgfsetroundjoin%
\pgfsetlinewidth{1.003750pt}%
\definecolor{currentstroke}{rgb}{1.000000,1.000000,1.000000}%
\pgfsetstrokecolor{currentstroke}%
\pgfsetdash{}{0pt}%
\pgfpathmoveto{\pgfqpoint{5.562516in}{0.557870in}}%
\pgfpathlineto{\pgfqpoint{5.562516in}{2.042604in}}%
\pgfusepath{stroke}%
\end{pgfscope}%
\begin{pgfscope}%
\definecolor{textcolor}{rgb}{0.150000,0.150000,0.150000}%
\pgfsetstrokecolor{textcolor}%
\pgfsetfillcolor{textcolor}%
\pgftext[x=5.562516in,y=0.425926in,,top]{\color{textcolor}\sffamily\fontsize{11.000000}{13.200000}\selectfont \(\displaystyle 0.75\)}%
\end{pgfscope}%
\begin{pgfscope}%
\pgfpathrectangle{\pgfqpoint{3.755891in}{0.557870in}}{\pgfqpoint{2.484109in}{1.484734in}}%
\pgfusepath{clip}%
\pgfsetroundcap%
\pgfsetroundjoin%
\pgfsetlinewidth{1.003750pt}%
\definecolor{currentstroke}{rgb}{1.000000,1.000000,1.000000}%
\pgfsetstrokecolor{currentstroke}%
\pgfsetdash{}{0pt}%
\pgfpathmoveto{\pgfqpoint{6.127086in}{0.557870in}}%
\pgfpathlineto{\pgfqpoint{6.127086in}{2.042604in}}%
\pgfusepath{stroke}%
\end{pgfscope}%
\begin{pgfscope}%
\definecolor{textcolor}{rgb}{0.150000,0.150000,0.150000}%
\pgfsetstrokecolor{textcolor}%
\pgfsetfillcolor{textcolor}%
\pgftext[x=6.127086in,y=0.425926in,,top]{\color{textcolor}\sffamily\fontsize{11.000000}{13.200000}\selectfont \(\displaystyle 1.00\)}%
\end{pgfscope}%
\begin{pgfscope}%
\definecolor{textcolor}{rgb}{0.150000,0.150000,0.150000}%
\pgfsetstrokecolor{textcolor}%
\pgfsetfillcolor{textcolor}%
\pgftext[x=4.997946in,y=0.235185in,,top]{\color{textcolor}\sffamily\fontsize{11.000000}{13.200000}\selectfont Specificity}%
\end{pgfscope}%
\begin{pgfscope}%
\pgfpathrectangle{\pgfqpoint{3.755891in}{0.557870in}}{\pgfqpoint{2.484109in}{1.484734in}}%
\pgfusepath{clip}%
\pgfsetroundcap%
\pgfsetroundjoin%
\pgfsetlinewidth{1.003750pt}%
\definecolor{currentstroke}{rgb}{1.000000,1.000000,1.000000}%
\pgfsetstrokecolor{currentstroke}%
\pgfsetdash{}{0pt}%
\pgfpathmoveto{\pgfqpoint{3.755891in}{0.625358in}}%
\pgfpathlineto{\pgfqpoint{6.240000in}{0.625358in}}%
\pgfusepath{stroke}%
\end{pgfscope}%
\begin{pgfscope}%
\definecolor{textcolor}{rgb}{0.150000,0.150000,0.150000}%
\pgfsetstrokecolor{textcolor}%
\pgfsetfillcolor{textcolor}%
\pgftext[x=3.429618in,y=0.572552in,left,base]{\color{textcolor}\sffamily\fontsize{11.000000}{13.200000}\selectfont \(\displaystyle 0.0\)}%
\end{pgfscope}%
\begin{pgfscope}%
\pgfpathrectangle{\pgfqpoint{3.755891in}{0.557870in}}{\pgfqpoint{2.484109in}{1.484734in}}%
\pgfusepath{clip}%
\pgfsetroundcap%
\pgfsetroundjoin%
\pgfsetlinewidth{1.003750pt}%
\definecolor{currentstroke}{rgb}{1.000000,1.000000,1.000000}%
\pgfsetstrokecolor{currentstroke}%
\pgfsetdash{}{0pt}%
\pgfpathmoveto{\pgfqpoint{3.755891in}{1.300237in}}%
\pgfpathlineto{\pgfqpoint{6.240000in}{1.300237in}}%
\pgfusepath{stroke}%
\end{pgfscope}%
\begin{pgfscope}%
\definecolor{textcolor}{rgb}{0.150000,0.150000,0.150000}%
\pgfsetstrokecolor{textcolor}%
\pgfsetfillcolor{textcolor}%
\pgftext[x=3.429618in,y=1.247431in,left,base]{\color{textcolor}\sffamily\fontsize{11.000000}{13.200000}\selectfont \(\displaystyle 0.5\)}%
\end{pgfscope}%
\begin{pgfscope}%
\pgfpathrectangle{\pgfqpoint{3.755891in}{0.557870in}}{\pgfqpoint{2.484109in}{1.484734in}}%
\pgfusepath{clip}%
\pgfsetroundcap%
\pgfsetroundjoin%
\pgfsetlinewidth{1.003750pt}%
\definecolor{currentstroke}{rgb}{1.000000,1.000000,1.000000}%
\pgfsetstrokecolor{currentstroke}%
\pgfsetdash{}{0pt}%
\pgfpathmoveto{\pgfqpoint{3.755891in}{1.975116in}}%
\pgfpathlineto{\pgfqpoint{6.240000in}{1.975116in}}%
\pgfusepath{stroke}%
\end{pgfscope}%
\begin{pgfscope}%
\definecolor{textcolor}{rgb}{0.150000,0.150000,0.150000}%
\pgfsetstrokecolor{textcolor}%
\pgfsetfillcolor{textcolor}%
\pgftext[x=3.429618in,y=1.922310in,left,base]{\color{textcolor}\sffamily\fontsize{11.000000}{13.200000}\selectfont \(\displaystyle 1.0\)}%
\end{pgfscope}%
\begin{pgfscope}%
\definecolor{textcolor}{rgb}{0.150000,0.150000,0.150000}%
\pgfsetstrokecolor{textcolor}%
\pgfsetfillcolor{textcolor}%
\pgftext[x=3.374062in,y=1.300237in,,bottom,rotate=90.000000]{\color{textcolor}\sffamily\fontsize{11.000000}{13.200000}\selectfont Sensitivity}%
\end{pgfscope}%
\begin{pgfscope}%
\pgfpathrectangle{\pgfqpoint{3.755891in}{0.557870in}}{\pgfqpoint{2.484109in}{1.484734in}}%
\pgfusepath{clip}%
\pgfsetbuttcap%
\pgfsetroundjoin%
\definecolor{currentfill}{rgb}{0.298039,0.447059,0.690196}%
\pgfsetfillcolor{currentfill}%
\pgfsetlinewidth{1.003750pt}%
\definecolor{currentstroke}{rgb}{0.298039,0.447059,0.690196}%
\pgfsetstrokecolor{currentstroke}%
\pgfsetdash{}{0pt}%
\pgfpathmoveto{\pgfqpoint{4.160196in}{1.919218in}}%
\pgfpathcurveto{\pgfqpoint{4.168433in}{1.919218in}}{\pgfqpoint{4.176333in}{1.922490in}}{\pgfqpoint{4.182157in}{1.928314in}}%
\pgfpathcurveto{\pgfqpoint{4.187981in}{1.934138in}}{\pgfqpoint{4.191253in}{1.942038in}}{\pgfqpoint{4.191253in}{1.950274in}}%
\pgfpathcurveto{\pgfqpoint{4.191253in}{1.958510in}}{\pgfqpoint{4.187981in}{1.966410in}}{\pgfqpoint{4.182157in}{1.972234in}}%
\pgfpathcurveto{\pgfqpoint{4.176333in}{1.978058in}}{\pgfqpoint{4.168433in}{1.981331in}}{\pgfqpoint{4.160196in}{1.981331in}}%
\pgfpathcurveto{\pgfqpoint{4.151960in}{1.981331in}}{\pgfqpoint{4.144060in}{1.978058in}}{\pgfqpoint{4.138236in}{1.972234in}}%
\pgfpathcurveto{\pgfqpoint{4.132412in}{1.966410in}}{\pgfqpoint{4.129140in}{1.958510in}}{\pgfqpoint{4.129140in}{1.950274in}}%
\pgfpathcurveto{\pgfqpoint{4.129140in}{1.942038in}}{\pgfqpoint{4.132412in}{1.934138in}}{\pgfqpoint{4.138236in}{1.928314in}}%
\pgfpathcurveto{\pgfqpoint{4.144060in}{1.922490in}}{\pgfqpoint{4.151960in}{1.919218in}}{\pgfqpoint{4.160196in}{1.919218in}}%
\pgfpathclose%
\pgfusepath{stroke,fill}%
\end{pgfscope}%
\begin{pgfscope}%
\pgfpathrectangle{\pgfqpoint{3.755891in}{0.557870in}}{\pgfqpoint{2.484109in}{1.484734in}}%
\pgfusepath{clip}%
\pgfsetbuttcap%
\pgfsetroundjoin%
\definecolor{currentfill}{rgb}{0.298039,0.447059,0.690196}%
\pgfsetfillcolor{currentfill}%
\pgfsetlinewidth{1.003750pt}%
\definecolor{currentstroke}{rgb}{0.298039,0.447059,0.690196}%
\pgfsetstrokecolor{currentstroke}%
\pgfsetdash{}{0pt}%
\pgfpathmoveto{\pgfqpoint{4.815826in}{1.852972in}}%
\pgfpathcurveto{\pgfqpoint{4.824063in}{1.852972in}}{\pgfqpoint{4.831963in}{1.856244in}}{\pgfqpoint{4.837787in}{1.862068in}}%
\pgfpathcurveto{\pgfqpoint{4.843610in}{1.867892in}}{\pgfqpoint{4.846883in}{1.875792in}}{\pgfqpoint{4.846883in}{1.884028in}}%
\pgfpathcurveto{\pgfqpoint{4.846883in}{1.892265in}}{\pgfqpoint{4.843610in}{1.900165in}}{\pgfqpoint{4.837787in}{1.905989in}}%
\pgfpathcurveto{\pgfqpoint{4.831963in}{1.911812in}}{\pgfqpoint{4.824063in}{1.915085in}}{\pgfqpoint{4.815826in}{1.915085in}}%
\pgfpathcurveto{\pgfqpoint{4.807590in}{1.915085in}}{\pgfqpoint{4.799690in}{1.911812in}}{\pgfqpoint{4.793866in}{1.905989in}}%
\pgfpathcurveto{\pgfqpoint{4.788042in}{1.900165in}}{\pgfqpoint{4.784770in}{1.892265in}}{\pgfqpoint{4.784770in}{1.884028in}}%
\pgfpathcurveto{\pgfqpoint{4.784770in}{1.875792in}}{\pgfqpoint{4.788042in}{1.867892in}}{\pgfqpoint{4.793866in}{1.862068in}}%
\pgfpathcurveto{\pgfqpoint{4.799690in}{1.856244in}}{\pgfqpoint{4.807590in}{1.852972in}}{\pgfqpoint{4.815826in}{1.852972in}}%
\pgfpathclose%
\pgfusepath{stroke,fill}%
\end{pgfscope}%
\begin{pgfscope}%
\pgfpathrectangle{\pgfqpoint{3.755891in}{0.557870in}}{\pgfqpoint{2.484109in}{1.484734in}}%
\pgfusepath{clip}%
\pgfsetbuttcap%
\pgfsetroundjoin%
\definecolor{currentfill}{rgb}{0.298039,0.447059,0.690196}%
\pgfsetfillcolor{currentfill}%
\pgfsetlinewidth{1.003750pt}%
\definecolor{currentstroke}{rgb}{0.298039,0.447059,0.690196}%
\pgfsetstrokecolor{currentstroke}%
\pgfsetdash{}{0pt}%
\pgfpathmoveto{\pgfqpoint{4.014501in}{1.927498in}}%
\pgfpathcurveto{\pgfqpoint{4.022737in}{1.927498in}}{\pgfqpoint{4.030637in}{1.930771in}}{\pgfqpoint{4.036461in}{1.936595in}}%
\pgfpathcurveto{\pgfqpoint{4.042285in}{1.942418in}}{\pgfqpoint{4.045557in}{1.950319in}}{\pgfqpoint{4.045557in}{1.958555in}}%
\pgfpathcurveto{\pgfqpoint{4.045557in}{1.966791in}}{\pgfqpoint{4.042285in}{1.974691in}}{\pgfqpoint{4.036461in}{1.980515in}}%
\pgfpathcurveto{\pgfqpoint{4.030637in}{1.986339in}}{\pgfqpoint{4.022737in}{1.989611in}}{\pgfqpoint{4.014501in}{1.989611in}}%
\pgfpathcurveto{\pgfqpoint{4.006265in}{1.989611in}}{\pgfqpoint{3.998365in}{1.986339in}}{\pgfqpoint{3.992541in}{1.980515in}}%
\pgfpathcurveto{\pgfqpoint{3.986717in}{1.974691in}}{\pgfqpoint{3.983444in}{1.966791in}}{\pgfqpoint{3.983444in}{1.958555in}}%
\pgfpathcurveto{\pgfqpoint{3.983444in}{1.950319in}}{\pgfqpoint{3.986717in}{1.942418in}}{\pgfqpoint{3.992541in}{1.936595in}}%
\pgfpathcurveto{\pgfqpoint{3.998365in}{1.930771in}}{\pgfqpoint{4.006265in}{1.927498in}}{\pgfqpoint{4.014501in}{1.927498in}}%
\pgfpathclose%
\pgfusepath{stroke,fill}%
\end{pgfscope}%
\begin{pgfscope}%
\pgfpathrectangle{\pgfqpoint{3.755891in}{0.557870in}}{\pgfqpoint{2.484109in}{1.484734in}}%
\pgfusepath{clip}%
\pgfsetbuttcap%
\pgfsetroundjoin%
\definecolor{currentfill}{rgb}{0.298039,0.447059,0.690196}%
\pgfsetfillcolor{currentfill}%
\pgfsetlinewidth{1.003750pt}%
\definecolor{currentstroke}{rgb}{0.298039,0.447059,0.690196}%
\pgfsetstrokecolor{currentstroke}%
\pgfsetdash{}{0pt}%
\pgfpathmoveto{\pgfqpoint{4.014501in}{1.919218in}}%
\pgfpathcurveto{\pgfqpoint{4.022737in}{1.919218in}}{\pgfqpoint{4.030637in}{1.922490in}}{\pgfqpoint{4.036461in}{1.928314in}}%
\pgfpathcurveto{\pgfqpoint{4.042285in}{1.934138in}}{\pgfqpoint{4.045557in}{1.942038in}}{\pgfqpoint{4.045557in}{1.950274in}}%
\pgfpathcurveto{\pgfqpoint{4.045557in}{1.958510in}}{\pgfqpoint{4.042285in}{1.966410in}}{\pgfqpoint{4.036461in}{1.972234in}}%
\pgfpathcurveto{\pgfqpoint{4.030637in}{1.978058in}}{\pgfqpoint{4.022737in}{1.981331in}}{\pgfqpoint{4.014501in}{1.981331in}}%
\pgfpathcurveto{\pgfqpoint{4.006265in}{1.981331in}}{\pgfqpoint{3.998365in}{1.978058in}}{\pgfqpoint{3.992541in}{1.972234in}}%
\pgfpathcurveto{\pgfqpoint{3.986717in}{1.966410in}}{\pgfqpoint{3.983444in}{1.958510in}}{\pgfqpoint{3.983444in}{1.950274in}}%
\pgfpathcurveto{\pgfqpoint{3.983444in}{1.942038in}}{\pgfqpoint{3.986717in}{1.934138in}}{\pgfqpoint{3.992541in}{1.928314in}}%
\pgfpathcurveto{\pgfqpoint{3.998365in}{1.922490in}}{\pgfqpoint{4.006265in}{1.919218in}}{\pgfqpoint{4.014501in}{1.919218in}}%
\pgfpathclose%
\pgfusepath{stroke,fill}%
\end{pgfscope}%
\begin{pgfscope}%
\pgfpathrectangle{\pgfqpoint{3.755891in}{0.557870in}}{\pgfqpoint{2.484109in}{1.484734in}}%
\pgfusepath{clip}%
\pgfsetbuttcap%
\pgfsetroundjoin%
\definecolor{currentfill}{rgb}{0.298039,0.447059,0.690196}%
\pgfsetfillcolor{currentfill}%
\pgfsetlinewidth{1.003750pt}%
\definecolor{currentstroke}{rgb}{0.298039,0.447059,0.690196}%
\pgfsetstrokecolor{currentstroke}%
\pgfsetdash{}{0pt}%
\pgfpathmoveto{\pgfqpoint{4.888674in}{1.811568in}}%
\pgfpathcurveto{\pgfqpoint{4.896910in}{1.811568in}}{\pgfqpoint{4.904810in}{1.814840in}}{\pgfqpoint{4.910634in}{1.820664in}}%
\pgfpathcurveto{\pgfqpoint{4.916458in}{1.826488in}}{\pgfqpoint{4.919731in}{1.834388in}}{\pgfqpoint{4.919731in}{1.842625in}}%
\pgfpathcurveto{\pgfqpoint{4.919731in}{1.850861in}}{\pgfqpoint{4.916458in}{1.858761in}}{\pgfqpoint{4.910634in}{1.864585in}}%
\pgfpathcurveto{\pgfqpoint{4.904810in}{1.870409in}}{\pgfqpoint{4.896910in}{1.873681in}}{\pgfqpoint{4.888674in}{1.873681in}}%
\pgfpathcurveto{\pgfqpoint{4.880438in}{1.873681in}}{\pgfqpoint{4.872538in}{1.870409in}}{\pgfqpoint{4.866714in}{1.864585in}}%
\pgfpathcurveto{\pgfqpoint{4.860890in}{1.858761in}}{\pgfqpoint{4.857618in}{1.850861in}}{\pgfqpoint{4.857618in}{1.842625in}}%
\pgfpathcurveto{\pgfqpoint{4.857618in}{1.834388in}}{\pgfqpoint{4.860890in}{1.826488in}}{\pgfqpoint{4.866714in}{1.820664in}}%
\pgfpathcurveto{\pgfqpoint{4.872538in}{1.814840in}}{\pgfqpoint{4.880438in}{1.811568in}}{\pgfqpoint{4.888674in}{1.811568in}}%
\pgfpathclose%
\pgfusepath{stroke,fill}%
\end{pgfscope}%
\begin{pgfscope}%
\pgfpathrectangle{\pgfqpoint{3.755891in}{0.557870in}}{\pgfqpoint{2.484109in}{1.484734in}}%
\pgfusepath{clip}%
\pgfsetbuttcap%
\pgfsetroundjoin%
\definecolor{currentfill}{rgb}{0.298039,0.447059,0.690196}%
\pgfsetfillcolor{currentfill}%
\pgfsetlinewidth{1.003750pt}%
\definecolor{currentstroke}{rgb}{0.298039,0.447059,0.690196}%
\pgfsetstrokecolor{currentstroke}%
\pgfsetdash{}{0pt}%
\pgfpathmoveto{\pgfqpoint{4.815826in}{1.811568in}}%
\pgfpathcurveto{\pgfqpoint{4.824063in}{1.811568in}}{\pgfqpoint{4.831963in}{1.814840in}}{\pgfqpoint{4.837787in}{1.820664in}}%
\pgfpathcurveto{\pgfqpoint{4.843610in}{1.826488in}}{\pgfqpoint{4.846883in}{1.834388in}}{\pgfqpoint{4.846883in}{1.842625in}}%
\pgfpathcurveto{\pgfqpoint{4.846883in}{1.850861in}}{\pgfqpoint{4.843610in}{1.858761in}}{\pgfqpoint{4.837787in}{1.864585in}}%
\pgfpathcurveto{\pgfqpoint{4.831963in}{1.870409in}}{\pgfqpoint{4.824063in}{1.873681in}}{\pgfqpoint{4.815826in}{1.873681in}}%
\pgfpathcurveto{\pgfqpoint{4.807590in}{1.873681in}}{\pgfqpoint{4.799690in}{1.870409in}}{\pgfqpoint{4.793866in}{1.864585in}}%
\pgfpathcurveto{\pgfqpoint{4.788042in}{1.858761in}}{\pgfqpoint{4.784770in}{1.850861in}}{\pgfqpoint{4.784770in}{1.842625in}}%
\pgfpathcurveto{\pgfqpoint{4.784770in}{1.834388in}}{\pgfqpoint{4.788042in}{1.826488in}}{\pgfqpoint{4.793866in}{1.820664in}}%
\pgfpathcurveto{\pgfqpoint{4.799690in}{1.814840in}}{\pgfqpoint{4.807590in}{1.811568in}}{\pgfqpoint{4.815826in}{1.811568in}}%
\pgfpathclose%
\pgfusepath{stroke,fill}%
\end{pgfscope}%
\begin{pgfscope}%
\pgfpathrectangle{\pgfqpoint{3.755891in}{0.557870in}}{\pgfqpoint{2.484109in}{1.484734in}}%
\pgfusepath{clip}%
\pgfsetbuttcap%
\pgfsetroundjoin%
\definecolor{currentfill}{rgb}{0.298039,0.447059,0.690196}%
\pgfsetfillcolor{currentfill}%
\pgfsetlinewidth{1.003750pt}%
\definecolor{currentstroke}{rgb}{0.298039,0.447059,0.690196}%
\pgfsetstrokecolor{currentstroke}%
\pgfsetdash{}{0pt}%
\pgfpathmoveto{\pgfqpoint{4.815826in}{1.861253in}}%
\pgfpathcurveto{\pgfqpoint{4.824063in}{1.861253in}}{\pgfqpoint{4.831963in}{1.864525in}}{\pgfqpoint{4.837787in}{1.870349in}}%
\pgfpathcurveto{\pgfqpoint{4.843610in}{1.876173in}}{\pgfqpoint{4.846883in}{1.884073in}}{\pgfqpoint{4.846883in}{1.892309in}}%
\pgfpathcurveto{\pgfqpoint{4.846883in}{1.900545in}}{\pgfqpoint{4.843610in}{1.908445in}}{\pgfqpoint{4.837787in}{1.914269in}}%
\pgfpathcurveto{\pgfqpoint{4.831963in}{1.920093in}}{\pgfqpoint{4.824063in}{1.923366in}}{\pgfqpoint{4.815826in}{1.923366in}}%
\pgfpathcurveto{\pgfqpoint{4.807590in}{1.923366in}}{\pgfqpoint{4.799690in}{1.920093in}}{\pgfqpoint{4.793866in}{1.914269in}}%
\pgfpathcurveto{\pgfqpoint{4.788042in}{1.908445in}}{\pgfqpoint{4.784770in}{1.900545in}}{\pgfqpoint{4.784770in}{1.892309in}}%
\pgfpathcurveto{\pgfqpoint{4.784770in}{1.884073in}}{\pgfqpoint{4.788042in}{1.876173in}}{\pgfqpoint{4.793866in}{1.870349in}}%
\pgfpathcurveto{\pgfqpoint{4.799690in}{1.864525in}}{\pgfqpoint{4.807590in}{1.861253in}}{\pgfqpoint{4.815826in}{1.861253in}}%
\pgfpathclose%
\pgfusepath{stroke,fill}%
\end{pgfscope}%
\begin{pgfscope}%
\pgfpathrectangle{\pgfqpoint{3.755891in}{0.557870in}}{\pgfqpoint{2.484109in}{1.484734in}}%
\pgfusepath{clip}%
\pgfsetbuttcap%
\pgfsetroundjoin%
\definecolor{currentfill}{rgb}{0.298039,0.447059,0.690196}%
\pgfsetfillcolor{currentfill}%
\pgfsetlinewidth{1.003750pt}%
\definecolor{currentstroke}{rgb}{0.298039,0.447059,0.690196}%
\pgfsetstrokecolor{currentstroke}%
\pgfsetdash{}{0pt}%
\pgfpathmoveto{\pgfqpoint{4.888674in}{1.852972in}}%
\pgfpathcurveto{\pgfqpoint{4.896910in}{1.852972in}}{\pgfqpoint{4.904810in}{1.856244in}}{\pgfqpoint{4.910634in}{1.862068in}}%
\pgfpathcurveto{\pgfqpoint{4.916458in}{1.867892in}}{\pgfqpoint{4.919731in}{1.875792in}}{\pgfqpoint{4.919731in}{1.884028in}}%
\pgfpathcurveto{\pgfqpoint{4.919731in}{1.892265in}}{\pgfqpoint{4.916458in}{1.900165in}}{\pgfqpoint{4.910634in}{1.905989in}}%
\pgfpathcurveto{\pgfqpoint{4.904810in}{1.911812in}}{\pgfqpoint{4.896910in}{1.915085in}}{\pgfqpoint{4.888674in}{1.915085in}}%
\pgfpathcurveto{\pgfqpoint{4.880438in}{1.915085in}}{\pgfqpoint{4.872538in}{1.911812in}}{\pgfqpoint{4.866714in}{1.905989in}}%
\pgfpathcurveto{\pgfqpoint{4.860890in}{1.900165in}}{\pgfqpoint{4.857618in}{1.892265in}}{\pgfqpoint{4.857618in}{1.884028in}}%
\pgfpathcurveto{\pgfqpoint{4.857618in}{1.875792in}}{\pgfqpoint{4.860890in}{1.867892in}}{\pgfqpoint{4.866714in}{1.862068in}}%
\pgfpathcurveto{\pgfqpoint{4.872538in}{1.856244in}}{\pgfqpoint{4.880438in}{1.852972in}}{\pgfqpoint{4.888674in}{1.852972in}}%
\pgfpathclose%
\pgfusepath{stroke,fill}%
\end{pgfscope}%
\begin{pgfscope}%
\pgfpathrectangle{\pgfqpoint{3.755891in}{0.557870in}}{\pgfqpoint{2.484109in}{1.484734in}}%
\pgfusepath{clip}%
\pgfsetbuttcap%
\pgfsetroundjoin%
\definecolor{currentfill}{rgb}{0.298039,0.447059,0.690196}%
\pgfsetfillcolor{currentfill}%
\pgfsetlinewidth{1.003750pt}%
\definecolor{currentstroke}{rgb}{0.298039,0.447059,0.690196}%
\pgfsetstrokecolor{currentstroke}%
\pgfsetdash{}{0pt}%
\pgfpathmoveto{\pgfqpoint{5.689999in}{1.728761in}}%
\pgfpathcurveto{\pgfqpoint{5.698236in}{1.728761in}}{\pgfqpoint{5.706136in}{1.732033in}}{\pgfqpoint{5.711960in}{1.737857in}}%
\pgfpathcurveto{\pgfqpoint{5.717784in}{1.743681in}}{\pgfqpoint{5.721056in}{1.751581in}}{\pgfqpoint{5.721056in}{1.759817in}}%
\pgfpathcurveto{\pgfqpoint{5.721056in}{1.768054in}}{\pgfqpoint{5.717784in}{1.775954in}}{\pgfqpoint{5.711960in}{1.781778in}}%
\pgfpathcurveto{\pgfqpoint{5.706136in}{1.787602in}}{\pgfqpoint{5.698236in}{1.790874in}}{\pgfqpoint{5.689999in}{1.790874in}}%
\pgfpathcurveto{\pgfqpoint{5.681763in}{1.790874in}}{\pgfqpoint{5.673863in}{1.787602in}}{\pgfqpoint{5.668039in}{1.781778in}}%
\pgfpathcurveto{\pgfqpoint{5.662215in}{1.775954in}}{\pgfqpoint{5.658943in}{1.768054in}}{\pgfqpoint{5.658943in}{1.759817in}}%
\pgfpathcurveto{\pgfqpoint{5.658943in}{1.751581in}}{\pgfqpoint{5.662215in}{1.743681in}}{\pgfqpoint{5.668039in}{1.737857in}}%
\pgfpathcurveto{\pgfqpoint{5.673863in}{1.732033in}}{\pgfqpoint{5.681763in}{1.728761in}}{\pgfqpoint{5.689999in}{1.728761in}}%
\pgfpathclose%
\pgfusepath{stroke,fill}%
\end{pgfscope}%
\begin{pgfscope}%
\pgfpathrectangle{\pgfqpoint{3.755891in}{0.557870in}}{\pgfqpoint{2.484109in}{1.484734in}}%
\pgfusepath{clip}%
\pgfsetbuttcap%
\pgfsetroundjoin%
\definecolor{currentfill}{rgb}{0.298039,0.447059,0.690196}%
\pgfsetfillcolor{currentfill}%
\pgfsetlinewidth{1.003750pt}%
\definecolor{currentstroke}{rgb}{0.298039,0.447059,0.690196}%
\pgfsetstrokecolor{currentstroke}%
\pgfsetdash{}{0pt}%
\pgfpathmoveto{\pgfqpoint{4.597283in}{1.861253in}}%
\pgfpathcurveto{\pgfqpoint{4.605519in}{1.861253in}}{\pgfqpoint{4.613419in}{1.864525in}}{\pgfqpoint{4.619243in}{1.870349in}}%
\pgfpathcurveto{\pgfqpoint{4.625067in}{1.876173in}}{\pgfqpoint{4.628339in}{1.884073in}}{\pgfqpoint{4.628339in}{1.892309in}}%
\pgfpathcurveto{\pgfqpoint{4.628339in}{1.900545in}}{\pgfqpoint{4.625067in}{1.908445in}}{\pgfqpoint{4.619243in}{1.914269in}}%
\pgfpathcurveto{\pgfqpoint{4.613419in}{1.920093in}}{\pgfqpoint{4.605519in}{1.923366in}}{\pgfqpoint{4.597283in}{1.923366in}}%
\pgfpathcurveto{\pgfqpoint{4.589047in}{1.923366in}}{\pgfqpoint{4.581147in}{1.920093in}}{\pgfqpoint{4.575323in}{1.914269in}}%
\pgfpathcurveto{\pgfqpoint{4.569499in}{1.908445in}}{\pgfqpoint{4.566226in}{1.900545in}}{\pgfqpoint{4.566226in}{1.892309in}}%
\pgfpathcurveto{\pgfqpoint{4.566226in}{1.884073in}}{\pgfqpoint{4.569499in}{1.876173in}}{\pgfqpoint{4.575323in}{1.870349in}}%
\pgfpathcurveto{\pgfqpoint{4.581147in}{1.864525in}}{\pgfqpoint{4.589047in}{1.861253in}}{\pgfqpoint{4.597283in}{1.861253in}}%
\pgfpathclose%
\pgfusepath{stroke,fill}%
\end{pgfscope}%
\begin{pgfscope}%
\pgfpathrectangle{\pgfqpoint{3.755891in}{0.557870in}}{\pgfqpoint{2.484109in}{1.484734in}}%
\pgfusepath{clip}%
\pgfsetbuttcap%
\pgfsetroundjoin%
\definecolor{currentfill}{rgb}{0.298039,0.447059,0.690196}%
\pgfsetfillcolor{currentfill}%
\pgfsetlinewidth{1.003750pt}%
\definecolor{currentstroke}{rgb}{0.298039,0.447059,0.690196}%
\pgfsetstrokecolor{currentstroke}%
\pgfsetdash{}{0pt}%
\pgfpathmoveto{\pgfqpoint{4.755987in}{1.850680in}}%
\pgfpathcurveto{\pgfqpoint{4.764223in}{1.850680in}}{\pgfqpoint{4.772123in}{1.853953in}}{\pgfqpoint{4.777947in}{1.859777in}}%
\pgfpathcurveto{\pgfqpoint{4.783771in}{1.865600in}}{\pgfqpoint{4.787044in}{1.873500in}}{\pgfqpoint{4.787044in}{1.881737in}}%
\pgfpathcurveto{\pgfqpoint{4.787044in}{1.889973in}}{\pgfqpoint{4.783771in}{1.897873in}}{\pgfqpoint{4.777947in}{1.903697in}}%
\pgfpathcurveto{\pgfqpoint{4.772123in}{1.909521in}}{\pgfqpoint{4.764223in}{1.912793in}}{\pgfqpoint{4.755987in}{1.912793in}}%
\pgfpathcurveto{\pgfqpoint{4.747751in}{1.912793in}}{\pgfqpoint{4.739851in}{1.909521in}}{\pgfqpoint{4.734027in}{1.903697in}}%
\pgfpathcurveto{\pgfqpoint{4.728203in}{1.897873in}}{\pgfqpoint{4.724931in}{1.889973in}}{\pgfqpoint{4.724931in}{1.881737in}}%
\pgfpathcurveto{\pgfqpoint{4.724931in}{1.873500in}}{\pgfqpoint{4.728203in}{1.865600in}}{\pgfqpoint{4.734027in}{1.859777in}}%
\pgfpathcurveto{\pgfqpoint{4.739851in}{1.853953in}}{\pgfqpoint{4.747751in}{1.850680in}}{\pgfqpoint{4.755987in}{1.850680in}}%
\pgfpathclose%
\pgfusepath{stroke,fill}%
\end{pgfscope}%
\begin{pgfscope}%
\pgfpathrectangle{\pgfqpoint{3.755891in}{0.557870in}}{\pgfqpoint{2.484109in}{1.484734in}}%
\pgfusepath{clip}%
\pgfsetbuttcap%
\pgfsetroundjoin%
\definecolor{currentfill}{rgb}{0.298039,0.447059,0.690196}%
\pgfsetfillcolor{currentfill}%
\pgfsetlinewidth{1.003750pt}%
\definecolor{currentstroke}{rgb}{0.298039,0.447059,0.690196}%
\pgfsetstrokecolor{currentstroke}%
\pgfsetdash{}{0pt}%
\pgfpathmoveto{\pgfqpoint{5.481863in}{1.876147in}}%
\pgfpathcurveto{\pgfqpoint{5.490099in}{1.876147in}}{\pgfqpoint{5.497999in}{1.879420in}}{\pgfqpoint{5.503823in}{1.885244in}}%
\pgfpathcurveto{\pgfqpoint{5.509647in}{1.891068in}}{\pgfqpoint{5.512919in}{1.898968in}}{\pgfqpoint{5.512919in}{1.907204in}}%
\pgfpathcurveto{\pgfqpoint{5.512919in}{1.915440in}}{\pgfqpoint{5.509647in}{1.923340in}}{\pgfqpoint{5.503823in}{1.929164in}}%
\pgfpathcurveto{\pgfqpoint{5.497999in}{1.934988in}}{\pgfqpoint{5.490099in}{1.938260in}}{\pgfqpoint{5.481863in}{1.938260in}}%
\pgfpathcurveto{\pgfqpoint{5.473627in}{1.938260in}}{\pgfqpoint{5.465727in}{1.934988in}}{\pgfqpoint{5.459903in}{1.929164in}}%
\pgfpathcurveto{\pgfqpoint{5.454079in}{1.923340in}}{\pgfqpoint{5.450806in}{1.915440in}}{\pgfqpoint{5.450806in}{1.907204in}}%
\pgfpathcurveto{\pgfqpoint{5.450806in}{1.898968in}}{\pgfqpoint{5.454079in}{1.891068in}}{\pgfqpoint{5.459903in}{1.885244in}}%
\pgfpathcurveto{\pgfqpoint{5.465727in}{1.879420in}}{\pgfqpoint{5.473627in}{1.876147in}}{\pgfqpoint{5.481863in}{1.876147in}}%
\pgfpathclose%
\pgfusepath{stroke,fill}%
\end{pgfscope}%
\begin{pgfscope}%
\pgfpathrectangle{\pgfqpoint{3.755891in}{0.557870in}}{\pgfqpoint{2.484109in}{1.484734in}}%
\pgfusepath{clip}%
\pgfsetbuttcap%
\pgfsetroundjoin%
\definecolor{currentfill}{rgb}{0.298039,0.447059,0.690196}%
\pgfsetfillcolor{currentfill}%
\pgfsetlinewidth{1.003750pt}%
\definecolor{currentstroke}{rgb}{0.298039,0.447059,0.690196}%
\pgfsetstrokecolor{currentstroke}%
\pgfsetdash{}{0pt}%
\pgfpathmoveto{\pgfqpoint{5.320557in}{1.910104in}}%
\pgfpathcurveto{\pgfqpoint{5.328793in}{1.910104in}}{\pgfqpoint{5.336694in}{1.913376in}}{\pgfqpoint{5.342517in}{1.919200in}}%
\pgfpathcurveto{\pgfqpoint{5.348341in}{1.925024in}}{\pgfqpoint{5.351614in}{1.932924in}}{\pgfqpoint{5.351614in}{1.941160in}}%
\pgfpathcurveto{\pgfqpoint{5.351614in}{1.949396in}}{\pgfqpoint{5.348341in}{1.957296in}}{\pgfqpoint{5.342517in}{1.963120in}}%
\pgfpathcurveto{\pgfqpoint{5.336694in}{1.968944in}}{\pgfqpoint{5.328793in}{1.972217in}}{\pgfqpoint{5.320557in}{1.972217in}}%
\pgfpathcurveto{\pgfqpoint{5.312321in}{1.972217in}}{\pgfqpoint{5.304421in}{1.968944in}}{\pgfqpoint{5.298597in}{1.963120in}}%
\pgfpathcurveto{\pgfqpoint{5.292773in}{1.957296in}}{\pgfqpoint{5.289501in}{1.949396in}}{\pgfqpoint{5.289501in}{1.941160in}}%
\pgfpathcurveto{\pgfqpoint{5.289501in}{1.932924in}}{\pgfqpoint{5.292773in}{1.925024in}}{\pgfqpoint{5.298597in}{1.919200in}}%
\pgfpathcurveto{\pgfqpoint{5.304421in}{1.913376in}}{\pgfqpoint{5.312321in}{1.910104in}}{\pgfqpoint{5.320557in}{1.910104in}}%
\pgfpathclose%
\pgfusepath{stroke,fill}%
\end{pgfscope}%
\begin{pgfscope}%
\pgfpathrectangle{\pgfqpoint{3.755891in}{0.557870in}}{\pgfqpoint{2.484109in}{1.484734in}}%
\pgfusepath{clip}%
\pgfsetbuttcap%
\pgfsetroundjoin%
\definecolor{currentfill}{rgb}{0.298039,0.447059,0.690196}%
\pgfsetfillcolor{currentfill}%
\pgfsetlinewidth{1.003750pt}%
\definecolor{currentstroke}{rgb}{0.298039,0.447059,0.690196}%
\pgfsetstrokecolor{currentstroke}%
\pgfsetdash{}{0pt}%
\pgfpathmoveto{\pgfqpoint{5.481863in}{1.842191in}}%
\pgfpathcurveto{\pgfqpoint{5.490099in}{1.842191in}}{\pgfqpoint{5.497999in}{1.845464in}}{\pgfqpoint{5.503823in}{1.851287in}}%
\pgfpathcurveto{\pgfqpoint{5.509647in}{1.857111in}}{\pgfqpoint{5.512919in}{1.865011in}}{\pgfqpoint{5.512919in}{1.873248in}}%
\pgfpathcurveto{\pgfqpoint{5.512919in}{1.881484in}}{\pgfqpoint{5.509647in}{1.889384in}}{\pgfqpoint{5.503823in}{1.895208in}}%
\pgfpathcurveto{\pgfqpoint{5.497999in}{1.901032in}}{\pgfqpoint{5.490099in}{1.904304in}}{\pgfqpoint{5.481863in}{1.904304in}}%
\pgfpathcurveto{\pgfqpoint{5.473627in}{1.904304in}}{\pgfqpoint{5.465727in}{1.901032in}}{\pgfqpoint{5.459903in}{1.895208in}}%
\pgfpathcurveto{\pgfqpoint{5.454079in}{1.889384in}}{\pgfqpoint{5.450806in}{1.881484in}}{\pgfqpoint{5.450806in}{1.873248in}}%
\pgfpathcurveto{\pgfqpoint{5.450806in}{1.865011in}}{\pgfqpoint{5.454079in}{1.857111in}}{\pgfqpoint{5.459903in}{1.851287in}}%
\pgfpathcurveto{\pgfqpoint{5.465727in}{1.845464in}}{\pgfqpoint{5.473627in}{1.842191in}}{\pgfqpoint{5.481863in}{1.842191in}}%
\pgfpathclose%
\pgfusepath{stroke,fill}%
\end{pgfscope}%
\begin{pgfscope}%
\pgfpathrectangle{\pgfqpoint{3.755891in}{0.557870in}}{\pgfqpoint{2.484109in}{1.484734in}}%
\pgfusepath{clip}%
\pgfsetbuttcap%
\pgfsetroundjoin%
\definecolor{currentfill}{rgb}{0.298039,0.447059,0.690196}%
\pgfsetfillcolor{currentfill}%
\pgfsetlinewidth{1.003750pt}%
\definecolor{currentstroke}{rgb}{0.298039,0.447059,0.690196}%
\pgfsetstrokecolor{currentstroke}%
\pgfsetdash{}{0pt}%
\pgfpathmoveto{\pgfqpoint{5.078599in}{1.859169in}}%
\pgfpathcurveto{\pgfqpoint{5.086835in}{1.859169in}}{\pgfqpoint{5.094735in}{1.862442in}}{\pgfqpoint{5.100559in}{1.868266in}}%
\pgfpathcurveto{\pgfqpoint{5.106383in}{1.874089in}}{\pgfqpoint{5.109655in}{1.881990in}}{\pgfqpoint{5.109655in}{1.890226in}}%
\pgfpathcurveto{\pgfqpoint{5.109655in}{1.898462in}}{\pgfqpoint{5.106383in}{1.906362in}}{\pgfqpoint{5.100559in}{1.912186in}}%
\pgfpathcurveto{\pgfqpoint{5.094735in}{1.918010in}}{\pgfqpoint{5.086835in}{1.921282in}}{\pgfqpoint{5.078599in}{1.921282in}}%
\pgfpathcurveto{\pgfqpoint{5.070362in}{1.921282in}}{\pgfqpoint{5.062462in}{1.918010in}}{\pgfqpoint{5.056638in}{1.912186in}}%
\pgfpathcurveto{\pgfqpoint{5.050814in}{1.906362in}}{\pgfqpoint{5.047542in}{1.898462in}}{\pgfqpoint{5.047542in}{1.890226in}}%
\pgfpathcurveto{\pgfqpoint{5.047542in}{1.881990in}}{\pgfqpoint{5.050814in}{1.874089in}}{\pgfqpoint{5.056638in}{1.868266in}}%
\pgfpathcurveto{\pgfqpoint{5.062462in}{1.862442in}}{\pgfqpoint{5.070362in}{1.859169in}}{\pgfqpoint{5.078599in}{1.859169in}}%
\pgfpathclose%
\pgfusepath{stroke,fill}%
\end{pgfscope}%
\begin{pgfscope}%
\pgfpathrectangle{\pgfqpoint{3.755891in}{0.557870in}}{\pgfqpoint{2.484109in}{1.484734in}}%
\pgfusepath{clip}%
\pgfsetbuttcap%
\pgfsetroundjoin%
\definecolor{currentfill}{rgb}{0.298039,0.447059,0.690196}%
\pgfsetfillcolor{currentfill}%
\pgfsetlinewidth{1.003750pt}%
\definecolor{currentstroke}{rgb}{0.298039,0.447059,0.690196}%
\pgfsetstrokecolor{currentstroke}%
\pgfsetdash{}{0pt}%
\pgfpathmoveto{\pgfqpoint{5.320557in}{1.876147in}}%
\pgfpathcurveto{\pgfqpoint{5.328793in}{1.876147in}}{\pgfqpoint{5.336694in}{1.879420in}}{\pgfqpoint{5.342517in}{1.885244in}}%
\pgfpathcurveto{\pgfqpoint{5.348341in}{1.891068in}}{\pgfqpoint{5.351614in}{1.898968in}}{\pgfqpoint{5.351614in}{1.907204in}}%
\pgfpathcurveto{\pgfqpoint{5.351614in}{1.915440in}}{\pgfqpoint{5.348341in}{1.923340in}}{\pgfqpoint{5.342517in}{1.929164in}}%
\pgfpathcurveto{\pgfqpoint{5.336694in}{1.934988in}}{\pgfqpoint{5.328793in}{1.938260in}}{\pgfqpoint{5.320557in}{1.938260in}}%
\pgfpathcurveto{\pgfqpoint{5.312321in}{1.938260in}}{\pgfqpoint{5.304421in}{1.934988in}}{\pgfqpoint{5.298597in}{1.929164in}}%
\pgfpathcurveto{\pgfqpoint{5.292773in}{1.923340in}}{\pgfqpoint{5.289501in}{1.915440in}}{\pgfqpoint{5.289501in}{1.907204in}}%
\pgfpathcurveto{\pgfqpoint{5.289501in}{1.898968in}}{\pgfqpoint{5.292773in}{1.891068in}}{\pgfqpoint{5.298597in}{1.885244in}}%
\pgfpathcurveto{\pgfqpoint{5.304421in}{1.879420in}}{\pgfqpoint{5.312321in}{1.876147in}}{\pgfqpoint{5.320557in}{1.876147in}}%
\pgfpathclose%
\pgfusepath{stroke,fill}%
\end{pgfscope}%
\begin{pgfscope}%
\pgfpathrectangle{\pgfqpoint{3.755891in}{0.557870in}}{\pgfqpoint{2.484109in}{1.484734in}}%
\pgfusepath{clip}%
\pgfsetbuttcap%
\pgfsetroundjoin%
\definecolor{currentfill}{rgb}{0.298039,0.447059,0.690196}%
\pgfsetfillcolor{currentfill}%
\pgfsetlinewidth{1.003750pt}%
\definecolor{currentstroke}{rgb}{0.298039,0.447059,0.690196}%
\pgfsetstrokecolor{currentstroke}%
\pgfsetdash{}{0pt}%
\pgfpathmoveto{\pgfqpoint{5.078599in}{1.901615in}}%
\pgfpathcurveto{\pgfqpoint{5.086835in}{1.901615in}}{\pgfqpoint{5.094735in}{1.904887in}}{\pgfqpoint{5.100559in}{1.910711in}}%
\pgfpathcurveto{\pgfqpoint{5.106383in}{1.916535in}}{\pgfqpoint{5.109655in}{1.924435in}}{\pgfqpoint{5.109655in}{1.932671in}}%
\pgfpathcurveto{\pgfqpoint{5.109655in}{1.940907in}}{\pgfqpoint{5.106383in}{1.948807in}}{\pgfqpoint{5.100559in}{1.954631in}}%
\pgfpathcurveto{\pgfqpoint{5.094735in}{1.960455in}}{\pgfqpoint{5.086835in}{1.963728in}}{\pgfqpoint{5.078599in}{1.963728in}}%
\pgfpathcurveto{\pgfqpoint{5.070362in}{1.963728in}}{\pgfqpoint{5.062462in}{1.960455in}}{\pgfqpoint{5.056638in}{1.954631in}}%
\pgfpathcurveto{\pgfqpoint{5.050814in}{1.948807in}}{\pgfqpoint{5.047542in}{1.940907in}}{\pgfqpoint{5.047542in}{1.932671in}}%
\pgfpathcurveto{\pgfqpoint{5.047542in}{1.924435in}}{\pgfqpoint{5.050814in}{1.916535in}}{\pgfqpoint{5.056638in}{1.910711in}}%
\pgfpathcurveto{\pgfqpoint{5.062462in}{1.904887in}}{\pgfqpoint{5.070362in}{1.901615in}}{\pgfqpoint{5.078599in}{1.901615in}}%
\pgfpathclose%
\pgfusepath{stroke,fill}%
\end{pgfscope}%
\begin{pgfscope}%
\pgfpathrectangle{\pgfqpoint{3.755891in}{0.557870in}}{\pgfqpoint{2.484109in}{1.484734in}}%
\pgfusepath{clip}%
\pgfsetbuttcap%
\pgfsetroundjoin%
\definecolor{currentfill}{rgb}{0.298039,0.447059,0.690196}%
\pgfsetfillcolor{currentfill}%
\pgfsetlinewidth{1.003750pt}%
\definecolor{currentstroke}{rgb}{0.298039,0.447059,0.690196}%
\pgfsetstrokecolor{currentstroke}%
\pgfsetdash{}{0pt}%
\pgfpathmoveto{\pgfqpoint{5.401210in}{1.901615in}}%
\pgfpathcurveto{\pgfqpoint{5.409446in}{1.901615in}}{\pgfqpoint{5.417346in}{1.904887in}}{\pgfqpoint{5.423170in}{1.910711in}}%
\pgfpathcurveto{\pgfqpoint{5.428994in}{1.916535in}}{\pgfqpoint{5.432267in}{1.924435in}}{\pgfqpoint{5.432267in}{1.932671in}}%
\pgfpathcurveto{\pgfqpoint{5.432267in}{1.940907in}}{\pgfqpoint{5.428994in}{1.948807in}}{\pgfqpoint{5.423170in}{1.954631in}}%
\pgfpathcurveto{\pgfqpoint{5.417346in}{1.960455in}}{\pgfqpoint{5.409446in}{1.963728in}}{\pgfqpoint{5.401210in}{1.963728in}}%
\pgfpathcurveto{\pgfqpoint{5.392974in}{1.963728in}}{\pgfqpoint{5.385074in}{1.960455in}}{\pgfqpoint{5.379250in}{1.954631in}}%
\pgfpathcurveto{\pgfqpoint{5.373426in}{1.948807in}}{\pgfqpoint{5.370154in}{1.940907in}}{\pgfqpoint{5.370154in}{1.932671in}}%
\pgfpathcurveto{\pgfqpoint{5.370154in}{1.924435in}}{\pgfqpoint{5.373426in}{1.916535in}}{\pgfqpoint{5.379250in}{1.910711in}}%
\pgfpathcurveto{\pgfqpoint{5.385074in}{1.904887in}}{\pgfqpoint{5.392974in}{1.901615in}}{\pgfqpoint{5.401210in}{1.901615in}}%
\pgfpathclose%
\pgfusepath{stroke,fill}%
\end{pgfscope}%
\begin{pgfscope}%
\pgfpathrectangle{\pgfqpoint{3.755891in}{0.557870in}}{\pgfqpoint{2.484109in}{1.484734in}}%
\pgfusepath{clip}%
\pgfsetbuttcap%
\pgfsetroundjoin%
\definecolor{currentfill}{rgb}{0.298039,0.447059,0.690196}%
\pgfsetfillcolor{currentfill}%
\pgfsetlinewidth{1.003750pt}%
\definecolor{currentstroke}{rgb}{0.298039,0.447059,0.690196}%
\pgfsetstrokecolor{currentstroke}%
\pgfsetdash{}{0pt}%
\pgfpathmoveto{\pgfqpoint{5.885127in}{1.621476in}}%
\pgfpathcurveto{\pgfqpoint{5.893364in}{1.621476in}}{\pgfqpoint{5.901264in}{1.624748in}}{\pgfqpoint{5.907088in}{1.630572in}}%
\pgfpathcurveto{\pgfqpoint{5.912912in}{1.636396in}}{\pgfqpoint{5.916184in}{1.644296in}}{\pgfqpoint{5.916184in}{1.652533in}}%
\pgfpathcurveto{\pgfqpoint{5.916184in}{1.660769in}}{\pgfqpoint{5.912912in}{1.668669in}}{\pgfqpoint{5.907088in}{1.674493in}}%
\pgfpathcurveto{\pgfqpoint{5.901264in}{1.680317in}}{\pgfqpoint{5.893364in}{1.683589in}}{\pgfqpoint{5.885127in}{1.683589in}}%
\pgfpathcurveto{\pgfqpoint{5.876891in}{1.683589in}}{\pgfqpoint{5.868991in}{1.680317in}}{\pgfqpoint{5.863167in}{1.674493in}}%
\pgfpathcurveto{\pgfqpoint{5.857343in}{1.668669in}}{\pgfqpoint{5.854071in}{1.660769in}}{\pgfqpoint{5.854071in}{1.652533in}}%
\pgfpathcurveto{\pgfqpoint{5.854071in}{1.644296in}}{\pgfqpoint{5.857343in}{1.636396in}}{\pgfqpoint{5.863167in}{1.630572in}}%
\pgfpathcurveto{\pgfqpoint{5.868991in}{1.624748in}}{\pgfqpoint{5.876891in}{1.621476in}}{\pgfqpoint{5.885127in}{1.621476in}}%
\pgfpathclose%
\pgfusepath{stroke,fill}%
\end{pgfscope}%
\begin{pgfscope}%
\pgfpathrectangle{\pgfqpoint{3.755891in}{0.557870in}}{\pgfqpoint{2.484109in}{1.484734in}}%
\pgfusepath{clip}%
\pgfsetbuttcap%
\pgfsetroundjoin%
\definecolor{currentfill}{rgb}{0.298039,0.447059,0.690196}%
\pgfsetfillcolor{currentfill}%
\pgfsetlinewidth{1.003750pt}%
\definecolor{currentstroke}{rgb}{0.298039,0.447059,0.690196}%
\pgfsetstrokecolor{currentstroke}%
\pgfsetdash{}{0pt}%
\pgfpathmoveto{\pgfqpoint{4.110764in}{1.927082in}}%
\pgfpathcurveto{\pgfqpoint{4.119000in}{1.927082in}}{\pgfqpoint{4.126900in}{1.930354in}}{\pgfqpoint{4.132724in}{1.936178in}}%
\pgfpathcurveto{\pgfqpoint{4.138548in}{1.942002in}}{\pgfqpoint{4.141820in}{1.949902in}}{\pgfqpoint{4.141820in}{1.958138in}}%
\pgfpathcurveto{\pgfqpoint{4.141820in}{1.966374in}}{\pgfqpoint{4.138548in}{1.974275in}}{\pgfqpoint{4.132724in}{1.980098in}}%
\pgfpathcurveto{\pgfqpoint{4.126900in}{1.985922in}}{\pgfqpoint{4.119000in}{1.989195in}}{\pgfqpoint{4.110764in}{1.989195in}}%
\pgfpathcurveto{\pgfqpoint{4.102528in}{1.989195in}}{\pgfqpoint{4.094628in}{1.985922in}}{\pgfqpoint{4.088804in}{1.980098in}}%
\pgfpathcurveto{\pgfqpoint{4.082980in}{1.974275in}}{\pgfqpoint{4.079707in}{1.966374in}}{\pgfqpoint{4.079707in}{1.958138in}}%
\pgfpathcurveto{\pgfqpoint{4.079707in}{1.949902in}}{\pgfqpoint{4.082980in}{1.942002in}}{\pgfqpoint{4.088804in}{1.936178in}}%
\pgfpathcurveto{\pgfqpoint{4.094628in}{1.930354in}}{\pgfqpoint{4.102528in}{1.927082in}}{\pgfqpoint{4.110764in}{1.927082in}}%
\pgfpathclose%
\pgfusepath{stroke,fill}%
\end{pgfscope}%
\begin{pgfscope}%
\pgfpathrectangle{\pgfqpoint{3.755891in}{0.557870in}}{\pgfqpoint{2.484109in}{1.484734in}}%
\pgfusepath{clip}%
\pgfsetbuttcap%
\pgfsetroundjoin%
\definecolor{currentfill}{rgb}{0.298039,0.447059,0.690196}%
\pgfsetfillcolor{currentfill}%
\pgfsetlinewidth{1.003750pt}%
\definecolor{currentstroke}{rgb}{0.298039,0.447059,0.690196}%
\pgfsetstrokecolor{currentstroke}%
\pgfsetdash{}{0pt}%
\pgfpathmoveto{\pgfqpoint{5.159251in}{1.856979in}}%
\pgfpathcurveto{\pgfqpoint{5.167488in}{1.856979in}}{\pgfqpoint{5.175388in}{1.860251in}}{\pgfqpoint{5.181212in}{1.866075in}}%
\pgfpathcurveto{\pgfqpoint{5.187036in}{1.871899in}}{\pgfqpoint{5.190308in}{1.879799in}}{\pgfqpoint{5.190308in}{1.888035in}}%
\pgfpathcurveto{\pgfqpoint{5.190308in}{1.896271in}}{\pgfqpoint{5.187036in}{1.904171in}}{\pgfqpoint{5.181212in}{1.909995in}}%
\pgfpathcurveto{\pgfqpoint{5.175388in}{1.915819in}}{\pgfqpoint{5.167488in}{1.919092in}}{\pgfqpoint{5.159251in}{1.919092in}}%
\pgfpathcurveto{\pgfqpoint{5.151015in}{1.919092in}}{\pgfqpoint{5.143115in}{1.915819in}}{\pgfqpoint{5.137291in}{1.909995in}}%
\pgfpathcurveto{\pgfqpoint{5.131467in}{1.904171in}}{\pgfqpoint{5.128195in}{1.896271in}}{\pgfqpoint{5.128195in}{1.888035in}}%
\pgfpathcurveto{\pgfqpoint{5.128195in}{1.879799in}}{\pgfqpoint{5.131467in}{1.871899in}}{\pgfqpoint{5.137291in}{1.866075in}}%
\pgfpathcurveto{\pgfqpoint{5.143115in}{1.860251in}}{\pgfqpoint{5.151015in}{1.856979in}}{\pgfqpoint{5.159251in}{1.856979in}}%
\pgfpathclose%
\pgfusepath{stroke,fill}%
\end{pgfscope}%
\begin{pgfscope}%
\pgfpathrectangle{\pgfqpoint{3.755891in}{0.557870in}}{\pgfqpoint{2.484109in}{1.484734in}}%
\pgfusepath{clip}%
\pgfsetbuttcap%
\pgfsetroundjoin%
\definecolor{currentfill}{rgb}{0.298039,0.447059,0.690196}%
\pgfsetfillcolor{currentfill}%
\pgfsetlinewidth{1.003750pt}%
\definecolor{currentstroke}{rgb}{0.298039,0.447059,0.690196}%
\pgfsetstrokecolor{currentstroke}%
\pgfsetdash{}{0pt}%
\pgfpathmoveto{\pgfqpoint{5.723822in}{1.874395in}}%
\pgfpathcurveto{\pgfqpoint{5.732058in}{1.874395in}}{\pgfqpoint{5.739958in}{1.877667in}}{\pgfqpoint{5.745782in}{1.883491in}}%
\pgfpathcurveto{\pgfqpoint{5.751606in}{1.889315in}}{\pgfqpoint{5.754878in}{1.897215in}}{\pgfqpoint{5.754878in}{1.905451in}}%
\pgfpathcurveto{\pgfqpoint{5.754878in}{1.913688in}}{\pgfqpoint{5.751606in}{1.921588in}}{\pgfqpoint{5.745782in}{1.927412in}}%
\pgfpathcurveto{\pgfqpoint{5.739958in}{1.933236in}}{\pgfqpoint{5.732058in}{1.936508in}}{\pgfqpoint{5.723822in}{1.936508in}}%
\pgfpathcurveto{\pgfqpoint{5.715585in}{1.936508in}}{\pgfqpoint{5.707685in}{1.933236in}}{\pgfqpoint{5.701861in}{1.927412in}}%
\pgfpathcurveto{\pgfqpoint{5.696037in}{1.921588in}}{\pgfqpoint{5.692765in}{1.913688in}}{\pgfqpoint{5.692765in}{1.905451in}}%
\pgfpathcurveto{\pgfqpoint{5.692765in}{1.897215in}}{\pgfqpoint{5.696037in}{1.889315in}}{\pgfqpoint{5.701861in}{1.883491in}}%
\pgfpathcurveto{\pgfqpoint{5.707685in}{1.877667in}}{\pgfqpoint{5.715585in}{1.874395in}}{\pgfqpoint{5.723822in}{1.874395in}}%
\pgfpathclose%
\pgfusepath{stroke,fill}%
\end{pgfscope}%
\begin{pgfscope}%
\pgfpathrectangle{\pgfqpoint{3.755891in}{0.557870in}}{\pgfqpoint{2.484109in}{1.484734in}}%
\pgfusepath{clip}%
\pgfsetbuttcap%
\pgfsetroundjoin%
\definecolor{currentfill}{rgb}{0.298039,0.447059,0.690196}%
\pgfsetfillcolor{currentfill}%
\pgfsetlinewidth{1.003750pt}%
\definecolor{currentstroke}{rgb}{0.298039,0.447059,0.690196}%
\pgfsetstrokecolor{currentstroke}%
\pgfsetdash{}{0pt}%
\pgfpathmoveto{\pgfqpoint{5.320557in}{1.891811in}}%
\pgfpathcurveto{\pgfqpoint{5.328793in}{1.891811in}}{\pgfqpoint{5.336694in}{1.895083in}}{\pgfqpoint{5.342517in}{1.900907in}}%
\pgfpathcurveto{\pgfqpoint{5.348341in}{1.906731in}}{\pgfqpoint{5.351614in}{1.914631in}}{\pgfqpoint{5.351614in}{1.922868in}}%
\pgfpathcurveto{\pgfqpoint{5.351614in}{1.931104in}}{\pgfqpoint{5.348341in}{1.939004in}}{\pgfqpoint{5.342517in}{1.944828in}}%
\pgfpathcurveto{\pgfqpoint{5.336694in}{1.950652in}}{\pgfqpoint{5.328793in}{1.953924in}}{\pgfqpoint{5.320557in}{1.953924in}}%
\pgfpathcurveto{\pgfqpoint{5.312321in}{1.953924in}}{\pgfqpoint{5.304421in}{1.950652in}}{\pgfqpoint{5.298597in}{1.944828in}}%
\pgfpathcurveto{\pgfqpoint{5.292773in}{1.939004in}}{\pgfqpoint{5.289501in}{1.931104in}}{\pgfqpoint{5.289501in}{1.922868in}}%
\pgfpathcurveto{\pgfqpoint{5.289501in}{1.914631in}}{\pgfqpoint{5.292773in}{1.906731in}}{\pgfqpoint{5.298597in}{1.900907in}}%
\pgfpathcurveto{\pgfqpoint{5.304421in}{1.895083in}}{\pgfqpoint{5.312321in}{1.891811in}}{\pgfqpoint{5.320557in}{1.891811in}}%
\pgfpathclose%
\pgfusepath{stroke,fill}%
\end{pgfscope}%
\begin{pgfscope}%
\pgfpathrectangle{\pgfqpoint{3.755891in}{0.557870in}}{\pgfqpoint{2.484109in}{1.484734in}}%
\pgfusepath{clip}%
\pgfsetbuttcap%
\pgfsetroundjoin%
\definecolor{currentfill}{rgb}{0.298039,0.447059,0.690196}%
\pgfsetfillcolor{currentfill}%
\pgfsetlinewidth{1.003750pt}%
\definecolor{currentstroke}{rgb}{0.298039,0.447059,0.690196}%
\pgfsetstrokecolor{currentstroke}%
\pgfsetdash{}{0pt}%
\pgfpathmoveto{\pgfqpoint{5.320557in}{1.830854in}}%
\pgfpathcurveto{\pgfqpoint{5.328793in}{1.830854in}}{\pgfqpoint{5.336694in}{1.834127in}}{\pgfqpoint{5.342517in}{1.839950in}}%
\pgfpathcurveto{\pgfqpoint{5.348341in}{1.845774in}}{\pgfqpoint{5.351614in}{1.853674in}}{\pgfqpoint{5.351614in}{1.861911in}}%
\pgfpathcurveto{\pgfqpoint{5.351614in}{1.870147in}}{\pgfqpoint{5.348341in}{1.878047in}}{\pgfqpoint{5.342517in}{1.883871in}}%
\pgfpathcurveto{\pgfqpoint{5.336694in}{1.889695in}}{\pgfqpoint{5.328793in}{1.892967in}}{\pgfqpoint{5.320557in}{1.892967in}}%
\pgfpathcurveto{\pgfqpoint{5.312321in}{1.892967in}}{\pgfqpoint{5.304421in}{1.889695in}}{\pgfqpoint{5.298597in}{1.883871in}}%
\pgfpathcurveto{\pgfqpoint{5.292773in}{1.878047in}}{\pgfqpoint{5.289501in}{1.870147in}}{\pgfqpoint{5.289501in}{1.861911in}}%
\pgfpathcurveto{\pgfqpoint{5.289501in}{1.853674in}}{\pgfqpoint{5.292773in}{1.845774in}}{\pgfqpoint{5.298597in}{1.839950in}}%
\pgfpathcurveto{\pgfqpoint{5.304421in}{1.834127in}}{\pgfqpoint{5.312321in}{1.830854in}}{\pgfqpoint{5.320557in}{1.830854in}}%
\pgfpathclose%
\pgfusepath{stroke,fill}%
\end{pgfscope}%
\begin{pgfscope}%
\pgfpathrectangle{\pgfqpoint{3.755891in}{0.557870in}}{\pgfqpoint{2.484109in}{1.484734in}}%
\pgfusepath{clip}%
\pgfsetbuttcap%
\pgfsetroundjoin%
\definecolor{currentfill}{rgb}{0.298039,0.447059,0.690196}%
\pgfsetfillcolor{currentfill}%
\pgfsetlinewidth{1.003750pt}%
\definecolor{currentstroke}{rgb}{0.298039,0.447059,0.690196}%
\pgfsetstrokecolor{currentstroke}%
\pgfsetdash{}{0pt}%
\pgfpathmoveto{\pgfqpoint{5.481863in}{1.874395in}}%
\pgfpathcurveto{\pgfqpoint{5.490099in}{1.874395in}}{\pgfqpoint{5.497999in}{1.877667in}}{\pgfqpoint{5.503823in}{1.883491in}}%
\pgfpathcurveto{\pgfqpoint{5.509647in}{1.889315in}}{\pgfqpoint{5.512919in}{1.897215in}}{\pgfqpoint{5.512919in}{1.905451in}}%
\pgfpathcurveto{\pgfqpoint{5.512919in}{1.913688in}}{\pgfqpoint{5.509647in}{1.921588in}}{\pgfqpoint{5.503823in}{1.927412in}}%
\pgfpathcurveto{\pgfqpoint{5.497999in}{1.933236in}}{\pgfqpoint{5.490099in}{1.936508in}}{\pgfqpoint{5.481863in}{1.936508in}}%
\pgfpathcurveto{\pgfqpoint{5.473627in}{1.936508in}}{\pgfqpoint{5.465727in}{1.933236in}}{\pgfqpoint{5.459903in}{1.927412in}}%
\pgfpathcurveto{\pgfqpoint{5.454079in}{1.921588in}}{\pgfqpoint{5.450806in}{1.913688in}}{\pgfqpoint{5.450806in}{1.905451in}}%
\pgfpathcurveto{\pgfqpoint{5.450806in}{1.897215in}}{\pgfqpoint{5.454079in}{1.889315in}}{\pgfqpoint{5.459903in}{1.883491in}}%
\pgfpathcurveto{\pgfqpoint{5.465727in}{1.877667in}}{\pgfqpoint{5.473627in}{1.874395in}}{\pgfqpoint{5.481863in}{1.874395in}}%
\pgfpathclose%
\pgfusepath{stroke,fill}%
\end{pgfscope}%
\begin{pgfscope}%
\pgfpathrectangle{\pgfqpoint{3.755891in}{0.557870in}}{\pgfqpoint{2.484109in}{1.484734in}}%
\pgfusepath{clip}%
\pgfsetbuttcap%
\pgfsetroundjoin%
\definecolor{currentfill}{rgb}{0.298039,0.447059,0.690196}%
\pgfsetfillcolor{currentfill}%
\pgfsetlinewidth{1.003750pt}%
\definecolor{currentstroke}{rgb}{0.298039,0.447059,0.690196}%
\pgfsetstrokecolor{currentstroke}%
\pgfsetdash{}{0pt}%
\pgfpathmoveto{\pgfqpoint{5.078599in}{1.891811in}}%
\pgfpathcurveto{\pgfqpoint{5.086835in}{1.891811in}}{\pgfqpoint{5.094735in}{1.895083in}}{\pgfqpoint{5.100559in}{1.900907in}}%
\pgfpathcurveto{\pgfqpoint{5.106383in}{1.906731in}}{\pgfqpoint{5.109655in}{1.914631in}}{\pgfqpoint{5.109655in}{1.922868in}}%
\pgfpathcurveto{\pgfqpoint{5.109655in}{1.931104in}}{\pgfqpoint{5.106383in}{1.939004in}}{\pgfqpoint{5.100559in}{1.944828in}}%
\pgfpathcurveto{\pgfqpoint{5.094735in}{1.950652in}}{\pgfqpoint{5.086835in}{1.953924in}}{\pgfqpoint{5.078599in}{1.953924in}}%
\pgfpathcurveto{\pgfqpoint{5.070362in}{1.953924in}}{\pgfqpoint{5.062462in}{1.950652in}}{\pgfqpoint{5.056638in}{1.944828in}}%
\pgfpathcurveto{\pgfqpoint{5.050814in}{1.939004in}}{\pgfqpoint{5.047542in}{1.931104in}}{\pgfqpoint{5.047542in}{1.922868in}}%
\pgfpathcurveto{\pgfqpoint{5.047542in}{1.914631in}}{\pgfqpoint{5.050814in}{1.906731in}}{\pgfqpoint{5.056638in}{1.900907in}}%
\pgfpathcurveto{\pgfqpoint{5.062462in}{1.895083in}}{\pgfqpoint{5.070362in}{1.891811in}}{\pgfqpoint{5.078599in}{1.891811in}}%
\pgfpathclose%
\pgfusepath{stroke,fill}%
\end{pgfscope}%
\begin{pgfscope}%
\pgfpathrectangle{\pgfqpoint{3.755891in}{0.557870in}}{\pgfqpoint{2.484109in}{1.484734in}}%
\pgfusepath{clip}%
\pgfsetbuttcap%
\pgfsetroundjoin%
\definecolor{currentfill}{rgb}{0.298039,0.447059,0.690196}%
\pgfsetfillcolor{currentfill}%
\pgfsetlinewidth{1.003750pt}%
\definecolor{currentstroke}{rgb}{0.298039,0.447059,0.690196}%
\pgfsetstrokecolor{currentstroke}%
\pgfsetdash{}{0pt}%
\pgfpathmoveto{\pgfqpoint{5.320557in}{1.883103in}}%
\pgfpathcurveto{\pgfqpoint{5.328793in}{1.883103in}}{\pgfqpoint{5.336694in}{1.886375in}}{\pgfqpoint{5.342517in}{1.892199in}}%
\pgfpathcurveto{\pgfqpoint{5.348341in}{1.898023in}}{\pgfqpoint{5.351614in}{1.905923in}}{\pgfqpoint{5.351614in}{1.914159in}}%
\pgfpathcurveto{\pgfqpoint{5.351614in}{1.922396in}}{\pgfqpoint{5.348341in}{1.930296in}}{\pgfqpoint{5.342517in}{1.936120in}}%
\pgfpathcurveto{\pgfqpoint{5.336694in}{1.941944in}}{\pgfqpoint{5.328793in}{1.945216in}}{\pgfqpoint{5.320557in}{1.945216in}}%
\pgfpathcurveto{\pgfqpoint{5.312321in}{1.945216in}}{\pgfqpoint{5.304421in}{1.941944in}}{\pgfqpoint{5.298597in}{1.936120in}}%
\pgfpathcurveto{\pgfqpoint{5.292773in}{1.930296in}}{\pgfqpoint{5.289501in}{1.922396in}}{\pgfqpoint{5.289501in}{1.914159in}}%
\pgfpathcurveto{\pgfqpoint{5.289501in}{1.905923in}}{\pgfqpoint{5.292773in}{1.898023in}}{\pgfqpoint{5.298597in}{1.892199in}}%
\pgfpathcurveto{\pgfqpoint{5.304421in}{1.886375in}}{\pgfqpoint{5.312321in}{1.883103in}}{\pgfqpoint{5.320557in}{1.883103in}}%
\pgfpathclose%
\pgfusepath{stroke,fill}%
\end{pgfscope}%
\begin{pgfscope}%
\pgfpathrectangle{\pgfqpoint{3.755891in}{0.557870in}}{\pgfqpoint{2.484109in}{1.484734in}}%
\pgfusepath{clip}%
\pgfsetbuttcap%
\pgfsetroundjoin%
\definecolor{currentfill}{rgb}{0.298039,0.447059,0.690196}%
\pgfsetfillcolor{currentfill}%
\pgfsetlinewidth{1.003750pt}%
\definecolor{currentstroke}{rgb}{0.298039,0.447059,0.690196}%
\pgfsetstrokecolor{currentstroke}%
\pgfsetdash{}{0pt}%
\pgfpathmoveto{\pgfqpoint{5.481863in}{1.900519in}}%
\pgfpathcurveto{\pgfqpoint{5.490099in}{1.900519in}}{\pgfqpoint{5.497999in}{1.903791in}}{\pgfqpoint{5.503823in}{1.909615in}}%
\pgfpathcurveto{\pgfqpoint{5.509647in}{1.915439in}}{\pgfqpoint{5.512919in}{1.923339in}}{\pgfqpoint{5.512919in}{1.931576in}}%
\pgfpathcurveto{\pgfqpoint{5.512919in}{1.939812in}}{\pgfqpoint{5.509647in}{1.947712in}}{\pgfqpoint{5.503823in}{1.953536in}}%
\pgfpathcurveto{\pgfqpoint{5.497999in}{1.959360in}}{\pgfqpoint{5.490099in}{1.962632in}}{\pgfqpoint{5.481863in}{1.962632in}}%
\pgfpathcurveto{\pgfqpoint{5.473627in}{1.962632in}}{\pgfqpoint{5.465727in}{1.959360in}}{\pgfqpoint{5.459903in}{1.953536in}}%
\pgfpathcurveto{\pgfqpoint{5.454079in}{1.947712in}}{\pgfqpoint{5.450806in}{1.939812in}}{\pgfqpoint{5.450806in}{1.931576in}}%
\pgfpathcurveto{\pgfqpoint{5.450806in}{1.923339in}}{\pgfqpoint{5.454079in}{1.915439in}}{\pgfqpoint{5.459903in}{1.909615in}}%
\pgfpathcurveto{\pgfqpoint{5.465727in}{1.903791in}}{\pgfqpoint{5.473627in}{1.900519in}}{\pgfqpoint{5.481863in}{1.900519in}}%
\pgfpathclose%
\pgfusepath{stroke,fill}%
\end{pgfscope}%
\begin{pgfscope}%
\pgfpathrectangle{\pgfqpoint{3.755891in}{0.557870in}}{\pgfqpoint{2.484109in}{1.484734in}}%
\pgfusepath{clip}%
\pgfsetbuttcap%
\pgfsetroundjoin%
\definecolor{currentfill}{rgb}{0.298039,0.447059,0.690196}%
\pgfsetfillcolor{currentfill}%
\pgfsetlinewidth{1.003750pt}%
\definecolor{currentstroke}{rgb}{0.298039,0.447059,0.690196}%
\pgfsetstrokecolor{currentstroke}%
\pgfsetdash{}{0pt}%
\pgfpathmoveto{\pgfqpoint{5.885127in}{1.613151in}}%
\pgfpathcurveto{\pgfqpoint{5.893364in}{1.613151in}}{\pgfqpoint{5.901264in}{1.616424in}}{\pgfqpoint{5.907088in}{1.622248in}}%
\pgfpathcurveto{\pgfqpoint{5.912912in}{1.628071in}}{\pgfqpoint{5.916184in}{1.635972in}}{\pgfqpoint{5.916184in}{1.644208in}}%
\pgfpathcurveto{\pgfqpoint{5.916184in}{1.652444in}}{\pgfqpoint{5.912912in}{1.660344in}}{\pgfqpoint{5.907088in}{1.666168in}}%
\pgfpathcurveto{\pgfqpoint{5.901264in}{1.671992in}}{\pgfqpoint{5.893364in}{1.675264in}}{\pgfqpoint{5.885127in}{1.675264in}}%
\pgfpathcurveto{\pgfqpoint{5.876891in}{1.675264in}}{\pgfqpoint{5.868991in}{1.671992in}}{\pgfqpoint{5.863167in}{1.666168in}}%
\pgfpathcurveto{\pgfqpoint{5.857343in}{1.660344in}}{\pgfqpoint{5.854071in}{1.652444in}}{\pgfqpoint{5.854071in}{1.644208in}}%
\pgfpathcurveto{\pgfqpoint{5.854071in}{1.635972in}}{\pgfqpoint{5.857343in}{1.628071in}}{\pgfqpoint{5.863167in}{1.622248in}}%
\pgfpathcurveto{\pgfqpoint{5.868991in}{1.616424in}}{\pgfqpoint{5.876891in}{1.613151in}}{\pgfqpoint{5.885127in}{1.613151in}}%
\pgfpathclose%
\pgfusepath{stroke,fill}%
\end{pgfscope}%
\begin{pgfscope}%
\pgfpathrectangle{\pgfqpoint{3.755891in}{0.557870in}}{\pgfqpoint{2.484109in}{1.484734in}}%
\pgfusepath{clip}%
\pgfsetbuttcap%
\pgfsetroundjoin%
\definecolor{currentfill}{rgb}{0.298039,0.447059,0.690196}%
\pgfsetfillcolor{currentfill}%
\pgfsetlinewidth{1.003750pt}%
\definecolor{currentstroke}{rgb}{0.298039,0.447059,0.690196}%
\pgfsetstrokecolor{currentstroke}%
\pgfsetdash{}{0pt}%
\pgfpathmoveto{\pgfqpoint{4.233044in}{1.902656in}}%
\pgfpathcurveto{\pgfqpoint{4.241280in}{1.902656in}}{\pgfqpoint{4.249180in}{1.905928in}}{\pgfqpoint{4.255004in}{1.911752in}}%
\pgfpathcurveto{\pgfqpoint{4.260828in}{1.917576in}}{\pgfqpoint{4.264101in}{1.925476in}}{\pgfqpoint{4.264101in}{1.933713in}}%
\pgfpathcurveto{\pgfqpoint{4.264101in}{1.941949in}}{\pgfqpoint{4.260828in}{1.949849in}}{\pgfqpoint{4.255004in}{1.955673in}}%
\pgfpathcurveto{\pgfqpoint{4.249180in}{1.961497in}}{\pgfqpoint{4.241280in}{1.964769in}}{\pgfqpoint{4.233044in}{1.964769in}}%
\pgfpathcurveto{\pgfqpoint{4.224808in}{1.964769in}}{\pgfqpoint{4.216908in}{1.961497in}}{\pgfqpoint{4.211084in}{1.955673in}}%
\pgfpathcurveto{\pgfqpoint{4.205260in}{1.949849in}}{\pgfqpoint{4.201988in}{1.941949in}}{\pgfqpoint{4.201988in}{1.933713in}}%
\pgfpathcurveto{\pgfqpoint{4.201988in}{1.925476in}}{\pgfqpoint{4.205260in}{1.917576in}}{\pgfqpoint{4.211084in}{1.911752in}}%
\pgfpathcurveto{\pgfqpoint{4.216908in}{1.905928in}}{\pgfqpoint{4.224808in}{1.902656in}}{\pgfqpoint{4.233044in}{1.902656in}}%
\pgfpathclose%
\pgfusepath{stroke,fill}%
\end{pgfscope}%
\begin{pgfscope}%
\pgfpathrectangle{\pgfqpoint{3.755891in}{0.557870in}}{\pgfqpoint{2.484109in}{1.484734in}}%
\pgfusepath{clip}%
\pgfsetbuttcap%
\pgfsetroundjoin%
\definecolor{currentfill}{rgb}{0.298039,0.447059,0.690196}%
\pgfsetfillcolor{currentfill}%
\pgfsetlinewidth{1.003750pt}%
\definecolor{currentstroke}{rgb}{0.298039,0.447059,0.690196}%
\pgfsetstrokecolor{currentstroke}%
\pgfsetdash{}{0pt}%
\pgfpathmoveto{\pgfqpoint{5.180065in}{1.819849in}}%
\pgfpathcurveto{\pgfqpoint{5.188301in}{1.819849in}}{\pgfqpoint{5.196201in}{1.823121in}}{\pgfqpoint{5.202025in}{1.828945in}}%
\pgfpathcurveto{\pgfqpoint{5.207849in}{1.834769in}}{\pgfqpoint{5.211122in}{1.842669in}}{\pgfqpoint{5.211122in}{1.850905in}}%
\pgfpathcurveto{\pgfqpoint{5.211122in}{1.859142in}}{\pgfqpoint{5.207849in}{1.867042in}}{\pgfqpoint{5.202025in}{1.872866in}}%
\pgfpathcurveto{\pgfqpoint{5.196201in}{1.878690in}}{\pgfqpoint{5.188301in}{1.881962in}}{\pgfqpoint{5.180065in}{1.881962in}}%
\pgfpathcurveto{\pgfqpoint{5.171829in}{1.881962in}}{\pgfqpoint{5.163929in}{1.878690in}}{\pgfqpoint{5.158105in}{1.872866in}}%
\pgfpathcurveto{\pgfqpoint{5.152281in}{1.867042in}}{\pgfqpoint{5.149009in}{1.859142in}}{\pgfqpoint{5.149009in}{1.850905in}}%
\pgfpathcurveto{\pgfqpoint{5.149009in}{1.842669in}}{\pgfqpoint{5.152281in}{1.834769in}}{\pgfqpoint{5.158105in}{1.828945in}}%
\pgfpathcurveto{\pgfqpoint{5.163929in}{1.823121in}}{\pgfqpoint{5.171829in}{1.819849in}}{\pgfqpoint{5.180065in}{1.819849in}}%
\pgfpathclose%
\pgfusepath{stroke,fill}%
\end{pgfscope}%
\begin{pgfscope}%
\pgfpathrectangle{\pgfqpoint{3.755891in}{0.557870in}}{\pgfqpoint{2.484109in}{1.484734in}}%
\pgfusepath{clip}%
\pgfsetbuttcap%
\pgfsetroundjoin%
\definecolor{currentfill}{rgb}{0.298039,0.447059,0.690196}%
\pgfsetfillcolor{currentfill}%
\pgfsetlinewidth{1.003750pt}%
\definecolor{currentstroke}{rgb}{0.298039,0.447059,0.690196}%
\pgfsetstrokecolor{currentstroke}%
\pgfsetdash{}{0pt}%
\pgfpathmoveto{\pgfqpoint{4.014501in}{1.927498in}}%
\pgfpathcurveto{\pgfqpoint{4.022737in}{1.927498in}}{\pgfqpoint{4.030637in}{1.930771in}}{\pgfqpoint{4.036461in}{1.936595in}}%
\pgfpathcurveto{\pgfqpoint{4.042285in}{1.942418in}}{\pgfqpoint{4.045557in}{1.950319in}}{\pgfqpoint{4.045557in}{1.958555in}}%
\pgfpathcurveto{\pgfqpoint{4.045557in}{1.966791in}}{\pgfqpoint{4.042285in}{1.974691in}}{\pgfqpoint{4.036461in}{1.980515in}}%
\pgfpathcurveto{\pgfqpoint{4.030637in}{1.986339in}}{\pgfqpoint{4.022737in}{1.989611in}}{\pgfqpoint{4.014501in}{1.989611in}}%
\pgfpathcurveto{\pgfqpoint{4.006265in}{1.989611in}}{\pgfqpoint{3.998365in}{1.986339in}}{\pgfqpoint{3.992541in}{1.980515in}}%
\pgfpathcurveto{\pgfqpoint{3.986717in}{1.974691in}}{\pgfqpoint{3.983444in}{1.966791in}}{\pgfqpoint{3.983444in}{1.958555in}}%
\pgfpathcurveto{\pgfqpoint{3.983444in}{1.950319in}}{\pgfqpoint{3.986717in}{1.942418in}}{\pgfqpoint{3.992541in}{1.936595in}}%
\pgfpathcurveto{\pgfqpoint{3.998365in}{1.930771in}}{\pgfqpoint{4.006265in}{1.927498in}}{\pgfqpoint{4.014501in}{1.927498in}}%
\pgfpathclose%
\pgfusepath{stroke,fill}%
\end{pgfscope}%
\begin{pgfscope}%
\pgfpathrectangle{\pgfqpoint{3.755891in}{0.557870in}}{\pgfqpoint{2.484109in}{1.484734in}}%
\pgfusepath{clip}%
\pgfsetbuttcap%
\pgfsetroundjoin%
\definecolor{currentfill}{rgb}{0.298039,0.447059,0.690196}%
\pgfsetfillcolor{currentfill}%
\pgfsetlinewidth{1.003750pt}%
\definecolor{currentstroke}{rgb}{0.298039,0.447059,0.690196}%
\pgfsetstrokecolor{currentstroke}%
\pgfsetdash{}{0pt}%
\pgfpathmoveto{\pgfqpoint{4.014501in}{1.935779in}}%
\pgfpathcurveto{\pgfqpoint{4.022737in}{1.935779in}}{\pgfqpoint{4.030637in}{1.939051in}}{\pgfqpoint{4.036461in}{1.944875in}}%
\pgfpathcurveto{\pgfqpoint{4.042285in}{1.950699in}}{\pgfqpoint{4.045557in}{1.958599in}}{\pgfqpoint{4.045557in}{1.966836in}}%
\pgfpathcurveto{\pgfqpoint{4.045557in}{1.975072in}}{\pgfqpoint{4.042285in}{1.982972in}}{\pgfqpoint{4.036461in}{1.988796in}}%
\pgfpathcurveto{\pgfqpoint{4.030637in}{1.994620in}}{\pgfqpoint{4.022737in}{1.997892in}}{\pgfqpoint{4.014501in}{1.997892in}}%
\pgfpathcurveto{\pgfqpoint{4.006265in}{1.997892in}}{\pgfqpoint{3.998365in}{1.994620in}}{\pgfqpoint{3.992541in}{1.988796in}}%
\pgfpathcurveto{\pgfqpoint{3.986717in}{1.982972in}}{\pgfqpoint{3.983444in}{1.975072in}}{\pgfqpoint{3.983444in}{1.966836in}}%
\pgfpathcurveto{\pgfqpoint{3.983444in}{1.958599in}}{\pgfqpoint{3.986717in}{1.950699in}}{\pgfqpoint{3.992541in}{1.944875in}}%
\pgfpathcurveto{\pgfqpoint{3.998365in}{1.939051in}}{\pgfqpoint{4.006265in}{1.935779in}}{\pgfqpoint{4.014501in}{1.935779in}}%
\pgfpathclose%
\pgfusepath{stroke,fill}%
\end{pgfscope}%
\begin{pgfscope}%
\pgfpathrectangle{\pgfqpoint{3.755891in}{0.557870in}}{\pgfqpoint{2.484109in}{1.484734in}}%
\pgfusepath{clip}%
\pgfsetbuttcap%
\pgfsetroundjoin%
\definecolor{currentfill}{rgb}{0.298039,0.447059,0.690196}%
\pgfsetfillcolor{currentfill}%
\pgfsetlinewidth{1.003750pt}%
\definecolor{currentstroke}{rgb}{0.298039,0.447059,0.690196}%
\pgfsetstrokecolor{currentstroke}%
\pgfsetdash{}{0pt}%
\pgfpathmoveto{\pgfqpoint{5.107217in}{1.811568in}}%
\pgfpathcurveto{\pgfqpoint{5.115454in}{1.811568in}}{\pgfqpoint{5.123354in}{1.814840in}}{\pgfqpoint{5.129178in}{1.820664in}}%
\pgfpathcurveto{\pgfqpoint{5.135001in}{1.826488in}}{\pgfqpoint{5.138274in}{1.834388in}}{\pgfqpoint{5.138274in}{1.842625in}}%
\pgfpathcurveto{\pgfqpoint{5.138274in}{1.850861in}}{\pgfqpoint{5.135001in}{1.858761in}}{\pgfqpoint{5.129178in}{1.864585in}}%
\pgfpathcurveto{\pgfqpoint{5.123354in}{1.870409in}}{\pgfqpoint{5.115454in}{1.873681in}}{\pgfqpoint{5.107217in}{1.873681in}}%
\pgfpathcurveto{\pgfqpoint{5.098981in}{1.873681in}}{\pgfqpoint{5.091081in}{1.870409in}}{\pgfqpoint{5.085257in}{1.864585in}}%
\pgfpathcurveto{\pgfqpoint{5.079433in}{1.858761in}}{\pgfqpoint{5.076161in}{1.850861in}}{\pgfqpoint{5.076161in}{1.842625in}}%
\pgfpathcurveto{\pgfqpoint{5.076161in}{1.834388in}}{\pgfqpoint{5.079433in}{1.826488in}}{\pgfqpoint{5.085257in}{1.820664in}}%
\pgfpathcurveto{\pgfqpoint{5.091081in}{1.814840in}}{\pgfqpoint{5.098981in}{1.811568in}}{\pgfqpoint{5.107217in}{1.811568in}}%
\pgfpathclose%
\pgfusepath{stroke,fill}%
\end{pgfscope}%
\begin{pgfscope}%
\pgfpathrectangle{\pgfqpoint{3.755891in}{0.557870in}}{\pgfqpoint{2.484109in}{1.484734in}}%
\pgfusepath{clip}%
\pgfsetbuttcap%
\pgfsetroundjoin%
\definecolor{currentfill}{rgb}{0.298039,0.447059,0.690196}%
\pgfsetfillcolor{currentfill}%
\pgfsetlinewidth{1.003750pt}%
\definecolor{currentstroke}{rgb}{0.298039,0.447059,0.690196}%
\pgfsetstrokecolor{currentstroke}%
\pgfsetdash{}{0pt}%
\pgfpathmoveto{\pgfqpoint{4.888674in}{1.811568in}}%
\pgfpathcurveto{\pgfqpoint{4.896910in}{1.811568in}}{\pgfqpoint{4.904810in}{1.814840in}}{\pgfqpoint{4.910634in}{1.820664in}}%
\pgfpathcurveto{\pgfqpoint{4.916458in}{1.826488in}}{\pgfqpoint{4.919731in}{1.834388in}}{\pgfqpoint{4.919731in}{1.842625in}}%
\pgfpathcurveto{\pgfqpoint{4.919731in}{1.850861in}}{\pgfqpoint{4.916458in}{1.858761in}}{\pgfqpoint{4.910634in}{1.864585in}}%
\pgfpathcurveto{\pgfqpoint{4.904810in}{1.870409in}}{\pgfqpoint{4.896910in}{1.873681in}}{\pgfqpoint{4.888674in}{1.873681in}}%
\pgfpathcurveto{\pgfqpoint{4.880438in}{1.873681in}}{\pgfqpoint{4.872538in}{1.870409in}}{\pgfqpoint{4.866714in}{1.864585in}}%
\pgfpathcurveto{\pgfqpoint{4.860890in}{1.858761in}}{\pgfqpoint{4.857618in}{1.850861in}}{\pgfqpoint{4.857618in}{1.842625in}}%
\pgfpathcurveto{\pgfqpoint{4.857618in}{1.834388in}}{\pgfqpoint{4.860890in}{1.826488in}}{\pgfqpoint{4.866714in}{1.820664in}}%
\pgfpathcurveto{\pgfqpoint{4.872538in}{1.814840in}}{\pgfqpoint{4.880438in}{1.811568in}}{\pgfqpoint{4.888674in}{1.811568in}}%
\pgfpathclose%
\pgfusepath{stroke,fill}%
\end{pgfscope}%
\begin{pgfscope}%
\pgfpathrectangle{\pgfqpoint{3.755891in}{0.557870in}}{\pgfqpoint{2.484109in}{1.484734in}}%
\pgfusepath{clip}%
\pgfsetbuttcap%
\pgfsetroundjoin%
\definecolor{currentfill}{rgb}{0.298039,0.447059,0.690196}%
\pgfsetfillcolor{currentfill}%
\pgfsetlinewidth{1.003750pt}%
\definecolor{currentstroke}{rgb}{0.298039,0.447059,0.690196}%
\pgfsetstrokecolor{currentstroke}%
\pgfsetdash{}{0pt}%
\pgfpathmoveto{\pgfqpoint{5.034370in}{1.852972in}}%
\pgfpathcurveto{\pgfqpoint{5.042606in}{1.852972in}}{\pgfqpoint{5.050506in}{1.856244in}}{\pgfqpoint{5.056330in}{1.862068in}}%
\pgfpathcurveto{\pgfqpoint{5.062154in}{1.867892in}}{\pgfqpoint{5.065426in}{1.875792in}}{\pgfqpoint{5.065426in}{1.884028in}}%
\pgfpathcurveto{\pgfqpoint{5.065426in}{1.892265in}}{\pgfqpoint{5.062154in}{1.900165in}}{\pgfqpoint{5.056330in}{1.905989in}}%
\pgfpathcurveto{\pgfqpoint{5.050506in}{1.911812in}}{\pgfqpoint{5.042606in}{1.915085in}}{\pgfqpoint{5.034370in}{1.915085in}}%
\pgfpathcurveto{\pgfqpoint{5.026133in}{1.915085in}}{\pgfqpoint{5.018233in}{1.911812in}}{\pgfqpoint{5.012409in}{1.905989in}}%
\pgfpathcurveto{\pgfqpoint{5.006585in}{1.900165in}}{\pgfqpoint{5.003313in}{1.892265in}}{\pgfqpoint{5.003313in}{1.884028in}}%
\pgfpathcurveto{\pgfqpoint{5.003313in}{1.875792in}}{\pgfqpoint{5.006585in}{1.867892in}}{\pgfqpoint{5.012409in}{1.862068in}}%
\pgfpathcurveto{\pgfqpoint{5.018233in}{1.856244in}}{\pgfqpoint{5.026133in}{1.852972in}}{\pgfqpoint{5.034370in}{1.852972in}}%
\pgfpathclose%
\pgfusepath{stroke,fill}%
\end{pgfscope}%
\begin{pgfscope}%
\pgfpathrectangle{\pgfqpoint{3.755891in}{0.557870in}}{\pgfqpoint{2.484109in}{1.484734in}}%
\pgfusepath{clip}%
\pgfsetbuttcap%
\pgfsetroundjoin%
\definecolor{currentfill}{rgb}{0.298039,0.447059,0.690196}%
\pgfsetfillcolor{currentfill}%
\pgfsetlinewidth{1.003750pt}%
\definecolor{currentstroke}{rgb}{0.298039,0.447059,0.690196}%
\pgfsetstrokecolor{currentstroke}%
\pgfsetdash{}{0pt}%
\pgfpathmoveto{\pgfqpoint{5.180065in}{1.869533in}}%
\pgfpathcurveto{\pgfqpoint{5.188301in}{1.869533in}}{\pgfqpoint{5.196201in}{1.872806in}}{\pgfqpoint{5.202025in}{1.878629in}}%
\pgfpathcurveto{\pgfqpoint{5.207849in}{1.884453in}}{\pgfqpoint{5.211122in}{1.892353in}}{\pgfqpoint{5.211122in}{1.900590in}}%
\pgfpathcurveto{\pgfqpoint{5.211122in}{1.908826in}}{\pgfqpoint{5.207849in}{1.916726in}}{\pgfqpoint{5.202025in}{1.922550in}}%
\pgfpathcurveto{\pgfqpoint{5.196201in}{1.928374in}}{\pgfqpoint{5.188301in}{1.931646in}}{\pgfqpoint{5.180065in}{1.931646in}}%
\pgfpathcurveto{\pgfqpoint{5.171829in}{1.931646in}}{\pgfqpoint{5.163929in}{1.928374in}}{\pgfqpoint{5.158105in}{1.922550in}}%
\pgfpathcurveto{\pgfqpoint{5.152281in}{1.916726in}}{\pgfqpoint{5.149009in}{1.908826in}}{\pgfqpoint{5.149009in}{1.900590in}}%
\pgfpathcurveto{\pgfqpoint{5.149009in}{1.892353in}}{\pgfqpoint{5.152281in}{1.884453in}}{\pgfqpoint{5.158105in}{1.878629in}}%
\pgfpathcurveto{\pgfqpoint{5.163929in}{1.872806in}}{\pgfqpoint{5.171829in}{1.869533in}}{\pgfqpoint{5.180065in}{1.869533in}}%
\pgfpathclose%
\pgfusepath{stroke,fill}%
\end{pgfscope}%
\begin{pgfscope}%
\pgfpathrectangle{\pgfqpoint{3.755891in}{0.557870in}}{\pgfqpoint{2.484109in}{1.484734in}}%
\pgfusepath{clip}%
\pgfsetbuttcap%
\pgfsetroundjoin%
\definecolor{currentfill}{rgb}{0.298039,0.447059,0.690196}%
\pgfsetfillcolor{currentfill}%
\pgfsetlinewidth{1.003750pt}%
\definecolor{currentstroke}{rgb}{0.298039,0.447059,0.690196}%
\pgfsetstrokecolor{currentstroke}%
\pgfsetdash{}{0pt}%
\pgfpathmoveto{\pgfqpoint{5.762847in}{1.687357in}}%
\pgfpathcurveto{\pgfqpoint{5.771083in}{1.687357in}}{\pgfqpoint{5.778983in}{1.690630in}}{\pgfqpoint{5.784807in}{1.696454in}}%
\pgfpathcurveto{\pgfqpoint{5.790631in}{1.702277in}}{\pgfqpoint{5.793904in}{1.710178in}}{\pgfqpoint{5.793904in}{1.718414in}}%
\pgfpathcurveto{\pgfqpoint{5.793904in}{1.726650in}}{\pgfqpoint{5.790631in}{1.734550in}}{\pgfqpoint{5.784807in}{1.740374in}}%
\pgfpathcurveto{\pgfqpoint{5.778983in}{1.746198in}}{\pgfqpoint{5.771083in}{1.749470in}}{\pgfqpoint{5.762847in}{1.749470in}}%
\pgfpathcurveto{\pgfqpoint{5.754611in}{1.749470in}}{\pgfqpoint{5.746711in}{1.746198in}}{\pgfqpoint{5.740887in}{1.740374in}}%
\pgfpathcurveto{\pgfqpoint{5.735063in}{1.734550in}}{\pgfqpoint{5.731791in}{1.726650in}}{\pgfqpoint{5.731791in}{1.718414in}}%
\pgfpathcurveto{\pgfqpoint{5.731791in}{1.710178in}}{\pgfqpoint{5.735063in}{1.702277in}}{\pgfqpoint{5.740887in}{1.696454in}}%
\pgfpathcurveto{\pgfqpoint{5.746711in}{1.690630in}}{\pgfqpoint{5.754611in}{1.687357in}}{\pgfqpoint{5.762847in}{1.687357in}}%
\pgfpathclose%
\pgfusepath{stroke,fill}%
\end{pgfscope}%
\begin{pgfscope}%
\pgfpathrectangle{\pgfqpoint{3.755891in}{0.557870in}}{\pgfqpoint{2.484109in}{1.484734in}}%
\pgfusepath{clip}%
\pgfsetbuttcap%
\pgfsetroundjoin%
\definecolor{currentfill}{rgb}{0.298039,0.447059,0.690196}%
\pgfsetfillcolor{currentfill}%
\pgfsetlinewidth{1.003750pt}%
\definecolor{currentstroke}{rgb}{0.298039,0.447059,0.690196}%
\pgfsetstrokecolor{currentstroke}%
\pgfsetdash{}{0pt}%
\pgfpathmoveto{\pgfqpoint{4.597283in}{1.861253in}}%
\pgfpathcurveto{\pgfqpoint{4.605519in}{1.861253in}}{\pgfqpoint{4.613419in}{1.864525in}}{\pgfqpoint{4.619243in}{1.870349in}}%
\pgfpathcurveto{\pgfqpoint{4.625067in}{1.876173in}}{\pgfqpoint{4.628339in}{1.884073in}}{\pgfqpoint{4.628339in}{1.892309in}}%
\pgfpathcurveto{\pgfqpoint{4.628339in}{1.900545in}}{\pgfqpoint{4.625067in}{1.908445in}}{\pgfqpoint{4.619243in}{1.914269in}}%
\pgfpathcurveto{\pgfqpoint{4.613419in}{1.920093in}}{\pgfqpoint{4.605519in}{1.923366in}}{\pgfqpoint{4.597283in}{1.923366in}}%
\pgfpathcurveto{\pgfqpoint{4.589047in}{1.923366in}}{\pgfqpoint{4.581147in}{1.920093in}}{\pgfqpoint{4.575323in}{1.914269in}}%
\pgfpathcurveto{\pgfqpoint{4.569499in}{1.908445in}}{\pgfqpoint{4.566226in}{1.900545in}}{\pgfqpoint{4.566226in}{1.892309in}}%
\pgfpathcurveto{\pgfqpoint{4.566226in}{1.884073in}}{\pgfqpoint{4.569499in}{1.876173in}}{\pgfqpoint{4.575323in}{1.870349in}}%
\pgfpathcurveto{\pgfqpoint{4.581147in}{1.864525in}}{\pgfqpoint{4.589047in}{1.861253in}}{\pgfqpoint{4.597283in}{1.861253in}}%
\pgfpathclose%
\pgfusepath{stroke,fill}%
\end{pgfscope}%
\begin{pgfscope}%
\pgfpathrectangle{\pgfqpoint{3.755891in}{0.557870in}}{\pgfqpoint{2.484109in}{1.484734in}}%
\pgfusepath{clip}%
\pgfsetbuttcap%
\pgfsetroundjoin%
\definecolor{currentfill}{rgb}{0.298039,0.447059,0.690196}%
\pgfsetfillcolor{currentfill}%
\pgfsetlinewidth{1.003750pt}%
\definecolor{currentstroke}{rgb}{0.298039,0.447059,0.690196}%
\pgfsetstrokecolor{currentstroke}%
\pgfsetdash{}{0pt}%
\pgfpathmoveto{\pgfqpoint{4.755987in}{1.876147in}}%
\pgfpathcurveto{\pgfqpoint{4.764223in}{1.876147in}}{\pgfqpoint{4.772123in}{1.879420in}}{\pgfqpoint{4.777947in}{1.885244in}}%
\pgfpathcurveto{\pgfqpoint{4.783771in}{1.891068in}}{\pgfqpoint{4.787044in}{1.898968in}}{\pgfqpoint{4.787044in}{1.907204in}}%
\pgfpathcurveto{\pgfqpoint{4.787044in}{1.915440in}}{\pgfqpoint{4.783771in}{1.923340in}}{\pgfqpoint{4.777947in}{1.929164in}}%
\pgfpathcurveto{\pgfqpoint{4.772123in}{1.934988in}}{\pgfqpoint{4.764223in}{1.938260in}}{\pgfqpoint{4.755987in}{1.938260in}}%
\pgfpathcurveto{\pgfqpoint{4.747751in}{1.938260in}}{\pgfqpoint{4.739851in}{1.934988in}}{\pgfqpoint{4.734027in}{1.929164in}}%
\pgfpathcurveto{\pgfqpoint{4.728203in}{1.923340in}}{\pgfqpoint{4.724931in}{1.915440in}}{\pgfqpoint{4.724931in}{1.907204in}}%
\pgfpathcurveto{\pgfqpoint{4.724931in}{1.898968in}}{\pgfqpoint{4.728203in}{1.891068in}}{\pgfqpoint{4.734027in}{1.885244in}}%
\pgfpathcurveto{\pgfqpoint{4.739851in}{1.879420in}}{\pgfqpoint{4.747751in}{1.876147in}}{\pgfqpoint{4.755987in}{1.876147in}}%
\pgfpathclose%
\pgfusepath{stroke,fill}%
\end{pgfscope}%
\begin{pgfscope}%
\pgfpathrectangle{\pgfqpoint{3.755891in}{0.557870in}}{\pgfqpoint{2.484109in}{1.484734in}}%
\pgfusepath{clip}%
\pgfsetbuttcap%
\pgfsetroundjoin%
\definecolor{currentfill}{rgb}{0.298039,0.447059,0.690196}%
\pgfsetfillcolor{currentfill}%
\pgfsetlinewidth{1.003750pt}%
\definecolor{currentstroke}{rgb}{0.298039,0.447059,0.690196}%
\pgfsetstrokecolor{currentstroke}%
\pgfsetdash{}{0pt}%
\pgfpathmoveto{\pgfqpoint{5.643169in}{1.859169in}}%
\pgfpathcurveto{\pgfqpoint{5.651405in}{1.859169in}}{\pgfqpoint{5.659305in}{1.862442in}}{\pgfqpoint{5.665129in}{1.868266in}}%
\pgfpathcurveto{\pgfqpoint{5.670953in}{1.874089in}}{\pgfqpoint{5.674225in}{1.881990in}}{\pgfqpoint{5.674225in}{1.890226in}}%
\pgfpathcurveto{\pgfqpoint{5.674225in}{1.898462in}}{\pgfqpoint{5.670953in}{1.906362in}}{\pgfqpoint{5.665129in}{1.912186in}}%
\pgfpathcurveto{\pgfqpoint{5.659305in}{1.918010in}}{\pgfqpoint{5.651405in}{1.921282in}}{\pgfqpoint{5.643169in}{1.921282in}}%
\pgfpathcurveto{\pgfqpoint{5.634932in}{1.921282in}}{\pgfqpoint{5.627032in}{1.918010in}}{\pgfqpoint{5.621208in}{1.912186in}}%
\pgfpathcurveto{\pgfqpoint{5.615385in}{1.906362in}}{\pgfqpoint{5.612112in}{1.898462in}}{\pgfqpoint{5.612112in}{1.890226in}}%
\pgfpathcurveto{\pgfqpoint{5.612112in}{1.881990in}}{\pgfqpoint{5.615385in}{1.874089in}}{\pgfqpoint{5.621208in}{1.868266in}}%
\pgfpathcurveto{\pgfqpoint{5.627032in}{1.862442in}}{\pgfqpoint{5.634932in}{1.859169in}}{\pgfqpoint{5.643169in}{1.859169in}}%
\pgfpathclose%
\pgfusepath{stroke,fill}%
\end{pgfscope}%
\begin{pgfscope}%
\pgfpathrectangle{\pgfqpoint{3.755891in}{0.557870in}}{\pgfqpoint{2.484109in}{1.484734in}}%
\pgfusepath{clip}%
\pgfsetbuttcap%
\pgfsetroundjoin%
\definecolor{currentfill}{rgb}{0.298039,0.447059,0.690196}%
\pgfsetfillcolor{currentfill}%
\pgfsetlinewidth{1.003750pt}%
\definecolor{currentstroke}{rgb}{0.298039,0.447059,0.690196}%
\pgfsetstrokecolor{currentstroke}%
\pgfsetdash{}{0pt}%
\pgfpathmoveto{\pgfqpoint{5.320557in}{1.901615in}}%
\pgfpathcurveto{\pgfqpoint{5.328793in}{1.901615in}}{\pgfqpoint{5.336694in}{1.904887in}}{\pgfqpoint{5.342517in}{1.910711in}}%
\pgfpathcurveto{\pgfqpoint{5.348341in}{1.916535in}}{\pgfqpoint{5.351614in}{1.924435in}}{\pgfqpoint{5.351614in}{1.932671in}}%
\pgfpathcurveto{\pgfqpoint{5.351614in}{1.940907in}}{\pgfqpoint{5.348341in}{1.948807in}}{\pgfqpoint{5.342517in}{1.954631in}}%
\pgfpathcurveto{\pgfqpoint{5.336694in}{1.960455in}}{\pgfqpoint{5.328793in}{1.963728in}}{\pgfqpoint{5.320557in}{1.963728in}}%
\pgfpathcurveto{\pgfqpoint{5.312321in}{1.963728in}}{\pgfqpoint{5.304421in}{1.960455in}}{\pgfqpoint{5.298597in}{1.954631in}}%
\pgfpathcurveto{\pgfqpoint{5.292773in}{1.948807in}}{\pgfqpoint{5.289501in}{1.940907in}}{\pgfqpoint{5.289501in}{1.932671in}}%
\pgfpathcurveto{\pgfqpoint{5.289501in}{1.924435in}}{\pgfqpoint{5.292773in}{1.916535in}}{\pgfqpoint{5.298597in}{1.910711in}}%
\pgfpathcurveto{\pgfqpoint{5.304421in}{1.904887in}}{\pgfqpoint{5.312321in}{1.901615in}}{\pgfqpoint{5.320557in}{1.901615in}}%
\pgfpathclose%
\pgfusepath{stroke,fill}%
\end{pgfscope}%
\begin{pgfscope}%
\pgfpathrectangle{\pgfqpoint{3.755891in}{0.557870in}}{\pgfqpoint{2.484109in}{1.484734in}}%
\pgfusepath{clip}%
\pgfsetbuttcap%
\pgfsetroundjoin%
\definecolor{currentfill}{rgb}{0.298039,0.447059,0.690196}%
\pgfsetfillcolor{currentfill}%
\pgfsetlinewidth{1.003750pt}%
\definecolor{currentstroke}{rgb}{0.298039,0.447059,0.690196}%
\pgfsetstrokecolor{currentstroke}%
\pgfsetdash{}{0pt}%
\pgfpathmoveto{\pgfqpoint{5.643169in}{1.867658in}}%
\pgfpathcurveto{\pgfqpoint{5.651405in}{1.867658in}}{\pgfqpoint{5.659305in}{1.870931in}}{\pgfqpoint{5.665129in}{1.876755in}}%
\pgfpathcurveto{\pgfqpoint{5.670953in}{1.882579in}}{\pgfqpoint{5.674225in}{1.890479in}}{\pgfqpoint{5.674225in}{1.898715in}}%
\pgfpathcurveto{\pgfqpoint{5.674225in}{1.906951in}}{\pgfqpoint{5.670953in}{1.914851in}}{\pgfqpoint{5.665129in}{1.920675in}}%
\pgfpathcurveto{\pgfqpoint{5.659305in}{1.926499in}}{\pgfqpoint{5.651405in}{1.929771in}}{\pgfqpoint{5.643169in}{1.929771in}}%
\pgfpathcurveto{\pgfqpoint{5.634932in}{1.929771in}}{\pgfqpoint{5.627032in}{1.926499in}}{\pgfqpoint{5.621208in}{1.920675in}}%
\pgfpathcurveto{\pgfqpoint{5.615385in}{1.914851in}}{\pgfqpoint{5.612112in}{1.906951in}}{\pgfqpoint{5.612112in}{1.898715in}}%
\pgfpathcurveto{\pgfqpoint{5.612112in}{1.890479in}}{\pgfqpoint{5.615385in}{1.882579in}}{\pgfqpoint{5.621208in}{1.876755in}}%
\pgfpathcurveto{\pgfqpoint{5.627032in}{1.870931in}}{\pgfqpoint{5.634932in}{1.867658in}}{\pgfqpoint{5.643169in}{1.867658in}}%
\pgfpathclose%
\pgfusepath{stroke,fill}%
\end{pgfscope}%
\begin{pgfscope}%
\pgfpathrectangle{\pgfqpoint{3.755891in}{0.557870in}}{\pgfqpoint{2.484109in}{1.484734in}}%
\pgfusepath{clip}%
\pgfsetbuttcap%
\pgfsetroundjoin%
\definecolor{currentfill}{rgb}{0.298039,0.447059,0.690196}%
\pgfsetfillcolor{currentfill}%
\pgfsetlinewidth{1.003750pt}%
\definecolor{currentstroke}{rgb}{0.298039,0.447059,0.690196}%
\pgfsetstrokecolor{currentstroke}%
\pgfsetdash{}{0pt}%
\pgfpathmoveto{\pgfqpoint{5.078599in}{1.876147in}}%
\pgfpathcurveto{\pgfqpoint{5.086835in}{1.876147in}}{\pgfqpoint{5.094735in}{1.879420in}}{\pgfqpoint{5.100559in}{1.885244in}}%
\pgfpathcurveto{\pgfqpoint{5.106383in}{1.891068in}}{\pgfqpoint{5.109655in}{1.898968in}}{\pgfqpoint{5.109655in}{1.907204in}}%
\pgfpathcurveto{\pgfqpoint{5.109655in}{1.915440in}}{\pgfqpoint{5.106383in}{1.923340in}}{\pgfqpoint{5.100559in}{1.929164in}}%
\pgfpathcurveto{\pgfqpoint{5.094735in}{1.934988in}}{\pgfqpoint{5.086835in}{1.938260in}}{\pgfqpoint{5.078599in}{1.938260in}}%
\pgfpathcurveto{\pgfqpoint{5.070362in}{1.938260in}}{\pgfqpoint{5.062462in}{1.934988in}}{\pgfqpoint{5.056638in}{1.929164in}}%
\pgfpathcurveto{\pgfqpoint{5.050814in}{1.923340in}}{\pgfqpoint{5.047542in}{1.915440in}}{\pgfqpoint{5.047542in}{1.907204in}}%
\pgfpathcurveto{\pgfqpoint{5.047542in}{1.898968in}}{\pgfqpoint{5.050814in}{1.891068in}}{\pgfqpoint{5.056638in}{1.885244in}}%
\pgfpathcurveto{\pgfqpoint{5.062462in}{1.879420in}}{\pgfqpoint{5.070362in}{1.876147in}}{\pgfqpoint{5.078599in}{1.876147in}}%
\pgfpathclose%
\pgfusepath{stroke,fill}%
\end{pgfscope}%
\begin{pgfscope}%
\pgfpathrectangle{\pgfqpoint{3.755891in}{0.557870in}}{\pgfqpoint{2.484109in}{1.484734in}}%
\pgfusepath{clip}%
\pgfsetbuttcap%
\pgfsetroundjoin%
\definecolor{currentfill}{rgb}{0.298039,0.447059,0.690196}%
\pgfsetfillcolor{currentfill}%
\pgfsetlinewidth{1.003750pt}%
\definecolor{currentstroke}{rgb}{0.298039,0.447059,0.690196}%
\pgfsetstrokecolor{currentstroke}%
\pgfsetdash{}{0pt}%
\pgfpathmoveto{\pgfqpoint{5.401210in}{1.893125in}}%
\pgfpathcurveto{\pgfqpoint{5.409446in}{1.893125in}}{\pgfqpoint{5.417346in}{1.896398in}}{\pgfqpoint{5.423170in}{1.902222in}}%
\pgfpathcurveto{\pgfqpoint{5.428994in}{1.908046in}}{\pgfqpoint{5.432267in}{1.915946in}}{\pgfqpoint{5.432267in}{1.924182in}}%
\pgfpathcurveto{\pgfqpoint{5.432267in}{1.932418in}}{\pgfqpoint{5.428994in}{1.940318in}}{\pgfqpoint{5.423170in}{1.946142in}}%
\pgfpathcurveto{\pgfqpoint{5.417346in}{1.951966in}}{\pgfqpoint{5.409446in}{1.955238in}}{\pgfqpoint{5.401210in}{1.955238in}}%
\pgfpathcurveto{\pgfqpoint{5.392974in}{1.955238in}}{\pgfqpoint{5.385074in}{1.951966in}}{\pgfqpoint{5.379250in}{1.946142in}}%
\pgfpathcurveto{\pgfqpoint{5.373426in}{1.940318in}}{\pgfqpoint{5.370154in}{1.932418in}}{\pgfqpoint{5.370154in}{1.924182in}}%
\pgfpathcurveto{\pgfqpoint{5.370154in}{1.915946in}}{\pgfqpoint{5.373426in}{1.908046in}}{\pgfqpoint{5.379250in}{1.902222in}}%
\pgfpathcurveto{\pgfqpoint{5.385074in}{1.896398in}}{\pgfqpoint{5.392974in}{1.893125in}}{\pgfqpoint{5.401210in}{1.893125in}}%
\pgfpathclose%
\pgfusepath{stroke,fill}%
\end{pgfscope}%
\begin{pgfscope}%
\pgfpathrectangle{\pgfqpoint{3.755891in}{0.557870in}}{\pgfqpoint{2.484109in}{1.484734in}}%
\pgfusepath{clip}%
\pgfsetbuttcap%
\pgfsetroundjoin%
\definecolor{currentfill}{rgb}{0.298039,0.447059,0.690196}%
\pgfsetfillcolor{currentfill}%
\pgfsetlinewidth{1.003750pt}%
\definecolor{currentstroke}{rgb}{0.298039,0.447059,0.690196}%
\pgfsetstrokecolor{currentstroke}%
\pgfsetdash{}{0pt}%
\pgfpathmoveto{\pgfqpoint{5.320557in}{1.910104in}}%
\pgfpathcurveto{\pgfqpoint{5.328793in}{1.910104in}}{\pgfqpoint{5.336694in}{1.913376in}}{\pgfqpoint{5.342517in}{1.919200in}}%
\pgfpathcurveto{\pgfqpoint{5.348341in}{1.925024in}}{\pgfqpoint{5.351614in}{1.932924in}}{\pgfqpoint{5.351614in}{1.941160in}}%
\pgfpathcurveto{\pgfqpoint{5.351614in}{1.949396in}}{\pgfqpoint{5.348341in}{1.957296in}}{\pgfqpoint{5.342517in}{1.963120in}}%
\pgfpathcurveto{\pgfqpoint{5.336694in}{1.968944in}}{\pgfqpoint{5.328793in}{1.972217in}}{\pgfqpoint{5.320557in}{1.972217in}}%
\pgfpathcurveto{\pgfqpoint{5.312321in}{1.972217in}}{\pgfqpoint{5.304421in}{1.968944in}}{\pgfqpoint{5.298597in}{1.963120in}}%
\pgfpathcurveto{\pgfqpoint{5.292773in}{1.957296in}}{\pgfqpoint{5.289501in}{1.949396in}}{\pgfqpoint{5.289501in}{1.941160in}}%
\pgfpathcurveto{\pgfqpoint{5.289501in}{1.932924in}}{\pgfqpoint{5.292773in}{1.925024in}}{\pgfqpoint{5.298597in}{1.919200in}}%
\pgfpathcurveto{\pgfqpoint{5.304421in}{1.913376in}}{\pgfqpoint{5.312321in}{1.910104in}}{\pgfqpoint{5.320557in}{1.910104in}}%
\pgfpathclose%
\pgfusepath{stroke,fill}%
\end{pgfscope}%
\begin{pgfscope}%
\pgfpathrectangle{\pgfqpoint{3.755891in}{0.557870in}}{\pgfqpoint{2.484109in}{1.484734in}}%
\pgfusepath{clip}%
\pgfsetbuttcap%
\pgfsetroundjoin%
\definecolor{currentfill}{rgb}{0.298039,0.447059,0.690196}%
\pgfsetfillcolor{currentfill}%
\pgfsetlinewidth{1.003750pt}%
\definecolor{currentstroke}{rgb}{0.298039,0.447059,0.690196}%
\pgfsetstrokecolor{currentstroke}%
\pgfsetdash{}{0pt}%
\pgfpathmoveto{\pgfqpoint{5.562516in}{1.893125in}}%
\pgfpathcurveto{\pgfqpoint{5.570752in}{1.893125in}}{\pgfqpoint{5.578652in}{1.896398in}}{\pgfqpoint{5.584476in}{1.902222in}}%
\pgfpathcurveto{\pgfqpoint{5.590300in}{1.908046in}}{\pgfqpoint{5.593572in}{1.915946in}}{\pgfqpoint{5.593572in}{1.924182in}}%
\pgfpathcurveto{\pgfqpoint{5.593572in}{1.932418in}}{\pgfqpoint{5.590300in}{1.940318in}}{\pgfqpoint{5.584476in}{1.946142in}}%
\pgfpathcurveto{\pgfqpoint{5.578652in}{1.951966in}}{\pgfqpoint{5.570752in}{1.955238in}}{\pgfqpoint{5.562516in}{1.955238in}}%
\pgfpathcurveto{\pgfqpoint{5.554280in}{1.955238in}}{\pgfqpoint{5.546379in}{1.951966in}}{\pgfqpoint{5.540556in}{1.946142in}}%
\pgfpathcurveto{\pgfqpoint{5.534732in}{1.940318in}}{\pgfqpoint{5.531459in}{1.932418in}}{\pgfqpoint{5.531459in}{1.924182in}}%
\pgfpathcurveto{\pgfqpoint{5.531459in}{1.915946in}}{\pgfqpoint{5.534732in}{1.908046in}}{\pgfqpoint{5.540556in}{1.902222in}}%
\pgfpathcurveto{\pgfqpoint{5.546379in}{1.896398in}}{\pgfqpoint{5.554280in}{1.893125in}}{\pgfqpoint{5.562516in}{1.893125in}}%
\pgfpathclose%
\pgfusepath{stroke,fill}%
\end{pgfscope}%
\begin{pgfscope}%
\pgfpathrectangle{\pgfqpoint{3.755891in}{0.557870in}}{\pgfqpoint{2.484109in}{1.484734in}}%
\pgfusepath{clip}%
\pgfsetbuttcap%
\pgfsetroundjoin%
\definecolor{currentfill}{rgb}{0.298039,0.447059,0.690196}%
\pgfsetfillcolor{currentfill}%
\pgfsetlinewidth{1.003750pt}%
\definecolor{currentstroke}{rgb}{0.298039,0.447059,0.690196}%
\pgfsetstrokecolor{currentstroke}%
\pgfsetdash{}{0pt}%
\pgfpathmoveto{\pgfqpoint{5.885127in}{1.612987in}}%
\pgfpathcurveto{\pgfqpoint{5.893364in}{1.612987in}}{\pgfqpoint{5.901264in}{1.616259in}}{\pgfqpoint{5.907088in}{1.622083in}}%
\pgfpathcurveto{\pgfqpoint{5.912912in}{1.627907in}}{\pgfqpoint{5.916184in}{1.635807in}}{\pgfqpoint{5.916184in}{1.644044in}}%
\pgfpathcurveto{\pgfqpoint{5.916184in}{1.652280in}}{\pgfqpoint{5.912912in}{1.660180in}}{\pgfqpoint{5.907088in}{1.666004in}}%
\pgfpathcurveto{\pgfqpoint{5.901264in}{1.671828in}}{\pgfqpoint{5.893364in}{1.675100in}}{\pgfqpoint{5.885127in}{1.675100in}}%
\pgfpathcurveto{\pgfqpoint{5.876891in}{1.675100in}}{\pgfqpoint{5.868991in}{1.671828in}}{\pgfqpoint{5.863167in}{1.666004in}}%
\pgfpathcurveto{\pgfqpoint{5.857343in}{1.660180in}}{\pgfqpoint{5.854071in}{1.652280in}}{\pgfqpoint{5.854071in}{1.644044in}}%
\pgfpathcurveto{\pgfqpoint{5.854071in}{1.635807in}}{\pgfqpoint{5.857343in}{1.627907in}}{\pgfqpoint{5.863167in}{1.622083in}}%
\pgfpathcurveto{\pgfqpoint{5.868991in}{1.616259in}}{\pgfqpoint{5.876891in}{1.612987in}}{\pgfqpoint{5.885127in}{1.612987in}}%
\pgfpathclose%
\pgfusepath{stroke,fill}%
\end{pgfscope}%
\begin{pgfscope}%
\pgfpathrectangle{\pgfqpoint{3.755891in}{0.557870in}}{\pgfqpoint{2.484109in}{1.484734in}}%
\pgfusepath{clip}%
\pgfsetbuttcap%
\pgfsetroundjoin%
\definecolor{currentfill}{rgb}{0.298039,0.447059,0.690196}%
\pgfsetfillcolor{currentfill}%
\pgfsetlinewidth{1.003750pt}%
\definecolor{currentstroke}{rgb}{0.298039,0.447059,0.690196}%
\pgfsetstrokecolor{currentstroke}%
\pgfsetdash{}{0pt}%
\pgfpathmoveto{\pgfqpoint{4.110764in}{1.927082in}}%
\pgfpathcurveto{\pgfqpoint{4.119000in}{1.927082in}}{\pgfqpoint{4.126900in}{1.930354in}}{\pgfqpoint{4.132724in}{1.936178in}}%
\pgfpathcurveto{\pgfqpoint{4.138548in}{1.942002in}}{\pgfqpoint{4.141820in}{1.949902in}}{\pgfqpoint{4.141820in}{1.958138in}}%
\pgfpathcurveto{\pgfqpoint{4.141820in}{1.966374in}}{\pgfqpoint{4.138548in}{1.974275in}}{\pgfqpoint{4.132724in}{1.980098in}}%
\pgfpathcurveto{\pgfqpoint{4.126900in}{1.985922in}}{\pgfqpoint{4.119000in}{1.989195in}}{\pgfqpoint{4.110764in}{1.989195in}}%
\pgfpathcurveto{\pgfqpoint{4.102528in}{1.989195in}}{\pgfqpoint{4.094628in}{1.985922in}}{\pgfqpoint{4.088804in}{1.980098in}}%
\pgfpathcurveto{\pgfqpoint{4.082980in}{1.974275in}}{\pgfqpoint{4.079707in}{1.966374in}}{\pgfqpoint{4.079707in}{1.958138in}}%
\pgfpathcurveto{\pgfqpoint{4.079707in}{1.949902in}}{\pgfqpoint{4.082980in}{1.942002in}}{\pgfqpoint{4.088804in}{1.936178in}}%
\pgfpathcurveto{\pgfqpoint{4.094628in}{1.930354in}}{\pgfqpoint{4.102528in}{1.927082in}}{\pgfqpoint{4.110764in}{1.927082in}}%
\pgfpathclose%
\pgfusepath{stroke,fill}%
\end{pgfscope}%
\begin{pgfscope}%
\pgfpathrectangle{\pgfqpoint{3.755891in}{0.557870in}}{\pgfqpoint{2.484109in}{1.484734in}}%
\pgfusepath{clip}%
\pgfsetbuttcap%
\pgfsetroundjoin%
\definecolor{currentfill}{rgb}{0.298039,0.447059,0.690196}%
\pgfsetfillcolor{currentfill}%
\pgfsetlinewidth{1.003750pt}%
\definecolor{currentstroke}{rgb}{0.298039,0.447059,0.690196}%
\pgfsetstrokecolor{currentstroke}%
\pgfsetdash{}{0pt}%
\pgfpathmoveto{\pgfqpoint{4.675334in}{1.865687in}}%
\pgfpathcurveto{\pgfqpoint{4.683570in}{1.865687in}}{\pgfqpoint{4.691470in}{1.868959in}}{\pgfqpoint{4.697294in}{1.874783in}}%
\pgfpathcurveto{\pgfqpoint{4.703118in}{1.880607in}}{\pgfqpoint{4.706391in}{1.888507in}}{\pgfqpoint{4.706391in}{1.896743in}}%
\pgfpathcurveto{\pgfqpoint{4.706391in}{1.904979in}}{\pgfqpoint{4.703118in}{1.912880in}}{\pgfqpoint{4.697294in}{1.918703in}}%
\pgfpathcurveto{\pgfqpoint{4.691470in}{1.924527in}}{\pgfqpoint{4.683570in}{1.927800in}}{\pgfqpoint{4.675334in}{1.927800in}}%
\pgfpathcurveto{\pgfqpoint{4.667098in}{1.927800in}}{\pgfqpoint{4.659198in}{1.924527in}}{\pgfqpoint{4.653374in}{1.918703in}}%
\pgfpathcurveto{\pgfqpoint{4.647550in}{1.912880in}}{\pgfqpoint{4.644278in}{1.904979in}}{\pgfqpoint{4.644278in}{1.896743in}}%
\pgfpathcurveto{\pgfqpoint{4.644278in}{1.888507in}}{\pgfqpoint{4.647550in}{1.880607in}}{\pgfqpoint{4.653374in}{1.874783in}}%
\pgfpathcurveto{\pgfqpoint{4.659198in}{1.868959in}}{\pgfqpoint{4.667098in}{1.865687in}}{\pgfqpoint{4.675334in}{1.865687in}}%
\pgfpathclose%
\pgfusepath{stroke,fill}%
\end{pgfscope}%
\begin{pgfscope}%
\pgfpathrectangle{\pgfqpoint{3.755891in}{0.557870in}}{\pgfqpoint{2.484109in}{1.484734in}}%
\pgfusepath{clip}%
\pgfsetbuttcap%
\pgfsetroundjoin%
\definecolor{currentfill}{rgb}{0.298039,0.447059,0.690196}%
\pgfsetfillcolor{currentfill}%
\pgfsetlinewidth{1.003750pt}%
\definecolor{currentstroke}{rgb}{0.298039,0.447059,0.690196}%
\pgfsetstrokecolor{currentstroke}%
\pgfsetdash{}{0pt}%
\pgfpathmoveto{\pgfqpoint{5.723822in}{1.865687in}}%
\pgfpathcurveto{\pgfqpoint{5.732058in}{1.865687in}}{\pgfqpoint{5.739958in}{1.868959in}}{\pgfqpoint{5.745782in}{1.874783in}}%
\pgfpathcurveto{\pgfqpoint{5.751606in}{1.880607in}}{\pgfqpoint{5.754878in}{1.888507in}}{\pgfqpoint{5.754878in}{1.896743in}}%
\pgfpathcurveto{\pgfqpoint{5.754878in}{1.904979in}}{\pgfqpoint{5.751606in}{1.912880in}}{\pgfqpoint{5.745782in}{1.918703in}}%
\pgfpathcurveto{\pgfqpoint{5.739958in}{1.924527in}}{\pgfqpoint{5.732058in}{1.927800in}}{\pgfqpoint{5.723822in}{1.927800in}}%
\pgfpathcurveto{\pgfqpoint{5.715585in}{1.927800in}}{\pgfqpoint{5.707685in}{1.924527in}}{\pgfqpoint{5.701861in}{1.918703in}}%
\pgfpathcurveto{\pgfqpoint{5.696037in}{1.912880in}}{\pgfqpoint{5.692765in}{1.904979in}}{\pgfqpoint{5.692765in}{1.896743in}}%
\pgfpathcurveto{\pgfqpoint{5.692765in}{1.888507in}}{\pgfqpoint{5.696037in}{1.880607in}}{\pgfqpoint{5.701861in}{1.874783in}}%
\pgfpathcurveto{\pgfqpoint{5.707685in}{1.868959in}}{\pgfqpoint{5.715585in}{1.865687in}}{\pgfqpoint{5.723822in}{1.865687in}}%
\pgfpathclose%
\pgfusepath{stroke,fill}%
\end{pgfscope}%
\begin{pgfscope}%
\pgfpathrectangle{\pgfqpoint{3.755891in}{0.557870in}}{\pgfqpoint{2.484109in}{1.484734in}}%
\pgfusepath{clip}%
\pgfsetbuttcap%
\pgfsetroundjoin%
\definecolor{currentfill}{rgb}{0.298039,0.447059,0.690196}%
\pgfsetfillcolor{currentfill}%
\pgfsetlinewidth{1.003750pt}%
\definecolor{currentstroke}{rgb}{0.298039,0.447059,0.690196}%
\pgfsetstrokecolor{currentstroke}%
\pgfsetdash{}{0pt}%
\pgfpathmoveto{\pgfqpoint{5.401210in}{1.900519in}}%
\pgfpathcurveto{\pgfqpoint{5.409446in}{1.900519in}}{\pgfqpoint{5.417346in}{1.903791in}}{\pgfqpoint{5.423170in}{1.909615in}}%
\pgfpathcurveto{\pgfqpoint{5.428994in}{1.915439in}}{\pgfqpoint{5.432267in}{1.923339in}}{\pgfqpoint{5.432267in}{1.931576in}}%
\pgfpathcurveto{\pgfqpoint{5.432267in}{1.939812in}}{\pgfqpoint{5.428994in}{1.947712in}}{\pgfqpoint{5.423170in}{1.953536in}}%
\pgfpathcurveto{\pgfqpoint{5.417346in}{1.959360in}}{\pgfqpoint{5.409446in}{1.962632in}}{\pgfqpoint{5.401210in}{1.962632in}}%
\pgfpathcurveto{\pgfqpoint{5.392974in}{1.962632in}}{\pgfqpoint{5.385074in}{1.959360in}}{\pgfqpoint{5.379250in}{1.953536in}}%
\pgfpathcurveto{\pgfqpoint{5.373426in}{1.947712in}}{\pgfqpoint{5.370154in}{1.939812in}}{\pgfqpoint{5.370154in}{1.931576in}}%
\pgfpathcurveto{\pgfqpoint{5.370154in}{1.923339in}}{\pgfqpoint{5.373426in}{1.915439in}}{\pgfqpoint{5.379250in}{1.909615in}}%
\pgfpathcurveto{\pgfqpoint{5.385074in}{1.903791in}}{\pgfqpoint{5.392974in}{1.900519in}}{\pgfqpoint{5.401210in}{1.900519in}}%
\pgfpathclose%
\pgfusepath{stroke,fill}%
\end{pgfscope}%
\begin{pgfscope}%
\pgfpathrectangle{\pgfqpoint{3.755891in}{0.557870in}}{\pgfqpoint{2.484109in}{1.484734in}}%
\pgfusepath{clip}%
\pgfsetbuttcap%
\pgfsetroundjoin%
\definecolor{currentfill}{rgb}{0.298039,0.447059,0.690196}%
\pgfsetfillcolor{currentfill}%
\pgfsetlinewidth{1.003750pt}%
\definecolor{currentstroke}{rgb}{0.298039,0.447059,0.690196}%
\pgfsetstrokecolor{currentstroke}%
\pgfsetdash{}{0pt}%
\pgfpathmoveto{\pgfqpoint{5.643169in}{1.848270in}}%
\pgfpathcurveto{\pgfqpoint{5.651405in}{1.848270in}}{\pgfqpoint{5.659305in}{1.851543in}}{\pgfqpoint{5.665129in}{1.857367in}}%
\pgfpathcurveto{\pgfqpoint{5.670953in}{1.863191in}}{\pgfqpoint{5.674225in}{1.871091in}}{\pgfqpoint{5.674225in}{1.879327in}}%
\pgfpathcurveto{\pgfqpoint{5.674225in}{1.887563in}}{\pgfqpoint{5.670953in}{1.895463in}}{\pgfqpoint{5.665129in}{1.901287in}}%
\pgfpathcurveto{\pgfqpoint{5.659305in}{1.907111in}}{\pgfqpoint{5.651405in}{1.910383in}}{\pgfqpoint{5.643169in}{1.910383in}}%
\pgfpathcurveto{\pgfqpoint{5.634932in}{1.910383in}}{\pgfqpoint{5.627032in}{1.907111in}}{\pgfqpoint{5.621208in}{1.901287in}}%
\pgfpathcurveto{\pgfqpoint{5.615385in}{1.895463in}}{\pgfqpoint{5.612112in}{1.887563in}}{\pgfqpoint{5.612112in}{1.879327in}}%
\pgfpathcurveto{\pgfqpoint{5.612112in}{1.871091in}}{\pgfqpoint{5.615385in}{1.863191in}}{\pgfqpoint{5.621208in}{1.857367in}}%
\pgfpathcurveto{\pgfqpoint{5.627032in}{1.851543in}}{\pgfqpoint{5.634932in}{1.848270in}}{\pgfqpoint{5.643169in}{1.848270in}}%
\pgfpathclose%
\pgfusepath{stroke,fill}%
\end{pgfscope}%
\begin{pgfscope}%
\pgfpathrectangle{\pgfqpoint{3.755891in}{0.557870in}}{\pgfqpoint{2.484109in}{1.484734in}}%
\pgfusepath{clip}%
\pgfsetbuttcap%
\pgfsetroundjoin%
\definecolor{currentfill}{rgb}{0.298039,0.447059,0.690196}%
\pgfsetfillcolor{currentfill}%
\pgfsetlinewidth{1.003750pt}%
\definecolor{currentstroke}{rgb}{0.298039,0.447059,0.690196}%
\pgfsetstrokecolor{currentstroke}%
\pgfsetdash{}{0pt}%
\pgfpathmoveto{\pgfqpoint{5.481863in}{1.839562in}}%
\pgfpathcurveto{\pgfqpoint{5.490099in}{1.839562in}}{\pgfqpoint{5.497999in}{1.842835in}}{\pgfqpoint{5.503823in}{1.848659in}}%
\pgfpathcurveto{\pgfqpoint{5.509647in}{1.854483in}}{\pgfqpoint{5.512919in}{1.862383in}}{\pgfqpoint{5.512919in}{1.870619in}}%
\pgfpathcurveto{\pgfqpoint{5.512919in}{1.878855in}}{\pgfqpoint{5.509647in}{1.886755in}}{\pgfqpoint{5.503823in}{1.892579in}}%
\pgfpathcurveto{\pgfqpoint{5.497999in}{1.898403in}}{\pgfqpoint{5.490099in}{1.901675in}}{\pgfqpoint{5.481863in}{1.901675in}}%
\pgfpathcurveto{\pgfqpoint{5.473627in}{1.901675in}}{\pgfqpoint{5.465727in}{1.898403in}}{\pgfqpoint{5.459903in}{1.892579in}}%
\pgfpathcurveto{\pgfqpoint{5.454079in}{1.886755in}}{\pgfqpoint{5.450806in}{1.878855in}}{\pgfqpoint{5.450806in}{1.870619in}}%
\pgfpathcurveto{\pgfqpoint{5.450806in}{1.862383in}}{\pgfqpoint{5.454079in}{1.854483in}}{\pgfqpoint{5.459903in}{1.848659in}}%
\pgfpathcurveto{\pgfqpoint{5.465727in}{1.842835in}}{\pgfqpoint{5.473627in}{1.839562in}}{\pgfqpoint{5.481863in}{1.839562in}}%
\pgfpathclose%
\pgfusepath{stroke,fill}%
\end{pgfscope}%
\begin{pgfscope}%
\pgfpathrectangle{\pgfqpoint{3.755891in}{0.557870in}}{\pgfqpoint{2.484109in}{1.484734in}}%
\pgfusepath{clip}%
\pgfsetbuttcap%
\pgfsetroundjoin%
\definecolor{currentfill}{rgb}{0.298039,0.447059,0.690196}%
\pgfsetfillcolor{currentfill}%
\pgfsetlinewidth{1.003750pt}%
\definecolor{currentstroke}{rgb}{0.298039,0.447059,0.690196}%
\pgfsetstrokecolor{currentstroke}%
\pgfsetdash{}{0pt}%
\pgfpathmoveto{\pgfqpoint{5.643169in}{1.909227in}}%
\pgfpathcurveto{\pgfqpoint{5.651405in}{1.909227in}}{\pgfqpoint{5.659305in}{1.912500in}}{\pgfqpoint{5.665129in}{1.918324in}}%
\pgfpathcurveto{\pgfqpoint{5.670953in}{1.924147in}}{\pgfqpoint{5.674225in}{1.932048in}}{\pgfqpoint{5.674225in}{1.940284in}}%
\pgfpathcurveto{\pgfqpoint{5.674225in}{1.948520in}}{\pgfqpoint{5.670953in}{1.956420in}}{\pgfqpoint{5.665129in}{1.962244in}}%
\pgfpathcurveto{\pgfqpoint{5.659305in}{1.968068in}}{\pgfqpoint{5.651405in}{1.971340in}}{\pgfqpoint{5.643169in}{1.971340in}}%
\pgfpathcurveto{\pgfqpoint{5.634932in}{1.971340in}}{\pgfqpoint{5.627032in}{1.968068in}}{\pgfqpoint{5.621208in}{1.962244in}}%
\pgfpathcurveto{\pgfqpoint{5.615385in}{1.956420in}}{\pgfqpoint{5.612112in}{1.948520in}}{\pgfqpoint{5.612112in}{1.940284in}}%
\pgfpathcurveto{\pgfqpoint{5.612112in}{1.932048in}}{\pgfqpoint{5.615385in}{1.924147in}}{\pgfqpoint{5.621208in}{1.918324in}}%
\pgfpathcurveto{\pgfqpoint{5.627032in}{1.912500in}}{\pgfqpoint{5.634932in}{1.909227in}}{\pgfqpoint{5.643169in}{1.909227in}}%
\pgfpathclose%
\pgfusepath{stroke,fill}%
\end{pgfscope}%
\begin{pgfscope}%
\pgfpathrectangle{\pgfqpoint{3.755891in}{0.557870in}}{\pgfqpoint{2.484109in}{1.484734in}}%
\pgfusepath{clip}%
\pgfsetbuttcap%
\pgfsetroundjoin%
\definecolor{currentfill}{rgb}{0.298039,0.447059,0.690196}%
\pgfsetfillcolor{currentfill}%
\pgfsetlinewidth{1.003750pt}%
\definecolor{currentstroke}{rgb}{0.298039,0.447059,0.690196}%
\pgfsetstrokecolor{currentstroke}%
\pgfsetdash{}{0pt}%
\pgfpathmoveto{\pgfqpoint{5.078599in}{1.874395in}}%
\pgfpathcurveto{\pgfqpoint{5.086835in}{1.874395in}}{\pgfqpoint{5.094735in}{1.877667in}}{\pgfqpoint{5.100559in}{1.883491in}}%
\pgfpathcurveto{\pgfqpoint{5.106383in}{1.889315in}}{\pgfqpoint{5.109655in}{1.897215in}}{\pgfqpoint{5.109655in}{1.905451in}}%
\pgfpathcurveto{\pgfqpoint{5.109655in}{1.913688in}}{\pgfqpoint{5.106383in}{1.921588in}}{\pgfqpoint{5.100559in}{1.927412in}}%
\pgfpathcurveto{\pgfqpoint{5.094735in}{1.933236in}}{\pgfqpoint{5.086835in}{1.936508in}}{\pgfqpoint{5.078599in}{1.936508in}}%
\pgfpathcurveto{\pgfqpoint{5.070362in}{1.936508in}}{\pgfqpoint{5.062462in}{1.933236in}}{\pgfqpoint{5.056638in}{1.927412in}}%
\pgfpathcurveto{\pgfqpoint{5.050814in}{1.921588in}}{\pgfqpoint{5.047542in}{1.913688in}}{\pgfqpoint{5.047542in}{1.905451in}}%
\pgfpathcurveto{\pgfqpoint{5.047542in}{1.897215in}}{\pgfqpoint{5.050814in}{1.889315in}}{\pgfqpoint{5.056638in}{1.883491in}}%
\pgfpathcurveto{\pgfqpoint{5.062462in}{1.877667in}}{\pgfqpoint{5.070362in}{1.874395in}}{\pgfqpoint{5.078599in}{1.874395in}}%
\pgfpathclose%
\pgfusepath{stroke,fill}%
\end{pgfscope}%
\begin{pgfscope}%
\pgfpathrectangle{\pgfqpoint{3.755891in}{0.557870in}}{\pgfqpoint{2.484109in}{1.484734in}}%
\pgfusepath{clip}%
\pgfsetbuttcap%
\pgfsetroundjoin%
\definecolor{currentfill}{rgb}{0.298039,0.447059,0.690196}%
\pgfsetfillcolor{currentfill}%
\pgfsetlinewidth{1.003750pt}%
\definecolor{currentstroke}{rgb}{0.298039,0.447059,0.690196}%
\pgfsetstrokecolor{currentstroke}%
\pgfsetdash{}{0pt}%
\pgfpathmoveto{\pgfqpoint{5.481863in}{1.900519in}}%
\pgfpathcurveto{\pgfqpoint{5.490099in}{1.900519in}}{\pgfqpoint{5.497999in}{1.903791in}}{\pgfqpoint{5.503823in}{1.909615in}}%
\pgfpathcurveto{\pgfqpoint{5.509647in}{1.915439in}}{\pgfqpoint{5.512919in}{1.923339in}}{\pgfqpoint{5.512919in}{1.931576in}}%
\pgfpathcurveto{\pgfqpoint{5.512919in}{1.939812in}}{\pgfqpoint{5.509647in}{1.947712in}}{\pgfqpoint{5.503823in}{1.953536in}}%
\pgfpathcurveto{\pgfqpoint{5.497999in}{1.959360in}}{\pgfqpoint{5.490099in}{1.962632in}}{\pgfqpoint{5.481863in}{1.962632in}}%
\pgfpathcurveto{\pgfqpoint{5.473627in}{1.962632in}}{\pgfqpoint{5.465727in}{1.959360in}}{\pgfqpoint{5.459903in}{1.953536in}}%
\pgfpathcurveto{\pgfqpoint{5.454079in}{1.947712in}}{\pgfqpoint{5.450806in}{1.939812in}}{\pgfqpoint{5.450806in}{1.931576in}}%
\pgfpathcurveto{\pgfqpoint{5.450806in}{1.923339in}}{\pgfqpoint{5.454079in}{1.915439in}}{\pgfqpoint{5.459903in}{1.909615in}}%
\pgfpathcurveto{\pgfqpoint{5.465727in}{1.903791in}}{\pgfqpoint{5.473627in}{1.900519in}}{\pgfqpoint{5.481863in}{1.900519in}}%
\pgfpathclose%
\pgfusepath{stroke,fill}%
\end{pgfscope}%
\begin{pgfscope}%
\pgfpathrectangle{\pgfqpoint{3.755891in}{0.557870in}}{\pgfqpoint{2.484109in}{1.484734in}}%
\pgfusepath{clip}%
\pgfsetbuttcap%
\pgfsetroundjoin%
\definecolor{currentfill}{rgb}{0.298039,0.447059,0.690196}%
\pgfsetfillcolor{currentfill}%
\pgfsetlinewidth{1.003750pt}%
\definecolor{currentstroke}{rgb}{0.298039,0.447059,0.690196}%
\pgfsetstrokecolor{currentstroke}%
\pgfsetdash{}{0pt}%
\pgfpathmoveto{\pgfqpoint{5.885127in}{1.604443in}}%
\pgfpathcurveto{\pgfqpoint{5.893364in}{1.604443in}}{\pgfqpoint{5.901264in}{1.607716in}}{\pgfqpoint{5.907088in}{1.613539in}}%
\pgfpathcurveto{\pgfqpoint{5.912912in}{1.619363in}}{\pgfqpoint{5.916184in}{1.627263in}}{\pgfqpoint{5.916184in}{1.635500in}}%
\pgfpathcurveto{\pgfqpoint{5.916184in}{1.643736in}}{\pgfqpoint{5.912912in}{1.651636in}}{\pgfqpoint{5.907088in}{1.657460in}}%
\pgfpathcurveto{\pgfqpoint{5.901264in}{1.663284in}}{\pgfqpoint{5.893364in}{1.666556in}}{\pgfqpoint{5.885127in}{1.666556in}}%
\pgfpathcurveto{\pgfqpoint{5.876891in}{1.666556in}}{\pgfqpoint{5.868991in}{1.663284in}}{\pgfqpoint{5.863167in}{1.657460in}}%
\pgfpathcurveto{\pgfqpoint{5.857343in}{1.651636in}}{\pgfqpoint{5.854071in}{1.643736in}}{\pgfqpoint{5.854071in}{1.635500in}}%
\pgfpathcurveto{\pgfqpoint{5.854071in}{1.627263in}}{\pgfqpoint{5.857343in}{1.619363in}}{\pgfqpoint{5.863167in}{1.613539in}}%
\pgfpathcurveto{\pgfqpoint{5.868991in}{1.607716in}}{\pgfqpoint{5.876891in}{1.604443in}}{\pgfqpoint{5.885127in}{1.604443in}}%
\pgfpathclose%
\pgfusepath{stroke,fill}%
\end{pgfscope}%
\begin{pgfscope}%
\pgfsetrectcap%
\pgfsetmiterjoin%
\pgfsetlinewidth{1.254687pt}%
\definecolor{currentstroke}{rgb}{1.000000,1.000000,1.000000}%
\pgfsetstrokecolor{currentstroke}%
\pgfsetdash{}{0pt}%
\pgfpathmoveto{\pgfqpoint{3.755891in}{0.557870in}}%
\pgfpathlineto{\pgfqpoint{3.755891in}{2.042604in}}%
\pgfusepath{stroke}%
\end{pgfscope}%
\begin{pgfscope}%
\pgfsetrectcap%
\pgfsetmiterjoin%
\pgfsetlinewidth{1.254687pt}%
\definecolor{currentstroke}{rgb}{1.000000,1.000000,1.000000}%
\pgfsetstrokecolor{currentstroke}%
\pgfsetdash{}{0pt}%
\pgfpathmoveto{\pgfqpoint{6.240000in}{0.557870in}}%
\pgfpathlineto{\pgfqpoint{6.240000in}{2.042604in}}%
\pgfusepath{stroke}%
\end{pgfscope}%
\begin{pgfscope}%
\pgfsetrectcap%
\pgfsetmiterjoin%
\pgfsetlinewidth{1.254687pt}%
\definecolor{currentstroke}{rgb}{1.000000,1.000000,1.000000}%
\pgfsetstrokecolor{currentstroke}%
\pgfsetdash{}{0pt}%
\pgfpathmoveto{\pgfqpoint{3.755891in}{0.557870in}}%
\pgfpathlineto{\pgfqpoint{6.240000in}{0.557870in}}%
\pgfusepath{stroke}%
\end{pgfscope}%
\begin{pgfscope}%
\pgfsetrectcap%
\pgfsetmiterjoin%
\pgfsetlinewidth{1.254687pt}%
\definecolor{currentstroke}{rgb}{1.000000,1.000000,1.000000}%
\pgfsetstrokecolor{currentstroke}%
\pgfsetdash{}{0pt}%
\pgfpathmoveto{\pgfqpoint{3.755891in}{2.042604in}}%
\pgfpathlineto{\pgfqpoint{6.240000in}{2.042604in}}%
\pgfusepath{stroke}%
\end{pgfscope}%
\begin{pgfscope}%
\definecolor{textcolor}{rgb}{0.150000,0.150000,0.150000}%
\pgfsetstrokecolor{textcolor}%
\pgfsetfillcolor{textcolor}%
\pgftext[x=4.997946in,y=2.125938in,,base]{\color{textcolor}\sffamily\fontsize{11.000000}{13.200000}\selectfont (b)}%
\end{pgfscope}%
\end{pgfpicture}%
\makeatother%
\endgroup%

    \caption{Distribution of DOR, sensitivity and specificity for the different peak-value classifiers trained to predict patient diagnosis.}
    \label{fig:pvmlc_ind_dor_sens_spec_dist}
\end{figure}

\newpage

\subsection{Comparisons}

\newpage

\section{Case Study: Segment Indication}

\subsection{Time-series Clustering}

\begin{figure}[htb]
    \centering
    % \includegraphics[width=\textwidth]{results/tsc_segm_ind_dor_sens_spec_dist.png}
    \caption{Distribution of DOR, sensitivity and specificity for the different TSC methods when classifying left ventrice segment indication.}
    \label{fig:tsc_segm_ind_dor_sens_spec_dist}
\end{figure}

\begin{figure}[htb]
    \centering
    % \includegraphics[width=\textwidth]{results/tsc_segm_ind_ari.png}
    %% Creator: Matplotlib, PGF backend
%%
%% To include the figure in your LaTeX document, write
%%   \input{<filename>.pgf}
%%
%% Make sure the required packages are loaded in your preamble
%%   \usepackage{pgf}
%%
%% Figures using additional raster images can only be included by \input if
%% they are in the same directory as the main LaTeX file. For loading figures
%% from other directories you can use the `import` package
%%   \usepackage{import}
%% and then include the figures with
%%   \import{<path to file>}{<filename>.pgf}
%%
%% Matplotlib used the following preamble
%%
\begingroup%
\makeatletter%
\begin{pgfpicture}%
\pgfpathrectangle{\pgfpointorigin}{\pgfqpoint{6.340000in}{2.340000in}}%
\pgfusepath{use as bounding box, clip}%
\begin{pgfscope}%
\pgfsetbuttcap%
\pgfsetmiterjoin%
\definecolor{currentfill}{rgb}{1.000000,1.000000,1.000000}%
\pgfsetfillcolor{currentfill}%
\pgfsetlinewidth{0.000000pt}%
\definecolor{currentstroke}{rgb}{1.000000,1.000000,1.000000}%
\pgfsetstrokecolor{currentstroke}%
\pgfsetdash{}{0pt}%
\pgfpathmoveto{\pgfqpoint{0.000000in}{-0.000000in}}%
\pgfpathlineto{\pgfqpoint{6.340000in}{-0.000000in}}%
\pgfpathlineto{\pgfqpoint{6.340000in}{2.340000in}}%
\pgfpathlineto{\pgfqpoint{0.000000in}{2.340000in}}%
\pgfpathclose%
\pgfusepath{fill}%
\end{pgfscope}%
\begin{pgfscope}%
\pgfsetbuttcap%
\pgfsetmiterjoin%
\definecolor{currentfill}{rgb}{0.917647,0.917647,0.949020}%
\pgfsetfillcolor{currentfill}%
\pgfsetlinewidth{0.000000pt}%
\definecolor{currentstroke}{rgb}{0.000000,0.000000,0.000000}%
\pgfsetstrokecolor{currentstroke}%
\pgfsetstrokeopacity{0.000000}%
\pgfsetdash{}{0pt}%
\pgfpathmoveto{\pgfqpoint{0.574769in}{0.557870in}}%
\pgfpathlineto{\pgfqpoint{6.240000in}{0.557870in}}%
\pgfpathlineto{\pgfqpoint{6.240000in}{2.240000in}}%
\pgfpathlineto{\pgfqpoint{0.574769in}{2.240000in}}%
\pgfpathclose%
\pgfusepath{fill}%
\end{pgfscope}%
\begin{pgfscope}%
\pgfpathrectangle{\pgfqpoint{0.574769in}{0.557870in}}{\pgfqpoint{5.665231in}{1.682130in}}%
\pgfusepath{clip}%
\pgfsetroundcap%
\pgfsetroundjoin%
\pgfsetlinewidth{1.003750pt}%
\definecolor{currentstroke}{rgb}{1.000000,1.000000,1.000000}%
\pgfsetstrokecolor{currentstroke}%
\pgfsetdash{}{0pt}%
\pgfpathmoveto{\pgfqpoint{0.897243in}{0.557870in}}%
\pgfpathlineto{\pgfqpoint{0.897243in}{2.240000in}}%
\pgfusepath{stroke}%
\end{pgfscope}%
\begin{pgfscope}%
\definecolor{textcolor}{rgb}{0.150000,0.150000,0.150000}%
\pgfsetstrokecolor{textcolor}%
\pgfsetfillcolor{textcolor}%
\pgftext[x=0.897243in,y=0.425926in,,top]{\color{textcolor}\sffamily\fontsize{11.000000}{13.200000}\selectfont \(\displaystyle 0.00\)}%
\end{pgfscope}%
\begin{pgfscope}%
\pgfpathrectangle{\pgfqpoint{0.574769in}{0.557870in}}{\pgfqpoint{5.665231in}{1.682130in}}%
\pgfusepath{clip}%
\pgfsetroundcap%
\pgfsetroundjoin%
\pgfsetlinewidth{1.003750pt}%
\definecolor{currentstroke}{rgb}{1.000000,1.000000,1.000000}%
\pgfsetstrokecolor{currentstroke}%
\pgfsetdash{}{0pt}%
\pgfpathmoveto{\pgfqpoint{1.866035in}{0.557870in}}%
\pgfpathlineto{\pgfqpoint{1.866035in}{2.240000in}}%
\pgfusepath{stroke}%
\end{pgfscope}%
\begin{pgfscope}%
\definecolor{textcolor}{rgb}{0.150000,0.150000,0.150000}%
\pgfsetstrokecolor{textcolor}%
\pgfsetfillcolor{textcolor}%
\pgftext[x=1.866035in,y=0.425926in,,top]{\color{textcolor}\sffamily\fontsize{11.000000}{13.200000}\selectfont \(\displaystyle 0.05\)}%
\end{pgfscope}%
\begin{pgfscope}%
\pgfpathrectangle{\pgfqpoint{0.574769in}{0.557870in}}{\pgfqpoint{5.665231in}{1.682130in}}%
\pgfusepath{clip}%
\pgfsetroundcap%
\pgfsetroundjoin%
\pgfsetlinewidth{1.003750pt}%
\definecolor{currentstroke}{rgb}{1.000000,1.000000,1.000000}%
\pgfsetstrokecolor{currentstroke}%
\pgfsetdash{}{0pt}%
\pgfpathmoveto{\pgfqpoint{2.834826in}{0.557870in}}%
\pgfpathlineto{\pgfqpoint{2.834826in}{2.240000in}}%
\pgfusepath{stroke}%
\end{pgfscope}%
\begin{pgfscope}%
\definecolor{textcolor}{rgb}{0.150000,0.150000,0.150000}%
\pgfsetstrokecolor{textcolor}%
\pgfsetfillcolor{textcolor}%
\pgftext[x=2.834826in,y=0.425926in,,top]{\color{textcolor}\sffamily\fontsize{11.000000}{13.200000}\selectfont \(\displaystyle 0.10\)}%
\end{pgfscope}%
\begin{pgfscope}%
\pgfpathrectangle{\pgfqpoint{0.574769in}{0.557870in}}{\pgfqpoint{5.665231in}{1.682130in}}%
\pgfusepath{clip}%
\pgfsetroundcap%
\pgfsetroundjoin%
\pgfsetlinewidth{1.003750pt}%
\definecolor{currentstroke}{rgb}{1.000000,1.000000,1.000000}%
\pgfsetstrokecolor{currentstroke}%
\pgfsetdash{}{0pt}%
\pgfpathmoveto{\pgfqpoint{3.803617in}{0.557870in}}%
\pgfpathlineto{\pgfqpoint{3.803617in}{2.240000in}}%
\pgfusepath{stroke}%
\end{pgfscope}%
\begin{pgfscope}%
\definecolor{textcolor}{rgb}{0.150000,0.150000,0.150000}%
\pgfsetstrokecolor{textcolor}%
\pgfsetfillcolor{textcolor}%
\pgftext[x=3.803617in,y=0.425926in,,top]{\color{textcolor}\sffamily\fontsize{11.000000}{13.200000}\selectfont \(\displaystyle 0.15\)}%
\end{pgfscope}%
\begin{pgfscope}%
\pgfpathrectangle{\pgfqpoint{0.574769in}{0.557870in}}{\pgfqpoint{5.665231in}{1.682130in}}%
\pgfusepath{clip}%
\pgfsetroundcap%
\pgfsetroundjoin%
\pgfsetlinewidth{1.003750pt}%
\definecolor{currentstroke}{rgb}{1.000000,1.000000,1.000000}%
\pgfsetstrokecolor{currentstroke}%
\pgfsetdash{}{0pt}%
\pgfpathmoveto{\pgfqpoint{4.772408in}{0.557870in}}%
\pgfpathlineto{\pgfqpoint{4.772408in}{2.240000in}}%
\pgfusepath{stroke}%
\end{pgfscope}%
\begin{pgfscope}%
\definecolor{textcolor}{rgb}{0.150000,0.150000,0.150000}%
\pgfsetstrokecolor{textcolor}%
\pgfsetfillcolor{textcolor}%
\pgftext[x=4.772408in,y=0.425926in,,top]{\color{textcolor}\sffamily\fontsize{11.000000}{13.200000}\selectfont \(\displaystyle 0.20\)}%
\end{pgfscope}%
\begin{pgfscope}%
\pgfpathrectangle{\pgfqpoint{0.574769in}{0.557870in}}{\pgfqpoint{5.665231in}{1.682130in}}%
\pgfusepath{clip}%
\pgfsetroundcap%
\pgfsetroundjoin%
\pgfsetlinewidth{1.003750pt}%
\definecolor{currentstroke}{rgb}{1.000000,1.000000,1.000000}%
\pgfsetstrokecolor{currentstroke}%
\pgfsetdash{}{0pt}%
\pgfpathmoveto{\pgfqpoint{5.741199in}{0.557870in}}%
\pgfpathlineto{\pgfqpoint{5.741199in}{2.240000in}}%
\pgfusepath{stroke}%
\end{pgfscope}%
\begin{pgfscope}%
\definecolor{textcolor}{rgb}{0.150000,0.150000,0.150000}%
\pgfsetstrokecolor{textcolor}%
\pgfsetfillcolor{textcolor}%
\pgftext[x=5.741199in,y=0.425926in,,top]{\color{textcolor}\sffamily\fontsize{11.000000}{13.200000}\selectfont \(\displaystyle 0.25\)}%
\end{pgfscope}%
\begin{pgfscope}%
\definecolor{textcolor}{rgb}{0.150000,0.150000,0.150000}%
\pgfsetstrokecolor{textcolor}%
\pgfsetfillcolor{textcolor}%
\pgftext[x=3.407384in,y=0.235185in,,top]{\color{textcolor}\sffamily\fontsize{11.000000}{13.200000}\selectfont ARI}%
\end{pgfscope}%
\begin{pgfscope}%
\pgfpathrectangle{\pgfqpoint{0.574769in}{0.557870in}}{\pgfqpoint{5.665231in}{1.682130in}}%
\pgfusepath{clip}%
\pgfsetroundcap%
\pgfsetroundjoin%
\pgfsetlinewidth{1.003750pt}%
\definecolor{currentstroke}{rgb}{1.000000,1.000000,1.000000}%
\pgfsetstrokecolor{currentstroke}%
\pgfsetdash{}{0pt}%
\pgfpathmoveto{\pgfqpoint{0.574769in}{0.557870in}}%
\pgfpathlineto{\pgfqpoint{6.240000in}{0.557870in}}%
\pgfusepath{stroke}%
\end{pgfscope}%
\begin{pgfscope}%
\definecolor{textcolor}{rgb}{0.150000,0.150000,0.150000}%
\pgfsetstrokecolor{textcolor}%
\pgfsetfillcolor{textcolor}%
\pgftext[x=0.366783in,y=0.505064in,left,base]{\color{textcolor}\sffamily\fontsize{11.000000}{13.200000}\selectfont \(\displaystyle 0\)}%
\end{pgfscope}%
\begin{pgfscope}%
\pgfpathrectangle{\pgfqpoint{0.574769in}{0.557870in}}{\pgfqpoint{5.665231in}{1.682130in}}%
\pgfusepath{clip}%
\pgfsetroundcap%
\pgfsetroundjoin%
\pgfsetlinewidth{1.003750pt}%
\definecolor{currentstroke}{rgb}{1.000000,1.000000,1.000000}%
\pgfsetstrokecolor{currentstroke}%
\pgfsetdash{}{0pt}%
\pgfpathmoveto{\pgfqpoint{0.574769in}{0.906137in}}%
\pgfpathlineto{\pgfqpoint{6.240000in}{0.906137in}}%
\pgfusepath{stroke}%
\end{pgfscope}%
\begin{pgfscope}%
\definecolor{textcolor}{rgb}{0.150000,0.150000,0.150000}%
\pgfsetstrokecolor{textcolor}%
\pgfsetfillcolor{textcolor}%
\pgftext[x=0.290741in,y=0.853331in,left,base]{\color{textcolor}\sffamily\fontsize{11.000000}{13.200000}\selectfont \(\displaystyle 20\)}%
\end{pgfscope}%
\begin{pgfscope}%
\pgfpathrectangle{\pgfqpoint{0.574769in}{0.557870in}}{\pgfqpoint{5.665231in}{1.682130in}}%
\pgfusepath{clip}%
\pgfsetroundcap%
\pgfsetroundjoin%
\pgfsetlinewidth{1.003750pt}%
\definecolor{currentstroke}{rgb}{1.000000,1.000000,1.000000}%
\pgfsetstrokecolor{currentstroke}%
\pgfsetdash{}{0pt}%
\pgfpathmoveto{\pgfqpoint{0.574769in}{1.254404in}}%
\pgfpathlineto{\pgfqpoint{6.240000in}{1.254404in}}%
\pgfusepath{stroke}%
\end{pgfscope}%
\begin{pgfscope}%
\definecolor{textcolor}{rgb}{0.150000,0.150000,0.150000}%
\pgfsetstrokecolor{textcolor}%
\pgfsetfillcolor{textcolor}%
\pgftext[x=0.290741in,y=1.201598in,left,base]{\color{textcolor}\sffamily\fontsize{11.000000}{13.200000}\selectfont \(\displaystyle 40\)}%
\end{pgfscope}%
\begin{pgfscope}%
\pgfpathrectangle{\pgfqpoint{0.574769in}{0.557870in}}{\pgfqpoint{5.665231in}{1.682130in}}%
\pgfusepath{clip}%
\pgfsetroundcap%
\pgfsetroundjoin%
\pgfsetlinewidth{1.003750pt}%
\definecolor{currentstroke}{rgb}{1.000000,1.000000,1.000000}%
\pgfsetstrokecolor{currentstroke}%
\pgfsetdash{}{0pt}%
\pgfpathmoveto{\pgfqpoint{0.574769in}{1.602671in}}%
\pgfpathlineto{\pgfqpoint{6.240000in}{1.602671in}}%
\pgfusepath{stroke}%
\end{pgfscope}%
\begin{pgfscope}%
\definecolor{textcolor}{rgb}{0.150000,0.150000,0.150000}%
\pgfsetstrokecolor{textcolor}%
\pgfsetfillcolor{textcolor}%
\pgftext[x=0.290741in,y=1.549865in,left,base]{\color{textcolor}\sffamily\fontsize{11.000000}{13.200000}\selectfont \(\displaystyle 60\)}%
\end{pgfscope}%
\begin{pgfscope}%
\pgfpathrectangle{\pgfqpoint{0.574769in}{0.557870in}}{\pgfqpoint{5.665231in}{1.682130in}}%
\pgfusepath{clip}%
\pgfsetroundcap%
\pgfsetroundjoin%
\pgfsetlinewidth{1.003750pt}%
\definecolor{currentstroke}{rgb}{1.000000,1.000000,1.000000}%
\pgfsetstrokecolor{currentstroke}%
\pgfsetdash{}{0pt}%
\pgfpathmoveto{\pgfqpoint{0.574769in}{1.950938in}}%
\pgfpathlineto{\pgfqpoint{6.240000in}{1.950938in}}%
\pgfusepath{stroke}%
\end{pgfscope}%
\begin{pgfscope}%
\definecolor{textcolor}{rgb}{0.150000,0.150000,0.150000}%
\pgfsetstrokecolor{textcolor}%
\pgfsetfillcolor{textcolor}%
\pgftext[x=0.290741in,y=1.898132in,left,base]{\color{textcolor}\sffamily\fontsize{11.000000}{13.200000}\selectfont \(\displaystyle 80\)}%
\end{pgfscope}%
\begin{pgfscope}%
\definecolor{textcolor}{rgb}{0.150000,0.150000,0.150000}%
\pgfsetstrokecolor{textcolor}%
\pgfsetfillcolor{textcolor}%
\pgftext[x=0.235185in,y=1.398935in,,bottom,rotate=90.000000]{\color{textcolor}\sffamily\fontsize{11.000000}{13.200000}\selectfont Occurance}%
\end{pgfscope}%
\begin{pgfscope}%
\pgfpathrectangle{\pgfqpoint{0.574769in}{0.557870in}}{\pgfqpoint{5.665231in}{1.682130in}}%
\pgfusepath{clip}%
\pgfsetbuttcap%
\pgfsetmiterjoin%
\definecolor{currentfill}{rgb}{0.298039,0.447059,0.690196}%
\pgfsetfillcolor{currentfill}%
\pgfsetfillopacity{0.400000}%
\pgfsetlinewidth{1.003750pt}%
\definecolor{currentstroke}{rgb}{1.000000,1.000000,1.000000}%
\pgfsetstrokecolor{currentstroke}%
\pgfsetstrokeopacity{0.400000}%
\pgfsetdash{}{0pt}%
\pgfpathmoveto{\pgfqpoint{0.832279in}{0.557870in}}%
\pgfpathlineto{\pgfqpoint{1.690648in}{0.557870in}}%
\pgfpathlineto{\pgfqpoint{1.690648in}{2.159899in}}%
\pgfpathlineto{\pgfqpoint{0.832279in}{2.159899in}}%
\pgfpathclose%
\pgfusepath{stroke,fill}%
\end{pgfscope}%
\begin{pgfscope}%
\pgfpathrectangle{\pgfqpoint{0.574769in}{0.557870in}}{\pgfqpoint{5.665231in}{1.682130in}}%
\pgfusepath{clip}%
\pgfsetbuttcap%
\pgfsetmiterjoin%
\definecolor{currentfill}{rgb}{0.298039,0.447059,0.690196}%
\pgfsetfillcolor{currentfill}%
\pgfsetfillopacity{0.400000}%
\pgfsetlinewidth{1.003750pt}%
\definecolor{currentstroke}{rgb}{1.000000,1.000000,1.000000}%
\pgfsetstrokecolor{currentstroke}%
\pgfsetstrokeopacity{0.400000}%
\pgfsetdash{}{0pt}%
\pgfpathmoveto{\pgfqpoint{1.690648in}{0.557870in}}%
\pgfpathlineto{\pgfqpoint{2.549016in}{0.557870in}}%
\pgfpathlineto{\pgfqpoint{2.549016in}{0.557870in}}%
\pgfpathlineto{\pgfqpoint{1.690648in}{0.557870in}}%
\pgfpathclose%
\pgfusepath{stroke,fill}%
\end{pgfscope}%
\begin{pgfscope}%
\pgfpathrectangle{\pgfqpoint{0.574769in}{0.557870in}}{\pgfqpoint{5.665231in}{1.682130in}}%
\pgfusepath{clip}%
\pgfsetbuttcap%
\pgfsetmiterjoin%
\definecolor{currentfill}{rgb}{0.298039,0.447059,0.690196}%
\pgfsetfillcolor{currentfill}%
\pgfsetfillopacity{0.400000}%
\pgfsetlinewidth{1.003750pt}%
\definecolor{currentstroke}{rgb}{1.000000,1.000000,1.000000}%
\pgfsetstrokecolor{currentstroke}%
\pgfsetstrokeopacity{0.400000}%
\pgfsetdash{}{0pt}%
\pgfpathmoveto{\pgfqpoint{2.549016in}{0.557870in}}%
\pgfpathlineto{\pgfqpoint{3.407384in}{0.557870in}}%
\pgfpathlineto{\pgfqpoint{3.407384in}{1.080271in}}%
\pgfpathlineto{\pgfqpoint{2.549016in}{1.080271in}}%
\pgfpathclose%
\pgfusepath{stroke,fill}%
\end{pgfscope}%
\begin{pgfscope}%
\pgfpathrectangle{\pgfqpoint{0.574769in}{0.557870in}}{\pgfqpoint{5.665231in}{1.682130in}}%
\pgfusepath{clip}%
\pgfsetbuttcap%
\pgfsetmiterjoin%
\definecolor{currentfill}{rgb}{0.298039,0.447059,0.690196}%
\pgfsetfillcolor{currentfill}%
\pgfsetfillopacity{0.400000}%
\pgfsetlinewidth{1.003750pt}%
\definecolor{currentstroke}{rgb}{1.000000,1.000000,1.000000}%
\pgfsetstrokecolor{currentstroke}%
\pgfsetstrokeopacity{0.400000}%
\pgfsetdash{}{0pt}%
\pgfpathmoveto{\pgfqpoint{3.407384in}{0.557870in}}%
\pgfpathlineto{\pgfqpoint{4.265753in}{0.557870in}}%
\pgfpathlineto{\pgfqpoint{4.265753in}{0.662350in}}%
\pgfpathlineto{\pgfqpoint{3.407384in}{0.662350in}}%
\pgfpathclose%
\pgfusepath{stroke,fill}%
\end{pgfscope}%
\begin{pgfscope}%
\pgfpathrectangle{\pgfqpoint{0.574769in}{0.557870in}}{\pgfqpoint{5.665231in}{1.682130in}}%
\pgfusepath{clip}%
\pgfsetbuttcap%
\pgfsetmiterjoin%
\definecolor{currentfill}{rgb}{0.298039,0.447059,0.690196}%
\pgfsetfillcolor{currentfill}%
\pgfsetfillopacity{0.400000}%
\pgfsetlinewidth{1.003750pt}%
\definecolor{currentstroke}{rgb}{1.000000,1.000000,1.000000}%
\pgfsetstrokecolor{currentstroke}%
\pgfsetstrokeopacity{0.400000}%
\pgfsetdash{}{0pt}%
\pgfpathmoveto{\pgfqpoint{4.265753in}{0.557870in}}%
\pgfpathlineto{\pgfqpoint{5.124121in}{0.557870in}}%
\pgfpathlineto{\pgfqpoint{5.124121in}{0.662350in}}%
\pgfpathlineto{\pgfqpoint{4.265753in}{0.662350in}}%
\pgfpathclose%
\pgfusepath{stroke,fill}%
\end{pgfscope}%
\begin{pgfscope}%
\pgfpathrectangle{\pgfqpoint{0.574769in}{0.557870in}}{\pgfqpoint{5.665231in}{1.682130in}}%
\pgfusepath{clip}%
\pgfsetbuttcap%
\pgfsetmiterjoin%
\definecolor{currentfill}{rgb}{0.298039,0.447059,0.690196}%
\pgfsetfillcolor{currentfill}%
\pgfsetfillopacity{0.400000}%
\pgfsetlinewidth{1.003750pt}%
\definecolor{currentstroke}{rgb}{1.000000,1.000000,1.000000}%
\pgfsetstrokecolor{currentstroke}%
\pgfsetstrokeopacity{0.400000}%
\pgfsetdash{}{0pt}%
\pgfpathmoveto{\pgfqpoint{5.124121in}{0.557870in}}%
\pgfpathlineto{\pgfqpoint{5.982489in}{0.557870in}}%
\pgfpathlineto{\pgfqpoint{5.982489in}{1.010617in}}%
\pgfpathlineto{\pgfqpoint{5.124121in}{1.010617in}}%
\pgfpathclose%
\pgfusepath{stroke,fill}%
\end{pgfscope}%
\begin{pgfscope}%
\pgfsetrectcap%
\pgfsetmiterjoin%
\pgfsetlinewidth{1.254687pt}%
\definecolor{currentstroke}{rgb}{1.000000,1.000000,1.000000}%
\pgfsetstrokecolor{currentstroke}%
\pgfsetdash{}{0pt}%
\pgfpathmoveto{\pgfqpoint{0.574769in}{0.557870in}}%
\pgfpathlineto{\pgfqpoint{0.574769in}{2.240000in}}%
\pgfusepath{stroke}%
\end{pgfscope}%
\begin{pgfscope}%
\pgfsetrectcap%
\pgfsetmiterjoin%
\pgfsetlinewidth{1.254687pt}%
\definecolor{currentstroke}{rgb}{1.000000,1.000000,1.000000}%
\pgfsetstrokecolor{currentstroke}%
\pgfsetdash{}{0pt}%
\pgfpathmoveto{\pgfqpoint{6.240000in}{0.557870in}}%
\pgfpathlineto{\pgfqpoint{6.240000in}{2.240000in}}%
\pgfusepath{stroke}%
\end{pgfscope}%
\begin{pgfscope}%
\pgfsetrectcap%
\pgfsetmiterjoin%
\pgfsetlinewidth{1.254687pt}%
\definecolor{currentstroke}{rgb}{1.000000,1.000000,1.000000}%
\pgfsetstrokecolor{currentstroke}%
\pgfsetdash{}{0pt}%
\pgfpathmoveto{\pgfqpoint{0.574769in}{0.557870in}}%
\pgfpathlineto{\pgfqpoint{6.240000in}{0.557870in}}%
\pgfusepath{stroke}%
\end{pgfscope}%
\begin{pgfscope}%
\pgfsetrectcap%
\pgfsetmiterjoin%
\pgfsetlinewidth{1.254687pt}%
\definecolor{currentstroke}{rgb}{1.000000,1.000000,1.000000}%
\pgfsetstrokecolor{currentstroke}%
\pgfsetdash{}{0pt}%
\pgfpathmoveto{\pgfqpoint{0.574769in}{2.240000in}}%
\pgfpathlineto{\pgfqpoint{6.240000in}{2.240000in}}%
\pgfusepath{stroke}%
\end{pgfscope}%
\end{pgfpicture}%
\makeatother%
\endgroup%

    \caption{ARI distribution of TSC methods when classifying classifying left ventrice segment indication.}
    \label{fig:tsc_segm_ind_ari}
\end{figure}

\newpage

\subsection{Deep Neural Network}

\begin{table}[htb]
    \centering
    \begin{tabular}{lrrrr}
        \toprule
        {}          &  Accuracy &  Sensitivity &  Specificity &  DOR \\
        Method      &           &              &              &      \\
        \midrule
        regular     &      0.74 &         0.80 &         0.68 & 8.65 \\
        downsampled &      0.74 &         0.74 &         0.75 & 8.38 \\
        upsampled   &      0.65 &         0.55 &         0.73 & 3.36 \\
        \bottomrule
    \end{tabular}
    \caption{Evaluation metrics of the NN for classifying the binary indication of individual segments in the left ventricle.}
    \label{tab:NN_segm_ind_perf}
\end{table}

\subsection{Comparisons}

