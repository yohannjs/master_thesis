\chapter{Review of The Literature} \label{chap:lit}

\textcite{hf_diagnosis_ml} attempt to diagnose heart failure with preserved \acrshort{ef} using a combination of a statistical unsupervised method, and a supervised classifier. Their dataset consists of the velocity, strain and strain-rate curves of the 18 regional segments of the left ventrice, and it consists of a 100 subjects. The patients were stress tested, meaning that the curves were extracted at rest, and after they had been exposed to a period of physical activity. This resulted in 36 curves of each parameter (18 at rest, and 18 during exercise). The patients were split into four groups: patients experiencing heat failure with preserved \acrshort{ef}, hypertensive patients, healthy control subjects and breathless control subjects. The machine learning algorithm that they developed attempted to predict which group each patient was affiliated to. The machine learning algorithm was composed of a unsupervised method called principle component analysis, and some variations of a supervised classifier called \acrfull{knn}. Principle component analysis was used to reduce the dimensionality, such that there were fewer dimensions for the \acrshort{knn} classifiers to consider. 
