\chapter{Review of The Literature} \label{chap:lit}

\textcite{hf_diagnosis_ml} attempt to diagnose heart failure with preserved \acrshort{ef} using a combination of a statistical unsupervised method, and a supervised classifier. Their dataset consists of the velocity, strain and strain-rate curves of the 18 regional segments of the left ventrice, and it consists of a 100 subjects. The patients were stress tested, meaning that the curves were extracted at rest, and after they had been exposed to a period of physical activity. This resulted in 36 curves of each parameter (18 at rest, and 18 during exercise). The patients were split into four groups: patients experiencing heat failure with preserved \acrshort{ef}, hypertensive patients, healthy control patients and breathless control patients. They also performed another partitioning of the patients where they combined the hypertensive patients and the healthy patients, and combined the breathless patients and the patients with heart failure with preserved \acrshort{ef}. The machine learning algorithm that they developed was trained at predicting two different target variables, group affiliation with four classes, and group affiliation with two classes. The machine learning algorithm was composed of a unsupervised method called principle component analysis, and some variations of a supervised classifier called \acrfull{knn}. Principle component analysis was used to reduce the dimensionality, such that there were fewer dimensions for the \acrshort{knn} classifiers to consider. In the four class classification problem their model attained an overall accuracy of 0.57, but attained an accuracy of 0.81 within the class of heart failure patients with preserved \acrshort{ef}. In the two class classification problem their model attained an accuracy of 0.85, a sensitivity of 0.86 and a specificity of 0.82. \bigskip


