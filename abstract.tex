\chapter*{Abstract}
\addcontentsline{toc}{chapter}{Abstract}%

The use of \acrlong{ef} in diagnosing of heart failure is well established in clinical cardiology. In the past few years clinicians have started using myocardial strain for diagnosing more often as well. The digitization of hospital databases, and collection of large amounts of echocardiographic data have opened up the possibility for application of machine learning algorithms to automate labor intensive tasks for clinicians such as data annotation and to assist clinicians with the diagnostic process. This work attempts to contribute to the latter. \bigskip

This thesis uses a dataset of 199 patients, that is part of the IMPROVE study which is an ongoing cardiology study. In the dataset there were 39 patients with \acrlong{nstemi}, 60 with \acrlong{stemi}, 70 with other heart diseases, and 30 control subjects. The dataset is also labelled by heart failure, and there were 100 patients with heart failure, and 99 patients without. For each patient there were given 3 \acrlong{gls} curves, and 18 \acrlong{rls} from the \acrlong{4ch}, \acrlong{2ch} and \acrlong{aplax} views yielded with transthoracic echocardiography. Each left ventrice segment was also given a label according to the wall motion score indicating the degree of dysfunction of each segment. \bigskip

