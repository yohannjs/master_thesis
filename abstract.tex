\section*{Abstract}
\addcontentsline{toc}{section}{Abstract}%

The use of left ventricle \acrfull{ef} in diagnosing heart failure is well established in clinical cardiology. In the past few years clinicians have started using myocardial strain for diagnosing more often as well. The digitization of hospital databases, and collection of large amounts of echocardiographic data have opened up the possibility for application of machine learning algorithms to automate labor intensive tasks for clinicians such as data annotation and to assist clinicians with the diagnostic process. This work attempts to contribute to the latter. \bigskip

This work has used a dataset of 199 patients, a part of the IMPROVE study which is an ongoing cardiology study. In the dataset there were 60 patients with \acrlong{stemi}, 39 with \acrlong{nstemi}, 70 with other heart diseases, and 30 control patients. The dataset is also labelled by heart failure, and there were 100 patients with heart failure and 99 patients without. For each patient there were given three \acrlong{gls} curves, and 18 \acrlong{rls} curves from the \acrlong{4ch}, \acrlong{2ch} and \acrlong{aplax} views yielded with transthoracic echocardiography. Each left ventricle segment was also given a label according to the wall motion score, indicating the degree of dysfunction of each segment. \bigskip

Three binary target variables are considered: Heart failure (Yes/No), patient diagnosis (Healthy/Unhealthy) and regional myocard segment indication (Normal/Abnormal). The main part of the work has been towards testing if \acrfull{tsc} and \acrfull{ann} could be applied to predict the three target variables when applied on longitudinal strain curves. To benchmark the \acrshort{tsc} model, regular clustering of point values was performed on peak systolic strain of the longitudinal strain curves in combination with \acrshort{ef}. To benchmark the \acrfull{ann} eleven different supervised classifiers were trained on peak values of longitudinal strain curves in combination with \acrshort{ef}. The models were evaluated with accuracy, sensitivity, specificity and \acrfull{dor}. \bigskip

It was clustering model applied to peak systolic global longitudinal strain in combination with \acrshort{ef} that performed best at predicting heart failure among patients. The model attained an accuracy of 0.76, sensitivity of 0.81, specificity of 0.72 and \acrshort{dor} of 10.85. However, it was found that all the models were outperformed by a simple \acrshort{ef} threshold classifier set at $45\%$, which attained an accuracy of 0.77, sensitivity of 0.86, specificity of 0.69 and \acrshort{dor} of 13.48. The model that performed best at predicting patient diagnosis was the \acrlong{knn} classifier trained on a combination of peak systolic global, and regional longitudinal strain values. It attained an accuracy of 0.93, a sensitivity of 0.95, a specificity of 0.82 and a \acrshort{dor} of 84.53. The model that performed best at predicting the indication of regional myocard segments was the \acrshort{ann}. It attained an accuracy of 0.74, sensitivity of 0.74, specificity of 0.75 and \acrshort{dor} of 8.38. \bigskip

It was found that future work to be done on this topic could include dimensionality reduction of the multiple strain curves used to represent the patients for the time-series clustering model. The architecture of the \acrshort{ann} was found to be too complex for the dataset at hand, so improvement could be gained by reducing the complexity of the architecture. The supervised classifiers were applied with fairly standard hyperparameters as they were meant to serve as a benchmark for the \acrshort{ann}, so further work could be put into optimizing the hyperparameters of the classifiers for the dataset at hand.

\clearpage
