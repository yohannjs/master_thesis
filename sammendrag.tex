\section*{Sammendrag}
\addcontentsline{toc}{section}{Sammendrag}%

Venstre-ventrikkels ejeksjonsfraksjon (EF) har lenge blitt brukt som en indikator på hjertetilstand av pasienter i klinisk kardiologi. De siste årene har bruken av myokardiell tøyning til diagnostikk også blitt mer utbredt. Digitaliseringen av sykehus sine databaser, og innsamling av store mengder ekkokardiografisk data har åpnet opp for muligheten for å anvende maskinlæringsalgoritmer for å automatisere tidkrevende arbeidsoppgaver som datamerking, og bruken av maskinlæringsalgoritmer for å sette diagnoser. Denne oppgaven forsøker å bidra til den sistnevnte anvendelsen. \bigskip

Dette arbeidet bruker et dataset som består av 199 pasienter, som er del av IMPROVE studiet som er et pågående kardiologisk studie. I datasettet er det 60 pasienter med ST-elevasjonsinfarkt, 39 pasienter med non-ST-elevasjonsinfarkt, 70 pasienter med andre hjerte-og karsykdommer og 30 friske kontrollpasienter. Datasettet er også delt i forhold til hvilke pasienter som har hjertesvikt, og det er 100 pasienter med hjertesvikt, og 99 pasienter uten hjertesvikt. For hver pasient har datasettet inneholdt tre globale longitudinale tøyningskurver, og 18 regionale longitudinale tøyningskurver. Disse kurvene er hentet fra de tre ultralydsnittene, 4-kammer snittet, 2-kammer snittet og det apikale-langaksesnittet, som er tilgjengelig ved  transthorakal ekkokardiografi. Hvert venstreventrikkels segment ble også gitt en ''Wall motion score'' som ga et inntrykk av graden av funksjonssvikt i segmentet. \bigskip

Det er tre binære målvariabler som vurderes i dette arbeidet: Hjertesvikt (Ja/Nei), Pasienthelse (Frisk/Syk), og tilstand til venstreventrikkelssegmenter (Normal/Unormal). Hoveddelen av arbeidet ble gjort for teste om tidsrekkeklynging og kunstige nevrale nettverk kan brukes for å forutse de tre målvariablene ved anvendelse på longitudinale tøyningskurver. For å danne et sammenligningsgrunnlag for tidsrekkeklyngemodellen ble klynging av punktverdier gjennomført på punkter ekstrahert fra de longitudinale tøyningskurvene under systolen i kombinasjon med EF. For å danne et sammenligningsgrunnlag for det kunstige nevrale nettverket ble det anvendt elleve forskjellige veiledede klassifiseringsalgoritmer på punktverdier ekstrahert fra de longitudinale tøyningskurvene i kombinasjon med EF. Modellene ble evaluert på deres nøyaktighet, sensitivitet, spesifisitet og med en indeks ved navn ''Diagnostic Odds Ratio'' (DOR). \bigskip

Klyngemodellen anvendt på punktverdier av tøyningskurver og EF var modellen som gjorde det best på å forutse hjertesvikt blant pasienter. Modellen opnådde en nøyaktighet på 0.76, en sensitivitet på 0.81, en spesifisitet på 0.72 og en DOR på 10.85. Det skal merkes at alle modellene ble utklassert av en enkel terskel-vurderingsalgoritme som forutså at alle pasienter med en EF under $45\%$ hadde hjertesvikt. Terskelvurderingsalgoritmen opnådde en nøyaktighet på 0.77, en sensitivitet på 0.86, en spesifisitet på 0.69 og en DOR på 13.48. Modellen som gjorde det best på å forutse pasienthelse var en veiledet klassifiseringsalgoritme som heter "K Nearest Neighbors". Den brukte en kombinasjon av punktverdier fra globale og regionale longitudinale tøyningskurver, og oppnådde en på nøyaktighet på 0.93, en sensitivitet på 0.95, en spesifisitet på 0.82 og en DOR på 84.53. Det kunstige nevrale nettverket var modellen som gjorde det best på å forutse tilstanden til venstreventrikkelssegmenter. Den opnådde en nøyaktighet på 0.74, en sensitivitet på 0.74, en spesifisitet på 0.75 og en DOR på 8.38. \bigskip

Det konkluderes med at fremtidig arbeid gjort på dette temaet kan se på metoder for å redusere antall kurver brukt for å representere hver enkelt pasient, spesielt for tidsrekkeklyngemodellen. Arkitekturen til det kunstige nevrale nettverket viste seg å være for komplekst for dette datasettet, så fremtidig arbeid kan også gå på å redusere kompleksiteten til arkitekturen eller å øke antall objekter i datasettet. De veiledede klassifiseringsalgoritmene ble brukt med ganske standardiserte hyperparametre, siden de i utgangspunktet kun var ment som et sammenligningsgrunnlag for det kunstige nevrale nettverket. Så videre arbeid kan også bli gjort på å tilpasse disse algoritmene mer til problemet, og datasettet for hånd.

\clearpage