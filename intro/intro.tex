\chapter{Introduction} \label{chap:intro}
\begin{comment}
    * What are machine learning models, and where are they applied?
    * What is echocardiography, and why is it useful?
    * Mention that it is commonplace among clinicians to extract strain curves from ultrasound videos.
    * Transition into the fact that one often uses peak strain values, and EF to diagnose heart failure, and other heart diseases.
    * Mention that the input data used in this thesis are \textbf{longitudinal strain curves, peak strain values, and EF}.
    * Mention that the target variables of this thesis are the binary variables: \textbf{Heart failure, patient diagnosis, and segment indication}.
\end{comment}

\section{Motivation}
\begin{comment}
    * Mention examples of where machine learning models have performed well
        * Computor vision.
        * Autonomous vehicles.
        * NLP
        * Speech recognition: Siri (Apple), Alexa (Amazon)
    * End with speech recognition because they are made up of time-series data.
    * Transition over to the statement that given that machine learning has performed so well in these other fields it warrants investigation to see if machine learning could assist with the diagnosis of patients potentially assisting clinicians.
    * Hence the objective of this thesis is to explore whether combination of supervised/unsupervised models and continous/peak-values could perform well at predicting the three different target variables
    * Illustrate this with the table you have been dreaming about.
\end{comment}

This will be the section on the motivation for the assignment.

\section{Objective} \label{sec:objective}

The objectives of this work can be summarized in the form of three questions.

\begin{tcolorbox}
    \textbf{Objectives}

    \begin{enumerate}
        \item Can an ML model be used to identify the following three binary target variables, using longitudinal strain from the left ventricle of the heart? Does the patient have heart failure, does the patient have a heart disease, are the individual segments of the patients left ventrice acting abnormally.
        \item Which type of machine learning is best suited for predicting the aforementioned target variables, supervised, or unsupervised learning models?
        \item Which type of dataset works best for a ML model to predict the target variables, a dataset consisting of longitudinal strain curves, or a dataset that consists of peak systolic longitudinal strain values in combination with EF?
    \end{enumerate}
\end{tcolorbox}

\section{Structure of Thesis}

Here the outline for the rest of the assignment will be given. \smallskip
