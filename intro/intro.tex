\chapter{Introduction} \label{chap:intro}
\begin{comment}
    * What are machine learning models, and where are they applied?
    * What is echocardiography, and why is it useful?
    * Mention that it is commonplace among clinicians to extract strain curves from ultrasound videos.
    * Transition into the fact that one often uses peak strain values, and EF to diagnose heart failure, and other heart diseases.
    * Mention that the input data used in this thesis are \textbf{longitudinal strain curves, peak strain values, and EF}.
    * Mention that the target variables of this thesis are the binary variables: \textbf{Heart failure, patient diagnosis, and segment indication}.
\end{comment}

Machine learning is a subcategory of artificial intelligence. Machine learning models differ from other types of artificial intelligence by the fact that they are not given a set of explicit rules on how the input data is related to the target variables. Instead they are given an objective, which is often to predict the target variable, with as little error as possible. The machine learning models then use the objective, and large amounts of data, ''learn'' how to best fulfill the objective. Machine learning is heavily applied in the fields of computor vision, speech recognition and natural language processing. Machine learning models can be divided into \textit{supervised learning}, \textit{unsupervised learning} and \textit{semi-supervised learning}. Machine learning models that fall under the category of supervised learning need a dataset that is labelled, meaning that it needs to know what answer is correct. Unsupervised machine learning models do not require a labelled dataset. Semi-supervised machine learning models use a combination of supervised and unsupervised learning. \bigskip

Echocardiography is a diagnostic tool applied in cardiology to assess the cardiovascular state of a patient. It uses ultrasound imaging to create two or three dimensional images of a patients heart which can be put together into videos and viewed in real-time. Since the ultrasound videos contain a lot of information, it is common to extract more information-dense curves and parameters from the videos. Specifically, parameters such as \acrfull{ef} is extracted to assess whether a patient is experiencing heart failure, and longitudinal strain curves of specific heart segments are extracted to assess the state of individual segments. Strain curves can also be further concentrated by only assessing their peak and trough values. In this work \acrshort{ef}, longitudinal strain curves and peak longitudinal strain values are used as input variables. Three binary variables are considered as target variables: Heart failure (Yes/No), patient diagnosis (Healthy/Unhealthy), and segment indication (Normal/Abnormal).

\section{Motivation} \label{sec:motivation}

Machine learning models have been successfully applied in computer vision contests such as the annual challanges hosted by ImageNet, where in 2015 contestants trained their models to differentiate between 20000 image classes, and used a dataset of 15 million images. Contestants scored if the correct label was among the top five predictions that the model outputed, and the best score attained was a classification error rate of $16.4\%$\footnote{http://image-net.org/challenges/LSVRC/2015/results}. Companies such as Tesla, and Google have also stated that they apply machine learning models in the computor vision of their autonomous cars, without going into the specifics of how well they perform. In speech recognition, it is also machine learning models that perform best at recognizing individual phonemes in recorded speech. The digitization of hospital databases, and collection of large amounts of echocardiographic data have opened up the possibility for application of machine learning algorithms to automate labor intensive tasks for clinicians such as data annotation and to assist clinicians with the diagnostic process. Machine learning models may even contribute to the discovery of new clinical parameters that can better predict the condition of patients with a heart condition.

\section{Objective} \label{sec:objective}

The main part of the work has been towards testing whether \acrfull{tsc} and \acrfull{ann} could be applied to predict the three target variables when applied on longitudinal strain curves. To benchmark the \acrshort{tsc} model, regular clustering of point values or \acrfull{pvc} was performed on peak values of the longitudinl strain curves in combination with \acrshort{ef}. To benchmark the \acrfull{ann} eleven different supervised classifiers were trained on peak values of longitudinal strain curves in combination with \acrshort{ef}. Since this work will test both supervised and unsupervised machine learning models, and strain curve and peak-strain datasets, one can say that the work is exploring the two-by-two grid of combinations illustrated in figure \ref{fig:objectives_diagram}.

\begin{figure}[H]
    \centering
    \includegraphics[width=0.5\textwidth]{intro/objectives_diagram.png}
    \caption{This is an illustration of the combinations of strain datasets and machine learning algorithms which will be tested in this work.}
    \label{fig:objectives_diagram}
\end{figure}

The objectives of this work can be summarized in the form of three questions:

\begin{tcolorbox}
    \textbf{Objectives}

    \begin{enumerate}
        \item Can a machine learning model be used to predict one of the three target variables assessed in this work using peak strain values or longitudinal strain curves?
        \item Which type of machine learning is best suited for predicting the aforementioned target variables, supervised or unsupervised learning models?
        \item Which type of input data works best for a machine learning model to predict the target variables, a dataset consisting of longitudinal strain curves or a dataset that consists of peak strain values in combination with \acrshort{ef}?
    \end{enumerate}
\end{tcolorbox}

\section{Structure}

The structure of this work is as follows: Chapter \ref{chap:strain} will explain the theory behind echocardiography, the technology used in ultrasound imaging, and outline the different heart diseases presented. Chapter \ref{chap:ml} describes the theory behind the machine learning models used. Chapter \ref{chap:lit} reviews the most recent work done on the topic. Chapter \ref{chap:data} explores the dataset. Chapter \ref{chap:method} details how every model in this work is configured, trained and evaluated. Chapter \ref{chap:results} presents the results of the individual models tested. A discussion of the results will be made in chapter \ref{chap:discussion} and a conclusion is given in chapter \ref{chap:conclusion}.