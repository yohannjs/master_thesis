\chapter{Myocardial Imaging and Echocardiography} \label{chap:strain}
\begin{comment}
[ ] Short about cardiology and echocardiography history
[ ] Short about the heart, and its anatomy
[ ] Even shorter about ultrasound, the different views, and what parts of the heart can be seen in them.
[ ] Method for extracting strain curves from ultrasound videos.
[ ] Explain the different diagnosises that will be encountered in this thesis.
[ ] Explain anatomical reasoning for why symptoms for certain diagnosis are evident in strain curves.
[ ] Summarize chapter
\end{comment}

This will be a kind of theory section about echocardiography, and strain imaging. \bigskip

\section{Basic Cardiology}
\begin{comment}
[ ] Short about cardiology and echocardiography history
[ ] Short about the heart, and its anatomy
\end{comment}
The heart is an autonomous muscle that is responsible for pumping oxygenated blood from the lungs, into the rest of the body and pumping unoxegenated blood from the rest of the body, into the lungs. The heart can be divided into four separate chambers: The right atrium, the left atrium, the right ventricle and the left ventricle. The atriums are responsible for pumping unoxegenated blood into the lungs, and the ventricles are responsible for pumping oxygenated blood into the body. One heart cycle is the time period it takes the heart muscles to make a full contraction and relaxation. The period of the heart cycle where the heart relaxes, and fills with blood is called the \textit{diastole}, and the period of the heart cycle when the heart contracts and pumps blood throughout the body is called the \textit{systole}. Cardiology is the branch of medicine that deals with the heart, and parts of the vascular system \cite{cardiology_wikipedia}. Cardiologists are doctors that specialize in the field of cardiology. Echocardiography is a diagnostic tool used in cardiology to take images of specific muscle tissue in the heart known as \textit{myocardial tissue}, using ultrasound imaging. 

\section{Introduction to Myocardial Imaging}
\begin{comment}
[ ] Even shorter about ultrasound, the different views, and what parts of the heart can be seen in them.
\end{comment}
Ultrasound imaging is a diagnostic tool that is popular because it can give videos in real-time, it is relatively inexpenisve and has a lower associated health-risk compared to its alternatives CITATION. In this section \textit{two dimensional B-mode ultrasound imaging} will be detailed, where the \textit{B} stands for \textit{brightness}. The frequency of the sound waves used in ultrasound imaging are in the range of 1 - 12 MHz, and the frequency chosen for wave pulses will decide the size of the objects that the method is able to resolute CITE BASIC ULTRASOUND FOR CLINICIANS. Ultrasound imaging works by emmitting pulses of ultrasound waves at myocardial tissue, the pulses are partially reflected by the different tissue structures, and are then sampled by a receiver upon return at the sond that transmitted them, as illustrated in figure %\ref{fig:us_reflect}.

% \begin{figure}
    % \centering
    % \includegraphics[width=0.99\textwidth]{echocardiography/US_reflection.png}
    % \caption{An illustration of how ultrsound pulses are partially reflected by many barriers of tissue. The horisontal arrows represent the pulses, where the relative sizes represent the amplitude of the pulse, and the vertical lines represent structures of tissue. Figure is inspired by.}
    % \label{fig:us_reflect}
% \end{figure}

Images created by two dimensional ultrasound imaging are polar 

\section{Myocardial Strain Estimation}
\begin{comment}
[ ] Method for extracting strain curves from ultrasound videos.
\end{comment}

\section{Heart Diseases}
\begin{comment}
[ ] Explain the different diagnosises that will be encountered in this thesis.
[ ] Explain anatomical reasoning for why symptoms for certain diagnosis are evident in strain curves.
\end{comment}

\section{Chapter Summary}
