\chapter{Myocardial Imaging and Echocardiography} \label{chap:strain}
\begin{comment}
[ ] Short about cardiology and echocardiography history
[ ] Short about the heart, and its anatomy
[ ] Even shorter about ultrasound, the different views, and what parts of the heart can be seen in them.
[ ] Method for extracting strain curves from ultrasound videos.
[ ] Explain the different diagnosises that will be encountered in this thesis.
[ ] Explain anatomical reasoning for why symptoms for certain diagnosis are evident in strain curves.
[ ] Summarize chapter
\end{comment}

This will be a kind of theory section about echocardiography, and strain imaging. \bigskip

\section{Basic Cardiology}
\begin{comment}
[ ] Short about cardiology and echocardiography history
[ ] Short about the heart, and its anatomy
\end{comment}
The heart is an autonomous muscle that is responsible for pumping oxygenated blood from the lungs, into the rest of the body and pumping unoxegenated blood from the rest of the body, into the lungs. The heart can be divided into four separate chambers: The right atrium, the left atrium, the right ventricle and the left ventricle. The atriums are responsible for pumping unoxegenated blood into the lungs, and the ventricles are responsible for pumping oxygenated blood into the body. One heart cycle is the time period it takes the heart muscles to make a full contraction and relaxation. The period of the heart cycle where the heart relaxes, and fills with blood is called the \textit{diastole}, and the period of the heart cycle when the heart contracts and pumps blood throughout the body is called the \textit{systole}. Cardiology is the branch of medicine that deals with the heart, and parts of the vascular system \cite{cardiology_wikipedia}. Cardiologists are doctors that specialize in the field of cardiology. 

\section{Introduction to Myocardial Imaging}
\begin{comment}
[ ] Even shorter about ultrasound, the different views, and what parts of the heart can be seen in them.
\end{comment}

\section{Myocardial Strain Estimation}
\begin{comment}
[ ] Method for extracting strain curves from ultrasound videos.
\end{comment}

\section{Heart Diseases}
\begin{comment}
[ ] Explain the different diagnosises that will be encountered in this thesis.
[ ] Explain anatomical reasoning for why symptoms for certain diagnosis are evident in strain curves.
\end{comment}

\section{Chapter Summary}
