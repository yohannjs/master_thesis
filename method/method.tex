\chapter{Method}

This is the section where we detail the specific models used. \bigskip

\section{Description of The Datasets}

Since the different ML models detailed in chapter REFERENCE require different types of input data the, datasets have been divided into two main categories: 
The peak-value datasets and the time-series datasets. \bigskip

\subsection{Time-series Datasets}

\begin{table*}[h]
    \centering
    \ra{1.3}
    \begin{tabular}{ lr }
        \toprule
        Input variables   & Shape \\
        \midrule
        Single RLS curves & (3600, 1) \\
        RLS curves        & (200, 18) \\
        GLS curves        & (200, 3)  \\
        Strain curves     & (200, 21) \\
        \bottomrule
    \end{tabular}
    \caption{Time-series datasets. The ''Shape'' parameter is indicates: (Number of objects in the dataset, Number of curves in each individual object).}
    \label{tab:ts_dsets}
\end{table*}

Table \ref{tab:ts_dsets} shows the different time-series datasets that will be used. 
All the datasets except \textit{Single RLS curves} will be used to predict the diagnosis of patients, and whether the patient has heart failure.
Recall that the different diagnosises are described in section REFERENCE, and there occurance in the data are illustrated in figure \ref{fig:hf_ind_dist}.
\textit{Single RLS curves} will be used to predict the segment indications shown in figure \ref{fig:segm_label_dis} and described in section REFERENCE. 
The point of classifying individual segments of a patients left ventricle is that if a single segment is found to be \textit{not normal}, 
this would also mean that the patient can be considered as \textit{not healthy}.
As mentioned in the description of table \ref{tab:ts_dsets} the ''Shape'' parameter shows how many objects each dataset has, and how many curves are associated to each object. 
Since each ultrasound examination takes ultrasound inspections from three views (four chamber, two chamber, and APLAX chamber), each patient has three views to estimate a GLS curve from. 
Since each GLS curve, also can be divided into six RLS curves, there is a total of 21 strain curves per patient. 
Both the RNN, and the TSC model have been applied on the datasets listed in table \ref{tab:ts_dsets}, 
however there have been made some significant changes to the multivariate time series to accomodate the DL model.
Even though RNNs are capable of handling objects of different sizes, the curves that make up a single object must be of the same size.
Since the strain curves extracted from different views often are of different lengths, the RNN has been limited to only using the strain curves associated with a single view
when evaluating the diagnosis, or heart failure status of one patient.

\subsection{Peak-value Datasets}

\begin{table*}[h]
    \centering
    \ra{1.3}
    \begin{tabular}{ lr }
        \toprule
        Input variables                     & Shape \\
        \midrule                              
        Single peak systolic RLS values     & (3600, 1) \\
        Peak systolic RLS values            & (200, 18) \\
        Peak systolic GLS values            & (200, 3)  \\
        Peak systolic strain values         & (200, 21) \\
        Peak systolic RLS, and EF values    & (200, 19) \\
        Peak systolic GLS, and EF values    & (200, 4)  \\
        Peak systolic strain, and EF values & (200, 22) \\
        \bottomrule
    \end{tabular}
    \caption{Peak-value datasets. The ''Shape'' parameter is indicates: (Number of objects in the dataset, Number of dimensions of each individual object).}
    \label{tab:pv_dsets}
\end{table*}

Table \ref{tab:pv_dsets} shows the different peak-value datasets. 
All the datasets with exception of \textit{Single peak systolic RLS values} will be used to predict the diagnosis of patients, and whether the patient has heart failure. 
\textit{Single peak systolic RLS values} is also the only peak-value dataset that is not suited for clustering, as a minimum of two dimensions is required to cluster a point-value dataset.
The reason that there are more peak-value datasets than there are time-series datasets, is that the peak-value version of three datasets in table \ref{tab:ts_dsets} have been combined 
with EF to determine whether a combination of peak systolic strain and EF can have a higher predictive power than strain alone.

\section{Case Studies}
The results of this paper will be presented in the form of three case studies. 
Each case study will compare model-performance with regard to its predictive power of a specific target variable. 
Recall that the three target variables that will be considered in this thesis are: Heart failure, patient diagnosis, and the indication of individual left ventricle segments.
As mentioned previously in this chapter there are four main machine learning models that will be tested, and many variations of these four models.
In each case study a brief discussion of performance different model variations within their category will be made, 
and then the best model variation will be used to compare the performance of the main models. 

\begin{itemize}
    \item Heart failure
    \item Patient diagnosis
    \item Left ventricle segment indication
\end{itemize}
