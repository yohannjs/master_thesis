\chapter{Method}

\section{Description of The Datasets} \label{sec:datasets}

Since the different \acrshort{ml} models require different types of input data the, datasets have been divided into two main categories: 
The peak-value datasets and the time-series datasets. \bigskip

\subsection{Time-series Datasets}

\begin{table*}[h]
    \centering
    \ra{1.3}
    \begin{tabular}{ rlr }
        \toprule
        Nr & Input variables   & Shape \\
        \midrule
        1  & Single \acrshort{rls} curves & $(3600,1)$ \\
        2  & \acrshort{rls} curves        & $(200,18)$ \\
        3  & \acrshort{gls} curves        & $(200,3)$  \\
        4  & Strain curves                & $(200,21)$ \\
        \bottomrule
    \end{tabular}
    \caption{Time-series datasets. The ''Shape'' parameter is indicates: (Number of objects in the dataset, Number of curves used to represent each individual object). The curve length is not included in the shape parameter because it differs for different curves.}
    \label{tab:ts_dsets}
\end{table*}

Table \ref{tab:ts_dsets} shows the different time-series datasets that will be used. 
All the datasets except \textit{Single \acrshort{rls} curves} will be used to predict whether or not the patient is diagnosed, and whether the patient has heart failure. Recall that the different diagnosises are described in section REFERENCE, and there occurance rate are illustrated in figure \ref{fig:hf_ind_dist}.\textit{Single \acrshort{rls} curves} will be used to predict the segment indications shown in figure \ref{fig:segm_label_dist} and described in section REFERENCE. The point of classifying individual segments of a patients left ventricle is that if a single segment is found to be \textit{not normal}, this would also mean that the patient can be considered as \textit{not healthy}. As mentioned in the description of table \ref{tab:ts_dsets} the ''Shape'' parameter shows how many objects each dataset has, and how many curves are associated to each object. Since each ultrasound examination takes ultrasound inspections from three views (four chamber, two chamber, and APLAX chamber), each patient has three views to estimate a \acrshort{gls} curve from. Since each \acrshort{gls} curve, also can be divided into six \acrshort{rls} curves, there is a total of 21 strain curves per patient. Since each patient has 18 \acrshort{rls} curves, there are $18 \times 200 = 3600$ curves that make up dataset number 1. For datasets two to three it will also be experimented with wether using data from a single view performs better than data from all views. For dataset two that means that the number of curves used to represent an object will be either 6 or 18, for dataset three it will be either 1 or 3 curves and for dataset four patients will be represented with either 7 or 21 curves. Both the \acrshort{ann}, and the \acrshort{tsc} model are applied on the datasets listed in table \ref{tab:ts_dsets}. \bigskip

\subsection{Peak-value Datasets}

\begin{table*}[h]
    \centering
    \ra{1.3}
    \begin{tabular}{ rlr }
        \toprule
        Nr & Input variables                                           & Shape \\
        \midrule                              
        1  & Peak systolic \acrshort{rls} values                       & $(200,18)$ \\
        2  & Peak systolic \acrshort{gls} values                       & $(200,3)$  \\
        3  & Peak systolic strain values                               & $(200,21)$ \\
        4  & Peak systolic \acrshort{rls}, and \acrshort{ef} values    & $(200,19)$ \\
        5  & Peak systolic \acrshort{gls}, and \acrshort{ef} values    & $(200,4)$  \\
        6  & Peak systolic strain, and \acrshort{ef} values            & $(200,22)$ \\
        \bottomrule
    \end{tabular}
    \caption{Peak-value datasets. The ''Shape'' parameter is indicates: (Number of objects in the dataset, Number of dimensions used to represent each individual object).}
    \label{tab:pv_dsets}
\end{table*}

Table \ref{tab:pv_dsets} shows the different peak-value datasets. All the datasets will be used to predict the diagnosis of patients, and whether the patient has heart failure. Single peak systolic \acrshort{rls} values were not considered suited for \acrshort{pvc} or \acrshort{pvsc} models to predict segment indication, because the best model one can hope for from a one dimensional point-value dataset is a form of threshold classifier. The reason that there are more peak-value datasets than there are time-series datasets, is that the peak-value version of three datasets in table \ref{tab:ts_dsets} have been combined with \acrshort{ef} to determine whether a combination of peak systolic strain, and \acrshort{ef} can have a higher predictive power than strain alone.

\section{Clustering} \label{sec:meth_clust}

The implementations of the two clustering models that are applied in this work are described together in the same section because conceptually, they are almost identical. It is only the method used to measure dissimilarity that separates the \acrshort{pvc} and \acrshort{tsc} models. The general implementation of the clustering models is illustrated in figure \ref{fig:clust_flow}. Time-series datasets are preprocessed before dissimilarity measurement, peak-value datasets are not. In the coming subsections the processes in each of the boxes in the flow diagram will be expanded upon.

\begin{figure}
    \centering
    \tikzstyle{IO}    = [draw, ellipse,fill=red!20, minimum height=2em]
\tikzstyle{block} = [rectangle, draw, fill=blue!20, text width=7em, text centered, rounded corners, minimum height=2em]
\tikzstyle{evl}   = [rectangle, draw, fill=green!20, text width=7em, text centered, minimum height=2em]
\tikzstyle{arrow} = [thin,->,>=stealth]

\begin{tikzpicture}
% Shapes
\node (input)   [IO] {Input data};
\node (preproc) [block, node distance=1.5cm, below of=input]   {Preprocessing};
\node (diss)    [block, node distance=1.5cm, below of=preproc] {Dissimilarity measurement};
\node (clust)   [block, node distance=2cm, below of=diss]  {Hierarchical Agglomerative Clustering};
\node (output)  [IO,    node distance=2cm, below of=clust] {Cluster assignments};
\node (ari)     [evl,   node distance=1.5cm, below of=output, xshift=-3cm] {ARI};
\node (dor)     [evl,   node distance=2.1cm, below of=output, xshift=3cm] {TP, TN, FP and FN};

% Arrows
\draw [arrow] (input)   --node[anchor=west] {}                     (preproc);
\draw [arrow] (preproc) --node[anchor=west] {Processed data}       (diss);
\draw [arrow] (diss)    --node[anchor=west] {Dissimilarity matrix} (clust);
\draw [arrow] (clust)   --node[anchor=west] {}                     (output);
\draw [arrow] (output)  --node[anchor=east] {$2 \rightarrow 9$}    (ari);
\draw [arrow] (output)  --node[anchor=west] {2}                    (dor);
\end{tikzpicture}

    \caption{A flow diagram to give an overview of how the \acrshort{pvc} and \acrshort{tsc} models are implemented and evaluated.}
    \label{fig:clust_flow}
\end{figure}

\clearpage

\subsection{Time-series Preprocessing}
Preprocessing of the time series is done because it is known that the \acrshort{dtw} distance is sensitive to absolute difference, and offsets of time series. In addition to clustering the longitudinal strain time series without preprocessing three forms of preprocessing were tested to see whether they could improve the predictive performance of the clustering algorithm: Normalization, scaling and Z-score normalization. The normalized version of a time series ($\{x_t\}_N$) is calculated by equation \eqref{eq:ts_norm}. The smallest recorded value in the time series ($\mathrm{min}\{x_t\}$) is subtracted from the time series ($\{x_t\}$), then the time series is divided by the difference between the highest recorded value ($\mathrm{max}\{x_t\}$), and lowest recorded value in the time series.

\begin{equation}
    \{x_t\}_N = \frac{\{x_t\} - \mathrm{min}\{x_t\}}{\mathrm{max}\{x_t\} - \mathrm{min}\{x_t\}}
    \label{eq:ts_norm}
\end{equation}

The ''scaled'' version of a times is calculated in a similar fashion. Scaling can be considered as normalizing a time series with regard to the highest, and lowest recorded values of the entire set of time series it is being compared to. If one lets $\{ \left \{ x_t \} \right \}$ represent the set of time series to be scaled, $\mathrm{min}\{ \left \{ x_t \} \right \}$ represent the smallest recorded value in the entire set of time series and $\mathrm{max}\{ \left \{ x_t \} \right \}$ represent the highest recorded value in the set of time series, the scaled version of a time series ($\{x_t\}_S$) is given by eqation \eqref{eq:ts_scale}.

\begin{equation}
    \{x_t\}_S = \frac{\{x_t\} - \mathrm{min}\{ \left \{ x_t \} \right \}}{\mathrm{max}\{ \left \{ x_t \} \right \} - \mathrm{min}\{ \left \{ x_t \} \right \}}
    \label{eq:ts_scale}
\end{equation}

The Z-score normalization is done by transforming each observation of a time series to it's Z-score. The Z-score of an individual time-series observation is calculated by subracting the expected value of the time series, and dividing by the standard deviation. The unbiased estimators used to calculate the expected value, and standard deviation of a time series are given in equations \eqref{eq:ev_est}, and \eqref{eq:std_est} respectively. The Z-score normalized version of a time series ($\{x_t\}_Z$) is calculated using equation \eqref{eq:ts_zscn}

\begin{equation}
    \hat{\mu} = \frac{1}{n} \sum^n_{t = 1} x_t
    \label{eq:ev_est}
\end{equation}

\begin{equation}
    \hat{\sigma} = \sqrt{\frac{1}{n - 1} \sum^n_{t = 1} (x_t - \hat{\mu})^2}
    \label{eq:std_est}
\end{equation}

\begin{equation}
    \{x_t\}_Z = \frac{\{x_t\} - \hat{\mu}}{\hat{\sigma}}
    \label{eq:ts_zscn}
\end{equation}

Figure \ref{fig:preproc} illustrates how the different preprocessing methods work on the \acrshort{4ch} \acrshort{gls} curves of four random patients. By comparing \ref{fig:preproc}a and \ref{fig:preproc}d one can see that scaling preserves both the relative offsets and relative size differences between the curves. From \ref{fig:preproc}b one can see that though normalization preserves the offsets of the curves, the relative sizes are not. From \ref{fig:preproc}c one can see that Z-score normalization preserves the offsets of the curves, the relative sizes are only preserved to a certain extent. In addition, the normalized, and scaled curves are constricted between 0 and 1, while the Z-score normalized curves are not. 

\begin{figure}
    \centering
    %% Creator: Matplotlib, PGF backend
%%
%% To include the figure in your LaTeX document, write
%%   \input{<filename>.pgf}
%%
%% Make sure the required packages are loaded in your preamble
%%   \usepackage{pgf}
%%
%% Figures using additional raster images can only be included by \input if
%% they are in the same directory as the main LaTeX file. For loading figures
%% from other directories you can use the `import` package
%%   \usepackage{import}
%% and then include the figures with
%%   \import{<path to file>}{<filename>.pgf}
%%
%% Matplotlib used the following preamble
%%
\begingroup%
\makeatletter%
\begin{pgfpicture}%
\pgfpathrectangle{\pgfpointorigin}{\pgfqpoint{6.400000in}{6.400000in}}%
\pgfusepath{use as bounding box, clip}%
\begin{pgfscope}%
\pgfsetbuttcap%
\pgfsetmiterjoin%
\definecolor{currentfill}{rgb}{1.000000,1.000000,1.000000}%
\pgfsetfillcolor{currentfill}%
\pgfsetlinewidth{0.000000pt}%
\definecolor{currentstroke}{rgb}{1.000000,1.000000,1.000000}%
\pgfsetstrokecolor{currentstroke}%
\pgfsetdash{}{0pt}%
\pgfpathmoveto{\pgfqpoint{0.000000in}{0.000000in}}%
\pgfpathlineto{\pgfqpoint{6.400000in}{0.000000in}}%
\pgfpathlineto{\pgfqpoint{6.400000in}{6.400000in}}%
\pgfpathlineto{\pgfqpoint{0.000000in}{6.400000in}}%
\pgfpathclose%
\pgfusepath{fill}%
\end{pgfscope}%
\begin{pgfscope}%
\pgfsetbuttcap%
\pgfsetmiterjoin%
\definecolor{currentfill}{rgb}{1.000000,1.000000,1.000000}%
\pgfsetfillcolor{currentfill}%
\pgfsetlinewidth{0.000000pt}%
\definecolor{currentstroke}{rgb}{0.000000,0.000000,0.000000}%
\pgfsetstrokecolor{currentstroke}%
\pgfsetstrokeopacity{0.000000}%
\pgfsetdash{}{0pt}%
\pgfpathmoveto{\pgfqpoint{0.693748in}{3.520370in}}%
\pgfpathlineto{\pgfqpoint{3.269001in}{3.520370in}}%
\pgfpathlineto{\pgfqpoint{3.269001in}{6.091667in}}%
\pgfpathlineto{\pgfqpoint{0.693748in}{6.091667in}}%
\pgfpathclose%
\pgfusepath{fill}%
\end{pgfscope}%
\begin{pgfscope}%
\pgfsetbuttcap%
\pgfsetroundjoin%
\definecolor{currentfill}{rgb}{0.000000,0.000000,0.000000}%
\pgfsetfillcolor{currentfill}%
\pgfsetlinewidth{0.803000pt}%
\definecolor{currentstroke}{rgb}{0.000000,0.000000,0.000000}%
\pgfsetstrokecolor{currentstroke}%
\pgfsetdash{}{0pt}%
\pgfsys@defobject{currentmarker}{\pgfqpoint{0.000000in}{-0.048611in}}{\pgfqpoint{0.000000in}{0.000000in}}{%
\pgfpathmoveto{\pgfqpoint{0.000000in}{0.000000in}}%
\pgfpathlineto{\pgfqpoint{0.000000in}{-0.048611in}}%
\pgfusepath{stroke,fill}%
}%
\begin{pgfscope}%
\pgfsys@transformshift{0.810805in}{3.520370in}%
\pgfsys@useobject{currentmarker}{}%
\end{pgfscope}%
\end{pgfscope}%
\begin{pgfscope}%
\definecolor{textcolor}{rgb}{0.000000,0.000000,0.000000}%
\pgfsetstrokecolor{textcolor}%
\pgfsetfillcolor{textcolor}%
\pgftext[x=0.810805in,y=3.423148in,,top]{\color{textcolor}\rmfamily\fontsize{12.000000}{14.400000}\selectfont \(\displaystyle 0\)}%
\end{pgfscope}%
\begin{pgfscope}%
\pgfsetbuttcap%
\pgfsetroundjoin%
\definecolor{currentfill}{rgb}{0.000000,0.000000,0.000000}%
\pgfsetfillcolor{currentfill}%
\pgfsetlinewidth{0.803000pt}%
\definecolor{currentstroke}{rgb}{0.000000,0.000000,0.000000}%
\pgfsetstrokecolor{currentstroke}%
\pgfsetdash{}{0pt}%
\pgfsys@defobject{currentmarker}{\pgfqpoint{0.000000in}{-0.048611in}}{\pgfqpoint{0.000000in}{0.000000in}}{%
\pgfpathmoveto{\pgfqpoint{0.000000in}{0.000000in}}%
\pgfpathlineto{\pgfqpoint{0.000000in}{-0.048611in}}%
\pgfusepath{stroke,fill}%
}%
\begin{pgfscope}%
\pgfsys@transformshift{1.348998in}{3.520370in}%
\pgfsys@useobject{currentmarker}{}%
\end{pgfscope}%
\end{pgfscope}%
\begin{pgfscope}%
\definecolor{textcolor}{rgb}{0.000000,0.000000,0.000000}%
\pgfsetstrokecolor{textcolor}%
\pgfsetfillcolor{textcolor}%
\pgftext[x=1.348998in,y=3.423148in,,top]{\color{textcolor}\rmfamily\fontsize{12.000000}{14.400000}\selectfont \(\displaystyle 20\)}%
\end{pgfscope}%
\begin{pgfscope}%
\pgfsetbuttcap%
\pgfsetroundjoin%
\definecolor{currentfill}{rgb}{0.000000,0.000000,0.000000}%
\pgfsetfillcolor{currentfill}%
\pgfsetlinewidth{0.803000pt}%
\definecolor{currentstroke}{rgb}{0.000000,0.000000,0.000000}%
\pgfsetstrokecolor{currentstroke}%
\pgfsetdash{}{0pt}%
\pgfsys@defobject{currentmarker}{\pgfqpoint{0.000000in}{-0.048611in}}{\pgfqpoint{0.000000in}{0.000000in}}{%
\pgfpathmoveto{\pgfqpoint{0.000000in}{0.000000in}}%
\pgfpathlineto{\pgfqpoint{0.000000in}{-0.048611in}}%
\pgfusepath{stroke,fill}%
}%
\begin{pgfscope}%
\pgfsys@transformshift{1.887191in}{3.520370in}%
\pgfsys@useobject{currentmarker}{}%
\end{pgfscope}%
\end{pgfscope}%
\begin{pgfscope}%
\definecolor{textcolor}{rgb}{0.000000,0.000000,0.000000}%
\pgfsetstrokecolor{textcolor}%
\pgfsetfillcolor{textcolor}%
\pgftext[x=1.887191in,y=3.423148in,,top]{\color{textcolor}\rmfamily\fontsize{12.000000}{14.400000}\selectfont \(\displaystyle 40\)}%
\end{pgfscope}%
\begin{pgfscope}%
\pgfsetbuttcap%
\pgfsetroundjoin%
\definecolor{currentfill}{rgb}{0.000000,0.000000,0.000000}%
\pgfsetfillcolor{currentfill}%
\pgfsetlinewidth{0.803000pt}%
\definecolor{currentstroke}{rgb}{0.000000,0.000000,0.000000}%
\pgfsetstrokecolor{currentstroke}%
\pgfsetdash{}{0pt}%
\pgfsys@defobject{currentmarker}{\pgfqpoint{0.000000in}{-0.048611in}}{\pgfqpoint{0.000000in}{0.000000in}}{%
\pgfpathmoveto{\pgfqpoint{0.000000in}{0.000000in}}%
\pgfpathlineto{\pgfqpoint{0.000000in}{-0.048611in}}%
\pgfusepath{stroke,fill}%
}%
\begin{pgfscope}%
\pgfsys@transformshift{2.425384in}{3.520370in}%
\pgfsys@useobject{currentmarker}{}%
\end{pgfscope}%
\end{pgfscope}%
\begin{pgfscope}%
\definecolor{textcolor}{rgb}{0.000000,0.000000,0.000000}%
\pgfsetstrokecolor{textcolor}%
\pgfsetfillcolor{textcolor}%
\pgftext[x=2.425384in,y=3.423148in,,top]{\color{textcolor}\rmfamily\fontsize{12.000000}{14.400000}\selectfont \(\displaystyle 60\)}%
\end{pgfscope}%
\begin{pgfscope}%
\pgfsetbuttcap%
\pgfsetroundjoin%
\definecolor{currentfill}{rgb}{0.000000,0.000000,0.000000}%
\pgfsetfillcolor{currentfill}%
\pgfsetlinewidth{0.803000pt}%
\definecolor{currentstroke}{rgb}{0.000000,0.000000,0.000000}%
\pgfsetstrokecolor{currentstroke}%
\pgfsetdash{}{0pt}%
\pgfsys@defobject{currentmarker}{\pgfqpoint{0.000000in}{-0.048611in}}{\pgfqpoint{0.000000in}{0.000000in}}{%
\pgfpathmoveto{\pgfqpoint{0.000000in}{0.000000in}}%
\pgfpathlineto{\pgfqpoint{0.000000in}{-0.048611in}}%
\pgfusepath{stroke,fill}%
}%
\begin{pgfscope}%
\pgfsys@transformshift{2.963576in}{3.520370in}%
\pgfsys@useobject{currentmarker}{}%
\end{pgfscope}%
\end{pgfscope}%
\begin{pgfscope}%
\definecolor{textcolor}{rgb}{0.000000,0.000000,0.000000}%
\pgfsetstrokecolor{textcolor}%
\pgfsetfillcolor{textcolor}%
\pgftext[x=2.963576in,y=3.423148in,,top]{\color{textcolor}\rmfamily\fontsize{12.000000}{14.400000}\selectfont \(\displaystyle 80\)}%
\end{pgfscope}%
\begin{pgfscope}%
\pgfsetbuttcap%
\pgfsetroundjoin%
\definecolor{currentfill}{rgb}{0.000000,0.000000,0.000000}%
\pgfsetfillcolor{currentfill}%
\pgfsetlinewidth{0.803000pt}%
\definecolor{currentstroke}{rgb}{0.000000,0.000000,0.000000}%
\pgfsetstrokecolor{currentstroke}%
\pgfsetdash{}{0pt}%
\pgfsys@defobject{currentmarker}{\pgfqpoint{-0.048611in}{0.000000in}}{\pgfqpoint{0.000000in}{0.000000in}}{%
\pgfpathmoveto{\pgfqpoint{0.000000in}{0.000000in}}%
\pgfpathlineto{\pgfqpoint{-0.048611in}{0.000000in}}%
\pgfusepath{stroke,fill}%
}%
\begin{pgfscope}%
\pgfsys@transformshift{0.693748in}{3.698853in}%
\pgfsys@useobject{currentmarker}{}%
\end{pgfscope}%
\end{pgfscope}%
\begin{pgfscope}%
\definecolor{textcolor}{rgb}{0.000000,0.000000,0.000000}%
\pgfsetstrokecolor{textcolor}%
\pgfsetfillcolor{textcolor}%
\pgftext[x=0.303703in,y=3.640983in,left,base]{\color{textcolor}\rmfamily\fontsize{12.000000}{14.400000}\selectfont \(\displaystyle -15\)}%
\end{pgfscope}%
\begin{pgfscope}%
\pgfsetbuttcap%
\pgfsetroundjoin%
\definecolor{currentfill}{rgb}{0.000000,0.000000,0.000000}%
\pgfsetfillcolor{currentfill}%
\pgfsetlinewidth{0.803000pt}%
\definecolor{currentstroke}{rgb}{0.000000,0.000000,0.000000}%
\pgfsetstrokecolor{currentstroke}%
\pgfsetdash{}{0pt}%
\pgfsys@defobject{currentmarker}{\pgfqpoint{-0.048611in}{0.000000in}}{\pgfqpoint{0.000000in}{0.000000in}}{%
\pgfpathmoveto{\pgfqpoint{0.000000in}{0.000000in}}%
\pgfpathlineto{\pgfqpoint{-0.048611in}{0.000000in}}%
\pgfusepath{stroke,fill}%
}%
\begin{pgfscope}%
\pgfsys@transformshift{0.693748in}{4.429378in}%
\pgfsys@useobject{currentmarker}{}%
\end{pgfscope}%
\end{pgfscope}%
\begin{pgfscope}%
\definecolor{textcolor}{rgb}{0.000000,0.000000,0.000000}%
\pgfsetstrokecolor{textcolor}%
\pgfsetfillcolor{textcolor}%
\pgftext[x=0.303703in,y=4.371508in,left,base]{\color{textcolor}\rmfamily\fontsize{12.000000}{14.400000}\selectfont \(\displaystyle -10\)}%
\end{pgfscope}%
\begin{pgfscope}%
\pgfsetbuttcap%
\pgfsetroundjoin%
\definecolor{currentfill}{rgb}{0.000000,0.000000,0.000000}%
\pgfsetfillcolor{currentfill}%
\pgfsetlinewidth{0.803000pt}%
\definecolor{currentstroke}{rgb}{0.000000,0.000000,0.000000}%
\pgfsetstrokecolor{currentstroke}%
\pgfsetdash{}{0pt}%
\pgfsys@defobject{currentmarker}{\pgfqpoint{-0.048611in}{0.000000in}}{\pgfqpoint{0.000000in}{0.000000in}}{%
\pgfpathmoveto{\pgfqpoint{0.000000in}{0.000000in}}%
\pgfpathlineto{\pgfqpoint{-0.048611in}{0.000000in}}%
\pgfusepath{stroke,fill}%
}%
\begin{pgfscope}%
\pgfsys@transformshift{0.693748in}{5.159903in}%
\pgfsys@useobject{currentmarker}{}%
\end{pgfscope}%
\end{pgfscope}%
\begin{pgfscope}%
\definecolor{textcolor}{rgb}{0.000000,0.000000,0.000000}%
\pgfsetstrokecolor{textcolor}%
\pgfsetfillcolor{textcolor}%
\pgftext[x=0.385299in,y=5.102033in,left,base]{\color{textcolor}\rmfamily\fontsize{12.000000}{14.400000}\selectfont \(\displaystyle -5\)}%
\end{pgfscope}%
\begin{pgfscope}%
\pgfsetbuttcap%
\pgfsetroundjoin%
\definecolor{currentfill}{rgb}{0.000000,0.000000,0.000000}%
\pgfsetfillcolor{currentfill}%
\pgfsetlinewidth{0.803000pt}%
\definecolor{currentstroke}{rgb}{0.000000,0.000000,0.000000}%
\pgfsetstrokecolor{currentstroke}%
\pgfsetdash{}{0pt}%
\pgfsys@defobject{currentmarker}{\pgfqpoint{-0.048611in}{0.000000in}}{\pgfqpoint{0.000000in}{0.000000in}}{%
\pgfpathmoveto{\pgfqpoint{0.000000in}{0.000000in}}%
\pgfpathlineto{\pgfqpoint{-0.048611in}{0.000000in}}%
\pgfusepath{stroke,fill}%
}%
\begin{pgfscope}%
\pgfsys@transformshift{0.693748in}{5.890428in}%
\pgfsys@useobject{currentmarker}{}%
\end{pgfscope}%
\end{pgfscope}%
\begin{pgfscope}%
\definecolor{textcolor}{rgb}{0.000000,0.000000,0.000000}%
\pgfsetstrokecolor{textcolor}%
\pgfsetfillcolor{textcolor}%
\pgftext[x=0.514929in,y=5.832558in,left,base]{\color{textcolor}\rmfamily\fontsize{12.000000}{14.400000}\selectfont \(\displaystyle 0\)}%
\end{pgfscope}%
\begin{pgfscope}%
\definecolor{textcolor}{rgb}{0.000000,0.000000,0.000000}%
\pgfsetstrokecolor{textcolor}%
\pgfsetfillcolor{textcolor}%
\pgftext[x=0.248148in,y=4.806018in,,bottom,rotate=90.000000]{\color{textcolor}\rmfamily\fontsize{12.000000}{14.400000}\selectfont Strain}%
\end{pgfscope}%
\begin{pgfscope}%
\pgfpathrectangle{\pgfqpoint{0.693748in}{3.520370in}}{\pgfqpoint{2.575253in}{2.571297in}}%
\pgfusepath{clip}%
\pgfsetrectcap%
\pgfsetroundjoin%
\pgfsetlinewidth{1.505625pt}%
\definecolor{currentstroke}{rgb}{0.121569,0.466667,0.705882}%
\pgfsetstrokecolor{currentstroke}%
\pgfsetdash{}{0pt}%
\pgfpathmoveto{\pgfqpoint{0.810805in}{4.909606in}}%
\pgfpathlineto{\pgfqpoint{0.837715in}{5.009442in}}%
\pgfpathlineto{\pgfqpoint{0.864624in}{5.137607in}}%
\pgfpathlineto{\pgfqpoint{0.891534in}{5.303902in}}%
\pgfpathlineto{\pgfqpoint{0.918444in}{5.494865in}}%
\pgfpathlineto{\pgfqpoint{0.945353in}{5.679574in}}%
\pgfpathlineto{\pgfqpoint{0.972263in}{5.823424in}}%
\pgfpathlineto{\pgfqpoint{0.999173in}{5.908722in}}%
\pgfpathlineto{\pgfqpoint{1.026082in}{5.942091in}}%
\pgfpathlineto{\pgfqpoint{1.052992in}{5.941784in}}%
\pgfpathlineto{\pgfqpoint{1.079901in}{5.924005in}}%
\pgfpathlineto{\pgfqpoint{1.106811in}{5.902753in}}%
\pgfpathlineto{\pgfqpoint{1.133721in}{5.890428in}}%
\pgfpathlineto{\pgfqpoint{1.160630in}{5.892809in}}%
\pgfpathlineto{\pgfqpoint{1.187540in}{5.904969in}}%
\pgfpathlineto{\pgfqpoint{1.214450in}{5.911575in}}%
\pgfpathlineto{\pgfqpoint{1.241359in}{5.894517in}}%
\pgfpathlineto{\pgfqpoint{1.268269in}{5.844482in}}%
\pgfpathlineto{\pgfqpoint{1.295179in}{5.764338in}}%
\pgfpathlineto{\pgfqpoint{1.322088in}{5.662964in}}%
\pgfpathlineto{\pgfqpoint{1.348998in}{5.550368in}}%
\pgfpathlineto{\pgfqpoint{1.375908in}{5.435144in}}%
\pgfpathlineto{\pgfqpoint{1.402817in}{5.321923in}}%
\pgfpathlineto{\pgfqpoint{1.429727in}{5.210320in}}%
\pgfpathlineto{\pgfqpoint{1.456636in}{5.097667in}}%
\pgfpathlineto{\pgfqpoint{1.483546in}{4.983696in}}%
\pgfpathlineto{\pgfqpoint{1.510456in}{4.871054in}}%
\pgfpathlineto{\pgfqpoint{1.537365in}{4.761590in}}%
\pgfpathlineto{\pgfqpoint{1.564275in}{4.655328in}}%
\pgfpathlineto{\pgfqpoint{1.591185in}{4.553648in}}%
\pgfpathlineto{\pgfqpoint{1.618094in}{4.458228in}}%
\pgfpathlineto{\pgfqpoint{1.645004in}{4.370831in}}%
\pgfpathlineto{\pgfqpoint{1.671914in}{4.293544in}}%
\pgfpathlineto{\pgfqpoint{1.698823in}{4.229239in}}%
\pgfpathlineto{\pgfqpoint{1.725733in}{4.180946in}}%
\pgfpathlineto{\pgfqpoint{1.752643in}{4.150458in}}%
\pgfpathlineto{\pgfqpoint{1.779552in}{4.137840in}}%
\pgfpathlineto{\pgfqpoint{1.806462in}{4.141755in}}%
\pgfpathlineto{\pgfqpoint{1.833371in}{4.158713in}}%
\pgfpathlineto{\pgfqpoint{1.860281in}{4.179740in}}%
\pgfpathlineto{\pgfqpoint{1.887191in}{4.191514in}}%
\pgfpathlineto{\pgfqpoint{1.914100in}{4.185875in}}%
\pgfpathlineto{\pgfqpoint{1.941010in}{4.166779in}}%
\pgfpathlineto{\pgfqpoint{1.967920in}{4.148212in}}%
\pgfpathlineto{\pgfqpoint{1.994829in}{4.144751in}}%
\pgfpathlineto{\pgfqpoint{2.021739in}{4.164043in}}%
\pgfpathlineto{\pgfqpoint{2.048649in}{4.205078in}}%
\pgfpathlineto{\pgfqpoint{2.075558in}{4.259021in}}%
\pgfpathlineto{\pgfqpoint{2.102468in}{4.314952in}}%
\pgfpathlineto{\pgfqpoint{2.129378in}{4.366553in}}%
\pgfpathlineto{\pgfqpoint{2.156287in}{4.413264in}}%
\pgfpathlineto{\pgfqpoint{2.183197in}{4.457854in}}%
\pgfpathlineto{\pgfqpoint{2.210106in}{4.503348in}}%
\pgfpathlineto{\pgfqpoint{2.237016in}{4.551042in}}%
\pgfpathlineto{\pgfqpoint{2.263926in}{4.600086in}}%
\pgfpathlineto{\pgfqpoint{2.290835in}{4.648629in}}%
\pgfpathlineto{\pgfqpoint{2.317745in}{4.694911in}}%
\pgfpathlineto{\pgfqpoint{2.344655in}{4.737850in}}%
\pgfpathlineto{\pgfqpoint{2.371564in}{4.777064in}}%
\pgfpathlineto{\pgfqpoint{2.398474in}{4.813378in}}%
\pgfpathlineto{\pgfqpoint{2.425384in}{4.854171in}}%
\pgfpathlineto{\pgfqpoint{2.452293in}{4.917717in}}%
\pgfpathlineto{\pgfqpoint{2.479203in}{5.026686in}}%
\pgfpathlineto{\pgfqpoint{2.506113in}{5.191696in}}%
\pgfpathlineto{\pgfqpoint{2.533022in}{5.398009in}}%
\pgfpathlineto{\pgfqpoint{2.559932in}{5.608273in}}%
\pgfpathlineto{\pgfqpoint{2.586841in}{5.780355in}}%
\pgfpathlineto{\pgfqpoint{2.613751in}{5.889180in}}%
\pgfpathlineto{\pgfqpoint{2.640661in}{5.936001in}}%
\pgfpathlineto{\pgfqpoint{2.667570in}{5.939834in}}%
\pgfpathlineto{\pgfqpoint{2.694480in}{5.922220in}}%
\pgfpathlineto{\pgfqpoint{2.721390in}{5.901167in}}%
\pgfpathlineto{\pgfqpoint{2.748299in}{5.890428in}}%
\pgfpathlineto{\pgfqpoint{2.775209in}{5.896020in}}%
\pgfpathlineto{\pgfqpoint{2.802119in}{5.911628in}}%
\pgfpathlineto{\pgfqpoint{2.829028in}{5.921483in}}%
\pgfpathlineto{\pgfqpoint{2.855938in}{5.910098in}}%
\pgfpathlineto{\pgfqpoint{2.882848in}{5.868840in}}%
\pgfpathlineto{\pgfqpoint{2.909757in}{5.797889in}}%
\pgfpathlineto{\pgfqpoint{2.936667in}{5.703420in}}%
\pgfpathlineto{\pgfqpoint{2.963576in}{5.593517in}}%
\pgfpathlineto{\pgfqpoint{2.990486in}{5.477352in}}%
\pgfpathlineto{\pgfqpoint{3.017396in}{5.362659in}}%
\pgfpathlineto{\pgfqpoint{3.044305in}{5.252797in}}%
\pgfpathlineto{\pgfqpoint{3.071215in}{5.146348in}}%
\pgfusepath{stroke}%
\end{pgfscope}%
\begin{pgfscope}%
\pgfpathrectangle{\pgfqpoint{0.693748in}{3.520370in}}{\pgfqpoint{2.575253in}{2.571297in}}%
\pgfusepath{clip}%
\pgfsetrectcap%
\pgfsetroundjoin%
\pgfsetlinewidth{1.505625pt}%
\definecolor{currentstroke}{rgb}{1.000000,0.498039,0.054902}%
\pgfsetstrokecolor{currentstroke}%
\pgfsetdash{}{0pt}%
\pgfpathmoveto{\pgfqpoint{0.810805in}{5.343090in}}%
\pgfpathlineto{\pgfqpoint{0.837715in}{5.348397in}}%
\pgfpathlineto{\pgfqpoint{0.864624in}{5.362262in}}%
\pgfpathlineto{\pgfqpoint{0.891534in}{5.397276in}}%
\pgfpathlineto{\pgfqpoint{0.918444in}{5.465259in}}%
\pgfpathlineto{\pgfqpoint{0.945353in}{5.566059in}}%
\pgfpathlineto{\pgfqpoint{0.972263in}{5.685322in}}%
\pgfpathlineto{\pgfqpoint{0.999173in}{5.796150in}}%
\pgfpathlineto{\pgfqpoint{1.026082in}{5.875747in}}%
\pgfpathlineto{\pgfqpoint{1.052992in}{5.913938in}}%
\pgfpathlineto{\pgfqpoint{1.079901in}{5.914767in}}%
\pgfpathlineto{\pgfqpoint{1.106811in}{5.890428in}}%
\pgfpathlineto{\pgfqpoint{1.133721in}{5.858131in}}%
\pgfpathlineto{\pgfqpoint{1.160630in}{5.828283in}}%
\pgfpathlineto{\pgfqpoint{1.187540in}{5.793359in}}%
\pgfpathlineto{\pgfqpoint{1.214450in}{5.733937in}}%
\pgfpathlineto{\pgfqpoint{1.241359in}{5.637632in}}%
\pgfpathlineto{\pgfqpoint{1.268269in}{5.514021in}}%
\pgfpathlineto{\pgfqpoint{1.295179in}{5.386128in}}%
\pgfpathlineto{\pgfqpoint{1.322088in}{5.273123in}}%
\pgfpathlineto{\pgfqpoint{1.348998in}{5.181896in}}%
\pgfpathlineto{\pgfqpoint{1.375908in}{5.102292in}}%
\pgfpathlineto{\pgfqpoint{1.402817in}{5.014035in}}%
\pgfpathlineto{\pgfqpoint{1.429727in}{4.904134in}}%
\pgfpathlineto{\pgfqpoint{1.456636in}{4.777201in}}%
\pgfpathlineto{\pgfqpoint{1.483546in}{4.646831in}}%
\pgfpathlineto{\pgfqpoint{1.510456in}{4.526044in}}%
\pgfpathlineto{\pgfqpoint{1.537365in}{4.421820in}}%
\pgfpathlineto{\pgfqpoint{1.564275in}{4.339871in}}%
\pgfpathlineto{\pgfqpoint{1.591185in}{4.284714in}}%
\pgfpathlineto{\pgfqpoint{1.618094in}{4.258985in}}%
\pgfpathlineto{\pgfqpoint{1.645004in}{4.260839in}}%
\pgfpathlineto{\pgfqpoint{1.671914in}{4.283573in}}%
\pgfpathlineto{\pgfqpoint{1.698823in}{4.321151in}}%
\pgfpathlineto{\pgfqpoint{1.725733in}{4.372884in}}%
\pgfpathlineto{\pgfqpoint{1.752643in}{4.442895in}}%
\pgfpathlineto{\pgfqpoint{1.779552in}{4.530497in}}%
\pgfpathlineto{\pgfqpoint{1.806462in}{4.622801in}}%
\pgfpathlineto{\pgfqpoint{1.833371in}{4.705210in}}%
\pgfpathlineto{\pgfqpoint{1.860281in}{4.781363in}}%
\pgfpathlineto{\pgfqpoint{1.887191in}{4.867972in}}%
\pgfpathlineto{\pgfqpoint{1.914100in}{4.969344in}}%
\pgfpathlineto{\pgfqpoint{1.941010in}{5.071497in}}%
\pgfpathlineto{\pgfqpoint{1.967920in}{5.157628in}}%
\pgfpathlineto{\pgfqpoint{1.994829in}{5.219029in}}%
\pgfpathlineto{\pgfqpoint{2.021739in}{5.259885in}}%
\pgfpathlineto{\pgfqpoint{2.048649in}{5.289648in}}%
\pgfpathlineto{\pgfqpoint{2.075558in}{5.312643in}}%
\pgfpathlineto{\pgfqpoint{2.102468in}{5.327292in}}%
\pgfpathlineto{\pgfqpoint{2.129378in}{5.331879in}}%
\pgfpathlineto{\pgfqpoint{2.156287in}{5.332137in}}%
\pgfpathlineto{\pgfqpoint{2.183197in}{5.337987in}}%
\pgfpathlineto{\pgfqpoint{2.210106in}{5.354949in}}%
\pgfpathlineto{\pgfqpoint{2.237016in}{5.379981in}}%
\pgfpathlineto{\pgfqpoint{2.263926in}{5.404483in}}%
\pgfpathlineto{\pgfqpoint{2.290835in}{5.419894in}}%
\pgfpathlineto{\pgfqpoint{2.317745in}{5.423591in}}%
\pgfpathlineto{\pgfqpoint{2.344655in}{5.419480in}}%
\pgfpathlineto{\pgfqpoint{2.371564in}{5.414756in}}%
\pgfpathlineto{\pgfqpoint{2.398474in}{5.415950in}}%
\pgfpathlineto{\pgfqpoint{2.425384in}{5.426682in}}%
\pgfpathlineto{\pgfqpoint{2.452293in}{5.451330in}}%
\pgfpathlineto{\pgfqpoint{2.479203in}{5.493486in}}%
\pgfpathlineto{\pgfqpoint{2.506113in}{5.551998in}}%
\pgfpathlineto{\pgfqpoint{2.533022in}{5.621096in}}%
\pgfpathlineto{\pgfqpoint{2.559932in}{5.694145in}}%
\pgfpathlineto{\pgfqpoint{2.586841in}{5.764749in}}%
\pgfpathlineto{\pgfqpoint{2.613751in}{5.824751in}}%
\pgfpathlineto{\pgfqpoint{2.640661in}{5.866517in}}%
\pgfpathlineto{\pgfqpoint{2.667570in}{5.890428in}}%
\pgfpathlineto{\pgfqpoint{2.694480in}{5.904392in}}%
\pgfpathlineto{\pgfqpoint{2.721390in}{5.912437in}}%
\pgfpathlineto{\pgfqpoint{2.748299in}{5.906747in}}%
\pgfpathlineto{\pgfqpoint{2.775209in}{5.872880in}}%
\pgfpathlineto{\pgfqpoint{2.802119in}{5.804391in}}%
\pgfpathlineto{\pgfqpoint{2.829028in}{5.709825in}}%
\pgfpathlineto{\pgfqpoint{2.855938in}{5.601609in}}%
\pgfpathlineto{\pgfqpoint{2.882848in}{5.483466in}}%
\pgfpathlineto{\pgfqpoint{2.909757in}{5.352321in}}%
\pgfpathlineto{\pgfqpoint{2.936667in}{5.204489in}}%
\pgfpathlineto{\pgfqpoint{2.963576in}{5.042055in}}%
\pgfusepath{stroke}%
\end{pgfscope}%
\begin{pgfscope}%
\pgfpathrectangle{\pgfqpoint{0.693748in}{3.520370in}}{\pgfqpoint{2.575253in}{2.571297in}}%
\pgfusepath{clip}%
\pgfsetrectcap%
\pgfsetroundjoin%
\pgfsetlinewidth{1.505625pt}%
\definecolor{currentstroke}{rgb}{0.172549,0.627451,0.172549}%
\pgfsetstrokecolor{currentstroke}%
\pgfsetdash{}{0pt}%
\pgfpathmoveto{\pgfqpoint{0.810805in}{5.117035in}}%
\pgfpathlineto{\pgfqpoint{0.837715in}{5.131951in}}%
\pgfpathlineto{\pgfqpoint{0.864624in}{5.145182in}}%
\pgfpathlineto{\pgfqpoint{0.891534in}{5.153645in}}%
\pgfpathlineto{\pgfqpoint{0.918444in}{5.156730in}}%
\pgfpathlineto{\pgfqpoint{0.945353in}{5.162410in}}%
\pgfpathlineto{\pgfqpoint{0.972263in}{5.186581in}}%
\pgfpathlineto{\pgfqpoint{0.999173in}{5.245347in}}%
\pgfpathlineto{\pgfqpoint{1.026082in}{5.344569in}}%
\pgfpathlineto{\pgfqpoint{1.052992in}{5.476623in}}%
\pgfpathlineto{\pgfqpoint{1.079901in}{5.626278in}}%
\pgfpathlineto{\pgfqpoint{1.106811in}{5.772812in}}%
\pgfpathlineto{\pgfqpoint{1.133721in}{5.890428in}}%
\pgfpathlineto{\pgfqpoint{1.160630in}{5.958522in}}%
\pgfpathlineto{\pgfqpoint{1.187540in}{5.974790in}}%
\pgfpathlineto{\pgfqpoint{1.214450in}{5.956270in}}%
\pgfpathlineto{\pgfqpoint{1.241359in}{5.922594in}}%
\pgfpathlineto{\pgfqpoint{1.268269in}{5.877992in}}%
\pgfpathlineto{\pgfqpoint{1.295179in}{5.811889in}}%
\pgfpathlineto{\pgfqpoint{1.322088in}{5.716314in}}%
\pgfpathlineto{\pgfqpoint{1.348998in}{5.595720in}}%
\pgfpathlineto{\pgfqpoint{1.375908in}{5.459015in}}%
\pgfpathlineto{\pgfqpoint{1.402817in}{5.310249in}}%
\pgfpathlineto{\pgfqpoint{1.429727in}{5.151435in}}%
\pgfpathlineto{\pgfqpoint{1.456636in}{4.988789in}}%
\pgfpathlineto{\pgfqpoint{1.483546in}{4.834020in}}%
\pgfpathlineto{\pgfqpoint{1.510456in}{4.695514in}}%
\pgfpathlineto{\pgfqpoint{1.537365in}{4.571002in}}%
\pgfpathlineto{\pgfqpoint{1.564275in}{4.450130in}}%
\pgfpathlineto{\pgfqpoint{1.591185in}{4.322611in}}%
\pgfpathlineto{\pgfqpoint{1.618094in}{4.184991in}}%
\pgfpathlineto{\pgfqpoint{1.645004in}{4.039658in}}%
\pgfpathlineto{\pgfqpoint{1.671914in}{3.894811in}}%
\pgfpathlineto{\pgfqpoint{1.698823in}{3.765657in}}%
\pgfpathlineto{\pgfqpoint{1.725733in}{3.672724in}}%
\pgfpathlineto{\pgfqpoint{1.752643in}{3.637247in}}%
\pgfpathlineto{\pgfqpoint{1.779552in}{3.673110in}}%
\pgfpathlineto{\pgfqpoint{1.806462in}{3.778390in}}%
\pgfpathlineto{\pgfqpoint{1.833371in}{3.929702in}}%
\pgfpathlineto{\pgfqpoint{1.860281in}{4.085599in}}%
\pgfpathlineto{\pgfqpoint{1.887191in}{4.202831in}}%
\pgfpathlineto{\pgfqpoint{1.914100in}{4.257538in}}%
\pgfpathlineto{\pgfqpoint{1.941010in}{4.255323in}}%
\pgfpathlineto{\pgfqpoint{1.967920in}{4.221468in}}%
\pgfpathlineto{\pgfqpoint{1.994829in}{4.186607in}}%
\pgfpathlineto{\pgfqpoint{2.021739in}{4.177525in}}%
\pgfpathlineto{\pgfqpoint{2.048649in}{4.206701in}}%
\pgfpathlineto{\pgfqpoint{2.075558in}{4.269095in}}%
\pgfpathlineto{\pgfqpoint{2.102468in}{4.361325in}}%
\pgfpathlineto{\pgfqpoint{2.129378in}{4.488569in}}%
\pgfpathlineto{\pgfqpoint{2.156287in}{4.654771in}}%
\pgfpathlineto{\pgfqpoint{2.183197in}{4.850213in}}%
\pgfpathlineto{\pgfqpoint{2.210106in}{5.046591in}}%
\pgfpathlineto{\pgfqpoint{2.237016in}{5.206128in}}%
\pgfpathlineto{\pgfqpoint{2.263926in}{5.302030in}}%
\pgfpathlineto{\pgfqpoint{2.290835in}{5.333973in}}%
\pgfpathlineto{\pgfqpoint{2.317745in}{5.322792in}}%
\pgfpathlineto{\pgfqpoint{2.344655in}{5.292899in}}%
\pgfpathlineto{\pgfqpoint{2.371564in}{5.261804in}}%
\pgfpathlineto{\pgfqpoint{2.398474in}{5.237727in}}%
\pgfpathlineto{\pgfqpoint{2.425384in}{5.220653in}}%
\pgfpathlineto{\pgfqpoint{2.452293in}{5.208054in}}%
\pgfpathlineto{\pgfqpoint{2.479203in}{5.200555in}}%
\pgfpathlineto{\pgfqpoint{2.506113in}{5.199685in}}%
\pgfpathlineto{\pgfqpoint{2.533022in}{5.203701in}}%
\pgfpathlineto{\pgfqpoint{2.559932in}{5.207633in}}%
\pgfpathlineto{\pgfqpoint{2.586841in}{5.207847in}}%
\pgfpathlineto{\pgfqpoint{2.613751in}{5.206013in}}%
\pgfpathlineto{\pgfqpoint{2.640661in}{5.211414in}}%
\pgfpathlineto{\pgfqpoint{2.667570in}{5.240744in}}%
\pgfpathlineto{\pgfqpoint{2.694480in}{5.309024in}}%
\pgfpathlineto{\pgfqpoint{2.721390in}{5.419386in}}%
\pgfpathlineto{\pgfqpoint{2.748299in}{5.560893in}}%
\pgfpathlineto{\pgfqpoint{2.775209in}{5.710934in}}%
\pgfpathlineto{\pgfqpoint{2.802119in}{5.834172in}}%
\pgfpathlineto{\pgfqpoint{2.829028in}{5.890428in}}%
\pgfpathlineto{\pgfqpoint{2.855938in}{5.863393in}}%
\pgfpathlineto{\pgfqpoint{2.882848in}{5.775198in}}%
\pgfpathlineto{\pgfqpoint{2.909757in}{5.671712in}}%
\pgfpathlineto{\pgfqpoint{2.936667in}{5.586053in}}%
\pgfpathlineto{\pgfqpoint{2.963576in}{5.513666in}}%
\pgfpathlineto{\pgfqpoint{2.990486in}{5.428587in}}%
\pgfpathlineto{\pgfqpoint{3.017396in}{5.313759in}}%
\pgfpathlineto{\pgfqpoint{3.044305in}{5.169652in}}%
\pgfpathlineto{\pgfqpoint{3.071215in}{5.005061in}}%
\pgfpathlineto{\pgfqpoint{3.098125in}{4.826768in}}%
\pgfpathlineto{\pgfqpoint{3.125034in}{4.642917in}}%
\pgfpathlineto{\pgfqpoint{3.151944in}{4.457568in}}%
\pgfusepath{stroke}%
\end{pgfscope}%
\begin{pgfscope}%
\pgfsetrectcap%
\pgfsetmiterjoin%
\pgfsetlinewidth{0.803000pt}%
\definecolor{currentstroke}{rgb}{0.000000,0.000000,0.000000}%
\pgfsetstrokecolor{currentstroke}%
\pgfsetdash{}{0pt}%
\pgfpathmoveto{\pgfqpoint{0.693748in}{3.520370in}}%
\pgfpathlineto{\pgfqpoint{0.693748in}{6.091667in}}%
\pgfusepath{stroke}%
\end{pgfscope}%
\begin{pgfscope}%
\pgfsetrectcap%
\pgfsetmiterjoin%
\pgfsetlinewidth{0.803000pt}%
\definecolor{currentstroke}{rgb}{0.000000,0.000000,0.000000}%
\pgfsetstrokecolor{currentstroke}%
\pgfsetdash{}{0pt}%
\pgfpathmoveto{\pgfqpoint{3.269001in}{3.520370in}}%
\pgfpathlineto{\pgfqpoint{3.269001in}{6.091667in}}%
\pgfusepath{stroke}%
\end{pgfscope}%
\begin{pgfscope}%
\pgfsetrectcap%
\pgfsetmiterjoin%
\pgfsetlinewidth{0.803000pt}%
\definecolor{currentstroke}{rgb}{0.000000,0.000000,0.000000}%
\pgfsetstrokecolor{currentstroke}%
\pgfsetdash{}{0pt}%
\pgfpathmoveto{\pgfqpoint{0.693748in}{3.520370in}}%
\pgfpathlineto{\pgfqpoint{3.269001in}{3.520370in}}%
\pgfusepath{stroke}%
\end{pgfscope}%
\begin{pgfscope}%
\pgfsetrectcap%
\pgfsetmiterjoin%
\pgfsetlinewidth{0.803000pt}%
\definecolor{currentstroke}{rgb}{0.000000,0.000000,0.000000}%
\pgfsetstrokecolor{currentstroke}%
\pgfsetdash{}{0pt}%
\pgfpathmoveto{\pgfqpoint{0.693748in}{6.091667in}}%
\pgfpathlineto{\pgfqpoint{3.269001in}{6.091667in}}%
\pgfusepath{stroke}%
\end{pgfscope}%
\begin{pgfscope}%
\definecolor{textcolor}{rgb}{0.000000,0.000000,0.000000}%
\pgfsetstrokecolor{textcolor}%
\pgfsetfillcolor{textcolor}%
\pgftext[x=1.981374in,y=6.175000in,,base]{\color{textcolor}\rmfamily\fontsize{12.000000}{14.400000}\selectfont (a) Regular}%
\end{pgfscope}%
\begin{pgfscope}%
\pgfsetbuttcap%
\pgfsetmiterjoin%
\definecolor{currentfill}{rgb}{1.000000,1.000000,1.000000}%
\pgfsetfillcolor{currentfill}%
\pgfsetlinewidth{0.000000pt}%
\definecolor{currentstroke}{rgb}{0.000000,0.000000,0.000000}%
\pgfsetstrokecolor{currentstroke}%
\pgfsetstrokeopacity{0.000000}%
\pgfsetdash{}{0pt}%
\pgfpathmoveto{\pgfqpoint{3.724747in}{3.520370in}}%
\pgfpathlineto{\pgfqpoint{6.300000in}{3.520370in}}%
\pgfpathlineto{\pgfqpoint{6.300000in}{6.091667in}}%
\pgfpathlineto{\pgfqpoint{3.724747in}{6.091667in}}%
\pgfpathclose%
\pgfusepath{fill}%
\end{pgfscope}%
\begin{pgfscope}%
\pgfsetbuttcap%
\pgfsetroundjoin%
\definecolor{currentfill}{rgb}{0.000000,0.000000,0.000000}%
\pgfsetfillcolor{currentfill}%
\pgfsetlinewidth{0.803000pt}%
\definecolor{currentstroke}{rgb}{0.000000,0.000000,0.000000}%
\pgfsetstrokecolor{currentstroke}%
\pgfsetdash{}{0pt}%
\pgfsys@defobject{currentmarker}{\pgfqpoint{0.000000in}{-0.048611in}}{\pgfqpoint{0.000000in}{0.000000in}}{%
\pgfpathmoveto{\pgfqpoint{0.000000in}{0.000000in}}%
\pgfpathlineto{\pgfqpoint{0.000000in}{-0.048611in}}%
\pgfusepath{stroke,fill}%
}%
\begin{pgfscope}%
\pgfsys@transformshift{3.841804in}{3.520370in}%
\pgfsys@useobject{currentmarker}{}%
\end{pgfscope}%
\end{pgfscope}%
\begin{pgfscope}%
\definecolor{textcolor}{rgb}{0.000000,0.000000,0.000000}%
\pgfsetstrokecolor{textcolor}%
\pgfsetfillcolor{textcolor}%
\pgftext[x=3.841804in,y=3.423148in,,top]{\color{textcolor}\rmfamily\fontsize{12.000000}{14.400000}\selectfont \(\displaystyle 0\)}%
\end{pgfscope}%
\begin{pgfscope}%
\pgfsetbuttcap%
\pgfsetroundjoin%
\definecolor{currentfill}{rgb}{0.000000,0.000000,0.000000}%
\pgfsetfillcolor{currentfill}%
\pgfsetlinewidth{0.803000pt}%
\definecolor{currentstroke}{rgb}{0.000000,0.000000,0.000000}%
\pgfsetstrokecolor{currentstroke}%
\pgfsetdash{}{0pt}%
\pgfsys@defobject{currentmarker}{\pgfqpoint{0.000000in}{-0.048611in}}{\pgfqpoint{0.000000in}{0.000000in}}{%
\pgfpathmoveto{\pgfqpoint{0.000000in}{0.000000in}}%
\pgfpathlineto{\pgfqpoint{0.000000in}{-0.048611in}}%
\pgfusepath{stroke,fill}%
}%
\begin{pgfscope}%
\pgfsys@transformshift{4.379997in}{3.520370in}%
\pgfsys@useobject{currentmarker}{}%
\end{pgfscope}%
\end{pgfscope}%
\begin{pgfscope}%
\definecolor{textcolor}{rgb}{0.000000,0.000000,0.000000}%
\pgfsetstrokecolor{textcolor}%
\pgfsetfillcolor{textcolor}%
\pgftext[x=4.379997in,y=3.423148in,,top]{\color{textcolor}\rmfamily\fontsize{12.000000}{14.400000}\selectfont \(\displaystyle 20\)}%
\end{pgfscope}%
\begin{pgfscope}%
\pgfsetbuttcap%
\pgfsetroundjoin%
\definecolor{currentfill}{rgb}{0.000000,0.000000,0.000000}%
\pgfsetfillcolor{currentfill}%
\pgfsetlinewidth{0.803000pt}%
\definecolor{currentstroke}{rgb}{0.000000,0.000000,0.000000}%
\pgfsetstrokecolor{currentstroke}%
\pgfsetdash{}{0pt}%
\pgfsys@defobject{currentmarker}{\pgfqpoint{0.000000in}{-0.048611in}}{\pgfqpoint{0.000000in}{0.000000in}}{%
\pgfpathmoveto{\pgfqpoint{0.000000in}{0.000000in}}%
\pgfpathlineto{\pgfqpoint{0.000000in}{-0.048611in}}%
\pgfusepath{stroke,fill}%
}%
\begin{pgfscope}%
\pgfsys@transformshift{4.918190in}{3.520370in}%
\pgfsys@useobject{currentmarker}{}%
\end{pgfscope}%
\end{pgfscope}%
\begin{pgfscope}%
\definecolor{textcolor}{rgb}{0.000000,0.000000,0.000000}%
\pgfsetstrokecolor{textcolor}%
\pgfsetfillcolor{textcolor}%
\pgftext[x=4.918190in,y=3.423148in,,top]{\color{textcolor}\rmfamily\fontsize{12.000000}{14.400000}\selectfont \(\displaystyle 40\)}%
\end{pgfscope}%
\begin{pgfscope}%
\pgfsetbuttcap%
\pgfsetroundjoin%
\definecolor{currentfill}{rgb}{0.000000,0.000000,0.000000}%
\pgfsetfillcolor{currentfill}%
\pgfsetlinewidth{0.803000pt}%
\definecolor{currentstroke}{rgb}{0.000000,0.000000,0.000000}%
\pgfsetstrokecolor{currentstroke}%
\pgfsetdash{}{0pt}%
\pgfsys@defobject{currentmarker}{\pgfqpoint{0.000000in}{-0.048611in}}{\pgfqpoint{0.000000in}{0.000000in}}{%
\pgfpathmoveto{\pgfqpoint{0.000000in}{0.000000in}}%
\pgfpathlineto{\pgfqpoint{0.000000in}{-0.048611in}}%
\pgfusepath{stroke,fill}%
}%
\begin{pgfscope}%
\pgfsys@transformshift{5.456383in}{3.520370in}%
\pgfsys@useobject{currentmarker}{}%
\end{pgfscope}%
\end{pgfscope}%
\begin{pgfscope}%
\definecolor{textcolor}{rgb}{0.000000,0.000000,0.000000}%
\pgfsetstrokecolor{textcolor}%
\pgfsetfillcolor{textcolor}%
\pgftext[x=5.456383in,y=3.423148in,,top]{\color{textcolor}\rmfamily\fontsize{12.000000}{14.400000}\selectfont \(\displaystyle 60\)}%
\end{pgfscope}%
\begin{pgfscope}%
\pgfsetbuttcap%
\pgfsetroundjoin%
\definecolor{currentfill}{rgb}{0.000000,0.000000,0.000000}%
\pgfsetfillcolor{currentfill}%
\pgfsetlinewidth{0.803000pt}%
\definecolor{currentstroke}{rgb}{0.000000,0.000000,0.000000}%
\pgfsetstrokecolor{currentstroke}%
\pgfsetdash{}{0pt}%
\pgfsys@defobject{currentmarker}{\pgfqpoint{0.000000in}{-0.048611in}}{\pgfqpoint{0.000000in}{0.000000in}}{%
\pgfpathmoveto{\pgfqpoint{0.000000in}{0.000000in}}%
\pgfpathlineto{\pgfqpoint{0.000000in}{-0.048611in}}%
\pgfusepath{stroke,fill}%
}%
\begin{pgfscope}%
\pgfsys@transformshift{5.994576in}{3.520370in}%
\pgfsys@useobject{currentmarker}{}%
\end{pgfscope}%
\end{pgfscope}%
\begin{pgfscope}%
\definecolor{textcolor}{rgb}{0.000000,0.000000,0.000000}%
\pgfsetstrokecolor{textcolor}%
\pgfsetfillcolor{textcolor}%
\pgftext[x=5.994576in,y=3.423148in,,top]{\color{textcolor}\rmfamily\fontsize{12.000000}{14.400000}\selectfont \(\displaystyle 80\)}%
\end{pgfscope}%
\begin{pgfscope}%
\pgfsetbuttcap%
\pgfsetroundjoin%
\definecolor{currentfill}{rgb}{0.000000,0.000000,0.000000}%
\pgfsetfillcolor{currentfill}%
\pgfsetlinewidth{0.803000pt}%
\definecolor{currentstroke}{rgb}{0.000000,0.000000,0.000000}%
\pgfsetstrokecolor{currentstroke}%
\pgfsetdash{}{0pt}%
\pgfsys@defobject{currentmarker}{\pgfqpoint{-0.048611in}{0.000000in}}{\pgfqpoint{0.000000in}{0.000000in}}{%
\pgfpathmoveto{\pgfqpoint{0.000000in}{0.000000in}}%
\pgfpathlineto{\pgfqpoint{-0.048611in}{0.000000in}}%
\pgfusepath{stroke,fill}%
}%
\begin{pgfscope}%
\pgfsys@transformshift{3.724747in}{3.637247in}%
\pgfsys@useobject{currentmarker}{}%
\end{pgfscope}%
\end{pgfscope}%
\begin{pgfscope}%
\definecolor{textcolor}{rgb}{0.000000,0.000000,0.000000}%
\pgfsetstrokecolor{textcolor}%
\pgfsetfillcolor{textcolor}%
\pgftext[x=3.419001in,y=3.579377in,left,base]{\color{textcolor}\rmfamily\fontsize{12.000000}{14.400000}\selectfont \(\displaystyle 0.0\)}%
\end{pgfscope}%
\begin{pgfscope}%
\pgfsetbuttcap%
\pgfsetroundjoin%
\definecolor{currentfill}{rgb}{0.000000,0.000000,0.000000}%
\pgfsetfillcolor{currentfill}%
\pgfsetlinewidth{0.803000pt}%
\definecolor{currentstroke}{rgb}{0.000000,0.000000,0.000000}%
\pgfsetstrokecolor{currentstroke}%
\pgfsetdash{}{0pt}%
\pgfsys@defobject{currentmarker}{\pgfqpoint{-0.048611in}{0.000000in}}{\pgfqpoint{0.000000in}{0.000000in}}{%
\pgfpathmoveto{\pgfqpoint{0.000000in}{0.000000in}}%
\pgfpathlineto{\pgfqpoint{-0.048611in}{0.000000in}}%
\pgfusepath{stroke,fill}%
}%
\begin{pgfscope}%
\pgfsys@transformshift{3.724747in}{4.104755in}%
\pgfsys@useobject{currentmarker}{}%
\end{pgfscope}%
\end{pgfscope}%
\begin{pgfscope}%
\definecolor{textcolor}{rgb}{0.000000,0.000000,0.000000}%
\pgfsetstrokecolor{textcolor}%
\pgfsetfillcolor{textcolor}%
\pgftext[x=3.419001in,y=4.046885in,left,base]{\color{textcolor}\rmfamily\fontsize{12.000000}{14.400000}\selectfont \(\displaystyle 0.2\)}%
\end{pgfscope}%
\begin{pgfscope}%
\pgfsetbuttcap%
\pgfsetroundjoin%
\definecolor{currentfill}{rgb}{0.000000,0.000000,0.000000}%
\pgfsetfillcolor{currentfill}%
\pgfsetlinewidth{0.803000pt}%
\definecolor{currentstroke}{rgb}{0.000000,0.000000,0.000000}%
\pgfsetstrokecolor{currentstroke}%
\pgfsetdash{}{0pt}%
\pgfsys@defobject{currentmarker}{\pgfqpoint{-0.048611in}{0.000000in}}{\pgfqpoint{0.000000in}{0.000000in}}{%
\pgfpathmoveto{\pgfqpoint{0.000000in}{0.000000in}}%
\pgfpathlineto{\pgfqpoint{-0.048611in}{0.000000in}}%
\pgfusepath{stroke,fill}%
}%
\begin{pgfscope}%
\pgfsys@transformshift{3.724747in}{4.572264in}%
\pgfsys@useobject{currentmarker}{}%
\end{pgfscope}%
\end{pgfscope}%
\begin{pgfscope}%
\definecolor{textcolor}{rgb}{0.000000,0.000000,0.000000}%
\pgfsetstrokecolor{textcolor}%
\pgfsetfillcolor{textcolor}%
\pgftext[x=3.419001in,y=4.514394in,left,base]{\color{textcolor}\rmfamily\fontsize{12.000000}{14.400000}\selectfont \(\displaystyle 0.4\)}%
\end{pgfscope}%
\begin{pgfscope}%
\pgfsetbuttcap%
\pgfsetroundjoin%
\definecolor{currentfill}{rgb}{0.000000,0.000000,0.000000}%
\pgfsetfillcolor{currentfill}%
\pgfsetlinewidth{0.803000pt}%
\definecolor{currentstroke}{rgb}{0.000000,0.000000,0.000000}%
\pgfsetstrokecolor{currentstroke}%
\pgfsetdash{}{0pt}%
\pgfsys@defobject{currentmarker}{\pgfqpoint{-0.048611in}{0.000000in}}{\pgfqpoint{0.000000in}{0.000000in}}{%
\pgfpathmoveto{\pgfqpoint{0.000000in}{0.000000in}}%
\pgfpathlineto{\pgfqpoint{-0.048611in}{0.000000in}}%
\pgfusepath{stroke,fill}%
}%
\begin{pgfscope}%
\pgfsys@transformshift{3.724747in}{5.039772in}%
\pgfsys@useobject{currentmarker}{}%
\end{pgfscope}%
\end{pgfscope}%
\begin{pgfscope}%
\definecolor{textcolor}{rgb}{0.000000,0.000000,0.000000}%
\pgfsetstrokecolor{textcolor}%
\pgfsetfillcolor{textcolor}%
\pgftext[x=3.419001in,y=4.981902in,left,base]{\color{textcolor}\rmfamily\fontsize{12.000000}{14.400000}\selectfont \(\displaystyle 0.6\)}%
\end{pgfscope}%
\begin{pgfscope}%
\pgfsetbuttcap%
\pgfsetroundjoin%
\definecolor{currentfill}{rgb}{0.000000,0.000000,0.000000}%
\pgfsetfillcolor{currentfill}%
\pgfsetlinewidth{0.803000pt}%
\definecolor{currentstroke}{rgb}{0.000000,0.000000,0.000000}%
\pgfsetstrokecolor{currentstroke}%
\pgfsetdash{}{0pt}%
\pgfsys@defobject{currentmarker}{\pgfqpoint{-0.048611in}{0.000000in}}{\pgfqpoint{0.000000in}{0.000000in}}{%
\pgfpathmoveto{\pgfqpoint{0.000000in}{0.000000in}}%
\pgfpathlineto{\pgfqpoint{-0.048611in}{0.000000in}}%
\pgfusepath{stroke,fill}%
}%
\begin{pgfscope}%
\pgfsys@transformshift{3.724747in}{5.507281in}%
\pgfsys@useobject{currentmarker}{}%
\end{pgfscope}%
\end{pgfscope}%
\begin{pgfscope}%
\definecolor{textcolor}{rgb}{0.000000,0.000000,0.000000}%
\pgfsetstrokecolor{textcolor}%
\pgfsetfillcolor{textcolor}%
\pgftext[x=3.419001in,y=5.449411in,left,base]{\color{textcolor}\rmfamily\fontsize{12.000000}{14.400000}\selectfont \(\displaystyle 0.8\)}%
\end{pgfscope}%
\begin{pgfscope}%
\pgfsetbuttcap%
\pgfsetroundjoin%
\definecolor{currentfill}{rgb}{0.000000,0.000000,0.000000}%
\pgfsetfillcolor{currentfill}%
\pgfsetlinewidth{0.803000pt}%
\definecolor{currentstroke}{rgb}{0.000000,0.000000,0.000000}%
\pgfsetstrokecolor{currentstroke}%
\pgfsetdash{}{0pt}%
\pgfsys@defobject{currentmarker}{\pgfqpoint{-0.048611in}{0.000000in}}{\pgfqpoint{0.000000in}{0.000000in}}{%
\pgfpathmoveto{\pgfqpoint{0.000000in}{0.000000in}}%
\pgfpathlineto{\pgfqpoint{-0.048611in}{0.000000in}}%
\pgfusepath{stroke,fill}%
}%
\begin{pgfscope}%
\pgfsys@transformshift{3.724747in}{5.974790in}%
\pgfsys@useobject{currentmarker}{}%
\end{pgfscope}%
\end{pgfscope}%
\begin{pgfscope}%
\definecolor{textcolor}{rgb}{0.000000,0.000000,0.000000}%
\pgfsetstrokecolor{textcolor}%
\pgfsetfillcolor{textcolor}%
\pgftext[x=3.419001in,y=5.916919in,left,base]{\color{textcolor}\rmfamily\fontsize{12.000000}{14.400000}\selectfont \(\displaystyle 1.0\)}%
\end{pgfscope}%
\begin{pgfscope}%
\pgfpathrectangle{\pgfqpoint{3.724747in}{3.520370in}}{\pgfqpoint{2.575253in}{2.571297in}}%
\pgfusepath{clip}%
\pgfsetrectcap%
\pgfsetroundjoin%
\pgfsetlinewidth{1.505625pt}%
\definecolor{currentstroke}{rgb}{0.121569,0.466667,0.705882}%
\pgfsetstrokecolor{currentstroke}%
\pgfsetdash{}{0pt}%
\pgfpathmoveto{\pgfqpoint{3.841804in}{4.637128in}}%
\pgfpathlineto{\pgfqpoint{3.868714in}{4.766472in}}%
\pgfpathlineto{\pgfqpoint{3.895623in}{4.932520in}}%
\pgfpathlineto{\pgfqpoint{3.922533in}{5.147968in}}%
\pgfpathlineto{\pgfqpoint{3.949443in}{5.395374in}}%
\pgfpathlineto{\pgfqpoint{3.976352in}{5.634678in}}%
\pgfpathlineto{\pgfqpoint{4.003262in}{5.821047in}}%
\pgfpathlineto{\pgfqpoint{4.030172in}{5.931557in}}%
\pgfpathlineto{\pgfqpoint{4.057081in}{5.974790in}}%
\pgfpathlineto{\pgfqpoint{4.083991in}{5.974392in}}%
\pgfpathlineto{\pgfqpoint{4.110901in}{5.951357in}}%
\pgfpathlineto{\pgfqpoint{4.137810in}{5.923824in}}%
\pgfpathlineto{\pgfqpoint{4.164720in}{5.907856in}}%
\pgfpathlineto{\pgfqpoint{4.191629in}{5.910941in}}%
\pgfpathlineto{\pgfqpoint{4.218539in}{5.926694in}}%
\pgfpathlineto{\pgfqpoint{4.245449in}{5.935254in}}%
\pgfpathlineto{\pgfqpoint{4.272358in}{5.913154in}}%
\pgfpathlineto{\pgfqpoint{4.299268in}{5.848329in}}%
\pgfpathlineto{\pgfqpoint{4.326178in}{5.744497in}}%
\pgfpathlineto{\pgfqpoint{4.353087in}{5.613160in}}%
\pgfpathlineto{\pgfqpoint{4.379997in}{5.467283in}}%
\pgfpathlineto{\pgfqpoint{4.406907in}{5.318002in}}%
\pgfpathlineto{\pgfqpoint{4.433816in}{5.171314in}}%
\pgfpathlineto{\pgfqpoint{4.460726in}{5.026725in}}%
\pgfpathlineto{\pgfqpoint{4.487636in}{4.880774in}}%
\pgfpathlineto{\pgfqpoint{4.514545in}{4.733117in}}%
\pgfpathlineto{\pgfqpoint{4.541455in}{4.587181in}}%
\pgfpathlineto{\pgfqpoint{4.568364in}{4.445361in}}%
\pgfpathlineto{\pgfqpoint{4.595274in}{4.307691in}}%
\pgfpathlineto{\pgfqpoint{4.622184in}{4.175957in}}%
\pgfpathlineto{\pgfqpoint{4.649093in}{4.052334in}}%
\pgfpathlineto{\pgfqpoint{4.676003in}{3.939104in}}%
\pgfpathlineto{\pgfqpoint{4.702913in}{3.838973in}}%
\pgfpathlineto{\pgfqpoint{4.729822in}{3.755661in}}%
\pgfpathlineto{\pgfqpoint{4.756732in}{3.693093in}}%
\pgfpathlineto{\pgfqpoint{4.783642in}{3.653594in}}%
\pgfpathlineto{\pgfqpoint{4.810551in}{3.637247in}}%
\pgfpathlineto{\pgfqpoint{4.837461in}{3.642318in}}%
\pgfpathlineto{\pgfqpoint{4.864371in}{3.664289in}}%
\pgfpathlineto{\pgfqpoint{4.891280in}{3.691531in}}%
\pgfpathlineto{\pgfqpoint{4.918190in}{3.706784in}}%
\pgfpathlineto{\pgfqpoint{4.945099in}{3.699480in}}%
\pgfpathlineto{\pgfqpoint{4.972009in}{3.674740in}}%
\pgfpathlineto{\pgfqpoint{4.998919in}{3.650685in}}%
\pgfpathlineto{\pgfqpoint{5.025828in}{3.646200in}}%
\pgfpathlineto{\pgfqpoint{5.052738in}{3.671194in}}%
\pgfpathlineto{\pgfqpoint{5.079648in}{3.724358in}}%
\pgfpathlineto{\pgfqpoint{5.106557in}{3.794246in}}%
\pgfpathlineto{\pgfqpoint{5.133467in}{3.866708in}}%
\pgfpathlineto{\pgfqpoint{5.160377in}{3.933561in}}%
\pgfpathlineto{\pgfqpoint{5.187286in}{3.994079in}}%
\pgfpathlineto{\pgfqpoint{5.214196in}{4.051848in}}%
\pgfpathlineto{\pgfqpoint{5.241106in}{4.110790in}}%
\pgfpathlineto{\pgfqpoint{5.268015in}{4.172580in}}%
\pgfpathlineto{\pgfqpoint{5.294925in}{4.236120in}}%
\pgfpathlineto{\pgfqpoint{5.321834in}{4.299012in}}%
\pgfpathlineto{\pgfqpoint{5.348744in}{4.358973in}}%
\pgfpathlineto{\pgfqpoint{5.375654in}{4.414604in}}%
\pgfpathlineto{\pgfqpoint{5.402563in}{4.465409in}}%
\pgfpathlineto{\pgfqpoint{5.429473in}{4.512456in}}%
\pgfpathlineto{\pgfqpoint{5.456383in}{4.565307in}}%
\pgfpathlineto{\pgfqpoint{5.483292in}{4.647636in}}%
\pgfpathlineto{\pgfqpoint{5.510202in}{4.788813in}}%
\pgfpathlineto{\pgfqpoint{5.537112in}{5.002596in}}%
\pgfpathlineto{\pgfqpoint{5.564021in}{5.269890in}}%
\pgfpathlineto{\pgfqpoint{5.590931in}{5.542303in}}%
\pgfpathlineto{\pgfqpoint{5.617841in}{5.765248in}}%
\pgfpathlineto{\pgfqpoint{5.644750in}{5.906239in}}%
\pgfpathlineto{\pgfqpoint{5.671660in}{5.966898in}}%
\pgfpathlineto{\pgfqpoint{5.698569in}{5.971865in}}%
\pgfpathlineto{\pgfqpoint{5.725479in}{5.949044in}}%
\pgfpathlineto{\pgfqpoint{5.752389in}{5.921769in}}%
\pgfpathlineto{\pgfqpoint{5.779298in}{5.907856in}}%
\pgfpathlineto{\pgfqpoint{5.806208in}{5.915100in}}%
\pgfpathlineto{\pgfqpoint{5.833118in}{5.935321in}}%
\pgfpathlineto{\pgfqpoint{5.860027in}{5.948090in}}%
\pgfpathlineto{\pgfqpoint{5.886937in}{5.933340in}}%
\pgfpathlineto{\pgfqpoint{5.913847in}{5.879888in}}%
\pgfpathlineto{\pgfqpoint{5.940756in}{5.787965in}}%
\pgfpathlineto{\pgfqpoint{5.967666in}{5.665572in}}%
\pgfpathlineto{\pgfqpoint{5.994576in}{5.523185in}}%
\pgfpathlineto{\pgfqpoint{6.021485in}{5.372685in}}%
\pgfpathlineto{\pgfqpoint{6.048395in}{5.224091in}}%
\pgfpathlineto{\pgfqpoint{6.075304in}{5.081757in}}%
\pgfpathlineto{\pgfqpoint{6.102214in}{4.943844in}}%
\pgfusepath{stroke}%
\end{pgfscope}%
\begin{pgfscope}%
\pgfpathrectangle{\pgfqpoint{3.724747in}{3.520370in}}{\pgfqpoint{2.575253in}{2.571297in}}%
\pgfusepath{clip}%
\pgfsetrectcap%
\pgfsetroundjoin%
\pgfsetlinewidth{1.505625pt}%
\definecolor{currentstroke}{rgb}{1.000000,0.498039,0.054902}%
\pgfsetstrokecolor{currentstroke}%
\pgfsetdash{}{0pt}%
\pgfpathmoveto{\pgfqpoint{3.841804in}{5.167727in}}%
\pgfpathlineto{\pgfqpoint{3.868714in}{5.175219in}}%
\pgfpathlineto{\pgfqpoint{3.895623in}{5.194793in}}%
\pgfpathlineto{\pgfqpoint{3.922533in}{5.244224in}}%
\pgfpathlineto{\pgfqpoint{3.949443in}{5.340199in}}%
\pgfpathlineto{\pgfqpoint{3.976352in}{5.482502in}}%
\pgfpathlineto{\pgfqpoint{4.003262in}{5.650871in}}%
\pgfpathlineto{\pgfqpoint{4.030172in}{5.807333in}}%
\pgfpathlineto{\pgfqpoint{4.057081in}{5.919703in}}%
\pgfpathlineto{\pgfqpoint{4.083991in}{5.973619in}}%
\pgfpathlineto{\pgfqpoint{4.110901in}{5.974790in}}%
\pgfpathlineto{\pgfqpoint{4.137810in}{5.940429in}}%
\pgfpathlineto{\pgfqpoint{4.164720in}{5.894834in}}%
\pgfpathlineto{\pgfqpoint{4.191629in}{5.852696in}}%
\pgfpathlineto{\pgfqpoint{4.218539in}{5.803392in}}%
\pgfpathlineto{\pgfqpoint{4.245449in}{5.719504in}}%
\pgfpathlineto{\pgfqpoint{4.272358in}{5.583546in}}%
\pgfpathlineto{\pgfqpoint{4.299268in}{5.409038in}}%
\pgfpathlineto{\pgfqpoint{4.326178in}{5.228486in}}%
\pgfpathlineto{\pgfqpoint{4.353087in}{5.068952in}}%
\pgfpathlineto{\pgfqpoint{4.379997in}{4.940162in}}%
\pgfpathlineto{\pgfqpoint{4.406907in}{4.827781in}}%
\pgfpathlineto{\pgfqpoint{4.433816in}{4.703186in}}%
\pgfpathlineto{\pgfqpoint{4.460726in}{4.548033in}}%
\pgfpathlineto{\pgfqpoint{4.487636in}{4.368836in}}%
\pgfpathlineto{\pgfqpoint{4.514545in}{4.184787in}}%
\pgfpathlineto{\pgfqpoint{4.541455in}{4.014267in}}%
\pgfpathlineto{\pgfqpoint{4.568364in}{3.867129in}}%
\pgfpathlineto{\pgfqpoint{4.595274in}{3.751438in}}%
\pgfpathlineto{\pgfqpoint{4.622184in}{3.673570in}}%
\pgfpathlineto{\pgfqpoint{4.649093in}{3.637247in}}%
\pgfpathlineto{\pgfqpoint{4.676003in}{3.639864in}}%
\pgfpathlineto{\pgfqpoint{4.702913in}{3.671959in}}%
\pgfpathlineto{\pgfqpoint{4.729822in}{3.725010in}}%
\pgfpathlineto{\pgfqpoint{4.756732in}{3.798043in}}%
\pgfpathlineto{\pgfqpoint{4.783642in}{3.896881in}}%
\pgfpathlineto{\pgfqpoint{4.810551in}{4.020553in}}%
\pgfpathlineto{\pgfqpoint{4.837461in}{4.150862in}}%
\pgfpathlineto{\pgfqpoint{4.864371in}{4.267203in}}%
\pgfpathlineto{\pgfqpoint{4.891280in}{4.374712in}}%
\pgfpathlineto{\pgfqpoint{4.918190in}{4.496981in}}%
\pgfpathlineto{\pgfqpoint{4.945099in}{4.640093in}}%
\pgfpathlineto{\pgfqpoint{4.972009in}{4.784307in}}%
\pgfpathlineto{\pgfqpoint{4.998919in}{4.905902in}}%
\pgfpathlineto{\pgfqpoint{5.025828in}{4.992584in}}%
\pgfpathlineto{\pgfqpoint{5.052738in}{5.050262in}}%
\pgfpathlineto{\pgfqpoint{5.079648in}{5.092280in}}%
\pgfpathlineto{\pgfqpoint{5.106557in}{5.124744in}}%
\pgfpathlineto{\pgfqpoint{5.133467in}{5.145424in}}%
\pgfpathlineto{\pgfqpoint{5.160377in}{5.151900in}}%
\pgfpathlineto{\pgfqpoint{5.187286in}{5.152265in}}%
\pgfpathlineto{\pgfqpoint{5.214196in}{5.160523in}}%
\pgfpathlineto{\pgfqpoint{5.241106in}{5.184470in}}%
\pgfpathlineto{\pgfqpoint{5.268015in}{5.219808in}}%
\pgfpathlineto{\pgfqpoint{5.294925in}{5.254399in}}%
\pgfpathlineto{\pgfqpoint{5.321834in}{5.276155in}}%
\pgfpathlineto{\pgfqpoint{5.348744in}{5.281374in}}%
\pgfpathlineto{\pgfqpoint{5.375654in}{5.275571in}}%
\pgfpathlineto{\pgfqpoint{5.402563in}{5.268901in}}%
\pgfpathlineto{\pgfqpoint{5.429473in}{5.270587in}}%
\pgfpathlineto{\pgfqpoint{5.456383in}{5.285738in}}%
\pgfpathlineto{\pgfqpoint{5.483292in}{5.320535in}}%
\pgfpathlineto{\pgfqpoint{5.510202in}{5.380048in}}%
\pgfpathlineto{\pgfqpoint{5.537112in}{5.462652in}}%
\pgfpathlineto{\pgfqpoint{5.564021in}{5.560201in}}%
\pgfpathlineto{\pgfqpoint{5.590931in}{5.663328in}}%
\pgfpathlineto{\pgfqpoint{5.617841in}{5.763002in}}%
\pgfpathlineto{\pgfqpoint{5.644750in}{5.847710in}}%
\pgfpathlineto{\pgfqpoint{5.671660in}{5.906673in}}%
\pgfpathlineto{\pgfqpoint{5.698569in}{5.940429in}}%
\pgfpathlineto{\pgfqpoint{5.725479in}{5.960143in}}%
\pgfpathlineto{\pgfqpoint{5.752389in}{5.971501in}}%
\pgfpathlineto{\pgfqpoint{5.779298in}{5.963467in}}%
\pgfpathlineto{\pgfqpoint{5.806208in}{5.915656in}}%
\pgfpathlineto{\pgfqpoint{5.833118in}{5.818967in}}%
\pgfpathlineto{\pgfqpoint{5.860027in}{5.685464in}}%
\pgfpathlineto{\pgfqpoint{5.886937in}{5.532690in}}%
\pgfpathlineto{\pgfqpoint{5.913847in}{5.365902in}}%
\pgfpathlineto{\pgfqpoint{5.940756in}{5.180759in}}%
\pgfpathlineto{\pgfqpoint{5.967666in}{4.972058in}}%
\pgfpathlineto{\pgfqpoint{5.994576in}{4.742742in}}%
\pgfusepath{stroke}%
\end{pgfscope}%
\begin{pgfscope}%
\pgfpathrectangle{\pgfqpoint{3.724747in}{3.520370in}}{\pgfqpoint{2.575253in}{2.571297in}}%
\pgfusepath{clip}%
\pgfsetrectcap%
\pgfsetroundjoin%
\pgfsetlinewidth{1.505625pt}%
\definecolor{currentstroke}{rgb}{0.172549,0.627451,0.172549}%
\pgfsetstrokecolor{currentstroke}%
\pgfsetdash{}{0pt}%
\pgfpathmoveto{\pgfqpoint{3.841804in}{5.117035in}}%
\pgfpathlineto{\pgfqpoint{3.868714in}{5.131951in}}%
\pgfpathlineto{\pgfqpoint{3.895623in}{5.145182in}}%
\pgfpathlineto{\pgfqpoint{3.922533in}{5.153645in}}%
\pgfpathlineto{\pgfqpoint{3.949443in}{5.156730in}}%
\pgfpathlineto{\pgfqpoint{3.976352in}{5.162410in}}%
\pgfpathlineto{\pgfqpoint{4.003262in}{5.186581in}}%
\pgfpathlineto{\pgfqpoint{4.030172in}{5.245347in}}%
\pgfpathlineto{\pgfqpoint{4.057081in}{5.344569in}}%
\pgfpathlineto{\pgfqpoint{4.083991in}{5.476623in}}%
\pgfpathlineto{\pgfqpoint{4.110901in}{5.626278in}}%
\pgfpathlineto{\pgfqpoint{4.137810in}{5.772812in}}%
\pgfpathlineto{\pgfqpoint{4.164720in}{5.890428in}}%
\pgfpathlineto{\pgfqpoint{4.191629in}{5.958522in}}%
\pgfpathlineto{\pgfqpoint{4.218539in}{5.974790in}}%
\pgfpathlineto{\pgfqpoint{4.245449in}{5.956270in}}%
\pgfpathlineto{\pgfqpoint{4.272358in}{5.922594in}}%
\pgfpathlineto{\pgfqpoint{4.299268in}{5.877992in}}%
\pgfpathlineto{\pgfqpoint{4.326178in}{5.811889in}}%
\pgfpathlineto{\pgfqpoint{4.353087in}{5.716314in}}%
\pgfpathlineto{\pgfqpoint{4.379997in}{5.595720in}}%
\pgfpathlineto{\pgfqpoint{4.406907in}{5.459015in}}%
\pgfpathlineto{\pgfqpoint{4.433816in}{5.310249in}}%
\pgfpathlineto{\pgfqpoint{4.460726in}{5.151435in}}%
\pgfpathlineto{\pgfqpoint{4.487636in}{4.988789in}}%
\pgfpathlineto{\pgfqpoint{4.514545in}{4.834020in}}%
\pgfpathlineto{\pgfqpoint{4.541455in}{4.695514in}}%
\pgfpathlineto{\pgfqpoint{4.568364in}{4.571002in}}%
\pgfpathlineto{\pgfqpoint{4.595274in}{4.450130in}}%
\pgfpathlineto{\pgfqpoint{4.622184in}{4.322611in}}%
\pgfpathlineto{\pgfqpoint{4.649093in}{4.184991in}}%
\pgfpathlineto{\pgfqpoint{4.676003in}{4.039658in}}%
\pgfpathlineto{\pgfqpoint{4.702913in}{3.894811in}}%
\pgfpathlineto{\pgfqpoint{4.729822in}{3.765657in}}%
\pgfpathlineto{\pgfqpoint{4.756732in}{3.672724in}}%
\pgfpathlineto{\pgfqpoint{4.783642in}{3.637247in}}%
\pgfpathlineto{\pgfqpoint{4.810551in}{3.673110in}}%
\pgfpathlineto{\pgfqpoint{4.837461in}{3.778390in}}%
\pgfpathlineto{\pgfqpoint{4.864371in}{3.929702in}}%
\pgfpathlineto{\pgfqpoint{4.891280in}{4.085599in}}%
\pgfpathlineto{\pgfqpoint{4.918190in}{4.202831in}}%
\pgfpathlineto{\pgfqpoint{4.945099in}{4.257538in}}%
\pgfpathlineto{\pgfqpoint{4.972009in}{4.255323in}}%
\pgfpathlineto{\pgfqpoint{4.998919in}{4.221468in}}%
\pgfpathlineto{\pgfqpoint{5.025828in}{4.186607in}}%
\pgfpathlineto{\pgfqpoint{5.052738in}{4.177525in}}%
\pgfpathlineto{\pgfqpoint{5.079648in}{4.206701in}}%
\pgfpathlineto{\pgfqpoint{5.106557in}{4.269095in}}%
\pgfpathlineto{\pgfqpoint{5.133467in}{4.361325in}}%
\pgfpathlineto{\pgfqpoint{5.160377in}{4.488569in}}%
\pgfpathlineto{\pgfqpoint{5.187286in}{4.654771in}}%
\pgfpathlineto{\pgfqpoint{5.214196in}{4.850213in}}%
\pgfpathlineto{\pgfqpoint{5.241106in}{5.046591in}}%
\pgfpathlineto{\pgfqpoint{5.268015in}{5.206128in}}%
\pgfpathlineto{\pgfqpoint{5.294925in}{5.302030in}}%
\pgfpathlineto{\pgfqpoint{5.321834in}{5.333973in}}%
\pgfpathlineto{\pgfqpoint{5.348744in}{5.322792in}}%
\pgfpathlineto{\pgfqpoint{5.375654in}{5.292899in}}%
\pgfpathlineto{\pgfqpoint{5.402563in}{5.261804in}}%
\pgfpathlineto{\pgfqpoint{5.429473in}{5.237727in}}%
\pgfpathlineto{\pgfqpoint{5.456383in}{5.220653in}}%
\pgfpathlineto{\pgfqpoint{5.483292in}{5.208054in}}%
\pgfpathlineto{\pgfqpoint{5.510202in}{5.200555in}}%
\pgfpathlineto{\pgfqpoint{5.537112in}{5.199685in}}%
\pgfpathlineto{\pgfqpoint{5.564021in}{5.203701in}}%
\pgfpathlineto{\pgfqpoint{5.590931in}{5.207633in}}%
\pgfpathlineto{\pgfqpoint{5.617841in}{5.207847in}}%
\pgfpathlineto{\pgfqpoint{5.644750in}{5.206013in}}%
\pgfpathlineto{\pgfqpoint{5.671660in}{5.211414in}}%
\pgfpathlineto{\pgfqpoint{5.698569in}{5.240744in}}%
\pgfpathlineto{\pgfqpoint{5.725479in}{5.309024in}}%
\pgfpathlineto{\pgfqpoint{5.752389in}{5.419386in}}%
\pgfpathlineto{\pgfqpoint{5.779298in}{5.560893in}}%
\pgfpathlineto{\pgfqpoint{5.806208in}{5.710934in}}%
\pgfpathlineto{\pgfqpoint{5.833118in}{5.834172in}}%
\pgfpathlineto{\pgfqpoint{5.860027in}{5.890428in}}%
\pgfpathlineto{\pgfqpoint{5.886937in}{5.863393in}}%
\pgfpathlineto{\pgfqpoint{5.913847in}{5.775198in}}%
\pgfpathlineto{\pgfqpoint{5.940756in}{5.671712in}}%
\pgfpathlineto{\pgfqpoint{5.967666in}{5.586053in}}%
\pgfpathlineto{\pgfqpoint{5.994576in}{5.513666in}}%
\pgfpathlineto{\pgfqpoint{6.021485in}{5.428587in}}%
\pgfpathlineto{\pgfqpoint{6.048395in}{5.313759in}}%
\pgfpathlineto{\pgfqpoint{6.075304in}{5.169652in}}%
\pgfpathlineto{\pgfqpoint{6.102214in}{5.005061in}}%
\pgfpathlineto{\pgfqpoint{6.129124in}{4.826768in}}%
\pgfpathlineto{\pgfqpoint{6.156033in}{4.642917in}}%
\pgfpathlineto{\pgfqpoint{6.182943in}{4.457568in}}%
\pgfusepath{stroke}%
\end{pgfscope}%
\begin{pgfscope}%
\pgfsetrectcap%
\pgfsetmiterjoin%
\pgfsetlinewidth{0.803000pt}%
\definecolor{currentstroke}{rgb}{0.000000,0.000000,0.000000}%
\pgfsetstrokecolor{currentstroke}%
\pgfsetdash{}{0pt}%
\pgfpathmoveto{\pgfqpoint{3.724747in}{3.520370in}}%
\pgfpathlineto{\pgfqpoint{3.724747in}{6.091667in}}%
\pgfusepath{stroke}%
\end{pgfscope}%
\begin{pgfscope}%
\pgfsetrectcap%
\pgfsetmiterjoin%
\pgfsetlinewidth{0.803000pt}%
\definecolor{currentstroke}{rgb}{0.000000,0.000000,0.000000}%
\pgfsetstrokecolor{currentstroke}%
\pgfsetdash{}{0pt}%
\pgfpathmoveto{\pgfqpoint{6.300000in}{3.520370in}}%
\pgfpathlineto{\pgfqpoint{6.300000in}{6.091667in}}%
\pgfusepath{stroke}%
\end{pgfscope}%
\begin{pgfscope}%
\pgfsetrectcap%
\pgfsetmiterjoin%
\pgfsetlinewidth{0.803000pt}%
\definecolor{currentstroke}{rgb}{0.000000,0.000000,0.000000}%
\pgfsetstrokecolor{currentstroke}%
\pgfsetdash{}{0pt}%
\pgfpathmoveto{\pgfqpoint{3.724747in}{3.520370in}}%
\pgfpathlineto{\pgfqpoint{6.300000in}{3.520370in}}%
\pgfusepath{stroke}%
\end{pgfscope}%
\begin{pgfscope}%
\pgfsetrectcap%
\pgfsetmiterjoin%
\pgfsetlinewidth{0.803000pt}%
\definecolor{currentstroke}{rgb}{0.000000,0.000000,0.000000}%
\pgfsetstrokecolor{currentstroke}%
\pgfsetdash{}{0pt}%
\pgfpathmoveto{\pgfqpoint{3.724747in}{6.091667in}}%
\pgfpathlineto{\pgfqpoint{6.300000in}{6.091667in}}%
\pgfusepath{stroke}%
\end{pgfscope}%
\begin{pgfscope}%
\definecolor{textcolor}{rgb}{0.000000,0.000000,0.000000}%
\pgfsetstrokecolor{textcolor}%
\pgfsetfillcolor{textcolor}%
\pgftext[x=5.012374in,y=6.175000in,,base]{\color{textcolor}\rmfamily\fontsize{12.000000}{14.400000}\selectfont (b) Normalized}%
\end{pgfscope}%
\begin{pgfscope}%
\pgfsetbuttcap%
\pgfsetmiterjoin%
\definecolor{currentfill}{rgb}{1.000000,1.000000,1.000000}%
\pgfsetfillcolor{currentfill}%
\pgfsetlinewidth{0.000000pt}%
\definecolor{currentstroke}{rgb}{0.000000,0.000000,0.000000}%
\pgfsetstrokecolor{currentstroke}%
\pgfsetstrokeopacity{0.000000}%
\pgfsetdash{}{0pt}%
\pgfpathmoveto{\pgfqpoint{0.693748in}{0.345370in}}%
\pgfpathlineto{\pgfqpoint{3.269001in}{0.345370in}}%
\pgfpathlineto{\pgfqpoint{3.269001in}{2.916667in}}%
\pgfpathlineto{\pgfqpoint{0.693748in}{2.916667in}}%
\pgfpathclose%
\pgfusepath{fill}%
\end{pgfscope}%
\begin{pgfscope}%
\pgfsetbuttcap%
\pgfsetroundjoin%
\definecolor{currentfill}{rgb}{0.000000,0.000000,0.000000}%
\pgfsetfillcolor{currentfill}%
\pgfsetlinewidth{0.803000pt}%
\definecolor{currentstroke}{rgb}{0.000000,0.000000,0.000000}%
\pgfsetstrokecolor{currentstroke}%
\pgfsetdash{}{0pt}%
\pgfsys@defobject{currentmarker}{\pgfqpoint{0.000000in}{-0.048611in}}{\pgfqpoint{0.000000in}{0.000000in}}{%
\pgfpathmoveto{\pgfqpoint{0.000000in}{0.000000in}}%
\pgfpathlineto{\pgfqpoint{0.000000in}{-0.048611in}}%
\pgfusepath{stroke,fill}%
}%
\begin{pgfscope}%
\pgfsys@transformshift{0.810805in}{0.345370in}%
\pgfsys@useobject{currentmarker}{}%
\end{pgfscope}%
\end{pgfscope}%
\begin{pgfscope}%
\definecolor{textcolor}{rgb}{0.000000,0.000000,0.000000}%
\pgfsetstrokecolor{textcolor}%
\pgfsetfillcolor{textcolor}%
\pgftext[x=0.810805in,y=0.248148in,,top]{\color{textcolor}\rmfamily\fontsize{12.000000}{14.400000}\selectfont \(\displaystyle 0\)}%
\end{pgfscope}%
\begin{pgfscope}%
\pgfsetbuttcap%
\pgfsetroundjoin%
\definecolor{currentfill}{rgb}{0.000000,0.000000,0.000000}%
\pgfsetfillcolor{currentfill}%
\pgfsetlinewidth{0.803000pt}%
\definecolor{currentstroke}{rgb}{0.000000,0.000000,0.000000}%
\pgfsetstrokecolor{currentstroke}%
\pgfsetdash{}{0pt}%
\pgfsys@defobject{currentmarker}{\pgfqpoint{0.000000in}{-0.048611in}}{\pgfqpoint{0.000000in}{0.000000in}}{%
\pgfpathmoveto{\pgfqpoint{0.000000in}{0.000000in}}%
\pgfpathlineto{\pgfqpoint{0.000000in}{-0.048611in}}%
\pgfusepath{stroke,fill}%
}%
\begin{pgfscope}%
\pgfsys@transformshift{1.348998in}{0.345370in}%
\pgfsys@useobject{currentmarker}{}%
\end{pgfscope}%
\end{pgfscope}%
\begin{pgfscope}%
\definecolor{textcolor}{rgb}{0.000000,0.000000,0.000000}%
\pgfsetstrokecolor{textcolor}%
\pgfsetfillcolor{textcolor}%
\pgftext[x=1.348998in,y=0.248148in,,top]{\color{textcolor}\rmfamily\fontsize{12.000000}{14.400000}\selectfont \(\displaystyle 20\)}%
\end{pgfscope}%
\begin{pgfscope}%
\pgfsetbuttcap%
\pgfsetroundjoin%
\definecolor{currentfill}{rgb}{0.000000,0.000000,0.000000}%
\pgfsetfillcolor{currentfill}%
\pgfsetlinewidth{0.803000pt}%
\definecolor{currentstroke}{rgb}{0.000000,0.000000,0.000000}%
\pgfsetstrokecolor{currentstroke}%
\pgfsetdash{}{0pt}%
\pgfsys@defobject{currentmarker}{\pgfqpoint{0.000000in}{-0.048611in}}{\pgfqpoint{0.000000in}{0.000000in}}{%
\pgfpathmoveto{\pgfqpoint{0.000000in}{0.000000in}}%
\pgfpathlineto{\pgfqpoint{0.000000in}{-0.048611in}}%
\pgfusepath{stroke,fill}%
}%
\begin{pgfscope}%
\pgfsys@transformshift{1.887191in}{0.345370in}%
\pgfsys@useobject{currentmarker}{}%
\end{pgfscope}%
\end{pgfscope}%
\begin{pgfscope}%
\definecolor{textcolor}{rgb}{0.000000,0.000000,0.000000}%
\pgfsetstrokecolor{textcolor}%
\pgfsetfillcolor{textcolor}%
\pgftext[x=1.887191in,y=0.248148in,,top]{\color{textcolor}\rmfamily\fontsize{12.000000}{14.400000}\selectfont \(\displaystyle 40\)}%
\end{pgfscope}%
\begin{pgfscope}%
\pgfsetbuttcap%
\pgfsetroundjoin%
\definecolor{currentfill}{rgb}{0.000000,0.000000,0.000000}%
\pgfsetfillcolor{currentfill}%
\pgfsetlinewidth{0.803000pt}%
\definecolor{currentstroke}{rgb}{0.000000,0.000000,0.000000}%
\pgfsetstrokecolor{currentstroke}%
\pgfsetdash{}{0pt}%
\pgfsys@defobject{currentmarker}{\pgfqpoint{0.000000in}{-0.048611in}}{\pgfqpoint{0.000000in}{0.000000in}}{%
\pgfpathmoveto{\pgfqpoint{0.000000in}{0.000000in}}%
\pgfpathlineto{\pgfqpoint{0.000000in}{-0.048611in}}%
\pgfusepath{stroke,fill}%
}%
\begin{pgfscope}%
\pgfsys@transformshift{2.425384in}{0.345370in}%
\pgfsys@useobject{currentmarker}{}%
\end{pgfscope}%
\end{pgfscope}%
\begin{pgfscope}%
\definecolor{textcolor}{rgb}{0.000000,0.000000,0.000000}%
\pgfsetstrokecolor{textcolor}%
\pgfsetfillcolor{textcolor}%
\pgftext[x=2.425384in,y=0.248148in,,top]{\color{textcolor}\rmfamily\fontsize{12.000000}{14.400000}\selectfont \(\displaystyle 60\)}%
\end{pgfscope}%
\begin{pgfscope}%
\pgfsetbuttcap%
\pgfsetroundjoin%
\definecolor{currentfill}{rgb}{0.000000,0.000000,0.000000}%
\pgfsetfillcolor{currentfill}%
\pgfsetlinewidth{0.803000pt}%
\definecolor{currentstroke}{rgb}{0.000000,0.000000,0.000000}%
\pgfsetstrokecolor{currentstroke}%
\pgfsetdash{}{0pt}%
\pgfsys@defobject{currentmarker}{\pgfqpoint{0.000000in}{-0.048611in}}{\pgfqpoint{0.000000in}{0.000000in}}{%
\pgfpathmoveto{\pgfqpoint{0.000000in}{0.000000in}}%
\pgfpathlineto{\pgfqpoint{0.000000in}{-0.048611in}}%
\pgfusepath{stroke,fill}%
}%
\begin{pgfscope}%
\pgfsys@transformshift{2.963576in}{0.345370in}%
\pgfsys@useobject{currentmarker}{}%
\end{pgfscope}%
\end{pgfscope}%
\begin{pgfscope}%
\definecolor{textcolor}{rgb}{0.000000,0.000000,0.000000}%
\pgfsetstrokecolor{textcolor}%
\pgfsetfillcolor{textcolor}%
\pgftext[x=2.963576in,y=0.248148in,,top]{\color{textcolor}\rmfamily\fontsize{12.000000}{14.400000}\selectfont \(\displaystyle 80\)}%
\end{pgfscope}%
\begin{pgfscope}%
\pgfsetbuttcap%
\pgfsetroundjoin%
\definecolor{currentfill}{rgb}{0.000000,0.000000,0.000000}%
\pgfsetfillcolor{currentfill}%
\pgfsetlinewidth{0.803000pt}%
\definecolor{currentstroke}{rgb}{0.000000,0.000000,0.000000}%
\pgfsetstrokecolor{currentstroke}%
\pgfsetdash{}{0pt}%
\pgfsys@defobject{currentmarker}{\pgfqpoint{-0.048611in}{0.000000in}}{\pgfqpoint{0.000000in}{0.000000in}}{%
\pgfpathmoveto{\pgfqpoint{0.000000in}{0.000000in}}%
\pgfpathlineto{\pgfqpoint{-0.048611in}{0.000000in}}%
\pgfusepath{stroke,fill}%
}%
\begin{pgfscope}%
\pgfsys@transformshift{0.693748in}{0.580886in}%
\pgfsys@useobject{currentmarker}{}%
\end{pgfscope}%
\end{pgfscope}%
\begin{pgfscope}%
\definecolor{textcolor}{rgb}{0.000000,0.000000,0.000000}%
\pgfsetstrokecolor{textcolor}%
\pgfsetfillcolor{textcolor}%
\pgftext[x=0.385299in,y=0.523016in,left,base]{\color{textcolor}\rmfamily\fontsize{12.000000}{14.400000}\selectfont \(\displaystyle -2\)}%
\end{pgfscope}%
\begin{pgfscope}%
\pgfsetbuttcap%
\pgfsetroundjoin%
\definecolor{currentfill}{rgb}{0.000000,0.000000,0.000000}%
\pgfsetfillcolor{currentfill}%
\pgfsetlinewidth{0.803000pt}%
\definecolor{currentstroke}{rgb}{0.000000,0.000000,0.000000}%
\pgfsetstrokecolor{currentstroke}%
\pgfsetdash{}{0pt}%
\pgfsys@defobject{currentmarker}{\pgfqpoint{-0.048611in}{0.000000in}}{\pgfqpoint{0.000000in}{0.000000in}}{%
\pgfpathmoveto{\pgfqpoint{0.000000in}{0.000000in}}%
\pgfpathlineto{\pgfqpoint{-0.048611in}{0.000000in}}%
\pgfusepath{stroke,fill}%
}%
\begin{pgfscope}%
\pgfsys@transformshift{0.693748in}{1.213680in}%
\pgfsys@useobject{currentmarker}{}%
\end{pgfscope}%
\end{pgfscope}%
\begin{pgfscope}%
\definecolor{textcolor}{rgb}{0.000000,0.000000,0.000000}%
\pgfsetstrokecolor{textcolor}%
\pgfsetfillcolor{textcolor}%
\pgftext[x=0.385299in,y=1.155810in,left,base]{\color{textcolor}\rmfamily\fontsize{12.000000}{14.400000}\selectfont \(\displaystyle -1\)}%
\end{pgfscope}%
\begin{pgfscope}%
\pgfsetbuttcap%
\pgfsetroundjoin%
\definecolor{currentfill}{rgb}{0.000000,0.000000,0.000000}%
\pgfsetfillcolor{currentfill}%
\pgfsetlinewidth{0.803000pt}%
\definecolor{currentstroke}{rgb}{0.000000,0.000000,0.000000}%
\pgfsetstrokecolor{currentstroke}%
\pgfsetdash{}{0pt}%
\pgfsys@defobject{currentmarker}{\pgfqpoint{-0.048611in}{0.000000in}}{\pgfqpoint{0.000000in}{0.000000in}}{%
\pgfpathmoveto{\pgfqpoint{0.000000in}{0.000000in}}%
\pgfpathlineto{\pgfqpoint{-0.048611in}{0.000000in}}%
\pgfusepath{stroke,fill}%
}%
\begin{pgfscope}%
\pgfsys@transformshift{0.693748in}{1.846474in}%
\pgfsys@useobject{currentmarker}{}%
\end{pgfscope}%
\end{pgfscope}%
\begin{pgfscope}%
\definecolor{textcolor}{rgb}{0.000000,0.000000,0.000000}%
\pgfsetstrokecolor{textcolor}%
\pgfsetfillcolor{textcolor}%
\pgftext[x=0.514929in,y=1.788604in,left,base]{\color{textcolor}\rmfamily\fontsize{12.000000}{14.400000}\selectfont \(\displaystyle 0\)}%
\end{pgfscope}%
\begin{pgfscope}%
\pgfsetbuttcap%
\pgfsetroundjoin%
\definecolor{currentfill}{rgb}{0.000000,0.000000,0.000000}%
\pgfsetfillcolor{currentfill}%
\pgfsetlinewidth{0.803000pt}%
\definecolor{currentstroke}{rgb}{0.000000,0.000000,0.000000}%
\pgfsetstrokecolor{currentstroke}%
\pgfsetdash{}{0pt}%
\pgfsys@defobject{currentmarker}{\pgfqpoint{-0.048611in}{0.000000in}}{\pgfqpoint{0.000000in}{0.000000in}}{%
\pgfpathmoveto{\pgfqpoint{0.000000in}{0.000000in}}%
\pgfpathlineto{\pgfqpoint{-0.048611in}{0.000000in}}%
\pgfusepath{stroke,fill}%
}%
\begin{pgfscope}%
\pgfsys@transformshift{0.693748in}{2.479268in}%
\pgfsys@useobject{currentmarker}{}%
\end{pgfscope}%
\end{pgfscope}%
\begin{pgfscope}%
\definecolor{textcolor}{rgb}{0.000000,0.000000,0.000000}%
\pgfsetstrokecolor{textcolor}%
\pgfsetfillcolor{textcolor}%
\pgftext[x=0.514929in,y=2.421398in,left,base]{\color{textcolor}\rmfamily\fontsize{12.000000}{14.400000}\selectfont \(\displaystyle 1\)}%
\end{pgfscope}%
\begin{pgfscope}%
\definecolor{textcolor}{rgb}{0.000000,0.000000,0.000000}%
\pgfsetstrokecolor{textcolor}%
\pgfsetfillcolor{textcolor}%
\pgftext[x=0.329744in,y=1.631018in,,bottom,rotate=90.000000]{\color{textcolor}\rmfamily\fontsize{12.000000}{14.400000}\selectfont Strain}%
\end{pgfscope}%
\begin{pgfscope}%
\pgfpathrectangle{\pgfqpoint{0.693748in}{0.345370in}}{\pgfqpoint{2.575253in}{2.571297in}}%
\pgfusepath{clip}%
\pgfsetrectcap%
\pgfsetroundjoin%
\pgfsetlinewidth{1.505625pt}%
\definecolor{currentstroke}{rgb}{0.121569,0.466667,0.705882}%
\pgfsetstrokecolor{currentstroke}%
\pgfsetdash{}{0pt}%
\pgfpathmoveto{\pgfqpoint{0.810805in}{1.649095in}}%
\pgfpathlineto{\pgfqpoint{0.837715in}{1.744202in}}%
\pgfpathlineto{\pgfqpoint{0.864624in}{1.866297in}}%
\pgfpathlineto{\pgfqpoint{0.891534in}{2.024716in}}%
\pgfpathlineto{\pgfqpoint{0.918444in}{2.206635in}}%
\pgfpathlineto{\pgfqpoint{0.945353in}{2.382595in}}%
\pgfpathlineto{\pgfqpoint{0.972263in}{2.519633in}}%
\pgfpathlineto{\pgfqpoint{0.999173in}{2.600891in}}%
\pgfpathlineto{\pgfqpoint{1.026082in}{2.632680in}}%
\pgfpathlineto{\pgfqpoint{1.052992in}{2.632387in}}%
\pgfpathlineto{\pgfqpoint{1.079901in}{2.615450in}}%
\pgfpathlineto{\pgfqpoint{1.106811in}{2.595205in}}%
\pgfpathlineto{\pgfqpoint{1.133721in}{2.583463in}}%
\pgfpathlineto{\pgfqpoint{1.160630in}{2.585732in}}%
\pgfpathlineto{\pgfqpoint{1.187540in}{2.597316in}}%
\pgfpathlineto{\pgfqpoint{1.214450in}{2.603609in}}%
\pgfpathlineto{\pgfqpoint{1.241359in}{2.587359in}}%
\pgfpathlineto{\pgfqpoint{1.268269in}{2.539694in}}%
\pgfpathlineto{\pgfqpoint{1.295179in}{2.463345in}}%
\pgfpathlineto{\pgfqpoint{1.322088in}{2.366773in}}%
\pgfpathlineto{\pgfqpoint{1.348998in}{2.259510in}}%
\pgfpathlineto{\pgfqpoint{1.375908in}{2.149743in}}%
\pgfpathlineto{\pgfqpoint{1.402817in}{2.041883in}}%
\pgfpathlineto{\pgfqpoint{1.429727in}{1.935566in}}%
\pgfpathlineto{\pgfqpoint{1.456636in}{1.828249in}}%
\pgfpathlineto{\pgfqpoint{1.483546in}{1.719676in}}%
\pgfpathlineto{\pgfqpoint{1.510456in}{1.612369in}}%
\pgfpathlineto{\pgfqpoint{1.537365in}{1.508088in}}%
\pgfpathlineto{\pgfqpoint{1.564275in}{1.406859in}}%
\pgfpathlineto{\pgfqpoint{1.591185in}{1.309995in}}%
\pgfpathlineto{\pgfqpoint{1.618094in}{1.219094in}}%
\pgfpathlineto{\pgfqpoint{1.645004in}{1.135836in}}%
\pgfpathlineto{\pgfqpoint{1.671914in}{1.062210in}}%
\pgfpathlineto{\pgfqpoint{1.698823in}{1.000951in}}%
\pgfpathlineto{\pgfqpoint{1.725733in}{0.954944in}}%
\pgfpathlineto{\pgfqpoint{1.752643in}{0.925900in}}%
\pgfpathlineto{\pgfqpoint{1.779552in}{0.913880in}}%
\pgfpathlineto{\pgfqpoint{1.806462in}{0.917609in}}%
\pgfpathlineto{\pgfqpoint{1.833371in}{0.933764in}}%
\pgfpathlineto{\pgfqpoint{1.860281in}{0.953796in}}%
\pgfpathlineto{\pgfqpoint{1.887191in}{0.965011in}}%
\pgfpathlineto{\pgfqpoint{1.914100in}{0.959640in}}%
\pgfpathlineto{\pgfqpoint{1.941010in}{0.941449in}}%
\pgfpathlineto{\pgfqpoint{1.967920in}{0.923761in}}%
\pgfpathlineto{\pgfqpoint{1.994829in}{0.920464in}}%
\pgfpathlineto{\pgfqpoint{2.021739in}{0.938842in}}%
\pgfpathlineto{\pgfqpoint{2.048649in}{0.977933in}}%
\pgfpathlineto{\pgfqpoint{2.075558in}{1.029322in}}%
\pgfpathlineto{\pgfqpoint{2.102468in}{1.082604in}}%
\pgfpathlineto{\pgfqpoint{2.129378in}{1.131761in}}%
\pgfpathlineto{\pgfqpoint{2.156287in}{1.176260in}}%
\pgfpathlineto{\pgfqpoint{2.183197in}{1.218737in}}%
\pgfpathlineto{\pgfqpoint{2.210106in}{1.262077in}}%
\pgfpathlineto{\pgfqpoint{2.237016in}{1.307512in}}%
\pgfpathlineto{\pgfqpoint{2.263926in}{1.354233in}}%
\pgfpathlineto{\pgfqpoint{2.290835in}{1.400477in}}%
\pgfpathlineto{\pgfqpoint{2.317745in}{1.444567in}}%
\pgfpathlineto{\pgfqpoint{2.344655in}{1.485473in}}%
\pgfpathlineto{\pgfqpoint{2.371564in}{1.522830in}}%
\pgfpathlineto{\pgfqpoint{2.398474in}{1.557423in}}%
\pgfpathlineto{\pgfqpoint{2.425384in}{1.596284in}}%
\pgfpathlineto{\pgfqpoint{2.452293in}{1.656822in}}%
\pgfpathlineto{\pgfqpoint{2.479203in}{1.760629in}}%
\pgfpathlineto{\pgfqpoint{2.506113in}{1.917824in}}%
\pgfpathlineto{\pgfqpoint{2.533022in}{2.114366in}}%
\pgfpathlineto{\pgfqpoint{2.559932in}{2.314671in}}%
\pgfpathlineto{\pgfqpoint{2.586841in}{2.478603in}}%
\pgfpathlineto{\pgfqpoint{2.613751in}{2.582275in}}%
\pgfpathlineto{\pgfqpoint{2.640661in}{2.626878in}}%
\pgfpathlineto{\pgfqpoint{2.667570in}{2.630529in}}%
\pgfpathlineto{\pgfqpoint{2.694480in}{2.613749in}}%
\pgfpathlineto{\pgfqpoint{2.721390in}{2.593694in}}%
\pgfpathlineto{\pgfqpoint{2.748299in}{2.583463in}}%
\pgfpathlineto{\pgfqpoint{2.775209in}{2.588790in}}%
\pgfpathlineto{\pgfqpoint{2.802119in}{2.603659in}}%
\pgfpathlineto{\pgfqpoint{2.829028in}{2.613048in}}%
\pgfpathlineto{\pgfqpoint{2.855938in}{2.602202in}}%
\pgfpathlineto{\pgfqpoint{2.882848in}{2.562898in}}%
\pgfpathlineto{\pgfqpoint{2.909757in}{2.495308in}}%
\pgfpathlineto{\pgfqpoint{2.936667in}{2.405312in}}%
\pgfpathlineto{\pgfqpoint{2.963576in}{2.300615in}}%
\pgfpathlineto{\pgfqpoint{2.990486in}{2.189952in}}%
\pgfpathlineto{\pgfqpoint{3.017396in}{2.080690in}}%
\pgfpathlineto{\pgfqpoint{3.044305in}{1.976032in}}%
\pgfpathlineto{\pgfqpoint{3.071215in}{1.874624in}}%
\pgfusepath{stroke}%
\end{pgfscope}%
\begin{pgfscope}%
\pgfpathrectangle{\pgfqpoint{0.693748in}{0.345370in}}{\pgfqpoint{2.575253in}{2.571297in}}%
\pgfusepath{clip}%
\pgfsetrectcap%
\pgfsetroundjoin%
\pgfsetlinewidth{1.505625pt}%
\definecolor{currentstroke}{rgb}{1.000000,0.498039,0.054902}%
\pgfsetstrokecolor{currentstroke}%
\pgfsetdash{}{0pt}%
\pgfpathmoveto{\pgfqpoint{0.810805in}{1.916529in}}%
\pgfpathlineto{\pgfqpoint{0.837715in}{1.923471in}}%
\pgfpathlineto{\pgfqpoint{0.864624in}{1.941607in}}%
\pgfpathlineto{\pgfqpoint{0.891534in}{1.987407in}}%
\pgfpathlineto{\pgfqpoint{0.918444in}{2.076332in}}%
\pgfpathlineto{\pgfqpoint{0.945353in}{2.208183in}}%
\pgfpathlineto{\pgfqpoint{0.972263in}{2.364184in}}%
\pgfpathlineto{\pgfqpoint{0.999173in}{2.509152in}}%
\pgfpathlineto{\pgfqpoint{1.026082in}{2.613268in}}%
\pgfpathlineto{\pgfqpoint{1.052992in}{2.663224in}}%
\pgfpathlineto{\pgfqpoint{1.079901in}{2.664309in}}%
\pgfpathlineto{\pgfqpoint{1.106811in}{2.632472in}}%
\pgfpathlineto{\pgfqpoint{1.133721in}{2.590226in}}%
\pgfpathlineto{\pgfqpoint{1.160630in}{2.551183in}}%
\pgfpathlineto{\pgfqpoint{1.187540in}{2.505501in}}%
\pgfpathlineto{\pgfqpoint{1.214450in}{2.427775in}}%
\pgfpathlineto{\pgfqpoint{1.241359in}{2.301804in}}%
\pgfpathlineto{\pgfqpoint{1.268269in}{2.140115in}}%
\pgfpathlineto{\pgfqpoint{1.295179in}{1.972825in}}%
\pgfpathlineto{\pgfqpoint{1.322088in}{1.825009in}}%
\pgfpathlineto{\pgfqpoint{1.348998in}{1.705680in}}%
\pgfpathlineto{\pgfqpoint{1.375908in}{1.601554in}}%
\pgfpathlineto{\pgfqpoint{1.402817in}{1.486111in}}%
\pgfpathlineto{\pgfqpoint{1.429727in}{1.342355in}}%
\pgfpathlineto{\pgfqpoint{1.456636in}{1.176321in}}%
\pgfpathlineto{\pgfqpoint{1.483546in}{1.005791in}}%
\pgfpathlineto{\pgfqpoint{1.510456in}{0.847797in}}%
\pgfpathlineto{\pgfqpoint{1.537365in}{0.711467in}}%
\pgfpathlineto{\pgfqpoint{1.564275in}{0.604274in}}%
\pgfpathlineto{\pgfqpoint{1.591185in}{0.532126in}}%
\pgfpathlineto{\pgfqpoint{1.618094in}{0.498471in}}%
\pgfpathlineto{\pgfqpoint{1.645004in}{0.500896in}}%
\pgfpathlineto{\pgfqpoint{1.671914in}{0.530633in}}%
\pgfpathlineto{\pgfqpoint{1.698823in}{0.579787in}}%
\pgfpathlineto{\pgfqpoint{1.725733in}{0.647456in}}%
\pgfpathlineto{\pgfqpoint{1.752643in}{0.739033in}}%
\pgfpathlineto{\pgfqpoint{1.779552in}{0.853621in}}%
\pgfpathlineto{\pgfqpoint{1.806462in}{0.974358in}}%
\pgfpathlineto{\pgfqpoint{1.833371in}{1.082153in}}%
\pgfpathlineto{\pgfqpoint{1.860281in}{1.181765in}}%
\pgfpathlineto{\pgfqpoint{1.887191in}{1.295053in}}%
\pgfpathlineto{\pgfqpoint{1.914100in}{1.427653in}}%
\pgfpathlineto{\pgfqpoint{1.941010in}{1.561273in}}%
\pgfpathlineto{\pgfqpoint{1.967920in}{1.673936in}}%
\pgfpathlineto{\pgfqpoint{1.994829in}{1.754252in}}%
\pgfpathlineto{\pgfqpoint{2.021739in}{1.807693in}}%
\pgfpathlineto{\pgfqpoint{2.048649in}{1.846624in}}%
\pgfpathlineto{\pgfqpoint{2.075558in}{1.876703in}}%
\pgfpathlineto{\pgfqpoint{2.102468in}{1.895865in}}%
\pgfpathlineto{\pgfqpoint{2.129378in}{1.901864in}}%
\pgfpathlineto{\pgfqpoint{2.156287in}{1.902203in}}%
\pgfpathlineto{\pgfqpoint{2.183197in}{1.909854in}}%
\pgfpathlineto{\pgfqpoint{2.210106in}{1.932042in}}%
\pgfpathlineto{\pgfqpoint{2.237016in}{1.964784in}}%
\pgfpathlineto{\pgfqpoint{2.263926in}{1.996835in}}%
\pgfpathlineto{\pgfqpoint{2.290835in}{2.016992in}}%
\pgfpathlineto{\pgfqpoint{2.317745in}{2.021828in}}%
\pgfpathlineto{\pgfqpoint{2.344655in}{2.016451in}}%
\pgfpathlineto{\pgfqpoint{2.371564in}{2.010271in}}%
\pgfpathlineto{\pgfqpoint{2.398474in}{2.011833in}}%
\pgfpathlineto{\pgfqpoint{2.425384in}{2.025872in}}%
\pgfpathlineto{\pgfqpoint{2.452293in}{2.058112in}}%
\pgfpathlineto{\pgfqpoint{2.479203in}{2.113254in}}%
\pgfpathlineto{\pgfqpoint{2.506113in}{2.189790in}}%
\pgfpathlineto{\pgfqpoint{2.533022in}{2.280174in}}%
\pgfpathlineto{\pgfqpoint{2.559932in}{2.375725in}}%
\pgfpathlineto{\pgfqpoint{2.586841in}{2.468078in}}%
\pgfpathlineto{\pgfqpoint{2.613751in}{2.546564in}}%
\pgfpathlineto{\pgfqpoint{2.640661in}{2.601196in}}%
\pgfpathlineto{\pgfqpoint{2.667570in}{2.632472in}}%
\pgfpathlineto{\pgfqpoint{2.694480in}{2.650738in}}%
\pgfpathlineto{\pgfqpoint{2.721390in}{2.661261in}}%
\pgfpathlineto{\pgfqpoint{2.748299in}{2.653818in}}%
\pgfpathlineto{\pgfqpoint{2.775209in}{2.609519in}}%
\pgfpathlineto{\pgfqpoint{2.802119in}{2.519932in}}%
\pgfpathlineto{\pgfqpoint{2.829028in}{2.396235in}}%
\pgfpathlineto{\pgfqpoint{2.855938in}{2.254684in}}%
\pgfpathlineto{\pgfqpoint{2.882848in}{2.100147in}}%
\pgfpathlineto{\pgfqpoint{2.909757in}{1.928603in}}%
\pgfpathlineto{\pgfqpoint{2.936667in}{1.735233in}}%
\pgfpathlineto{\pgfqpoint{2.963576in}{1.522762in}}%
\pgfusepath{stroke}%
\end{pgfscope}%
\begin{pgfscope}%
\pgfpathrectangle{\pgfqpoint{0.693748in}{0.345370in}}{\pgfqpoint{2.575253in}{2.571297in}}%
\pgfusepath{clip}%
\pgfsetrectcap%
\pgfsetroundjoin%
\pgfsetlinewidth{1.505625pt}%
\definecolor{currentstroke}{rgb}{0.172549,0.627451,0.172549}%
\pgfsetstrokecolor{currentstroke}%
\pgfsetdash{}{0pt}%
\pgfpathmoveto{\pgfqpoint{0.810805in}{1.942035in}}%
\pgfpathlineto{\pgfqpoint{0.837715in}{1.956951in}}%
\pgfpathlineto{\pgfqpoint{0.864624in}{1.970182in}}%
\pgfpathlineto{\pgfqpoint{0.891534in}{1.978645in}}%
\pgfpathlineto{\pgfqpoint{0.918444in}{1.981730in}}%
\pgfpathlineto{\pgfqpoint{0.945353in}{1.987410in}}%
\pgfpathlineto{\pgfqpoint{0.972263in}{2.011581in}}%
\pgfpathlineto{\pgfqpoint{0.999173in}{2.070347in}}%
\pgfpathlineto{\pgfqpoint{1.026082in}{2.169569in}}%
\pgfpathlineto{\pgfqpoint{1.052992in}{2.301623in}}%
\pgfpathlineto{\pgfqpoint{1.079901in}{2.451278in}}%
\pgfpathlineto{\pgfqpoint{1.106811in}{2.597812in}}%
\pgfpathlineto{\pgfqpoint{1.133721in}{2.715428in}}%
\pgfpathlineto{\pgfqpoint{1.160630in}{2.783522in}}%
\pgfpathlineto{\pgfqpoint{1.187540in}{2.799790in}}%
\pgfpathlineto{\pgfqpoint{1.214450in}{2.781270in}}%
\pgfpathlineto{\pgfqpoint{1.241359in}{2.747594in}}%
\pgfpathlineto{\pgfqpoint{1.268269in}{2.702992in}}%
\pgfpathlineto{\pgfqpoint{1.295179in}{2.636889in}}%
\pgfpathlineto{\pgfqpoint{1.322088in}{2.541314in}}%
\pgfpathlineto{\pgfqpoint{1.348998in}{2.420720in}}%
\pgfpathlineto{\pgfqpoint{1.375908in}{2.284015in}}%
\pgfpathlineto{\pgfqpoint{1.402817in}{2.135249in}}%
\pgfpathlineto{\pgfqpoint{1.429727in}{1.976435in}}%
\pgfpathlineto{\pgfqpoint{1.456636in}{1.813789in}}%
\pgfpathlineto{\pgfqpoint{1.483546in}{1.659020in}}%
\pgfpathlineto{\pgfqpoint{1.510456in}{1.520514in}}%
\pgfpathlineto{\pgfqpoint{1.537365in}{1.396002in}}%
\pgfpathlineto{\pgfqpoint{1.564275in}{1.275130in}}%
\pgfpathlineto{\pgfqpoint{1.591185in}{1.147611in}}%
\pgfpathlineto{\pgfqpoint{1.618094in}{1.009991in}}%
\pgfpathlineto{\pgfqpoint{1.645004in}{0.864658in}}%
\pgfpathlineto{\pgfqpoint{1.671914in}{0.719811in}}%
\pgfpathlineto{\pgfqpoint{1.698823in}{0.590657in}}%
\pgfpathlineto{\pgfqpoint{1.725733in}{0.497724in}}%
\pgfpathlineto{\pgfqpoint{1.752643in}{0.462247in}}%
\pgfpathlineto{\pgfqpoint{1.779552in}{0.498110in}}%
\pgfpathlineto{\pgfqpoint{1.806462in}{0.603390in}}%
\pgfpathlineto{\pgfqpoint{1.833371in}{0.754702in}}%
\pgfpathlineto{\pgfqpoint{1.860281in}{0.910599in}}%
\pgfpathlineto{\pgfqpoint{1.887191in}{1.027831in}}%
\pgfpathlineto{\pgfqpoint{1.914100in}{1.082538in}}%
\pgfpathlineto{\pgfqpoint{1.941010in}{1.080323in}}%
\pgfpathlineto{\pgfqpoint{1.967920in}{1.046468in}}%
\pgfpathlineto{\pgfqpoint{1.994829in}{1.011607in}}%
\pgfpathlineto{\pgfqpoint{2.021739in}{1.002525in}}%
\pgfpathlineto{\pgfqpoint{2.048649in}{1.031701in}}%
\pgfpathlineto{\pgfqpoint{2.075558in}{1.094095in}}%
\pgfpathlineto{\pgfqpoint{2.102468in}{1.186325in}}%
\pgfpathlineto{\pgfqpoint{2.129378in}{1.313569in}}%
\pgfpathlineto{\pgfqpoint{2.156287in}{1.479771in}}%
\pgfpathlineto{\pgfqpoint{2.183197in}{1.675213in}}%
\pgfpathlineto{\pgfqpoint{2.210106in}{1.871591in}}%
\pgfpathlineto{\pgfqpoint{2.237016in}{2.031128in}}%
\pgfpathlineto{\pgfqpoint{2.263926in}{2.127030in}}%
\pgfpathlineto{\pgfqpoint{2.290835in}{2.158973in}}%
\pgfpathlineto{\pgfqpoint{2.317745in}{2.147792in}}%
\pgfpathlineto{\pgfqpoint{2.344655in}{2.117899in}}%
\pgfpathlineto{\pgfqpoint{2.371564in}{2.086804in}}%
\pgfpathlineto{\pgfqpoint{2.398474in}{2.062727in}}%
\pgfpathlineto{\pgfqpoint{2.425384in}{2.045653in}}%
\pgfpathlineto{\pgfqpoint{2.452293in}{2.033054in}}%
\pgfpathlineto{\pgfqpoint{2.479203in}{2.025555in}}%
\pgfpathlineto{\pgfqpoint{2.506113in}{2.024685in}}%
\pgfpathlineto{\pgfqpoint{2.533022in}{2.028701in}}%
\pgfpathlineto{\pgfqpoint{2.559932in}{2.032633in}}%
\pgfpathlineto{\pgfqpoint{2.586841in}{2.032847in}}%
\pgfpathlineto{\pgfqpoint{2.613751in}{2.031013in}}%
\pgfpathlineto{\pgfqpoint{2.640661in}{2.036414in}}%
\pgfpathlineto{\pgfqpoint{2.667570in}{2.065744in}}%
\pgfpathlineto{\pgfqpoint{2.694480in}{2.134024in}}%
\pgfpathlineto{\pgfqpoint{2.721390in}{2.244386in}}%
\pgfpathlineto{\pgfqpoint{2.748299in}{2.385893in}}%
\pgfpathlineto{\pgfqpoint{2.775209in}{2.535934in}}%
\pgfpathlineto{\pgfqpoint{2.802119in}{2.659172in}}%
\pgfpathlineto{\pgfqpoint{2.829028in}{2.715428in}}%
\pgfpathlineto{\pgfqpoint{2.855938in}{2.688393in}}%
\pgfpathlineto{\pgfqpoint{2.882848in}{2.600198in}}%
\pgfpathlineto{\pgfqpoint{2.909757in}{2.496712in}}%
\pgfpathlineto{\pgfqpoint{2.936667in}{2.411053in}}%
\pgfpathlineto{\pgfqpoint{2.963576in}{2.338666in}}%
\pgfpathlineto{\pgfqpoint{2.990486in}{2.253587in}}%
\pgfpathlineto{\pgfqpoint{3.017396in}{2.138759in}}%
\pgfpathlineto{\pgfqpoint{3.044305in}{1.994652in}}%
\pgfpathlineto{\pgfqpoint{3.071215in}{1.830061in}}%
\pgfpathlineto{\pgfqpoint{3.098125in}{1.651768in}}%
\pgfpathlineto{\pgfqpoint{3.125034in}{1.467917in}}%
\pgfpathlineto{\pgfqpoint{3.151944in}{1.282568in}}%
\pgfusepath{stroke}%
\end{pgfscope}%
\begin{pgfscope}%
\pgfsetrectcap%
\pgfsetmiterjoin%
\pgfsetlinewidth{0.803000pt}%
\definecolor{currentstroke}{rgb}{0.000000,0.000000,0.000000}%
\pgfsetstrokecolor{currentstroke}%
\pgfsetdash{}{0pt}%
\pgfpathmoveto{\pgfqpoint{0.693748in}{0.345370in}}%
\pgfpathlineto{\pgfqpoint{0.693748in}{2.916667in}}%
\pgfusepath{stroke}%
\end{pgfscope}%
\begin{pgfscope}%
\pgfsetrectcap%
\pgfsetmiterjoin%
\pgfsetlinewidth{0.803000pt}%
\definecolor{currentstroke}{rgb}{0.000000,0.000000,0.000000}%
\pgfsetstrokecolor{currentstroke}%
\pgfsetdash{}{0pt}%
\pgfpathmoveto{\pgfqpoint{3.269001in}{0.345370in}}%
\pgfpathlineto{\pgfqpoint{3.269001in}{2.916667in}}%
\pgfusepath{stroke}%
\end{pgfscope}%
\begin{pgfscope}%
\pgfsetrectcap%
\pgfsetmiterjoin%
\pgfsetlinewidth{0.803000pt}%
\definecolor{currentstroke}{rgb}{0.000000,0.000000,0.000000}%
\pgfsetstrokecolor{currentstroke}%
\pgfsetdash{}{0pt}%
\pgfpathmoveto{\pgfqpoint{0.693748in}{0.345370in}}%
\pgfpathlineto{\pgfqpoint{3.269001in}{0.345370in}}%
\pgfusepath{stroke}%
\end{pgfscope}%
\begin{pgfscope}%
\pgfsetrectcap%
\pgfsetmiterjoin%
\pgfsetlinewidth{0.803000pt}%
\definecolor{currentstroke}{rgb}{0.000000,0.000000,0.000000}%
\pgfsetstrokecolor{currentstroke}%
\pgfsetdash{}{0pt}%
\pgfpathmoveto{\pgfqpoint{0.693748in}{2.916667in}}%
\pgfpathlineto{\pgfqpoint{3.269001in}{2.916667in}}%
\pgfusepath{stroke}%
\end{pgfscope}%
\begin{pgfscope}%
\definecolor{textcolor}{rgb}{0.000000,0.000000,0.000000}%
\pgfsetstrokecolor{textcolor}%
\pgfsetfillcolor{textcolor}%
\pgftext[x=1.981374in,y=3.000000in,,base]{\color{textcolor}\rmfamily\fontsize{12.000000}{14.400000}\selectfont (c) Z-score normalized}%
\end{pgfscope}%
\begin{pgfscope}%
\pgfsetbuttcap%
\pgfsetmiterjoin%
\definecolor{currentfill}{rgb}{1.000000,1.000000,1.000000}%
\pgfsetfillcolor{currentfill}%
\pgfsetlinewidth{0.000000pt}%
\definecolor{currentstroke}{rgb}{0.000000,0.000000,0.000000}%
\pgfsetstrokecolor{currentstroke}%
\pgfsetstrokeopacity{0.000000}%
\pgfsetdash{}{0pt}%
\pgfpathmoveto{\pgfqpoint{3.724747in}{0.345370in}}%
\pgfpathlineto{\pgfqpoint{6.300000in}{0.345370in}}%
\pgfpathlineto{\pgfqpoint{6.300000in}{2.916667in}}%
\pgfpathlineto{\pgfqpoint{3.724747in}{2.916667in}}%
\pgfpathclose%
\pgfusepath{fill}%
\end{pgfscope}%
\begin{pgfscope}%
\pgfsetbuttcap%
\pgfsetroundjoin%
\definecolor{currentfill}{rgb}{0.000000,0.000000,0.000000}%
\pgfsetfillcolor{currentfill}%
\pgfsetlinewidth{0.803000pt}%
\definecolor{currentstroke}{rgb}{0.000000,0.000000,0.000000}%
\pgfsetstrokecolor{currentstroke}%
\pgfsetdash{}{0pt}%
\pgfsys@defobject{currentmarker}{\pgfqpoint{0.000000in}{-0.048611in}}{\pgfqpoint{0.000000in}{0.000000in}}{%
\pgfpathmoveto{\pgfqpoint{0.000000in}{0.000000in}}%
\pgfpathlineto{\pgfqpoint{0.000000in}{-0.048611in}}%
\pgfusepath{stroke,fill}%
}%
\begin{pgfscope}%
\pgfsys@transformshift{3.841804in}{0.345370in}%
\pgfsys@useobject{currentmarker}{}%
\end{pgfscope}%
\end{pgfscope}%
\begin{pgfscope}%
\definecolor{textcolor}{rgb}{0.000000,0.000000,0.000000}%
\pgfsetstrokecolor{textcolor}%
\pgfsetfillcolor{textcolor}%
\pgftext[x=3.841804in,y=0.248148in,,top]{\color{textcolor}\rmfamily\fontsize{12.000000}{14.400000}\selectfont \(\displaystyle 0\)}%
\end{pgfscope}%
\begin{pgfscope}%
\pgfsetbuttcap%
\pgfsetroundjoin%
\definecolor{currentfill}{rgb}{0.000000,0.000000,0.000000}%
\pgfsetfillcolor{currentfill}%
\pgfsetlinewidth{0.803000pt}%
\definecolor{currentstroke}{rgb}{0.000000,0.000000,0.000000}%
\pgfsetstrokecolor{currentstroke}%
\pgfsetdash{}{0pt}%
\pgfsys@defobject{currentmarker}{\pgfqpoint{0.000000in}{-0.048611in}}{\pgfqpoint{0.000000in}{0.000000in}}{%
\pgfpathmoveto{\pgfqpoint{0.000000in}{0.000000in}}%
\pgfpathlineto{\pgfqpoint{0.000000in}{-0.048611in}}%
\pgfusepath{stroke,fill}%
}%
\begin{pgfscope}%
\pgfsys@transformshift{4.379997in}{0.345370in}%
\pgfsys@useobject{currentmarker}{}%
\end{pgfscope}%
\end{pgfscope}%
\begin{pgfscope}%
\definecolor{textcolor}{rgb}{0.000000,0.000000,0.000000}%
\pgfsetstrokecolor{textcolor}%
\pgfsetfillcolor{textcolor}%
\pgftext[x=4.379997in,y=0.248148in,,top]{\color{textcolor}\rmfamily\fontsize{12.000000}{14.400000}\selectfont \(\displaystyle 20\)}%
\end{pgfscope}%
\begin{pgfscope}%
\pgfsetbuttcap%
\pgfsetroundjoin%
\definecolor{currentfill}{rgb}{0.000000,0.000000,0.000000}%
\pgfsetfillcolor{currentfill}%
\pgfsetlinewidth{0.803000pt}%
\definecolor{currentstroke}{rgb}{0.000000,0.000000,0.000000}%
\pgfsetstrokecolor{currentstroke}%
\pgfsetdash{}{0pt}%
\pgfsys@defobject{currentmarker}{\pgfqpoint{0.000000in}{-0.048611in}}{\pgfqpoint{0.000000in}{0.000000in}}{%
\pgfpathmoveto{\pgfqpoint{0.000000in}{0.000000in}}%
\pgfpathlineto{\pgfqpoint{0.000000in}{-0.048611in}}%
\pgfusepath{stroke,fill}%
}%
\begin{pgfscope}%
\pgfsys@transformshift{4.918190in}{0.345370in}%
\pgfsys@useobject{currentmarker}{}%
\end{pgfscope}%
\end{pgfscope}%
\begin{pgfscope}%
\definecolor{textcolor}{rgb}{0.000000,0.000000,0.000000}%
\pgfsetstrokecolor{textcolor}%
\pgfsetfillcolor{textcolor}%
\pgftext[x=4.918190in,y=0.248148in,,top]{\color{textcolor}\rmfamily\fontsize{12.000000}{14.400000}\selectfont \(\displaystyle 40\)}%
\end{pgfscope}%
\begin{pgfscope}%
\pgfsetbuttcap%
\pgfsetroundjoin%
\definecolor{currentfill}{rgb}{0.000000,0.000000,0.000000}%
\pgfsetfillcolor{currentfill}%
\pgfsetlinewidth{0.803000pt}%
\definecolor{currentstroke}{rgb}{0.000000,0.000000,0.000000}%
\pgfsetstrokecolor{currentstroke}%
\pgfsetdash{}{0pt}%
\pgfsys@defobject{currentmarker}{\pgfqpoint{0.000000in}{-0.048611in}}{\pgfqpoint{0.000000in}{0.000000in}}{%
\pgfpathmoveto{\pgfqpoint{0.000000in}{0.000000in}}%
\pgfpathlineto{\pgfqpoint{0.000000in}{-0.048611in}}%
\pgfusepath{stroke,fill}%
}%
\begin{pgfscope}%
\pgfsys@transformshift{5.456383in}{0.345370in}%
\pgfsys@useobject{currentmarker}{}%
\end{pgfscope}%
\end{pgfscope}%
\begin{pgfscope}%
\definecolor{textcolor}{rgb}{0.000000,0.000000,0.000000}%
\pgfsetstrokecolor{textcolor}%
\pgfsetfillcolor{textcolor}%
\pgftext[x=5.456383in,y=0.248148in,,top]{\color{textcolor}\rmfamily\fontsize{12.000000}{14.400000}\selectfont \(\displaystyle 60\)}%
\end{pgfscope}%
\begin{pgfscope}%
\pgfsetbuttcap%
\pgfsetroundjoin%
\definecolor{currentfill}{rgb}{0.000000,0.000000,0.000000}%
\pgfsetfillcolor{currentfill}%
\pgfsetlinewidth{0.803000pt}%
\definecolor{currentstroke}{rgb}{0.000000,0.000000,0.000000}%
\pgfsetstrokecolor{currentstroke}%
\pgfsetdash{}{0pt}%
\pgfsys@defobject{currentmarker}{\pgfqpoint{0.000000in}{-0.048611in}}{\pgfqpoint{0.000000in}{0.000000in}}{%
\pgfpathmoveto{\pgfqpoint{0.000000in}{0.000000in}}%
\pgfpathlineto{\pgfqpoint{0.000000in}{-0.048611in}}%
\pgfusepath{stroke,fill}%
}%
\begin{pgfscope}%
\pgfsys@transformshift{5.994576in}{0.345370in}%
\pgfsys@useobject{currentmarker}{}%
\end{pgfscope}%
\end{pgfscope}%
\begin{pgfscope}%
\definecolor{textcolor}{rgb}{0.000000,0.000000,0.000000}%
\pgfsetstrokecolor{textcolor}%
\pgfsetfillcolor{textcolor}%
\pgftext[x=5.994576in,y=0.248148in,,top]{\color{textcolor}\rmfamily\fontsize{12.000000}{14.400000}\selectfont \(\displaystyle 80\)}%
\end{pgfscope}%
\begin{pgfscope}%
\pgfsetbuttcap%
\pgfsetroundjoin%
\definecolor{currentfill}{rgb}{0.000000,0.000000,0.000000}%
\pgfsetfillcolor{currentfill}%
\pgfsetlinewidth{0.803000pt}%
\definecolor{currentstroke}{rgb}{0.000000,0.000000,0.000000}%
\pgfsetstrokecolor{currentstroke}%
\pgfsetdash{}{0pt}%
\pgfsys@defobject{currentmarker}{\pgfqpoint{-0.048611in}{0.000000in}}{\pgfqpoint{0.000000in}{0.000000in}}{%
\pgfpathmoveto{\pgfqpoint{0.000000in}{0.000000in}}%
\pgfpathlineto{\pgfqpoint{-0.048611in}{0.000000in}}%
\pgfusepath{stroke,fill}%
}%
\begin{pgfscope}%
\pgfsys@transformshift{3.724747in}{0.408011in}%
\pgfsys@useobject{currentmarker}{}%
\end{pgfscope}%
\end{pgfscope}%
\begin{pgfscope}%
\definecolor{textcolor}{rgb}{0.000000,0.000000,0.000000}%
\pgfsetstrokecolor{textcolor}%
\pgfsetfillcolor{textcolor}%
\pgftext[x=3.419001in,y=0.350141in,left,base]{\color{textcolor}\rmfamily\fontsize{12.000000}{14.400000}\selectfont \(\displaystyle 0.3\)}%
\end{pgfscope}%
\begin{pgfscope}%
\pgfsetbuttcap%
\pgfsetroundjoin%
\definecolor{currentfill}{rgb}{0.000000,0.000000,0.000000}%
\pgfsetfillcolor{currentfill}%
\pgfsetlinewidth{0.803000pt}%
\definecolor{currentstroke}{rgb}{0.000000,0.000000,0.000000}%
\pgfsetstrokecolor{currentstroke}%
\pgfsetdash{}{0pt}%
\pgfsys@defobject{currentmarker}{\pgfqpoint{-0.048611in}{0.000000in}}{\pgfqpoint{0.000000in}{0.000000in}}{%
\pgfpathmoveto{\pgfqpoint{0.000000in}{0.000000in}}%
\pgfpathlineto{\pgfqpoint{-0.048611in}{0.000000in}}%
\pgfusepath{stroke,fill}%
}%
\begin{pgfscope}%
\pgfsys@transformshift{3.724747in}{0.788260in}%
\pgfsys@useobject{currentmarker}{}%
\end{pgfscope}%
\end{pgfscope}%
\begin{pgfscope}%
\definecolor{textcolor}{rgb}{0.000000,0.000000,0.000000}%
\pgfsetstrokecolor{textcolor}%
\pgfsetfillcolor{textcolor}%
\pgftext[x=3.419001in,y=0.730390in,left,base]{\color{textcolor}\rmfamily\fontsize{12.000000}{14.400000}\selectfont \(\displaystyle 0.4\)}%
\end{pgfscope}%
\begin{pgfscope}%
\pgfsetbuttcap%
\pgfsetroundjoin%
\definecolor{currentfill}{rgb}{0.000000,0.000000,0.000000}%
\pgfsetfillcolor{currentfill}%
\pgfsetlinewidth{0.803000pt}%
\definecolor{currentstroke}{rgb}{0.000000,0.000000,0.000000}%
\pgfsetstrokecolor{currentstroke}%
\pgfsetdash{}{0pt}%
\pgfsys@defobject{currentmarker}{\pgfqpoint{-0.048611in}{0.000000in}}{\pgfqpoint{0.000000in}{0.000000in}}{%
\pgfpathmoveto{\pgfqpoint{0.000000in}{0.000000in}}%
\pgfpathlineto{\pgfqpoint{-0.048611in}{0.000000in}}%
\pgfusepath{stroke,fill}%
}%
\begin{pgfscope}%
\pgfsys@transformshift{3.724747in}{1.168509in}%
\pgfsys@useobject{currentmarker}{}%
\end{pgfscope}%
\end{pgfscope}%
\begin{pgfscope}%
\definecolor{textcolor}{rgb}{0.000000,0.000000,0.000000}%
\pgfsetstrokecolor{textcolor}%
\pgfsetfillcolor{textcolor}%
\pgftext[x=3.419001in,y=1.110639in,left,base]{\color{textcolor}\rmfamily\fontsize{12.000000}{14.400000}\selectfont \(\displaystyle 0.5\)}%
\end{pgfscope}%
\begin{pgfscope}%
\pgfsetbuttcap%
\pgfsetroundjoin%
\definecolor{currentfill}{rgb}{0.000000,0.000000,0.000000}%
\pgfsetfillcolor{currentfill}%
\pgfsetlinewidth{0.803000pt}%
\definecolor{currentstroke}{rgb}{0.000000,0.000000,0.000000}%
\pgfsetstrokecolor{currentstroke}%
\pgfsetdash{}{0pt}%
\pgfsys@defobject{currentmarker}{\pgfqpoint{-0.048611in}{0.000000in}}{\pgfqpoint{0.000000in}{0.000000in}}{%
\pgfpathmoveto{\pgfqpoint{0.000000in}{0.000000in}}%
\pgfpathlineto{\pgfqpoint{-0.048611in}{0.000000in}}%
\pgfusepath{stroke,fill}%
}%
\begin{pgfscope}%
\pgfsys@transformshift{3.724747in}{1.548758in}%
\pgfsys@useobject{currentmarker}{}%
\end{pgfscope}%
\end{pgfscope}%
\begin{pgfscope}%
\definecolor{textcolor}{rgb}{0.000000,0.000000,0.000000}%
\pgfsetstrokecolor{textcolor}%
\pgfsetfillcolor{textcolor}%
\pgftext[x=3.419001in,y=1.490888in,left,base]{\color{textcolor}\rmfamily\fontsize{12.000000}{14.400000}\selectfont \(\displaystyle 0.6\)}%
\end{pgfscope}%
\begin{pgfscope}%
\pgfsetbuttcap%
\pgfsetroundjoin%
\definecolor{currentfill}{rgb}{0.000000,0.000000,0.000000}%
\pgfsetfillcolor{currentfill}%
\pgfsetlinewidth{0.803000pt}%
\definecolor{currentstroke}{rgb}{0.000000,0.000000,0.000000}%
\pgfsetstrokecolor{currentstroke}%
\pgfsetdash{}{0pt}%
\pgfsys@defobject{currentmarker}{\pgfqpoint{-0.048611in}{0.000000in}}{\pgfqpoint{0.000000in}{0.000000in}}{%
\pgfpathmoveto{\pgfqpoint{0.000000in}{0.000000in}}%
\pgfpathlineto{\pgfqpoint{-0.048611in}{0.000000in}}%
\pgfusepath{stroke,fill}%
}%
\begin{pgfscope}%
\pgfsys@transformshift{3.724747in}{1.929007in}%
\pgfsys@useobject{currentmarker}{}%
\end{pgfscope}%
\end{pgfscope}%
\begin{pgfscope}%
\definecolor{textcolor}{rgb}{0.000000,0.000000,0.000000}%
\pgfsetstrokecolor{textcolor}%
\pgfsetfillcolor{textcolor}%
\pgftext[x=3.419001in,y=1.871137in,left,base]{\color{textcolor}\rmfamily\fontsize{12.000000}{14.400000}\selectfont \(\displaystyle 0.7\)}%
\end{pgfscope}%
\begin{pgfscope}%
\pgfsetbuttcap%
\pgfsetroundjoin%
\definecolor{currentfill}{rgb}{0.000000,0.000000,0.000000}%
\pgfsetfillcolor{currentfill}%
\pgfsetlinewidth{0.803000pt}%
\definecolor{currentstroke}{rgb}{0.000000,0.000000,0.000000}%
\pgfsetstrokecolor{currentstroke}%
\pgfsetdash{}{0pt}%
\pgfsys@defobject{currentmarker}{\pgfqpoint{-0.048611in}{0.000000in}}{\pgfqpoint{0.000000in}{0.000000in}}{%
\pgfpathmoveto{\pgfqpoint{0.000000in}{0.000000in}}%
\pgfpathlineto{\pgfqpoint{-0.048611in}{0.000000in}}%
\pgfusepath{stroke,fill}%
}%
\begin{pgfscope}%
\pgfsys@transformshift{3.724747in}{2.309256in}%
\pgfsys@useobject{currentmarker}{}%
\end{pgfscope}%
\end{pgfscope}%
\begin{pgfscope}%
\definecolor{textcolor}{rgb}{0.000000,0.000000,0.000000}%
\pgfsetstrokecolor{textcolor}%
\pgfsetfillcolor{textcolor}%
\pgftext[x=3.419001in,y=2.251386in,left,base]{\color{textcolor}\rmfamily\fontsize{12.000000}{14.400000}\selectfont \(\displaystyle 0.8\)}%
\end{pgfscope}%
\begin{pgfscope}%
\pgfsetbuttcap%
\pgfsetroundjoin%
\definecolor{currentfill}{rgb}{0.000000,0.000000,0.000000}%
\pgfsetfillcolor{currentfill}%
\pgfsetlinewidth{0.803000pt}%
\definecolor{currentstroke}{rgb}{0.000000,0.000000,0.000000}%
\pgfsetstrokecolor{currentstroke}%
\pgfsetdash{}{0pt}%
\pgfsys@defobject{currentmarker}{\pgfqpoint{-0.048611in}{0.000000in}}{\pgfqpoint{0.000000in}{0.000000in}}{%
\pgfpathmoveto{\pgfqpoint{0.000000in}{0.000000in}}%
\pgfpathlineto{\pgfqpoint{-0.048611in}{0.000000in}}%
\pgfusepath{stroke,fill}%
}%
\begin{pgfscope}%
\pgfsys@transformshift{3.724747in}{2.689505in}%
\pgfsys@useobject{currentmarker}{}%
\end{pgfscope}%
\end{pgfscope}%
\begin{pgfscope}%
\definecolor{textcolor}{rgb}{0.000000,0.000000,0.000000}%
\pgfsetstrokecolor{textcolor}%
\pgfsetfillcolor{textcolor}%
\pgftext[x=3.419001in,y=2.631635in,left,base]{\color{textcolor}\rmfamily\fontsize{12.000000}{14.400000}\selectfont \(\displaystyle 0.9\)}%
\end{pgfscope}%
\begin{pgfscope}%
\pgfpathrectangle{\pgfqpoint{3.724747in}{0.345370in}}{\pgfqpoint{2.575253in}{2.571297in}}%
\pgfusepath{clip}%
\pgfsetrectcap%
\pgfsetroundjoin%
\pgfsetlinewidth{1.505625pt}%
\definecolor{currentstroke}{rgb}{0.121569,0.466667,0.705882}%
\pgfsetstrokecolor{currentstroke}%
\pgfsetdash{}{0pt}%
\pgfpathmoveto{\pgfqpoint{3.841804in}{1.734606in}}%
\pgfpathlineto{\pgfqpoint{3.868714in}{1.834442in}}%
\pgfpathlineto{\pgfqpoint{3.895623in}{1.962607in}}%
\pgfpathlineto{\pgfqpoint{3.922533in}{2.128902in}}%
\pgfpathlineto{\pgfqpoint{3.949443in}{2.319865in}}%
\pgfpathlineto{\pgfqpoint{3.976352in}{2.504574in}}%
\pgfpathlineto{\pgfqpoint{4.003262in}{2.648424in}}%
\pgfpathlineto{\pgfqpoint{4.030172in}{2.733722in}}%
\pgfpathlineto{\pgfqpoint{4.057081in}{2.767091in}}%
\pgfpathlineto{\pgfqpoint{4.083991in}{2.766784in}}%
\pgfpathlineto{\pgfqpoint{4.110901in}{2.749005in}}%
\pgfpathlineto{\pgfqpoint{4.137810in}{2.727753in}}%
\pgfpathlineto{\pgfqpoint{4.164720in}{2.715428in}}%
\pgfpathlineto{\pgfqpoint{4.191629in}{2.717809in}}%
\pgfpathlineto{\pgfqpoint{4.218539in}{2.729969in}}%
\pgfpathlineto{\pgfqpoint{4.245449in}{2.736575in}}%
\pgfpathlineto{\pgfqpoint{4.272358in}{2.719517in}}%
\pgfpathlineto{\pgfqpoint{4.299268in}{2.669482in}}%
\pgfpathlineto{\pgfqpoint{4.326178in}{2.589338in}}%
\pgfpathlineto{\pgfqpoint{4.353087in}{2.487964in}}%
\pgfpathlineto{\pgfqpoint{4.379997in}{2.375368in}}%
\pgfpathlineto{\pgfqpoint{4.406907in}{2.260144in}}%
\pgfpathlineto{\pgfqpoint{4.433816in}{2.146923in}}%
\pgfpathlineto{\pgfqpoint{4.460726in}{2.035320in}}%
\pgfpathlineto{\pgfqpoint{4.487636in}{1.922667in}}%
\pgfpathlineto{\pgfqpoint{4.514545in}{1.808696in}}%
\pgfpathlineto{\pgfqpoint{4.541455in}{1.696054in}}%
\pgfpathlineto{\pgfqpoint{4.568364in}{1.586590in}}%
\pgfpathlineto{\pgfqpoint{4.595274in}{1.480328in}}%
\pgfpathlineto{\pgfqpoint{4.622184in}{1.378648in}}%
\pgfpathlineto{\pgfqpoint{4.649093in}{1.283228in}}%
\pgfpathlineto{\pgfqpoint{4.676003in}{1.195831in}}%
\pgfpathlineto{\pgfqpoint{4.702913in}{1.118544in}}%
\pgfpathlineto{\pgfqpoint{4.729822in}{1.054239in}}%
\pgfpathlineto{\pgfqpoint{4.756732in}{1.005946in}}%
\pgfpathlineto{\pgfqpoint{4.783642in}{0.975458in}}%
\pgfpathlineto{\pgfqpoint{4.810551in}{0.962840in}}%
\pgfpathlineto{\pgfqpoint{4.837461in}{0.966755in}}%
\pgfpathlineto{\pgfqpoint{4.864371in}{0.983713in}}%
\pgfpathlineto{\pgfqpoint{4.891280in}{1.004740in}}%
\pgfpathlineto{\pgfqpoint{4.918190in}{1.016514in}}%
\pgfpathlineto{\pgfqpoint{4.945099in}{1.010875in}}%
\pgfpathlineto{\pgfqpoint{4.972009in}{0.991779in}}%
\pgfpathlineto{\pgfqpoint{4.998919in}{0.973212in}}%
\pgfpathlineto{\pgfqpoint{5.025828in}{0.969751in}}%
\pgfpathlineto{\pgfqpoint{5.052738in}{0.989043in}}%
\pgfpathlineto{\pgfqpoint{5.079648in}{1.030078in}}%
\pgfpathlineto{\pgfqpoint{5.106557in}{1.084021in}}%
\pgfpathlineto{\pgfqpoint{5.133467in}{1.139952in}}%
\pgfpathlineto{\pgfqpoint{5.160377in}{1.191553in}}%
\pgfpathlineto{\pgfqpoint{5.187286in}{1.238264in}}%
\pgfpathlineto{\pgfqpoint{5.214196in}{1.282854in}}%
\pgfpathlineto{\pgfqpoint{5.241106in}{1.328348in}}%
\pgfpathlineto{\pgfqpoint{5.268015in}{1.376042in}}%
\pgfpathlineto{\pgfqpoint{5.294925in}{1.425086in}}%
\pgfpathlineto{\pgfqpoint{5.321834in}{1.473629in}}%
\pgfpathlineto{\pgfqpoint{5.348744in}{1.519911in}}%
\pgfpathlineto{\pgfqpoint{5.375654in}{1.562850in}}%
\pgfpathlineto{\pgfqpoint{5.402563in}{1.602064in}}%
\pgfpathlineto{\pgfqpoint{5.429473in}{1.638378in}}%
\pgfpathlineto{\pgfqpoint{5.456383in}{1.679171in}}%
\pgfpathlineto{\pgfqpoint{5.483292in}{1.742717in}}%
\pgfpathlineto{\pgfqpoint{5.510202in}{1.851686in}}%
\pgfpathlineto{\pgfqpoint{5.537112in}{2.016696in}}%
\pgfpathlineto{\pgfqpoint{5.564021in}{2.223009in}}%
\pgfpathlineto{\pgfqpoint{5.590931in}{2.433273in}}%
\pgfpathlineto{\pgfqpoint{5.617841in}{2.605355in}}%
\pgfpathlineto{\pgfqpoint{5.644750in}{2.714180in}}%
\pgfpathlineto{\pgfqpoint{5.671660in}{2.761001in}}%
\pgfpathlineto{\pgfqpoint{5.698569in}{2.764834in}}%
\pgfpathlineto{\pgfqpoint{5.725479in}{2.747220in}}%
\pgfpathlineto{\pgfqpoint{5.752389in}{2.726167in}}%
\pgfpathlineto{\pgfqpoint{5.779298in}{2.715428in}}%
\pgfpathlineto{\pgfqpoint{5.806208in}{2.721020in}}%
\pgfpathlineto{\pgfqpoint{5.833118in}{2.736628in}}%
\pgfpathlineto{\pgfqpoint{5.860027in}{2.746483in}}%
\pgfpathlineto{\pgfqpoint{5.886937in}{2.735098in}}%
\pgfpathlineto{\pgfqpoint{5.913847in}{2.693840in}}%
\pgfpathlineto{\pgfqpoint{5.940756in}{2.622889in}}%
\pgfpathlineto{\pgfqpoint{5.967666in}{2.528420in}}%
\pgfpathlineto{\pgfqpoint{5.994576in}{2.418517in}}%
\pgfpathlineto{\pgfqpoint{6.021485in}{2.302352in}}%
\pgfpathlineto{\pgfqpoint{6.048395in}{2.187659in}}%
\pgfpathlineto{\pgfqpoint{6.075304in}{2.077797in}}%
\pgfpathlineto{\pgfqpoint{6.102214in}{1.971348in}}%
\pgfusepath{stroke}%
\end{pgfscope}%
\begin{pgfscope}%
\pgfpathrectangle{\pgfqpoint{3.724747in}{0.345370in}}{\pgfqpoint{2.575253in}{2.571297in}}%
\pgfusepath{clip}%
\pgfsetrectcap%
\pgfsetroundjoin%
\pgfsetlinewidth{1.505625pt}%
\definecolor{currentstroke}{rgb}{1.000000,0.498039,0.054902}%
\pgfsetstrokecolor{currentstroke}%
\pgfsetdash{}{0pt}%
\pgfpathmoveto{\pgfqpoint{3.841804in}{2.168090in}}%
\pgfpathlineto{\pgfqpoint{3.868714in}{2.173397in}}%
\pgfpathlineto{\pgfqpoint{3.895623in}{2.187262in}}%
\pgfpathlineto{\pgfqpoint{3.922533in}{2.222276in}}%
\pgfpathlineto{\pgfqpoint{3.949443in}{2.290259in}}%
\pgfpathlineto{\pgfqpoint{3.976352in}{2.391059in}}%
\pgfpathlineto{\pgfqpoint{4.003262in}{2.510322in}}%
\pgfpathlineto{\pgfqpoint{4.030172in}{2.621150in}}%
\pgfpathlineto{\pgfqpoint{4.057081in}{2.700747in}}%
\pgfpathlineto{\pgfqpoint{4.083991in}{2.738938in}}%
\pgfpathlineto{\pgfqpoint{4.110901in}{2.739767in}}%
\pgfpathlineto{\pgfqpoint{4.137810in}{2.715428in}}%
\pgfpathlineto{\pgfqpoint{4.164720in}{2.683131in}}%
\pgfpathlineto{\pgfqpoint{4.191629in}{2.653283in}}%
\pgfpathlineto{\pgfqpoint{4.218539in}{2.618359in}}%
\pgfpathlineto{\pgfqpoint{4.245449in}{2.558937in}}%
\pgfpathlineto{\pgfqpoint{4.272358in}{2.462632in}}%
\pgfpathlineto{\pgfqpoint{4.299268in}{2.339021in}}%
\pgfpathlineto{\pgfqpoint{4.326178in}{2.211128in}}%
\pgfpathlineto{\pgfqpoint{4.353087in}{2.098123in}}%
\pgfpathlineto{\pgfqpoint{4.379997in}{2.006896in}}%
\pgfpathlineto{\pgfqpoint{4.406907in}{1.927292in}}%
\pgfpathlineto{\pgfqpoint{4.433816in}{1.839035in}}%
\pgfpathlineto{\pgfqpoint{4.460726in}{1.729134in}}%
\pgfpathlineto{\pgfqpoint{4.487636in}{1.602201in}}%
\pgfpathlineto{\pgfqpoint{4.514545in}{1.471831in}}%
\pgfpathlineto{\pgfqpoint{4.541455in}{1.351044in}}%
\pgfpathlineto{\pgfqpoint{4.568364in}{1.246820in}}%
\pgfpathlineto{\pgfqpoint{4.595274in}{1.164871in}}%
\pgfpathlineto{\pgfqpoint{4.622184in}{1.109714in}}%
\pgfpathlineto{\pgfqpoint{4.649093in}{1.083985in}}%
\pgfpathlineto{\pgfqpoint{4.676003in}{1.085839in}}%
\pgfpathlineto{\pgfqpoint{4.702913in}{1.108573in}}%
\pgfpathlineto{\pgfqpoint{4.729822in}{1.146151in}}%
\pgfpathlineto{\pgfqpoint{4.756732in}{1.197884in}}%
\pgfpathlineto{\pgfqpoint{4.783642in}{1.267895in}}%
\pgfpathlineto{\pgfqpoint{4.810551in}{1.355497in}}%
\pgfpathlineto{\pgfqpoint{4.837461in}{1.447801in}}%
\pgfpathlineto{\pgfqpoint{4.864371in}{1.530210in}}%
\pgfpathlineto{\pgfqpoint{4.891280in}{1.606363in}}%
\pgfpathlineto{\pgfqpoint{4.918190in}{1.692972in}}%
\pgfpathlineto{\pgfqpoint{4.945099in}{1.794344in}}%
\pgfpathlineto{\pgfqpoint{4.972009in}{1.896497in}}%
\pgfpathlineto{\pgfqpoint{4.998919in}{1.982628in}}%
\pgfpathlineto{\pgfqpoint{5.025828in}{2.044029in}}%
\pgfpathlineto{\pgfqpoint{5.052738in}{2.084885in}}%
\pgfpathlineto{\pgfqpoint{5.079648in}{2.114648in}}%
\pgfpathlineto{\pgfqpoint{5.106557in}{2.137643in}}%
\pgfpathlineto{\pgfqpoint{5.133467in}{2.152292in}}%
\pgfpathlineto{\pgfqpoint{5.160377in}{2.156879in}}%
\pgfpathlineto{\pgfqpoint{5.187286in}{2.157137in}}%
\pgfpathlineto{\pgfqpoint{5.214196in}{2.162987in}}%
\pgfpathlineto{\pgfqpoint{5.241106in}{2.179949in}}%
\pgfpathlineto{\pgfqpoint{5.268015in}{2.204981in}}%
\pgfpathlineto{\pgfqpoint{5.294925in}{2.229483in}}%
\pgfpathlineto{\pgfqpoint{5.321834in}{2.244894in}}%
\pgfpathlineto{\pgfqpoint{5.348744in}{2.248591in}}%
\pgfpathlineto{\pgfqpoint{5.375654in}{2.244480in}}%
\pgfpathlineto{\pgfqpoint{5.402563in}{2.239756in}}%
\pgfpathlineto{\pgfqpoint{5.429473in}{2.240950in}}%
\pgfpathlineto{\pgfqpoint{5.456383in}{2.251682in}}%
\pgfpathlineto{\pgfqpoint{5.483292in}{2.276330in}}%
\pgfpathlineto{\pgfqpoint{5.510202in}{2.318486in}}%
\pgfpathlineto{\pgfqpoint{5.537112in}{2.376998in}}%
\pgfpathlineto{\pgfqpoint{5.564021in}{2.446096in}}%
\pgfpathlineto{\pgfqpoint{5.590931in}{2.519145in}}%
\pgfpathlineto{\pgfqpoint{5.617841in}{2.589749in}}%
\pgfpathlineto{\pgfqpoint{5.644750in}{2.649751in}}%
\pgfpathlineto{\pgfqpoint{5.671660in}{2.691517in}}%
\pgfpathlineto{\pgfqpoint{5.698569in}{2.715428in}}%
\pgfpathlineto{\pgfqpoint{5.725479in}{2.729392in}}%
\pgfpathlineto{\pgfqpoint{5.752389in}{2.737437in}}%
\pgfpathlineto{\pgfqpoint{5.779298in}{2.731747in}}%
\pgfpathlineto{\pgfqpoint{5.806208in}{2.697880in}}%
\pgfpathlineto{\pgfqpoint{5.833118in}{2.629391in}}%
\pgfpathlineto{\pgfqpoint{5.860027in}{2.534825in}}%
\pgfpathlineto{\pgfqpoint{5.886937in}{2.426609in}}%
\pgfpathlineto{\pgfqpoint{5.913847in}{2.308466in}}%
\pgfpathlineto{\pgfqpoint{5.940756in}{2.177321in}}%
\pgfpathlineto{\pgfqpoint{5.967666in}{2.029489in}}%
\pgfpathlineto{\pgfqpoint{5.994576in}{1.867055in}}%
\pgfusepath{stroke}%
\end{pgfscope}%
\begin{pgfscope}%
\pgfpathrectangle{\pgfqpoint{3.724747in}{0.345370in}}{\pgfqpoint{2.575253in}{2.571297in}}%
\pgfusepath{clip}%
\pgfsetrectcap%
\pgfsetroundjoin%
\pgfsetlinewidth{1.505625pt}%
\definecolor{currentstroke}{rgb}{0.172549,0.627451,0.172549}%
\pgfsetstrokecolor{currentstroke}%
\pgfsetdash{}{0pt}%
\pgfpathmoveto{\pgfqpoint{3.841804in}{1.942035in}}%
\pgfpathlineto{\pgfqpoint{3.868714in}{1.956951in}}%
\pgfpathlineto{\pgfqpoint{3.895623in}{1.970182in}}%
\pgfpathlineto{\pgfqpoint{3.922533in}{1.978645in}}%
\pgfpathlineto{\pgfqpoint{3.949443in}{1.981730in}}%
\pgfpathlineto{\pgfqpoint{3.976352in}{1.987410in}}%
\pgfpathlineto{\pgfqpoint{4.003262in}{2.011581in}}%
\pgfpathlineto{\pgfqpoint{4.030172in}{2.070347in}}%
\pgfpathlineto{\pgfqpoint{4.057081in}{2.169569in}}%
\pgfpathlineto{\pgfqpoint{4.083991in}{2.301623in}}%
\pgfpathlineto{\pgfqpoint{4.110901in}{2.451278in}}%
\pgfpathlineto{\pgfqpoint{4.137810in}{2.597812in}}%
\pgfpathlineto{\pgfqpoint{4.164720in}{2.715428in}}%
\pgfpathlineto{\pgfqpoint{4.191629in}{2.783522in}}%
\pgfpathlineto{\pgfqpoint{4.218539in}{2.799790in}}%
\pgfpathlineto{\pgfqpoint{4.245449in}{2.781270in}}%
\pgfpathlineto{\pgfqpoint{4.272358in}{2.747594in}}%
\pgfpathlineto{\pgfqpoint{4.299268in}{2.702992in}}%
\pgfpathlineto{\pgfqpoint{4.326178in}{2.636889in}}%
\pgfpathlineto{\pgfqpoint{4.353087in}{2.541314in}}%
\pgfpathlineto{\pgfqpoint{4.379997in}{2.420720in}}%
\pgfpathlineto{\pgfqpoint{4.406907in}{2.284015in}}%
\pgfpathlineto{\pgfqpoint{4.433816in}{2.135249in}}%
\pgfpathlineto{\pgfqpoint{4.460726in}{1.976435in}}%
\pgfpathlineto{\pgfqpoint{4.487636in}{1.813789in}}%
\pgfpathlineto{\pgfqpoint{4.514545in}{1.659020in}}%
\pgfpathlineto{\pgfqpoint{4.541455in}{1.520514in}}%
\pgfpathlineto{\pgfqpoint{4.568364in}{1.396002in}}%
\pgfpathlineto{\pgfqpoint{4.595274in}{1.275130in}}%
\pgfpathlineto{\pgfqpoint{4.622184in}{1.147611in}}%
\pgfpathlineto{\pgfqpoint{4.649093in}{1.009991in}}%
\pgfpathlineto{\pgfqpoint{4.676003in}{0.864658in}}%
\pgfpathlineto{\pgfqpoint{4.702913in}{0.719811in}}%
\pgfpathlineto{\pgfqpoint{4.729822in}{0.590657in}}%
\pgfpathlineto{\pgfqpoint{4.756732in}{0.497724in}}%
\pgfpathlineto{\pgfqpoint{4.783642in}{0.462247in}}%
\pgfpathlineto{\pgfqpoint{4.810551in}{0.498110in}}%
\pgfpathlineto{\pgfqpoint{4.837461in}{0.603390in}}%
\pgfpathlineto{\pgfqpoint{4.864371in}{0.754702in}}%
\pgfpathlineto{\pgfqpoint{4.891280in}{0.910599in}}%
\pgfpathlineto{\pgfqpoint{4.918190in}{1.027831in}}%
\pgfpathlineto{\pgfqpoint{4.945099in}{1.082538in}}%
\pgfpathlineto{\pgfqpoint{4.972009in}{1.080323in}}%
\pgfpathlineto{\pgfqpoint{4.998919in}{1.046468in}}%
\pgfpathlineto{\pgfqpoint{5.025828in}{1.011607in}}%
\pgfpathlineto{\pgfqpoint{5.052738in}{1.002525in}}%
\pgfpathlineto{\pgfqpoint{5.079648in}{1.031701in}}%
\pgfpathlineto{\pgfqpoint{5.106557in}{1.094095in}}%
\pgfpathlineto{\pgfqpoint{5.133467in}{1.186325in}}%
\pgfpathlineto{\pgfqpoint{5.160377in}{1.313569in}}%
\pgfpathlineto{\pgfqpoint{5.187286in}{1.479771in}}%
\pgfpathlineto{\pgfqpoint{5.214196in}{1.675213in}}%
\pgfpathlineto{\pgfqpoint{5.241106in}{1.871591in}}%
\pgfpathlineto{\pgfqpoint{5.268015in}{2.031128in}}%
\pgfpathlineto{\pgfqpoint{5.294925in}{2.127030in}}%
\pgfpathlineto{\pgfqpoint{5.321834in}{2.158973in}}%
\pgfpathlineto{\pgfqpoint{5.348744in}{2.147792in}}%
\pgfpathlineto{\pgfqpoint{5.375654in}{2.117899in}}%
\pgfpathlineto{\pgfqpoint{5.402563in}{2.086804in}}%
\pgfpathlineto{\pgfqpoint{5.429473in}{2.062727in}}%
\pgfpathlineto{\pgfqpoint{5.456383in}{2.045653in}}%
\pgfpathlineto{\pgfqpoint{5.483292in}{2.033054in}}%
\pgfpathlineto{\pgfqpoint{5.510202in}{2.025555in}}%
\pgfpathlineto{\pgfqpoint{5.537112in}{2.024685in}}%
\pgfpathlineto{\pgfqpoint{5.564021in}{2.028701in}}%
\pgfpathlineto{\pgfqpoint{5.590931in}{2.032633in}}%
\pgfpathlineto{\pgfqpoint{5.617841in}{2.032847in}}%
\pgfpathlineto{\pgfqpoint{5.644750in}{2.031013in}}%
\pgfpathlineto{\pgfqpoint{5.671660in}{2.036414in}}%
\pgfpathlineto{\pgfqpoint{5.698569in}{2.065744in}}%
\pgfpathlineto{\pgfqpoint{5.725479in}{2.134024in}}%
\pgfpathlineto{\pgfqpoint{5.752389in}{2.244386in}}%
\pgfpathlineto{\pgfqpoint{5.779298in}{2.385893in}}%
\pgfpathlineto{\pgfqpoint{5.806208in}{2.535934in}}%
\pgfpathlineto{\pgfqpoint{5.833118in}{2.659172in}}%
\pgfpathlineto{\pgfqpoint{5.860027in}{2.715428in}}%
\pgfpathlineto{\pgfqpoint{5.886937in}{2.688393in}}%
\pgfpathlineto{\pgfqpoint{5.913847in}{2.600198in}}%
\pgfpathlineto{\pgfqpoint{5.940756in}{2.496712in}}%
\pgfpathlineto{\pgfqpoint{5.967666in}{2.411053in}}%
\pgfpathlineto{\pgfqpoint{5.994576in}{2.338666in}}%
\pgfpathlineto{\pgfqpoint{6.021485in}{2.253587in}}%
\pgfpathlineto{\pgfqpoint{6.048395in}{2.138759in}}%
\pgfpathlineto{\pgfqpoint{6.075304in}{1.994652in}}%
\pgfpathlineto{\pgfqpoint{6.102214in}{1.830061in}}%
\pgfpathlineto{\pgfqpoint{6.129124in}{1.651768in}}%
\pgfpathlineto{\pgfqpoint{6.156033in}{1.467917in}}%
\pgfpathlineto{\pgfqpoint{6.182943in}{1.282568in}}%
\pgfusepath{stroke}%
\end{pgfscope}%
\begin{pgfscope}%
\pgfsetrectcap%
\pgfsetmiterjoin%
\pgfsetlinewidth{0.803000pt}%
\definecolor{currentstroke}{rgb}{0.000000,0.000000,0.000000}%
\pgfsetstrokecolor{currentstroke}%
\pgfsetdash{}{0pt}%
\pgfpathmoveto{\pgfqpoint{3.724747in}{0.345370in}}%
\pgfpathlineto{\pgfqpoint{3.724747in}{2.916667in}}%
\pgfusepath{stroke}%
\end{pgfscope}%
\begin{pgfscope}%
\pgfsetrectcap%
\pgfsetmiterjoin%
\pgfsetlinewidth{0.803000pt}%
\definecolor{currentstroke}{rgb}{0.000000,0.000000,0.000000}%
\pgfsetstrokecolor{currentstroke}%
\pgfsetdash{}{0pt}%
\pgfpathmoveto{\pgfqpoint{6.300000in}{0.345370in}}%
\pgfpathlineto{\pgfqpoint{6.300000in}{2.916667in}}%
\pgfusepath{stroke}%
\end{pgfscope}%
\begin{pgfscope}%
\pgfsetrectcap%
\pgfsetmiterjoin%
\pgfsetlinewidth{0.803000pt}%
\definecolor{currentstroke}{rgb}{0.000000,0.000000,0.000000}%
\pgfsetstrokecolor{currentstroke}%
\pgfsetdash{}{0pt}%
\pgfpathmoveto{\pgfqpoint{3.724747in}{0.345370in}}%
\pgfpathlineto{\pgfqpoint{6.300000in}{0.345370in}}%
\pgfusepath{stroke}%
\end{pgfscope}%
\begin{pgfscope}%
\pgfsetrectcap%
\pgfsetmiterjoin%
\pgfsetlinewidth{0.803000pt}%
\definecolor{currentstroke}{rgb}{0.000000,0.000000,0.000000}%
\pgfsetstrokecolor{currentstroke}%
\pgfsetdash{}{0pt}%
\pgfpathmoveto{\pgfqpoint{3.724747in}{2.916667in}}%
\pgfpathlineto{\pgfqpoint{6.300000in}{2.916667in}}%
\pgfusepath{stroke}%
\end{pgfscope}%
\begin{pgfscope}%
\definecolor{textcolor}{rgb}{0.000000,0.000000,0.000000}%
\pgfsetstrokecolor{textcolor}%
\pgfsetfillcolor{textcolor}%
\pgftext[x=5.012374in,y=3.000000in,,base]{\color{textcolor}\rmfamily\fontsize{12.000000}{14.400000}\selectfont (d) Scaled}%
\end{pgfscope}%
\end{pgfpicture}%
\makeatother%
\endgroup%

    \caption{Four plots of three random \acrshort{4ch} \acrshort{gls} curves that are preprocessed in the three different ways. (a) no preprocessing, (b) normalization, (c) Z-score normalization and (d) scaling}
    \label{fig:preproc}
\end{figure}

\subsection{Dissimilarity measurement}

When estimating dissimilarity between patients represented by a peak-value dataset Euclidean distance was used. To measure the dissimilarity between longitudinal strain curves in the \acrshort{tsc} model \acrshort{dtw} distanse was used. Recall that the \acrshort{dtw} distance between to time series is the length of the shortest \acrshort{dtw} path between them. To calculate the \acrshort{dtw} distance the \textbf{dtaidistance} library was used. The \textbf{dtaidistance} library is used by the DTAI Research Group to measure distances between time series. To encapsulate all the dissimilarity between patients in a single matrix, one first has to calculate one matrix of \acrshort{dtw} distances for each of the time series used to represent patients. Say that a patient was represented using the \acrshort{gls} curves in the three views. To calculate the dissimilarity matrix one would first estimate the \acrshort{dtw} distance between all the \acrshort{4ch}/\acrshort{gls} curves, then all the \acrshort{2ch}/\acrshort{gls} curves and finally all the \acrshort{aplax}/\acrshort{gls} curves. By adding the three matrices of \acrshort{dtw} distances together one gets the dissimilarity matrix. \bigskip

\subsection{Hierarchical Agglomorative Clustering}

\subsection{Cluster Assignment Evaluation}

\section{Artificial Neural Network} \label{sec:meth_nn}

\section{Peak-value Supervised Classifiers} \label{sec:meth_pvsc}


